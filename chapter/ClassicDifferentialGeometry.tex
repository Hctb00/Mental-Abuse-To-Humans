\chapter{古典微分几何}
\section{曲线论}

自从古希腊的数学家们建立了几何学,人们渐渐熟悉了在欧氏空间中计算距离,夹角,面积,体积.随后笛卡尔提出了坐标系的概念,在坐标系中计算上述内容更加形式化.所谓几何性质,或者说某种意义上讲纯粹的几何性质,是指那些不依赖于坐标系的选取的,或者说不是代数角度也不是分析角度的,几何对象所特有的性质.例如平面中的一个圆周,我们既可以在选定坐标系后表述成一个方程;也可以按照参数曲线的方式表示成一个函数;但是我们也可以不依赖于坐标系的,表示为到一个定点距离固定的点构成的集合.本节我们的核心就是探究曲线的不依赖于坐标系选取的,纯粹几何学意义上的,对它的形状的系统描述.不过通常来讲直接抛弃坐标系处理几何对象十分麻烦,于是我们的战略是先选定一个坐标系,描述一种性质,然后验证这个性质不依赖于坐标系的选取.

根据不同的需求,会定义一条欧氏空间$X=\mathbb{R}^n$中的曲线为$\mathbb{R}$中的开/闭区间到$X$的连续/可导/连续可导/光滑的映射.在本节里我们把开区间到$X$的光滑(即任意阶导数均存在)的映射称为曲线.通常把曲线参数化表示为$q(t):[a,b]\to\mathbb{R}^n$,$t\mapsto(x_1(t),x_2(t),\cdots)$.那么光滑条件就是在讲$x_i(t)$都是任意阶可导的函数.一个曲线上一点处导数为0等价于讲这一点处曲线没有切向量.称处处导数非0的曲线为正则曲线.

曲线长度.阿基米德最早给出了圆周的长度公式,随后直到1600年左右,牛顿第一个给出了一般曲线长度的定义.从物理角度讲,曲线的参数表示相当于位置关于时间的函数,那么参数表示的导函数就表示每个点的速度,于是移动距离就是积分式$\int_a^b|\phi'(t)|\mathrm{d}t$.

一个曲线称为弧长参数曲线,如果它的导函数的模长总是1.那么任意正则曲线可以换元成为弧长参数曲线.事实上正则这个条件保证了弧长函数$s(t)=\int_{t_0}^t|\phi'(t)|\mathrm{d}t$满足$\frac{\mathrm{d}s}{\mathrm{d}t}=|\phi'(t)|>0$,而弧长函数是从$[a,b]$到$[0,s(b)]$的双射,于是按照整体逆映射定理,存在$s$的连续可微逆映射$t:[0,s(b)]\to[a,b]$,于是$\psi(s)=\phi(t(s))$是换了参数的曲线,满足$\left|\frac{\mathrm{d}\phi(f(s))}{\mathrm{d}s}\right|=\left|\frac{\mathrm{d}\phi(f(s))}{\mathrm{d}t}\frac{\mathrm{d}t}{\mathrm{d}s}\right|=|\phi'(t)|\cdot\frac{1}{|\phi'(t)|}=1$.今后我们只考虑弧长参数曲线,因为我们主要关注的是曲线的几何性质,而不是它虽时间变化如何运动.

曲率.曲率度量的是曲线的弯曲程度.对于平面曲线,曲率的思想就是寻找曲线在一个点附近的用圆弧的最佳逼近,正如切线是曲线在一个点附近用直线的最佳逼近.此时这个圆弧的半径就度量了曲线的弯曲程度.Huygens曾运用纯几何方法考虑了很多一般曲线的曲率,牛顿似乎是最早的定义曲率概念的人.

给定欧氏空间$\mathbb{R}^k$中的一条弧长参数的正则曲线$\gamma(s)$,记$T(s)=\gamma'(s)$,这是取值在单位球面上的曲线,称为曲线的单位切向量.那么$T(s)$总和$T'(s)$是垂直的,这是因为从$T(s)\cdot T(s)=1$求导可得$T'(s)\cdot T(s)=0$.

现在如果$T(s)$是固定的映射,那么曲线$\gamma(s)$是一条直线,这说明$T(s)$的变化程度可以度量曲线的弯曲称度.我们就定义$\kappa(s)=|T'(s)|$为曲线$\gamma(s)$的曲率函数.如果曲率总不为0,并且欧氏空间是三维的,我们定义曲线在一点$s$处的法向量和副法向量分别为$N(s)=\frac{T'(s)}{|T'(s)|}$和$B(s)=T(s)\times N(s)$.此时$\{T(s),N(s),B(s)\}$是$\mathbb{R}^3$中的一组标准正交基.称它为曲线的Frenet标架.

Frenet标架随点在曲线上的移动而发生变化,为了描述标架发生的变化,需要探究$T,N,B$的导数,我们记$X_1=T$,$X_2=N$,$X_3=B$.记$\frac{\mathrm{d}X_i}{\mathrm{d}s}=\sum_ja_{ij}X_j$.记系数矩阵为$A$,它是一个三阶实方阵.

系数矩阵$A$是一个斜对称矩阵,于是$a_{ji}=-a_{ij}$,并且$a_ii\equiv0$.证明,按照标准正交性,有$X_i\cdot X_j=\delta_{ij}$.求导得到$X_i'\cdot X_j+X_i\cdot X_j'=0$.带入上述定义的方程,得到$\left(\sum_ka_{ik}X_k\right)\cdot X_j+X_i\cdot\left(\sum_ka_{jk}X_k\right)$.这整理得到$a_{ij}+a_{ji}=0$.也即$A$是斜对称矩阵.

而按照曲率的定义有$T'(s)=\kappa(s)N(s)$,说明$a_{13}=0$,于是$a_{31}=0$,这说明$B'(s)=-a_{23}N$以及$N'(s)=-\kappa(s)T+a_{23}B$.定义$a_{23}$为挠率函数,记作$\tau(s)$.综上我们得到Frenet-Serret方程,如果弧长参数曲线$\gamma(s)$满足曲率函数处处非0,那么有等式:
$$\begin{array}{cccc}
T'(s)=&&\kappa(s)N&\\
N'(s)=&-\kappa(s)T&&+\tau(s)B\\
B'(s)=&&-\tau(s)N&\end{array}$$

挠率函数的几何意义.固定弧长参数的一个取值$s$,那么曲线在该点的移动方向是$T(s)$,并且该点局部上的运动相当于一个圆心位于$N(s)$方向的圆弧.这两个向量确定的平面在古典几何学里被称为密切平面,而$B(s)$是垂直于这个平面的向量.倘若曲线一直位于这个密切平面内,那么$B(s)$将会是一个定值,于是它的导数,$-\tau(s)N$恒为0,于是挠率函数恒为0.逆命题也是成立的,即挠率函数恒为0说明曲线恒落在一个平面内,因为按照Frenet-Serret方程说明$B(s)$是固定的,于是曲线落在平面$P=\{x\mid(x-\gamma(s_0))\cdot B(s)=0\}$上.于是挠率函数度量的是曲线离恒落在一个平面上的差距,或者说某种意义上的曲线的"扭曲称度".

按照克莱因的观点,几何就是处理几何对象在各种运动群下不变的性质.欧氏空间的运动群是指欧氏变换群,它的元是欧氏空间上的等距变换,即平移旋转对称变换的复合.矩阵表示是$Ox+b$,这里$O$是欧氏空间中的一个正交变换.有时也会考虑欧氏群的一种特殊的子群作为运动群,这个特殊的子群称为特殊欧氏群,它是由全体刚体变换构成的子群,也即只由平移旋转进行复合的等距变换.刚体变换的矩阵表示是$Ax+b$,这里$A$是行列式为1的正交变换.

现在我们把曲线放置这个观点之下.首先,和物理角度不同,我们不关心运动随时间发生的变化,因而我们选取了特殊的弧长参数曲线,并由此推出了两个涉及到曲线的函数曲率函数$\kappa(s)$和挠率函数$\tau(s)$.对称变换会改变挠率函数的正负,但是不改变曲率函数.刚体变换既不改变曲率函数也不改变挠率函数.于是曲率和挠率两个概念在特殊欧氏群这个运动群下是曲线的不变量.令人惊讶的是,这两个不变量在相差一个刚体变换的意义下完全决定了三维欧氏空间中的曲线,此即曲线论基本定理.

曲线论基本定理.
\begin{enumerate}
	\item 设$\gamma$是弧长参数曲线,设它的曲率函数处处不取0.那么曲率和挠率函数在相差一个刚体变换的意义下完全决定了该曲线.换句话讲,如果$\sigma(s)$是另一条具有相同曲率函数和挠率函数的弧长参数曲线,那么存在刚体变换$M$满足$\sigma(s)=M\circ\gamma(s)$.
	\item 反过来,如果$\kappa(s)>0$和$\tau(s)$都是光滑函数,那么存在一条弧长参数曲线$\gamma(s)$以这两个函数分别为曲率函数和挠率函数.
\end{enumerate}
\begin{proof}
	占坑.
\end{proof}

【后续占坑,快考试了.】

\section{曲面的局部理论}

曲面定义的核心思想是局部同胚于$\mathbb{R}^2$中的开集.延续这个观点就会把曲面论划分为两部分,第一部分是局部理论,处理的是同胚于$\mathbb{R}^2$中开集的曲面的局部部分,这时我们不考虑曲面的整体拓扑,而是仅仅考虑局部上的性质;第二部分则是整体理论,处理的是把所有这些局部部分粘合成整体曲面后的整体性质.本节我们先来处理第一部分.

$\mathbb{R}^3$中的一个正则参数曲面是指一个光滑函数$s(u,v):U\subset\mathbb{R}^2\to\mathbb{R}^3$,这里$U$是$\mathbb{R}^2$的一个连通开集.满足$s$是$U$到$s(U)$的同胚映射,并且$s_u=\frac{\partial s}{\partial u}$和$s_v=\frac{\partial s}{\partial v}$在$U$上处处是一对线性无关的向量组.

参数面线定义的一些注解.首先,和曲线一样我们定义参数曲面是一个光滑映射,为了在曲线曲面上做微积分,光滑条件是需要的,我们会在给出一些概念之后解释定义中要求光滑和要求$C^k$的区别.要求$s$是单射的目的是排除曲面自交的情况,自交情况下曲面在自交点处的切平面可能是不唯一的,并且这个要求符合于我们后面会给出的把曲面看作$\mathbb{R}^3$的子集而不是映射这个观点.有些书不会要求$s$是同胚,我们这里要求同胚主要是因为一些地方需要这个条件,例如曲面面积的定义中.连通性主要是为定向概念服务,尽管严格的定向概念要在整体曲面上定义,我们会暂时给出一个版本的定向概念.最后一个条件保证了曲面每个点处都有切平面,这个条件还可以等价的写作对每个点$p\in U$,$s$在该点处诱导的微分映射$\mathbb{R}^2\to\mathbb{R}^3$是单射.

参数曲面可以存在不同的参数表示.给定两个参数曲线$q_1(u,v):U\subset\mathbb{R}^2\to\mathbb{R}^3$和$q_2(s,t):O\subset\mathbb{R}^2\to\mathbb{R}^3$.称它们是同一个曲面的不同参数表示,如果存在$O$和$U$之间的光滑同胚$h$,满足$q_2(s,t)=q_1\circ h(s,t)$.我们更关注的是不依赖参数表示选取的曲面的性质,在这个观点下把曲面理解为$\mathbb{R}^3$的存在参数表示的子集是更为合理的.

定向和参数表示.给定曲面$S$的两个参数表示,按照定义这等价于存在两个$\mathbb{R}^2$上开集之间的光滑同胚,满足其中一个参数表示复合这个光滑同胚等于另一个.注意光滑同胚意味着,它作为可微映射的雅各比行列式总是不取0的,否则诱导的切映射不会是双射.这个结果结合雅各比行列式是连通开集上的连续函数,说明一个光滑同胚的雅各比行列式要么恒为正要么恒为负.另外我们可以说明尽管固定两个参数表示后可以选取不同的光滑同胚,但是它们的雅各比行列式的正负号是固定的,这只要注意到参数表示按照定义有二维切空间,说明它们的雅各比矩阵都是行满秩的,于是等价于说$A,B$同型行满秩矩阵,存在可逆矩阵$X,Y$满足$AX=B$和$AY=B$,那么$X,Y$的行列式总是同号的.现在我们在$S$的全部参数表示上定义一个等价关系,两个参数表示称为等价的如果过渡它们的光滑同胚的雅各比行列式是正的.于是结合我们刚刚所讨论的,这的确构成了一个等价关系.它的等价类有两个.在取定一个参数表示之后,全体过渡到这个参数表示的光滑同胚的雅各比行列式取正的那些参数表示构成一个等价类,取负的那些参数表示也构成一个等价类,并且不再存在其他的参数表示.

在曲线论中我们曾指出,我们关注的是几何对象不依赖于坐标系选取的性质,但是完全不运用坐标系会让我们无法对几何对象做深入的探究,因此我们的思路是先借助坐标系,或者说曲面的参数表示,来得到一种关于曲面的性质,然后再考虑是否依赖于参数表示的问题.此时会出现两种情况,如果这个性质不随任意参数表示的选取而改变,就称它不依赖于参数表示的选取;如果这个性质不随上述定义的等价的参数表示的选取而改变,就称它不依赖于保定向的参数表示的选取.

曲面的例子.曲面最简单的例子之一是蒙日(Monge)块.即给定开集$U\subset\mathbb{R}^2$上的光滑映射$f$,则三个$U\to\mathbb{R}^3$的映射的图像构成了曲面$s(u,v)=(u,v,f(u,v))$,$s(u,v)=(u,f(u,v),v)$,$s(u,v)=(f(u,v),u,v)$.以第一个为例,我们来证明$s(u,v)$的确是曲面.首先$s$是光滑的双射,并且按照$\frac{\partial(x,y)}{\partial(u,v)}=1$说明两个偏导数作为切向量在$U$上处处线性无关.最后只要说明$s$是同胚,为此注意到$s$的逆映射就是像集到前两个分量的投影映射,这自然是连续的.

给定曲面$S$,任取$p\in S$,该点处的切平面定义为两个偏导数$s_u,s_v$张成的$\mathbb{R}^3$的二维子空间,记作$T_pS$.切平面中的元称为点$p$处的切向量.我们断言曲面固定一点处的切空间不依赖于参数的选取.
\begin{proof}
	
	给定曲面$S$的两个参数表示$s_1(s,t):U\to\mathbb{R}^3$和$s_2(u,v):V\to\mathbb{R}^3$.按照定义这等价于说存在$U\to V$的光滑同胚$h=(u(s,t),v(s,t))$,满足$s_1(u,v)=s_2\circ h(u,v)$.链式法则说明$\frac{\partial s_1}{\partial s}=\frac{\partial s_2}{\partial u}\frac{\partial u}{\partial s}+\frac{\partial s_2}{\partial v}\frac{\partial v}{\partial s}$.于是得到$\frac{\partial s_1}{\partial s}\subset\mathrm{span}\{\frac{\partial s_2}{\partial u},\frac{\partial s_2}{\partial v}\}$.同理$\frac{\partial s_1}{\partial t}\subset\mathrm{span}\{\frac{\partial s_2}{\partial u},\frac{\partial s_2}{\partial v}\}$.于是$s,t$坐标下的切平面包含于$u,v$坐标下生成的切平面,于是二者相同,即固定点的切空间不依赖于参数的选取.
\end{proof}

切空间的曲线角度描述.给定曲面$S$,称$\mathbb{R}^3$中的一条曲线$\alpha:I\to S$在曲面$S$上,如果存在平面中的一条曲线$\gamma(t):I\to U$,满足$\alpha(t)=s(\gamma(t))$.曲线$S$上一点$p$处的切空间恰好就是由全部过该点的位于$S$内的参数曲线在该点处切向量构成的集合.
\begin{proof}
	
	一方面,按照定义任一过点$p$的位于$S$内的参数曲线$q(t)$可以表示为$q(t)=s(u(t),v(t))$.不妨设$p=(u_0,v_0)$,并且$I$包含点0,并且$u(0)=u_0$,$v(0)=v_0$.求导可得这样的曲线在该点的切向量总落在曲面在该点的切平面内.
	
	另一方面,设$u(t)=at+u_0$,$v(t)=bt+v_0$,那么过该点的全部切线为$\frac{\partial s}{\partial u}a+\frac{\partial s}{\partial v}b$.
\end{proof}

曲面上的可微函数.给定曲面$S$,给定一个映射$f:S\to\mathbb{R}$.一个合理的对$f$可微性的定义是,取$S$的参数表示$s(u,v):U\subset\mathbb{R}^2\to\mathbb{R}^3$.约定$f$在点$s(p)\in S$处是$C^k$的(即k阶连续可微),如果$f\circ s$作为$U\to\mathbb{R}$的映射在点$p$处是$C^k$的.那么首要需要验证的是对$f$的可微性不依赖于参数表示的选取.为此设曲面$S$有两个不同的参数表示$s^1(u,v):U\to S$和$s^2(s,t):V\to S$,于是存在光滑同胚$h:U\to V$满足$s^1=s^2\circ h$.那么自然有$f\circ s^1$在点$p\in U$处是$C^k$可微等价于$f\circ s^2$在点$h(p)\in V$处是$C^k$可微.

这里我们补充一个关于曲面定义的注解.我们在曲面定义中要求参数表示是光滑的函数,在定义同一个曲面的不同参数表示时要求过渡映射是光滑同胚.这样定义的曲面通常称为$C^{\infty}$曲面或者光滑曲面.如果我们约定曲面定义中的参数表示以及不同参数表示之间的过渡映射是$C^k$的,则称定义出来的曲面为$C^k$曲面.注意在$C^k$曲面上,我们就无法定义出曲面上的$C^n$可微函数,这里$n>k$.

类似可定义两曲面之间的可微映射.给定曲面$S$和$N$,设参数表示分别是$s:U\to S$和$t:V\to N$,给定映射$f:S\to N$,称它在点$s(p)\in S$处$C^k$可微,如果$t^{-1}\circ f\circ s$作为$U\to V$的映射在点$p$是$C^k$可微的.同样可以验证这个定义不依赖于参数表示的选取.

可微映射的微分.给定两个曲面$S,N$,设参数表示同上一段,给定在点$a\in S$可微的映射$f:S\to N$,按照微积分中对可微映射的微分的定义,合理的想法是,$f$在点$a$处的微分是指一个线性映射$df_a:T_aS\to T_{f(a)}N$.那么如何定义这个线性映射呢?或许最几何直观的角度是运用切空间的曲线角度定义.我们知道点$a$处的一个向量$v$是切向量,等价于存在曲面上的过点$a$的曲线$\gamma:(-\varepsilon,\varepsilon)\to S$,$\gamma(0)=p$,满足该向量可以表示为$\gamma'(0)$.现在$f\circ\gamma$是曲面$N$上的一条过点$f(a)$的曲线,并且满足$f\circ\gamma(0)=f(a)$.所以这条曲线也定义了$T_{f(a)}N$中的一个向量$(f\circ\gamma)'(0)$.我们就把$df_a(v)$定义为这个向量.需要验证下这个定义的良性,即不依赖于符合条件的$\gamma$的选取,在明确这一点之后,容易看出这个映射是线性映射.【微分不依赖于参数表示的选取】

第一基本形式.设想一种生活在某个曲面上的生物,随着科学和科技的发展,产生了两个工具:一把可以度量任意小长度的尺子,一个可以度量任意小的两条曲线之间角度的量角器.严格的说,前者是指它们可以通过积分手段计算曲面上曲线的长度;后者是指它们可以计算两个曲线切向量之间的夹角.这两个工具允许它们在每个点做出直角坐标系,尽管在我们看来这些坐标轴是弯曲的.经验会告诉它们不需要对单位长度的概念太过纠结,尽管在我们看来判断两段曲线长度相同比运用它们的工具要麻烦得多.另外经验还会告诉它们在它们的世界画出法向量是不可能的.这两个工具允许这种生物计算所有我们在欧氏空间中熟知的度量:长度角度面积.作为局外人的我们,无法获取它们的工具,更无法进入到它们的曲面中,那么如何该做它们世界中的度量问题呢?第一基本形式回答的就是这个问题.曲面的第一基本形式蕴含的信息等价于上述生物通过上述两种工具可获取的全部度量信息.

我们先给出第一基本形式的定义,随后给出曲面上度量的计算方法.和欧氏空间中一样,度量需要内积来提供.一个自然的思路是对曲面上每个点的切空间上赋予一种内积,而曲面作为$\mathbb{R}^3$的子集,它每个点的切空间都是$\mathbb{R}^3$的子空间,于是这个内积我们可以自然的取$\mathbb{R}^3$上的典范内积,这就得到了第一基本形式.

给定参数曲面$S$为$s(u,v):U\subset\mathbb{R}^2\to\mathbb{R}^3$.记$\langle -,-\rangle$为$\mathbb{R}^3$上的典范内积.定义$U$上的函数$E=\langle s_u,s_u\rangle$,$F=\langle s_u,s_v\rangle$,$G=\langle s_v,s_v\rangle$.按照柯西不等式有$EG\ge F^2$.于是对称矩阵$\left(\begin{array}{cc} E&F\\ F&G\end{array}\right)$是正定矩阵,在每个点$s(p)$的切空间中取这个对称矩阵诱导的内积,称为曲面$S$的第一基本形式.它通常会表示为$Edu^2+2Fdudv+Gdv^2$,这里的$du$和$dv$应该理解为足够小的变量.

我们首要需要说明的是,第一基本形式不依赖于参数的选取.
\begin{proof}

假设有光滑同胚$u=u(s,t),v=v(s,t)$.两个参数表示为$s^1(s,t)$和$s^2(u,v)$,那么有$s_1(s,t)=s_2(u(s,t),v(s,t))$.按照链式法则,得到:
$$\left(du,dv\right)=(ds,dt)\left(\begin{array}{cc} \frac{\partial u}{\partial s}&\frac{\partial v}{\partial s}\\\frac{\partial u}{\partial t}&\frac{\partial v}{\partial t}\end{array}\right)$$

这里右侧的$2\times2$矩阵就是该光滑同胚的Jacobi矩阵,记作$J$.现在有如下等式,这里元素乘积要理解为向量在$\mathbb{R}^3$中的典范内积:
$$\left(\begin{array}{cc} E&F\\ F&G\end{array}\right)=\left(\begin{array}{c} s^2_u\\s^2_v\end{array}\right)(s^2_u,s^2_v);\left(\begin{array}{cc} E'&F'\\ F'&G'\end{array}\right)=\left(\begin{array}{c} s^1_s\\s^1_t\end{array}\right)(s^1_s,s^1_t)$$

综上得到:
$$(du,dv)\left(\begin{array}{cc} E&F\\ F&G\end{array}\right)\left(\begin{array}{c} du\\dv\end{array}\right)=(ds,dt)J\left(\begin{array}{cc} E&F\\ F&G\end{array}\right)J^T\left(\begin{array}{c}ds\\dt\end{array}\right)=(ds,dt)\left(\begin{array}{cc} E'&F'\\ F'&G'\end{array}\right)\left(\begin{array}{c}ds\\dt\end{array}\right)$$
\end{proof}

曲线长度公式.给定曲面$S$,设曲线$\alpha:I\to S$在曲面$S$上,按照定义这等价于说存在平面中的一条曲线$\gamma(t):I\to U$,满足$\alpha(t)=s(\gamma(t))$.如果记第一基本形式为$I(-,-)$,那么按照曲线在每一点的切向量都在曲面在该点的切平面内,得到公式$s(t)=\int_a^b|\alpha'(t)|\mathrm{d}t=\int_a^b\sqrt{I(\alpha'(t),\alpha'(t))}\mathrm{d}t$.按照曲面上曲线的定义,这里$\alpha(t)$可以理解为$s(u(t),v(t))$,它是$\mathbb{R}^2$开集中的一条曲线到$S$的映射.这导致曲线长度具有公式:
$$s(t)=\int_a^b\sqrt{E(u')^2+2Fu'v'+G(v')^2}\mathrm{d}t$$

如果曲面上的两个曲线$\alpha:I\to S$和$\beta:I\to S$在点$t=t_0$相交,曲线在交点的夹角定义为交点处两个切向量的夹角.于是按照定义这个夹角$\theta$满足$\cos\theta=\frac{\langle \alpha'(t_0),\beta'(t_0)\rangle}{|\alpha'(t_0)||\beta'(t_0)|}$.特别的,两个充当坐标轴的曲线的夹角$\theta$就满足$\cos\theta=\frac{F}{\sqrt{EG}}$.特别的,曲面的一个特定参数表示的坐标曲线总是垂直的当且仅当$F(u,v)=0,\forall (u,v)\in U$.注意最后描述的这个性质依赖于参数表示的选取.

这里我们补充一下曲面上曲线长度公式的含义.先考虑我们熟知的平面$\mathbb{R}^2$,在其上确定一个点$p$,如果我们把两个坐标$u,v$都改变一个非常小的量$du$和$dv$,那么我们知道根据勾股定理,点$p$改变的距离$ds$满足$ds^2=du^2+dv^2$.现在考虑一个弯曲的曲面$S$,在其上也确定一个点$p$,但是此时坐标不再保长度和夹角,于是当我们对$u,v$改变足够小量的时候,曲面上的点发生的变化不再满足勾股定理.而曲线长度公式告诉我们,在一般的曲面上改变足够小的变量应该满足的等式为$ds^2=Edu^2+2Fdudv+Gdv^2$.这也就是平面和一般曲面上度量结构的本质区别,即第一基本形式的区别.

最后一个曲面上的度量问题是曲面面积的求法.历史上人们先想到曲面面积应该类似于曲线长度一样,用规则的几何图形去逼近.曲线用折线逼近,自然的会考虑用内接多面体逼近曲面,当多面体的所有面的最大直径趋于0时,内接于曲面的面的和将会趋于曲面的面积.随后19世纪末施瓦兹给出了一个具体例子,说明了这个定义下即便圆柱体的表面积也无法求出(见微积分学教程第三卷曲面面积-施瓦兹的例子).他用全等的小三角形内接圆柱体,发现极限依赖于小三角形边长的比例.

那么是什么区别导致了这个定义和曲线的折线逼近具有本质的不同?我们知道曲线曲面在一点处附近的逼近应该是它的切线或者切平面.对于曲线的情况,只要所作折线的各个弦足够小,则每个弦的方向和其对应弧上任一点处切线的方向相差足够小,于是这种无穷小的弦可以越来越精确的逼近对应的弧元素的长度.相反的,在曲面情况下尽管多面体的面越来越小,但是它可以不与对应曲面片上任一点的切平面方向不接近,在这种情况下多面体的面不能替代曲面片.施瓦兹的例子很好的体现了这种情况:圆柱面的切面都是直立的,而内接于柱面的小三角形总可以不接近这个方向.

于是,当我们用多面体内接曲面的时候,不仅需要多面体的面的直径最大值趋于0,还要求每个面的方向和对应曲面片上任一点的切平面的方向越来越接近.对此思路的严格的改进描述是:用分段光滑曲线网将曲面分割为若干部分$S_i$,对每个曲面片$S_i$,任取一个点做切平面,使得曲面片$S_i$在切平面上做垂直投影.当各曲面片的直径最大值趋于0的时候,这些垂直投影的面积和的极限称为曲面的面积.

经过一些推导(见微积分学教程第三卷),可得曲面面积和第一基本形式有关:设$s:U\subset\mathbb{R}^2\to\mathbb{R}^3$是曲面,设$R\subset S$是有界区域,那么$Q=s^{-1}(R)$是$U$的有界区域(注意$s$是同胚).那么$R$的面积为:$\iint_Q\sqrt{EG-F^2}dudv$.

高斯映射.曲面上点$s(p)$处切空间是$\mathbb{R}^3$中的平面,它的正交补,也即过点$s(p)$的垂直于切平面的直线称为点$s(p)$处的法空间或者法线.记$n=\frac{s_u\times s_v}{|s_u\times s_v|}$,它在每个点处的取值是该点处的一个单位法向量.另外$n$还可以视为曲面到$\mathbb{R}^3$上单位球面上的映射,此时成它为曲面的高斯映射.注意这里记号$n$同时表示曲面上点的单位法向量和高斯映射.

我们接下来要做的是对曲面弯曲程度进行刻画,或者说类似曲线的情况定义一种符合形势的曲面的曲率概念.为此我们从三个不同的几何角度给出刻画.首先是把曲面上一个点附近的弯曲程度用该点附近的点到该点的切平面的距离来刻画,由此会得出第二基本形式的概念;接下来会考虑曲面上的曲线,通过曲线的曲率给出曲面一个点沿某个切方向的弯曲程度的刻画,这会得出法曲率的概念;最后我们会把曲面的弯曲程度理解为高斯映射发生的变化(也即它的微分),【】

第二基本形式.给定曲面$S$,记参数表示为$s(u,v)$,取定一个点$s(u_0,v_0)$,记这个点的切平面为$\pi$.很自然的,点$s(u_0,v_0)$附近曲面的弯曲程度,可以用附近的点到切平面$\pi$的距离来刻画.记$s(u_0,v_0)$附近的点为$s(u_0+\Delta u,v_0+\Delta v)$,则该点到平面$\pi$的有向距离为:
$$\delta(\Delta u,\Delta v)=\left(s(u_0+\Delta u,v_0+\Delta v)-s(u_0,v_0)\right)\cdot n$$

将右侧的函数差做泰勒展开式,其中第一项为$s_u\mid_{(u_0,v_0)}\Delta u+s_v\mid_{(u_0,v_0)}\Delta v$,这是切平面中的向量,将它和单位法向量$n$做数量积就为0,于是当点足够接近$s(u_0,v_0)$的时候,该点和平面$\pi$的距离被泰勒展开的第二项和$n$的数量积所决定,也即:
$$\delta(\Delta u,\Delta v)=\frac{1}{2}\left(L\Delta u^2+2M\Delta u\Delta v+N\Delta v^2\right)+o(\Delta u^2+\Delta v^2)$$

其中$L=s_{uu}\mid_{(u_0,v_0)}\cdot n$;$M=s_{uv}\mid_{(u_0,v_0)}\cdot n$;$N=s_{vv}\mid_{(u_0,v_0)}\cdot n$.按照$s_u\cdot n=s_v\cdot n=0$,它们还可以表示为$L=-s_u\cdot n_u$;$M=-s_u\cdot n_v=-s_v\cdot n_u$;$N=-s_v\cdot n_v$.

首先要解决的依然是是否依赖参数表示选取的问题.第二基本形式不依赖于保定向的参数表示的选取.当选取不保定向的参数表示时,第二基本形式会整体变号.
\begin{proof}
	
	设光滑同胚为$u=u(s,t)$和$v=v(s,t)$.保定向的含义是$\frac{\partial(u,v)}{\partial(s,t)}>0$.另外我们记这个雅各比行列式对应的矩阵为$J$.注意到$s^1_s\times s^1_t=|J|s^2_u\times s^2_v$,于是两个参数表示下的单位法向量是相同的.再注意到$(du,dv)=(ds,dt)J$以及$\left(\begin{array}{cc} L'&M'\\ M'&N'\end{array}\right)=J\left(\begin{array}{cc} L&M\\ M&N\end{array}\right)J^T$.就得到:
	$$(du,dv)\left(\begin{array}{cc} L&M\\ M&N\end{array}\right)\left(\begin{array}{c} du\\ dv\end{array}\right)=(ds,dt)\left(\begin{array}{cc} L'&M'\\ M'&N'\end{array}\right)\left(\begin{array}{c} ds\\ dt\end{array}\right)$$
\end{proof}

第二基本形式和曲面形状.考虑函数$f(u,v)=\langle s(u,v)-s(u_0,v_0),n(u_0,v_0)\rangle$,它在点$(u_0,v_0)$的两个偏导数为0,而在该点的二阶导数方阵,也即黑塞矩阵,就是第二基本形式作为双线性型的矩阵.因此,如果$LN>M^2$,那么曲面在该点附近的形状是凸面或凹面,具体是哪个取决于法向量的方向;如果$LN<M^2$,那么曲面在该点附近的形状是马鞍面.

给定曲面$S$上的一个点$p$,我们的目的是理解点$p$附近曲面的弯曲程度.除了考虑$p$附近的点到$p$的切平面上的距离,还可以考虑$S$上的经过点$p$的特殊的曲线.如果我们仅取法向量所在平面束与曲面所截的过点$p$的曲线,则此时这些曲线都是平面曲线,并且此时曲线的曲率理应蕴含了曲面弯曲程度的信息.

设曲面$S$的参数表示是$r(u,v):U\to S$.取$(u_0,v_0)\in U$,取$U$中过该点的一条弧长参数曲线$(u(s),v(s))$.不妨约定$(u(0),v(0))=(u_0,v_0)$.于是$r(s)=r(u(s),v(s))$是曲面$S$上的过点$r_0=r(u_0,v_0)$的一条曲线.那么曲线$r(s)$在点$r_0$处的单位切向量为$r_u\frac{\mathrm{d}r}{\mathrm{d}s}\mid_{s=0}+r_v\frac{\mathrm{d}v}{\mathrm{d}s}\mid_{s=0}$.它在点$r_0$处的曲率向量为:
$$\frac{\mathrm{d}^2r}{\mathrm{d}s^2}\mid_{s=0}=\left(r_u\frac{\mathrm{d}^2u}{\mathrm{d}s^2}+r_v\frac{\mathrm{d}^2v}{\mathrm{d}s^2}+r_{uu}\left(\frac{\mathrm{d}u}{\mathrm{d}s}\right)^2+2r_{uv}\frac{\mathrm{d}u}{\mathrm{d}s}\frac{\mathrm{d}v}{\mathrm{d}s}+r_{vv}\left(\frac{\mathrm{d}v}{\mathrm{d}s}\right)^2\right)\mid_{s=0}$$

这告诉我们曲面上一般曲线的曲率向量不再是曲面的切向量.对此我们可以把曲率向量在曲面标架下进行分解.为了探究曲面沿一条曲线方向的弯曲程度,一个合理的做法是把曲率向量在曲面的法向量上做分解,也即$k_n=\langle \frac{\mathrm{d}^2r}{\mathrm{d}s^2},n\rangle$.整理一下它就是:
$$k_n=L\left(\frac{\mathrm{d}u}{\mathrm{d}s}\right)^2+2M\frac{\mathrm{d}u}{\mathrm{d}s}\frac{\mathrm{d}v}{\mathrm{d}s}+N\left(\frac{\mathrm{d}v}{\mathrm{d}s}\right)^2$$

这告诉我们,曲率向量在法向量的分解只和曲线$r(s)$在点$r_0$的切向量以及曲面自身的第二基本形式有关.我们称这里的$k_n$为曲面$S$在点$r_0$的关于单位切向量$\left(\frac{\mathrm{d}u}{\mathrm{d}s},\frac{\mathrm{d}v}{\mathrm{d}s}\right)\mid_{s=0}$的法曲率.换句话讲,曲面$S$的法曲率$k_n$在曲面上的每个点处是关于单位切向量的二次型.任取单位切向量为$v=ar_u+br_v$,称沿这条切向量的法曲率为$k_n(v)=La^2+2Mab+Nb^2$.法曲率度量的是曲面在一个点沿某个切方向的弯曲程度.

这里做一个技术性补充.我们所定义的法曲率是单位切向量上的二次型,它度量的是沿一个切向量方向时曲面的弯曲程度,实际上和这个切向量的长度无关,因此我们期望对一个非0的未必单位长度的切向量$v$,把$k_n(v)$定义为$k_n(\frac{v}{|v|})$.经过简单推导得到此时法曲率的定义为,对任意非0切向量$v=ar_u+br_v$,有$k_n(v)=\frac{La^2+2Mab+Nb^2}{Ea^2+2Fab+Gb^2}$.即第二基本形式和第一基本形式的比值.

总结一下,通过曲线刻画弯曲程度我们得到了第二基本形式更为本质的理解:在固定一个点$p(u,v)\in S$后,将第二基本形式视为关于$(du,dv)$的二次型,则对于$T_pS$中的单位切向量$(du,dv)$,有$S$沿这个切方向的弯曲程度,也就是法曲率,就是二次型在$(du,dv)$处的取值.另外,切方向$v$和$-v$的法曲率是相同的.

两个具体例子.
\begin{enumerate}
	\item 曲面是平面(的一部分)当且仅当它的第二基本形式恒为0.
	\begin{proof}
		
		我们知道曲面是平面等价于单位法向量是固定的.于是倘若曲面是平面,那么$n_u=n_v=\equiv0$,这说明第二基本形式中的$L,M,N$都恒为0.反过来如果第二基本形式恒为0,那么得到$s_u\cdots n_u=0$,$s_u\cdot n_v=s_v\cdot n_u=0$,$s_v\cdot,n_v=0$.又因为$n$是单位向量,对$(n,n)=1$求偏导得到$n_u\cdot n=n_v\cdot n=0$,这些等式说明了$n_u$和$n_v$在$s_u,s_v,n$上的投影都是0,但是它们三个向量构成了标架,于是两个偏导数恒为0,即$n$是固定的向量.
	\end{proof}
    \item 曲面是球面(的一部分)当且仅当它的法曲率在每个点的每个非0向量处恒为一个相同的非0数.这个数即半径的负倒数.
    \begin{proof}
    	
    	若曲面$S$落在一个球面上,记参数表示为$s(u,v)$,那么必存在一个向量$s_0$和一个正常数$R$满足$(s(u,v)-s_0,s(u,v)-s_0)=R^2$.做微分得到$\mathrm{d}s\cdot(s(u,v)-s_0)=0$,说明单位法向量恰好就是$n=\frac{1}{R}(s(u,v)-s_0)$.于是得到第二基本形式为$-\mathrm{d}s\cdot\mathrm{d}n=-\frac{1}{R}\mathrm{d}s\cdot\mathrm{d}s$.而这里的$\mathrm{d}s\cdot\mathrm{d}s$恰好就是第一基本形式.
    	
    	反过来,如果存在非0常数$c$使得第二基本形式和第一基本形式的比值恒为$c$,那么得到$(L-cE)\mathrm{d}u^2+2(M-cF)\mathrm{d}u\mathrm{d}v+(N-cG)\mathrm{d}v^2$.按照$du,dv$的任意性,这说明恒有$L=cE$,$M=cF$,$N=cG$.这些等式可推出$n_u+cs_u$与$u_v+cs_v$在$s_u,s_v,n$上的投影都是0,于是说明恒有$n_u+cs_u=n_v+cs_v=0$,这说明$d(n+cs)=0$,于是不妨设$n+cs=cs_0$,这说明了$(s(u,v)-s_0,s(u,v)-s_0)=\frac{1}{c^2}$,即$S$落在球心$s_0$,半径$\frac{1}{|c|}$的球面上.
    \end{proof}
\end{enumerate}

如果曲面在点$p$的一个切方向上的法曲率为0,这说明沿这个方向的曲面的弯曲程度可以视为直线,称这样的切方向为曲面在该点的渐进方向.那么当固定点$(u,v)$时,渐进方向恰好就是$Ldu^2+2Mdudv+Ndv^2=0$的解.于是一点处有实渐进方向的充要条件就是$LN-M^2\le 0$.

我们提及过,如果我们把单位法向量视为三维欧氏空间的单位球面上的点,那么单位法向量作为映射,它实际上把曲面映射到了单位球面.当曲面在一个点附近弯曲得很厉害的时候,高斯映射在该点附近发生的变化也就较大,这说明高斯映射在一个点附近发生的变化可以描述曲面的弯曲程度.而一个可微映射在一个点附近发生的变换自然可以用在该点处的微分描述.

给定$S$中的点$p$,记单位球面为$\Sigma$,记高斯映射在点$p$处的微分为$dn_p(-)$.先来求它的表达式.按照切空间的曲线定义,可设$(u(t),v(t))$是参数区域$U$内的一条曲线,那么$s(u(t),v(t))$是曲面上的曲线,高斯映射沿这条曲线求微分,得到的是$\frac{\mathrm{d}n(u(t),v(t))}{\mathrm{d}t}=n_u\frac{\mathrm{d}u}{\mathrm{d}t}+n_v\frac{\mathrm{d}v}{\mathrm{d}t}$.注意$\Sigma$上的每个点,视为向量都是曲面$\Sigma$的法向量,说明$S$在点$p$处的切平面和$\Sigma$在$n(p)$处的切平面是平行的.因此我们可以不妨把高斯映射的微分视为$T_pS$上的线性变换:$\left(\frac{du}{dt},\frac{dv}{dt}\right)\mapsto n_u\frac{du}{dt}+n_v\frac{dv}{dt}$.换句话讲,固定曲面上的点$(u,v)$,那么高斯映射在该点的微分就是$dn_p:v=as_u+bs_v\mapsto an_u+bn_v$.我们称$-dn_p$为曲面$S$在点$p$处的Weingarten映射,记作$W_p(-)$.

Weingarten映射的性质.
\begin{enumerate}
	\item Weingarten映射不依赖于曲面参数表示的选取.
	\item 对曲面$S$的任意单位切向量$v$,沿$v$方向的法曲率可表示为$k_n(v)=(W(v),v)$.事实上设$v=as_u+bs_v$,那么有$(W(v),v)=-(an_u+bn_v,as_u+bs_v)=a^2L+2abM+b^2N=k_n(v)$.
	\item Weingarten映射作为切平面上的线性变换是对称的线性变换(有的书叫自共轭的),也即$\forall u,w\in T_pS$,有$(W(v),w)=(v,W(w))$.
\end{enumerate}

Weingarten映射在曲面上每一点处是对称的线性变换,根据线性代数理论,它的特征多项式在实数域上分裂,此时会出现两种情况,要么有两个不相同的实特征值,要么具有一个代数重数(和几何重数)为2的实特征值.取单位特征向量$v$,取对应的特征值为$k$,那么有$(W(v),v)=(kv,v)=k$,说明Weingarten映射的特征值总是法曲率.称Weingarten映射在一点处的两个计重数意义下的特征值为曲面在该点的两个主曲率.线性代数理论告诉我们,由于线性变换是对称的,所以所属不同特征值的特征向量必然正交,这说明如果两个主曲率不同,那么两个主曲率对应的切方向是正交的,它们称为主方向.

主曲率是法曲率的最值,欧拉公式.无论Weingarten映射在曲面一点$p$处是有重特征值还是有两个不同特征值,我们都可以取两个特征值$k_1,k_2$对应的单位特征向量为$e_1,e_2$,并且可以约定它们正交.现在任取一个$T_pS$中的单位向量$v$,那么它可以表示为$v=\cos\theta e_1+\sin\theta e_2$.这得到$W(v)=k_1\cos\theta e_1+k_2\sin\theta e_2$,于是有$k_n(v)=(W(v),v)=k_1\cos^2\theta+k_2\sin^2\theta$,这称为欧拉公式.这个公式说明主曲率恰好就是法曲率的两个最值,主方向恰好就是取最值的方向.当Weingarten映射具有两个不同特征值时,此时恰有一个方向取法曲率最大值,也恰有一个方向取法曲率最小值,并且两个方向相互垂直;当Weingarten映射具有两个相同的特征值时,此时每一个切方向上的法曲率都是相同的.这等价于讲该点处的第二基本形式和第一基本形式的比值是定值.

下面我们来求主曲率.为此先来探究Weingarten映射在参数表示下的矩阵表示,换句话说在基$\{s_u,s_v\}$下的矩阵表示.为此设$W(s_u,s_v)=(s_u,s_v)\left(\begin{array}{cc} a&b\\ c&d\end{array}\right)$.将两个等式分别对$s_u,s_v$做内积,得到:
$$\left(\begin{array}{cc} a&b\\ c&d\end{array}\right)=\left(\begin{array}{cc} E&F\\ F&G\end{array}\right)^{-1}\left(\begin{array}{cc} L&M\\ M&N\end{array}\right)=\frac{1}{EG-F^2}\left(\begin{array}{cc} LG-MF&ME-LF\\ MG-NF&NE-MF\end{array}\right)$$

主曲率是这个矩阵的特征值,所以它们是下述方程的两个根:
$$k^2-\frac{LG-2MF+NE}{EG-F^2}k+\frac{LN-M^2}{EG-F^2}=0$$

称Weingarten映射迹的一半为平均曲率,称Weingarten映射的行列式为高斯曲率,那么平均曲率的公式为$H=\frac{LG-2MF+NE}{2(EG-F^2)}$,高斯曲率的公式为$K=\frac{LN-M^2}{EG-F^2}$.

\section{曲面的标架理论}
\section{曲面的内蕴理论}


