\chapter{簇论}
\section{簇}
\subsection{仿射簇}
\begin{enumerate}
	\item 仿射空间与多项式映射.任取域$k$,把$k^n$称为域$k$上的$n$维仿射空间,记作$\mathbb{A}^n_k$,或者在不产生歧义的前提下记作$\mathbb{A}^n$.其中$\mathbb{A}_k^1$称为仿射直线,$\mathbb{A}_k^2$称为仿射平面.任取$n$元多项式$f\in k[x_1,x_2,\cdots,x_n]$,确定了一个从$k^n$到$k$的映射,称为多项式映射.注意在有限域上不同多项式可能诱导出相同的多项式映射,例如$\mathbb{F}_p$中的多项式$x^p-x$和零多项式作为映射是相同的.不过对于无限域(特别的对于代数闭域)的情况这不会发生.把$k^n$称为域$k$上的$n$维仿射空间,记作$\mathbb{A}_k^n$,于是一个$k$上的$n$多项式诱导的映射可以理解为$k$上$n$维仿射空间上的函数.
	\item 仿射代数集.给定$R=k[x_1,x_2,\cdots,x_n]$的子集$\alpha$,那么$\alpha$中所有多项式的公共零点构成了$\mathbb{A}^n$的子集,记作$Z(\alpha)$.称$\mathbb{A}^n$的能表示为$Z(\alpha)$形式的子集为仿射代数集(affine algebraic sets).仿射代数集的一些基本性质:
	\begin{enumerate}
		\item 如果多项式环的子集满足$\alpha_1\subset\alpha_2$,那么有$Z(\alpha_2)\subseteq Z(\alpha_1)$.
		\item 如果子集$\alpha'$在多项式环中生成的理想为$\alpha$,那么$Z(\alpha')=Z(\alpha)=Z(\sqrt{\alpha})$.于是仿射代数集总可以表示为$Z(\alpha)$,其中$\alpha$是多项式环的理想,也可以要求$\alpha$是根理想.
		\item 按照希尔伯特基定理,域$k$上的$n$元多项式环是一个诺特环,说明理想$\alpha$总是有限生成的,换句话说,探究公共零点时总是只涉及到有限个多项式.
	\end{enumerate}
    \item 仿射代数集上的拓扑.全体仿射代数集满足闭集公理,这个拓扑称为Zariski拓扑.
    \begin{enumerate}
    	\item 先验证满足闭集公理.
    	\begin{proof}
    		
    		首先验证全集和空集:$Z((1))=\emptyset$,$Z((0))=\mathbb{A}^n$.
    		
    		\qquad
    		
    		验证有限并:设$\alpha_1,\alpha_2$是$R$的两个理想,那么有$Z(\alpha_1)\cup Z(\alpha_2)=Z(\alpha_1\alpha_2)$.一方面从$\alpha_1\alpha_2\subset\alpha_1\cap\alpha_2$得到$Z(\alpha_1)\cup Z(\alpha_2)\subseteq Z(\alpha_1\alpha_2)$;另一方面假设仿射空间中的点$x\not\in Z(\alpha_1)\cup Z(\alpha_2)$,那么存在$f\in\alpha_1$和$g\in\alpha_2$均不以$x$为零点,于是$fg\in\alpha_1\alpha_2$也不以$x$为零点.
    		
    		\qquad
    		
    		验证任意交:设$\{\alpha_i,i\in I\}$是$R$中的一族理想,那么有$\cap_iZ(\alpha_i)=Z(\sum_i\alpha_i)$.一方面从$\alpha_i\subset\sum_i\alpha_i,\forall i\in I$得到$Z(\sum_i\alpha_i)\subset\cap_iZ(\alpha_i)$;另一方面任取$x\in\cap_iZ(\alpha_i)$,那么$x$自然是$\sum_i\alpha_i$中每个多项式的零点.
    	\end{proof}
    	\item 这个拓扑适用于任意一般域上的仿射空间.在$\mathbb{C}$和$\mathbb{R}$的情况下这个拓扑是粗糙于标准拓扑的,因为多项式映射在标准拓扑下是连续的,于是代数集在标准拓扑下总是闭集.
    	\item 拓扑基.任取单个多项式$f\in R=k[x_1,x_2,\cdots,x_n]$,记开集$D(f)=\mathbb{A}^n-Z((f))$.我们来验证这构成了Zariski拓扑的拓扑基,称它们为主开集.
    	\begin{proof}
    		
    		先验证每个开集可以表示为主开集的并:$\mathbb{A}^n-Z(I)=\mathbb{A}^n-\cap_{f\in I}Z((f))=\cup_{f\in I}D(f)$.再验证包含某个固定点$x$的全体主开集是$x$的局部基:$x\in D(f)\cap D(g)=\mathbb{A}^n-Z((f))\cup Z((g))=\mathbb{A}^n-Z((fg))=D(fg)$.
    	\end{proof}
    	\item 一维情况.按照域上有限个单元多项式的根集是有限集,说明$\mathbb{A}^1$上的Zariski拓扑是余有限拓扑.另外$\mathbb{A}^n$上的直线$\{at+b\mid t\in k\}$的Zariski拓扑都是余有限拓扑.事实上任取仿射代数集$Z(f_1,f_2,\cdots,f_n)=\cap_iZ(f_i)$,注意到$Z(f_i)\cap\{at+b\}$的点的个数就是$Z(f_i(at+b))$的点的个数,这总是有限的.
    	\item 分离公理.余有限拓扑一般不是Hausdorff空间,于是Zariski拓扑一般没有Hausdorff性,不过Zariski拓扑总是$T_1$空间,这是因为单点$(a_1,a_2,\cdots,a_n)$总是$R$的极大理想$(x_1-a_1,x_2-a_2,\cdots,x_n-a_n)$对应的代数集.
    	\item 有限域上仿射空间的Zariski拓扑总是离散拓扑.上一条说明仿射空间总是$T_1$空间,但是有限域的情况下仿射空间总是有限点集,于是此时拓扑是离散的.
    	\item 拟紧致性(在代数几何中通常把开覆盖总有有限子覆盖称为拟紧致性,而把拟紧致的Hausdorff空间称为紧致的).Zariski拓扑总是诺特空间,这个条件是指闭集降链总会终止.另外空间诺特等价于全部子空间均是拟紧致的,于是仿射空间的任意子空间都是拟紧致的.
    	\begin{proof}
    		
    		假设存在闭集的降链$Y_1\supset Y_2\supset\cdots\supset$,这得到了$R=k[x_1,x_2,\cdots,x_n]$中的理想升链$I(Y_1)\subseteq I(Y_2)\subset\cdots$,按照$R$是诺特环说明这个理想链是终止的,再结合$Y_i=Z(I(Y_i))$说明闭集降链终止.注意这个证明并没有用到希尔伯特零点定理.
    	\end{proof}
    \end{enumerate}
    \item 仿射代数簇.仿射空间的不可约闭子集称为仿射代数簇(affine algebraic varieties).这里不可约性是指不能表示为更小的两个闭集的并.下面给出不可约性和代数簇的一些基本性质:
    \begin{enumerate}
    	\item 一个空间是不可约的,当且仅当任意两个非空开集的交非空,当且仅当任意非空开集都是稠密集,当且仅当任意真闭子集都不包含内点.
    	\item 不可约空间的开子集总是不可约的,不可约子空间的闭包总是不可约的.
    	\item 下文会证明一个代数集是不可约的当且仅当它对应的根理想是素理想.
    	\item 诺特空间$X$的每个非空闭子集$Y$可以分解为有限个不可约闭子集的并$Y=Y_1\cup Y_2\cup\cdots\cup Y_r$.如果我们约定这里的$Y_i$两两之间没有包含关系,此时分解是唯一的,此时这些$Y_i$称为$Y$的不可约分支.它们实际上就是$Y$的不可约子空间在包含序下的极大元.另外注意到和连通分支或者道路分支不同,$Y$的不同不可约分支可能是有交的.
    	\item 于是上一条说明每个仿射代数集可以唯一的表示为有限个仿射代数簇的并,满足任意两个仿射代数簇之间没有包含关系.
    \end{enumerate}
    \item 关于积.给定两个仿射代数集$X\subseteq\mathbb{A}^n$和$Y\subseteq\mathbb{A}^m$,设$X$被多项式族$\{f_i(T_1,\cdots,T_n)\}$定义,$Y$被多项式族$\{g_j(S_1,\cdots,S_m)\}$定义,我们约定$X\times Y$是$\mathbb{A}^{n+m}$的闭子集,被$\{F_i(T_1,\cdots,T_n,S_1,\cdots,S_m)=f_i(T_1,\cdots,T_n),G_j(T_1,\cdots,T_n,S_1,\cdots,S_m)=g_j(S_1,\cdots,S_m)\}$定义.于是在这个定义下,我们有$\mathbb{A}^n\times\mathbb{A}^m=\mathbb{A}^{n+m}$.另外我们在后文会证明这的确是范畴意义下的积.
    \begin{enumerate}
    	\item 仿射代数集上的积的拓扑并不是积拓扑,例如我们来证明$\mathbb{A}^2$上的Zariski拓扑并不是两个$\mathbb{A}_k^1$的积拓扑(实际上是纤维积):首先多项式$x_1-x_2$的零点集就是$\mathbb{A}^2$上的对角线,于是它是闭集;而点集拓扑理论告诉我们如果这的确是积拓扑,则对角线是闭集等价于$\mathbb{A}^1$是Hausdorff空间,但是当$k$是无限域时$\mathbb{A}_k^1$作为余有限拓扑空间不会是Hausdorff空间.
    	\item 如果$X,Y$都是仿射代数簇,那么$X\times Y$也是仿射代数簇.
    	\begin{proof}
    		
    		假设$X\times Y$是可约的,那么它存在真闭子集$Z_1,Z_2$,满足并集是整个$X\times Y$.那么对任意$x\in X$,有$X\times Y$的闭子集$\{x\}\times Y$同构于$Y$,从而也是不可约的,于是从$\left(\{x\}\times Y\right)\cap Z_1$和$\left(\{x\}\times Y\right)\cap Z_2$都是$\{x\}\times Y$的闭子集,并且并是整个$\{x\}\times Y$,就说明要么$\{x\}\times Y\subseteq Z_1$,要么$\{x\}\times Y\subseteq Z_2$.我们用$X_1$表示那些满足前一个式子的$x\in X$构成的子集,用$X_2$表示那些满足后一个式子的$x\in X$构成的子集.下面说明$X_1$是闭集,为此只需说明$X_y=\{x\in X\mid(x,y)\in Z_1\}$是闭集,就有$X_1=\cap_{y\in Y}X_y$是闭集.但是因为$\left(X\times\{y\}\right)\cap Z_1=X_y\times y$得到$X_y\times\{y\}$是$X\times\{y\}$的闭集,在同构下就得到$X_y$是$X$的闭集.同理我们得到$X_2$也是闭集,于是按照$X$是不可约的得到$X_1=X$或者$X_2=X$,前者得到$X\times Y=Z_1$,后者得到$X\times Y=Z_2$,于是$X\times Y$是不可约的.
    	\end{proof}
    \end{enumerate}
    \item 关于映射$I(-)$.给定代数闭域上的仿射空间$\mathbb{A}^n$的子集$Y$,定义它对应的理想为$I(Y)=\{P\in k[x_1,x_2,\cdots,x_n]\mid P(x)=0,\forall x\in Y\}$.这里整理下两个映射$Z(-)$和$I(-)$所满足的性质:
    \begin{enumerate}
    	\item (一次复合是反序的)它们都是反序的映射:如果$\alpha_1\subseteq \alpha_2$是$R=k[x_1,x_2,\cdots,x_n]$中的两个子集,那么有$Z(\alpha_2)\subseteq Z(\alpha_1)$;如果$Y_1\subseteq Y_2$是$\mathbb{A}^n$中的两个子集,那么有$I(Y_2)\subseteq I(Y_1)$.
    	\item 对$R$中的任意理想$\alpha_1,\alpha_2$,有$Z(\alpha_1\alpha_2)=Z(\alpha_1)\cup Z(\alpha_2)$;对仿射空间的任意子集$Y_1,Y_2$,有$I(Y_1\cup Y_2)=I(Y_1)\cap I(Y_2)$.
    	\item $Z(\alpha)$总是$\mathbb{A}^n$中的闭集(事实上,这只是定义);$I(Y)$总是$R=k[x_1,x_2,\cdots,x_n]$的根理想.
    	\item (二次复合使点集变大)对每个子集$\alpha\subseteq R$,总有$\alpha\subseteq I(Z(\alpha))$;对每个子集$Y\subset\mathbb{A}^n$,总有$Y\subseteq Z(I(Y))$.
    	\item (一次复合与三次复合相同)第一条和第四条就得到:对每个子集$\alpha\subseteq R$总有$Z(\alpha)=Z(I(Z(\alpha)))$;对每个子集$Y\subset\mathbb{A}^n$总有$I(Y)=I(Z(I(Y)))$.
    	\item (二次复合的具体表达式)对每个理想$\alpha\subseteq R$,总有$I(Z(\alpha))=\sqrt{\alpha}$(此为希尔伯特零点定理:设$k$是代数闭域,设$\alpha$是$R=k[x_1,x_2,\cdots,x_n]$中的理想,取$f\in R$是一个$n$元多项式,满足它以$Z(\alpha)$中每个点为零点,那么存在某个次幂$f^r\in\alpha$);对每个子集$Y\subset\mathbb{A}^n$,有$Z(I(Y))=\overline{Y}$.
    	\begin{proof}
    		
    		一方面按照$Y\subseteq Z(I(Y))$以及后者是闭集,得到$\overline{Y}\subseteq Z(I(Y))$;另一方面任取包含$Y$的闭集$Z(\alpha)$,得到$Z(I(Y))\subseteq Z(I(Z(\alpha)))=Z(\alpha)$.
    	\end{proof}
    \end{enumerate}
    \item 对应定理.记$k$是一个代数闭域,$R=k[x_1,x_2,\cdots,x_n]$.
    \begin{enumerate}
    	\item $Z(-)$和$I(-)$是$\mathbb{A}^n$中的代数集和$R$中的根理想之间的反序一一对应.事实上,按照上面给出的这两个映射的二次复合的具体表达式,直接得到它们限制在这两种子集上是反序的一一对应.
    	\item 在上述对应下,代数簇和素理想之间是反序一一对应的.特别的,按照零理想是多项环的素理想,就得到每个$\mathbb{A}^n$都是不可约空间.
    	\begin{proof}
    		
    		一方面,如果$P$是素理想,需要说明$Z(P)$是一个代数簇.若否则有$Z(P)=A\cup B$,其中$A,B$是$Z(P)$内的两个更小的代数集.按照零点定理,以及素理想是根理想,得到$P=I(Z(P))=I(A\cup B)=I(A)\cap I(B)$.由prime aviodance lemma,得到$P$就是$I(A)$,$I(B)$中的一个,不妨设$P=I(A)$,于是$A=Z(P)$,这和$A$严格包含于$Z(P)$矛盾.
    		
    		另一方面,如果$X$是代数簇,需要证明$I(X)$是素理想.若否存在两个多元多项式$f,g\not\in I(X)$,但是$fg\in I(X)$.于是存在$X$中的点不是$f$的零点,也存在$X$中的点不是$g$的零点.于是$Z(f)\cap X$和$Z(g)\cap X$是$X$中的两个严格更小的代数集.我们断言它们的并是$X$,这就会和$X$是代数簇矛盾:$(Z(f)\cap X)\cup(Z(g)\cap X)=Z(fg)\cup X=X$.
    	\end{proof}
    	\item 在上述对应下,单点和极大理想之间是一一对应的.记$x=(a_1,a_2,\cdots,a_n)\in\mathbb{A}^n$,那么有$I(\{x\})=(x_1-a_1,x_2-a_2,\cdots,x_n-a_n)$(首先必然有$(x_1-a_1,x_2-a_2,\cdots,x_n-a_n)\subseteq I(\{x\})$,但是左侧已经是极大理想,于是二者相同).反过来按照弱希尔伯特零点定理,多项式的极大理想均具有这样的形式.
    	\item 我们之前证明过代数集总可以唯一的分解为不可约分支的并.按照第二条的对应定理,不可约分支实际上就对应于坐标环的极小素理想.
    	\item 给定仿射代数集$X\subset\mathbb{A}^n$,称$k[X_1,X_2,\cdots,X_n]/I(X)$为它的仿射坐标环.利用坐标环,对应定理也可以描述为:
    	\begin{itemize}
    		\item 代数集$Y$上的闭子集和坐标环$\mathbb{A}(Y)$中的根理想一一对应.
    		\item 代数集$Y$上的不可约闭子集和坐标环$\mathbb{A}(Y)$中的素理想一一对应.
    		\item 代数集$Y$上的单点和坐标环$A(Y)$的极大理想一一对应.即$x=(a_1,a_2,\cdots,a_n)\in Y$对应于$\mathbb{A}(Y)$中的极大理想$(x_1-a_1,x_2-a_2,\cdots,x_n-a_n)/I(Y)$.
    	\end{itemize}
    \end{enumerate}
\end{enumerate}
\subsection{射影簇}
\begin{enumerate}
	\item 射影空间.在$\mathbb{A}^{n+1}-\{0\}$上定义一个等价关系,两个点$x,y$等价当且仅当存在一个$\lambda\in k$使得$x=\lambda y$.全体等价类构成的集合记作$\mathbb{P}_k^n$或者$\mathbb{P}^n$,称为$n$维射影空间.换句话讲它是$\mathbb{A}_k^{n+1}$上全体一维子空间构成的集合.每个一维子空间被其上任一非零点所决定.点$(x_0,x_1,\cdots,x_n)\in k^{n+1}-\{0\}$所在的一维子空间记作$(x_0:x_1:\cdots:x_n)$,它称为射影空间上的齐次坐标.
	\item 齐次多项式.考虑多项式环$S=k[x_0,x_1,\cdots,x_n]$,它经次数分次构成了一个分次环.其中全体$d\ge0$次齐次多项式构成的子群记作$S_d$.齐次多项式的零点集满足,它如果包含某个非零点,则包含了该点所在的一维子空间中的全部非零元,于是齐次多项式的零点集可视为射影空间中的子集.
	\item 关于$Z_+(-)$.给定$k[x_0,x_1,\cdots,x_n]$中的齐次理想$\alpha$,定义$Z_+(\alpha)$为$\alpha$中全部齐次多项式在射影空间$\mathbb{P}^n$中的公共零点集.称$\mathbb{P}^n$的子集$Y$是射影代数集(projective algebraic sets),如果存在某个齐次理想$I$满足$Y=Z_+(I)$.称不可约的射影代数集为射影代数簇(projective algebraic varieties).
	\begin{enumerate}
		\item 按照交换代数的内容,诺特分次环的齐次理想$\alpha$可以表示为有限个齐次元生成的,于是尽管齐次理想$\alpha$中经常含有无穷个齐次元,但当我们考虑公共零点集$Z_+(\alpha)$的时候仅需考虑有限个齐次多项式的公共零点.按照这个性质我们通常也会约定$Z_+(-)$定义在由齐次多项式构成的有限集合上的映射,但是本质含义是不变的.
		\item 如果齐次理想$\alpha_1\subset\alpha_2$,那么有$Z_+(\alpha_2)\subseteq Z_+(\alpha_1)$.
		\item 射影代数集满足闭集公理,这个拓扑称为射影空间上的Zariski拓扑.这个证明和仿射情况是一样的.
		\item 不可约分支.我们马上看到射影空间总是诺特的,于是按照上节不可约性中的一个定理,得到射影代数集总可以唯一的表示为有限个射影簇的并,满足任意两个射影簇之间没有包含关系.
	\end{enumerate}
    \item 射影空间上Zariski拓扑的一些性质:
    \begin{enumerate}
    	\item 拓扑基.对任意齐次多项式$f$,记$D(f)-\mathbb{P}^n-Z_+((f))$,全体$D(f)$构成了Zariski拓扑的拓扑基,称它们为主开集.这个证明和仿射情况一样.
    	\item 一维情况.一个二元的齐次多项式可以在换元下变为一元多项式,此时它的零点集是$\mathbb{P}^1$中的有限点集,于是$\mathbb{P}^1$上的拓扑就是余有限拓扑.
    	\item 分离公理.余有限拓扑一般不是Hausdorff的,于是Zariski拓扑一般不是$T_2$的,但是它同样总是$T_1$的:给定单点$(a_0:a_1:\cdots:a_n)$,不妨设这个表示中$a_t$非零,于是$x_ia_t-x_ta_i,\forall i$生成的齐次理想对应 零点集恰好是该单点集.
    	\item 诺特性.和仿射情况一样,射影空间总是诺特的.这导致射影空间的全部子空间总是拟紧致的.这个证明和仿射情况一样.
    \end{enumerate}
    \item 射影版本的希尔伯特零点定理.给定齐次理想$\alpha\subseteq S=k[x_0,x_1,\cdots,x_n]$,其中$k$是代数闭域,如果$f$是一个次数大于零的齐次多项式,并且以$Z_+(\alpha)$中每个点为零点,那么存在某个次幂$f^r\in\alpha$.
    \begin{proof}
    	
    	按照$C(Z_+(\alpha))=Z(\alpha)$,从$f$以$Z_+(\alpha)$中每个点为零点得到$f$以$Z(\alpha)\subset\mathbb{A}^{n+1}$中每个点为零点,于是按照仿射版本的零点定理,得到存在某个次幂$f^r\in\alpha$.
    \end{proof}
    \item 关于$I(-)$.我们考虑代数闭域上的射影空间.给定$\mathbb{P}^n$的子集$Y$,定义$I(Y)$为以$Y$为零点集的全部$S=k[x_0,x_1,\cdots,x_n]$中的齐次多项式生成的理想(按照$S$是诺特的,这个理想也可以被有限个齐次多项式生成).下面给出$I(-)$和$Z_+(-)$的一些性质:
    \begin{enumerate}
    	\item (一次复合是反序的)它们都是反序的映射:如果$\alpha_1\subseteq \alpha_2$是$S=k[x_0,x_1,\cdots,x_n]$中的两个齐次理想,那么有$Z_+(\alpha_2)\subseteq Z_+(\alpha_1)$;如果$Y_1\subseteq Y_2$是$\mathbb{P}^n$中的两个子集,那么有$I(Y_2)\subseteq I(Y_1)$.
    	\item 对$S$中的任意理想$\alpha_1,\alpha_2$,有$Z_+(\alpha_1\alpha_2)=Z_+(\alpha_1)\cup Z_+(\alpha_2)$;对射影空间的任意子集$Y_1,Y_2$,有$I(Y_1\cup Y_2)=I(Y_1)\cap I(Y_2)$.
    	\item $Z_+(\alpha)$总是$\mathbb{P}^n$中的闭集(这只是定义);$I(Y)$总是$S=k[x_0,x_1,\cdots,x_n]$的齐次根理想.
    	\item (二次复合使点集变大)对每个齐次理想$\alpha\subseteq S$,总有$\alpha\subseteq I(Z_+(\alpha))$;对每个子集$Y\subset\mathbb{P}^n$,总有$Y\subseteq Z_+(I(Y))$.
    	\item (一次复合与三次复合相同)第一条和第四条就得到:对每个子集$\alpha\subseteq S$总有$Z_+(\alpha)=Z_+(I(Z_+(\alpha)))$;对每个子集$Y\subset\mathbb{P}^n$总有$I(Y)=I(Z_+(I(Y)))$.
    	\item (二次复合的具体表达式)对每个理想$\alpha\subseteq S$,总有$I(Z_+(\alpha))=\sqrt{\alpha}$(此为射影版本的希尔伯特零点定理,需要代数闭域条件);对每个子集$Y\subset\mathbb{P}^n$,有$Z_+(I(Y))=\overline{Y}$.最后这条的证明和仿射情况一样.
    \end{enumerate}
    \item 仿射锥.定义$\theta:\mathbb{A}^{n+1}-\{0\}\to\mathbb{P}^n$为,把点映射为它在$\mathbb{P}^n$中的等价类.对射影代数集$Y$,它的仿射锥定义为$C(Y)=\theta^{-1}(Y)\cup\{0\}\subset\mathbb{A}^{n+1}$.
    \begin{enumerate}
    	\item 对齐次理想$\alpha$,总有$Z_+(\alpha)$的仿射锥就是$Z(\alpha)$,即$C(Z_+(\alpha))=Z(\alpha)$.特别的,这一条说明了仿射锥总是仿射集.
    	\item 射影集$Y$和它仿射锥对应的理想相同,即$I(Y)=I(C(Y))$.
    	\begin{proof}
    		
    		首先必然有$I(Y)\subseteq I(C(Y))$.现在任取$f\in I(C(Y))$,即$C(Y)$中每个点都是$f$的零点.做齐次分解$f=f_0+f_1+\cdots+f_t$,那么要证$f\in I(Y)$等价于证明每个$f_i$都是以$Y\subset\mathbb{P}^n$为零点集的齐次多项式.
    		
    		首先$0\in C(Y)$得到$f_0=0$.现在任取$x_0\in C(Y)$,任取$\lambda\in k$,得到$\lambda x_0\in C(Y)$,于是总有$\lambda f_1(x_0)+\lambda^2f_2(x_0)+\cdots+\lambda^tf_t(x_0)=0$.按照$k$是无限域,得到$f_i(x_0)=0,\forall i,x_0\in C(Y)$成立,于是$f_i\in I(Y)$,于是$f=\sum_if_i\in I(Y)$.
    	\end{proof}
    	\item 射影簇的仿射锥是仿射簇.在给出对应定理后就得到$Y$是射影簇等价于$I(Y)$是素理想,于是上一条得到$I(C(Y))$是素理想,于是得到$C(Y)$是仿射簇.
    \end{enumerate}
    \item 对应定理.设$k$是代数闭域.
    \begin{enumerate}
    	\item 关于无关理想的一个引理.设$\alpha\subseteq S=k[x_0,x_1,\cdots,x_n]$是齐次理想,那么如下三个条件等价:
    	\begin{enumerate}
    		\item $Z_+(\alpha)=\emptyset$.
    		\item $\sqrt{\alpha}$要么是$S$要么是$S_+=\oplus_{d>0}S_d=(x_0,x_1,\cdots,x_n)$.
    		\item $\alpha$包含了某个$S_d,d>0$.
    	\end{enumerate}
    	\begin{proof}
    		
    		1推2,如果$Z_+(\alpha)=\emptyset$,那么它的仿射锥$Z(\alpha)\subset\mathbb{A}^{n+1}$要么是空集要么是单点集$\{0\}$.这导致齐次理想$m=(x_0,x_1,\cdots,x_n)=I(\{0\})\subseteq I(Z(a))=\sqrt{\alpha}$.但是$m$本身是极大理想,这说明$\sqrt{\alpha}$要么是$m$要么是单位理想.
    		
    		2推3,若$\sqrt{\alpha}$是单位理想,那么$\alpha$也是单位理想,于是它自然包含全部的$S_d,d>0$.若$\sqrt{\alpha}=m$,则存在足够大的正整数$s$使得$x_i^s\in\alpha,\forall 0\le i\le n$.于是取$d=(n+1)s$就有$S_d\subset\alpha$.
    		
    		3推1,如果$\alpha$包含某个$S_d,d>0$,那么$Z(\alpha)\subset\{0\}$,于是$Z_+(\alpha)=\emptyset$.
    	\end{proof}
    	\item $I(-)$和$Z_+(-)$是$\mathbb{P}^n$中代数集和$S=k[x_0,x_1,\cdots,x_n]$中不是$S_+=\oplus_{d>0}S_d=(x_0,x_1,\cdots,x_n)$的齐次根理想之间的反序一一对应.按照这个性质,我们称$S_+$为$S$的无关理想.
    	\item 在上述对应下,射影代数簇和不为无关理想的齐次素理想之间是反序一一对应的,特别的这说明射影空间$\mathbb{P}^n$总是不可约空间.这个证明和仿射情况一样,注意到一个齐次理想$\alpha$是素理想当且仅当,对齐次元$f,g$,从$fg\in\alpha$推出$f$和$g$中必有一个$\in\alpha$.
    \end{enumerate}
    \item 射影代数集是一个预簇(从而每个点附近都有仿射代数集开邻域),射影代数簇是一个不可约预簇(从而每个点附近都有仿射簇开邻域).
    \begin{enumerate}
    	\item 对每个正整数$0\le i\le n$,记$H_i=Z_+(x_i)=\{(x_0:x_1:\cdots:x_n)\mid x_i=0\}\subset\mathbb{P}^n$,记开集$U_i=H_i^c$.它是射影空间中那些齐次坐标中第$i$分量不为零的点构成的集合.按照齐次坐标中至少有一个分量不取零,得到$\cup_iU_i=\mathbb{P}^n$.
    	\item 构造映射$\varphi_i:U_i\to\mathbb{A}^n$为$(x_0:x_1:\cdots:x_n)\mapsto(\frac{x_0}{x_i},\cdots,\frac{x_{i-1}}{x_i},\frac{x_{i+1}}{x_i},\cdots,\frac{x_n}{x_i})$,这个定义是良性的,取值不依赖于齐次坐标的具体选取,它自然是一个双射,下面验证它实际上是同胚:
    	\begin{proof}
    		
    		不妨只证明$i=0$的情况,为方便起见略去$\varphi_0$和$U_0$的角标.我们来证明$\varphi$和$\varphi^{-1}$都是闭映射.对$n+1$元齐次多项式$f$,定义$\alpha(f)=f(1,y_1,y_2,\cdots,y_n)$是一个$n$元未必齐次的多项式.对$n$元多项式$g$,定义$\beta(g)=x_0^{\deg g}g(\frac{x_1}{x_0},\frac{x_2}{x_0},\cdots,\frac{x_n}{x_0})$是一个$n+1$元齐次多项式.我们要证明的便是,在$\varphi$和$\varphi^{-1}$的作用下,$U$中的代数集和$\mathbb{A}_k^n$中的代数集,映射为的新的集合,就是相应理想经$\alpha,\beta$映射下的理想生成的代数集.
    		
    		一方面任取$U$的闭子集$Y$,设它在$\mathbb{P}_k^n$中的闭包为$\overline{Y}$,于是有$\overline{Y}=V(b)$,其中$b$是一组$n+1$元齐次多项式生成的齐次理想.那么$\alpha(b)$是一组$n$元多项式,设它们生成的$k[y_1,y_2,\cdots,y_n]$中的理想为$a$,需要验证的是$\varphi(Y)=V(a)$:一方面对$Y$中的点$x=(x_0:x_1:\cdots:x_n)$,有$\varphi(x)=(\frac{x_1}{x_0},\cdots,\frac{x_n}{x_0})$,它是生成$a$的那些多项式的零点;另一方面设$x=(x_1,x_2,\cdots,x_n)\in Z(a)$,那么$(1:x_1:x_2:\cdots:x_n)\in Y$满足在$\varphi$下的像是$(x_1,x_2,\cdots,x_n)$.
    		
    		另一方面任取$\mathbb{A}_k^n$中的闭子集$V=V(a)$,那么$\beta(a)$是一组$n+1$元齐次多项式,设它生成的$k[x_0,x_1,\cdots,x_n]$中的(齐次)理想是$b$,需要验证的是$\varphi^{-1}(V)=V_+(b)\cap U$.这个类似上一段最后部分的证明.
    	\end{proof}
    	\item 特别的,上一条中的同胚实际上是一个同构.它把$U_0$中开集上的局部正则函数$g(x_0,x_1,\cdots,x_n)/h(x_0,x_1,\cdots,x_n)$,其中$g$和$h$是次数相同的齐次多项式,对应为仿射情况的局部正则函数$g(1,y_1,\cdots,y_n)/h(1,y_1,\cdots,y_n)$.
    	\item 设$X$是射影代数簇,记$X_i=X\cap U_i$,它是$U_i\cong\mathbb{A}^n$中的闭子集:如果记$X$由齐次多项式$F_1,F_2,\cdots,F_m$定义的代数集,其中$F_i(X_0,X_1,\cdots,X_n)$是$n_i$次齐次多项式,那么$X_0$是被$X_0^{-n_i}F_i=F_i(1,T_1,T_2,\cdots,T_n)=0,1\le i\le m$定义的仿射代数集.这里把代数集改为代数簇也是成立的因为不可约空间的开子集仍然是不可约的.于是由于每个$X_i$都是仿射簇,得到$X$是预簇.另外这里$X_i$作为射影簇的开子集的截面环就是它作为仿射代数簇的坐标环$\mathbb{A}[X_i]\cong S(Y)_{(x_i)}$,这是局部化$S(Y)_{x_i}$自然的赋予分次结构后的零次子环.
    \end{enumerate}
    \item 射影闭包.设$X\subset\mathbb{A}^n$为仿射代数集,我们把$\mathbb{A}^n$同构于$\mathbb{P}^n$的开子集$U_0=\{[x_0:x_1:\cdots:x_n]\mid x_0\not=0\}$.考虑$X$在$\mathbb{P}^n$中的闭包$\overline{X}$,它称为$X$的射影闭包.$\overline{X}$被$n+1$元齐次多项式族$S_0^{n}F(S_1/S_0,\cdots,S_n/S_0),1\le i\le m$所定义,其中$F\in I(X),\deg F=n$.于是仿射代数集的投射闭包是一个射影代数集.另外按照不可约子集的闭包仍是不可约的,说明仿射簇的闭包是射影簇.
\end{enumerate}
\subsection{正则函数和映射}

仿射代数集上的正则函数.设$X$是$k$上仿射代数集,映射$f:X\to k$称为正则函数,如果存在多项式$p(T_1,T_2,\cdots,T_n)\in k[T_1,T_2,\cdots,T_n]$使得$f(x)=p(x),\forall x\in X$.两个多项式表示同一个正则函数当且仅当它们的差落在$I(X)$中.于是正则函数环恰好就是仿射坐标环$k[T_1,T_2,\cdots,T_n]/I(X)$.
\begin{enumerate}
	\item 明显的如果$X$是单点集$\{(x_1,\cdots,x_n)\}$,那么$k[X]=k[X_1,\cdots,X_n]/(X_1-x_1,\cdots,X_n-x_n)\cong k$.
	\item 设$X,Y$是两个仿射代数集,那么$k[X\times Y]=k[X]\otimes_kk[Y]$.
	\begin{proof}
		
		我们明显可以构造典范同态$\varphi:k[X]\otimes_kk[Y]\to k[X\times Y]$为$\varphi(\sum_if_i\otimes g_i)(x,y)=\sum_if_i(x)g_i(y)$.这明显是一个满射.为证它是单射,不妨设$\{f_i\}$和$\{g_i\}$分别是线性无关的,那么我们要证明的就是$\{f_i(x)g_j(x)\}$是线性无关的,这是显然的:从$\sum_{i,j}c_{ij}f_i(x)g_j(y)=0$得到对任意$i$有$\sum_jc_{ij}g_j(y)=0$,进而得到$c_{ij}\equiv0$.
	\end{proof}
	\item 正则函数也满足类似的零点定理:设$f,g_1,\cdots,g_m$都是正则函数,设$\{g_1,\cdots,g_m\}$的公共零点都是$f$的零点,则存在正整数$r$使得$f^r\in(g_1,\cdots,g_m)$.
	\begin{proof}
		
		设$f$被一个多项式$F(T)$表示,设$g_i$被多项式$G_i(T)$表示,再设$\{F_1,\cdots,F_l\}$定义了仿射代数集$X$.那么$\{G_1,\cdots,G_m,F_1,\cdots,F_l\}$的公共零点一定是$F$的公共零点.于是按照我们普通版本的零点定理,就存在正整数$r$使得$F^r\in(G_1,\cdots,G_m,F_1,\cdots,F_l)$.这说明$f^r\in(g_1,\cdots,g_m)$.
	\end{proof}
\end{enumerate}

仿射代数集的开子集上的正则函数.设$Y$是仿射代数集,设$U$是一个开集,一个函数$f:U\to k$在点$x\in U$处正则是指存在点$p$的开邻域$U_p\subseteq U$,使得存在多项式$g,h\in k[x_1,x_2,\cdots,x_n]$,并且有$h(x)\not=0,\forall x\in U_p$和$f(x)=g(x)/h(x),\forall x\in U_p$.如果函数在$U$上处处正则,就称$f$是$U$上的正则函数.
\begin{enumerate}
	\item 正则函数总是连续映射,这里我们把$k$等同于赋予了Zariski拓扑的$\mathbb{A}_k^1$.
	\begin{proof}
		
		给定仿射集$Y$开子集$U$上的正则函数$f:U\to k$,需要验证的是$k$上闭集的原像是闭集.但是一维仿射空间上的真闭子集就是有限子集,因此我们只需验证每个单点$a\in k$的原像总是$U$上的闭集.
		
		为了说明$f^{-1}(\{a\})$是闭集,我们只要证明对$U$的一个预先取定的开覆盖$\{U_i\}$,总有$U_i\cap f^{-1}(\{a\})$是$U_i$中的闭集.因为这导致$f^{-1}(\{a\})^c\cap U_i$是$U$中开集,对$i$取并得到$f^{-1}(\{a\})^c$是$U$中开集.
		
		不妨就把开覆盖取为正则函数定义中的特殊开集:那些满足正则函数在其上可表示为有理分式函数$g/h$的开集.取定这样一个开集$W$,于是$f^{-1}(\{a\})\cap W=\{x\in W\mid g(x)/h(x)=a\}$.但是这个集合实际上就是$Z(g-ah)\cap W$,它是$W$中的闭集.
	\end{proof}
	\item 如果$Y$是代数簇,设$f,g$是开子集$U$上的正则函数,并且在某个非空开子集上$f=g$,那么在整个$U$上恒有$f=g$.这是因为不可约集的开子集仍然是不可约的,而$f-g=0$的点集包含一个开子集的闭子集,于是它只能是整个$X$.
	\item 考虑全集作为开集,这里定义的正则函数就是仿射代数集$X$上的正则函数(即多项式函数).
	\begin{proof}
		
		一方面任取多项式函数$p(x)$,它在整个空间$X$上可表示为$p(x)/1$,于是它是我们这里局部定义的正则函数.另一方面如果映射$f(x):X\to k$满足这里我们局部定义的正则函数.对每个点$x\in X$,取定义中的开邻域$U_x$,使得存在多项式函数$p_x,q_x$,使得$f=p_x/q_x$,并且在$U_x$上恒有$q_x\not=0$.此时在$U_x$上恒有$q_xf=p_x$.我们可以选取多项式函数在$X-U_x$上取零,而在$U_x$恒不为零,再同时对$p_x,q_x$乘上这个函数.于是我们不妨假定$q_xf=p_x$在整个$X$上成立.
		
		考虑全体$q_x$生成的理想$I$,按照域上多项式环是诺特环,这个理想是有限生成的,可记$I=(g_{x_1},g_{x_2},\cdots,g_{x_N})$.我们断言这些生成元是没有公共零点的,否则全体$q_x$存在公共零点$x_0$,导致$q_{x_0}(x_0)=0$和选取矛盾.于是按照希尔伯特零点定理,$I$只能是单位理想.于是存在多项式$u_1,u_2,\cdots,u_N$满足$\sum_{1\le i\le N}u_ig_{x_i}=1$.最后按照$\sum_{1\le i\le N}p_{x_i}u_i=f\sum_{1\le i\le N}q_{x_i}u_i=f$,说明这是多项式函数.
	\end{proof}
\end{enumerate}

仿射代数集上的结构层.仿射代数集$X$的开集$U$上的全体正则函数构成了一个环,记作$\mathscr{O}_X(U)$,限制映射自然的定义为映射在更小开集上的限制,此时$\mathscr{O}_X(\cdot)$是$X$上的一个层.
\begin{enumerate}
	\item 主开集上的截面环.设$h\in k[x_1,x_2,\cdots,x_n]$,那么有$k$代数同构$k[Y]_h\cong\mathscr{O}_Y(D(h))$.这里$k[Y]=k[x_1,x_2,\cdots,x_n]/I(X)$,它称为仿射代数集$Y$的仿射坐标环,有时也记作$\mathbb{A}[Y]$.特别的,取非零常多项式$h$,得到仿射代数集上的正则函数环恰好是仿射坐标环.
	\begin{proof}
		
		定义同态$\varphi:k[Y]_h\to\mathscr{O}_Y(D(h))$为把$\overline{g}/\overline{h}^s$映射为正则函数$g(x)/h(x)^s$.先来验证定义良性和单射:$g(x)/h(x)^s\equiv0,\forall x\in D(h)$当且仅当$g(x)\equiv0,x\in D(h)$,当且仅当$g(x)h(x)\equiv0,x\in D(h)$,当且仅当$g(x)h(x)\equiv0,x\in Y$,也即在$k[Y]$上有$gh=0$.
		
		容易说明$\varphi$是一个$k$代数同态.下面验证$\varphi$是一个满射.任取$D(h)$上的正则函数$f$,按照定义,存在开覆盖$D(h)=\cup_iY_i$,以及元素$g_i,h_i\in k[Y]$,其中$h_i$不在$Y_i$上取零,并且对每个$i$有$f\mid_{Y_i}=g_i/h_i$.不妨约定每个$Y_i$是主开集$=D(a_i)$,否则可以继续把每个$Y_i$分解为主开集的并.那么按照条件,有$D(a_i)\subseteq D(h_i)$,于是存在一个正整数$N$使得$a_i^N=h_ig_i'$,其中$g_i'\in k[Y]$.于是在$D(a_i)$上就有$f=gI/h_i=g_ig_i'/a_i^N$.注意到$D(a_i^N)=D(a_i)$,于是把$g_i$替换为$g_ig_i'$,把$h_i$替换为$a_i^N$,我们不妨约定$Y$有开覆盖$\cup_iD(h_i)$,使得总有$f\mid_{Y_i}=g_i/h_i$.另外按照主开集是拟紧的,可约定这个开覆盖$D(h)=\cup_iD(h_i)$是有限的.
		
		在主开集相交的地方$D(h_i)\cap D(h_j)=D(h_ih_j)$有$g_i/h_i=g_j/h_j$,于是在$k[Y]$上有$h_ih_j^2g_i=h_i^2h_jg_j$.现在$D(h)=\cup_iD(h_i)=\cup_iD(h_i^2)$说明$Z((h))=Z((h_1^2,h_2^2,\cdots,h_m^2))$.于是有正整数$N$使得$h^N=\sum_{1\le i\le m}a_ih_i^2$.我们断言$f$可被$\sum_ia_ig_ih_i/h^N$诱导.
		
		设$x\in D(h)$,不妨设$x\in D(h_j)$,那么在$k[Y]$上有$h_j^2\sum_ia_ig_ih_i=\sum_ia_ig_jh_i^2h_j=g_jh_jh^N$.由此得到在每个$D(h_j)$上都有$fh^N=\sum_ia_ig_ih_i$.
	\end{proof}
	\item stalk.给定点$x\in Y$,那么存在典范的同构$\mathscr{O}_{Y,x}\cong k[Y]_{m_x}$.这个映射即把$x$附近的正则函数映射为它在$x$处的取值.它是一个局部环,唯一的极大理想即在$x$处取零的正则函数构成的理想.
	\begin{proof}
		
		$$\mathscr{O}_{Y,x}=\lim\limits_{\substack{\rightarrow\\h(x)\not=0}}\mathscr{O}_Y(D(h))\cong\lim\limits_{\substack{\rightarrow\\h\not\in m_x}}k[Y]_h\cong k[Y]_{m_x}$$
	\end{proof}
\end{enumerate}

射影代数集的开子集上的正则函数.给定$k$上的射影代数集$Y$,设$U\subseteq Y$是开子集,考虑函数$f:U\to k$.称它在点$x_0\in U$处正则,如果存在$x_0$的开邻域$U_0\subseteq U$,以及次数相同的齐次多项式$g,h\in k[x_0,x_1,\cdots,x_n]$,其中$h$在$U_0$上处处非零,使得$f(x)=g(x)/h(x),x\in U_0$.注意这里是因为约定了$g,h$次数相同,才保证了$g/h$的确是$U$上的函数.如果$f$在$Y$上处处正则,就称它是正则函数.开集$U$上的全体正则函数及零映射构成了一个环,记作$\mathscr{O}_Y(U)$,把限制映射自然的定义为映射在更小开集上的限制,这样$\mathscr{O}_Y(\cdot)$构成了$Y$上的一个层.
\begin{enumerate}
	\item 正则函数总是连续映射,这里我们把$k$等同于赋予了Zariski拓扑的$\mathbb{A}_k^1$.
	\item 如果$f,g$射影代数簇$X$上的正则函数,并且$f=g$在某个非空开子集上成立,那么在整个簇$X$上恒有$f=g$.事实上按照连续性和$X$的不可约性,得到$f-g=0$的点是$X$中闭的稠密集,这只能是整个$X$.
	\item 我们解释过仿射簇总可以视为射影簇的开子集(比方说仿射簇是它射影闭包的开子集),这里定义的正则函数和仿射簇上定义的正则函数(即仿射坐标环中的元)是一致的.
\end{enumerate}

射影代数集上的结构层.射影代数集$Y$的开子集$U$上的全体正则函数构成一个环,限制映射依旧定义为正则函数在更小开子集上的限制,这定义了$Y$上的一个结构层,记作$\mathscr{O}_Y(\bullet)$.
\begin{enumerate}
	\item 设$Y$是射影簇,任取$p\in Y$,记$m_p\subseteq S(Y)$为全体以$p$为零点的齐次多项式生成的理想,那么茎可以表示为$\mathscr{O}_{Y,p}=S(Y)_{(m_p)}$.
	\begin{proof}
		
		不妨设$p\in Y_i$,那么有$\mathscr{O}_{Y,p}\cong\mathbb{A}[Y_i]_{m_p'}$,这里$m_p'$是$\mathbb{A}[Y_i]$的对应于$p$的极大理想.在典范同构$\mathbb{A}[Y_i]\cong S(Y)_{(x_i)}$下,有$m_p'$映射为$m_pS(Y)_{(x_i)}$,结合$x_i\not\in m_p$,得到$\mathbb{A}[Y_i]_{m_p'}\cong S(Y)_{(m_p)}$.
	\end{proof}
	\item 设$Y$是射影簇,那么有$\mathscr{O}_Y(Y)=k$.注意这个结论对射影代数集是不成立的,Rees和Nagata构造过一个射影代数集,它的正则函数环甚至不是有限生成$k$代数.另外这个结论也依赖于基域是代数闭域.
	\begin{proof}
		
		任取一个整体正则函数$f\in\mathscr{O}_Y(Y)$.设$Y_i=Y\cap D(x_i)$,我们解释过$\mathbb{A}[Y_i]\cong S(Y)_{(x_i)}$,于是在每个$Y_i$上$f$可表示为$g_i/x_i^{N_i}$,其中$g_i\in S(Y)$是$N_i$次齐次元.现在把$\mathscr{O}_Y(Y)$视为$S(Y)$商域的子环(如果$Y$不是不可约的,那么没法取商域,此时没法整体的考虑$f$),此时有$x_i^{N_i}f\in S(Y)_{N_i}$.于是取足够大的$N\ge\sum_iN_i$,就有$S(Y)_Nf\subseteq S(Y)_N$.归纳得到$S(Y)_Nf^q\subseteq S(Y)_N$.于是特别的有$f^q\in x_0^{-N}S(Y)$.这说明$S(Y)[f]$是$S(Y)$的有限生成模$x_0^{-N}S(Y)$的子模.按照$S(Y)$是诺特的,得到$x_0^{-N}S(Y)$是诺特模,得到$S(Y)[f]$是$S(Y)$有限模.这说明$f$是$S(Y)$上的整元.于是有$a_i\in S(Y)$使得$f^m+a_1f^{m-1}+\cdots+a_m=0$.取这些$a_i$的零次分支代替,依旧得到这个等式.但是$S(Y)_0=k$,得到$f$是$k$上的代数元,按照$k$是代数闭域得到$f\in k$.
	\end{proof}
	\item 主开集上的截面环.设$h\in\mathbb{A}[X]$是非零齐次元,那么$\mathscr{O}_X(D(h)\cap X)\cong S(X)_{(h)}$.
\end{enumerate}

$k$值函数层空间,各种簇及态射的定义.
\begin{enumerate}
	\item 给定拓扑空间$X$,设$\mathscr{O}_X$是其上的一个$k$值函数层,也就说开集$U$上的每个截面是由某些$U\to k$的函数构成的环,限制映射定义为$k$值函数在更小的开子集上的限制.两个这样的层空间之间的态射$\varphi:X\to Y$定义为一个连续映射,使得对每个$s\in\mathscr{O}_Y(V)$,都有$s\circ\varphi\in\mathscr{O}_X(\varphi^{-1}(V))$.两个赋予同一个域$k$值函数层的空间称为同构的,如果它们之间存在互为逆映射的态射.这里$\varphi^{\#}:s\mapsto s\circ\varphi$是$\mathscr{O}_Y(V)\to\mathscr{O}_X(\varphi^{-1}(V))$的环同态.于是我们定义的态射的确是环空间范畴上的态射.
	\item 我们定义的仿射簇,仿射集,射影簇,射影集和拟射影簇(此为射影空间闭子集的开子集上赋予簇结构,和仿射簇与射影簇不同的是,拟射影簇未必是不可约的)上的结构层都是$k$值函数层.据此可以重新定义仿射簇,仿射集,射影簇,射影集和拟射影集为同构于(我们之前定义的)仿射簇,仿射集,射影簇,射影集和拟射影集的$k$值函数层空间.另外它们之间的态射就定义为它们作为$k$值函数层空间之间的态射,这样的态射也称为正则映射.
	\item 如果一个$k$值函数层空间$(X,\mathscr{O}_X)$满足存在开覆盖$\{U_i\}$使得每个$(U_i,\mathscr{O}_X\mid_{U_i})$都是仿射代数集,就称$(X,\mathscr{O}_X)$是预簇.预簇之间的态射(或者称为正则映射)就是它作为$k$值函数层空间之间的态射.按照我们的定义,仿射代数集,射影代数集,拟射影簇都是预簇.如果$X$是不可约预簇,那么它的非空开子集都是不可约的,于是此时它被一族仿射簇覆盖.另外有的书上预簇要求定义中的仿射代数集改为仿射簇,也有的书要求预簇的定义中是被有限个仿射簇覆盖,有时候要求不可约条件.
\end{enumerate}

态射的具体描述.
\begin{enumerate}
	\item 仿射情况.代数集之间的态射.给定两个代数集$X,Y$,如果映射$\varphi:X\to Y$满足如下两个等价条件的任意一个,就称它是代数集之间的态射,或者称为正则映射.另外在如下任一条件成立时,$\varphi$总是连续的.
	\begin{enumerate}
		\item 存在$X$上的$m$个正则函数$f_1,f_2,\cdots,f_m$,使得$\varphi(x)=(f_1(x),f_2(x),\cdots,f_m(x)),\forall x\in X$.
		\item $s\mapsto s\circ\varphi$把开集$V\subseteq Y$上的正则函数映射为开集$f^{-1}(V)\subseteq X$上的正则函数.
		\item $s\mapsto s\circ\varphi$把$Y$上的正则函数映射为$X$上的正则函数.
	\end{enumerate}
	\begin{proof}
		
		1推2,给定满足条件的映射$\varphi:X\to Y$,如果$s$是$Y$上正则函数,那么$s\circ\varphi$也是一个多项式函数,并且满足$X$所满足的多项式组,于是它是$X$上的正则函数.2推3平凡.3推1,假设$\varphi:X\to Y$是满足条件的映射,那么它诱导了$\mathbb{A}[Y]\to\mathbb{A}[X]$的同态$\varphi^{\#}$.记$y_i$为$Y$上的坐标函数,记$\varphi^{\#}(y_i)=s_i\in\mathbb{A}[X]$,那么$\varphi'(x)=(s_1(x),s_2(x),\cdots,s_m(x))$是$X\to Y$的正则函数.我们断言$\varphi=\varphi'$:事实上$y_i\circ\varphi=\varphi^{\#}(y_i)=y_i\circ\varphi'$,于是$\varphi$和$\varphi'$的每个分量对应相同,于是它们相同,于是$\varphi$是一个正则映射.
		
		下面说明它是一个连续映射:任取$Y$中的闭子集$E$,设它由有限个多项式$f_1,f_2,\cdots,f_r\in k[y_1,y_2,\cdots,y_m]$决定.那么$\varphi^{-1}(E)=Z(f_1\circ\varphi,f_2\circ\varphi,\cdots,f_r\circ\varphi)$是闭集.
	\end{proof}
    \item 仿射代数集和正则映射构成一个范畴.我们接下来解释这个范畴和有限生成既约$k$代数范畴是逆变范畴等价的;仿射代数簇和正则映射构成的范畴和有限生成$k$整环范畴是逆变范畴等价的.
    \begin{enumerate}
    	\item 一个环可以作为某个代数集的仿射坐标环当且仅当它是既约(即没有非平凡的幂零元)的有限生成$k$代数.类似的一个环可以作为某个代数簇的仿射坐标环当且仅当它是有限生成$k$代数和整环.
    	\begin{proof}
    		
    		我们要运用到:理想$I\subseteq R$是$R$的根理想当且仅当$R/I$是既约环.于是必要性是直接的.充分性则要用到零点定理:给定有限生成$k$代数,它可以表示为$k[x_1,x_2,\cdots,x_n]/\alpha$,按照它是既约环得到$\alpha$是根理想,于是从零点定理得到$\alpha=I(Z(\alpha))$,于是这个环是代数集$Z(\alpha)\subset\mathbb{A}^n$的坐标环.
    	\end{proof}
    	\item 如果$X$是$k$预簇,$Y$是$k$仿射簇,那么总有如下典范双射:
    	$$\alpha:\mathrm{Hom}_{\textbf{pVar}}(X,Y)\cong\mathrm{Hom}_{\textbf{Alg}(k)}(A[Y],\mathscr{O}_X(X))$$
    	\begin{proof}
    		
    		任取正则映射$\varphi:X\to Y$,对$Y$上的任意正则函数$u$,明显的$u\circ\varphi$是$X$上的正则函数,据此我们定义了映射$\varphi^*:u\mapsto u\circ\varphi$,并它明显是一个$k$代数同态.于是我们定义了集合映射$\alpha$.
    		
    		\qquad
    		
    		下面任取$k$代数同态$h:A[Y]\to\mathscr{O}_X(X)$.设$Y\subseteq\mathbb{A}^n$,于是有$A[Y]=k[X_1,\cdots,X_n]/I(Y)$.记$X_i$在$A[Y]$中的像是$x_i$,再记$h(x_i)=\xi_i$是$X$上的整体正则函数.我们取$\psi:X\to\mathbb{A}^n$为$P\mapsto(\xi_1(P),\cdots,\xi_n(P))$.由于这个映射的分量都是正则函数,所以只要我们证明了$\psi(X)\subseteq Y$,则它是正则映射.按照$Y=Z(I(Y))$,归结为证明对任意$P\in X$和任意$f\in I(Y)$都有$f(\psi(P))=0$,但是$f(\psi(P))=h(f(x_1,\cdots,x_n))(P)=0$.于是$\alpha(\psi)=h$,这说明了$\alpha$是满射.最后说明它是单射,这是因为如果两个态射$f,g:X\to Y$诱导了相同的$k$代数同态,那么$f,g$视为分量映射时的分量正则函数是对应一致的,于是$f=g$.
    	\end{proof}
    	\item 按照范畴的语言,两个范畴(逆变)等价当且仅当它们之间存在完全忠实本质满(逆变)函子.函子$F$定义为把代数集映射为坐标环,把代数集之间的正则映射按照上一条的方式对应为坐标环之间的$k$代数同态.第一条解释了这个函子是本质满的,上一条解释了这个函子限制在每个Hom集合上都是双射,于是两个范畴是逆变范畴等价的.
    	\item 设$k$是代数闭域,设$f:X\to Y$是仿射簇之间的正则映射,那么它诱导了$k$代数同态$f^*:k[Y]\to k[X]$.我们断言$f$是支配映射(此即$\overline{f(X)}=Y$)当且仅当$f^*$是单射.
    	\begin{proof}
    		
    		$f^*$是单射等价于讲对$Y$上任意正则函数$u$都有$f^*(u)=0$,这等价于讲对任意$u$都有$f(X)$中的点都是$u$的零点,但是正则映射的零点集是闭集,所以这等价于讲对任意$u$都有$\overline{f(X)}$落在$u$的零点集,也等价于讲对任意$u$都有$u\in I(\overline{f(X)})$,于是等价于讲$k[Y]=I(\overline{f(X)})$,等价于$\overline{f(X)}=Y$.
    	\end{proof}
    \end{enumerate}
    \item 终端为仿射空间的态射.设$X$是预簇,给定态射$\varphi:X\to\mathbb{A}^n$,记这个$n$维仿射空间上典范的整体正则(多项式)函数为$t_1,t_2,\cdots,t_n$,那么$\varphi^{\#}(t_i)=t_i\circ\varphi$是$X$上的整体正则函数,记作$s_i,1\le i\le n$.于是$\varphi$可以表示为$\varphi(x)=(s_1(x),s_2(x),\cdots,s_n(x))$.反过来给定$X$上$n$个整体正则函数就对应一个$X\to\mathbb{A}^n$的态射.于是终端为仿射空间的态射总是被源端的若干正则函数所决定.
    \item 终端为仿射簇的态射.类似的,设$X$是预簇,$Y\subset\mathbb{A}^n$是仿射簇,态射$\varphi:X\to Y$蕴含的信息等价于$X$上$n$个正则函数$s_1,s_2,\cdots,s_n$,使得对每个$F\in I(X)$,都有$F(s_1(x),s_2(x),\cdots,s_n(x))=0,\forall x\in X$.这件事在上面逆变范畴等价中证明过了.
    \item 终端为拟射影簇的态射.设$X$是预簇,$Y\subset\mathbb{P}^n$是拟射影簇,给定映射$\varphi:X\to Y$,它是态射当且仅当对每个点$x\in X$,任取包含$\varphi(x)$的$\mathbb{A}^n\cong D(x_i)\subset\mathbb{P}^n$,都存在$x$的开邻域$U$满足$\varphi(U)\subseteq D(x_i)$,并且$\varphi$限制在$U$上是终端为仿射簇的态射.这里我们需要证明下这个定义不依赖于包含$\varphi(x)$的$D(x_i)$的选取.
    \begin{proof}
    	
    	设$\varphi(x)$在$D(x_i)$中的坐标表示为$(y_0,\cdots,\hat{y_i},\cdots,y_n)$.如果$\varphi(x)$也落在$D(x_j)$中,$\varphi(x)$此时的坐标表示为$(z_0,\cdots,\hat{z_j},\cdots,z_n)$,其中$k\not=i,j$的时候$z_k=y_0/y_j$,而$z_j=1/y_j$.于是如果$\varphi$限制在$U\to D(x_i)\cong\mathbb{A}^n$的坐标表示为$(s_0,\cdots,\hat{s_i},\cdots,s_n)$,那么它限制在$U\to D(x_j)\cong\mathbb{A}^n$的坐标表示为$(t_0,\cdots,\hat{t_j},\cdots,t_n)$,其中$k\not=i,j$的时候$g_k=f_0/f_j$,而$g_j=1/f_j$.于是适当选取$x$更小的开邻域$U'\subseteq U$使得$f_j$在其上不取零,此时这些分量都是$X$上的正则函数,于是它也是正则映射.
    \end{proof}
    \item 不可约拟射影簇到射影空间的态射.设$X$是不可约拟射影簇,给定态射$\varphi:X\to\mathbb{P}^n$,任取点$x\in X$,任取$\varphi(x)$所在的$D(x_i)$,为方便起见不妨设为$D(x_0)$,那么存在$x$的开邻域$U_x$使得$\varphi:U_x\to D(x_0)$可以被$n$个正则函数$f_1,f_2,\cdots,f_n$所表示.于是适当缩小$x$的开邻域$U_x$,按照正则函数的定义存在齐次多项式$P_i,Q_i$使得在$U_x$上有$f_i=P_i/Q_i$和$Q_i(x)\not=0$,并且满足$\deg P_i=\deg Q_i,\forall 0\le i\le n$.我们可以适当乘以因式使得$f_i=P_i/Q_i$具有公分母,即$f_i=F_i/F_0,\forall 1\le i\le n$.于是这些$F_i,0\le i\le n$是具有相同次数的齐次多项式.按照拟射影簇是拟紧的,存在有限个$U_x$覆盖了整个空间$X$,按照不可约条件,不同$U_x$的交总是非空的,这确保了这些$F_0,F_1,\cdots,F_n$可以延拓到整个$X$上.于是这样的态射$\varphi$可表示为$[F_0(x):F_1(x):\cdots:F_n(x)]$,它们是同次数齐次多项式.这样的表示不是唯一的,另一个表示$[G_0:G_1:\cdots:G_n]$和它对应相同的态射当且仅当$F_iG_j=F_jG_i,\forall i,j$.
    \item 不可约拟射影簇到拟射影簇的态射.设$X$是不可约拟射影簇,$Y\subset\mathbb{P}^n$是拟射影簇,态射$\varphi:X\to Y$可以表示为$[F_0:F_1:\cdots:F_n]$,它们是同次数齐次多项式,并且满足$I(Y)$中的齐次多项式.
\end{enumerate}

正则映射的例子.
\begin{enumerate}
	\item $\mathbb{P}^n$的一个$d$维线性子空间$E$是指被一次齐次多项式$L_1,L_2,\cdots,L_{n-d}$定义的射影代数集,满足如果记$L_i=\sum_{0\le j\le n}a_{ij}x_j,a_{ij}\in k$,那么$\left(a_{ij}\right)$是一个秩为$n-d$的矩阵.一个中心在$E$的投影映射是指$\mathbb{P}^n\to\mathbb{P}^{n-d-1}$的有理映射$\pi:x\mapsto[L_1(x):L_2(x):\cdots:L_{n-d}(x)]$.这实际上是开集$\mathbb{P}^n-E$到$\mathbb{P}^{n-d-1}$的正则映射.
	\item Veronese嵌入.考虑全体$k[S_0,S_1,\cdots,S_n]$的$m$次齐次多项式,它构成了$k$上$N+1=\left(\begin{array}{cc}m+n\\m\end{array}\right)$维的线性子空间.于是存在从全体$\mathbb{P}^n$的$m$次超曲面到$\mathbb{P}^N$的双射为把$F=\sum_{i_0+i_1+\cdots+i_n=m}a_{i_0,i_1,\cdots,i_n}S_0^{i_0}S_1^{i_1}\cdots S_n^{i_n}\mapsto(a_{i_0,i_1,\cdots,i_n})$.构造正则映射$v_m:\mathbb{P}^n\to\mathbb{P}^N$为$[u_0:u_1:\cdots:u_n]\mapsto(v_{i_0,i_1,\cdots,i_n}=u_0^{i_0}u_1^{i_1}\cdots u_n^{i_n})_{i_0+i_1+\cdots+i_n=m}$.这称为$\mathbb{P}^n$的$m$次的Veronese嵌入.
	\begin{itemize}
		\item 我们构造的这个正则映射的像集被齐次方程$v_{i_0,i_1,\cdots,i_n}v_{j_0,j_1,\cdots,j_n}=v_{k_0,k_1,\cdots,k_n}v_{l_0,l_1,\cdots,l_n}$使得$i_0+j_0=k_0+l_0,\cdots,i_n+j_n=k_n+l_n$.
		\item 如果$F=\sum a_{i_0,i_1,\cdots,i_n}u_0^{i_0}u_1^{i_1}\cdots u_n^{i_n}$是$m$次齐次多项式,设$H$是由它定义的$\mathbb{P}^n$的超曲面,那么$v_m(H)\subseteq v_m(\mathbb{P}^n)\subset\mathbb{P}^N$是$v_m(\mathbb{P}^n)$和$\mathbb{P}^N$的由$\sum a_{i_0,i_1,\cdots,i_n}v_{i_0,i_1,\cdots,i_n}=0$定义的超平面的交.
	\end{itemize}
\end{enumerate}
\subsection{有理函数和映射}

函数域.设$X$是一个不可约预簇,它的函数域定义为它的一般茎(generic stalk):$k(X)=\lim\limits_{\substack{\rightarrow\\U\not=\emptyset}}\mathscr{O}_X(U)$.这里正向极限的指标$U$跑遍全部非空开子集.函数域记作$k(X)$,它的元素称为预簇上的有理函数.
\begin{enumerate}
	\item 关于共尾(cofinal)有向集.设$I$是一个有向集,它的子集$J$称为共尾子集,如果对任意$i\in I$,存在$j\in J$使得$i\le j$.那么$J$也是一个有向集.另外如果设$(G_i,f_{ij})_{i,j\in I}$是一个指标集为有向集的正向系统.取$I$的共尾子集$J$,它对应了正向系统是指$(G_i,f_{ij})_{i,j\in J}$,那么这两个正向系统的极限存在性是一致的,并且正向极限存在时它们是典范同构的.
	\item 如果$X$是不可约的环空间,它的全部非空开子集构成一个有向集,任取非空开集$U$,那么$U$的全部非空开子集构成了一个共尾子有向集.这说明预簇的函数域同构于它任意开子预簇的函数域.
	\item 不可约预簇上的函数域是一个域.
	\begin{proof}
		
		按照上一条这归结为证明仿射簇上函数域是域.设$(U,f)$所在的等价类为$[U,f]$,定义加法和乘法为$[U,f]+[V,g]=[U\cap V,f+g]$和$[U,f][V,g]=[U\cap V,fg]$.它的加法零元由全体$(U,0)$构成.它的乘法幺元由全体$(U,1)$构成,这里1表示在$U$上处处取$1_k$的正则映射.最后验证每个非零元都有逆:设$[U,f]$是一个非零元,那么$f$至少在某个点$p\in U$处不取零,记$V=U\cap Z(f)^c$是$U$的开子集,那么$[V,1/f]$是$[U,f]$的逆.
	\end{proof}
	\item 不可约的预簇的函数域是它任一开子仿射簇的坐标环的商域,也是它任一茎的商域.这个性质依旧归结为证明仿射簇情况,下面会给出证明.
	\item 有理函数是极大延拓的正则函数.任取等价类$[U,f]$,取其中两个不同的元素$(U,f)$和$(V,g)$,按照不可约条件有$U\cap V$非空,按照等价类定义有$f,g$在$U\cap V$的某个开子集上的限制相同,于是按照正则函数的唯一性,$f,g$在整个$U\cap V$上限制是相同的,于是这两个函数可以统一延拓为$U\cup V$上的正则函数.据此,如果我们把这个等价类中出现的全部开子集并起来,正则函数可以延拓到这个并$U_0$上.并且这个正则函数不能在$X$上继续延拓,否则这样更大的开集理应出现在等价类中.于是每个有理函数可视为极大延拓的正则函数,这个极大定义域称为有理函数的定义域.
	\item 有理函数的唯一性定理.如果不可约预簇$X$上的两个有理函数在某个都有定义的非空开子集上取值相同,那么它们是相同的有理函数.
	\begin{proof}
		
		假设它们是不同的有理函数,把它们视为不可延拓的正则函数,它们的定义域的交是一个不可约开子集,条件说这个不可约集存在开子集上两个正则函数取值相同,于是不可约性说明在定义域的交集上两个函数取值相同,但是这导致两个极大正则函数的定义域延拓为它们定义域的并,这矛盾.
	\end{proof}
\end{enumerate}

仿射簇上的函数域.
\begin{enumerate}
	\item 仿射簇上的函数域是仿射坐标环的商域,也是每个点局部环的商域(剩余域).于是仿射簇上的有理函数都可以表示为两个正则函数(多项式函数)的商.
	\begin{proof}
	
	    按照正向极限的定义,仿射簇$X$上的一般茎可以这样描述:考虑全体对$(U,f)$,其中$U$是$X$的非空开子集,$f$是$U$上的正则函数.两个对$(U,f)$和$(V,g)$称为等价的,如果在$U\cap V$上有$f=g$.每个等价类称为$X$上的一个有理函数.任取点$p\in X$,它的茎的定义是全体等价类$[U,f]$,这里$U$要覆盖点$p$,于是定义$\mathscr{O}_{X,p}$的正向系统是定义函数域的正向系统的子系统.按照正向系统的泛性质,得到同态$\varphi_x:\mathscr{O}_{X,p}\to k(X)$.它就是把茎中的等价类$[U,f]$映射为函数域中的等价类$[U,f]$.这个映射是单射,于是它可提升为$\mathrm{Frac}(\mathscr{O}_{X,p})=\mathrm{Frac}(\mathbb{A}[X])\to k(X)$的域同态.这个同态不依赖于点$p\in X$的选取.它是单射因为域同态总是单射,它是满射因为函数域中的每个等价类$[U,f]$都是某个茎中的等价类.于是我们证明了仿射簇的函数域就是它坐标环的商域,也是每个茎的商域.
	    $$\xymatrix{\mathrm{Frac}(\mathbb{A}[X])\ar[drr]&&\\\mathscr{O}_{X,p}\ar[rr]\ar[u]&&k(X)\\\mathscr{O}_X(U_i),p\in U_i\ar[u]\ar[urr]&&}$$
	\end{proof}
    \item 正则点.给定$X$上的有理函数$f$,称它在点$p\in X$正则,如果它可以表示为多项式函数的商$f=g/h$,使得$h(p)\not=0$.取仿射簇上一个有理函数$f$,那么它在某个开集$U$上处处正则,当且仅当$f$是$U$上的正则函数.
    \begin{proof}
    	
    	我们要证明的是如果一个有理函数在开集$U$上处处正则,那么它是一个正则函数.按照定义,对每个点$p\in U$,有正则函数$f_p$和$g_p$,满足$\varphi=f_p/g_p$,并且$g_p(p)\not=0$.考虑$\{g_p\mid p\in X\}$生成的理想为$I$,它是有限生成的,可取$p_1,p_2,\cdots,p_N\in U$使得$I=(g_{p_1},g_{p_2},\cdots,g_{p_N})$.我们断言$Z(I)$是空集,因为如果$p_0\in Z(I)$,导致每个$g_{p_i}(p_0)=0$,但是这导致$g_{p_0}(p_0)=0$矛盾.于是$I$是单位理想,于是有正则函数$u_1,u_2,\cdots,u_N$满足$\sum_{1\le i\le N}u_ig_{p_i}=1$.两边乘以$\varphi$得到$\varphi=\sum_{1\le i\le N}u_if_{p_i}$,于是$\varphi$是正则函数.
    \end{proof}
    \item 有理函数的定义域.有理函数的所有正则点$U$是这个多项式函数在仿射坐标环商域中不同分式表示的分母不取零的点的并,它是一个开集,并且恰好是这个有理函数的定义域.
	\item 仿射簇$X$上的任意开集$U$,和任意点$p\in X$,就有典范的单射链$\mathscr{O}_X(U)\to\mathscr{O}_{X,x}\to k(X)$.换句话讲截面环和茎都可以视为有理函数域的子环.在这个观点下我们可以从函数域着手,定义开集$U$上的正则函数为全体定义域包含开集$U$的有理函数构成的子环;定义点$p$处的茎为全体在点$p$处有定义的有理函数构成的子环.
\end{enumerate}

不可约拟射影簇上的函数域.
\begin{enumerate}
	\item 我们解释过不可约预簇上的函数域就是它每个仿射开子集上坐标环的商域,于是如果记$X\subseteq\mathbb{P}^n$是不可约拟射影簇,设它在$\mathbb{P}^n$中的闭包是$Y$,那么$Y$就是$\mathbb{P}^n$的射影簇.$Y$的函数域就是$Y\cap D(x_i)$上齐次坐标环$S(Y)_{(x_i)}$的商域,也即$S(Y)_{(0)}$,这表示齐次坐标环商域的零次子环.换句话讲不可约拟射影簇上的有理函数总可以表示为两个次数相同的齐次多项式的商$P/Q$,使得$Q(x)\not\equiv0$.
	\item 类似的定义拟射影簇$X$上有理函数$f$在一点$p\in X$正则,如果$f$可表示为两个相同次数的齐次多项式的商$f=P/Q$,使得$Q(p)\not=0$.一个拟射影簇上的有理函数如果在某个开子集上处处正则,那么它是这个开集上的正则函数.
	\item 拟射影簇上有理函数全部正则点构成开子集,它就是这个有理函数的定义域.
\end{enumerate}

不可约簇上的有理映射.拟射影簇满足可分公理,于是我们这里定义的有理映射适用于拟射影簇之间.
\begin{enumerate}
	\item 设$X$是不可约预簇,$Y$是不可约簇.考虑全体对$(U,\varphi_U)$,其中$U\subseteq X$是开子预簇,$\varphi_U:U\to Y$是正则映射(预簇之间的态射),定义$(U,\varphi_U)$和$(V,\varphi_V)$等价,如果$\varphi_U$和$\varphi_V$在$U\cap V$上是一致的.每个等价类记作$[U,\varphi_U]$,称为$X\to Y$的有理映射.
	\item 为了验证这的确是一个等价关系,需要如下结论:按照$Y$是簇,满足可分公理,于是$\{z\in X\mid\varphi(z)=\psi(z)\}$是$X$中的闭集,但是这是不可约空间上的一个包含了开集的闭子集,它只能是全集.
	\item 有理映射并不是实际意义上的$X\to Y$的映射.它可以理解为极大延拓的正则映射.即如果把一个等价类$\{(U,\varphi_U)\}$中所有开集$U$并起来得到$W$,那么这些$\varphi_U$可延拓为$W\to Y$的正则映射.
	\item 仿射情况.仿射簇之间的有理映射$\varphi:X\to Y\subset\mathbb{A}^m$就是$m$个$X$上的有理函数$\varphi_1,\varphi_2,\cdots,\varphi_m$,它们的定义域的交是一个非空开集,满足这个开集中的每个点$x$都有$\varphi(x)=(\varphi_1(x),\varphi_2(x),\cdots,\varphi_m(x))\in Y$.那么任意一组有理函数$\{\varphi_i\}$是$X\to Y$的有理映射当且仅当把每个$\varphi_i$视为$k(X)$中的元后,这组元带入定义$Y$的多项式族总得到$k(X)$中的零元.
	\begin{proof}
		
		充分性,如果对每个$u(T_1,T_2,\cdots,T_m)\in I(Y)$,都有$u(\varphi_1,\varphi_2,\cdots,\varphi_m)=0$.那么对$\{\varphi_i\}$的公共定义域的每个点$x$,都有$u(\varphi_1(x),\varphi_2(x),\cdots,\varphi_m(x))=0$,于是$(\varphi_1(x),\varphi_2(x),\cdots,\varphi_m(x))\in Y$.
		
		必要性,如果$\varphi:X\to Y$是有理映射,那么对每个$u\in I(Y)$,都有$u(\varphi_1(x),\varphi_2(x),\cdots,\varphi_m(x))$在$\{\varphi_i\}$的公共定义域上取零,于是它在一个非空开集上取零,按照$X$是不可约的,得到它在整个$X$上处处取零,于是在$k(X)$上有$u(\varphi_1,\varphi_2,\cdots,\varphi_m)=0$.
	\end{proof}
    \item 设$\varphi:X\to Y$是有理映射,记定义域为$U$,如果$\varphi(U)$在$Y$中稠密,就称$\varphi$是支配有理映射.如果$\varphi:X\to Y$是两个仿射簇之间的支配有理映射,设$\varphi$的定义域是开集$U$,那么正则映射$\varphi\mid U:U\to Y$可表示为$x\mapsto(\varphi_1(x),\varphi_2(x),\cdots,\varphi_m(x))$,其中$\varphi_i$都是$X$上的正则函数,并且$\varphi_i$是$Y$对应理想中多项式的根(上一条结论).这个正则映射被一个环同态$k[Y]\to k[X]$,$f\mapsto f(\varphi_1,\varphi_2,\cdots,\varphi_m)$诱导.我们断言这里$\varphi^*$是单射.
    \begin{proof}
    	
    	假设$\varphi^*(f)=0$,此即$f(\varphi(x))=0,\forall x\in U$.按照$\varphi$的像集是$Y$的稠密集,说明$f$的零点集包含一个稠密集,但是$f$是连续的,这迫使$f$处处为零.
    \end{proof}
    \item 类似的,拟射影簇之间的有理映射$\varphi:X\to Y\subset\mathbb{P}^n$可以表示为$x\mapsto[F_0(x):F_1(x):\cdots:F_n(x)]$,其中$F_i(x)$是次数相同的齐次多项式,并且它们满足$I(Y)$中的齐次多项式.
    \item 不可约簇之间的有理映射如果存在逆有理映射,就称它是双有理映射,如果$X$和$Y$之间有双有理映射,就称它们是双有理等价的.
\end{enumerate}

我们接下来证明一个重要结论,不可约拟射影簇和支配有理映射构成的范畴和基域(代数闭域)$k$上的有限生成域扩张范畴是逆变范畴等价的.
\begin{enumerate}
	\item 首先是对象之间的对应.对不可约拟射影簇$X$,它对应于函数域$k(X)$.我们解释过不可约预簇上的函数域就是它仿射开子集的函数域,但是仿射簇上的函数域是有限生成的,说明不可约拟射影簇的函数域在基域上总是有限生成的.反过来我们需要说明基域$k$的每个有限生成扩张域$L$都是某个不可约拟射影簇的函数域.设$L=k(y_1,y_2,\cdots,y_n)$,记$B=k[y_1,y_2,\cdots,y_n]$,那么$B$是多项式环$k[T_1,T_2,\cdots,T_n]$的商,于是$B$是某个仿射簇$X$的坐标环,于是$L$是$X$的函数域.
	\item 现在给定不可约拟射影簇之间的支配有理映射$\varphi:X\to Y$,设它的极大定义域为$U$,任取有理函数$f\in k(Y)$,设它的极大定义域为$V$.按照$\varphi(U)$在$Y$中稠密,说明$\varphi^{-1}(V)$是非空的$X$中的开子集.于是$f\circ\varphi$是$\varphi^{-1}(V)$上的正则函数,它唯一的对应于$X$上的一个有理函数.这定义了一个$k$代数同态$k(Y)\to k(X)$.
	\item 另外我们之前证明过仿射簇的情况下,支配有理映射对应的极大正则映射,对应的坐标环之间的环同态是单射,于是这个单射可以提升为固定$k$域同态$k(Y)\to k(X)$.此即在仿射情况下上一条中有理映射对应的$k$代数同态.
	\item 设$X,Y,Z$都是不可约拟射影簇.如果$\varphi:X\to Y$和$\psi:Y\to Z$是两个支配有理映射,那么$\psi\circ\varphi:X\to Z$也是支配的有理映射.此时上一条所定义的诱导的域同态满足$(\psi\circ\varphi)^*=\varphi^*\circ\psi^*$.
	\begin{proof}
		
		容易验证两个有理映射的复合仍然是有理映射.假设$\varphi$的定义域为$U$,$\psi$的定义域为$V$,那么有理映射$\psi\circ\varphi$的定义域为$U\cap\varphi^{-1}(V)$.现在假设$\overline{\varphi(U)}=Y$和$\overline{\psi(V)}=Z$.任取$Z$中的开集$T$,那么不可约性说明$\psi^{-1}(T)\cap V$是$Y$的非空开子集.按照$\overline{\varphi(U)}=Y$,说明存在$x\in U$,使得$\varphi(x)\in\psi^{-1}(T)\cap Y$,于是$x\in U\cap\varphi^{-1}(V)$,并且$\psi\circ\varphi(x)\in T$.于是$\psi\circ\varphi$是支配的.
	\end{proof}
	\item 最后我们验证第二条构造的是全体支配有理映射和全体$k(Y)\to k(X)$的$k$代数同态之间的双射.为此我们直接构造逆映射.任取$k$代数同态$\theta:k(Y)\to k(X)$,取$Y$的仿射开覆盖$\{Y_i\}$,记$\{y_1,y_2,\cdots,y_n\}$为$\mathbb{A}[Y_i]$的生成元.那么$\theta(y_i)$都是$X$上的有理函数,可取开集$U_i\subseteq X$上满足每个$\theta(y_i)$在其上有定义.于是$\theta$诱导了$k$代数同态$\mathbb{A}[Y]\to\mathscr{O}_X(U_i)$.这个$k$代数同态对应了簇之间的态射$\varphi_i:U_i\to Y$.现在对每个指标$i$我们定义了态射$\varphi_i$,可验证它们在$U_i$相交的地方是一致的,从而它们可以粘合为一个态射$\varphi$.这个映射和上一条的映射是互为逆映射的.
	\item 给定两个不可约拟射影簇$X$和$Y$,那么如下条件互相等价:
	\begin{itemize}
		\item $X$和$Y$是双有理等价的.
		\item 有$k$代数同构$k(X)\cong k(Y)$.
		\item 存在开子集$U\subseteq X$和$V\subseteq Y$,使得$U$和$V$作为簇是同构的.
	\end{itemize}
	\begin{proof}
		
		我们证明的范畴等价说明这里第一条和第三条是等价的.现在假设$X$和$Y$是双有理等价的,设有理映射$\varphi:X\to Y$和$\psi:Y\to X$互为逆映射,适当缩小这两个有理映射的定义域,就使得它们在其上的限制是两个互为逆映射的正则函数.反过来如果存在开子集$U\subseteq X$和$V\subseteq Y$作为预簇是同构的,那么它们的坐标环作为$k$代数是同构的,所以它们的商域是同构的.
	\end{proof}
\end{enumerate}

我们接下来要证明不可约拟射影簇总是双有理等价于射影空间中的某个超曲面.射影空间$\mathbb{P}^n(k)$中的超曲面是指被一个齐次多项式定义的射影代数集,这个齐次多项式的次数称为这个超曲面的次数.
\begin{enumerate}
	\item 一个域扩张$k\subseteq F$称为可分生成的,如果它存在超越基$\{x_i\}\subseteq F$,使得$F$在$k(x_i)$上是可分代数扩张.此时称$\{x_i\}$是扩张的一组可分超越基.关于可分生成有这样两个结论:
	\begin{itemize}
		\item 代数闭域上的有限生成扩张总是可分生成的.
		\item 一个有限生成并且可分生成的域扩张的生成元集中总包含一组可分超越基.
		\item 于是代数闭域上的有限生成扩张的生成元集中总包含一组可分超越基.
	\end{itemize}
	\item 设$X$是$r$维不可约拟射影簇,那么它双有理等价于$\mathbb{A}^{r+1}$中的一个超曲面.取射影闭包,它也双有理等价于$\mathbb{P}^{r+1}$中的一个超曲面.
	\begin{proof}
		
		设$L$是拟射影簇$X$的函数域,那么它是基域$k$的有限生成扩张,记$L=k(t_1,t_2,\cdots,t_m)$.$k$是代数闭域,于是上一条说明这是可分生成的,并且可取超越基$\{x_1,x_2,\cdots,x_r\}\subset\{t_1,t_2,\cdots,t_m\}$使得$L$是$k(x_1,x_2,\cdots,x_r)$的有限可分扩张.此时$k(x_1,x_2,\cdots,x_r)\subseteq L$是有限可分代数扩张.按照本原元定理,有本原元$y$,使得$L=k(x_1,x_2,\cdots,x_r,y)$.现在$y$满足一个$k(x_1,x_2,\cdots,x_r)$系数的不可约多项式,通分去分母,得到一个不可约多项式$f(X_1,X_2,\cdots,X_r,Y)=0$.这定义了$\mathbb{A}^{r+1}$中的一个超曲面,它的函数域是$L$.于是它双有理等价于$X$.
	\end{proof}
\end{enumerate}
\subsection{预簇和簇}

预簇.如果一个$k$值函数层空间$(X,\mathscr{O}_X)$满足存在开覆盖$\{U_i\}$使得每个$(U_i,\mathscr{O}_X\mid_{U_i})$都是仿射代数集,就称$(X,\mathscr{O}_X)$是预簇.预簇之间的态射(或者称为正则映射)就是它作为$k$值函数层空间之间的态射.
\begin{enumerate}
	\item 我们知道仿射代数集的仿射开子集构成拓扑基,于是预簇的仿射开子集(此为同构于仿射代数集的开子集)也构成拓扑基.考虑不可约预簇,它的仿射开子集一定是不可约的,所以不可约预簇的仿射簇开子集构成了拓扑基.
	\item 如果预簇$X$是被一族仿射簇覆盖,那么$X$上的连通性和不可约性等价.
	\begin{proof}
		
		一方面不可约空间总是连通的,反过来设$X$是连通的,任取$X$的非空开子集$V$,我们只要证明$V$是稠密的即可,这归结为证明$X$的所有仿射开子集都和$V$有交.设所有和$V$有交的仿射开子集的并为$U_1$,设所有和$V$无交的仿射开子集的并为$U_2$.那么明显有$U_1\cup U_2=X$.假设$y\in U_1\cap U_2$,那么存在$y$的仿射开邻域$W_1,W_2$,满足$W_1$和$V$有交,$W_2$和$V$无交.那么$W_1\cap V$是仿射簇$W_1$的非空开集,按照$W_1$是不可约的,就有$W_1\cap V$是$W_1$的稠密开集.按照$W=W_1\cap W_2$是$W_1$的非空开集,它就和稠密开子集$W_1\cap V$有交,但是这导致$W_2\cap V$有交矛盾.于是这样的$y$不存在,于是$U_1\cap U_2=\emptyset$,于是按照连通性得到$U_2=\emptyset$.
	\end{proof}
    \item 预簇$X$是诺特的当且仅当它能被有限个仿射代数集覆盖.于是特别的,诺特预簇(以及诺特预簇的闭子集)只有有限个不可约分支.
    \begin{proof}
    	
    	充分性是因为有限个诺特空间的并仍然是诺特的.必要性是因为诺特空间是拟紧的,所以定义中的仿射开覆盖可以取有限子覆盖.
    \end{proof}
    \item 开子预簇.
    \begin{enumerate}
    	\item 设$(X,\mathscr{O}_X)$是仿射代数集,设$f\in k[X]$,那么$(D(f),\mathscr{O}_X\mid_{D(f)})$也是仿射代数集.并且如果$X$是仿射簇的时候$D(f)$也是仿射簇.
    	\begin{proof}
    		
    		我们把$X$视为一个仿射空间$\mathbb{A}^n$的闭子集,记$\mathfrak{a}=I(X)$是$k[X_1,\cdots,X_n]$的理想,设$f\in k[X]$在$k[X_1,\cdots,X_n]$中的一个提升为$f_1$.取$k[X_1,\cdot,X_{n+1}]$的理想$\mathfrak{b}$是由$\mathfrak{a}$和$1-f_1X_{n+1}$生成.记$Y$是$\mathbb{A}^{n+1}$的被$\mathfrak{b}$定义的仿射代数集.我们只需证明$(Y,\mathscr{O}_Y)\cong(D(f),\mathscr{O}_X\mid_{D(f)})$,并且如果$\mathfrak{a}$是素理想则有$\mathfrak{b}$是素理想(于是$X$是不可约的可以推出$D(f)$也是不可约的).
    		
    		第二个命题是容易的,按照$k[X_1,\cdots,X_{n+1}]/\mathfrak{b}=(k[X])_f$,从右侧是整环得到左侧是整环,于是$\mathfrak{b}$是素理想.至于第一个命题,定义$\alpha:(Y,\mathscr{O}_Y)\to(X,\mathscr{O}_X)$为$(x_1,\cdots,x_n,x_{n+1})\mapsto(x_1,\cdots,x_n)$,这是一个正则单映射,它的像集恰好是$D(f)$.容易构造逆映射$(D(f),\mathscr{O}_X\mid_{D(f)})\to(Y,\mathscr{O}_Y)$为$(x_1,\cdots,x_n)\mapsto(x_1,\cdots,x_n,1/f_1(x_1,\cdots,x_n))$.得到这个同构.
    	\end{proof}
        \item 如果$(X,\mathscr{O}_X)$是预簇,设$U\subseteq X$是开集,那么$(U,\mathscr{O}_X\mid_U)$也是预簇,这称为$X$在开子集$U$上的开子预簇.另外明显的$X$的不可约性和诺特性都可以传递给$U$.
        \begin{proof}
        	
        	按照定义$X$被一族仿射代数集$\{U_i\}$覆盖,上一条告诉我们仿射代数集的主开集都是仿射代数集,于是仿射代数集有仿射开子集构成的拓扑基,于是每个$U\cap U_i$都可以被一族仿射代数集覆盖,进而$U$可以被一族仿射代数集覆盖.
        \end{proof}
    \end{enumerate}
    \item 闭子预簇.比如说对于仿射代数集$X$,则$X$可以视为某个仿射空间$\mathbb{A}^n$的闭子集,那么$X$的闭子集$Y$当然也能视为$\mathbb{A}^n$的闭子集,从而具备结构层.我们要证明的是这种定义不依赖于嵌入$X\to\mathbb{A}^n$的选取,从而闭子预簇概念不依赖于所在的大空间的选取.
    \begin{enumerate}
    	\item 仿射情况.设$(X,\mathscr{O}_X)$是仿射代数集,设$Y\subseteq X$是闭子集,对$Y$的开子集$V$,定义$\mathscr{O}_Y(V)$为$V$上的全体$k$值函数,满足对任意$x\in V$,都存在$x$在$X$中的开邻域$U$,以及一个函数$F\in\mathscr{O}_X(U)$,使得$f=F\mid_{U\cap V}$.我们断言$(Y,\mathscr{O}_Y)$是一个仿射代数集.
    	\begin{proof}
    		
    		我们不妨设$X\subseteq\mathbb{A}^n$,那么$X$的闭子集$Y$也是一个仿射代数集,把结构层记作$\mathscr{O}_Y'$,那么我们要证明的是$\mathscr{O}_Y$和$\mathscr{O}_Y'$是一致的.首先按照包含映射$(Y,\mathscr{O}_Y')\to(X,\mathscr{O}_X)$是态射,于是$\mathscr{O}_X$中的函数在$Y$的开子集上的限制的确是$\mathscr{O}_Y'$中的函数,这就导致$\mathscr{O}_Y\subseteq\mathscr{O}_Y'$.反过来任取$g\in R=k[X_1,\cdots,X_n]$,任取$f'\in\mathscr{O}_Y'(D_Y(g))$,那么它一定是某个$f\in\mathscr{O}_X(D_X(g))=R_g$的限制,但是这按照定义有$f'\in\mathscr{O}_Y(D_Y(g))$.综上我们证明了总有$\mathscr{O}_Y\subseteq\mathscr{O}_Y'$并且它们在主开集上一致,于是它们是相同的层.
    	\end{proof}
        \item 一般情况.设$(X,\mathscr{O}_X)$是预簇,设$Y\subseteq X$是闭子集,对$Y$的开子集$V$,定义$\mathscr{O}_Y(V)$为$V$上的全体$k$值函数,满足对任意$x\in V$,都存在$x$在$X$中的开邻域$U$,以及一个函数$F\in\mathscr{O}_X(U)$,使得$f=F\mid_{U\cap V}$.那么$(Y,\mathscr{O}_Y)$是一个预簇.
    \end{enumerate}
\end{enumerate}

预簇的积.取定两个$k$预簇$X,Y$,它们的积对象$(X\times Y,p:X\times Y\to X,q:X\times Y\to Y)$定义为一个$k$预簇,满足对任意$k$预簇$W$,总有典范映射$\mathrm{Hom}(W,X\times Y)\to\mathrm{Hom}(W,X)\times\mathrm{Hom}(W,Y)$,$\varphi\mapsto(p\circ\varphi,q\circ\varphi)$是同构.
\begin{enumerate}
	\item 取$W=\mathbb{A}^0_k$是单点集,那么正则映射$\varphi:\mathbb{A}_k^0\to X$被$X$中的点所刻画,这说明$X\times Y$如果存在,则它在集合层面上就是$X$和$Y$的笛卡尔积.
	\item 仿射代数集的积.
	\begin{enumerate}
		\item 设$X$和$Y$是域$k$上的两个仿射代数集.记$X\subset\mathbb{A}^{n_1}$由多项式集$\{f_i(x_1,\cdots,x_{n_1}),1\le i\le m_1\}$定义;记$Y\subset\mathbb{A}^{n_2}$由多项式集$\{g_j(y_1,\cdots,y_{n_2}),1\le j\le m_2\}$定义.设它们的坐标环分别是$R$和$S$.记$X\times Y\subset\mathbb{A}^{n_1+n_2}$由多项式集$\{f_i,g_j\}$定义,它的坐标环就是$R\otimes_kS$,这是$X,Y$在预簇范畴中的积对象.在$k$是代数闭域时,如果$X,Y$都是仿射簇,那么$X\times Y$也是仿射簇.
		\begin{proof}
			
			先证明$X\times Y$作为$k$仿射代数集的坐标环的确是$R\otimes_kS$:
			$$k[x_1,\cdots,x_{n_1},y_1,\cdots,y_{n_2}]/(f_i,g_j)\cong k[x_1,\cdots,x_{n_1}]/(f_i)\otimes_kk[y_1,\cdots,y_{n_2}]/(g_j)=R\otimes_kS$$
			
			再说明构造的$X\times Y$的确是不可约的,为此等价于证明$R\otimes_kS$是整环,这是一个代数结论:代数闭域上两个整环的张量积是整环.
			
			\qquad
			
			最后证明这的确是积对象.首先投影映射$p:X\times Y\to X,q:X\times Y\to Y$是典范的分别由坐标环之间的同态$\mathbb{A}[X]\to\mathbb{A}[X\times Y]$和$\mathbb{A}[Y]\to\mathbb{A}[X\times Y]$诱导的态射.任取$k$预簇$W$以及态射$r:W\to X$和$s:W\to Y$,按照我们构造的$X\times Y$的确是集合层面的笛卡尔积,导致恰好存在唯一的集合映射$t:W\to X\times Y$满足$r=p\circ t$和$s=q\circ t$.于是仅需要做的是证明这里$t$是态射.这是一个终端为仿射簇的映射,我们解释过验证它的正则性等价于验证它诱导了映射$t^*:\Gamma(X\times Y,\mathscr{O}_{X\times Y})\to\Gamma(W,\mathscr{O}_W)$.但是我们解释了$X\times Y$上的整体截面被$R$和$S$生成,按照$r,s$都是态射,这些生成元被$t^*$映为$W$上整体截面,于是$X\times Y$上整体截面总被$t^*$映为$W$上整体截面.
		\end{proof}
		\item 按照$X\times Y$的坐标环为$R\otimes_kS$,说明$X\times Y$上主开集的形式为$D(\sum f_i(x)g_j(y))$,其中$f_i\in R$,$g_j\in S$.
		\item $\mathscr{O}_{X\times Y}$在点$(x,y)$的局部环就是$\mathscr{O}_{X,x}\otimes_k\mathscr{O}_{Y,y}$在极大理想$m_x\mathscr{O}_{Y,y}+m_y\mathscr{O}_{X,x}$的局部化.
		\begin{proof}
			
			只要说明$R\otimes_kS$中的全部在$(x,y)$处取零的元构成的理想恰好就是$m_x\mathscr{O}_{Y,y}+m_y\mathscr{O}_{X,x}$:一方面这里的元自然在$(x,y)$处取零;另一方面任取$h=\sum_if_i(X)g_i(Y)$使得它在$(x,y)$处取零,记$f_i(x)=a_i$和$g_i(y)=b_i$,那么$\sum_ia_ib_i=0$,于是$h$落在这个理想中:
			$$\sum_if_i(X)g_i(Y)=\sum_if_i(X)g_i(Y)-\sum_ia_ib_i=\sum_i(f_i(X)-a_i)g_i(Y)+\sum_ia_i(g_i(Y)-b_i)\in m_x\mathscr{O}_{Y,y}+m_y\mathscr{O}_{X,x}$$
		\end{proof}
		\item $\mathbb{A}^m_k$和$\mathbb{A}^n_k$的积就是$\mathbb{A}^{m+n}_k$.这件事说明尽管仿射簇的积在集合层面是笛卡尔积,它在拓扑层面不是积.
	\end{enumerate}
	\item 预簇的积.
	\begin{enumerate}
		\item 设$X$和$Y$是$k$上的预簇,它们的纤维积存在.
		\begin{proof}
			
			先构造笛卡尔积集合$X\times Y$,对任意仿射开子集$U\subseteq X$和$V\subseteq Y$,对任意有限子集$f_i\in\mathscr{O}_X(U)$和$g_i\in\mathscr{O}_X(V)$,记集合$(U\times V)_{\sum_if_ig_i}$表示$U\times V$上所有不为$\sum_if_ig_i$零点的点构成的子集.我们断言全体这种形式的子集构成拓扑基.它覆盖了全空间因为每对仿射开子集的笛卡尔积可以表示为这种形式.另外$(U\times V)_{\sum_if_ig_i}\cap(U'\times V')_{\sum_jf_j'g_j'}$包含了每个$(U''\times V'')_{\sum_if_ig_i\sum_jf_j'g_j'}$,其中$U''\subseteq U\cap U'$和$V''\subseteq V\cap V'$都是仿射开子集.另外如果$U$和$V$是仿射的,这种拓扑基定义了纤维积$U\times V$上的拓扑.
			
			现在设$L$是$k(X)\otimes_kk(Y)$的商域(这个张量积是整环因为它是代数闭域上两个整环的张量积).任取$x\in X$和$y\in Y$,记$\mathscr{O}_{X,x}\otimes_k\mathscr{O}_{Y,y}$在理想$m_x\mathscr{O}_{Y,y}+m_y\mathscr{O}_{X,x}$处的局部化为$\mathscr{O}_{x,y}$.定义结构层为$\mathscr{O}_{X\times Y}(U)=\cap_{(x,y)\in U}\mathscr{O}_{x,y}$.并且这个结构层在每对仿射簇的纤维积$U\times V$上和仿射簇的纤维积的截面环是一致的.容易验证$X\times Y$是连通的并且可被有限个仿射簇覆盖,于是它是一个预簇.
			
			最后我们来验证它满足纤维积的泛性质.任取预簇的态射$r:Z\to X$和$s:Z\to Y$,按照$X\times Y$在集合层面是笛卡尔积,说明存在唯一的集合之间的映射$(r,s):Z\to X\times Y$使得$p_1\circ(r,s)=r$和$p_2\circ(r,s)=s$.现在仅需验证$(r,s)$是一个态射.任取仿射开子集$U\subseteq X$和$V\subseteq Y$,记$Z_{U,V}=p_1^{-1}(U)\cap p_2^{-1}(V)$.这种开集覆盖了整个$X\times Y$,于是只需验证$(r,s)\mid Z_{U,V}$是态射,但是按照这是仿射情况的纤维积,说明它是态射.
		\end{proof}
		\item 设$X_1$和$X_2$是两个预簇,设$Y_i$是$X_i$的子预簇,设$(X_1\times X_2,p_1,p_2)$是纤维积,那么$Y_1\times Y_2=p_1^{-1}(Y_1)\cap p_2^{-1}(Y_2)$是$X_1\times X_2$的子预簇.
	\end{enumerate}
	\item 射影簇的积.
	\begin{enumerate}
		\item 两个射影簇的纤维积仍然是射影簇.Segre嵌入.
		\begin{proof}
			
			射影簇是射影空间的闭子集,于是仅需验证$\mathbb{P}^n\times\mathbb{P}^m$可嵌入到$\mathbb{P}^{(n+1)(m+1)-1}$中.取$\mathbb{P}^n$上的齐次坐标$\{x_0,x_1,\cdots,x_n\}$和$\mathbb{P}^m$上的齐次坐标$\{y_0,y_1,\cdots,y_m\}$.再取$\mathbb{P}^{(n+1)(m+1)-1}$的齐次坐标为$\{z_{ij},0\le i\le n,0\le j\le m\}$.
			
			构造态射$\varphi:\mathbb{P}^n\times\mathbb{P}^m\to\mathbb{P}^{(n+1)(m+1)-1}$为,对每个$z=(x_0,\cdots,x_n,y_1,\cdots,y_m)\in\mathbb{P}^n\times\mathbb{P}^m$,那么$\varphi(z)\in\mathbb{P}^{(n+1)(m+1)-1}$定义为第$i,j$分量取$x_iy_j$的齐次坐标.这有意义是因为存在某个$x_i$和$y_j$非零,于是这个$i,j$分量非零.另外注意到$\varphi^{-1}(D(z_{ij}))=D(x_i)\times D(y_j)$.
			
			现在证明$\varphi$是单射.如果有$0\not=\lambda\in k$使得$x_iy_j=\lambda x_i'y_j',\forall i,j$.需要验证存在非零的$a,b\in k$使得$x_i=ax_i'$和$y_j=by_j'$.不妨设$x_0$和$y_0$都不为零,那么$x_0'$和$y_0'$也都不是零.此时有$x_i/x_0=x_iy_0/x_0y_0=x_i'y_0'/x_0'y_0'=x_i'/x_0'$.同理有$y_j/y_0=y_j'/y_0'$.于是得到$\varphi$是单射.
			
			再说明$\varphi$限制在$D(x_i)\times D(y_j)$上是到$D(z_{ij})$的某个闭子预簇的同构.不妨设$i=j=0$,在$D(x_0)$上有仿射坐标$s_i=x_i/x_0$,在$D(y_0)$上有仿射坐标$t_j=y_j/y_0$,在$D(z_{00})$上有仿射坐标$r_{ij}=z_{ij}/z_{00}$.在这些坐标下,$\varphi$把$(s,t)$映射为$(r_{ij})$,当$i,j\ge1$的时候$R_{ij}=s_it_j$;当$ij=0$的时候有$R_{i0}=s_i$和$R_{0j}=t_j$.于是$\varphi$的像集是$R_{ij}=R_{i0}R_{0j},i,j\ge1$确定的$D(z_{ij})$的闭子集.它的仿射坐标环是$k[r_{ij}]/(r_{ij}-r_{i0}r_{0j}),i,j\ge1$.这个环同构于$k[r_{i0},r{0j}],i,j\ge1$.于是在$\varphi^*$下这个环同构于$k[s_i,t_j]=k[s_i]\otimes_kk[t_i],i,j\ge1$,也即$D(x_0)\times D(y_0)$的坐标环.
			
			如果设$Z=\varphi(\mathbb{P}^n\times\mathbb{P}^m)$,我们证明了每个$Z\cap D(z_{ij})$都同构于一个仿射簇,于是这个同构是整体的,并且$Z$是$\mathbb{P}^{(n+1)(m+1)-1}$中的闭子集,这说明这个纤维积是射影代数集.
			
			最后容易验证像集$Z$恰好被齐次多项式组$\{z_{ij}z_{kl}=z_{kj}z_{il},0\le i,k\le n,0\le j,l\le m\}$所定义,容易验证这是不可约射影代数集,于是同构的$Z$也是不可约的,于是它是射影簇.
		\end{proof}
		\item Segre嵌入最简单的情况是$n=m=1$,此时嵌入$\mathbb{P}^1\times\mathbb{P}^1\to\mathbb{P}^3$的像集的齐次坐标环是$k[w,x,y,z]/(xy-zw)$.这说明$\mathbb{P}^3$中非退化的二次型定义的射影簇都同构于$\mathbb{P}^1\times\mathbb{P}^1$.
		\item 借助Segre嵌入得到:
		\begin{itemize}
			\item 一个子集$X\subset\mathbb{P}^n\times\mathbb{P}^m$是闭子簇当且仅当它被一组多项式$G_K(u_i,v_j),0\le i\le n,0\le j\le m,1\le k\le t$所定义,使得这里$G_k$分别视为$\{u_i\}$变量和$\{v_j\}$变量时(也即把另外一组变量视为常数)都是齐次多项式.
			\item 一个子集$X\subset\mathbb{P}^n\times\mathbb{A}^m$是闭子簇当且仅当它被一组多项式$G_K(u_i,v_j),0\le i\le n,1\le j\le m,1\le k\le t$所定义,使得这里$G_k$视为$\{u_i\}$变量时(也即把$\{v_j\}$变量视为常数)是齐次多项式.
			\item 类似的结论对$\mathbb{P}^{n_1}\times\cdots\times\mathbb{P}^{n_s}\times\mathbb{A}^{m_1}\times\mathbb{A}^{m_t}$成立.
		\end{itemize}
	\end{enumerate}
\end{enumerate}

簇.设$X$是预簇,它称为一个簇,如果对任意预簇$Y$和任意态射$f,g:Y\to X$,都有$\{y\in Y\mid f(y)=g(y)\}$都是$Y$的闭子集.这个额外的条件称为可分公理或者Hausdorff公理.
\begin{enumerate}
	\item 设预簇$X$上的的对角映射为两个恒等映射$\mathrm{id}_X:X\to X$确定的映射$\Delta:X\to X\times X$.记$\Delta(X)=\{z\in X\times X\mid p_1(z)=p_2(z)\}$,那么$X$是可分的当且仅当$\Delta(X)$是$X\times X$中的闭子集.
	\begin{proof}
		
		必要性,在可分定义中取$Y=X\times X$,取$f=p_1$和$g=p_2$,那么$\Delta(X)=\{(x,x)\mid x\in X\}$是$Y$的闭子集.充分性,任取$f,g:Y\to X$,它唯一决定了态射$\varphi:Y\to X\times X$使得如下图表交换,那么有$\{y\in Y\mid f(y)=g(y)\}=\varphi^{-1}(\Delta(X))$是闭子集.
		$$\xymatrix{&Y\ar@/^1pc/[ddr]^f\ar@/_1pc/[ddl]_g\ar[d]^{\varphi}&\\&X\times X\ar[dr]^{p}\ar[dl]_{q}&\\X&&X}$$
	\end{proof}
    \item 反例.取$X_1=X_2=\mathbb{A}_k^1$是两条仿射线,记坐标分别为$x_1$和$x_2$,记$U_i=X_i-\{x_i\},i=1,2$是两个开子集,将$X_1,X_2$按照恒等映射$U_1\to U_2$粘合,得到带两个原点的仿射线$X$.考虑两个态射$f_i:\mathbb{A}_k^1\to X$分别同构的映射到$X_1$和$X_2$,此时$\{x\in\mathbb{A}_k^1\mid f_1(x)=f_2(x)\}=U=\mathbb{A}^1_k-\{x\}$不是$\mathbb{A}_k^1$的闭子集.
	\item 设$Y$是簇$X$的子预簇,那么$Y$也是一个簇.这是因为$\Delta(Y)=\Delta(X)\cap Y\times Y$是$Y\times Y$中的闭子集.
	\item 两个簇的纤维积是簇,这是因为任取$f,g:Z\to X\times Y$,那么如下集合是两个闭集的交,于是是闭集.
	$$\{z\in Z\mid f(z)=g(z)\}=\{z\in Z\mid p_1f(z)=p_1g(z)\}\cap\{z\in Z\mid p_2f(z)=p_2g(z)\}$$
	\item 仿射簇是簇.这是因为任取预簇$Y$,任取两个态射$f,g:Y\to X$,那么$\{y\in Y\mid f(y)=g(y)\}$恰好是$\{sf-sg,s\in\Gamma(X,\mathscr{O}_X)\}$的公共零点集,于是这是闭子集.
	\item 上一条说明,可分定义中的$Z=\{z\in Y\mid f(z)=g(z)\}$总是局部闭子集.因为任取点$f(z)=g(z)$的仿射开子集$V$,记$Z=\{z\in Z\mid f(z)=g(z)\}$,那么$Z\cap V$是$V$中的闭子集.
	\item 态射的图像.设$f:X\to Y$是预簇之间的态射,并且$Y$是簇,设$(\mathrm{id_X},f)$为纤维积泛性质确定的态射$X\to X\times Y$,它称为$f$的图像态射.它的像集记作$\Gamma_f$,那么它是同构于$X$的$X\times Y$的闭子预簇.这里逆映射为$p_1:X\times Y\to X$.这里我们仅需证明$\Gamma_f$是闭子集:
	\begin{proof}
		
		记$X,Y$的积对象是$(X\times Y,p_X,p_Y)$,考虑两个态射$g=p_Y,h=f\circ p_X:X\times Y\to Y$,那么有$\{(x,y)\in X\times Y\mid g(x,y)=h(x,y)\}=\{(x,y)\in X\times Y\mid y=f(x)\}=\Gamma_f$,按照可分公理得到这是$X\times Y$的闭子集.
	\end{proof}
	\item 射影簇是一个簇.这里我们证明更一般的结论:如果$X$是预簇,如果任意的$x,y\in X$都有仿射开子集包含了这两个点,那么$X$是一个簇.于是对于射影簇,它的任意两个点都存在一个超平面不包含这两个点,这个超平面由一个一次齐次多项式所定义,它的补集是一个主开集,是仿射的.
	\begin{proof}
		
		假设$X$不是可分的,也即可取两个态射$f,g:Y\to X$使得$Z=\{z\in Y\mid f(z)=g(z)\}$不是闭集.任取$z\in\overline{Z}$,记$x=f(z)$和$y=g(z)$.按照条件,存在仿射开子集$V$同时包含了$x$和$y$,记$V=f^{-1}(V)\cap g^{-1}(V)$.按照$V$是仿射的,它是可分的,于是$Z\cap U=\{z\in U\mid f(z)=g(z)\}$在$U$中闭.于是$z\in Z$.
	\end{proof}
    \item 设$X$是不可约簇,设$U,V$是两个仿射开子集,它们的坐标环分别是$R$和$S$,那么$U\cap V$是坐标环为$RS\subseteq k(X)$的仿射簇.
    \begin{proof}
    	
    	$U\times V$是$X\times X$的仿射开子集,它的坐标环为$R\otimes_kS$.记$Z=\Delta(X)$,那么$Z\cap(U\times V)$是仿射簇$U\times V$的闭子集,并且经$\Delta$同构于$U\cap V$.按照$U\cap V$是不可约的,于是$Z\cap(U\times V)$是不可约的,于是它是$U\times V$的闭子簇,于是按照仿射簇的闭子簇是仿射簇,得到$U\cap V$是仿射簇.
    	
    	最后求$U\cap V$的坐标环$T$,按照$U\cap V$同构于$Z$的仿射开子集,说明$T$是$R\otimes_kS\to k(Z),(r\otimes s)\mapsto rs$的像,此即$RS$.
    \end{proof}
\end{enumerate}
\subsection{有限映射}

有限映射.设$X,Y$是仿射簇,一个正则映射$f:X\to Y$等同于给出一个环同态$f^*:k[Y]\to k[X]$,这个同态使得$k[X]$是有限型$k[Y]$代数,于是$f^*$是整同态等价于$k[X]$是$k[Y]$上有限模,如果这个等价条件成立,就称这个正则映射$f:X\to Y$是有限映射.
\begin{enumerate}
	\item 整同态的复合还是整同态,于是两个有限映射的复合是有限映射.
	\item 有限映射具有有限纤维.换句话讲如果$f:X\to Y$是有限映射,任取$y\in Y$,那么$f^{-1}(y)$要么是空集要么是有限点集.注意下一条会证明空集情况也不会发生.
	\begin{proof}
		
		设$X\subset\mathbb{A}^n$,记仿射空间的坐标为$\{t_1,t_2,\cdots,t_n\}$,把它们视为$X$上的正则函数,归结为证明每个$t_i$在集合$f^{-1}(y)$上只取有限个点.按照有限映射的定义,存在$a_i\in k[Y]$使得$t_i^k+a_1t_i^{k-1}+\cdots+a_k=0$.任取$x\in f^{-1}(y)$,得到$t_i(x)^k+a_1(y)t_i(x)^{k-1}+\cdots+a_k(y)=0$,而这只有有限个点$x$满足这个方程.
	\end{proof}
    \item 有限映射总是闭映射.另外对于有限映射$f$,有$f^*$是单射当且仅当$f$是满射
    \begin{proof}
    	
    	设$I=(p_1,p_2,\cdots,p_r)$,设$V(I)$是$X$的闭子集,记$J=(f^*)^{-1}(I)$,那么$k[Y]/J\to k[X]/I$依旧是整同态,于是$f(V(I))=V(J)$.特别的取$I=(0)$得到$f(X)=Y$当且仅当$\ker f^*=0$,也即$f^*$是单射.
    \end{proof}
    \item 有限映射是一个局部性质:如果$f:X\to Y$是仿射簇之间的正则映射,如果对每个点$y\in Y$都有仿射开邻域$V$,使得$U=f^{-1}(V)$是仿射的,并且$f$限制在$U\to V$是有限映射,那么$f:X\to Y$是有限映射.
\end{enumerate}

预簇之间的正则映射$f:X\to Y$称为有限映射,如果对$Y$中每个点$y$都存在仿射簇开邻域$V$,使得$U=f^{-1}(V)$是仿射簇,并且$f$限制在$U\to V$上是仿射簇之间的有限映射.按照仿射簇之间的有限映射是局部性质,这个定义吻合于我们之前的仿射簇之间有限映射的定义.
\begin{enumerate}
	\item 有限映射是保复合的;有限映射具有有限纤维;有限映射总是满射.
	\item 如果$f:X\to Y$是预簇之间的正则映射,使得$f(X)$在$Y$中稠密(即支配条件),那么$f(X)$包含了$Y$的某个开子集.
	\begin{proof}
		
		问题归结为$X,Y$都是仿射簇的情况.支配条件导致$k[Y]\subseteq k[X]$.设域扩张$k(Y)\subseteq k(X)$的超越维数是$r$,选取$\{u_1,u_2,\cdots,u_r\}\subseteq k[X]$是$k(Y)$上的代数无关集.于是$k[Y]\subseteq k[Y][u_1,u_2,\cdots,u_r]\subseteq k[X]$,并且$k[Y][u_1,u_2,\cdots,u_r]=k[Y\times\mathbb{A}^r]$.于是$f$被分解为$g\circ h$,其中$h:X\to Y\times\mathbb{A}^r$和$g:Y\times\mathbb{A}^r\to Y$.
		
		每个$v\in k[X]$都是$k[Y\times\mathbb{A}^r]$上的代数元,于是可取$a\in k[Y\times\mathbb{A}^r]$使得$av$是$k[Y\times\mathbb{A}^r]$上的整元.记$k[X]$上的坐标函数为$v_1,v_2,\cdots,v_m$,选取$a_i\in k[Y\times\mathbb{A}^r]$使得每个$a_iv_i$都是$k[Y\times\mathbb{A}^r]$的整元.记$a=a_1a_2\cdots a_m$,那么每个函数$a_i$都是主开集$D(a)\subseteq Y\times\mathbb{A}^r$的可逆函数,于是$D(h^*(a))\subseteq X$上的函数$v_i$都是$k[Y\times\mathbb{A}^r][1/a]$上的整元.这说明限制映射$h:D(h^*(a))\to D(a)$是有限映射,我们解释过有限映射都是满射,于是$h(D(h^*(a)))=D(a)$,于是$D(a)\subseteq h(X)$.最后问题归结为$g(D(a))$包含了$Y$的某个开集.
		
		记$a=F(y,T)=\sum F_{\alpha}(y)T^{\alpha}$,其中$y$是$Y$上坐标函数组,$T$是$\mathbb{A}^r$是坐标函数组.考虑那些使得$F_{\alpha}(y)$不全为零的点$y$,都可以选取$T$的取值$\tau$使得$F(y,\tau)\not=0$,这导致$\cup D(F_{\alpha})\subseteq g(D(a))$.完成证明.
	\end{proof}
\end{enumerate}

诺特正规化引理.
\begin{enumerate}
	\item 设$X$是射影簇,那么存在它到某个射影空间的有限映射.
	\begin{proof}
		
		不妨设$X\subsetneqq\mathbb{P}^n$,那么可选取$x\in\mathbb{P}^n-\{x\}$,记$x$被$L_0=L_1=\cdots=L_{n-1}=0$定义,那么可构造正则映射$\varphi:X\to\mathbb{P}^{n-1}$为$z\mapsto[L_0(z):L_1(z):\cdots:L_{n-1}(z)]$.那么$\varphi(X)\subset\mathbb{P}^{n-1}$是射影簇,并且$\varphi:X\to\varphi(X)$是有限映射.如果$\varphi(X)\subsetneqq\mathbb{P}^{n-1}$继续做相同的事情,有限步骤后会终止,此时构造的这些有限映射的复合是有限映射,并且它把$X$映入某个射影空间.
	\end{proof}
	\item 设$X$是不可约仿射簇,那么存在它到某个仿射空间的有限映射.
	\begin{proof}
		
		设$X\subset\mathbb{A}^n\subset\mathbb{P}^n$,设它的射影闭包为$\overline{X}$,按照上一条有有限映射$\varphi:\overline{X}\to\mathbb{P}^m$,它限制为$X\to\mathbb{A}^m$的有限映射.
	\end{proof}
\end{enumerate}
\subsection{维数}

仿射簇的维数.预簇的维数都定义为它作为拓扑空间的组合维数,也即不可约闭集严格包含链长度的上确界.
\begin{enumerate}
	\item 按照对应定理,仿射代数集的不可约闭子集和坐标环的素理想是反序一一对应的,于是仿射代数集的维数恰好是仿射坐标环的Krull维数.例如按照$\dim k[x_1,x_2,\cdots,x_n]=n$,得到$\mathbb{A}^n$的维数就是$n$.另外给定$\mathrm{A}^n$中的代数簇$Y$,设它对应的$R=k[x_1,x_2,\cdots,x_n]$中的素理想是$\mathfrak{p}$,那么有$\mathrm{ht}(\mathfrak{p})+\dim Y=n$.
	\item 诺特正规化引理告诉我们,如果$X$是仿射簇,那么它的维数恰好是扩张维数$[k(X):k]$.
	\item 诺特正规化引理还告诉我们,对仿射簇$X$,存在唯一的$n$使得存在有限满映射$X\to\mathbb{A}^n$,并且这里$n$恰好就是$X$的维数.进而如果$X,Y$分别是维数是$n,m$的仿射簇,那么$X\times Y$的维数是$n+m$.
	\begin{proof}
		
		按照上一条存在有限满映射$X\to\mathbb{A}^n$和$Y\to\mathbb{A}^m$,那么积映射$X\times Y\to\mathbb{A}^{n+m}$就也是有限满映射,并且这里$X\times Y$仍然是仿射簇,于是上一条说明$X\times Y$的维数是$n+m$.
	\end{proof}
	\item 对仿射代数集的开子集$Y$有$\dim Y=\dim\overline{Y}$.于是特别的,仿射簇的开子集的维数和仿射簇的维数相同.
	\begin{proof}
		
		按照$Y$是$\overline{Y}$的子空间,我们已经证明过了$\dim Y\le\dim\overline{Y}$.不妨约定$\dim Y$有限,否则此时两个维数都是无穷,此时约定等式成立.设$Z_0\subseteq Z_1\subset\cdots\subseteq Z_n$是$Y$中极大长度的不可约闭集链,于是$\dim Y=n$.我们断言$\overline{Z_0}\subset\overline{Z_1}\subset\cdots\subset\overline{Z_n}$是$\overline{Y}$中的不可延长的严格包含的不可约闭集链.它是严格包含的是因为$Z_i=\overline{Z_i}\cap Y$.现在说明它不可延长,如果有不可约闭集$W\subset\overline{Y}$满足严格包含链$\overline{Z_i}\subseteq W\subset\overline{Z_{i+1}}$,那么$W\cap Y$是$Y$中的不可约闭集,于是从$Z_i\subseteq W\cap Y\subseteq Z_{i+1}$得到这两个包含符号必然有一个是等号.如果$W\cap Y=Z_{i+1}$,那么$Z_{i+1}\subseteq W$,这里$W$是$\overline{Y}$中闭集,于是$\overline{Z_{i+1}}=W$和$W$的选取矛盾;如果$W\cap Y=Z_i$,按照$Y$是准仿射簇,存在开集$U$使得$Y=\overline{Y}\cap U$,于是$Z_i=W\cap U$是不可约集$W$的开子集,于是它是$W$的稠密子集,得到$\overline{Z_i}\cap W=W$,导致$\overline{Z_i}=W$同样和$W$的选取矛盾.
		
		\qquad
		
		现在$Z_0$必然是一个单点集$\{x\}$,否则可以添加单点集延长这个严格包含链的长度.现在$\{x\}$对应于$\overline{Y}$坐标环$A(\overline{Y})$中的极大理想$m$,于是得到$\dim\overline{Y}=\dim A(\overline{Y})=\mathrm{ht}(m)$.最后按照对应定理存在以$m$为最大元的长度$n$的素理想的严格包含链,并且这个链不可再延拓,于是按照$A(\overline{Y})$总是universal catenary的,导致$\mathrm{ht}(m)=n$,完成证明.
	\end{proof}
\end{enumerate}

射影簇的维数.给定射影代数集$Y$,它的齐次坐标环定义为$S(Y)=S/I(Y)$.
\begin{enumerate}
	\item 诺特正规化引理告诉我们,对于射影簇$Y$,存在唯一的$n$使得存在有限满映射$Y\to\mathbb{P}^n$,并且这里$n$恰好就是$Y$的维数.进而如果$X,Y$是两个维数分别为$n,m$的射影簇,那么$X\times Y$是维数是$n+m$的射影簇.
	\item 设$Y$是射影簇,记齐次坐标为$S(Y)$,那么$\dim S(Y)=\dim Y+1$.特别的有$\dim(\mathbb{P}^n)=n$.
	\begin{proof}
		
		记$\varphi_i$是我们之前定义的同构$U_i\to\mathbb{A}^n$,那么$Y_i=\varphi_i(Y\cap U_i)$.记$S(Y)_{x_i}$为$S(Y)$在点$x_i$处的分式化,这个分式化也是分次环,记$S(Y)_{x_i,0}$为分式化的零次元和零元构成的子环.我们解释过有$\mathbb{A}[Y_i]\cong S(Y)_{(x_i)}$.由此得到$\mathbb{A}[Y_i][x_0,x_0^{-1}]\cong S(Y)_{x_0}$,取商域得到$k(Y_i)(x_0)\cong\mathrm{Frac}(S(Y))$.按照$k$上有限生成整环的维数就是它商域在$k$上的超越维数,另外我们解释过不可约的预簇上每个仿射开子集的坐标环的商域都是函数域,得到$\dim\mathbb{A}[Y_i]+1=\dim S(Y)$.
	\end{proof}
	\item 设$Y$是射影簇,那么仿射锥$C(Y)$是仿射簇,此时有维数公式$\dim C(Y)=\dim Y+1$.
	\begin{proof}
		
		事实上按照$I(C(Y))=I(Y)$得到$\dim C(Y)=\dim k[X_0,X_1,\cdots,X_n]/I(C(Y))=\dim S/I(Y)$.其中最后一项就是$\dim S(Y)$,按照上一条得到它就是$\dim Y+1$.
	\end{proof}
	\item 设$Y\subset\mathbb{P}^n$是准仿射锥,那么有$\dim Y=\dim\overline{Y}$.
	\begin{proof}
		
		记$U_i$是齐次坐标中第$i$分量非零的$\mathbb{P}^n$中的点构成的开集.那么$U_i$同胚于$\mathbb{A}^n$.记$Y_i=Y\cap U_i$,那么$Y_i$在$U_i$中的闭包$Y_i'=\overline{Y}\cap U_i$.于是按照$U_i$是$\mathbb{P}$的开覆盖得到$\dim Y=\max_i\dim Y_i$.而我们证明过仿射情况中准仿射簇取闭包维数不变,于是$\max_i\dim Y_i=\max_i\dim Y_i'=\dim\overline{Y}$.
	\end{proof}
\end{enumerate}

预簇的维数.
\begin{enumerate}
	\item 不可约预簇的维数恰好是它函数域在$k$上的超越维数.我们解释过不可约预簇$X$上每个开子集的函数域都是$k(X)$本身,于是不可约预簇的维数恰好是它任意开子集的维数.
	\begin{proof}
		
		设$X$是不可约预簇,取它的仿射开覆盖$\{U_i\}$,我们解释过这里每个$U_i$的函数域都是$k(X)$.我们还证明过仿射情况下维数恰好是函数域在基域$k$上的超越维数,这说明每个$\dim U_i$都是$k\subseteq k(X)$的超越维数$n$.我们还解释过由于$\{U_i\}$是$X$的开覆盖,导致$\dim X=\sup_i\dim U_i$,于是$\dim X=n$.
	\end{proof}
	\item 设基域$k$是代数闭域,对于不可约预簇$X$如下三个条件等价:
	\begin{itemize}
		\item $\dim X=0$.
		\item 函数域$k(X)\cong k$.
		\item $X$是单点集.
	\end{itemize}
	\begin{proof}
		
		上一条直接得到1和2是等价的.3推1是直接的.最后1推3因为此时$X$的所有仿射开子集都是单点集,但是不可约条件要求非空开集的交总是非空的,导致$X$本身是仿射的.
	\end{proof}
\end{enumerate}

关于超曲面(krull主理想定理).$\mathbb{A}^n$或者$\mathbb{P}^n$的超曲面就定义为被单个非常值多项式所定义的代数集.一个预簇$X$的超曲面$Y$指的是存在$X$的非零的整体正则函数$F$,使得$Y$是闭子概型$\{x\in X\mid f(x)=0\}$(这是闭子集因为正则函数都是连续的),被$F$定义的超曲面记作$X_F$.
\begin{enumerate}
	\item Krull主理想定理.如果$R$是有限生成$k$整环,取$F\in R$,如果$\mathfrak{p}$是$(F)$的极小素理想,那么$\dim R-1=\dim R/\mathfrak{p}$.
	\item 不可约预簇上的超曲面总是等余维数1的.换句话讲它的每个不可约分支作为全空间的闭子集的余维数都是1(闭子集$Y\subseteq X$的余维数定义为$\dim X-\dim Y$).这件事基本上归结为仿射情况,而仿射情况基本上就是Krull主理想定理.
	\begin{proof}
		
		设$X$是不可约预簇$X'$的超曲面,任取$X'$的仿射开子集$U$,那么我们解释过$\dim X'=\dim U$.另一方面如果$Z$是$X$的不可约分支,那么只要$Z\cap U$非空它就是$U$的不可约分支,并且此时$Z\cap U$作为不可约预簇$Z$的非空开子集,就有$\dim Z=\dim Z\cap U$.于是问题归结为设$X'$本身已经是仿射簇.此时在上一条中把$R$取为$I(X')$,再取定义$X$的整体正则函数$F\in R$,这就得证.
	\end{proof}
	\item 反过来,如果$X$是仿射簇,满足$k[X]$是UFD(比方说$X=\mathbb{A}^n$),如果$X$的闭子集$Z$具有等余维数1,那么$Z$一定是$X$的超曲面.这件事基本上就是UFD的高度1素理想一定是主理想.结合上一条我们得到一个仿射簇$X\subseteq\mathbb{A}^n$具有等余维数1当且仅当$X$是$\mathbb{A}^n$的超曲面.另外去掉这里UFD的条件结论是不成立的:设$Q\subseteq\mathbb{P}^3$是非奇异二次射影曲面,那么存在这样的曲线$C\subseteq Q$不被单个方程所定义:设$Q$的方程是$G$,选取两条无交的曲线,分别被$F_1,F_2$定义,那么$G=F_1=F_2=0$理应没有解,但是我们在下文会看到$n+1$个变量的$n$个齐次方程总是有解的.
	\begin{proof}
		
		归结为设$Z$本身是不可约的.那么$I(Z)$是$k[T_1,\cdots,T_n]/I(X)$的高度1的素理想,但是按照UFD的高度1素理想是主理想,于是$I(Z)=(F)$.
	\end{proof}
	\item 射影锥$Y\subseteq\mathbb{P}^n$具有等余维数1当且仅当$Y$是$\mathbb{P}^n$的超曲面.
	\begin{proof}
		
		充分性上面证明过了,对于必要性,归结为设$Y$本身是不可约的,此时仿射锥$Y\subset\mathbb{P}^n$的维数是$n-1$,那么$\dim C(Y)=n$.于是$C(Y)\subset\mathbb{A}^{n+1}$是一个超曲面,于是它对应一个不可约多项式生成的主理想,进而$Y$对应的根理想是不可约的齐次多项式生成的理想.
	\end{proof}
	\item 设$X\subseteq\mathbb{P}^{n_1}\times\cdots\mathbb{P}^{n_k}$是拟射影簇,并且$X$在其中具有等余维数1,那么$X$是被单个方程定义的代数集,这个方程满足在$k$个变量集合上分别是齐次的.
\end{enumerate}

超曲面的交(krull高度定理).当我们考虑多个多项式定义的代数集时(因为希尔伯特基定理,我们总可以只考虑有限个多项式定义的代数集),等价于考虑多个超曲面的交.设$X\subset\mathbb{P}^N$是闭子集,取$F\in k[S_0,S_1,\cdots,S_N]$是一个齐次多项式,使得它在$X$上不恒为零,记$X_F$表示$X$的被$F=0$定义的闭子集.
\begin{enumerate}
	\item 设$X\subset\mathbb{P}^N$是射影代数集,对任意正整数$m$,都可以选取$m$次齐次多项式$G(u_0,u_1,\cdots,u_N)$使得在$X$的每个不可约分支上$G$恒不为零.
	\begin{proof}
		
		对$X$的每个不可约分支$X_i,1\le i\le l$选取点$x_i=[a_{i0},a_{i1},\cdots,a_{iN}]$,记$L_i=a_{i0}u_0+a_{i1}u_1+\cdots+a_{iN}u_N,1\le i\le l$.选取$k^{N+1}-\cup_{1\le i\le l}(L_i=0)$中的点$[c_0:c_1:\cdots:c_N]$,取$L=c_0u_0+c_1u_1+\cdots+c_Nu_N$,那么$L(x_i)\not=0,\forall 1\le i\le l$.那么$L^m$满足要求.
	\end{proof}
	\item 设$X$是射影簇,设$F$是$X$上的非零齐次多项式,那么$\dim X_F=\dim X-1$.后文会证明$X_F$是等余维数1的.
	\begin{proof}
		
		设$\dim X=n$.我们解释过不可约空间的真闭子集的维数必须严格小于全空间的维数,于是$\dim X_F=\dim X^{(1)}\le n-1$.可继续选取$X^{(1)}$上的在每个不可约分支上不恒为零的齐次多项式$F_1$,记$X^{(2)}=X^{(1)}_{F_1}$,那么有$\dim X^{(2)}\le\dim X^{(1)}-1$.继续构造下去,得到$F=F_0,F_1,\cdots,F_n$,使得$X^{(n+1)}=\emptyset$,于是$\{F_0,F_1,\cdots,F_n\}$没有公共零点.
		
		\qquad
		
		现在构造映射$\varphi:X\to\mathbb{P}^n$,$x\mapsto[F_0(x):F_1(x):\cdots:F_n(x)]$,这是一个有限满映射(因为$F_i$没有公共零点,证明可考虑Veronese嵌入).假设$\dim X^{(1)}<n-1$,那么$X^{(n)}=\emptyset$,于是$F_0,F_1,\cdots,F_{n-1}$没有公共零点,导致$[0:0:\cdots:0:1]$不在$\varphi(X)=\mathbb{P}^n$中,矛盾.
	\end{proof}
	\item 这个定理说明如果$X$是射影簇,如果$0\le n\le\dim X$,那么总存在$X$的维数为$n$的闭子簇,这个定理还说明射影簇的维数(按照超越维数定义的维数)恰好就是它不可约闭子集严格包含链长度的上确界.
	\item 设$X\subset\mathbb{P}^N$是$n$维射影簇,那么$N-n-1$是$\mathbb{P}^N$的极大维数的和$X$不交的线性子空间.
	\begin{proof}
		
		设$E\subset\mathbb{P}^N$是维数$s$的线性子空间.假设$s\ge N-n$,那么$E$可以被$\le n$个线性方程定义,反复使用$\dim X_F=\dim X-1$得到$\dim X\cap E\ge0$,导致$X\cap E$非空.于是$s\le N-n-1$.接下来可以选取1次齐次多项式组$\{L_0,L_1,\cdots,L_n\}$使得它们在$X$上没有公共零点,此时$E$的维数是$N-n-1$,并且$X\cap E$是空集.
	\end{proof}
	\item 另外这个定理还证明了Krull高度定理的射影簇版本:设$X\subset\mathbb{P}^n$是被$r$个齐次多项式定义的射影代数集,那么它的每个不可约分支的余维数不超过$r$.换句话讲如果$r\le n$,那么$\mathbb{P}^n$中$r$个齐次多项式必然有公共零点.特别的,这说明$\mathbb{P}^2$中任意两条曲线总是有交的.
	\item 推论.射影平面$\mathbb{P}^2$中被单个齐次多项式$F$定义的曲线的inflexion点是指$F=H(F)=0$的点,这里$H(F)$表示$F$诱导的Hessian矩阵的行列式,如果$\deg F=n\ge3$,那么$\deg H(F)=3(n-2)\ge1$,于是此时$F=H(F)=0$总有公共零点.换句话讲,射影平面中每个次数$\ge3$的曲线都有inflexion点.
\end{enumerate}

Tsen定理.
\begin{enumerate}
	\item Tsen定理.设$F(x_1,x_2,\cdots,x_n)$是$n$元$m$次齐次型,其中$m<n$,它的系数是$k[t]$中的元,那么$F(x_1,x_2,\cdots,x_n)=0$有多项式解$x_i=p_i(t)$.
	\begin{proof}
		
		设$x_i=\sum_{0\le j\le l}u_{ij}t^j$.带入$F(x_1,x_2,\cdots,x_n)$后得到一个关于$t$的多项式,它的所有系数理应为零,这得到一个方程组.如果$F$的所有系数视为$t$的多项式的次数的最大值为$k$,那么这个方程组至多有$ml+k+1$个方程.而这里未定元$u_{ij}$的个数有$n(l+1)$,由于$n>m$,选取足够大的$l$会使得得到$n(l+1)\ge ml+k+1$,于是按照我们的定理得到这有非零解.
	\end{proof}
	\item Tsen定理的例子.我们解释过单个多项式$q(x_0:x_1:x_2;t)=\sum_{0\le i,j\le2}a_{ij}(t)x_ix_j$可以定义$\mathbb{P}^2\times\mathbb{A}^1$上的一个二次曲面$X$,其中$a_{ij}(t)\in k[t]$,并且$t_0,t_1,t_2$是$\mathbb{P}^2$上坐标,$t$是$\mathbb{A}^1$上坐标.考虑映射$X\to\mathbb{A}^1$为$(x_0:x_1:x_2;a)\mapsto a$.每个点$a\in\mathbb{A}^1$的纤维是一个二次曲线$q(x_0:x_1:x_2;a)=0$,因此$X$称为一个二次曲线束.那么Tsen定理告诉我们存在正则映射$\varphi:\mathbb{A}^1\to X$,使得$\varphi(a)$是$a$的纤维中的点.
	\item 称二次曲线束非退化,如果定义中的$\det|a_{ij}(t)|$不恒为零.称二次曲线束有理,如果它双有理等价于$\mathbb{P}^2$,换句话讲它的函数域是$k(x;t)$.那么$\mathbb{A}^1$上的非退化二次曲线束总是有理曲面.
	\item 设$X$是射影簇,设$F$是$X$上不恒为零的齐次形式,那么$X_F$是纯余维数1的.
	\item 设$X\subset\mathbb{P}^N$是拟射影簇,设$F$是$X$上不为零的齐次形式,那么$X_F$是纯余维数1的.
	\begin{proof}
		
		$\overline{X}$是射影簇,$X$是它的开子集,我们解释过$\dim\overline{X}=\dim X$,记$\overline{X}_F=\cup_iY_i$,其中$\dim Y_i=\dim X-1$,那么$X_F=\cup_i(Y_i\cap X)$.其中$Y_i\cap X$要么是空集要么是$Y_i$的非空开子集,在第二种情况下$\dim Y_i\cap X=\dim X-1$,导致$X_F$是等余维数1的.
	\end{proof}
    \item 设$X\subset\mathbb{P}^N$是$n$维的拟射影簇,设$Y\subseteq X$是某个$m$齐次形式的零点集,那么$Y$的每个不可约分支的维数$\ge n-m$.
    \item 设$X,Y\subset\mathbb{P}^N$是两个拟射影簇,满足$\dim X=n$和$\dim Y=m$,那么$X\cap Y$的每个不可约分支都满足$\dim Z\ge n+m-N$.特别的,如果$X,Y$是射影簇使得$n+m\ge N$,那么$X\cap Y$非空.
    \item 更一般的,如果$Y_i$是预簇$X$的有限个子簇,那么有余维数的不等式:
    $$\mathrm{codim}_X\cap_{i=1}^rY_i\le\sum_{i=1}^r\mathrm{codim}_XY_i$$
\end{enumerate}

纤维的维数.考虑不可约预簇之间的正则映射$f:X\to Y$,任取$y\in Y$,纤维集$f^{-1}(y)$是$X$的闭子预簇.
\begin{enumerate}
	\item 设$f:X\to Y$是不可约预簇之间的满正则映射,设$\dim X=n$,$\dim Y=m$,那么对$f^{-1}(y)$的每个不可约分支都有$\dim F\ge n-m$,并且存在$Y$的非空开子集$U$使得$\dim f^{-1}(y)=n-m,\forall y\in U$.特别的,第二个结论说明$m\le n$.
	\begin{proof}
		
		不可约条件使得维数是一个仿射局部性质,于是不妨设$Y$是仿射的,设$Y\subset\mathbb{A}^N$,取$Y$的极大长度的不可约闭子集链$Y=Y^{(1)}\supsetneqq\cdots\supsetneqq Y^{(m+1)}=\emptyset$.于是这里$Y^{(m)}$是一个单点集$\{y\}$.于是$Y$上的单点$\{y\}$被$m$个都不为零的多项式$g_1,g_2,\cdots,g_m$定义,于是$f^{-1}(y)$被$m$个正则函数$f^*(g_i),1\le i\le m$定义.按照$f$是满射,得到$f^*$是单射,于是$f^*(g_i)$都不为零.按照我们证明的$\dim X_F=\dim X-1$,得到$\dim F\ge n-m$.
		
		\qquad
		
		问题归结为仿射情况,设$X,Y$都是仿射簇,此时未必有$f$是满射,但是它是支配映射.于是此时$f$诱导了单射$f^*:k[Y\to k[X]]$.记$k[Y]=k[y_1,y_2,\cdots,y_s]$和$k[X]=k[x_1,x_2,\cdots,x_t]$.按照$\dim Y=m$,$\dim X=n$,于是$m\le n$,并且$k[X]$在$k(Y)$上具有超越维数$n-m$.可不妨设$x_1,x_2,\cdots,x_{n-m}$在$k(Y)$上代数无关.那么其余的$v_i$是$k(Y)[x_1,x_2,\cdots,x_{n-m}]$的代数元.于是有关系式:
		$$F_i(x_i;x_1,x_2,\cdots,x_{n-m};y_1,y_2,\cdots,y_s)=0,\forall n-m+1\le i\le t$$
		
		记$x_i$在$f^{-1}(y)\cap V$上的限制为$\overline{x_i}$,那么有$k[f^{-1}(y)\cap X]=k[\overline{x_1},\cdots\overline{x_N}]$.把$F_i$视为关于变量$x_i,x_1,x_2,\cdots,x_{n-m}$的系数是关于$y_1,y_2,\cdots,y_s$的多项式,这定义了$Y$的闭子预簇$Y_i$.记$E=\cup_iY_i$,记$U=Y-E$,那么$U$是非空开集.按照代数无关性,任取$y\in U$有$F_i(T_i;T_1,T_2,\cdots,T_{n-m};y_1(y),\cdots,y_s(y))$不是恒为零的.于是每个$\overline{x_i}$在$\overline{x_1},\cdots,\overline{x_{n-m}}$上代数相关.这导致$\dim f^{-1}(y)\le n-m$,结合第一个结论得证.
	\end{proof}
    \item $Y_k=\{y\in Y\mid\dim f^{-1}(y)\ge k\}$总是$Y$上的闭子集.
    \item 设$f:X\to Y$是射影代数集之间的正则满映射,如果$Y$是不可约的,并且所有纤维$f^{-1}(y)$都是不可约并且维数全部相同,那么$X$是不可约的.
    \begin{proof}
    	
    	取不可约分支分解$X=\cup_iX_i$(诺特条件保证这是有限分解).我们解释过射影簇在正则映射下的像是闭集,于是这里$f(X_i)$都是$Y$的闭子集.按照$f$是满射得到$Y=\cup_if(X_i)$,按照$Y$是不可约的,说明存在某个$i$使得$Y=f(X_i)$.
    	
    	\qquad
    	
    	设所有纤维的维数是统一的$n$,如果$X_i$满足$Y=f(X_i)$,按照上面结论有非空开子集$U_i\subseteq Y$,使得$\dim f^{-1}(y)=n_i,\forall y\in U_i$.对于那些满足$f(X_i)\subsetneqq Y$的$X_i$,记$U_i=Y-f(X_i)$.考虑全部交$\cap U_i$,不可约性保证这是非空的,任取其中的点$y$,按照$f^{-1}(y)$是不可约的,导致存在某个$i$使得$f^{-1}(y)\subseteq X_i$,不妨设$i=0$.记$f_0:X_0\to Y$为$f$限制在$X_0$上.这导致$f^{-1}(y)=f_0^{-1}(y)$.于是有$n=n_0$.
    	
    	\qquad
    	
    	现在$f_0$是满射,对任意$y\in Y$,上面证明了$\dim f_0^{-1}(y)\ge n_0=n$.但是$f_0^{-1}(y)$是不可约空间$f^{-1}(y)$的闭子集,而$\dim f^{-1}(y)=n$,这迫使$f_0^{-1}(y)=f^{-1}(y)$.于是$X_0=X$,于是$X$是不可约的.
    \end{proof}
\end{enumerate}
\subsection{完全簇}

称一个簇$X$是完全簇,如果对任意簇$Y$都有投影映射$p:X\times Y\to Y$是闭映射.
\begin{enumerate}
	\item 我们来解释这个概念的动机.先回顾消除理论.设有一组$k$系数多项式
	$$\{f_i(X_0,\cdots,X_n;Y_1,\cdots,Y_m),1\le i\le r\}$$
	
	这些多项式关于第一组变量$\{X_0,\cdots,X_n\}$是齐次的.那么我们可以找到一组关于第二组变量的$k$系数多项式$\{g_j(Y_1,\cdots,Y_m),1\le j\le s\}$满足如下性质:设$(a_1,\cdots,a_m)\in\mathbb{A}_k^m$,那么$g_j(a_1,\cdots,a_m)=0,\forall1\le j\le s$当且仅当存在$(b_0,\cdots,b_n)\in\mathbb{P}^n$满足$f_i(b_0,\cdots,b_n;a_1,\cdots,a_m)=0,\forall1\le i\le r$.
	\item 这件事转化为仿射簇的语言就是说:设$\{f_i\}$定义了闭子集$X\subseteq\mathbb{P}^n\times\mathbb{A}^m$,设$p:\mathbb{P}^n\times\mathbb{A}^m\to\mathbb{A}^m$是投影映射,那么有$p(X)$是$\mathbb{A}^m$的闭子集.我们知道$\mathbb{P}^n\times\mathbb{A}^m$的闭子集一定可以表示为这样一组多项式$\{f_i\}$的零点集,于是这个消除定理等价于讲投影映射$p:\mathbb{P}^n\times\mathbb{A}^m\to\mathbb{A}^m$总是一个闭映射.
	\item 一些性质.
	\begin{enumerate}
		\item 如果$f:X\to Y$是簇之间的正则映射,$X$是完全簇,那么$f(X)$是$Y$的闭子集,它作为闭子簇是完全簇.
		\begin{proof}
			
			因为$Y$满足分离公理,于是图像$\Gamma_f=\{(x,f(x))\mid x\in X\}$是$X\times Y$的闭子集,于是按照$p:X\times Y\to Y$是闭映射,得到$f(X)$是$Y$的闭子集.于是问题归结为如果$f:X\to Y$是簇之间的满正则映射,并且$X$是完全簇,那么$Y$也是完全簇:任取簇$Z$,那么$g:X\times Z\to Y\times Z$也是满射,再记$p:Y\times Z\to Z$,任取$Y\times Z$的闭子集$F$,则$g^{-1}(F)$是$X\times Z$的闭子集,并且按照$g$是满射就有$g(g^{-1}(F))=F$,于是$g\circ p$是闭映射就得到$p\circ g(g^{-1}(F))=p(F)$是闭集,于是$p$是闭映射.
		\end{proof}
		\item 如果$X,Y$都是完全簇,那么$X\times Y$也是完全簇.因为$(X\times Y)\times Z\to Z$是闭映射的复合$X\times(Y\times Z)\to X\times Z\to Z$.
		\item 如果$X$是完全簇,那么它的闭子集$Y$也是完全簇.因为$Y\times Z$是$X\times Z$的闭子集,而闭映射在闭子集上的限制当然是闭映射.
		\item 不可约完全簇上的正则函数只有常值函数.
		\begin{proof}
			
			设$X$是不可约完全簇,它的正则函数就是正则映射$X\to\mathbb{A}^1$,但是我们解释了这个像集是$\mathbb{A}^1$的闭子集,并且不可约空间的连续像还是不可约的,但是$\mathbb{A}^1$的不可约闭子集只有单点和全集,倘若像集是全集,(a)就导致$\mathbb{A}^1$也是完全簇,但是这矛盾.
		\end{proof}
		\item 仿射簇是完全簇当且仅当它是单点集.
		\begin{proof}
			
			假设这个仿射簇存在至少两个点,那么至少存在一个坐标函数不是常值的,但是上一条解释了不可约完全簇的正则函数只有常值函数,这就说明仿射簇如果是完全簇必然只能是单点集.
		\end{proof}
	\end{enumerate}
    \item 射影簇是完全簇:设$X$是射影簇,$Y$是预簇(或者拟射影簇),投影映射$p:X\times Y\to Y$是闭映射,即它把闭子集映射为闭子集.
    \begin{proof}
    	
    	首先不妨设$X=\mathbb{P}^n$,因为如果$X\subset\mathbb{P}^n$是射影簇,那么$X\times Y$是$\mathbb{P}^n\times Y$的闭子集,于是如果$Z$是$X\times Y$的闭子集,那么它也是$\mathbb{P}^n\times Y$的闭子集.于是一旦证明投影映射$\mathbb{P}^n\times Y\to Y$是闭映射,就得到$X\times Y\to Y$是闭映射.
    	
    	\qquad
    	
    	其次不妨设$Y=\mathbb{A}^m$.因为按照$Y$有仿射簇的开覆盖,而闭集是一个局部性质,于是不妨设$Y$是仿射簇.但是如果仿射簇$Y\subset\mathbb{A}^m$,那么$\mathbb{P}^n\times Y$是$\mathbb{P}^n\times\mathbb{A}^m$的闭子集.于是不妨设$Y=\mathbb{A}^m$.
    	
    	\qquad
    	
    	我们解释过$\mathbb{P}^n\times\mathbb{A}^m$的闭子集$Z$可以被一组多项式$\{g_i(u;y),1\le i\le t\}$定义,每个多项式都是关于变量组$u$的齐次多项式.把投影映射在$Z$上的限制记作$p:Z\to\mathbb{A}^m$.任取$y_0\in\mathbb{A}^m$,那么$p^{-1}(y_0)$被多项式组$\{g_i(u;y_0),1\le i\le t\}$定义.于是问题归结为证明那些使得$\{g_i(u;y_0)\}$具有非零公共零点的$y_0\in\mathbb{A}^m$构成了一个闭子集$T$.
    	
    	\qquad
    	
    	我们还解释过$\{g_i(u;y_0)\}$具有公共零点当且仅当理想$I=\left(g_1(u;y_0),\cdots,g_t(u;y_0)\right)$不包含无关理想,也等价于$I$不包含每个$I_s,s\ge1$(这个$I_s$表示变量$u$构成的次数为$s$的单项式生成的$k$线性空间).于是问题归结为对每个$s\ge1$,那些使得$I$不包含$I_s$的点$y_0\in\mathbb{A}^m$构成的子集$T_s$是闭子集,这样得到$T=\cap_{s\ge1}T_s$是闭子集.
    	
    	\qquad
    	
    	记$g_i(u;y)$关于变量组$u$的次数为$k_i$,记$\{M^{\alpha}\}_{\alpha}$表示全体次数为$s$的关于变量组$u=\{u_0,u_1,\cdots,u_n\}$的单项式.于是条件$I_s\subseteq I$表示对每个这样的单项式$M^{\alpha}$都可表示为$\sum_{1\le i\le t}g_i(u;y_0)F_{i,\alpha}(u)$.取每个$F_{i,\alpha}$的齐次分支,不妨设每个$F_{i,\alpha}(u)$是次数为$s-k_i$的齐次多项式(如果$s<k_i$自然有$F_{i,\alpha}(u)=0$).另外按照全体不同单项式是一组基,于是不妨设$F_{i,\alpha}(u)$要么是0,要么是次数为$s-k_i>0$的单项式.于是问题归结为证明每个单项式$M^{\alpha}$可以表示为多项式组$\{g_i(u;y_0)N_i^{\beta}\}$的线性组合,这里$N_i^{\beta}$是次数为$s-k_i>0$的单项式.而这等价于讲全体$\{g_i(u;y_0)N_i^{\beta}\}$,其中$N_i^{\beta}$跑遍次数为$s-k_i>0$的单项式,生成了整个$s$次齐次多项式空间$I_s$.
    	
    	\qquad
    	
    	如果把向量组$\{g_i(u;y_0)N_i^{\beta}\}$表示为基$\{M^{\alpha}\}_{\alpha}$下的矩阵,那么使得$I_s\subseteq I$的点集$y_0$恰好是这个矩阵存在某个$\dim I_s$阶子式不为零的点构成的集合.于是使得$I$不包含$I_s$的点$y_0$构成的集合恰好是使得这个矩阵所有$\dim I_s$阶子式为零的点构成的集合,于是$T_s$被一组$u$系数多项式定义,于是$T_s$是$\mathbb{A}^m$的闭子集,完成证明.
    \end{proof}
    \item 设$X$是射影簇,设$f:X\to Y$是拟射影簇之间的正则映射,那么$f(X)$是$Y$的闭子集.事实上设$f$的图像为$\Gamma_f$,设投影映射$X\times Y\to Y$记作$p$,那么有$f(X)=p(\Gamma_f)$,我们解释过$Y$是簇的情况下$\Gamma_f$是$X\times Y$的闭子集,按照上一条得到$p(\Gamma_f)$是$Y$的闭子集,于是$f(X)$是$Y$的闭子集.
    \item 射影簇上的正则函数总是常值函数,换句话讲代数闭域$k$上射影簇上的正则函数环恰好就是$k$,这件事事实上我们已经证明过了,这里我们用上述定理给出第二种证明.
    \begin{proof}
    	
    	设$X$是射影簇,其上正则函数可视为正则映射$f:X\to\mathbb{A}^1$,取$\mathbb{A}^1$的射影闭包,复合典范的包含映射$\mathbb{A}^1\to\mathbb{P}^1$得到一个正则映射$\overline{f}:X\to\mathbb{P}^1$.于是按照上一条结论得到$\overline{f}(X)$是$\mathbb{P}^1$的闭子集,并且按照$f(X)=\overline{f}(X)$说明这个像集不包含无穷远点$x_{\infty}\in\mathbb{P}^1$.按照$\mathbb{A}^1$上是余有限拓扑,这说明像集$f(X)$要么是整个$\mathbb{A}^1$要么是有限点集.第一种情况导致$\overline{f}(X)$不是$\mathbb{P}^1$的闭子集.于是$f(X)$只能是有限点集$\{\alpha_1,\alpha_2,\cdots,\alpha_t\}$,于是$X=\cup f^{-1}(\alpha_i)$,于是$t>1$会和不可约性矛盾,导致$f(X)$是单点集,于是$f$是常值的.
    \end{proof}
    \item 一个从射影簇到仿射簇的正则映射$f:X\to Y$的像集是单点集.设$Y\subset\mathbb{A}^m$,那么$f$可以表示为$f(x)=(s_1(x),s_2(x),\cdots,s_m(x))$,这里每个$s_i$都是$X$上的正则函数,上一条说明了每个$s_i(x)$只能是常值函数,这导致$f$的像集是单点集.
\end{enumerate}
\newpage
\section{局部性质}
\subsection{切空间和奇点}

仿射代数集上的切空间.
\begin{enumerate}
	\item 交点的重数.设$X\subset\mathbb{A}^N$是仿射代数集.不妨设零点在$X$中,否则可以做平移.取$0\not=a\in\mathbb{A}^N$作为方向,该方向的直线为$L=\{ta\mid t\in k\}$.如果记$I(X)$被$\{F_1,F_2,\cdots,F_m\}$生成,那么$X\cap L$被$F_1(ta)=\cdots=F_m(ta)=0$所定义.如果记$f(t)$为所有多项式$\{F(ta)\mid F(T)\in I(X)\}$的最大公因式,等价于$\{F_1(ta),F_2(ta),\cdots,F_m(ta)\}$的最大公因式(于是不依赖于$I(X)$生成元集的选取),那么$t=0$在$f(t)$中的重数理解为$0$作为$X$和$L$的交点$\{0\}$的重数.另外如果$X$本身是仿射空间,此时所有$F_i\equiv0$,那么对任意次数$r$都有$t^r\mid F_i(ta)$,此时就约定交点$\{0\}$的重数是$+\infty$.
	\item 约定同上,如果$X$和$L$在点$\{0\}$的重数$\ge2$,我们就称直线$L$(或者理解为方向$a$)是0处的切线.全体切线构成的集合称为切空间,记$X$在点$x$处的切空间为$\mathrm{T}_{X,x}$.
	\item 设$\{F_1,F_2,\cdots,F_m\}$生成了整个$I(X)$,设$0\in X$,那么每个$F_i(T)$的常数项都是0,可记$F_i=L_i+G_i$,其中$L_i$的次数恰为1,而$G_i$的次数$\ge2$,那么有$F_i(at)=tL_i(a)+G_i(ta)$,其中$G_i(ta)$被$t^2$整除.于是方向$a$是0的切线当且仅当$L_i(a)=0,\forall i$.这说明0处切空间就是$L_1(a)=L_2(a)=\cdots=L_m(a)=0$定义的仿射集,并且由于这些$L_i$都是一次的,所以切空间的确是$\mathbb{A}^n$的线性子空间.
	\item 仿射代数集上正则函数的微分.任取$A[X]$中的元$g$,它是某个多项式在$X$上的限制,选取这样的多项式$G$,定义$d_xg=d_xG$,那么这个定义不依赖于表示$g$的多项式$G$的选取.这是因为同一个$g$的不同选取相差一个$(F_1,F_2,\cdots,F_m)$中的元$F$,于是可记$F=A_1F_1+A_2F_2+\cdots+A_mF_m$,那么从$F_i(x)=0$得到$d_xF=A_1(x)d_xF_1+\cdots+A_m(x)d_xF_m$,但是又有$d_xF_i=0$,导致$d_xF=0$.这样定义的$d_xg$称为正则函数$g$的微分.它同样满足微分的两条性质$d_x(f+g)=d_xf+d_xg$和$d_x(fg)=f(x)d_xg+g(x)d_xf$.
	\item 我们断言微分映射$\mathrm{d}_x$是从$\mathfrak{m}_x/\mathfrak{m}_x^2$到$x\in X$处切空间$\mathrm{T}_{X,x}$的对偶空间的同构.
	\begin{proof}
		
		$d_xg$可视为$T_{X,x}$(经典定义)上的线性函数.于是$d_x$诱导了$\mathfrak{n}_x=\{f\in k[X]\mid f(x)=0\}$到$T_{X,x}^*$(对偶空间)的同态.我们断言这是一个满同态,并且核为$\mathfrak{n}_x^2$,换句话讲$d_x$诱导了同构$\mathfrak{n}_x/\mathfrak{n}_x^2\cong T_{X,x}^*$.另外$A[X]$在$x$对应极大理想的局部化就是$\mathscr{O}_{X,x}$,于是有$\mathfrak{n}_x/\mathfrak{n}_x^2\cong\mathfrak{m}_x/\mathfrak{m}_x^2$,于是这两种定义是等价的.
		
		\qquad
		
		证明满射:任取$T_{X,x}$上的线性函数$\sum_{1\le i\le N}\lambda_i(T_i-x_i)$,这个线性函数本身作为$k[X]$中的元的微分是它本身.证明核是$\mathfrak{n}_x^2$:不妨设$x=0$,设$g\in n_x$满足$d_xg=0$,选取$g$的一个多项式表示$G$,按照定义$d_xG=0$,而切空间被$\{d_xF_i\}$定义,它们都是线性的,于是存在一些系数$\lambda_i$使得$d_xG=\sum_{1\le i\le m}\lambda_id_xF_i$.记$G_1=G-\sum_{1\le i\le m}\lambda_iF_i$,那么$G_1$没有不超过1次的项,导致$G_1\in(T_1,T_2,\cdots,T_m)^2$.但是$G_1$和$G$在$X$上的限制相同,于是$G_1$也是$g$的一个多项式表示,于是$g\in(t_1,t_2,\cdots,t_N)^2=n_x^2$,其中$t_i$是$T_i$在$X$上的限制.
	\end{proof}
    \item 切空间的内蕴定义.按照上一条结论,我们称$k$线性空间$\mathfrak{m}_x/\mathfrak{m}_x^2$为点$x$的余切空间,称$\mathfrak{m}_x/\mathfrak{m}_x^2$的对偶空间为$x\in X$处的切空间,记作$\mathrm{T}_{X,x}$.由此我们可以定义预簇上一个点的切空间.
    \item 切映射.设$f:X\to Y$是两个仿射代数集之间的正则映射,那么$f$诱导了一个环同态$f^*:k[Y]\to k[X]$,如果取$x\in X$,$y=f(x)$,那么自然有$f^*(m_y)\subseteq m_x$,于是$f^*$诱导了环同态$m_y/m_y^2\to m_x/m_x^2$.它的对偶映射是切空间之间的同态$T_{X,x}\to T_{Y,y}$,这称为$f$诱导的切映射,记作$d_xf$.于是从预簇到切空间和从正则映射到切映射构成了一个函子.
    \item 另外内蕴定义中我们把切空间定义为基域上的一个线性空间,没有视为某个大空间的子空间.但是如果$X$是仿射代数集,选取嵌入$X\subset\mathbb{A}^N$,那么切映射诱导了切空间的嵌入$T_{X,x}\to T_{\mathbb{A}^N,0}=\mathbb{A}^N$,此时切空间的确可视为仿射空间的子空间,这也就是仿射代数集上切空间的经典定义.
\end{enumerate}

奇点和正则点.
\begin{enumerate}
	\item 切丛.设$X\subset\mathbb{A}^N$是仿射簇,考虑$\mathbb{A}^N\times X$的子集$TX=\{(a,x)\mid x\in X,a\in T_{X,x}\}$,按照切空间的多项式描述,$TX$是$\mathbb{A}^N\times X$的一个闭子集,于是它是仿射簇,它称为$X$的切丛.
	\item 仿射簇上正则和奇异的定义.存在正则映射$TX\to X$为投影映射在$TX$上的限制$(a,x)\mapsto x$.按照纤维上维数的性质,我们有:记$s=\min_{x\in X}\dim T_{X,x}$,那么使得$\dim T_{X,x}=s$的点$x\in X$构成了一个非空开子集,这样的点称为正则点.于是使得$\dim T_{X,x}>s$的点$x$构成了$X$的闭子簇,这样的点称为奇异点.
	\item 下一条会证明对于不可约预簇,在正则点定义中的$s$恰好是$\dim X$.另外对于仿射簇$X$,我们有$\dim X=\dim\mathscr{O}_{X,x}$,于是对于仿射簇,一个点$x\in X$是正则点当且仅当$\dim\mathrm{T}_{X,x}=\dim\mathscr{O}_{X,x}$,此即$\mathscr{O}_{X,x}$是正则局部环.
	\item 对于不可约预簇$X$,有$\dim X=\min_{x\in X}\dim T_{X,x}$.对于一般可约的预簇这个结论未必成立,例如取$X=X_1\cup X_2$,使得$\dim X_1=1,\dim X_2=2$,如果$x\in X_1-X_2$是$X_1$的一个非奇点,那么$\dim\mathrm{T}_{X,x}=1$,但是$\dim X=2$.出现这种情况是因为那些不经过点$x$的不可约分支不影响$\dim\mathrm{T}_{X,x}$但是影响$\dim X$.
	\begin{proof}
		
		这个结论是局部的,于是不妨设$X$是仿射簇.设$s=\min_{x\in X}\dim T_{X,x}$.我们解释过存在双有理映射$\varphi:X\to Y$,其中$Y$是某个仿射空间的超曲面.于是存在$X$的开子集$U$和$Y$的开子集$V$使得它们是同构的.于是倘若我们证明了超曲面满足这个性质,那么$\min_{x\in U}\dim T_{X,x}=\min_{y\in V}\dim T_{Y,y}=\dim Y=\dim X$.
		
		\qquad
		
		现在设不可约超曲面$X$被不可约多项式$F$定义,那么点$x$处切空间被$\sum_{1\le i\le n}\frac{\partial F}{\partial T_i}(x)(T_i-x_i)=0$定义.于是为证明此时$s=\min_{x\in X}\dim T_{X,x}=n-1$,其中$\dim X=n$,只需说明这里$\frac{\partial F}{\partial T_i}(x)$不全为零.但是如果它全为零,特征零的情况下有$F$是常数,它不是不可约多项式;在特征$p>0$的情况下有$F=F_1^p$(这需要基域$k$是代数闭的,但是我们总假定这件事),这同样和$F$不可约矛盾.
	\end{proof}
	\item 正则点的经典定义.设$X\subset\mathbb{A}^n$是仿射簇,设$F_1,F_2,\cdots,F_t$是定义$X$的多项式组,设$\dim X=r$,那么点$x\in X$是正则点当且仅当矩阵$\left(\frac{\partial F_i}{\partial T_j}(x)\right)_{1\le i\le t,1\le j\le n}$的秩是$n-r$.
	\begin{proof}
		
		设$\theta$是从$k[T_1,T_2,\cdots,T_n]$的极大理想$n_x=(T_1-x_1,T_2-x_2,\cdots,T_n-x_n)$到$k^n$的映射,定义为把多项式$f$映射为$\left(\frac{\partial f}{\partial T_1}(x),\frac{\partial f}{\partial T_2}(x),\cdots,\frac{\partial f}{\partial T_n}(x)\right)$.这诱导了同构$n_p/n_p^2\cong k^n$.现在把映射$\theta$限制在$I(X)$上,那么$I(X)$中的多项式$F(T_1,T_2,\cdots,T_n)$可以表示为$f(F_1,F_2,\cdots,F_t)$,这使得映射$\theta$可视为先把$F$映射为$\left(\frac{\partial f}{\partial F_1}(x),\frac{\partial f}{\partial F_2}(x),\cdots,\frac{\partial f}{\partial F_t}(x)\right)^t$,再左乘这里的矩阵$\left(\frac{\partial F_i}{\partial T_j}(x)\right)$.于是这个矩阵的秩就是$\theta(I(X))$作为$k^n\cong n_x/n_x^2$的$k$线性子空间的秩.考虑如下$k$线性空集的短正合列:
		$$\xymatrix{0\ar[r]&\frac{I(X)+n_x}{n_x^2}\ar[r]&\frac{n_x}{n_x^2}\ar[r]&\frac{n_x}{I(X)+n_x^2}\ar[r]&0}$$
		
		这里$x$处的局部环就是$R=k[T_1,T_2,\cdots,T_n]$商去$I(X)$后对$n_x$做局部化.于是如果记$\mathscr{O}_{X,x}$极大理想为$m_x$,那么$m_x/m_x^2\cong n_x/(I(X)+n_x^2)$.于是上述短正合列得到这个矩阵的秩是$\dim_kn_x/n_x^2-\dim_km_x/m_x^2=n-\dim_km_x/m_x^2$,于是它的秩为$n-r$当且仅当$\dim_km_x/m_x^2=\dim\mathbb{A}[X]$,也即当且仅当$x$是正则点.
	\end{proof}
    \item 一般预簇上的正则和奇异的定义.设$X$是预簇,点$x\in X$称为正则点,如果满足$\dim_k\mathrm{T}_{X,x}=\dim\mathscr{O}_{X,x}$,这也等价于讲$\mathscr{O}_{X,x}$是正则局部环.如果$X$的每个点都是正则的,则称$X$是非奇异预簇.
\end{enumerate}

参数系统.设$X$是预簇,设$x\in X$是正则点,设$n=\dim X$,一组元$u_1,u_2,\cdots,u_n\in\mathfrak{m}_x$称为$x$处的局部参数系统,如果它们在$\mathfrak{m}_x/\mathfrak{m}_x^2$中的像构成线性空间的一组基,这里$\mathfrak{m}_x$是$\mathscr{O}_{X,x}$的极大理想.
\begin{enumerate}
    \item 按照NAK引理,点$x$处的参数系统生成了极大理想$\mathfrak{m}_x$(于是对于一般的点$x\in X$,它的切空间维数,也称为嵌入维数,是不小于局部环的维数的,当二者相等时该点恰为正则点).
	\item 设$x\in X$是预簇的正则点,那么$u_1,\cdots,u_n\in\mathfrak{m}_x$是一组参数系统当且仅当$\mathrm{d}_xu_1,\cdots,\mathrm{d}_xu_n$是$\mathrm{T}_{X,x}$上线性无关的线性函数.换句话讲方程组$\mathrm{d}_xu_1=\cdots=\mathrm{d}_xu_n=0$在$\mathrm{T}_{X,x}$上只有零解.
	\item 设$x\in X$是预簇的正则点,设$u_1,\cdots,u_n$是$x$的参数系统,并且我们约定这些$u_i$都在$X$上正则.记$X_i=V(u_i)$,那么我们断言$x$在每个$X_i$中都是正则的,并且如果记$X_i$在点$x$的切空间为$\mathrm{T}_{X_i,x}$,则有$\cap_i\mathrm{T}_{X_i,x}=\{0\}$.另外存在$x$的开邻域满足这些$X_i$的交只有点$x$本身.
	\begin{proof}
		
		因为切空间和正则点都是局部性质,我们用$x$的仿射簇开邻域代替$X$不会影响命题,于是不妨设$X$本身是仿射代数集,那么此时$u_1,\cdots,u_n$都是$X$上的正则函数.设$u_i=0$定义的$X$的超曲面为$X_i$,记$I=I(X)$和$I_i=I(X_i)$,取$u_i$作为$X$上正则函数所对应的一个多项式为$F_i$,那么有$(I,F_i)\subseteq I_i$,这个包含关系说明,如果记$\mathrm{d}_xF_i=0$定义的直线$L_i\subseteq\mathrm{T}_{X,x}$,那么有$\mathrm{T}_{X_i,x}\subseteq L_i$.按照$\mathrm{d}_xu_i$是余切空间中的非零元,得到$L_i\not=\mathrm{T}_{X,x}$,于是$L_i$作为切空间的超曲面,维数必然是$n-1$,于是得到$\dim\mathrm{T}_{X_i,x}\le\dim L_i=n-1$.反过来有$\dim\mathrm{T}_{X_i,x}\ge\dim X_i=n-1$(最后一个等式是因为$X_i$作为$X$的超曲面是等余维数1的),综上得到$\dim\mathrm{T}_{X_i,x}=n-1$.进而有$\dim\mathrm{T}_{X_i,x}=\dim X_i$,于是$x$在每个$X_i$内都是正则点.这里$\mathrm{T}_{X_i,x}$就是$\mathrm{d}_xu_i=0$在$\mathrm{T}_{X,x}$上的零点集,所以从$\{u_1,\cdots,u_n\}$构成参数系统得到$\mathrm{d}_xu_1=\cdots=\mathrm{d}_xu_n=0$只有零解,此即$\cap_i\mathrm{T}_{X_i,x}=\{0\}$.最后假设$\cap_iX_i$不止含有点$x$,那么存在这个交的某个不可约分支$Y$满足$\dim Y>0$,但是这导致维数非零的空间$\mathrm{T}_{Y,x}$落在$\cap_i\mathrm{T}_{X_i,x}=\{0\}$中,这矛盾.
	\end{proof}
\end{enumerate}

横截.设$X$是预簇,设$Y_1,\cdots,Y_r$是它有限个子预簇(预簇$X$的子预簇指的是形如$U\cap F$的子集,其中$U$是$X$的开集,$F$是$X$的闭集,簇结构是这样取的,先取$F$为$X$的闭子预簇,再取$U\cap F$为$F$的开子预簇),称这些子预簇在一个$X$上的正则点$x\in\cap_iY_i$处是横截(transversal),如果有余维数等式:
$$\mathrm{codim}_{T_{X,x}}\cap_{i=1}^rT_{Y_i,x}=\sum_{i=1}^r\mathrm{codim}_XY_i$$
\begin{enumerate}
	\item 例如一个非奇异曲面上的两条曲线在某个交点处是横截当且仅当这两条曲线在交点位置的切线是不同的.
	\item 设$X$的子预簇$\{Y_1,\cdots,Y_r\}$在点$x\in Y=\cap_iY_i$处是横截,那么我们断言$x$在每个$Y_i$中是正则点,并且$x$在$Y$中耶是正则点.
	\begin{proof}
		
		我们之前解释过对子预簇恒有余维数不等式$\mathrm{codim}_X\cap_{i=1}^rY_i\le\sum_{i=1}^r\mathrm{codim}_XY_i$,用在线性子空间上得到:
		\begin{align*}
			\sum_{i=1}^r\mathrm{codim}_XY_i&=\mathrm{codim}_{\mathrm{T}_{X,x}}\left(\cap_{i=1}^r\mathrm{T}_{Y_i,x}\right)\\&\le\sum_{i=1}^r\mathrm{codim}_{\mathrm{T}_{X,x}}\mathrm{T}_{Y_i,x}\\&\le\sum_{i=1}^r\mathrm{codim}_XY_i
		\end{align*}
		
		其中最后一个不等式是因为$\dim X=\dim\mathrm{T}_{X,x}$和$\dim Y_i\le\dim\mathrm{T}_{Y_i,x}$.于是这些不等式都取等,于是$x$在每个$Y_i$中都是正则点.类似的从$\mathrm{T}_{Y,x}\subseteq\cap_{i=1}^r\mathrm{T}_{Y_i,x}$得到$x$也在$Y$中是正则点.
	\end{proof}
	\item 我们在参数系统里证明了:设$X$是仿射代数集,设$x$是正则点,如果可以取$x$的参数系统$\{u_1,u_2,\cdots,u_n\}$都是整个$X$上的正则函数,那么$X$的子簇$V(u_1),V(u_2),\cdots,V(u_n)$在点$x$处是横截.
\end{enumerate}

泰勒级数.设$X$是预簇,设$x\in X$的嵌入维数是$n$(此为切空间维数),记$u_1,\cdots,u_n\in\mathfrak{m}_x$满足在$\mathfrak{m}_x/\mathfrak{m}_x^2$中构成一组$k$基.一个有理函数$f\in\mathscr{O}_{X,x}$的泰勒级数指的是$k$上的一个$n$元形式幂级数$\Phi=\sum_{i\ge0}F_i$,其中$F_i\in k[T_1,\cdots,T_n]$是次数为$i$的齐次分量.满足如果记$S_k=\sum_{i=0}^kF_i,\forall k\ge0$,那么总有$f-S_k(u_1,\cdots,u_n)\in\mathfrak{m}_x^{k+1}$.
\begin{enumerate}
	\item 例如在$\mathbb{A}^1$上取坐标函数$t$,取点$x=0\in\mathbb{A}^1$,那么$\mathfrak{m}_x=(t)$.此时泰勒级数就是我们熟悉的$\sum_{i=1}^{\infty}a_it^i$.对任意有理函数$f(t)=\frac{P(t)}{Q(t)}$,我们可以通过等式$P(t)-Q(t)\sum_{i=0}^ka_it^i\equiv0(\mathrm{mod}t^{k+1})$来求得$\{a_i\}$,从而$\sum_{i=0}^{\infty}a_it^i$就是$f(t)$对应的泰勒级数.比方说我们熟悉的$\frac{1}{1-t}=\sum_{i=0}^{\infty}t^i$.另外通过求$\{a_i\}$还能说明这里对应于$f$的泰勒级数是唯一的,事实上我们会证明如果$x$是正则点,那么$f$对应的泰勒级数总是唯一的.
	\item 每个函数$f\in\mathscr{O}_{X,x}$都至少存在一个泰勒级数.
	\begin{proof}
	
	记$f(x)=a_0\in k$,记$f_1=f-a_0$,那么$f_1\in\mathfrak{m}_x$,按照定义$\{u_1,\cdots,u_n\}\subseteq\mathfrak{m}_x$满足它们在$\mathfrak{m}_x/\mathfrak{m}_x^2$中构成一组基.那么存在$a_1,\cdots,a_n\in k$使得$f_2=f_1-\sum_{i=1}^na_iu_i\in\mathfrak{m}_x^2$.那么又存在$g_j,h_j\in\mathfrak{m}_x$满足$f_2=\sum_jg_jh_j$,取$b_{ji},c_{ji}\in k$满足$g_j-\sum_{i=1}^nb_{ji}u_i\in\mathfrak{m}_x^2$和$h_j-\sum_{i=1}^nc_{ji}u_i\in\mathfrak{m}_x^2$.于是我们有$f_3=f_2-\sum_{i_1,i_2}\left(\sum_{j=1}^nb_{ji_1}c_{ji_2}\right)u_{i_1}u_{i_2}\in\mathfrak{m}_x^3$.继续操作下去得到$f$对应的泰勒级数.
	\end{proof}
	\item 如果$x$是正则点,那么它局部环中的每个函数都恰有唯一的泰勒级数.
	\begin{proof}
		
		归结为证明函数$f=0$的泰勒级数恰好是零.如果记局部参数系统$\{u_1,u_2,\cdots,u_n\}$,这归结为证明如果一个$k$次数的齐次型$F_k(T_1,T_2,\cdots,T_n)$满足$F_k(u_1,u_2,\cdots,u_n)\in m_x^{k+1}$,那么有$F_k(T_1,T_2,\cdots,T_n)=0$.
		
		\qquad
		
		如果这不成立,即$F_k(T_1,T_2,\cdots,T_n)\not=0$,适当做一个线性变换可以不妨约定$F_k$中$T_n^k$的系数非零.记$F_k(T_1,T_2,\cdots,T_n)=\alpha T_n^k+G_1(T_1,T_2,\cdots,T_{n-1})T_n^{k-1}+\cdots+G_k(T_1,T_2,\cdots,T_{n-1})$,其中$G_i$是关于$\{T_1,T_2,\cdots,T_{n-1}\}$的次数为$i$的齐次形式.于是有:
		$$F_k(u_1,u_2,\cdots,u_n)=\alpha u_n^k+G_1(u_1,u_2,\cdots,u_{n-1})u_n^{k-1}+\cdots+G_k(u_1,u_2,\cdots,u_{n-1})$$
		
		我们解释过局部参数系统是局部环极大理想的生成元集,于是$m_x^{k+1}$中的元可以表示为$m_x$系数的关于$u_1,u_2,\cdots,u_n$的$k$次齐次形式,于是有如下等式,其中$\mu\in m_x$,而$H_i$是次数为$i$的齐次形式.
		$$F_k(u_1,u_2,\cdots,u_n)=\mu u_n^k+H_1(u_1,u_2,\cdots,u_{n-1})u_n^{k-1}+\cdots+H_k(u_1,u_2,\cdots,u_{n-1})$$
		
		这两个式子做差,说明$(\alpha-\mu)u_n^k\in(u_1,u_2,\cdots,u_{n-1})$,但是这里$0\not=\alpha\in k$,导致$\alpha-\mu\not\in m_x$,于是$(\alpha-\mu)^{-1}\in\mathscr{O}_{X,x}$,导致$u_n^k\in(u_1,u_2,\cdots,u_{n-1})$,但是这不能成立,因为导致$V(u_1)\cap\cdots\cap V(u_{n-1})\subseteq V(u_n)$,导致$T_1\cap\cdots\cap T_{n-1}\subseteq T_n$,其中$T_i$是$x$在$V(u_i)$中的切空间,导致$T_1\cap T_2\cap\cdots\cap T_n=T_1\cap\cdots\cap T_{n-1}$,但是后者的维数$\ge1$,导致前者的维数$\ge1$,但是我们解释过此时前者的交是零.
	\end{proof}
    \item 于是对于正则点$x$,有单同态$\tau:\mathscr{O}_{X,x}\to k[[T_1,T_2,\cdots,T_n]]$.特别的正则点的局部环$\mathscr{O}_{X,x}$总是整环(但是事实上一般的正则局部环也总是整环).
    \item 如果$x\in X$是诺特预簇上的正则点,那么恰好存在唯一的不可约分支经过点$x$.
    \begin{proof}
    	
    	设不过点$x$的不可约分支为$\{Z_1,Z_2,\cdots,Z_r\}$(有限性因为诺特条件).那么$X'=X-\cup_iZ_i$是开子集,于是可以选取$x$的落在$X'$中的仿射簇开邻域$U$.那么有典范单同态$k[U]\subseteq\mathscr{O}_{X,x}$(它是单射因为$U$是不可约的,如果$U$上两个正则函数$f,g$在更小的$x$的开邻域上一致,那么满足$f-g=0$的点是$U$的包含一个开子集的闭子集,不可约性导致这只能是整个$U$).【】
    \end{proof}
    \item 上一条可以从另一个角度证明诺特预簇$X$的奇异点构成了闭子集.
    \begin{proof}
    	
    	取$X$的全部不可约分支$\{X_1,X_2,\cdots,X_r\}$,每个$X_i$都是闭子集,上一条说明全体奇点是集合$\cup_{i\not=j}(X_i\cap X_j)$,这是一个闭子集.
    \end{proof}
\end{enumerate}







子簇的局部方程.
\begin{itemize}
	\item 设$Y\subseteq X$是子簇,称$f_1,f_2,\cdots,f_m\in\mathscr{O}_{X,x}$是子簇$Y$在$x\in X$附近的局部方程,如果存在$x$的仿射开邻域$X'$,使得$f_1,f_2,\cdots,f_m\in k[X']$,并且如果记$Y'=Y\cap X'$是$X'$的子簇,那么在$k[X']$中有$I(Y')=(f_1,f_2,\cdots,f_m)$.
	\item 设$Y\subseteq X$是子簇,任取$x\in X$,记$I_{Y,x}\subset\mathscr{O}_{X,x}$由这样的有理函数$f\in\mathscr{O}_{X,x}$构成,满足存在$x$的开邻域$U$使得$f$在$U\cap Y$上恒取零.于是$I_{Y,x}$是$\mathscr{O}_{X,x}$的理想.
\end{itemize}
\begin{enumerate}
	\item 如果$X$是仿射簇,$Y\subseteq X$是子簇,那么$I_{Y,x}$由形如$u/v$的局部函数构成,其中$u,v\in k[X]$,满足$u\in I(Y)$和$v(x)\not=0$.并且如果额外的有$Y$的所有不可约分支都经过点$x$,那么有$I(Y)=I(Y,x)\cap k[X]$.
	
	先设$f\in\mathscr{O}_{X,x}$,那么有$u,v\in k[X]$使得$f=u/v$,并且$v(x)\not=0$.再设存在$x$的开邻域$U$使得
	
	
	111.4,133,180,99.2,302
	
	
	\item 函数$f_1,f_2,\cdots,f_m\in\mathscr{O}_{X,x}$是子簇$Y$在点$x$附近的局部方程当且仅当有$I(Y,x)=(f_1,f_2,\cdots,f_m)$.
	\begin{proof}
		
		如果存在$x$的仿射开邻域$U$,记$Y'=Y\cap U$,使得有代表元$f_1,\cdots,f_m\in k[U]$,那么在$k[U]$中就有$I(Y')=(f_1,\cdots,f_m)$.按照定义就有$f_i\in I_{Y,x}$.反过来如果$f\in I_{Y,x}$,也即存在$x\in Y$的开邻域$V$使得$f$在$V$上恒为零.【】
		
		如果$I_{Y,x}=(f_1,\cdots,f_m)$,其中$f_i\in\mathscr{O}_{X,x}$.记$I(Y)=(g_1,\cdots,g_s)$,其中$g_i\in k[X]$.按照$g_i\in I_{Y,x}$,说明存在$h_{ij}\in\mathscr{O}_{X,x}$使得$g_i=\sum_{j=1}^mh_{ij}f_j$.这些$f_i$和$h_{ij}$都在$x$的某个开邻域上正则,所以可以统一的取一个主开集$x\in U=X-V(g),g\in k[X]$,使得这些函数都在$U$上正则.那么$k[U]$中的元具有形式$u/g^l$,其中$u\in k[X],l\ge0$.所以有$I(Y)k[U]=k[U]g_1+\cdots+k[U]g_s\subseteq I_{Y,x}$.记$Y'=Y\cap U$,下面断言$I(Y)k[U]=I(Y')$.一旦这成立,说明$I(Y')\subset(f_1,\cdots,f_m)$,但是另一侧的包含关系是平凡的,就得证.
		
		\qquad
		
		下面证明断言.一方面有$I(Y)k[U]\subseteq I(Y')$,另一方面$I(Y')$中的元可以表示为$v=u/g^l$,其中$u\in k[X]$,那么$u=vg^l\in I(Y)$,但是$1/g^l\in k[U]$,导致$v=(vg^l)(1/g^l)\in I(Y)k[U]$.于是引理得证.
	\end{proof}
    \item $X$的每个余1维不可约子簇$Y$在任意非奇点上都存在局部方程.
    \item 推论.如果$X$是非奇异簇,如果$\varphi:X\to\mathbb{P}^n$是有理映射,那么非正则点构成的闭子集是余维数$\ge2$的.
    \begin{proof}
    	
    	我们解释过有理映射的非正则点构成闭子集.这个命题是局部的,所以不妨设$X$是非奇异仿射簇.可记$\varphi=(f_0:\cdots:f_n)$,其中$f_i\in k(X)$.取$x\in X$,我们可以适当乘以一个合适的有理函数使得全体$f_i\in\mathscr{O}_{X,x}$,并且它们在$\mathscr{O}_{X,x}$中没有公因式.现在$\varphi$的非正则点恰好是$f_0=\cdots=f_n=0$的零点集,记作$Y$.如果$Y$是余维数1的,那么可取局部方程$I_{Y,x}=(g)$,但是有$I(Y)=(f_0,f_1,\cdots,f_n)\subseteq I_{Y,x}=(g)$,导致$\{g_i\}$在$\mathscr{O}_{X,x}$中有公因式$g$,这矛盾.命题得证.
    \end{proof}
    \item 推论.从非奇异曲线到射影空间的有理映射总是正则映射.
    \item 推论.如果两条非奇异射影曲线是双有理等价的,那么它们是同构的.
\end{enumerate}

\newpage
\subsection{爆破}

$\mathbb{P}^n$在点$\xi=(1:0:\cdots:0)$处的爆破(blowup).记$\mathbb{P}^n$的齐次坐标为$\{x_0,x_1,\cdots,x_n\}$,记$\mathbb{P}^{n-1}$的齐次坐标为$\{y_1,y_2,\cdots,y_n\}$.那么$\mathbb{P}^n\times\mathbb{P}^{n-1}$中的点可记作$(x,y)=(x_0:\cdots:x_n;y_1:\cdots:y_n)$.考虑被$x_iy_j=x_jy_i,\forall 1\le i,j\le n$定义的$\mathbb{P}^n\times\mathbb{P}^{n-1}$的闭子簇,记作$D$.投影映射$\mathbb{P}^n\times\mathbb{P}^{n-1}\to\mathbb{P}^n$在$D$上的限制记作$\sigma:D\to\mathbb{P}^n$,它称为$\mathbb{P}^n$在点$\xi$处的爆破.
\begin{enumerate}
	\item $\sigma$是双有理映射.
	\begin{proof}
		
		如果点$(x_0:\cdots:x_n)\not=\xi$,那么$(x,y)\in D$就要满足$(y_1:\cdots:y_n)=(x_1:\cdots:x_n)$.于是$\sigma^{-1}:\mathbb{P}^n-\{\xi\}\to D$,$(x_0:\cdots:x_n)\mapsto((x_0:\cdots:x_n),(x_1:\cdots:x_n))$是$\sigma$的逆映射.
		
		如果$(x_0:\cdots:x_n)=\xi$,那么任意$y_i$都要满足$D$的定义中的方程,所以有$\sigma^{-1}(\xi)=\{\xi\}\times\mathbb{P}^{n-1}$.于是$\sigma$可限制为$\mathbb{P}^n-\{\xi\}\to D-\{\xi\}\times\mathbb{P}^{n-1}$的同构,所以$\sigma$是双有理映射.
	\end{proof}
    \item 设$D$的由$x_0\not=0,y_i\not=0$定义的开集为$U_i$,齐次性可不妨设$x_0=y_i=1$.那么用来定义$D$的方程就变成了$x_j=y_jx_i,j\not=i$.所以$U_i$同构于坐标为$\{y_1,\cdots,\hat{y_i},x_i,\cdots,y_n\}$的仿射空间$\mathbb{A}^n$.特别的$D$是非奇异的.
    \item $D$是不可约的.
    \begin{proof}
    	
    	我们已经解释了$D-\{\xi\}\times\mathbb{P}^{n-1}$同构于$\mathbb{P}^n-\{\xi\}$.这里$\mathbb{P}^n-\{\xi\}$是不可约空间$\mathbb{P}^n$的开子集(因为单点都是闭点),所以也是不可约的.所以$D-\{x_i\}\times\mathbb{P}^{n-1}$是不可约的.它在$D$中的闭包也是不可约的,所以问题归结为这个闭包是整个$D$,换句话讲归结为$\{\xi\}\times\mathbb{P}^{n-1}\subset\overline{D-\{\xi\}\times\mathbb{P}^{n-1}}$.
    	
    	设$L$是经过点$\xi$的直线,可记$L$被$x_j=\alpha_jx_i$定义,其中$i$是固定的,$j$取遍不为$i$的指标.那么$\sigma^{-1}$限制在直线$L$上为$(x_0:\cdots:x_n)\mapsto(x_0:\cdots:x_n;\alpha_1:\cdots:1:\cdots:\alpha_n)$,其中$\alpha_i$的位置是1.所以$\sigma^{-1}$在$L$上的限制是正则映射,并且像$\sigma^{-1}(L)\subseteq D$和$\{\xi\}\times\mathbb{P}^{n-1}$的交是$(\xi:\alpha_1:\cdots:1:\cdots:\alpha_n)$.换句话讲,尽管$\sigma^{-1}$在点$\xi$处不是正则的,但是把$\sigma^{-1}$限制在$L$上,我们可以延拓定义$\xi$处的取值.并且选取不同的过点$\xi$的直线$L$可以得到$\{\xi\}\times\mathbb{P}^{n-1}$中的全部点.对每条这样的直线$L$有$\sigma^{-1}(L)\subset\overline{D-\{\xi\}\times\mathbb{P}^{n-1}}$,这导致$\{\xi\}\times\mathbb{P}^{n-1}\subset\overline{D-\{\xi\}\times\mathbb{P}^{n-1}}$.
    \end{proof}
\end{enumerate}

一般的爆破.设$X$是拟射影簇,取$\xi\in X$是非奇异点,设$\{u_1,\cdots,u_n\}$是$X$上的正则函数,满足$u_1=\cdots=u_n=0$具有唯一的解$\{\xi\}$,并且$\{u_1,\cdots,u_n\}$构成了$X$在点$\xi$的局部参数系统.考虑积$X\times\mathbb{P}^{n-1}$,设这里$\mathbb{P}^{n-1}$的坐标为$\{t_1,t_2,\cdots,t_n\}$.考虑这个积上的由$u_i(x)t_j=u_j(x)t_i,\forall 1\le i,j\le n$定义的闭子簇$Y$.考虑投影映射$X\times\mathbb{P}^{n-1}\to X$在$Y$上的限制$\sigma:Y\to X$,它称为$X$在点$\xi$处的局部爆破.
\begin{enumerate}
	\item 这里的新定义并不适用于射影的情况,因为这里的新定义需要存在非常值的整体正则函数.所以严格的讲这里的定义并不包含上面$X=\mathbb{P}^n$的定义.但是这两种定义经如下方式联系在一起:选取$X\subset\mathbb{P}^n$是由$x_0\not=0$定义的仿射开子集,记$Y=\sigma^{-1}(X)$,那么$\sigma:Y\to X$符合我们这里新定义的爆破.
	\item 和之前的情况一样,$\sigma:Y\to X$可限制为$Y-\{\xi\}\times\mathbb{P}^{n-1}\cong X-\{\xi\}$的同构.
	\item $Y$是非奇异的.
	\begin{proof}
		
		任取$y\in\sigma^{-1}(\xi)$,记$y=(\xi;t_1:\cdots:t_n)$.那么存在某个指标$i$使得$t_i\not=0$.对$j\not=i$记$s_j=t_j/t_i$.那么定义$Y$的方程变为$u_j=u_is_i,j\not=i$.所以$y\in Y$处极大理想是$m_y=(s_1-s_1(y),\cdots,u_i-u_i(y),\cdots,s_n-s_n(y))$.所以$\dim\mathrm{T}_{Y,y}\le n$.按照$\dim X=n$,得到$\dim X-\{\xi\}=\dim Y-\{\xi\}\times\mathbb{P}^{n-1}=n$.这说明$y\in Y-\{\xi\}\times\mathbb{P}^n$都是非奇异的.
		
		按照$Y=\{\xi\}\times\mathbb{P}^{n-1}\cup\sigma^{-1}(X-\{\xi\})$.要么$Y$本身是不可约的,这等价于讲$Y=\overline{\sigma^{-1}(X-\{\xi\})}$;要么还存在一个不可约分支同构于$\mathbb{P}^{n-1}$.在第二种情况下必须有$\overline{\sigma^{-1}(X-\{\xi\})}$和$\{\xi\}\times\mathbb{P}^{n-1}$的交非空,不然会导致$\sigma^{-1}(X-\{\xi\})$是$X\times\mathbb{P}^{n-1}$的闭子集.我们解释过到射影簇分量的投影映射$X\times Y\to X$是闭映射,所以导致$X-\{\xi\}$是$X$的闭子集.
		
		
	\end{proof}
\end{enumerate}









\newpage
\subsection{正规预簇}

一个不可约仿射簇称为正规的,如果它的坐标环$k[X]$是正规整环,一个不可约拟射影簇称为正规的如果它的每个点都有正规的仿射开邻域.当我们提及正规簇的时候总是指满足这个条件的不可约拟射影簇.
\begin{enumerate}
	\item 非正规簇的例子:考虑$\mathbb{A}^2$的由$y^2=x^2+x^3$定义的仿射曲线,有理函数$t=y/x\in k(X)$满足$t^2=1+x$,于是它是坐标环上的整元,但是这个有理函数不在坐标环中.
	\item 正规且奇异的例子:设$X\subset\mathbb{A}^3$由$x^2+y^2=z^2$定义,那么$(0,0,0)$是奇异点,因为Jaocbian矩阵的秩是0.我们可以证明整体开集本身是正规的仿射开集:即$k[x,y,z]/(x^2+y^2-z^2)$是正规整环.
	\item 一个不可约簇$X$是正规的当且仅当它的所有局部环$\mathscr{O}_{X,x}$都是正规整环.这件事是一个交换代数结论,即一个整环(前提必须是整环,否则局部上是整环甚至得不到原本环是整环)是正规整环当且仅当它在所有极大理想处的局部化都是正规整环.于是正规是一个局部性质,即如果不可约簇$X$是正规的,那么它的不可约开子集都是正规的.
	\item 如果$X$是正规簇,那么它在每个不可约子簇$Y$处的局部环$\mathscr{O}_{X,Y}$(此即$A[X]$在$I(Y)$处的局部化,也即那些至少在$Y$上一个点正则的有理函数构成的$k(X)$的子环,如果$Y$取为单点集$\{x\}$时这就是局部环$\mathscr{O}_{X,x}$本身)都是正规整环.反过来如果$X$是不可约的,并且在每个点处的局部环都是正规整环,那么$X$是正规预簇.
	\begin{proof}
		
		首先第二个命题是一个交换代数结论,即一个整环(前提必须是整环,否则局部上是整环甚至得不到原本环是整环)是正规整环当且仅当它在所有极大理想处的局部化都是正规整环.或者也可以这样证明:任取有理函数$\alpha\in k(X)$使得它在$k[X]$上整,那么有$a_i\in k[X]$使得$\alpha^n+a_1\alpha^{n-1}+\cdots+a_n=0$.但是这个等式放在$\mathscr{O}_{X,x}$里同样成立,因为$a_i$处处是正则的.所以$\alpha\in\mathscr{O}_{X,x}$.但是有$\cap_{x\in X}\mathscr{O}_{X,x}=k[X]$,就得到$\alpha\in k[X]$.
		
		\qquad
		
		下面说明第一个命题.命题是局部的,不妨设$X$是正规的仿射簇,设$Y\subseteq X$是不可约子簇,记局部环为$\mathscr{O}_{X,Y}$.任取有理函数$\alpha\in k(X)$使得它在$\mathscr{O}_{X,Y}$上整,此即存在$a_i\in\mathscr{O}_{X,Y}$使得:
		$$\alpha^n+a_1\alpha^{n-1}+\cdots+a_n=0$$
		
		其中$a_i\in\mathscr{O}_{X,Y}$就可以表示为$a_i=b_i/c_i$,其中$b_i,c_i\in k[X]$,使得$c_i\not\in I(Y)$.记$d_0=c_1\cdots c_n$,把上述方程乘以$d_0$,就得到:
		$$d_0\alpha^n+d_1\alpha{n-1}+\cdots+d_n=0$$
		
		其中$d_i=c_1\cdots b_i\cdots c_n\in k[X]$,且$d_0\not\in I(Y)$.再把这个等式乘以$d_0^{n-1}$,得到$d_0\alpha$在$k[X]$上整.按照$k[X]$是正规的,就有$d_0\alpha\in k[X]$,所以有$\alpha=\beta/d_0\in\mathscr{O}_{X,Y}$.这得到$\mathscr{O}_{X,Y}$是正规的.
	\end{proof}
	\item 非奇异簇总是正规的.因为我们解释过正则局部环总是UFD,而UFD总是正规整环,于是局部环都是正规整环,就得到它是正规簇.
	\item 如果$X$是正规簇,$Y\subseteq X$是余维数1的子簇,那么存在$X$的仿射开子集$X'$,满足$Y'=X'\cap Y$非空,并且$Y'$对应的理想在$k[X']$中是主的.
	\begin{proof}
		
		【】
	\end{proof}
    \item 正规簇的奇异点构成的子簇具有余维数$\ge2$,特别的,这说明对于代数曲线,正规和非奇异是等价的条件.另外如果簇的奇异点集构成的闭子集是余维数$\ge2$的,我们就称这样的簇在余维数1上非奇异,那么这一条就是说正规簇总在余维数1上非奇异.
    \begin{proof}
    	
    	【】
    \end{proof}
\end{enumerate}





\newpage
\section{除子和微分形式}
\subsection{除子}

Weil除子.约定我们提到的簇总是指拟射影簇,即射影簇的开子集,于是特别的我们提到的簇总是不可约的.
\begin{itemize}
	\item 设$X$是一个簇,它的全体余维数1的不可约闭子簇生成的自由阿贝尔群记作$\mathrm{Div}X$.它的元素称为$X$上的Weil除子,换句话讲除子可以表示为$D=k_1C_1+k_2C_2+\cdots+k_rC_r$,其中$k_i$是整数,$C_i$是$X$上余维数1的不可约闭子簇.
	\item 如果全部$k_i=0$则称它是零除子;如果全体$k_i$非负并且不全为零,就称它是正除子或者有效(effective)除子;如果$D-D'$是正除子或者零除子,我们记$D\ge D'$;称余维数1的不可约闭子簇本身为素除子.
	\item 如果$D=\sum_ik_iC_i$不是零除子,并且全部$k_i\not=0$,称$\cup_iC_i$是除子$D$的支集,记作$\mathrm{Supp}D$.
\end{itemize}

主除子.给定簇$X$上的有理函数$f$,我们要定义和它相伴的除子.
\begin{enumerate}
	\item 设$X$在余维数1上非奇异,换句话讲它的奇异点构成的闭子集是余维数$\ge2$的.任取$X$的素除子$C$,我们解释过余维数1的子簇的非奇点附近总存在局部方程,于是可取一个仅由非奇点构成的仿射开子集$U\subseteq X$,使得$C'=U\cap C$非空,并且这个交由单个方程定义,记作$I(C')=(\pi)$.按照NAK引理,有$\cap_{k\ge0}(\pi^k)=0$,所以对每个$f\in k[U]$都存在唯一的$k$使得$f\in(\pi^k),f\not\in(\pi^{k+1})$.记作$v_C^U(f)=k$,这是一个离散赋值,因为它满足:
	$$v_C^U(fg)=v_C^U(f)+v_C^U(g)$$
	$$v_C^U(f+g)\ge\min\{v_C(f),v_C(g)\}$$
	
	按照$X$是不可约的,我们有$k(X)=\mathrm{Frac}k[U]$,所以这个赋值可以唯一延拓到整个$k(X)$,即定义$v_C^U(f/g)=v_C^U(f)-v_C^U(g)$.下面要说明这个定义不依赖于$U$的选取:【】
	\item 如果$X$是正规预簇,那么每个$\mathscr{O}_{X,C}$都是离散赋值环,此时$\mathrm{ord}_C$恰好是对应的赋值.
	\item 给定有理函数$f\in k(X)$,只存在有限个不可约余维数1闭子簇$C$使得$\mathrm{ord}_C(f)\not=0$.
	\item 定义有理函数$f$对应的Weil除子为$\sum_C\mathrm{ord}_C(f)C$,记作$\mathrm{Div}f$.可以表示为一个有理函数的Weil除子的除子称为主除子.如果$\mathrm{Div}f=\sum_ik_iC_i$,记$\mathrm{Div}_0f=\sum_{k_i>0}k_iC_i$,记$\mathrm{Div}_{\infty}f=\sum_{k_i<0}-k_iC_i$.它们分别称为$f$的零点除子和极点除子.此时满足$\mathrm{Div}f=\mathrm{Div}_0f-\mathrm{Div}_{\infty}f$.另外对$f\in k$平凡的有$\mathrm{Div}f=0$.
	\item 如果$f\in k[X]$,那么自然有$\mathrm{Div}f\ge0$.反过来对不可约非奇异预簇$X$,逆命题是成立的:如果$f$是$X$上的有理函数,满足$\mathrm{Div}f\ge0$,那么$f$是$X$上的正则函数.
	\begin{proof}
		
		假设$f$在某个点$x\in X$处不是正则的,那么有$f=g/h$,其中$g,h\in\mathscr{O}_{X,x}$,但是$g/h\not\in\mathscr{O}_{X,x}$.按照$\mathscr{O}_{X,x}$是UFD,不妨设$g,h$的唯一分解中没有公共不可约因子.于是可取一个不可约元$\pi\in\mathscr{O}_{X,x}$,满足$\pi\mid h$但$\pi\not\mid g$.考虑$x$某个开邻域中的余维数1的不可约子簇$V(\pi)$,记它在$X$中的闭包为$C$,那么$\mathrm{ord}_C(f)<0$,这和条件矛盾,于是$f$是$X$上的正则函数.
	\end{proof}
    \item 特别的,上一条结合射影簇上整体正则函数都是常值函数,说明不可约非奇异射影簇$X$上如果整体函数$f$满足$\mathrm{Div}f\ge0$,那么$f$是$X$上的常值函数.特别的,这还说明不可约分期一射影簇$X$上的有理函数被它的除子在相差一个常数意义下唯一决定:如果有理函数$f,g$满足$\mathrm{Div}f=\mathrm{Div}g$,那么$\mathrm{Div}f/g=0$,导致$f=\alpha g$,其中$\alpha\in k$.
\end{enumerate}

Weil除子的一些例子.
\begin{enumerate}
	\item 设$X=\mathbb{A}^n$,每个余维数1的不可约闭子簇可以被一个不可约多项式$F\in k[X]$定义,此时有$C=\mathrm{Div}F$,于是特别的仿射空间上的每个素除子都是主除子.
	\item 设$X=\mathbb{P}^n$,每个余维数1的不可约闭子簇可以被一个不可约齐次多项式定义.于是特别的,如果$f\in k(\mathbb{P}^n)$是$X$上的有理函数,它可以表示为$f=G/H$,其中$G,H$是次数相同的齐次多项式,做分解$G=\prod G_i^{\alpha_i}$和$H=\prod H_j^{\beta_j}$.记$H_i=0$和$L_j=0$分别对应的不可约超曲面为$C_i$和$D_i$,那么有$\mathrm{Div}f=\sum_i\alpha_iC_i-\sum_j\beta_jD_j$.
	\item 另外在上一条的表示中我们看到如果$\mathrm{Div}f=\sum k_iC_i$,其中$C_i$被齐次多项式$H_i$定义,那么有$\sum_ik_i\deg H_i=0$.这个逆命题也是平凡成立的:如果Weil除子$\sum_ik_iC_i$满足$\sum_ik_i\deg H_i=0$,其中$H_i$定义了超曲面$C_i$,那么取$f=\prod H_i^{k_i}$就得到$\mathrm{Div}f=\sum_ik_iC_i$.我们称$\sim_ik_i\deg H_i$为除子$\sum_ik_iC_i$的次数,那么我们证明的是$\mathbb{P}^n$上一个Weil除子是主除子当且仅当它的次数为零.
\end{enumerate}

Weil除子类群.全体主除子构成了Weil除子群$\mathrm{Div}X$的子群$P(X)$,商群$\mathrm{Div}(X)/P(X)$称为Weil除子类群,记作$\mathrm{Cl}(X)$,它的元素称为Weil除子类.两个Weil除子落在同一个陪集称为线性等价的,换句话讲$D_1\sim D_2$当且仅当存在有理函数$f$使得$D_1-D_2=\mathrm{Div}f$.例如我们有:
$$\mathrm{Cl}(\mathbb{A}^n)=0;\mathrm{Cl}(\mathbb{P}^n)=\mathbb{Z};\mathrm{Cl}(\mathbb{P}^{n_1}\times\mathbb{P}^{n_2}\times\cdots\times\mathbb{P}^{n_k})=\mathbb{Z}^k$$

我们接下来讨论Cartier除子,为了区分这两种除子,我们把Weil除子记作$\mathrm{WDiv}$,把Cartier除子记作$\mathrm{CDiv}$.不可约预簇$X$上的一个Cartier除子(或者称为局部主除子)是指一族有理函数$f_i\in k(U_i)$,其中$\{U_i\}$是$X$的开覆盖,满足每个$f_i$都不是恒为零的;$f_i/f_j$和$f_j/f_i$都是$U_i\cap U_j$上的正则函数.约定$\{f_i\in k(U_i)\}$和$\{g_i\in k(V_i)\}$是相同的Cartier除子如果$f_i/g_j$和$g_j/f_i$都在$U_i\cap V_j$上是正则函数.
\begin{enumerate}
	\item 我们来解释如果添加非奇异条件,那么Cartier除子和Weil除子是等价的.设$X$是不可约非奇异预簇,于是任取素除子$C\subseteq X$,任取点$x\in X$,那么存在$x$的开邻域$U$使得$C$可以被一个局部方程$\pi$所定义.于是如果设$D=\sum_ik_iC_i$是一个Weil除子,每个$C_i$在$U$上可以表示为一个局部方程$\pi_i$,取$f=\prod_i\pi_i^{k_i}$,那么有$D=\mathrm{WDiv}f$.选取这样的开子集$U$使得它们构成$X$的一个开覆盖$\{U_i\}$,那么在每个$U_i$上有$D=\mathrm{WDiv}f_i$,这满足Cartier除子的条件.反过来任取一个Cartier除子$\{f_i\in k(U_i)\}$,对每个素除子$C$,如果$U_i\cap C$非空,记$k_C=\mathrm{ord}_C(f_i)$,按照定义中的$f_i/f_j,f_j/f_i$在$U_i\cap U_j$上正则,说明$k_C$的这个定义不依赖于和$C$的交非空的$U_i$的选取.由此定义了一个Weil除子$D=\sum k_CC$.
	\item 如果存在有理函数$f$满足$f=f_i,\forall i$,这样的Cartier除子称为主除子.
	\item 除子的回拉.设$\varphi:X\to Y$是非奇异不可约预簇之间的正则映射,设$D$是$Y$上的一个除子,满足$\varphi(X)\not\subset\mathrm{Supp}D$.首先我们要说明有理函数的回拉不是恒为零的映射,考虑$y\in\varphi(X)$满足$f$在$y$处正则并且$f(y)\not=0$构成的点集$V$,如果有$\varphi(X)\not\subset\mathrm{Supp}(\mathrm{Div}f)$,那么$U$是一个非空开子集,于是$\varphi^*(f)$是$\varphi^{-1}(V)$上的正则函数,并且处处不取零,于是$\varphi^*(f)$是$X$上的不恒为零的有理函数.现在设$D$是一个Cartier除子,那么$Y=\cup_iU_i$,$D$由一族有理映射$f_i\in k(U_i)$构成,考虑那些满足$\varphi(X)\cap U_i$非空的$U_i$,我们断言$\varphi(X)\cap U_i\not\subset\mathrm{Supp}(\mathrm{Div}f_i)$:按照$X$是不可约的,得到$\overline{\varphi(X)}$是$Y$的不可约子集,假设这个结论不成立,那么理应有$\overline{\varphi(X)}\subset\mathrm{Supp}(\mathrm{Div}f_i)$,结合$\mathrm{Supp}(\mathrm{Div}f_i)\cap U_i=\mathrm{Supp}D\cap U_i$得到$\varphi(X)\subset\mathrm{Supp}D$,这矛盾.于是$\varphi^*(f_i)$构成了$X$上的一个Cartier除子,这称为$D$的回拉或者逆像.
	\item 如果$\varphi(X)$在$Y$中稠密,那么$Y$上每个Cartier除子都存在回拉.这个回拉是一个群同态$\varphi^*:\mathrm{Div}Y\to\mathrm{Div}X$.这个同态把主除子打到主除子,于是它诱导了Cartier除子类群的同态$\mathrm{Cl}(Y)\to\mathrm{Cl}(X)$.
	\item Picard群.给定两个Cartier除子$\{f_i\in k(U_i)\}$和$\{g_j\in k(V_j)\}$,它们的乘积定义为$\{f_ig_j\in k(U_i\cap V_j)\}$.在这个乘法下全体主除子构成子群,商群称为$X$的Picard群.
\end{enumerate}

曲线上除子的次数.设$X$是非奇异射影曲线,那么它的余维数1闭子簇就是曲线上的单点,于是$X$上的除子$D$可以表示为$D=\sum k_ix_i$,其中$x_i\in X$.我们定义$D$的次数为$\sum_ik_i$,记作$\deg D$.
\begin{enumerate}
	\item 如果$f:X\to Y$是非奇异射影曲线之间的正则满映射,那么$\deg f=\deg f^*(y)$,这里$y$表示乘数1的$Y$上除子,$f^*(y)$是除子的回拉.左侧是正则映射的次数.
	\item 非奇异射影曲线上的主除子的次数总是零.
	\begin{proof}
		
		非常值有理函数$f\in k(X)$定义了一个正则映射$f:X\to\mathbb{P}^1$,并且有$f^*(0)=\mathrm{Div}_0f$,这里记号$f^*(0)$同样表示零除子的回拉.类似的有$f^*(\infty)=\mathrm{Div}_{\infty}f$,于是按照上一条得到:
		\begin{align*}
		\deg(\mathrm{Div}f)&=\deg(\mathrm{Div}_0f)-\deg(\mathrm{Div}_{\infty}f)\\&=\deg(f^*(0))-\deg(f^*(\infty))\\&=\deg f-\deg f=0
		\end{align*}
	\end{proof}
    \item 一个非奇异射影曲线$X$是有理的当且仅当$\mathrm{Cl}^0(X)=0$.
\end{enumerate}

除子的Riemann-Roch空间和维数.给定非奇异预簇$X$上的除子$D$,考虑$X$上的零有理函数以及全体满足$\mathrm{Div}f+D\ge0$的有理函数$f$,这构成了基域$k$上的线性空间,它称为$X$上关于除子$D$的Riemann-Roch空间,记作$\mathscr{L}(D)$或者$\mathscr{L}(X,D)$.它的维数称为除子$D$的维数,记作$l(D)$.
\begin{enumerate}
	\item 我们定义过除子线性等价$D_1\sim D_2$是指存在有理函数$g\in k(X)$使得$D_1-D_2=\mathrm{Div}g$.我们断言线性等价的除子具有相同的维数.于是我们定义的维数实际是定义在除子类上的.
	\begin{proof}
		
		如果$f\in\mathscr{L}(D_1)$,也即$\mathrm{Div}(f)+D_1\ge0$,那么有$\mathrm{Div}(fg)+D_2=\mathrm{Div}f+D_1\ge0$,于是$fg\in\mathscr{L}(D_2)$,这说明$g\mathscr{L}(D_1)=\mathscr{L}(D_2)$于是于是两个线性空间是同构的.
	\end{proof}
    \item 如果$D$是非奇异射影曲线$X$上的除子,那么$\mathscr{L}(D)$是有限维$k$线性空间.
    \begin{proof}
    	
    	首先我们解释问题归结为设$D$是正除子.设$D=D_1-D_2$,其中$D_1,D_2$都是正除子,那么有$\mathscr{L}(D)\subset\mathscr{L}(D_1)$,因为如果$f\in\mathscr{L}(D)$,那么$\mathrm{Div}f+D\ge0$,于是$\mathrm{Div}f+D_1\ge0$,于是$f\in\mathscr{L}(D_1)$.于是$\mathscr{L}(D_1)$有限维推出$\mathscr{L}(D)$有限维.
    	
    	现在设$D\ge0$,设点$x$出现在$D$中,并且重数为$r>0$,于是可记$D=rx+D_1$,再记$(r-1)x+D_1=D'$,设$x\in X$处的局部参数为$t$,对函数$f\in\mathscr{L}(D)$,记$\lambda(f)=(t^rf)(x)$,那么$\lambda$是$\mathscr{L}(D)\to k$的线性函数,它的核是$\mathscr{L}(D')$,这个操作重复$\deg D$次,于是$\mathscr{L}(0)$是线性空间$\mathscr{L}(D)$上$\deg D$格线性形式的公共零点集.但是我们解释过$\mathscr{L}(0)=k$,于是得到$l(D)\le\deg D+1$,于是这是有限维的.
    \end{proof}
    \item 另外上一条的证明中可以看出如果$D_1<D_2$,那么$l(D_2)\le l(D_1)+\deg(D_2-D_1)$.
    \item 如果$X=\mathbb{P}^1$,此时所有除子都是线性等价于某个$rx_{\infty}$的,于是$\mathscr{L}(D)$是次数$\le r$的多项式空间,于是$l(D)=r+1$.
\end{enumerate}


\newpage
\subsection{微分形式}

给定开集$U$上的正则函数$f$,我们定义过对$x\in U$,有$d_xU$是切空间$T_{X,x}$上的线性函数.于是$df:x\mapsto d_xf$是$U\to\coprod_{x\in U}T_{X,x}^*$的映射.$X$上的一个正则微分形式是指一个映射$\varphi:X\to\coprod_{x\in X}T_{X,x}^*$,满足$\varphi(x)\in T_{X,x}^*$,并且对每个点$x\in X$,都存在一个开邻域$U$,使得$\varphi$限制在$U$上可以表示为$\sum_if_i\mathrm{d}g_i$,其中$f_i,g_i\in k[U]$.$X$上的全体正则微分形式构成了一个$k[X]$模,记作$\Omega[X]$.
\begin{enumerate}
	\item 微分映射$f\mapsto\mathrm{d}f$是$k[X]\to\Omega[X]$的交换群同态$\mathrm{d}(f+g)=\mathrm{d}f+\mathrm{d}g$,满足$\mathrm{d}(fg)=f\mathrm{d}g+g\mathrm{d}f$.这说明如果$F\in k[T_1,T_2,\cdots,T_n]$和$f_1,f_2,\cdots,f_n\in k[X]$,那么有:
	$$\mathrm{d}(F[f_1,f_2,\cdots,f_n])=\sum_{i=1}^n\frac{\partial F}{\partial T_i}(f_1,f_2,\cdots,f_n)\mathrm{d}f_i$$
	\item 考虑仿射空间$\mathbb{A}^n$,对每个点$x\in X$微分$d_xt_1,d_xt_2,\cdots,d_xt_n$都构成了$T_{X,x}^*$的一组基.这说明每个正则微分形式$\varphi\in\Omega[\mathbb{A}^n]$都可以表示为$\sum_{i=1}^n\varphi_i\mathrm{d}t_i$,其中$\varphi_i$都是$\mathbb{A}^n$上的$k$值函数.每个点$x\in X$存在开邻域$U$使得这些$\varphi_i$限制在$U$上是正则函数,这说明这些$\varphi_i$实际都是$\mathbb{A}^n$上的整体正则函数.于是仿射空间上的正则微分形式都可以表示为$\varphi=\sum_{i=1}^n\varphi_i\mathrm{d}t_i$,其中$\varphi_i$都是$\mathbb{A}^n$上的正则函数(多项式).
	\item 射影线$\mathbb{P}^1$上的正则微分形式都是平凡的.记$\mathbb{P}^1=\mathbb{A}_0^1\cup\mathbb{A}_1^1$.任取$\mathbb{P}^1$上的正则微分形式$\varphi$,那么在$\mathbb{A}_0^1$上可以表示为$P(t)\mathrm{d}t$,在$\mathbb{A}_1^1$上可以表示为$Q(u)\mathrm{d}u$,并且这里$u=t^{-1}$.于是有$\mathrm{d}u=-\mathrm{d}t/t^2$,设$\deg Q=n$,那么在$\mathbb{A}_0^1\cap\mathbb{A}_1^1$中有$P(t)\mathrm{d}t=-\frac{1}{t^2}Q(1/t)\mathrm{d}t$,于是有$P(t)=\frac{-Q^*(t)}{t^{n+2}}$,其中$Q^*(t)=t^nQ(1/t)$,这说明$Q^*(0)\not=0$(因为$Q$的首系数非零),这导致$P=Q=0$,于是$\Omega[\mathbb{P}^1]=0$.
	\item 设$X$是仿射簇,设$A=k[X]$是坐标环,设$\Omega=\Omega[X]$是正则微分形式模,设$\mathrm{d}$是微分映射,那么$\Omega$是全体$\mathrm{d}f,f\in A$生成的$A$模.
	\begin{proof}
		
		任取$\varphi\in\Omega$,对每个点$x\in X$存在$f_{i,x},g_{i,x}\in\mathscr{O}_{X,x}$满足$\varphi(x)=\sum_if_{i,x}\mathrm{d}g_{i,x}$.每个$u\in\mathscr{O}_{X,x}$可以表示为$u=v/w$,其中$v,w\in A$,并且$w(x)\not=0$,考虑全体分母的最小公倍式$p_x$,那么$p_x(x)\not=0$,并且有$p_x\varphi=\sum r_{i,x}\mathrm{d}h_{i,x}$,其中$r_{i,x},h_{i,x}\in A$.现在按照$p_x(x)\not=0$对每个$x\in X$成立,存在有限个点$x$使得$1=\sum p_xq_x$,于是$\varphi=\sum_x\sum_i q_xr_{i,x}\mathrm{d}h_{i,x}$.
	\end{proof}
    \item 仿射簇$X$的每个非奇异点$x$都存在仿射开邻域$U$使得$\Omega[U]$是自由$k[U]$模,并且它的秩为$\dim_xX$.
    \begin{proof}
    	
    	设$X\subset\mathbb{A}^N$,设根理想$I(X)$由$F_1,F_2,\cdots,F_m$生成,那么在$X$上恒有$F_i\equiv0$,于是有$\sum_{j=1}^N\frac{\partial F_i}{\partial T_j}\mathrm{d}t_j=0$.如果$x\in X$是非奇异点,记$\dim_xX=n$,那么矩阵$\left((\partial F_i/\partial T_j)(x)\right)$具有秩$N-n$.不妨设$t_1,t_2,\cdots,t_n$是点$x$处的局部参数系统,于是每个$\mathrm{d}t_j$可以表示为$\mathrm{d}t_i,1\le i\le n$的有理函数系数的线性和.
    	
    	选取$x$的开邻域使得这些有理函数都是正则的,那么此时$d_yt_i,1\le i\le n$构成了$T_{X,y}^*$的基对任意$y\in U$成立.任取$\varphi\in\Omega[U]$,此时它可以唯一的表示为$\varphi=\sum_{i=1}^n\varphi_i\mathrm{d}t_i$,其中$\varphi_i$是$U$上的$k$值函数.但是我们解释过$\varphi$可以表示为正则函数系数的$\mathrm{d}t_i,1\le i\le N$的线性组合,把那些$\mathrm{d}t_i,i\ge n+1$表示为$\mathrm{d}t_i,1\le i\le n$的正则函数系数线性组合,说明这些$\varphi_i$都是$U$上的正则函数.这说明$\Omega[U]=\sum_{i=1}^nk[U]\mathrm{d}t_i$.
    	
    	最后说明这个和是直和.如果有$\sum_{i=1}^ng_i\mathrm{d}t_i=0$,不妨设$g_n\not=0$,那么$\mathrm{d}t_i,1\le i\le n$在$g_n\not=0$这个开集上是线性相关的,这就和$d_yt_i\in T_{X,y}^*,\forall y\in U$的线性无关性相矛盾,于是$\Omega[U]=\oplus_{i=1}^nk[U]\mathrm{d}t_i$.
    \end{proof}
    \item 如果$u_1,u_2,\cdots,u_n$构成了正则点$x$的局部参数系统,那么存在$x$的某个开邻域$U$使得$\mathrm{d}u_1,\mathrm{d}u_2,\cdots,\mathrm{d}u_n$生成了$k[U]$模$\Omega[U]$.
    \begin{proof}
    	
    	按照上一条,选取$x$的开邻域$U$使得$\mathrm{d}t_i,1\le i\le n$构成了自由模$\Omega[U]$的一组基,那么可记$\mathrm{d}u_i=\sum_{j=1}^ng_{ij}\mathrm{d}t_j$.按照$\{u_i\}$是局部参数系统,说明$\det|g_{ij}(x)|\not=0$,于是在$\det|g_{ij}|\not=0$这个开邻域$U'$上就有$\mathrm{d}u_i,1\le i\le n$生成了$\Omega[U']$.
    \end{proof}
    \item 微分模的代数描述.设$X$是非奇异仿射簇,记坐标环$A=k[X]$,记正则微分形式模为$\Omega$,那么$\Omega$是由生成元集$\{\mathrm{d}f\mid f\in A\}$和关系集$\{\mathrm{d}(f+g)-\mathrm{d}f-\mathrm{d}g,\mathrm{d}(fg)-f\mathrm{d}g-g\mathrm{d}f,\mathrm{d}\alpha,\alpha\in k\}$确定的$A$模.
    \begin{proof}
    	
    	设这个生成元集与关系集确定的$A$模为$R$.于是存在典范的同态$\gamma:R\to\Omega$,这是满的因为我们证明过仿射簇上的正则微分形式模被全体$\{\mathrm{d}f\mid f\in A\}$生成.最后只需验证它是单射.假设$\varphi\in R$满足$\gamma(\varphi)=0$,和之前命题证明一样,存在$p\in A$使得$p(x)\not=0$,并且有$p\varphi=\sum g_i\mathrm{d}t_i$,其中$g_i\in A$.于是如果$\gamma(\varphi)=0$,导致$\sum g_i\mathrm{d}t_i$在$\Omega$中为零.这导致这些$g_i=0$,从而$p\varphi=0$,于是对每个$x\in X$存在函数$p\in A$使得$p(x)\not=0$单射$p\varphi=0$,这导致$\varphi=0$,单射得证.
    \end{proof}
\end{enumerate}

微分$p$形式.我们之前定义的$X$上的正则微分1形式$\omega$是指一个把$X$中的点$x$映射为$T_{X,x}^*$中的元的映射,一般的正则微分$p$形式$\omega$是指一个把$X$中的点$x$映射为外积$\bigwedge^pT_{X,x}^*$,换句话讲$x$处的取值是$T_{X,x}$上的一个斜对称$p$重线性型.之前定义的微分1形式要额外满足对每个点$x\in X$存在开邻域$U$使得$\omega$限制在$U$上可表示为$\sum_if_i\mathrm{d}g_i$,其中$f_i,g_i\in k[U]$.这里要额外满足对每个$x\in X$存在开邻域$U$使得$\omega$在$U$上的限制可以表示为$\mathrm{d}f_1\wedge\mathrm{d}f_2\wedge\cdots\wedge\mathrm{d}f_p$的$k[U]$系数的线性组合,其中$f_i\in k[U]$.预簇$X$上的全体正则微分$p$形式构成一个$k[X]$模,记作$\Omega^p[X]$.
\begin{enumerate}
	\item 按照外代数的乘法,如果$\omega_1\in\Omega^r[X]$,$\omega_2\in\Omega^s[X]$,那么$\omega_1\wedge\omega_2\in\Omega^{r+s}[X]$.
	\item 设$X$是$n$维预簇,设$x\in X$是非奇点,那么存在$x$的开邻域$U$使得$\Omega^r[U]$是一个秩为$\left(\begin{array}{c}n\\r\end{array}\right)$的$k[U]$自由模.
	\begin{proof}
		
		我们之前解释了对非奇点$x$,存在开邻域$U$使得存在$U$上的正则函数$\{u_1,u_2,\cdots,u_n\}$满足$\mathrm{d}_yu_i$构成了$T_{X,y}^*$的一组基对任意$y\in U$成立.于是每个$\varphi\in\Omega^r[U]$可以唯一的表示为如下形式,其中$\varphi_{i_1,i_2,\cdots,i_r}$是$U$上的$k$值函数.
		$$\varphi=\sum\varphi_{i_1,i_2,\cdots,i_r}\mathrm{d}u_{i_1}\wedge\mathrm{d}u_{i_2}\wedge\cdots\wedge\mathrm{d}u_{i_r}$$
		
		适当缩小$x$的开邻域$U$可以保证在$U$上$\varphi$可表示为$\sum g_{i_1,i_2,\cdots,i_r}\mathrm{d}f_{i_1}\wedge\cdots\wedge\mathrm{d}f_{i_r}$,把每个$\mathrm{d}f_{i_j}$表示为$\mathrm{d}u_k$的正则函数系数线性组合,说明$\varphi$可以表示为$\sum\varphi'_{i_1,i_2,\cdots,i_r}\mathrm{d}u_{i_1}\wedge\mathrm{d}u_{i_2}\wedge\cdots\wedge\mathrm{d}u_{i_r}$,这说明全体$\mathrm{d}u_{i_1}\wedge\mathrm{d}u_{i_2}\wedge\cdots\wedge\mathrm{d}u_{i_r}$生成了$k[U]$模$\Omega^r[U]$.
		
		最后说明这个和是直和,即如果$\sum g_{i_1,i_2,\cdots,i_r}\mathrm{d}u_{i_1}\wedge\mathrm{d}u_{i_2}\wedge\cdots\wedge\mathrm{d}u_{i_r}=0$,那么有$\sum g_{i_1,i_2,\cdots,i_r}(x)\mathrm{d}_xu_{i_1}\wedge\mathrm{d}_xu_{i_2}\wedge\cdots\wedge\mathrm{d}_xu_{i_r}=0$对任意$x\in U$成立.但是$\{\mathrm{d}_xu_i\}$是$T_{X,x}^*$的一组基,于是$\{\mathrm{d}_xu_{i_1}\wedge\mathrm{d}_xu_{i_2}\wedge\cdots\wedge\mathrm{d}_xu_{i_r}\}$是$\bigwedge^rT_{X,x}^*$的一组基,这说明全体$g_{i_1,i_2,\cdots,i_r}=0$,这说明和是直和.
	\end{proof}
    \item 特别的,如果$\dim X=n$,那么对每个点$x$存在开邻域$U$使得$\Omega^n[U]$是秩为1的$k[U]$上自由模,此时$\Omega^n[U]$中的元总可以表示为$\omega=g\mathrm{d}u_1\wedge\mathrm{d}u_2\wedge\cdots\wedge\mathrm{d}u_n$,其中$g$是$U$上的正则函数.这个表示依赖于基$\{u_1,u_2,\cdots,u_n\}$.假设还有一组基$\{v_1,v_2,\cdots,v_n\}$,记$\mathrm{d}u_i=\sum_{j=1}^nh_{ij}\mathrm{d}v_j$,记$\det|h_{ij}(x)|$是基变换的Jacobian行列式,记作$J(\frac{u_1,u_2,\cdots,u_n}{v_1,v_2,\cdots,v_n})$,那么这个行列式是$U$上的正则函数,并且在$U$上处处非零.此时微分形式的不同表示满足:
    $$g\mathrm{d}u_1\wedge\cdots\wedge\mathrm{d}u_n=gJ(\frac{u_1,u_2,\cdots,u_n}{v_1,v_2,\cdots,v_n})\mathrm{d}v_1\wedge\cdots\wedge\mathrm{d}v_n$$
    \item 另外注意尽管微分$n$形式局部上(在选取一组基的前提下)可以表示为一个正则函数,但是这些局部表示未必可以粘合为一个整体正则函数.
\end{enumerate}

有理微分形式.
\begin{enumerate}
	\item 引理.一个正则微分形式取零的点构成的集合是闭子集.特别的,如果不可约预簇上的一个正则微分形式在一个开集上处处取零,那么它在整个预簇上恒取零.
	\begin{proof}
		
		按照闭集是一个局部性质,我们只需验证任一点$x$的足够小开邻域满足这个性质.不妨设这样的$U$满足存在$u_1,u_2,\cdots,u_n\in k[U]$使得$\{\mathrm{d}u_{i_1}\wedge\cdots\wedge\mathrm{d}u_{i_r},1\le i_1<i_2<\cdots<i_r\le n\}$构成$\Omega^r[U]$的一组基.于是在$U$上正则微分形式$\omega$可以表示为正则函数系数的线性组合,而这些正则函数的公共零点集是一个闭子集.
	\end{proof}
    \item 定义.考虑不可约预簇$X$上的全体对$(\omega,U)$,其中$U\subseteq X$是开子集,$\omega\in\Omega^r[U]$,两个对$(\omega,U)$和$(\omega',U')$约定为等价的,如果在$U\cap U'$上有$\omega=\omega'$.那么按照上一条引理,这是一个等价关系,它的等价类称为有理微分$r$形式.
    \item 一个有理微分$r$形式的正则点构成一个开子集,称为这个有理微分形式的定义域.
    \item 全体有理微分$r$形式构成了函数域$k(U)$上的线性空间,记作$\Omega^r(X)$.特别的,有$\Omega^0(X)=k(X)$.我们断言$\Omega^r(X)$是维数为$\left(\begin{array}{c}n\\r\end{array}\right)$的$k(X)$线性空间.
    \begin{proof}
    	
    	我们解释过存在开集$U$使得$\Omega^r[U]$是自由$k[U]$模,并且存在$u_1,u_2,\cdots,u_n\in k[U]$使得$\{\mathrm{d}u_{i_1}\wedge\cdots\wedge\mathrm{d}u_{i_r},1\le i_1<i_2<\cdots<i_r\le n\}$构成了$k[U]$的一组基,于是任取有理微分$r$形式$\omega$,存在开子集$U'\subseteq U$使得$\omega$限制在$U'$上可以唯一的表示为$\sum g_{i_1,i_2,\cdots,i_r}\mathrm{d}u_{i_1}\wedge\cdots\wedge\mathrm{d}u_{i_r}$,其中$g_{i_1,\cdots,i_r}$是$U'$上的正则函数.这就说明$\{\mathrm{d}u_{i_1}\wedge\cdots\wedge\mathrm{d}u_{i_r},1\le i_1<i_2<\cdots<i_r\le n\}$是$\Omega^r(X)$上的一组基.
    \end{proof}
    \item 如果$u_1,u_2,\cdots,u_n$是$k\subseteq k(X)$的一组可分超越基,那么$\{\mathrm{d}u_{i_1}\wedge\cdots\wedge\mathrm{d}u_{i_r},1\le i_1<i_2<\cdots<i_r\le n\}$构成了$k(X)$线性空间$\Omega^r(X)$的一组基.
\end{enumerate}

有理微分形式的除子.设$\omega$是非奇异预簇$X$上的有理微分$n$形式选取$X$的有限仿射开覆盖$\{U_i\}$使得在每个开集$U_i$上有表示$\omega=g^{(i)}\mathrm{d}u_1^{(i)}\wedge\cdots\wedge\mathrm{d}u_n^{(i)}$.那么在交集$U_i\cap U_j$上就有$g^{(j)}=g^{(i)}J\left(\frac{u_1^{(i)},u_2^{(i)},\cdots,u_n^{(i)}}{u_1^{(j)},u_2^{(j)},\cdots,u_n^{(j)}}\right)$,这里Jacobian行列式$J$总是$U_i\cap U_j$上的处处非零的正则函数,于是有理函数族$\{g^{(i)}\in k(U_i)\}$是$X$上的Cartier除子,它称为有理$n$形式$\omega$诱导的除子,记作$\mathrm{Div}\omega$.
\begin{enumerate}
	\item 如果$\omega$是$n$维非奇异预簇$X$上的有理$n$形式,$f\in k(X)$是有理函数,那么有$\mathrm{Div}(f\omega)=\mathrm{Div}f+\mathrm{Div}\omega$.
	\item $\mathrm{Div}\omega\ge0$当且仅当$\omega$是$X$上的正则$n$形式.
	\item 典范除子类.我们解释过$\Omega^n(X)$是$k(X)$上一维线性空间,于是任意两个非零有理$n$形式$\omega_1,\omega_2$都满足存在有理函数$f$使得$\omega_2=f\omega_1$,于是任意两个非零有理$n$形式都是线性等价的,这个等价类称为$X$上的典范除子类.记作$K$或者$K_X$.
	\item 固定一个有理$n$形式$\omega_1\in\Omega^n(X)$,那么每个有理$n$形式$\omega$可以表示为$\omega=f\omega_1$,第二条说明$\omega$是正则$n$形式当且仅当$\mathrm{Div}f+\mathrm{Div}\omega_1\ge0$,也即当且仅当$f\in\mathscr{L}(\mathrm{Div}(\omega_1))$,于是我们证明了$\Omega^n[X]=\mathscr{L}(\mathrm{Div}(\omega_1))$.于是有$\dim_k\Omega^n[X]=l(K_X)$是典范除子类的维数.
	\item 亏格.设$X$是非奇异射影曲线,设$D$是其上一个除子,我们解释过此时$l(D)$是有限的.于是特别的对典范除子类$K_X$就有$\dim_k\Omega^1[X]=l(K_X)$是有限的,这个数字称为$X$的亏格,记作$g=g(X)$.
\end{enumerate}



\newpage
\subsection{Riemann-Roch定理}

定理内容.设$X$是非奇异射影曲线,设$D$是任意除子,设$K$是$X$上的典范除子,设$g$是$X$的亏格,那么有等式:$$l(D)-l(K-D)=\deg D-g+1$$

在给出证明前还需要引入一些概念.不可约非奇异曲线$X$上的分布(distribution)是指一个映射,它把$X$中的每个点$x$映射为一个有理函数$r_x\in k(X)$,使得$v_x(r_x)\ge0$对几乎所有点$x\in X$成立.
\begin{enumerate}
	\item 容易定义分布的加法和乘法,并且结果仍然是分布.全体分布构成了基域$k$上的一个线性空间,记作$\mathscr{R}$.
	\item 对$X$上的除子$D$,记$\mathscr{R}(D)$表示这样的分布$r$构成的$\mathscr{R}$的线性子空间:如果记$D=\sum n_ix_i$,满足$v_{x_i}(r_{x_i})+n_i\ge0$,并且对任意$x\not=x_i$都有$v_x(r_x)\ge0$.
	\item 对每个有理函数$f\in k(X)$,我们可以把它视为一个处处取$f$的分布.那么这个对应是从$k(X)$到$\mathscr{R}$的单同态.
	\item 给定非奇异射影曲线$X$,对全部除子$D$,有$\deg D+1-l(D)$是上有界的.
	\item 定理.对非奇异射影曲线$X$上的除子$D$,空间$\bigwedge(D)=\mathscr{R}/(\mathscr{R}(D)+k(X))$总是基域上的有限维线性空间.这里$k(X)$是按照第三条视为$\mathscr{R}$的线性子空间.
	\begin{proof}
		
		假设除子$D'\ge D$,我们先证明$\mathscr{R}(D')/\mathscr{R}(D)$的维数是$\deg D'-\deg D$,从而它是有限维的:设$D'-D=\sum m_iP_i$,任取分布$r\in\mathscr{R}(D')$,条件$r\in\mathscr{R}(D)$等价于讲,如果在每个$P_i$处$r_{P_i}$的洛朗展开为$a_{m'}t^{m'}+\cdots+a_{m'+m_i-1}t^{m'+m_i-1}+\cdots$,这里$m'=v(r_{P_i})$,那么有$a_{m'}=\cdots=a_{m'+m_i-1}=0$.在点$P_i$处的条件个数有$m_i$个,一共的条件个数是$\sum m_i=\deg D'-\deg D$,这些条件明显是线性无关的,于是维数是$\deg D'-\deg D$.
		
		接下来证明$(\mathscr{R}(D')+k(X))/(\mathscr{R}(D)+k(X))$的维数是$(\deg D'-l(D'))-(\deg D-l(D))$,从而它是有限维的:考虑典范满同态$\mathscr{R}(D')\to(\mathscr{R}(D')+k(X))/(\mathscr{R}(D)+k(X))$,它的核是$\mathscr{R}(D')\cap(\mathscr{R}(D)+k(X))$.但是一般来讲如果三个子空间$A,B,C$满足$B\subseteq A$,那么$A\cap(B+C)=B+(A\cap C)$,结合$\mathscr{R}(D')\cap k(X)=\mathscr{L}(D')$,于是有$(\mathscr{R}(D')+k(X))/(\mathscr{R}(D)+k(X))\cong\mathscr{R}(D')/(\mathscr{R}+\mathscr{L}(D'))$.另外我们有$(\mathscr{R}(D)+\mathscr{L}(D'))/(\mathscr{R}(D))\cong\mathscr{L}(D')/\mathscr{L}(D)$.两个商空间相除的维数是两个商空间维数的差,这得到结论.
		
		我们解释过$l(D)-\deg D$是下有界的,设取除子$D_0$时取到最小整数,即对任意除子$D$有$l(D)-\deg D\ge l(D_0)-\deg D_0$,特别的,取$D=D_0$和$D'\ge D$,那么上一段证明了$(\mathscr{R}(D')+k(X))/(\mathscr{R}(D_0)+k(X))$的维数只能是零,换句话讲$\mathscr{R}(D')+k(X)=\mathscr{R}(D)+k(X)$.但是对任意分布$r\in\mathscr{R}$,都可以取到除子$D'$使得$r\in\mathscr{R}(D')$,并且此时必然满足$D'\ge D_0$.于是$\cup\mathscr{R}(D')=\mathscr{R}$,这里的并取遍满足$D'\ge D_0$的除子$D'$.于是有$\cup(\mathscr{R}(D')+k(X))=\mathscr{R}$,于是只能有$\mathscr{R}(D_0)+k(X)=\mathscr{R}$.这说明定理中的空间是有限维的.
	\end{proof}
    \item 对非奇异射影曲线$X$上的除子$D$,记$\lambda(D)=\dim\bigwedge(D)$,那么有$l(D)-\lambda(D)=\deg D-\lambda(0)+1$.
    \begin{proof}
    	
    	在这个新的记号$\lambda(D)$下,上一个定理证明中我们证明了对除子$D'\ge D$总有$\lambda(D)-\lambda(D')=(\deg D'-l(D'))-(\deg D-l(D))$.这个等式实际上对所有除子$D$和$D'$成立,因为我们总可以选取一个除子$D''$满足$D''\ge D$和$D''\ge D'$.再取$D'=0$,结合$l(0)=1$和$\deg 0=0$得到结论.
    \end{proof}
\end{enumerate}

按照上面最后一个命题,为证明Riemann-Roch定理,只需验证$\lambda(D)=l(K-D)$,这两个数字分别是空间$\bigwedge(D)$和$\mathscr{L}(K-D)$的维数.我们只需验证它们互为对偶空间.为此我们需要构造$\bigwedge(D)\times\mathscr{L}(K-D)\to k$的双线性型,并且它满足非退化条件:如果$(u,v)=0$对任意$v$成立当且仅当$u=0$;$(u,v)=0$对任意$u$成立当且仅当$v=0$.
\begin{enumerate}
	\item 这里$\mathscr{L}(K-D)$同构于满足$\mathrm{Div}\omega\ge D$的有理1形式$\omega$构成的$\Omega^1(D)$的线性子空间.这个空间我们记作$\Omega^1(D)$.
	\item 留数.选取点$x\in X$处的局部参数$t$,每个有理1形式$\omega$可以表示为$f\mathrm{d}t$,其中$f\in k(X)$.考虑$f$在点$x$处的洛朗级数$f=a_{-m}t^{-m}+\cdots+a_0+a_1t+\cdots$.这里$a_{-1}$称为有理1形式$\omega$在点$x$处的留数,记作$\mathrm{Res}_x\omega$.
	\item 无论特征有限还是无限,留数是一个内蕴概念,它不依赖于局部参数的选取.
	\item 现在我们构造$\mathscr{R}\times\Omega^1(X)\to k$的双线性型为,$(r,\omega)=\sum_x\mathrm{Res}_x(r_x\omega)$.这个定义是良性的因为对几乎所有的$x\in X$都有$r_x\in\mathscr{O}_{X,x}$并且$\omega$和$r_x\omega$都在点$x$正则,于是洛朗展开中的负次数项系数都是零.最后我们把这个二次型视为$\bigwedge(D)\times\Omega^1(D)\to k$的双线性型.需要验证两个条件:
	\begin{itemize}
		\item $(\mathscr{R}(D),\Omega^1(D))=0$:任取$r\in\mathscr{R}(D)$和$\omega\in\Omega^1(D)$,那么$r_x\omega$在点$x$处正则,于是$\mathrm{Res}_x(r_x\omega)=0$对全部$x\in X$成立.
		\item $(k(X),\Omega^1(D))=0$:选取$f\in k(X)$和$\omega\in\Omega^1(D)$,这个等式即$\sum_x\mathrm{Res}_x\omega=0$,此为留数定理,只对射影曲线成立.
	\end{itemize}
    \item 最后我们证明非退化条件.
    \begin{itemize}
    	\item 如果$\omega\in\Omega^1(D)$使得$(r,\omega)=0$对任意$r\in\bigwedge(D)=\mathscr{R}/(\mathscr{R}(D)+k(X))$成立,那么$\omega=0$:问题归结为设$(r,\omega)=0$对任意$r\in\mathscr{R}$成立,假设$\omega\not=0$,那么存在某个点$x$出现在$\mathrm{Div}\omega$中,使得系数$n\not=0$.于是存在局部参数$t$使得$\omega$可以表示为$f\mathrm{d}t$,其中$v_x(f)=n$.选取分布$r$满足$r_x=t^{-n-1}$,对每个$y\not=x$有$r_y=0$,那么$(r,\omega)=\mathrm{Res}_(r_x\omega)\not=0$,这矛盾.
    	\item 等价于证明$\omega\mapsto (-,\omega)$是从$\Omega^1(D)$到对偶空间$\bigwedge(D)^*$的满射.考虑全体$\mathscr{R}/k(X)$上的线性函数,满足在某个$\mathrm{R}(D)$上为零,构成的空间记作$\bigwedge$.现在$\mathscr{R}/k(X)$是$k(X)$模,我们断言这个模是一维的:任取有理微分1形式$\omega\not-0\in\Omega^1(X)$,定义了一个非零函数$\varphi\in\bigwedge$.于是归结为证明任意非零函数$\varphi,\psi\in\bigwedge$是在$k(X)$上线性相关的.假设它们是线性无关的.不妨设存在除子$D$使得$\varphi,\psi\in\mathscr{R}(D)$,可取足够大的正整数$n$,取函数$f,g\in\mathscr{L}(nP)$,按照线性无关条件,$f\varphi+g\psi$总是不同的函数,于是这构成了维数为$2l(nP)$的空间,并且这个空间落在$\bigwedge(D-nP)$中,于是有$2l(nP)\le\lambda(D-nP)$.但是我们证明过$l(D)-\lambda(D)=\deg D+c$,这里$c$是一个不依赖于$D$的固定常数.于是如果$n>\deg D$,那么$l(D-nP)=0$,导致$\lambda(D-nP)=n+c'$,但是$l(nP)\ge n+c''$,矛盾.
    	
    	这是一维的说明每个$\varphi\in\bigwedge$都被某个微分形式$f\omega$定义,于是问题归结为证明$(r,\omega)=0$对任意$r\in\mathscr{R}(D)$成立,那么$\omega\in\Omega^1(D)$:假设$\omega\not\in\Omega^1(D)$,那么存在点$x$出现在除子$D$中使得系数$n\not=0$.选取$x$的局部参数$t$使得$\omega=f\mathrm{d}t$,使得$v_x(f)<n$.构造分布$r$满足$r_x=t^{-v_x(f)-1}$和$r_y=0,\forall y\not=x$,那么$r\in\mathscr{R}(D)$,但是$(r,\omega)\not=0$,这矛盾.
    \end{itemize}
\end{enumerate}

Riemann-Roch定理的一些推论.
\begin{itemize}
	\item 取$D=K$,按照$l(0)=1$和$l(K)=g$,说明$\deg K=2g-2$.
	\item 如果$\deg D>2g-2$,那么$l(D)=\deg(D)-g+1$.
	\item 如果$g=0$那么$X\cong\mathbb{P}^1$
	\item 如果$g=1$那么$X$同构于$\mathbb{P}^2$的cubic曲线.
\end{itemize}

\newpage
\section{代数群}
\subsection{基本内容}

一个$k$代数群$G$是指$k$代数簇范畴中的群对象,也即群结构上的乘法运算$G\times G\to G$(这里$G\times G$当然取的是Zariski拓扑)和取逆映射$G\to G$都是$k$代数簇态射.两个$k$代数群之间的态射或者$k$态射是指它是$k$代数簇态射同时也是群同态.
\begin{enumerate}
	\item 例子.
	\begin{enumerate}[(1)]
		\item 加法群$\mathbb{G}_{a,k}$.此即$\mathbb{A}^1_k$赋予加法.
		\item 乘法群$\mathbb{G}_{m,k}$.此即$\mathbb{A}^1_k-\{0\}$赋予乘法.
		\item 一般线性群$\mathrm{GL}(n,k)$.首先$\mathrm{M}(n,k)$可以视为空间$\mathbb{A}_k^{n^2}$.进而$\mathrm{GL}(n,k)$可以视为$\mathbb{A}_k^{n^2}$的主开集$D(\det(T_{ij}))$.这个主开集上矩阵乘法是被多项式描述的,从而是代数簇态射,于是一般线性群可以视为$k$代数群.此时有$\mathrm{GL}(1,k)=\mathbb{G}_{m,k}$.
	\end{enumerate}
	\item 关于$G$的单位分支.代数群$G$的单位元所在的连通分支称为它的单位分支,记作$G^0$.
	\begin{enumerate}[(1)]
		\item 我们知道$k$代数簇上总有正则点,进而按照代数群上平移总是同构,说明$k$代数群$G$的每个点都是正则点.特别的这说明$G$上不可约分支和连通分支是一致的.
		\item $G^0$是$G$的正规子群,并且陪集分解恰好是$G$的连通分支分解.这里$G^0$甚至是有限指数的,因为$G$是诺特空间.
		\item $G$的有限指数闭子群一定也是开子群,进而它一定包含了$G^0$.
		\item 于是一个代数群$G$是连通的当且仅当$G=G^0$.
	\end{enumerate}
	\item 关于子群.设$G$是$k$代数群.
	\begin{enumerate}[(1)]
		\item 引理.如果$U,V$是$G$的任意两个稠密开子集,那么$UV=G$.
		\begin{proof}
			
			任取$x\in G$,那么$xV^{-1}$也是稠密开集,于是存在$y\in xV^{-1}\cap U$.进而存在$z\in V$使得$x=yz$,于是$UV=G$.
		\end{proof}
		\item 设$H$是$G$的任意子群,那么$\overline{H}$仍然是$G$的子群(这件事对拓扑群已经成立了).特别的如果$H$是可构造的子群,那么$H$一定是闭子群.
		\begin{proof}
			
			事实上如果$H$是可构造集,那么它包含了$\overline{H}$的某个稠密开子集$U$.于是按照上面引理有$\overline{H}=UU\subseteq HH=H$.
		\end{proof}
		\item 设$A,B$是$G$的两个闭子群,如果$B\subseteq\mathrm{N}_G(A)$(正规化子),那么$AB$仍然是$G$的闭子群.
		\begin{proof}
			
			从$B\subseteq\mathrm{N}_G(A)$得到$AB$是$G$的子群.并且它作为$A\times B$在$G\times G\to G$下的像,得到$AB$是可构造子集.于是上一条得证.
		\end{proof}
	\end{enumerate}
	\item 设$\varphi:G\to G'$是代数群同态.
	\begin{enumerate}[(1)]
		\item $\ker\varphi$是$G$的闭子群.
		\item $\mathrm{Im}\varphi$是$G'$的闭子群.
		\item $\varphi(G^0)=\varphi(G)^0$.
		\item $\dim G=\dim\ker\varphi+\dim\mathrm{Im}\varphi$.
	\end{enumerate}
    \begin{proof}
    	
    	(1)是因为连续性.(2)是因为$\varphi(G)$是$G'$的可构造子群.(3)是因为$\varphi$限制在$G^0\to G'$上仍然是代数群态射,进而(2)说明$\varphi(G^0)$是$G'$的闭子群,并且它还是连通的,于是$\varphi(G^0)\subseteq\varphi(G)^0$.另一方面$\varphi(G^0)$作为$\varphi(G)$的有限指数闭子群,它就得包含$\varphi(G)^0$.(4)是因为按照维数和纤维的性质存在$x\in\varphi(G)$使得$\dim G=\dim\varphi(G)+\dim\varphi^{-1}(x)$.但是这里都是代数群,所以这个等式实际上对任意$x\in\varphi(G)$成立,特别的取$x=0$得到结论.
    \end{proof}
    \item 子集生成的闭子群.设$G$是$k$代数群,设$M$是$G$的子集,全体包含$M$的闭子群的交称为子集$M$生成的闭子群,记作$\mathscr{A}(M)$.
    \begin{enumerate}[(1)]
    	\item 设$G$是$k$代数群,设$\{f_i:X_i\to G\mid i\in I\}$是一族代数簇态射,其中每个$X_i$都是不可约簇,并且幺元$e=e_G$落在每个$Y_i=f_i(X_i)$中.取$M=\cup_{i\in I}Y_i$.那么:
    	\begin{enumerate}[(i)]
    		\item $\mathscr{A}(M)$是$G$的连通闭子群.
    		\item 存在有限个指标$i_1,\cdots,i_n$使得$\mathscr{A}(M)=Y_{i_1}^{e_1}\cdots Y_{i_n}^{e_n}$,其中$e_i=\pm1$.
    	\end{enumerate}
        \begin{proof}
        	
        	不妨扩大这族代数簇态射使得它包含了每个$f_i^{-1}$.对$I$的有限序列$a=(a_1,\cdots,a_n)$,记$Y_a=Y_{a_1}\cdots Y_{a_n}$,按照它是不可约簇$X_{a_1}\times\cdots\times X_{a_n}$在$G$中的像,于是它是$G$的不可约的可构造子集.进而$\overline{Y_a}$是包含$e$的不可约簇.按照$G$是诺特空间,全体可以表示为$\overline{Y_a}$的不可约闭子集中存在一个极大元,把对应的有限序列$a$依旧记作$a$.
        	
        	\qquad
        	
        	取$I$中的两个有限序列$b=(b_1,\cdots,b_n),c=(c_1,\cdots,c_m)$,记$(b,c)=(b_1,\cdots,b_n,c_1,\cdots,c_m)$.我们断言有$\overline{Y_b}\overline{Y_c}\subseteq\overline{Y_{(b,c)}}$.事实上任取$x\in Y_c$,那么右乘$x$这个同构就把$Y_b$映入$Y_{(b,c)}$,进而把$\overline{Y_b}$映入$\overline{Y_{(b,c)}}$.于是得到$\overline{Y_b}Y_c\subseteq\overline{Y_{(b,c)}}$.于是对$x\in\overline{Y_b}$就有左乘$x$把$Y_c$映入$\overline{Y_{(b,c)}}$,进而它把$\overline{Y_c}$映入$\overline{Y_{(b,c)}}$.综上$\overline{Y_b}\overline{Y_c}\subseteq\overline{Y_{(b,c)}}$.
        	
        	\qquad
        	
        	按照$\overline{Y_a}$是极大的,对任意$I$中的有限序列$b$就有$\overline{Y_a}\subseteq\overline{Y_a}\overline{Y_b}\subseteq\overline{Y_{(a,b)}}=\overline{Y_a}$.取$b=a$和$b$为$a$的逆序列(此即如果记$a=(a_1,\cdots,a_n)$,我们之前扩充$I$使得每个$f_{a_i}^{-1}$也在指标$I$中,把它对应的指标记作$b_i$,那么取$b=(b_1,\cdots,b_n)$),就得到$\overline{Y_a}$是$G$的闭子群,并且包含了全部$Y_i$(否则$Y_a$再乘上$Y_i$取闭包和极大性矛盾).进而有$\overline{Y_a}=\mathscr{A}(M)$.最后按照$Y_a$是可构造子集,并且它在$\overline{Y_a}$中稠密(因为不可约性),就有$\overline{Y_a}=Y_aY_a=Y_{(a,a)}$.
        \end{proof}
        \item 推论.设$G$是代数群,设$Y_i$是一族连通闭子群,并且在群意义下生成了整个$G$,那么$G$是连通的.
    \end{enumerate}
\end{enumerate}
\subsection{代数群作用}

设$G$是代数群,设$X$是代数簇(簇要求分离条件),那么$G$在$X$上的群作用是指取定一个代数簇态射$\varphi:G\times X\to X$满足:
\begin{enumerate}[(a)]
	\item $g_1(g_2x)=(g_1g_2)x,\forall g_1,g_2\in G,x\in X$.
	\item $ex=x,\forall x\in X$.
\end{enumerate}
\begin{enumerate}
	\item 设$G$作用在代数簇$X$上.
	\begin{enumerate}[(1)]
		\item 设$Y,Z\subseteq X$是子集,其中$Z$是闭子集.那么$\mathrm{Tran}_G(Y,Z)=\{g\in G\mid gY\subseteq Z\}$是$G$的闭子集.
		\item 对任意$x\in X$,稳定子$G_x=\{g\in G\mid gx=x\}$是$G$的闭子群.特别的,$X$的子集$Y$的中心化子$\mathrm{C}_G(Y)=\cap_{y\in Y}G_y$是$G$的闭子群.
		\item 对任意$g\in G$,稳定子集$X^g=\{x\in X\mid gx=x\}$是$X$的闭子集.特别的$X^G=\cap_{g\in G}X^g$是$X$的闭子集.
		\item 如果$G$是连通的,那么$G$作用在$X$的不可约分支集合上是平凡作用.特别的如果$X$是有限集合(此时它是离散拓扑),那么$G$在$X$上的作用总是平凡的.
	\end{enumerate}
	\begin{proof}
		
		(1)是因为$\mathrm{Tran}_G(Y,Z)=\cap_{y\in Y}\mathrm{Tran}_G(\{y\},Z)$.于是归结为设$Y=\{y\}$是单点集合,那么$\mathrm{Tran}_G(Y,Z)$就是闭子集$Z$在映射$\varphi_y:G\to X$,$g\mapsto gy$的原像,从而是闭子集.(2)是因为$G_x=\mathrm{Tran}_G(\{x\},\{x\})$.(3)是因为$X^g$是对角线在连续映射$X\to X\times X$,$x\mapsto (x,gx)$下的原像,这里对角线是闭子集是簇的分离公理.(4):任取$X$的一个不可约分支$X_0$,这是闭子集,按照(1)有$H=\mathrm{Tran}_G(X_0,X_0)$是闭子群.由于$G$在$X$的全体(有限个)不可约分支集合上作用,于是$H$在$G$中是有限指数的,但是有限指数闭子群也是开子群,从$G$连通就得到$H=G$.
	\end{proof}
	\item 推论.设$G$是代数群,设$H$是闭子群,那么$\mathrm{N}_G(H)$和$\mathrm{C}_G(H)$都是闭子群.特别的对任意$g\in G$就有$\mathrm{C}_G(g)$是闭子群.
	\begin{proof}
		
		$\mathrm{C}_G(H)$是闭子群上一条得证了,对$g\in G$取$H=\langle g\rangle$就有$\mathrm{C}_G(H)=\mathrm{C}_G(g)$.最后让$G$共轭作用在$X=G$上,那么有$\mathrm{N}_G(H)=\mathrm{Tran}_G(H,H)$,上一条得到这是闭子群.
	\end{proof}
	\item 取$V\cong\mathbb{A}_k^n$是$n$维仿射空间,我们称一个群同态$G\to\mathrm{GL}(V)$为$G$的线性表示,称一个代数群态射$\rho:G\to\mathrm{GL}(V)$称为代数群$G$的有理表示.这也即仿射空间$V$上赋予了一个$G$作用结构,此时也称$V$是一个$G$模.下面是表示论的一些约定:
	\begin{enumerate}[(1)]
		\item 如果$\varphi:G\to\mathrm{GL}(V)$是有理表示,它的对偶表示$\varphi^*:G\to\mathrm{GL}(V^*)$定义为$(gf)(v)=f(g^{-1}v)$.
		\item 设$\varphi:G\to\mathrm{GL}(V)$和$\psi:G\to\mathrm{GL}(W)$是同一个代数群的两个有理表示,它们的张量积$\varphi\otimes\psi:G\to\mathrm{GL}(V\otimes W)$定义为仿射空间$V\otimes W$上赋予$G$作用为$g(v\otimes w)=(gv)\otimes(gw)$.
		\item 我们有典范同构$V^*\otimes V\cong\mathrm{End}(V)$为把$f\otimes v$对应于$V$上自同态$w\mapsto f(w)v$.进而一个有理表示$\varphi:G\to\mathrm{GL}(V)$就诱导了$G\to\mathrm{GL}(\mathrm{End}(V))$,这个态射也即$gv=gvg^{-1}$,其中$v\in\mathrm{End}(V)$.
		\item 如果$G$作用在两个代数簇$X,Y$上,那么一个代数簇态射$\varphi:X\to Y$称为$G$不变的,如果有$\varphi(gx)=g\varphi(x)$
	\end{enumerate}
    \item 闭轨道存在性.设代数群作用在非空的代数簇$X$上.那么$X$的每个轨道都是正则局部闭子集,并且它的拓扑边界是一些维数相较于$Y$严格更小的轨道的并.特别的,极小维数的轨道一定是闭的,于是闭轨道总是存在的.
    \begin{proof}
    	
    	取$y\in X$,它的轨道是$Y=Gy$.按照$Y$是$G\to X$,$g\mapsto gy$的像,所以它是可构造子集.于是$Y$包含了$\overline{Y}$的某个开子集.但是$G$在$Y$上的作用是可迁的,就得到$Y$是$\overline{Y}$的开子集,同样的原因结合$Y$上至少存在一个正则点说明$Y$是正则的.我们有$\overline{Y}-Y$是闭子集,但是这个闭子集在$G$作用下不变,于是它是维数严格小于$\dim\overline{Y}$的一些轨道的并.
    \end{proof}
    \item 半直积.
    \begin{enumerate}[(1)]
    	\item 设代数群$G$作用在代数群$N$上,使得每个$g\in G$是$N$上的代数群自同构.那么在代数簇$N\times G$上赋予乘法$(x_1,y_1)(x_2,y_2)=(x_1(y_1x_2),y_1y_2)$构成一个代数群,记作$N\rtimes G$.我们有代数群的短正合列:
    	$$\xymatrix{1\ar[r]&N\ar[r]^i&N\rtimes G\ar[r]^{\pi}&G\ar[r]\ar@/^1pc/[l]^{\sigma}&e}$$
    	其中$i$是$n\mapsto(n,e_G)$使得$N$是$N\rtimes G$的正规子群,$\pi$是$(n,g)\mapsto g$,$\sigma$,$g\mapsto(e_N,g)$是$\pi$的截面.
    	\item 反过来如果给定代数群$G'$,取一个正规闭子群$N$和闭子群$G$,使得$N\cap G=\{e\}$.那么典范态射$N\rtimes G\to G'$,$(n,g)\mapsto ng$是同构.
    	\item 例如上三角矩阵群$\mathrm{T}(n,k)$是对角矩阵群$G=\mathrm{D}(n,k)$经对角线全为1的上三角矩阵群$N=\mathrm{U}(n,k)$的半直积.
    \end{enumerate}
    \item 设代数群$G$作用在仿射簇$X$上.那么$G$就也作用在坐标环$K[X]$上:记$g^{-1}$对应的$X$上的自同构所诱导的$K[X]$上的自同构为$\tau_g$.此即对$f\in K[X]$和$x\in X$有$(\tau_gf)(x)=f(g^{-1}x)$.进而$\tau:G\to\mathrm{GL}(K[X])$,$g\mapsto\tau_g$就是群同态.另外这里$\tau_g$实际上是$K[X]$的$K$代数同构.
    \item 代数群在坐标环上的左/右平移作用.
    \begin{enumerate}[(1)]
    	\item 设代数群$G$左(或者右)平移作用在自身上.那么$G$上左乘$g$和右乘$g^{-1}$对应于$K[G]$上的左乘$g^{-1}$和右乘$g$的映射,分别记作$L_g$和$R_g$.进而得到群同态$L:G\to\mathrm{GL}(K[G])$和$R:G\to\mathrm{GL}(K[G])$分别为$(L_gf)(x)=f(g^{-1}x)$和$(R_gf)(x)=f(xg)$.这里$L_g$和$R_{g'}$是交换的.
    	\item 我们断言如果$H\le G$是闭子群,设理想$I\subseteq K[G]$由那些$H$上取值为零的正则函数构成,那么有$H=\{g\in G\mid R_g(I)\subseteq I\}$.
    	\begin{proof}
    		
    		一方面任取$h,h'\in H$,任取$f\in I$,那么有$(R_hf)(h')=f(h'h)=0$.于是有$R_hf\subseteq I$.反过来如果$g\in G$满足$R_g(I)\subseteq I$,对任意$f\in I$,有$R_gf(e)=f(g)=0$,于是$g\in H$(零点定理).
    	\end{proof}
        \item 函子性.设$H\le G$是闭子群,设$h\in H$,那么有如下交换图表:
        $$\xymatrix{K[G]\ar[rr]^{R_h}\ar[d]&&K[G]\ar[d]\\K[H]\ar[rr]_{R_h}&&K[H]}$$
        \item 设$\varphi:G\to G'$是代数群之间的满态射,那么$\varphi^*$是单射,把$K[G']$映为$K[G]$的子环.那么任取$g\in G$,就有$R_{\varphi(g)}$是$R_g$在子环$K[G']$上的限制.类似的有$D_{\mathrm{d}\varphi(v)}$是$D_v$在子环$K[G']$上的限制.
    \end{enumerate}
    \item 引理.设代数群$G$作用在仿射簇$X$上,设$F$是$K[X]$的有限维子空间.
    \begin{enumerate}[(1)]
    	\item 存在$K[X]$的有限维$G$不变$K$子空间$E$包含了$F$,这里$K[X]$上的$G$作用是之前定义的$\tau_g$.
    	\item $F$是$G$不变的当且仅当$\varphi^*F\subseteq K[G]\otimes_KF$,其中$\varphi:G\times X\to X$是$(g,x)\mapsto g^{-1}x$.
    \end{enumerate}
    \begin{proof}
    	
    	(1):不妨设$F=Kf$是一维的,其中$f\in K[X]$.记$\varphi^*f=\sum_i\alpha_i\otimes\beta_i\in K[G]\otimes_KK[X]$.任取$g\in G$和$x\in X$,我们有$(\tau_gf)(x)=f(g^{-1}x)=f\circ\varphi(g,x)=\varphi^*f(g,x)=\sum_i\alpha_i(g)\beta_i(x)$.于是有$\tau_xf=\sum_i\alpha_i(x)\beta_i$.这些$\beta_i$张成了$K[X]$的有限维子空间,它包含了全部$\tau_gf$.于是全体$\{\tau_gf\mid g\in G\}$张成了$K[X]$的有限维子空间,并且这明显是$G$不变的.
    	
    	\qquad
    	
    	(2):一方面如果$\varphi^*F\subseteq K[G]\otimes_KF$,那么任取$f\in F$,就有$\varphi^*f=\sum_i\alpha_i\otimes\beta_i$,其中$\beta_i\in F$.进而$\tau_gf=\sum_i\alpha_i(g)\beta_i\in F$.于是$F$是$G$不变的.反过来如果$F$是$G$不变的,选取一组$K$基$f_1,\cdots,f_n$,把它延拓为$K[X]$上的一组基$\{f_i\}\coprod\{f_j'\}$.任取$f\in F$,记$\varphi^*f=\sum\alpha_i\otimes f_i+\sum\alpha_j'\otimes f_j'$.那么有$\tau_gf=\sum\alpha_i(g)f_i+\sum\alpha_j'(g)f_j'$.按照它在$F$中,就有$\alpha_j'(g)=0,\forall g\in G$,于是这些$\alpha_j'=0$,于是$\varphi^*f\in K[G]\otimes_KF$.
    \end{proof}
    \item 代数群上的Cayley定理.设$G$是仿射代数群,那么它同构于某个$\mathrm{GL}(n,K)$的闭子群.
    \begin{proof}
    	
    	选取$K$代数$K[G]$的一组代数生成元$\{f_1,\cdots,f_n\}$,设它们张成的$K$子空间为$F$.让$G$右平移作用在自身上.那么上面引理的(1)说明存在$K[G]$的有限维$G$不变子空间$E$包含了$F$.适当给$\{f_1,\cdots,f_n\}$添加有限个元,可设它们是也$E$的一组基.取$\varphi:G\times G\to G$为$(x,y)\mapsto yx$.那么上面引理的(2)说明存在$m_{ij}\in K[G]$使得$\varphi^*f_i=\sum_jm_{ij}\otimes f_j$.进而有$R_gf_i=\sum_km_{ij}(g)f_j$.于是$R_g$限制在$E$上关于基$\{f_1,\cdots,f_n\}$的矩阵是$(m_{ij}(g))$.这诱导了$\psi:G\to\mathrm{GL}(n,K)$为$g\mapsto(m_{ij}(g))$是代数群态射.
    	
    	\qquad
    	
    	我们有$f_i(g)=f_i(eg)=\sum_jm_{ij}(g)f_j(e)$,也即$f_i=\sum_jf_j(e)m_{ij}$.于是这些$m_{ij}$代数张成了整个$K[G]$,所以它们不能全为零,于是$\psi$是单射.我们解释过代数群之间态射的像是闭子群.记闭子群$\mathrm{Im}\psi=G'\le\mathrm{GL}(n,K)$.最后验证$\psi$限制为是$\psi_0:G\to G'$的簇同构.而这是因为$\psi_0^*:K[G']\to K[G]$,$T_{ij}\mapsto m_{ij}$是同构.
    \end{proof}
\end{enumerate}
\subsection{李代数}

设$G$是仿射代数群,它的坐标环记作$A=K[G]$,记$A$上的全体$K$导数构成的集合是$\mathrm{Der}_K(A)$,这是一个李代数.考虑全体左不变$K$导数构成的李子代数$\mathscr{L}(G)=\{D\in\mathrm{Der}_K(A)\mid D\circ L_g=L_g\circ D,\forall g\in G\}$,它称为$G$的李代数.
\begin{enumerate}
	\item 等价描述.设$G$是代数群,设$A=K[G]$,设$\mathrm{T}_eG$是幺元的切空间.
	\begin{enumerate}[(1)]
		\item 构造$\theta:\mathscr{L}(G)\to\mathrm{T}_eG$为$(\theta D)(f)=(Df)(e),\forall f\in A$.那么这是$K$线性同构.
		\item 进而$\mathrm{T}_eG$是一个李代数,记作$\mathfrak{g}$.它的乘法是这样描述的:把$G$上的乘法记作$\mu:G\times G\to G$,那么$\mu^*:A\to A\otimes_KA$.设$v,w\in\mathrm{T}_eG$,它们可以视为$A\to K$的导数,那么$K$线性映射$v\otimes w:A\otimes_KA\to K$就是$f\otimes f'\mapsto(vf)(wf')$.进而取$v\cdot w=(v\otimes w)\circ\mu^*:A\to K$.那么上述同构$\theta:\mathscr{L}(G)\cong\mathrm{T}_eG$就把$D_v\circ D_w$映射到$v\cdot w$.
		\item 如果$\varphi:G\to G'$是代数群态射,那么它的微分$\mathrm{d}\varphi_e:\mathfrak{g}\to\mathfrak{g}'$是李代数同态.
	\end{enumerate}
	\begin{proof}
		
		(1):对幺元$e$处的切向量$v$,记导数$D_v$为$D_v(f)(g)=v(L_{g^{-1}}f)$.$D_v$是导数是因为对任意$f_1,f_2\in A$和任意$g\in G$有:
		\begin{align*}
			D_v(f_1f_2)(g)&=v(L_{g^{-1}}(f_1f_2))\\&=v((L_{g^{-1}}f_1)(L_{g^{-1}}f_2))\\&=v(L_{g^{-1}}f_1)f_2(g)+v(L_{g^{-1}}f_2)f_1(g)\\&=D_v(f_1)f_2(g)+D_v(f_2)f_1(g)
		\end{align*}
		$D_v$是左不变的是因为对任意$f\in A$和$g_1,g_2\in G$有:
		\begin{align*}
			(D_v(L_{g_1}(f)))(g_2)&=v(L_{g_2^{-1}}L_{g_1}f)\\&=v(L_{g_2^{-1}g_1}f)\\&=D_vf(g_1^{-1}g_2)=(L_{g_1}(D_vf))(g_2)
		\end{align*}
		同态$\eta:v\mapsto D_v$是$K$线性的,容易验证它和$\theta$互逆.下面验证(2):记$\mu^*f=\sum_if_i\otimes f_i'$,那么对任意$g\in G$有:
		\begin{align*}
			D_vf(g)&=v(L_{g^{-1}}f)=v(\sum_if_i(g)f_i')\\&=\sum_if_i(g)v(f_i')
		\end{align*}
		于是$D_vf=\sum_iv(f_i')f_i$.进而有:
		\begin{align*}
			\theta(D_v\circ D_w)(f)(e)&=D_v\circ D_wf(e)\\&=D_v(\sum_iw(f_i')f_i)(e)\\&=\sum_iw(f_i')v(f_i)\\&=(v\otimes w)\sum_if_i\otimes f_i'\\&=(v\cdot w)(f)
		\end{align*}
		(3):设$\varphi:G\to G'$是代数群态射,我们要证明$\mathrm{d}\varphi_e:\mathfrak{g}\to\mathfrak{g}'$保李括号,任取$v,w\in\mathfrak{g}$,任取$f'\in K[G']$,记$v'=\mathrm{d}\varphi_e(v)$和$w'=\mathrm{d}\varphi_e(w)$和$f=\varphi^*f'$,我们有:
		\begin{align*}
			[v',w'](f')&=D_{v'}\circ D_{w'}f'(e)-D_{w'}\circ D_{v'}f'(e)\\&=v'(D_{w'}f')-w'(D_{v'}f')\\&=v(\varphi^*D_{w'}f')-w(\varphi^*D_{v'}f')
		\end{align*}
	    \begin{align*}
	    	(\mathrm{d}\varphi_e[v,w])f'&=[v,w]f\\&=D_v\circ D_wf(e)-D_w\circ D_vf(e)\\&=v(D_wf)-w(D_vf)
	    \end{align*}
	    于是问题归结为证明$\varphi^*(D_{v'}f')=D_vf$,也即$\varphi^*(D_{\mathrm{d}\varphi_e(v)}f')=D_v(\varphi^*f')$.任取$g_1\in G$有:
	    \begin{align*}
	    	\varphi^*(D_{\mathrm{d}\varphi_e(v)}f')(g_1)&=D_{\mathrm{d}\varphi_e(v)}f'(\varphi(g_1))\\&=\mathrm{d}\varphi_e(v)(L_{\varphi(g_1^{-1})}f')\\&=v(\varphi^*L_{\varphi(g_1^{-1})}f')\\&=v(L_{g_1^{-1}}\varphi^*f')\\&=D_v(\varphi^*f')(g_1)
	    \end{align*}
	    其中倒数第二个等号是因为,对任意$g_2\in G$有:
	    \begin{align*}
	    	\varphi^*L_{\varphi(g_1^{-1})}f'(g_2)&=L_{\varphi(g_1^{-1})}f'(\varphi(g_2))\\&=f'(\varphi(g_1g_2))\\&=\varphi^*f'(g_1g_2)\\&=L_{g_1^{-1}}\varphi^*f'(g_2)
	    \end{align*}
	\end{proof}
    \item 伴随表示.设代数群$G$共轭作用在自身上,也即对任意$g\in G$和$g'\in G$有$\mathrm{Int}(g)g'=gg'g^{-1}$.记微分$\mathrm{Ad}(g)=\mathrm{d}\mathrm{Int}(g)$,那么这是李代数$\mathfrak{g}$上的自同构.于是我们构造了$\mathrm{Ad}:G\to\mathrm{Aut}(\mathfrak{g})\subseteq\mathrm{GL}(\mathfrak{g})$,这是$G$的一个线性表示(后面会证明这实际上是代数群态射),称为$G$的伴随表示(adjoint representation).我们断言$\mathrm{Ad}(x)$在$\mathscr{L}(G)$上的作用为$\mathrm{Ad}(g)(D)=R_gDR_g^{-1}$.因为$R_g$和$L_{g'}$可交换,这个结果仍然在$\mathscr{L}(G)$中.
    \begin{proof}
    	
    	任取$g_1,g_2\in G$和$f\in A$有:
    	\begin{align*}
    		\mathrm{Ad}(g_1)(D)f(g_2)&=D(L_{g_2^{-1}}\mathrm{Int}(g_1)^*f)(e)\\&=D(L_{g_2^{-1}}L_{g_1^{-1}}R_{g_1^{-1}}f)(e)\\&=D(R_{g_1^{-1}}f)(g_2g_1)\\&=R_{g_1}DR_{g_1^{-1}}f(g_2)
    	\end{align*}
    \end{proof}
    \item 例子.
    \begin{enumerate}[(1)]
    	\item 取加法群$G=\mathbb{G}_a$,它的李代数是一维的,于是李括号恒为零.有$K[G]=K[T]$.我们断言$D=\frac{\mathrm{d}}{\mathrm{d}T}$是左不变的,于是它张成了整个李代数:任取多项式$p(T)$和$g\in G$,那么有$L_gDf(T)=f'(T-g)$和$DL_gf(T)=f'(T-g)$.
    	\item 取乘法群$G=\mathbb{G}_m$,它的李代数也是一维的,于是李括号也恒为零.有$K[G]=K[T,T^{-1}]$.取$D=\frac{T\mathrm{d}}{\mathrm{d}T}$,那么它是左不变的,进而张成整个李代数:$L_gD(T^n)=L_g(nT^n)=ng^{-n}T^n$和$DL_g(T^n)=D(g^{-n}T^n)=ng^{-n}T^n$.
    	\item 群一般线性群$G=\mathrm{GL}(n,K)$.它是$\mathbb{A}_k^{n^2}$的开子空间,所以它的维数也是$n^2$.于是幺元切空间的一组基是$\frac{\partial}{\partial T_{ij}}$.进而切空间对应于$\mathrm{M}(n,K)$中的元.任取两个切向量$v=(v_{ij})$和$w=(w_{ij})$,那么有$v\cdot w=\left((v\otimes w)\sum_hT_{ih}\otimes T_{hj}\right)_{ij}=\left(\sum_hv_{ih}w_{hj}\right)_{ij}$.于是$\mathrm{M}(n,K)$上的乘法吻合于矩阵乘法,于是$G$的李代数$\mathfrak{gl}(n,K)$就是$\mathrm{M}(n,K)$上赋予典范李括号$[M,N]=MN-NM$.
    \end{enumerate}
    \item 子群和李子代数.设$H$是代数群$G$的闭子群,典范包含态射$\eta:H\to G$,那么$\eta^*$是典范商同态$K[G]\to K[H]=K[G]/I$,其中$I$是那些在$H$上为零的正则函数构成的理想.进而$\mathrm{d}\eta_e:\mathrm{T}_eH\to\mathrm{T}_eG$是单射,它的像集是那些满足$v(I)=0$的切向量$v$构成的子空间.进而李代数同态$\mathrm{d}\eta:\mathfrak{h}\to\mathfrak{g}$是单射.也即此时$H$的李代数是$G$的李代数的子代数.
    \item 引理.设$H\le G$是闭子群,设$I$是$K[G]$的那些在$H$上取零的正则函数构成的理想.那么有$\mathfrak{h}=\{v\in\mathfrak{g}\mid D_v(I)\subseteq I\}$.
    \begin{proof}
    	
    	一方面设$v\in\mathfrak{h}$,取$f\in I$和$h\in H$,那么有$(D_vf)(h)=v(L_{h^{-1}}f)=0$,因为$L_{h^{-1}}f\in I$.于是$D_vf\in I$.反过来任取$v\in\mathfrak{g}$使得$D_vI\subseteq I$.那么对任意$f\in I$就有$(D_vf)(e)=v(f)=0$.于是$v(I)=0$,于是$v\in\mathfrak{h}$.
    \end{proof}
    \item 例子.
    \begin{enumerate}[(1)]
    	\item 考虑$\mathrm{GL}(n,K)$的由上三角矩阵构成的闭子群$\mathrm{T}(n,K)$.它是$\mathbb{A}^{n(n+1)/2}_K$被对角元乘积非零所定义的主开集.于是幺元的切空间就是$\mathbb{A}^{n(n+1)/2}_K$本身.于是可逆上三角矩阵群$\mathrm{T}(n,K)$的李代数就是上三角矩阵环$\mathfrak{t}(n,K)$.
    	\item 考虑$\mathrm{GL}(n,K)$的由对角矩阵构成的闭子群$\mathrm{D}(n,K)$.它的李代数$\mathfrak{d}(n,K)$就是全体对角矩阵构成的矩阵环.
    	\item 考虑$\mathrm{GL}(n,K)$的由对角线上均为1的上三角矩阵构成的闭子群$\mathrm{U}(n,K)$.它的李代数$\mathfrak{u}(n,K)$是全体严格上三角矩阵构成的矩阵环.
    	\item 考虑$\mathrm{GL}(n,K)$的行列式为1的矩阵构成的闭子群$\mathrm{SL}(n,K)$.它是$\mathrm{M}(n,K)$被单个多项式$f=\det(T_{ij})-1$所定义的.它幺元的切向量就要满足$\sum_{i,j}\frac{\partial f}{\partial T_{ij}(e)}T_{ij}=\sum_iT_{ii}=0$,此即迹零矩阵空间$\mathfrak{sl}(n,K)$.但是二者的维数都是$n^2-1$,所以$\mathrm{SL}(n,K)$的李代数就是$\mathfrak{sl}(n,K)$.
    \end{enumerate}
    \item 设$G$是代数群.
    \begin{enumerate}[(1)]
    	\item 设$\mu:G\to G$是乘法,那么对任意$v,w\in\mathfrak{g}$有$\mathrm{d}\mu_{(e,e)}(v,w)=v+w$.
    	\item 设$l:G\to G$是取逆,那么对任意$t\in\mathfrak{g}$有$\mathrm{d}l_e(v)=-v$.
    \end{enumerate}
    \begin{proof}
    	
    	(1):有典范同构$\mathrm{T}_{(e,e)}(G\times G)\cong\mathrm{T}_eG\oplus\mathrm{T}_eG=\mathfrak{g}\oplus\mathfrak{g}$.取$f\in K[G]$,记$\mu^*f=\sum_if_i\otimes f_i'$.于是有$f(gg')=\sum_if_i(g)f_i'(g')$.进而有$f=\sum f_i'(e)f_i=\sum_if_i(e)f_i'$.任取$(v,w)\in\mathfrak{g}\oplus\mathfrak{g}$,就有:
    	\begin{align*}
    		\mathrm{d}\mu_{(e,e)}(v,w)(f)&=(v,w)\mu^*f\\&=(v,w)\sum_if_i\otimes f_i'\\&=\sum_iv(f_i)f_i'(e)+\sum_iw(f_i')f_i(e)\\&=(v+w)(f)
    	\end{align*}
    	(2):考虑复合$\xymatrix{G\ar[r]^{(1,l)}&G\times G\ar[r]^{\mu}&G}$.它是恒为$e$的同态,于是微分为零映射.于是有:
    	\begin{align*}
    		0&=\mathrm{d}(\mu\circ(1,l))_e(v)\\&=\mathrm{d}\mu_{e,e}\left(\mathrm{d}(1,l)_e(v)\right)\\&=\mathrm{d}\mu_{(e,e)}(v,\mathrm{d}l_e(v))\\&=v+\mathrm{d}l_e(v)
    	\end{align*}
    \end{proof}
    \item 我们之前定义过$G$左/右平移作用在$K[G]$上:$(L_hf)(g)=f(h^{-1}g)$和$(R_hf)(g)=f(gh)$.也即两个群同态$R,L:G\to\mathrm{GL}(K[G])$.但是这未必是代数群态射,因为$K[G]$可能是无限维的.但是$K[G]$是那些被全体$R_h$(或者改为全体$L_h$)固定的有限维子空间的并.设$E$是这样的子空间,取一组基$\{f_1,\cdots,f_n\}$.记$R_hf_i=\sum_jm_{ij}(h)f_j$.换句话讲设$\varphi:G\times E\to E$满足$\varphi^*f_i=\sum_jm_{ij}\otimes f_j$.于是$\psi(h)=(m_{ij}(h))$是$G\to\mathrm{GL}(n,K)$的(关于这组基的)表示.
    \begin{enumerate}[(1)]
    	\item $K[G]$的被$\{m_{ij}\}$生成的子空间包含了$E$并且是左平移不变的:$(L_{h^{-1}}f_i)(g)=f_i(hg)=R_gf_i(h)=\sum_jm_{ij}(g)f_j(h)$.
    	\item $\mathrm{d}\psi_e:\mathfrak{g}\to\mathfrak{gl}(n,K)$是$v\mapsto(v(m_{ij}))$.
    	\item $D_v(E)\subseteq E$并且关于$\{f_1,\cdots,f_n\}$的矩阵就是$(v(m_{ij}))$:因为$D_v(f_i)(g)=v(L_{g^{-1}}f_i)=v(\sum_jf_j(g)m_{ij})=\sum_jf_j(g)v(m_{ij})$.
    \end{enumerate}
    \item 关于伴随表示.固定$h\in G$,那么$\mathrm{Ad}(h)$是李代数$\mathfrak{g}$上的自同构,我们解释过它可以表示为$D\mapsto R_hDR_h^{-1}$.
    \begin{enumerate}[(1)]
    	\item 考虑特殊情况$G=\mathrm{GL}(n,K)$,那么$\mathfrak{g}=\mathfrak{gl}(n,K)$.任取$g\in\mathrm{GL}(n,K)$和$v\in\mathfrak{gl}(n,K)$,我们断言$\mathrm{Ad}(g)(v)=gvg^{-1}$.进而(在选取一组基后)$\mathrm{Ad}:\mathrm{GL}(n,K)\to\mathrm{GL}(n^2,K)$是代数群同态.
    	\begin{proof}
    		
    		设$\mathfrak{gl}(n,K)$的坐标函数是$\{T_{ij}\}$,记$T=(T_{ij})$,任取$h\in G$.那么有:
    		\begin{align*}
    			(R_gT_{ij})(h)&=T_{ij}(hg)\\&=\sum_sh_{is}g_{sj}\\&=\sum_sT_{is}(h)g_{sj}\\&=(Tx)_{ij}(h)
    		\end{align*}
    	    于是$R_gT_{ij}$就是矩阵$Tg$的$(i,j)$项.类似的$L_gT_{ij}$就是矩阵$g^{-1}T$的$(i,j)$项.进而有:
    		\begin{align*}
    			(D_vT_{ij})(h)&=v(L_{h^{-1}}T_{ij})\\&=v(\sum_sh_{is}T_{sj})\\&=\sum_sh_{is}v(T_{sj})\\&=\sum_sh_{is}v_{sj}\\&=\sum_sT_{is}(h)v_{sj}\\&=(Tv)_{ij}(h)
    		\end{align*}
    		于是$D_vT_{ij}$是$Tv$的$(i,j)$项.进而有:
    		\begin{align*}
    			\mathrm{Ad}(g)(v)(T_{ij})&=R_g\circ D_v\circ R_g^{-1}(T_{ij})\\&=\sum_{r,s,t}\left(T_{ir}g_{rs}v_{st}g_{tj}^{-1}\right)\\&=gvg^{-1}(T_{ij})
    		\end{align*}
    	\end{proof}
        \item 考虑一般情况.把任意仿射代数群$G$视为某个$\mathrm{GL}(n,K)$的闭子群.此时$G$的李代数就等同于$\mathfrak{gl}(n,K)$的某个李子代数.我们知道代数群态射限制到闭子群和微分可交换,并且$\mathrm{Int}_G(g)$是$\mathrm{Int}_{\mathrm{GL}(n,K)}(g)$限制到$G$,于是$\mathrm{Ad}_G(g)$就是$\mathrm{Ad}_{\mathrm{GL}(n,K)}(g)$限制到$\mathfrak{g}$.综上得到:对任意代数群$G$,有$\mathrm{Ad}:G\to\mathrm{GL}(\mathfrak{g})$是代数群态射.并且如果$G$是仿射代数群,则$\mathrm{Ad}(g)$就是经$g\in G$的共轭同态.
        \item $\mathrm{Ad}:G\to\mathrm{GL}(\mathfrak{g})$的微分记作$\mathrm{ad}:\mathfrak{g}\to\mathrm{gl}(\mathfrak{g})$,它可以表示为$v(w)=[v,w]$,其中$v,w\in\mathfrak{g}$.
        \begin{proof}
        	
        	问题归结为设$G=\mathrm{GL}(n,K)$.【】
        	
        \end{proof}
        \item 李代数的Jacobi恒等式说明$\mathrm{ad}v$是$\mathfrak{g}$上的导数.另外$\ker\mathrm{ad}$恰好是$\mathfrak{g}$的中心,即和所有元可交换的元构成的集合.另外任取$G$的中心元$g$,那么$\mathrm{Int}g=1$,于是$\mathrm{Ad}(g)=1$,也即$Z(G)\subseteq\ker\mathrm{Ad}$.当$K$的特征为零时可以证明这是等式,特征$p$有反例.
        \item 推论.设$H\le G$是正规闭子群,那么李代数$\mathfrak{h}$是$\mathfrak{g}$的理想,也即对任意$v\in\mathfrak{g}$和$w\in\mathfrak{h}$有$[v,w]\in\mathfrak{h}$.
        \begin{proof}
        	
        	$H$是正规的说明$\mathrm{Int}(g),\forall g\in G$总固定$H$,进而$\mathrm{Ad}(g),\forall g\in G$总固定$\mathfrak{h}$.选取$\mathfrak{h}$的一组基,把它扩充为$\mathfrak{g}$的一组基,那么在这组基下$\mathrm{Ad}(g)$可以统一的表示为$2\times2$的分块上三角矩阵.进而它的微分$\mathrm{ad}(v)$也具有相同的矩阵形式,特别的有$[v,\mathfrak{h}]=\mathrm{ad}(g)(\mathfrak{h})\subseteq\mathfrak{h}$.
        \end{proof}
        \item 推论.设$H\le G$是闭子群,设$N=\mathrm{N}_G(H)$是正规化子.那么有$\mathfrak{n}\subseteq\mathfrak{n}_{\mathfrak{g}}(\mathfrak{h})=\{v\in\mathfrak{g}\mid[v,\mathfrak{h}]\subseteq\mathfrak{h}\}$.
        \begin{proof}
        	
        	我们解释过闭子群的正规化子$N$一定是闭子群.把上一条用在$N$的正规子群$H$上即可.
        \end{proof}
    \end{enumerate}
    \item 换位子.
    \begin{enumerate}[(1)]
    	\item 设$\gamma_h:G\to G$为$g\mapsto ghg^{-1}h^{-1}$(这当然不是同态),那么对任意$v\in\mathfrak{g}$有$(\mathrm{d}\gamma_h)_e(v)=(1-\mathrm{Ad}h)(v)$.
    	\begin{proof}
    		
    		取$\psi=\mathrm{Int}(h)\circ l$,那么$\gamma_h$可以表示为复合:
    		$$\xymatrix{G\ar[r]^{1,\psi}&G\times G\ar[r]^{\mu}&G}$$
    		这里$\mathrm{d}\psi_e(v)=\mathrm{d}(\mathrm{Int}h)(-v)=-\mathrm{Ad}(h)(v)$.进而有:
    		\begin{align*}
    			(\mathrm{d}\gamma_h)_e(v)&=\mathrm{d}\mu_{(e,e)}(v,-\mathrm{Ad}(h)(v))\\&=v-\mathrm{Ad}(h)(v)
    		\end{align*}
    	\end{proof}
        \item 设$A,B$是代数群$G$的闭子群,设$[a,b],a\in A,b\in B$生成的子群$[A,B]$的闭包为$C$,这是$G$的闭子群.那么它的李代数$\mathfrak{c}$包含全体形如$w-\mathrm{Ad}(a)(w),v-\mathrm{Ad}(b)(v),[v,w]$的元,其中$a\in A,b\in B,v\in\mathfrak{a}$和$w\in\mathfrak{b}$.特别的,后文会证明$H=[G,G]$总是闭的,于是总有$[\mathfrak{g},\mathfrak{g}]\subseteq\mathfrak{h}$.另外后文会证明特征零的时候事实上有$[\mathfrak{g},\mathfrak{g}]=\mathfrak{h}$.
        \begin{proof}
        	
        	任取$a\in A$,有$\gamma_a$把$B$映入$C$,于是$1-\mathrm{Ad}(a)$把$\mathfrak{b}$映入$\mathfrak{c}$.类似的$1-\mathrm{Ad}(b)$把$\mathfrak{a}$映入$\mathfrak{c}$.这解决了前两种情况.
        	
        	\qquad
        	
        	固定$v\in\mathfrak{a}$,把$\mathfrak{c}$视为仿射空间,那么它是自身的切空间考虑态射$\varphi:B\to\mathfrak{c}$为$b\mapsto v-\mathrm{Ad}(b)(v)$.它把$e$映为0,它的微分是$\mathrm{d}\varphi_e:\mathfrak{b}\to\mathfrak{c}$为$w\mapsto[v,w]$,于是这落在$\mathfrak{c}$中.
        \end{proof}
        \item $\gamma_g:G\to G$的核就是闭子群$\mathrm{C}_G(g)=\{h\in G\mid gh=hg\}$.我们断言有$\mathscr{L}(\mathrm{C}_G(g))\subseteq\mathfrak{c}_{\mathfrak{g}}(g)=\{v\in\mathfrak{g}\mid\mathrm{Ad}(g)(v)=v\}$(此为$g$的无穷小中心).如果$G=\mathrm{GL}(n,K)$那么这个包含式取等.
        \begin{proof}
        	
        	第一个命题是因为$(\mathrm{d}\gamma_g)_e=1-\mathrm{Ad}(g)$.第二个命题是因为我们解释过此时$\mathrm{Ad}(g)(v)=v$的元$v$就是满足$gv=vg$的元.于是$\mathrm{C}_G(g)$恰好由$\mathfrak{c}_{\mathfrak{g}}(g)$中的可逆矩阵构成.并且$\mathrm{C}_G(g)$是$\mathfrak{c}_{\mathfrak{g}}(g)$视为仿射空间的主开集,于是它们具有相同的(切空间)维数,于是这个包含关系取等.
        \end{proof}
    \end{enumerate}
    \item 设$G$是代数群.
    \begin{enumerate}[(1)]
    	\item 设$\varphi:G\to\mathrm{GL}(V)$是有理表示.$G$典范作用在$V^*$上为$(gf)(x)=f(g^{-1}x)$.进而$\mathfrak{g}$典范作用在$V^*$上为$(vf)(x)=-f(vx)$,这里$vx$理解为$\mathrm{d}\varphi(v)(x)$.
    	\begin{proof}
    		
    		【】
    	\end{proof}
    	\item 设$\varphi:G\to\mathrm{GL}(V)$和$\psi:G\to\mathrm{GL}(W)$是有理表示.$G$典范作用在$V\otimes W$上为$g(x\otimes y)=(gx)\otimes(gy)$.进而$\mathfrak{g}$典范作用在$V\otimes W$上为$v(x\otimes y)=(vx)\otimes y+x\otimes(vy)$,依旧把$vx$理解为$\mathrm{d}\varphi(v)(x)$.
    	\begin{proof}
    		
    		【】
    	\end{proof}
    	\item 进而一个有理表示$\varphi:G\to\mathrm{GL}(V)$就可以延拓为$V$上的张量代数$\mathrm{T}V$的表示.另外定义外代数的那个理想是在$\mathrm{GL}(V)$下不变的,进而诱导了外代数$\bigwedge V$上的表示.
    	\item 设$A$是一个有限维$K$代数,未必是结合代数,设$G$是由$A$上的$K$代数同构构成的$\mathrm{GL}(A)$的闭子群.那么$\mathfrak{g}$由$A$上的全部导数构成.
    	\begin{proof}
    		
    		【】
    	\end{proof}
    \end{enumerate}
\end{enumerate}
\subsection{商代数群}

\begin{enumerate}
	\item 引理.设$V_2\subseteq V_1=K^n$是$d$维子空间,那么$L_2=\bigwedge^dV_2$是$L_1=\bigwedge^dV_1$的一维子空间.取$x\in\mathrm{GL}(V_1)$和$v\in\mathfrak{gl}(V_1)$,有$\bigwedge^dx$是$L_1\to L_1$的自同构,$\mathrm{d}\bigwedge^d(v)$是$L_1\to L_1$的自同态,那么:
	\begin{enumerate}[(1)]
		\item $(\bigwedge^dx)L_2=L_2$当且仅当$x(V_2)=V_2$.
		\item $(\mathrm{d}\bigwedge^d)(v)L_2\subseteq L_2$当且仅当$v(V_2)\subseteq V_2$.
	\end{enumerate}
    \begin{proof}
    	
    	【】
    \end{proof}
    \item Chevalley定理.设$G$是代数群,设$H$是闭子群,那么存在$G$的有理表示$\varphi:G\to\mathrm{GL}(V)$以及一个一维子空$L\subseteq V$,满足$H=\{g\in G\mid gL=L\}$和$\mathfrak{h}=\{v\in\mathfrak{g}\mid vL\subseteq L\}$,这里$vL$理解为$\mathrm{d}\varphi(v)(L)$.
    \begin{proof}
    	
    	设闭子群$H$对应于$K[G]$的根理想$I$,这是有限生成理想,我们解释过它的一组生成元集可以包含在某个有限维$R_g,g\in G$不变的$K[G]$的子空间$W$中.取$M=W\cap I$,于是$M$也生成了理想$I$.我们之前解释过$H=\{g\in G\mid R_g(I)=I\}$,于是$M$在$R_g,g\in G$下不变.我们还解释过此时就有$D_v(W)\subseteq W$,进而$D_v(M)\subseteq M$对任意$v\in\mathfrak{h}$成立.取$V=\bigwedge^dW$和$L=\bigwedge^dM$,其中$d=\dim M$,取$G$在$K[G]$上右平移作用,那么按照上面引理,问题归结为证明$H=\{g\in G\mid R_gM=M\}$和$\mathfrak{h}=\{v\in\mathfrak{g}\mid D_v(M)\subseteq M\}$.
    	
    	\qquad
    	
    	取$g\in G$满足$R_gL=L$,按照理想$I$被$M$生成,就有$R_g(I)=R_g(MK[G])=R_g(M)R_g(K[G])=MA=I$.进而有$g\in H$.反过来如果$h\in H$,那么$R_h(M)=R_h(W)\cap R_h(I)=W\cap I=M$.再取$v\in\mathfrak{g}$满足$D_v(M)\subseteq M$,进而有:$D_v(I)=D_v(MK[G])\subseteq(D_vM)K[G]+M(D_vK[G])\subseteq MK[G]=I$.于是$v\in\mathfrak{h}$.反过来我们已经解释了如果$v\in\mathfrak{h}$,那么$D_v(M)\subseteq M$.
    \end{proof}
    \item 推论.
    \begin{enumerate}[(1)]
    	\item 给定闭子群$H\le G$,那么存在$G$在某个射影空间$\mathbb{P}(V)$上的作用$\varphi$,使得$H$恰好是某个直线$[L]$的稳定子群.进而此时$G/H$一一对应于$[L]$所在轨道中的元.这个轨道是$\mathbb{P}(V)$的正则局部闭子簇.
    	\item 构造$\pi:G\to\mathbb{P}(V)$为$g\mapsto g[L]=[gL]$.取典范的$\sigma:V-\{0\}\to\mathbb{P}(V)$,取$L$的任意非零向量$x$,把它扩充为$V$的一组基$\{x_1=x,x_2,\cdots,x_n\}$,那么$\sigma$可以表示为$(a_1,\cdots,a_n)\mapsto(a_2/a_1,\cdots,a_n/a_1)$,进而有$\mathrm{d}\sigma_x$是满射,并且核就是$L$.再取$\omega:\mathrm{GL}(V)\to V$为$\omega(\alpha)=\alpha(x)$.它在上述基下就是把一个矩阵映为它的第一列.特别的$\omega$是线性映射,它的微分就是自身,也即$\mathrm{d}\omega_e(v)=v(x),\forall v\in\mathfrak{gl}(V)$.
    	\item $\pi$可以表示为$\xymatrix{G\ar[r]^{\varphi}&\mathrm{GL}(V)\ar[r]^{\omega}&V\ar[r]^{\sigma\mid_{V-\{0\}}}&\mathbb{P}(V)}$.进而有$\mathrm{d}\pi_e$把$v\in\mathfrak{g}$映为$\mathrm{d}\varphi(v)(x)$在$\mathrm{T}_{[L]}\mathbb{P}(V)$中的像.进而按照Chevalley定理有$\ker\mathrm{d}\pi_e=\mathfrak{h}$.
    	\item 另外比较维数【?】得到$\mathrm{d}\pi_e$是满射,进而$\pi$是可分态射.
    \end{enumerate}
    \item 特征标.设$G$是代数群,它的特征标(character)是指一维有理表示(也即代数群态射)$\chi:G\to\mathbb{G}_m$.例如$\det:\mathrm{GL}(n,K)\to\mathbb{G}_m$.
    \begin{enumerate}[(1)]
    	\item 两个特征标的乘积$(\chi_1\chi_2)(g)=\chi_1(g)\chi_2(g)$也是特征标.进而$G$上全体特征标$X(G)$构成一个交换群,称为$G$的特征群.它可以视为$K[G]$的子集.特征群函子$G\mapsto X(G)$是正合函子.
    	\item $\mathbb{G}_a$没有非平凡特征标.因为,比方说,$\mathbb{G}_a$可以实现为矩阵群$\left(\begin{array}{cc}1&\lambda\\0&1\end{array}\right),\lambda\in K$.其中的元素都是幂单的(此即矩阵$x$满足$x-1$的某个次幂为零).但是$\mathbb{G}_m$中的元都是半单的,按照矩阵的Jordan分解,同时是幂单和半单的矩阵只有单位矩阵.
    	\item $\mathrm{SL}(n,K)$也没有非平凡特征标.
    	\item 半稳定元.设$\varphi:G\to\mathrm{GL}(V)$是一个有理表示,对$\chi\in X(\varphi(G))$,定义$V_{\chi}=\{x\in V\mid gx=\chi(g)x,\forall g\in G\}$,这是$V$的$G$不变子空间,它的非零元称为关于$\chi$的半稳定元.反过来如果$0\not=x\in V$,使得它所在的直线$L$是$G$不变的,那么定义$\chi(g)v=gv$得到一个$\chi\in X(G)$.另外注意$X(\varphi(G))\to X(G)$是单射,所以$X(\varphi(G))$可以典范的视为$X(G)$的子群.
    	\item 设$\varphi:G\to\mathrm{GL}(V)$是有理表示,那么子空间$\{V_{\chi}\mid\chi\in X(G)\}$是线性无关的.特别的只有有限个$\chi$使得$V_{\chi}$非零.
    	\begin{proof}
    		
    		设$n\ge2$是最小的正整数,使得存在$x_i\in V_{\chi_i},1\le i\le n$,$\chi_i$两两不同,使得$x_1+\cdots+x_n=0$.按照$\chi_i$两两不同,存在某个$g\in G$使得$\chi_1(g)\not=\chi_2(g)$.进而有$0=\varphi(g)\sum x_i=\sum\chi_i(g)x_i$.进而有$\sum\chi_1^{-1}(g)\chi_i(g)x_i=0$.其中$x_1$的系数为1,$x_2$的系数非1,所以把这个式子减去$\sum x_i=0$得到涉及到$\le n-1$个元的求和,这和$n$的最小性相矛盾.
    	\end{proof}
        \item 设$\varphi:G\to\mathrm{GL}(V)$是有理表示,设$N\le G$是正规子群,取$\chi\in X(\varphi(N))$,那么我们断言$\varphi(G)$中的每个元都把$V_{\chi}$映入某个$V_{\chi'},\chi'\in X(\varphi(N))$.由此可定义$G$在$X(\varphi(N))$上的典范作用$g\chi=\chi'$.
        \begin{proof}
        	
        	不妨设$\varphi$是单射,把$G$视为$\mathrm{GL}(V)$的子群.任取$g\in G$,$n\in N$,$x\in V_{\chi}$,那么有:
        	\begin{align*}
        		ngx&=g(g^{-1}ng)x\\&=g\chi(g^{-1}ng)x\\&=\chi(g^{-1}ng)gx
        	\end{align*}
        	于是取$\chi'(n)=\chi(g^{-1}ng)$就满足$g(V_{\chi})\subseteq V_{\chi'}$.
        \end{proof}
    \end{enumerate}
    \item 设$G$是代数群,设$N\le G$是闭正规子群,那么存在有理表示$\psi:G\to\mathrm{GL}(W)$使得$N=\ker\psi$和$\mathfrak{n}=\ker\mathrm{d}\psi$.
    \begin{proof}
    	
    	我们之前已经解释了存在一个有理表示$\varphi:G\to\mathrm{GL}(V)$,以及$V$的一个一维子空间$L$,使得$N=\{g\in G\mid gL=L\}$和$\mathfrak{n}=\{v\in\mathfrak{g}\mid vL\subseteq L\}$.由于$N$中的元都固定了$L$,于是它对应了一个特征标$\chi_0\in X(\varphi(N))\subseteq X(N)$.考虑那些有限个非零的$V_{\chi},\chi\in X(\varphi(N))$的直和$V'$,它包含了$L$.并且按照$N$是正规子群,就有$G$在这些$\{V_{\chi},\chi\in X(\varphi(N))\}$上是置换作用.于是不妨设$V=V'$,此时$N$在每个$V_{\chi}$上的作用都是数乘.
    	
    	\qquad
    	
    	考虑$\mathrm{End}(V)=\mathfrak{gl}(V)$的全部固定每个$V_{\chi},\chi\in X(\varphi(N))$的自同态子空间$W=\coprod_{\chi}\mathrm{End}(V_{\chi})$.我们知道$\mathrm{GL}(V)$伴随作用在$\mathfrak{gl}(V)$上,也即共轭作用.按照$\varphi(G)$在$\{V_{\chi}\}$上是置换作用,就有$\varphi(G)$固定了$W$.于是子表示是群同态$\psi:G\to\mathrm{GL}(W)$.它是如下代数群态射的复合限制在不变子空间$W$上,从而是有理表示:
    	$$\xymatrix{G\ar[r]^{\varphi}&\mathrm{GL}(V)\ar[r]^{\mathrm{Ad}}&\mathfrak{gl}(V)}$$
    	
    	计算$\ker\psi$.任取$n\in N$,那么$n$在每个$V_{\chi}$上的作用都是数乘,所以它和$\mathfrak{gl}(V)$中的矩阵可交换,也即$n\in\ker\psi$.反过来任取$g\in\ker\psi$,那么$g$固定了每个$V_{\chi}$,并且和$\mathrm{End}V_{\chi}$可交换,迫使$g$在每个$V_{\chi}$上是数乘,特别的$g$固定了$L\subseteq V_{\chi_0}$,于是$g\in N$.
    	
    	\qquad
    	
    	计算$\ker\mathrm{d}\psi$.任取$w\in\mathfrak{n}$,那么$\mathrm{d}\varphi(w)$作用在每个$V_{\chi}$上都是数乘.这样的矩阵和$\mathfrak{gl}(V)$中每个元可交换,进而$\mathrm{ad}(\mathrm{d}\varphi(w))=0$,于是$w\in\ker\mathrm{d}\psi$.反过来任取$v\in\ker\mathrm{d}\psi$,那么$\mathrm{ad}(\mathrm{d}\varphi(v))$在$W$上的作用为零,于是$\mathrm{d}\varphi(v)$固定每个$V_{\chi}$并且和每个$V_{\chi}$交换,于是它在每个$V_{\chi}$上的作用是数乘,特别的$\mathrm{d}\varphi(v)L\subseteq L$,于是$v\in\mathfrak{n}$.
    \end{proof}
    \item 商代数群.设$H\le G$是代数群的正规闭子群,那么存在代数群$Y$和代数群态射$\pi:G\to Y$,使得它的每个非空纤维都是$H$的某些左陪集的无交并,满足如下泛性质:对任意代数簇$X$和态射$\varphi:G\to X$,使得$X$的非空纤维都是$H$的某些左陪集的无交并,那么存在唯一的态射$\psi:Y\to X$使得如下图表交换.我们称$(\pi,Y)$就是$G$关于$H$的商代数群,把$Y$依旧记作$G/H$.
    $$\xymatrix{Y\ar[rr]^{\psi}&&X\\G\ar[u]\ar[urr]_{\varphi}&&}$$
    \begin{enumerate}[(1)]
    	\item 取$Y=\mathrm{Im}\psi$,这是仿射代数群,把$\psi$限制为$\pi:G\to Y$,我们断言$(\pi,Y)$就是$H$的商代数群.
    	\begin{proof}
    		
    		任取满足上述条件的$\varphi:G\to X$,下面构造唯一的$\psi$.
    		\begin{itemize}
    			\item 集合上$\psi$是唯一的,因为它必须把$\pi(g)$映为$\varphi(g)$.这也说明了$\psi$一旦存在必然唯一.
    			\item 连续性.问题归结为证明$Y$上是商拓扑,也即$V\subseteq Y$是开集当且仅当$\pi^{-1}(V)$是开集.由于我们处理的是群,这件事归结为证明$\pi$是开映射:一方面满开映射一定是商映射,反过来如果$U\subseteq G$是开集,那么$\pi^{-1}(\pi(U))=UH$是开子集,进而$\pi(U)$是开集.
    			
    			\qquad
    			
    			我们知道不可约代数簇之间的支配态射$\pi:G\to Y$如果满足对任意不可约闭子集$W\subseteq Y$,都有$\pi^{-1}(W)$是等维数$\dim G-\dim Y+\dim W$的,则有$\pi$是开映射.按照$Y$是正则的,它的局部环都是整闭的,进而可以选取$Y$的非空开子集$V\subseteq\pi(G)$,使得只要$W$和$V$有交,如果$Z$是$\pi^{-1}(W)$的和$\pi^{-1}(V)$有交的不可约分支,那么$\dim Z=\dim W+\dim G-\dim Y$.但是这里$G$在$Y$上的作用是可迁的,就有$\{gU\}$覆盖了整个$Y$,于是$\pi^{-1}(W)$的所有分支都有维数$\dim W+\dim G-\dim Y$.
    			\item 结构层.最后还需要验证对任意开子集$U\subseteq X$都有$\psi$诱导了同态$\mathscr{O}_X(U)\to\mathscr{O}_Y(\psi^{-1}(U))$.我们知道$\varphi^*$把$\mathscr{O}_X(U)$映入$\mathscr{O}_G(\varphi^{-1}(U))$,并且像集中的正则函数是在每个左陪集$gH$上是常值映射【?】
    		\end{itemize}
    	\end{proof}
        \item 设$G$是代数群,设$H$是闭子群,设$Y$是一个带$G$作用的代数簇,设$\pi:G\to Y$是满射可分$G$不变态射,并且纤维恰好都是$H$的左陪集,那么$Y$在同构意义下是唯一存在的,它也记作$G/H$.
        \item 如果基域是特征零的.
        \begin{enumerate}[(i)]
        	\item 李代数和核可交换:设$\varphi:G\to G'$是代数群态射,那么$\ker\mathrm{d}\varphi=\mathscr{L}(\ker\varphi)$.
        	\item 李代数和交可交换:如果$A,B$是代数群$G$的两个闭子群,记$C=A\cap B$,那么$\mathfrak{a}\cap\mathfrak{b}=\mathfrak{c}$.
        \end{enumerate}
        \begin{proof}
        	
        	(i):不妨设$\varphi$是满射.按照$\varphi$是可分的,上一条说明有典范代数群态射$G\to G/\ker\varphi$.进而我们解释过有$\ker\mathrm{d}\varphi=\mathscr{L}(\ker\varphi)$.
        	
        	\qquad
        	
        	(ii):按照可分性,取典范态射$\pi:G\to G/B$,那么我们解释过$\ker\mathrm{d}\pi_e=\mathfrak{b}$.把$\pi$限制为$\pi':A\to\pi(A)$,这个态射的纤维是$C$的左陪集,进而$\pi'$可以典范的视为$A\to A/C$,进而有$\ker\mathrm{d}\pi_e'=\mathfrak{c}$.另一方面$\ker\mathrm{d}\pi_e'=\mathfrak{a}\cap\ker\mathrm{d}\pi_e=\mathfrak{a}\cap\mathfrak{b}$.
        \end{proof}
    \end{enumerate}
\end{enumerate}
\subsection{李对应}
\begin{enumerate}
	\item 设基域的特征为零,设$G$是代数群,设$H,N$是两个闭子群,其中$N$是连通闭子群,那么$N\subsetneqq H$当且仅当$\mathfrak{n}\subsetneqq\mathfrak{h}$.特别的,如果$G$是连通代数群,那么$H\mapsto\mathfrak{h}$是从$G$的连通闭子群集合到$\mathfrak{g}$的李子代数集合的单射.
	\begin{proof}
		
		这件事是因为李代数和闭子群的交可交换,另外在$G$连通的时候,如果$H$是一个真闭子群,那么必然有$\mathfrak{h}\subsetneqq\mathfrak{g}$.
	\end{proof}
    \item 闭子群和对应的李子代数固定相同的子空间.设$V$是有限维仿射空间,设$W\subseteq V$是子空间,设$G_W=\{g\in\mathrm{GL}(V)\mid gW=W\}$和$\mathfrak{g}_W=\{v\in\mathfrak{gl}(V)\mid v(W)\subseteq W\}$.再对$x\in V$定义$G_x=G_{\langle x\rangle}=\{g\in\mathrm{GL}(V)\mid gx=x\}$和$\mathfrak{g}_x=\mathfrak{g}_{\langle x\rangle}=\{v\in\mathfrak{gl}(V)\mid v(x)=0\}$.那么我们知道$G_W$和$G_x$都是$\mathrm{GL}(V)$的闭子群,$\mathfrak{g}_W$和$\mathfrak{g}_x$都是$\mathfrak{gl}(V)$的李子代数.我们断言有$\mathscr{L}(G_W)=\mathfrak{g}_W$和$\mathscr{L}(G_x)=\mathfrak{g}_x$.特别的,如果基域的特征为零,如果$H\le\mathrm{GL}(V)$是闭子群,那么$\mathscr{L}(H_W)=\mathfrak{h}\cap\mathfrak{g}_W$,其中$H_W=H\cap G_W$.
    \begin{proof}
    	
    	选取$V$的一组基$\{x_1,\cdots,x_n\}$,使得$\{x_1,\cdots,x_m\}$是$W$的一组基.那么$\mathfrak{g}_W$是维数为$n(n-m)+m^2$的$\mathbb{A}^{n^2}$的闭子簇.而$G_W$是$\mathfrak{g}_W$的主开集.所以它们具有相同的维数,但是$\mathscr{L}(G_W)\subseteq\mathfrak{g}_W$,于是这个包含关系取等号.
    \end{proof}
    \item 推论.【】
    \item 正规和理想.设基域的特征为零,设$G$是连通代数群,设$H$是连通闭子群,那么$\mathfrak{h}$是$\mathfrak{g}$的理想当且仅当$H$是$G$的闭子群.
    \begin{proof}
    	
    	充分性已经证明过了.必要性:设$\mathfrak{h}$是$\mathfrak{g}$的理想.记$N=\{g\in G\mid\mathrm{Ad}(g)(\mathfrak{h})=\mathfrak{h}\}$.那么有$\mathfrak{n}=\{v\in\mathfrak{g}\mid\mathrm{ad}(v)(\mathfrak{h})\subseteq\mathfrak{h}\}=\mathfrak{g}$,最后一个等号因为$\mathfrak{h}$是理想.于是有$N=G$.任取$g\in G$,就有$\mathscr{L}(\mathrm{Int}(g)(H))=\mathrm{Ad}(g)(\mathfrak{h})=\mathfrak{h}$.于是$H$和$gHg^{-1}$具有相同的李代数的连通闭子群,所以它们相同.
    \end{proof}
    \item 中心和中心化子.设基域的特征为零,设$G$是连通代数群.
    \begin{enumerate}[(1)]
    	\item 设$g\in G$,那么有$\mathscr{L}(\mathrm{C}_G(g))=\mathfrak{c}_{\mathfrak{g}}(g)$,这里$\mathfrak{c}_{\mathfrak{g}}(g)=\{v\in\mathfrak{g}\mid\mathrm{Ad}(g)(v)=v\}$.
    	\item $\ker\mathrm{Ad}=Z(G)$,并且它的李代数是$\ker\mathrm{ad}=\mathfrak{z}(\mathfrak{g})$.特别的,这件事说明基域特征为零时,一个连通代数群$G$是交换群当且仅当它的李代数是交换代数.
    \end{enumerate}
    \begin{proof}
    	
    	(1):我们解释过$G=\mathrm{GL}(n,K)$时在基域是任意特征时这件事都成立.下面设基域的特征为零,设$G$是$\mathrm{GL}(n,K)$的闭子群,那么有$\mathrm{C}_G(g)=G\cap\mathrm{C}_{\mathrm{GL}(n,K)}(g)$和$\mathfrak{c}_{\mathfrak{g}}(g)=\mathfrak{g}\cap\mathfrak{c}_{\mathfrak{gl}(n,K)}(g)$.于是按照特征零时李代数和闭子群的交可交换得证.
    	
    	\qquad
    	
    	(2):按照特征零,李代数和核可交换,有$\mathscr{L}(\ker\mathrm{Ad})=\ker\mathrm{ad}=\mathfrak{z}(\mathfrak{g})$.对$g\in Z(G)$,有$\mathrm{Int}(g)=1$,于是微分$\mathrm{Ad}(g)=1$.反过来如果$\mathrm{Ad}(g)=1$,那么有$\mathfrak{c}_{\mathfrak{g}}(g)=\mathfrak{g}$,按照(1)就有$\mathscr{L}(\mathrm{C}_G(g))=\mathfrak{g}$.于是$\mathrm{C}_G(g)=G$,也即$g\in Z(G)$.
    \end{proof}
    \item 正规化子.设基域的特征为零.设$G$是连通代数群,$H$是连通闭子群,那么$\mathscr{L}(\mathrm{N}_G(H))=\mathfrak{n}_{\mathfrak{g}}(\mathfrak{h})$,这里$\mathfrak{n}_{\mathfrak{g}}(\mathfrak{h})=\{v\in\mathfrak{g}\mid[v,\mathfrak{h}]\subseteq\mathfrak{h}\}$.
\end{enumerate}
\subsection{Jordan分解}

\begin{enumerate}
	\item 矩阵情况.设$V$是域$K$上的有限维线性空间,设$f\in\mathrm{End}(V)$,称$f$是幂零的,如果存在某个次幂为零;称$f$是半单的,如果它可以对角化,也即它的极小多项式没有重根;称$f$是幂单的,如果$f-1$是幂零变换.
	\begin{enumerate}[(1)]
		\item 我们有如下(加性)Jordan分解定理:
		\begin{enumerate}[(i)]
			\item 存在唯一的$f_s,f_n\in\mathrm{End}(V)$,满足$f=f_s+f_n$,并且$f_s$是半单的,$f_n$是幂零的,并且$f_sf_n=f_nf_s$.这个分解称为$f$的Jordan分解.
			\item 存在多项式$p(T)$和$q(T)$使得$f_s=p(f)$和$f_n=q(f)$.特别的和$f$可交换的线性变换一定与$f_s$和$f_n$可交换.
			\item 如果$A\subseteq B\subseteq V$是子空间,如果$f(B)\subseteq A$,那么$f_s$和$f_n$满足相同的事情.
			\item 如果$ff'=f'f$,那么$(f+f')_s=f_s+f'_s$和$(f+f')_n=f_n+f'_n$.
		\end{enumerate}
	    \item 如果$f$是可逆变换,那么$f_s$也是可逆的,记$f_u=f_s^{-1}f=1+f_s^{-1}f_n$,由此得到乘性的Jordan分解$f=f_sf_u$:
	    \begin{enumerate}[(i)]
	    	\item 存在唯一的$f_s,f_u\in\mathrm{GL}(V)$满足$f=f_sf_u$,其中$f_s$是半单的,$f_u$是幂单的(此即$f_u-1$是幂零的).这个分解称为$f$的乘性Jordan分解.特别的同时是半单和幂单的变换是单位变换.
	    	\item 和$f$可交换的线性变换都和$f_s,f_u$可交换.
	    	\item 如果子空间$W\subseteq V$是$f$不变的,那么它也是$f_s,f_u$不变的.
	    	\item 如果$ff'=f'f$,那么$(ff')_s=f_sf'_s$和$(ff')_u=f_uf'_u$.
	    \end{enumerate}
        \item 尽管半单和幂单变换对于$V$无限维的时候没有意义,但是我们可以把它理解为有限子空间构成的正向系统上的半单变换族或者幂单变换族.我们把这个分解依旧记作$f=f_sf_u$.
        \item 设基域的特征$p>0$,我们断言$f$是幂单变换当且仅当存在某个自然数$t$使得$f^{p^t}=1$.因为存在足够大的次幂$p^t$使得$f^{p^t}-1=(f-1)^{p^t}=0$.
        \item 设基域的特征为零.如果$f$是幂零变换,那么$\exp(f)=\sum_{n\ge0}f^n/n!$是有限求和,并且它是$1+$幂零矩阵的形式,所以它必然是幂单变换.反过来如果$f$是幂单变换,那么$f-1$是幂零的,进而可定义$\ln(f)=\sum_{n\ge0}(-1)^{n+1}(f-1)^n/n$.对幂单变换$f$有$\exp(\ln(f))=f$;对幂零变换$f$有$\ln(\exp(f))=f$.
        \item 设基域的特征为零,设$f\not=1$是$V$上幂单变换,那么$\varphi:\mathbb{G}_a\to\mathrm{GL}(V)$,$a\mapsto\exp(af)$是单代数群态射,它就是$\mathrm{GL}(V)$的包含$f$的最小的闭子群.特别的,$\mathrm{GL}(V)$的包含一个幂单元的连通闭一维子群必然同构于$\mathbb{G}_a$.
        \begin{proof}
        	
        	$\varphi$是代数群之间的态射,它是单射是因为$af=\ln(\exp(af))=\sum_{n\ge0}(-1)^{n+1}(\varphi(a)-1)^n/n$是$\varphi(a)$的多项式$P(\varphi(a))$,进而$P(\varphi(a))-a$的某个次幂是零,于是$\varphi(a)$的逆是$\varphi(a)$的某个多项式.最后我们知道幂单变换在乘法下的阶数不能是有限的,于是幂单变换在$\mathrm{GL}(V)$中生成的子群的闭包的维数至少为1(因为零维是有限集合),迫使这个闭包必须等同于1维的闭子群$\varphi(\mathbb{G}_a)$.
        \end{proof}
	\end{enumerate}
	\item 设$g\in G=\mathrm{GL}(n,K)$,那么$R_g\in\mathrm{GL}(K[G])$具有乘性Jordan分解$R_{g_s}R_{g_u}$.设$v\in\mathfrak{g}=\mathfrak{gl}(n,K)$,那么$D_v$具有加性Jordan分解$D_{v_s}+D_{v_n}$.
	\begin{proof}
		
		我们知道$K[G]$是那些被$\{R_g\mid g\in G\}$固定的有限维子空间的并,也是那些被$\{D_v\mid v\in\mathfrak{g}\}$固定的有限维子空间的并.在这些有限维子空间上$g$和$D_v$相应的加性/乘性Jordan分解都存在:$R_g=R_{g_s}R_{g_u}$和$D_v=D_{v_s}+D_{v_n}$.所以问题归结为证明$R_{g_s}$和$D_{v_s}$是半单的,$R_{g_u}$是幂单的,$D_{v_n}$是幂零的.
		
		\qquad
		
		$K[G]=K[T_{ij}]_{d}$,其中$d=\det(T_{ij})$.我们知道$R_g,L_g$作用在$K[T_{ij}]$上为,$R_gT_{ij}$是$Tg$的$(i,j)$分量;$L_gT_{ij}$是$g^{-1}T$的$(i,j)$分量.此时有$R_g(d)(f)=\det g\deg f$,于是$R_g(d)=\deg g\cdot d$,也即$d$是$R_g$的关于$\deg g$的特征向量.于是$d$在$K[T]$中张成的子空间是$G$不变的,于是按照右平移作用下代数群的不变子空间也是对应李代数的不变子空间,得到它也是$\mathfrak{g}$不变的.并且按照迹是行列式的微分,得到$D_v(d)=\mathrm{Tr}(v)d$.综上$R_g\mid K[T]$,$D_v\mid K[T]$的半单性,幂单性,幂零性都能传递给$R_g\mid K[G]$和$D_v\mid K[G]$,综上问题归结为把$R_g$和$D_v$视为$K[T]$上的线性变换.
		
		\qquad
		
		【】
	\end{proof}
	\item 代数群和李代数上元素的Jordan分解.设$G$是代数群.
	\begin{enumerate}[(1)]
		\item 对$g\in G$,存在唯一的$g_s,g_u\in G$,使得$g=g_sg_u$,其中$g_s,g_u$可交换,并且$R_{g_s}$是半单的,$R_{g_u}$是幂单的.$g_s$和$g_u$分别称为$g$的半单部分和幂单部分.
		\item 对$v\in\mathfrak{g}$,存在唯一的$v_s,v_n\in\mathfrak{g}$,使得$v=v_s+v_n$,其中$v_s,v_n$乘法可交换,并且$D_{v_s}$是半单的,$D_{v_n}$是幂零的.$v_s$和$v_n$分别称为$v$的半单部分和幂零部分.
		\item 代数群态射和Jordan分解可交换:如果$\varphi:G\to G'$是代数群态射,那么$\varphi(g)_s=\varphi(g_s)$,$\varphi(g)_u=\varphi(g_u)$,$\mathrm{d}\varphi(v)_s=\mathrm{d}\varphi(v_s)$和$\mathrm{d}\varphi(v)_n=\mathrm{d}\varphi(v_n)$.
		\item 设$G$是代数群,记$G_s=\{g\in G\mid g=g_s\}$和$G_u=\{g\in G\mid g=g_u\}$.这是内蕴定义的,也即不依赖到$\mathrm{GL}(n,K)$的嵌入的选取,它们的交是幺元$e$.类似的定义$\mathfrak{g}_s$和$\mathfrak{g}_n$.这里$G_u$和$\mathfrak{g}_n$都是闭子集.因为$\mathrm{GL}(n,K)$中的全体幂单元构成闭子集(因为它满足$(x-1)^n=0$);$\mathfrak{gl}(n,K)$的全体幂零元构成闭子集(因为满足$x^n=0$).
	\end{enumerate}
    \begin{proof}
    	
    	把$G$视为某个$\mathrm{GL}(n,K)$的闭子群.设理想$I\subseteq K[\mathrm{GL}(n,K)]$定义了闭子群$G$.我们解释过$g\in\mathrm{GL}(n,K)$落在$G$中等价于$R_g$固定$I$;$v\in\mathfrak{gl}(n,K)$落在$\mathfrak{g}$中等价于$D_v$固定$I$.任取$g\in G$,取$g$(作为矩阵)的Jordan分解是$g=g_sg_u$,其中$g_s,g_u\in\mathrm{GL}(n,K)$.我们已经解释了$(R_g)_s=R_{g_s}$和$(R_g)_u=R_{g_u}$.我们还解释过由于$I$是$R_g$和$D_v$不变的,就有$I$也是$R_{g_s},R_{g_u},D_{v_s},D_{v_n}$不变的,从而$g_s,g_u\in G$和$v_s,v_n\in\mathfrak{g}$.这就证明了(1)和(2).
    	
    	\qquad
    	
    	任取代数群态射$\varphi:G\to G'$,它分解为满态射$G\to\varphi(G)$和单态射$\varphi(G)\to G'$.于是问题归结为证明$\varphi$是单态射和满态射的情况.如果$\varphi$是单射,按照上一段相同的做法就得证.下面设$\varphi$是满态射.此时$R_{\varphi(g)}$是$R_g$在子环$K[G']\subseteq K[G]$上的限制.所以从$g$是半单元/幂单元就得到它在子空间上的限制$\varphi(g)$是半单元/幂单元.于是分解$R_{\varphi(g)}=R_{\varphi(g_s)}R_{\varphi(g_u)}$就已经是$R_{\varphi(g)}$的Jordan分解.于是$\varphi(g_s)=\varphi(g)_s$和$\varphi(g_u)=\varphi(g)_u$.李代数的部分是类似的.
    \end{proof}
    \item 交换代数群的结构定理.设$G$是交换代数群.
    \begin{enumerate}[(1)]
    	\item $G_s$和$G_u$都是$G$的闭子群.
    	\item 如果$G$连通则$G_s,G_u$都是连通的.
    	\item 典范态射$\varphi:G_s\times G_u\to G$是代数群同构.
    	\item 有$\mathscr{L}(G_s)=\mathfrak{g}_s$,$\mathscr{L}(G_u)=\mathfrak{g}_n$.
    \end{enumerate}
    \begin{proof}
    	
    	我们之前解释过在$f,f'\in\mathrm{GL}(n,K)$交换时有$(ff')_s=f_sf'_s$和$(ff')_u=f_uf'_u$,于是此时$G_s$和$G_u$都是子群.已经解释了$G_u$是闭子集.我们知道交换的矩阵族是可以同时上三角化的,于是可以不妨把$G$嵌入到某个$\mathrm{T}(n,K)$中.此时有$G_s\subseteq\mathrm{D}(n,K)$,迫使$G_s=G\cap\mathrm{D}(n,K)$,于是$G_s$也是闭子群.这解决了(1).
    	
    	\qquad
    	
    	$\varphi$是代数群态射也是双射.下面证明它的逆也是代数群态射,这归结为证明$g\mapsto g_s$和$g\mapsto g_u$都是代数群态射.按照$g_u=g_s^{-1}g$,从前者是代数群态射就得到后者也是.前者是代数群态射是因为这里$g$视为上三角矩阵,那么$g_s$必然是它的对角线部分,所以$g\mapsto g_s$就是线性的,所以是态射,这证明了(3).
    	
    	\qquad
    	
    	(2)是因为$G\to G_s$和$G\to G_u$都是满态射,所以当$G$连通时满射像是连通的.最后证明(4):按照$G_s\subseteq\mathrm{D}(n,K)$和$G_u\subseteq\mathrm{U}(n,K)$,就有$\mathscr{L}(G_s)\subseteq\mathfrak{d}(n,K)$和$\mathscr{L}(G_u)\subseteq\mathfrak{n}(n,K)$(此为严格上三角矩阵构成的子空间).它们分别是半单和幂零的,于是有$\mathscr{L}(G_s)\subseteq\mathfrak{g}_s$和$\mathscr{L}(G_u)\subseteq\mathfrak{g}_n$.但是按照$\varphi$是同构,得到$\mathscr{L}(G_s)\oplus\mathscr{L}(G_u)=\mathfrak{g}$.结合Jordan分解有$\mathfrak{g}=\mathfrak{g}_s\oplus\mathfrak{g}_n$,就得到这两个包含关系实际上取等.
    \end{proof}
\end{enumerate}
\subsection{对角群}

一个代数群称为对角群(diagonalizable),如果它是交换代数群,并且$G=G_s$.
\begin{enumerate}
	\item 基本性质.
	\begin{enumerate}[(1)]
		\item 一个代数群是对角群当且仅当它可以嵌入到某个$\mathrm{D}(n,K)=\mathbb{G}_m^{\times n}$.因为一个交换可对角化矩阵族一定可以同时对角化.
		\item 对角群的子群和态射像都是对角群.
	\end{enumerate}
	\item 等价描述:$d$-群.一个代数群$G$称为$d$-群,如果$K[G]$存在一组基由特征标构成.
	\begin{enumerate}[(1)]
		\item 设$G$的基域是$K$,我们知道群同态集合$\mathrm{Hom}_{\textbf{Grp}}(G,K^*)$在$G$上$K$值函数空间中是$K$线性无关的.于是一个代数群$G$是$d$-群也等价于讲$X(G)$构成了$K[G]$的一组基.
		\item $\mathrm{D}(n,K)$上有特征标$\chi_i:\mathrm{diag}\{a_1,\cdots,a_n\}\mapsto a_i$.并且$X(\mathrm{D}(n,K))$就是由$\{\chi_1,\cdots,\chi_n\}$生成的自由阿贝尔群.$\mathrm{D}(n,K)$是$d$-群.特征标$\chi_1^{r_1}\cdots\chi_n^{r_n}$对应于$K[G]=K[T_1,T_1^{-1},\cdots,T_n,T_n^{-1}]$中的$T_1^{r_1}\cdots T_n^{r_n}$.
		\item $d$-群之间的代数群态射和特征群之间的群同态一一对应.设$\varphi:G\to G'$是$d$-群之间的态射,它诱导了$K$代数同态$\varphi^*:K[G']\to K[G]$,它可以限制为乘法群同态$X(G')\to X(G)$,反过来因为特征群是一组基,所以$X(G')\to X(G)$的群同态可以延拓为一个$K$代数同态$K[G']\to K[G]$,进而定义了一个代数群态射$G\to G'$.
		\item $d$-群和对角群等价.其中对角群是$d$-群的部分我们会证明如下断言:如果$H\le G$是一个$d$-群的闭子群,那么$H$也是$d$-群,并且是$G$的某些特征标的核的交(然后取$G=\mathrm{D}(n,K)$即可).
		\begin{proof}
			
			先证明我们的断言.设包含态射$H\to G$诱导的$K$代数同态为$\varphi:K[G]\to K[H]$.明显的$G$上的特征标限制在$H$上是$H$的特征标,于是$K[H]$也是被特征标生成的,于是$H$是$d$-群.设理想$I\subseteq K[G]$定义了$H$,记$f=\sum_ia_i\chi_i\in I$,那么这些$\chi_i$限制在$H$上是相同的,并且$\sum_ia_i=0$.于是$I$被这样的元线性生成.可记$f=\chi_1h$,其中$h=\sum_ia_i(\chi_1^{-1}\chi_i-1)$.于是$I$被形如$\theta-1$的元作为理想生成,其中$\theta=\chi_1^{-1}\chi_i$,而$\theta-1$的核就是$\theta$的核,于是$H$是$G$的若干特征标的核的交.
			
			\qquad
			
			反过来设$G$是$d$-群.那么存在$X(G)$的有限子集$\{\chi_1,\cdots,\chi_n\}$代数生成了$K[G]$.定义$\varphi:G\to\mathbb{G}_m^{\times n}\cong\mathrm{D}(n,K)$为$g\mapsto(\chi_1(g),\cdots,\chi_n(g))$,这是代数群态射.按照这些$\chi_i$代数生成了整个$K[G]$,所以$\varphi$是单射.于是$G$是对角群.
		\end{proof}
	\end{enumerate}
	\item 对角群的结构定理.
	\begin{enumerate}[(1)]
		\item 设$G$是对角群,那么$X(G)$是$X(\mathrm{D}(n,K))\cong\mathbb{Z}^n$(写成加法)的同态像,所以$X(G)$是有限生成阿贝尔群.所以它可以分解为$\mathbb{Z}^r\oplus B$,其中$r$是$X(G)$的秩,$B$是一个有限子群(扭部分).如果基域的特征$p\not=0$,那么基域上就没有非平凡的$p$次单位根,进而$X(G)$就不能有$p$扭部分,这至少说明了不是每个有限生成阿贝尔群都能作为某个对角群的特征群.
		\item 设$G$是连通代数群,那么$X(G)$是无扭阿贝尔群.这件事是因为任取$G$的特征标$\chi$,那么$\chi(G)$是$\mathbb{G}_m$的连通子群,但是后者仅有的连通子群是自身和平凡群,所以$\chi$要么平凡要么满射,所以$n\chi=0$迫使$n=0$.
		\item 结构定理.设$G$是对角群,那么有$G=G^0\times H$,其中$G^0$是单位分支,它总是一个环面(此即同构于某个$\mathrm{D}(n,K)$的代数群),而$H$是一个阶数和$p=\mathrm{char}K$互素的有限代数群(它不是唯一的).特别的一个连通分解群是环面.
		\begin{proof}
			
			设$G$嵌入到$D=\mathrm{D}(n,K)$中.设$X(G)$的秩为$r$,那么$X(G)\to X(G^0)$的核是$X(G/G^0)$,按照$G/G^0$是有限群,得到$X(G/G^0)$是有限群.但是我们解释过$X(G^0)$是无扭阿贝尔群,于是$X(G^0)$恰好就是$X(G)$的无扭部分.
			
			\qquad
			
			下面考虑包含态射$G^0\to D$,它诱导了满同态$\varphi:\mathbb{Z}^n=X(D)\to X(G^0)=\mathbb{Z}^r$,按照自由模是投射模,就有$X(D)=\ker\varphi\oplus\mathbb{Z}^r$.取$\ker\varphi$作为自由阿贝尔群的一组基$\chi_1,\cdots,\chi_{n-r}$,再取$\mathbb{Z}^r$的一组基$\chi_{n-r+1},\cdots,\chi_n$.那么$g\mapsto\mathrm{diag}\{\chi_1(g),\cdots,\chi_n(g)\}$是$D$上的自同构,并且它把$G^0$对应为前$n-r$对角元为1的$n$阶对角矩阵构成的子群$D^0$.取后$r$个对角元为1的矩阵构成的子群为$D'$,那么有$D=D^0\times D'$.取$H=D'\cap G\cong G/G^0$,就得到$G=G^0\times H$.由于我们解释过$X(G)$没有$p$扭部分,于是$H$的阶数和$p$互素.
		\end{proof}
	    \item 对角群和无$p$扭有限生成阿贝尔群的对应.设对角群范畴为$\mathscr{D}$,设无$p$扭有限生成阿贝尔群范畴为$\mathscr{A}$.我们有特征群函子$X:\mathscr{D}\to\mathscr{A}$.这实际上是一个范畴等价函子,它的拟逆具有如下描述:给定有限生成阿贝尔群$A$,取它在基域$K$上的群代数$K[A]$,条件$A$无$p$扭等价于$K[A]$没有非平凡幂零元.按照$K[A]$是Hopf代数,它本身可以视为一个仿射代数群,记作$D(A)$.任取$a\in A$,$K$代数同态$K[T,T^{-1}]\to A$,$T\mapsto a$定义了一个特征标$D(A)\to\mathbb{G}_m$.于是$A\subseteq X(D(A))$,但是$A$本身线性张成了整个坐标环$K[D(A)]$,迫使$D(A)$是对角群,进而$A\cong X(D(A))$.函子$A\mapsto D(A)$是$X$的拟逆函子.
	\end{enumerate}
\end{enumerate}
\subsection{可解代数群}
\begin{enumerate}
	\item 群论引理.设$G$是群.
	\begin{enumerate}[(1)]
		\item 对$x,y\in G$,记$[x,y]=xyx^{-1}y^{-1}$是换位子.对子群$A,B$,把$[a,b],a\in A,b\in B$生成的子群记作$[A,B]$.如果$A,B$是正规子群,那么$[A,B]$也是.
		\item 如果$[G:Z(G)]=n<\infty$,那么$[G,G]$是有限的.
		\begin{proof}
			
			记$G$的换位子集合为$S$,那么$S$群生成了整个$[G,G]$.但是这里$[x,y]$只依赖于$x$和$y$在模掉$Z(G)$下的像,于是$|S|\le n^2$.我们有$[x_1,y_1][x_2,y_2]=[x_2,y_2][x_1,y_1]^z=[x_2,y_2][x_1^z,y_1^z]$(这里$a^z=zaz^{-1}$),其中$z=[x_2,y_2]$.于是一个换位子乘积中出现的相同项可以经这个操作变成次幂.我们断言$S$中某个元的$n+1$次幂可以写成$S$中$n$个元乘积,据此得到$|[G,G]|\le n^3$是有限的.最后证明断言:按照$[x,y]^n\in Z(G)$,得到$[x,y]^{n+1}=y^{-1}[x,y]^ny[x,y]=y^{-1}[x,y]^{n-1}[x,y^2]y$,得证.
		\end{proof}
		\item 设$A,B\le G$是正规子群,设$S=\{[a,b]\mid a\in A,b\in B\}$是有限集,那么$[A,B]$是有限集.这件事也可以推出上一条.
		\begin{proof}
			
			不妨设$G=AB$.我们还可以约化到$C=[A,B]$交换的情况:有$G$共轭作用在集合$S$上,也即$[a,b]^g=[a^g,b^g]$.据此得到群同态$G\to\mathrm{Aut}(S)$.把核记作$H$,并且它的指数有限.记$C=[A,B]$,那么$H\cap C$落在$C$的中心里,并且具有有限指数.按照前两条,$[C,C]$是$G$的有限正规子群.用$G/[C,C]$替换$G$,不妨设$C$是阿贝尔群.
			
			\qquad
			
			任取$a\in A$和$c\in C$,那么形如$[a,c]$的元互相交换,并且按照$C$可交换得到$[a,c]^2=(aca^{-1}c^{-1})=(aca^{-1})^2c^{-2}=[a,c^2]$.于是$|[A,C]|\le2|S|$是有限的正规子群.进而用$G/[A,C]$替换$G$可设$A,C$中的元可交换,特别的$A$是交换的.那么任取$a\in A$和$b\in B$,就有$[a,b]^2=a^2(ba^{-1}b^{-1})^2=[a^2,b]$,于是$|[A,B]|\le2|S|$是有限的.
		\end{proof}
	\end{enumerate}
    \item 设$A,B$是代数群$G$的闭子群.
    \begin{enumerate}[(1)]
    	\item 如果$A$是连通的,那么$[A,B]$是连通闭子群.
    	\item 如果$A,B$是正规闭子群,那么$[A,B]$是闭(正规)子群.特别的$[G,G]$总是闭(正规)子群.
    \end{enumerate}
    \begin{proof}
    	
    	(1):对任意$b\in B$,取态射$\varphi_b:A\to G$为$a\mapsto aba^{-1}b^{-1}$.按照$\varphi_b(e)=e$,以及$A$是连通的,就有$\{\varphi_b\mid b\in B\}$生成的闭子群是连通的,这个闭子群可以表示为$\varphi_{b_1}(A)^{e_1}\cdots\varphi_{b_n}(A)^{e_n}$,其中$e_i=\pm1$,进而它必然是$[A,B]$.
    	
    	\qquad
    	
    	(2):按照(1)已经有$[A^0,B]$和$[A,B^0]$都是闭连通正规子群(其中$A^0\le G$正规是因为$gA^0g^{-1}\subseteq A$也是包含幺元的连通子集,迫使$gA^0g^{-1}\subseteq A$).进而$C=[A^0,B][A,B^0]$也是闭正规子群(我们解释过如果$A,B$是两个闭子群,$A$包含在$B$的正规化子中,那么$AB$是闭子群).于是问题归结为一个群论问题:验证$[A,B]/C$是有限群.由于$A^0$在$G/C$中的像和$B$中的元可交换,$B^0$在$G/C$中的像和$A$中的元可交换.结合$A/A^0$和$B/B^0$都是有限的,说明任取$a\in A$和$b\in B$,就有$[a,b]$在$[A,B]/C$中的像是有限的.也即$[A/C,B/C]\subseteq G/C$是有限子群,进而上面引理说明$[A,B]$是有限群.
    \end{proof}
    \item 可解群.一个代数群$G$称为可解群如果它作为群是可解的.此即它的导出列$\mathscr{D}^0G=G$,$\mathscr{D}^{i+1}G=[\mathscr{D}^iG,\mathscr{D}^iG]$,在有限次之后终止于$\{e\}$.这件事等价于讲存在闭子群列$G=G_0\supseteq G_1\supseteq\cdots\supseteq G_n=\{e\}$,使得$[G_i,G_i]\subseteq G_{i+1}$.
    \begin{enumerate}[(1)]
    	\item 可解群的子群和满射像(商代数群)都是可解群.反过来如果$N\le G$是正规闭子群,使得$N$和$G/N$都是可解群,那么$G$是可解群.另外如果$A,B\le G$是正规闭可解子群,那么$AB$也是.
    	\item 有限可解群的结构已经很复杂了,不能期待可解代数群有结构定理.但是我们会证明连通可解代数群一定是$\mathrm{T}(n,K)$的闭子群.这里我们先证明$T=\mathrm{T}(n,K)$是可解群:我们知道$[T,T]$落在$U=\mathrm{U}(n,K)$中.记$A\subseteq\mathrm{M}(n,K)$是全体上三角矩阵构成的子代数.记$I\subseteq A$是全体严格上三角矩阵构成的子集,这是$A$的双边理想.记$U_r=1+I^r$.这里$I^r$被全体$E_{ij},j-i\ge r$线性生成.有$U_n=\{e\}$.那么$U_r$是$U$的闭正规子群,并且满足$[U_s,U_t]\subseteq U_{s+t}$,于是$U$是可解群.
    	\item 引理.设$A,B\subseteq G$是代数群的子群,满足$A\subseteq B$,那么有$[\overline{A},\overline{B}]=\overline{[A,B]}$.
    	\begin{proof}
    		
    		记态射$\varphi:G\times G\to G$为$(g,h)\mapsto ghg^{-1}h^{-1}$.按照$A\times B$在$\overline{A}\times\overline{B}$中稠密(Zariski拓扑下),得到$\varphi(A\times B)$在$\varphi(\overline{A}\times\overline{B})$中稠密.按照$[\overline{A},\overline{B}]$是$G$的闭子群(这件事需要$A\subseteq B$),并且被$\varphi(\overline{A}\times\overline{B})$群生成,于是它包含了$\overline{[A,B]}$.反过来$\overline{[A,B]}$是$\varphi(A\times B)$生成的群的闭包,从而包含了$\varphi(\overline{A}\times\overline{B})$,进而$\overline{[A,B]}$包含了$[\overline{A},\overline{B}]$.
    	\end{proof}
    	\item 设$G$是代数群,设$H\le G$是可解子群,那么$\overline{H}$也是可解群.
    	\begin{proof}
    		
    		我们有子群列$H=H_0\supseteq H_1\supseteq\cdots\supseteq H_n=\{e\}$,满足$[H_i,H_i]\subseteq H_{i+1}$.于是有$[\overline{H_i},\overline{H_i}]\subseteq\overline{H}_{i+1}$.于是子群列$\overline{H}=\overline{H_0}\supseteq\cdots\supseteq\overline{H_n}=\{e\}$保证$\overline{H}$是可解群.
    	\end{proof}
    \end{enumerate}
    \item 幂零群.一个代数群$G$称为幂零群如果它作为群是幂零的.此即它的幂零列$\mathscr{C}^0G=G$,$\mathscr{C}^{i+1}G=[G,\mathscr{C}^iG]$,在有限项后终止于$\{e\}$.
    \begin{enumerate}[(1)]
    	\item 阿贝尔群是幂零群;幂零群是可解群;幂零群的子群和商代数群都是幂零群;如果$G/Z(G)$是幂零群,那么$G$也是幂零群;如果$G$是幂零群,并且$n$是最小的自然数满足$\mathscr{C}^nG\not=\{e\}$,那么$\mathscr{C}^nG\subseteq Z(G)$,特别的这说明只要幂零群$G$非平凡,就有$Z(G)\not=e$;设$G$是幂零群,设$H\le G$是真子群,那么$H$严格包含在$\mathrm{N}_G(H)$中.
    	\item 设$G$是正维数的连通幂零代数群,那么$Z(G)$具有正维数,并且对任意真闭子群$H\le G$,都有$\dim H<\dim\mathrm{N}_G(H)$.特别的,这说明对余维数1的真闭子群$H$,总有$H$是正规子群.
    	\begin{proof}
    		
    		按照$G$是连通的,我们解释过$\mathscr{C}^iG$都是连通闭子群,取$n$是最大的自然数使得$\mathscr{C}^nG\not=\{e\}$,那么有$\mathscr{C}^nG\subseteq Z(G)$,进而有$Z(G)^0$非平凡,所以它是正维数的,因为零维连通代数簇是单点集.
    		
    		\qquad
    		
    		下面任取真闭子群$H\le G$,记$Z=Z(G)^0$,这是正维数的.如果$Z\subseteq H$,那么做商对维数归纳就能操作下去.如果$Z\not\subseteq H$,那么$ZH$是$\mathrm{N}_G(H)$的子群,并且$\dim H<\dim ZH\le\dim\mathrm{N}_G(H)$.
    	\end{proof}
        \item 如果$H\le G$是代数群的幂零子群,那么$\overline{H}$也是幂零的,这件事依旧是因为当子群$A\subseteq B$时有$[\overline{A},\overline{B}]=\overline{[A,B]}$.
    \end{enumerate}
    \item 幂单群.一个代数群称为幂单群,如果它由幂单元构成.例如$\mathrm{U}(n,K)$.代数群$G$的一个(未必闭的)子群称为幂单子群,如果它由幂单元构成.我们会证明幂单群可以刻画为$\mathrm{U}(n,K)$的子群.进而幂单群总是幂零群.
    \begin{enumerate}[(1)]
    	\item 设$G$是$\mathrm{GL}(V)$的幂单子群,其中$V$是非平凡的有限维线性空间.那么$G$在$V$中总有一个公共特征向量,也即存在$V$的一维$G$不变子空间.
    	\begin{proof}
    		
    		对$V$的维数做归纳,不妨设$V$是不可约$G$模(也即不能写成更小的$G$子模的直和).按照$G$是幂单的,任取$g\in G$就有$\mathrm{Tr}(g)=\mathrm{Tr}(1)=(\dim V)1_K$.类似的任取$g,h\in G$,从$g-e$幂零得到$\mathrm{Tr}(h)=\mathrm{Tr}(gh)=\mathrm{Tr}(h)+\mathrm{Tr}((g-1)h)$.进而$\mathrm{Tr}((g-1)h)=0$.设$G$在$K$线性下张成的$\mathrm{End}(V)$的子代数为$R$,那么有$\mathrm{Tr}((g-1)R)=0$.但是按照Burnside定理,$\mathrm{End}(V)$的一个子代数$R$如果在$V$上的作用是不可约的,那么必然有$R=\mathrm{End}(V)$.于是对任意$h\in\mathrm{End}(V)$都有$\mathrm{Tr}((g-1)h)=0$.这迫使$g-1=0$,也即$G=\{e\}$,此时不可约性导致$\dim V=1$.
    	\end{proof}
        \item 设$G\subseteq\mathrm{GL}(n,K)$是幂单子群,选取一维$G$不变子空间$V_1$,那么$G$作用在$V/V_1$上仍然是幂单的,反复归纳下去,就得到$G$共轭同构于$\mathrm{U}(n,K)$的子群.特别的,有$G$是幂零群.另外按照$\mathrm{U}(n,K)$是$\mathrm{GL}(n,K)$的闭子群,说明如果$H\le G$是代数群的幂单子群,那么$\overline{H}$也是幂单子群.
        \item 如果基域的特征$p\not=0$,那么$\mathrm{GL}(V)$中的一个自同构是幂单的当且仅当它的阶数是$p$的次幂.于是幂单群是幂零群推广了有限$p$群是幂零群这件事.
    \end{enumerate}
    \item Lie-Kolchin定理.设$G\subseteq\mathrm{GL}(V)$是连通可解子群,其中$V$是非平凡有限维空间.那么$G$在$V$上有公共特征向量.特别的,连通可解代数群$G$共轭同构于$\mathrm{T}(n,K)$的某个闭子群.
    \begin{proof}
    	
    	我们解释过$G$在$\mathrm{GL}(V)$中的闭包仍然是可解的,所以问题归结为设$G$是$\mathrm{GL}(V)$的闭子群.我们来对$n=\dim V$归纳.如果$n=1$,此时$V$的任意非零向量都是$G$的公共特征向量.如果$n>1$,我们断言它总存在非平凡的不变子空间.为此只需验证如果$V$是不可约$G$模,那么$\dim V=1$.
    	
    	\qquad
    	
    	为此我们来对$G$的导出长度(此即最小的自然数$d$使得$\mathscr{D}^dG=\{e\}$)做归纳.如果$d=1$,此时$G$是交换代数群,我们解释过此时它总有公共特征向量.下面设$d\ge2$.记$G'=[G,G]$,这是$G$的闭正规连通子群,并且它的导出长度恰好是$d-1$.按照归纳假设,就有$G'$存在公共特征向量.全体这样的公共特征向量线性生成的子空间记作$W\subseteq V$,那么$G'$在$W$上的作用是对角的.这里$W$是$G$不变空间,因为任取$g\in G$和$G'$的公共特征向量$x$,任取换位子$[g_1,g_2]$,那么有$[g_1,g_2]gx=g[g_1^{g^{-1}},g_2^{g^{-1}}]x=\lambda(gx)$.于是按照$V$的不可约性,就有$W=V$.于是$G'$在$V$上的作用是对角的(此即$G'$可同时对角化),于是$G'$交换,也即$d\le2$.
    	
    	\qquad
    	
    	我们断言$G'\subseteq Z(G)$:选取$V$的一组基使得$G'$可同时对角化,任取$g'\in G'$,$g\in G$,那么$gg'g^{-1}\in G'$,所以代数群态射$G\to G'$,$g\mapsto gg'g^{-1}$的像集是有限(零维)且连通的,迫使这个像集只能是$\{g'\}$,也即$g'\in Z(G)$.
    	
    	\qquad
    	
    	按照Schur引理,代数闭域$K$上代数$R$的单模$M$如果作为$K$模是有限的,那么$M$上的自同态都是数乘.取群代数$R=K[G]$,取$M=V$,$G'$中的元都是$V$上的$G$同态,于是$G'$在$V$上的作用都是数乘.但是换位子的行列式是1,所以$G'$作为数乘对应的元都是基域$K$上的$n$次单位根,于是$G'$是有限集合.但是$G'$是连通的,迫使$G'=\{e\}$,于是归结到$d=1$的情况.
    \end{proof}
\end{enumerate}
\newpage
\section{椭圆曲线}
\subsection{椭圆曲线上的几何}
\subsubsection{Weierstrass方程}

Weierstrass方程的定义.
\begin{enumerate}
	\item 固定一个完全域$K$,记它的代数闭包为$\overline{K}$.一个Weierstrass方程指的是$\mathbb{P}_{\overline{K}}^2$上的三元齐次方程:
	$$E:Y^2Z+a_1XYZ+a_3YZ^2=X^3+a_2X^2Z+a_4XZ^2+a_6Z^3$$
	
	这个方程定义的代数集如果是非奇异的,则称为椭圆曲线.称零点$O=[0,1,0]$是它的基点(base point).称$E$是定义在$K$上的,如果$a_1,\cdots,a_6\in K$.
	\item 在$D_+(Z)$上变成仿射情况.我们记$x=X/Z$和$y=Y/Z$,那么Weierstrass方程变成如下形式,它在仿射平面中的零点集只和射影情况相差了一个无穷远点$O=[0,1,0]$.对于仿射情况的Weierstrass方程,当我们提及它的零点集,或者提及它定义的曲线时总约定添加了这个无穷远点,换句话讲我们指的是射影情况的零点集.
	$$E:y^2+a_1xy+a_3y=x^3+a_2x^2+a_4x+a_6$$
	\item 做约化.设$\mathrm{char}(\overline{K})\not=2$,做代换$y\mapsto(y-a_1x-a_3)/2$,那么Weierstrass方程变为:
	$$E:y^2=4x^3+b_2x^2+2b_4x+b_6$$
	
	其中:
	$$b_2=a_1^2+4a_2,b_4=2a_4+a_1a_3,b_6=a_3^2+4a_6$$
	
	再记:
	\begin{align*}
		b_8&=a_1^2a_6+4a_2a_6-a_1a_3a_4+a_2a_3^2-a_4^2\\
		c_4&=b_2^2-24b_4\\
		c_6&=-b_2^3+36b_2b_4-216b_6\\
		\Delta&=-b_2^2b_8-8b_4^3-27b_6^2+9b_2b_4b_6\\
		j&=c_4^3/\Delta\\
		\omega&=\frac{\mathrm{d}x}{2y+a_1x+a_3}=\frac{\mathrm{d}y}{3x^2+2a_2x+a_4-a_1y}
	\end{align*}
	
	此时我们有:
	$$4b_8=b_2b_6-b_4^2,1728\Delta=c_4^3-c_6^2$$
	
	如果$\mathrm{char}(\overline{K})\not=2,3$,做代换:
	$$(x,y)\mapsto\left(\frac{x-3b_2}{36},\frac{y}{108}\right)$$
	
	那么Weierstrass方程变为:
	$$E:y^2=x^3-27c_4x-54c_6$$
	
	我们称$\Delta,j,\omega$分别为Weierstrass方程的判别式,$j$-不变量和不变微分形式.如果Weierstrass方程是:
	$$E:y^2=x^3+Ax+B$$
	
	那么它的$j$不变量是$j=-1728\frac{(4A)^3}{\Delta}$,它的判别式为$\Delta=-16(4A^3+27B^2)$.
	\item 我们解释了在$\mathrm{char}(\overline{K})\not=2,3$时,Weierstrass方程在仿射变换下具有简单的形式$y^2=x^3+Ax+B$.但是比方说对于$\mathbb{Q}$上的椭圆曲线,一个重要的手段是把问题归结到模素数$p$下,这就涉及到特征为2,3的情况.所以我们要尽可能处理全部特征的情况.
	\item 如果我们约定无穷远处的直线(此即$\mathbb{P}_{\overline{K}}^2$中的$Z=0$)和$E$的交只有基点$[0,1,0]$,那么使得Weierstrass方程仍具有相同形式的变换只有:
	$$(x,y)\mapsto(u^2x'+r,u^3y'+u^2sx'+t),u,r,s,t\in\overline{K},u\not=0$$
	
	并且此时Weierstrass方程的基本量具有如下关系:
	\begin{align*}
		ua_1'&=a_1+2s\\
		u^2a_2'&=a_2-sa_1+3r-s^2\\
		u^3a_3'&=a_3+ra_1+2t\\
		u^4a_4'&=a_4-sa_3+2ra_2-(t+rs)a_1+3r^2-2st\\
		u^6a_6'&=a_6+ra_4+r^2a_2+r^3-ta_3-t^2-rta_1\\
		u^2b_2'&=b_2+12r\\
		u^4b_4'&=b_4+rb_2+6r^2\\
		u^6b_6'&=b_6+2rb_4+r^2b_2+4r^3\\
		u^8b_8'&=b_8+3rb_6+3r^2b_4+r^3b_2+3r^4\\
		u^4c_4'&=c_4\\
		u^6c_6'&=c_6\\
		u^{12}\Delta'&=\Delta\\
		j'&=j\\
		u^{-1}\omega'&=\omega
	\end{align*}
	
	特别的,如果$\mathrm{char}(\overline{K})\not=2,3$,如果Weierstrass方程具有形式$y^2=x^3+Ax+B$,那么保此形式的仿射变换只有:
	$$(x,y)=(u^2x',u^3y'),u\in\overline{K}^*$$
	
	并且此时有:
	$$u^4A'=A,u^6B'=B,u^{12}\Delta'=\Delta$$
\end{enumerate}

基本性质.
\begin{enumerate}
	\item 关于奇点.设$E:f(x,y)=y^2+a_1xy+a_3y-x^3-a_2x^2-a_4x-a_6$,如果$P=(x_0,y_0)$是一个奇点,此即有:
	$$f(P)=\frac{\partial f}{\partial x}(P)=\frac{\partial f}{\partial y}(P)=0$$
	
	那么按照Taylor展开,我们有:
	$$f(x,y)=((y-y_0)-\alpha(x-x_0))((y-y_0)-\beta(x-x_0))-(x-x_0)^3$$
	
	于是点P处就有切线$y-y_0=\alpha(x-x_0)$和$y-y_0=\beta(x-x_0)$.如果$\alpha\not=\beta$,也即这两条切线不同,就称奇点$P$是一个结点(node,此即图像上看是打结的点);如果$\alpha=\beta$,也即这两条切线相同,就称奇点$P$是一个尖点(cusp,此即图像上看是尖凸的点).设$E$是一个Weierstrass方程,设$\mathrm{char}(\overline{K})\not=2$,我们有:
	\begin{enumerate}
		\item $E$是非奇异的(也即有奇点)当且仅当$\Delta=0$.特别的对于$y^2=4x^3+b_2x^2+2b_4x+b_6$形式的曲线,它是非奇异的当且仅当右侧三次多项式没有重根,并且奇点恰好是$(x_0,0)$,其中$x_0$是重根.
		\item $E$恰有一个结点当且仅当$\Delta=0$和$c_4\not=0$.
		\item $E$恰有一个尖点当且仅当$\Delta=c_4=0$.
	\end{enumerate}
	\begin{proof}
		
		因为我们约定过$E$视为它的零点集时总附带无穷远点$O=[0,1,0]$,所以我们先证明这个无穷远点总不是奇点.设$\mathbb{P}_{\overline{K}}^2$上的齐次方程:
		$$F(X,Y,Z)=Y^2Z+a_1XYZ+a_3YZ^2-X^3-a_2X^2Z-a_4XZ^2-a_6Z^3=0$$
		
		记$O=[0,1,0]$,那么$\frac{\partial F}{\partial Z}(O)=1\not=0$,所以$O$总是$E$的非奇点.下面设$E$有奇点$P_0=(x_0,y_0)$,那么做变换$x=x'+x_0$和$y=y'+y_0$不改变$c_4$和$\Delta$.所以我们可以不妨设$a_6=f(0,0)=0$,$a_4=\frac{\partial f}{\partial x}(0,0)=0$和$a_3=\frac{\partial f}{\partial y}(0,0)=0$.于是有:
		$$E:f(x,y)=y^2+a_1xy-a_2x^2-x^3=0$$
		
		我们有基本量$c_4=b_2^2-24b_4=(a_1^2+4a_2)^2$和$\Delta=0$.并且$c_4$就是二次多项式$y^2+a_1xy-a_2x^2$的判别式的平方,所以$P_0$的两条切线不同当且仅当这个判别式不为零,当且仅当$c_4=0$.这就证明了(a),(b),(c)的必要性.最后证明充分性,只需证明如果$E$是非奇异的,那么有$\Delta\not=0$.按照$\mathrm{char}(\overline{K})\not=2$,做代换可以使得$E$具有形式:
		$$E:y^2=4x^3+b_2x^2+2b_4x+b_6$$
		
		于是$E$是奇异的当且仅当存在$(x_0,y_0)\in E$使得$2y_0=12x_0^2+2b_2x_0^2+2b_4=0$.换句话讲,奇点恰好具有形式$(x_0,0)$,其中$x_0$是$4x^3+b_2x^2+2b_4x+b_6$的重根,但是$\Delta$按照定义就是这个三次多项式的判别式,于是$E$是奇异的当且仅当$\Delta=0$.
	\end{proof}
	\item 两条椭圆曲线(按照定义这是非奇异的,也即$\Delta\not=0$)在$\overline{K}$上同构当且仅当它们有相同的$j$-不变量.我们只证明$\mathrm{char}(\overline{K})\not=2,3$的情况.
	\begin{proof}
		
		我们前面默认了两条椭圆曲线同构等价于对应的Weierstrass方程有仿射变换,并且这样的变换不改变$j$-不变量,这说明了必要性.对于充分性,设$E$和$E'$具有相同的$j$-不变量.由于我们假定了特征非2,3.所以可设:
		$$E:y^2=x^3+Ax+B$$
		$$E':{y'}^2={x'}^3+A'x'+B'$$
		
		按照$j(E)=j(E')$,我们就有:
		$$\frac{(4A)^3}{4A^3+27B^2}=\frac{(4A')^3}{4{A'}^3+27{B'}^2}$$
		
		展开等价于$A^3{B'}^2={A'}^3B^2$.我们的目标是找形如$(x,y)=(u^2x',u^3y')$的同构.考虑如下三种情况:
		\begin{itemize}
			\item $A=0$,此等价于$j=0$.于是$A'=0$,按照$\Delta,\Delta'\not=0$,我们有$B,B'\not=0$,取$u=(B/B')^{1/6}$就得到同构.
			\item $B=0$,此等价于$j=1728$.于是$j'\not=0$,于是$A'\not=0$,于是$B'=0$,取$u=(A/A')^{1/4}$得到同构.
			\item $AB\not=0$,此等价于$j\not=0,1728$.此时$A'B'=0$,取$u=(A/A')^{1/4}=(B/B')^{1/6}$得到同构.
		\end{itemize}
	\end{proof}
	\item 设$j_0\in\overline{K}$,那么存在定义在$K(j_0)$上的(此即系数都在$K(j_0)$中)椭圆曲线满足它的$j$-不变量是$j_0$.
	\begin{proof}
		
		先设$j_0\not=0,1728$,对任意特征的$\overline{K}$,考虑椭圆曲线:
		$$E:y^2+xy=x^3-\frac{36}{j_0-1728}x-\frac{1}{j_0-1728}$$
		
		此时有:
		$$\Delta=\frac{j_0^3}{(j_0-1728)^3},j=j_0$$
		
		如下两个例子补充$j_0=0,1728$的情况.并且如果特征为2,那么$1728=0$,此时第一个曲线是非奇异的;如果特征为3,也有$1728=0$,此时第二个曲线是非奇异的,所以对任意特征这两个例子都补充了缺失的情况.
		$$E:y^2+y=x^3,\Delta=-27,j=0$$
		$$E:y^2=x^3+x,\Delta=-64,j=1728$$
	\end{proof}
	\item 设$E$是椭圆曲线,那么它的关于Weierstrass方程的微分不变量$\omega$总是全纯和不取零的,换句话讲除子$\mathrm{div}(\omega)=0$.
	\begin{proof}
		
		设$P=(x_0,y_0)\in E$,设$E:F(x,y)=y^2+a_1xy+a_3y-x^3-a_2x^2-a_4x-a_6=0$,那么我们有:
		$$\omega=\frac{\mathrm{d}(x-x_0)}{F_y(x,y)}=-\frac{\mathrm{d}(y-y_0)}{F_x(x,y)}$$
		
		这说明$P$不会是$\omega$的极点,否则$F_y(P)=F_x(P)=0$,导致$P$是一个奇点矛盾.接下来考虑映射$E\to\mathbb{P}^1$为$[x,y,1]\mapsto[x,1]$,这是一个次数为2的映射【】.于是$\mathrm{ord}_P(x-x_0)\le2$,并且这个不等式取等号当且仅当$F(x_0,y)$具有二重根.于是要么$\mathrm{ord}_P(x-x_0)=1$,要么$\mathrm{ord}_P(x-x_0)=2$并且$F_y(x_0,y_0)=0$.无论哪种情况,我们有$\mathrm{ord}_P(\omega)=\mathrm{ord}_P(x-x_0)-\mathrm{ord}_P(F_y)-1=0$,于是$P$总不是零点.所以我们只剩验证无穷远点$O$不是$\omega$的极点和零点.
		
		\qquad
		
		设$t$是$O$点的uniformizer,我们解释过$\mathrm{ord}_O(x)=-2$和$\mathrm{ord}_O(y)=-3$,于是有$x=t^{-2}f$和$y=t^{-3}g$,其中$f,g$是$E$上的有理函数,均不以$O$为零点或极点.我们有:
		$$\omega=\frac{\mathrm{d}x}{F_y(x,y)}=\frac{-2t^{-3}f+t^{-2}f'}{2t^{-3}g+a_1t^{-2}f+a_3}\mathrm{d}t=\frac{-2f+tf'}{2g+a_1tf+a_3t^3}\mathrm{d}t$$
		
		其中$f'=\mathrm{d}f/\mathrm{d}t$.我们知道一般的如果$f$在点$O$正则,那么$f'$也在点$O$正则.那么如果$\mathrm{char}(K)\not=2$,则该式在$O$处正则并且不为零,于是$\mathrm{ord}_O(\omega)=0$.如果$\mathrm{char}(K)=2$,我们用$\omega=\frac{\mathrm{d}y}{F_x(x,y)}$类似的得到结论.
	\end{proof}
	\item 设由Weierstrass方程给出的曲线$E$是奇异的,我们断言存在次数为1的有理映射$\varphi:E\to\mathbb{P}_{\overline{K}}^1$,此即$E$是双有理等价于$\mathbb{P}_{\overline{K}}^1$.
	\begin{proof}
		
		做平移变换,可设奇点是$(0,0)$,也即有$a_3=a_4=a_6=0$,也即$E:y^2+a_1xy=x^3+a_2x^2$.如果记$t=y/x$,那么有$t^2+a_1t=x+a_2$,这说明$x=t^2+a_1t-a_2,y=xt$都在$\overline{K}(t)$中.所以只要取有理映射:
		$$E\to\mathbb{P}_{\overline{K}}^1$$
		$$(x,y)\mapsto[x,y]$$
		
		它就有逆有理映射:
		$$\mathbb{P}_{\overline{K}}^1\to E$$
		$$[1,t]\mapsto(t^2+a_1t-a_2,t^3+a_1t^2-a_2t)$$
	\end{proof}
\end{enumerate}

Legendre形式.这是Weierstrass方程的另一种形式,它是指具有如下形式的方程:
$$y^2=x(x-1)(x-\lambda)$$
\begin{enumerate}
	\item 设$\mathrm{char}(\overline{K})\not=2$.那么每个$\overline{K}$上的椭圆曲线都同构于一个Legendre形式的椭圆曲线:
	$$E_{\lambda}:y^2=x(x-1)(x-\lambda),\lambda\in\overline{K},\lambda\not=0,1$$
	\begin{proof}
		
		按照特征$\not=2$,有Weierstrass方程可以变换为$E:y^2=4x^3+b_2x^2+2b_4x+b_6$.把$y$替换为$2y$,在代数闭域$\overline{K}$上就有分解$y^2=(x-e_1)(x-e_2)(x-e_3)$.按照椭圆曲线是非奇异的,这里$e_1,e_2,e_3$两两不同.再做代换$x=(e_2-e_1)x'+e_1$和$y=(e_2-e_1)^{3/2}y'$,就有${y'}^2=x'(x'-1)(x'-\lambda)$,其中$\lambda=\frac{e_3-e_1}{e_2-e_1}\in\overline{K}$,并且$\lambda\not=0,1$.
	\end{proof}
	\item 设$E_{\lambda}:y^2=x(x-1)(x-\lambda)$是椭圆曲线,那么$\lambda\not=0,1$.那么它的$j$-不变量为:
	$$j(E_{\lambda})=2^8\frac{(\lambda^2-\lambda+1)^3}{\lambda^2(\lambda-1)^2}$$
	\item 考虑映射$\overline{K}-\{0,1\}\to\overline{K}$,$\lambda\mapsto j(E_{\lambda})$.这总是满射,并且:
	\begin{itemize}
		\item 如果$j\not=0,1728$,那么这个映射是六对应一(换句话讲纤维都是六元集).
		\item 如果$\mathrm{char}(K)\not=3$,并且$j=0$,那么这个映射是二对应一.
		\item 如果$\mathrm{char}(K)\not=3$,并且$j=1728\not=0$,那么这个映射是三对应一.
		\item 如果$\mathrm{char}(K)=3$,并且$j=0=1728$,那么这个映射是一对应一.
	\end{itemize}
	\begin{proof}
		
		我们解释过椭圆曲线同构当且仅当它们的$j$-不变量相同.假设有$j(E_{\lambda})=j(E_{\mu})$,那么它们对应的Legendre形式的Weierstrass方程可以经代换$x=u^2x'+r$和$y=u^3y'$转化.于是我们有:
		$$x(x-1)(x-\lambda)=\left(x+\frac{r}{u^2}\right)\left(x+\frac{r-1}{u^2}\right)\left(x+\frac{r-\lambda}{u^2}\right)$$
		
		我们有六种方式对应这个等式两侧的一次多项式,它们对应了$\mu$的六种选取:
		$$\mu\in\left\{\lambda,\frac{1}{\lambda},1-\lambda,\frac{1}{1-\lambda},\frac{\lambda}{\lambda-1},\frac{\lambda-1}{\lambda}\right\}$$
		
		所以只要这六个数两两不同,那么$\lambda\mapsto j(E_{\lambda})$就是六对应一;当特征$\not=3$,$\lambda\in\{-1,2,1/2\}$时它是三对应一,这种情况对应于$j=1728$;当特征$\not=3$,$\lambda^2-\lambda+1=0$时它是二对应一,这种情况对应于$j=0$.最后如果$\mathrm{char}(K)=3$,那么仅有的使得这六个数中存在相同的项的情况是$\lambda=-1$,并且此时六个数都相同,于是此时是一对应一.
	\end{proof}
\end{enumerate}
\subsubsection{群结构}

设$E\subseteq\mathbb{P}^2_{\overline{K}}$是由Weierstrass方程给出的椭圆曲线,记$O=[0,1,0]$是基点.任取直线$L\subseteq\mathbb{P}^2_{\overline{K}}$,按照Bezout定理就有$L$和$E$有三个交点(计重数).下面设$P,Q\in E$,设经$P,Q$的直线为$L$,如果$P=Q$我们就取$L$是$E$的过该点的切线.设$L$和$E$的第三个交点为$R$.再取$R$和$O$确定的直线是$L'$,它和$E$的第三个交点记作$P+Q$.
\begin{enumerate}
	\item $(E,+)$是一个交换群.
	\begin{enumerate}
		\item 设直线$L$和$E$的交点$P,Q,R$,那么有$(P+Q)+R=O$.
		\item 交换律.对任意$P,Q\in E$有$P+Q=Q+P$.
		\item 基点$O$就是幺元,即对任意$P\in E$有$P+O=P$.
		\item 逆元.设$P\in E$,它的逆元$-P$就是$O$和$P$确定的直线和$E$的第三个交点.
		\item 结合律.对任意$P,Q,R\in E$有$(P+Q)+R=P+(Q+R)$.
	\end{enumerate}
	\item 设$E\subseteq\mathbb{P}^2_{\overline{K}}$是定义在$K$上的Weierstrass方程所定义的椭圆曲线,那么如下子集是它的子群:
	$$E(K)=\{(x,y)\in K^2\mid y^2+a_1xy+a_3y=x^3+a_2x^2+a_4x+a_6\}\cup\{O\}$$
	\begin{proof}
		
		这件事是因为如果$P,Q$是两个坐标在域$K$中的点,那么过这两个点的直线的系数也在$K$中,又因为$E$本身定义在$K$上,所以第三个交点的坐标也在$K$中,于是$P+Q$的坐标也在$K$中.
	\end{proof}
	\item 我们来给出$P+Q$和$-P$的具体表达式.设$E$是由如下Weierstrass方程定义的椭圆曲线:
	$$E:y^2+a_1xy+a_3y=x^3+a_2x^2+a_4x+a_6$$
	\begin{enumerate}
		\item 设$P_0=(x_0,y_0)\in E$,那么:
		$$-P_0=(x_0,-y_0-a_1x_0-a_3)$$
		\item 设$P_i\in E$,设$P_1+P_2=P_3$,记$P_i=(x_i,y_i),i=1,2,3$.如果$x_1=x_2$且$y_1+y_2+a_1x_2+a_3=0$,那么有$P_1+P_2=O$.否则我们定义:
		$$\lambda=\left\{\begin{array}{cc}\frac{y_2-y_1}{x_2-x_1}&x_1\not=x_2\\\frac{3x_1^2+2a_2x_1+a_4-a_1y_1}{2y_1+a_1x_1+a_3}&x_1=x_2\end{array}\right.$$
		$$\nu=\left\{\begin{array}{cc}\frac{y_1x_2-y_2x_1}{x_2-x_1}&x_1\not=x_2\\\frac{-x_1^3+a_4x_1+2a_6-a_3y_1}{2y_1+a_1x_1+a_3}&x_1=x_2\end{array}\right.$$
		
		那么$y=\lambda x+\nu$是过$P_1$和$P_2$的直线(在$P_1=P_2$时理解为该点的切线).此时我们有:
		\begin{align*}
			x_3&=\lambda^2+a_1\lambda-a_2-x_1-x_2\\y_3&=-(\lambda+a_1)x_3-\nu-a_3
		\end{align*}
		\item 特别的,如果$P_1\not=\pm P_2$,此即等价于$x_1\not=x_2$.此时有:
		$$x_3=\left(\frac{y_2-y_1}{x_2-x_1}\right)^2+a_1\left(\frac{y_2-y_1}{x_2-x_1}\right)-a_2-x_1-x_2$$
		
		如果$P_1=P_2=P=(x,y)$,那么有:
		$$x_3=\frac{x^4-b_4x^2-2b_6x-b_8}{4x^3+b_2x^2+2b_4x+b_6}$$
	\end{enumerate}
	\item 考虑$\mathbb{Q}$上由$y^2=x^3+17$定义的椭圆曲线$E$.取$P_1=(-2,3)$,$P_2=(-1,4)$,$P_3=(2,5)$,$P_4=(4,9)$,$P_5=(8,23)$,$P_6=(43,282)$,$P_7=(52,375)$和$P_8=(5234,378661)$.这里有两件不是特别平凡的事情,一个是$E(\mathbb{Q})$作为阿贝尔群被$P_1$和$P_2$生成,进而有$E(\mathbb{Q})\cong\mathbb{Z}\times\mathbb{Z}$.另一个是$\{\pm P_1,\cdots,\pm P_8\}$是$y^2=x^3+17$的全部$\mathbb{Z}$点.这两件事对应了两个比较重要的定义,Mordell-Weil定理断言$\mathbb{Q}$上椭圆曲线的有理点群总是有限生成的;Siegel定理断言椭圆曲线上的整点集合总是有限集合.
\end{enumerate}

奇异的情况.设$E$是代数闭域$\overline{K}$上被Weierstrass方程定义的曲线.它的全体非奇异点构成的子集记作$E_{\mathrm{ns}}$,称为$E$的非奇异部分.
\begin{enumerate}
	\item 如果$E$是定义在域$K$上的(此即Weierstrass方程的系数在域$K$中),把$E(K)$的全体非奇异点构成的子集记作$E_{\mathrm{ns}}(K)$.我们断言$E_{\mathrm{ns}}(K)$按照我们定义的加法同样构成阿贝尔群.
	\begin{proof}
		
		设$E$是奇异曲线,那么我们解释过它恰有一个奇点$S$,如果$L$是直线,那么$E\cap L$中奇点的重数至少为2.我们解释过基点$O$总是非奇点.假设$P,Q$是非奇点,我们断言$P+Q$和$-P$一定也是非奇点:$-P$按照定义是$P$和$O$所在直线$L$和$E$的第三个交点,如果这是奇点,那么$P\not=-P$,这导致$L\cap E$在计重数意义下至少有4个点,这和Bezout定理矛盾;同理,如果记$P$和$Q$所在直线$L'$和$E$的第三个交点为$R$,那么$R=-(P+Q)$,我们只需说明$R$不是奇点,而这是因为,不妨设$R\not=P,Q$,倘若$R$是奇点则重数至少为2,则$L'\cap E$在计重数意义下至少有4个点,这也和Bezout定理矛盾.
	\end{proof}
	\item 如果$E$的唯一奇点$S$是结点,此即基本量$c_4\not=0$,也即$E$在点$S$处的两条切线是不同的,记作$y=\alpha_ix+\beta_i,i=1,2$.那么有阿贝尔群的同构:
	$$E_{\mathrm{ns}}\to\overline{K}^*$$
	$$(x,y)\mapsto\frac{y-\alpha_1x-\beta_1}{y-\alpha_2x-\beta_2}$$
	\begin{proof}
		
		不妨设奇点是$S=(0,0)$,那么Weierstrass方程为$y^2+a_1xy=x^3+a_2x^2$.适当选取$s\in\overline{K}$,用$y+sx$替换$y$可以消去等式中的$x^2$项.还原回齐次形式,我们可设Weierstrass方程为:
		$$E:Y^2Z+AXYZ-X^3=0$$
		
		而$S=[0,0,1]$是结点等价于讲$A\not=0$.奇点$S$的两条切线为$Y=0$和$Y+AX=0$.我们构造映射:
		$$E_{\mathrm{ns}}\to\overline{K}^*$$
		$$[X,Y,Z]\mapsto 1+\frac{AX}{Y}$$
		
		再做代换$X=A^2(X'-Y')$,$Y=A^3Y'$,$Z=Z'$,把新的坐标同样记作$X,Y,Z$,那么方程变成$E:XYZ-(X-Y)^3=0$.奇点仍然是$S=[0,0,1]$,基点变成$O=[1,1,0]$,映射变成$f:[X,Y,Z]\mapsto\frac{X}{Y}$.考虑仿射平面$D_+(Y)$,也即做变换$Y=1$,$x=X/Y$和$z=Z/Y$,那么方程变为$E:xz-(x-1)^3=0$,并且奇点变成了新的仿射坐标的无穷远点,上述映射在仿射坐标下为$f:E_{\mathrm{ns}}\to\overline{K}^*$,$(x,z)\mapsto x$.这有逆映射$\overline{K}^*\to E_{\mathrm{ns}}$,$t\mapsto\left(t,\frac{(t-1)^3}{t}\right)$,所以$f$是双射.接下来只需证明$f$是一个同态.我们断言如果$P,Q,R$是同一条直线和$E_{\mathrm{ns}}$的三个交点,那么$f(P)f(Q)f(R)=1$.一旦这成立,那么从$f(P+Q)f(R)=1=f(P)f(Q)f(R)$得到$f(P+Q)=f(P)f(Q)$.证明我们的断言:设直线$z=ax+b$和$E_{\mathrm{ns}}$交于$(x_i,z_i),i=1,2,3$,那么$x_1,x_2,x_3$是方程$x(ax+b)-(x-1)^3=0$的根,韦达定理就得到$x_1x_2x_3=1$.
	\end{proof}
	\item 如果$E$的唯一奇点$S$是尖点,此即基本量$c_4=0$,也即$E$在点$S$处的两条切线是相同的,记作$y=\alpha x+\beta$.那么有如下阿贝尔群的同构,其中$x_S$表示$S$的$x$坐标:
	$$E_{\mathrm{ns}}\to\overline{K}^+$$
	$$(x,y)\mapsto\frac{x-x_S}{y-\alpha x-\beta}$$
	\begin{proof}
		
		和上一条证明一样的,我们可以约化到奇点$S=(0,0)$,方程为$E:Y^2Z+AXYZ-X^3=0$的情况,这里$S=(0,0)$是尖点导致$A=0$,那么$E$的关于点$S=[0,0,1]$的两条重合切线就是$Y=0$.考虑如下映射:
		$$f:E_{\mathrm{ns}}\to\overline{K}^+$$
		$$[X,Y,Z]\mapsto\frac{X}{Y}$$
		
		取仿射坐标$D_+(Y)$,也即做代换$Y=1$,$x=X/Y$和$z=Z/Y$,那么方程变成$E:z-x^3=0$,映射变成$f:E_{\mathrm{ns}}\to\overline{K}^+$,$(x,z)\mapsto x$.此时唯一的奇点在新的仿射坐标下变成无穷远点.$f$有逆映射$t\mapsto(t,t^3)$,于是$f:E_{\mathrm{ns}}\to\overline{K}^+$是双射.最后还剩下证明$f$是同态,和上一条一样归结为证明如果$P,Q,R$是同一条直线和$E_{\mathrm{ns}}$的交点,那么$f(P)+f(Q)+f(R)=0$.为此设直线$z=ax+b$和$E_{\mathrm{ns}}$的三个交点的坐标是$(x_i,z_i),i=1,2,3$.那么$x_1,x_2,x_3$就是$ax+b-x^3=0$的三个根,按照韦达定理有$x_1+x_2+x_3=0$.
	\end{proof}
\end{enumerate}
\subsubsection{椭圆曲线}

一个椭圆曲线指的是带基点$O$的亏格1的非奇异曲线$E$,记作$(E,O)$,或者不引起歧义的时候简记作$E$.我们称$(E,O)$是定义在域$K$上的,如果$E$是$K$上的曲线,并且$O\in E(K)$.我们下面给出Weierstrass方程和椭圆曲线的联系,设$E$是定义在域$K$上的椭圆曲线.
\begin{enumerate}
	\item 存在$E$上的正则函数$x,y\in K(E)$(这里$K(E)$表示$E$在域$K$上的函数域,光滑曲线上的有理函数总是正则函数),使得如下映射
	$$\varphi:E\to\mathbb{P}^2_{\overline{K}}$$
	$$P\mapsto[x(P),y(P),1]$$
	
	是从$K$上的曲线$E$到由某个$K$系数的Weierstrass方程
	$$C:Y^2+a_1XY+a_3Y=X^3+a_2X^2+a_4X+a_6$$
	
	所定义的曲线的同构,并且满足$\varphi(O)=[0,1,0]$.我们称正则函数$x,y$是椭圆曲线$E$的Weierstrass坐标.
	\begin{proof}
		
		按照Riemann-Roch定理,对于亏格$g$的曲线上除子$D$,如果$\deg D>2g-2$,那么有$l(D)=\deg D-g+1$.考虑$E$上的除子$n(O)$,其中$n$取正整数,那么有$l(n(O))=n$.一般的如果$C$是域$K$上的光滑曲线,如果$D$是除子,那么按照定义$\mathscr{L}(D)$是$\overline{K}$上的线性空间,它一定存在由域$K$上正则函数构成的基【】.所以按照$l(2(O))=2$,可以找到$x\in K(E)$使得$\{1,x\}$是$\mathscr{L}(2(O))$的一组基.再按照$\mathscr{L}(2(O))\subseteq\mathscr{L}(3(O))$,结合$l(3(O))=3$,又可以找到$y\in K(E)$使得$\{1,x,y\}$是$\mathscr{L}(3(O))$的一组基.另外按照我们的选取,这里$x$必须恰好在点$O$处是二阶极点,$y$必须恰好在点$O$处是三阶极点.按照定义$\mathscr{L}(6(O))$包含的是那些唯一可能的极点是$O$,并且次数不超过6的有理函数构成的空间,这包含了七个元素$\{1,x,y,x^2,xy,y^2,x^3\}$,所以它们必须是$\overline{K}$线性相关的,但是它们都是域$K$上的有理函数,所以它们是$K$线性相关的,也即存在不全为零的$A_1,\cdots,A_7\in K$满足:
		$$A_1+A_2x+A_3y+A_4x^2+A_5xy+A_6y^2+A_7x^3=0$$
		
		另外这里$A_6A_7\not=0$,否则这个等式的每个单项以点$O$为极点的次数都是不同的,会导致$A_i$全部为零矛盾.把$x$和$y$分别替换为$-A_6A_7x$和$A_6A_7^2y$则得到Weierstrass方程.记它对应的代数集为$C$,那么我们有有理映射$\varphi:E\to C$,$P\mapsto[x(P),y(P),1]$.这是正则映射因为源端为光滑曲线的有理映射一定是正则映射(另外我们始终约定$K$是完全域,所以其上曲线是非奇异的等价于光滑的).另外射影曲线之间的正则映射要么是满射要么是常值映射,所以这里$\varphi$是满射.另外按照$y$相比$x$在极点$O$的次数更高,所以$\varphi(O)=[0,1,0]$.一般的两条光滑射影曲线之间次数1的正则映射一定是同构(这里的次数的定义是,设$\varphi:C_1\to C_2$是两条曲线之间的正则映射,它诱导了函数域之间的域扩张$\varphi^*:K(C_2)\to K(C_1)$,定义$\deg\varphi=[K(C_1):K(C_2)]$).所以我们接下来只需证明$\varphi$的次数为1,以及$C$是光滑曲线.
		
		\qquad
		
		先证明$\varphi:E\to C\subseteq\mathbb{P}^2_{\overline{K}}$的次数为1,也即$K(E)=K(x,y)$.考虑映射$[x,1]:E\to\mathbb{P}^1_{\overline{K}}$,$P\mapsto[x(P),1]$.一般的如果$\varphi:C_1\to C_2$是两条光滑射影曲线之间的非常值映射,那么对任意$Q\in C_2$,总有$\deg\varphi=\sum_{P\in\varphi^{-1}(Q)}\mathrm{ord}_P(\varphi^*t_{\varphi(P)})$,其中$t_{\varphi(P)}\in K(C_2)$是$\varphi(P)$处的uniformizer.把这个结论用在$[x,1]$和$Q=[x(O),1]=[1,0]$上,按照$x$只有极点$O$并且次数2,得到$\deg[x,1]=2$,也即$[K(E):K(x)]=2$.同理$[K(E):K(y)]=3$.那么$[K(E):K(x,y)]$同时整除2和3,于是它是1.
		
		\qquad
		
		最后证明$C$是光滑的(因为$K$是完全域,光滑等价于非奇异).假设$C$是奇异曲线,那么我们解释过存在次数1的有理映射$\psi:E\to\mathbb{P}^1_{\overline{K}}$.那么复合有理映射$\psi\circ\varphi:E\to\mathbb{P}^1_{\overline{K}}$是光滑曲线之间的次数为1的有理映射,所以这是一个同构.但是$E$的亏格是1,而$\mathbb{P}^1$的亏格是0,这矛盾.
	\end{proof}
	\item 特别的,上一条的证明告诉我们如果$E$是域$K$上的椭圆曲线,如果$\{x,y\}$是Weierstrass坐标,那么$K(E)=K(x,y)$.
	\item 设$\{x,y\}$是椭圆曲线$(E,O)$的Weierstrass坐标,那么点$O$是$x$的二阶极点,也是$y$的三阶极点.按照$O$是它们的唯一的极点,前面证明里解释了有$[K(E):K(x)]=2$和$[K(E):K(y)]=3$.
	\begin{proof}
		
		按照前面命题的同构,归结为证明$X/Z={\varphi^*}^{-1}(x)$和$Y/Z={\varphi^*}^{-1}(y)$在点$[0,1,0]$分别是二阶极点和三阶极点.把方程齐次化,再取$Y=1$的仿射坐标,那么方程变为$E:Z+a_1XZ+a_3Z^2=X^3+a_2X^2Z+a_4XZ^2+a_6Z^3$.此时基点$[0,1,0]$的坐标变为$(0,0)$.我们需要算$X$和$Z$在点$(0,0)$的次数.$K[X,Z]$中的在$(0,0)$处取零的多项式构成的理想就是$(X,Z)$.所以如果上述方程除了$Z$以外的单项式都至少在点$(0,0)$为二阶零点,这导致$Z$在点$(0,0)$至少为二阶零点,但是这又导致除了$Z$以外的单项式至少在点$(0,0)$是三阶零点,进而$Z$至少在点$(0,0)$为三阶零点.另外因为曲线是非奇异的,如果把Weierstrass方程记作$f(X,Z)$,那么$(K[X,Z]/f(X,Z))_{(X,Z)}$是DVR,所以它的极大理想$(x,z)$理应包含uniformizer,换句话讲$Z,X$中至少有一个在点$(0,0)$是一阶零点,则它必须是$X$,进而有$Z$在点$(0,0)$恰为三阶零点.于是$X/Z$是二阶极点,$1/Z$是三阶极点.
	\end{proof}
	\item 设椭圆曲线$E$的两个Weierstrass坐标为$\{x,y\}$和$\{x',y'\}$,那么它们的Weierstrass方程之间可以经如下变换得到:
	$$X=u^2X'+r,Y=u^3Y'+su^2X'+t,u\in K^*,r,s,t\in K$$
	\begin{proof}
		
		我们解释了$\{x,x'\}$在点$O$是二阶极点,$\{y,y'\}$在点$O$是三阶极点.于是$\{1,x\}$和$\{1,x'\}$都是$\mathscr{L}(2(O))$的基,$\{1,x,y\}$和$\{1,x',y'\}$都是$\mathscr{L}(3(O))$的基.于是存在$u_1,u_2\in K^*$和$r,s_2,t\in K$使得$x=u_1x'+r$和$y=u_2y'+s_2x'+t$.因为$(x,y)$和$(x',y')$都满足Weierstrass方程,所以$Y^2$和$X^3$的系数得相同,于是$u_1^3=u_2^2$,取$u=u_1^{1/2}=u_2^{1/3}$和$s=s_2/u^2$,那么有$x=u^2x'+r$和$y=u^3y'+su^2x'+t$.
	\end{proof}
	\item 反过来,域$K$上的Weierstrass方程如果定义了非奇异曲线,那么它是基点为$O=[0,1,0]$的域$K$上的椭圆曲线.
	\begin{proof}
		
		设非奇异曲线$E$被一个Weierstrass方程所定义.我们解释过不变微分$\omega=\frac{\mathrm{d}x}{2y+a_1x+a_3}$是全纯和不取零的,也即它没有极点和零点,也即$\mathrm{div}(\omega)=0$.那么Riemann-Roch定理说明$2g-2=\deg\mathrm{div}(\omega)=0$,于是$E$具有亏格1,选取基点为$O$即可.
	\end{proof}
\end{enumerate}

用除子描述椭圆曲线的群法则.
\begin{enumerate}
	\item 引理.设$C$是代数闭域$\overline{K}$上亏格为1的曲线,设$P,Q\in C$,那么除子$(P)$和$(Q)$线性等价当且仅当$P=Q$.
	\begin{proof}
		
		如果$(P)$和$(Q)$线性等价,也即存在$f\in\overline{K}(C)$使得$\mathrm{div}(f)=(P)-(Q)$,那么$f\in\mathscr{L}((Q))$.按照$\deg(P)=1>2g-2=0$,Riemann-Roch定理说明$\dim\mathscr{L}((Q))=\deg D-g+1=1$,这迫使$\mathscr{L}((Q))=\overline{K}$,即常值函数构成的空间,于是$f\in\overline{K}$,于是$P=Q$.
	\end{proof}
	\item 设$(E,O)$是代数闭域$\overline{K}$上的椭圆曲线,记$\mathrm{Div}^0(E)$表示$E$上次数为零的除子构成的群,我们知道一般的光滑曲线上的主除子的次数总是零,所以主除子构成的群是$\mathrm{Div}^0(E)$,它们的商群记作$\mathrm{Pic}^0(E)$,称为皮卡群的零次部分.我们断言$E$上之前定义的几何角度的群结构和这里代数角度定义的群$\mathrm{Pic}^0(E)$是典范同构的.
	\begin{enumerate}
		\item 对任意零次除子$D\in\mathrm{Div}^0(E)$,存在唯一的点$P\in E$,使得$D$和$(P)-(O)$线性等价.据此我们可以定义一个映射$\sigma:\mathrm{Div}^0(E)\to E$,并且这是一个满射.
		\begin{proof}
			
			因为$E$的亏格为1,按照$\deg(D+(O))=1>2g-2=0$,Riemann-Roch定理说明有$\dim\mathscr{L}(D+(O))=\deg(D+(O))-g+1=1$.设$f\in\overline{K}(E)$是$\mathscr{L}(D+(O))$的非零元,那么它是这个空间的一组基.按照$\mathrm{div}(f)\ge-D-(O)$和$\deg(\mathrm{div}(f))=0$,所以$\mathrm{div}(f)$必须具有形式$-D-(O)+(P)$,其中$P\in E$.于是有$D$和$(P)-(O)$线性等价.这解决了$P$的存在性.至于唯一性,假设还有一个点$P'\in E$满足结论,那么$(P)$线性等价于$D+(O)$线性等价于$(P')$,于是按照上面引理就有$P=P'$.最后$\sigma$是满射因为任取$P\in E$,那么明显有$\sigma((P)-(O))=P$.
		\end{proof}
		\item 设$D_1,D_2\in\mathrm{Div}^0(E)$,那么$\sigma(D_1)=\sigma(D_2)$当且仅当$D_1$和$D_2$线性等价.于是$\sigma$诱导了双射$\mathrm{Pic}^0(E)\cong E$,这个映射同样记作$\sigma$.它的逆映射记作$\kappa:E\to\mathrm{Pic}^0(E)$,它把点$P$映射为$(P)-(O)$所在的除子类.
		\begin{proof}
			
			设$D_1,D_2\in\mathrm{Div}^0(E)$,记$P_i=\sigma(D_i),i=1,2$.按照定义有$(P_1)-(P_2)$线性等价于$D_1-D_2$.于是$D_1$线性等价于$D_2$当且仅当$(P_1)$线性等价于$(P_2)$,按照引理这当且仅当$P_1=P_2$.
		\end{proof}
		\item $\kappa$是群同态,这完成我们命题的证明.
		\begin{proof}
			
			设椭圆曲线$E$被一个Weierstrass方程定义,我们需要证明的是对$P,Q\in E$,有$\kappa(P+Q)=\kappa(P)+\kappa(Q)$.设$\mathbb{P}^2_{\overline{K}}$中的直线$L:f(X,Y,Z)=\alpha X+\beta Y+\gamma Z$是$P$和$Q$定义的直线.设第三个交点为$R$.再设$R$和$O$确定的直线为$L':f'(X,Y,Z)=\alpha' X+\beta' Y+\gamma' Z$.再按照$Z=0$和$E$的交点是重数为3的$O$,所以有:
			$$\mathrm{div}(f/Z)=(P)+(Q)+(R)-3(O)$$
			$$\mathrm{div}(f'/Z)=(R)+(P+Q)-2(O)$$
			
			于是我们有$(P+Q)-(P)-(Q)+(O)=\mathrm{div}(f'/f)$是主除子,另外明显有$\kappa(O)$是主除子类.于是有$\kappa(P+Q)=\kappa(P)+\kappa(Q)$.
		\end{proof}
	\end{enumerate}
	\item 推论.设$E$是椭圆曲线,设$D=\sum n_p(P)$是除子,那么$D$是主除子当且仅当在整数加法下有$\sum n_P=0$,并且在$E$的加法下(此即$E$上几何层面定义的加法)有$\sum n_PP=O$.
	\begin{proof}
		
		光滑曲线上的主除子一定是零次的.设$D$是一个零次除子,那么$D$是主除子当且仅当$\sigma(D)=O$,当且仅当在$E$的加法下有$\sum n_P\sigma((P)-(O))=O$,也即在$E$的加法下有$\sum n_PP=O$.
	\end{proof}
	\item 推论.设$C$是$\overline{K}$上的光滑曲线,那么我们有基本正合列:
	$$\xymatrix{1\ar[r]&\overline{K}^*\ar[r]&\overline{K}(C)^*\ar[r]^{\mathrm{div}}&\mathrm{Div}^0(C)\ar[r]&\mathrm{Pic}^0(C)\ar[r]&0}$$
	
	结合上面的同构得到正合列:
	$$\xymatrix{1\ar[r]&\overline{K}^*\ar[r]&\overline{K}(C)^*\ar[r]^{\mathrm{div}}&\mathrm{Div}^0(C)\ar[r]^{\sigma}&E\ar[r]&0}$$
	
	另外一般的如果$C$是域$K$上的曲线,那么有正合列:
	$$\xymatrix{1\ar[r]&K^*\ar[r]&K(C)^*\ar[r]^{\mathrm{div}_K}&\mathrm{Div}_K^0(C)\ar[r]&\mathrm{Pic}_K^0(C)}$$
	
	如果$C$的亏格为1,并且$K$值点$C(K)$非空,那么最后一个映射$\mathrm{Div}_K^0(C)\to\mathrm{Pic}_K^0(C)$是满射.于是在我们这个情况下如果对正合列取$G_{\overline{K}/K}$不变,就得到正合列:
	$$\xymatrix{1\ar[r]&K^*\ar[r]&K(E)^*\ar[r]^{\mathrm{div}_K}&\mathrm{Div}_K^0(E)\ar[r]^{\sigma}&E(K)\ar[r]&0}$$
\end{enumerate}

我们接下来证明一个基本事实,椭圆曲线是簇范畴上的群对象,也即一个阿贝尔簇.更具体的讲,设$E$是域$K$上的椭圆曲线,那么$E$上的加法$E\times E\to E$,$(P,Q)\mapsto P+Q$和取逆元的映射$E\to E$,$P\mapsto -P$都是簇之间的态射.
\begin{proof}
	
	我们之前给出过椭圆曲线上加法的具体表达式,对点$(x,y)$,它的逆元是$(x,-y-a_1x-a_3)$,按照$E$是光滑的,光滑曲线为源端的有理映射是正则映射(也即簇之间的态射),所以取逆元的映射的确是一个簇之间的态射.按照加法的具体表达式,$(P,Q)\mapsto P+Q$的仅有的可能非正则的点是$(P,P)$,$(P,-P)$,$(P,O)$,$(O,P)$,因为其余点的加法的结果的表达式中$\lambda$和$\nu$都是一个常数.
	
	\qquad
	
	固定一个点$Q\not=0$,那么按照我们的具体表达式,就有$\tau_Q:E\to E$,$P\mapsto P+Q$是有理映射,按照源端仍然是一个光滑曲线,这个有理映射是正则映射.而它明显有正则的逆映射$P\mapsto P-Q$,所以$\tau$总是一个同构.接下来我们考虑如下态射的复合,其中$Q_1,Q_2\in E$:
	$$\xymatrix{E\times E\ar[rr]^{\tau_{Q_1}\times\tau_{Q_2}}&&E\times E\ar[r]^+&E\ar[r]^{\tau_{Q_1}^{-1}}&E\ar[r]^{\tau_{Q_2}^{-1}}&E}$$
	$$\xymatrix{(P_1,P_2)\ar@{|->}[r]&(P_1+Q_1,P_2+Q_2)\ar@{|->}[r]&P_1+Q_1+P_2+Q_2\ar@{|->}[r]&P_1+P_2+Q_2\ar@{|->}[r]&P_1+P_2}$$
	
	所以只要$\varphi$有定义的点都和加法是一致的.按照$\tau_Q$总是同构,所以$\varphi$仅有可能的没有定义的点是:
	$$(P-Q_1,P-Q_2),(P-Q_1,-P-Q_2),(P-Q_1,-Q_2),(-Q_1,P-Q_2)$$
	
	但是这里$Q_1,Q_2$是任意的,当它们跑遍所有点的时候$\varphi$有定义的点跑遍整个$E\times E$,这说明加法是定义在整个$E\times E$上的,所以加法是态射.
\end{proof}
\subsubsection{同源}

两条椭圆曲线之间的态射$\varphi:E_1\to E_2$称为同源(isogeny),如果它把基点映射为基点.一般的光滑曲线之间的态射要么是满射要么是常值映射,所以一个同源$\varphi:E_1\to E_2$要么满足$\varphi(E_1)=\{O\}$,要么满足$\varphi(E_1)=E_2$.如果两条椭圆曲线之间存在非常值的同源态射,就称这两条椭圆曲线是同源的(isogenous).
\begin{enumerate}
	\item 我们在后面会证明同源是椭圆曲线上的等价关系,唯一需要证明的是对称性,此为对偶同源的概念.后文也会证明同源是阿贝尔群同态,所以同源就是椭圆曲线作为阿贝尔簇的态射.
	\item 设$\varphi,\psi:E_1\to E_2$是两条椭圆曲线之间的两个同源,我们之前证明过椭圆曲线上的加法是态射,于是$\varphi+\psi:E_1\to E_2$也是态射,并且也把基点映射为基点,所以它也是一个同源.于是全体同源构成的集合$\mathrm{Hom}(E_1,E_2)$是一个阿贝尔群.如果$E_1=E_2=E$,把复合作为乘法,则$\mathrm{End}(E)=\mathrm{Hom}(E,E)$是一个环,称为椭圆曲线$E$的自同态环,这个环的单位群称为$E$的自同构群,记作$\mathrm{Aut}(E)$.另外如果$E_1,E_2,E$是定义在域$K$上的,则相应的域$K$上的同源构成的群或者环记作:
	$$\mathrm{Hom}_K(E_1,E_2),\mathrm{End}_K(E),\mathrm{Aut}_K(E)$$
\end{enumerate}

设$E$是椭圆曲线,对任意$m\in\mathbb{Z}$,我们用$[m]:E\to E$表示数乘$m$的同源,具体地讲如果$m>0$则为$P\mapsto mP$,如果$m<0$则为$P\mapsto(-m)(-P)$,并且定义$[0](P)=O,\forall P\in E$.另外如果$E$是定义在域$K$上的,则$[m]$总是定义在域$K$上的.
\begin{enumerate}
	\item 设$E$是定义在域$K$上的椭圆曲线,设$0\not=m\in\mathbb{Z}$,那么$[m]:E\to E$总不是常值映射.另外我们会在后文证明更强的结论:$\deg[m]=m^2,\forall m\in\mathbb{Z}$.
	\begin{proof}
		
		我们先证明$[2]$不是常值映射.按照我们之前给出的加法公式,如果点$P=(x,y)$满足$2P=O$,那么有$4x^3+b_2x^2+2b_4x+b_6=0$.倘若$\mathrm{char}(K)\not=2$,那么满足这个等式的$x$至多有限个,导致至多有限个点$P$满足$2P=O$,但是曲线上一定有无限个点,所以此时$[2]$不是常值映射.倘若$\mathrm{char}(K)=2$,则上述等式变为$b_2x^2+b_6=0$,如果$E$中每个点都满足这个等式,那么必须有$b_2=b_6=0$,但是此时有$\Delta=0$,和$E$是非奇异的矛盾.综上总有$[2]$不是常值映射.接下来按照$[mn]=[m]\circ[n]$,所以问题归结为证明对奇数$m$有$[m]$非常值映射.
		
		\qquad
		
		先设$\mathrm{char}(K)\not=2$,那么$4x^3+b_2x^2+2b_4x+b_6$整除$x^4-b_4x^2-2b_6x-b_8$会得到$\Delta=0$矛盾.于是存在$x_0\in\overline{K}$作为第一个多项式的零点的次数(如果不是零点约定零次)高于$x_0$作为第二个多项式零点的次数.再取$y_0\in\overline{K}$使得$P_0=(x_0,y_0)\in E$.于是椭圆曲线上加法公式告诉我们$[2]P_0=O$.那么对于奇数$m$就有$[m]P_0=P_0\not=O$,于是$[m]$不是常值映射.
		
		\qquad
		
		如果$\mathrm{char}(K)=2$,我们可以找到一个点$P_0$在椭圆曲线上的加法的阶数为3,也即$P_0,2P_0\not=O$,$3P_0=O$.这说明如果$m$和3互素则$[m]$不是常值映射.【】
	\end{proof}
	\item 设$E_1,E_2$是椭圆曲线,那么$\mathrm{Hom}(E_1,E_2)$是无挠$\mathbb{Z}$模(这里$\mathrm{Hom}(E_1,E_2)$作为$\mathbb{Z}$模按照定义有$m\varphi=[m]\circ\varphi$,另外PID上无挠模就等价于自由模).换句话讲如果$\varphi\in\mathrm{Hom}(E_1,E_2)$,$m\in\mathbb{Z}$满足$[m]\circ\varphi=[0]$,那么要么$m=0$要么$\varphi=[0]$.
	\begin{proof}
		
		对$[m]\circ\varphi=[0]$取次数(如果$\varphi:E_1\to E_2$是射影曲线之间的态射,则它诱导了函数域的扩张$\varphi^*:K(E_2)\to K(E_1)$,这总是有限扩张,它的次数称为态射$\varphi$的次数,并且约定常值态射的次数为零),得到$\deg[m]\deg\varphi=0$,结合上一条得到要么$m=0$要么$\deg\varphi=0$,也即$\varphi$是常值映射.
	\end{proof}
	\item 设$E$是椭圆曲线,那么自同态环$\mathrm{End}(E)$是一个特征零并且没有非平凡零因子的未必交换的环.
	\begin{proof}
		
		上一条说明了$\mathrm{End}(E)$是特征零的.假设有$E$的自同态$\varphi,\psi$满足$\varphi\circ\psi=[0]$,取次数得到$\deg\varphi\deg\psi=0$,所以$\deg\varphi=0$或者$\deg\psi=0$,于是$\varphi=[0]$或者$\psi=[0]$.
	\end{proof}
	\item 设$E$是椭圆曲线,设$m\ge1$是整数,$E$的$m$-挠子群定义为$E[m]=\{P\in E\mid [m]P=O\}$;$E$的挠子群定义为$E_{\mathrm{tors}}=\cup_{m=1}^{\infty}E[m]$.如果$E$是定义在域$K$上的,我们就记域$K$上的$m$-挠子群和挠子群分别为$E[m](K)=E[m]\cap E(K)$和$E_{\mathrm{tors}}(K)=E_{\mathrm{tors}}\cap E(K)$.
	\item 设$\mathrm{Char}(K)=0$.在很多情况下映射$[\bullet]:\mathbb{Z}\to\mathrm{End}(E)$是同构.倘若$\mathrm{End}(E)$严格比$\mathbb{Z}$多,我们就称$E$具有复乘(complex multiplication,简记作CM).具有复乘的椭圆曲线通常有很多特殊性质.另一方面我们会看到有限域上的椭圆曲线总有复乘.
	\item 一些例子:
	\begin{enumerate}
		\item 设$\mathrm{char}(K)\not=2$,设$i\in\overline{K}$是四次本原单位根,也即$i^2=-1$.考虑域$K$上的椭圆曲线$E:y^2=x^3-x$,那么它有复乘,因为有自同态$[i]:(x,y)\mapsto(-x,iy)$.并且$[i]$定义在$K$上当且仅当$i\in K$.于是即便$E$定义在$K$上,$\mathrm{End}_K(E)$可能严格小于$\mathrm{End}(E)$.另外容易计算有$[i]\circ[i]=[-1]$,所以存在环同态$\mathbb{Z}[i]\to\mathrm{End}(E)$为$m+ni\mapsto[m]+[n]\circ[i]$.如果$\mathrm{char}(K)=0$,则这个同态是同构【】.于是此时$E$的自同构群为$\mathbb{Z}[i]^*$是四阶循环群.
		\item 设$\mathrm{char}(K)\not=2$,设$a,b\in K$满足$b\not=0$和$r=a^2-4b\not=0$,考虑两条椭圆曲线:
		$$E_1:y^2=x^3+ax^2+bx$$
		$$E_2:Y^2=X^3-2aX^2+rX$$
		
		存在两个二次的同源:
		$$\varphi:E_1\to E_2$$
		$$(x,y)\mapsto\left(\frac{y^2}{x^2},\frac{y(b-x^2)}{x^2}\right)$$
		$$\widehat{\varphi}:E_2\to E_1$$
		$$(X,Y)\mapsto\left(\frac{Y^2}{4X^2},\frac{Y(r-X^2)}{8X^2}\right)$$
		
		那么有$\widehat{\varphi}\circ\varphi=[2]$和$\varphi\circ\widehat{\varphi}=[2]$.这是对偶同源的例子,后文会详细介绍.
		\item 设$K$是特征$p>0$的域,设$q=p^r$,设$E$是域$K$上被一个Weierstrass方程定义的椭圆曲线.我们用$E^{(q)}$表示把$E$的Werierstrass方程的系数都替换为$q$次幂得到的$K$上的曲线,那么有$\Delta(E^{(q)})=\Delta(E)^q$和$j(E^{(q)})=j(E)^q$.于是特别的$E^{(q)}$仍然是$K$上的椭圆曲线.并且我们有同源$\varphi_q:E\to E^{(q)}$为$(x,y)\mapsto(x^q,y^q)$,它称为Frobenius态射.
		
		\qquad
		
		下面设$K=\mathbb{F}_q$,那么$K$上元素的$q$次幂是本身,于是$E^{(q)}=E$,此时Frobenius态射$\varphi_q$是$E$上的自同态,称为$E$的Frobenius自同态.被$\varphi_q$固定的点集恰好是$E(\mathbb{F}_q)$,我们后面会证明这是有限群.
		\item 设$E$是定义在$K$上的椭圆曲线,对$Q\in E$,我们定义过平移态射$\tau_Q:E\to E$,$P\mapsto P+Q$,这是一个同构因为它有逆态射$\tau_{-Q}$.但是它一般不是同源,除非$Q=O$.考虑椭圆曲线之间的态射$F:E_1\to E_2$,那么$\varphi:\tau_{-F(O)}\circ F$总是一个同源.于是椭圆曲线之间的态射总可以分解为同源复合上一个平移态射:
		$$F=\tau_{F(O)}\circ\varphi$$
	\end{enumerate}
	\item 同源是阿贝尔簇之间的态射,换句话讲它虽然定义为簇之间的态射,但是它自动是一个阿贝尔群同态.
	\begin{proof}
		
		设$\varphi:E_1\to E_2$是椭圆曲线之间的同源,我们要证明的是对任意$P,Q\in E_1$,有$\varphi(P+Q)=\varphi(P)+\varphi(Q)$.倘若$\varphi$是常值映射,那么没什么需要证明的.下面设$\varphi$不是常值映射,则它诱导了映射$\varphi_*:\mathrm{Pic}^0(E_1)\to\mathrm{Pic}^0(E_2)$把$\sum n_i(P_i)$所在的除子类映射为$\sum n_i(\varphi(P_i))$所在的除子类.由于$\varphi$把基点映射为基点,所以$\varphi_*$是群同态,再考虑如下交换图表,我们解释过其中$\kappa_i$是群同构,于是$\varphi$是群同态.
		$$\xymatrix{E_1\ar[rr]^{\varphi}\ar[d]^{\cong}_{\kappa_1}&&E_2\ar[d]^{\kappa_2}_{\cong}\\\mathrm{Pic}^0(E_1)\ar[rr]_{\varphi_*}&&\mathrm{Pic}^0(E_2)}$$
	\end{proof}
	\item 推论.设$\varphi:E_1\to E_2$是椭圆曲线之间的非常值同源,那么$\ker\varphi=\varphi^{-1}(O)$是一个有限群.
	\begin{proof}
		
		按照同源都是阿贝尔群同态,所以$\ker\varphi$是$E_1$的子群.另外一般的光滑曲线之间的非常值态射$\varphi:C_1\to C_2$,它一定是满射,并且对任意$Q\in C_2$,我们有$\sum_{P\in\varphi^{-1}(Q)}e_{\varphi}(P)=\deg\varphi$,其中$e_{\varphi}(P)\ge1$.由于态射的次数总是有限的,于是$\varphi^{-1}(Q)$总是有限集,特别的这里$\ker\varphi$是有限群.
	\end{proof}
\end{enumerate}

椭圆函数域和Galois理论.
\begin{enumerate}
	\item 回顾$e_{\varphi}(P)$.设$\varphi:C_1\to C_2$是光滑曲线之间的非常值态射,设$P\in C_1$,那么$\varphi$在点$P$的分歧指数定义为$\mathrm{ord}_P(\varphi^*t_{\varphi(P)})$,其中$t_{\varphi(P)}\in K(C_2)$是$\varphi(P)$处的uniformizer.于是总有$e_{\varphi(P)}\ge1$,倘若$e_{\varphi(P)}=1$,就称$\varphi$在点$P$处是非分歧的.下面设$\varphi:C_1\to C_2$是光滑曲线之间的非常值态射,则$\varphi$必然是满射.
	\begin{enumerate}
		\item 对任意$Q\in C_2$,我们有$\sum_{P\in\varphi^{-1}(Q)}E_{\varphi}(P)=\deg\varphi$.
		\item 对除了一个有限点集以外的所有$Q\in C_2$都有$|\varphi^{-1}(Q)|=\deg_s(\varphi)$.
		\item 设$\psi:C_2\to C_3$是另一个光滑曲线之间的非常值态射,那么对任意$P\in C_1$有$e_{\psi\circ\varphi}(P)=e_{\varphi}(P)e_{\psi}(\varphi(P))$.
	\end{enumerate}
	\item 设$\varphi:E_1\to E_2$是非常值的同源,那么对任意$Q\in E_2$,我们有$|\varphi^{-1}(Q)|=\deg_s\varphi$.对任意$P\in E_1$,有$e_{\varphi}(P)=\deg_i\varphi$.
	\begin{proof}
		
		我们知道对除了一个有限点集以外的所有$Q\in E_2$都有$|\varphi^{-1}(Q)|=\deg_s\varphi$.下面设$Q,Q'\in E_2$,按照$\varphi$是满射,存在$R\in E_1$使得$\varphi(R)=Q'-Q$.于是按照同源是同态,就有双射$\varphi^{-1}(Q)\to\varphi^{-1}(Q')$为$P\mapsto P+R$.
		
		\qquad
		
		下面设$P,P'\in E_1$满足$\varphi(P)=\varphi(P')=Q$.记$R=P'-P$,则$\varphi(R)=O$,于是$\varphi\circ\tau_R=\varphi$.于是有$e_{\varphi}(P)=e_{\varphi\circ\tau_R}(P)=e_{\varphi}(\tau_R(P))e_{\tau_R}(P)$.又因为$\tau_R$是同构,则$e_{\tau_R}(P)=1$,于是$e_{\varphi}(P)=e_{\varphi}(P')$.换句话讲$\varphi^{-1}(Q)$中点具有相同的分歧指数.则下式说明$e_{\varphi}(P)=\deg_i\varphi$:
		\begin{align*}
			(\deg_s\varphi)(\deg_t\varphi)&=\deg\varphi\\&=\sum_{P\in\varphi^{-1}(Q)}e_{\varphi}(P)\\&=|\varphi^{-1}(Q)|e_{\varphi}(P)\\&=(\deg_s\varphi)e_{\varphi}(P)
		\end{align*}
	\end{proof}
	\item 设$\varphi:E_1\to E_2$是非常值的同源,任取$T\in\ker\varphi$,则$\tau_T$是$E_1$上的自同构,它诱导了$\overline{K}(E_1)$上的自同构$\tau_T^*:f\mapsto f\circ\tau_T$.我们断言$\tau_T^*$实际上落在$\mathrm{Aut}(\overline{K}(E_1)/\overline{K}(E_2))$中,并且如下映射实际上是一个群同构:
	$$\ker\varphi\to\mathrm{Aut}(\overline{K}(E_1)/\overline{K}(E_2))$$
	$$T\mapsto\tau_T^*$$
	\begin{proof}
		
		首先$\tau_T^*$固定$\overline{K}(E_2)$是因为,任取$f\in\overline{K}(E_2)$,则从$T\i\ker\varphi$,导致$\varphi\circ\tau_T=\varphi$,于是$\tau_T^*(\varphi^*f)=\varphi^*f$(我们把$\overline{K}(E_2)$典范的视为$\overline{K}(E_1)$的子群,就是把$f\in\overline{K}(E_2)$视为$\varphi^*f\in\overline{K}(E_1)$).这说明$\tau_T^*$固定了$\overline{K}(E_2)$(严格说是固定了$\varphi^*\overline{K}(E_2)$).于是我们的确定义了一个同态$\ker\varphi\to\mathrm{Aut}(\overline{K}(E_1)/\overline{K}(E_2))$.最后说明它是同构.我们前面证明了$|\ker\varphi|=\deg_s\varphi$,另一方面按照Galois基本定理得到$|\mathrm{Aut}(\overline{K}(E_1)/\overline{K}(E_2))|\le\deg_s\varphi$.所以只需证明$T\mapsto\tau_T^*$是单射.假设有$\tau_T^*$是$E_1$上的常值映射,那么$E_1$上的每个有理函数$f$都满足$f(T)=f(O)$,比方说把$f$选取为坐标函数,就导致$T=O$,这说明$T\mapsto\tau_T^*$是单射.
	\end{proof}
	\item 设$\varphi:E_1\to E_2$是非常值同源,如果$\varphi$是可分的(此即它诱导的有限域扩张$\varphi^*:\overline{K}(E_2)\to\overline{K}(E_1)$是可分扩张),那么$\varphi$是非分歧的(此即对任意$P\in E_1$有$e_{\varphi}(P)=1$),并且有$|\ker\varphi|=\deg\varphi$,并且$\varphi^*\overline{K}(E_2)\subseteq\overline{K}(E_1)$是有限Galois扩张.
	\begin{proof}
		
		如果$\varphi$是可分的,则$\deg\varphi=\deg_s\varphi=|\varphi^{-1}(Q)|,\forall Q\in E_2$,从$|\varphi^{-1}(Q)|e_{\varphi}(P)=\deg\varphi$,其中$P\in E_1$得到$e_{\varphi}(P)=1,\forall P\in E_1$,于是$\varphi$是非分歧的.再取$Q=O$,得到$|\ker\varphi|=\deg\varphi$,则前面命题说明$|\mathrm{Aut}(\overline{K}(E_1)/\overline{K}(E_2))|=[\overline{K}(E_1):\varphi^*\overline{K}(E_2)]$,这说明$\varphi^*\overline{K}(E_2)\subseteq\overline{K}(E_1)$是Galois扩张.
	\end{proof}
	\item 推论.设$\varphi:E_1\to E_2$和$\psi:E_1\to E_3$是椭圆曲线之间的两个非常值同源,设$\varphi$是可分的.如果还满足$\ker\varphi\subseteq\ker\psi$,则存在唯一的同源$\lambda:E_2\to E_3$满足$\psi=\lambda\circ\varphi$.
	\begin{proof}
		
		因为$\varphi$是可分的,前面命题说明了$\overline{K}(E_1)$是$\varphi^*\overline{K}(E_2)$的有限Galois扩张.并且$\ker\varphi\subseteq\ker\psi$说明有如下域扩张:
		$$\psi^*\overline{K}(E_3)\subseteq\varphi^*\overline{K}(E_2)\subseteq\overline{K}(E_1)$$
		
		这里函数域的域扩张$\psi^*\overline{K}(E_3)\subseteq\varphi^*\overline{K}(E_2)$唯一的对应了一个态射$\lambda:E_2\to E_3$,于是它满足$\lambda\circ\varphi=\psi$,带入基点说明$\lambda$是同源.
	\end{proof}
	\item 设$E$是椭圆曲线,设$\Phi$是$E$的一个有限子群,那么在同源同构意义下存在唯一的椭圆曲线$E'$,和唯一的可分同源$\varphi:E\to E'$,满足$\ker\varphi=\Phi$.
	\begin{proof}
		
		每个$T\in\Phi$对应于$\overline{K}(E)$的一个自同构$\tau_T^*$.设$\overline{K}(E)^{\Phi}$是被$\Phi$中所有元固定的$\overline{K}(E)$的子域.那么Galois理论告诉我们$\overline{K}(E)$是$\overline{K}(E)^{\Phi}$的有限Galois扩张,并且Galois群同构于$\Phi$.又因为$\overline{K}\subseteq\overline{K}(E)$的超越维数是1,导致$\overline{K}\subseteq\overline{K}(E)^{\Phi}$的超越维数也是1,于是存在同构意义下唯一的$\overline{K}$上的光滑曲线$C$,以及一个有限态射$\varphi:E\to C$满足$\varphi^*\overline{K}(C)=\overline{K}(E)^{\varphi}$.我们断言$\varphi$是非分歧态射,一旦这成立,按照Hurwitz不等式,就得到$C$的亏格也是1,于是$C$是椭圆曲线,把$\varphi(O)$取为$C$的基点,这就得证.
		
		\qquad
		
		所以问题归结为证明$\varphi$是非分歧的.设$P\in E$和$T\in\Phi$,对任意$f\in\overline{K}(C)$,我们有:
		$$f(\varphi(P+T))=((\tau_T^*\circ\varphi^*)f)(P)=(\varphi^*f)(P)=f(\varphi(P))$$
		
		于是把$f$选取为坐标函数就导致$\varphi(P+T)=\varphi(P),\forall P\in E,T\in\Phi$.设$Q\in C$,选取$P\in E$满足$\varphi(P)=Q$,那么有$P+\Phi\subseteq\varphi^{-1}(Q)$.另一方面我们知道$|\varphi^{-1}(Q)|\le\deg\varphi=|\Phi|$,这个取等当且仅当$\varphi$是非分歧的.在$T$跑遍$\Phi$中的元时有$P+T$两两不同,所以$|\varphi^{-1}(Q)|\ge|\Phi|$,于是这实际上是等式,于是$\varphi$是非分歧的.
	\end{proof}
	\item 上一条中的$E'$通常记作$E/\Phi$.一般的,如果$X$是簇,如果$G$是$X$的某些自同构构成的有限群,则商$X/G$总是一个簇.另外如果$E$定义在域$K$上,如果$\Phi$是$G_{\overline{K}/K}$不变的,那么这里$E/\Phi$和典范态射$\varphi:E\to E/\Phi$都是定义在$K$上的.
\end{enumerate}
\subsubsection{不变微分}

设定义在域$K$上的椭圆曲线$E:y^2+a_1xy+a_3y=x^3+a_2x^2+a_4x+a_6$,我们定义过它的不变微分是$\omega=\frac{\mathrm{d}x}{2y+a_1x+a_3}$.并且我们解释过它在$E$上既没有零点也没有极点.
\begin{enumerate}
	\item 首先我们证明不变微分是在平移变换下不变的,这是它名字的由来.具体地讲,设$E$是被某个Weierstrass方程定义的椭圆曲线,设$\omega$是不变微分,对任意$Q\in E$,记平移变换$\tau_Q:E\to E$,那么有$\tau_Q^*\omega=\omega$.
	\begin{proof}
		
		这件事可以按照加法公式比较繁琐的推出,也可以按照如下方式证明:记$\Omega_E$是$E$上有理函数的微分形式构成的空间,那么它是$\overline{K}(E)$上一维空间.于是存在$a_Q\in\overline{K}(E)^*$,使得$\tau_Q^*\omega=a_Q\omega$(这里$a_Q\not=0$是因为$\tau_Q^*$是同构).我们有:
		\begin{align*}
			\mathrm{div}(a_Q)&=\mathrm{div}(\tau_Q^*\omega)-\mathrm{div}(\omega)\\&=\tau_Q^*\mathrm{div}(\omega)-\mathrm{\div}(\omega)=0
		\end{align*}
		
		最后一个等式是因为我们解释过$\mathrm{div}(\omega)=0$.于是$a_Q$是$E$上的没有极点和零点的正则函数,一般的光滑曲线上没有零点和极点的正则函数是常值的,于是我们有$a_Q$是一个不为零的常值函数(或者我们可以考虑有理映射$f:E\to\mathbb{P}^1_{\overline{K}}$为$Q\mapsto[a_Q,1]$,这不是一个满射,因为它不会取到$[0,1]$和$[1,0]$,所以它必须是常值映射).于是就有$a_Q\equiv a_O=1$,这就说明$\tau_Q^*\omega=\omega$.
	\end{proof}
	\item 设$\varphi,\psi:E\to E'$是两条椭圆曲线之间的同源,设$\omega$是$E$上的不变微分,那么有:
	$$(\varphi+\psi)^*\omega=\varphi^*\omega+\psi^*\omega$$
	
	明显的,第一个加号是$\mathrm{Hom}(E',E)$上的加法,这依赖于椭圆曲线$E$上的加法;第二个加号是$\Omega_{E'}$上的加法.
	\begin{proof}
		
		首先明显$\varphi=[0]$或者$\psi=[0]$时等式成立.下面设$\varphi+\psi=[0]$,那么有$\psi^*=(-\varphi)^*=\varphi^*\circ[-1]^*$.于是此时归结为证明$[-1]^*\omega=-\omega$.按照加法公式,我们有$[-1](x,y)=(x,-y-a_1x-a_3)$,进而有:
		\begin{align*}
			[-1]^*\left(\frac{\mathrm{d}x}{2y+a_1x+a_3}\right)&=\frac{\mathrm{d}x}{2(-y-a_1x-a_3)+a_1x+a_3}\\&=-\frac{\mathrm{d}x}{2y+a_1x+a_3}
		\end{align*}
		
		综上我们解决了$\varphi,\psi,\varphi+\psi$中某个为$[0]$的情况.下面设它们三个都不为$[0]$.设$(x_1,y_1)$和$(x_2,y_2)$是$E$的两个Weierstrass坐标,使得它们满足的代数方程只有同一个Weierstrass方程.或者更精确的讲,要求$([x_1,y_1,1],[x_2,y_2,1])$定义了$E\times E\subseteq\mathbb{P}^2_{\overline{K}}\times\mathbb{P}^2_{\overline{K}}$.我们记$(x_1,y_1)+(x_2,y_2)=(x_3,y_3)$,那么$x_3,y_3$是关于$x_1,x_2,y_1,y_2$的有理分式.再对任意$(x,y)$,我们记$\omega(x,y)=\frac{\mathrm{d}x}{2y+a_1x+a_3}$为相应的不变微分.那么$\omega(x_3,y_3)$做展开就是$\overline{K}(x_1,y_1,x_2,y_2)$系数的关于$\mathrm{d}x_1$,$\mathrm{d}x_2$,$\mathrm{d}y_1$,$\mathrm{d}y_2$的线性组合.另一方面我们知道$(x_1,y_1)$和$(x_2,y_2)$都满足同一个Weierstrass方程,于是微分得到$(2y_i+a_1x_i+a_3)\mathrm{d}y_i=(3x_i^2+2a_2x_i+a_4-a_1y_i)\mathrm{d}x_i,\forall i=1,2$.于是$\omega(x_3,y_3)$就可以表示为$\overline{K}(x_1,x_2,y_1,y_2)$系数的关于$\mathrm{d}x_1$和$\mathrm{d}x_2$的线性组合.于是存在有理函数$f,g$满足:
		$$\omega(x_3,y_3)=f(x_1,y_1,x_2,y_2)\omega(x_1,y_1)+g(x_1,y_1,x_2,y_1)\omega(x_2,y_2)$$
		
		我们断言这里$f=g=1$.这当然可以用繁琐的计算得到,不过这里我们可以避开计算:取$(x_2,y_2)$为一个固定点$Q\in E$(把$x_2,y_2$理解为常值函数),那么此时$\mathrm{d}x_2=0$,于是$\omega(x_2,y_2)=0$.按照不变微分是平移不变的,就有$\omega(x_3,y_3)=\tau_Q^*\omega(x_1,y_1)=\omega(x_1,y_1)$.带入到我们上面的等式中,得到$f(x_1,y_1,x(Q),y(Q))=1$.这说明$f$不依赖于$x_1,y_1$,于是$f\in\overline{K}(x_2,y_2)$.但是我们这里$Q$也是任意取的,所以$f\equiv1$.同理得到$g\equiv1$.
		
		\qquad
		
		最后取$E'$上的Weierstrass坐标为$(x',y')$,记$\varphi(x',y')=(x_1,y_1)$,记$\psi(x',y')=(x_2,y_2)$,再记$(\varphi+\psi)(x',y')=(x_3,y_3)$,那么按照我们前面证明的$f=g=1$,就得到$\omega(x_3,y_3)=\omega(x_1,y_1)+\omega(x_2,y_2)$.于是有$(\omega\circ(\varphi+\psi))(x',y')=(\omega\circ\varphi)(x',y')+(\omega\circ\psi)(x',y')$,此即$(\varphi+\psi)^*\omega=\varphi^*\omega+\psi^*\omega$.
	\end{proof}
	\item 推论.设$\omega$是椭圆曲线$E$上的不变微分,设$m\in\mathbb{Z}$,那么有$[m]^*\omega=m\omega$.
	\item 推论.设$E$是域$K$上的椭圆曲线,设$m\in\mathbb{Z}$满足在$K$上有$m\not=0$,那么$[m]:E\to E$是非常值态射.这件事我们之前证明过了,这里我们借助上面定理给出另一种证明.
	\begin{proof}
		
		设$\omega$是$E$上的不变微分,按照$K$中有$m\not=0$,于是$[m]^*\omega=m\omega\not=0$.于是$[m]$不是常值映射.
	\end{proof}
	\item 回顾可分的态射:曲线之间的态射称为可分的,如果它诱导的函数域扩张是有限可分扩张(曲线之间态射诱导的函数域扩张一定是有限扩张,所以它要么是有限可分扩张要么是有限不可分扩张).设$C$是曲线,它的亚纯微分形式构成了$\overline{K}(C)$上的一维线性空间,记作$\Omega_C$,并且$x\in\overline{K}(C)$使得$\mathrm{d}x$构成它的一组基当且仅当$\overline{K}(C)/\overline{K}(x)$是有限可分扩张.如果$\varphi:C_1\to C_2$是非常值态射,那么$\varphi$是可分的当且仅当$\varphi^*:\Omega_{C_2}\to\Omega_{C_1}$是单射,当且仅当$\varphi^*$是非零的.
	\item 推论.设$E$是定义有限域$\mathbb{F}_q$上的椭圆曲线,记这个有限域的特征为素数$p$,我们之前定义过$E$上的Frobenius态射$\varphi$为$(x,y)\mapsto(x^q,y^q)$.设$m,n\in\mathbb{Z}$,那么我们断言$m+n\varphi:E\to E$(严格写是$[m]+[n]\circ\varphi$)是可分态射当且仅当$p\not\mid m$.特别的,我们有$1-\varphi$总是$E$上的可分态射.
	\begin{proof}
		
		记$\omega$为$E$的不变微分,我们知道一个态射$\psi:E\to E$是纯不可分的当且仅当$\psi^*\omega=0$.记$\psi=m+n\varphi$,按照前面定理有$(m+n\varphi)^*\omega=m\omega+n\varphi^*\omega$.这里Frobenius态射$\varphi$是不可分的,于是$\varphi^*\omega=0$,或者这件事也可以直接计算:
		$$\varphi^*\left(\frac{\mathrm{d}x}{2y+a_1x+a_3}\right)=\frac{\mathrm{d}(x^q)}{2y^q+a_1x^q+a_3}=\frac{qx^{q-1}\mathrm{d}x}{2y^q+a_1x^q+a_3}=0$$
		
		于是我们有$(m+n\varphi)^*\omega=m\omega$,这为零当且仅当$p\mid m$,于是得证.
	\end{proof}
	\item 推论.设$E$是定义在$K$上的椭圆曲线,设$\omega$是$E$的不变微分,对任意$\varphi\in\mathrm{End}(E)$,由于$\Omega_E$是一维$\overline{K}(E)$线性空间,我们有$\varphi^*\omega=a_{\varphi}\omega$,其中$a_{\varphi}\in\overline{K}(E)$.
	\begin{enumerate}
		\item 我们断言这里$a_{\varphi}\in\overline{K}$,并且由此定义的映射$\varphi\mapsto a_{\varphi}\in\overline{K}$是$\mathrm{End}(E)\to\overline{K}$的环同态.
		\item 这个环同态的核由$E$的全体不可分态射构成.
		\item 如果$\mathrm{char}(K)=0$,则$\mathrm{End}(E)$是交换环.
	\end{enumerate}
	\begin{proof}
		
		我们有$\mathrm{div}(a_{\varphi})=\mathrm{div}(\varphi^*\omega)-\mathrm{div}(\omega)=\varphi^*\mathrm{div}(\omega)-\mathrm{div}(\omega)=0$.一般的光滑曲线上没有极点和零点的亚纯函数一定是常值函数,于是$a_{\varphi}\in\overline{K}$.接下来验证环同态是容易的:
		$$a_{\varphi+\psi}\omega=(\varphi+\psi)^*\omega=\varphi^*\omega+\psi^*\omega=a_{\varphi}\omega+a_{\psi}\omega=(a_{\varphi}+a_{\psi})\omega$$
		$$a_{\varphi\circ\psi}\omega=(\varphi\circ\psi)^*\omega=\psi^*(\varphi^*\omega)=\psi^*(a_{\varphi}\omega)=a_{\varphi}\psi^*\omega=a_{\varphi}a_{\psi}\omega$$
		
		这证明了(a).另外$a_{\varphi}=0$当且仅当$\varphi^*\omega=0$,我们解释过这等价于$\varphi$是不可分的,这证明了(b).最后如果$\mathrm{char}(K)=0$,则$E$上每个自同态都是可分的,于是(b)说明$\mathrm{End}(E)$可以嵌入到交换环$\overline{K}$中,于是$\mathrm{End}(E)$是交换环,这证明了(c).
	\end{proof}
\end{enumerate}
\subsubsection{对偶同源}
\begin{enumerate}
	\item 设$\varphi:E_1\to E_2$是椭圆曲线之间的非常值同源,次数记作$m$,那么存在唯一的同源$\widehat{\varphi}:E_2\to E_1$,使得$\widehat{\varphi}\circ\varphi=[m]$.这个唯一的同源$\widehat{\varphi}$称为$\varphi$的对偶同源.我们可以补充定义$\widehat{[0]}=[0]$.
	\begin{proof}
		
		我们先来证明唯一性.假设有两个同源$\widehat{\varphi},\widehat{\varphi}':E_2\to E_1$满足结论,我们解释过同源都是阿贝尔群同态,于是有$(\widehat{\varphi}-\widehat{\varphi}')\circ\varphi=[m]-[m]=[0]$.这里$\varphi$不是常值映射,所以它是满射(光滑曲线之间的态射要么是常值映射要么是满射),但是$\widehat{\varphi}-\widehat{\varphi}'$不是满射,这迫使它是常值映射,而$\widehat{\varphi}$和$\widehat{\varphi}'$又都是同源,于是$\widehat{\varphi}=\widehat{\varphi}'$.
		
		\qquad
		
		设$\psi:E_2\to E_3$是另一个非常值同源,次数记作$n$,倘若我们证明了$\widehat{\psi}$和$\widehat{\varphi}$都存在,那么$(\widehat{\varphi}\circ\widehat{\psi})\circ(\psi\circ\varphi)=\widehat{\varphi}\circ[n]\circ\varphi=[n]\circ\widehat{\varphi}\circ\varphi=[nm]$.并且$\widehat{\varphi}\circ\widehat{\psi}$也是非常值同源,于是$\widehat{\psi\circ\varphi}=\widehat{\varphi}\circ\widehat{\psi}$存在.由于光滑曲线之间态射的次数总是有限的,对应的函数域扩张可以分解为可分扩张复合上纯不可分扩张,而有限纯不可分扩张可以分解为若干纯不可分单扩张的复合,而这对应于Frobenius态射,如果设$p$是素数,记$q=p^r$,那么Frobenius态射$E\to E^{(q)}$,$(x,y)\mapsto(x^q,y^q)$的次数就是$q$.并且这个Frobenius态射又可以分解为$r$个$(x,y)\mapsto(x^p,y^p)$的复合.综上我们只需证明$\varphi$是可分的态射和它是素数$p$次的Frobenius态射的情况.
		
		\qquad
		
		先设$\varphi$是可分态射.我们解释过此时有$|\ker\varphi|=\deg\varphi=m$.进而有$\ker\varphi\subseteq\ker[m]$,我们还解释过可分条件下这导致存在一个同源$\widehat{\varphi}:E_2\to E_1$满足$\widehat{\varphi}\circ\varphi=[m]$,这得到存在性.
		
		\qquad
		
		再设$\varphi$是素数$p$次的Frobenius态射.此时$\mathrm{char}(K)=p$,我们有$[p]^*\omega=p\omega=0$,于是$[p]$不是$E$上的可分的态射.于是我们有同源分解$[p]=\psi\circ\varphi^e,e\ge1$,其中$\psi$是可分的态射.于是我们取$\widehat{\varphi}=\psi\circ\varphi^{e-1}$就满足$\widehat{\varphi}\circ\varphi=[p]$,得到存在性.
	\end{proof}
	\item 对偶同源$\widehat{\varphi}:E_2\to E_1$的加法就是如下复合:
	$$\xymatrix{E_2\ar[r]^{\kappa_2}_{\cong}&\mathrm{Div}^0(E_2)\ar[r]^{\varphi^*}&\mathrm{Div}^0(E_1)\ar[r]^{\mathrm{sum}}&E_1\\Q\ar@{|->}[r]&(Q)-(O)\ar@{|->}[r]&\sum n_P(P)\ar@{|->}[r]&\sum[n_P]P}$$
	\begin{proof}
		\begin{align*}
			\mathrm{sum}\circ\varphi^*((Q)-(O))&=\sum_{P\in\varphi^{-1}(Q)}[e_{\varphi}(P)]P-\sum_{T\in\varphi^{-1}(O)}[e_{\varphi}(T)]T\\&=[\deg_i\varphi]\left(\sum_{P\in\varphi^{-1}(Q)}P-\sum_{T\in\varphi^{-1}(O)}T\right)\\&=[\deg_i\varphi][\deg_s\varphi]P\\&=(\deg\varphi)P
		\end{align*}
		
		这里倒数第二个等号是因为首先有$|\varphi^{-1}(Q)|=|\varphi^{-1}(O)|=\deg_s\varphi$,接下来我们可以给$\varphi^{-1}(O)=\{P_i\}$和$\varphi^{-1}(Q)=\{T_i\}$一个对应,使得相对应的$P_i-T_i$总是一个固定的$\varphi^{-1}(Q)$中的元$P$,并且这里$P$可以预先约定好.另一方面我们有$\widehat{\varphi}(Q)=\widehat{\varphi}\circ\varphi(P)=(\deg\varphi)P$.得证.
	\end{proof}
	\item 对偶同源的一些基本性质.设$\varphi:E_1\to E_2$是椭圆曲线之间的同源.
	\begin{enumerate}
		\item 我们有$\deg\varphi=\deg\widehat{\varphi}$,并且$\widehat{\widehat{\varphi}}=\varphi$.
		\item 对任意$m\in\mathbb{Z}$,我们有$\widehat{[m]}=[m]$,和$\deg[m]=m^2$.
		\item 设$\lambda:E_2\to E_3$是另一个同源,那么有:
		$$\widehat{\lambda\circ\varphi}=\widehat{\varphi}\circ\widehat{\lambda}$$
		\item 设$\psi:E_1\to E_2$是另一个同源,那么有:
		$$\widehat{\varphi+\psi}=\widehat{\varphi}+\widehat{\psi}$$
	\end{enumerate}
	\begin{proof}
		
		(c)是平凡的,我们先承认(d)成立【】.那么按照$\widehat{[1]}=[1]$,归纳就有$\widehat{[m+1]}=\widehat{[m]}+\widehat{[1]}=[m+1]$.另一方面我们有$[m^2]=\widehat{[m]}\circ[m]=[\deg[m]]$,于是$\deg[m]=m^2$,这解决了(b).最后设$m=\deg\varphi$,那么有$m^2=\deg[m]=\deg(\widehat{\varphi}\circ\varphi)=(\deg\varphi)(\deg\widehat{\varphi})=m\deg\widehat{\varphi}$,进而得到$\deg\widehat{\varphi}=m=\deg\varphi$.另外按照$\varphi\circ\widehat{\varphi}\circ\varphi=\varphi\circ[m]=[m]\circ\varphi$得到$\left(\varphi\circ\widehat{\varphi}-[m]\right)\circ\varphi=[0]$,不妨设$\varphi$是非常值的(因为常值的时候明显有$\widehat{[0]}=[0]$,此时(a)是平凡的),于是它是满射,于是$\varphi\circ\widehat{\varphi}=[m]$,这解决了(a).
	\end{proof}
	\item 定义.设$A$是阿贝尔群,一个映射$d:A\to\mathbb{R}$称为二次型(quadratic form),如果它满足如下条件:
	\begin{enumerate}
		\item $\forall\alpha\in A$有$d(\alpha)=d(-\alpha)$.
		\item 映射$A\times A\to\mathbb{R}$,$(\alpha,\beta)\mapsto d(\alpha+\beta)-d(\alpha)-d(\beta)$是$\mathbb{Z}$双线性型.
	\end{enumerate}
	
	称二次型$d$是正定的(positive definite),如果它还满足:
	\begin{enumerate}[start=3]
		\item $\forall\alpha\in A$有$d(\alpha)\ge0$.
		\item $d(\alpha)=0$当且仅当$\alpha=0$.
	\end{enumerate}
	\item 设$E_1,E_2$是椭圆曲线,那么次数映射$\deg:\mathrm{Hom}(E_1,E_2)\to\mathbb{Z}$是一个正定二次型.
	\begin{proof}
		
		都是平凡的,这里我们验证下条件(b),因为$\mathbb{Z}$可以嵌入到$\mathrm{End}(E_1)$中,所以我们可以在$\mathrm{End}(E_1)$中验证这个$\mathbb{Z}$双线性型:
		\begin{align*}
			[\deg(\varphi+\psi)-\deg\varphi-\deg\psi]&=[\deg(\varphi+\psi)]-[\deg\varphi]-[\deg\psi]\\&=\widehat{(\varphi+\psi)}\circ(\varphi+\psi)-\widehat{\varphi}\circ\varphi-\widehat{\psi}\circ\psi\\&=\widehat{\varphi}\circ\psi+\widehat{\psi}\circ\varphi
		\end{align*}
		
		最后这个式子是分别关于位置$\varphi$和$\psi$是$\mathbb{Z}$线性的.
	\end{proof}
	\item 设$E$是椭圆曲线,对$0\not=m\in\mathbb{Z}$,我们定义过$m$-挠子群$E[m]=\ker[m]=\{P\in E\mid [m]P=O\}$.
	\begin{enumerate}
		\item 设在$K$中有$m\not=0$,换句话讲要么$\mathrm{char}(K)=0$,要么$p=\mathrm{char}(K)>0$并且满足$p\not\mid m$.此时我们有:
		$$E[m]=\frac{\mathbb{Z}}{m\mathbb{Z}}\times\frac{\mathbb{Z}}{m\mathbb{Z}}$$
		\item 如果$\mathrm{char}(K)=p>0$,那么要么对任意$e\ge1$恒有$E[p^e]=\{O\}$;要么对任意$e\ge1$有$E[p^e]=\frac{\mathbb{Z}}{p^e\mathbb{Z}}$.
	\end{enumerate}
	\begin{proof}
		
		按照$[m]^*\omega=m\omega\not=0$,我们有$[m]$是有限可分态射,我们解释过此时$|E[m]|=|\ker[m]|=\deg[m]=m^2$.对任意$d\mid m$,我们有子群$E[d]\subseteq E[m]$的阶是$|E[d]|=d^2$,有限阿贝尔群结构定理告诉我们此时有$E[m]\cong\frac{\mathbb{Z}}{m\mathbb{Z}}\times\frac{\mathbb{Z}}{m\mathbb{Z}}$.这是(a).
		
		\qquad
		
		设$\varphi$是$E$上的$p$次Frobenius态射,那么$|E[p^e]|=|\ker[p^e]|=\deg_s[p^e]=(\deg_s[p])^e=(\deg_s(\widehat{\varphi}\circ\varphi))^e=(\deg_s\widehat{\varphi})^e$,这里最后一个等式是因为$\varphi$是纯不可分的态射.另外我们有$\deg\widehat{\varphi}=\deg\varphi=p$是一个素数,于是要么$\widehat{\varphi}$是纯不可分的,也即$\deg_s\widehat{\varphi}=1$,那么$|E[p^e]|=1,\forall e\ge1$,也即$E[p^e]=\{O\}$;要么$\widehat{\varphi}$是可分扩张,于是$\deg_s\widehat{\varphi}=p$,那么$|E[p^e]|=p^e,\forall e\ge1$,按照有限阿贝尔群结构定理得到$E[p^e]=\frac{\mathbb{Z}}{p^e\mathbb{Z}},\forall e\ge1$(因为比方说如果$E[p^e]$是至少2个非平凡循环群的直和,那么$E[p^e]$就要被$[p^{e-1}]$零化,所以理应有$E[p^e]\subseteq E[p^{e-1}]$,但是阶数对不上).
	\end{proof}
\end{enumerate}
\subsubsection{Tate模}

设$E$是椭圆曲线,设$\ell\in\mathbb{Z}$是一个素数,定义$E$的$\ell$-adic Tate模定义为逆向极限$T_{ell}(E)=\lim\limits_{\leftarrow}E[\ell^n]$,其中逆像系统的典范同态是$[\ell]:E[\ell^{n+1}]\to E[\ell^n]$.
\begin{enumerate}
	\item 我们之前解释过如果$\mathrm{char}(K)=0$或者$\mathrm{char}(K)\not=0,\ell$,那么有$E[\ell^n]=\frac{\mathbb{Z}}{\ell^n\mathbb{Z}}$;如果$\mathrm{char}(K)=\ell$,那么要么恒有$E[\ell^n]=\{O\},\forall n\ge1$,要么有$E[\ell^n]=\frac{\mathbb{Z}}{\ell^n\mathbb{Z}},\forall n\ge1$.于是$E[\ell^n]$总是一个$\frac{\mathbb{Z}}{\ell^n\mathbb{Z}}$模,于是$T_{\ell}(E)$总是一个$\mathbb{Z}_{\ell}$模(下表).又因为这里典范同态$[\ell]:E[\ell^{n+1}]\to E[\ell^n]$总是满射,于是$T_{\ell}(E)$上赋予的逆向极限拓扑等同于它作为$\mathbb{Z}_{\ell}$模的$\ell$-adic拓扑.
	$$T_{\ell}(E)\cong\left\{\begin{array}{cc}\mathbb{Z}_{\ell}\times\mathbb{Z}_{\ell}&\ell\not=\mathrm{char}(K)\\\{0\}\text{或者}\mathbb{Z}_{\ell}&\ell=\mathrm{char}(K)>0\end{array}\right.$$
	\item 我们有$G=G_{\overline{K}/K}$作用在$E[m]$上,因为对任意$\sigma\in G$和$P\in E[m]$,有$[m](P^{\sigma})=([m]P)^{\sigma}=O^{\sigma}=O$.所以每个$E[\ell^n]$可以视为一个$G$模,那么$[\ell]:E[\ell^{n+1}]\to E[\ell^n]$也是一个$G$模同态,也即有如下交换图表:
	$$\xymatrix{G\times E[\ell^{n+1}]\ar[rr]\ar[d]&&E[\ell^{n+1}]\ar[d]\\G\times E[\ell^n]\ar[rr]&&E[\ell^n]}$$
	
	进而$G$可以作用在逆向极限$T_{\ell}(E)$上.并且按照$G$作用在每个离散群$E[\ell^n]$上都是连续作用,导致它诱导的$T_{\ell}(E)$上的作用也是联系的.我们把这个连续群表示称为$G_{\overline{K}/K}$的关于$E$的$\ell$-adic表示:
	$$\rho_{\ell}:G_{\overline{K}/K}\to\mathrm{Aut}(T_{\ell}(E))$$
	
	今后我们总约定$\ell\not=\mathrm{char}(K)$,这涵盖了$\mathrm{char}(K)=0$的情况,此时$T_{\ell}(E)$是$\mathbb{Z}_{\ell}$上的二维线性空间,于是在选取一组基的前提下,$\ell$-adic表示为:
	$$\rho_{\ell}:G_{\overline{K}/K}\to\mathrm{GL}_2(\mathbb{Z}_{\ell})$$
	\item 按照$\mathbb{Z}_{\ell}\subseteq\mathbb{Q}_{\ell}$.把$T_{\ell}(E)$上的$\ell$-adic表示复合上$\mathrm{Aut}(T_{\ell}(E))\to\mathrm{Aut}(T_{\ell}(E))\otimes_{\mathbb{Z}_{\ell}}\mathbb{Q}_{\ell}$.于是$G_{\overline{K}/K}$在Tate模上的$\ell$-adic表示总可以对应为一个特征为零的域(也就是$\mathbb{Q}_{\ell}$)上的二维表示.
	\item 记$\bm{\mu}_{\ell^n}\subseteq\overline{K}^*$表示$\overline{K}$的全体$\ell^n$次单位根构成的乘法群.那么我们有典范同态$\bm{\mu}_{\ell^{n+1}}\to\bm{\mu}_{\ell^n}$为$\zeta\mapsto\zeta^{\ell}$,对此取逆向极限$T_{\ell}(\bm{\mu})=\varprojlim\bm{\mu}_{\ell^n}$,这称为乘法群$\overline{K}^*$的Tate模.我们有$\bm{\mu}_{\ell^n}\cong\mathbb{Z}/\ell^n\mathbb{Z}$(域的乘法群的有限子群一定是循环群,群论里的循环准则),进而有$T_{\ell}(\bm{\mu})\cong\mathbb{Z}_{\ell}$.另外$G_{\overline{K}/K}$典范的作用在$\bm{\mu}_{\ell^n}$上,并且使得$\bm{\mu}_{\ell^{n+1}}\to\bm{\mu}_{\ell^n}$,$\zeta\mapsto\zeta^{\ell}$是一个$G_{\overline{K}/K}$模同态,进而有$G_{\overline{K}/K}$可以可以作用在$T_{\ell}(\bm{\mu})$上,这就得到了一个一维表示:
	$$G_{\overline{K}/K}\to\mathrm{Aut}(T_{\ell}(\bm{\mu}))\cong\mathbb{Z}_{\ell}^*$$
	\item 后文会介绍对于复数域上的椭圆曲线$E$,我们可以把$m$-挠子群$E[m]$典范的等同于同调群$\mathrm{H}_1(E,\mathbb{Z}/m\mathbb{Z})$,进而把Tate模$T_{\ell}(E)$典范的等同于$\mathrm{H}_1(E,\mathbb{Z}_{\ell})$.
	\item Tate模和同源.设$\varphi:E_1\to E_2$是椭圆曲线之间的同源,那么对任意正整数$n$有$\varphi$诱导了同态$E_1[\ell^n]\to E_2[\ell^n]$,进而诱导了Tate模之间的$\mathbb{Z}_{\ell}$模同态$\varphi_{\ell}:T_{\ell}(E_1)\to T_{\ell}(E_2)$.综上我们有典范的交换群同态:
	$$\mathrm{Hom}(E_1,E_2)\to\mathrm{Hom}_{\mathbb{Z}_{\ell}}(T_{\ell}(E_1),T_{\ell}(E_2))$$
	
	当$E_1=E_2=E$时这是一个环同态:
	$$\mathrm{End}(E)\to\mathrm{End}(T_{\ell}(E))$$
	
	那么这两个典范同态总是单射.这里我们证明更强的一件事:如果$\ell\not=\mathrm{char}(K)$是素数,那么如下典范$\mathbb{Z}_{\ell}$模同态总是单射.
	$$\mathrm{Hom}(E_1,E_2)\otimes\mathbb{Z}_{\ell}\to\mathrm{Hom}_{\mathbb{Z}_{\ell}}(T_{\ell}(E_1),T_{\ell}(E_2))$$
	$$\varphi=\varphi_{\ell}$$	
	\begin{proof}
		
		
	\end{proof}
	\item 推论.设$E_1,E_2$是椭圆曲线,那么$\mathrm{Hom}(E_1,E_2)$是秩至多为4的自由$\mathbb{Z}$模.
	\begin{proof}
		
		我们解释过$\mathrm{Hom}(E_1,E_2)$一定是自由阿贝尔群(PID上无挠模等价于自由模),于是我们有:
		$$\mathrm{rank}_{\mathbb{Z}}\mathrm{Hom}(E_1,E_2)=\mathrm{rank}_{\mathbb{Z}_{\ell}}\mathrm{Hom}(E_1,E_2)\otimes\mathbb{Z}_{\ell}$$
		
		按照上一条结论,我们有:
		$$\mathrm{rank}_{\mathbb{Z}_{\ell}}\mathrm{Hom}(E_1,E_2)\otimes\mathbb{Z}_{\ell}\le\mathrm{rank}_{\mathbb{Z}_{\ell}}\mathrm{Hom}(T_{\ell}(E_1),T_{\ell}(E_2))$$
		
		但是这里$T_{\ell}(E_1)$和$T_{\ell}(E_2)$都是秩为2的自由$\mathbb{Z}_{\ell}$模,于是$\mathrm{rank}_{\mathbb{Z}_{\ell}}\mathrm{Hom}(T_{\ell}(E_1),T_{\ell}(E_2))=4$.
	\end{proof}
	\item 我们解释过$\mathrm{Hom}_K(E_1,E_2)$由那些和$G_{\overline{K}/K}$中元素可交换的同源构成的阿贝尔群,再定义$\mathrm{Hom}_K(T_{\ell}(E_1),T_{\ell}(E_2))$由那些既是$\mathbb{Z}_{\ell}$模同态,也是$G_{\overline{K}/K}$模同态的映射构成.那么我们依旧有典范单同态:
	$$\mathrm{Hom}_K(E_1,E_2)\otimes\mathbb{Z}_{\ell}\to\mathrm{Hom}_K(T_{\ell}(E_1),T_{\ell}(E_2))$$
	$$\varphi=\varphi_{\ell}$$
	
	一件不太平凡的事情是在很多情况下这个同态实际上是同构,例如在$K$是有限域或者数域的情况下.
\end{enumerate}














\newpage
\section{阿贝尔簇}
\subsection{解析理论(复环面)}
\subsubsection{复环面}

复线性空间$\mathbb{C}^g$上的(完备)格$\wedge$指的是它的极大秩离散子群,也即秩为$2g$的自由阿贝尔群.那么$\wedge$自然的作用在$\mathbb{C}^g$上为$gv=g+v,\forall g\in\wedge,v\in\mathbb{C}^g$.称作用的商$X=\mathbb{C}^g/\wedge$为一个复环面(torus,复数是tori),这是一个$g$维阿贝尔复李群.
\begin{enumerate}
	\item 复环面是一个复流形,是因为复流形的商准则:$\wedge$作用在$\mathbb{C}^g$上是自由的,此即如果存在$g\in\wedge,v\in V$使得$gv=v$,那么$g=0$;作用也是严格不连续的,此即对任意紧子集$K_1,K_2\subseteq\mathbb{C}^g$,都有$\{g\in\wedge\mid gK_1\cap K_2\not=\emptyset\}$.在这两个条件下商$\mathbb{C}^g/\wedge$是复流形.
	\item 等价定义.$X$是$g$维复环面当且仅当它是$g$维连通紧复李群.
	\begin{proof}
		
		必要性是直接的.充分性:先验证$g$维连通紧复李群$X$是阿贝尔的:考虑换位子态射$\varphi:X\times X\to X$,$(x,y)\mapsto xyx^{-1}y^{-1}$(从李群知道这是态射).任取$x\in X$,按照$\varphi(x,1)=1$,以及连续性,任取$1_X$的有界的坐标开邻域$U$,那么存在$x$的开邻域$V_x$和$1_X$的开邻域$W_x$使得$\varphi(V_x,W_x)\subseteq U$.按照$X$是紧的,可取有限个点$\{x_1,\cdots,x_n\}$,使得$\{V_{x_1},\cdots,V_{x_n}\}$覆盖了整个$X$.记$W=\cap_{1\le i\le n}W_{x_i}$.那么有$\varphi(X,W)\subseteq U$.但是任取$w\in W$,那么$\varphi(-,w)$是$X\to U$的全纯映射,它的坐标函数都是全纯的,我们知道紧复流形上的全纯函数如果有界必然常值,结合$\varphi(1,w)=1$就得到$\varphi(X,W)=\{1\}$,进而开子集$W$和$X$中的全部元可交换.但是我们知道李群的中心元构成闭子群$Z(G)$,拓扑群的子群如果包含了开子群则必然是开子群,于是$Z(G)$既开又闭,连通性保证$Z(G)=G$.
		
		\qquad
		
		取$X$的李代数$\mathrm{Lie}(X)=\mathrm{T}_eX$,我们有指数映射$\exp:\mathrm{Lie}(X)\to X$,按照$X$是阿贝尔的,这个映射是李群之间的态射.它是局部同胚,于是$\mathrm{Im}\exp$包含了$1_X$的某个开子群(因为比方说设$\mathrm{Im}\exp$包含了$1_X$的对称开邻域$V$,那么$\cup_{n\ge1}V^n$是开子群,并且包含在$\mathrm{Im}\exp$里),拓扑群的开子群也是闭的,连通性就保证$\mathrm{Im}\exp=X$.这里$X$是紧的导致$\ker\exp$必然是一个格.
	\end{proof}
	\item 关于基本群.设$X=V/\wedge$是复环面,那么典范商映射$\pi:V\to X$是万有覆盖空间,其上的Deck变换就是$V$上关于$\wedge$中的点的平移变换,于是$X$的基本群就是$\wedge$本身.特别的按照$\pi_1(X)=\wedge$是交换的,就有它们典范的同构于同调群$\mathrm{H}_1(X,\mathbb{Z})$.另外按照$\pi$是局部同构,就有$V$典范的同构于李代数$\mathrm{T}_eX$.此时$\pi:V\to X$就是指数映射.
	\item 矩阵表示.取定$V$的一组基$\{e_1,\cdots,e_g\}$,取定$\wedge$(作为自由阿贝尔群)的一组基$\{\lambda_1,\cdots,\lambda_{2g}\}$,那么$\lambda_i=\sum_j\lambda_{ji}e_j$.记矩阵:
	$$M'=\left(\begin{array}{ccc}\lambda_{11}&\cdots&\lambda_{1,2g}\\\vdots&\ddots&\vdots\\\lambda_{g1}&\cdots&\lambda_{g,2g}\end{array}\right)$$
	再记$M=\left(\begin{array}{c}M'\\\overline{M'}\end{array}\right)\in\mathrm{M}(2g,\mathbb{C})$.那么具有该形式的矩阵描述一个复环面当且仅当$M$是可逆矩阵.
\end{enumerate}
\subsubsection{同态}

同态.两个复环面之间的同态是指一个全纯映射$f:X=V/\wedge\to X'=V'/\wedge'$同时也是(阿贝尔)群同态.复环面$X$上的关于元素$x_0\in X$的平移变换是指$t_{x_0}:X\to X$,$x\mapsto x+x_0$.
\begin{enumerate}
	\item 设$h:X\to X'$是复环面之间的全纯映射,那么$f=t_{-h(0)}$,$x\mapsto h(x)-h(0)$是复环面之间的态射.并且此时存在唯一的$\mathbb{C}$线性映射$F:V\to V'$,满足$F(\wedge)\subseteq\wedge'$,并且它诱导到商上就是$f$.
	\begin{proof}
		
		考虑全纯映射$f\circ\pi:V\to X'$,它提升为覆盖空间$\pi':V'\to X'$上的全纯映射,记作$F:V\to V'$.此时有$F(0)=0$,并且$F$诱导在商上就是$f$.于是对任意$v\in V$和$\lambda\in\wedge$有$F(v+\lambda)-F(v)\in\wedge'$.但是连通性和连续性迫使$v\mapsto F(v+\lambda)-F(v)$在$V$上是常值的,于是$F(v+\lambda)=F(v)+F(\lambda),\forall\lambda\in\wedge$和$v\in V$.于是$F$的偏导数是关于$\wedge$周期的,这个格的基本网孔是有界的,紧集上的全纯函数是常值的,于是$F$的偏导数是常值的,进而$F$是线性的,进而$F$诱导的$f$是交换群同态,从而$f$是复环面之间的同态.
	\end{proof}
	\item 上一条说明有阿贝尔群之间的典范单映射$f\mapsto F$,称为解析表示:
	$$\rho_a:\mathrm{Hom}_{\textbf{Tori}}(X,X')\to\mathrm{Hom}_{\mathbb{C}}(V,V')$$
	另外$F$限制在$\wedge$上是到$\wedge'$的交换群同态$F_{\wedge}$,这个同态完全决定了$F$和$f$,进而有典范单射$f\mapsto F_{\wedge}$,称为有理表示:
	$$\rho_r:\mathrm{Hom}_{\textbf{Tori}}(X,X')\to\mathrm{Hom}_{\mathbb{Z}}(\wedge,\wedge')$$
	如果$X''=V''/\wedge''$是第三个复环面,设$f\in\mathrm{Hom}_{\textbf{Tori}}(X,X')$和$f'\in\mathrm{Hom}_{\textbf{Tori}}(X',X'')$.那么有$\rho_a(ff')=\rho_a(f)\rho_a(f')$和$\rho_r(ff')=\rho_r(f)\rho_r(f')$.
	\item 特别的,$\rho_r$是单射说明自由阿贝尔群$\mathrm{Hom}_{\textbf{Tori}}(X,X')$的秩不超过$4gg'$.
	\item 设$f:X=V/\wedge\to X'=V'/\wedge'$是同态,取定$V,V',\wedge,\wedge'$的一组基,那么$X$和$X'$分别被矩阵$M\in\mathrm{M}(g\times2g,\mathbb{C})$和$M'\in\mathrm{M}(g'\times2g',\mathbb{C})$表示.进而$\rho_a(f)$被一个矩阵$A\in\mathrm{M}(g'\times g,\mathbb{C})$表示;$\rho_r(f)$被一个矩阵$R\in\mathrm{M}(2g'\times2g,\mathbb{Z})$表示.此时条件$\rho_a(f)(\wedge)\subseteq\wedge'$就等价于条件$AM=M'R$.反过来给定满足这个条件的$A,R$,那么在之前取定的$V,V',\wedge,\wedge'$的基下就定义了一个同态$X\to X'$.记$\mathrm{End}_{\mathbb{Q}}(X)=\mathrm{End}_{\textbf{Tori}}(X)\otimes_{\mathbb{Z}}\mathbb{Q}$,类似的定义$\mathrm{End}_{\mathbb{C}}(X)$.把$\rho_a\otimes1_{\mathbb{C}}$仍然记作$\rho_a$,把$\rho_r\otimes1_{\mathbb{Q}}$仍然记作$\rho_r$.那么有:
	$$\rho_r\otimes1_{\mathbb{C}}=\rho_a\oplus\overline{\rho_a}:\mathrm{End}_{\mathbb{C}}(X)\to\mathrm{End}_{\mathbb{C}}(\wedge\otimes_{\mathbb{Z}}\mathbb{C})\cong\mathrm{End}_{\mathbb{C}}(V\times V)$$
	\begin{proof}
		
		取定$V,\wedge$的一组基,那么$X$对应于矩阵$M\in\mathrm{M}(g\times2g,\mathbb{C})$.任取$f\in\mathrm{End}_{\textbf{Tori}}(X)$,记$\rho_a(f)$和$\rho_r(f)$对应的矩阵分别为$A\in\mathrm{M}(g,\mathbb{C})$和$R\in\mathrm{M}(2g,\mathbb{Z})$.从$AM=MR$得到:
		$$\left(\begin{array}{cc}A&0\\0&\overline{A}\end{array}\right)\left(\begin{array}{c}M\\\overline{M}\end{array}\right)=\left(\begin{array}{c}M\\\overline{M}\end{array}\right)R$$
		我们之前解释了$\left(\begin{array}{c}M\\\overline{M}\end{array}\right)$是可逆矩阵.
	\end{proof}
	\item 设$f:X\to X'$是复环面之间的同态,那么$\mathrm{im}f=F(V)/F(V)\cap\wedge'$也是复环面(其中$F$是$f$的延拓);$\ker f$是$X$的闭子群,并且单位分支$(\ker f)_0$也是复环面,并且在$\ker f$中具有有限指数.
	\begin{proof}
		
		$\mathrm{im}f$是复环面是因为它是连通紧复李群,或者$F(V)\cap\wedge'$是$F(V)$的格(因为它离散,它包含了$F(\wedge)$导致它$\mathbb{R}$生成了整个$F(V)$).按照$\ker f$是紧集的闭子集,所以仍然是紧的,所以只有有限个连通分支.只剩下证明$(\ker f)_0$是复环面.这是因为有$(\ker f)_0=F^{-1}(\wedge')_0/(F^{-1}(\wedge')\cap\wedge)$.并且$F^{-1}(\wedge')_0$是线性子空间:如果$v\in F^{-1}(\wedge')_0$,那么存在有限折点的折线段使得$v$和$w$相连,取临近$v$的线段,另一个端点记作$w$,那么有$(1-t)v+tw\in F^{-1}(\wedge')_0,\forall0\le t\le1$,进而有$(1-t)F(v)+tF(w)\in\wedge'$.但是$\wedge'$是离散的,迫使$(1-t)F(v)+tF(w)$是常值的,也即$F(w)=F(v)=0$,特别的有$0$到$v$的连线在$F^{-1}(\wedge')$中,进而它必须在$F^{-1}(\wedge')_0$中.最后按照$\ker f$是紧集的闭子集,于是紧,迫使$(\ker f)_0$是紧的,进而$F^{-1}(\wedge')\cap\wedge$是$F^{-1}(\wedge')_0$的完备格.
	\end{proof}
	\item 积.设$X=V/\wedge$和$X'=V'/\wedge'$是复环面,它们的积对象是$X\times X'=(V\times V')/(\wedge\times\wedge')$.
	\item 同源.两个复环面之间的同源(isogeny)是指一个具有有限核的满同态$f:X\to X'$.这也等价于讲$f$是满同态并且$\dim X=\dim X'$.任取复环面$X$的有限子群$\Gamma$,那么$X/\Gamma$仍然是复环面,因为如果记$X=V/\wedge$和典范商映射$\pi:V\to X\to X/\Gamma$,那么有$X/\Gamma=V/\pi^{-1}(\Gamma)$.进而有典范商映射$X\to X/\Gamma$是同源.
	\item 对复环面的同态$f:X\to X'$,它可以做如下stein分解,其中$g$是核为复环面的同态,$h$是同源:
	$$\xymatrix{X\ar[rr]^f\ar[dr]_g&&X'\\&X/(\ker f)_0\ar[ur]_h&}$$
	\item 次数.设$f:X\to X'$是同态,它的次数$\deg f$定义为,当$\ker f$是有限集时,定义为$|\ker f|$,否则定义$\deg f=0$.
	\begin{enumerate}[(1)]
		\item 如果$f$是同源,那么有$\deg f=[\wedge':\rho_r(f)(\wedge)]$.
		\item 如果$f$是$X\to X$的同源,那么有$\deg f=\deg\rho_r(f)$.
		\item 如果$f':X'\to X''$是另一个同态,那么有$\deg f'f=\deg f\deg f'$.特别的同源的复合仍然是同源.
	\end{enumerate}
    \item 对复环面$X$和整数$n$,记$[n]:X\to X$为$x\mapsto nx$.
    \begin{enumerate}[(1)]
    	\item $[n]$总是满射,也即$X$作为阿贝尔群总是可除群.
    	\item 我们有$\ker[n]=\frac{1}{n}\wedge/\wedge\cong\wedge/n\wedge\cong(\mathbb{Z}/n\mathbb{Z})^{2g}$.
    	\item 于是$[n]$总是一个同源,并且次数为$n^{2g}$.
    \end{enumerate}
    \item 对偶同源.设$f:X\to X'$是指数为$e$的同源(这里指数指的是$\ker f$的指数,也即最小的正整数$n$使得$nx=0,\forall x\in\ker f$),那么存在在同构意义下唯一的同源$g:X'\to X$,使得$gf=[e]$和$fg=[e]$.
    \begin{proof}
    	
    	按照$\ker f\subseteq\ker[e]=X[e]$,存在唯一的同态$g:X'\to X$使得$gf=[e]$,这个$g$也是同源.任取$x'\in\ker g$,那么存在$x\in\ker[e]_X$满足$f(x)=x'$,并且有$ex'=ef(x)=f(ex)=0$,于是$\ker g\subseteq X'[e]$.于是存在同源$f':X\to X'$使得$f'g=[e]_{X'}$.最后$f'[e]_X=f'gf=[e]_{X'}f=f[e]_X$,从$[e]_X$是满射得到$f=f'$.
    \end{proof}
    \item 推论.
    \begin{enumerate}[(1)]
    	\item 同源定义了复环面上的一个等价关系.
    	\item 按照$\mathrm{Hom}_{\textbf{Tori}}(X,X')$是自由阿贝尔群,它可以嵌入到$\mathrm{Hom}_{\mathbb{Q}}(X,X')=\mathrm{Hom}_{\textbf{Tori}}(X,X')\otimes_{\mathbb{Z}}\mathbb{Q}$中.那么$\mathrm{End}_{\textbf{Tori}}(X)$中的元$f$是同源当且仅当它在$\mathrm{End}_{\mathbb{Q}}(X)$中可逆.
    \end{enumerate}
\end{enumerate}
\subsubsection{上同调}

设$X=V/\wedge$是复环面.作为实流形$X$可以视为$2g$个圈$S_1\cong S_1^{(i)}=\lambda_i\mathbb{R}/\lambda_i\mathbb{Z}$的直积,其中$\{\lambda_1,\cdots,\lambda_{2g}\}$是自由阿贝尔群$\wedge$的一组基.
\begin{enumerate}
	\item 同调群.我们解释过有$\pi_1(X)=\mathrm{H}_1(X,\mathbb{Z})=\wedge$.按照$\mathrm{H}_1(X,\mathbb{Z})$是自由阿贝尔群,从K\"unneth公式得到$\mathrm{H}_n(X,\mathbb{Z})=\oplus_{p+q=n}\left(\mathrm{H}_p(X,\mathbb{Z})\otimes\mathrm{H}_q(X,\mathbb{Z})\right)$,进而这是秩为$\left(\begin{array}{c}2g\\n\end{array}\right)$的自由阿贝尔群.
	\item 上同调群.按照泛系数定理,有$\mathrm{H}^1(X,\mathbb{Z})=\mathrm{Hom}(\pi_1(X),\mathbb{Z})$.我们断言对任意正整数$n$,被cup积诱导的典范映射$\bigwedge^n\mathrm{H}^1(X,\mathbb{Z})\to\mathrm{H}^n(X,\mathbb{Z})$是同构.进而$\mathrm{H}^n(X,\mathbb{Z})$也是秩为$\left(\begin{array}{c}2g\\n\end{array}\right)$的自由阿贝尔群.
	\begin{proof}
		
		我们要证明的是对任意正整数$m$有如下典范映射是同构:
		$$\gamma_m^n:\bigwedge^n\mathrm{H}^1(S_1^m,\mathbb{Z})\to\mathrm{H}^n(S_1^m,\mathbb{Z})$$
		为此我们对$m$做归纳,$m=1$时$S_1$的$\mathbb{Z}$系数上同调环是$\mathbb{Z}[x]/(x^2)$,此时同构是直接的.下面设$m>1$,设对任意$p<m$有$\gamma_p^n$是同构.按照K\"unneth公式和cup积可交换,有如下交换图表:
		$$\xymatrix{\oplus_{p+q=n}\left(\bigwedge^p\mathrm{H}^1(S_1^{m-1,\mathbb{Z}}\otimes\bigwedge^q\mathrm{H}^1(S_1,\mathbb{Z}))\right)\ar[rr]^{\gamma}\ar@{=}[d]&&\oplus_{p+q=n}\left(\mathrm{H}^p(S_1^{m-1},\mathbb{Z})\otimes\mathrm{H}^q(S_1,\mathbb{Z})\right)\ar[dd]_{\beta_m}\\\bigwedge^n\left(\mathrm{H}^1(S_1^{m-1},\mathbb{Z})\oplus\mathrm{H}^1(S_1,\mathbb{Z})\right)\ar[d]_{\alpha_m}&&\\\bigwedge^n\mathrm{H}^1(S_1^m,\mathbb{Z})\ar[rr]^{\gamma_m^n}&&\mathrm{H}^n(S_1^m,\mathbb{Z})}$$
		按照归纳假设,有$\gamma$是同构,按照$\mathrm{H}^p(S_1^{m-1},\mathbb{Z})$和$\mathrm{H}^q(S_1,\mathbb{Z})$都是自由阿贝尔群,从K\"unneth公式得到$\alpha_m$和$\beta_m$都是同构.于是图表交换得到$\gamma_m^n$是同构.
	\end{proof}
    \item 按照$\mathrm{H}^n(X,\mathbb{Z})$和$\mathrm{H}_n(X,\mathbb{Z})$都是自由阿贝尔群,泛系数定理告诉我们对任意阿贝尔群$R$有$\mathrm{H}^n(X,R)=\mathrm{H}^n(X,\mathbb{Z})\otimes_{\mathbb{Z}}R$和$\mathrm{H}_n(X,R)=\mathrm{H}_n(X,\mathbb{Z})\otimes_{\mathbb{Z}}R$.
    \item 记$\mathrm{Alt}^n_{\mathbb{R}}(V,C)=\bigwedge^n\mathrm{Hom}_{\mathbb{Z}}(\wedge,\mathbb{Z})\otimes_{\mathbb{Z}}\mathbb{C}$,这是$V$上所有取值在$\mathbb{C}$的$\mathbb{R}$交错$n$形式构成的空间,那么有典范同构:
    $$\mathrm{H}^n(X,\mathbb{C})\cong\mathrm{Alt}_{\mathbb{R}}^n(V,\mathbb{C})=\bigwedge^n\mathrm{Hom}_{\mathbb{R}}(V,\mathbb{C})\cong\bigwedge^n\mathrm{H}^1(X,\mathbb{C})$$
    按照De Rham定理有典范同构:
    $$\mathrm{H}^n(X,\mathbb{C})\cong\mathrm{H}^n_{\mathrm{DR}}(X)=\frac{\{\text{闭}n\text{形式}\}}{\{\text{恰当}n\text{形式}\}}$$
    \item 不变形式.我们对$\mathrm{H}^n_{\mathrm{DR}}(X)$中每个$n$形式类选取一个代表元.固定$\wedge=\mathrm{H}_1(X,\mathbb{Z})$的一组基$\{\lambda_1,\cdots,\lambda_{2g}\}$.定义$v=\sum_{1\le i\le2g}x_i(v)\lambda_i,\forall v\in V$,这里$\{x_i\}$就是$V$关于上述基的实坐标函数.称$V$或者$X$上的复值光滑形式$\omega$是不变的,如果对任意$v\in V$或者对任意$x\in X$有$t_v^*\omega=\omega$或者$t_x^*\omega=\omega$,这里$t_x$或者$t_v$是平移变换.例如$\{\mathrm{d}x_1,\cdots,\mathrm{d}x_{2g}\}$是$V$上的不变形式.进而它们定义了$X$上的不变形式,仍然记作$\mathrm{d}x_i$.在De Rham同构下,这些$\{\mathrm{d}x_i\}$就对应于$\mathrm{H}^1(X,\mathbb{C})$的一组基.按照构造有$\{\mathrm{d}x_i\}$和$\{\lambda_i\}$互相对偶.我们解释过上同调的cup积对应于微分形式在外代数中的乘积.于是$\mathrm{d}x_{i_1}\wedge\cdots\wedge\mathrm{d}x_{i_n},i_1<\cdots<i_n$就对应了$\mathrm{H}^n(X,\mathbb{C})$的一组基.由于这些形式张成的空间就是全体不变$n$形式构成的空间$\mathrm{IF}^n(X)$.于是我们有典范同构$\mathrm{H}^n(X,\mathbb{C})\cong\mathrm{IF}^n(X)$.
\end{enumerate}
\subsubsection{Hodge分解}

\begin{enumerate}
	\item 记$\Omega_X^p$是复流形$X$上的全纯$p$形式层.记$V=\mathrm{T}_0X$,记$V$的复对偶空间为$\Omega$,我们断言有典范同构$\left(\bigwedge^p\Omega\right)\otimes_{\mathbb{C}}\mathscr{O}_X\cong\Omega_X^p$.特别的有$\Omega_X^p$是秩$\left(\begin{array}{c}g\\p\end{array}\right)$的$\mathscr{O}_X$模层.
	\begin{proof}
		
		对$x\in X$考虑平移变换$t_{-x}$,它诱导了切空间同构$\mathrm{d}t_{-x}:\mathrm{T}_xX\to\mathrm{T}_0X$,它的对偶是同构$(\mathrm{d}t_{-x})^*:\Omega=\Omega_{X,0}^1\to\Omega_{X,x}^1$.任取$\varphi\in\bigwedge^p\Omega$,定义$X$上的一个平移不变全纯$p$形式$\omega_{\varphi}$为$(\omega_{\varphi})_x=\left(\bigwedge^p(\mathrm{d}t_{-x})^*\right)\varphi$.于是$\varphi\to\omega_{\varphi}$诱导了一个同构:
		$$\left(\bigwedge^p\Omega\right)\otimes_{\mathbb{C}}\mathscr{O}_X\to\Omega_X^p$$
	\end{proof}
    \item 定义$\mathrm{IF}^{p,q}(X)$.选取$V$的一组基$\{e_1,\cdots,e_g\}$,对应的坐标函数记作$\{v_1,\cdots,v_g\}$.那么$\{\mathrm{d}v_i,\mathrm{d}\overline{v_j}\}$在$\mathbb{C}$上线性无关,并且它们是$\mathrm{IF}^1(X)$的一组基.对重指标$I=(i_1<\cdots<i_p)$,定义$\mathrm{d}v_I=\mathrm{d}v_{i_1}\wedge\cdots\wedge\mathrm{d}v_{i_p}$和$\mathrm{d}\overline{v}_I=\mathrm{d}\overline{v}_{i_1}\wedge\cdots\wedge\mathrm{d}\overline{v}_{i_p}$.如果$\varphi\in\mathrm{IF}^n(X)$可以表示为$\varphi=\sum_{|I|=p,|J|=q}\alpha_{IJ}\mathrm{d}v_I\wedge\mathrm{d}\overline{v}_J$,其中$\alpha_{IJ}\in\mathbb{C}$和$p+q=n$,就称$\varphi$是一个$(p,q)$型不变形式.这个定义不依赖于基的选取.于是我们有典范同构:
    $$\mathrm{IF}^n(X)=\bigoplus_{p+q=n}\mathrm{IF}^{p,q}(X)$$
    按照上一条有$\mathrm{IF}^{p,0}(X)=\mathrm{H}^0(X,\Omega_X^p)=\bigwedge^p\Omega$.类似的有$\mathrm{IF}^{0,q}(X)\cong\bigwedge^q\overline{\Omega}$.并且我们有典范同构$\bigwedge^p\Omega\otimes\bigwedge^q\overline{\Omega}\cong\mathrm{IF}^{p,q}(X)$,此即把$\varphi_1\otimes\varphi_2$映为$\omega_{\varphi_1\otimes\varphi_2}=\omega_{\varphi_1}\wedge\omega_{\varphi_2}$.综上我们得到典范同构:$$\mathrm{H}^n(X,\mathbb{C})\cong\bigoplus_{p+q=n}\mathrm{IF}^{p,q}(X)\cong\bigoplus_{p+q=n}\bigwedge^p\Omega\otimes\bigwedge^q\overline{\Omega}$$
    \item Dolbeault上同调.记$\mathscr{A}_X^{p,q}$是$X$上的光滑$(p,q)$型形式层,我们有Dolbeault预解:
    $$\xymatrix{0\ar[r]&\Omega_X^p\ar[r]&\mathscr{A}_X^{p,0}\ar[r]^{\overline{\partial}}&\mathscr{A}_X^{p,1}\ar[r]^{\overline{\partial}}&\cdots}$$
    进而得到同构:
    $$\mathrm{H}^q(X,\Omega_X^p)\cong\mathrm{H}^{p,q}(X)=\frac{\{\overline{\partial}\text{-闭}(p,q)\text{形式}\}}{\overline{\partial}\mathrm{H}^0(X,\mathscr{A}_X^{p,q-1})}$$
    \item 我们可以证明$\mathrm{H}^{p,q}(X)$和$\mathrm{IF}^{p,q}(X)$都典范等同于$(p,q)$型调和形式群.所以它们是典范同构的.进而有复环面上的Hodge分解定理:
    \begin{enumerate}[(1)]
    	\item 对任意$n\ge0$,有如下典范同构:
    	$$\mathrm{H}^n(X,\mathbb{C})\cong\bigoplus_{p+q=n}\mathrm{H}^q(X,\Omega_X^p)$$
    	\item 对任意$(p,q)$,记$V=\mathrm{T}_0X$,记$\Omega$是$V$的复对偶空间,记$\overline{\Omega}=\mathrm{Hom}_{\overline{\mathbb{C}}}(V,\mathbb{C})$,那么存在如下典范同构,特别的有$h^p(X,\Omega_X^q)=h^q(X,\Omega_X^p)$:
    	$$\mathrm{H}^q(X,\Omega_X^p)\cong\bigwedge^p\Omega\otimes\bigwedge^q\overline{\Omega}$$
    \end{enumerate}
\end{enumerate}
\subsubsection{线丛}
\begin{enumerate}
	\item 回顾向量丛.设$X$是复流形,它的一个秩为$r$的全纯向量丛是指一个复流形$E$以及一个全纯映射$\pi:E\to X$,使得它的纤维都是$r$维复线性空间,并且满足:
	\begin{enumerate}[(1)]
		\item 存在$X$的开覆盖$\{U_i\}$,使得$\varphi_i:\pi^{-1}(U_i)\cong U_i\times\mathbb{C}^r$,并且$\pi\circ\varphi_i^{-1}:U_i\times\mathbb{C}^r\to U_i$就是投影映射.换句话讲在每个$U_i$上都是平凡向量丛.
		\item 对任意指标$i,j$有$\varphi_i\circ\varphi_j^{-1}:U_j\times\mathbb{C}^r\to U_i\times\mathbb{C}^r$在第二个分量上是$\mathbb{C}$线性的.
	\end{enumerate}
	\item 回顾复流形$X$上的线丛是指秩1向量丛,全体向量丛的同构类在张量积下构成交换群,称为$X$的皮卡群,记作$\mathrm{Pic}(X)$.它可以典范同构于上同调群$\mathrm{H}^1(X,\mathscr{O}_X^*)$.我们有如下指数映射$\exp$得到的$X$上层的短正合列:
	$$\xymatrix{0\ar[r]&\underline{\mathbb{Z}}\ar[r]&\mathscr{O}_X\ar[r]^{\exp}&\mathscr{O}_X^*\ar[r]&0}$$
	进而诱导了上同调群的长正合列:
	$$\xymatrix{\cdots\ar[r]&\mathrm{H}^1(X,\underline{\mathbb{Z}})\ar[r]&\mathrm{H}^1(X,\mathscr{O}_X)\ar[r]&\mathrm{H}^1(X,\mathscr{O}_X^*)\ar[r]^{c_1}&\mathrm{H}^2(X,\underline{\mathbb{Z}})\ar[r]&\cdots}$$
	例如这可以说明$\mathrm{Pic}(\mathbb{C}^n)=0$:取$X=\mathbb{C}^n$,一方面按照庞加莱引理有$\mathrm{H}^1(X,\mathscr{O}_X)=0$,另一方面按照$X$是可缩空间,它的奇异上同调就平凡$\mathrm{H}^2(X,\mathbb{Z})=0$.
	\item 设$X$是复流形,设$\pi_1(X)$是基本群,设$\pi:\widetilde{X}\to X$是万有覆盖,那么$\mathrm{H}^0(\widetilde{X},\mathscr{O}_{\widetilde{X}}^*)$上典范的具备一个$\pi_1(X)$模结构.进而可以考虑群上同调$\mathrm{H}^1(\pi_1(X),\mathrm{H}^0(\widetilde{X},\mathscr{O}_{\widetilde{X}}^*))$,这里$\mathrm{H}^0(\widetilde{X},\mathscr{O}_{\widetilde{X}}^*)$就是$\widetilde{X}$上的可逆全纯函数环,一个映射$\pi(X)\to\mathrm{H}^0(\widetilde{X},\mathscr{O}_{\widetilde{X}}^*)$就可以理解为一个映射$\pi_1(X)\times\widetilde{X}\to\mathbb{C}^*$.进而这个上同调的1余圈集合是:$$\mathrm{Z}^1(\pi_1(X),\mathrm{H}^0(\widetilde{X},\mathscr{O}_{\widetilde{X}}^*))=\{f:\pi_1(X)\times\widetilde{X}\to\mathbb{C}^*\text{全纯}\mid f(\lambda\mu,\widetilde{x})=f(\lambda,\mu\widetilde{x})f(\mu,\widetilde{x})\}$$
	\item 定义$\mathrm{Z}^1(\pi_1(X),\mathrm{H}^0(\widetilde{X},\mathscr{O}_{\widetilde{X}}^*))\to\mathrm{Pic}(X)$如下:任取1-余圈$f:\pi_1(X)\times\widetilde{X}\to\mathbb{C}^*$,构造$\widetilde{X}$上秩1平凡向量丛$\pi:\widetilde{X}\times\mathbb{C}\to\widetilde{X}$的$\pi_1(X)$作用为$\lambda(\widetilde{x},t)=(\lambda\widetilde{x},f(\lambda,\widetilde{x})t)$.这个作用是自由和严格不连续的,所以商$L=\widetilde{X}\times\mathbb{C}/\pi_1(X)$是复流形.进而$\pi$诱导了全纯线丛$p:L\to X$.这定义为$f$的像,这是一个交换群同态,它诱导了【Christina Birkenhake-Complex abelian varietties,appendix B】如下函子性的同构:
	$$\mathrm{H}^1(\pi_1(X),\mathrm{H}^0(\widetilde{X},\mathscr{O}_{\widetilde{X}}^*))\cong\ker\left(\xymatrix{\mathrm{H}^1(X,\mathscr{O}_X^*)\ar[r]^{\pi^*}&\mathrm{H}^1(\widetilde{X},\mathscr{O}^*_{\widetilde{X}})}\right)$$
	特别的,如果$X=V/\wedge$是复环面,那么$\pi^*$是零映射,于是有函子性的同构:
	$$\mathrm{H}^1(\pi_1(X),\mathrm{H}^0(\widetilde{X},\mathscr{O}_{\widetilde{X}}^*))\cong\mathrm{H}^1(X,\mathscr{O}_X^*)$$
	\item 第一陈类.设$L$是复环面$X$上的线丛,它的第一陈类是指$c_1(L)\in\mathrm{H}^2(X,\mathbb{Z})$.我们解释过有典范同构$\mathrm{H}^2(X,\mathbb{Z})\cong\bigwedge^2\mathrm{Hom}(\wedge,\mathbb{Z})$,于是第一陈类可以视为$\wedge$上的$\mathbb{Z}$值交错2-形式.
	\item 存在典范同构$\mathrm{H}^2(X,\mathbb{Z})\cong\bigwedge^2\mathrm{Hom}(\wedge,\mathbb{Z})$,把复环面$X$上的线丛$L$的第一陈类$c_1(L)$映为如下$\mathbb{Z}$值交错2-形式,其中$f$是$L$所对应的$\mathrm{Z}^1(\pi_1(X),\mathrm{H}^0(\widetilde{X},\mathscr{O}_{\widetilde{X}}^*))$中的一个代表元,而$g:\wedge\times V\to\mathbb{C}$是满足$f=\exp(2\pi ig)$的对第二个分量全纯的映射.
	$$E_L(\lambda,\mu)=g(\mu,v+\lambda)+g(\lambda,v)-g(\lambda,v+\mu)-g(\mu,v),\forall v\in V$$
	\item 设$E:V\times V\to\mathbb{R}$是交错2-形式,那么$E$对应于复环面$X$上某个全纯线丛的第一陈类(这个对应就是说,它是$L$对应的某个交错2-形式$E:\wedge\times\wedge\to\mathbb{Z}$经$\mathbb{R}$线性延拓得到的$V\times V\to\mathbb{R}$),当且仅当$E(\wedge,\wedge)\subseteq\mathbb{Z}$,并且$E(iv,iw)=E(v,w),\forall v,w\in V$.
	\begin{proof}
		
		考虑如下图表,其中$l$是典范嵌入,$\beta_2,\gamma_2$是hodge分解中的典范同构,$p$是相应的投影映射,这个图表交换是因为$\mathrm{H}^2(X,\mathbb{Z})\to\mathrm{H}^2(X,\mathscr{O}_X)$的确经$\mathrm{H}^2(X,\mathbb{C})$的hodge分解投影到$\mathrm{H}^{0,2}(X)$.
		$$\xymatrix{\mathrm{H}^1(X,\mathscr{O}_X^*)\ar[rr]^{c_1}&&\mathrm{H}^2(X,\mathbb{Z})\ar[rr]\ar[d]^l&&\mathrm{H}^2(X,\mathscr{O}_X)\ar@{=}[d]\\&&\mathrm{H}^2(X,\mathbb{C})\ar[rr]^p\ar[d]^{\beta_2^{-1}}\ar[rr]^p&&\mathrm{H}^{0,2}(X)\ar[d]^{\gamma_2^{-1}}\\&&\wedge^2\Omega\oplus\left(\Omega\otimes\overline{\Omega}\right)\oplus\wedge^2\overline{\Omega}\ar[rr]^p&&\wedge^2\overline{\Omega}}$$
		现在设$L\in\mathrm{H}^1(X,\mathscr{O}_X^*)$,记hodge分解$\beta_2^{-1}\circ l\circ c_1(L)=E=E_1+E_2+E_3$.上述图表的交换性告诉我们$E_3=0$,又按照$E$的取值是实数,有$E_1=\overline{E_3}$,进而$E_1=0$,进而$E=E_2$,这得到必要性.充分性是类似的.
	\end{proof}
    \item 满足上一条中的$E(\wedge,\wedge)\subseteq\mathbb{Z}$,并且$E(iv,iw)=E(v,w),\forall v,w\in V$的实值交错2-形式$E$,恰好一一对应于$V$上Hermitian形式,这个对应为:
    $$E(v,w)\mapsto H(v,w)=E(iv,w)+iE(v,w)$$
    $$H(v,w)\mapsto\mathrm{Im}H(v,w)$$
    \item 定义复环面$X$上的N\'eron-Severi群,简称NS群,记作$\mathrm{NS}(X)$,是全体第一陈类构成的群,也即$c_1:\mathrm{H}^1(X,\mathscr{O}_X^*)\to\mathrm{H}^2(X,\mathbb{Z})$的像.按照上两条,此即$V$上满足$E(\wedge,\wedge)\subseteq\mathbb{Z}$和$E(iv,iw)=E(v,w),\forall v,w\in V$的实值交错2-形式构成的群,或者$V$上的满足$\mathrm{Im}H(\wedge,\wedge)\subseteq\mathbb{Z}$的Hermitian形式构成的群.
    \item 取$\mathrm{NS}(X)$中的一个元,把它视为一个Hermitian形式$H:V\times V\to\mathbb{C}$,使得$\mathrm{Im}H(\wedge,\wedge)\subseteq\mathbb{Z}$.记$\mathbb{S}^1=\{z\in\mathbb{C}\mid|z|=1\}$.定义$H$的一个半特征(semicharacter)是指一个映射$\chi:\wedge\to\mathbb{S}^1$,满足:
    $$\chi(\lambda+\mu)=\chi(\lambda)\chi(\mu)\exp(\pi i\mathrm{Im}H(\lambda,\mu)),\forall\lambda,\mu\in\wedge$$
    特别的,$\wedge$的取值在$\mathbb{S}^1$的特征恰好就是NS群中零元对应的半特征.称$(H,\chi)$是一个半特征对,全体半特征对构成的集合记作$\mathrm{P}(\wedge)$,它在乘法$(H_1,\chi_1)\times(H_2,\chi_2)=(H_1+H_2,\chi_1\chi_2)$下构成群,并且投影$p:(H,\chi)\mapsto H$和同态$l:\chi\mapsto(0,\chi)$构成了短正合列(其实就是说每个NS群中的元都有半特征):
    $$\xymatrix{1\ar[r]&\mathrm{Hom}(\wedge,\mathbb{S}^1)\ar[r]^l&\mathrm{P}(\wedge)\ar[r]^p&\mathrm{NS}(X)\ar[r]&1}$$
    \item 构造$\varphi:\mathrm{P}(\wedge)\to\mathrm{Pic}(X)$如下(后文会证明这是同构):任取半特征对$(H,\chi)$,定义:
    $$a=a_{(H,\chi)}:\wedge\times V\to\mathbb{C}^*$$
    $$(\lambda,v)\mapsto\chi(\lambda)\to\chi(\lambda)\exp(\pi H(v,\lambda)+\frac{\pi}{2}H(\lambda,\lambda))$$
    这个映射$a$落在$\mathrm{Z}^1(\wedge,\mathrm{H}^0(\pi_1(X),\mathscr{O}_V^*))$中,进而对应了$X$上的一个线丛$L(H,\chi)\cong V\times\mathbb{C}/\wedge$,其中$\wedge$在$V\times\mathbb{C}$上的作用是$\lambda(v,t)=(v+\lambda,a_{(H,\chi)}(\lambda,v)t)$.这是一个群同态,并且有如下交换图表:
    $$\xymatrix{\mathrm{P}(\wedge)\ar[rr]^{\varphi}\ar[dr]_p&&\mathrm{Pic}(X)\ar[dl]^{c_1}\\&\mathrm{NS}(X)&}$$
    \item Appell-Humbert定理.设$X=V/\wedge$是复环面,记$c_1$的核为$\mathrm{Pic}^0(X)$,此为第一陈类平凡的线丛构成的子群,我们有如下短正合列的典范同构:
    $$\xymatrix{1\ar[r]&\mathrm{Hom}(\wedge,\mathbb{S}^1)\ar[r]\ar[d]^{\cong}&\mathrm{P}(\wedge)\ar[d]^{\cong}\ar[r]&\mathrm{NS}(X)\ar[r]\ar@{=}[d]&\\1\ar[r]&\mathrm{Pic}^0(X)\ar[r]&\mathrm{Pic}(X)\ar[r]&\mathrm{NS}(X)\ar[r]&0}$$
    \item 特别的,上一条说明任取$X$上的线丛$L$,记$c_1(L)=H$,那么存在关于$H$的唯一半特征$\chi$使得$L\cong L(H,\chi)$.定义$L$对应的典范因子(factor,指的是它作为1-余圈)为:
    $$a_L=a_{L(H,\chi)}\in\mathrm{Z}^1(\wedge,\mathrm{H}^0(V,\mathscr{O}_V^*))$$
    $$(\lambda,v)\mapsto\chi(\lambda)\exp(\pi\mathrm{H}(v,\lambda)+\frac{\pi}{2}H(\lambda,\lambda))$$
    \item 
    
\end{enumerate}











