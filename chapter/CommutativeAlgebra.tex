\chapter{交换代数}
\section{基本工具}
\subsection{NAK引理}
\subsubsection{行列式技巧}

如果$A$模$M$是有限生成的,我们通常简称$M$是有限$A$模.有限$A$模上的一个标准技巧是所谓的"行列式技巧".
\begin{enumerate}
	\item 设$M$是由$n$个元生成的$A$模,设理想$I\subseteq A$和$M$上自同态$\varphi$满足$\varphi(M)\subseteq IM$,那么存在一个次数为$n$的多项式$p(x)=x^n+p_1x^{n-1}+\cdots+p_n$,其中$p_j\in I^j$,使得$p(\varphi)=0$.
	\begin{proof}
		
		记$m_1,m_2,\cdots,m_n\in M$生成了整个$M$,按照条件对每个$1\le i\le n$有$\varphi(m_i)=\sum_ja_{ij}m_j$.把$M$视为$A[X]$模,这里定义$Xm=\varphi(m)$.那么有:
		$$\left(\begin{array}{ccc}X&&\\&\ddots&\\&&X\end{array}\right)
		\left(\begin{array}{c}m_1\\\vdots\\m_n\end{array}\right)
		=\left(\begin{array}{ccc}a_{11}&\cdots&a_{1n}\\\vdots&&\vdots\\a_{n1}&\cdots&a_{nn}\end{array}\right)\left(\begin{array}{c}m_1\\\vdots\\m_n\end{array}\right)$$
		
		记$T=(a_{ij})$,记列向量为$v$,上述等式即$(xE_n-T)v=0$,左侧乘以$(xE-T)$的伴随矩阵,记$p(x)=\det(xE-T)$,得到$p(x)E_nv=0$,即$p(x)m_i=0,\forall 1\le i\le n$,这就说明了$p(\varphi)=0$,并且按照构造$p$的系数满足要求.
	\end{proof}
	\item 特别的,当$I$取单位理想时套入上述证明,这就得到经典的Cayley-Hamilton定理,即一个有限维线性空间上的线性变换被它自身的特征多项式零化.
	\item 特别的,如果把$\varphi$取为恒等映射,得到结论:给定有限$A$模$M$,给定理想$I$,那么$M=IM$当且仅当存在$a\in I$满足$(1+a)M=0$.
	\item 给定有限$A$模$M$,如果$\alpha$是$M$上的一个满同态,那么它是同构.在线性代数中考虑维数公式可以得到这一结果.
	\begin{proof}
		
		把$M$看作$A[X]$模,这里$X$的作用定义为$Xm=\alpha(m)$.取理想$I=(X)\subseteq A[X]$,$\alpha$是满同态说明$IM=M$,于是上一条说明存在$a\in I=(X)$使得$(1+a)M=0$,记$a=-Xq(X)$,这说明在$M$上有$q(\alpha)\alpha=\mathrm{Id}$,这就说明$\alpha$是可逆的.
	\end{proof}
\end{enumerate}
\subsubsection{NAK引理}

这是一个很常用的技术引理,它有多种描述.这个定理通常称为Nakayama引理,但是他本人认为这个定理应该归功于Krull和Azumaya.他们谁先提出的交换环版本的结论已经无从考证,因此本文就把这个定理称为NAK引理.
\begin{enumerate}
	\item 如果$I$是一个位于Jacobson根的理想,$M$是有限$A$模,那么$IM=M$推出$M=0$.
	\begin{proof}
		
		这只要注意到对上一命题中的$a$如果在$J(R)$中,那么$a+1$必然是单位.
	\end{proof}
	\item 如果$M/N$是有限$A$模,满足$M=N+IM$,其中$I$位于Jacobson根中,那么$M=N$.
	\begin{proof}
		
		这只要把上一命题用在$M/N$上.注意到$I(M/N)=(IM+N)/N$.
	\end{proof}
	\item 如果$I$是一个位于Jacobson根的理想,$M$是有限$A$模,如果$m_1,\cdots,m_n\in M$映射到$M/IM$中生成了$M/IM$,那么$m_1,\cdots,m_n$生成了模$M$.
	\begin{proof}
		
		设$N$是由$m_1,m_2,\cdots,m_n$生成的$M$的子模,从$M$有限得到$M/N$有限,而条件说明了$M=N+IM$,从上一条就说明$M=N$.
	\end{proof}
    \item 反例.NAK引理对于非有限模未必成立.例如取一个局部整环$(A,m,k)$,设$A$不是域,那么可取$m$中的非零元$x$,那么$k$中每个元$y$可以表示为$y=x(y/x)$,于是$k=mk$,但是$k\not=0$.
\end{enumerate}
\subsubsection{局部环上模的基}

给定$R$模$M$,称它的一个生成元集是极小生成元集,如果满足每个真子集都不是$M$的生成元集.那么一般来讲有限模上不同的极小生成元集的元素个数不一定相同.不过对于局部环这是成立的:设$(A,m,k)$是局部环,设$M$是有限$A$模.那么$M'=M/mM$是$k=A/m$上的有限维线性空间,设维数是$n$,那么:
\begin{enumerate}
	\item 取$M'$的一组基$u_1',u_2',\cdots,u_n'$,任取相应的$M$中的原像$u_1,u_2,\cdots,u_n$,那么这是$M$的一个极小生成元集.
	\begin{proof}
		
		NAK引理中的最后一条已经证明了此时$\{u_1,u_2,\cdots,u_n\}$生成了$M$.现在假设去掉某个$u_i$后仍然生成了整个$M$,这就导致$\{u_1',u_2',\cdots,u_n'\}$去掉$u_i'$后仍然是$M'$中的一组基,矛盾.
	\end{proof}
	\item 反过来$M$的每个极小生成元集都可以按照上一条的方式得到.特别的,$(A,m)$上的有限模$M$的极小生成元集合的元素个数是固定的,它就是$\dim_kM'$.
	\begin{proof}
		
		假设$\{u_1,u_2,\cdots,u_n\}$是$M$的一组极小生成元集.设$u_i$在$M'$中的像是$u_i'$,那么$\{u_1',u_2',\cdots,u_n'\}$生成了整个$M'$.现在我们断言它们线性无关,因为若否,它的某个真子集生成了整个$M'$,那么从第一条得到这个真子集对应的$\{u_1,u_2,\cdots,u_n\}$的真子集生成了整个$M$,这就和预先假定的极小性矛盾.
	\end{proof}
	\item 取$M$的任意两组极小生成元集,那么过渡矩阵是$A$上的一个可逆方阵.
	\begin{proof}
		
		设过渡矩阵为$T=(a_{ij})$.记$a_{ij}$在$k=A/m$中的像是$a_{ij}'$,那么$T'=(a_{ij}')$是$k$上两组基之间的过渡矩阵,于是它的行列式非零,于是$\det T$不在极大理想$m$中,按照$(A,m)$是局部环得到$\det T$是$A$中单位元,于是它是$A$上的可逆矩阵.
	\end{proof}
\end{enumerate}
\newpage
\subsection{链条件}
\subsubsection{诺特模和阿廷模}
\begin{enumerate}
	\item 交换环$R$的模$M$称为诺特模,如果它满足如下等价描述的任意一个:
		\begin{enumerate}
			\item $M$的每个子模都是有限生成的.
			\item $M$的子模满足acc条件,即每一条子模升链总会终止,即若有子模链$M_1\subseteq M_2\subset\cdots$,那么存在正整数$n$使得对于任意正整数$r$有$M_n=M_ {n+r}$.
			\item $M$的子模满足极大条件,即每个子模构成的集合含有在包含序下的极大元.
		\end{enumerate}
	\item 对偶的,交换环$R$的模$M$称为阿廷模,如果它满足如下等价条件中的任一:
		\begin{enumerate}
			\item $M$的子模满足dcc条件,即每一条子模降链总会终止.
			\item $M$的子模满足极小条件,即每个子模构成的集合含有在包含序下的极小元.
		\end{enumerate}
	\item 另外环可视为自身的模,如果环作为自身模是诺特模或者阿廷模,就称环是诺特环或者阿廷环.注意环作为自身模的时候,环的理想和这个模的子模是相同的概念.于是例如有诺特环等价于讲每个理想都是有限生成的.
\end{enumerate}

下面给出一些基本例子.这些例子说明模上诺特条件和阿廷条件是无关的,也说明了存在非阿廷的诺特环,但是没有给出非诺特的阿廷环,这是因为实际上阿廷环总是诺特的.
\begin{enumerate}
	\item $\mathbb{Z}$是诺特环,但不是阿廷环,例如有严格降链$(n)\supsetneqq(n^2)\supsetneqq(n^3)\cdots$.
	\item 有限环只有有限个理想,于是它总是诺特环和阿廷环.同理有限交换群作为$\mathbb{Z}$模也总是诺特模和阿廷模.
	\item 考虑交换群$M=\mathbb{Z}[1/p]/\mathbb{Z}$,其中$p$是一个素数.我们先来求它的全部子群,也即作为$\mathbb{Z}$模的全部子模.给定子群$G$,先说明假如它包含一个元$a/p^n+\mathbb{Z}$,其中$a/p^n$是既约的分数,那么$G$包含了$1/p^n+\mathbb{Z}$.事实上按照$(a,p^n)=1$,说明存在两个整数$x,y$使得$ax+p^ny=1$,导致$x(a/p^n)+y=1/p^n\in G$.现在设$S$是由这样的正整数$n$构成的集合,存在$G$中的一个元的表示的既约形式为$a/p^n+\mathbb{Z}$.倘若$S$没有上界,那么对任意的$a/p^n+\mathbb{Z}$,可取$S$中足够大的正整数$m$使得$m>n$,导致$a/p^n+\mathbb{Z}$是$1/p^m+\mathbb{Z}$的倍数,这说明此时$G=M$本身.倘若$S$是空集,此时$G=\{0\}$,最后倘若$G$有上界,设最大元是$r$,那么$G$恰好是$\langle 1/p^r+\mathbb{Z}\rangle$,记这个子群为$M_r$,于是$M$的全部子群就是$0,M_r,M$,这说明$M$是阿廷$\mathbb{Z}$模但不是诺特$\mathbb{Z}$模.
	\item $N=\mathbb{Z}[1/p]$既不是诺特$\mathbb{Z}$模也不是阿廷$\mathbb{Z}$模,这是因为此时有短正合列$0\to\mathbb{Z}\to N\to M\to0$,按照下面的短正合列准则,$\mathbb{Z}$非阿廷和$M$非诺特导致$N$既不是阿廷的也不是诺特的.
\end{enumerate}
\subsubsection{短正合列准则}

给定$R$模短正合列如下图,那么$B$是诺特/阿廷模当且仅当$A$和$C$都是诺特/阿廷模.
$$\xymatrix{0\ar[r]&A\ar[r]^{f}&B\ar[r]^{g}&C\ar[r]&0}$$
\begin{proof}

必要性.这里$A$同构于$B$的某个子模,$C$同构于$B$的某个商模,按照子模对应定理,无论从诺特情况的有限生成定义,还是从链条件定义都可直接推出必要性.

充分性.先考虑诺特情况的有限生成定义,任取$B$的子模$B_0$,那么$f^{-1}(B_0)$是$A$的一个子模,可设$\{a_1,a_2,\cdots,a_s\}$生成了这个子模,类似的$g(B_0)$是$C$的一个子模,可设$\{c_1,c_2,\cdots,c_t\}$生成了这个子模.按照$g$是满射,可取$\{b_1,b_2,\cdots,b_t\}$使得$g(b_i)=c_i$.我们断言$\{f(a_1),f(a_2),\cdots,f(a_s),b_1,b_2,\cdots,b_t\}$生成了整个$B_0$.事实上任取$b\in B_0$,那么$g(b)\in g(B_0)$,于是有$g(b)=\sum_im_ic_i$,于是$b-\sum_im_ib_i\in\ker g=\mathrm{im}f$中,于是$b=\sum_im_ib_i+\sum_jn_jf(a_j)$,这就得到$B_0$是有限生成的.

再考虑链条件定义.假设给定了$B$的一个子模链$B_1\subseteq B_2\subset\cdots$(类似的有降链条件),取$A_i=f^{-1}(B_i)$和$C_i=g(B_i)$,于是$\{A_i\}$和$\{C_i\}$分别是$A$和$C$的子模链.按照链条件,不妨设$n>N$的时候总有$A_n=A_{n+1}$和$C_n=C_{n+1}$.现在对任意正整数$n$,把$f,g$限制在更小的子模上会得到一个短正合列$\xymatrix{0\ar[r]&A_n\ar[r]&B_n\ar[r]&C_n\ar[r]&0}$.分别取包含映射$l_1,l_2,l_3$,会得到如下交换图.现在$l_1,l_3$都是同构,于是按照短五引理,得到$l_2$也是同构.
$$\xymatrix{0\ar[r]&A_n\ar[r]\ar[d]_{l_1}&B_n\ar[r]\ar[d]_{l_2}&C_n\ar[r]\ar[d]_{l_3}&0\\0\ar[r]&A_{n+1}\ar[r]&B_{n+1}\ar[r]&C_{n+1}\ar[r]&0}$$
\end{proof}

注意上述证明中我们得到了这样一个结论,如果短正合列两侧的模都是有限生成的,那么中间的模也是有限生成的.例如我们可以得到:如果$M,N$是$L$的子模,并且$M\cap N$和$M+N$都是有限生成的,那么$M$和$N$均为有限生成模.这只要考虑短正合列$0\to M\cap N\to M\to(M+N)/N\to0$即可.

借助上述短正合列准则可得诺特性和阿廷性的一些结论:
\begin{enumerate}
	\item 有限个模的直和是诺特/阿廷模当且仅当每个分量都是诺特/阿廷模.
	\item 诺特/阿廷环上有限生成模是诺特/阿廷的.关于这一条我们可以说得更多:诺特环上的模是有限生成模当且仅当是诺特模当且仅当是有限表示模.
	\item 一个诺特/阿廷模的子模和商都是诺特/阿廷模.诺特/阿廷环的商环是诺特/阿廷环.
	\item 但是诺特/阿廷环的子环未必是诺特/阿廷的,最简单的例子是分别考虑非诺特以及非阿廷的整环,它的商域是域,自然同时是诺特和阿廷的.还有一个例子是,按照下面的希尔伯特基定理得到$k[x,y]$是诺特的,这里$k$是一个域,那么它的子环$k[x,xy,xy^2,\cdots]$不是诺特环,因为$(x,xy,xy^2,\cdots)$不是有限生成的.
	\item 诺特环$R$上有限生成代数作为环是诺特环.这是希尔伯特基定理的推论.
	\item 分式化.如果$R$是诺特环,它的任一分式化$S^{-1}R$也是诺特的.这只要按照$S^{-1}R$的任一理想升链$J_1\subseteq J_2\subset\cdots$,它们在典范映射下的原像$\varphi^{-1}(J_i)=I_i$是$R$的理想列,于是$J_i=S^{-1}I_i$,并且有升链$I_1\subseteq I_2\subset\cdots$.于是这个升链终止,就得到原升链终止.另外我们也可以用所有理想均有限生成这个等价定义:$S^{-1}R$中的理想具有形式$S^{-1}I$,这里$I$是$R$的理想,于是按照$R$是诺特环得到$I=(a_1,a_2,\cdots,a_n)$,于是$S^{-1}I=(a_1/1,a_2/1,\cdots,a_n/1)$.
\end{enumerate}
\subsubsection{Cohen定理}

实际上全体素理想是有限生成的,就已经保证了所有理想都是有限生成的:(Cohen定理)如果环$R$的所有素理想是有限生成的,那么$R$的每个理想都是有限生成的,换句话讲$R$是诺特环.
\begin{proof}
	
	给定环$R$,设全体素理想是有限生成的,如果存在不是有限生成的理想,考虑全体这样的理想构成的集合$S$,赋予包含序,容易证明对于每个升链,全体其中的理想的并就是它的上界,于是按照Zorn引理,集合$S$存在一个极大元$I$.特别的,$I\not=R$,并且$I$不会是素理想,这说明存在$a,b\not\in I$满足$ab\in I$.那么按照极大性,$(a)+I$是有限生成的,记生成元为$w_ia+x_i,1\le i\le n$.现在取$B=\mathrm{Ann}((I+(a))/I)=(I:a)=\{r\in R\mid ra\in I\}$,那么$(b)+I\subseteq B$,另外$b\not\in I$,导致$B$是有限生成的,记生成元为$y_1,\cdots,y_m$,对$z\in I$,有$z\in I+(a)$,于是有$z=a_1x_1+\cdots+a_nx_n+ya$,这里$a_i,y$都是$R$中元,于是$ya\in I$,于是$y\in B$,于是$y=\sum b_jy_j$,这就导致$I$由$\{x_1,\cdots,x_n,ay_1,\cdots,ay_m\}$生成,这矛盾.于是所有理想都是有限生成的.
\end{proof}
\subsubsection{希尔伯特基定理}

这里的基是生成元的意思.如果交换环$R$是诺特环,那么多项式环$R[x]$同样是诺特环.注意这个结论实际上是充要的,因为从$R[x]$诺特可得到商环$R[x]/(x)\cong R$诺特.另外归纳法说明基定理实际上说明诺特环$R$上有限生成代数(作为环)总是诺特的.
\begin{proof}

任给$R[x]$的非零理想$I$,$R$中的能够成为$I$中某个多项式首系数的元构成了$R$的理想$J$.按照$R$是诺特的,可记$J=(r_1,r_2,\cdots,r_n)$.设$I$中的多项式$p_i$以$r_i$为首系数.取$R[x]$的理想$I_1=(p_1,p_2,\cdots,p_n)$.设$\max\{\deg p_i,1\le i\le n\}=d$.记此时小于$d$的多项式构成的$R[x]$的子集为$M_d$,那么它是一个有限生成$R$模,于是它是诺特$R$模.可验证$I=I\cap M_d+I_1$.现在$I\cap M_d$是诺特$R$模$M_d$的子模,于是它是有限生成的$R$模,取生成元集$\{q_1,q_2,\cdots,q_m\}$.设它们在$R[x]$中生成的理想为$I_2\subseteq I$,于是得到$I=I\cap M_d+I_1\subseteq I_2+I_1\subseteq I$.这就说明了$I=I_1+I_2=(p_1,p_2,\cdots,p_n,q_1,q_2,\cdots,q_m)$.完成证明.
\end{proof}

形式幂级数环具有类似性质.如果$R$是诺特环,那么形式幂级数环$R[[x]]$同样是诺特环.
\begin{proof}

对$R$上一个形式幂级数$f$,它可以唯一的表示为$f=a_rx^r+a_{r+1}x^{r+1}+\cdots$,其中$a_r\not=0$.称$a_r$是$f$的首系数.

取$R[[x]]$的非零理想$I$.需要说明的是$I$是有限生成的.为此考虑$R$中能够成为$I$中某个幂级数的首系数的元构成的集合$J$,那么$J$是$R$的一个理想,按照$R$的诺特性说明$J$有限生成,记作$(a_1,a_2,\cdots,a_s)$.设$a_i$对应的$J$中的幂级数为$f_i$.不妨约定所有$f_i$具有统一的最小非零项的次数$n$,否则可以以合适的$x^r$乘以若干$f_i$,这不会改变$f_i\in I$.

取同态$\pi:R[[x]]\to R[x]$为$\sum_{i\ge0}c_ix^i\mapsto\sum_{i=0}^{n-1}c_ix^i$.它的像空间就是$M_n=\{f\in R[x]\mid\deg f<n\}$,它是有限生成$R$模,于是$M_n$是诺特模.那么它的$R$子模$\pi(I)$是有限生成的.于是可取$I$中有限个幂级数$\{h_1,h_2,\cdots,h_t\}$使得$\{\pi(h_i)\}$的$R$线性组合就是$\pi(I)$.

我们断言$I=(f_1,f_2,\cdots,f_s,h_1,h_2,\cdots,h_t)$.任取$f\in I$,那么$\pi(f)$可以表示为$\pi(h_i)$的$R$线性组合,于是存在线性组合$h=r_1h_1+r_2h_2+\cdots+r_th_t$使得$\pi(f-h)=0$,这等价于讲$f$和$h$的从0次项到$n-1$次项系数是对应相同的.于是接下来需要说明$f-h\in(f_1,f_2,\cdots,f_s)$.等价于证明如果$f$的非零首项的次数为$n$(大于$n$的时候只要对每个$f_i$再乘以合适的$x^r$),那么$f$可以表示为$\{f_1,f_2,\cdots,f_s\}$的$R[[x]]$线性组合$f=g_1f_1+g_2f_2+\cdots+g_sf_s$.

记$f=\sum_{n\ge0}c_ix^i$,$f_j=\sum_{i\ge n}d_{i,j}x^i$,$g_j=\sum_{i\ge0}b_{i,j}x^i$.那么有:
$$c_{n+t}=\sum_{p=0}^t\sum_{j=1}^sb_{p,j}d_{n+t-p,j}$$

下面我们对$t\ge0$归纳证明存在系数族$\{b_{p,j}\in R\mid 1\le j\le s,0\le p\le t\}$,使得上述等式列对$0,1,\cdots,t$均成立.

初始步骤.需要验证$t=0$,注意$c_n\in J=(a_1,a_2,\cdots,a_s)=(d_{n,1},d_{n,2},\cdots,d_{n,s})$,于是存在$\{b_{0,1},b_{0,2},\cdots,b_{0,s}\}$满足:
$$c_n=b_{0,1}d_{n,1}+b_{0,2}d_{n,2}+\cdots+b_{0,s}d_{n,s}$$

归纳步骤.假设已经构造了$\{b_{p,j}\in R\mid 1\le j\le j,0\le p\le t-1\}$满足前$t$个等式,需要构造$\{b_{t,1},b_{t,2},\cdots,b_{t,s}\}$满足$c_{n+t}$的表达式.考虑如下形式幂级数:
$$g=f-\sum_{j=1}^sb_{0,j}f_j-\sum_{j=1}^kb_{1,j}xf_j-\cdots-\sum_{j=1}^kb_{t-1,j}x^{t-1}f_j$$

那么$g\in I$.按照归纳假设满足的前$t$个等式,说明$g$的$x^i$系数均为0,其中$i<n+t$.而$g$的首系数为$c=c_{n+1}-\sum_{p=0}^{t-1}\sum_{j=1}^sb_{p,j}d_{n+t-p,j}$.于是$c$可以表示为$\{d_{n,1},d_{n,2},\cdots,d_{n,k}\}$的$R$线性组合,记作$b_{t,1},b_{t,2},\cdots,b_{t,s}\in R$,满足$c=\sum_{j=1}^sb_{t,j}d_{n,j}$,于是得到第$t+1$个等式成立,这就完成归纳:
$$c_{n+t}=\sum_{p=0}^{t-1}\sum_{j=1}^sb_{p,j}d_{n+t-p,j}+\sum_{j=1}^sb_{t,j}d_{n,j}=\sum_{p=0}^{t}\sum_{j=1}^sb_{p,j}d_{n+t-p,j}$$
\end{proof}
\subsubsection{诺特环上的极小素理想}
\begin{enumerate}
	\item 和极大理想一样,按照Zorn引理可验证包含一个理想的全体素理想在包含序下总存在极小元.本质上是验证一个素理想的降链的交仍然是一个素理想.
	\item 在诺特环上可以说明极小素理想的个数总是有限的,另外按照取商可以看出这个结论等价于讲对任意理想,包含这个理想的全部素理想的极小元个数是有限的.
	\begin{proof}
		
		设诺特交换环$R$上全体不满足这个性质的理想构成的集合为$S$,假设$S$非空,诺特性说明$S$中有极大元$I$.那么$I$自身必然不会是素理想,否则包含它的素理想的极小元恰有一个,就是它自身.于是存在环$R$中的两个元$a,b\not\in I$使得$ab\in I$.记$J_1=I+(a)$和$J_2=I+(b)$.现在任取包含$I$的一个极小素理想$P$,从$ab\in I\subseteq P$得到$a,b$中至少一个属于$P$,不妨设$a\in P$,那么$J_1\subseteq P$,倘若存在素理想$Q$满足$J_1\subseteq Q\subseteq P$,那么$I\subseteq Q\subseteq P$,按照$P$的极小性得到$P=Q$,这说明$I$的极小素理想要么是$J_1$的极小素理想要么是$J_2$的极小素理想,于是这导致$I$的极小素理想集合包含于两个有限集合的并,于是$I$的极小素理想集合是有限集合,这矛盾,于是$S$是空集.
	\end{proof}
    \item 极小素理想总是由零因子构成.取环$R$的一个极小素理想$P$,考虑局部化$S=R_P$,它恰有一个素理想$P_P$.于是这就是$S$的幂零根.于是对任意$x\in P$,存在$s\in R-P$有$sx^n=0$,取$n$是最小的满足这个等式的正整数,那么$(sx^{n-1})x=0$说明$x$是零因子.
\end{enumerate}
\subsubsection{模的长度}

$R$模$M$的合成链是指这样一列严格包含的子模列$0=M_n\subsetneqq M_{n-1}\subsetneqq\cdots\subsetneqq M_0=M$,满足$M_i/M_{i+1},0\le i\le n-1$都是单模(单模约定非零).如果模存在合成链,就称它是有限长度模.一个严格包含链的长度约定为严格包含号的个数.借助下文的引理,可说明有限长度模上不同合成链的长度相同,它称为这个模的长度,记作$l(M)$.
\begin{enumerate}
	\item 如果模$M$没有合成链,约定$l(M)=\infty$.
	\item 于是模$M$是有限长度模当且仅当$l(M)$是有限自然数.
	\item $l(M)=0$当且仅当$M=0$.
\end{enumerate}
	
引理.任取一个合成链$0=M_r\subsetneqq M_{r-1}\subsetneqq\cdots\subsetneqq M_0=M$.如果有严格包含的子模链$N_s\subsetneqq N_{s-1}\subsetneqq\cdots\subsetneqq N_0=M$,那么有$s\le r$.
\begin{proof}
	
	对$r$归纳.若$r=1$,那么$M$是$R$单模,它的子模只有零模和自身,于是此时$s\le1$.假设$r\ge2$,取$M'=M/M_{r-1}$,$N_i'=(N_i+M_{r-1})/M_{r-1}\cong N_i/M_{r-1}\cap N_i$.于是得到$M'$的子模链$N_s'\subseteq N_{s-1}'\subset\cdots\subseteq N_0'=M'$.其中相邻两个子模的商模为$N'/N_{i+1}'=(N_i+M_{r-1})/(N_{i+1}+M_{r-1})$,它满足如下短正合列:
	$$\xymatrix{0\ar[r]&(N_i\cap M_{r-1})/(N_{i+1}\cap M_{r-1})\ar[r]& N_i/N_{i+1}\ar[r]&N_i'/N_{i+1}'\ar[r]&0}$$
	
	按照$M_{r-1}$是单模,说明$(N_i\cap M_{r-1})/(N_{i+1}\cap M_{r-1})$要么是零模要么同构于$M_{r-1}$.是零模的情况等价于$M_{r-1}\subseteq N_{i+1}$或者$M_{r-1}\cap N_i=\{0\}$.是$M_{r-1}$的情况等价于$M_{r-1}\subseteq N_i$且$M_{r-1}\cap N_{i+1}=\{0\}$.于是$N_i'/N_{i+1}'=0$的情况当且仅当短正合列第二项和第三项同构,按照$N_{i+1}\subsetneqq N_i$说明第三项不为零,于是$N_i'/N_{i+1}'=0$当且仅当$M_{r-1}\subseteq N_i$且$M_{r-1}\cap N_{i+1}=\{0\}$,于是至多仅有一个指标$i$使得$N_i'/N_{i+1}'=0$.
	
	最后,倘若得$N_i'/N_{i+1}'$均不为0,那么$N_s'\subseteq N_{s-1}'\subset\cdots\subseteq N_0'=M'$是严格升链,它的长度是$s$,归纳假设说明$s\le r-1$,于是$s\le r$.倘若恰好存在一个指标$i$使得$N_i'/N_{i+1}'=0$,那么这个关于$N_i'$的严格升链在划去一项的情况下是严格升链,于是按照归纳假设得到$s-1\le r-1$,这得到$s\le r$.完成证明.
\end{proof}

长度的等价定义.模$M$的长度就是它的全部严格包含的子模链的长度的上确界.特别的,$l(M)=\infty$当且仅当$M$存在任意长度的链.另外有限长度模上的每个链都可以延拓为合成链.
\begin{proof}
	
	假设$M$是有限长度模,引理\ref{lem:模的长度的引理}就说明$l(M)$就是全部链长度的上确界.现在假设$M$没有合成链,我们断言这个上确界是$\infty$.若否,设这个上确界是有限数$n$,于是存在一个链$M=M_0\supsetneqq M_1\supsetneqq\cdots\supsetneqq M_n=0$.按照$M$没有合成链,说明某个$M_i/M_{i+1}$不是单模,于是按照子模对应定理,存在$M$的子模$M_{i+1}\subsetneqq M_0\subsetneqq M_i$.于是上述链会延长为长度$n+1$的链,这和上确界是$n$矛盾.
\end{proof}

长度关于正合列的可加性.这个证明只要注意到$M'$的合成链和$M''$的合成链分别在$M$中的像和原像可以拼接为$M$的一个合成链.
\begin{enumerate}
	\item 如果有模的短正合列$0\to M'\to M\to M''\to0$,那么有$l(M)=l(M')+l(M'')$.特别的,总有$l(M)=l(N)+l(M/N)$.
	\item 另外按照长度为零当且仅当模是零模,说明有限长度模的真子模和非平凡商模的长度是严格小于它的长度的.
	\item 按照长正合列总可以拆为若干短正合列,可证明如果有长正合列$0\to M_1\to\cdots\to M_n\to0$,那么有等式$\sum_{i=1}^n(-1)^il(M_i)=0$.
\end{enumerate}

有限长度模的等价描述.$M$是有限长度模当且仅当它即是诺特模又是阿廷模.
\begin{proof}
	
必要性是直接的,否则会找到任意长度的严格升链或降链,导致$l(M)=\infty$.充分性,记$M=M_0$,按照升链条件可以取极大真子模$M_1$,再取$M_1$的极大真子模$M_2$,由此得到一个降链,再按照降链条件得到这个链有限长度,于是此时终端必然是零模,这就构造出一个有限长度的合成链.
\end{proof}

长度视为线性空间维数的推广.给定$k$线性空间$V$,那么$V$是有限长度$k$模当且仅当它是有限维线性空间,并且此时模的长度恰好是它作为线性空间的维数.
\begin{proof}
	
	线性空间的一维子空间是单模,于是任取一组基$\{e_1,e_2,\cdots,e_n\}$,得到$0\subsetneqq\langle e_1\rangle\subsetneqq\langle e_1,e_2\rangle\subsetneqq\cdots\subsetneqq \langle e_1,e_2,\cdots,e_n\rangle=V$是合成链,于是$V$是有限长度模.另外此时长度就是空间维数.
	
	设$V$是有限长度模,假设它不是有限维线性空间,可取一组可数的线性无关组$\{e_n,n\ge1\}$,记$U_n=\langle e_1,e_2,\cdots,e_n\rangle$,那么$U_n$构成了严格包含的长度任意的子模链,这和有限长度条件矛盾.
\end{proof}

有限长度模总是有限生成模.
\begin{proof}
	
	对长度$n$归纳,首先$n=0$的时候模$M=0$没什么需要证的.假设对长度$n-1$的模成立,取长度为$n$的模$M$,取一个合成链$M=M_0\supsetneqq M_1\supsetneqq\cdots\supsetneqq M_n=0$,按照归纳假设$M_1$是有限生成模,按照$M/M_1$是单模,它被任意非零元生成,于是任取$M_1$的生成元集,并上一个$M-M_1$中的元就是$M$的生成元集,完成归纳.
\end{proof}
\subsubsection{Jordan-H\"older定理}

给定模$M$的一个合成链,称相邻两项的商这个单模为合成链的合成因子,Jordan-H\"older定理说的是,有限长度模$M$的任意两个合成链的合成因子在不计顺序意义下是相同的,并且每个合成因子出现的个数是相同的,于是有限长度模被有限个计重数意义下的单模完全决定.
\begin{enumerate}
	\item 为证明合成因子是相同的,我们来说明有限长度模的每个合成链的合成因子在不计重数意义下和支集$\mathrm{Supp}(M)$是一一对应的,即合成因子$M_i/M_{i+1}$对应于它的零化子$\mathrm{Ann}(M_i/M_{i+1})$,这个零化子就是单模$M_i/M_{i+1}$同构的唯一的$R/m$中的$m$.
	\begin{proof}
		
		设$M$有合成链$M=M_0\supsetneqq M_1\supsetneqq\cdots\supsetneqq M_r=0$,任取素理想$p$,得到链$(M)_p=(M_0)_p\supset(M_1)_p\supset\cdots\supset(M_r)_p$.如果记$M_{i-1}/M_i\cong R/m_i$,其中$m_i$是极大理想,那么$(R/m_i)_p$在$p\not=m_i$时是零;在$p=m_i$时是$R/m_i$.于是$M_p\not=0$当且仅当某个$(M_{i-1}/M_i)_p=(M_{i-1})_p/(M_i)_p$不为零,这等价于$p$是某个$m_i$.于是$\mathrm{Supp}(M)$是$m_i$中那些两两不同的极大理想构成的集合.
	\end{proof}
    \item 特别的,上一条证明里说明有限长度模的支集中的素理想都是极大理想.
    \item 对于有限长度模$M$,有模同构$M\cong\prod_{m\in\mathrm{Supp}(M)}M_m$.
    \begin{proof}
    	
    	取典范映射$\varphi:M\to\prod_{m\in\mathrm{Supp}(M)}M_m$.按照同构是局部性在,为证明它是同构,只要说明对每个极大理想$p$,它诱导的映射$\varphi_p$是同构.但是有$\varphi_p:M_p\to(\prod_mM_m)_p=M_p$明显是同构.
    \end{proof}
    \item 最后说明对于不同的合成链,同一个合成因子的重数相同,为此只需证明这个重数恰好就是$M_p$的长度,这里$p$是该合成因子对应的极大理想.
    \begin{proof}
    	
    	任取$M$的合成链$M=M_0\supsetneqq M_1\supsetneqq\cdots\supsetneqq M_r=0$,设$p$是一个合成因子对应的素理想,得到$(M)_p=(M_0)_p\supset(M_1)_p\supset\cdots\supset(M_r)_p$.这里$M_i/M_{i+1}\cong R/p$等价于$(M_i/M_{i+1})_p\not=0$,等价于$(M_{i+1})_p\subsetneqq(M_i)_p$.于是取局部化后划去相等的包含号,每个严格包含号对应着一个合成因子$R/p$.这说明$l(R_p)$恰好就是合成因子$R/p$出现的重数.
    \end{proof}
    \item 我们在后文会证明对于诺特环上的有限生成模,它的支集和伴随素理想集的极小元相同,这里我们说明了有限长度模上支集由极大理想构成.而有限长度模必然是有限生成模.这就证明了:对于诺特环上有限生成模,它的支集和伴随素理想集合是相同的(有限集合),并且都由极大理想构成.
\end{enumerate}
\subsubsection{阿廷环}
\begin{enumerate}
	\item 阿廷环是零维环,即它的每个素理想都是极大理想.
	\begin{proof}
		
		给定阿廷环$R$的一个素理想$P$,需要说明的是$R/P$是域.首先这个商环也是阿廷环,并且它是一个整环.于是问题归结为阿廷整环总是域.任取阿廷整环的一个非零元$r$,那么按照降链条件,存在一个正整数$n$满足$(r^n)=(r^{n+1})$,导致存在环中的元$s$满足$r^n=r^{n+1}s$,按照整环条件,说明$rs=1$,于是$r$是单位,于是阿廷整环是域.
	\end{proof}
    \item 阿廷环只有有限个极大理想.为此我们来证明存在有限个极大理想的乘积是零,一旦这个结论得证,任取一个极大理想,它包含了这有限个极大理想的乘积,于是导致它必然和其中一个极大理想相同.
    \begin{proof}
    	
    	考虑极大理想的有限乘积的集合,极小条件说明这个集合里存在一个极小元$m'=m_1m_2\cdots m_r$.那么对任一极大理想$m$有$m'=mm'\subseteq m$.另外由于$m'^2\subseteq m'$同样是有限个极大理想的乘积,极小性得出$m'=m'^2$.
    	
    	现在假设$m'$不为零,记$S$是全体不零化$m'$的理想构成的集合,即$S=\{n\mid nm'\not=0\}$,那么$S$非空,因为$m'\in S$.取$S$中的极小元$n$.那么有$(nm')m'=nm'\not=0$,按照$nm'\subseteq n$以及$n$的极小性说明$nm'=n$.按照定义存在一个元$f\in n$使得$fm'\not=0$,但是因为$n$极小,得到$n=(f)$,那么按照$nm'=n$得到存在一个$g\in m'$使得$f=fg$,于是$(1-g)f=0$,但是因为$g$在每个极大理想中,于是$1-g$是单位,那么$f=0$,这矛盾.
    \end{proof}
    \item Akizuki-Hopkins定理.$R$是阿廷环当且仅当$R$是诺特的并且维数零.于是尽管在模上阿廷条件和诺特条件是独立的,在环上阿廷环总是诺特环,也就等价于环作为自身模是有限长度模.这还说明了环作为自身模是有限长度模当且仅当环是阿廷环.
    \begin{proof}
    	
    	必要性,已经证明了阿廷环维数是零,现在需要证明阿廷环是诺特环.为此只需证明它作为自身的模是有限长度模.为此我们构造一个链使得相邻子模的商是有限长度模,于是这个链可以添加有限个子模成为一个合成链.已经证明过阿廷环存在有限个极大理想的乘积为零,不妨记作$m_1m_2\cdots m_n=0$.那么有$R$的链$0=m_1m_2\cdots m_n\subseteq m_1m_2\cdots m_{n-1}\subset\cdots\subseteq m_1\subseteq A$.现在$M/Mm_k=m_1m_2\cdots m_{k-1}/m_1m_2\cdots m_k$是域$R/m_k$上的线性空间.其中$M$是$R$上的阿廷模,说明$M/m_kM$是$R/m_k$上的阿廷模,但是线性空间的情况下等价于$M/m_kM$是有限长度的$R/m_k$上的模.我们断言此时$M/m_kM$作为$R$模也是有限长度的,否则有任意长度的$R$的子模链$0=M_r/m_kM\subsetneqq M_{r-1}/m_kM\subsetneqq\cdots\subsetneqq M_0/m_kM$.注意这个子模链中每个子模也是$R/m_k$上的模,这就和$M/m_kM$是$R/m_k$上的有限长度模矛盾.
    	
    	充分性,我们先来证明添加一个条件下结论成立,即如果$R$是诺特零维环,并且只有有限个极大理想,那么它是阿廷环.先设全部极大理想为$m_1,m_2,\cdots,m_r$,按照条件它们也就是全部素理想.于是Jacobson根和幂零根相同,记作$I$.按照两个不同的极大理想是互素的,说明$I=\cap m_i=\prod m_i$.诺特性说明每个$m_i$是有限生成的,于是存在一个足够大的正整数$n$使得$I^n=0$.现在构造如下子模链,模仿必要性中的证明,得到相邻两个子模的商是$R$上有限长度模,这得到$R$作为自身模是阿廷的.
    	$$0=I^n\subset\cdots Im_1\subseteq I=m_1m_2\cdots m_r\subset\cdots\subseteq m_1m_2\subseteq m_1\subseteq R$$
    	
    	于是完整的证明充分性需要说明诺特零维环只有有限个素理想.这只要注意到诺特环上极小素理想只有有限个,而零维意味着每个素理想都是极大理想也都是极小素理想.
    \end{proof}
    \item 阿廷环的结构定理.每个阿廷环可以分解为有限个阿廷局部环的直积,并且这个分解是在同构意义下唯一的.
    \begin{proof}
    	
    	存在性.我们之前证明过阿廷环中只有有限个极大理想,并且在证明中我们论证了阿廷环中存在有限个极大理想的乘积是零.设$A$的全部极大理想是$\{m_1,m_2,\cdots,m_r\}$,于是存在某个正整数$d$使得$(m_1m_2\cdots m_r)^d=0$.由于$i\not=j$时$m_i^d+m_j^d=1$,于是满足中国剩余定理的条件,于是有$A=A/(m_1m_2\cdots m_r)^d\cong\prod_{1\le i\le r}A/m_i^d$.再注意到每个$A/m_i^d$是具有唯一素理想的诺特环,于是它是局部阿廷环.
    	
    	唯一性.上一段已经给出了一个分解,我们来证明对任意的同构$A\cong\prod_{1\le i\le s}A_i$,其中每个$A_i$是局部阿廷环,总有$A_i$差一个排序意义下同构于$A/m_i^d$.按照$A_i$是局部阿廷环,它唯一的素理想(于是极大)记作$n_i$.环的乘积$\prod_{1\le i\le s}A_i$的极大理想必然具有形式$A_1\times\cdots\times n_i\times\cdots\times A_s$.于是$A$恰有$s$个不同极大理想,这证明了$s=r$.不妨设$m_i$在同构下对应的极大理想是$A_1\times\cdots\times n_i\times\cdots\times A_r$,于是$m_i^d=A_1\times\cdots\times n_i^d\times\cdots\times A_r$.于是从$\prod_{1\le i\le r}m_i^d=0$得到每个$n_i^d=0$.于是有$A/m_i^d\cong(\prod_{1\le i\le r}A_i)/(A_1\times\cdots\times n_i^d\times\cdots\times A_r)\cong A_i$.
    \end{proof}
\end{enumerate}

给定阿廷环$A$,设$M$是$A$模,那么如下四个条件等价:
\begin{enumerate}
	\item $M$是有限长度模.
	\item $M$是诺特模.
	\item $M$是有限生成模.
	\item $M$是阿廷模.
\end{enumerate}
\begin{proof}
	
	按照阿廷环也是诺特环,于是1和2和3的等价性是直接的.而3推4也是直接的,于是只需验证4推出3.假设结论不成立,设$S$是全体$M$的非有限生成子模构成的集合,那么这是一个非空集合.按照极小条件,可取$S$中的极小元$N$.那么$N$作为$M$的子模同样是阿廷模,并且按照定义$N$的每个真子模都是有限生成模.我们断言$I=\mathrm{Ann}(N)$是一个素理想:任取$a,b\in A$满足$ab\in I$.如果$a\not\in I$,取$N'=\{x\in N\mid ax=0\}$,这是$N$的一个真子模,于是它是有限$A$模.下面考虑短正合列$0\to N'\to N\to aN\to0$,按照$N'$有限生成,$N$非有限生成,说明$aN$非有限生成,而它是$M$的子模,按照$N$是$S$的极小元,这得到$N=aN$,因而有$b\in I=\mathrm{Ann}(N)$.于是我们得到$I$是素理想.
	
	下面按照$A$是阿廷环,于是$I$同样是一个极大理想,于是$N$可视为$A/I$阿廷模,而这里$A/I$是一个域,其上的阿廷模应该是有限生成模,这矛盾,于是$M$是有限生成模.
\end{proof}
\newpage
\subsection{准素分解}
\subsubsection{模的支集}

给定$R$模$M$,$M$的支集$\mathrm{Supp}(M)$定义为$\mathrm{Spec}(R)$的子集,由满足$M_P\not=0$的素理想$P$构成.首先$M_P=0$等价于讲,对任意$m\in M$,存在$S=R-P$中的元$s$使得$sm=0$,于是$M_P\not=0$等价于讲存在$m\in M$,使得对任意$S=R-P$中的元$s$恒有$sm\not=0$.这等价于讲$\mathrm{Ann}(m)$和$S$不交,也即$\mathrm{Ann}(m)\subseteq P$.于是素理想$P\in\mathrm{Supp}(M)$当且仅当存在$M$中非零元$m$,使得$\mathrm{Ann}(m)\subseteq P$.下面给出支集的一些性质:
\begin{enumerate}
	\item 非零模$M$上支集总是非空的,任取模上非零元$m$,那么$\mathrm{Ann}(m)$是一个真理想,于是可取包含它的极大理想$P$,那么按照等价定义得到$P$在$M$的支集中.
	\item 另外按照等价定义,说明如果$P\in\mathrm{Supp}(M)$,那么$V(P)\subset\mathrm{Supp}(M)$.
	\item 对理想$I$,总有$\mathrm{Supp}(A/I)=V(I)$.
	\item 给定$R$模的短正合列$0\to L\to M\to N\to0$,按照局部化保正合性,对任意素理想$P$总有$0\to L_P\to M_P\to N_P\to0$.于是$M_P\not=0$当且仅当$L_P\not=0$和$N_P\not=0$中至少一个成立,这说明了$\mathrm{Supp}(M)=\mathrm{Supp}(L)\cup\mathrm{Supp}(N)$.
	\item 上一条说明模$M$的子模和商的支集都是$M$支集的子集,另外如果$N$是$M$的子模,就有$\mathrm{Supp}(M)=\mathrm{Supp}(N)\cup\mathrm{Supp}(M/N)$.
	\item 如果有子模的和式$M=\oplus_iM_i$,那么有$\cup_i\mathrm{Supp}(M_i)=\mathrm{Supp}(M)$.首先按照上一条说明并是包含于$M$支集的,对于另一侧的包含关系:按照分式化保和,如果素理想$P\not\in\cup_i\mathrm{Supp}(M_i)$,那么$(M_i)_P=0,\forall i$,于是得到$M_P=0$,导致$P\not\in\mathrm{Supp}(M)$.
	\item 对支集中的每个素理想$P$,按照定义存在$m\in M$使得$\mathrm{Ann}(M)\subset\mathrm{Ann}(m)\subseteq P$.这说明总有$\mathrm{Supp}(M)\subseteq V(\mathrm{Ann}(M))$.我们断言当$M$是有限生成模的时候这个包含关系取等,于是此时支集总是素谱上的闭子集.
	\begin{proof}
		
		任取素理想$P$满足$M_P=0$,取$m$的有限的生成元集为$\{m_i\}$,那么按照$M_P=0$知存在$s_i\in R-P$使得$s_im_i=0$,于是记$s=\prod_is_i\not=0$就有$sm=0,\forall m\in M$成立,这说明$\mathrm{Ann}(M)\cap(R-P)$非空集,于是$P\not\in V(\mathrm{Ann}(M))$.
	\end{proof}
	\item 按照张量积和局部化的关系式$(M\otimes_RN)_P\cong M_P\otimes_{R_P}N_P$,以及当$M,N$是有限生成模的时候,$M\otimes_AN\not=0$当且仅当$M,N$均不为零模.就得到$\mathrm{Supp}(M\otimes_RN)\subset\mathrm{Supp}(M)\cap\mathrm{Supp}(N)$,并且在$M,N$都是有限生成$R$模的时候包含号取等.
	\item 分式化的支集.给定环$R$上的乘性闭子集$S$,把$\mathrm{Spec}(S^{-1}R)$视为$\mathrm{Spec}(R)$的子集,此即$R$中所有和$S$不交的素理想构成的子集.那么有$\mathrm{Supp}(S^{-1}R)=\mathrm{Supp}(R)\cap\mathrm{Spec}(S^{-1}R)$.
	\begin{proof}
		
		任取$S^{-1}R$支集中的素理想$S^{-1}P$,那么存在元$a/s,a\in P,s\in S$使得$\mathrm{Ann}(a/s)\subseteq S^{-1}P$.我们来验证$\mathrm{Ann}(a)\subseteq P$:如果$r\in R$满足$ra=0$,那么有$(r/1)(a/s)=0$,于是$r/1\in S^{-1}P$,于是$r\in P$.
		
		反过来任取$R$的和$S$无交的素理想$P$,假设它在支集中,即存在$\mathrm{Ann}(a)\subseteq P$,我们来验证$\mathrm{Ann}(a/1)\subseteq S^{-1}P$:如果$r/s\in S^{-1}R$满足$ra/s=0$,此即存在$s'\in S$使得$s'ra=0\in P$,于是$s'r\in P$,而从$s'$和$P$不交得到$r\in P$,于是$r/s\in S^{-1}P$.
	\end{proof}
	\item 如果$M$是有限生成$A$模,而$I$是一个理想,那么$\mathrm{Supp}(M/IM)=V(I+\mathrm{Ann}(M))$.事实上$\mathrm{Supp}(M/IM)=\mathrm{Supp}(A/I\otimes_AM)=\mathrm{Supp}(A/I)\cap\mathrm{Supp}(M)=V(I)\cap V(\mathrm{Ann}(M))=V(I+\mathrm{Ann}(M))$.
	\item 如果$f:A\to B$是环同态,而$M$是有限生成$A$模,那么有$\mathrm{Supp}(B\otimes_AM)=(f')^{-1}(\mathrm{Supp}(M))$.
	\begin{proof}
		
		首先有和式$M=\sum_{x\in M}Ax$,于是有$\mathrm{Supp}(M)=\cup_{x\in M}\mathrm{Supp}(Ax)$.于是得到$\mathrm{Supp}(B\otimes_AM)=\cup_{x\in M}\mathrm{Supp}(B\otimes_AAx)$.于是我们只需验证$\mathrm{Supp}(B\otimes_AAx)=(f')^{-1}(\mathrm{Supp}(Ax))$.
		
		$Ax$是一个循环模,于是它同构于$A$模$A/I$,其中$I$是某个理想,那么有$B\otimes_AAx\cong B/I^e$.而我们知道$\mathrm{Supp}(A/I)=V(I)$和$\mathrm{Supp}(B/I^e)=V(I^e)$,于是问题归结为$(f')^{-1}(V(I))=V(I^e)$,这是我们证明过的.
	\end{proof}
\end{enumerate}
\subsubsection{零化子}

给定交换环$R$上的模$M$,任取$M$的子集$S$,$S$的零化子定义为$\mathrm{Ann}(S)=\{r\in R\mid\forall s\in S,rs=0\}$.这是$R$的一个理想.如果模$M$满足$\mathrm{Ann}(M)=\{0\}$,则称$M$是忠实模.下面给出一些基本性质:
\begin{enumerate}
	\item 如果$S\subseteq T$,那么有$\mathrm{Ann}(T)\subset\mathrm{Ann}(S)$.
	\item 给定$M$的子集$S$,设它生成的子模为$N$,那么有$\mathrm{Ann}(S)=\mathrm{Ann}(N)$.事实上一方面$S\subseteq N$得到$\mathrm{Ann}(N)\subset\mathrm{Ann}(S)$,另一方面如果$r\in R$零化了$S$中每个元,那么也零化了任意一个$\sum r_is_i,r_i\in R,s_i\in S$,于是$\mathrm{Ann}(N)\subset\mathrm{Ann}(S)$.
	\item 给定$R$模$M$,那么$M$还是$R/\mathrm{Ann}(M)$模,即约定$(r+\mathrm{Ann}(M))m=rm$.此时$M$作为$R/\mathrm{Ann}(M)$模是自动忠实的.
	\item 零化子可视为双线性函数上正交补概念的特例.给定$R$模上的双线性型$F:M\times N\to P$,那么$S\subseteq M$的零化子就是$\mathrm{Ann}(S)=\{n\in N\mid\forall s\in S,F(s,n)=0\}$.那么$\mathrm{Ann}(-)$满足Galois关系,即零化子是反序的取值是子模的映射;满足二次复合使点集变大;满足一次和三次复合固定点集.另外注意到如果取双线性型是$V\times V\to K$的线性空间上的内积,那么零化子这个概念实际上就是子集的正交补这个概念.
\end{enumerate}
\subsubsection{伴随素理想}

给定$R$模$M$,$M$的伴随素理想是指可以表示为$\mathrm{Ann}(x)=\{r\in R\mid rx=0\},x\in M$的素理想.那么如果$P$是$M$的伴随素理想,就有模同态$f:R\to M,r\mapsto rx$,按照同构定理得到$R/P$同构于$M$的一个子模.另一方面,如果存在一个素理想$P$使得$R/P$同构于$M$的一个子模,取$1+P$对应的子模中的元$x$,那么有$\mathrm{Ann}(x)=P$.这就说明:素理想$P$是$M$的伴随素理想当且仅当$M$存在一个子模同构于$R/P$.$M$的全部伴随素理想的集合记作$\mathrm{Ass}(M)$.

存在性问题.即便对于交换局部环,也会存在非空有限生成模不存在伴随素理想.不过链条件下伴随素理想是总存在的:设$R$是诺特环,那么$R$模$M$非空当且仅当$\mathrm{Ass}(M)$非空.事实上诺特性说明$\mathscr{A}=\{\mathrm{Ann}(x)\mid x\in M\backslash\{0\}\}$在包含序下存在极大元,而这样的极大元总是素理想,于是它就是$M$的伴随素理想.
\begin{proof}
	
	设$I=\mathrm{Ann}(m)$是一个极大元,下证$I$是素理想.假设有$ab\in I$,其中$a,b\in R$,那么有$abm=0$,假设$a\not\in I$,那么$am\not=0$,那么$\mathrm{Ann}(m)\subset\mathrm{Ann}(am)$,于是按照极大性得到这个包含关系实际取等号,于是按照$abm=0$得到$b\in\mathrm{Ann}(am)=\mathrm{Ann}(m)$,得到$bm=0$,得到$b\in I$,于是$I$是素理想.
\end{proof}

关于伴随素理想的一些性质:
\begin{enumerate}
	\item 如果$R$模$M$本身已经具有形式$R/P$,其中$P$是素理想,任取它的非零元$s+P$,倘若$r\in R$零化了这个元,等价于$rs\in P$,但是按照$P$是素理想且$s\not\in P$,导致$r\in P$,这就说明任取$M$的非零元$m$,恒有$\mathrm{Ann}(m)=P$,这导致$\mathrm{Ass}(R/P)=\{P\}$.
	\item 设$N$是$M$的子模,那么$m\in N$视为$N$中元和视为$M$中元不影响它零化子构成的理想,这说明恒有$\mathrm{Ass}(N)\subset\mathrm{Ass}(M)$.
	\item 设$P$是$M$的伴随素理想,那么存在单同态$R/P\to M$,记像集为$E$,倘若$E\cap N$不为0,任取一个非零元$n$,按照上一条说明$\mathrm{Ann}(n)=P$,否则有$E\cap N=0$,这导致$R/P\to M\to M/N$是单射,于是$P\in\mathrm{Ass}(M/N)$.综上得到了:
	$$\mathrm{Ass}(M)\subset\mathrm{Ass}(N)\cup\mathrm{Ass}(M/N)$$
	\item 上两条可以直接得出,如果$M=\oplus_{1\le i\le r}M_i$,那么$\mathrm{Ass}(M)=\cup_i\mathrm{Ass}(M_i)$.特别的,在短正合列分裂的时候,上一条最后的关系式里包含号是取等的.特别的,如果$M$是有限自由$A$模,我们有$\mathrm{Ass}_A(A)=\mathrm{Ass}_A(M)$.
\end{enumerate}

和支集不同的是,$\mathrm{Ass}(M)\subset\mathrm{Ass}(N)\cup\mathrm{Ass}(M/N)$并不一定取等号,这里主要问题在于$\mathrm{Ass}(M/N)$未必包含于$\mathrm{Ass}(M)$.例如$\mathbb{Z}$作为$\mathbb{Z}$模的伴随素理想只有$(0)$,但是它的商$\mathbb{Z}/2\mathbb{Z}$的伴随素理想只有$(2)$.

同样和支集不同的是,如果$p\in\mathrm{Ass}(M)$不代表$V(p)\subset\mathrm{Ass}(M)$.例如取$R=k[x,y]$是域上的二元多项式环,取$R$模$M=R\oplus R/(x,y)$,那么$\mathrm{Ass}(M)=\mathrm{Ass}(R)\cup\mathrm{Ass}(R/(x,y))=\{(0),(x,y)\}$,但是它并不包含素理想$(x)$.

关于分式化:
\begin{enumerate}
	\item 给定$R$上的乘性闭子集$S$,那么$\mathrm{Spec}(S^{-1}R)$可视为$\mathrm{Spec}(R)$的子集,此时$S^{-1}R$模$N$视为$R$模时伴随素理想是不变的,即$\mathrm{Ass}_{S^{-1}R}(N)=\mathrm{Ass}_R(N)$.这里把$N$视为$R$模是指$rn$约定为$(r/1)n$.
	\begin{proof}
		
		一方面如果有伴随素理想$S^{-1}P=\mathrm{Ann}(n)$,倘若$r\in R$满足$(r/1)n=0$,等价于$r/1\in S^{-1}P$等价于$r\in P$,于是$N$视为$R$模的时候有$\mathrm{Ann}(n)=P$.
		
		另一方面如果$N$视为$R$模有伴随素理想$P=\mathrm{Ann}(n)$,倘若$(r/s)n=0$,等价于$(r/1)n=(s/1)(r/s)n=0$,等价于$r\in P$,等价于$r/s\in S^{-1}P$.注意此时有$P\cap S=\emptyset$,否则取一个$s$,得到$n=(s/s)n=(1/s)(s/1)n=0$矛盾.
	\end{proof}
	\item 如果$P$是$M$的伴随素理想并且$P\cap S=\emptyset$,那么$S^{-1}P$是$S^{-1}M$的伴随素理想;反过来如果$S^{-1}P$是$S^{-1}M$的伴随素理想,并且$P$是有限生成的,那么$P$是$M$的伴随素理想.
	\begin{proof}
		
		先设$P$是$M$的伴随素理想,于是存在单同态$R/P\to M$,这诱导了单同态$S^{-1}R/S^{-1}P=S^{-1}(R/P)\to S^{-1}M$.所以倘若$P\cap S=\emptyset$,那么$S^{-1}P$是$S^{-1}R$的素理想,这导致$S^{-1}P$是$S^{-1}M$的伴随素理想.
		
		反过来如果$S^{-1}P$是$S^{-1}M$的伴随素理想,那么有$S^{-1}P=\mathrm{Ann}(m/t)$.记$P=(x_1,x_2,\cdots,x_n)$,那么有$x_im/t=0$,也即存在$s_i\in S$使得$s_ix_im=0$,取$s=\prod_is_i$,那么$x_i\in\mathrm{Ann}(sm)$,导致$P\subset\mathrm{Ann}(sm)$.
		
		任取$b\in\mathrm{Ann}(sm)$,即$bsm=0$,导致在$S^{-1}R$中有$bsm/st=0$,这说明$b/1\in S^{-1}P$,于是$b\in P$,这就得到$P=\mathrm{Ann}(sm)$.
	\end{proof}
	\item 于是如果$R$是诺特环,$R$模$M$就满足$\mathrm{Ass}(S^{-1}M)=\mathrm{Ass}(M)\cap\mathrm{Spec}(S^{-1}R)$
\end{enumerate}

诺特环$R$上有限生成模$M$的伴随素理想个数是有限的.为此只需说明这样的模$M$存在子模链$0=M_0\subseteq M_1\subset\cdots\subseteq M_n=M$,满足相邻子模的商$M_i/M_{i-1}\cong R/P_i$,其中$P_i$是一个素理想.一旦这个命题得到证明,从$\mathrm{Ass}(M)\subset\mathrm{Ass}(N)\cup\mathrm{Ass}(M/N)$归纳就得到$\mathrm{Ass}(M)\subset\cup_{1\le i\le n}\mathrm{Ass}(M_i/M_{i-1})\subset\{P_1,P_2,\cdots,P_n\}$.
\begin{proof}
	
	记全体满足这个条件的$M$的子模构成的集合为$S$,那么$S$非空,因为包含零模.按照诺特条件,$S$中存在包含序下的极大元$N$,假设$M/N\not=0$,那么按照诺特环上模非零当且仅当伴随素理想存在,可取$M/N$的子模$M'/N$同构于$R/P$,其中$P$是一个素理想,于是$N\subsetneqq M'$,这导致$M'$也满足条件,和极大性矛盾,这说明$M=N$.
\end{proof}

伴随素理想和支集的一些关系:
\begin{enumerate}
	\item 伴随素理想集合是支集的子集.我们解释过$P$在$M$的支集中当且仅当存在$m\in M$使得$\mathrm{Ann}(m)\subseteq P$,于是如果$P$本身就是某个$\mathrm{Ann}(m)$,自然有落在支集中.
	\item 如果$M$是诺特环$R$的模,那么它的支集中的极小元和伴随素理想中的极小元是一致的.
	\begin{proof}
		
		一方面设$P$是有限集合$\mathrm{Ass}(M)$中的极小元,假设存在素理想$P'\subsetneqq P$满足$M_{P'}\not=0$,那么$\mathrm{Ass}(M)\cap\mathrm{Spec}(R_{P'})=\mathrm{Ass}(M_{P'})$是非空的,这就和$P$是$\mathrm{Ass}(M)$中的极小元矛盾.
		
		另一方面设$Q$是$M$支集中的极小元,就有$\mathrm{Supp}(M_Q)=\{Q\}$,而第一条说明$\mathrm{Ass}(M_Q)$是它的子集,从$M_Q\not=0$和$R$诺特就说明$\mathrm{Ass}(M_Q)=\{Q\}$,于是$Q\in\mathrm{Ass}(M)$.
	\end{proof}
	\item 上一条结合支集就是支集中全部极小元生成的素谱中的闭集的并,得到:如果$R$是诺特环,那么对$R$模$M$恒有$\cup_{q\in\mathrm{Ass}(M)}V(q)=\mathrm{Supp}(M)$.
\end{enumerate}

设$R$是诺特环,$M$是有限生成$R$模,就把上述$\mathrm{Supp}(M)$和$\mathrm{Ass}(M)$中一致的极小素理想称为孤立素理想,把不是孤立素理想的$M$的相伴素理想称为嵌入素理想.孤立素理想的意义在于提供了支集上的不可约分支分解:

已经说明了在条件下,$M$的孤立素理想个数有限,记作$P_1,P_2,\cdots,P_s$,并且支集中的每个素理想都包含了某个$P_i$,于是得到$\mathrm{Supp}(M)=\cup_{1\le i\le s}V(P_i)$.另外我们证明过$v(I)$不可约当且仅当$\sqrt{I}$是素理想,于是这里每个$V(P_i)$都是不可约的闭集,另外不存在$i,j$使得$P_i\in V(P_j)$,这说明这些不可约闭集是互相不包含的,于是这就是支集的不可约分支分解.

特别的,如果$R$是诺特环,取$R$作为自身的模,此时支集恰好就是整个素谱$\mathrm{Spec}(R)$,并且支集的极小元恰好就是$R$的极小素理想,它只有有限个,记作$P_1,P_2,\cdots,P_s$,于是此时有$\mathrm{Spec}(R)=\cup_{1\le i\le s}V(P_i)$是素谱的不可约分支分解.注意这里我们说明了诺特环上极小素理想只有有限个.

理想的情况.给定环$R$,它的理想就是$R$模$R$的子模,因而当我们讨论理想的准素分解时,实际上是模情况的特例,此时理想$I$的伴随素理想集合即模$R/I$的是素理想的零化子,也即形如$\mathrm{Ann}(x+I)$的素理想,也即形如$\sqrt{(I:x)}$的素理想.这里$(I:x)$表示的是理想$\{r\in R\mid rx\in I\}$.

理想情况的极小素理想.诺特环上理想$I$的极小素理想实际上就是包含$I$的素理想中的极小元:按照下文给出的准素分解,记$I=\cap_iq_i$,其中$q_i$是$p_i$准素理想,那么从$I\subseteq p$取根理想得到$\cap_ip_i\subseteq p$,这导致某个$p_i\subseteq p$,于是只要取$p$本身是包含$I$的极小素理想,就说明它必然是某个$p_i$.

下面给出一些例子:
\begin{enumerate}
	\item 设$R=k[x,y]$是域上的二元多项式环,任取非常数的多项式$f\in R$,记$R$模$M=R/(f)$,我们断言$M$没有嵌入素理想.取唯一分解$f=\lambda p_1^{e_1}p_2^{e_2}\cdots p_r^{e_r}$,其中$\lambda\in k-\{0\}$,$p_i$是两两不相伴(即不相差一个单位)的不可约多项式,$e_i$是正整数.任取$g\in A$,记$g'=g+(f)\in M$,那么有$\mathrm{Ann}(g')=(f/(f,g))$,于是它是素理想当且仅当$g=(f/p_i),1\le i\le r$.于是此时$M$仅有的伴随素理想是$\{P_1,P_2,\cdots,P_s\}$,其中$P_i=(p_i)$.并且它们都是孤立素理想.
	\item $R$的记号同上,这回取$R$模$N=R/(x^2,xy)$.记$x,y$在商中的像是$x'$和$y'$,那么有$\mathrm{Ann}(N)=(x)$,于是$\mathrm{Supp}(N)=V(x)\subset\mathrm{Spec}(R)$.这说明$N$恰有一个孤立素理想$p=(x)$.另外$\mathrm{Ann}(x')=(x,y)$是$R$的极大理想,于是它是嵌入素理想.现在说明$N$不存在其它的嵌入素理想,任取素理想$q\in V(x)$,那么$q$可对应于$k[x,y]/(x)\cong k[y]$的素理想,这是一个主理想整环,于是$q$对应于$(p(y))$,其中$p(y)\in k[y]$是首一的不可约多项式,于是$q=(x,p(y))$.假设$q=(x,p(y))$是某个$\mathrm{Ann}(g(x,y)+(x^2,xy))$,那么有$xg(x,y)\in (x^2,xy)$和$p(y)g(x,y)\in (x^2,xy)$.前者说明$g(x,y)$中每个单项式要么包含$x$要么包含$y$,后者说明$p(y)$的常数项必然是零,此时要想$p(y)$不可约只能有$p(y)=y$,综上得到此时只能有$q=(x,y)$.于是有$\mathrm{Ass}(M)=\{(x),(x,y)\}$.
\end{enumerate}
\subsubsection{准素分解}

零因子和幂零元.称$r\in R$是模$M$的零因子,如果存在$m\in M,m\not=0$使得$rm=0$.那么全体$M$的零因子构成的集合$D(M)=\cup_{m\in M,m\not=0}\mathrm{Ann}(m)$.称$r\in R$是模$M$上的幂零元,如果存在某个正整数$n\ge1$使得$r^nM=0$,于是全体$M$的幂零元构成的集合$\mathrm{nil}(M)=\sqrt{\mathrm{Ann}(M)}$,也即全体包含$\mathrm{Ann}(M)$的素理想的交.按照定义,幂零元总是零因子,因为任取幂零元$a$,设$n$是最小的正整数使得$a^nM=0$,那么存在某个$m\in M$使得$a^{n-1}m\not=0$,否则和$n$的最小性矛盾,这就导致$a(a^{n-1}m)=0$,于是$a$是零因子.

零因子和幂零元与支集和伴随素理想集的关系:
\begin{enumerate}
	\item 全部伴随素理想的并就是零因子集合.按照定义,$M$的伴随素理想中的元都是$M$的零因子,反过来任取零因子$r\not=0$,那么有$m\not=0,m\in M$使得$rm=0$,于是$r\in\mathrm{Ann}(m)$,按照上一段,包含$\mathrm{Ann}(m)$的形如$\mathrm{Ann}(x),x\not=0$的极大元是$M$的伴随素理想,这说明了全部$M$的伴随素理想的并恰好就是$M$的零因子集合$D(M)$.
	\item 支集中的素理想都是包含$\mathrm{Ann}(M)$的素理想,于是支集中全部素理想的交包含了幂零元集合$\mathrm{nil}(M)$.倘若$M$是有限生成的,我们证明过此时支集恰好就是$V(\mathrm{Ann}(M))$,于是此时恰好有$\cap_{P\in\mathrm{Supp}(M)}P=\mathrm{nil}(M)$.
	\item 另外这个交可以替换为支集中全部极小元的交,按照诺特环上支集的极小元和伴随素理想的极小元一致,这就得到对于诺特环上的有限生成模$M$,总有$\mathrm{nil}(M)=\cap_{P\in\mathrm{Ass}(M)}P$.
\end{enumerate}

准素子模.设$M$是$R$模,称$M$的真子模$N$是准素子模,如果$R$模$M/N$的幂零元和零因子是一致的,即有等式$\sqrt{\mathrm{Ann}(M/N)}=D(M/N)$.由于幂零元总是零因子,于是这个要求等价于讲$M/N$的所有零因子都是幂零元,也等价于讲从$x\not\in N$和$ax\in N$推出存在某个正整数$n$使得$a^nM\subseteq N$.注意按照定义准素子模必然是真子模,否则幂零元集合是整个环$R$,但是不存在零因子.

对于诺特环上的有限生成模$M$,取真子模$N$,有$M/N$也是有限生成的,它的零因子集合是全部伴随素理想的并,它的幂零元集合是全部伴随素理想的交,这说明:诺特环上有限生成模$M$的真子模$N$的伴随素理想恰好有一个$P$,当且仅当$N$是准素子模.此时称准素子模$N$是$P$准素的.$P$就是幂零元集合$\sqrt{\mathrm{Ann}(M/N)}$.
\begin{enumerate}
	\item 如果$N$是$p$准素子模,那么$p$是$V(\mathrm{Ann}(M/N))$的极小元.
	\item 另外这里的$\mathrm{Ann}(M/N)$是一个准素理想:若$ab\in\mathrm{Ann}(M/N)$,假设$a\not\in\mathrm{Ann}(M/N)$,于是存在一个$m+N\not=N$,使得$am\not\in N$,那么$bam=0$说明$b$是一个零因子,按照零因子都是幂零元,就得到$b\in\sqrt{Ann(M/N)}$,于是$\mathrm{Ann}(M/N)$是准素理想.
	\item 如果$M$是诺特环上的有限生成模,如果两个真子模$N,N'$都是$P$准素子模,按照嵌入$M/N\cap N'\to M/N\oplus M/N'$,得到$N\cap N'$同样是$P$准素子模.但是无限个$p$准素子模的交未必仍然是$p$准素子模,例如考虑理想的情况,$\mathbb{Z}$上$(2^n),n\ge1$都是2准素理想,但是它们的交是零理想,并不是2准素理想.
	\item 如果诺特环上有限生成模$M$的真子模$N$满足$P=\sqrt{\mathrm{Ann}(M/N)}$已经是极大理想,那么$N$自动是准素子模.这是因为$M/N$的支集是$V(\mathrm{Ann}(M/N))$,而这里的全部素理想的交是根$\sqrt{\mathrm{Ann}(M/N)}$,于是如果它本身是极大理想,说明支集恰有这一个元素构成,于是此时$N$是准素子模.
\end{enumerate}

准素理想.当$R$视为自身的模时,真子模即$R$的真理想$I$,此时$R/I$上的零因子即满足存在$S=R-I$中的元$s$使得$rs\in I$的$r$,$R/I$中的幂零元即满足存在某个正整数$n$使得$r^n\in I$的元$r$.于是$I$是准素理想等价于如下要求:若$xy\in I$,则要么$x\in I$要么存在某个正整数$n$使得$y^n\in I$.准素理想的一些基本性质:
\begin{enumerate}
	\item 如果环是诺特环,任取真理想$I$,$R/I$自然是有限生成模,于是$I$是准素理想当且仅当$\mathrm{Ass}(R/I)$恰由一个素理想构成,此时素理想就是$\sqrt{\mathrm{Ann}(R/I)}=\sqrt{I}$.注意直接从定义也可以得出准素理想的根理想是素理想.
	\item 但是反过来,如果诺特环上一个理想的根理想是素理想,并不能说明这个理想是准素的.例如取$R=k[x,y,z]/(xy-z^2)$,按照理想对应定理有$p=(x',z')$是$R$的素理想,取$q=p^2$.那么有$\sqrt{q}=p$,但是从$x'y'=z'^2\in p^2=q$,且$x'\not\in q,y'^n\in q$说明$q$不是准素理想.
	\item 不过如果一个理想的根理想是极大理想,可说明它是准素的:设$R$有理想$I$,使得$m=\sqrt{I}$是极大理想,那么任取$xy\in I$,设$x\not\in I$,倘若$y$的任何次幂都不在$I$中,那么$y$不在$m$中,导致$m+(y)$是单位理想,于是有$s+ry=1$,其中$s\in m$,假设$s^n\in I$,那么有$x=x(s+ry)^n\in I$,这矛盾.
	\item 如果$m$是诺特环$R$的极大理想,那么每个包含某个$m^n$的理想都是$m$准素的:设$m^n\subseteq I$,那么有$m=\sqrt{m^n}\subset\sqrt{I}$,于是只能有$\sqrt{I}=m$,结合上一条得到$I$是准素理想.这里把$m$改成素理想是不成立的,第二条就是个反例.
	\item 另外在诺特环上,如果$m$是极大理想,那么倘若$p$的根理想是$m$,可存在一个正整数$n$使得$m^n\subseteq p\subseteq m$,反过来也是成立的,即诺特环上理想$p$的根理想是极大理想$m$当且仅当存在一个正整数$n$使得$m^n\subseteq p\subseteq m$.结合上一条得到$m$准素理想的一个完全刻画.
	\item 不过极大理想的准素理想并不都是从取次幂得到的.例如$R=k[x,y]$上有极大理想$(x,y)$,但是它的准素理想$(x,y^2)$不是取次幂得到的.
	\item 设$R$是诺特环,设$P$是素理想,那么典范映射$\varphi:R\to R_P$的核就是全部$P$准素理想的交.事实上一方面如果$\varphi(r)=0$,即存在$s\in S=R-P$使得$rs=0$,任取$P$准素理想$Q$,从$rs=0\in Q\subseteq P$得到$s^n\not\in Q$对任意正整数$n$,这导致$r\in Q$.另一方面如果$r$在全部$P$准素理想的交中,按照分式化的准素理想的对应,记$A_p$的极大理想$p_1=pA_p$,那么$\varphi(r)$落在全部$p_1$的准素理想的交中.按照Krull相交定理,有$\cap_{n\ge1}p_1^n=\{0\}$,这里每个$p_1^n$都是$p_1$准素理想,这就得到$\varphi(r)=0$.
\end{enumerate}

关于局部化典范映射的核.按照Atiyah书上的记号,我们对素理想$p$,把局部化的典范映射$A\to A_p$的核记作$S_p(0)$.它有如下基本性质:
\begin{enumerate}
	\item $S_p(0)\subseteq p$;另外如果素理想$p_1\subseteq p$,那么有$S_p(0)\subseteq S_{p_1}(0)$.
	\item $\sqrt{S_p(0)}=p$当且仅当$p$是$A$的极小素理想.事实上,任取和$A-p$不交的素理想$q$,换句话说任取包含于$p$的素理想$q$,和上一条一样我们可以证明$S_p(0)\subseteq q\subseteq p$,于是这样的素理想$q$总包含了$\sqrt{S_p(0)}$,于是倘若这个根等同于$p$意味着$p$自身是一个极小素理想,这就完成了必要性的证明.对于充分性,如果$p$是$A$的极小素理想,假设可取到一个元$a\in p-\sqrt{S_p(0)}$,那么集合$S_0=\{sa^n\mid s\in A-p,n\ge0\}$是一个严格包含着$S$的乘性闭子集,并且它不含零元:如果某个$sa^n=0$,那么$a^n\in S_p(0)$,导致$a\in\sqrt{S_p(0)}$矛盾.于是$S_0^{-1}A$非零环,于是它至少存在一个素理想$S_0^{-1}q$,按照理想对应定理,这里的$q$是真包含于$p$的$A$的素理想,这就和$p$的极小性矛盾.
	\item 当$p$取遍$\mathrm{Supp}(A)$中全部素理想时,$S_p(0)$的交是零理想.若否,假设可取非零元$a\in\cap_{p\in\mathrm{Supp}(A)}S_p(0)$.取关于真理想$\mathrm{Ann}(a)$的极小素理想$q$,按照支集定义这个$q$必然在支集中,于是存在$s\in A-q$使得$as=0$,这导致$s\in\mathrm{Ann}(a)\subseteq q$矛盾!
	\item 如果$p$是$A$的极小素理想,那么$S_p(0)$是最小的$p$准素理想.事实上第二条说明了$\sqrt{S_p(0)}=p$,现在验证$S_p(0)$是准素理想:任取$xy\in S_p(0)$,此即存在某个$s\in A-p$使得$sxy=0$,倘若$y\not\in S_p(0)$,于是$sy\not=0$,于是$x/1$是$A_p$的一个零因子.现在从$p$是极小素理想,说明$A_p$中唯一的素理想是$pA_p$,而环上零因子集合是若干素理想的并,说明这唯一的极大理想$pA_p$恰好就是全体零因子构成的集合,于是全体零因子同样是全体幂零元,于是存在某个次幂使得$x^n\in S_p(0)$.这就得到$S_p(0)$是$p$准素理想.
	
	现在说明$S_p(0)$包含在每个$p$准素理想中.假设$q$是$p$准素理想,那么有$q\subseteq p$,于是如果$x\in S_p(0)$,那么存在$s\in A-p$使得$sx=0\in q$,倘若$x\not\in q$,会导致某个次幂$s^n\in q\subseteq p$,导致$s\in p$矛盾,于是$x\in q$,也即$S_p(0)\subseteq q$.
	\item 记$p$取遍极小素理想时$S_p(0)$的交为$S(0)$,那么$S(0)$落在幂零根中.事实上如果$a\in A$不是幂零元,那么不含$a$的全部素理想中存在极小元,任取一个$q$,它实际上也是$A$的极小素理想.现在$a\not\in q$导致$a\not\in S_q(0)$,于是$a$不在$S(0)$中.
\end{enumerate}

不可约分解和准素分解.给定$R$模$M$和一个子模$N$,称$N$是不可分解模,如果不存在$M$的两个子模$N_i\not=N,i=1,2$使得$N=N_1\cap N_2$,即不能表示为两个严格更大的子模的交,否则称$N$是可分解模.子模$N$的一个分解是指把它表示为有限个$M$的子模$N_i$的交$N=\cap_{1\le i\le s}N_i$.称分解是不可缩短的,如果对每个$1\le i\le s$有$N\not\subset\cap_{j\not=i}N_j$.如果要求每个$N_i$都是不可约模,就称分解是$N$的不可约分解;如果要求每个$N_i$是准素子模,就称分解是$N$的准素分解.

对于诺特环上的有限生成模,不可约真子模总是准素子模.只需假设非零真子模$N$不是准素的,则$N$是可约模.由于$N$不是准素的,那么$M/N$至少存在两个不同的伴随素理想$p_i,i=1,2$.于是$M/N$存在两个子模$N_i,i=1,2$分别同构于$R/p_i,i=1,2$.我们断言这两个子模的交是零元,否则取交中的一个非零元,它的零化子既是$p_1$又是$p_2$,这和$p_1\not=p_2$矛盾.于是$N+N_1$和$N+N_2$是两个严格包含$N$的子模,并且它们的交就是$N$,于是$N$是可约模.

不可约分解的存在性.如果$M$是诺特环上的有限生成模,那么任意子模$N$有不可约分解,经删去可去掉的分支,说明任意子模有不可缩短的不可约分解.
\begin{proof}
	
	假设不满足条件的子模存在,设全体这样的子模构成的集合是$S$,按照诺特性,说明$S$具有极大元$N$,那么$N$必然是可约的,否则它自身就构成了一个不可约分解.于是存在两个严格更大的子模$N_1,N_2$满足$N=N_1\cap N_2$.按照极大性每个$N_i$可以表示为有限个不可约子模的交,导致$N$也可以,这矛盾.
\end{proof}

至此我们证明了准素分解的存在性.现在任取真子模$N$的准素分解$N=N_1\cap N_2\cap\cdots\cap N_r$.记$\mathrm{Ass}(N/N_i)=\{p_i\}$,那么从单射$M/N\to\oplus_{1\le i\le r}M/N_i$就得到包含关系$\mathrm{Ass}(M/N)\subset\{p_1,p_2,\cdots,p_r\}$.如果这个准素分解是不可缩短的,那么该包含号取等号.
\begin{proof}
	
	假设分解是不可缩短的.我们的思路是对每个$p_i$找到$M/N$中的一个元$y+N$使得$\mathrm{Ann}(y+N)=p_i$,以$i=1$为例.从$N_2\cap N_3\cap\cdots\cap N_r\not=N$,说明可取非零元$x+N\in\cap_{2\le i\le r}N_i/N$,那么$x+N_1\not=N_1$,否则有$x\in \cap_iN_i=N$和$x+N\not=N$矛盾.
	
	按照$N_1\subseteq M$是准素子模,有$\sqrt{\mathrm{Ann}(M/N_1)}=p_1$是有限生成的素理想,于是可取最小的自然数$n$使得$p_1^{n+1}(x+N)\in N_1/N$,于是必然存在$p_1^n(x+N)$中的一个元$y+N\not\in N_1/N$.按照构造知道$p_1\subset\mathrm{Ann}(y+N)$.反过来先注意到$y+N_1\not=N_1$.于是如果有$r(y+N)=N$,那么$r(y+N_1)=N_1$说明$r$是$M/N_1$的零因子,按照$N_1$是准素子模,这得到$r\in p_1$,综上有$\mathrm{Ann}(y+N)=p_1$.
\end{proof}

最短长度准素分解.设$M$是诺特环上的有限生成模,如果$M$的真子模$N$有准素分解$N=\cap_{1\le i\le s}N_i$,记$\mathrm{Ass}(M/N_i)=\{p_i\}$.倘若存在$p_i=p_j=p$,那么$N_i\cap N_j$同样是$p$准素子模,据此我们可以把对应的伴随素理想相同的那些$N_i$交在一起,这会得到一个准素分解$N=\cap_{1\le i\le r}N_i'$,其中每个$N_i'$是$p_i'$准素的,并且$p_i'$两两不同.称这样的准素分解为最短长度的准素分解,称每个$N_i'$是$N$的$p_i'$准素分支.并且准素分支个数恰好就是$\mathrm{Ass}(M/N)$中的元素个数.

关于分解的唯一性,条件同样是诺特环上的有限生成模$M$,给定真子模$N$,取最短长度的准素分解$N=N_1\cap N_2\cap\cdots\cap N_r$:
\begin{enumerate}
	\item 如果$p$是$M/N$的孤立素理想,那么$p$准素分支恰好就是$\varphi_p^{-1}(N_p)$,这里$\varphi_p$是典范映射$M\to M_p$.于是孤立素理想对应的准素分支总是唯一的.
	
	事实上不妨设$p=p_1$,那么得到$N_{p_1}=(N_1)_{p_1}\cap(N_2)_{p_1}\cap\cdots\cap(N_r)_{p_1}$.按照$p_1$是极小素理想,于是对任意的$p_j,j\not=1$,有$p_1$的补集和$p_j$有交,导致$M_{p_1}/(N_j)_{p_1}=(M/N_j)_{p_1}=0$,这说明$N_{p_1}=(N_1)_{p_1}$.于是$\varphi_p^{-1}(N_p)=\varphi^{-1}((N_1)_p)$.最后证明$\varphi^{-1}((N_1)_p)=N_1$.
	
	右侧自然包含于左侧,下面只需验证左侧包含于右侧,任取$(N_1)_p$中的元$n/s=m/1$,需要说明$m\in N_1$.有$ss'm=s'n\in N_1$,倘若$m\not\in N_1$,那么$ss'$是$M/N_1$中的非零元$m+N_1$的零化子,于是$ss'\in P$,但是$s,s'\not\in p$,这和$p$是素理想矛盾.
	\item 但是嵌入素理想对应的准素分支未必是唯一的.例如取环$R=k[x,y]$,取理想$I=(x^2,xy)$,取$M=R/I$,记$x,y$在$M$中的像是$x'$和$y'$.那么我们证明过有支集$\mathrm{Supp}(M)=V(x)$,伴随素理想集为$\mathrm{Ass}(M)=\{(x),(x,y)\}$.其中$(x)$是孤立素理想,而$(x,y)$是嵌入素理想.那么孤立素理想$p=(x)$的部分准素分支是唯一的,它就是$R\to R_p$在$I_p$下的原像,但是按照$R_p$中$y/1$是单位元,说明$I_p=(x^2,x)_p=(x)_p=pR_p$,而$pR_p$的原像自然是$p=(x)$,于是这部分准素分支是$(x)$.
	
	对于嵌入理想$(x,y)$,注意$R/(x^2,y)$和$R/(x^2,xy,y^n),n\ge1$的支集中只有极大理想$(x,y)$,说明$(x^2,y)$和$(x^2,xy,y^n),n\ge1$都是$(x,y)$准素理想,这一事实还可以从它们均包含某个$(x,y)^n$直接得出.现在我们断言$I=(x)\cap(x^2,y)$和$I=(x)\cap(x^2,xy,y^n)$都是不可缩短的准素分解.事实上从$I\subset(x^2,xy,y^n)\subset(x^2,y)$,只需验证$(x)\cap(x^2,y)\subseteq I$.任取交中的$f$,那么$F=Gx=Ax^2+By$,其中$G,A,B\in R$.于是$x(G-Ax)=By$,于是$x\mid B$,记$B=B'x$,那么$F=Ax^2+B'xy\in I$.
\end{enumerate}

分式化的准素分解:
\begin{enumerate}
	\item 设$R$是诺特环,$S$是$R$的一个乘性闭子集,$p$是素理想,$M$是$R$模,$Q$是它的$p$准素子模,如果$S\cap p$非空,那么$S^{-1}Q=S^{-1}M$,并且$Q^S=M$;如果$S\cap p$是空的,那么$S^{-1}Q$是$S^{-1}M$的$S^{-1}p$准素子模,并且$Q^S=Q$.
	\begin{proof}
		
		首先$S^{-1}q\in\mathrm{Ass}(S^{-1}(M/Q))$当且仅当$q\cap S$空并且$q\in\mathrm{Ass}(M/Q)$,于是如果$Q$本身是$p$准素子模,且$S\cap p$非空,那么$\emptyset=\mathrm{Ass}(S^{-1}(M/Q))=\mathrm{Ass}(S^{-1}M/S^{-1}Q)$,这导致$S^{-1}M=S^{-1}Q$.而如果$S\cap p$是空集,就有$\mathrm{Ass}(S^{-1}(M/Q))=\{S^{-1}p\}$,于是$S^{-1}Q$是$S^{-1}M$的$S^{-1}p$准素子模.
		
		按照$Q^S=\varphi_S^{-1}(S^{-1}Q)$,如果$S^{-1}Q=S^{-1}M$,就得到$Q^S=M$.现在假设$S\cap p$是空集,任取$m\in Q^S$,于是存在$s\in S$使得$sm\in Q$,但是$s\not\in p$,$p$已经是$M/Q$的零因子集合,这导致$m\in Q$,于是$Q^S\subseteq Q$,结合$Q\subseteq Q^S$就得到$Q=Q^S$.
	\end{proof}
	\item 如果$R$是诺特环,$S$是乘性闭子集,$M$是有限生成$R$模,设$M$的真子模$N$有不可缩短的准素分解$N=Q_1\cap Q_2\cap\cdots\cap Q_r$.记$Q_i$是$p_i$准素的,不妨重排使得$S\cap p_i=\emptyset$只对$i\le h$成立,那么$S^{-1}N$的准素分解为$S^{-1}N=S^{-1}Q_1\cap\cdots\cap S^{-1}Q_s$,并且有$N^S=Q_1\cap Q_2\cap\cdots\cap Q_s$.
\end{enumerate}

一些特例.
\begin{enumerate}
	\item 理想的准素分解.设$R$是诺特环,$I$是一个真理想,记$R/I$的全部不同的伴随素理想为$\{p_1,p_2,\cdots,p_r\}$,那么$I$可以表示为$I=J_1\cap J_2\cap\cdots\cap J_r$,其中$J_i$是一个$p_i$准素理想.
	\item 特别的,设$R$是诺特环,如果$I$本身是一个根理想,那么它已经是全体包含自身的极小素理想的交,于是这就是它的准素分解.结合零点定理对应到代数集上,这等价于每个代数集可以分解为有限个代数簇的并,也即拓扑空间的不可约分解.
	\item 另外PID上有限生成模的结构定理可视为准素分解的特例.因为有限生成模即有限生成自由模的商模,于是该结构定理即探究$R^n/M$形式的模的结构,这归结为探究$R^n$的子模$M$的结构,于是可考虑准素分解,此时【】
\end{enumerate}

多项式环上的准素分解.
\begin{enumerate}
	\item 设$A$是环,我们知道如果$p$是$A$上素理想,那么按照$A[x]/p[x]\cong(A/p)[x]$是整环,就得到$p[x]$是$A[x]$的素理想.现在假设在$A$中有$q$是$p$准素理想,那么在$A[x]$中就有$q[x]$是$p[x]$准素理想.
	\begin{proof}
		
		需要验证的是$A[x]/q[x]\cong(A/q)[x]$非零并且零因子都是幂零元.为此注意到$(A/q)[x]$中的零因子即满足系数全部是零因子的多项式$f(x)$.而按照$q$是$p$准素理想,于是$A/q$中零因子均为幂零元,于是$f(x)$是$(A/q)[x]$中的幂零元.最后需要说明$q[x]$的根理想即$p[x]$.一方面$p$是$q$的根理想说明$p$中每个元的某个次幂都落在$q$中,因此$p[x]$中每个元的某个次幂会落在$q[x]$中(取系数次幂的和);另一方面倘若$f(x)$的某个次幂落在$q[x]$中,不妨设$f(x)=a_nx^n+\cdots+a_0$,那么有$a_n$的某个次幂落在$q$中,即$a_n\in p$,于是$f(x)-a_nx^n$同样落在$q[x]$根理想中,归纳操作下去,得到$f(x)$全部系数都落在$p$中,于是$f(x)\in p[x]$.
	\end{proof}
	\item 上述结论说明:如果$I=\cap_{i=1}^nq_i$是$A$中理想$I$的极小准素分解,那么$I[x]=\cap_{i=1}^nq_i[x]$是$I[x]$在$A[x]$中的极小准素分解.
	\item 特别的,如果$p$是$I$的极小素理想,那么$p[x]$是$I[x]$的极小素理想.这是因为极小素理想就是伴随素理想集合中的极小元,而如果$p$是$\mathrm{Ann}(I)=\{p_1,p_2,\cdots,p_r\}$中的极小元,必然就有$p[x]$是$\mathrm{Ann}(I[x])=\{p_1[x],p_2[x],\cdots,p_r[x]\}$中的极小元.
\end{enumerate}
\newpage
\section{环的扩张}
\subsection{平坦扩张}

如果结构映射$A\to B$使得$B$是平坦$A$模/忠实平坦$A$模,就称$B$是平坦$A$代数/忠实平坦$A$代数.
\begin{enumerate}
	\item 基环的变换.设$A$是环,$B$是$A$代数,如果$M$是平坦/忠实平坦$A$模,那么$B\otimes_AM$是平坦/忠实平坦$B$模.证明就是一步张量积的结合律.
	\item 基环变换下的张量积.设$B$是$A$代数,$M,N$是$B$模,那么$M\otimes_BN$是$M\otimes_AN$模去由$\{bm\otimes n-m\otimes bn\mid m\in M,n\in N,b\in B\}$生成的$A$子模.于是特别的:
	\begin{enumerate}
		\item 设$S$是$A$上乘性闭子集,设$M,N$都是$S^{-1}A$模,那么如下等式说明生成的那个子模为零,从而有$M\otimes_AN=M\otimes_{S^{-1}A}N$.
		$$\frac{a}{s}m\otimes n=\frac{am}{s}\otimes\frac{sn}{s}=\frac{sm}{s}\otimes\frac{an}{s}=m\otimes\frac{a}{s}n$$
		\item 设$I$是$A$的理想,设$M,N$都是$A/I$模(此即$A$模且被$I$零化),那么如下等式说明生成的那个子模为零,于是有$M\otimes_AN=M\otimes_{A/I}N$.
		$$((a+I)m)\otimes n=((a+I)m)\otimes((1_A+I)n)=((1_A+I)m)\otimes((a+I)n)=m\otimes((a+I)m)$$
		\item 补充一件事,如果$B$是$A$代数,$M$是$A$模,$N$是$B$模,那么$M\otimes_AN$自然具备$B$模结构,即$b(m\otimes n)=m\otimes bn$,设$S$是$B$的乘性闭子集,那么$S^{-1}(M\otimes_AN)=M\otimes_AN\otimes_BS^{-1}B=M\otimes_AS^{-1}N$.特别的,如果取局部化为在素理想$q\in\mathrm{Spec}(B)$处的局部化,此时$N_q$是$A_p$模,于是$M\otimes_AN_q=M\otimes_A(A_p\otimes_{A_p}N_q)=M_p\otimes_{A_p}N_q$,其中$p=q\cap A$.
	\end{enumerate}
	\item 平坦是一个局部性质.设$B$是$A$代数,设$M$是$B$模,那么$M$是$A$平坦模当且仅当对$B$的每个素理想(极大理想)$q$,记$p=q\cap A$(这个记号表示$p=f^{-1}(q)$,其中$f:A\to B$是结构映射),那么每个$M_q$都是平坦$A_p$模.这个结论包含两个特殊情况:如果取$B=A$,结构映射取恒等,得到$A$模$M$平坦当且仅当对每个素理想$p\in\mathrm{Spec}(A)$有$M_p$是$A_p$平坦模;如果取$M$是$B$模$B$,得到一个$A$代数$B$是平坦的当且仅当对每个素理想$q\in\mathrm{Spec}(A)$,记$p=q\cap A$,那么每个$B_q$都是$A_p$平坦代数.
	\begin{proof}
		
		必要性.任取$B$的素理想$q$,结构映射诱导了环同态$A_p\to B_q$,按照上一条得到$A_p$模上的函子$-\otimes_{A_p}M_q=-\otimes_AM_q=(-\otimes_AM)\otimes_BB_q$.这是两个正合函子的复合,于是正合.
		
		充分性.设对每个$B$的极大理想$q$,都有$M_q$是$A_p$平坦模.任取$A$模的正合列$0\to N'\to N$,设$N'\otimes_AM\to N\otimes_AM$的核为$K$,得到$B$模的正合列$0\to K\to N'\otimes_A M\to N\otimes_A M$.任取$B$的极大理想$q$,局部化得到正合列$0\to K_q\to N'\otimes_AM_q\to N\otimes_AM_q$.这里$N\otimes_AM_q=N_p\otimes_{A_p}M_q$和$N'\otimes_AM_q=(N')_p\otimes_{A_p}M_q$,于是条件说明$K_q=0$对任意极大理想$q\in\mathrm{Spec}(B)$成立,于是$K=0$.
	\end{proof}
	\item 平坦性的传递性.设$A$是环,$B$是$A$代数也是平坦$A$模,如果$N$是平坦$B$模,那么$N$是平坦$A$模.特别的,这说明平坦同态的复合仍为平坦同态.
	\begin{proof}
		
		需要验证$N\otimes_A-$是正合函子,而它自然同构于$(N\otimes_BB)\otimes_A-)$,又自然同构于$N\otimes_B(B\otimes_A-)$,按照$B$是平坦$A$模得到$B\otimes_A-$是正合函子,按照$N$是平坦$B$模得到$N\otimes_B-$是正合函子,于是它们的复合是正合函子.于是$N$是平坦$A$模.
	\end{proof}
\end{enumerate}

纤维.
\begin{enumerate}
	\item 给定环同态$f:A\to B$,它诱导了素谱之间的连续映射$^af:\mathrm{Spec}(B)\to\mathrm{Spec}(A)$,任取$p\in\mathrm{Spec}(A)$,记$\kappa(p)=A_p/pA_p=(A/p)_p$,它称为$p$处的剩余类域.有如下交换图:
	$$\xymatrix{B\ar[rr]^{1\otimes\pi}&&B\otimes_A\kappa(p)\\A\ar[u]^f\ar[rr]^{\pi}&&\kappa(p)\ar[u]_{f\otimes1}}$$
	
	取素谱,得到如下交换图:
	$$\xymatrix{\mathrm{Spec}(B)\ar[d]_{^af}&&\mathrm{Spec}(B\otimes_A\kappa(p))\ar[d]^{^a(f\otimes1)}\ar[ll]_{^a(1\otimes\pi)}\\\mathrm{Spec}(A)&&\mathrm{Spec}(\kappa(p))\ar[ll]_{^a\pi}}$$
	\item 我们断言$^a(1\otimes\pi)$诱导了$\mathrm{Spec}(B\otimes_A\kappa(p))$到$(^af)^{-1}(p)$的双射,据此称$B\otimes_A\kappa(p)$为$f$在$p$处的纤维环,称$\mathrm{Spec}(B\otimes_A\kappa(p))$是$f$在$p$处的纤维.
	\begin{proof}
		
		记$S=A-p$,那么$B\otimes_A\kappa(p)=(B\otimes_AA_p)\otimes_{A_p}(A_p/pA_p)=S^{-1}B/pS^{-1}B$,于是$1\otimes\pi:B\to B\otimes_A\kappa(p)$即典范映射$B\to S^{-1}B/pS^{-1}B$.现在$\mathrm{Spec}(S^{-1}B/pS^{-1}B)$中的素理想即$S^{-1}B$中的素理想$q'$满足$q'\cap A_p=pA_p$.按照分式化素理想的对应定理,此即$B$中素理想$q$使得$q\cap A=p$(因为如果$q'\cap A_p=pA_p$,那么$q'$对应的$B$中素理想肯定和$f(q)$无交).于是$^a(1\otimes\pi)$即把这样的$S^{-1}B/qS^{-1}B$中素理想一一对应为$B$中的满足$q\cap A=p$的素理想.
	\end{proof}
	\item 取$p\in\mathrm{Spec}(A)$,取$q\in\mathrm{Spec}(B)$是在$p$上的素理想(即$q\cap A=p$).按照上一条,记$q'$为$q$在纤维环$B\otimes_A\kappa(p)$中唯一对应的素理想.取$B$模$M$,那么$B\otimes_Ak(p)$模$M\otimes_A\kappa(p)$在素理想$q'$处的局部化自然同构于$M_q\otimes_A\kappa(p)$.
	\begin{proof}
		
		记$S=A-p$,取$q\in\mathrm{Spec}(B)$是在$p$上的素理想,那么$S^{-1}q$是$S^{-1}B=B\otimes_AA_p$的素理想.那么$q'=S^{-1}q/pS^{-1}B$.$S^{-1}B$模$S^{-1}M$在$S^{-1}q$处的局部化为$(S^{-1}M)_{S^{-1}q}\cong M_q$,于是$S^{-1}B/pS^{-1}B=B\otimes_A\kappa(p)$模$M\otimes_A\kappa(p)=S^{-1}M/pS^{-1}M$在$q'$处的局部化为:
		$$(S^{-1}M/pS^{-1}M)_{S^{-1}q/pS^{-1}B}\cong(S^{-1}M)_{s^{-1}q}/p(S^{-1}M)_{S^{-1}q}\cong M_q\otimes_A\kappa(p)$$
	\end{proof}
\end{enumerate}

忠实平坦性和素谱之间连续映射的满射性.
\begin{enumerate}
	\item 设$f:A\to B$是环同态,设$M$是忠实平坦$A$模,那么$(^af)(\mathrm{Supp}(M))=\mathrm{Spec}(A)$.
	\begin{proof}
		
		任取$p\in\mathrm{Spec}(A)$,按照忠实平坦性,得到$M\otimes_A\kappa(p)\not=0$.于是$M\otimes_A\kappa(p)$作为$B\otimes_A\kappa(p)$模($M\otimes_A\kappa(p)=M\otimes_B(B\otimes_A\kappa(p))$)的支集非空.于是存在$B\otimes_A\kappa(p)$的素理想$q'$使得$M\otimes_A\kappa(p)$在$q'$处的局部化不为零.按照上面定理,存在$B$中素理想$q$使得$(M\otimes_A\kappa(p))_{q'}=M_q\otimes_A\kappa(p)\not=0$,于是$q\in\mathrm{Supp}(M)\cap(^af)^{-1}(p)$,此即$a^f(\mathrm{Supp}(M))=\mathrm{Spec}(A)$.
	\end{proof}
	\item 设$f:A\to B$是环同态,设$M$是有限$B$模,那么$M$是$A$上忠实平坦模当且仅当$M$是平坦$A$模,并且$(^af)(\mathrm{Supp}(M))$视为$\mathrm{Spec}(A)$子集时包含全部闭点(极大理想).
	\begin{proof}
		
		必要性是上一条的推论.下面验证充分性.设$M$是平坦$A$模,并且$^af(\mathrm{Supp}(M))$包含了全部$A$的极大理想.为证明$M$是忠实平坦$A$模,仅需验证对每个$A$的极大理想$m$,总有$M/mM\not=0$.按照条件,可取$m$的落在$\mathrm{Supp}(M)$中的提升理想$q$,设$q'$是$q$在$B\otimes_A\kappa(m)$中对应的素理想,那么$(M/mM)_{q'}=M_q\otimes_A\kappa(m)=M_q/mM_q$(这是之前给出的定理).按照$M_q\not=0$,并且它是有限$B$模,并且$mB_q$在$B_q$的极大理想中,于是NAK引理得到$M_q/mM_q\not=0$,于是$M/mM\not=0$,于是$M$是忠实平坦$A$模.
	\end{proof}
	\item 推论.由第一条知,如果$M$是忠实平坦$A$模,那么$\mathrm{Supp}(M)=\mathrm{Spec}(A)$.
	\item 推论.由第一条知,如果$B$是忠实平坦$A$代数,那么$^af$是满射.(实际上条件还可以放宽,只要$B$作为$A$代数,$B$有模作为$A$模时是忠实平坦的,那么结构映射诱导的连续映射就是满的).
	\item 推论.环同态$f:A\to B$诱导的素谱之间的连续映射是满射,换个说法即对$A$中每个素理想$p$,总存在它的提升理想(即$B$中素理想$q$满足$q\cap A=p$).如果环同态$f:A\to B$满足对每个素理想$p\in\mathrm{Spec}(A)$都有提升理想,就说$f:A\to B$满足提升条件.于是前几条我们说明了忠实平坦映射总是满足提升条件的.
	\item 推论.在第二条中取$A,B$是局部环,$f$是局部映射,$M=B$,说明局部环之间的局部同态是平坦的等价于是忠实平坦的.
	\item 例子.现在我们很容易提出一个不是忠实平坦的平坦模.取环$A$的乘性闭子集$S$,考虑典范的$A\to S^{-1}A$,我们知道局部化总是平坦的,即$S^{-1}A$是平坦$A$代数.但是它是忠实平坦得满足典范的$\mathrm{Spec}(S^{-1}A)\to\mathrm{Spec}(A)$是满射,只要取$S$含某些非零因子,这就不能实现,此时$S^{-1}A$是平坦但非忠实平坦的$A$代数.
\end{enumerate}

在平坦性下,原本不能和张量积可交换的一些算子变得可交换.
\begin{enumerate}
	\item 设$M$是平坦$A$模,设$N-1,N_2$是$A$模$N$的两个子模,那么作为$N\otimes M$的子模有$(N_1\cap N_2)\otimes_AM=(N_1\otimes_AM)\cap(N_2\otimes_AM)$.
	\begin{proof}
		
		构造$\varphi:N\to N/N_1\oplus N/N_2$为$n\mapsto(n+N_1,n+N_2)$,得到正合列$0\to N_1\cap N_2\to N\to N/N_1\oplus N/N_2$.于是张量$M$后依旧是正合列,于是得到$(N_1\cap N_2)\otimes_AM=\ker\varphi=(N_1\otimes_AM)\cap(N_2\otimes_AM)$.
	\end{proof}
	\item 设$B$是平坦$A$代数,设$I,J$是$A$的两个理想,那么$(I\cap J)B=IB\cap JB$.
	\item 如果$A$模$M$有子模$N_1,N_2$,我们用记号$(N_1:N_2)_A$表示$A$的理想$\{a\in A\mid aN_2\subseteq N_1\}$.设$B$是平坦$A$代数,设$I,J$是$A$的两个理想,并且$J$有限生成,那么$(I:J)_AB=(IB:JB)_B$.
	\begin{proof}
		
		设$J=Aa_1+Aa_2+\cdots+Aa_n$,那么$(I:J)=\cap_i(I:Aa_i)$.于是利用上一条归结为证明$J$是主理想的情况.为此考虑正合列$0\to(I:Aa)\to A\to A/I$,这里最后一个映射取为数乘$a$的映射$\varphi$.张量$B$后仍然是一个正合列,于是$(I:Aa)_AB=\ker\varphi=(IB:aB)_B$.
	\end{proof}
\end{enumerate}

设$f:A\to B$是忠实平坦映射.
\begin{enumerate}
	\item 设$M$为$A$模,那么$M\to M\otimes_AB$,$m\mapsto m\otimes1_B$是单射.事实上任取$0\not=m\in M$,有$(Am)\otimes_AB=(m\otimes1_B)B$.忠实平坦性说明$(Am)\otimes_AB\not=0$,于是$m\otimes1\not=0$,于是单射.
	\item 特别的,如果$f:A\to B$是忠实平坦的,那么它是单射.
	\item 特别的,对$A$的理想$I$总有$I^{ec}=I$,此即$IB\cap A=I$.这只要取$M=A/I$,得到$M\otimes_AB=B/IB$.从$A/I\to B/IB$是单射得到$IB\cap A=I$.
\end{enumerate}

平坦性和解方程组.
\begin{enumerate}
	\item 设$M$是平坦$A$模.设$\textbf{A}=(a_{ij})$是$A$上的$r\times n$矩阵,设$\textbf{x}=(x_1,x_2,\cdots,x_n)^t\in M^n$是齐次方程组的解$\textbf{Ax}=0$.那么存在$A$中的$n\times s$的矩阵$\textbf{B}$,以及一个列向量$\textbf{y}=(y_1,y_2,\cdots,y_s)^t\in M^s$使得$\textbf{AB}=0$和$\textbf{x}=\textbf{BY}$.换句话讲,以$A$中的元为系数的齐次方程组在平坦模$M$中的解,一定可以表示为它在$A$中的解的$M$线性组合.
	\begin{proof}
		
		设模同态$\varphi:A^n\to A^r$可表示为上述矩阵$\textbf{A}$,记$K=\ker\varphi$,那么有正合列$0\to K\to A^n\to A^r$,按照平坦性得到正合列$0\to K\otimes_AM\to M^n\to M^r$,记这里最后一个映射是$\varphi_M$.按照条件有$\textbf{x}\in\varphi_M$,正合性说明存在$K$中的元$b_1,b_2,\cdots,b_s$和$M$中的元$y_1,y_2,\cdots,y_s$使得$(x_1,x_2,\cdots,x_n)=(1\otimes i)(\sum_{1\le i\le s}y_i\otimes b_i)$.把$b_i\in A^n$写作坐标表示$b_i=(b_{1i},b_{2i},\cdots,b_{ni})$,记$\textbf{B}=(b_{ij})$,得到$\textbf{AB}=0$,并且$\textbf{x}=\textbf{By}$.
	\end{proof}
	\item 反过来,如果上一条中的条件对$r=1$成立,那么$A$模$M$是平坦模.
	\begin{proof}
		
		任取有限生成理想$I=Aa_1+Aa_2+\cdots+Aa_n\in A$,我们需要验证$\varphi:I\otimes_AM\to M$是单射.任取$\sum_ia_i\otimes x_i\in\ker\varphi$,其中$x_i\in M$,于是有$\sum_ia_ix_i=0$.取$\textbf{A}=(a_i)$是$1\times n$矩阵,记$\textbf{x}=(x_1,x_2,\cdots,x_n)^t$是$M^n$中的元.于是$\textbf{Ax}=0$.
		
		按照条件就有$s$个$A$中长度为$n$的列向量$\textbf{y}_i=(y_{1i},y_{2i},\cdots,y_{ni})^t$,满足$\textbf{Ay}_i=0$,并且存在$m_j\in M$使得$\textbf{x}=\textbf{y}_1m_1+\cdots+\textbf{y}_sm_s$.于是:
		$$\sum_ia_i\otimes x_i=\sum_i\sum_ja_i\otimes y_{ij}m_j=\sum_j\left(\sum_ia_iy_{ij}\right)\otimes m_j=0$$
	\end{proof}
\end{enumerate}

平坦性和Tor函子.设$M$是$A$模,那么如下条件等价:
\begin{enumerate}
	\item $M$是平坦$A$模.
	\item 对任意$A$模$M,N$有$\mathrm{Tor}_1^A(M,N)=0$.
	\item 对任意有限生成例子$I$有$\mathrm{Tor}_1^A(M,A/I)=0$.
\end{enumerate}

局部环上有限模是平坦的等价于是忠实平坦的等价于是自由模.首先自由模必然是忠实平坦模,忠实平坦模必然是平坦模.下面我们仅需验证局部环上有限平坦模是自由模.为此,记$M$是$(A,m)$上的有限平坦模,取它的有限子集$\{x_1,x_2,\cdots,x_n\}$,使得它们在$M/mM$中的像是作为$A/m$线性空间的线性无关组,我们断言这可推出$x_1,x_2,\cdots,x_n$是$A$线性无关的.一旦这成立,只要取$M$的极小生成元集,我们解释过局部环上模的极小生成元集在$M/mM$中是一组基,这就说明极小生成元集总是$M$上的一个基,于是此时该有限平坦模是自由模.
\begin{proof}
	
	对$n$归纳.首先$n=1$时,假设$x_1\in M$不是$A$线性无关的,等价于讲存在$0\not=a\in A$使得$ax_1=0$,于是平坦性和解方程组那个等价描述告诉我们存在$b_1,b_2,\cdots,b_s\in A$,使得$ab_i=0$,并且$x\in\sum_ib_iM$.按照条件有$x_1\not\in mM$,于是必然有某个$b_i\not\in m$,于是$b_i$是单位元,于是从$ab_i=0$得到$a=0$,矛盾.于是$x_1$是$A$线性无关的.
	
	设$n>1$,假设有线性组合$\sum_ia_ix_i=0$.同样按照平坦性和解方程组的那个等价描述,说明存在$b_{ij}\in A$和$y_j\in M$使得$\sum_ia_ib_{ij}=0$和$x_i=\sum_jb_{ij}y_j$.按照$x_n\not\in mM$,说明$b_{nj},j$中必然至少存在一个是$A$中单位元,不妨设$b_{nt}$,导致从$\sum_ia_ib_{it}=0$得到$a_n$可以表示为$a_1,a_2,\cdots,a_{n-1}$的$A$线性组合$a_n=\sum_{1\le i\le n-1}a_ic_i,c_i\in A$,于是$a_1(x_1+c_1x_n)+\cdots+a_{n-1}(x_{n-1}+c_{n-1}x_n)=0$.按照$x_i+c_ix_n,1\le i\le n-1$这$n-1$个元在$M/mM$中是线性无关的,于是归纳假设说明$a_1=a_2=\cdots=a_{n-1}=0$,这又推出$a_n-0$,于是完成归纳.
\end{proof}

Hom函子的平坦基变换公式.
\begin{enumerate}
	\item 设$M,N$是$A$模,设$B$是平坦$A$代数,设$M$是有限表出$A$模,那么有平坦基变换公式:
	$$\mathrm{Hom}_A(M,N)\otimes_AB\cong\mathrm{Hom}_B(M\otimes_AB,N\otimes_AB)$$
	\begin{proof}
		
		按照$M$是有限表出模,存在自由$A$模$L_1,L_2$使得有正合列$L_1\to L_2\to M\to0$,于是按照函子$\mathrm{Hom}_A(-,N)$的左正合性,以及$-\otimes_AB$的正合性,得到如下交换图,其中两行都是正合列,于是短五引理说明存在所求的同构.
		$$\xymatrix{0\ar[r]&\mathrm{Hom}_A(M,N)\otimes_AB\ar[r]\ar[d]&\mathrm{Hom}_A(L_1,N)\otimes_AB\ar[r]\ar[d]_{\cong}&\mathrm{Hom}_A(L_2,N)\otimes_AB\ar[d]_{\cong}\\0\ar[r]&\mathrm{Hom}_B(M\otimes_AB,N\otimes_AB)\ar[r]&\mathrm{Hom}_B(L_1\otimes_AB,N\otimes_AB)\ar[r]&\mathrm{Hom}_B(L_2\otimes_AB,N\otimes_AB)}$$
	\end{proof}
	\item 特别的,按照局部化是平坦的,说明在有限表示条件下局部化和Hom函子可交换:如果$M,N$是$A$模,其中$M$是有限表示模,对任意素理想$p$总有$\mathrm{Hom}_A(M,N)\otimes_AA_p\cong\mathrm{Hom}_{A_p}(M_p,N_p)$.
	\item 特别的,如果$M$是有限表示模,那么它是投射模是一个局部性质.结合我们解释过局部环上投射模总是自由的,说明一个有限表示模是投射模当且仅当它的每个局部化是自由模.
	\begin{proof}
		
		必要性,事实上我们证明过如果$M$是投射$A$模,那么对任意乘性闭子集$S$,总有$S^{-1}M$是$S^{-1}A$投射模.(比方说,$M$是投射模等价于它是自由模的一个直和项,这个性质是保局部化的).
		
		充分性,设$N_1\to N_2\to0$是$A$模的正合列,设$\mathrm{Hom}_A(M,N_1)\to\mathrm{Hom}_A(M,N_2)$的余核为$C$,上一条说明$C_m$是$\mathrm{Hom}_{A_m}(M_m,(N_1)_m)\to\mathrm{Hom}_{A_m}(M_m,(N_2)_m)$的余核,按照条件$C_m=0$,于是$C=0$.
	\end{proof}
\end{enumerate}

设$f:A\to B$和$g:B\to C$都是环同态,如果$g\circ f$是平坦映射,$g$是忠实平坦映射,那么$f$是平坦映射.
\begin{proof}
	
	任取单同态$M\to N$,按照$g\circ f$是平坦映射,于是诱导的同态$M\otimes_AC\to N\otimes_AC$是单射.现在$-\otimes_AC$自然同构于$(-\otimes_AB)\otimes_BC$,于是$0\to(M\otimes_AB)\otimes_BC\to(N\otimes_AB)\otimes_BC$是正合列.现在按照$C$是忠实平坦$B$模,说明上述复形是正合列当且仅当$0\to M\otimes_AB\to N\otimes_AB$是正合列,这说明了$B$是平坦$A$模,即$f$是平坦映射.
\end{proof}
\newpage
\subsection{整扩张}
\subsubsection{整性}

整性.整性概念是将整数概念推广到一般环上的成功尝试.如果$S$是一个$A$代数,称一个元$s\in S$在$A$上整,或者是$A$上的整元,如果存在一个$A[x]$中的首一多项式$p(x)$满足在$S$中有$p(s)=0$.这时候称$p(x)$是$s$在$A$上的整方程.如果$S$中每个元都是$A$上整元,那么称$S$在$A$上整.特别的,如果$S$是$A$的扩环,此时结构同态取为嵌入,此时如果$S$在$A$上整,就称$A\subseteq S$是环的整扩张.先来给出等价描述,给定环$A$和$A$(交换)代数$S$,给定$s\not=0\in S$,那么如下条件等价:
\begin{enumerate}
	\item $s$是$A$上的整元.
	\item 子代数$A[s]$是$A$上的有限模.
	\item 存在子代数$C$,满足$s\in C$,并且$C$作为$A$模是有限模.
\end{enumerate}
\begin{proof}
	
	1推2,如果设$s$的整方程是$x^n+a_1x^{n-1}+\cdots+a_n$,那么$A[s]$作为$A$模被$1,s,s^2,\cdots,s^{n-1}$生成.(题外话,这组生成元实际上是一组基,于是$s$是整元时$A[s]$是$A$有限自由模).
	
	2推3只要取$C=A[s]$.3推1,设$\varphi$是有限$A$模$C$上左乘$s$的同态,按照行列式技巧,$\varphi$满足一个首一多项式,把它作用至$1_A$上,就得到$s$所满足的一个首一多项式.
\end{proof}

给定$A$代数$S$,那么如下三个条件是等价的:
\begin{enumerate}
	\item $S$是有限型的$A$代数,并且$S$在$A$上整.
	\item $S=A[x_1,x_2,\cdots,x_n]$,其中每个$x_i$都是$A$上整元.
	\item $S$是有限$A$代数.
\end{enumerate}
\begin{proof}
	
	3推1是上一定理(另外有限代数一定是有限型代数).1推2是平凡的,现在假设2成立,需要证明$A[x_1,x_2,\cdots,x_n]$是$R$上的有限代数.如果设$x_i$满足的$R$系数首一多项式的次数是$t_i$,那么$\{\prod_{1\le i\le n}x_i^{j_i}\mid 0\le j_i\le t_i-1\}$生成了整个$A[x_1,x_2,\cdots,x_n]$,于是$S$是有限$A$代数.
\end{proof}

这里总结下整元和整性的一些性质:
\begin{enumerate}
	\item 给定三个环$A,B,C$,给定两个结构同态$\varphi:A\to B$和$\psi:B\to C$,这诱导了$A$代数$B$和$B$代数$C$两个结构.并且复合$\psi\circ\varphi$诱导了$A$代数$C$.如果$c\in C$是$A$上的整元,那么$c$也是$B$上的整元,这是因为$A$上零化$c$的首一多项式也是$B$上零化$c$的首一多项式.于是这也说明了如果$C$在$A$上整,那么$C$在$B$上整.
	\item 记号同上一条,如果$B$在$A$上整,$C$在$B$上整,我们断言$C$在$A$上整.任取$c\in C$,需要说明$c$是$A$上的整元,首先按照$c$是$B$上的整元,可以取一组$b_0,b_1,\cdots,b_n\in B$使得$b_nc^n+b_{n-1}c^{n-1}+\cdots+b_0=0$.考虑$D=A[b_0,b_1,\cdots,b_n]$,它是$A$上的有限代数.另外$D[c]$作为$D$模是有限生成的.这说明$D[c]$作为$A$模有限生成,导致$c$在$A$上整.
	\item 上两条可以总结为,在第一条的记号下,$C$在$A$上整当且仅当$C$在$B$上整且$B$在$A$上整.
	\item 如果$S$是$A$代数,$X$是$S$的子集,设$X$的全部有限子集构成的集合是$F$,那么有$A[X]=\cup_{Y\in F}A[Y]$.这一事实说明如果$X$是$S$的由$A$上整元构成的子集,那么$A[X]$仍然在$A$上整.
	\item 整性保有限直积.如果$S_1,S_2,\cdots,S_n$都是整的$A$代数,那么$S=\prod_{1\le i\le n}S_i$具备由结构同态的直积诱导的$A$代数结构.任取$s=s_1s_2\cdots s_n\in S$,其中$s_i\in S_i$,设$s_i$满足的$A$系数的首一多项式是$p_i(x)$,取$p(x)=\prod_ip_i(x)$,这仍然是首一多项式,并且满足$p(s)=0$.
\end{enumerate}

正规性.
\begin{enumerate}
	\item 给定$A$代数$S$,那么全部$S$的在$A$上整的元构成了一个子代数.事实上如果$s_1,s_2\in S$在$A$上整,那么$A[s_1,s_2]$是$A$的有限模,于是$A[s_1,s_2]$在$A$上整,特别的$s_1\pm s_2$和$s_1s_2$都在$A$上整.
	\item 这个子代数称为$A$在$S$中的整闭包或者正规化.如果这个正规化是$A$自身,就称$A$在$S$中正规.
	\item 一个重要的特殊情况是如果$A$是整环,记商域为$F$.这时$F$中全部$A$上的整元构成的中间环就称为整环$A$的整闭包或者正规化.倘若整环和正规化相同,就称它是正规整环.
	\item 对于一般交换环$A$,称它为正规环,如果它在每个素理想$p$处的局部化$A_p$都是正规整环.
\end{enumerate}

正规化的一些性质:
\begin{enumerate}
	\item 正规化是正规的.设$S$是$A$代数,设$A$在$S$中的正规化是$A'$,我们断言$A'$在$S$中是正规的.事实上,任取$x\in S$在$A'$上整,于是$A'[x]$在$A'$上整,于是$A'[x]$在$A$上整,于是$x\in A'$.
	\item 给定整环的整扩张$A\subseteq B$,记它们分别的正规化$\overline{A},\overline{B}$,那么它们满足$\overline{A}\subset\overline{B}$.事实上任取$a\in\mathrm{Frac}(A)$,使得$a$是$A$上的整元,则存在$A$系数首一多项式零化$a$,但是这个首一多项式还是$B$中的,于是$a$同样是$B$上的整元.
	\item 正规化和有限直积可交换.设$A=\prod_iA_i$和$S=\prod_iS_i$是有限个环的直积,其中$S_i$是$A_i$代数.我们断言$A$在$S$中正规当且仅当每个$A_i$在$S_i$中正规.这一事实说明$A$的正规化恰好就是每个$A_i$的正规化的直积,即正规化和有限直积可交换.
	\begin{proof}
		
		容易说明$x\in S$在$A$中整当且仅当每个$x_i$在$R_i$中整.于是如果$A$在$S$中整闭,得到每个$A_i$在$S_i$中整闭.反过来如果每个$A_i$在$S_i$中整闭,那么任取$s\in S$使得$s$在$A$中整,就得到$s\in A$.
	\end{proof}
	\item UFD是正规整环.设$A$是一个UFD,设它的商域是$F$,倘若$F$中的一个元$a/b$不在$A$中,即存在一个不可约元$p$满足$p$整除$b$但是不整除$a$,但是$a/b$却是$A$上的整元,那么存在$A$中有限个元$a_i$使得:
	$(a/b)^n+a_1(a/b)^{n-1}+\cdots+a_n=0$,这得到$a^n+a_1a^ {n-1}b+\cdots+a_nb^n=0$,但是这导致了$p$整除$a$,矛盾.
	\item 设$A$是正规整环,$K$是商域,设$K\subseteq L$是代数扩张,一个元$\alpha\in L$在$A$上整当且仅当$\alpha$在$K$上的首一极小多项式的系数都落在$A$中.
	\begin{proof}
		
		充分性就是定义.现在设$\alpha\in L$在$A$上整,设$\alpha$在$K$上的首一极小多项式是$f(X)=X^n+a_1X^{n-1}+\cdots+a_{n-1}X+a_0\in K[X]$.取$L$的一个代数闭包为$\overline{L}$,那么在$\overline{L}[X]$中存在分解$F(X)=\prod_{1\le i\le n}(X-\alpha_i)$,不妨设$\alpha_1=\alpha$.按照$K[\alpha]\cong K[\alpha_i],\forall i$,说明每个$\alpha_i$都是$A$上的整元,于是$\mathbb{Z}[\alpha_1,\alpha_2,\cdots,\alpha_n]\subseteq A$.但是$F(X)$的系数都落在$\mathbb{Z}[\alpha_i]$中,这就得到$f(X)\in A[X]$.
	\end{proof}
\end{enumerate}

这里给出一些正规化的例子.
\begin{enumerate}
	\item 域上多项式环是UFD,于是正规.一个不平凡的事实是,交换环正规当且仅当它的单元多项式环是正规的.
	\item $R=\mathbb{Z}[\sqrt{5}]$不是UFD,因为有$(1+\sqrt{5})(1-\sqrt{5})=-2\cdot2$,这里$1\pm\sqrt{5}$和2是不相伴的不可约元.如果取$S=\mathbb{Z}[\frac{1+\sqrt{5}}{2}]$,这是一个包含了$R$的UFD(实际上甚至是PID),于是上一定理说明$R$是正规整环.于是$R$的正规化$R'$包含于$S$的正规化$S$自身中.另一方面$\frac{1+\sqrt{5}}{2}$被首一多项式$x^2-x-1$零化,这就说明实际上$R'=S$.
	\item 上一条更一般的结论:设$d$是一个无平方因子的非零整数.记$K=\mathbb{Q}(\sqrt{d})$,设$\mathbb{Z}$在$K$中的正规化是$R$,那么当$d=4k+1$时$R=\mathbb{Z}(\frac{1+\sqrt{d}}{2})$,当$d=4k+2$或者$4k+3$时$R=\mathbb{Z}(\sqrt{d})$.
	\item 设$k$是域,取一元多项式环$k[t]$,取它的子环$R=k[t^2,t^3]$,那么有$\mathrm{Frac}(R)=k(t)$.记$R$的正规化为$R'$.$t$满足$R$系数的首一多项式$X^2-t^2$.于是$k[t]\subseteq R'$.但是按照$k[t]$是正规的,从$R\subseteq k[t]$就得到$R'\subseteq k[t]$,这说明$R$的正规化就是$k[t]$.
\end{enumerate}

商,分式化和整扩张的联系.
\begin{enumerate}
	\item 给定环的整扩张$R_1\subseteq R_2$,设$J$是$R_2$的理想,设$I=J\cap R_1$,那么$R_2/J$是$R_1/I$的整扩张.事实上任取$r\in R_2$,对$r^n+a_1r^{n-1}+\cdots+r_n=0$模去$J$就得证.
	\item 给定环的整扩张$R_1\subseteq R_2$,设$S$是$R_1$的乘性闭子集,那么$S^{-1}R_2$是$S^{-1}R_1$的整扩张.
	\begin{proof}
		
		任取$a/s\in S^{-1}R_2$,由于$a\in R_2$在$R_1$上整,于是存在一组系数$r_i\in R_1$使得$a^n+r_1a^{n-1}+\cdots+r_n=0$,导致$(a/s)^n+r_1/s(a/s)^{n-1}+\cdots+r_n/s^n=0$,于是$a/s$在$S^{-1}R_2$上整.
	\end{proof}
	\item 正规化和分式化可交换.给定环的整扩张$R\subseteq T$,设$R$在$T$中的正规化为$R'$,取$R$的乘性闭子集$S$,那么$S^{-1}R$在$S^{-1}T$中的正规化是$S^{-1}R'$.事实上上一条说明$S^{-1}R'$在$S^{-1}R$上整,现在任取$S^{-1}T$中的元$a/s$在$S^{-1}R$上整,也即存在等式$(a/s)^n+(r_1/s_1)(a/s)^{n-1}+\cdots+(r_n/s_n)=0$,于是得到$s_1s_2\cdots s_na$在$R$上整,于是它在$R'$中,于是$a/s=(s_1s_2\cdots s_na)/(s_1s_2\cdots s_ns)\in S^{-1}R'$,得证.
	\item 特别的,上一条说明,给定整环$A$,取局部化$A_p$,如果$A$在$F$中的正规化是$S$,那么$A_p$在$F$中的正规化是$S_p$.
\end{enumerate}

我们只定义了整环的正规性.对于一般的(交换)环$A$,称它是正规环,如果对每个素理想$p$,局部化$A_p$都是正规整环.
\begin{enumerate}
	\item 先来说明这个新定义的概念包含了正规整环的概念.换句话讲需要验证两件事:一个正规整环在每个素理想处的局部化都是正规整环;如果一个整环在每个素理想处的局部化都是正规整环,那么它是正规的.这也等价于讲,给定整环$A$,它是正规的是一个局部性质.事实上我们取$A$在商域$F$中的正规化$S$,那么$A$是正规的等价于嵌入映射$A\to S$是满射,而满射是一个局部性质,于是等价于对每个素理想(或者对每个极大理想)$p$,总有$A_p\to S_p$是满射,但是我们解释过$S_p$就是$A_p$在$F$中的正规化,于是这等价于讲$A_p$都是正规的.
	\item 设$A$是诺特环,如果对每个素理想$p$的局部化$A_p$都是整环,如果记$A$的全部极小素理想为(我们证明过诺特环上极小素理想总是有限的)$\{p_1,p_2,\cdots,p_r\}$,那么有$A\cong\prod_{1\le i\le r}A/p_i$.这个直和结论等价于讲素谱$\mathrm{Spec}(A)$的不可约分支和连通分支是一致的.
	\begin{proof}
		
		按照每个$A_p$都是整环,特别的它是既约环,既约是一个局部性质,于是$A$是既约环,于是$\cap_ip_i=\{0\}$.于是典范同态$\alpha:A\to\prod_{1\le i\le r}A/p_i$是单射.
		
		为说明$\alpha$是满射,只需验证每个局部同态$\alpha_p$是同构.任取素理想$p$,有$\alpha_p:A_p\to\prod_{1\le i\le r}(A/p_i)_p$.由于$A_p$是整环,于是$p$只包含了至多一个不是自身的素理想,不妨设它包含的极小素理想是$p_1$,那么有$p_i\not\subseteq p,\forall i\ge2$.于是$i\ge2$时有$(A/p_i)_p=0$.于是局部映射转化为$A_p\to (A/p_1)_p=A_p/p_1A_p$.按照$A_p$是整环,得到极小素理想$p_1A_p=0$,于是$\alpha_p$是同构,按照同构是局部性质得到$\alpha$是同构.
	\end{proof}
	\item 一个诺特环$A$是正规环当且仅当它是有限个正规整环的积.
	\begin{proof}
		
		充分性,环的直积$A_1\times\cdots\times A_n$的素理想具有形式$A_1\times\cdots\times p_i\times\cdots\times A_n$,其中$q_i$是$A_i=A/p_i$的素理想.积环在这个素理想处的局部化恰好就是$(A_i)_{q_i}$,它是正规整环的局部化,我们解释过这仍然是正规的.
		
		必要性,按照上一条,记$A$的全部极小素理想是$\{p_1,p_2,\cdots,p_r\}$,那么就有$A\cong A/p_1\times A/p_2\times\cdots\times A/p_r$.需要验证每个$A/p_i$都是正规的.于是不妨约定$A$本身是整环,它在每个素理想处的局部化都是正规的,我们解释过这等价于$A$本身正规.    	
	\end{proof}
\end{enumerate}
\subsubsection{上升定理和下降定理}

环扩张的一些定义.
\begin{enumerate}
	\item 称环扩张$A\subseteq B$满足提升条件,如果对于$A$的每个素理想$P$均存在$B$的素理想$Q$提升了$P$(这是指$P=Q\cap A$).
	\item 称环扩张$A\subseteq B$满足不可比条件,如果$B$中两个素理想$Q_1,Q_2$均是$A$中同一个素理想$P$的提升,那么$Q_1,Q_2$之间没有包含关系,换句话说如果$Q_1\subseteq Q_2$是$P$的提升,那么$Q_1=Q_2$.
	\item 称环扩张$A\subseteq B$满足上升条件,如果对$A$的任意素理想升链$P_1\subseteq P_2\subset\cdots\subseteq P_n$,和$B$的素理想升链$Q_1\subseteq Q_2\subset\cdots\subseteq Q_m$,其中$m<n$并且每个$Q_i$是$P_i$的提升,那么$B$上的这个素理想升链可以延拓至$Q_1\subseteq Q_2\subset\cdots\subseteq Q_n$,满足每个$Q_i$都是$P_i$的提升.按照归纳法,这个条件总成立归结为对$n=2$,$m=1$的情况成立.
	\item 称环扩张$A\subseteq B$满足下降条件,如果对$A$的任意素理想降链$P_1\supset P_2\supset\cdots\supset P_n$,和$B$的素理想降链$Q_1\supset Q_2\supset\cdots\supset Q_m$,其中$m<n$并且每个$Q_i$是$P_i$的提升,那么$B$上的这个素理想降链可以延拓至$Q_1\supset Q_2\supset\cdots\supset Q_n$,满足每个$Q_i$都是$P_i$的提升.同样按照归纳法,这个结论总成立归结为对$n=2$,$m=1$的情况成立.
\end{enumerate}

商和分式化下的提升素理想.
\begin{enumerate}
	\item 如果$A\subseteq B$是环扩张,设$B$的素理想$q$提升了$A$的素理想$p$,那么$A/p\subseteq B/q$是环扩张.那么素理想$q'/q\subseteq B/q$提升了素理想$p'/p\subseteq A/p$当且仅当包含了$q$的素理想$q'\subseteq B$提升了包含了$p$的素理想$p'\subseteq A$.这个结论只要注意到如下交换图:
	$$\xymatrix{A\ar[rr]\ar[d]&&B\ar[d]\\A/p\ar[rr]&&B/q}$$
	\item 如果$A\subseteq B$是环扩张,取$A$的乘性闭子集$S$,那么$S$也是$B$中的一个乘性闭子集.考虑环扩张$S^{-1}A\subseteq S^{-1}B$.那么素理想$S^{-1}q\subseteq S^{-1}B$提升了素理想$S^{-1}p\subseteq S^{-1}A$当且仅当和$S$无交的素理想$q\subseteq B$提升了和$S$无交的素理想$p\subseteq A$.这个结论只要注意到如下交换图:
	$$\xymatrix{A\ar[rr]\ar[d]&&B\ar[d]\\S^{-1}A\ar[rr]&&S^{-1}B}$$
\end{enumerate}

给定整环的整扩张$A\subseteq B$,那么$B$是域当且仅当$A$是域.
\begin{proof}
	
	一方面,如果$B$是域,任取非零元$a\in A$,那么$1/a\in B$,于是它在$A$上整,于是存在$A$中一组元$a_0,a_1,\cdots,a_n$满足$(1/a)^n+a_1(1/a)^{n-1}+\cdots+a_n=0$,这导致$1/a=-(a_1+a_2a+\cdots+a_na^{n-1})\in A$.
	
	另一方面,如果$A$是域,任取非零元$b\in B$,按照它是$A$上整元说明可取次数最小的首一多项式零化$b$:$b^n+a_1b^{n-1}+\cdots+a_n=0$,其中$a_i\in A$,不妨约定$a_n\not=0$,否则按照$B$是整环得到$b^{n-1}+a_1b^{n-2}+\cdots+a_{n-1}=0$和次数最小矛盾,于是有$bc=1$,其中$c=-a_n^{-1}(b^{n-1}+a_1b^{n-2}+\cdots+a_{n-1})\in B$,于是$B$是域.
\end{proof}

整扩张$A\subseteq B$满足提升条件,不可比条件,上升条件.上升定理即整扩张满足上升条件.
\begin{enumerate}
	\item 引理.设$B$的素理想$q$提升了$A$的素理想$p$,那么$p$是极大理想当且仅当$q$是极大理想.事实上有$A/p\subseteq B/q$是整扩张,于是其中一个是域当且仅当另一个也是域.
	\item 提升条件.任取$A$的素理想$p$,记$A$的乘性闭子集$S=A-p$,考虑整扩张$S^{-1}A\subseteq S^{-1}B$.任取$S^{-1}B$的极大理想,按照引理,它拉回到$S^{-1}A=A_p$上必然是唯一的极大理想$pA_p$.于是存在$B$中的素理想提升了$p$.
	\item 不可比条件.设$B$中素理想$q_1\subseteq q_2$均为$A$中素理想$p$的提升.取乘性闭子集$S=A-p$,那么有整扩张$S^{-1}A\subseteq S^{-1}B$.并且$S^{-1}q_1\subseteq S^{-1}q_2$均为$pA_p$的提升素理想.但是$pA_p$是$S^{-1}A=A_p$的极大理想,引理告诉我们$S^{-1}q_1\subseteq S^{-1}q_2$应该都为$S^{-1}B$的极大理想,从而只能有$q_1=q_2$.
	\item 上升条件.设$A$有素理想$p_1\subseteq p_2$,设$B$中素理想$q_1$提升了素理想$p_1$.考虑整扩张$A/p_1\subseteq B/q_1$,按照提升条件,存在$B/q_1$中的素理想$q_2/q_1$提升了素理想$p_2/p_1$,于是$q_2$提升了$p_2$,并且$q_1\subseteq q_2$.
\end{enumerate}

提升条件换一种说法即结构映射诱导的素谱之间的连续映射是满射.我们给出过另一个满足这个条件的映射是忠实平坦映射:如果$\varphi:A\to B$是忠实平坦映射,任取$A$的素理想$p$,那么$B\otimes_A\kappa(p)$不是零环,任取它的极大理想,考虑如下交换图,说明存在$B$中的素理想提升了$p$.
$$\xymatrix{\mathrm{Spec}(B\otimes_A\kappa(p))\ar[rr]\ar[d]&&\mathrm{Spec}(B)\ar[d]\\\mathrm{Spec}(\kappa(p))\ar[rr]&&\mathrm{Spec}(A)}$$

设$A\subseteq B$是正规整环之间的整扩张,分别记$A,B$的商域为$K,L$.设$K\subseteq L$是一个正规扩张(维数可以无穷),那么所有提升了$p\subset\mathrm{Spec}(A)$的素理想是互相共轭的,换句话讲,如果$q_1,q_2$均为$p$的提升素理想,那么存在$\sigma\in G=\mathrm{Gal}(L/K)$满足$\sigma(p_1)=p_2$.
\begin{proof}
	
	先设$[L:K]<\infty$,记$G=\{\sigma_1,\cdots,\sigma_r\}$.如果对任意指标$j$都有$q_2\not=\sigma_j^{-1}(q_1)$.按照不可比条件,有$q_2\not\subseteq\sigma_j^{-1}(q_1)$.按照Prime avoidance引理,可取$x\in q_2$满足$x\not\in$每个$\sigma_j^{-1}(q_1)$中.记$y=(\prod_j\sigma_j(x))^a$,当$\mathrm{char}K=0$时取$a=1$,当$\mathrm{char}K=p$时取$a=p^v$,其中$v$足够大.$y$是$x$全部共轭元的初等对称多项式,尽管扩张未必是可分的,但是当$v$足够大时可以保证$y\in K$,但是$y\in B$导致$y$在$A$上整,所以按照$A$的正规性有$y\in A$.但是这里$\sigma_j$中必有一个是$L$上的恒等映射,所以$y\in q_2$,说明$y\in q_2\cap A=p\subseteq q_1$,但是这和全体$\sigma_j(x)\not\in q_1$矛盾.
	
	\qquad
	
	再设$[L:K]=\infty$,记$K'\subseteq L$是$G$固定的中间域,那么$L$在$K'$上Galois,并且$K\subseteq K'$是纯不可分扩张.如果$K'\not=K$,必然有$\mathrm{char} K=p>0$.记$A$在$K'$中的整闭包是$A'$.我们断言$A'$中唯一的提升$p$的素理想是$p'=\{x\in A'\mid\exists v\ge0 x^{p^v}\in p\}$:首先$p'$的确是$A'$的素理想,并且$A'$中任意提升了$p$的素理想$q$都满足$p'\subseteq q$.反过来任取$a\in q$,按照$K'/K$是纯不可分扩张,有$a$的极小多项式是$x^{p^v}-c$,按照$a$是$A$上的整元,有$c\in A$,那么$c=a^{p^v}\in q$,于是$c\in A\cap q=p$,于是$a\in p'$,这得到$q\subseteq p'$,也即$q=p'$.
	
	\qquad
	
	通过把$K$替换为$K'$,我们不妨设$L$已经在$K$上是(无限)Galois扩张.对每个包含在$L$中的有限Galois扩张$K\subseteq L'$,记$F(L')=\{\sigma\in G\mid\sigma(q_1\cap L')=q_2\cap L'\}$.我们已经证明了有限扩张的情况,所以至少有$F(L')$非空.但是非空有限集合构成的投射系统的逆向极限总是非空的,所以$\lim\limits_{\leftarrow}F(L')$非空,选取一个元$\sigma$,就满足$\sigma(q_1)=q_2$.
\end{proof}

两个下降定理.
\begin{enumerate}
	\item 设$A\subseteq B$是整环之间的整扩张,$A$是正规的,那么这个扩张满足下降条件.
	\begin{proof}
		
		设$A$的商域为$K$,设$L$是包含了$B$的$K$的正规扩张.设$A$在$L$中的正规化为$C$.设$p_1\subseteq p_2$是$A$中两个素理想,设$B$中素理想$q_2$提升了$p_2$.按照$B\subseteq C$也是整扩张,存在$C$中的素理想$Q_2$提升了$B$中素理想$q_2$.按照$A\subseteq C$是整扩张,存在$C$中素理想$Q_1'$提升了$A$中素理想$p_1$,按照上升定理,还存在$C$中包含了$Q_1'$的素理想$Q_2'$提升了$p_2$.于是我们得到$C$中两个素理想$Q_2$和$Q_2'$均提升了$p_2$.按照上面定理,可取$\sigma\in\mathrm{Gal}(L/K)$,使得$\sigma(Q_2')=Q_2$.于是$Q_1=\sigma(Q_1')$是包含于$Q_2$的$C$中素理想,并且提升了$p_1$.最后取$q_1=Q_1\cap B$,他是包含于$q_2=Q_2\cap B$的素理想,并且提升了$p_2$.
	\end{proof}
	\item 设$B$是平坦$A$代数,那么满足下降条件.
	\begin{proof}
		
		设$A$中有素理想$p_1\subseteq p_2$,设$B$中有素理想$q_2$满足$q_2\cap A=p_2$.我们知道诱导的局部映射$A_{p_2}\to B_{q_2}$是平坦的,局部环上平坦和忠实平坦等价,于是这个局部映射诱导的素谱之间的连续映射是满射,于是存在素理想$q_1\subseteq q_2$使得$q_1\cap A=p_1$.
	\end{proof}
\end{enumerate}

关于维数.
\begin{enumerate}
	\item 如果结构映射$\varphi:A\to B$使得$B$在$A$上整,此时$\varphi$满足不可比条件但未必满足提升条件,说明$B$中的素理想链回拉到$A$上仍然是同样长度的素理想链(即不会出现某些相邻项相同).于是如果设$B$中素理想$q$提升了$A$中素理想$p$,那么总有$\mathrm{ht}(q)\le\mathrm{ht}(p)$和$\mathrm{coht}(q)\le\mathrm{coht}(p)$.特别的,有$\dim B\le\dim A$.
	\item 如果$A\subseteq B$是整扩张,此时扩张同时满足不可比条件和提升条件,于是不仅上一条中$B$的素理想链拉回到$A$上是同样长度的素理想链,反过来$A$中素理想链也可以提升为$B$中同样长度的素理想链.这说明如果$B$的素理想$q$提升了$A$的素理想$p$,此时有$\mathrm{ht}(q)=\mathrm{ht}(p)$和$\mathrm{coht}(q)=\mathrm{coht}(p)$.特别的,有$\dim B=\dim A$.
\end{enumerate}

设有环扩张$A\subseteq B$,设$f\in A[X]$是首一多项式,并且$f$在$B$中分解为$gh$,那么$g,h$的系数都是$A$中的整元.特别的,如果$A$是正规整环,$F$是它的商域,那么$A$上的多项式$f(x)$不可约当且仅当它视为$F$中多项式是不可约的.这一事实推广了UFD及其商域上多项式不可约性的一致性.
\begin{proof}
	
	设$A_1=B[X]/(g)$,设$x_1=X+(g)\in A_1$,那么$1,x_1,x_1^2,\cdots$构成了$A_1$在$B$上的自由基,于是$B\subseteq A_1$.按照$g(x_1)=0$,于是$g$在$A_1$中分解为$(X-x_1)g_1$.反复这个操作可得到$B$的扩环$C$,使得在$C$中$g,h$分解为一次因式乘积,于是它们的根都是$A$上整元,于是$g,h$的系数作为这些整元的初等对称多项式也是$A$上的整元.
\end{proof}

设$A\subseteq B$是整环的整扩张,并且其中$A$是正规整环,那么这个包含映射诱导的素谱之间的连续映射$f:\mathrm{Spec}(B)\to\mathrm{Spec}(A)$是一个开映射.具体的讲,设$t\in B$,设它的整方程(约定极小次数的)为$F(X)=X^n+a_1X^{n-1}+\cdots+a_n\in A[X]$,那么$f(D(t))=\cup_{1\le \le n}D(a_i)$.
\begin{proof}
	
	一方面,如果$q\in D(t)$,从$0=t^n+a_1t^{n-1}+\cdots+a_n=0\in q$得到至少存在某个$a_i\not\in q\cap A=f(q)$,也即$f(q)\in D(a_i)$.
	
	另一方面,设$p\in\cup_{1\le i\le n}D(a_i)$.记$A$的商域为$K$,记$C=A[t]\subseteq B$,记$D_c(t)\subset\mathrm{Spec}(C)$为全体和$t$不交的$C$的素理想构成的集合,我们有如下交换图:
	$$\xymatrix{\mathrm{Spec}(B)\ar[r]&\mathrm{Spec}(C)\ar[r]&\mathrm{Spec}(A)\\V(pB)\cap D(t)\ar[r]\ar[u]&V(pC)\cap D_c(t)\ar[r]\ar[u]&V(p)\ar[u]}$$
	
	这里垂直的映射都是包含映射.容易验证$V(pB)\cap D(t)$是$V(pC)\cap D_C(t)$在$\mathrm{Spec(B)}\to\mathrm{Spec}(C)$下的原像(用一下整扩张$C\subseteq B$满足提升条件).我们断言$V(pB)\cap D(t)$非空,这就等价于$V(pC)\cap D_C(t)$非空.我们说明过对于正规整环$A$,它商域上的代数元在商域里的极小多项式的系数都在$A$中,于是特别的$F(X)$恰好就是$t$的极小多项式,于是有作为$A$模的同构$A[X]/(F(X))\cong A[t]=C$.于是特别的,$C$作为$A$模是有限自由模,并且它的一组基为$1,t,t^2,\cdots,t^{n-1}$.如果$V(pC)\cap D_C(t)=\emptyset$,这等价于$t\in\sqrt{pC}$,于是存在正整数$m\ge n$使得$t^m=\sum_{1\le i\le n}b_t^{n-i},b_i\in p$.特别的,这说明$X^m-\sum_{1\le i\le n}b_iX^{n-i}$被$F(X)$整除,并且它们的商的那个多项式也是$A$系数的.于是特别的在$(A/p)[X]$中有$\overline{F}(X)\mid X^m$.但是按照至少存在某个$a_i\not\in p$,说明$\overline{F}(X)\mid X^m$不能成立.于是$t\not\in\sqrt{pC}$,于是$V(pC)\cap D_C(t)$非空,于是$V(pB)\cap D(t)$非空.
	
	取$q'\in V(pB)\cap D(t)$,记$p'=q'\cap A$,那么有$p\subseteq p'$.于是按照下降定理,存在$B$的素理想$q$满足$q$提升了$p$,并且$q\subseteq q'$.于是$t\not\in q$,于是$p=f(q)\in f(D(t))$.这就说明了$\cup_{1\le i\le n}D(a_i)\subseteq f(D(t))$
\end{proof}

\subsubsection{诺特正规化引理}

诺特正规化引理给出了域上有限型代数的基本结构.经典版本:给定域$k$上一个有限型代数$R=k[x_1,\cdots,x_n]$,那么存在一个整数$r$,和$R$上的代数无关集$y_1,\cdots,y_r$,满足$R$在$S=k[y_1,\cdots,y_r]$上是整的,这等价于说$R$在$S$上是有限的.这里的$r$被$R$唯一决定,它就是$R$的Krull维数.另外当$R$是整环的时候,我们证明过$r$同样是$k$扩张到$R$商域的超越维数,并且按照$\mathrm{Frac}(R)$是$k(y_1,y_2,\cdots,y_r)$的代数扩张,说明$r$同样是$k\subseteq k(y_1,y_2,\cdots,y_r)$的超越维数.
\begin{proof}
	
	证明定理的前半段我们来对$n$归纳.$n=0$时$R$即域$k$,此时代数无关集是空集,并且域的Krull维数为0.现在假设$n>0$,我们来证明存在$R$的子环$R'$,它由$n-1$个元生成,并且$R$在$R'$上有限.那么按照归纳假设,存在$R'$中的代数无关集$\{y_1,y_2,\cdots,y_r\}$满足$R'$在$S=k[y_1,y_2,\cdots,y_r]$上整,于是$R$在$S$上整.
	
	如果$\{x_1,x_2,\cdots,x_n\}$本身已经在$k$上代数无关,则就取$\{y_i\}=\{x_i\}$就得证.现在不妨设$\{x_1,x_2,\cdots,x_n\}$不是代数无关集,那么存在非平凡的关系$\sum_ja_{(j)}x_1^{j_1}\cdots x_n^{j_n}=0$,其中系数$a_{(j)}\in k$并且非零.取$m_2,\cdots,m_n$为正整数,再记$y_i=x_i-x_1^{m_i},2\le i\le n$,做代换$x_i=y_i+x_1^{m_i},2\le i\le n$,得到:
	$$\sum_j c_{(j)}x_1^{j_1+m_2j_2+\cdots+m_nj_n}+f(y_1,y_2,\cdots,y_n)=0$$
	
	其中$f$是一个不包含$x_1$次幂这个单项式的多项式,那么只要取一个足够大的正整数$d$,再取$(m_2,m_3,\cdots,m_n)=(d,d^2,\cdots,d^{n-1})$,就保证上述和式的每个项里,$x_1$的系数都是不同的,于是这是一个$x_1$在$R'=k[y_2,\cdots,y_n]$上的整方程.再注意到每个$y_i,i\ge2$都在$k[x_1,y_2,\cdots,y_n]$上整,于是$R$在$R'$上是整的.
	
	定理后半部分的证明.【】
\end{proof}

诺特正规化引理(加细版本).给定域$k$和有限型$k$代数$R=k[x_1,\cdots,x_n]$,取$R$的一个真理想链$I_1\subseteq I_2\subset\cdots I_r$,那么存在$R$上的一个代数无关组$t_1,\cdots,t_v$满足:
\begin{enumerate}
	\item $R$在$P=k[t_1,\cdots,t_v]$上有限.
	\item 对每个$i=1,2,\cdots,r$,存在一个$h_i$满足$I_i\cap P=\langle t_1,\cdots,t_{h_i}\rangle$.
	\item 如果$k$是无限域,那么可以把每个$t_i$取为$x_i$的$k$线性组合.
\end{enumerate}
\begin{proof}
	
	贼长,见Allen Altman的<A Term of Commutative Algebra>88页.
\end{proof}

零点定理的几个版本.
\begin{enumerate}
	\item Zariski引理.给定域$k$和一个有限生成$k$代数$R$,如果$R$是域,那么$R$是$k$的有限扩张.按照正规化引理,存在$R$的一个多项式子环$P=k[t_1,\cdots,t_v]$,满足$P\subseteq R$是有限扩张也是整扩张.但是这导致$P$是域,于是$v=0$,于是$P=k$,于是$k\subseteq R$是有限扩张.
	\item 取代数闭域$k$,取有限生成代数$R=k[x_1,\cdots,x_n]$,取$R$的极大理想$m$,那么$K=R/m$是一个域,按照Zariski零点定理,得到$K$是$k$的有限扩张,但是$k$代数闭,这导致$k=K$,现在取$a_i\in k$为$x_i$在$K$中的像,取$n=(x_1-a_1,\cdots,x_n-a_n)$,那么$n\subseteq m$.另一方面$k[x_1,x_2,\cdots,x_n]/(x_1-a_1,\cdots,x_n-a_n)\cong k$说明$(x_1-a_1,\cdots,x_n-a_n)$是一个极大理想,于是我们得出代数闭域上有限生成代数的极大理想必然形如$(x_1-a_1,\cdots,x_n-a_n)$.
	\item Hilbert零点定理.给定域$k$,给定有限型$k$代数$R$,任取$R$的一个真理想$I$,那么$I$的根理想是全体包含$I$的极大理想的交.
	\begin{proof}
		
		不妨设$I=0$,否则以商代数代替$R$,于是需要验证的是$R$的幂零根就是Jacobson根.一侧的包含关系是直接的,幂零根必然包含于Jacobson根中,现在假设$f$不在幂零根中,那么$R_f\not=0$,于是可取$R_f$的一个极大理想$n$,设它对应的$R$中的素理想是$m$.从$R$是有限型$k$代数得到$R_f$是有限型$k$代数,于是按照Zariski引理,得到$R_f/n$是$k$的有限扩张,记$K=R/m$,那么$K$是$R_f/n$的子代数,于是$K$同样是$k$的整扩张,导致$K$是域,于是$m$是$R$的极大理想,并且$m$不含$f$,于是$f$不在全部极大理想的交中,这就得到Jacobson根包含于幂零根.
	\end{proof}
\end{enumerate}

本节最后给出零点定理的最一般形式.称一个环是Jacobson环如果它满足如下等价描述的任一条:
\begin{enumerate}
	\item 每个素理想都是若干极大理想的交.
	\item 每个真理想$I$的根理想都是包含了$I$的所有极大理想的交.
\end{enumerate}

$R$是Jacobson环等价于让$R$满足这样一个条件:给定$R$的一个非极大理想的素理想$P$,给定$f\not\in P$,那么$PR_f$总不会是极大理想.
\begin{proof}
	
	一方面如果$R$是Jacobson环,取非极大的素理想$P$,取$f\not\in P$,那么$f\not\in m$,这里$m$是一个包含$P$的极大理想,于是按照局部化素理想对应关系,看到$PR_f$不是极大理想.反过来,任取理想$I$,取$f$不在$I$的根理想中,那么$(R/I)_f$非0,于是存在极大理想$n$.记$m$为$n$在$R$中的原像,那么$I\subseteq m$并且$f\not\in m$,但是从$R_f/mR_f=(R/I)_f/n$看出$m$是极大理想,这说明$R$是Jacobson环.
\end{proof}

如果$P$是$R$的素理想,并且如果$S=R/P$包含一个非0元$b$使得$S[b^{-1}]$是一个域,那么$S$是一个域.
\begin{proof}
	
	一方面如果$R$是Jacobson环,那么它的分式化$S$同样是Jacobson环,再结合$S$是整环,看到$S$的Jacobson根是0.而局部化$S[b^ {-1}]$的素理想对应于$S$中的不含元$b$的素理想,但是既然$S[b^{-1}]$是域,知道唯一个这样的素理想是$(0)$,这说明$(0)$必然是$S$的极大理想,于是$S$是域.反过来,取$R$的一个素理想$Q$,取全部包含$Q$的极大理想的交为$I$,需要证明$Q=I$.不然如果$I\not=Q$,取一个元$f\in I-Q$,那么考虑全体包含$Q$但不包含$f$的素理想,对每一个升链,看到它的并同样包含$Q$但不包含$f$,另外这个并是素理想,于是按照Zorn引理看到总存在一个极大的包含$Q$但不包含$f$的素理想$P$,按照假设知道$P$不会是一个极大理想,因为不然它应该包含$f$,考虑$S=R/P$,这不会是一个域,但是按照局部化的理想对应关系看到$P$对应到$R[f^ {-1}]$的极大理想,这导致了$S[f^{-1}]$是一个域,这和条件矛盾,从而必然有$I=Q$.
\end{proof}

\textbf{零点定理的一般形式}:给定一个Jacobson环$R$,如果$R'$是一个有限型$R$代数,那么$R'$同样是一个Jacobson环,并且,如果$m'\subseteq R'$是一个极大理想,取$m=m'\cap R$,那么$m$是$R$的一个极大理想,并且$R'/m'$是$R/m$的有限扩张.特别的,按照域总是Jacobson环,Hilbert零点定理即本定理的特例.
\newpage
\subsection{完备化}
\subsubsection{线性拓扑}

$A$模$M$上的线性拓扑是指0元存在由子模构成的开邻域基的拓扑.换句话讲这样的拓扑是指存在有向集作为指标集的一族子模$\{M_i\}$,使得全体$(x+M_i)$,其中$x\in M$和$i\in I$,构成了$M$上的拓扑基.
\begin{enumerate}
	\item 首先验证这的确是一个拓扑基.一方面这些基元素的并是整个$M$,另一方面对任意的基元素$x+M_i$和$y+M_j$,设存在交中的元$z$,那么按照$I$是有向集,可取指标$k\ge i,j$,于是有$M_k\subseteq M_i\cap M_j$,于是$z\in z+M_k\subset(x+M_i)\cap(y+M_j)$.
	\item 对于$M$上的线性拓扑,定义中的每个子模$M_i$都是既开又闭的子集.这是因为$M-M_i=\cup_{x\in M-M_i}(x+M_i)$.
	\item 这里线性的含义是,不同点处的开邻域基是线性平移关系,即零元的开邻域基恰好就是$\{M_i\}$,而点$x$的开邻域基为$\{x+M_i\}$.
	\item 对于$M$上的线性拓扑,加法和数乘都是连续映射,具体的讲,映射$\pm:M\otimes M\to M$,$(x,y)\mapsto x\pm y$是连续映射;对固定的$a\in A$,映射$M\to M$,$x\mapsto ax$是连续映射;另外考虑$M=A$的情况,此时$A$的子模恰好就是理想,此时映射$A\times A\to A$,$(a,b)\mapsto ab$是连续映射.
	\begin{proof}
		
		以加法为例,任取$x+y$的开邻域$x+y+M_i$,那么有$(x+M_i)+(y+M_i)\subseteq x+y+M_i$.
	\end{proof}
	\item 对于$M$上的线性拓扑,取子模$N$,它的子空间拓扑恰好是由$\{N\cap M_i\}$诱导的线性拓扑.它的闭包$\overline{N}=\cap_{i\in N}(N+M_i)$:事实上$x\in\overline{N}$当且仅当对每个$i\in I$有$(x+M_i)\cap N$非空,当且仅当对每个$i\in I$有$x\in N+M_i$.
	\item $M$上的线性拓扑是Hausdorff的当且仅当$\cap_iM_i=\{0\}$,当且仅当零元是闭点,当且仅当空间是$T_1$的.
	\begin{proof}
		
		按照每个$M_i$是既开又闭的,于是$\cap_iM_i=\{0\}$推出零元是闭点,按照我们指出的线性拓扑下不同点的开邻域基是平移关系,说明零元是闭点推出每个点都是闭点.最后如果$M$是$T_1$的,那么零元是闭点,于是$\{0\}$的闭包是自身,上一条中取$N=\{0\}$就得到$\cap_iM_i=\{0\}$.
		
		假设$\cap_iM_i=\{0\}$,取两个不同的点$x,y\in M$,于是$x-y\not=0$,于是存在$i_0\in I$使得$x-y\not\in M_{i_0}$,这说明$(x+M_{i_0})\cap(y+M_{i_0})$是空集,于是$M$是Hausdorff空间.
		
		假设$\cap_iM_0\not=\{0\}$,取非零元$x$,那么不存在基元素分离点$x$和零元,这导致空间不是Hausdorff的.
	\end{proof}
	\item $M$的子模$N$是开集当且仅当它包含了某个$M_i$:充分性,此时有$N=\cup_{x\in N}(x+M_i)$;必要性,如果$N$是开集,那么它包含了零元的某个基元素$M_i$.
	\item 取赋予线性拓扑的模$M$的子模$N$,那么商空间$M/N$恰好是经$\{M_i+N/N\}_{i\in I}$诱导的线性拓扑.特别的,如果$N$是某个$M_i$,那么$M/M_i$是离散拓扑;另外如果取$N=\cap_iM_i$,那么$M/\cap_iM_i$是Hausdorff的,它称为关于$M$的可分模.
\end{enumerate}
\subsubsection{线性拓扑的完备化}

设$M$是$A$模,其上有由有向集作为指标集的子模族$\{M_i\}$定义的线性拓扑.$M$关于这个线性拓扑的完备化定义为$\widehat{M}=\varprojlim M/M_i$,其中每个$M/M_i$取离散拓扑,它是$\prod_iM/M_i$的子空间.完备化中的元素可以表示为$(m_i+M_i)_{i\in I}$,满足只要$i\le j$,就有$m_i-m_j\in M_i$.
\begin{enumerate}
	\item 记典范映射$\psi:M\to\widehat{M}$为把$m\in M$映射为完备化中的元$(m)$.记投影映射$p_i:\widehat{M}\to M/M_i$,这个投影映射自然是连续的.记$M^*_i=\ker p_i$,从投影映射的连续性以及$M/M_i$上是离散拓扑,就得到$M^*_i$是完备化上既开又闭的子集.
	\item 容易看出$\ker\psi=\cap_iM_i$,并且复合$p_i\circ\psi$恰好就是典范映射$M\to M/M_i$,这导致$\psi^{-1}(M_i^*)=M_i$.因此如果$M$是Hausdorff空间,这等价于$\ker\psi=\{0\}$,此时$\psi$是单射,并且$M\to\psi(M)$是闭映射,这说明$M$和$\psi(M)$同胚.
	\item $\widehat{M}$上的拓扑即经$\{M_i^*\}_{i\in I}$诱导的线性拓扑.
	\begin{proof}
		
		我们要验证的是两个拓扑是一致的,一方面按照完备化拓扑的定义,每个$x+M_i^*$都是$\widehat{M}$中的开集,于是线性拓扑下的开集都是完备化定义的开集.另一方面,需要验证的是对$\widehat{M}$中的每个非空开集$U$,以及每个点$x\in U$,存在某个$x+M_i^*$包含于$U$.首先存在$I$的有限子集$I_0$,以及$M/M_i$的子集$U_i$,使得$x\in\widehat{M}\cap\left(\prod_{i\in I_0}U_i\times\prod_{i\in I-I_0}M/M_i\right)\subseteq U$.按照\textbf{有向集}的条件,可取一个$i\in I$使得$i>i_0,\forall i_0\in I_0$.于是$M_i^*$的以$I_0$为指标的分量全部是零,导致$x+M_i^*\subseteq U$,这就完成证明.
	\end{proof}
	\item 自然投影$p_i:\widehat{M}\to M/M_i$是满射,特别的,这诱导了同构$\widehat{M}/M_i^*\cong M/M_i$.另外$\psi(M)$总在$\widehat{M}$中稠密,并且$\widehat{M}$总是Hausdorff空间,或者等价的说有$\cap_iM_i^*=\{0\}$.
	\begin{proof}
		
		首先$p_i$是满射因为$p_i\circ\psi:M\to M/M_i$是满射.于是对任意$\widehat{M}$中的元$x$,对任意$i$,可取$M$中的$y$使得$x-\psi(y)\in M_i^*$,此即$\psi(M)$在$\widehat{M}$中稠密.最后按照$M_i^*$的定义有$\cap_iM_i^*=\{0\}$.
	\end{proof}
	\item 完备化只依赖于$M$上的线性拓扑.这句话是指,如果给定两族$M$的子模族,它们诱导了相同的$M$上的线性拓扑,那么它们的完备化是同构的.
	\begin{proof}
		
		假设$\{M_i\}_{i\in I}$和$\{M_j'\}_{j\in J}$是两个子模族,诱导了$M$上相同的拓扑,分别用$\widehat{M}$和$\widehat{M}'$表示对应的完备化.按照拓扑相同,对每个$j\in J$,可取某个$i\in I$使得$M_i\subseteq M_j'$.于是存在典范映射$M/M_i\to M/M_j'$,这得到了复合$\widehat{M}\to M/M_i\to M/M_j'$,这个复合映射不依赖于$i$的选取,按照逆向极限的泛映射性质,得到同态$\widehat{M}\to\widehat{M'}$.对偶的得到$\widehat{M'}\to\widehat{M}$,验证它们互为逆映射,就得到完备化之间的同构.
	\end{proof}
	\item 称赋予线性拓扑的$A$模$M$是完备的,如果典范映射$\psi:M\to\widehat{M}$是双射.此时双射是等价于同胚的,因为有$\psi^{-1}(M^*_i)=M_i$.另外完备化总是完备的模:我们证明过$\cap_iM_i^*=\{0\}$,说明$\widehat{M}$到它完备化的典范映射是单射.另外按照$\widehat{M}/M_i^*\cong M/M_i$可得到典范映射是满射.
	\item 柯西列.给定$A$模$M$,取子模列$M_0=M\supset M_1\supset M_2\supset\cdots$,此时指标集$\mathbb{N}$是一个有向集.称$M$中的一个序列$(m_n)_{n\ge0}$是柯西列,如果对任意正整数$N$,存在正整数$N'$,使得只要$n,n'>N'$,就有$m_n-m_{n'}\in M_{N}$.元素$m\in M$称为$(m_n)$的极限,如果对任意给定的正整数$N$,存在正整数$N'$使得当$n\ge N'$时有$m-m_n\in M_{N}$,如果每个柯西列都存在唯一的极限,就称模$M$柯西完备.
	\item 一个$R$模$M$的全体柯西列$M^*$按照逐项相加和逐项数乘是环$R$的模.其中全体以0为极限的柯西列构成它的一个子模$M^*_0$.它们的商模$M^*/M^*_0$同构于完备化$\widehat{M}$.在这个同构下,从模$M$到完备化$\widehat{M}$的典范映射即把每个元$x\in M$映射为常序列$(x)$在商中的像.另外,柯西完备和我们所定义的模的完备性是等价的.
	\begin{proof}
		
		我们直接构造从$M^*/M^*_0$到$\lim\limits_{\leftarrow}M/M_n$的同构.对任意给定的柯西列$(m_v)$,对固定的$n$,按照柯西列定义,对于足够大的$v$有$m_v$在$M/M_n$中的像是固定的,记这个像为$q_n=q_n'+M_n$.我们断言从$M/M_{n+1}$到$M/M_n$的典范映射$x+M_{n+1}\mapsto x+M_n$把$q_{n+1}$映射为$q_n$:设$v>N_1$的时候恒有$m_v-q_n'\in M_n$,设$v>N_2$的时候恒有$m_v-q_{n+1}'\in M_{n+1}$,于是$v>\max\{N_1,N_2\}$的时候有$q_n-q_{n+1}\in M_{n}$.这就说明$(q_n)$是逆向极限中的元,我们定义$M^*/M^*_0\to\lim\limits_{\leftarrow}M/M_n$即把$(m_v)$映射为$(q_n)$.
		
		这个映射的线性是直接验证的.它是单射因为$(m_v)$极限是零当且仅当全体$q_n=0$.最后来说明它是满射,任取完备化中的元$(q_n)$,记$q_n=m_n+M_n$,那么对正整数$N$,只要$n>N$,就有$m_n-m_N\in M_N$,这说明$(m_n)$是一个柯西列,并且它在$M/M_n$中最终固定的值即$q_n$,于是$(m_n)$的像是$(q_n)$,这完成了满射的证明.
		
		我们之前定义的典范映射为把$M$中的元$m$映射为完备化中的元$(m+M_n)_{n\ge0}$,于是按照上述同构,此即把$M$中的元$m$映射为$M^*/M^*_0$中的$(m)+M^*_0$.最后说明完备性是一致的,如果模$M$是完备的,那么典范映射是双射,于是每个柯西列可以表示为某个常值列$(m)$和一个极限零柯西列的和,于是每个柯西列都收敛;反过来如果模$M$是柯西完备的,那么典范映射$M\to M^*/M^*_0$必然是双射,其中单射是因为,如果非单射,那么存在非零元$m$使得$(m)$收敛于零,而它必然又收敛于$m$,这和柯西列极限唯一矛盾.
	\end{proof}
\end{enumerate}
\subsubsection{$I$-adic完备化}

给定$A$模$M$,它的一个形如$M=M_0\supset M_1\supset\cdots$的子模链称为$M$的滤过,记作$(M_n)$.取定$A$的理想$I$,滤过$(M_n)$称为$I$-滤过,如果对每个自然数$n$有$IM_n\subseteq M_{n+1}$;滤过$(M_n)$称为$I$-稳定滤过,如果对足够大的自然数$n$恒有$IM_n=M_{n+1}$.例如$(I^nM)$是一个$I$-稳定滤过.固定理想$I$和模$M$,那么$M$上的全部$I$-稳定滤过诱导了相同的线性拓扑,这个拓扑称为$I$-adic拓扑.于是$A$模$M$上的$I$-adic拓扑也就是滤过$(M_n)$定义的线性拓扑,它的完备化$\widehat{M}$称为$I$-adic完备化.特别的,取$M=A$得到$A$的完备化$\widehat{A}$是一个环.另外如果$M$是$A$模,那么$\widehat{M}$是$\widehat{A}$模.最后,如果模$M$在$I$-adic拓扑下是完备的,就称$M$是$I$-adic完备模.类似的定义一个环是$I$-adic完备环.
\begin{enumerate}
	\item 
	\begin{enumerate}[(1)]
		\item 我们来验证$M$上的$I$-adic稳定滤过都诱导了相同的线性拓扑.
		\begin{proof}
			
			假设$(M_n)$是$M$上的$I$-稳定滤过,我们来证明它和典范的$I$-稳定滤过$(M_n'=I^nM)$诱导了相同的拓扑.一方面按照定义对每个$n$有$M_n'=I^nM\subseteq M_n$,另一方面不妨设$n\ge n_0$的时候有$IM_n=M_{n+1}$,于是对于任意正整数$n$有$M_{n+n_0}=I^nM_{n_0}\subseteq I^nM=M_n'$,这说明零元处的关于两个线性拓扑的局部基是一致的,因而二者诱导了相同的拓扑.
		\end{proof}
		\item 对$A$和$M$同时赋予$I$-adic拓扑,此时数乘也是连续映射.这只需注意到$(x+I^n)(m+I^nM)\subseteq xm+I^nM$.
		\item 设$A$是$I$-adic完备环,那么有$I$落在$A$的Jacobson根中:任取$a\in I$,那么只需验证对每个$b\in A$有$1+ab$是$A$中的单位元.现在$1-(ab)+(ab)^2-\cdots$是$A$上的一个柯西列,按照完备性得到它收敛,而它就是$1+ab$的逆,得证.
		\item 如果$M$是$I$-adic完备模,任取$a\in I$,那么数乘$1+a$是$M$的自同构.事实上按照$M$是$I$-adic完备的,说明它也是$\widehat{A}$模,于是只需验证$1+a$在$\widehat{A}$中是可逆元.为此记$I_i^*=\ker(\widehat{A}\to A/I^i)$,那么每个$a^i$落在$I_i^*$中,导致$\sum_{i\ge0}(-1)^ia^i$是$\widehat{A}$中的柯西列,它的极限的逆元即$1+a$.
		\item 如果$A$是诺特环,对任意理想$I$,有NAK引理保证$\cap_nI^n=0$,于是此时$A$上的$I$-adic拓扑总是可分的.
	\end{enumerate}
	\item Hensel引理.这件事是说完备局部环上首一多项式在剩余域上的互素分解一定来自它本身的分解.
	\begin{enumerate}[(1)]
		\item 设$(A,m,k)$是局部环,设$A$是$m$-adic完备的.设$F(X)\in A[X]$是首一多项式,记$\overline{F}(X)\in k[X]$是$F(X)$在剩余类域中的像,如果在$k[X]$中有多项式$g,h$互素并且$\overline{F}=gh$,那么存在$A[X]$中的首一多项式$G,H$,使得$\overline{G}=g$,$\overline{H}=h$,并且$F=GH$.
		\begin{proof}
			
			我们来对$n$归纳的构造首一多项式$G_n,H_n\in A[X]$,使得$\overline{G_n}=g$和$\overline{H_n}=h$,并且$F\equiv G_nH_n(\mod m^n)$,并且有$G_n\equiv G_{n-1}(\mod m^{n-1})$和$H_n\equiv H_{n-1}(\mod m^{n-1})$.一旦归纳构造出这样的序列$\{F_n\}$和$\{G_n\}$,所有$F_n$的次数必然是相同的,否则比方说$F_n$和$F_{n+1}$次数不同,按照它们都是首一的,说明$F_n-F_{n+1}$是首一的,但是按照构造这个差在$(m[x])^n$中,这就矛盾.同理全体$G_n$的次数都是相同的.它们作为柯西列的极限分别记作$F,G$,那么这是首一的多项式,并且有$\overline{G}=g$和$\overline{H}=h$,并且$F=GH$.
			
			先构造$n=1$,直接取$G_1,H_1\in A[X]$是$g,h$在$A[X]$中的首一的任意提升.下面假设以及构造了满足要求的$G_n,H_n$,其中$n\ge1$,特别的有$F-G_nH_n\equiv0\mod(m^n)$,于是可记$F-G_nH_n=\sum_{1\le i\le s}w_iU_i(X)$,其中$w_i\in m^n$并且$U_i\in A[X]$.这里$F-G_nH_n$的次数肯定严格小于$F$的次数,因为$F$是首一的,如果次数没变,这个差也是首一的,就和系数落在$m^n$矛盾.于是我们不妨约定对每个$i$总有$\deg U_i<\deg F$.按照$g,h$在$k[X]$中互素,于是可取$k[X]$中多项式$p_i,q_i$使得$\overline{U_i}=gp_i+hq_i$,并且这个分解中可不妨约定$\deg p_i<\deg h$和$\deg q_i<\deg g$.取提升多项式$P_i,Q_i\in A[X]$使得$\overline{P_i}=p_i$和$\overline{Q_i}=q_i$,并且约定$\deg P_i=\deg p_i$和$\deg Q_i=\deg q_i$.于是有:
			$$F-G_nH_n\equiv\sum_{1\le i\le s}w_i(G_nP_i+H_nQ_i)(\mod m^{n+1})$$
			$$F-(G_n+\sum_iw_iQ_i)(H_n+\sum_iw_iP_i)\equiv0(\mod m^{n+1})$$
			
			取$G_{n+1}=G_n+\sum_iw_iQ_i$和$H_{n+1}=H_n+\sum_iw_iP_i$.按照$\deg Q_i<\deg g=\deg G_n$得到$G_{n+1}$是首一的多项式,同理$H_{n+1}$也是首一的多项式.另外按照$w_i\in m^n$得到$G_{n+1}\equiv G_n(\mod m^n)$和$H_{n+1}\equiv H_n(\mod m^n)$,这就完成构造.
		\end{proof}
		\item 推论.设$(A,m,k)$是$m$-adic完备的局部环,设$F(X)\in A[X]$是首一的多项式,设$\overline{F}(X)$是$F(X)$在$k[X]$中的像,如果$\overline{F}(X)$在$k$上有一个单重根$\lambda$,那么$F(X)$在$A$中存在一个单重根$a$,满足$\overline{a}=\lambda$.
	\end{enumerate}
    \item Artin-Rees引理.本质上讲,这个引理是在说取赋予$I$-adic拓扑的模$M$,取子模$N$,那么$N$上的子空间拓扑恰好就是$N$上的$I$-adic拓扑.
    \begin{enumerate}[(1)]
    	\item 一般版本.设$A$是诺特环,设$I$是$A$的理想,设$M$是有限$A$模,设$(M_n)$是$M$的$I$-稳定滤过,取$M$的子模$N$,那么$(N\cap M_n)$是$N$的一个$I$-稳定滤过.
    	\item 经典版本.上一条中取$M_n=I^nM$,得到:设$A$是诺特环,设$M$是有限$A$模,设$N$是$M$子模,设$I$是$A$理想,那么存在正整数$c$使得对每个$n>c$恒有:
    	$$I^nM\cap N=I^{n-c}(I^cM\cap N)$$
    	\begin{proof}
    		
    		这个等式中右侧包含于左侧是总成立的.于是只需找到一个正整数$c$使得$n>c$时$I^nM\cap N\subseteq I^{n-c}(I^cM\cap N)$.按照诺特条件,记$I=(a_1,a_2,\cdots,a_r)$.记$M=Aw_1+Aw_2+\cdots+Aw_s$.那么$I^nM$中的元可以表示为$\sum_{i=1}^sf_i(a_1,a_2,\cdots,a_r)w_i$,其中$f_i(X_1,X_2,\cdots,X_r)\in B=A[X_1,X_2,\cdots,X_r]$是$n$次齐次多项式.对每个$n>0$,记$B_n\subseteq B$为全体$n$次齐次多项式构成的$B$的子模.记$J_n=\{(f_1,f_2,\cdots,f_s)\in B_n^s\mid \sum_{i=1}^sf_i(a_1,a_2,\cdots,a_r)w_i\in N\}$.
    		
    		再记$C\subseteq B^n$是由$\cup_{n>0}J_n$生成的$B$子模.按照$B$是诺特环,$B^s$是有限$B$模,说明$C$是有限$B$模,于是$C$可由$\cup_{n>0}J_n$中有限个元生成,记作$C=Bu_1+Bu_2+\cdots+Bu_t$,其中$u_i=(u_{i1},u_{i2},\cdots,u_{is})\in J_{d_i},\forall 1\le i\le t$.记$c=\max\{d_1,d_2,\cdots,d_t\}$.
    		
    		若$n>c$,并且$\eta\in I^nM\cap N$,可记$\eta=\sum_{1\le i\le s}f_i(a_1,a_2,\cdots,a_s)w_i$,其中$(f_1,f_2,\cdots,f_s)\in J_n$.于是有$(f_1,f_2,\cdots,f_s)=\sum_jp_j(X_1,X_2,\cdots,X_s)u_j$,其中$p_j\in B$.按照每个$f_i$是齐次的,于是不妨约定$p_j$是$n-d_j$次齐次的(否则选取适当的齐次分支).于是有:
    		$$\eta=\sum_{1\le i\le s}f_i(a_1,a_2,\cdots,a_r)w_i=\sum_j\sum_{1\le i\le s}p_j(a_1,a_2,\cdots,a_r)u_{ji}(a_1,a_2,\cdots,a_r)w_i$$
    		$$\in\sum_jI^{n-d_j}(I^{d_j}M\cap N)$$
    		
    		最后按照$c\ge d_j$,就有$\eta\in\sum_jI^{n-d_j}(I^{d_j}M\cap N)\subset_j(I^{n-c}(I^{c-d_j}(I^{d_j}M\cap N)))\subseteq I^{n-c}(I^cM\cap N)$.完成证明.
    	\end{proof}
    \end{enumerate}
    \item $I$-adic完备化函子.设$A$是环,$I$是理想,那么$I$-adic完备化函子是$\textbf{Mod}(A)\to\textbf{Mod}(\widehat{A})$的函子.它把$A$模$M$映为$I$-adic完备化$\widehat{M}$,把$A$模同态$f:M\to N$映为由诱导的$M/I^nM\to N/I^nN$得到的同态$\widehat{f}:\widehat{M}\to\widehat{N}$.并且如果$f$关于$I$-adic拓扑连续,那么$\widehat{f}$也是连续的.
    \begin{enumerate}[(1)]
    	\item 正合性.设$A$是诺特环,那么$I$-adic完备化函子在有限模范畴上是正合函子.
    	\begin{proof}
    		
    		取有限$A$模$M$和任意子模$N$,那么有逆向系统的短正合列:
    		$$0\to(N/N\cap I^nM)_n\to(M/I^nM)_n\to(M/(I^nM+N))_n\to0$$
    		
    		按照Artin-Rees引理有$(N\cap I^nM)$也是$N$上的$I$-adic滤过,于是得到完备化的正合列:
    		$$0\to\widehat{N}\to\widehat{M}\to\widehat{M/N}\to0$$
    	\end{proof}
    	\item 张量表示.设$A$是诺特环,设$I$是理想,对有限$A$模$M$,记$M$和$A$的$I$-adic完备化为$\widehat{M}$和$\widehat{A}$,那么有典范的同构$M\otimes_A\widehat{A}\cong\widehat{M}$.特别的$I$-adic完备诺特环上的有限模自动是$I$-adic完备的.
    	\begin{proof}
    		
    		取$M$的一个有限表示$A^p\to A^q\to M\to0$.按照上一条的正合性,得到完备化和有限直和可交换,并且有正合列$\widehat{A}^p\to\widehat{A}^q\to\widehat{M}\to0$.考虑如下交换图,它的两行都是正合列,于是短五引理说明前两个垂直映射是同构推出第三个垂直映射是同构,完成证明.
    		$$\xymatrix{\widehat{A}^p\ar[r]&\widehat{A}^q\ar[r]&\widehat{M}\ar[r]&0\\A^p\otimes_A\widehat{A}\ar[r]\ar[u]_{\cong}&A^q\otimes_A\widehat{A}\ar[r]\ar[u]_{\cong}&M\otimes_A\widehat{A}\ar[r]\ar[u]&0}$$
    	\end{proof}
        \item 推论.设$A$是诺特环,$I$是理想,$M$和$N$是两个有限模,考虑$I$-adic完备化,那么有函子性的同构:
        $$\widehat{M\otimes_AN}\cong\widehat{M}\otimes_{\widehat{A}}\widehat{N}$$
        $$\widehat{\mathrm{Hom}_A(M,N)}\cong\mathrm{Hom}_{\widehat{A}}(\widehat{M},\widehat{N})$$
        \begin{proof}
        	
        	第二件事是因为有如下一般事实:设$A$是环,$B$是平坦$A$代数,$M$和$N$是$A$模,其中$M$是有限表示$A$模,那么如下典范同态是同构:
        	$$\mathrm{Hom}_A(M,N)\otimes_AB\cong\mathrm{Hom}_B(M\otimes_AB,N\otimes_AB)$$
        \end{proof}
    	\item 记诺特环$A$的$I$-adic完备化是$\widehat{A}$,那么$\widehat{A}$是平坦$A$代数.
    	\begin{proof}
    		
    		按照理想准则,需要验证对任意理想$J$,有$J\otimes_A\widehat{A}\to\widehat{A}$是单射.但是上一条我们证明了$J\otimes_A\widehat{A}=\widehat{J}$,正合性里我们证明了$\widehat{J}\to\widehat{A}$是单射($A$作为$A$模是有限的,于是可用正合性那条的结论).
    	\end{proof}
    \end{enumerate}
    \item Krull相交定理.
    \begin{enumerate}[(1)]
    	\item Kummer相交定理.设$A$是诺特环,$I$是理想,$M$是有限$A$模,记$M$的$I$-adic完备化为$\widehat{M}$,那么$M\to\widehat{M}$的核$E=\cap_{n\ge1}I^nM$恰好由那些被$1+I$中某个元零化的$x\in M$构成.
    	\item 推论.设$A$是诺特环,$I$是包含于$\mathrm{Rad}(A)$的理想,那么对任意有限$A$模$M$,有$M$上的$I$-adic拓扑是可分的,并且任意子模都是闭子集.这里前一件事只要用上一条引理,后一件事只要注意到对$M$的子模$N$,有$M/N$仍然是$I$-adic可分的,于是$N$是$M$的闭子集.
    	\item 推论.设$A$是诺特整环,设$I$是一个真理想,那么$\cap_{n\ge0}I^n=0$.
    \end{enumerate}
    \item 诺特环的$I$-adic完备化总是诺特的.事实上我们甚至可以给出诺特环关于一个理想完备化的具体表述.
    \begin{enumerate}[(1)]
    	\item 引理.设$A$是诺特环,设$I,J$是理想,$M$是有限$A$模,下面完备化都取为$I$-adic完备化,记典范映射$\psi:M\to\widehat{M}$,那么$\widehat{(JM)}=J\widehat{M}$,并且它就是$\psi(JM)$在$\widehat{M}$中的闭包.另外有$\widehat{(M/JM)}\cong\widehat{M}/J\widehat{M}$.
    	\begin{proof}
    		
    		按照$I$-adic完备化在有限模上的正合性,以及Artin-Rees引理,说明$\widehat{JM}$就是$\widehat{M}\to\widehat{M/JM}$的核,于是它就是$\psi(JM)$在$\widehat{M}$中的闭包.现在设$J=(a_1,a_2,\cdots,a_r)$,定义$\varphi:M^r\to M$为$(x_1,x_2,\cdots,x_r)\mapsto\sum_ia_ix_i$,于是$\varphi(M^r)=JM$,于是有正合列$M^r\to M\to M/JM\to0$.取$I$-adic完备化,得到$\widehat{M}^r\to\widehat{M}\to\widehat{(M/JM)}\to0$.这里$\mathrm{im}\widehat{\varphi}=J\widehat{M}$,于是正合性说明$J\widehat{M}=\widehat{JM}$.另外同态基本定理得到$\widehat{(M/JM)}\cong\widehat{M}/J\widehat{M}$.
    	\end{proof}
    	\item 设$A$是诺特环,设$I=(a_1,a_2,\cdots,a_n)$是一个理想,我们断言$A$关于$I$的完备化同构于$A[[X_1,X_2,\cdots,X_n]]/(X_1-a_1,X_2-a_2,\cdots,X_n-a_n)$,于是特别的它是诺特的.
    	\begin{proof}
    		
    		记$B=A[X_1,X_2,\cdots,X_n]$,记$I'=(X_1,X_2,\cdots,X_n)$,记$J=(X_1-a_1,X_2-a_2,\cdots,X_n-a_n)$,那么$B/J\cong A$.那么$A$关于$I$的完备化也就是$B/J$关于$I'$的完备化.于是上一引理说明$\widehat{A}=\widehat{B}/\widehat{JB}=\widehat{B}/J\widehat{B}=A[[X_1,X_2,\cdots,X_n]]/(X_1-a_1,X_2-a_2,\cdots,X_n-a_n)$.
    	\end{proof}
        \item 设$(A,\mathfrak{m})$是诺特局部环,那么$A$的$\mathfrak{m}$-adic完备化$\widehat{A}$也是诺特局部环,它的极大理想是$\mathfrak{m}\widehat{A}$,并且$\widehat{A}$与$A$有相同的剩余类域.更一般的,对任意$n\ge1$总有$\widehat{A}/\mathfrak{m}^n\widehat{A}\cong A/\mathfrak{m}^n$.
        \begin{proof}
        	
        	先证明$\widehat{A}/m^n\widehat{A}\cong A/m^n$.考虑短正合列$0\to m^n\to A\to A/m^n\to0$,诺特条件说明它们都是有限$A$模,于是$m$-adic完备化后保正合性,得到短正合列$0\to\widehat{(m^n)}\to\widehat{A}\to\widehat{(A/m^n)}\to0$.其中我们证明过$\widehat{(m^n)}=m^n\widehat{A}$.另外$A/m^n$是离散拓扑,它本身已经完备,于是$\widehat{A}/m^n\widehat{A}\cong\widehat{(A/m^n)}\cong\widehat{A/m^n}$.
        	
        	上面同构式取$n=1$,说明$m\widehat{A}$是$\widehat{A}$的极大理想.我们已经证明过诺特环关于任意真理想的完备化还是诺特的.最后我们来说明$m\widehat{A}$是$\widehat{A}$唯一的极大理想.任取$a\in\widehat{A}-m\widehat{A}$,仅需验证$a$是$\widehat{A}$中可逆元.按照$\widehat{A}/m\widehat{A}\cong A/m$,存在$a_0\in A$和$b\in m\widehat{A}$使得$a=a_0+b$(这里$A$视为$\widehat{A}$的子环),按照$a$和$a_0$在$\widehat{A}/m\widehat{A}\cong A/mA$中有相同的像,说明$a_0$是$A$中可逆元,于是$a_0$也是$\widehat{A}$中可逆元.而$b\in m\widehat{A}$,说明$b^n$极限为零,于是$a=a_0+b$是可逆元.(这是因为,归结为$a_0=1$的情况,此时$1-b+b^2-b^3+\cdots$是柯西列,它收敛,并且是$1+b$的逆元).
        \end{proof}
    \end{enumerate}
    \item Zariski环.给定诺特环$A$,给定理想$I$,对$A$赋予$I$-adic拓扑,那么如下三个条件互相等价,并且条件成立时称拓扑环$A$为Zariski环.例如诺特局部环$(A,\mathfrak{m})$赋予$\mathfrak{m}$-adic拓扑是Zariski环.
    \begin{enumerate}[(a)]
    	\item $I\subset\mathrm{Rad}(A)$.
    	\item $A$的每个理想都是闭子集.
    	\item $I$-adic完备化$\widehat{A}$是忠实平坦的$A$代数.
    	\item 有限$A$模在$I$-adic拓扑下总是Hausdorff的.
    	\item 有限$A$模的任意子模在$I$-adic拓扑下总是闭子集.
    \end{enumerate}
    \begin{proof}
    	
    	(a)推(e)是Krull相交定理,(e)推(b)平凡,(b)推(c):我们解释过$\widehat{A}$是平坦$A$代数,于是仅需验证对每个极大理想$m$有$m\widehat{A}\not=\widehat{A}$.按照零理想是闭集,说明$\cap_n m^n\widehat{A}=0$,于是典范映射$\psi:A\to\widehat{A}$是单射.于是$A\subset\widehat{A}$的子空间拓扑和$A$上的$I$-adic拓扑是一致的.于是$mA$是$A$中闭集并且它在$\widehat{A}$中闭包就是$m\widehat{A}$.于是$m\widehat{A}\cap A=mA\not=A$,于是$m\widehat{A}\not=\widehat{A}$.
    	
    	\qquad
    	
    	
    	(c)和(d)等价是因为(c)等价于对任意有限$A$模$M$有$M\to\widehat{M}$是单射,也即对任意有限$A$模$M$有它的$I$-adic拓扑是可分的.最后证明(c)推(a):按照$\widehat{A}$是忠实平坦$A$代数,说明典范映射$\psi:A\to\widehat{A}$是单射.并且对每个极大理想$m\subseteq A$,有$m\widehat{A}\not=\widehat{A}$,于是有$m\widehat{A}\cap A=m$(忠实平坦下有$m^{ec}=m$).再按照$\widehat{A}$是$I\widehat{A}$完备的,于是$I\widehat{A}\subset\mathrm{Rad}(\widehat{A})$,于是$m\widehat{A}$是$\widehat{A}$的闭集.于是$m$是$A$的闭集,按照闭包公式就有$\cap_n(m+I^n)=m$.这个等式推出$I\subseteq m$,按照极大理想$m$的任意性,于是$I\subset\mathrm{Rad}(A)$.
    \end{proof}
    \item 半局部环关于大根的完备化.设$A$是半局部环,它只有有限个不同极大理想$m_1,m_2,\cdots,m_r$,记$I=\mathrm{Rad}(A)=m_1m_2\cdots m_r$,那么$A$的$I$-adic完备化$\widehat{A}$可以表示为如下直积,其中$A_i=A_{m_i}$,$\widehat{A_i}$是$A_i$关于$m_iA_i$的完备化.
    $$\widehat{A}=\widehat{A_1}\times\widehat{A_2}\times\cdots\times\widehat{A_r}$$
    \begin{proof}
    	
    	首先$i\not=j$时$m_i+m_j=1$,这可以推出$\forall n\ge1$有$m_i^n+m_j^n=A$,因为比方说设$a\in m_i$和$b\in m_j$使得$a+b=1_A$,那么$(a+b)^{2n}=1_A$,展开左侧后可写作$m_i^n+m_j^n$中的元.另外有$I^n=m_1^nm_2^n\cdots m_r^n$,于是中国剩余定理的条件满足,就有同构式$A/I^n\cong A/m_1^n\times A/m_2^n\times\cdots\times A/m_r^n$.取逆向极限,得到$\widehat{A}\cong\bigtimes_{1\le i\le r}\lim_{\leftarrow}(A/m_i^n)$.按照$m_i\subseteq A$是极大理想,于是有同构$A/m_i^n\cong A_i/m_i^nA_i$,这就得到$\lim_{\leftarrow}(A/m_i^n)\cong\lim_{\leftarrow}(A_i/m_i^nA_i)=\widehat{A_i}$.
    \end{proof}
\end{enumerate}
\subsubsection{可容环}

设$A$是拓扑环.
\begin{itemize}
	\item 称$A$是拓扑是线性拓扑,如果它存在一族由理想构成的0元的开邻域基.
	\item 称$A$是预可容环,如果存在一个开理想$J$称为定义理想(就是指定义了拓扑的理想,它当然不是唯一的),使得对0元的任意开邻域$V$,都存在正整数$n$使得$J^n\subseteq V$(和进制环的区别就在于这些$J^n,n\ge2$未必是开的).
	\item 称$A$是可容环,如果它有定义理想$J$,并且拓扑是Hausdorff和完备的.
\end{itemize}
\begin{enumerate}
	\item 如果$J$是定义理想,对任意开理想$\mathfrak{a}$,都有$J\cap\mathfrak{a}$是定义理想.特别的预可容环$A$的全部定义理想构成了0元的开邻域基.
	\item 拓扑幂零元.设$A$是拓扑环,一个元$x\in A$称为拓扑幂零元,如果序列$(x^n)$的极限为零.
	\begin{enumerate}[(1)]
		\item 设$A$是线性拓扑环,$x\in A$是拓扑幂零的(此即序列$(x^n)$的极限为零)当且仅当对$A$的任意开理想$J$都有$x$在$A/J$中的像是幂零的.进而全体拓扑幂零元构成了$A$的一个理想,记作$\mathfrak{T}$.
		\item 设$A$是预可容环,设$J$是定义理想,那么$x\in A$是拓扑幂零的当且仅当$x$在$A/J$中的像是幂零的.进而$\mathfrak{T}$是$A/J$的幂零根在$A$中的逆像,于是它是开的.
	\end{enumerate}
	\item 设$A$是预可容环,$J$是一个定义理想.
	\begin{enumerate}[(1)]
		\item 一个理想$I$包含在某个定义理想中当且仅当存在某个正整数$n$使得$I^n\subseteq J$.
		\item 一个元素$x\in A$包含在某个定义理想中当且仅当它是拓扑幂零元.
	\end{enumerate}
	\item 预可容环上的开素理想包含了所有定义理想.
	\item 最大定义理想.设$A$是预可容理想,设$J$是一个定义理想,如下命题互相等价.满足这个条件的$J_0$称为最大定义理想,它存在当且仅当$A/J$的幂零根是幂零的,此时$J_0$恰好是全体拓扑幂零元构成的理想$\mathfrak{T}$.特别的,预可容诺特环总存在最大定义理想.
	\begin{enumerate}[(1)]
		\item $J_0$是$A$的最大的定义理想.
		\item $J_0$是$A$的一个极大的定义理想.
		\item $J_0$是一个定义理想,并且$A/J_0$是既约的.
	\end{enumerate}
    \begin{proof}
    	
    	(1)推(2)平凡.(2)推(3)是因为如果$x\in A$是$A/J_0$的幂零元,那么$x$是拓扑幂零元,进而$J_0+Ax$也是定义理想(因为如果$x^n\in J_0$,那么$(J_0+Ax)^n\subseteq J_0$,并且$J_0+Ax$是开的),于是只要$J_0$是极大的定义理想,就有$x\in J_0$.最后(3)推(1)是因为一样的理由说明所有拓扑幂零元都包含在$J_0$中.最后明显的如果最大定义理想存在它一定是拓扑幂零元构成的理想,于是最大定义理想存在等价于讲拓扑幂零元理想$\mathfrak{T}$是定义理想,而这等价于讲存在某个次幂$\mathfrak{T}^n\subseteq J$,按照$\mathfrak{T}$就是$A/J$的幂零根在$A$中的原像,于是这等价于讲$A/J$的幂零根是幂零的.
    \end{proof}
    \item 设$A$是预可容环,如果定义理想$J$满足$(J^n)$构成了0元的开邻域基,那么$A$的任意定义理想都满足这个性质.
    \begin{proof}
    	
    	因为存在正整数$m$使得$J^m\subseteq J'$,进而$J^{mn}\subseteq {J'}^n$,由于$J^{mn}$是开的就得到${J'}^n$是开的.
    \end{proof}
    \item 设$A$是可容环,$J$是一个定义理想,那么$J$包含在$A$的Jacobson根中.
    \begin{proof}
    	
    	任取$x\in J$,任取$a\in A$,归结为证明$1-ax$是$A$的可逆元,但是按照$A$是J-adic完备的,它的逆元就是$1+(ax)+(ax)^2+\cdots$.
    \end{proof}
\end{enumerate}
\subsubsection{进制环}

一个预可容环称为预进制环,如果存在(之前解释过这里存在可以改为任意)定义理想$J$使得$\{J^n\mid n\ge1\}$构成了0元的开邻域基.进制环是指Hausdorff且完备的预进制环.
\begin{enumerate}
	\item 一个线性拓扑环$A$是可容环,当且仅当它是一个由离散环构成的逆向系统$(A_i,u_{ij})$的极限,其中指标集$I$是有向集,并且包含一个最小元记作0,并且典范同态$u_i:A\to A_i$都是满射,并且同态$u_{0i}:A_i\to A_0$的核$J_i$都是幂零的.此时总有$u_0:A\to A_0$的核$J$恰好是$\varprojlim J_i$,并且它是一个定义理想.
	\begin{proof}
		
		必要性.取定一个定义理想$J_0$,取全体包含在$J_0$中的定义理想$\{J_i\}$,取$A_i=A/J_i$,取典范的$u_{ij}:A/J_j\to A/J_i$,那么$(A_i,u_{ij})$是满足条件的逆向系统.我们有典范的同态$\varphi:A\to\varprojlim A_i$,按照$A$是Hausdorff且完备的,就有$\varphi$是拓扑同构.
		
		\qquad
		
		充分性用逆向极限的定义验证$A$以$J$为定义理想并且是Hausdorff且完备的.
	\end{proof}
	\item 设$A$是可容环,设$J$是包含在$A$的某个定义理想中的理想,并且$(J^n)$都是$A$的闭理想,那么$A$在$J$-adic拓扑下还是Hausdorff且完备的.
	\begin{proof}
		
		$A$上的$J$-adic拓扑比原本的拓扑细,这直接得到$J$-adic拓扑是Hausdorff的.至于完备性是基于如下一般事实【bourbaki.Top.III.3.prop10,cor1】:设$G$是交换群,设$\mathscr{T}_1$和$\mathscr{T}_2$是$G$上的和群结构兼容的两个拓扑,并且$\mathscr{T}_1$细于$\mathscr{T}_2$,并且存在0元在$\mathscr{T}_1$下的一族开邻域基使得它们在$\mathscr{T}_2$下是闭的,那么如果$A\subseteq G$赋予子空间拓扑的时候在$\mathscr{T}_2$下是完备的,则它在$\mathscr{T}_1$下也是完备的.
	\end{proof}
	\item 设$A$是可容环,设$J$是包含在$A$的某个定义理想中的理想.考虑连续同态(终端赋予离散拓扑)$u_n:A\to A/J^n$,它诱导了极限同态$u:A\to\widehat{A}$,这里$\widehat{A}$表示$A$的$J$-adic完备化,这个同态未必是连续的.反过来由于$A$上的$J$-adic拓扑比原本拓扑细,于是$A$上恒等映射是从$A$赋予$J$-adic拓扑到$A$赋予原本拓扑的连续映射,这诱导了完备化的连续映射$v:\widehat{A}\to A$.如果$(J^n)$都是$A$的闭理想,上一条得到$A$的$J$-adic拓扑已经完备,于是此时$A$上这两个拓扑是一致的当且仅当$u$是连续的.
	\item 设$A$是可容环,设$J$是定义理想,并且$\{J^n\}$是$A$的闭理想($J$开理想导致它已经是闭理想),那么$A$是诺特环当且仅当$A/J$是诺特的并且$J/J^2$是有限$A/J$模.
	\begin{proof}
		
		条件保证了$A$的$J$-adic拓扑是完备的,于是$A$是诺特的等价于讲$\mathrm{gr}^J(A)$是诺特的.
	\end{proof}
    \item 设$(A_i,u_{ij})$是离散环构成的逆向系统,指标集是有向集并且包含一个最小元0,记$\ker\left(u_{0i}:A_i\to A_0\right)=J_i$,它们满足:
    \begin{enumerate}[(a)]
    	\item 对任意$i\le j$都有$u_{ij}$是满射,并且核是$J_j^{i+1}$.
    	\item $J_1/J_1^2$是$A_0=A_1/J_1$有限模.
    \end{enumerate}
    
    设$A=\varprojlim A_i$,设典范同态$u_i:A\to A_i$的核为$J^{(i+1)}$,那么有:
    \begin{enumerate}[(1)]
    	\item $A$是进制环,并且以$J=J^{(1)}$为定义理想.
    	\item 对任意$n\ge1$有$J^{(n)}=J^n$.
    	\item $J/J^2$同构于$J_1=J_1/J_1^2$,从而它是$A_0=A/J$上的有限模.
    	\item 此时$A$是诺特的当且仅当$A_0$是诺特的.
    	\item 反过来如果$A$是进制环,$J$是定义理想,并且$J/J^2$是有限$A/J$模,那么取$A_i=A/J^{i+1}$满足了这里所有的条件.于是这实际上描述了所有这样类型的进制环,特别的这包含了全部诺特进制环.
    \end{enumerate}
    \begin{proof}
    	
    	按照$u_{ij}$总是满射,得到$u_i$总是满射,结合$J_j^{j+1}=0$,从第一条【ref】得到$A$是可容环.有$J^{(n)}$构成了$A$的0元的开邻域基,于是(2)能推出(1)成立.又按照$J=\varprojlim J_i$,以及$J\to J_i$都是满射,于是$J/J^2\cong J_1/J_1^2$,也即(2)能推出(3).另外(4)和(5)都是直接推论,于是问题归结为证明(2).
    	
    	\qquad
    	
    	$J^{(n)}$作为$A\to A_{n-1}$的核,它恰好是$A$的满足$k<n$的时候有$x_k=0$的元素$(x_k)_{k\ge0}$构成,于是$J^{(n)}J^{(m)}\subseteq J^{(n+m)}$.而$J^{(n)}/J^{(n+1)}$同构于$J^{(n)}$在$A_n$上的投影,但是按照$J^{(n)}=\varprojlim_{i>n}J_i^n$,这个投影就是$J_n^n$.【】
    	
    \end{proof}
    \item 设上一条的条件成立.对每个指标$i$取一个$A_i$模$M_i$,对$i\le j$设$v_{ij}:M_j\to M_i$是一个$u_{ij}$同态,使得$(M_i,v_{ij})$构成逆向系统,再设$M_0$是有限$A_0$模,每个$v_{ij}$是满射,核恰好是$J_j^{i+1}M_j$,记$M=\varprojlim M_i$.那么$M$是有限$A$模并且$u_i$满同态$v_i:M\to M_i$的核就是$J^{n+1}M$,特别的$M_n\cong M/J^{n+1}M$.
    \begin{proof}
    	
    	【】
    \end{proof}
    \item 如果$(N_i,w_{ij})$是另一个满足上一条条件的逆向系统,记$N=\varprojlim N_i$,那么从逆向系统$(h_i)$,其中$h_i:M_i\to N_i$是$A_i$同态,到$A$模同态$h:M\to N$(此时它在$J$-adic拓扑下必然连续)之间存在一一对应.
    \item 特别的,按照第五条,不需要$A$的理想$I$是有限的,只需要$I/I^2$是有限$A/I$模(也等价于有限$A$模),就有$\varprojlim_nA/I^{n+1}$是$A$的$I$-adic完备化$\widehat{A}$,并且此时$\widehat{A}$的定义理想就是$J$在$A\to\widehat{A}$下像集的闭包$\overline{J}$.类似的对$A$模$M$只需要$M/IM$是有限$A/I$模,就有$\varprojlim_nM/I^{n+1}M$是$M$的$I$-adic完备化$\widehat{M}$.
\end{enumerate}
\subsubsection{局部环上的拟有限模}

设$(A,\mathfrak{m},\kappa)$是局部环,一个$A$模$M$称为拟有限模(quasi-finite),如果$M/\mathfrak{m}M$是$\kappa$上的有限维线性空间.
\begin{enumerate}
	\item 设$M$是拟有限$A$模.
	\begin{enumerate}[(1)]
		\item 如果$A$是诺特环,那么在$\mathfrak{m}$-adic完备化下$\widehat{M}$是有限$\widehat{A}$模.
		\begin{proof}
			
			因为$A$是诺特环,有$\mathfrak{m}/\mathfrak{m}^2$是有限$\kappa$模,按照拟有限模的定义,有$M/\mathfrak{m}M$是有限$\kappa$模,我们之前解释过此时$\mathfrak{m}$-adic完备化满足$\widehat{M}=\varprojlim_nM/\mathfrak{m}^{n+1}M$和$\widehat{A}=\varprojlim_nA/\mathfrak{m}^{n+1}$.特别的此时$\widehat{M}$是有限$\widehat{A}$模.
		\end{proof}
		\item 另外如果$A$是完备诺特局部环,$M$在$\mathfrak{m}$-adic拓扑下可分(也即$\cap_n\mathfrak{m}^nM=0$),那么$M$是有限$A$模.
		\begin{proof}
			
			因为$A$完备导致$M$已经是自身的$\mathfrak{m}$-adic完备化.
		\end{proof}
	\end{enumerate}
    \item 设$(A,\mathfrak{m})$和$(B,\mathfrak{n})$是局部环,设$B$是诺特的,设$\varphi:A\to B$是局部环同态,设$M$是有限$B$模.如果$M$是拟有限$A$模,那么$M$上的$\mathfrak{m}$-adic拓扑和$\mathfrak{n}$-adic拓扑是一致的,特别的$M$的$\mathfrak{m}$-adic拓扑是可分的.
    \begin{proof}
    	
    	按照$M/\mathfrak{m}M$是有限长度$\kappa$模,得到$M/\mathfrak{m}M$是有限长度$A$模,于是它自然也是有限长度$B$模.我们断言$\mathfrak{n}$是唯一的包含$\mathrm{Ann}_B(M/\mathfrak{m}M)=\mathfrak{m}B+\mathrm{Ann}_B(M)$的素理想:因为$M/\mathfrak{m}M$是有限$B$模,所以包含这个零化子的素理想构成的集合就是$\mathrm{Supp}_B(M/\mathfrak{m}M)$,我们知道有限长度模可以做正合列分解,使得它同时分解了伴随素理想集合,所以不妨设$M/\mathfrak{m}M$本身是单模,但是这迫使它同构于$B/\mathfrak{n}$,完成断言的证明.按照$B$是诺特环以及上面断言,【P.Samuel交换代数】上的一个结论说明此时$\mathfrak{m}B+\mathrm{Ann}_B(M)$是$B$的$\mathfrak{n}$-adic拓扑的一个定义理想,于是存在正整数$a$满足$\mathfrak{n}^a\subseteq\mathfrak{m}B+\mathrm{Ann}_B(M)\subseteq\mathfrak{n}$,进而对任意$b\ge1$有$\mathfrak{n}^{ab}M\subseteq(\mathfrak{m}B+\mathrm{Ann}_B(M))^bM=\mathfrak{m}^bM\subseteq\mathfrak{n}^bM$,于是$M$上的$\mathfrak{m}$-adic拓扑和$\mathfrak{n}$-adic拓扑是一致的.
    \end{proof}
    \item 推论.设$(A,\mathfrak{m})$和$(B,\mathfrak{n})$是诺特局部环,设$A$是完备的,设$\varphi:A\to B$是局部环同态,设$M$是有限$B$模,并且是拟有限$A$模.那么此时$M$是有限$A$模.
    \begin{proof}
    	
    	这件事是因为$A$是完备诺特局部环,$M$是$A$上拟有限并且$\mathfrak{m}$-adic可分的,我们解释过这些条件推出$M$是有限$A$模.
    \end{proof}
    \item 设$(A,\mathfrak{m})$和$(B,\mathfrak{n})$是局部环,设$B$是诺特的,设$\varphi:A\to B$是局部环同态.此时$B$是拟有限$A$模等价于讲如下两个条件成立:
    \begin{enumerate}[(1)]
    	\item $\mathfrak{m}B$是$B$的定义理想.
    	\item $B/\mathfrak{n}$是有限$A/\mathfrak{m}$模,或者说剩余域扩张是有限的.
    \end{enumerate}

    在条件成立时,每个有限$B$模都是拟有限$A$模.
    \begin{proof}
    	
    	【】
    \end{proof}
    \item 设$(A,\mathfrak{m})$和$(B,\mathfrak{n})$是局部环,设$B$是诺特的,设$\varphi:A\to B$是局部环同态,设$M$是有限$B$模,$M$是拟有限$A$模,设$\mathfrak{b}=\mathrm{Ann}_B(M)$,那么$B/\mathfrak{b}$是拟有限$A$模.
    \begin{proof}
    	
    	不妨设$M$非零.$M$也是$B/\mathfrak{b}$模,用$B/\mathfrak{b}$替换$B$,不妨设$B$模$M$的零化子平凡.按照$M/\mathfrak{m}M$是$A/\mathfrak{m}A$有限模,它的商$M/\mathfrak{n}M$也是有限$A/\mathfrak{m}$模.按照NAK引理有$M/\mathfrak{n}M$非零模.综上$M/\mathfrak{n}M$是非零的有限$A/\mathfrak{m}$模和有限$B/\mathfrak{n}$模,这迫使$B/\mathfrak{n}$作为$A/\mathfrak{m}$模也是有限的,也即$B$是拟有限$A$模.
    \end{proof}
\end{enumerate}
\subsubsection{限制幂级数环}

\begin{enumerate}
	\item 设$A$是完备线性拓扑环,取由开理想构成的0的开邻域基$\{J_i\}$,那么有典范同构$A\cong\varprojlim_iA/J_i$.对任意指标$i$取$B_i=(A/J_i)[T_1,\cdots,T_r]$,那么$\{B_i\}$构成离散环的逆向系统,它的极限记作$A{T_1,\cdots,T_r}$.考虑$A[[T_1,\cdots,T_r]]$的满足$\lim c_{\alpha}=0$的形式幂级数$\sum_{\alpha}c_{\alpha}T^{\alpha}$,称为限制幂级数.全体限制幂级数构成的子环记作$A'$,它的拓扑是这样定义的:任取$A$的0元的开邻域$V$,记$V'=\{\sum_{\alpha}c_{\alpha}T^{\alpha}\mid c_{\alpha}\in V,\forall\alpha\}$,当$V$跑遍$A$的零元的开邻域时,定义$\{V'\}$是$A'$的零元的开邻域基,这定义了$A'$上的可分拓扑.我们断言有典范拓扑同构$A'\cong A{T_1,\cdots,T_r}$.特别的,这说明我们构造的$A{T_1,\cdots,T_r}$不依赖于0元开邻域基的选取.
    \begin{proof}
    	
    	先构造$\varphi:A{T_1,\cdots,T_r}\to A'$:对$\alpha\in\mathbb{N}^r$和指标$i$,定义$\varphi_{i,\alpha}:B_i=(A/J_i)[T_1,\cdots,T_r]\to A/J_i$为把多项式映为$T^{\alpha}$的系数,它的极限是$\varphi_{\alpha}:A{T_1,\cdots,T_r}\to A$.定义$\varphi:A{T_1,\cdots,T_r}\to A'$为把$f=\sum_{\alpha}c_{\alpha}T^{\alpha}$映为$\sum_{\alpha}\varphi_{\alpha}(f)T^{\alpha}$,这个像是限制幂级数:记$f$在$B\to B_i$下的像是$f_i$,记$S_i$是那些$\alpha\in\mathbb{N}^r$,使得$f_i$的$T^{\alpha}$的系数非零,那么$S_i$是有限集合,于是如果我们任取$J_i$,当$J_j\subseteq J_i$和$\alpha\not\in S_i$的时候有$\varphi_{i,\alpha}(f_j)\in J_i$,取极限得到$\varphi_{\alpha}(f)\in J_i$,其中$\alpha\not\in S_i$,这就得到像是限制幂级数.
    	
    	\qquad
    	
    	反过来构造$\psi:A'\to A{T_1,\cdots,T_r}$:记典范同态$\theta_i:A\to A/J_i$,任取$f=\sum_{\alpha}c_{\alpha}T^{\alpha}$,对指标$i$,只有有限个$\alpha$使得$\theta(c_{\alpha})\not=0$,于是可以定义$\psi_i:A'\to B_i$为把$f$映为$\sum_{\alpha}\theta_i(c_{\alpha})T^{\alpha}$,它的极限是联系同态$\psi:A'\to A{T_1,\cdots,T_r}$.最后验证它们互为逆映射即可.
    \end{proof}
    \item 我们有:
    $$\left(A/J_i)[T_1,\cdots,T_r]\right)[T_{r+1},\cdots,T_s]\cong(A/J_i)[T_1,\cdots,T_s]$$
    
    进而取逆向极限得到:
    $$\left(A{T_1,\cdots,T_r}\right){T_{r+1},\cdots,T_s}\cong A{T_1,\cdots,T_s}$$
    \item 设$A$是完备线性拓扑环,它的限制幂级数环$A{T_1,\cdots,T_r}$被如下泛性质描述:对任意完备线性拓扑环$B$,和任意连续同态$u:A\to B$,使得对任意取定的$(b_1,\cdots,b_r)\in B^r$,都存在唯一的连续同态$\overline{u}:A{T_1,\cdots,T_r}\to B$,满足$\overline{u}(a)=u(a),a\in A$和$\overline{u}(T_j)=b_j$.
    $$\xymatrix{A{T_1,\cdots,T_r}\ar[rr]^{\exists_!\overline{u}}&&B\\A\ar[u]\ar@/_1pc/[urr]_u&&}$$
    \item 设$A$是完备线性拓扑环,记$A'=A{T_1,\cdots,T_r}$.
    \begin{enumerate}[(1)]
    	\item 如果$A$是可容环,那么$A'$也是.
    	\item 如果$A$是进制环,$J$是定义理想,并且$J/J^2$是有限$A/J$模.记$J'=JA'$.那么$A'$是以$J'$为定义理想的进制环,并且$J'/{J'}^2$是有限$A'/J'$模.如果额外的$A$是诺特的,那么$A'$也是诺特的.
    \end{enumerate}
    \begin{proof}
    	
    	【】
    \end{proof}
    \item 描述限制幂级数环的商.设$A$是$J$-adic诺特环,设$B$是可容环,设$\varphi:A\to B$是连续同态,如下命题互相等价:
    \begin{enumerate}[(a)]
    	\item $B$是诺特的$JB$-adic环,并且$B/JB$是有限型$A/J$代数.
    	\item $B$作为$A$代数是拓扑同构于$\varprojlim B_n$的,这里$B_n=B_m/J^{n+1}B_m,\forall m\ge n$,并且有$B_1$是有限型$A_1=A/J^2$代数.
    	\item $B$作为$A$代数是拓扑同构于$A{T_1,\cdots,T_r}$的商的.
    \end{enumerate}
    \begin{proof}
    	
    	【】
    \end{proof}
\end{enumerate}



\subsubsection{完备分式化}

设$A$是线性拓扑环,设理想族$\{J_i\}$构成了0元的开邻域基,记典范同态$u_i:A\to A_i=A/J_i$,如果$J_i\subseteq J_j$,就记典范同态$u_{ji}:A_i\to A_j$.设$S\subseteq A$是一个乘性闭子集,记$S_i=u_i(S)$,那么有$u_{ji}(S_i)=S_j$,于是$S_i^{-1}A_i$和典范的$S_i^{-1}A_i\to S_j^{-1}A_j$构成逆向系统,它的极限记作$A\{S^{-1}\}$,称为$A$关于$S$的完备分式化.
\begin{enumerate}
	\item $A\{S^{-1}\}$拓扑同构于$S^{-1}A$的以$\{S^{-1}J_i\}$为0元开邻域基的拓扑的完备化,于是特别的$A\{S^{-1}\}$不依赖于0元开邻域基$\{J_i\}$的选取.
	\begin{proof}
		
		这件事是因为有典范同构$S^{-1}A/S^{-1}J_i\cong S_i^{-1}A_i$.
	\end{proof}
    \item 典范同态$A\to S^{-1}A$是连续的,进而复合上$S^{-1}A\to A\{S^{-1}\}$也是连续的,把这个复合同态$A\to A\{S^{-1}\}$称为完备分式化的典范同态,它就是$A\to S_i^{-1}A_i$诱导的同态.$A\{S^{-1}\}$和典范同态$\rho:A\to A\{S^{-1}\}$满足如下泛性质:对任意Hausdorff且完备的线性拓扑换$A$和任意连续同态$u:A\to B$,都存在唯一的连续同态$u':A\{S^{-1}\}\to B$使得如下图表交换.
    $$\xymatrix{A\{S^{-1}\}\ar[rr]^{u'}&&B\\A\ar[u]^{\rho}\ar@/_1pc/[urr]_u&&}$$
    \item 设$A$是线性拓扑环,取0元的一族开邻域基为$\{J_i\}$,设$S$是乘性闭子集,记$S$在$\widehat{A}$中的像是$\widehat{S}$,那么$A\{S^{-1}\}$典范的等同于$\widehat{A}\{\widehat{S}^{-1}\}$.这件事是因为$\widehat{S}_i^{-1}\widehat{A}_i=S_i^{-1}A_i$.
    \item 如果在$A$中0元不落在$S$的闭包中,那么$A\{S^{-1}\}$一定不是零.
    \begin{proof}
    	
    	只需说明$S^{-1}A$中0不落在$\{1\}$的闭包中(从而$0\not=1$),而这是因为如果0在$\{1\}$闭包中,就有1落在每个$S^{-1}J_i$中,进而$J_i\cap S$总非空,进而在$A$中0落在$S$的闭包中.
    \end{proof}
    \item 设$\varphi:A\to B$是线性拓扑环之间的连续同态,分别取$A$和$B$的乘性闭子集$S$和$T$,使得$\varphi(S)\subseteq T$,那么按照泛性质存在唯一的连续同态$\varphi':A\{S^{-1}\}\to B\{T^{-1}\}$使得如下图表交换.
    $$\xymatrix{A\{S^{-1}\}\ar[rr]^{\varphi'}&&B\{T^{-1}\}\\A\ar[u]\ar[rr]^{\varphi}&&B\ar[u]}$$
    \item 推论.同一个线性拓扑环$A$上的两个乘性闭子集如果满足$S\subseteq T$,那么存在典范的连续同态$\rho_{T,S}:A\{S^{-1}\}\to A\{T^{-1}\}$,它也就是同态$S^{-1}A\to T^{-1}A$的完备化.如果$A$上三个乘性闭子集满足$S\subseteq T\subseteq U$,那么有$\rho_{U,S}=\rho_{U,T}\circ\rho_{T,S}$.
    \item 设$S_1,S_2$是线性拓扑环$A$上的两个乘性闭子集,设$S_2$在$A\{S_1^{-1}\}$中的像是$S_2'$,那么有典范同构$A\{(S_1S_2)^{-1}\}\cong A\{S_1^{-1}\}\{S_2^{-1}\}$.
    \item 理想对应.设$A$是线性拓扑环.
    \begin{enumerate}[(1)] 	
    	\item 设$\mathfrak{a}$是$A$的开理想,那么我们选取的$A$的0元的开邻域基$\{J_i\}$可以要求它们总包含在$\mathfrak{a}$中.进而$S^{-1}\alpha$也是$S^{-1}A$的开理想,把它的完备化记作$\mathfrak{a}\{S^{-1}\}=\varprojlim(S^{-1}\mathfrak{a}/S^{-1}J_i)$,这是$A\{S^{-1}\}$的开理想,因为它包含了$J_i\{S^{-1}\}$,并且有$A\{S^{-1}\}/\mathfrak{a}\{S^{-1}\}$典范同构于$S^{-1}A/S^{-1}\mathfrak{a}=S^{-1}(A/\mathfrak{a})$.
    	\item 反过来设$\mathfrak{b}$是$A\{S^{-1}\}$的开理想,那么存在某个指标$i$使得$J_i\{S^{-1}\}\subseteq\mathfrak{b}$,于是$\mathfrak{b}/J_i\{S^{-1}\}$是$A\{S^{-1}\}/J_i\{S^{-1}\}=S^{-1}A/S^{-1}J_i$的理想,于是存在$A$的理想$\mathfrak{a}$使得$\mathfrak{b}/J_i\{S^{-1}\}=S^{-1}\mathfrak{a}/S^{-1}J_i$,进而$J_i\subseteq\mathfrak{a}$,于是$\mathfrak{a}$是$A$的开理想,进而$\mathfrak{b}/J_i\{S^{-1}\}=\mathfrak{a}\{S^{-1}\}/J_i\{S^{-1}\}$,于是$\mathfrak{b}=\mathfrak{a}\{S^{-1}\}$.
    	\item 映射$\mathfrak{p}\mapsto\mathfrak{p}\{S^{-1}\}$是从$A$的和$S$不交的开素理想到$A\{S^{-1}\}$的开素理想之间的保序一一对应.并且有典范同构$A\{S^{-1}\}/\mathfrak{p}\{S^{-1}\}\cong S^{-1}A/S^{-1}\mathfrak{p}\cong S^{-1}(A/\mathfrak{p})$.
    	\begin{proof}
    		
    		$A\to S^{-1}A$把$A$的和$S$不交的开素理想和$S^{-1}A$的开素理想一一对应;$S^{-1}A\to A\{S^{-1}\}$把$S^{-1}A$的开理想和$A\{S^{-1}\}$的开理想一一对应.
    	\end{proof}
    \end{enumerate}
    \item 设$A$是线性拓扑环,设$S$是乘性子集,记$A'=\{S^{-1}\}$.
    \begin{enumerate}[(1)]
    	\item 如果$A$是可容环,那么$A'$也是可容环,并且对$A$的任意定义理想$J$,都有$J'=J\{S^{-1}\}$是$A'$的定义理想.
    	\item 如果$A$是进制环,设$J$是定义理想,并且$J/J^2$是$A/J$上有限模,那么$A'$也是进制环,并且$J'/{J'}^2$是$A'/J'$有限模,并且有$(J\{S^{-1}\})^n=J^n\{S^{-1}\}$.特别的如果$A$是诺特进制环,则$A'$也是.
    	\begin{proof}
    		
    		取$A_i=A/J^{i+1}$,对$i\le hj$取典范同态$u_{ij}:A_j\to A_i$.设$S$在$A\to A_i$下的像为$S_i$,设$A_i'=S_i^{-1}A_i$,设$u_{ij}$诱导的典范同态$u_{ij}':A_j'\to A_i'$.按照定义有$A_i'$的逆向极限就是$A'$.问题归结为验证进制环的逆向极限描述的条件:
    		\begin{itemize}
    			\item $u_{ij}'$是满射,并且核为$S_j^{-1}(J^{i+1}/J^{j+1})={J'}_j^{i+1}$,其中$J_j'=S_j^{-1}(J/J^{j+1})$.
    			\item $J_1'/{J_1'}^2=S_1^{-1}(J/J^2)$是$A_1'=S_1^{-1}(A/J)$上有限模.
    		\end{itemize}
    	\end{proof}
    \end{enumerate}
    \item 平坦性.
    \begin{enumerate}[(1)]
    	\item 设$A$是诺特进制环,$S$是乘性子集,那么$A\{S^{-1}\}$是平坦$A$代数.
    	\begin{proof}
    		
    		因为这是分式化和$J$-adic完备化的复合,它们都是平坦的.
    	\end{proof}
        \item 设$A$是诺特进制环,有乘性子集$S_1\subseteq S_2$,那么$A\{S_2^{-1}\}$是平坦$A\{S_1^{-1}\}$代数.
        \begin{proof}
        	
        	因为$A\{S_2^{-1}\}=A\{S_1^{-1}\}\{{S_2'}^{-1}\}$.
        \end{proof}
    \end{enumerate}
    \item 设$A$是线性拓扑环,设$f\in A$,记乘性子集$S_f=\{1,f,f^2,\cdots\}$,记$A_{\{f\}}=A\{S_f^{-1}\}$.对$A$的理想$\mathfrak{a}$,记$\mathfrak{a}_{\{f\}}=\mathfrak{a}\{S_f^{-1}\}$.如果$g\in A$,则按照泛性质有典范的连续同态$A_{\{f\}}\to A_{\{fg\}}$.取$A$的乘性子集$S$,那么$\{A_{\{f\}}\}$和相应的$A_{\{f\}}\to A_{\{g\}}$就构成了环的正向系统,这个极限记作$A_{\{S\}}$.于是按照正向极限的泛性质,典范同态$A_{\{f\}}\to A\{S^{-1}\}$就定义了典范同态$A_{\{S\}}\to A\{S^{-1}\}$.
    \begin{enumerate}[(1)]
    	\item 如果$A$是诺特的线性拓扑环,那么$A_{\{S\}}\to A\{S^{-1}\}$是平坦的.
    	\begin{proof}
    		
    		有$A_{\{f\}}\to A\{S^{-1}\}$是平坦的,并且平坦模的正向极限还是平坦的.
    	\end{proof}
        \item 设$A$是可容环,设$\mathfrak{p}$是一个开素理想,设$S=A-\mathfrak{p}$.那么$A\{S^{-1}\}$和$A_{\{S\}}$都是局部环,并且上述典范同态$A_{\{S\}}\to A\{S^{-1}\}$是局部环同态,并且它们的剩余域都典范同构于$A/\mathfrak{p}$的商域.
        \begin{proof}
        	
        	设$J\subseteq\mathfrak{p}$是$A$的一个定义理想,那么$S^{-1}J\subseteq S^{-1}\mathfrak{p}=\mathfrak{p}A_{\mathfrak{p}}$,于是$A_{\mathfrak{p}}/S^{-1}J$是局部环.【】
        	
        \end{proof}
        \item 推论.设$A$是诺特进制环,设乘性子集$S$是某个开素理想的补集,则$A\{S^{-1}\}$和$A_{\{S\}}$都是诺特的,并且$A_{\{S\}}\to A\{S^{-1}\}$是忠实平坦的.
        \begin{proof}
        	
        	之前解释了此时$A\{S^{-1}\}$是诺特的,局部环同态是平坦态射等价于忠实平坦态射,最后如果$A\to B$是忠实平坦的,那么这是单射,并且$\mathfrak{a}\mapsto\mathfrak{a}B$是从$A$的理想集到$B$的理想集的保序单射,于是从$B$诺特得到$A$诺特.
        \end{proof}
    \end{enumerate}
\end{enumerate}
\subsubsection{完备张量积}

设$A$是线性拓扑环,$M,N$是两个线性拓扑$A$模.考虑三元组$(J,V,W)$,这里$J,V,W$分别是$A,M,N$的零元的开子模,满足$JM\subseteq V$和$JN\subseteq W$,那么此时$M/V$和$N/W$都是$A/J$模.那么当$(J,V,W)$跑遍所有满足上述条件的三元组时,就有$\{A/J\}$是环的逆向系统,并且$\{(M/V)\otimes_{A/J}(N/W)\}$是上述环的逆向系统上的模的逆向系统.进而得到$\widehat{A}$上的一个模,称为$M$和$N$在$A$上的完备张量积,记作$\widehat{M\otimes_AN}$.
\begin{enumerate}
	\item 按照$M/V\cong\widehat{M}/\widehat{V}$,于是完备张量积$\widehat{M\otimes_AN}$也等同于$\widehat{\widehat{M}\otimes_{\widehat{A}}\widehat{N}}$.
	\item 我们有典范同构:
	$$(M/V)\otimes_{A/J}(N/W)\cong(M/V)\otimes_A(N/W)\cong(M\otimes_AN)/(\mathrm{Im}(V\otimes_AN)+\mathrm{Im}(M\otimes_AW))$$
	
	于是完备张量积$\widehat{M\otimes_AN}$典范同构于$M\otimes_AN$的以$\{\mathrm{Im}(V\otimes_AN)+\mathrm{Im}(M\otimes_AW)\}$为0元的开邻域基的完备化,其中$V$和$W$分别跑遍$M$和$N$的开子模.我们把$M\otimes_AN$上的这个拓扑称为$M$和$N$上拓扑的张量积拓扑.
	\item 态射的完备化.设$u:M\to M'$和$v:N\to N'$是线性拓扑$A$模之间的连续同态,那么$u\otimes v:M\otimes_AN\to M'\otimes_AN'$关于张量积拓扑也是连续的,于是这诱导了完备化之间的连续同态,记作$u\widehat{\otimes}v:\widehat{M\otimes_AN}\to\widehat{M'\otimes_AN'}$.综上张量完备化是$A$上线性拓扑模上的二元函子.
	\item 代数的完备张量积.设$B$和$C$是两个线性拓扑$A$代数,那么$B\otimes_AC$上以$\mathrm{Im}(\mathfrak{b}\otimes_AC)+\mathrm{Im}(B\otimes_A\mathfrak{c})$为0元的开邻域基的拓扑称为张量积拓扑,这里$\mathfrak{b}$和$\mathfrak{c}$分别跑遍$B$和$C$的开理想.那么$B\otimes_AC$关于这个张量积拓扑的完备化也就是$A/J$代数的逆向系统$\{(B/\mathfrak{b})\otimes_{A/J}(C/\mathfrak{c})\}$的极限.这个代数称为$B$和$C$的完备张量积,记作$\widehat{B\otimes_AC}$.
	\item 代数的完备张量积的泛性质.典范同态$B\to B\otimes_AC$,$b\mapsto b\otimes1$和$C\to B\otimes_AC$,$c\mapsto1\otimes c$在张量积拓扑下都是连续的,它们和完备化的复合记作$\rho_1:B\to\widehat{B\otimes_AC}$和$\rho_2:C\to\widehat{B\otimes_AC}$.代数完备张量积$(\widehat{B\otimes_AC},\rho_1,\rho_2)$满足如下泛性质:对任意Hausdorff且完备的线性拓扑$A$代数$D$,和任意连续$A$同态$u:B\to D$和$v:C\to D$,都存在唯一的连续$A$同态$w:\widehat{B\otimes_AC}\to D$满足如下图表交换:
	$$\xymatrix{&B\ar[d]_{\rho_1}\ar@/^1pc/[ddr]^u&\\C\ar[r]^{\rho_2}\ar@/_1pc/[drr]_v&\widehat{B\otimes_AC}\ar@{-->}[dr]^w&\\&&D}$$
	\item 设$B$和$C$是两个预可容$A$代数,那么$\widehat{B\otimes_AC}$是可容环.并且如果$\mathfrak{b}$和$\mathfrak{c}$分别是$B$和$C$的定义理想,那么$\mathfrak{d}=\mathrm{Im}(\mathfrak{b}\otimes_AC)+\mathrm{Im}(B\otimes_A\mathfrak{c})$在$\widehat{B\otimes_AC}$中的像的闭包是$\widehat{B\otimes_AC}$的定义理想.
	\begin{proof}
		
		因为$\mathfrak{d}^{2n}\subseteq\mathrm{Im}(\mathfrak{b}^n\otimes_AC)+\mathrm{Im}(B\otimes_A\mathfrak{c}^n)$.从$\mathfrak{b}^n$和$\mathfrak{c}^n$趋于0得到$\mathfrak{d}^n$趋于0.
	\end{proof}
    \item 设$A$是进制环,设$J$是定义理想,设$M$是有限$A$模,赋予$J$-adic拓扑,那么对任意诺特进制$A$代数$B$,都有$B\otimes_AM$典范同构于$\widehat{B\otimes_AM}$.
    \begin{proof}
    	
    	取$B$的一个定义理想$\mathfrak{b}$,记结构同态$\varphi:A\to B$,那么$\varphi^{-1}(\mathfrak{b})$是$A$的开理想,于是存在正整数$m$使得$J^m\subseteq\varphi^{-1}(\mathfrak{b})$,进而$J^mB\subseteq\mathfrak{b}$.进而$\mathrm{Im}(B\otimes_AJ^{mn})=\mathrm{Im}(J^{mn}B\otimes_AM)\subseteq\mathfrak{b}^n(B\otimes_AM)$,于是$B\otimes_AM$上的张量积拓扑就是$\mathfrak{b}$-adic拓扑.最后按照$B$是$\mathfrak{b}$-adic完备的,它的有限模$B\otimes_AM$自然也是$\mathfrak{b}$-adic完备的.
    \end{proof}
\end{enumerate}
\newpage
\section{赋值环}
\subsection{一般赋值环}

一个整环$A$称为赋值环,如果它商域$k$中每个非零元$x$满足如果$x\not\in R$,那么$x^{-1}\in R$.如果我们记$A^{-1}$为$A$中所有非零元在$k$中的逆构成的集合,那么这个条件等价于讲$A\cup A^{-1}=k$.此时也会称$A$是$k$的赋值环.
\begin{enumerate}
	\item 例子:
	\begin{itemize}
		\item 如果整环$A$本身就是域,称为平凡赋值环.
		\item 设$p\in\mathbb{Z}$是素数,那么局部化$\mathbb{Z}_{(p)}$是$\mathbb{Q}$的赋值环.
		\item 设$A$是$k$的赋值环,那么对每个中间环$A\subseteq A'\subseteq k$,都有$A'$也是$k$的赋值环.
	\end{itemize}
    \item 赋值环的一些基本性质.
    \begin{enumerate}[(1)]
    	\item 赋值环总是局部环,它的极大理想即$m=\{x^{-1}\in k\mid x\in k-A\}\cup\{0\}$,也即$A$扣去单位群$A^*$.
    	\begin{proof}
    		
    		只需验证这里的$m$是一个理想(因为全体非单位元构成的子集如果是环的理想,那么它必然是唯一的极大理想).任取$x,y\in m$和$a\in A$.假设$ax\not\in m$,也即$ax\in A^*$,于是$x^{-1}=a(ax)^{-1}\in A$,这说明$x\in A^*$矛盾.再不妨设$x,y$均非零元,否则$x+y\in m$是直接的.此时有$x/y$和$y/x$中某个落在$A$中,不妨设$y/x\in A$,于是有$x+y=(1+(y/x))x\in A$.这就得到$m$是理想.
    	\end{proof}
    	\item 赋值环的一个等价描述.设$A$是整环,那么它是赋值环当且仅当它的全体理想在包含序下是一个全序,换句话讲对任意理想$I,J$,要么$I\subseteq J$要么$J\subseteq I$.另外整环$A$是赋值环也等价于它的全体主理想在包含序下是全序集,也即对任意主理想$(a),(b)$,要么$(a)\subset(b)$,要么$(b)\subset(a)$.
    	\begin{proof}
    		
    		必要性,如果$A$是赋值环,如果存在理想$I,J$互不包含,那么存在元$x\in I,x\not\in J$,还存在元$y\in J,y\not\in I$.现在$x/y\in A$导致$x\in J$矛盾,而$y/x\in A$导致$y\in I$也矛盾,于是$I,J$必然有包含关系.
    		
    		充分性,对非零元$x,y\in A$有$x/y\in A$等价于$(x)\subseteq (y)$,于是存在包含关系说明$x/y$和$y/x$中必然有一个在$A$中.
    	\end{proof}
    	\item 赋值环是正规整环.
    	\begin{proof}
    		
    		设$A$是关于$k$的赋值环,任取$x\in k$是$A$上整元,也即有等式$x^n+r_1x^{n-1}+\cdots+r_n=0$,其中$r_i\in A$.倘若$x\in A$那没什么需要证的,倘若$x^{-1}\in A$那么有$x=-(r_1+r_2x^{-1}+\cdots+r_nx^{1-n})\in A$,这说明了正规性.
    	\end{proof}
    	\item 赋值环$A$上的所有有限生成理想都是主理想(满足这个条件的整环称为Bezout环).于是诺特赋值环必然是一个PID.
    	\begin{proof}
    		
    		按照归纳法,只需证明理想$(a_1,a_2)$是主理想,但是我们解释过$(a_1)\subseteq (a_2)$和$(a_2)\subseteq (a_1)$必然有一个成立,于是$(a_1,a_2)=(a_1)$或者$(a_2)$.
    	\end{proof}
    	\item Remark.一般赋值环未必只有两个素理想,但是DVR一定只有两个素理想.
    \end{enumerate}
    \item 赋值环的扩张.设$A$是$k$上的赋值环,取中间环$A\subseteq A'\subseteq k$,那么$A'$也是$k$的赋值环.设$A\not=A'$,设$A$和$A'$的极大理想分别是$m$和$m'$,那么有:
    \begin{enumerate}[(1)]
    	\item $m'\subseteq m\subseteq A\subseteq A'$,并且有$m'\not=m$.
    	\item $m'$是$A$的素理想,并且有$A'=A_{m'}$.
    	\item $A/m'$是关于$A'/m'$的赋值环.
    	\item 任取域$A'/m'$的赋值环$S'$,设它在$A'$中的原像是$S$,那么$S'$也是关于$k$的赋值环,它称为$A'$和$S'$的合成.
    \end{enumerate}
    \begin{proof}
    	
    	第一条.设$x\in m'$,那么$x^{-1}\not\in A'$,于是$x^{-1}\not\in A$,于是$x\in m$,这说明$m'\subseteq m$.再取$y\in A'-A$,那么$y^{-1}\not\in m$但$y^{-1}\in m'$,于是$m'\not=m$.
    	
    	第二条.按照$m'\subseteq A$,于是$m'=m'\cap A$,于是$m'$是$A$的素理想.任取$A-m'$中的元都是$A'$中的可逆元,说明$A_p\subseteq A'$.反过来任取$x\in A'-A$,那么$x^{-1}\in A$,于是$x^{-1}\in A-m'$,于是$x=1/x^{-1}\in A_{m'}$,这就得到$A_{m'}=A'$.
    	
    	第三条.按照$m'=m'\cap A$,说明$A/m'$是$A'/m'$的子环.记典范映射$\varphi:A'\to A'/m'$,我们期望验证的是$\varphi(A)$是赋值环.任取$\varphi(x)\not\in\varphi(A)$,那么$x\not\in A$,那么$\varphi(x^{-1})=\varphi(x)^{-1}\in A/m'$,于是$A/m'$是赋值环.
    	
    	第四条.按照$m'\subseteq S$和$S/m'=S'$.如果$x\in R'$,$x\not\in S$,那么$\varphi(x)$不在$S'$中,于是$\varphi(x^{-1})\in S'$,于是$x^{-1}\in S$.再设如果$x\in k-R'$,那么$x^{-1}\in m'\subseteq S$.
    \end{proof}
    \item 推论.设$(A,m)$是关于域$k$的赋值环,那么:
    \begin{enumerate}[(1)]
    	\item 存在从$\mathrm{Spec}(A)$到全体$A\subseteq k$的赋值中间环的反序双射$p\mapsto A_p$,它的逆映射为把赋值中间环$A'$映射为它的极大理想(此时极大理想本身是$A$的素理想).
    	\item 存在从包含于$A$的全体$k$赋值环到全体关于$A/m$的赋值环之间的保序双射,它把包含于$A$的关于$k$的赋值环$A'$映射为$A/m$的赋值环$A'/m$,逆映射为把$A/m$的赋值环$S'$映射为它在典范映射$A\to A/m$下的原像.
    \end{enumerate}
    \item 赋值环和控制偏序.给定域$k$,我们会对$k$的全部局部子环上赋予一个偏序关系,称为控制,并证明$k$的赋值环即在控制关系下的极大元.给定局部环$(A,m)$和$(B,n)$称$B$控制了$A$,如果$A\subseteq B$并且包含映射是一个局部映射,换句话讲$m=n\cap A$,此时记$A\le B$.如果$A\not=B$并且$B$控制了$A$,就记作$A<B$.
    \begin{enumerate}[(1)]
    	\item 给定域$k$的局部子环$(A,m)$,那么存在$k$的赋值环$(B,n)$控制了$A$.
    	\begin{proof}
    		
    		记$S$为全体$k$这样的子环$B$构成的集合,它满足包含$A$,并且$1_B\not\in pB$.那么$S$是非空集合,因为$A\in S$.赋予$S$包含序,任取全序子集$\{B_i,i\in I\}$,它的并记作$B$,如果有$1_B\in pB$,可设$1_B\in pB_i$,按照$1_B=1_{B_i}$,就和$B_i\in S$矛盾.于是$B\in S$.按照Zorn引理,可取$S$中极大元$B$.按照$pB\not=B$,可取$B$的极大理想$m$包含了$pB$.于是$B\subseteq B_m$并且$1_B\not\in pB_m$,这说明$B_m\in S$,按照极大性得到$B=B_m$,也即$B$是局部环.
    		
    		最后验证$B$是关于$k$的赋值环.任取$x\in k$满足$x\not\in B$.于是$B\subsetneqq B[x]\subseteq k$.于是极大性说明$1\in pB[x]$.于是有等式$1=a_0+a_1x+\cdots+a_nx^n$,其中$a_i\in pB$.按照$a_0\in pB\subseteq m$,说明$1-a_0$是$B$中单位,于是我们可以改进这个等式为$1=b_1x+b_2x^2+\cdots+b_nx^n$的形式,其中$b_i\in pB$.我们设这个等式是满足条件的次数最小的等式.同理如果$x^{-1}\not\in B$,也可取一个次数最小的具有同样形式的等式$1=c_1x^{-1}+c_2x^{-2}+\cdots+c_mx^{-m}$,其中$c_i\in pB$.
    		
    		不妨$n\ge m$,否则以$x^{-1}$代替$x$.在等式$1=c_1x^{-1}+c_2x^{-2}+\cdots+c_mx^{-m}$两侧乘以$b_nx^n$得到$b_nx^n=b_nc_1x^{n-1}+b_nc_2x^{n-2}+\cdots+b_nc_mx^{n-m}$.倘若$n>m$,那么利用这个等式带入$1=b_1x+b_2x^2+\cdots+b_nx^n$消去最高次项,它的常数项仍然在左侧,于是这和$n$的最小性相矛盾;如果$n=m$,带入后得到一个次数为$n-1$的多项式,按照$1-b_nc_m$是单位,仍然可以把它改进为满足相同条件的次数为$n-1$的多项式,这又和$n$的定义相矛盾.于是说明$x$和$x^{-1}$中至少存在一个落在$B$中,于是$B$是赋值环.
    	\end{proof}
    	\item 稍加改进我们可以证明,给定域$k$的局部子环$(A,m)$,那么存在$k$的赋值环$(B,n)$控制了$A$,并且可要求它们的剩余类域扩张$A/m\subseteq B/n$是代数扩张.这个证明只要把上面证明中的$S$改为全部这样的$k$的局部子环,它控制了$A$,并且剩余类域扩张是代数扩张.
    	\item 反过来,我们断言如果存在$k$的局部子环$(B,n)$控制了$k$的赋值环$(A,m)$,那么$A=B$.这就说明了赋值环恰好是控制偏序下的极大元.事实上任取$b\in B$,假设$b\not\in A$,那么$b^{-1}\in A$,于是$b^{-1}$只能是$A$的非单位元,于是$b^{-1}\in m\subseteq n$,这导致$b\not\in B$矛盾.
    \end{enumerate}
    \item 赋值环和正规化.设$k$是域,设$A$是子环,设$A$在$k$中的正规化为$B$,那么$B$是全部$k$的包含了$A$的赋值环的交.特别的,这说明正规整环必然是它商域的全部包含该正规整环的赋值环的交.
    \begin{proof}
    	
    	设全部包含了$A$的$k$的赋值环的交为$B'$,那么$B\subseteq B'$.现在任取$x\in k$不在$A$上整,我们只需找到一个包含$A$的赋值环$C$不包含$x$即可.取$y=x^{-1}$,那么$yA[y]$作为$A[y]$的理想不包含1,否则把等式写出来,适当乘以$x$的次幂会得到$x$的$A$系数首一的零化多项式,这和假设矛盾.于是按照上一个定理的证明,存在一个赋值环$(C,m_C)$,使得$A[y]\subseteq C$,并且$y\in m_C$.最后这个属于关系说明$x\not\in C$,这就得证.
    \end{proof}
    \item Zariski-Riemann空间.
    \begin{enumerate}[(1)]
    	\item 设$k$是域,$A$是$k$的子环,称$k$的赋值环$R$在$A$中有中心,如果$A\subseteq R$,此时称$A$的素理想$m_R\cap A$为$R$的中心.全体在$A$中有中心的$k$赋值环构成的集合记作$\mathrm{Zar}(k,A)$,它称为$k$在$A$上的Zariski空间或者Zariski-Riemann曲面.
    	\item 定义拓扑.对$x_1,x_2,\cdots,x_n\in k$,记$U(x_1,x_2,\cdots,x_n)=\mathrm{Zar}(k,A[x_1,x_2,\cdots,x_n])$,此即$k$的全体包含了$A$和$\{x_1,x_2,\cdots,x_n\}$的赋值环构成的集合.自然有$U(x_1,x_2,\cdots,x_n)\cap U(y_1,y_2,\cdots,y_m)=U(x_1,x_2,\cdots,x_n,y_1,y_2,\cdots,y_m)$,于是全体集合$\{U(x_1,x_2,\cdots,x_n)\mid n\in\mathbb{N}^+,x_i\in k\}$构成构成拓扑基,它作为$\mathrm{Zar}(k,A)$上的拓扑.
    	\item $\mathrm{Zar}(k,A)$是拟紧空间.
    	\begin{proof}
    		
    		开集转化为闭集语言,我们仅需验证对任意一族满足有限交条件(即其中任意有限个集合的交非空)的闭集族$\mathscr{A}$,它的交是非空的.按照Zorn引理,可构造极大的满足有限交条件并且包含$\mathscr{A}$的闭集族$\mathscr{B}$.极大性说明$\mathscr{B}$满足如下三个条件:
    		\begin{enumerate}
    			\item 若$F_1,F_2,\cdots,F_r\in\mathscr{B}$,那么$\cap_{1\le i\le r}F_i\in\mathscr{B}$.
    			\item 若$Z_1,Z_2,\cdots,Z_s$是$\mathrm{Zar}(k,A)$中的闭集,并且$\cup_{1\le i\le s}Z_i\in\mathscr{B}$,那么至少存在某个$Z_i\in\mathscr{B}$.事实上条件$\cup_{1\le i\le s}Z_i\in\mathscr{B}$说明至少存在一个$Z_i$,使得对任意$F_j\in\mathscr{B},1\le j\le r$,总有$Z_i\cap(\cap_{1\le j\le r}F_j)$非空,否则对每个$i$取这样的有限集,它们全部的交是空集,就和$\cup_{1\le i\le s}Z_i\in\mathscr{B}$矛盾.
    			\item $F$是$\mathrm{Zar}(k,A)$中的闭集,并且它包含了$\mathscr{B}$中的某个元,那么$F\in\mathscr{B}$.
    		\end{enumerate}
    		
    		于是为证明$\cap\mathscr{A}$非空,仅需证明$\cap\mathscr{B}$非空.任取$F\in\mathscr{B}$,它的补集$F^c$是若干开集的并$\cup_iU_i$,于是$F=\cap_iU_i^c$.于是第三条知每个$U_i^c\in\mathscr{B}$.另外按照$U(x_1,x_2,\cdots,x_n)^c=\cup_iU(x_i)^c$,第二条说明了$\mathscr{B}$包含了至少一个形如$U(x)^c$的元素.
    		
    		记$\Gamma=\{y\in k\mid U(y)^c\in\mathscr{B}\}$,那么$\Gamma$非空.并且对$R\in\mathrm{Zar}(k,A)$,有$R\in U(y^{-1})^c$当且仅当$y^{-1}\not\in R$,当且仅当$y\in m_R$.于是有$\cap\mathscr{B}=\cap_{y\in\Gamma}U(y^{-1})^c=\{R\in\mathrm{Zar}(k,A)\mid\Gamma\subseteq m_R\}$.
    		
    		记$I\subseteq A[\Gamma]$是由$\Gamma$生成的理想.我们断言$I$不会是单位理想.倘若$1\in I$,那么存在有限个$y_1,y_2,\cdots,y_r\in\Gamma$,使得$1\in\sum_{1\le i\le r}y_iA[y_1,y_2,\cdots,y_r]$.按照有限交条件,$\cap_{1\le i\le r}U(y_i^{-1})^c$非空,任取其中的一个赋值环$R$,那么满足$y_1,y_2,\cdots,y_r\in m_R$,这导致$1\in\sum_{1\le \le r}y_iA[y_1,y_2,\cdots,y_r]\subseteq m_R$矛盾.
    		
    		最后按照$I$是真理想,我们证明过赋值环是控制偏序集下的极大元,于是存在$k$赋值环$R$包含了$A[\Gamma]$,于是这个$R$落在$\{R\in\mathrm{Zar}(k,A)\mid\Gamma\subseteq m_R\}=\cap\mathscr{B}$中,证毕.
    	\end{proof}
    \end{enumerate}
\end{enumerate}
\subsection{赋值环和赋值}
\begin{enumerate}
	\item 全序交换群.给定交换群$H$,称它是全序的,如果$H$存在一个加法下封闭的子集$H^+$,满足存在无交并$H=H^+\cup(-H^+)\cup\{0\}$.如果$x,y\in H$满足$x-y\in H^+$,就记作$x>y$.用$x\ge y$表示$x>y$或者$x=y$.几个注解:
	\begin{enumerate}[(1)]
		\item 首先如果给定$H$的加法下封闭的子集$H^+$,那么此时$\ge$未必是一个全序关系,可能存在$x,y\in H$之间没有关系.条件$H=H^+\cup(-H^+)\cup\{0\}$即等价于讲对任意$x,y\in H$,那么$x\ge y$和$y\ge x$至少一个成立.
		\item 如果我们不要求关系$H=H^+\cup(-H^+)\cup\{0\}$是无交并,那么可能存在这样的情况发生:存在$x\not=y$,同时满足$x>y$和$y>x$.约定它是无交并后我们有所谓的Cantor-Bernstein定理:任取$x,y\in H$,那么$x>y$,$x<y$和$x=y$恰有一个成立.
		\item 我们也可以直接定义一个全序交换群即一个交换群上赋予一个全序关系$\ge$,使得全序结构和加法结构是兼容的,即对任意$a,b,c,d\in H$,从$a\ge c$和$b\ge d$可推出$a+c\ge b+d$.
	\end{enumerate}
	\item 添加无穷大.设$H$是全序交换群,取一个元$\infty\not\in H$,称它为无穷大.我们补充$H\cup\{\infty\}$上的全序关系为,$\forall x\in H$有$x\le\infty$.补充加法为$\infty+x=x+\infty=\infty+\infty=\infty$.此时$H\cup\{\infty\}$构成一个全序半群.
	\item 加法赋值.给定域$k$,其上的赋值是指一个全序交换群$H$,以及一个映射$v:k\to H\cup\{\infty\}$,满足$\forall x,y\in k$,有如下三条成立,我们称子群$v(k^*)\subseteq H$为$v$的赋值群.注意这有一个最简单的例子,即$v(0)=\infty$,$v(x)=0,\forall x\not=0$,这个加性赋值称为平凡赋值,提及赋值我们总排除平凡赋值的情况.
	\begin{enumerate}[(a)]
		\item $v(xy)=v(x)+v(y)$.
		\item $v(x+y)\ge\min\{v(x),v(y)\}$.
		\item $v(x)=\infty\Leftrightarrow x=0$.
	\end{enumerate}
	\item 一些简单推论.
	\begin{enumerate}[(1)]
		\item 第一条说明从$k^*\to H$是一个群同态,于是有$v(1_k)=0$和$v(x)=-v(x^{-1})$.
		\item 我们断言$v(-1)=0$,于是特别的总有$v(x)=v(-x)$.首先从$0=v(1)=v(-1)+v(-1)$得到$2v(-1)=0$.倘若$v(-1)<0$,那么$0=v(-1)+v(-1)<v(-1)$矛盾.同理$v(-1)>0$矛盾.
		\item 如果$v(x)>0$,那么$v(1+x)=0$.事实上$0=v(1)=v(1+x-x)\ge\min\{v(1+x),v(-x)\}$,这里$v(-x)=v(x)>0$,于是只能有$v(1+x)\le0$.但是另一方面$v(1+x)\ge\min\{v(1),v(x)\}\ge0$,于是$v(1+x)=0$.
		\item 上一条可推出,如果$x+y\not=0$,那么$v(x+y)=\min\{v(x),v(y)\}$.不妨设$v(x)>v(y)$,则$y\not=0$,取$z=x/y$,那么$v(x+y)=v(y(1+z))$.其中$v(z)=v(x)-v(y)>0$,于是$v(1+z)=0$,于是$v(x+y)=v(y(1+z))=v(y)=\min\{v(x),v(y)\}$.
	\end{enumerate}
	\item 赋值环和赋值对应.
	\begin{enumerate}[(1)]
		\item 设$v:k\to H\cup\{\infty\}$是$k$上的赋值.取$R_v=\{x\in k\mid v(x)\ge0\}$,取$m_v=\{x\in k\mid v(x)>0\}$.那么$R_v$是$k$上的赋值环,它的唯一极大理想是$m_v$.反过来,$k$的每个赋值环都可以由$k$上某个赋值诱导.
		\begin{proof}
			
			设$R$是$k$上的赋值环.取$G=\{xR\mid x\in k^*\}$.在$G$上赋予偏序为$xR\le yR\Leftrightarrow xR\supset yR\Leftrightarrow x^{-1}y\in R$.按照赋值环的性质,这个偏序关系实际上是全序的.赋予$G$上乘法$xR\cdot yR=(xy)R$,这使得$G$成为一个全序交换群.
			
			最后构造$v:k\to G\cup\{\infty\}$为把$x\in k^*$映射为$xR$,把$0$映射为$\infty$.此时有$v$是赋值,并且$R=R_v$和$m_R=m_v$.
		\end{proof}
	    \item 设$R$是$k$上赋值环,设$v$和$v'$是$k$上两个赋值,它们都诱导了赋值环$R$,即$R_v=R_{v'}=R$.那么如果记$v$和$v'$的赋值群分别为$H$和$H'$,则存在保序的交换群同构$\varphi:H\to H'$,使得$v'=\varphi\circ v$.
	    \begin{proof}
	    	
	    	设$G$是第二条所定义的由赋值环$R$诱导的赋值群,我们只需验证存在满足条件的同构$\varphi:G\to H$,但是这仅需构造$xR\mapsto v(x)$即可.
	    \end{proof}
	\end{enumerate}
	\item 赋值群是$\mathbb{R}$全序子群的准则.
	\begin{enumerate}[(1)]
		\item 设$H$是全序交换群,那么它保序同构于$\mathbb{R}$的全序子群,当且仅当对任意$a,b\in H$,$a>0$,存在自然数$n$使得$na>b$.
		\begin{proof}
			
			必要性是直接的,因为$\mathbb{R}$的加法子群必然满足阿基米德条件.下面证明充分性.不妨约定$H\not=\{0\}$.取$x\in H$满足$x>0$,对每个$y\in H$,存在良性定义的整数$n\in\mathbb{Z}$,使得$nx\le y<(n+1)x$.现在对每个自然数$r$,记$n_r\in\mathbb{Z}$为良性定义的整数,使得$n_rx\le 10^ry<(n_r+1)x$.于是都没个$r\ge0$,就有$10n_r\le n_{r+1}\le 10n_r+9$,于是$0\le n_{r+1}/10^{r+1}-n_r/10^r\le\frac{9}{10^{r+1}}$.于是序列$(n_r/10^r)$是一个柯西列.它的极限我们记作$\varphi(y)$.这个$\varphi$就是$H\to\mathbb{R}$的映射.并且有$n_r/10^r\le\varphi(y)\le(n_r+1)/10^r$.
			
			验证$\varphi$是保序映射.取$y<y'$,$y,y'\in H$,我们断言$\varphi(y)<\varphi(y')$.事实上,取自然数$r$使得$10^r(y'-y)>2x$,记$n_r$和$n_r'$是整数分别满足$n_rx\le 10^ry<(n_r+1)x$和$n_r'x\le 10^ry'<(n_r'+1)x$.于是有$n_rx\le 10^ry<(n_r+1)x<n_r'x\le 10^ry'<(n_r'+1)x$.于是$\varphi(y)\le\frac{n_r+1}{10^r}<\frac{n_r'}{10^r}\le\varphi(y')$.这得到严格保序,并且推出$\varphi$是单射.
			
			验证$\varphi$是同态.若$y,y'\in H$,分别对应于柯西列$(n_r/10^r)$和$(n_r'/10^r)$.那么有$(n_r+n_r')x\le y+y'<(n_r+n_r'+2)x$.于是如果$(m_r)$是由$y+y'$定义的整数序列,就有$n_r+n_r'\le m_r\le n_r+n_r'+1$,于是$\frac{n_r+n_r'}{10^r}\le\frac{m_r}{10^r}\le\frac{n_r+n_r'+1}{10^r}$,对$r$取极限,得到$\varphi(y)+\varphi(y')=\varphi(y+y')$.
		\end{proof}
		\item 设$R$是域$k$上的赋值环,设赋值群为$H$,那么$H$保序同构于$\mathbb{R}$的全序子群当且仅$R$具有Krull维数1.
		\begin{proof}
			
			必要性.按照$H\not=0$,说明$R$不是一个域.设$p$是一个不同于极大理想$m_R$的$R$的素理想.取$x\in p-\{0\}$,取$y\in m-p$.于是$v(y)>0$.按照条件知存在正整数$n$使得$v(y^n)=nv(y)>v(x)$.于是$y^n/x\in R$,于是$y^n=(y^n/x)x\in(x)\subseteq p$,于是$y\in p$矛盾.这说明$p=0$,于是$R$的Krull维数是1.
			
			充分性.任取$a,b\in H$,设$a>0$.不妨也设$b>0$,否则取$n=1$就有$nv(a)>v(b)$.取$x,y\in m$使得$v(x)=a$和$v(y)=b$.按照条件$m$是唯一的素理想包含了$x$和$y$,于是有$\sqrt{(y)}=m$.于是存在正整数$n$使得$x^n\in(y)$,于是$nv(x)\ge v(y)$.
		\end{proof}
	\end{enumerate}
\end{enumerate}
\subsection{非阿基米德赋值}

如果赋值群是$\mathbb{R}$的全序子群,那么赋值还可以描述为绝对值.称域$K$上的一个绝对值,是从$K$到$\mathbb{R}_{\ge0}$的函数$x\mapsto|x|$,满足如下前三个条件.如果第三条替换为更强的第四条,称它是非阿基米德绝对值,不满足第四条的绝对值称为阿基米德绝对值.
\begin{enumerate}[(a)]
	\item $|x|=0$当且仅当$x=0$.
	\item $|xy|=|x||y|$.
	\item $|x+y|\le|x|+|y|$.
	\item $|x+y|\le\max\{|x|,|y|\}$.
\end{enumerate}
\begin{enumerate}
	\item 基本性质.
	\begin{enumerate}
		\item 从前两条得出绝对值是从$K^*$到$R_{>0}$的群同态,取$y=1_k,x\not=0$得到$|1|=1$.另外$K^*$上的单位根的绝对值总是1.于是有$|-1|=1$,$|-x|=|x|$.
		\item 最平凡的情况是对每个非0元$x\in K$定义绝对值为$|x|=1$.称这个绝对值为平凡绝对值.它实际上对应于平凡赋值,提及绝对值时我们总排除平凡绝对值的情况.
		\item 给定$K$上的一个绝对值$|\bullet|$,那么$d(x,y)=|x-y|,x,y\in K$是$K$上的一个度量,于是诱导了$K$上一个拓扑,平凡绝对值诱导了离散拓扑,我们排除了这一情况.我们来证明绝对值诱导的拓扑使这个域成为\textbf{拓扑域},这是指域上的加法和乘法都是$K\times K\to K$的连续映射,取加法负元是$K\to K$的连续映射,取乘法逆元是$K^*\to K$的连续映射.
		\begin{proof}
			
			按照$K$和$K\times K$都是度量空间,它们满足第一可数公理,于是满足Heine归结原理的条件.取$K\times K$上的点列$(a_n,b_n)\to (a,b)$,那么有$K$中$a_n\to a$和$b_n\to b$.于是$|a_n+b_n-a-b|\le|a_n-a|+|b_n-b|$就得到加法的连续性.如果$a_n\to a_0$,断言$|a_n|$有界,事实上对于正数1,可以取$N$使得$n>N$时候有$|a_n-a_0|<1$,由此得到$|a_n|\le|a_0|+1,\forall n>N$,于是取$M=\max\{a_1,\cdots,a_N,|a_0|+1\}$,就有$|a_n|\le M,\forall n\ge1$.那么,对于$(a_n,b_n)\to (a,b)$,看到$|a_nb_n-ab|\le|a_n-a||b_n|+|a||b_n-b|$就得到乘法的连续性.取加法负元的映射是连续的,只要注意到$a_n\to a$,那么$|(-a_n)-(-a)|\to0$.最后说明取乘法逆元是连续映射.取$K^*$中的一列$a_n\to a$,按照$|a_n|$有下界,即$|a_n|\ge M>0$,那么$|\frac{1}{a_n}-\frac{1}{a}|=\frac{|a_n-a|}{|a_n||a|}\le\frac{|a_n-a|}{M|a|}$,这得到$1/a_n\to1/a$,完成证明.
		\end{proof}
	\end{enumerate}
	\item 绝对值的等价.如果两个绝对值诱导的拓扑相同,就说两个绝对值等价.域$K$上绝对值的等价类称为素位.设$|\bullet|_1$和$|\bullet|_2$是域$K$上的两个绝对值,如下条件互相等价:
	\begin{itemize}
		\item $|\bullet|_1$和$|\bullet|_2$诱导了$K$上相同的拓扑.
		\item $|\alpha|_1<1$则$|\alpha|_2<1$.
		\item 存在一个实数$a>0$满足$|\bullet|_1=|\bullet|_2^a$.
	\end{itemize}

    和非阿基米德绝对值等价的绝对值只能是非阿基米德绝对值,和阿基米德绝对值等价的绝对值只能是阿基米德绝对值.于是阿基米德性不止是对绝对值来讲的,还是对绝对值等价类来讲的.如果一个绝对值的等价类由阿基米德绝对值构成,就称它是一个无穷素位;如果等价类由非阿基米德绝对值构成,就称它是一个有限素位.另外注意对阿基米德绝对值$|\bullet|$,对$a>0$,未必有$|\bullet|^a$仍然是绝对值,但是如果$|\bullet|$是非阿基米德绝对值,那么对任意$a>0$都有$|\bullet|^a$是非阿基米德绝对值.
	\begin{proof}
		
		1推2,按照$|a^n|=|a|^n$,于是$a^n\to0$当且仅当$|a|<1$,于是按照$|\bullet|_1$和$|\bullet|_2$诱导相同的拓扑,同一个点列的敛散性是相同的,于是$|a|_1<1\Leftrightarrow|a|_2<1$.2推3,假如$|\bullet|_1$是平凡绝对值,那么条件说明不存在非0元在$|\bullet|_2$下是小于1或者大于1的,于是$|\bullet|_2$也是平凡绝对值.现在设$|\bullet|_1$不是非平凡的,于是可取$y\in K$使得$|y|_1>1$.任取非0的$x\in K$,那么存在一个实数$b$使得$|x|_1=|y|_1^b$,取有理数列$m_i/n_i>b$并且趋于$b$.得到$|x|_1=|y|_1^b<|y|_1^{m_i/n_i}$,导致$|\frac{x^{n_i}}{y^{m_i}}|_1<1$,于是$|\frac{x^{n_i}}{y^{m_i}}|_2<1$,于是$|x|_2<|y|_2^{m_i/n_i}$,让$i$趋于无穷,得到$|x|_2\le|y|_2^b$,再取小于$b$的趋于$b$的有理数列,得到$|x|_2\ge|y|_2^b$,于是$|x|_2=|y|_2^b$对任意的非0元$x$成立.于是设$|y|_1=|y|_2^a$,就有对任意非0元$x$有$|x|_1=|y|_1^b=|y|_2^{ab}=|x|_2^a$.最后3推1是直接的.
	\end{proof}
    \item 逼近定理.设$|\bullet|_1,\cdots,|\bullet|_n$是域$K$上两两不等价的绝对值,任取$a_1,a_2,\cdots,a_n\in K$,对每个$\varepsilon>0$,存在$x\in K$使得$|x-a_i|_i<\varepsilon,\forall i$.
    \begin{proof}
    	
    	首先我们证明可取$z\in K$使得$|z|_1>1$和$|z|_i>1,\forall i\ge2$.先证明$n=2$的情况,可取$\alpha,\beta\in K$使得$|\alpha|_1<1,|\alpha|_2\ge1$和$|\beta|_2<1,|\beta_j|_1\ge1$.记$z=\beta/\alpha$,就有$|z|_1>1$和$|z|_2<1$.现在假设$n>3$,假设已经构造了$z\in K$使得$|z|_1>1$和$|z|_i<1,\forall 2\le i\le n-1$.如果$|z|_n\le1$,取$y\in K$使得$|y|_1>1$和$|y|_n<1$,可取足够大的$m$使得$z^my$满足条件.如果$|z|_n>1$,数列$\{t_m=\frac{z^m}{1+z^m}\}$在$|\bullet|_1$和$|\bullet|_n$下收敛到1,而在$|\bullet|_i,2\le i\le n-1$下收敛到0.于是取足够大的$m$就使得$t_mz$满足条件.
    	
    	对每个$i$,可取$z_i\in K$使得$|z_i|_i>1$和$|z_i|_j<,\forall j\not=i$.那么序列$t_m=\frac{z_i^m}{1+z_i^m}$满足在$|\bullet|_i$下收敛到1,在$|\bullet|_j,j\not=i$下收敛到0.于是对每个$\varepsilon>0$,对每个$i$,可取足够大的$m_i$,使得$x=\sum_ia_iz_i^{m_i}$满足条件.
    \end{proof}
	\item 一个绝对值$|\bullet|$是非阿基米德的,当且仅当这个绝对值在$\{m1\mid m\in\mathbb{Z}\}$上的取值是有界的.特别的,特征非零的域上的绝对值必然是非阿基米德绝对值.
	\begin{proof}
		
		首先如果$|\bullet|$是非阿基米德的,那么有$|m|=|1+1+\cdots+1|\le|1|=1$.反过来,如果存在正整数$N$使得$|m1|\le N$.那么得到$|x+y|^n\le\sum_r|C_n^r||x|^r|y|^{n-r}$其中$|x| $和$|y|$都可以放缩到$\max\{|x|,|y|\}$,而$C_n^r$是整数,于是赋值放缩到条件要求的$N$.于是得到$|x+y|\le N^{1/n}(n+1)^{1/n}\max\{|x|,|y|\}$. 让$n\to+\infty$,得到这个系数趋于1,于是得到$|x+y|\le\max\{|x|,|y|\}$.由此得到赋值是非阿基米德的.
	\end{proof}
    \item 我们可以把域$K$上的绝对值$|\bullet|$延拓到函数域$K(X)$上:给定多项式$f(t)=a_0+a_1t+\cdots+a_nt^n$,定义$|f|=\max\{|a_0|,|a_1|,\cdots,|a_n|\}$.此时强三角不等式$|f+g|\le\max\{|f|,|g|\}$是直接的,另外可验证$|fg|=|f||g|$.
    \item 非阿基米德绝对值上的一些性质.设$(K,|\bullet|)$赋予了非阿基米德绝对值.
    \begin{enumerate}[(1)]
    	\item 如果$|x|\not=|y|$,那么总有$|x+y|=\max\{|x|,|y|\}$.换句话讲每个三角形都是等腰三角形.
    	\begin{proof}
    		
    		不妨设$|y|<|x|$.按照$x=(x+y)+(-y)$,得到$|x|\le\max\{|x+y|,|y|\}=|x+y|\le|x|$.
    	\end{proof}
    	\item $K$上的一个级数$\sum_{n\ge0}a_n$是柯西列当且仅当$\lim\limits_{n\to\infty}a_n=0$.
    	\item 每个圆内的每个点都是它的圆心:记$B(x,r)=\{y\in K\mid |y-x|<r\}$是一个圆盘,任取其中的点$y$和$z$,有$|z-y|=|(z-x)+(x-y)|<r$,于是$B(x,r)\subseteq B(y,r)$.对偶的$B(y,r)\subseteq B(x,r)$.特别的,如果$K$中两个圆相交,那么其中一个圆包含另一个圆.
    	\item 进而,圆盘$B(x,r)=\{y\in K\mid |y-x|<r\}$和$\overline{B}(x,r)=\{y\in K\mid |y-x|\le r\}$和$\partial B(x,r)=\{y\in K\mid |y-x|=r\}$总是既开又闭的.
    	\item $(K,|\bullet|)$上的拓扑是全不连通的.
    	\begin{proof}
    		
    		设子集$A\subseteq K$至少包含了两个不同点$\{x,y\}$,取$d=\frac{1}{2}|x-y|$,那么$A_1=B(x,d)\cap A$和$A_2=A-A_1$是$A$两个非空不交既开又闭子集,分别包含$x$和$y$,从而$A$非连通.
    	\end{proof}
    \end{enumerate}
    \item 赋值群是$\mathbb{R}$的全序子群的赋值的等价类是一一对应于非阿基米德绝对值的等价类的.为此只需对非阿基米德绝对值$|\bullet|$取$v(x)=-\ln|x|$.反过来取定一个实数$\gamma>1$,那么加性赋值$v$对应于非阿基米德赋值$\gamma^{-v(x)}$.另外如果任取另一个$\gamma_1>1$,那么有$\gamma=\gamma_1^s$,得到$\gamma^{-v(x)}$和$\gamma_1^{-v(x)}$总是等价的.于是在等价意义下加性赋值对应的非阿基米德赋值不依赖于底数的选取,只需是大于1的实数即可.
\end{enumerate}



\newpage
\subsection{DVR}

一个赋值环称为离散赋值环(DVR),如果它的赋值群同构于$\mathbb{Z}$.换句话讲,离散赋值环是一个整环$R$,满足存在非平凡的赋值$v:F\to\mathbb{Z}\cup\{\infty\}$,这里$F=\mathrm{Frac}(R)$,使得$R=R_v$.这里离散的意义在于,$\mathbb{Z}$是$\mathbb{R}$的离散子群.

均匀化参数.设$(R,m)$是DVR,按照赋值非平凡,有$v(k^*)$是$\mathbb{Z}$的子群,记作$r\mathbb{Z},r>0$.就称赋值为$r$的$m$中的元$t$为$R$的均匀化参数.此时$R$的每个理想都可以表示为$(t^n)$的形式,并且$(t)=m$.

DVR在赋值环中的描述.设$R$是赋值环,不是域,那么如下条件等价,于是特别的,DVR的Krull维数为1.
\begin{enumerate}
	\item $R$是DVR.
	\item $R$是PID.
	\item $R$是诺特环.
\end{enumerate}
\begin{proof}
	
	1推2按照均匀化参数的性质直接得到.2推3平凡.我们证明过诺特赋值环是PID,于是3推2得证.最后证明2推1.记极大理想$m=(\pi)$,按照$R$不是域说明$\pi\not=0$.于是有严格包含降链$m\supset m^2\supset m^2\supset\cdots$(严格包含因为,否则存在$\pi^i=\pi^{i+1}u$,按照$R$是整环得到$\pi$是单位,这矛盾).我们先来证明$I=\cap_{i\ge0}m^i=0$.
	
	按照$R$是PID,设$I=(x)\subseteq m$,那么存在$y\in R$使得$x=\pi y$.同理,按照$x\in(\pi^i)=m^i$,说明存在$y_i\in R$使得$x=\pi^iy_i$.于是有$0=\pi y-\pi^iy_i=\pi(y-\pi^{i-1}y_i)$.于是整环条件说明$y=\pi^iy_i\in m^{i-1}$,于是$y\in\cap_{i\ge1}m^{i-1}=I$.于是可取$z\in R$使得$y=xz$,于是$0=x-\pi y=x(1-\pi z)$,但是这里$1-\pi z$是单位,导致$x=0$.于是$I=0$.
	
	现在定义赋值.任取$x\in R-\{0\}$,恰好存在一个自然数$v_x$满足$x\in m^{v_x}-m^{v_x+1}$.记$k=\mathrm{Frac}(R)$,对$k$中每个元非零$z=x/y$,定义$v_z=v_x-v_y$,容易验证这个定义不依赖于$z$的分式表示.再令$v_0=\infty$,验证这是一个赋值$v_R:k\to\mathbb{Z}\cup\{\infty\}$.得证.
	
	最后注意到PID的Krull维数不超过1,不是域说明维数非零,于是DVR的Krull维数是1.
\end{proof}

注意一个赋值环如果极大理想是主理想,它未必是DVR.不过我们即将证明添加诺特条件这是成立的:诺特局部环如果极大理想非零并且是主理想,那么它是DVR.另外我们在证明中会看到如果$(R,m)$是DVR,那么$m-m^2$中的每个元$x$都满足$m=(x)$.

DVR在一般环中的描述.设$R$是环,那么如下条件两两等价.另外维数论中我们会证明DVR实际上就是一维正则局部环.
\begin{enumerate}
	\item $R$是DVR.
	\item $R$是局部PID,并且不是域.
	\item $R$是诺特局部环,维数非零,并且极大理想是主理想.
	\item $R$是一维正规诺特局部环.
\end{enumerate}
\begin{proof}
	
	1推2推3我们给出过.下面证明3推1.记$m=xR$.首先$x$不是幂零的,否则$m$是$R$的唯一的素理想,导致$\dim R=0$矛盾.于是有严格包含的降链$R=x^0R\supset xR\supset x^2R\supset\cdots$.按照$R$是诺特环,由Krull相交定理得到$\cap_{n\ge0}m^n=0$.下面构造赋值.对每个非零元$y\in R$,存在唯一的自然数$n_y$使得$y\in x^{n_y}R-x^{n_{y+1}}R$.记$y=x^{n_y}u_y$,那么$u_y\not\in m$,于是$u_y$是单位.假设$z\in R$是另一个非零元,那么有$yz=x^{n_y+n_z}u_yu_z\not=0$(因为$x$不是幂零元).于是这说明了$R$是整环.再记$k=\mathrm{Frac}(R)$,对$w=y/z\in k$,定义$n_w=n_y-n_z$.这个定义不依赖于分式表示的选取.验证$w\mapsto n_w$是$k^*\to\mathbb{Z}$的赋值,导致$R$是DVR.
	
	我们证明过DVR是一维诺特局部环,也验证过DVR是正规整环,于是1推4得证.最后验证4推3,也即条件4下极大理想是主理想.记$k=\mathrm{Frac}(R)$,取$x\in m-\{0\}$,按照$\dim R=1$,说明唯一包含$x$的素理想是$m$.于是$R/xR$的伴随素理想仅有$m$一个(伴随素理想的存在性由诺特条件保证).记$y\in R$使得$\mathrm{Ann}_R(\overline{y})=m$.换句话讲$m=\{a\in R\mid ay\in xR\}$.记$z=yx^{-1}\in k$,那么$z\not\in R$.但是$zm$是$R$的一个理想.倘若$zm\subseteq m$,那么$R[z]$是$R$上有限模,导致$z$在$R$上整,按照正规条件说明$z\in R$,矛盾.于是$zm=R$,特别的,存在$t\in m$使得$zt=1$,于是$m=mzt=(mz)t\subseteq Rt\subseteq m$,于是$m=Rt$.
\end{proof}

设$R$是诺特整环,$p$是非零素理想,如果$p$是可逆的(即如果设$I^{-1}=(R:I)_k$,那么$II^{-1}=R$),那么$\mathrm{ht}(p)=1$,并且$R_p$是DVR.
\begin{proof}
	
	我们证明过如果$I$是整环$R$的分式理想,那么对任意素理想$p$有$(IR_p)^{-1}=I^{-1}R_p$,我们还证明过$R$的理想$I$可逆的一个等价描述是它是有限生成的,并且对每个素理想$p$有$IR_p$是$R_p$的主理想.
	
	这里$p$是可逆的素理想,于是$pR_p$是$R_p$的可逆素理想,并且它是一个主理想,于是$R_p$是维数不为零的(因为$p$不是零)局部诺特整环,于是$R_p$是DVR.
\end{proof}

定理.
\begin{enumerate}
	\item 设$R$是诺特整环,记商域为$k$,那么如下两个条件等价:
	\begin{enumerate}
		\item 对每个非零非单位的元$b\in R$,有$R/bR$的伴随素理想都是极小的,并且高度都为1.
		\item $R_p$对全部高度为1的素理想$p$取交是$R$.
	\end{enumerate}
	\begin{proof}
		
		1推2,设$x=a/b\in k$,其中$a,b\in R$,$b\not=0$,满足$x\in R_p$对任意高度为1的素理想$p$成立.特别的,这说明对任意这样的素理想$p$总有$a\in bR_p$.不妨设$b$不是单位元,否则已经有$x=ab^{-1}\in R$.记$R/bR$的全部伴随素理想为$\{p_1,p_2,\cdots,p_r\}$,记$bR=q_1\cap q_2\cap\cdots\cap q_r$是最短准素分解,其中$q_i$是$p_i$准素理想.按照条件有么个$p_i$都是高度1的素理想,于是它们都是极小素理想.于是这里$q_i$被$xR$唯一决定,它就是$bR_{p_i}$在$R\to R_{p_i}$下的原像.于是$a\in q_i,\forall i$,于是$a\in bR$,也即$x\in R$.
		
		2推1,任取$0\not=b\in R$,并且$b$不是单位,设$\{p_1,p_2,\cdots,p_r\}$是包含$b$的高度为1的素理想,它们也是包含$b$的极小素理想.我们断言典范映射$R/bR\to\oplus_{1\le i\le r}(R/bR)_{p_i}\cong\oplus_{1\le i\le r}R_{p_i}/bR_{p_i}$是单射:设$a\in R$在这个映射下为零,也就是说$a\in bR_{p_i}$对每个$i$成立.假设$p$是一个异于$p_i$的高度为1的素理想,那么$b\in R_p^*$,于是自然有$a\in bR_p$.按照上一段我们证明的方法,得到$x=a/b$满足$x\in R_p$对任意高度1的素理想$p$成立.于是条件得到$x\in R$,于是$a=xb\in bR$,也即这个映射是单射.
		
		这个典范映射是单射说明$\mathrm{Ass}(R/bR)\subset\cup_{1\le i\le r}\mathrm{Ass}_R(R_{p_i}/bR_{p_i})=\cup_{1\le i\le r}\mathrm{Ass}_{R_{p_i}}(R_{p_i}/bR_{p_i})=\{p_1,p_2,\cdots,p_r\}$.这里最后一个等式是因为每个$R_{p_i}$是维数1的局部整环,于是$p_iR_{p_i}$是唯一包含$b$的素理想.
	\end{proof}
	\item 设$R$是诺特正规整环,记商域为$k$,那么上一条中的条件都满足.
	\begin{proof}
		
		我们只需验证条件下,取$x\in R$非零非单位,那么$\mathrm{Ass}(R/xR)$由高度1的素理想构成.任取$p\in\mathrm{Ass}(R/xR)$,按照$x\subseteq p$说明$p$非零.按照$R$是诺特的说明$p\in\mathrm{Ass}(R_p/xR_p)$.于是以$R_p$替换$R$,不妨约定$R$是维数非零的正规诺特局部整环,并且$p$是极大理想.记$y\in R$使得$\mathrm{Ann}_R(y)=p$,那么$p=\{a\in R\mid ay\in xR\}$.特别的,$z=yx^{-1}\in k$不在$R$中.我们断言$zp=R$:一方面$zp=ypx^{-1}\subseteq xRx^{-1}\subseteq R$;另一方面如果$zp\subseteq p$,按照$R$是诺特环,说明$z$是$R$上整元,于是条件得到$z\in R$矛盾,于是$zp=R$.
		
		最后说明$\mathrm{ht}(p)=1$,只需验证作为$R$的分式理想,它是可逆的(我们证明过如果$p$是诺特整环的非零素理想,如果它可逆,那么它的高度1).但是按照$zp=R$,说明$z\in p^{-1}$,于是$pp^{-1}=R$,得证.
	\end{proof}
	\item 一个注解.更一般的,对诺特环$R$,如果$x\in R$不是零因子也不是单位,那么包含$x$的极小素理想的高度为1,这是Krull主理想定理.
	\item 设$R$是诺特整环,那么如下两个条件等价:
	\begin{enumerate}
		\item $R$是正规的.
		\item 对每个高度为1的素理想$p$,有$R_p$是DVR,并且对每个非零非单位的元$x\in R$,有$R/xR$的伴随素理想的高度都是1.
	\end{enumerate}
	\begin{proof}
		
		1推2按照第一条和第二条得证.2推1,第一条说明条件推出$R$是$R_p$让$p$取遍高度1的素理想时的交.这里每个$R_p$都是DVR,于是是正规的,于是交也是正规的.
	\end{proof}
\end{enumerate}
\newpage
\subsection{戴德金整环}

分式理想和可逆理想的定义.设$R$是整环,$k$是它的商域.
\begin{itemize}
	\item 分式理想.称$R$模$k$的子模$I$是$R$的分式理想,如果存在$a\in R$使得$aI\subset R$.分式理想可以理解为具有公共分母的理想.
	\item 可逆分式理想.称$R$的分式理想$I$是可逆理想,如果$I^{-1}=(R:I)_k=\{a\in K\mid aI\subset R\}$满足$I^{-1}I=R$.
\end{itemize}
\begin{enumerate}
	\item 分式理想定义中的$I\to aI$,$x\mapsto ax$是一个$R$模同构.于是如果整环$R$是诺特的,那么分式理想可以等价的定义为$R$模$k$的有限生成子模.
	\item 任取$a\in k^*$,那么$I=aR$是一个可逆分式理想,它的逆就是$\frac{1}{a}R$,$I$也称为主分式理想.
	\item 如果$I$是$R$的分式理想,那么$I^{-1}=(R:I)$也是分式理想.事实上任取非零元$a\in I$,那么$aR\subset I$,于是$aI^{-1}\in R$.
	\item 整环$R$的全部分式理想构成一个(交换)乘法半群,全部可逆分式理想构成一个交换群,幺元都是$R$.另外如果$I$和$J$是分式理想,那么$IJ$是可逆分式理想当且仅当$I$和$J$都是可逆的.
	\begin{proof}
		
		只需验证必要性.必要性只需验证此时有$(IJ)^{-1}=I^{-1}J^{-1}$.一方面$I^{-1}J^{-1}\subset(IJ)^{-1}$.另一方面从$IJ(IJ)^{-1}\subset R$得到$I(IJ)^{-1}\subset J^{-1}$以及$J(IJ)^{-1}\subset I^{-1}$,于是$(IJ)^{-1}=(IJ)^{-1}(IJ)^{-1}(IJ)\subset I^{-1}J^{-1}$.
	\end{proof}
	\item 设$I$是$R$模$k$的子模,那么$I$是可逆分式理想等价于要求存在$a_1,\cdots,a_n\in I$和$q_1,\cdots,q_n\in k$,满足每个$q_iI\subset R$,并且有$1=\sum_{i=1}^{n}q_ia_i$.并且此时有$I=(a_1,\cdots,a_n)$,即可逆分式理想必然是有限生成的模.
	\begin{proof}
		
		事实上如果这个条件成立,那么$I$必然是$R$模$Q$的子模,现在我们断言$I$是有限生成的,对任意的$t\in I$,有$t=\sum_{i=1}^{n}q_ia_it$,其中$q_it\in R$,这说明$a_1,\cdots,a_n$生成了整个理想.接下来说明存在一个分式理想$I^{-1}$满足$II^{-1}=R$.取$q_1,\cdots,q_n$生成的$R$模$Q$的子模记作$J$,那么$J$是分式理想,按照条件得到$IJ=R$.反过来,如果$I$是可逆的分式理想,易证这个条件成立.
		
		最后说明$I=(a_1,a_2,\cdots,a_n)$,一方面$(a_1,a_2,\cdots,a_n)\subset I$,另一方面对于任意$b\in I$,有$b=\sum(bq_i)a_i$,而$bq_i\in q_iI\subset R$,于是$b\in(a_1,a_2,\cdots,a_n)$.
	\end{proof}
	\item 设$R$是整环,如果$(p_i)_{1\le i\le r}$和$(q_j)_{1\le j\le s}$是两组可逆素理想,满足$p_1p_2\cdots p_r=q_1q_2\cdots q_s$,那么$r=s$,并且重排素理想的顺序后得到$p_i=q_i$.换句话讲,如果一个可逆分式理想表示为有限个素理想的乘积,那么这种分解是唯一的.
	\begin{proof}
		
		假设$p_1$在$\{p_1,p_2,\cdots,p_r\}$中是极小的,那么从$q_1q_2\cdots q_s\subset p_1$,得到某个$q_t\subset p_1$.同理可取某个$p_u\subset q_t$,于是$p_u\subset q_t\subset p_1$,极小性说明$p_u=p_1$,于是$p_1=q_t$,按照可逆条件,在等式两边分别约去$p_1$和$q_t$,归纳操作下去即可.
	\end{proof}
\end{enumerate}

设$R$是整环,设$I$是$R$的分式理想,那么如下三个条件两两等价.
\begin{enumerate}
	\item $I$是可逆分式理想.
	\item $I$是投射$R$模.
	\item $I$是有限模,并且对每个极大理想(也可以改成对每个素理想)$m$,总有$R_m$的分式理想$IR_m$是主分式理想.
\end{enumerate}
\begin{proof}
	
	1推2,按照定义,存在$a_1,a_2,\cdots,a_n\in I$,以及$q_1,q_2,\cdots,q_n\in k$,使得$q_iI\subset R$,并且$1=\sum_{1\le i\le n}q_ia_i$.取$R$模同态$\varphi_i:I\to R$为$x\mapsto q_ix$.那么$\forall x\in I$都有$\sum_{1\le i\le n}\varphi_i(x)a_i=\sum_{1\le i\le n}q_iba_i=b$.于是$\{a_i,\varphi_i\}$是$I$的一组投射基,于是$I$是投射模.
	
	2推1,假设$I$是投射模,取投射基为$\{a_i\in I\}$和相同指标集的模同态集$\{\varphi_i:I\to R\}$,也即满足对任意$x\in I$有$\varphi_i(x)$只对有限个指标不取零,并且$x=\sum_i\varphi_i(x)a_i$.记$k=\mathrm{Frac}(R)$,对非零元$x\in I$,记$q_i=\varphi_i(x)/x\in k$.那么首先$q_i$不依赖于非零元$x$的选取:如果$y\in I$是另一个非零元,那么$y\varphi_i(x)=\varphi_i(xy)=x\varphi_i(y)$.于是有$q_iI\subset R$.于是按照投射基的定义,仅有有限个指标使得$q_i$不为零.我们把为零的指标划去.从投射基的定义得到$1=\sum_iq_ia_i$.于是$I$是可逆分式理想.
	
	1推3,设$I=(a_1,a_2,\cdots,a_n)$,有$q_i\in k$使得$q_i\in I^{-1}$,并且$1=\sum_ia_iq_i$.于是至少存在一个指标使得$a_tq_t\in R_p^*$,于是$IR_p=\sum_ia_ia_tq_tR_p=a_tR_p$.
	
	3推1,设$I=(a_1,a_2,\cdots,a_n)$,先证明对每个素理想$p$有$(IR_p)^{-1}=I^{-1}R_p$:一方面$I^{-1}R_p\subset(IR_p)^{-1}$,反过来任取$x\in(IR_p)^{-1}$,那么对任意$i$有$xa_i\in R_p$,于是存在$t\in R-p$使得$txa_i\in R$对任意$i$成立.于是$tx\in I^{-1}$,于是$x\in I^{-1}R_p$.完成断言的证明.接下来假设$I$不是可逆的,那么$II^{-1}$是$R$的真理想,于存在某个极大理想$m$包含了$II^{-1}$,于是$IR_m(IR_m)^{-1}=IR_mI^{-1}R_m\subset mR_m$,导致$IR_m$不是可逆的,矛盾.
\end{proof}

戴德金整环.一个整环称为戴德金整环,如果它的每个非零理想都是可逆的,或等价的讲,每个理想都是投射模.设$R$是整环,那么有如下等价描述:
\begin{enumerate}
	\item $R$是戴德金整环.
	\item $R$是域,或者是一维正规诺特整环.
	\item $R$的每个非零理想都可以表示为有限个素理想的乘积.这里我们约定空集的乘积是单位理想.特别的,这一条说明在戴德金整环中任取非零的$x\in R$,那么只存在有限个素理想包含了$x$.
\end{enumerate}
\begin{proof}
	
	1推2,可逆分式理想都是有限生成的,于是按照定义戴德金整环的理想都是有限生成的,于是$R$是诺特的.假设$R$不是域,任取$R$的非零素理想$p$,我们证明过诺特整环上的非零可逆素理想的高度都是1,并且$R_p$是DVR,于是每个$R_p$都是正规的,于是按照正规性是局部的,得到$R$是正规的.
	
	2推1,域的情况是平凡的,现在设$R$是一维正规诺特整环.取$R$的极大理想$m$,那么$R_m$是一维诺特正规局部整环,也即DVR.于是特别的,$R_m$的每个非零理想都是可逆的,于是$R$的每个非零理想在$R_m$中生成的理想都是主理想,我们证明过这导致$R$的每个非零理想都是可逆的,于是$R$是戴德金整环.
	
	1推3,设$I\subseteq R$是非零真理想,设极大理想$m_1$包含了$I$,那么有$I\subseteq Im_1^{-1}\subseteq R$,我们断言不会出现$I=Im_1^{-1}$的情况:等价于$Im_1=I$,按照NAK引理,存在$a\in R$满足$a\equiv1(\mod m_1)$使得$aI=0$,按照$I\not=0$以及$R$是整环,得到$a=0$,于是$1\in m$矛盾.于是有$I$真包含于$Im_1^{-1}$.如果$Im_1^{-1}=R$,得到$I=m_1$已经写成了素理想的积.如果$Im_1^{-1}$是真理想,可取极大理想$m_2$包含了它,于是可以继续考虑$Im_1^{-1}m_2^{-1}$,由此得到一个理想的升链$I\subseteq Im_1^{-1}\subseteq Im_1^{-1}m_2^{-1}\subset\cdots$.按照$R$是诺特的,这个升链会在有限步后终止,也即有极大理想$m_1,m_2,\cdots,m_r$使得$Im_1^{-1}m_2^{-1}\cdots m_r^{-1}=R$,也即$I=m_1m_2\cdots m_r$,完成证明.
	
	3推1,我们只需验证每个非零素理想是可逆的,这样按照每个非零真理想可以表示为有限个素理想的乘积,就得到每个非零真理想是可逆的.我们断言为证明这个命题,仅需验证$R$的每个可逆素理想都是极大的:一旦这成立,任取非零素理想$p$,取$0\not=a\in p$,记$aR=p_1p_2\cdots p_r$,其中每个$p_i$都是素理想.我们知道主分式理想都是可逆的,于是每个$p_i$都是可逆的素理想.于是得到$p_1p_2\cdots p_r\subseteq p$,于是至少存在某个$p_i\subseteq p$,按照$p_i$是极大理想,得到$p_i=p$,于是任意非零素理想都是可逆的.
	
	现在任取$R$的非零可逆素理想,设有理想$I$真包含了$p$,我们需要证明$I=R$.不妨设$I=p+aR$,其中$a\not\in p$,按照$p$是可逆的,为了证明$I=R$,仅需验证$pI=p$.记$I^2=p_1p_2\cdots p_r$和$p+a^2R=q_1q_2\cdots q_s$,其中$p_i$和$q_j$都是素理想.现在可逆分式理想$I^2R/p=a^2R/p$存在两种素理想的分解$\overline{p_1}\overline{p_2}\cdots\overline{p_r}=\overline{q_1}\overline{q_2}\cdots\overline{q_s}$,按照我们证明过的可逆素理想的乘积表示是唯一的,得到$r=s$,并且重拍素理想的顺序后会得到$p_i=q_i$.于是有$I^2=p+a^2R$,按照$I^2=p^2+ap+a^2R$,得到$p\subseteq p^2+ap+a^2R$,由于$a\not\in p$,得到$p\subseteq p^2+ap+a^2p$(任取$x\in p$,按照$x\in p^2+ap+a^2R$,得到$x=s_1s_2+as_3+a^2r$,其中$s_i\in p$,从$a\not\in p$得到$r\in p$,于是$x\in p^2+ap+a^2p$).于是$p\subset(p+aR)p$,于是$p\subseteq pI$,而$pI\subseteq p$,得到$pI=p$,完成证明.
\end{proof}
\begin{enumerate}
	\item 在代数数论里会定义戴德金整环是一维的,这实际上是排除了域的情况.另外对于戴德金整环,上面第三条中的分解是唯一的,因为我们证明过可逆理想写成有限个可逆素理想的乘积是唯一的.
	\item 戴德金整环必然是诺特环.戴德金整环的非零理想都是可逆的,我们证明过可逆分式理想都是有限生成的,于是戴德金整环是诺特环.
	\item 一个整环是戴德金整环当且仅当每个可除模都是内射模,即可除模和内射模等价.
	\begin{proof}
		
		假设每个可除$R$模是内射的,任取内射$R$模$E$,那么$E$可除.可除模的每个商都是可除的,于是$R$上每个内射模的商都是内射的,于是整环是戴德金整环.反过来,假设$R$是戴德金整环,而$E$是可除$R$模,按照Baer准则,只要证明存在$R\to E$提升了图表:
		$$\xymatrix{
			&E&\\
			0\ar[r]&I\ar[u]^f\ar[r]_i&R\ar[ul]
		}$$
		
		不妨设$I$非0,那么$I$可逆:存在$a_1,\cdots,a_n\in I$和$q_1,\cdots,q_n\in Q$,其中$Q$是$R$的商域,满足$q_iI\subset R$和$1=\sum_iq_ia_i$.按照$E$是可除模,存在$e_i\in E$使得$f(a_i)=a_ie_i$.那么对任意的$b\in I$,就得到$f(b)=b\sum q_ia_ie_i$,于是取$e=\sum q_ia_ie_i\in E$,就有$f(b)=be,\forall b\in I$.取$g:R\to E$为$r\mapsto er$,那么$g$延拓了$f$,于是$E$是内射模.
	\end{proof}
    \item 戴德金整环是UFD当且仅当是PID当且仅当理想类群是平凡群.事实上我们可以证明如果$R$是UFD,那么一个非0理想是投射的当且仅当是主理想.
    \begin{proof}
    	
    	先说明第二个命题可以证明第一个命题,给定UFD的戴德金整环$R$,那么每个非0理想都是投射的,于是都是主理想,于是是PID.
    	
    	对于一个整环$R$上的非0主理想,它是同构于$R$的,于是它自由,于是是投射的.反过来,如果给UFD上的一个投射的非0理想$I$,那么$I$是可逆理想,于是存在一族$a_1,\cdots,a_n\in I$和$q_1,\cdots,q_n\in Q$,满足$\sum q_ia_i=1$和$q_iI\subset R$.记$q_i=b_i/c_i$,可以约定$b_i,c_i\in R$是没有公共非单位因子,于是按照$q_ia_j\in R$得到$c_i\mid a_j,\forall i,j$.取$c$为$c_1,\cdots,c_n$的最小公倍数,那么$c=c\sum b_ia_i/c_i=\sum(b_ic/c_i)a_i\in I$.于是$(c)\subset I$.另一方面,按照$c_i\mid a_j$得到$c\mid a_j$,于是每个$a_j\in (c)$,于是$I\subset (c)$.
    \end{proof}
\end{enumerate}

戴德金整环上分式理想的赋值.
\begin{enumerate}
	\item 我们之前证明过一个诺特整环上的可逆素理想的高度总是1,并且在这样的素理想处的局部化是DVR.按照戴德金整环的每个素理想都是可逆的,说明戴德金整环在每个素理想处的局部化都是DVR.
	\item 戴德金整环$R$是诺特的,设商域$K$,于是它的每个分式理想等价于是$R$模$K$的有限生成子模.于是每个分式理想可以表示为$\frac{1}{a}I=I(a)^{-1}$的形式,其中$a\in A$和$I\subseteq A$是理想.于是戴德金整环上每个非零分式理想可以唯一的表示为有限个素理想的次幂的乘积,这里次幂可以取整数.对素理想$p$记$R_p$上的离散赋值为$v_p$,对戴德金整环$R$的非零分式理想$I$,容易验证有$I=\prod_pp^{v_p(I)}$,其中$p$跑遍全部非零素理想.
	\item 对戴德金整环$R$上分式理想$I$和$J$有:
	\begin{itemize}
		\item $v_p(IJ)=v_p(I)+v_p(J)$.
		\item $v_p((J:I))=v_p(JI^{-1})=v_p(J)-v_p(I)$.
		\item $v_p(I+J)=\inf\{v_p(I),v_p(J)\}$.
		\item $v_p(xR)=v_p(x)$.
	\end{itemize}
\end{enumerate}

逼近定理.
\begin{enumerate}
	\item 设$p_i,1\le i\le r$是戴德金整环$R$的不同素理想,设$x_i\in\mathrm{Frac}(R)=K,1\le i\le r$,设$n_i,1\le i\le r$是取定的整数.那么存在$x\in K$满足对任意$i$有$v_{p_i}(x-x_i)\ge n_i$,并且对任意$q\not=p_1,\cdots,p_r$有$v_q(x)\ge0$.
	\begin{proof}
		
		首先假设$x_i\in R$,不妨设$n_i\ge0$,否则适当变大$n_i$不影响条件,我们来找$R$中满足条件的元$x$.有$p_i^{n_i}+\prod_{j\not=i}p_j^{n_j}=R$,所以存在$t_i\in p_i^{n_i}$和$y_i\in\prod_{j\not=i}p_j^{n_j}$,使得$x_i=t_i+y_i$.我们断言$x=\sum_it_i$满足结论:$v_{p_i}(x-x_i)=v_{p_i}(\sum_{j\not=i}t_j-y_i)\ge n_i$和$v_q(x)=v_q(\sum_it_i)\ge0$因为$n_i\ge0$.
		
		对于一般情况,如果$x_i$未必在$A$中,记$x_i=a_i/s$,其中$a_i\in A$,$0\not=s\in A$,记$x=a/s$,那么$a$就要满足$v_{p_i}(a-a_i)\ge n_i+v_{p_i}(s)$和$v_q(a)\ge v_q(s),q\not=p_i$.这归结为上一种情况,从而得证.
	\end{proof}
    \item 如果戴德金整环$R$只有有限个素理想,那么它是主理想整环.
    \begin{proof}
    	
    	只需证明每个素理想都是主理想.任取非零素理想$p$,按照逼近定理可以选取$x\in R$满足$v_p(x)=1$和$v_q(x)=0,\forall q\not=p$,于是有$xA=p$.
    \end{proof}
\end{enumerate} 

有限扩张.设$A$是诺特整环,设$K$是它的商域,设$K\subseteq L$是有限扩张,设$B$是$A$在$L$中的整闭包.
\begin{enumerate}
	\item 如果额外的$A$还是整闭的,并且$K\subseteq L$是可分扩张,那么$B$是有限$A$模.特别的,这个证明说明了$B^*$(这是$B$关于迹二次型$\mathrm{T}$的对偶模)也是有限$A$模.
	\begin{proof}
		
		记$\mathrm{T}:L\to K$表示迹映射,按照$K\subseteq L$是有限可分扩张,说明迹二次型$\mathrm{T}(xy)$是非退化的.固定$K$的一个代数闭包,$x\in B$的共轭元仍然都在$A$上整,并且$x$的所有共轭元的和是$\mathrm{T}(x)$,但是$\mathrm{T}(x)\in K$,说明有$\mathrm{T}(x)\in A$.
		
		下面取$L$在$K$上的一组基为$\{e_i\}$,可以要求$e_i\in B$,否则统一乘以一个$B$中的非零元不影响它是一组$K$基.记$\{e_i\}$生成的自由$A$模为$V$.对$A$模$L$的每个子模$M$,用$M^*$表示所有$x\in L$,使得$\mathrm{T}(xM)\subseteq A$.那么$M^*$也是一个$A$模,并且有$V\subseteq B\subseteq B^*\subseteq V^*$.这里$\{e_i\}$关于非退化二次型$\mathrm{T}(xy)$的对偶基就是$V^*$的一组基.所以$V^*$是有限$A$模,但是$A$是诺特的,所以子模$B$也是有限模.
	\end{proof}
	\item 如果$A$是域上有限型代数或者完备离散赋值环,仍然有$B$是有限$A$模.【】
	\item 如果$A$是戴德金整环,那么$B$也是.
	\begin{proof}
		
		戴德金整环是整闭的,所以第一条说明$B$是有限$A$模,所以$B$也是诺特和整闭的.按照戴德金整环的等价描述,只需验证$B$的维数不超过1.假设可取$B$的素理想链$q_0\subseteq q_1\subseteq q_2$,那么得到$A$中的素理想链$q_0\cap A\subseteq q_1\cap A\subseteq q_2\cap A$,这个链的长度不能超过1,所以两个包含号至少有一个是等号.按照$B$在$A$上整,素理想链满足不可比条件,所以第一个链同样的两个包含号至少有一个是等号,这得到$B$的维数不超过1.
	\end{proof}
	\item 引理.设$A$是一维诺特整环,记商域是$K$,设$M$是无挠$A$模(这是指只要$0\not=m\in M$和$0\not=a\in A$,就有$am\not=0$),记$r=\dim_k(M\otimes_AK)<\infty$(这也就是$M$的在$A$上线性无关的极大组的元素个数,可以称为未必自由的模$M$的秩),那么对每个$0\not=a\in A$,有$l(M/aM)\le rl(A/aA)$.
	\begin{proof}
		
		先设$M$是有限$A$模,取$M$的在$A$上线性无关的极大组$\{\xi_1,\cdots,\xi_r\}$.记子模$E=\sum A\xi_i$.对每个$\eta\in M$,都可以找到一个非零的元$t\in A$,使得$t\eta\in E$,否则$\eta$也可以添加到$\{\xi_i\}$中线性无关,这会和极大性矛盾.考虑$A$模$C=M/E$,按照$M$是有限$A$模得到$C$也是有限$A$模,所以可以找到一个非零元$t\in A$使得$tC=0$.按照$C$是诺特环$A$上的有限模,可以找到链$C=C_0\supseteq C_1\supseteq\cdots\supseteq C_m=0$,使得每个$C_i/C_{i+1}\cong A/p_i$,其中$p_i\in\mathrm{Spec}A$.于是$t\in$每个$p_i$.按照$A$是一维的,这里$p_i$都是极大理想,所以$l(C)=m<\infty$.对$0\not=a\in A$,我们有正合列:
		$$\xymatrix{E/a^nE\ar[r]&M/a^nM\ar[r]&C/a^nC\ar[r]&0}$$
		
		于是有$l(M/a^nM)\le l(E/a^nE)+l(C),\forall n>0$.按照$M$和$E$都是无挠$A$模,得到$a^iM/a^{i+1}M\cong M/aM$和$a^iE/a^{i+1}E\cong E/aE$.所以有$nl(M/aM)\le nl(E/aE)+l(C),\forall n>0$,所以有$l(E/aE)\ge l(M/aM)$.再按照$E\cong A^r$,得到$l(E/aE)=rl(A/aA)$.至此我们完成了$M$是有限$A$模情况下引理的证明.
		
		\qquad
		
		下面设$M$未必是有限$A$模.取$\overline{M}=M/aM$的有限子模$\overline{N}=A\overline{\omega_1}+\cdots+A\overline{\omega_s}$.取$\overline{\omega_i}$的一个提升为$\omega_i$,记$M_1=\sum A\omega_i$.那么有$l(\sum A\overline{\omega_i})=l(M_1/M_1\cap aM)\le l(M_1/aM_1)\le rl(A/aA)$.这个不等式右侧不依赖$\overline{N}$,所以$\overline{M}$实际上是有限生成(长度的),并且有$l(\overline{M})\le rl(A/aA)$.完成证明.	
	\end{proof}
	\item Krull-Akizuki定理.设$A$是一维诺特整环,商域为$K$,设$K\subseteq L$是有限扩张,设$B$是一个环,满足$A\subseteq B\subseteq L$,那么$B$是维数不超过1的诺特环,并且如果$J$是$B$的非零理想,那么$B/J$是有限长度的$A$模.
	\begin{proof}
		
		不妨设$L=\mathrm{Frac}(B)$,否则要用一个更小的域替换$L$,这不改变$K\subseteq L$是有限扩张.记$[L:K]=r$.那么$B$是$A$的秩为$r$(此为在$A$上线性无关的极大组的元素个数)的无挠模,所以按照引理,对任意$0\not=a\in A$,都有$B/aB$在$A$上有限长度.如果$J\subseteq B$是非零理想,任取$0\not=b\in J$,按照$b$在$K$上代数,有$b$满足$A$系数多项式$a_mb^m+\cdots+a_1b+a_0=0,a_i\in A$.按照$B$是整环,说明$a_0\not=0$.所以$0\not=a_0\in J\cap A$,所以$l_A(B/J)\le l_A(B/a_0B)<\infty$.
		
		\qquad
		
		从$l_B(J/a_0B)\le l_A(J/a_0B)\le l_A(B/a_0B)<\infty$说明$J/a_0B$是有限$B$模.所以$J$必须是有限$B$模,这说明$B$是诺特环.如果$q\subseteq B$是非零素理想,那么$B/q$是阿廷整环,阿廷因为有限长度,但是有限整环是域,说明$q$是$B$的极大理想,于是$B$是一维的.
	\end{proof}
	\item 推论.设$A$是一维诺特整环,$k$是商域,设$k\subseteq L$是有限扩张,记$A$在$L$中的正规化是$B$,那么$B$是一维的戴德金整环.并且对每个素理想$p\subseteq A$,只存在有限个$B$中的素理想提升了$p$.
	\begin{proof}
		
		上一条已经证明了$B$是维数不超过1的诺特环,而正规化必然是整闭的,于是$B$是维数不超过1的整闭诺特整环,即戴德金整环.最后我们来证明$p\in\mathrm{Spec}A$的提升素理想恰好就是$pB$作为戴德金整环的非零理想做素理想唯一分解$pB=q_1^{a_1}\cdots q_r^{a_r},a_i>0$中的$q_i$:一方面如果$q\in\mathrm{Spec}B$满足$q\cap A=p$,那么$pB\subseteq q$,于是某个$q_i\subseteq q$,但是它们都是极大理想,所以相同.另一方面对每个$p_i$都有$p\subseteq pB\cap A\subseteq q_i\cap A$,这里$q_i\cap A$是$A$的素理想,按照$p$是$A$的极大理想,得到这个包含链实际上是等式,所以$p=pB\cap A=q_i\cap A$.
	\end{proof}
\end{enumerate}
\newpage
\subsection{Krull环}

Krull环定义.设$A$是整环,商域记作$K$,称$A$为Krull环,如果存在$K$的一族DVR记作$\{R_{\lambda}\}_{\lambda\in\wedge}$,满足如下两个条件:
\begin{itemize}
	\item $A=\cap_{\lambda}R_{\lambda}$.
	\item 对每个$0\not=a\in A$,对除了有限个$\lambda\in\wedge$以外都有$aR_{\lambda}=R_{\lambda}$.
\end{itemize}
\begin{enumerate}
	\item 条件$aR_{\lambda}=R_{\lambda}$等价于讲$a$是$R_{\lambda}$的单位,也即$R_{\lambda}$上的规范离散赋值$v_{\lambda}(a)=0$.于是Krull环定义中的第二条等价于讲:对任意$x\in K^*$,只存在有限个$\lambda\in\wedge$使得$v_{\lambda}(x)\not=0$.
	\item 如果$\wedge=\emptyset$,那么$\cap_{\lambda\in\wedge}R_{\lambda}=K$,于是$K$本身也是一个Krull环.
	\item 完全整闭环(completely integrally closed ring).
	\begin{enumerate}
		\item 设$A$是整环,$K$是商域,一个元$x\in K$称在$A$上几乎整,如果存在$0\not=a\in A$,满足对任意$n\ge0$有$ax^n\in A$.我们断言如果$x\in K$在$A$上整那么它在$A$上几乎整,如果$A$是诺特环那么逆命题成立.
		\begin{proof}
			
			先设$x\in K$在$A$上整,那么存在$a_1,\cdots,a_n\in A$使得$x^n+a_1x^{n-1}+\cdots+a_n=0$.对次数做归纳知$x^t,t\ge n$总可以表示为$1,x,\cdots,x^{n-1}$的$A$线性组合.按照$x\in K=\mathrm{Frac}(A)$,有$x=c/d$,其中$c,d\in A$.取$a=d^{n-1}$,得到每个$ax^i,0\le i\le n-1$都在$A$中,这说明每个$ax^i,i\ge0$都在$A$中,于是$x$在$A$上几乎整.
			
			\qquad
			
			反过来假设$A$是诺特环,并且存在$0\not=a\in A$使得$ax^i,i\ge0$都在$A$中.那么有理想升链$(a)\subseteq(a,ax)\subseteq(a,ax,ax^2)\subseteq\cdots$.按照诺特条件这个升链会终止,于是有某个正整数$n$使得$ax^n\in(a,ax,\cdots,ax^{n-1})$.于是存在$a_0,a_1,\cdots,a_{n-1}\in A$使得$aa_0+aa_1x+\cdots+aa_{n-1}x^{n-1}=ax^n$.于是在$K$中有$x^n=a_0+a_1x+\cdots+a_{n-1}x^{n-1}$,于是$x$在$A$上整.
		\end{proof}
	    \item 设$A$是整环,它称为完全整闭环,如果每个几乎整元$x\in\mathrm{Frac}(A)$都在$A$中.于是上一条说明完全整闭环都是整闭环,并且对于诺特环这两个概念一致.于是DVR总是完全整闭环,完全整闭环的交仍然是完全整闭的,于是Krull环总是完全整闭环.
	\end{enumerate}
    \item 如果$A$是Krull环,商域记作$K$,任取$K$的子域$K'$,那么$A\cap K'$仍是Krull环.
    \begin{proof}
    	
    	不妨设$K'$本身是$A\cap K'$的商域,否则可以用$\mathrm{Frac}(A\cap K')$替换$K'$,这不改变$A\cap K'$.设用来定义$A$的DVR族为$\{R_{\lambda},\lambda\in\Lambda\}$.那么有$A\cap K'\subseteq R_{\lambda}\cap K'\subseteq K'$.一般的,如果$A$是商域为$K$的DVR,如果环$R$满足$A\subseteq R\subseteq K$,那么$R$要么是DVR,要么有$R=K$(比方说,设$\pi$是素元,设使得$\varepsilon/\pi^r\in R$的最大的自然数为$r$,其中$\varepsilon\in A$是单位,如果$r=0$则$R=K$,如果$r>0$那么$1/\pi^r\in R$是$R$的素元,使得$R$是DVR).所以如果记$\Lambda'\subseteq\Lambda$是使得$R_{\lambda}\cap K'\not=K'$的指标$\lambda$,则有$\cap_{\lambda\in\Lambda'}R_{\lambda}=A$,并且有$R_{\lambda}\cap K'$是DVR,并且有$\cap_{\lambda\in\Lambda'}(R_{\lambda}\cap K')=A\cap K'$.于是$A\cap K'$是Krull环.
    \end{proof}
    \item 一个诺特整环是Krull环当且仅当整闭.于是特别的,戴德金整环总是Krull环,我们会在后文证明维数$\le1$的Krull环自动是诺特的,于是维数$\le1$的Krull环等价于戴德金整环.
    \begin{proof}
    	
    	我们已经证明了诺特整环如果是Krull环,那么它完全整闭也是整闭的.反过来如果$A$是诺特正规整环,我们解释过当$p$取遍高度1素理想时有$A_p$是DVR,并且$\cap A_p=A$.下面任取$0\not=a\in A$,不妨设它不是单位,否则自然有$aA_p=A_p$.设$p$是高度1素理想,那么$a\in pA_p$等价于$a\in p$,按照$p$高度1,说明$a\in pA_p$等价于$p$是包含$a$的极小素理想.但是按照诺特条件,这样的极小素理想个数是有限的,这得到Krull环定义中的第二条,综上我们证明了诺特正规整环是Krull环.
    \end{proof}
\end{enumerate}

有限个DVR定义的Krull环.
\begin{enumerate}
	\item 引理.设$A$是Krull环,设用来定义它的一族DVR为$\{R_{\lambda}\}_{\lambda\in\wedge}$.设$S\subseteq A$是乘性闭子集,那么$S^{-1}A$仍是Krull环,用来定义它的一族DVR可取$\{R_{\lambda}\}_{\lambda\in\Gamma}$,其中$\Gamma=\{\lambda\in\wedge\mid S^{-1}A\subseteq R_{\lambda}\}$.
	\begin{proof}
		
		对每个$\lambda\in\wedge$,记$R_{\lambda}$的极大理想为$m_{\lambda}$,按照$S\subseteq A\subseteq R_{\lambda}$.按照$\Gamma$的定义,有$\lambda\in\Gamma$当且仅当$S\subseteq R_{\lambda}-m_{\lambda}$,也即$S\cap m_{\lambda}=\emptyset$.于是恰好有$\cap_{\lambda\in\Gamma}R_{\lambda}=S^{-1}A$.
	\end{proof}
    \item 引理(Nagata).设$K$是域,设$R_1,\cdots,R_n$是$K$的有限个赋值环,设$A=\cap R_i$.对$a\in K$,断言存在整数$s\ge2$使得$(1+a+\cdots+a^{s-1})^{-1}$和$a(1+a+\cdots+a^{s-1})^{-1}\in A$.
    \begin{proof}
    	
    	有$(1-a)(1+a+\cdots+a^{s-1})=1-a^s$.记$R$是某个$R_i$,我们断言存在整数$d\ge2$,使得只要$s\ge2$和$d\not\mid s$,就有$(1+a+\cdots+a^{s-1})^{-1}$和$a(1+a+\cdots+a^{s-1})^{-1}\in R$.一旦这件事成立,对$a\in K$就可以取一族正整数$d_i\ge2$,但是这只有有限个整数,所以一定可取$s\ge2$使得$d_i\not\mid s$,这就得到结论成立.最后为了证明我们的断言,分如下四种情况考虑.记$m$为$R$的极大理想,记剩余类域$k=R/m$.
    	\begin{enumerate}
    		\item 如果$a\not\in R$且$a^{-1}\in m$.有$1-a^{-s}$是$R$的单位.于是对$s\ge2$有$(1+\cdots+a^{s-1})^{-1}=\frac{1-a}{1-a^s}=\frac{a^{-s}-a^{1-s}}{a^{-s}-1}$是$R$中的元.类似有$a(1+a+\cdots+a^{s-1})^{-1}=\frac{a^{1-s}-a^{2-s}}{a^{-s}-1}$同样在$s\ge2$时是$R$中的元.此时$d$可取任意$\ge2$的整数.
    		\item 如果$a\in R$且$a$在$k=R/m$中的像$\overline{a}$不是单位根.那么$1-a^s\in R-m=R^*$,于是这两个元仍然对任意$s$都在$R$中.此时$d$依旧可取任意$\ge2$的整数.
    		\item 如果$a\in R$且$a$在$k=R/m$中的像$\overline{a}$是阶数$d\ge2$的单位根.那么当$d\not\mid s$有$1-a^s\in R-m=R^*$.于是此时这两个元都在$R$中.
    		\item 如果$a\in R$且$\overline{a}$是阶数1的单位根,也即$a\in 1+m$.设$s\ge2$是在$k$中可逆的整数,那么$1+a+\cdots+a^{s-1}\equiv s(\mathrm{mod}m)$在$A$中可逆.于是取$d=\mathrm{char}k\ge2$满足条件.
    	\end{enumerate}
    \end{proof}
    \item 引理.设$A$是环,设$I,J$是两个理想,对任意极大理想$m\subseteq A$有$IA_m=JA_m$,那么有$I=J$.
    \begin{proof}
    	
    	只需验证$I\subseteq J$,对称性得到$I=J$.任取$x\in I$,任取极大理想$m$,从$x\in IA_m=JA_m$得到存在$y_m\in J$和$s_m\in A-m$使得在$A_m$中有$x/1=y_m/s_m$.于是存在$t_m\in A-m$满足$t_ms_mx=y_mt_m\in A$.我们断言当$m$取遍$A$的极大理想时$t_ms_m$生成了单位理想.否则它生成的真理想要包含在某个极大理想$m_0$中,但是这导致$t_{m_0}s_{m_0}\in m_0$矛盾.于是存在$A$的有限个极大理想$\{m_1,\cdots,m_n\}$和$c_i\in A$使得$1=\sum_{i=1}^nc_it_{m_i}s_{m_i}$.于是有$x=\sum_{i=1}^nc_it_{m_i}s_{m_i}x=\sum_{i=1}^nc_iy_{m_i}t_{m_i}\in J$.这得到$x\in J$,于是有$I\subseteq J$.
    \end{proof}
    \item 设$K$是域,设$R_1,\cdots,R_n$是$K$的赋值环,满足$i\not=j$时$R_i,R_j$总没有包含关系.记$R_i$的极大理想是$m_i$,记$A=\cap_iR_i$.断言$A$是全部极大理想为$p_i=m_i\cap A$的半局部环,并且$A_{p_i}=R_i$.特别的,如果$R_i$都是DVR,那么$A$是PID.
    \begin{proof}
    	
    	如果$x\in A-p_i$,从$x\not\in m_i\cap A$得到$x\not\in m_i$,所以有$A-p_i\subseteq R_i-m_i$,于是有$A_{p_i}\subseteq R_i$.反过来任取$a\in R_i$,按照引理可取$s\ge2$,使得$u=(1+a+\cdots+a^{s-1})^{-1}$满足$u,au\in A$.按照$a\in R_i$,得到$1+\cdots+a^{s-1}\in R_i$,于是$u\in R_i$是单位.于是$u\in A-p_i$,于是$a=ua/u\in A_{p_i}$.综上得到$A_{p_i}=R_i$.
    	
    	\qquad
    	
    	下面证明$A$的全部极大理想即全体$\{p_i\}$.首先上一段得到的结论也说明当$i\not=j$时有$p_i$和$p_j$没有包含关系,另外由于赋值环都是局部环,说明每个$p_i$都是$A$的极大理想.问题归结为证明每个真理想$I\subsetneqq A$都被某个$p_i$包含.假设这不成立,选取这样的真理想$I$,那么有$I\not\subseteq\cup_{i=1}^np_i$.于是可取$x\in I$使得$x\not\in$每个$p_i$.于是$x$在每个$R_i$中是可逆的,于是$x^{-1}\in\cap R_i=A$,导致$x$是$A$的单位,这和$I$是真理想矛盾.
    	
    	\qquad
    	
    	最后假设每个$R_i$都是DVR,记uniformizer为$\pi_i\in R_i$.任取非零理想$I\subseteq A$,存在正整数$v_i$使得$IA_{p_i}=m_i^{v_i}$.由于这里$p_i$是两两不同的$A$的极大理想,按照中国剩余定理存在$x\in A$使得$x\equiv\pi_i^{v_i}(\mathrm{mod}m_i^{v_i+1}),\forall i$.于是有$JA_{p_i}=xA_{p_i},\forall i$.再按照之前的引理得到$J=xA$,于是$A$是PID.
    \end{proof}
\end{enumerate}

Krull环的刻画.
\begin{enumerate}
	\item 设$A$是Krull环,设商域为$K$,那么对$A$的每个高度1素理想$p$,都有$A_p$是DVR,并且$A_p$总出现在定义Krull环$A$的DVR族中.另外当$p$跑遍高度1素理想时总有$A=\cap A_p$,于是Krull环的全体高度1素理想处的局部化是定义Krull环的一族DVR,并且任意定义了$A$的Krull环都包含这族DVR.换句话讲尽管定义一个Krull环的DVR族未必是唯一的,但是在极小意义下是唯一的.
	\begin{proof}
		
		先证每个$A_p$都是DVR,并且$A_p$出现在任意定义Krull环$A$的DVR族中.任取$\{R_{\lambda}\}_{\lambda\in\wedge}$是定义Krull环$A$的一族DVR.对素理想$p\subseteq A$,我们解释过$\{R_{\lambda}\}_{\lambda\in\wedge'}$是定义Krull环$A_p$的一族DVR,其中$\wedge'=\{\lambda\in\wedge\mid A_p\subseteq R_{\lambda}\}$.特别的,对$\lambda\in\wedge'$,就有$A-p\subseteq R_{\lambda}^{\times}$.所以$(A-p)\cap m_{\lambda}=\emptyset$.于是$A\cap m_{\lambda}\subseteq p$.如果$A\cap m_{\lambda}=\{0\}$,导致$K\subseteq R_{\lambda}$矛盾.于是$A\cap m_{\lambda}$不是零,按照$p$的高度1只能有$A\cap m_{\lambda}=p$.于是$p\subseteq m_{\lambda}$对任意$\lambda\in\wedge'$成立.于是如果固定$0\not=x\in p$,那么对每个$\lambda\in\wedge'$都有$v_{\lambda}(x)>0$,但是按照Krull环定义中第二条,应该只有有限个指标使得$v_{\lambda}(x)\not=0$,于是这里$\wedge'$是有限集.于是$A_p$是有限个DVR的交.但是$A_p$是局部环,只有一个极大理想,按照有限DVR交的Krull环的结构,这里$\wedge'$如果元素多于一个,这里的DVR之间就有包含关系.但是1维赋值环是它商域的真子环中的极大元,这就导致$\wedge'$是单元集合,于是$A_p$本身是DVR.
		
		\qquad
		
		下面证明命题中的等式.只需证明$p$跑遍高度1素理想时有$\cap A_p\subseteq A$.任取$x=a'/a\in\cap A_p$,其中$a',a\in A$.不妨设$a$非零非单位.特别的,对每个高度1素理想$p$都有$a'\in aA_p$.按照$A$是Krull环,只存在有限个$R_{\lambda}$使得$aR_{\lambda}\not=R_{\lambda}$.设这有限个DVR为$R_1,\cdots,R_n$(存在是因为如果都有$aR_{\lambda}=R_{\lambda}$会导致$a\in A$是单位).记$q_i=A\cap aR_i$和$p_i=A\cap m_i$.那么有$aA=\cap_{\lambda\in\wedge}(aR_{\lambda}\cap A)=q_1\cap\cdots\cap q_n$.不妨设这个交是不可缩短的,换句话讲$q_i$之间没有互相包含关系.我们断言这里每个$p_i$的高度都是1,否则的话不妨设$p_1$的高度$>1$.那么$A_{p_1}$是维数为$\mathrm{ht}(p_1)>1$的被$\{R_{\lambda}\}_{\lambda\in\wedge_1}$定义的Krull环,其中$\wedge_1=\{\lambda\in\wedge\mid A_{p_1}\subseteq R_{\lambda}\}$.按照$\dim A_{p_1}>1$,结合有限DVR交的Krull环的结构知,这里$\wedge_1$必然是无限集.于是特别的可取$R'=$某个$R_{\lambda_1}$,其中$\lambda_1\in\wedge_1$,使得$aR'=R'$.记$p'=A\cap m_{R'}$,我们断言$p'\not=\{0\}$,否则的话$K\subseteq R'$矛盾.从$A_{p_1}\subseteq R'$和$(A-p_1)\cap m_{R'}=\emptyset$得到$p'\subseteq p_1$.按照$R_1$是DVR,有足够大的正整数$v$使得$m_{R_1}^v\subseteq aR_1$,于是$p_1^v\subseteq q_1$.从$aA=q_1\cap\cdots\cap q_n$是不可缩短的,得到$q_2\cap\cdots\cap q_n\not\subseteq aA$,于是存在整数$i\ge0$使得$p_1^i\cap q_2\cap\cdots\cap q_n\not\subseteq aA$和$p_1^{i+1}\cap q_2\cap\cdots\cap q_n\subseteq aA$.取$b\in(p_1^i\cap q_2\cap\cdots\cap q_n)-aA$,得到$b\not\in aA$且$bp_1\subseteq aA$.于是$(b/a)p'\subseteq(b/a)p_1\subseteq A\subseteq R'$.按照$a\in R'$可逆.就有$(b/a)p'\subseteq A\cap m_{R'}=p'\subseteq A$.于是对任意$0\not=c\in p'$和任意$n\ge0$,都有$(b/a)^nc\in p'\subseteq A$.按照$A$是完全整闭环,得到$b/a\in A$,导致$b\in aA$和$b$的选取矛盾,这个矛盾说明我们一开始设的$p_i$高度不全为1矛盾.于是这里$R_i=A_{p_i}$,导致$a'\in A\cap aR_i=q_i$,于是$a'\in aA=q_1\cap\cdots\cap q_n$,也即$x=a'/a\in A$.		
	\end{proof}
    \item 设$A$是整环,商域记作$K$,那么$A$是Krull环当且仅当如下两个条件同时成立.
    \begin{enumerate}
    	\item 对$A$的每个高度1素理想$p$,有$A_p$是DVR.
    	\item 对每个$0\not=a\in A$,主理想$aA$可以表示为有限个高度1的准素理想的交.这里准素理想的高度定义为它根理想(是素理想)的高度.
    \end{enumerate}
    \begin{proof}
    	
    	先设$A$是Krull环,我们已经证明了对每个高度1素理想$p$有$A_p$是DVR,甚至得到了$\{A_p\}$其中$p$取遍$A$的高度1素理想是定义$A$的一族DVR.对每个$0\not=a\in A$只包含在有限个高度1素理想中,设为$p_1,\cdots,p_n$.记$q_i=A\cap aA_{p_i}$.按照准素理想的回拉仍然是准素的,说明每个$q_i$都是$A$的准素理想.必要性最后只需验证$aA=q_1\cap\cdots\cap q_n$.一方面明显有$aA\subseteq q_1\cap\cdots\cap q_n$.另一方面任取$b\in q_1\cap\cdots\cap q_n$,那么$b\in aA_{p_i},\forall i$.如果$p$是不为任一$p_i$的高度1的素理想,就有$a\in A_p^{\times}$,于是$b\in aA_p$.于是$b/a\in A_p$对任意高度1素理想$p$成立,于是$b/a\in\cap A_p=A$.于是$b\in aA$.
    	
    	\qquad
    	
    	下面证明充分性.假设整环$A$同时满足这两个条件.先证明有$A=\cap A_p$,其中$p$取遍高度1素理想.为此任取$x=b/a\in\cap A_p$,其中$a,b\in A$,$a\not=0$.记$aA=q_1\cap\cdots\cap q_n$是第二条中的准素分解,于是每个$p_i=\sqrt{q_i}$的高度1.不妨设这里$i\not=j$时$p_i\not=p_j$,这可实现是因为同一个素理想的准素理想的有限交仍然是该素理想的准素理想.再不妨设$i\not=j$时$p_i$与$p_j$没有包含关系,那么导致$i\not=j$时$q_i$与$q_j$也没有包含关系.我们断言有$q_i=A\cap aA_{p_i}$,这是因为$aA_{p_i}=\left(\cap_{j=1}^nq_j\right)A_{p_i}=\cap_{j=1}^nq_jA_{p_i}=q_iA_{p_i}$,其中第二个等式因为局部化是平坦的,第三个等式因为$q_j\not\subseteq p_i,i\not=j$.按照$q_i$是准素理想,得到$q_i=A\cap q_iA_{p_i}=A\cap aA_{p_i}$于是断言成立.回到证明,有$b\in A\cap aA_{p_i}=q_i$,得到$b\in q_1\cap\cdots\cap q_n=aA$,于是有$x=b/a\in A$.至此得到$A=\cap A_p$,其中$p$跑遍高度1素理想.下面断言$p_i,1\le i\le n$是仅有的包含$a$的高度1素理想:如果$p$是包含$a$的高度1素理想,那么$q_1\cap\cdots\cap q_n\subseteq q$,导致某个$\sqrt{q_i}=p_i\subseteq p$,考虑高度得到$p_i=p$.这得到Krull定义中的第二条,于是$A$是Krull环.
    \end{proof}
\end{enumerate}

Krull环的更多例子和性质.
\begin{enumerate}
	\item 引理.设$A$是完全整闭整环,那么$A[X]$和$A[[X]]$也是完全正规整环.我们解释过诺特条件下完全整闭和整闭是一致的,于是如果$A$是诺特完全整闭整环,那么$A[X]$和$A[[X]]$是诺特整闭整环.
	\begin{proof}
		
		设$f\in K(X)$和$0\not=g\in A[X]$,使得$f^rg\in A[X],\forall r\ge0$成立.我们要证明$f\in A[X]$.但是按照$K[X]$是PID,它是完全整闭环,所以已经有$f\in K[X]$.记$f(X)=a_0X^n+\cdots+a_{n-1}X+a_n\in K[X]$,设$a_0\not=0$.再记$g(X)=b_0X^m+\cdots+b_{m-1}X+b_m\in A[X]$,设$b_0\not=0$.那么从$f^rg\in A[X]$得到最高次项系数有$a_0^rb_0\in A,\forall r\ge0$.于是按照$A$是完全整闭环,得到$a_0\in A$.接下来有$f-a_0X^n$仍然是$A[X]$上的完全整元,因为对任意$r\ge0$有$(f-a_0X^n)^rg\in A[X]$.于是对$f$的阶数做归纳,得到$f\in A[X]$,于是$A[X]$是完全整闭的.
		
		\qquad
		
		对于$A[[X]]$的情况.首先$K[[X]]$是一个DVR,于是它也是完全整闭的.于是对任意$\alpha\in\mathrm{Frac}(A[[X]])\subseteq K[[X]][X^{-1}]=\mathrm{Frac}(K[[X]])$,如果有$\beta\in A[[X]]\subseteq K[[X]]$使得$\alpha^r\beta\in A[[X]]$.那么已经有$\alpha\in K[[X]]$.设$\alpha=a_sX^s+a_{s+1}X^{s+1}+\cdots\in K[[X]]\cap\mathrm{Frac}(A[[X]])$,设$a_s\not=0$.再设$\beta(X)=b_tX^t+b_{t+1}X^{t+1}+\cdots\in A[[X]]$,设$b_t\not=0$.那么从$\alpha^r\beta\in A[[X]]$得到$a_s^rb_t\in A,\forall r\ge0$.于是按照$A$是完全整闭的,得到$a_s\in A$,接下来$\alpha-a_sX^s$同样是$A[[X]]$上几乎整元因为$(\alpha-a_sX^s)^r\beta\in A[[X]],\forall r\ge0$.于是归纳得到$\alpha$的每个系数都在$A$中,于是有$\alpha\in A[[X]]$,于是$A[[X]]$是完全整闭环.
	\end{proof}
    \item 设$A$是整环,商域记作$K$.
    \begin{enumerate}
    	\item 设$L/K$是域扩张,如果$\{A_i\}_{i\in I}$是一族包含在$L$中的Krull环,满足$A=\cap_{i\in I}A_i$,并且对任意$0\not=a\in A$有$aA_i=A_i$只对有限个指标不成立,那么$A$是Krull环.
    	\item 如果$A$是Krull环,那么$A[X]$和$A[[X]]$都是Krull环.
    \end{enumerate}
    \begin{proof}
    	
    	第一件事是容易的,因为$K\subseteq\mathrm{Frac}(A_i)$,有$K\cap A_i$仍然是Krull环,所以把$A_i$替换为$K\cap A_i$不改变两个条件.所以不妨设每个$A_i$的商域都是$K$,于是同一个域的Krull环的交自然还是Krull环.
    	
    	\qquad
    	
    	下面证明第二条.首先按照$K[X]$是PID,它也是Krull环.任取$A$的高度1素理想$p$,那么有$A_p$是DVR,于是按照引理有$A_p[X]$是诺特整闭整环.但是我们解释过诺特整环上Krull环等价于整闭,所以$A_p[X]$是Krull环.下面当$p$跑遍高度1素理想时有$A[X]=\cap A_p[X]\subseteq K[X]$.我们断言有$A_p[X]=A_p[X]_{pA_p[X]}\cap K[X]$:一般的,设$R$是DVR,极大理想为$m$,商域为$K$,那么有$R[X]\subseteq R[X]_{m[X]}\cap K[X]$.反过来如果$\alpha=f/g\in R[X]_{m[x]}$,其中$f\in R[X]$,$g\in R[X]-m[X]$,并且设$\alpha\in K[X]$,那么可记$\alpha=h/c$,其中$h\in R[X]-m[X]$,$c\in R-\{0\}$.那么有$gh=fc\in R[X]$.但是$gh\in A[X]-m[X]$,导致$c$必须是$R$中可逆元,于是$\alpha=hc^{-1}\in A[X]$.于是$R[X]=R[X]_{m[X]}\cap K[X]$.
    	
    	\qquad
    	
    	回到证明.按照断言,上述等式可以写为$A[X]=\left(\cap A_p[X]_{pA_p[X]}\right)\cap K[X]$,其中$p$跑遍高度1素理想.再取$0\not=f\in A[X]$,因为$f$只有有限个非零系数,所以除了有限个高度1素理想以外,对其余高度1素理想$p$都有$f$的非零系数在$A_p$中可逆.于是$f\in A_p[X]_{pA_p[X]}$可逆,从第一条($K[X]$也是Krull环)就得到$A[X]$是Krull环.
    	
    	\qquad
    	
    	再考虑$A[[X]]$.任取$A$的高度1素理想$p$,按照引理$A_p[[X]]$仍然是诺特整闭整环,于是它是Krull环.同样有$A[[X]]=\cap A_p[[X]]$,其中$p$跑遍高度1素理想.按照$A_p[[X]][X^{-1}]\cap K[[X]]=A_p[[X]]$,于是上述等式可以写为$A[[X]]=\left(\cap A_p[[X]][X^{-1}]\right)\cap K[[X]]$.任取$0\not=f\in A[[X]]$,记$f=a_sX^s+a_{s+1}X^{s+1}+\cdots$,设$0\not=a_s\in A$,在$A_p[[X]][X^{-1}]$中单位恰好是$a_s$不在$p$中的$f$.但是只有有限个高度1素理想包含了非零元$a_s$,于是只存在有限个高度1素理想$p$使得$f$不在$A_p[[X]][X^{-1}]$中可逆.于是按照第一条得到$A[[X]]$是Krull环.
    \end{proof}
    \item 设$A$是诺特整环,商域记作$K$,设$A$在$K$中的整闭包是$\overline{A}$,那么$\overline{A}$是Krull整环.更一般的如果$L/K$是有限扩张,那么$A$在$L$中的整闭包$B$也是Krull整环.另外注意一个诺特整环在它商域的整闭包未必还是诺特的.
    \begin{proof}
    	
    	见Nagata的<local rings>的33.10.
    \end{proof}
    \item 设$A$是整环,那么如下两个命题互相等价.
    \begin{enumerate}
    	\item $A$是维数$\le1$的Krull环.
    	\item $A$是戴德金整环.
    \end{enumerate}
    \begin{proof}
    	
    	这个等价性我们已经证明了大部分内容.唯独需要证明的是如果$A$是1维Krull整环,那么它自动是诺特的.为此我们证明对任意$0\not=a\in A$非单位,有$A/aA$是诺特环.记$aA=q_1\cap\cdots\cap q_r$是不可缩短的准素分解,记$p_i=\sqrt{q_i}$,那么$i\not=j$时有$p_i\not=p_j$.按照$A$是1维整环有每个$p_i$是极大理想.于是对$i\not=j$有$p_i+p_j=R$.于是按照中国剩余定理,有$A/aA\cong\oplus_{i=1}^rA/q_i$.这里$A/q_i$是极大理想为$p_i/q_i$的局部环,于是$A/q_i\cong A_{p_i}/q_iA_{p_i}$是DVR的商环,于是诺特,于是$A/aA\cong\oplus_{i=1}^rA_{p_i}/q_iA_{p_i}$是诺特的.
    \end{proof}
    \item Krull环上的中国剩余定理.设$A$是Krull环,商域记作$K$,取$A$的$r$个不同的高度1素理想$p_1,\cdots,p_r$,对每个$1\le i\le r$,记$v_i:K^*\to\mathbb{Z}$为$A_{p_i}$上的离散赋值.任取整数$e_1,\cdots,e_r\in\mathbb{Z}$,那么存在$x\in K$满足$v_i(x)=e_i,\forall 1\le i\le r$,并且对不为$p_i$的高度1素理想$p$有$x\in A_p$.
    \begin{proof}
    	
    	先断言有$y_1\in A$满足$v_1(y_1)=1$和$v_i(y_1)=0,\forall 2\le i\le r$.为此记$S=A-\cup_ip_i$,这是$A$的一个乘性闭子集,考虑$S^{-1}A$,它是全部极大理想为$S^{-1}p_i,1\le i\le r$的半局部环.按照中国剩余定理,可取$\alpha=a/s\in S^{-1}A$,其中$a,s\in A$,$s\not\in\cup_{i=1}^rp_i$,使得$v_1(\alpha)=1$且$v_i(\alpha)=0,\forall 2\le i\le r$.那么$y_1=a$就满足要求.类似的对每个$1\le j\le r$选取$y_j\in A$使得$v_j(y_j)=1$和$v_i(y_j)=0,\forall i\not=j$.取$y=y_1^{e_1}\cdots y_r^{e_r}\in K^*$.按照Krull环的定义使得$y\not\in A_p$的高度1素理想$p$只能有有限个,设为$q_1,\cdots,q_s$.重复上述操作,选取$t_j\in A$使得$t_j\in q_j$且$v_i(t_j)=0,\forall 1\le i\le r$.再取$x=y(t_1\cdots t_s)^n$,那么有$v_i(x)=e_i,\forall 1\le i\le r$.选取足够大的$n$就导致对$q_i$上的赋值$w$有$w(x)\ge0$,也即$x\in A_{q_i}$,于是$x$满足要求.
    \end{proof}
    \item 尽管维数$\le1$时Krull环自然是诺特的,但是高维情况却未必如此.对此我们有如下准则:设$A$是Krull环,那么$A$是诺特环当且仅当对每个高度1素理想$p$有$A/p$是诺特环.
    \begin{proof}
    	
    	只需验证充分性.设Krull环$A$满足对任意高度1素理想$p$有$A/p$是诺特环.对每个非负整数$n$记$p^{(n)}=A\cap p^nA_p$.我们来证明对每个非单位元$0\not=a\in A$有$A/aA$是诺特的.按照$A$是Krull环,我们解释过有准素分解$aA=p_1^{(n_1)}\cap\cdots\cap p_r^{(n_r)}$.于是有单射$A/aA\to\oplus_{i=1}^rA/p_i^{(n_i)}$.于是环$\oplus_{i=1}^rA/p_i^{(n_i)}$是环$A/aA$的扩张,并且作为$A/aA$模是有限模.按照Eakin-Nagata定理:如果$B$是诺特环,有子环$A\subseteq B$使得$B$是有限$A$模,那么$A$也是诺特环.为证明$A/aA$是诺特环,只需证明$A/p_i^{(n_i)}$是诺特环.我们简记$p_i$为$p$,$n_i$为$n$.按照Krull环上的中国剩余定理,可取$x\in K$使得$v_p(x)=1$且$v_q(x)\le0$对任意不为$p$的高度1素理想$q$.记$B=A[x]\subseteq K$,我们断言有$p=xB\cap A$:一方面有$xB\subseteq pA_p$,于是$p=pA_p\cap A\supset xB\cap A$;反过来对$y\in p$,有$y/x\in\cap A_q=A$,其中$q$取遍高度1素理想,所以$y\in xA\subseteq xB\cap A$,于是断言成立.
    	
    	\qquad
    	
    	于是包含映射$A\subseteq B=A[x]$诱导了同构$A/p\cong B/xB$.于是$B/xB$是诺特环,于是它是诺特$B$模,按照$B$是整环,归纳得到$B/x^iB$是诺特$B$模,于是$B/x^iB$是诺特环.最后按照$x^nB\cap A\subseteq x^nA_p\cap A=p^{(n)}$,为说明$A/p^{(n)}$是诺特环只需说明$A/(x^nB\cap A)$是诺特环.但是$A/(x^nB\cap A)$是$B/x^nB$的子环,且$B/x^nB$作为$A/(x^nB\cap A)$模是有限的,于是再按照Eakin-Nagata定理得到$A/(x^nB\cap A)$是诺特环,完成证明.
    \end{proof}
\end{enumerate}



















\newpage
\section{维数论}
\subsection{Krull维数}

基本定义.
\begin{enumerate}
	\item 非零环$A$的Krull维数(简称维数)定义为它素理想严格包含链长度的上确界.这里链$P_0\subsetneqq P_1\subsetneqq\cdots\subsetneqq P_n$的长度约定为严格包含号的个数,即$n$.这个定义等价于环$A$素谱的组合维数,即不可约闭子集严格包含链长度的上确界.环$A$的维数记作$\dim A$.
	\item 对素理想$p\in\mathrm{Spec}(A)$,定义它的高度是$A$中以$p$为包含意义下最大端的严格包含的素理想链的长度的上确界.这个定义等价于局部化$A_p$的维数.记作$\mathrm{ht}(p)=\dim A_p$.
	\item 对素理想$p\in\mathrm{Spec}(A)$,定义它的余高度是$A$中以$p$为包含意义下最小端的严格包含的素理想链的长度的上确界.这个定义等价于商环$A/p$的维数.记作$\mathrm{coht}(p)=\dim A/p$.
	\item $A$模$M$的维数约定为$\dim M=\dim(A/\mathrm{Ann}(M))$,即素谱的闭子集$V(\mathrm{Ann}(M))$的组合维数.如果$M$是有限$A$模,那么这个闭子集恰好是$M$的支集.
	\item 对一般理想$I\subseteq A$,它的高度就是$A$模$A/I$的高度,也即$I$的全部极小素理想的高度的下确界.也即$V(I)$的组合维数.
	\item 零环的维数可约定为$-1$或者$-\infty$,但是这无关紧要.
\end{enumerate}

一些例子.
\begin{enumerate}
	\item 环的维数是零等价于要求每个素理想都是极大理想.
	\item 整环的维数是1当且仅当它的非零素理想都是极大理想,例如$\mathbb{Z}$和域上的多项式环$k[x]$的维数都是1.
	\item 零维诺特环就是阿廷环.
	\item Nagata给出过维数$\infty$的诺特环:设$A=k[X_1,X_2,\cdots]$是可数个未定元的域$k$上的多项式环,取严格递增的正整数列$\{m_n\}$,满足$m_{i+1}-m_i>m_i-m_{i-1},\forall i>1$.取$p_i=(x_{m_i+1},x_{m_i+2},\cdots,x_{m_{i+1}})$.记$S$是全体$p_i$的并在$A$中补集,它是一个乘性闭子集.那么按照$S^{-1}p_i$的维数是$m_{i+1}-m_i\to\infty$,得到$S^{-1}A$的维数无穷.重点是可验证$S^{-1}A$是诺特环.
	\item 在后面我们会证明诺特局部环的维数总是有限的,于是诺特环素理想的高度总是有限的.
\end{enumerate}

catenary条件.一个严格递增的素理想链$P_0\subsetneqq P_1\subsetneqq\cdots\subsetneqq P_n$称为饱和的,如果不能在中间任一位置添加素理想使得它仍然构成严格饱和的链.称一个环$A$是catenary的,如果对任意两个素理想$p\subsetneqq q$,以它们作为初始端和终端的严格包含的饱和素理想链的长度是固定的(包含总是无穷长度的情况).

引理.设$A$是域$k$上的有限生成代数,记作$A=k[a_1,a_2,\cdots,a_n]$.设$A$是整环,那么生成元集$\{a_1,a_2,\cdots,a_n\}$中的极大代数无关集总是域扩张$k\subset\mathrm{Frac}(A)$的超越基.
\begin{proof}
	
	设$F=\mathrm{Frac}(A)$,那么$F=k(a_1,a_2,\cdots,a_n)$.如果$a_i$全部是$k$上代数元,那么此时$k\subseteq F$是代数扩张,超越基是空集.现在取$\{a_1,a_2,\cdots,a_n\}$中的极大代数无关集,不妨记作$\{a_1,a_2,\cdots,a_m\}$.那么每个$a_i,i\ge m+1$都在$k(a_1,a_2,\cdots,a_m)$上代数,否则$\{a_1,a_2,\cdots,a_m,a_i\}$也是代数无关集和极大性矛盾.于是特别的$k(a_1,a_2,\cdots,a_m)\subseteq k(a_1,a_2,\cdots,a_n)$是代数扩张,于是$\{a_1,a_2,\cdots,a_m\}$是$k\subseteq F$的超越基.
\end{proof}

引理.设$k$是域,设$A$是有限生成的$k$代数并且是整环,记$A$的商域为$F$,记$F$在$k$上的超越维数为$r$.那么$A$是域当且仅当$r=0$,即当且仅当$k\subseteq F$是代数扩张.
\begin{proof}
	
	充分性.如果$r=0$,那么$A$中每个元都是$k$上的代数元,任取$\alpha\in A\backslash\{0\}$,那么存在不可约的非零多项式$f(X)=a_nX^n+a_{n-1}X^{n-1}+\cdots+a_0\in k[X]$,$a_n\not=0$,按照$f$不可约说明$a_0\in k$不为零,那么有$-a_0^{-1}(a_n\alpha^{n-1}+\cdots+a_1\alpha)=1$,这说明$A$中每个元都有逆元,于是$A$自身是域.
	
	必要性.设$A=k[a_1,a_2,\cdots,a_n]$是域,超越维数$r>0$.上一个引理说明$\{a_1,a_2,\cdots,a_n\}$中包含了一个$k\subseteq A$的超越基,不妨设为$\{a_1,a_2,\cdots,a_r\}$.记$K=k(a_1,a_2,\cdots,a_r)$,那么$K\subseteq A$是有限扩张,即$A$是$K$上有限模.
	
	我们断言可取$g_1\in k[a_1,a_2,\cdots,a_r]\backslash\{0\}$使得$A$是$k[a_1,a_2,\cdots,a_r]_{g_1}$上的有限模.事实上按照每个$a_i$都是$K$上的代数元,这些元只有有限个,于是可取足够大的正整数$N$,使得存在次数$\le N$的多项式$F_i(T)\in K[T]$,满足$a_i^{N+1}=F_i(a_i)\in A$,对任意$1\le i\le n$成立.现在取$0\not=g_1\in k[a_1,a_2,\cdots,a_r]$使得每个$g_1F_i(T)$的系数都落在$k[a_1,a_2,\cdots,a_r]$中,换句话讲$g_1$可以取做所有$F_i(T)$的系数(作为$K=k(a_1,a_2,\cdots,a_r)$中的元)的分母的乘积.于是每个$F_i[T]$的系数可视为$k[a_1,a_2,\cdots,a_r]_{g_1}$中的元.于是$A$作为$k[a_1,a_2,\cdots,a_r]_{g_1}$模可被$a_1^{e_1}a_2^{e_2}\cdots a_n^{e_n}$,其中$0\le e_1,e_2,\cdots,e_n\le N$生成,于是$A$是有限$k[a_1,a_2,\cdots,a_r]_{g_1}$模.
	
	现在取$A$作为$K$模的一组基$f_1,f_2,\cdots,f_s\in A$,考虑$k[a_1,a_2,\cdots,a_r]_{g_1}$模同态$\varphi:(k[a_1,a_2,\cdots,a_r]_{g_1})^s\to A$为$(\mu_1,\mu_2,\cdots,\mu_s)\mapsto\sum_{i=1}^s\mu_if_i$.我们断言存在$0\not=g_2\in k[a_1,a_2,\cdots,a_r]_{g_1}$使得局部化$\varphi_{g_2}$是同构.事实上取$C=\mathrm{coker}(\varphi)$,那么它是有限$k[a_1,a_2,\cdots,a_r]_{g_1}$模.于是它的支集$\mathrm{Supp}(C)$是$\mathrm{Spec}(k[a_1,a_2,\cdots,a_r]_{g_1})$的闭子集.于是支集在整个素谱中的补集是非空开集(非空因为至少零理想在这个补集里).于是可取主开集$D(g_2),g_2\in k[a_1,a_2,\cdots,a_r]_{g_1}$满足$D(g_2)\cap\mathrm{Supp}(C)=\emptyset$.于是$C_{g_2}=0$,它说明$\varphi_{g_2}$是满的,而单射因为$k[a_1,a_2,\cdots,a_r]_{g_1}$是整环,于是$\varphi_{g_2}$是同构.我们不妨记$g_2\in l[a_1,a_2,\cdots,a_r]$,记$g=g_1g_2$.
	
	总结一下,我们找到了元$0\not=g\in k[a_1,a_2,\cdots,a_r]$满足$A=k[a_1,a_2,\cdots,a_n]$在$B=k[a_1,a_2,\cdots,a_r]_{g}=k[a_1,a_2,\cdots,a_r][1/g]$上有限并且自由.但是$B$不是域:取$h\in k[a_1,a_2,\cdots,a_r]$和$g$互素的不可约多项式,那么$h$在$B$中不存在逆.于是$hB$是$B$的真子模.按照$A$在$B$上有限自由,得到$hA\subsetneqq A$也是$B$模$A$的真子模.但是按照$A$是域得到$hA=A$,这矛盾.
\end{proof}

设$k$是域,$A$是整环并且是有限生成$k$代数,那么有$\dim A=\mathrm{tr.deg}_k(A)$.按照$A$的条件,它同构于$R/p$,其中$R$是$k$上某个多项式环$R=k[X_1,X_2,\cdots,X_n]$,那么这个结论就是在讲素理想$p$的余高度恰好就是剩余类域$\kappa(p)$在$k$上的超越维数.粗略的讲,这个证明就是在给出多项式环$R=k[X_1,X_2,\cdots,X_n]$上的素理想严格升链和相应剩余类域扩张链的对应.
\begin{proof}
	
	记$A=k[X_1,X_2,\cdots,X_n]/p$,其中$p$是$k[X_1,X_2,\cdots,X_n]=R$的素理想.如果有$R$中素理想$p\subseteq p'$,记相应的剩余类域是$F=\kappa(p)=(R/p)_p$和$F'=\kappa(p')=(R/p')_{p'}$,那么$F'\subseteq F$,于是$\mathrm{tr.deg}(F')\le\mathrm{tr.deg}(F)$.我们断言如果$p\subsetneqq p'$,那么这个超越维数的不等式是不能取等号的.若否,假设$F$和$F'$的超越维数相同,记作$r$.我们记$A=k[a_1,a_2,\cdots,a_n]=R/p$和$A'=k[b_1,b_2,\cdots,b_n]=R/p'$.如果记$\{b_1,b_2,\cdots,b_r\}$是$A'$的一组超越基,那么相应的$\{a_1,a_2,\cdots,a_r\}$也是$A$的一组超越基.现在记$S=k[X_1,X_2,\cdots,X_r]\backslash\{0\}$.按照$k[X_1,X_2,\cdots,X_r]\to A'$,$X_i\mapsto b_i$是单射(代数无关性),说明$S\cap p'=\emptyset$,于是$S\cap p=\emptyset$.于是在$S^{-1}R=k(X_1,X_2,\cdots,X_r)[X_{r+1},\cdots,X_n]$中有素理想$S^{-1}p\subsetneqq S^{-1}p'$.但是$(S^{-1}R)/(S^{-1}p)\cong S^{-1}(R/p)\cong k(a_1,a_2,\cdots,a_r)[a_{r+1},\cdots,a_n]$,这是一个整环并且是$k(a_1,a_2,\cdots,a_r)$上的代数扩张,于是上一个引理得到$S^{-1}R/S^{-1}p$是域,也即$S^{-1}p$是$S^{-1}R$的极大理想,这导致$S^{-1}p=S^{-1}p'$矛盾.
	
	下面给出$A$中严格包含的素理想链$0=p_0\subsetneqq p_1\subsetneqq\cdots\subsetneqq p_s$.这对应于剩余类域的扩张链$\kappa(p_s)\subset\kappa(p_{s-1})\subset\cdots\subset\kappa(p_0)$.其中$\kappa(p_0)=\mathrm{Frac}(A)=F$.于是上一段得到$\mathrm{tr.deg}_kF\ge s$,取右侧上确界得到$\dim A\le\mathrm{tr.deg}(A)$.
	
	最后证明$\mathrm{te.deg}_k(A)\le\dim A$.为此我们来对$r=\mathrm{tr.deg}_k(A)$归纳.首先$r=0$时$A$是整环并且在$k$上代数,考虑$A$中元素的极小多项式就得到$A$实际上是域,于是$\dim A=0$.现在设$r>0$,并且命题对$r-1$成立.记$A=k[a_1,a_2,\cdots,a_n]$,设$a_1$是$k$上的一个超越元(因为$r>0$知这必然取到).取$S=k[a_1]\backslash\{0\}\subseteq A$,那么$S^{-1}(A)=k(a_1)[a_2,a_3,\cdots,a_n]$在$k(a_1)$上的超越维数是$r-1$.于是按照归纳假设,有$r-1\le\dim(S^{-1}A)$,于是$\dim S^{-1}A=r-1$.取长度为$r-1$的严格包含的$S^{-1}A$中素理想链$0=q_0\subsetneqq q_1\subsetneqq\cdots\subsetneqq q_{r-1}$.那么每个$q_i=S^{-1}p_i$,其中$p_i\in\mathrm{Spec}(A)$,并且$S\cap p_i=\emptyset$.现在只要说明$p_{r-1}$不是极大理想,那么这个链就可以至少延长一个长度,这就导致$\dim A\ge r$完成归纳.而$p_{r-1}$非极大理想等价于$A/p_{r-1}$不是域.但是按照$S$和$p_{r-1}$不交,得到$a_1'\in A/p_{r-1}$是$k$上的超越元,于是$A/p_{r-1}$不是域.
\end{proof}

\begin{enumerate}
	\item 推论.如果$k$是域,那么$\dim k[X_1,X_2,\cdots,X_n]=n$.
	\item $R=k[X_1,X_2,\cdots,X_n]$上的素理想$P$是极大理想当且仅当它的剩余类域在$k$上是代数扩张.
	\item 如果$p$是$R=k[X_1,X_2,\cdots,X_n]$的极大理想,那么高度$\mathrm{ht}(p)=\dim R=n$.
	\begin{proof}
		
		取$k$的代数闭包为$\overline{k}$,取$B=\overline{k}[X_1,X_2,\cdots,X_n]$,那么$pB$是$B$的真理想.那么$pB$包含于$B$的某个极大理想$\overline{p}$中.按照弱零点定理有$\overline{p}=(X_1-a_1',X_2-a_2',\cdots,X_n-a_n')$,其中$(a_1',a_2',\cdots,a_n')\in\overline{k}^n$.记$p_i=(X_1-a_1',\cdots,X_i-a_i')\cap R$,那么$p_i$为$R$的素理想,并且得到素理想链$0=p_0\subseteq p_1\subset\cdots\subseteq p_n=p$.倘若能够说明$p_i$相邻两项是严格包含关系,会得到$p$的高度$\ge n$,但是恒有$\mathrm{ht}(p)\le\dim R=n$,这就能完成证明.现在有单同态:
		$$R_i=R/p_i\to\overline{R_i}=\overline{k}[X_1,X_2,\cdots,X_n]/(X_1-a_1',X_2-a_2',\cdots,X_n-a_n')$$
		
		其中$\overline{R_i}$中的每个元都是$R_i$上的代数元,这导致$\mathrm{trdeg}_k(R_i)=\mathrm{trdeg}_k(\overline{R_i})=\mathrm{trdeg}_{\overline{k}}(\overline{R_i})=n-i$.这说明$p_i\subsetneqq p_{i+1}$.
	\end{proof}
	\item 对多元多项式环$R=k[X_1,X_2,\cdots,X_n]$,任意素理想$p$满足维数公式$\mathrm{ht}(p)+\mathrm{coht}(p)=\dim R$.
	\begin{proof}
		
		首先已经得到了$\dim R=n$,另外$\mathrm{coht}(p)=\dim R/p$恰好就是$r=\mathrm{trdeg}_k(R/p)$.于是问题等价于证明$\mathrm{ht}(p)=n-r$.另外我们知道给$A/p=k[a_1,a_2,\cdots,a_n]$的生成元适当排列会使得$\{a_1,a_2,\cdots,a_r\}$就是超越基,此时如果取$S=k[x_1,x_2,\cdots,x_r]\backslash\{0\}$会使得$S^{-1}p$是$S^{-1}R$的极大理想.于是可不妨设$p$是极大理想,否则以$S^{-1}R$替代$R$.但是此时$p$的余高度是零,上一定理得到它的高度是$n$,这就得证.
	\end{proof}
	\item 更一般的,如果$A$是有限生成$k$代数并且是整环,那么对$A$的任意素理想$p$,总有$\mathrm{ht}(p)+\mathrm{coht}(p)=\dim A$.
	\begin{proof}
		
		按照上一条证明中同样的手段,可归结为证明$A$的极大理想$p$的高度总是$\dim A$.设$\dim A=r$,按照诺特正规化引理,有单同态$k[X_1,X_2,\cdots,X_r]\to A$,使得$A$在$k[X_1,X_2,\cdots,X_n]$整.记$p'=k[X_1,X_2,\cdots,X_r]\cap p$,那么$p'$是$k[X_1,X_2,\cdots,X_r]$中的极大理想.按照上升定理和上一条结论,得到$\mathrm{ht}(p)=\mathrm{ht}(p')=r$.
	\end{proof}
	\item 如果$k$是域,那么是有限生成$k$代数$A$总是catenary的.(此即域总是universally catenary,一个环称为universally catenary,如果它的每个有限生成代数都是catenary的)
	\begin{proof}
		
		我们只需证明,如果$p'\subsetneqq p$是$A$中饱和的素理想链,那么$\mathrm{coht}(p')-\mathrm{coht}(p)=1$.因为一旦证明这个结论,那么任意两个素理想$p'\subsetneqq p$延拓而成的饱和的素理想链的长度必然是$\mathrm{coht}(p')-\mathrm{coht}(p)$.
		
		不妨设$p'=0$并且$A$是整环,否则可以用$A/p'$替换$A$.那么按照$0\subsetneqq p$是饱和的素理想链得到$\mathrm{ht}(p)=1$.于是上一条结论得到$\mathrm{coht}(p)=\dim A-\mathrm{ht}(p)=\mathrm{coht}(p')-1$.
	\end{proof}
\end{enumerate}

给定环$R$,取多项式环$R[X]$,那么有如下维数的关系,特别的,$R$的维数无穷当且仅当$R[X]$的维数无穷.
$$1+\dim R\le\dim R[x]\le 1+2\dim R$$
\begin{proof}
	
	取$R$上一个素理想链$P_0\subsetneqq\cdots\subsetneqq P_n$,那么注意到有$R[X]$上的素理想链$P_0R[X]\subsetneqq\cdots\subsetneqq P_nR[X]\subsetneqq P_nR[X]+(X)$,于是左边不等式成立.现在取$R$的素理想$P$,取$R[X]$上一列素理想链$Q_0\subsetneqq\cdots\subsetneqq Q_r$满足$Q_i\cap R=P$,那么看到这导致存在一个长度为$r$的$R[X]/PR[X]= (R/P)[X]$上的素理想链.那么这个素理想链也必然在$(R/P)_P[X]$上,注意到$(R/P)_P$是PID,于是这个多项式环是$PID$,那么$r\le1$.现在取$R[X]$中的素理想链$Q_0\subsetneqq\cdots\subsetneqq Q_m$,那么它收缩到$R$上得到一个素理想链$P_0\subsetneqq\cdots\subsetneqq P_n$,按照所证,至多有两个相邻的$Q_i$收缩到相同的$P_j$,也就是说$m+1\le 2(n+1)$,也就是右侧不等式.
\end{proof}
\newpage
\subsection{维数论基本定理}
\subsubsection{Hilbert函数}

基本定义.
\begin{enumerate}
	\item 对给定交换半群$G$,一个$G$-分次环是一个环$R$,它作为阿贝尔群存在直和分解$R=\oplus_{i\in G}R_i$,使得$\forall i,j\in G$,有$R_iR_j\subseteq R_{i+j}$.
	\item 给定$G$-分次环$R$,一个$R$模$M$称为$G$-分次模,如果存在阿贝尔群的直和分解$M=\oplus_{i\in G}M_i$,满足$\forall i,j\in G$,有$R_iM_j\subseteq M_{i+j}$.
	\item 分次环或者分次模中的元素按照定义中的直和做分解称为元素的齐次分解,每个$R_i$或者$M_i$中的元素称为分次环或者分次模中的$i$次齐次项,齐次分解中的齐次项称为元素的齐次分量.
	\item 分次子模.把分次模$M$的子模$N$称为分次子模,或者齐次子模,如果它满足如下三个等价条件之一.特别的,当考虑分次环作为自身的模的时候,分次子模称为分次理想.
	\begin{enumerate}
		\item $N$有直和分解$N=\oplus_{i\in G}N\cap M_i$.
		\item $N$是由$M$中齐次元生成的$A$模.
		\item 对$N$中每个元的齐次分解,齐次分量总在$N$中.
	\end{enumerate}
	\begin{proof}
		
		2推3,设$N=(m_j)$,其中每个$m_i$都是$M$中的齐次元,任取$N$中的元$n$,设它的齐次分解为$n=\sum n_i$,那么有$R$中的元$r_j$使得$\sum n_i=\sum r_jm_j$,这是一个有限和,将$r_j$做齐次分解,在等式两边取$\deg n_i$的项就得到$n_i\in N$.3推1,必然有$\oplus_{i\in G}N\cap M_i\subseteq N$,现在任取$n\in N$,做齐次分解$n=\sum_i n_i$,按照条件$n_i\in M_n\cap N$.1推2,$\cup_i(N\cap M_i)$中的元都是齐次元,这个子集自然生成整个$N$.
	\end{proof}
\end{enumerate}

一些基本性质.
\begin{enumerate}
	\item $R_0$是$R$的一个子环,于是从$R_0M_i\subseteq M_i$得到每个$M_i$是$R_0$模.
	\begin{proof}
		
		首先$R_0$自身是阿贝尔群,并且从$R_0R_0\subseteq R_0$得到乘法封闭.接下来只需验证乘法幺元在$R_0$中.记幺元1的齐次分解为$1=\sum x_i$.任取$R$中一个元$r$,取齐次分解$r=\sum r_i$,那么$r_i=r_i\cdot1=\sum_j x_jr_i$,比较次数得到当$j\not=0$的时候有$r_ix_j=0$,于是对$i$求和得到$rx_j=0$,这个式子对任意$r$成立,因而取$r=1$就得到$j\not=0$时有$x_j=0$,这说明了$1\in R_0$.
	\end{proof}
	\item 如果$N$是分次模$M$的齐次子模,那么$M/N$也是一个分次$R$模.我们先来验证$\oplus_{i\in G}(M_i+N)/N$是一个直和,也即分量两两交平凡:如果有$x\in M_i$和$y\in M_j$使得$x+N=y+N$,此即$x-y\in N$,但是按照齐次子模的定义,$x,y\in N$.据此得到直和分解$M/N=\oplus_{i\in G}(M_n+N)/N=\oplus_{i\in G}M_i/M_i\cap N$.于是$M/N$是一个分次$R$模.
	\item 我们可以定义分次结构之间的态射,即如果分次环$A$有两个分次模$M,N$,模同态$f:M\to N$称为分次模同态,如果它把$M$中的齐次元映射为$N$中的同次数的齐次元.那么我们断言分次模同态的核总会是一个分次子模.
	\begin{proof}
		
		取$x\in\ker f$,记它的齐次分解为$x=\sum_ix_i$,为了证明$\ker f$是$M$的分次子模,只需验证这里每个$x_i\in\ker f$.为此,从$f(x)=0$得到$\sum_if(x_i)=0$,这里按照$f$把齐次元映射为同次数的齐次元,说明上个等式左侧依旧是一个齐次分解,这导致每个$f(x_i)=0$.
	\end{proof}
	\item 诺特分次环上的分次理想可以被有限个齐次元生成.
\end{enumerate}

但是通常来讲我们只会用到$G=\mathbb{N}$的情况,即自然数集合上赋予常义加法.此时对于分次环$R=\oplus_{n\ge0}R_n$,子集$\oplus_{n\ge1}R_n$构成了$R$的理想,记作$R^+$,并且有同构式$R/R^+\cong R_0$.今后提及分次环特指$\mathbb{N}$-分次环.一个最基本的例子是,考虑环$R_0$上的多元多项式环$R=R_0[X_1,X_2,\cdots,X_n]$,齐次元约定为齐次多项式,次数约定为齐次多项式的常义次数,这构成一个分次环.另外我们也可以对未定元$X_i$赋予权重$d_i$,此时单项式$X_1^{e_1}X_2^{e_2}\cdots X_n^{e_n}$的次数为$\sum_i d_ie_i$.
\begin{enumerate}
	\item 分次理想的根理想仍然是分次理想.
	\begin{proof}
		
		设$I$是分次环$R$的分次理想,记$J=\sqrt{I}$.按照分次理想的等价定义,需要验证的是对$J$中每个多项式,它的每个齐次分支都落在$J$中.为此任取$f\in J$,于是存在某个正整数$n$使得$f^n\in I$.取$f$的次数最高的齐次分支$f_0$,那么$f_0^n$就是$f^n$的次数最高的齐次分支,于是按照$I$是齐次理想,说明$f_0^n\in I$,于是$f_0\in J$.接下来对$f-f_0$重复操作,至多经$\deg f$步就得到$f$的全部齐次分支均在$J$中.
	\end{proof}
	\item 分次理想$p$是素理想当且仅当对两个齐次元$f,g$,从$fg\in p$推出$f,g$至少一个$\in p$.
	\begin{proof}
		
		任取两个多项式$f,g$满足$fg\in p$,考虑$f,g$最高次的齐次分支$f_0,g_0$,那么有$f_0g_0\in p$,按照条件不妨设$f_0\in p$,于是$(f-f_0)g\in p$,于是对$\deg fg$归纳即可得到命题成立
	\end{proof}
\end{enumerate}

分次环关于一个齐次元的局部化.
\begin{enumerate}
	\item 设$A=\oplus_{n\ge\mathbb{Z}}A_n$是分次环,设$f$是次数为$d$的元.我们要给局部化$A_f$上赋予一个典范分次结构.记$A_{f,n}=\{a/f^k\in A_f\mid k\in\mathbb{N}^{\ge0},a\in A_{n+kd}\}\subseteq A_f$.这是一个子群,我们断言有$A_f=\oplus_{n\in\mathbb{Z}}A_{f,n}$,并且这是一个分次结构.
	\begin{proof}
		
		首先$A_{f,n}$是子群:如果$a_1\in A_{n+k_1d},a_2\in A_{n+k_2d}$,不妨设$k_1\ge k_2$,那么$a_1/f^{k_1}+a_2/f^{k_2}=(a_1+a_2f^{k_1-k_2})/f^{k_1}\in A_{f,n}$.
		
		有$A_f=\sum_{n\in\mathbb{Z}}A_{f,n}$以及$A_{f,m}A_{f,n}\subseteq A_{f,m+n}$.最后只需验证$A_f=\sum_nA_{f,n}$是直和.
		
		如果有$\sum_{n\in\mathbb{Z}}h_n=0$,其中$h_n\in A_{f,n}$,并且对几乎全部$n$有$h_n=0$.可记$h_n=a_n/f^{k_n}$,其中$a_n\in A_{n+k_nd}$.可取$l$足够大,大于$h_n\not=0$的那些项的$k_n$的最大值,并且在$A$中有$\sum_nf^{l-k_n}a_n=0$.现在$f^{l-k_n}a_n$的次数是$n+ld$,对不同的$n$这些单项次数不同,所以每个$f^{l-k_n}a_n=0$.所以每个$h_n=a_n/f^{k_n}=0$.
	\end{proof}
    \item 上一条定义的分次结构约定为$A_f$的典范的分次结构.$A_f$的零次子环记作$A_{(f)}$.
    \item 如果$A$是分次环,$f\in A_d,g\in A_e$,其中$d,e\ge1$,那么存在典范同构:
    $$A_{(fg)}\cong (A_{(f)})_{f^{-e}g^d},\frac{a}{(fg)^k}\mapsto\left(\frac{g^d}{f^e}\right)^{-k}\frac{g^{(d-1)k}a}{f^{(e+1)k}},a\in A_{(d+e)k}$$
\end{enumerate}

滤过诱导的分次环和I-adic滤过.
\begin{enumerate}
	\item 我们定义过环$A$上的滤过,即一个理想降链$A=J_0\supset J_1\supset J_2\supset\cdots$,满足$J_iJ_j\subseteq J_{i+j}$.给定滤过后,我们这样定义一个分次环:记$\mathrm{gr}_n(A)=J_n/J_{n+1},n\ge0$,再记$\mathrm{gr}(A)=\oplus_{n\ge0}\mathrm{gr}_n(A)$.这个直和自然是一个阿贝尔群,另外我们可以定义乘法使得它是一个分次环,即定义$(x+J_{n+1})(y+J_{m+1})=xy+I_{n+m+1}\in\mathrm{gr}_{n+m}(A)$.这个定义是良性的:如果$x-x'\in J_{n+1}$和$y-y'\in J_{m+1}$,那么有$xy-x'y'=(x-x')y+x'(y-y')\in J_{m+n+1}$.分次环$\mathrm{gr}(A)$称为$A$的关于滤过$(J_i)$的分次环.
	\item 诱导分次环和完备化.给定环$A$上的滤过$(J_n)$,那么滤过诱导了$A$上的线性拓扑.记完备化$\widehat{A}=\lim\limits_{\leftarrow}A/J_n$,记$J_n'=\ker(\widehat{A}\to A/J_n)$,我们解释过有$\widehat{A}/J_n'\cong A/J_n$.于是$(J_n')$诱导的分次环$\mathrm{gr}(\widehat{A})$和$(J_n)$诱导的分次环$\mathrm{gr}(A)$是同构的.
	\item $I$-adic滤过诱导的分次环.我们定义过环$A$上关于理想$I$的adic滤过为$A=I^0\supset I^1\supset\cdots$,此时它诱导的分次环即$B=\oplus_{n\ge0}I^n/I^{n+1}$.这里$B$通常会记作$\mathrm{gr}_I(A)$或者$\mathrm{gr}^I(A)$等.此时$B$有如下性质:
	\begin{enumerate}
		\item $B_n=I^n/I^{n+1}$中的元具有这样一个特性:它可以表示为$B_1=I/I^2$中$n$个元的乘积.这一事实导致,$B$必然是子集$B_1$在子环$B_0=A/I$上生成的代数.
		\item 更进一步的,假设$I=(x_1,x_2,\cdots,x_r)$有限生成理想,设$x_i$在$B_1=I/I^2$中的像是$y_i$,那么就有$B=(A/I)[y_1,y_2,\cdots,y_r]$.于是此时诱导的分次环$B$是多项式环$(A/I)[X_1,X_2,\cdots,X_r]$的商环.另外按照$(A/I)[X_1,X_2,\cdots,X_r]$到$B$的典范满同态实际上是一个分次模同态,于是此时的核是多项式环的一个分次理想,于是此时$B$上的分次结构恰好是多项式环上典范分次的商结构所诱导的分次结构.
	\end{enumerate}
\end{enumerate}

我们接下来要给一种特殊的分次模上定义希尔伯特级数和希尔伯特多项式等概念.这个条件有点复杂,仅仅诺特分次环上有限生成分次模也是不够的.为了体会这个条件,我们先来给出几个性质:
\begin{enumerate}
	\item 分次环诺特的充要条件:分次环$R=\oplus_{n\ge0}R_n$是诺特环当且仅当$R_0$是诺特环并且$R$是有限生成$R_0$代数.事实上,如果取有限生成了$R^+$的齐次元集合为$x_1,x_2,\cdots,x_r$,那么有$R=R_0[x_1,x_2,\cdots,x_r]$.
	\begin{proof}
		
		这里后推前只要用到希尔伯特基定理.下面证明前推后.假设$R$是诺特环,按照$R/R^+\cong R_0$得到$R_0$是诺特环.现在$R^+$是$R$的齐次理想,按照$R$诺特它可以被有限个元在$R$上生成,按照齐次理想的定义,这有限个元做齐次分解后,全部齐次分量都在$R^+$中,于是得到$R^+$可以被有限个齐次元在$R$上生成,不妨记作$x_1,x_2,\cdots,x_r$.我们来证明$R=R_0[x_1,x_2,\cdots,x_r]$,这就得到$R$是$R_0$上有限生成代数.为此只需验证每个$R_n\subseteq R_0[x_1,x_2,\cdots,x_r]$.
		
		我们来对$n$归纳,当$n=0$的时候自然有$R_0\subseteq R_0[x_1,x_2,\cdots,x_r]$.现在假设$n>0$,并且对小于$n$的指标$i$总有$R_i\subseteq R_0[x_1,x_2,\cdots,x_n]$.设$x_i$的次数是$d_i$,那么就有$R_n=\sum_ix_iR_{n-d_i}$,这里约定当$k<0$的时候$R_k=0$.这个验证是直接的,右侧包含于左侧平凡,反过来任取左侧中的元$f\in R_n$,按照$f\in R^+$得到$R$中的元$f_i$使得$f=\sum_ix_if_i$,现在把$f_i$替换为它的次数为$\deg f-d_i$的分量,如果这个次数小于零则替换为零,此时仍保证该等式成立.这导致$f$落在右侧.于是按照归纳假设,每个$R_{n-d_i}$落在$R_0[x_1,x_2,\cdots,x_r]$,于是$R_n\subseteq R_0[x_1,x_2,\cdots,x_r]$.
	\end{proof}
	\item 给定诺特分次环$R=\oplus_nR_n$,给定有限生成$R$分次模$M=\oplus M_n$,那么每个$M_n$都是有限生成$R_0$模.
	\begin{proof}
		
		按照上一条可记$R=R_0[x_1,x_2,\cdots,x_r]$,这里$x_i$是$R^+$中的次数记作$d_i$的齐次元.再设$M$作为$R$模可被有限个齐次元$m_1,m_2,\cdots,m_s$生成.那么对每个$M_n$中的元$m$,它可以表示为$m=\sum_if_im_i$,这里$f_i\in R$.现在把$f_i$做齐次分解,把$f_i$替换为次数为$\deg m-\deg m_i$的齐次分量能够保证此时等式仍然成立.但是每个$f_i$是$S_i=\{x_1^{t_1}x_2^{t_2}\cdots x_r^{t_r}\mid \sum_jd_jt_j=\deg m-\deg m_i\}$中元素的$R_0$线性组合,于是$M_n$中每个元都是有限集合$S=\cup_{1\le i\le s}\{x_1^{t_1}x_2^{t_2}\cdots x_r^{t_r}m_i\mid\sum_jd_jt_j=\deg m-\deg m_i\}$中元素的$R_0$线性组合,并且$S\subseteq M_n$,因而$M_n$是有限$R_0$模.
	\end{proof}
\end{enumerate}

上面的性质说明如果约定如下条件:$R=\oplus_{n\ge0}R_n$是诺特分次环,而$M=\oplus_{n\ge0}M_n$是有限分次模,并且$R_0$是阿廷环.这个条件我们记作$(*)$条件,那么此时每个$M_n$将会是有限长度$R_0$模(即同时为诺特模和阿廷模),此时长度$l(M_n)$是自然数,我们约定此时分次模$M$的希尔伯特级数为$P(M,t)=\sum_{n\ge0}l(M_n)t^n\in\mathbb{Z}[[t]]$.

我们接下来给出的一个惊人的结论是,尽管希尔伯特级数仅仅是一个形式幂级数,但是在我们的条件$(*)$下,它实际上总会是一个有理分式函数.例如例如考虑域$k$上的多元多项式环$R=k[x_1,x_2,\cdots,x_r]$,赋予常义的分次,那么$R_n$的长度,或者说$R_n$作为$R_0=k$线性空间的维数,恰好就是它里面不同单项式的个数,也即$x_1^{e_1}x_2^{e_2}\cdots x_r^{e_r},\sum_ie_i=n$.这个结果可以表示为组合数$\left(\begin{array}{c}n+r-1\\r-1\end{array}\right)$.于是此时的希尔伯特级数为:
$$\sum_{n\ge0}\left(\begin{array}{c}n+r-1\\r-1\end{array}\right)t^n=(1-t)^{-r}$$

现在给出一般结论.如果$R=\oplus_{n\ge0}R_n$是诺特分次环,而$M=\oplus_{n\ge0}M_n$是有限分次$R$模,并且假设$R_0$是阿廷环,那么$M$的希尔伯特级数是一个有理分式函数.更精细的结果是,在相同条件下,如果设$R=R_0[x_1,x_2,\cdots,x_r]$,其中$x_i$是$R$中次数为$d_i$的齐次元,那么存在一个$\mathbb{Z}$系数多项式$f(t)$,使得希尔伯特级数可以表示为$P(M,t)=\frac{f(t)}{\prod_{1\le i\le r}(1-t^{d_i})}$.
\begin{proof}
	
	我们来对$r$归纳.$r=0$的时候即$R=R_0$,此时$R$上有限模$M=\oplus_{n\ge0}$是有限长度模.按照长度公式$l(M)=\sum_{n\ge0}l(M_n)$,说明当$n$足够大的时候恒有$l(M_n)=0$,此即$M_n=0$,因此此时希尔伯特级数恰好是一个多项式,满足结论(注意对空集取乘积是1).
	
	现在假设$r>0$,左乘$x_r$是一个$R_0$模同态$M_n\to M_{n+d_r}$,设这个同态的核是$K_n$,余核是$L_{n+d_r}$.这就对每个$n\ge0$得到一个短正合列:
	$$\xymatrix{0\ar[r]&K_n\ar[r]&M_n\ar[r]^{\cdot x_r}&M_{n+d_r}\ar[r]&L_{n+d_r}\ar[r]&0}$$
	
	现在取$K=\oplus_{n\ge0}K_n$和$L=\oplus_{n\ge0}L_n$,这里为了使得正合列始终成立,我们约定$L_n=M_n,n<d_r$.我们断言$L=M/x_rM$,事实上取$M=\oplus_{n\ge0}M_n\to L=\oplus_{n\ge0}L_n=\oplus_{n\ge0}M_n/M_{n-d_r}$的同态为,$n<d_r$的时候为$M_n\to L_n=M_n$的恒等映射,而$n\ge d_r$的时候把$M_n$中的元$m$映射为$L_n$中的元$m+x_rM_{n-d_r}$,这个映射的核自然是$x_rM$,因而有$L\cong M/x_rM$.于是此时$K$是$M$的子模,而$L$是$M$的商,于是它们都是有限$R$模.接下来按照$x_rK=x_rL=0$,于是$K$和$L$均可视为$R/x_rR$模,并且此时仍为有限模.于是归纳假设对$P(K,t)$和$P(L,t)$成立.现在从正合列得到模的长度的等式$l(K_n)-l(M_n)+l(M_{n+_r})-l(L_{n+d_r})=0,n\ge0$,这个等式乘以$t^{n+d_r}$,并对$n\ge0$求和,这就得到$t^{d_r}P(K,t)-t^{d_r}P(M,t)+P(M,t)-P(L,t)=g(t)$,其中$g(t)$是求和时缺失的某些项的和,它是一个整系数多项式.因而按照归纳假设,整理得到$P(M,t)$是结论中所要求的形式.
\end{proof}

希尔伯特多项式.
\begin{enumerate}
	\item 特别的,在相同条件下,如果$R$可以被有限个1次齐次元在$R_0$上生成,此时上述定理中的$d_i$可以全部取为1,此时希尔伯特级数可以表示为$P(M,t)=f(t)(1-t)^{-r}$.
	\item 一般的,如果希尔伯特级数$P(M,t)$可以表示为$f(t)(1-t)^{-d}$,其中$f\in\mathbb{Z}[t],d\ge0$,并且如果$d>0$则约定$f(1)\not=0$,此时我们记$d=d(M)$.另外如果我们做形式幂级数展开$(1-t)^{-d}=\sum_{n\ge0}\left(\begin{array}{c}d+n-1\\d-1\end{array}\right)t^n$,把它代入希尔伯特级数,设$f(t)=\sum_{0\le i\le s}a_it^i$,比较两边$n$次项系数,比较好算的情况是$n\ge s-1+d$,此时得到如下等式:
	$$l(M_n)=a_0\left(\begin{array}{c}d+n-1\\d-1\end{array}\right)+a_1\left(\begin{array}{c}d+n-2\\d-2\end{array}\right)+\cdots+a_s\left(\begin{array}{c}d+n-s-1\\d-1\end{array}\right)$$
	\item 于是我们证明了希尔伯特多项式定理:如果在上述定理中$d_i=1$,记$d=d(M)$,那么存在一个$d-1$次有理系数多项式$\varphi_M(t)$,它的首系数是$\frac{f(1)}{(d-1)!}$,满足当$n\ge s+1-d$的时候恒有$l(M_n)=\varphi_M(n)$,这里$s$是多项式$(1-t)^dP(M,t)$的次数.
	\item 但是注意,对于不再全为1的一般的$d_i$,当$n$足够大的时候$l(M_n)$未必符合于多项式函数.
\end{enumerate}

我们整理一下三个概念.给定诺特分次环$R=\oplus_{n\ge0}R_n$上的有限分次模$M=\oplus_{n\ge0}M_n$,约定$R_0$是阿廷环,此时定义关于$M$的形式幂级数$P(M,t)=\sum_{n\ge0}l(M_n)t^n$是$M$的希尔伯特级数.数值函数$n\to l(M_n)$称为$M$的希尔伯特函数.如果额外的要求,$R$可以被有限个1次元在$R_0$上生成,设最小的使得$(1-t)^dP(M,t)$是多项式的次数为$d=d(M)$,那么当$n$足够大的时候希尔伯特函数的取值恰好符合一个多项式,它称为希尔伯特多项式.

Hilbert多项式的例子.设$k$是域,设$R=k[X_1,\cdots,X_r]$.
\begin{enumerate}
	\item 如果$M=R$,那么$M_n$具有长度$\left(\begin{array}{c}n+r-1\\r-1\end{array}\right)$.于是有:
	$$\varphi_M(X)=\frac{(X+r-1)\cdots(X+1)}{(r-1)!}$$
	\item 设$F\in R$是非零的次数$s$的齐次多项式,记$M=R/(F)$,它的Hilbert多项式记作$\varphi_F(n)$.我们有如下短正合列:
	$$\xymatrix{0\ar[r]&k[X_1,\cdots,X_r]_{n-s}F\ar[r]&k[X_1,\cdots,X_r]_n\ar[r]&(k[X_1,\cdots,X_r]/F)_n\ar[r]&0}$$
	
	于是$l(M_n)=l(R_n)-l(R_{n-s}),n\ge s$.于是有:
	\begin{align*}
		\varphi_F(n)&=\frac{(n+r-1)\cdots(n+1)}{(r-1)!}-\frac{(n-s+r-1)\cdots(n-s+1)}{(r-1)!}\\&=\frac{s}{(r-2)!}n^{r-2}+\text{低次项}
	\end{align*}

    于是对于非零理想$I\subseteq R$,记$N=k[X_1,\cdots,X_r]/I$,就有$\deg(\varphi_N)\le r-2$.
\end{enumerate}
\subsubsection{Samuel函数}

设$R$是半局部诺特环,设Jacobson根为$m$,称$R$的理想$I$为定义理想,如果存在某个$v>0$使得$m^v\subseteq I\subseteq m$,这个条件等价于讲$I$-adic拓扑和$m$-adic拓扑是一致的.这里的定义理想就是指"定义了$m$-adic拓扑的理想".另外对于局部环$(A,m)$,理想$I$是定义理想当且仅当$\sqrt{I}=m$.

现在任取有限$R$模$M$,定义$R'=\mathrm{gr}_I(R)=\oplus_{n\ge0}I^n/I^{n+1}$和$M'=\mathrm{gr}_I(M)=\oplus_{n\ge0}I^nM/I^{n+1}M$.我们来说明这个分次环和分次模满足上一节中的条件,于是之前的操作均可以应用其上:$P(M,t)=\sum_{n\ge0}l(M_n')t^n$为有理分式函数,且当$n$足够大的时候$n\mapsto l(M_n')$是一多项式,这里模的长度取为它作为$R_0'=R/I$模的长度,这个多项式的次数是$d(M')-1$,这里的$d(M')$表示的是最小的次数$d$使得$(1-t)^dP(M,t)$是多项式.
\begin{enumerate}
	\item 首先$R_0'=R/I$是阿廷环.需要验证的是包含了$I$的素理想$P$总是极大理想,此时有$m^v\subseteq P$,于是得到$m\subseteq P$.现在按照半局部性设全部极大理想是$m_1,m_2,\cdots,m_r$,于是$\cap_im_i\subseteq P$,这得到了某个$m_i\subseteq P$得到$P$是极大理想.
	\item 按照$R$是诺特环,可设$I=(x_1,x_2,\cdots,x_r)$.设$y_i$是$x_i$在$A_1'=I/I^2$中的像,那么有$R'=R_0'[y_1,y_2,\cdots,y_r]$,于是从希尔伯特基定理得到$R'$是诺特分次环.再按照$M$是有限$R$模,可设$M=(m_1,m_2,\cdots,m_s)$,那么有$M'=\sum R'm_i'$,其中$m_i'$是$m_i$在$M_0'=M/IM$中的像,于是$M'$是有限分次$R'$模.
	\item 另外按照$R'=R_0'[y_1,y_2,\cdots,y_r]$,并且这里$y_i$是$R'$中的一次元,于是此时$R'$可被有限个1次元在$R_0'$上生成,于是此时$d(M')$有意义.
\end{enumerate}

Samuel函数.我们定义上述$M$的Samuel函数$\chi^I_M:\mathbb{N}\to\mathbb{N}$为$n\mapsto l(M/I^{n+1}M)$,这里模的长度取为它作为$R_0'=R/I$模.
\begin{enumerate}
	\item Samuel函数是有意义的,换句话讲每个$M/I^{n+1}M$是有限长度$R/I$模,为此只需注意到阿廷环上有限生成模必然同时阿廷和诺特,于是它是有限长度模.
	\item 另外实际上有$l(M/I^{n+1}M)=\sum_{i=0}^nl(M_i')$.为此只要考虑一组短正合列:
	$$\xymatrix{0\ar[r]&I^iM/I^{i+1}M\ar[r]&M/I^{i+1}M\ar[r]&M/I^iM\ar[r]&0},1\le i\le n$$
	
	这得到等式$l(M_i')=l(I^iM/I^{i+1}M)=l(M/I^{i+1}M)-l(M/I^iM)$,对$1\le i\le n$求和就得到$l(M/I^{n+1}M)=\sum_{i=0}^nl(M_i')$.
	\item 设$M\not=0$,那么当$n$足够大时$\chi_M^I(n)$为一多项式,它的次数即$d(M')$,也即$M'$的Hilbert多项式的次数$+1$,它的首系数是$\frac{1}{d(M')}$.最后这里的$d(M')$与定义理想$I$的选取无关,换句话讲任取其它的定义理想不影响$d(M')$的取值.
	\begin{proof}
		
		记$M'$的Hilbert多项式为$a_nn^d+a_{n-1}n^{d-1}+\cdots+a_1n+a_0$,记希尔伯特函数为$\varphi(n)$,不妨设$n>N$的时候有上述二者相同,于是存在一个常数$C$,使得$n>N$的时候恒有$\chi_M^I(n)=\sum_{i=0}^n\varphi(n)=\sum_{i=0}^n(a_ii^d+\cdots+a_0)+C$,这个求和是一个$d+1$次多项式,并且首系数是$\frac{1}{d}$.
		
		取两个定义理想$I,J$,记$d_I$和$d_J$分别为它们定义出来的$d(M')$,也即当$n$足够大的时候Samuel函数作为的多项式的次数.按照$I$-adic和$J$-adic拓扑相同,可取正整数$a,b$使得$I^a\subseteq J$和$J^b\subseteq I$.于是可构造典范的满同态$M/(I^a)^{n+1}\to M/J^{n+1}$,这得到长度不等式$l(M/(I^a)^{n+1})\ge l(M/J^{n+1})$,此即$\chi_M^I(an+a-1)\ge\chi_M^J(n)$.但是当$n$足够大的时候不等式两侧都是关于$n$的正首系数多项式,要想这个不等式恒成立,必然有左侧的次数不小于右侧的次数,即$d_I\ge d_J$.同理构造满同态$M/(J^b)^{n+1}\to M/I^{n+1}$会得到$d_J\ge d_I$,这就得到了$d_I=d_J$.
	\end{proof}
	\item 记$R$是半局部诺特环,$M$为有限$R$模,今后就用$d(M)$表示$M$的Samuel函数在$n$足够大时成为的多项式的次数.假设有有限$R$模的短正合列$0\to M'\to M\to M''\to0$,那么有$d(M)=\max\{d(M'),d(M'')\}$,并且$\chi^I_M-\chi^I_{M''}$与$\chi_{M'}^I$具有相同的(正的)首系数.特别的,这说明对模取子模和取商会使得$d(-)$不增.
	\begin{proof}
		
		不妨约定$M'$为$M$的子模,且$M''=M/M'$.对定义理想$I$,按照同构基本定理有$M''/I^nM''=M/(M'+I^nM)$.考虑如下短正合列:
		$$\xymatrix{0\ar[r]&(M'+I^nM)/I^nM\ar[r]&M/I^nM\ar[r]&M/(M'+I^nM)\ar[r]&0}$$
		
		于是得到$l(M/I^nM)=l(M''/I^nM'')+l(M'/M'\cap I^nM)$.记最后一项是$\varphi(n)$,得到$\chi_M^I(n)-\chi_{M''}^I(n)=\varphi(n)$.于是当$n$足够大时$\varphi(n)$同样是一个多项式.
		
		下面按照Artin-Rees引理,存在正整数$c>0$使得$n\ge c$的时候恒有$I^nM'\subseteq M'\cap I^nM=I^{n-c}(I^cM\cap M')\subseteq I^{n-c}M'$,这诱导了两个满同态$M'/I^nM'\to M'/M'\cap I^nM\to M'/I^{n-c}M'$,这就得到了长度不等式$l(M'/I^nM')\ge l(M'/M'\cap I^nM)\ge l(M'/I^{n-c}M')$,于是当$n$足够大的时候有不等式$\chi_{M'}^I(n-c)\le\varphi(n)\le\chi_{M'}(n)$,这个不等式两侧是同次数同首系数的多项式,中间项在$n$足够大的时候也是不等式,这就迫使中间项作为不等式的时候次数也与两侧相同,首系数也与两侧相同.于是$\chi_M^I(n)$的次数是$\chi_{M''}^I(n)$次数和$\chi_{M'}^I(n)$次数的最大值,并且$\chi^I_M-\chi^I_{M''}$与$\chi_{M'}^I$具有相同的(正的)首系数.
	\end{proof}
\end{enumerate}
\subsubsection{基本定理}

$\delta(M)$.设$R$是半局部环,$m$是Jacobson根,$M$是有限$R$模.定义$\delta(M)=\inf\{n\in\mathbb{N}\mid\exists x_1,x_2,\cdots,x_n\in m,l_R(M/(x_1,x_2,\cdots,x_n)M)<+\infty\}$.
\begin{enumerate}
	\item $\delta(M)=0$等价于讲$M$本身是有限长度模.
	\item 取$I$为定义理想,那么我们证明过$M/IM$作为$R/I$模是有限长度的,下一条会解释这等价于$M/IM$作为$R$模是有限长度的,于是$\delta(M)$的值必然不超过定义理想$I$的生成元集的元素个数.
	\item 这一条是一个注解.给定$R$模$M$,取理想$I\subset\mathrm{Ann}(M)$,那么$M$可同时作为$R$模和作为$R/I$模,并且相应的子模作为集合是完全相同的,这说明关于子模链的所有条件二者都是一致的.特别的,$l_R(M)=l_{R/I}(M)$.
	\item 现在设$R$是局部诺特环,取理想$I\subseteq R$,那么$R/I$是有限长度$R$模等价于讲$R/I$是阿廷环,这等价于讲包含$I$的唯一素理想是$R$的唯一极大理想$m$,这等价于$\sqrt{I}=m$,等价于$I$是$m$准素理想.这说明对于局部环$R$,有$\delta(R)$表示所有能作为$m$准素理想的生成元集的那些集合的元素个数最小元.
\end{enumerate}

维数论基本定理.给定半局部诺特环$R$,设$M$是有限$R$模,非零模,那么有$\dim M=d(M)=\delta(M)$.这里$\dim M$定义为环$R/\mathrm{Ann}(M)$的维数,此即$\mathrm{Spec}(R)$的闭子集$V(\mathrm{Ann}(M))=\mathrm{Supp}(M)$的组合维数(即不可约闭子集降链长度的上确界).
\begin{proof}
	
	我们的思路是依次归纳的证明不等式链$\dim(M)\le d(M)\le\delta(M)\le\dim(M)$.
	
	第一步,证明对半局部诺特环$R$总有$\dim(R)\le d(R)$.首先$d(R)=0$时,按照定义$\chi_R^m(n)$在$n$足够大时是常值函数,故当$n$足够大时$m^n/m^{n+1}=0$,那么由NAK引理得到$m^n=0$,于是$R$中素理想都是极大理想,于是$\dim(R)=0$.现在假设$d(R)>0$,不妨设$\dim(R)>0$,否则结论已经成立.任取素理想严格升链$P_0\subseteq P_1\subset\cdots\subseteq P_e$,只要验证$e\le d(R)$.取一个元$x\in P_1-P_0$,考虑$R/P_0$上的模同态为数乘$x$,这个同态的余核记作$B$,于是得到短正合列$0\to R/P_0\to R/P_0\to B\to0$.于是得到$d(R/P_0)=\max\{d(R/P_0),d(B)\}$,并且$\chi_{R/P_0}^m-\chi_B^m$和$\chi_{R/P_0}^m$具有相同的次数和首系数.这两个结果说明了必然有$d(B)<d(R/P_0)$,否则如果$d(B)=d(R/P_0)$导致$\chi_{R/P_0}^m-\chi_B^m$和$\chi_{R/P_0}^m$的首系数必然不相同;如果$d(B)>d(R/P_0)$导致$\chi_{R/P_0}^m-\chi_B^m$的次数高于$\chi_{R/P_0}^m$的次数.另外我们说明过取商使得$d(-)$不增,综上得到不等式$d(B)<d(R/P_0)\le d(R)$.现在$P_1/P_0\subseteq P_2/P_0\subset\cdots\subseteq P_e/P_0$是$B$中的素理想严格升链,于是按照归纳假设得到$e-1\le d(B)$,于是$e-1\le d(R)-1$得到$e\le d(R)$,于是取上确界得到$\dim(R)\le d(R)$完成归纳.
	
	第二步,证明条件下恒有$\dim(M)\le d(M)$.回顾一点知识:如果$R$是诺特环,$M$是有限$R$模,那么可构造链$0=M_0\subseteq M_1\subset\cdots\subseteq M_n=M$,使得对每个$1\le i\le n$有$M_i/M_{i-1}\cong R/p_i$,其中全体$\{p_i\}$就是$R$模$M$的全体伴随素理想构成的集合$\mathrm{Ass}(M)$.另外对于有限$R$模$M$,有$\mathrm{Supp}(M)$和$\mathrm{Ass}(M)$的极小素理想是一致的,如果记这些一致的极小素理想为$\{q_1,q_2,\cdots,q_u\}$,那么$\mathrm{Supp}(M)=\cup_{1\le i\le u}V(p_i)$,这里$V(p_i)$就是$\mathrm{Supp}(M)$的不可约分支.而不可约分支总是闭集,一个空间的组合维数是不可约闭集降链长度的上确界,于是空间的组合维数恰好就是全部不可约闭集上组合维数的上确界.因而我们有$\dim(M)=\max_{1\le i\le u}(\dim R/p_i)$.现在从短正合列族$0\to M_i\to M_{i+1}\to M_{i+1}/M_i\to0$,得到$d(M)=\max\{d(M_{n-1}),d(M_n/M_{n-1})\}$,而这里$d(M_n/M_{n-1})=d(R/p_n)$,再反复带入短正合列得到的$d(-)$的公式,得到$d(M)=\max_{1\le i\le u}(d(R/p_i))$.最后从第一步得到每个$d(R/p_i)\ge\dim(R/p_i)$,于是得到$d(M)\ge\dim(M)$.
	
	第三步,证明条件下恒有$d(M)\le\delta(M)$.首先$\delta(M)=0$等价于讲$M$是有限长度$R$模.我们知道取商的时候模的长度不增,这说明$l(M/I^nM)$有上界$l(M)$,而当$n$足够大的时候$l(M/I^nM)$是多项式,有界的多项式自然是常值多项式,这就导致$d(M)=0$.现在假设$\delta(M)=s>0$,按照定义可取$x_1,x_2,\cdots,x_s\in m$,使得$l(M/(x_1,x_2,\cdots,x_s)M)<+\infty$.设$M_i=M/(x_1,x_2,\cdots,x_i)M,1\le i\le s$,我们断言$\delta(M_i)=s-i$,若否有某个$\delta(M_i)\not=s-i$,按照$M_i/(x_{i+1},\cdots,x_s)$是有限长度$R$模,说明$\delta(M_i)\le s-i$,假设有$t<s-i$个元$y_1,y_2,\cdots,y_t\in m$使得$M_i/(y_1,y_2,\cdots,y_t)$是有限长度$R$模,那么有$M/(y_1,y_2,\cdots,y_t,x_1,\cdots,x_i)$是有限长度$R$模,导致$\delta(M)\le t+i<s$,这就和$\delta(M)=s$矛盾,于是有$\delta(M_i)=s-i$.现在按照归纳假设,有$d(M_1)\le\delta(M_1)=s-1$.现在考虑如下短正合列:
	$$\xymatrix{0\ar[r]&\frac{x_1M+m^nM}{m^nM}\ar[r]&\frac{M}{m^nM}\ar[r]&\frac{M}{x_1M+m^nM}=\frac{M_1}{m^nM_1}\ar[r]&0}$$
	
	它说明了有模的长度的等式$l(M_1/m^nM_1)=l(M/m^nM)-l(x_1M+m^nM/m^nM)$.现在有满同态$M/m^{n-1}M\to x_1M/(x_1M\cap m^nM)$为左乘$x_1$,于是得到长度公式$l(M/m^{n-1}M)\ge l(x_1M/(x_1M\cap m^nM))$.带入上面等式得到$l(M_1/m^nM_1)\ge l(M/m^nM)-l(M/m^{n-1}M)$,按照$n$足够大时这三个长度都是多项式,于是得到当$n$足够大的时候有$\chi_{M_1}(n)\ge\chi_M(n)-\chi_M(n-1)$.这里右侧是一个$d(M)-1$次多项式,于是得到$d(M_1)\ge d(M)-1$,于是$d(M)-1\le s-1$得到$d(M)\le s$,完成归纳.
	
	第四步,证明$\delta(M)\le\dim(M)$.注意这里$\dim(M)$的有限性从第一步得到的:按照定义有$\dim(M)\le\dim(R)$,第一步得到$\dim(R)\le d(R)$,这里$d(R)$是一个多项式的次数必然是有限数.首先$\dim(M)=0$得到$R/\mathrm{Ann}(M)$是零维环,于是包含$\mathrm{Ann}(M)$的$R$的素理想都是极大理想,于是$\sqrt{\mathrm{Ann}(M)}$是某些极大理想的交,于是Jacobson根$m$包含于$\sqrt{\mathrm{Ann}(M)}$,按照诺特条件,可找到正整数$n$使得$m^n\subset\mathrm{Ann}(M)$,于是$m^nM=0$,于是$l(M/m^nM)=l(M)<+\infty$,于是$\delta(M)=0$.下设$\dim(M)>0$,取$\mathrm{Ann}(M)$的全部极小素理想为$\{p_1,p_2,\cdots,p_t\}$,这里极小素理想的有限性是由诺特条件保证的.那么$V(p_1),V(p_2),\cdots,V(p_t)$是$\mathrm{Supp}(M)$的全部不可约分支,因而有$\dim(M)$就是某些$\dim(R/p_i)$.(不可约分支就是极大的不可约子空间,它总是闭集,于是空间$\mathrm{Supp}(M)$的组合维数必然是它全部不可约分支的组合维数中的最大值).设这些$i$的指标构成集合$S$,于是从假设的$\dim(M)>0$得到$P_i,i\in S$不会是极大理想,于是它们也不会包含Jacobson根$m$.现在取$x\in m-\cup_{i\in S}p_i$,记$M_1=M/x_1M$,那么有$\mathrm{Supp}(M_1)\subset\mathrm{Supp}(M)$,并且二者都是闭集,于是组合维数前者不超过后者,于是$\dim(M_1)\le\dim(M)$.这里取等会产生矛盾:如果取等,设这个相等的维数是$r$,于是可取$R$中的包含了$\mathrm{Ann}(M)+(x)$的素理想升链$Q_0\subseteq Q_2\subset\cdots\subseteq Q_r$,但是任意补上一个$p_i$会导致$R/\mathrm{Ann}(M)$中存在长度为$r+1$的素理想链,这和$\dim M=r$矛盾.最后归纳假设得到$\delta(M)-1\le\delta(M_1)\le\dim(M_1)\le\dim(M)-1$得到$\delta(M)\le\dim(M)$,完成归纳.
\end{proof}

一些推论.
\begin{enumerate}
	\item 设$R$是诺特环,设$I=(a_1,a_2,\cdots,a_r)\subseteq R$是理想,设$p$是$I$的极小素理想(即包含着$I$的素理想中的极小元),那么有$\mathrm{ht}(p)\le r$.这个定理存在逆命题,见第三条.特别的,取$r=1$得到Krull主理想定理:对$a\in R$,包含$a$的素理想的极小元的高度$\le1$.
	\begin{proof}
		
		考虑$R_p$的理想$IR_p$,包含它的唯一的素理想是$R_p$的唯一的极大理想$pR_p$,于是$IR_p$是$pR_p$准素理想.按照$R_p$是局部环,得到$\delta(R_p)$不超过$IR_p$生成元集的元素个数,也就是$\delta(R_p)\le r$,但是按照基本定理得到$\mathrm{ht}(p)=\dim(R_p)=\delta(R_p)\le r$,完成证明.
	\end{proof}
	\item 我们不借助维数论基本定理证明如下基本结论.
	\begin{enumerate}
		\item Krull主理想定理.设$A$是诺特环,设$f\in A$是非可逆元,那么$fA$的极小素理想$\mathfrak{p}$都满足$\mathrm{ht}(\mathfrak{p})\le1$.
		\begin{proof}
			
			对$\mathfrak{p}$做局部化,我们不妨设$(A,\mathfrak{p})$是诺特局部环.倘若$f$包含在$A$的每个极小素理想中,那么$\mathrm{\mathfrak{p}}=0$满足结论.下面设$f$至少不包含在一个$A$的极小素理想$\mathfrak{q}$中,让$A$商掉这个极小素理想,不妨设$(A,\mathfrak{p})$是诺特局部整环.下面设$\mathfrak{q}\subseteq\mathfrak{p}$是素理想链,我们需要证明$\mathfrak{q}=0$.
			
			\qquad
			
			由于$A/fA$的唯一素理想是$\mathfrak{p}/fA$,于是$A/fA$是阿廷环,考虑$\mathfrak{q}_n=\mathfrak{q}^nA_f\cap A$,它在$A/fA$中的像要稳定,于是存在正整数$n_0$,当$n\ge n_0$时就有$\mathfrak{q}_n\subseteq\mathfrak{q}_{n+1}+fA$.下面设$n\ge n_0$,任取$x\in\mathfrak{q}_n$,那么存在$y\in A$使得$x-fy\in\mathfrak{q}_{n+1}\subseteq\mathfrak{q}_n$.于是$fy\in\mathfrak{q}_n$,由于$f^r\not\in\mathfrak{q},\forall r\ge1$,从$fy\in\mathfrak{q}_n$得到$y\in\mathfrak{q}_n$.于是从$\mathfrak{q}_n\subseteq\mathfrak{q}_{n+1}+f\mathfrak{q}_n\subseteq\mathfrak{q}_{n+1}+\mathfrak{p}\mathfrak{q}_n$.那么从NAK引理得到$\mathfrak{q}_n=\mathfrak{q}_{n+1}$.按照$\mathfrak{q}_nA_f=\mathfrak{q}^nA_f$,得到$\mathfrak{q}^{n_0}A_f=\cap_{n\ge n_0}\mathfrak{q}^nA_f$.按照Artin-Rees引理,如果$A$是诺特环,$I$是理想,$M$是有限$A$模,那么$\cap{n\ge0}I^nM=\{x\in M\mid\exists\alpha\in I,(1+\alpha)x=0\}$,这里如果$a/f^s\in A_f$使得存在$\alpha\in\mathfrak{q}$使得$(1+\alpha)a/f^s=0$,但是$1+\alpha$是单位,导致$a/f^s=0$,于是这里$\mathfrak{q}^{n_0}A_f=\cap_{n\ge n_0}\mathfrak{q}^nA_f=0$,这导致$\mathfrak{q}=0$.
		\end{proof}
	    \item 设$A$是诺特环,设$f\in A$,如果$\mathfrak{p}_0\subsetneqq\cdots\subsetneqq\mathfrak{p}_n$是$A$的素理想链,其中$n\ge1$,满足$f\in\mathfrak{p}_n$,那么存在$A$的素理想链$\mathfrak{q}_1\subsetneqq\cdots\subsetneqq\mathfrak{q}_n$,使得$\mathfrak{p}_n=\mathfrak{q}_n$,并且$f\in\mathfrak{q}_1$(这里不是$\mathfrak{q}_0$).
	    \begin{proof}
	    	
	    	归结为证明$n=2$的情况.假设$f\not\in\mathfrak{p}_1$,那么$f\not\in\mathfrak{p}_0$,设$\mathfrak{q}_1$是$\mathfrak{p}_0+fA$的包含在$\mathfrak{p}_2$中的极小素理想,那么$\mathfrak{p}_0\subsetneqq\mathfrak{q}_1$,倘若$\mathfrak{q}_1=\mathfrak{p}_2$,就导致$\mathfrak{q}_1$在$A/\mathfrak{p}_0$中的高度至少是2,和上一条结论矛盾,于是$\mathfrak{p}_1$满足要求.
	    \end{proof}
        \item 设$A$是诺特环,$I$是被$r$个元生成的真理想,如果$\mathfrak{p}$是$I$的极小素理想,那么$\mathrm{ht}(I)\le r$.
        \begin{proof}
        	
        	我们来对$r$归纳,$r=1$我们已经证明过了.下面设$r\ge2$,设$f_1,\cdots,f_r$生成了理想$I$,那么$I/f_rA$被$A/f_rA$中$r-1$个元生成,并且$\mathfrak{p}$在$A/f_rA$中的像是$I/f_rA$的极小素理想,按照归纳假设就有$\mathrm{ht}(\mathfrak{p}/f_rA)\le r-1$.设$A$的素理想链$\mathfrak{p}_0\subsetneqq\cdots\subsetneqq\mathfrak{p}_n$满足$\mathfrak{p}=\mathfrak{p}_n$.那么上一条说明存在素理想链$\mathfrak{q}_1\subsetneqq\cdots\subsetneqq\mathfrak{q}_n$使得$f_r\in\mathfrak{q}_1$,这个素理想链在$A/f_rA$中的像也是素理想链,于是$n-1\le\mathrm{ht}(\mathfrak{p}/f_rA)\le r-1$,于是得到$n\le r$,左侧取上确界得到$\mathrm{ht}(\mathfrak{p})\le r$.
        \end{proof}
        \item 另外如果$(A,\mathfrak{m},k)$是诺特局部环,那么$\dim A\le\dim_k\mathfrak{m}/\mathfrak{m}^2$,特别的诺特局部环的维数都是有限的.
        \begin{proof}
        	
        	如果记$e=\dim_k\mathfrak{m}/\mathfrak{m}^2$,按照NAK引理,有$\mathfrak{p}$可被$e$个元生成,上一条就得到$\dim A=\mathrm{ht}(\mathfrak{m})\le e$.
        \end{proof}
        \item 设$(A,\mathfrak{m})$是诺特局部环,设$f\in\mathfrak{m}$,那么有$\dim(A/fA)\ge\dim A-1$,并且如果$f$不在$A$的每个极小素理想中,那么这个不等式取等号.
        \begin{proof}
        	
        	先设$\dim A=n$,那么存在素理想链$\mathfrak{p}_0\subsetneqq\cdots\subsetneqq\mathfrak{p}_n=\mathfrak{m}$,并且有$f\in\mathfrak{p}_n$.于是存在$A$的素理想链$\mathfrak{q}_1\subsetneqq\cdots\subsetneqq\mathfrak{q}_n$,使得$\mathfrak{p}_n=\mathfrak{q}_n$,并且$f\in\mathfrak{q}_1$.于是$\dim A/fA\ge n-1$.
        	
        	\qquad
        	
        	下面设$f$不落在$A$的每个极小素理想中,任取$f$的极小素理想$\mathfrak{p}$,主理想定理说明$\mathrm{ht}(\mathfrak{p})\le1$,但是由于$\mathfrak{p}$不是极小素理想,导致$\mathrm{ht}(\mathfrak{p})\not=0$,于是只能有$\mathfrak{p}$的高度是1.于是有$\dim A/fA+1\le\dim A$.这得到等式.
        \end{proof}
	\end{enumerate}
	\item 设$R$是诺特环,取高度$r$的素理想$p$,则:
	\begin{enumerate}
		\item 可取由$r$个元生成的理想$I$,使得$p$是$I$的极小素理想.
		\begin{proof}
			
			按照$p$的高度$r$,有$R_p$的维数$r$,于是基本定理说明$\delta(R_p)=r$,于是可取$a_1,a_2,\cdots,a_r\in pR_p$使得$l(R_p/(a_1,a_2,\cdots,a_r))<\infty$.记$a_i=b_i/s_i$,其中$b_i\in R$和$s_i\in R-p$,取$I=(b_1,b_2,\cdots,b_r)$为一个真理想,那么有$\{a_1,a_2,\cdots,a_r\}$在$R_p$中生成的理想恰好就是$IR_p$.现在$R_p/IR_p$长度有限,于是$pR_p$是$R_p$中唯一包含了$IR_p$的素理想,按照分式化的理想对应定理,得到在$R$中有$p$是$I$的极小素理想.
		\end{proof}
		\item 若$b_1,b_2,\cdots,b_s\in p$,那么有$\mathrm{ht}(P/(b_1,b_2,\cdots,b_s))\ge r-s$.
		\begin{proof}
			
			记$\overline{R}=R/(b_1,b_2,\cdots,b_s)$,记$\overline{p}=p/(b_1,b_2,\cdots,b_s)$,它是$\overline{R}$中的素理想,记这个素理想的高度是$t$,需要证明的是$t\ge r-s$.现在按照上一条可取$c_1,c_2,\cdots,c_t\in p$,使得$\overline{p}$是理想$(\overline{c_1},\overline{c_2},\cdots,\overline{c_t})$的极小素理想,这里$\overline{c_i}=c_i+(b_1,b_2,\cdots,b_s)\in p/(b_1,b_2,\cdots,b_s)$.于是$p$是$(b_1,\cdots,b_s,c_1,\cdots,c_t)$的极小素理想(如果$p/J$是$I/J$的极小素理想,那么$p$是$I+J$的极小素理想),于是$r=\mathrm{ht}(p)\le t+s$.
		\end{proof}
		\item 如果$a_1,a_2,\cdots,a_r$使得它们生成的理想以$p$为极小素理想,那么对每个$1\le i\le r$有$\mathrm{ht}(P/(a_1,a_2,\cdots,a_i))=r-i$.
		\begin{proof}
			
			从上一条知$\mathrm{ht}(P/(a_1,a_2,\cdots,a_i))\ge r-i$.另一方面按照理想对应定理,有$p/(a_1,a_2,\cdots,a_i)$是$I/(a_1,a_2,\cdots,a_i)$的极小素理想,而后者可由$r-s$个元生成,因而主理想定理得到$\mathrm{ht}(P/(a_1,a_2,\cdots,a_i))\le r-s$.
		\end{proof}
	\end{enumerate}
\end{enumerate}

域上分次代数的维数.
\begin{enumerate}
	\item 引理.设$A=\oplus_{n\ge0}A_n$是诺特分次环.
	\begin{enumerate}
		\item 如果$I\subseteq A$是齐次理想,设$p\subseteq A$是$I$的素除子(prime divisor,此即$A$模$A/I$的伴随素理想),那么$p$也是齐次的.特别的,诺特分次环的极小素理想总是齐次的,因为极小素理想是零理想的伴随素理想.
		\begin{proof}
			
			按照定义存在$A$分次环$A/I$中的元$x$使得$p=\mathrm{Ann}(x)$.任取$a\in p$,记齐次分解$a=a_p+a_{p+1}+\cdots+a_s$和$x=x_0+\cdots+x_l$.从$ax=0$得到$a_px_0=0$,$a_{p+1}x_0+a_px_1=0$,$a_{p+2}x_0+a_{p+1}x_1+a_px_2=0$等.归纳得到$a_p^{i+1}x_i=0,0\le i\le l$,于是$a_p^{l+1}x=0$,于是$a_p^{l+1}\in p$,按照$p$是素理想有$a_p\in p$,于是$a_{p+1}+\cdots+a_s\in p$,对下指标归纳得到每个$a_i\in p$,这说明$p$是齐次的.
		\end{proof}
	    \item 如果$p\subseteq A$是高度$r$的齐次素理想,那么存在长度$r$的齐次素理想链$p=p_0\supsetneqq p_1\supsetneqq\cdots\supsetneqq p_r$.
	    \begin{proof}
	    	
	    	先取$p$为终端的长度$r$的未必齐次的素理想链$p=q_0\supsetneqq q_1\supsetneqq\cdots\supsetneqq q_r$.那么$q_r$一定是极小素理想,所以按照上一条有$q_r$一定是齐次的.于是不妨可设$A$是整环,否则可以用$A/q_r$替换$A$.
	    	
	    	\qquad
	    	
	    	下面对$r=\mathrm{ht}(p)$做归纳.如果$r=0$有$p=(0)$必然是齐次的,下设$r>0$.取非零齐次元$b_1\in p$,我们解释过有$\mathrm{ht}(p/(b_1))\ge r-1$.假设这是严格大于的,那么$p/(b_1)$要包含长度$\ge r$的素理想链,再加上零理想这个素理想就导致$\mathrm{ht}(p)\ge r+1$矛盾.所以有$\mathrm{ht}(p/(b_1))=p-1$.取它的长度为$p-1$的素理想链,记这个链中最小的是$q/(b_1)$,那么它一定是$A/(b_1)$的极小素理想,于是按照上一条有$q/(b_1)$是齐次的,于是原像$q$也是齐次的,于是$p/q$是$A/q$的长度为$r-1$的素理想.按照归纳假设,有长度为$r-1$的齐次理想链$p=p_1\supsetneqq\cdots\supsetneqq p_{r-1}=q$.再添加零理想得到以$p$为终端的长度$r$的齐次素理想链.
	    \end{proof}
	\end{enumerate}
    \item 设$k$为域,记$R=k[\xi_1,\xi_2,\cdots,\xi_r]$是有限生成分次$k$代数,其中$\xi_i$都是1次元(这导致了,比方说,$\xi_i$中不含$k$中的元).记$m=(\xi_1,\xi_2,\cdots,\xi_r)$是$R$的极大理想.记$A=R_m$,记$n=mR_m$.我们有如下断言:
    \begin{enumerate}
    	\item 记$\chi$为局部环$A$的Samuel函数,记$\varphi$为$R$的Hilbert函数,那么有$\varphi(i)=\chi(i)-\chi(i-1)$.特别的,这说明$\deg(\varphi)=\deg(\chi)-1=\dim(A)=\mathrm{ht}(m)-1$.
    	\begin{proof}
    		
    		按照定义$\chi(i)=l(A/n^{i+1})$,$\varphi(i)=l(m^i/m^{i+1})$.另外有$\chi(i)-\chi(i-1)=l(n^i/n^{i+1})$.于是问题归结为一个初等代数问题:证明典范映射$n^i/n^{i+1}\to m^i/m^{i+1}$是同构.单射只需验证$A\cap m^{i+1}A=A\cap n^{i+1}=m^{i+1}$.一方面有$m^{i+1}\subseteq A\cap n^{i+1}$.另一方面任取$x=a/s\in A\cap n^{i+1}$,其中$a\in m^{i+1}$和$s\in A-m$.那么有$t\in A-m$使得$t(xs-a)=0\in R$.单射$m$是极大理想导致$m^{i+1}+(ts)=R$,于是可取$a'\in m^{i+1}$和$b\in R$使得$a'+bst=1$,导致$x=xa'+xbst=xa'+bta\in m^{i+1}$,于是$A\cap n^{i+1}\subseteq m^{i+1}$.
    		
    		\qquad
    		
    		再证明满射.任取$x=a/s\in n^i=m^iA$,其中$a\in m^i$和$s\in A-m$,按照$m\subseteq R$是极大理想,有$m^n+(s)=R$,于是可取$y\in m^i$和$b\in R$使得$y+bs=1$.于是$x=a/s=a(y+bs)/s=ay/s+ba\in m^i+n^{i+1}$.这得到满射.
    	\end{proof}
    	\item $\dim R=\mathrm{ht}(m)=\dim A=\deg\varphi+1$.
    	\begin{proof}
    		
    		按照$m$是$A$的唯一极大理想,说明$\dim A=\mathrm{ht}(m)$.另外维数论基本定理说明$\dim A$恰好就是$\chi(i)$的次数,从上一条就得到$\dim A=\deg\varphi+1$.于是唯一需要验证的是$\dim R=\mathrm{ht}(m)$.
    		
    		先设$R$是整环,我们之前给出过域上有限生成代数并且是整环的条件下,它的维数就是$k\subset\mathrm{Frac}(R)$的超越次数.并且超越基可在$\{\xi_1,\xi_2,\cdots,\xi_r\}$中选取,把这个次数记作$t$,那么按照代数无关集生成的同次数的不同单项式总是线性无关的,得到模的长度公式$l(R_n)\ge\left(\begin{array}{c}t+n-1\\n-1\end{array}\right)$,这导致$\deg\varphi\ge t-1$.于是有$\mathrm{ht}(m)=1+\deg\varphi\ge\mathrm{tr.deg}_k(R)=\dim(R)\ge\mathrm{ht}(m)$,完成证明.
    		
    		现在设$R$是一般的交换环.记$R$的所有极小素理想为$\{p_1,p_2,\cdots,p_t\}$,那么有$\dim R=\max_{1\le i\le t}\{\dim R/p_i\}$.不妨记$\dim R=\dim R/p_1$,这就能够把问题约化为整环$R/p_1$的情况.但是在整环的情况下我们是借助了$R$的分次环结构给出的证明,于是这里我们需要验证下$p_1$是齐次理想,进而$R/p_1$是分次环,我们在下文会证明齐次理想的伴随素理想总是齐次的,于是特别的极小素理想也是齐次的.最后我们断言$p_1\subseteq m$,否则有$x\in p_1-m$,记$x$的齐次分解为$x_0+x_1+\cdots$,那么$x$不在$m$中等价于讲$x_0\not=0$,但是按照$p_1$是齐次理想得到$x_0\in p_1$,但是$x_0$作为$k$中非零元在$R$中可逆,这和$p_1$是素理想矛盾.于是我们得到$\dim R=\dim(R/p_1)=\mathrm{ht}(m/p_1)\le\mathrm{ht}(m)\le\dim R$.完成证明
    	\end{proof}
    	\item 作为分次环有$\mathrm{gr}^n(A)\cong R$.这一条只需验验证$R_t=m^t/m^{t+1}\to\mathrm{gr}_t^n(A)=n^t/n^{t+1}$是同构的.但是我们已经证明了$R_i\cong m^i/m^{i+1}\cong n^i/n^{i+1},\forall i$.
    \end{enumerate}
    \item 推论.设$(A,m)$是诺特局部环,那么$\dim A=\dim(\mathrm{gr}^mA)$.另外这个命题实际上把$m$改为任意真理想都是成立的,我们会在后文给出证明.
    \begin{proof}
    	
    	记$R=\mathrm{gr}^m(A)$,记$R$的Hilbert多项式为$\varphi$,基本定理得到$\dim A=\deg\varphi+1$.但是上一个定理有$\dim R=\deg(\varphi)+1$.于是$\dim A=\dim R$.
    \end{proof}
\end{enumerate}
\newpage
\subsection{参数系统和重数}

参数系统.记$(A,m)$是诺特局部环,$M$是维数$s$的有限$A$模.那么按照维数的一个定义,存在$x_1,x_2,\cdots,x_s\in m$,使得$M/(x_1,x_2,\cdots,x_s)M$是有限长度模.按照定义集合$\{x_1,x_2,\cdots,x_s\}$中没有相同的元,我们就称它是$M$的一组参数系统.对于环的情况,我们之前证明过对于$d$维诺特局部环$(A,m)$,必然存在一个$m$准素理想被$r$个元生成,并且不能被更小个数的元生成.这样的生成某个$m$准素理想的$r$元集合就称为$(A,m)$的一个参数系统.这里环上的定义是模上定义的特例,因为$l(A/(x_1,x_2,\cdots,x_d))$有限长度等价于$A/I$是阿廷环,这里$I=(x_1,x_2,\cdots,x_d)$.导致$A/I$是局部阿廷环,于是它的零理想的根理想就是极大理想,也即$I$是$m$准素理想.

设$(A,m)$是诺特局部环,设$\{x_1,x_2,\cdots,x_s\}$是$A$的参数系统.
\begin{enumerate}
	\item $\dim(A/(x_1,x_2,\cdots,x_i))=r-i$,并且有$\mathrm{ht}(x_1,x_2,\cdots,x_i)\le i,\forall 1\le i\le s$.这个结论我们在上一节已经证明过了.这里我们解释下一般理想的高度的定义:给定环$R$的理想$I$,它的高度约定为它全部极小素理想的高度的最小值.这也是$V(I)$的组合维数.
	\item 存在$(A,m)$的一组参数系统$\{y_1,y_2,\cdots,y_s\}$,使得任取子集$F\subset\{y_1,y_2,\cdots,y_s\}$都有$F$生成的理想的高度恰好就是$|F|$.
	\begin{proof}
		
		首先对于$s=0$,此时$A$是域,空集是满足条件的参数系统.现在假设$r=1$,那么存在$y_1\in m$,使得$A/(y_1)$是有限长度的,于是它是阿廷环,于是包含$(y_1)$的素理想必然是极大理想,按照$A$是局部环,说明包含$y_1$的素理想只有$m$.现在对于$F=\emptyset$,有$F$生成了零理想,而按照极小素理想的定义,包含$\{0\}$的极小素理想的高度必然是0;再对于$F=\{y_1\}$,此时我们已经说明了包含$y_1$的素理想只有$m$,此时它的高度不能为0,否则此时$A$是域,维数是0,于是$\mathrm{ht}(m)=1$.
		
		现在设$r=2$,对于更一般的$r$做法是类似的.先取$A$的全部极小素理想为$\{p_{0,1},p_{0,2},\cdots,p_{0,e_0}\}$,这里不会包含极大理想,否则此时维数是零和约定的$r=2$矛盾.按照prime avoidance引理从$m\not\subseteq p_{0,i}$就得到$m\not\subset\cup_{1\le i\le e_0}p_{0,i}$.取$y_1\in m-\cup_{1\le i\le e_0}p_{0,i}$,按照主理想定理有$\mathrm{ht}(y_1)\le1$,如果它是零那么理应存在一个$A$的极小素理想包含$y_1$和它的定义矛盾.于是$\mathrm{ht}(y_1)=1$.再取$\{p_{1,1},p_{1,2},\cdots,p_{1,e_1}\}$是$(y_1)$的全部极小素理想,那么主理想定理同样得到$\mathrm{ht}(p_{1,j})\le1$,如果取零,那么$y_1$将会落在某个极小素理想中同样矛盾,于是得到$\mathrm{ht}(p_{1,j})=1$.现在有$m\not=p_{1,j}$否则$\dim R=\mathrm{ht}(m)=1$和$r=2$矛盾.于是按照prime avoidance引理,从$m\not\subseteq p_{1,j}$得到$m\not\subset\cup_{i=0,1}\cup_{1\le j\le e_i}p_{i,j}$.取$y_2$在$m$扣去这个并集中.那么按照主理想定理有$\mathrm{ht}(y_2)\le1$,如果取零则$y_2$包含于某个$A$的极小素理想中同样矛盾.于是$\mathrm{ht}(y_2)=1$.最后我们验证$\mathrm{ht}(y_1,y_2)=2$.首先任取$(y_1,y_2)$的极小素理想$p$,则从$y_1\in p$得到$p\not=p_{1,j}$,于是按照$p_{1,j}$是$y_1$的极小素理想,说明$p$真包含了某个$p_{1,j}$,于是$\mathrm{ht}(p)\ge\mathrm{ht}(p_{1,j})+1\ge2$.反过来按照主理想定理有$\mathrm{ht}(y_1,y_2)\le2$,这就得到$\mathrm{ht}(y_1,y_2)=2$完成证明.
	\end{proof}
    \item 一般的,对于局部环$A$,可能存在一组参数系数$\{x_1,x_2,\cdots,x_r\}$,使得存在某个$i$使得$\mathrm{ht}(x_1,x_2,\cdots,x_i)<i$(主理想定理说明理想的高度不超过生成元个数).例如取$R=k[[X,Y,Z]]$,取$I=(X)\cap(Y,Z)$,取$A=R/I$,用形式幂级数环是因为此时$A$是局部环.记$X,Y,Z$在$A$中的像是$x,y,z$.那么$(x)$和$(y,z)$是$A$的极小素理想,并且有$A/(x)\cong k[[Y,Z]]$和$A/(y,z)\cong k[[X]]$.于是$\dim A=2$.我们断言$\{y,x+z\}$是$A$的参数系统.一旦这成立,就有$\mathrm{ht}(y)=0<1$,得到反例.为此只需证明$x,y,z$在$A/(y,x+z)$中的像是$k$上的整元,这就导致$A/(y,x+z)$是有限长度的.而这是因为在商环中有$x^2=x(x+z)\equiv0$和$z^2=z(x+z)\equiv0$.
\end{enumerate}

正则局部环.设$(A,m,k)$是诺特局部环,按照主理想定理,$m$不能被小于$\dim A$个元素生成,如果$m$恰好可以被$\dim A$个元素生成,就称它是正则局部环.正则局部环上的参数系统称为正则参数系统.
\begin{enumerate}
	\item 这一条仅仅是一个强调,当我们提及正则局部环的时候总假定诺特条件.
	\item 一个等价定义.给定诺特局部环$(A,m,k)$,它的嵌入维数定义为$m/m^2$作为$k=A/m$线性空间的维数.按照NAK引理,一族元可以生成$m$当且仅当这些元素可以生成$R/m$线性空间$m/m^2$.于是嵌入维数是一个能够生成$m$的元素个数,导致嵌入维数总是大于等于$\dim A$.当这个不等式取等的时候,等价于讲恰好$\dim A$个元素可以生成$m$.于是诺特局部环$A$是正则局部环等价于讲它的维数和嵌入维数相同.
	\item 一个例子.记$k$是域,取多元形式幂级数环$R=k[[X_1,X_2,\cdots,X_n]]$,那么它的可逆元恰好是常数项非零的元,于是它是局部环,唯一的极大理想恰好是$m=(X_1,X_2,\cdots,X_n)$.那么此时嵌入维数恰好是$n$,因为$X_i+m^2$是$m/m^2$的一组基.而一方面按照存在长度$n$的素理想升链,得到$\dim R\ge n$,另一方面主理想定理得到$\dim R=\mathrm{ht}(m)\le n$,于是$\dim R=n$,于是$R$是一个正则局部环.
	\item 1维正则局部环等价于DVR.因为我们解释过DVR等价于维数$>0$的诺特局部环,并且极大理想是主理想.
\end{enumerate}

设$(A,m)$是$n$维正则局部环,取$x_1,x_2,\cdots,x_i\in m$,那么如下三个条件等价.特别的,正则参数系统恰好是维数个$m$中的元构成的集合,使得它们的像构成了$m/m^2$中的一组基,或者等价的讲,使得它们生成了整个$m$.特别的,这个结论说明如果$(A,\mathfrak{m})$是正则局部环,如果$f\in\mathfrak{m}-\{0\}$,那么$A/fA$是正则局部环当且仅当$f\not\in\mathfrak{m}^2$.
\begin{enumerate}
	\item $\{x_1,x_2,\cdots,x_i\}$是某个正则参数系的子集.
	\item $\{x_1,x_2,\cdots,x_i\}$在$m/m^2$中线性无关.
	\item $A/(x_1,x_2,\cdots,x_i)$是$n-i$维的正则局部环.
\end{enumerate}
\begin{proof}
	
	1推2,我们不妨设$\{x_1,x_2,\cdots,x_n\}$是$(A,m)$的一个正则参数系,那么它在$m/m^2$的像集合在$k=A/m$上生成了整个线性空间,既然$m/m^2$作为$A/m$线性空间的维数是$n$,于是只能有该像集合是一组基.特别的$\{x_1,x_2,\cdots,x_i\}$在$m/m^2$中的像是线性无关的.
	
	2推1,设集合$\{x_1,x_2,\cdots,x_n\}\subseteq m$使得它在$m/m^2$中的像$\{\overline{x_1},\overline{x_2},\cdots,\overline{x_n}\}$是一组基,那么NAK引理得到$\{x_1,x_2,\cdots,x_n\}$生成了整个$m$.此时它就是正则参数系.
	
	1推3,我们之前证明过此时$\dim R/(x_1,x_2,\cdots,x_i)=n-i$.另外有$\{x_{i+1},\cdots,x_n\}$生成了这个商环的唯一极大理想,于是正则.
	
	3推1,设$R/(x_1,x_2,\cdots,x_i)$的极大理想是被$\{y_1,y_2,\cdots,y_{n-i}\}$的像集生成的,那么$m$会由$\{x_1,x_2,\cdots,x_i,y_1,\cdots,y_{n-i}\}$生成,于是这个集合是一个参数系统.
\end{proof}

正则局部环是整环.于是特别的,零维诺特局部环是正则的当且仅当它是域.
\begin{proof}
	
	记$R$是正则局部环.首先如果$\dim R=0$,那么极大理想被空集生成,于是此时$R$是域.再设$\dim R=1$,此时极大理想$m=(x)$,任取一个极小素理想$p$,我们需要证明$p=\{0\}$.按照NAK引理这等价于证明$p=mp$.其中$mp\subseteq p$是平凡的,为证$p\subseteq mp$,任取$y\in p\subseteq (x)=m$,可记$y=xz$,按照$x\not\in p$和$p$是素理想,得到$z\in p$,于是$y=xz\in mp$.
	
	最后设$\dim R\ge2$,假设对$\dim R-1$维的局部正则环都是整环.设$R$的全部极小素理想为$\{p_1,p_2,\cdots,p_r\}$,那么有$m\not\subseteq p_i$,否则和维数$\ge2$矛盾.另外$m\not\subseteq m^2$,否则由NAK引理得到$m=0$得到$\dim R=0$矛盾.于是按照改进版本的prime avoidance引理,得到$m\not\subset\left(m^2\cup(\cup_ip_i)\right)$.于是可取一个元$x\in m$不在这个并集中.那么从$x\not\in m^2$得到它在$m/m^2$中的像是一个线性无关集,按照上一个定理,得到$R/(x_1)$是$\dim R-1$维的正则局部环.按照归纳假设有$R/(x_1)$是整环,于是$(x_1)$是$R$的素理想.现在按照$x_1$的选取结合主理想定理,有$\mathrm{ht}(x_1)=1$.于是可取包含于$(x_1)$中的极小素理想$p$,那么有$p=x_1p\subseteq mp\subseteq p$,按照NAK引理得到$p=0$,于是$R$是整环.
\end{proof}

设$(A,m,k)$是$d$维的正则局部环,那么有$\mathrm{gr}^m(A)\cong k[X_1,X_2,\cdots,X_d]$.于是$A$的Samuel函数$\chi(n)=\left(\begin{array}{c}n+d\\d\end{array}\right)$.
\begin{proof}
	
	按照正则局部环的定义,记$m=(x_1,x_2,\cdots,x_d)$.构造同态$f:k[X_1,X_2,\cdots,X_d]\to\mathrm{gr}^m(A)$为$X_i$映射为$\overline{x_i}\in m/m^2$.于是这是满射,导致$\mathrm{gr}^m(A)\cong k[X_1,X_2,\cdots,X_d]/\ker f$.假设$\ker f\not=0$,可取$\ker f$中的齐次元$F$,于是存在满同态$k[X_1,X_2,\cdots,X_d]/(F)\to k[X_1,X_2,\cdots,X_d]/\ker f$.这导致二者的希尔伯特多项式满足$\varphi_F(n)\ge\varphi_{\ker f}(n)$.但是我们解释过只要$F\not=0$就有$\deg\varphi_F(n)\le d-2$,于是$A$的Samuel多项式的次数至多为$d-1$,和$\dim A=n$矛盾.
\end{proof}

解析无关性.给定诺特局部环$(A,m)$,元素$y_1,y_2,\cdots,y_r\in m$称为解析无关的,如果对每个$A$系数的齐次多项式$F$,从$F(y_1,y_2,\cdots,y_r)=0$总能推出$F$的系数都在$m$中.
\begin{enumerate}
	\item 如果$y_1,y_2,\cdots,y_r$是解析无关的,并且$A$包含了一个域$k$,那么对每个非零齐次多项式$F\in k[Y_1,Y_2,\cdots,Y_r]$,总有$F(y_1,y_2,\cdots,y_r)\not=0$.
	\item 设$(A,m)$是$d$维诺特局部环,每个参数系统$\{x_1,x_2,\cdots,x_d\}$总是解析无关的.
	\begin{proof}
		
		记$I=(x_1,x_2,\cdots,x_d)\subseteq A$.因为$\{x_i\}$是参数系统,有$\sqrt{I}=m$,于是$I$是定义理想,于是Samuel函数是$\chi_A^I(n)=l(A/I^{n+1})$,当$n$足够大时它是次数为$d$的多项式.记剩余域$\kappa=A/m$,对$n$次齐次多项式$f\in k[X_1,\cdots,X_d]$,称它为$I$中的零形式,如果对每个(存在一个就保证每一个都成立)$f$在$A[X_1,\cdots,X_d]$的提升$F$,有$F(x_1,\cdots,x_d)\in I^nm$.设$I$的全体零形式生成的理想为$J\subseteq k[X_1,\cdots,X_d]$.考虑同态$k[X_1,\cdots,X_d]\to\oplus_nI^n/I^nm$为把多项式$f$做齐次分解$\sum_nf_n$后,每个$f_n$映射为$f_n(x_1,\cdots,x_d)\in I^n/I^nm$,那么这个同态的核是$J$,于是有$k[X_1,\cdots,X_d]/J\cong\oplus_nI^n/I^nm$.
		
		\qquad
		
		下面记$k[X_1,\cdots,X_d]/J$的Hilbert多项式为$\varphi$,于是有$n$足够大时有$\varphi(n)=l(I^n/I^nm)$.但是按照NAK引理,有$l(I^n/I^nm)$是$I^n$作为$A$模的极小生成元集的元素个数.于是有$A/I$模的满同态$(A/I)^{l(I^n/I^nm)}\to I^n/I^{n+1}=I^n\otimes_AA/I$.于是有$\varphi(n)l(A/I)\ge l(I^n/I^{n+1})=\chi_A^I(n)-\chi_A^I(n-1)$.于是有$\deg(\varphi)\ge\deg(\chi_A^I)-1=d-1$.但是我们解释过当$J\not=0$时有$\deg(\varphi)\le d-2$.这个矛盾说明$J=0$,也即$\{x_1,\cdots,x_d\}$是解析无关的.
	\end{proof}
\end{enumerate}

下面定义模的重数.先给出两个引理.
\begin{enumerate}
	\item $\forall d\in\mathbb{N}$,取$f_d(n)=\sum_{i=1}^ni^d$,那么这是一个次数为$d+1$,首系数为$\frac{1}{d+1}$的多项式.
	\item 记$\chi:\mathbb{N}\to\mathbb{Z}$的数值函数,它在$n$足够大的时候是一个多项式(此时多项式系数必然在$\mathbb{Q}$中,克莱姆法则),如果这个多项式的次数$\le d$,那么存在整数$e$使得$n$足够大时有$\chi(n)=\frac{e}{d!}n^d+\cdots$.
	\begin{proof}
		
		对$d$归纳,当$d=0$时$\chi(n)$在$n$足够大时是常值的,此时$e$就是这个常数.下面设$d>0$,按照归纳假设有$\chi(n)-\chi(n-1)=\frac{e}{(d-1)!}n^{d-1}+\cdots$,对这个式子求和得到$\chi(n)=\frac{e}{d!}n^d+\cdots$.
	\end{proof}
\end{enumerate}

给定局部环$(A,m)$,记维数$d$,设$M$是有限$A$模,取$A$的定义理想$q$(对于局部环,定义理想的定义事实上就是$\sqrt{q}=m$),按照上述引理存在唯一的整数$e$满足$\chi_M^q(n)=\frac{e}{d!}n^d+\cdots$(实际上按照Samuel函数的定义,这里的$e\ge0$),我们称$e$为$M$的关于$q$的重数,记作$e(q,M)$.若$M=A$,记$e(q,A)$为$e(q)$.另外对极大理想$m$,记$e(m)$为$e(A)$.下面给出一些基本性质(后几条均可用第二条推出):
\begin{enumerate}
	\item 按照Samuel函数的定义,得到$e(q,M)=\lim_{n\to+\infty}\frac{l(M/q^nM)}{n^d/d!}$.
	\item 我们知道$\dim(M)$就是Samuel多项式的次数,于是有:
	$$\left\{\begin{array}{cc}\dim(M)<d&\Rightarrow e(q,M)=0\\ \dim(M)=d&\Rightarrow e(q,M)>0\end{array}\right.$$
	\item 对于局部环,设$q$是定义理想,我们知道对每个$r\ge1$都有$q^r$是定义理想.此时有$e(q^r,M)=e(q,M)\cdot r^d$.
	\item 如果$q'\subseteq q$是两个定义理想,那么$e(q,M)\le e(q',M)$.
	\item 当$\dim A=0$时,有$e(q,M)=l(M)$.
	\item 对于$d$维的正则局部环$A$,我们证明过$\mathrm{gr}^m(A)\cong k[X_1,X_2,\cdots,X_d]$,于是此时有$\chi_A^m(n)=\frac{1}{d!}n^d+\cdots$,于是$e(A)=1$.
\end{enumerate}

下面给出重数的更多性质.记$(A,m)$是诺特局部环,$q$是定义理想.
\begin{enumerate}
	\item 重数对短正合列有可加性:如果$0\to M'\to M\to M''\to0$是有限$A$模构成的短正合列,那么有$e(q,M)=e(q,M')+e(q,M'')$.
	\begin{proof}
		
		对任意$n\in\mathbb{N}$,考虑如下短正合列:
		$$\xymatrix{0\ar[r]&M'/M'\cap q^nM\ar[r]&M/q^nM\ar[r]&M''/q^nM''\ar[r]&0}$$
		
		按照模的长度对短正合列的可加性,得到$l(M/q^nM)-l(M''/q^nM'')=l(M'/M'\cap q^nM)$.按照Artin-Rees引理,得到$q^nM'\subseteq M'\cap q^nM\subseteq q^{n-c}M'$,于是得到$l(M'/q^nM)\ge l(M'/M'\cap q^nM)\ge l(M'/q^{n-c}M')$.于是得到:
		$$\frac{l(M'/q^nM')}{n^d/d!}\ge\frac{l(M/q^nM)-l(M''/q^nM'')}{n^d/d!}\ge\frac{l(M'/q^{n-c}M')}{n^d/d!}$$
		
		令$n\to+\infty$,得到$e(q,M')\ge e(q,M)-e(q,M'')\ge e(q,M')$,得证.
	\end{proof}
	\item 记$\{p_1,p_2,\cdots,p_t\}$是所有满足$\dim(A/p_i)=\dim A$的极小素理想.记$M$是有限$A$模,那么有$e(q,M)=\sum_{i=1}^te(\overline{q}_i,A/p_i)l_{A_{p_i}}(M_{p_i})$,其中$\overline{q}_i$是$q$在$A/p_i$中对应的理想.注意这里按照$A_{p_i}$都是Artin环以及$M$是有限$A$模就得到每个$l(M_{p_i})<\infty$.这个命题允许我们把求模的重数约化为求理想的重数.
	\begin{proof}
		
		记$\sigma(M)=\sum_{i=1}^tl(M_{p_i})$.我们来对$\sigma(M)$做归纳.首先如果$\sigma(M)=0$,那么每个$l(M_{p_i})=0$,于是每个$M_{p_i}=0$.于是这些$p_i$不在$\mathrm{Supp}(M)$中,于是$\dim(M)=\dim(\mathrm{Supp}(M))<\dim(A)$,于是$e(q,M)=0=\sum_{i=1}^te(\overline{q}_i,A/p_i)l(M_{p_i})$.
		
		下面设$\sigma(M)>0$.那么$M_{p_i}$不能全部为零.不妨记$M_{p_1}\not=0$,即$p_1\in\mathrm{Supp}(M)$,于是$p_1$是支集中的极小素理想,于是$p_1\in\mathrm{Ass}(M)$.此即存在子模$M'\subseteq M$使得$M'\cong A/p_1$.下面求$\sigma(M')$.当$i=1$时$(A/p_1)_{p_1}$是一个域;当$i\not=1$时$(A/p_1)_{p_i}$是零环.于是$\sigma(M')=1$.于是得到$e(q,M')=e(q,A/p_1)=\sum_{i=1}^te(\overline{q}_i,A/p_i)l(M'_{p_i})$.再取$M''=M/M'$,于是模的长度对短正合列的可加性得到$\sigma(M'')=\sigma(M)-1$.按照归纳假设得到$e(q,M'')=\sum_{i=1}^te(q,M'')l(M''_{p_i})$.把它和上一个和式相加完成归纳.
	\end{proof}
	\item 上一条的特例.如果$A$是局部整环,$M$是有限$A$模,那么$e(\mathfrak{q},M)=e(\mathfrak{q},A)\cdot\mathrm{rank}(M)$.这里的$\mathrm{rank}(M)=\dim_K(M\otimes_AK)$,其中$k=\mathrm{Frac}(A)$.此即非自由模$M$的秩的概念.
	\begin{proof}
		
		因为$A$是整环,唯一的极小素理想是零理想,此时$K=A_{(0)}$是$A$的商域,按照上一条有$e(\mathfrak{q},M)=e(\mathfrak{q})\cdot l(M_{(0)})$.但是这里$M_{(0)}$作为$K$模的维数恰好是$\mathrm{rank}(M)$.
	\end{proof}
	\item 设$(A,m)$是诺特局部环,记参数系统为$\{x_1,x_2,\cdots,x_d\}$.设某个定义理想$q$包含了这个参数系统,满足$x_i\in q^{v_i},v_i\ge1,\forall i$.那么对任意有限$A$模$M$和任意$1\le s\le d$,有$e(q/(x_1,x_2,\cdots,x_s),M/(x_1,x_2,\cdots,x_s))\ge v_1v_2\cdots v_s e(q,M)$.
	\begin{proof}
		
		只需证明$s=1$的情况,对于$s>1$的情况只要反复操作$s=1$即可.记$A_1=A/(x_1)$,$M_1=M/x_1M$,$q_1=q/(x_1)$,那么$q_1$也是$A_1$的定义理想.现在按照参数系统的性质,得到$\dim A_1=\dim A-1$.
		
		考虑短正合列$0\to x_1M+q^nM/q^nM\to M/q^nM\to M/x_1M+q^nM\to0$,这得到了模的长度的等式$l(M_1/q_1^nM_1)=l(M/x_1M+q^nM)=l(M/q^nM)-l(x_1M+q^nM/q^nM)$.其中$x_1M+q^nM/q^nM\cong x_1M/x_1M\cap q^nM$.考虑左乘$x_1$得到的满同态$M/q^{n-v_1}M\to x_1M/x_1M\cap q^nM$.于是得到$l(x_1M/x_1M\cap q^nM)\le l(M/q^{n-v_1}M)$.结合上一等式得到$l(M_1/q_1^nM_1)\ge l(M/q^nM)-l(M/q^{n-v_1}M)$.
		
		现在当$n$足够大的时候右侧第一项为$\frac{e}{d!}n^d+\cdots$,第二项为$\frac{e}{d!}(n-v_1)^d+\cdots$.于是右侧的$d-1$次项为$\frac{ev_1}{(d-1)!}n^{d-1}$.两边除以$n^{d-1}/(d-1)!$取极限,就得到$e(q',M')\ge v_1e(q,M)$,得证.
	\end{proof}
	\item 上一条的特例.如果$q$就是由参数系统生成的理想(此时$q$必然是一个定义理想,因为参数系统生成的理想是$m$准素的),取$M=A$和$s=d$,得到$l(A/q)\ge e(q)$.如果还满足$x_i\in m^v,\forall i$,那么有$l(A/q)\ge v^de(m)$.
	\item 记$(A,m)$是$d$维的诺特局部环,记$\{x_1,x_2,\cdots,x_d\}$是参数系统,记$q$是一个定义理想,记$M$是有限$A$模.现在取$A'=A/(x_1)$,$q'=q/(x_1)$和$M'=M/x_1M$.那么此时$\dim A'=\dim A-1$.一般做不到$e(q,M)=e(q',M')$.能够使得这个等式成立的一个条件是$x_1$不是$M$的零因子,也即$M$上的同态$m\mapsto x_1m$是单射.
	\begin{proof}
		
		首先上一个定理得到$e(q',M')\ge e(q,M)$.给定$A$模$M$的子模$N$和元$r\in A$,我们记$(N:x)$表示集合$\{m\in M\mid xm\in N\}$.现在考虑如下短正合列:
		$$\xymatrix{0\ar[r]&\frac{(q^nM:x_1)}{q^{n-1}M}\ar[r]&\frac{M}{q^{n-1}M}\ar[r]^{\cdot x_1}&\frac{x_1M+q^nM}{q^nM}\ar[r]&0}$$
		
		这就得到$l(M'/(q')^nM')=l(M/q^nM)-l(x_1M+q^nM/q^nM)=l(M/q^nM)-l(M/q^{n-1}M)+l((q^nM:x_1)/q^{n-1}M)$.下面记$\alpha=(x_2,x_3,\cdots,x_d)$,那么有$q=x_1A+\alpha$,于是得到$q^n=x_1q^{n-1}+\alpha^n$.于是$(q^nM:x_1)=q^{n-1}M+(\alpha^nM:x_1)$(这里用到了$x_1$不是$M$上零因子).由Artin-Rees引理,存在$c$使得$\alpha^nM\cap x_1M=\alpha^{n-c}(\alpha^cM\cap x_1M)$.特别的,这得到$(\alpha^nM:x_1)\subset\alpha^{n-c}M$(这里也用到了$x_1$不是$M$上的零因子).于是得到$\frac{(q^nM:x_1)}{q^{n-1}M}=\frac{q^{n-1}M+(\alpha^nM:x_1)}{q^{n-1}M}\subset\frac{q^{n-1}M+\alpha^{n-c}M}{q^{n-1}M}\cong\frac{\alpha^{n-c}M}{q^{n-1}M\cap\alpha^{n-c}M}$.这里最后一项是一个$A/q^{c-1}$模,因为$q^{c-1}$在其上的作用是零作用.
		
		按照$\alpha$由$d-1$个元生成,得到$\alpha^{n-c}$可由$\left(\begin{array}{c}n-c+d-2\\d-2\end{array}\right)$个元生成,记$M$可被$r$个元生成,于是$\alpha^{n-c}M$可被$N=r\left(\begin{array}{c}n-c+d-2\\d-2\end{array}\right)$个元生成,现在作为$A/q^{c-1}$模有满同态$(A/q^{c-1})^N\to\alpha^{n-c}M/(\alpha^{n-c}M\cap q^{n-1}M)$.于是得到长度公式$l(\alpha^{n-c}M/\alpha^{n-c}M\cap q^{n-1}M)\le r\left(\begin{array}{c}n-c+d-2\\d-2\end{array}\right)l(A/q^{c-1})$.重点是这里右侧是关于$n$的$d-2$次多项式.于是等式$l(M'/(q')^nM)=l(M/q^nM)-l(M/q^{n-1}M)+l(\frac{(q^nM:x_1)}{q^{n-1}M})$右侧在$n$足够大的时候前两项是$\frac{e}{d!}n^d-\frac{e}{d!}(n-1)^d+\cdots$,第三项放缩为一个$d-2$次多项式,于是右侧可写作$\frac{e}{(d-1)!}n^{d-1}+\cdots$.这里$e=e(q,M)$,两边除去$n^{d-1}/(d-1)!$取极限$n\to\infty$就得到$e(q',M')=e(q,M)$.
	\end{proof}
	\item Lech引理.设$A$是$d$维的诺特局部环,设$\{x_1,x_2,\cdots,x_d\}$是一个参数系统,取$\mathfrak{q}=(x_1,x_2,\cdots,x_d)$,设$M$是有限$A$模,那么有:
	$$e(\mathfrak{q},M)=\lim_{\min(v_i)\to\infty}\frac{l(M/(x_1^{v_1},x_2^{v_2},\cdots,x_d^{v_d})M)}{v_1v_2\cdots v_d}$$
	\begin{proof}
		
		我们对$d$归纳证明这件事.如果$d=0$,那么$\dim M=0$(因为$M$的维数是$V(\mathrm{Ann}(M))$的维数,这不超过$\dim A$).此时有$e(\mathfrak{q},M)=l(M)$,参数系统是空集,所以等式右侧也是$l(M)$.如果$d=1$,我们解释过有$e(\mathfrak{q},M)=\lim\limits_{n\to\infty}\frac{l(M/\mathfrak{q}^nM)}{n}$.
		
		\qquad
		
		下面设$d>1$,记$N_j=\{m\in M\mid x_1^jm=0\}$,那么有$N_1\subseteq N_2\subseteq\cdots$,按照诺特条件有足够大的$c>0$使得$\forall i\ge0$有$N_{i+c}=N_c$.记$M'=x_1^cM$,那么$x_1$不会是$M'$的零因子:如果$x_1(x_1^cm)=0$,那么$m\in N_{c+1}=N_c$,所以$x_1^cm=0$.下面考虑数乘$x_1^c$诱导的短正合列:
		$$\xymatrix{0\ar[r]&N_c\ar[r]&M\ar[r]^{x_1^c}&M'\ar[r]&0}$$
		
		因为这里$N_c$是$A/x_1^cA$模,所以$\dim N_c\le\dim A/x_1^cA<d$,所以有$e(\mathfrak{q},N_c)=0$.另外我们解释过乘数对短正合列有可加性,于是$e(\mathfrak{q},M)=e(\mathfrak{q},M')$.另一方面考虑如下短正合列:
		$$\xymatrix{0\ar[r]&N_c/N_c\cap(x_1^{v_1},\cdots,x_d^{v_d})M\ar[r]&M/(x_1^{v_1},\cdots,x_d^{v_d})M\ar[r]&M'/(x_1^{v_1},\cdots,x_d^{v_d})M'\ar[r]&0}$$
		
		于是有:
		$$l(M/(x_1^{v_1},\cdots,x_d^{v_d})M)-l(M'/(x_1^{v_1},\cdots,x_d^{v_d})M')=l(N_c/N_c\cap(x_1^{v_1},\cdots,x_d^{v_d})M)\le l(N_c/(x_1^{v_1},\cdots,x_d^{v_d})N_c)$$
		
		如果$v_1\ge c$,得到$x_1^{v_1}N_c=0$,并且$N_c$是$d-1$维诺特局部环$A/x_1^cA$上的模,所以按照归纳假设有:
		$$e(\mathfrak{q}',N_c)=\lim\limits_{\min\{v_i\}\to\infty}\frac{l(N_c/(x_2^{v_2},\cdots,x_d^{v_d})N_c)}{v_2v_3\cdots v_d}$$
		
		这里$\mathfrak{q}'=(\overline{x_2},\cdots,\overline{x_d})\subseteq A/x_1^cA$.于是存在常数$C>0$,当$\min\{x_i\}\to\infty$时有:
		$$l(N_c/(x_1^{v_1},\cdots,x_d^{v_d})N_c)\le Cv_2\cdots c_d$$
		
		于是有:
		$$\lim\limits_{\min\{v_i\}\to\infty}\frac{l(M/(x_1^{v_1},\cdots,x_d^{v_d})M)-l(M'/(x_1^{v_1},\cdots,x_d^{v_d})M')}{v_1\cdots v_d}=0$$
		
		这说明我们可不妨用$M'$代替$M$来证明命题.换句话讲,不妨设$x_1$不是$M$上的零因子.那么按照上一条,有$e(\mathfrak{q},M)=e(\overline{\mathfrak{q}},\overline{M})$.其中$\overline{\mathfrak{q}}=\mathfrak{q}/x_1A\subseteq A/x_1A$,而$\overline{M}=M/x_1M$.再记$E=(x_2^{v_2},\cdots,x_d^{v_d})M$和$F=M/E$.我们之前解释过有$e(\mathfrak{q},M)v_1\cdots v_d\le l(M/(x_1^{v_1},\cdots,x_d^{v_d})M)=l(F/x_1^{v_1}F)$.下面有:
		$$l(F/x_1^{v_1}F)=\sum_{i=1}^{v_1}l(x_1^{i-1}F/x_1^iF)\le v_1l(F/x_1F)=v_1l(\overline{M}/(x_2^{v_2},\cdots,x_d^{v_d})\overline{M})$$
		
		这两个不等式得到:
		$$e(\mathfrak{q},M)\le\frac{l(M/(x_1^{v_1},\cdots,x_d^{v_d})M)}{v_1\cdots v_d}\le\frac{l(\overline{M}/(x_2^{v_2},\cdots,x_d^{v_d})\overline{M})}{v_2\cdots v_d}$$
		
		但是按照归纳假设,这个不等式右侧在$\min\{v_i\}\to\infty$时极限是$e(\overline{\mathfrak{q}},\overline{M})$,按照$x_1$不是$M$的零因子,这个乘数也就是$e(\mathfrak{q},M)$,这就完成归纳.
	\end{proof}
\end{enumerate}

我们已经看到了参数系统生成的定义理想具有多种良好性质,接下来说明某些条件下求一般的定义理想的重数可以约化为求参数系统生成的定义理想的重数.对任意交换环$A$,设$\alpha\subseteq A$是理想,称理想$\alpha'$为$\alpha$的约化,如果$\alpha'\subset\alpha$,并且存在$r$使得$\alpha^{r+1}=\alpha'\alpha^r$.于是$\forall n\in\mathbb{N}$都有$\alpha^{n+r}=(\alpha')^n\alpha^r$.
\begin{enumerate}
	\item $(A,m)$是诺特局部环,记$q$是定义理想(局部环上定义理想等价于根理想是唯一极大理想),取$\alpha$为$q$的约化,那么$\alpha$也是定义理想,并且有$e(q,M)=e(\alpha,M)$.
	\begin{proof}
		
		取$r\in\mathbb{N}$使得$q^{r+1}=\alpha q^r$,于是$q^{r+n}=\alpha^nq^r$.于是有:$l(M/\alpha^{n+r}M)\ge l(M/q^{n+r}M)=l(M/\alpha^nq^rM)\ge l(M/\alpha^nM)$.除去$n^d/d!$取极限$n\to+\infty$就得到$e(q,M)=e(\alpha,M)$.
	\end{proof}
	\item $(A,m)$是$d$维诺特局部环,约定$k=A/m$是无限域,取$q=(u_1,u_2,\cdots,u_s)$是$A$的一个定义理想,记$\{y_1,y_2,\cdots,y_d\}$是“足够一般”(sufficiently general)的$\{u_1,u_2,\cdots,u_s\}$的$A$线性组合,那么理想$(y_1,y_2,\cdots,y_d)$是$q$的约化,并且$\{y_1,y_2,\cdots,y_d\}$是一组参数系统.这里$\{y_1,y_2,\cdots,y_d\}$是$\{u_1,u_2,\cdots,u_s\}$的足够一般的线性组合是指,存在有限个$sd$元多项式$F_l(\Lambda_{ij}),1\le i\le d,1\le j\le s,1\le l\le N$【】
	\begin{proof}
		
		
	\end{proof}
\end{enumerate}
\newpage
\subsection{环扩张的维数}

关于纤维的维数.
\begin{enumerate}
	\item 先回顾一些内容.给定环同态$f:A\to B$,它诱导了素谱之间的连续映射$f^*:\mathrm{Spec}(B)\to\mathrm{Spec}(A)$.任取素理想$p\in\mathrm{Spec}A$,那么存在自然的同胚$(f^*)^{-1}(p)\cong\mathrm{Spec}(B\otimes_A\kappa(p))$.按照这个性质,我们把$B\otimes_A\kappa(p)\cong B_p/pB_p$称为$f$在点$p$处的纤维环.
	\begin{itemize}
		\item 如果$(A,m)$是局部环,称$\mathrm{Spec}(B\otimes_A\kappa(m))$为$f$的闭纤维.
		\item 如果$A$是整环,此时零理想是素理想,称$\mathrm{Spec}(B\otimes_A\kappa(0))$为$f$的一般纤维(general或者generic).
	\end{itemize}
    \item 给定诺特环之间的环同态$\varphi:A\to B$,记$p\in\mathrm{Spec}(B)$,记$q=\varphi^{-1}(p)$.有不等式$\mathrm{ht}(p)\le\mathrm{ht}(q)+\dim(B_p/qB_p)$.注意这里$B_p/qB_p$就是$B_p/pB_p\cong B\otimes_A\kappa(q)$在素理想$\overline{p}$处的局部化.
    \begin{proof}
    	
    	不妨将$A$换为$A_q$,$B$换为$B_p$,此时条件均不变,并且此时$\varphi$是局部环同态.于是不妨设$(A,q)$和$(B,p)$均为局部环,并且$\varphi$是局部环同态.需要证明的是$\dim B\le\dim A+\dim B/qB$.
    	
    	为此取$A$的一组参数系统$\{x_1,x_2,\cdots,x_r\}\subseteq q$,取$y_1,y_2,\cdots,y_s\in p$使得它们构成$B/qB$的一组参数系统.需要证明的是$\dim B\le r+s$.我们之前证明过一个结论:一个局部环$(A,m)$的维数恰好是全部$m$准素理想的全部生成元集的个数的最小值.于是为证这个不等式只需证明$\sqrt{(x_1,x_2,\cdots,x_r,y_1,y_2,\cdots,y_s)}=p$,这里我们把$x_i$理解为$B$中的$\varphi(x_i)$.
    	
    	\qquad
    	
    	按照$(x_1,x_2,\cdots,x_r)$是$q$准素理想,于是存在某个次幂使得$q^u\subset(x_1,x_2,\cdots,x_r)$.同理$(p/qB)^v\subset(y_1',y_2',\cdots,y_s')$,这里$y_i'$是$y_i\in B$在$B/qB$中的像.于是$p^v\subseteq qB+(y_1,y_2,\cdots,y_s)$.于是得到$(q)^{ab}\subset(qB+(y_1,y_2,\cdots,y_s))^a\subseteq q^aB+(y_1,y_2,\cdots,y_s)\subset(x_1,x_2,\cdots,x_r,y_1,y_2,\cdots,y_s)$.
    \end{proof}
    \item 如果诺特环之间的环同态$\varphi:A\to B$满足下降条件(例如$\varphi$是平坦扩张,或者例如$\varphi$是整扩张且$B$是整环而$A$是整闭的),那么上一条这个不等式取等.
    \begin{proof}
    	
    	延续上一条的证明,不妨设$(A,q)$和$(B,p)$都是局部环,并且$\varphi$是局部环同态.只需证明如果$\varphi$满足下降条件,那么有$\dim B\ge\dim A+\dim B/qB$.
    	
    	\qquad
    	
    	设$\dim A=r$和$\dim B/qB=s$.那么可取$A$中的严格包含的素理想链$q=q_0\supset q_1\supset\cdots\supset q_r$.还可以取$B$中的严格包含的并且包含理想$qB$的素理想链$p=p_0\supset p_1\supset\cdots\supset p_s$.从$qB\subseteq p_s$和$q$本身已经是$A$中极大理想,得到$q=\varphi^{-1}(p_s)$,于是下降定理得到$B$中素理想$p_s$还可以延拓为严格包含的$B$中素理想链$p_s\supset p_{s+1}\supset\cdots\supset p_{s+r}$.这导致$\dim B\ge r+s$.
    \end{proof}
\end{enumerate}

上升条件和下降条件会导致纤维环维数的大小变化.
\begin{enumerate}
	\item 设$\varphi:A\to B$是诺特环同态,并且满足上升条件,如果$A$中有素理想$q\subseteq p$,那么有维数不等式$\dim(B\otimes_Ak(p))\ge\dim(B\otimes_Ak(q))$.
	\begin{proof}
		
		设$\dim(B\otimes_Ak(q))=r$,设$p/q$作为$A/q$的素理想的高度是$s$.于是按照$\mathrm{Spec}(B\otimes_Ak(q))$同胚于$\varphi^*$在点$q$处的纤维,说明可取$B$中长度为$r$的素理想链$Q_0\subseteq Q_1\subset\cdots\subseteq Q_r$,满足每个$Q_i$在$q$上(这是在说$\varphi^{-1}(Q_i)=q$).再按照定义取$A$中的严格包含的素理想链$q=p_0\subseteq p_1\subset\cdots\subseteq p_s=p$.于是按照上升条件,$Q_i$构成的严格包含的素理想链可延拓为$Q_0\subseteq Q_1\subset\cdots\subseteq Q_{r+s}$.这里$Q_0$在$q$上说明了$\mathrm{ht}(Q_{r+s}/qB)\ge r+s$.这里$Q_{r+s}$是在$p=p_s$上的素理想.
		
		接下来按照上一个定理,考虑环同态$A/q\to B/qB$,得到$\mathrm{ht}(Q_{r+s}/qB)\le\mathrm{ht}(p/q)+\dim(B_{Q_{r+s}}/pB_{Q_{r+s}})$.这里最后一项$B_{Q_{r+s}}/pB_{Q_{r+s}}$是$B\otimes_Ak(p)$的局部化,于是有$\dim(B_{Q_{r+s}}/pB_{Q_{r+s}})\le\dim(B\otimes_Ak(p))$,结合上一段的不等式,得到$\dim(B\otimes_Ak(p))\ge r=\dim(B\otimes_Ak(q))$.
	\end{proof}
	\item 设$\varphi:A\to B$是诺特环同态,并且满足下降条件,如果$A$中有素理想$q\subseteq p$,那么有维数不等式$\dim(B\otimes_Ak(p))\le\dim(B\otimes_Ak(q))$.
	\begin{proof}
		
		尽管是对偶命题,这个命题的证明比上一个要繁琐一点.取$A$中$p,q$之间的不可延长的素理想严格包含链$p=p_0\supset p_1\supset\cdots\supset p_r=q$,即$\mathrm{ht}(p_i/p_{i+1})=1$.只需证明$\dim(B\otimes_Ak(p_i))\le\dim(B\otimes_Ak(p_{i+1}))$即可.于是不妨干脆假定$p,q$之间不能加入新的素理想.
		
		按照纤维环的素谱同胚于素理想处的纤维,只需证明如果取$B$中的在$p$上的严格包含的素理想链$p_0\subseteq p_1\subset\cdots\subseteq p_r$,那么可构造在$q$上的$B$中的严格包含的素理想链$q_0\subseteq q_1\subset\cdots\subseteq q_r$.我们来归纳构造$q_i$,但是在实际归纳过程中我们要求证明更强的归纳假设条件$q_i\subseteq p_i$以及$\mathrm{ht}(q_i/q_{i-1})=1$.初始步骤$q_0$的构造仅仅是下降条件:从$q\subseteq p$和$p_0$在$p$上,就得到存在在$q$上的$q_0$满足$q_0\subseteq p_0$.
		
		构造$q_1$.不妨$q\not=p$,取$x\in p-q$,取$q_0+xB\subseteq B$的全部极小素理想$g_1,g_2,\cdots,g_s$.那么主理想定理说明有$\mathrm{ht}(g_i/q_0)=1,\forall i$(主理想定理说明这个高度$\le1$,但是$A/q_0$是整环,此时零理想必然是素理想,于是高度$\ge1$).从$q_0\subsetneqq p_0\subsetneqq p_1$得到$\mathrm{ht}(p_1/q_0)\ge2$.于是$p_1\not\subseteq g_i,\forall i$,于是$p_1\not\subset\cup_ig_i$.再取$y\in p_1-\cup_ig_i$,取$B$中$q_0+yB$的并且包含于$p_1$的极小素理想为$q_1$.于是按照主理想定理有$\mathrm{ht}(q_1/q_0)=1$.
		
		我们断言$q_1$在$q$上.事实上从$q_0\subseteq q_1\subseteq p_1$得到$q=\varphi^{-1}(q_0)\subset\varphi^{-1}(q_1)\subset\varphi^{-1}(p_1)=p$,按照我们假设的$q$和$p$之间没有其它素理想,得到要么$\varphi^{-1}(q_1)=p$要么$=q$.假设$\varphi^{-1}(q_1)=p$,那么$x\in p=\varphi^{-1}(q_1)$,于是$q_1$包含$q_0+xB$,按照$\mathrm{ht}(q_1/q_0)=1$得到$q_1$是包含$q_0+xB$的极小素理想,于是$q_1$是某个$g_i$,这就和$q_1$包含了$y\not\in g_i$矛盾.于是$q_1$在$q$上.
	\end{proof}
\end{enumerate}

关于多项式环和形式幂级数环的维数.
\begin{enumerate}
	\item 引理.设$A$是环,如果$\mathfrak{M}\subseteq A[[X_1,\cdots,X_n]]$是极大理想,那么$m=\mathfrak{M}\cap A$是$A$的极大理想,并且此时有$\mathfrak{M}=(m,X_1,\cdots,X_n)$.另外注意这件事是不平凡的,对一般的环同态$A\to B$(即便多项式环的情况),不能保证极大理想的原像是极大理想,例如考虑$A$是一个DVR,设$B=A[X]$,设$\pi\in A$是素元,那么$(\pi X-1)\subseteq A[X]$是极大理想,但是$(\pi X-1)\cap A=(0)$.
	\begin{proof}
		
		事实上对每个形式幂级数$f$,总有$1+X_if$是可逆元,于是每个$X_i$落在Jacobson根中,于是$(X_1,X_2,\cdots,X_n)\subseteq\mathfrak{M}$,然后从$A[[X_1,X_2,\cdots,X_n]]/\mathfrak{M}=(A[[X_i]]/(X_i))/(\mathfrak{M}/(x_i))=A/(\mathfrak{M}\cap A)$就得到$\mathfrak{M}\cap A$是$A$的极大理想.
	\end{proof}
	\item 设$A$是诺特环,那么有$\dim A[X_1,X_2,\cdots,X_n]=\dim A[[X_1,X_2,\cdots,X_n]]=\dim A+n$.
	\begin{proof}
		
		首先反复运用希尔伯特基定理,说明仅需证明$n=1$的情况.先来说明多项式环的情况.取$A[X]$中的素理想$p$,记$q=p\cap A$,它是$A$中的素理想.于是我们之前证明的定理给出了$\mathrm{ht}(p)\le\mathrm{ht}(q)+\dim(A[X]_p/qA[X]_p)$.这里$A[X]_p/qA[X]_p=(A[X]\otimes_Ak(q))_{p'}=(k(p)[X])_{p'}$.这里$k(p)[X]$的维数是1,所以局部化的维数不超过1.于是$\mathrm{ht}(p)\le\mathrm{ht}(q)=1\le\dim A+1$,于是左侧取上确界得到$\dim A[X]\le\dim A+1$.反过来任取$A$中素理想$p$,那么$pA[X]$是$A[X]$的素理想(因$A[X]/pA[X]\cong(A/p)[X]$是整环),于是从$pA[X]\subsetneqq(p,X)$得到$\dim A[X]\ge\dim A+1$.
		
		\qquad
		
		接下来证明形式幂级数环的情况.这个证明不能模仿的理由是一般来讲$A[[X]]\otimes_AB\not\cong B[[X]]$.但是我们总有对$A$的极大理想$m$总有$A[[X]]\otimes_Ak(m)\cong A[[X]]\otimes_AA/m\cong A/m[[X]]$.另外上述引理说明:如果$m$是形式幂级数环$A[[X_1,X_2,\cdots,X_n]]$中的极大理想,那么有$m\cap A$是$A$中的极大理想,反过来任取$A$的极大理想$n$就有$(n,X_i)$是$A[[X_i]]$的极大理想,换句话讲$A$和形式幂级数环的极大理想是一一对应的.
		
		\qquad
		
		现在取$A[[X]]$的极大理想$m$,记$m'=m\cap A$,那么$m=(m',X)$.于是有$A[[X]]_m/m'A[[X]]_m=(A[[X]]\otimes_Ak(m'))_{m}=k(m')[[X]]_{m}$这是一个主理想整环,于是维数不超过1.于是得到$\mathrm{ht}(m)\le\mathrm{ht}(m')+1\le\dim A+1$于是取左侧上确界得到$\dim A[[X]]\le\dim A+1$.反过来如果$n$是$A$的极大理想,那么$(n,X)$是$A[[X]]$的极大理想,并且有$(n,X)\cap A=n$.现在注意$A[[X]]$是$A$上平坦的,于是它满足下降条件,于是有维数公式$\mathrm{ht}(n)=\mathrm{ht}(m)+\dim(X)$,这里$X=A[[X]]_n/mA[[X]]_n\cong k(m)[[X]]$,它的维数恰好是1(因为$k(m)$是域).这就得到$\mathrm{ht}(m)+1=\mathrm{ht}(n)\le\dim A[[X]]$.再取上确界得到$\dim A+1\le\dim A[[X]]$.
	\end{proof}
	\item $A\to A[X]$的任意纤维环的维数恰好是$\dim A$.因为任取$A$的素理想$p$,那么纤维环是$A[X]\otimes_A\kappa(p)=\kappa(p)[X]$,维数恰好是1.
	\item $A\to A[[X]]$的闭纤维的维数$\dim(A[[X]]\otimes k(m))=1$.但是和上一条多项式环不同的是,存在这样的纤维的维数大于1.例如取$A=k[Y,Z]$,这里$k$是一个域,考虑$A\to A[[X]]$.我们来证明它的一般纤维的维数是2:
	\begin{proof}
		
		我们知道$\mathrm{Frac}(k[[X]])$在$k[X]$上的超越维数是无限的,所以特别的可以取$u(X),v(X)\in k[[X]]$在$k(X)$上是代数无关的.考虑满同态$\varphi:A[[X]]\to k[[X]]$为$X\mapsto X$,$Y\mapsto u(X)$,$z\mapsto v(X)$.它的核记作$\mathfrak{P}$,按照$u(X),v(X)$在$k(X)$上(从而在$k$上)代数无关,所以有$\mathfrak{P}\cap A=(0)$.所以$A[[X]]$的素理想$\mathfrak{P}$落在$A\to A[[X]]$的一般纤维中(也即是零理想的提升素理想).另一方面有$A[[X]]$中长度为2的素理想链$(0)\subsetneqq(Y-u(X))\subsetneqq\mathfrak{P}$,并且这些素理想都是$(0)\subseteq A$的提升素理想,于是有$\dim(A[[X]]\otimes_AK)\ge2$,其中$K=\mathrm{Frac}(A)$.反过来有$A[[X]]/\mathfrak{P}\cong k[[X]]$的维数是1,所以$\mathfrak{P}$的余高度是1.但是我们有$\dim A[[X]]=\dim A+1=3\ge\mathrm{ht}(\mathfrak{P})+\mathrm{coht}(\mathfrak{P})$,于是有$\mathrm{ht}(\mathfrak{P})\le2$.综上得到$\mathrm{ht}(\mathfrak{P})=2$.我们之前的引理解释过$A[[X]]$的极大理想总包含了$A$的极大理想,所以这里一般纤维$A[[X]]\otimes_AK$中的素理想都不是极大的,所以至少有$3=\dim(A[[X]])\ge\dim(A[[X]]\otimes_AK)+1$,于是有$\dim(A[[X]]\otimes_AK)\le2$,综上得到$\dim(A[[X]]\otimes_AK)=2$.
	\end{proof}
\end{enumerate}

维数不等式和维数公式.
\begin{enumerate}
	\item 设$A$是诺特整环,设$A\subseteq B$是环扩张(即单同态),并且$B$是整环(未必诺特),记$\mathfrak{P}\in\mathrm{Spec}(B)$,记$\mathfrak{p}=\mathfrak{P}\cap A$是$A$中素理想,那么有如下不等式,这里$\mathrm{tr.deg}_AB$表示的是$\mathrm{tr.deg}_{\mathrm{Frac}(A)}\mathrm{Frac}(B)$.另外如果$B=A[X_1,X_2,\cdots,X_n]$,则这个不等式取等号.
	$$\mathrm{ht}(\mathfrak{P})+\mathrm{tr.deg}_{\kappa(\mathfrak{p})}\kappa(\mathfrak{P})\le\mathrm{ht}(\mathfrak{p})+\mathrm{tr.deg}_A(B)$$
	\begin{proof}
		
		我们先证明归结为$B$是有限型$A$代数的情况.首先设不等式右侧是有限的,否则没什么需要证的.假设正整数$m,t$满足$\mathrm{ht}(\mathfrak{P})\ge m$和$\mathrm{tr.deg}_{\kappa(\mathfrak{p})}\kappa(\mathfrak{P})\ge t$.那么存在$B$中长度$m$的素理想链$\mathfrak{P}=\mathfrak{P}_0\supsetneq\mathfrak{P}_1\supsetneq\cdots\supsetneq\mathfrak{P}_m$.取$a_i\in\mathfrak{P}_i-\mathfrak{P}_{i+1},0\le i\le m-1$.再设$c_1,c_2,\cdots,c_t\in B$使得它们在$\mathrm{mod}\mathfrak{P}$下的像在$A/\mathfrak{p}$中是超越的.取$C$是$A$代数$B$的被$\{a_i,0\le i\le m-1\}$和$\{c_i,1\le i\le t\}$生成的子代数.那么有$C$中的长度$m$的素理想链$C\cap\mathfrak{P}=C\cap\mathfrak{P}_0\supsetneq C\cap\mathfrak{P}_1\supsetneq\cdots\supsetneq C\cap\mathfrak{P}_m$.于是$C$中的素理想$C\cap\mathfrak{P}$的高度$\ge m$.类似的有$\mathrm{tr.deg}_{\kappa(\mathfrak{p})}(\kappa(C\cap\mathfrak{P}))\ge t$.所以相同的不等式关系对$C$和素理想$C\cap\mathfrak{P}$成立.于是一旦我们证明命题对$A$的有限型代数$C$成立,就有:
		$$m+t\le\mathrm{ht}(C\cap\mathfrak{P})+\mathrm{tr.deg}_{\kappa(\mathfrak{p})}(\kappa(C\cap\mathfrak{P}))\le\mathrm{ht}(\mathfrak{p})+\mathrm{tr.deg}_AC\le\mathrm{ht}(\mathfrak{p})+\mathrm{tr.deg}_AB$$
		
		再让$m$和$t$变动得到不等式对$B$成立.于是我们可设$B$是有限型$A$代数.另外这里超越维数是对于扩张链有加性的,所以还可以不妨设$B$是$A$上单扩张的代数,也即设$B=A[x]$.我们还可以用$A_{\mathfrak{p}}$替代$A$,用$B\otimes_AA_{\mathfrak{p}}=A_{\mathfrak{p}}[x]$替代$B$,因为这不改变对应素理想的高度,也不改变对应剩余域(从而不改变相应超越维数).于是不妨设$(A,\mathfrak{p})$是局部环.记$k=A/\mathfrak{p}$,记$B=A[X]/\mathfrak{Q}$.下面分两种情况:
		
		\qquad
		
		如果$\mathfrak{Q}=(0)$.此即$x=X$是$A$上超越元,那么$B=A[X]$是多项式环,我们解释过此时有$\mathrm{ht}(\mathfrak{P})=\mathrm{ht}(\mathfrak{p})+\dim(B_{\mathfrak{P}}/\mathfrak{p}B_{\mathfrak{P}})$.但是这里$B/\mathfrak{p}B=k[X]$,所以$\overline{\mathfrak{P}}$要么是$k[X]$的零理想,要么是高度1的素理想.前者也即$\mathfrak{P}=pB$,于是$B_{\mathfrak{P}}/\mathfrak{p}B_{\mathfrak{P}}=k(X)$是域,于是$\mathrm{ht}(\mathfrak{p})=\mathrm{ht}(\mathfrak{P})$,且$\mathrm{tr.deg}_{\kappa(\mathfrak{p})}\kappa(\mathfrak{P})=\mathrm{tr.deg}_kk(X)=1$和$\mathrm{tr.deg}_AB=\mathrm{tr.deg}_KK(X)=1$,于是此时不等式取等号.类似的后者情况该不等式也甚至取等号.
		
		\qquad
		
		如果$\mathfrak{Q}\not=(0)$,那么$B$是在$A$上添加了一个代数元,于是$\mathrm{tr.deg}_AB=0$.因为$A\subseteq B=A[X]/\mathfrak{Q}$是单同态,所以在$A[X]$中有$A\cap\mathfrak{Q}=(0)$.记$K=\mathrm{Frac}(A)$,那么$\mathrm{ht}(\mathfrak{Q})=\mathrm{ht}(\mathfrak{Q}K[X])=1$.如果记$\mathfrak{P}'\subseteq A[X]$是$\mathfrak{P}$在$A[X]$中的原像,那么$\mathfrak{P}=\mathfrak{P}'/\mathfrak{Q}$和$\kappa(\mathfrak{P})=\kappa(\mathfrak{P}')$.结合$\mathfrak{Q}=(0)$情况我们已经证明了,就有:
		\begin{align*}
			\mathrm{ht}(\mathfrak{P})&\le\mathrm{ht}(\mathfrak{P}')-\mathrm{ht}(\mathfrak{Q})\\&=\mathrm{ht}(\mathfrak{P}')-1\\&=\left(\mathrm{ht}(\mathfrak{p})+\mathrm{tr.deg}_A(A[X])-\mathrm{tr.deg}_{\kappa(\mathfrak{p})}\kappa(\mathfrak{P}')\right)-1\\&=\mathrm{ht}(\mathfrak{p})-\mathrm{tr.deg}_{\kappa(\mathfrak{p})}(\kappa(\mathfrak{P}))
		\end{align*}
	\end{proof}
    \item 给定整环之间的扩张$A\to B$,其中$A$是诺特的,我们把上一命题中的不等式称为$A\to B$的维数不等式.如果$A\to B$使得对任意$B$的素理想$p$有上述维数不等式是取等的,就称$A\to B$满足维数公式.于是上一条说明如果$A$是诺特整环,那么$A\to A[X_1,X_2,\cdots,X_n]$满足维数公式.
\end{enumerate}

catenary环的补充.
\begin{enumerate}
	\item 回顾定义.诺特环$A$称为catenary的,如果对任意素理想$p\subseteq q$,它们作为初始端和终端的任意的不可延长的素理想严格包含链$p=p_0\subsetneqq p_1\subsetneqq\cdots\subsetneqq p_r=q$的长度都是相同的.那么这个条件下相同的长度必然是$\mathrm{ht}(q/p)$(这里$q/p$作为$A/p$的素理想).称诺特环$A$是universally catenary的,如果$A$的每个有限生成代数都是catenary的.
	\item 如果$A$是catenary的,那么$A$的商环和分式化总是catenary的.于是$A$是universlly catenary的当且仅当全体$A[X_1,X_2,\cdots,X_n],n\ge1$是catenary的.我们会在下文证明也等价于$A[X]$是catenary的.
	\item 如果$A$诺特,对任意素理想$p\subseteq q$,总有$\mathrm{ht}(q/p)=\mathrm{ht}(q)-\mathrm{ht}(p)$成立,那么$A$是catenary的.事实上任取两个素理想之间的不可延长的素理想严格包含链$p=p_0\subsetneqq p_1\subsetneqq\cdots\subsetneqq p_r=q$,不可延长条件得到每个$\mathrm{ht}(p_{i+1}/p_i)=1,\forall 0\le i\le r-1$.于是条件说明$\mathrm{ht}(p_{i+1})-\mathrm{ht}(p_i)=1$,于是得到链长度$r=\mathrm{ht}(q)-\mathrm{ht}(p)=\mathrm{ht}(q/p)$是固定的.
	\item 上一条的逆命题是不成立的.但是如果$A$是诺特整环,那么从$A$是catenary的得到对任意素理想$p\subseteq q$,有$\mathrm{ht}(q/p)=\mathrm{ht}(q)-\mathrm{ht}(p)$.【】
	\item 设$A$诺特,取全部极小素理想$\{p_i\}$,那么$A$是catenary的当且仅当每个$A/p_i$都是catenary的.事实上任取$A$中两个素理想$p\subseteq q$,为了验证任意它们之间的不可延长的素理想严格包含链的长度固定,只需取$p$内的极小素理想$p_t$,然后把素理想链视为$A/p_t$上即可.
	\item (Ratliff).一个诺特环$A$是universally catenary的当且仅当对任意$A$的素理想$p$,对任意的有限生成的整环之间的单扩张$A/p\subseteq B$都满足维数公式.
	\begin{proof}
		
		必要性.如果$A$是universally catenary的,那么$A/\mathfrak{p}$也是universally catenary的.所以用$A/\mathfrak{p}$替换$A$,我们不妨设$A$是整环.记$B=A[X_1,X_2,\cdots,X_n]/\mathfrak{Q}$.任取$B$的素理想$\mathfrak{P}$,记$\mathfrak{P}=\mathfrak{P}'/\mathfrak{Q}$,其中$\mathfrak{P}'$是$A[X_1,\cdots,X_n]$的素理想.记$\mathfrak{p}=A\cap\mathfrak{P}$,它也是$A\cap\mathfrak{P}'$.因为$A$是universally catenary的,所以$A[X_1,\cdots,X_n]$是catenary的,于是有$\mathrm{ht}(\mathfrak{P})=\mathrm{ht}(\mathfrak{P}')-\mathrm{ht}(\mathfrak{Q})$.另外$A\to A[X_1,\cdots,X_n]$满足维数公式,于是有:
		$$\mathrm{ht}(\mathfrak{P}')+\mathrm{tr.deg}_{\kappa(\mathfrak{p})}\kappa(\mathfrak{P}')=\mathrm{ht}(\mathfrak{p})+\mathrm{tr.deg}_A(A[X_1,\cdots,X_n])$$
		
		类似的,考虑$A[X_1,\cdots,X_n]$的素理想$\mathfrak{Q}$和$A$的素理想$A\cap\mathfrak{Q}=(0)$(这是因为$A\to B=A[X_1,\cdots,X_n]/\mathfrak{Q}$是单射),有维数公式:
		$$\mathrm{ht}(\mathfrak{Q})+\mathrm{tr.deg}_{\kappa(A\cap\mathfrak{Q})}\kappa(\mathfrak{Q})=\mathrm{ht}(A\cap\mathfrak{Q})+\mathrm{tr.deg}_A(A[X_1,\cdots,X_n])$$
		
		这里$\kappa(A\cap\mathfrak{Q})=\mathrm{Frac}(A)$和$\kappa(\mathfrak{Q})=\mathrm{Frac}(B)$,所以上述两个等式就得到:
		$$\mathrm{ht}(\mathfrak{P}')+\mathrm{tr.deg}_{\kappa(\mathfrak{p})}\kappa(\mathfrak{P}')-\mathrm{ht}(\mathfrak{Q})-\mathrm{tr.deg}_A(B)=\mathrm{ht}(\mathfrak{p})$$
		
		这里$\kappa(\mathfrak{P})=\kappa(\mathfrak{P}')=\mathrm{Frac}(B/\mathfrak{P})$和$\mathrm{ht}(\mathfrak{P})=\mathrm{ht}(\mathfrak{P}')-\mathrm{ht}(\mathfrak{Q})$,带入就得到:
		$$\mathrm{ht}(\mathfrak{P})+\mathrm{tr.deg}_{\kappa(\mathfrak{p})}\kappa(\mathfrak{P})=\mathrm{ht}(\mathfrak{p})+\mathrm{tr.deg}_AB$$
		
		充分性.假设$A$不是universally catenary的,那么存在有限型$A$代数$B$不是catenary的.不妨设$B$是整环(事实上我们解释过$A$是universally catenary当且仅当每个$B=A[X_1,\cdots,X_n],n\ge1$是catenary的).设结构同态$A\to B$的核是$\mathfrak{p}$,因为$B$是整环,所以$\mathfrak{p}$是$A$的素理想.因为$B$不是catenary的,所以可设有$B$是素理想$\mathfrak{P}\subsetneqq\mathfrak{Q}$,使得$\mathrm{ht}(\mathfrak{Q}/\mathfrak{P})=d<\mathrm{ht}(\mathfrak{Q})-\mathrm{\mathfrak{P}}$.记$h=\mathrm{ht}(\mathfrak{P})$,那么$d+h<\mathrm{ht}(\mathfrak{Q})$.选取$a_1,\cdots,a_h\in\mathfrak{P}$使得$\mathfrak{P}$是$I=(a_1,\cdots,a_h)\subseteq B$的极小素理想,那么$\mathfrak{P}\in\mathrm{Ass}(B/I)$.记最短准素分解$I=\mathfrak{q}_1\cap\cdots\cap\mathfrak{q}_r$,其中$\sqrt{\mathfrak{q}_1}=\mathfrak{P}$,再取$b\in\mathfrak{Q}\mathfrak{q}_2\cdots\mathfrak{q}_r-\mathfrak{P}$(因为$\mathfrak{Q}\not\subseteq\mathfrak{P}$和$\mathfrak{q}_i\not\subseteq\mathfrak{P},i\ge2$,所以有$\mathfrak{Q}\mathfrak{q}_2\cdots\mathfrak{q}_r\not\subseteq\mathfrak{P}$).
		
		\qquad
		
		我们断言对任意$v\ge1$有$(I:b^vB)=\mathfrak{q}_1$:一方面如果$xb^v\in I$,那么$xb^v\in\mathfrak{q}_1$,由于$b\not\in\mathfrak{P}$,所以$b$的任意次幂都不在$\mathfrak{q}_1$中,所以$x\in\mathfrak{q}_1$(因为$\mathfrak{q}_1$是准素的).于是$(I:b^vB)\subseteq\mathfrak{q}_1$.另一方面,从$b\mathfrak{q}_1\subseteq\mathfrak{q}_1\cdots\mathfrak{q}_r\subseteq I$,得到$\mathfrak{q}_1\subseteq(I:b^vB)$.完成断言的证明.
		
		\qquad
		
		下面取$y_i=a_i/b\in\mathrm{Frac}(B),1\le i\le h$,取$C=B[y_1,\cdots,y_h]\subseteq\mathrm{Frac}(B)$,再取$J=(y_1,y_2,\cdots,y_h)C$和$\mathfrak{M}=J+\mathfrak{Q}C=J+\mathfrak{Q}$(因为有$C=B+J$).我们来证明$J\cap B=\mathfrak{q}_1$:一方面意$z\in J$可以表示为$u/b^k$,其中$u\in I^k$,所以如果$z\in J\cap B$,就有足够大的$v$使得$zb^v\in I$,于是$z\in(I:v^bB)=\mathfrak{q}_1$(上述断言),这说明$J\cap B\subseteq\mathfrak{q}_1$.另一方面有$b\mathfrak{q}_1\subseteq I$,所以$\mathfrak{q}_1\subseteq (1/b)I\subseteq J$,所以$\mathfrak{q}_1\subseteq J\cap B$.综上得到$J\cap B=\mathfrak{q}_1$.
		
		\qquad
		
		于是有$\mathfrak{M}\cap B=(J+\mathfrak{Q})\cap B=(J\cap B)+\mathfrak{Q}=\mathfrak{Q}$.于是有$C/J=(B+J)/J\cong B/\mathfrak{q}_1$和$C/\mathfrak{M}=(B+\mathfrak{M})/\mathfrak{M}\cong B/\mathfrak{Q}$.特别的有$\mathfrak{M}$是$C$的素理想,并且$C_{\mathfrak{M}}/JC_{\mathfrak{M}}\cong B_{\mathfrak{Q}}/\mathfrak{q}_1B_{\mathfrak{Q}}$是$d$维局部环,并且$J$被$h$个元生成,所以$C_{\mathfrak{M}}$的每个包含$JC_{\mathfrak{M}}$的极小素理想的高度$\le h$,于是有$\mathrm{ht}(\mathfrak{M})=\dim(C_{\mathfrak{M}})\le d+h<\mathrm{ht}(\mathfrak{Q})$.
		
		\qquad
		
		现在$B$和$C$具有相同的商域,并且$\kappa(\mathfrak{Q})=\kappa(\mathfrak{M})$.但是如果$A/\mathfrak{p}\to B$和$A/\mathfrak{p}\to C$同时满足维数公式的话,也即:
		$$\mathrm{ht}(\mathfrak{M})+\mathrm{tr.deg}_{\kappa(\mathfrak{M}\cap(A/\mathfrak{p}))}\kappa(\mathfrak{M})=\mathrm{ht}(\mathfrak{M}\cap(A/\mathfrak{p}))+\mathrm{tr.deg}_{A/\mathfrak{p}}C$$
		$$\mathrm{ht}(\mathfrak{Q})+\mathrm{tr.deg}_{\kappa(\mathfrak{Q}\cap(A/\mathfrak{p}))}\kappa(\mathfrak{Q})=\mathrm{ht}(\mathfrak{Q}\cap(A/\mathfrak{p}))+\mathrm{tr.deg}_{A/\mathfrak{p}}B$$
		
		这导致$\mathrm{ht}(\mathfrak{Q})=\mathrm{ht}(\mathfrak{M})$,这矛盾.
	\end{proof}
    \item 特别的,上一条说明一个诺特环$A$是泛catenary环当且仅当对每个极小素理想$\mathfrak{p}$都有$A/\mathfrak{p}$是泛catenary环.这也说明诺特环$A$是泛catenary环当且仅当$A_{\mathrm{red}}=A/\mathrm{nil}(A)$是泛catenary环.
    \item 定理.设$A$是诺特环,那么$A$是泛catenary的当且仅当$A$是catenary的,并且对任意本质有限型(这是指它是有限型代数的分式化)整局部代数$(B',\mathfrak{q},\kappa')$,如果记$\mathfrak{q}$的回拉是素理想$\mathfrak{p}$,记$A_{\mathfrak{p}}$在$B'$中的像是$A'$,记$A'$的剩余域为$\kappa$,那么有维数等式:
    $$\dim A'+\mathrm{tr.deg}_{A'}(B')=\dim B'+\mathrm{tr.deg}_{\kappa}(\kappa')$$
    \item 设诺特环$A$是泛catenary的,那么它的每个分式化$S^{-1}A$都是泛catenary的.因为catenary环的分式化一定是catenary环.逆命题也成立:上一条说明对诺特环$A$,如果任意局部化$A_{\mathfrak{p}}$都是泛catenary的,那么$A$是泛catenary的:取$S=A-\mathfrak{p}$,那么$S^{-1}B$是有限型$A_{\mathfrak{p}}$代数,并且有$B_{\mathfrak{q}}=(S^{-1}B)_{S^{-1}\mathfrak{q}}$.
\end{enumerate}

Rees环.设$A$是环,设$I\subseteq A$是理想,设$t$是未定元,记子环$R_+=R_+(A,I)=\oplus_{n\ge0}I^nt\subseteq A[t]$和环$R=R(A,I)=R_+[t^{-1}]=\{\sum_{n\in\mathbb{Z}}c_nt^n\in A[t,t^{-1}]\mid c_n\in I^n,n\ge0\}$.我们称$R(A,I)$是关于$A$的理想$I$的Rees环.
\begin{enumerate}
	\item 用几何语言讲,$\mathrm{Proj}(R_+)$是$\mathrm{Spec}A$关于闭子概型$V(I)$的爆破(blowup).
	\item 设$A$是环,$I\subseteq A$是理想,记$R_+=R_+(A,I)$和$R=R(A,I)$,那么:
	\begin{enumerate}
		\item 如果$A$是诺特的,则$R_+$和$R$都是诺特的.
		\item 有典范同构$\mathrm{gr}^I(A)\cong R_+/IR_+$.
		\item 有典范同构$R/(1-t^{-1})R\cong A$和$R/t^{-1}R\cong\mathrm{gr}^I(A)$.
	\end{enumerate}
    \begin{proof}
    	
    	第一件事是因为按照$A$诺特,可记$I=(a_1,\cdots,a_r)$,那么$R_+=A[a_1t,a_2t,\cdots,a_rt]$,这是诺特环上有限型代数,所以是诺特的.同样的原因导致$A=A_+[t^{-1}]$是诺特的.
    	
    	\qquad
    	
    	第二件事是因为$R_+/IR_+\cong\oplus_{n\ge0}I^nt/I^{n+1}t$.
    	
    	\qquad
    	
    	第三件事.考虑满同态$A[t,t^{-1}]\to A$为把$t,t^{-1}$均映射为1,它的核是$(1-t)A[t,t^{-1}]=(1-t^{-1})A[t,t^{-1}]$.于是限制在子环上诱导了满的环同态$R/(1-t^{-1})R\to A$.它是单射等价于证明$R\cap(1-t^{-1})A[t,t^{-1}]=(1-t^{-1})R$,右侧包含于左侧是平凡的,反过来如果$f=\sum_{n\in\mathbb{Z}}c_nt^n\in(1-t^{-1})A[t,t^{-1}]$,那么存在$g=\sum_{n\in\mathbb{Z}}d_nt^n$使得$f=(1-t^{-1})g$.那么系数满足$c_n=d_n-d_{n+1}$,于是有$d_n=\sum_{i\ge n}c_i$.当$n$足够大时有$d_n=0$,对$n\ge0$有$d_n\in I^n$,于是$g\in R$,于是$f\in (1-t^{-1})R$.于是我们证明了诱导的$R/(1-t^{-1})R\to A$是同构.对于第二个同构,只需注意到$t^{-1}R=\{\sum_nc_n't^n\in A[t,t^{-1}]\mid c_n'\in I^{n+1},n\ge0\}$,于是有$R/t^{-1}R\cong\oplus_nI^nt/I^{n+1}t\cong\mathrm{gr}^I(A)$.
    \end{proof}
    \item 引理.设$R\to S$是环同态,设$M$是$R$模,$N$是$S$模,并且$N$视为$R$模是平坦的,那么有:
    $$\mathrm{Ass}_S(M\otimes_RN)\supseteq\cup_{\mathfrak{p}\in\mathrm{Ass}_R(M)}\mathrm{Ass}_S(N/\mathfrak{p}N)$$
    
    如果额外的$R$是诺特的,那么这个包含式是取等号的.另外我们总有$\mathrm{Ass}_R(N/\mathfrak{p}N)=\{p\}$,因为$N/\mathfrak{p}N$的伴随素理想至少包含了$\mathfrak{p}$,另一方面由于$N$在$R$上平坦,得到$N/\mathfrak{p}N=N\otimes_AA/\mathfrak{p}$是$A/\mathfrak{p}$上平坦的,但是$A/\mathfrak{p}$是整环,所以$R-\mathfrak{p}$中的元就是$N/\mathfrak{p}N$上的正则元,导致$N/\mathfrak{p}N$的伴随素理想不能包含$R-\mathfrak{p}$中的元,于是伴随素理想只有$\mathfrak{p}$本身.
    \begin{proof}
    	
    	设$\mathfrak{p}\in\mathrm{Ass}_R(M)$,那么有$R$模的单同态$R/\mathfrak{p}\to M$,因为$N$是平坦$R$模,于是有单同态$N/\mathfrak{p}N\cong(R/\mathfrak{p})\otimes_RN\to M\otimes_RN$.于是有$\mathrm{Ass}_S(N/\mathfrak{p}N)\subseteq\mathrm{Ass}_S(M\otimes_RN)$.这证明了命题的前半部分.
    	
    	\qquad
    	
    	现在设$R$是诺特环,把$M$写作所有有限$R$子模$\{M_i,i\in I\}$的并,于是$M\otimes_RN=\cup_iM_i\otimes_RN$,$\mathrm{Ass}_R(M)=\cup_{i\in I}\mathrm{Ass}_R(M_i)$和$\mathrm{Ass}_S(M\otimes_RN)=\cup_{i\in I}\mathrm{Ass}_R(M_i\otimes_RN)$.所以归结为$M$是有限$R$模的情况.
    	
    	\qquad
    	
    	任取$\mathfrak{q}\in\mathrm{Ass}_S(M\otimes_RN)$,记$\mathfrak{p}=R\cap\mathfrak{q}$.我们断言$\mathfrak{p}\in\mathrm{Ass}_R(M)$:有$\mathfrak{q}S_{\mathfrak{q}}\in\mathrm{Ass}_{S_{\mathfrak{q}}}(M\otimes_RN_{\mathfrak{q}})=\mathrm{Ass}_{S_{\mathfrak{q}}}(M_{\mathfrak{p}}\otimes_{R_{\mathfrak{p}}}S_{\mathfrak{q}})$.而如果$\mathfrak{p}\not\in\mathrm{Ass}_R(M)$,就有$\mathfrak{p}R_{\mathfrak{p}}\not\in\mathrm{Ass}_{R_{\mathfrak{p}}}(M_{\mathfrak{p}})$.又因为$M_{\mathfrak{p}}$的零因子集合恰好就是$\cup_{\mathfrak{p}'\in\mathrm{Ass}_{R_{\mathfrak{p}}}(M_{\mathfrak{p}})}\mathfrak{p}'$.所以可不妨设$\mathfrak{p}R_{\mathfrak{p}}$包含了非零因子,否则的话有唯一极大理想$\mathfrak{p}R_{\mathfrak{p}}$恰好是全体零因子集合,按照素理想avoidance引理,导致这唯一极大理想恰好是一个伴随素理想,就和$\mathfrak{p}R_{\mathfrak{p}}\not\in\mathrm{Ass}_{R_{\mathfrak{p}}}(M_{\mathfrak{p}})$矛盾.假设有$x\in\mathfrak{p}R_{\mathfrak{p}}$不是$M_{\mathfrak{p}}$的零因子,那么按照$N_{\mathfrak{q}}$在$R_{\mathfrak{p}}$上平坦,得到$x$在$\mathfrak{q}S_{\mathfrak{q}}$中的像也不是$M_{\mathfrak{p}}\otimes_{R_{\mathfrak{p}}}N_{\mathfrak{q}}$的零因子,这就和$\mathfrak{q}S_{\mathfrak{q}}\in\mathrm{Ass}_{S_{\mathfrak{p}}}(M_{\mathfrak{p}}\otimes_{R_{\mathfrak{p}}}N_{\mathfrak{q}})$相矛盾.于是我们完成了断言$\mathfrak{p}\in\mathrm{Ass}_R(M)$的证明.
    	
    	\qquad
    	
    	因为$R$是诺特环,$M$是有限$R$模,所以存在$M$的滤过$0=M_0\subsetneqq M_1\subsetneqq\cdots\subsetneqq M_r=M$.使得每个$M_i/M_{i-1}\cong A/\mathfrak{p_i}$,其中$\mathfrak{p}_i\in\mathrm{Spec}(R)$.此时有$\mathrm{Ass}_R(M)\subseteq\{\mathfrak{p}_1,\cdots\mathfrak{p}_r\}$.由于$N$是平坦$R$模,所以有$M\otimes_RN$的如下滤过:
    	$$0=M_0\otimes_RN\subsetneqq M_1\otimes_RN\subsetneqq\cdots\subsetneqq M_r\otimes_RN=M\otimes_RN$$
    	
    	这里商是$(M_i\otimes_RN)/(M_{i-1}\otimes_RN)\cong N/\mathfrak{p}_iN$.这就导致$\mathfrak{q}\in\mathrm{Ass}_S(M\otimes_RN)\subsetneqq\cup_{i=1}^r\mathrm{Ass}_S(N/\mathfrak{p}_iN)$.取$\mathfrak{q}\in\mathrm{Ass}_S(M\otimes_RN)$,设$\mathfrak{q}\in\mathrm{Ass}_S(N/\mathfrak{p}_iN)$,记$\mathfrak{p}=R\cap\mathfrak{q}$,那么我们前面证明了$\mathfrak{p}\in\mathrm{Ass}_R(R/\mathfrak{p}_i)$,迫使$\mathfrak{p}=\mathfrak{p}_i$,于是我们证明了另一侧的包含关系:
    	$$\mathrm{Ass}_S(M\otimes_RN)\subseteq\cup_{\mathfrak{p}\in\mathrm{Ass}_R(M)}\mathrm{Ass}_S(N/\mathfrak{p}N)$$
    	
    \end{proof}
    \item 我们在前文承诺过证明这样一件事:设$A$是诺特环,设$I=(a_1,\cdots,a_r)$是真理想,设$R=R(A,I)$,设$G=\mathrm{gr}^I(A)$,那么有$\dim R=\dim A+1$和$\dim G\le\dim A$.如果额外的$A$是局部环,那么有$\dim G=\dim A$.
    \begin{proof}
    	
    	设$\mathfrak{a}\subseteq A$是理想,记$\mathfrak{a}'=\mathfrak{a}A[t,t^{-1}]\cap R$,也即:
    	$$\mathfrak{a}'=\{\sum_nc_nt^n\in A[t,t^{-1}]\mid c_n\in I^n\cap\mathfrak{a},n\ge0,c_n\in\mathfrak{a},n<0\}$$
    	
    	那么有$R/\mathfrak{a}'=(A/\mathfrak{a})[t^{-1},\overline{a_1}t,\cdots\overline{a_r}t]=R(A/\mathfrak{a},I(A/\mathfrak{a}))$.其中$\overline{a_i}$是$a_i$在$A/\mathfrak{a}$的像.并且有:
    	\begin{enumerate}
    		\item 有$\mathfrak{a}'\cap A=\mathfrak{a}$,于是对$A$的两个不同的理想$\mathfrak{a}_1\not=\mathfrak{a}_2$,就有$\mathfrak{a}_1'\not=\mathfrak{a}_2'$.
    		\item 如果$\mathfrak{p}$是$A$是素理想,那么$\mathfrak{p}'$是$R$的素理想:因为有同构$A[t,t^{-1}]/\mathfrak{p}A[t,t^{-1}]\cong(A/\mathfrak{p})[t,t^{-1}]$所以是整环,而$R/\mathfrak{p}'$是这个同构的环的子环.
    		\item 如果$\mathfrak{q}$是$A$的准素理想,那么$\mathfrak{q}'$是$R$的准素理想:按照上述引理,有$\mathfrak{q}A[t,t^{-1}]\subseteq A[t,t^{-1}]$是准素理想:因为$\mathfrak{q}$是$A$的准素理想得到$\mathrm{Ass}_A(A/\mathfrak{q})=\{\mathfrak{p}\}$,而按照引理有$\mathrm{Ass}_{A[t,t^{-1}]}(A/\mathfrak{q}\otimes_AA[t,t^{-1}])=\cup_{\mathfrak{p}'\in\mathrm{Ass}_A(A/\mathfrak{q})}\mathrm{Ass}_{A[t,t^{-1}]}(A[t,t^{-1}]/\mathfrak{p}'A[t,t^{-1}])=\mathrm{Ass}_{A[t,t^{-1}]}(A[t,t^{-1}]/\mathfrak{p}A[t,t^{-1}])=\{\mathfrak{p}A[t,t^{-1}]\}$.另外准素理想的原像还是准素的,所以如果$(0)=\mathfrak{q}_1\cap\cdots\cap\mathfrak{q}_n$是$A$的零理想的准素分解,那么有$(0)=\mathfrak{q}_1'\cap\cdots\cap\mathfrak{q}_n'$是$R$的零理想的准素分解.
    	\end{enumerate}
    
        结合这里(a)和(c),说明如果$\mathfrak{p}_{0i},1\le i\le m$是$A$的全部极小素理想,那么$\mathfrak{p}_{0i}',1\le i\le m$是$R$的全部极小素理想.另外如果$A$的素理想$\mathfrak{p}$的高度是$h$,取长度$h$的素理想链$\mathfrak{p}=\mathfrak{p}_0\supsetneqq\cdots\supsetneqq\mathfrak{p}_h$,于是(a)和(b)说明有$R$的长度$h$的素理想链$\mathfrak{p}'=\mathfrak{p}_0'\supsetneqq\cdots\supsetneqq\mathfrak{p}_h'$.这说明$\mathrm{ht}(\mathfrak{p})\le\mathrm{ht}(\mathfrak{p}')$,取上确界得到$\dim A\le\dim R$.反过来取$R$的素理想$\mathfrak{P}$,记$\mathfrak{p}=\mathfrak{P}\cap A$.记$\mathfrak{p}_{0i}'$是$\mathfrak{P}$包含的全部的满足$\mathrm{ht}(\mathfrak{P})=\mathrm{ht}(\mathfrak{P}/\mathfrak{p}_{0i}')$的极小素理想,按照$(a)$有$\mathfrak{p}_{0i}=\mathfrak{p}_{0i}'\cap A$,于是$A/\mathfrak{p}_{0i}\to R/\mathfrak{p}_{0i}'$是单射.按照维数不等式有:
        \begin{align*}
        	\mathrm{ht}(\mathfrak{P})=\mathrm{ht}(\mathfrak{P}/\mathfrak{p}_{0i}')&\le\mathrm{ht}(\mathfrak{p}/\mathfrak{p}_{0i})+\mathrm{tr.deg}_{A/\mathfrak{p}_{0i}}(R/\mathfrak{p}_{0i}')-\mathrm{tr.deg}_{\kappa(\mathfrak{p})}\kappa(\mathfrak{P})\\&\le\mathrm{ht}(\mathfrak{p}/\mathfrak{p}_{0i})+1-\mathrm{tr.deg}_{\kappa(\mathfrak{p})}\kappa(\mathfrak{P})\\&\le\mathrm{ht}(\mathfrak{p})+1
        \end{align*}
    
        于是有$\dim R\le\dim A+1$.反过来有$A[t,t^{-1}]=R[t]=R[1/t^{-1}]$是$R$的局部化,所以有$\dim R\ge\dim A[t,t^{-1}]\ge\dim A+1$,于是有$\dim R=\dim A+1$.另一方面,我们解释过$G\cong R/t^{-1}R$,于是有$\dim G=\dim R/t^{-1}R$.由于$t^{-1}\in R$不是零因子,说明它不包含在任一极小素理想中(极小素理想由零因子构成),所以包含$t^{-1}R$的极小素理想的高度至少为1,这导致$\dim R/t^{-1}R\le\dim R-1=\dim A$.至此我们证明了命题的第一部分.
        
        \qquad
        
        现在证明命题的第二部分.对$A$的素理想$\mathfrak{p}=(a_1,a_2,\cdots,a_r)$,记$\alpha_i=\overline{a_i}\in A/\mathfrak{p}$,我们之前证明了$R/\mathfrak{p}'=(A/\mathfrak{p})[t^{-1},\alpha_1t,\cdots,\alpha_rt]$.于是有$\kappa(\mathfrak{p}')=\kappa(\mathfrak{p})(t)$和$\mathrm{tr.deg}_{\kappa(\mathfrak{p})}\kappa(\mathfrak{p}')=1$.考虑$\mathfrak{p}$和$\mathfrak{p}'$的维数不等式有:
        $$\mathrm{ht}(\mathfrak{p}')=\mathrm{ht}(\mathfrak{p}'/\mathfrak{p}_{0i}')\le\mathrm{ht}(\mathfrak{p}/\mathfrak{p}_{0i})+\mathrm{tr.deg}_{A/\mathfrak{p}_{0i}}R/\mathfrak{p}_{0i}'-\mathrm{tr.deg}_{\kappa(\mathfrak{p})}\kappa(\mathfrak{p}')\le\mathrm{ht}(\mathfrak{p})$$
        
        结合之前证明的$\mathrm{ht}(\mathfrak{p})\le\mathrm{ht}(\mathfrak{p}')$,就得到这两个高度实际相等.现在取包含$I$的极大理想$m\subseteq A$,那么有$R/m'=(A/m)[t^{-1}]$(这里$\overline{a_i}$被$m$零化了).所以$\mathfrak{M}=(m',t^{-1})$是$R$的极大理想,并且$\mathfrak{M}\not=m'$,于是$\mathrm{ht}(\mathfrak{M})>\mathrm{ht}(m')$,于是$\mathrm{ht}(\mathfrak{M})\ge\mathrm{ht}(m)+1$.另一方面按照维数不等式有$\mathrm{ht}(\mathfrak{M})\le\mathrm{ht}(m)+1=\mathrm{ht}(m')+1$,于是实际上这些不等式都取等的:$\mathrm{ht}(\mathfrak{M})=\mathrm{ht}(m)+1=\mathrm{ht}(m')+1$.下面断言$\mathrm{ht}(\mathfrak{M}/t^{-1}R)=\mathrm{ht}(\mathfrak{M})-1$:一方面按照$t^{-1}$是$R$的非零因子,它不会包含在极小素理想中,于是有$\mathrm{ht}(\mathfrak{M}/t^{-1}R)\le\mathrm{ht}(\mathfrak{M})-1$.反过来如果$\delta(R_{\mathfrak{M}}/t^{-1}R_{\mathfrak{M}})=n$,也即存在$n$个元在$R_{\mathfrak{M}}/t^{-1}R_{\mathfrak{M}}$模去后是有限长度的,所以这$n$个元带上$t^{-1}$会使得$R_{\mathfrak{M}}$在模去它们后是有限长度的,于是有$\mathrm{ht}(\mathfrak{M})=\delta(R_{\mathfrak{M}})\le n+1=\mathrm{ht}(\mathfrak{M}/t^{-1}R)+1$,综上得到断言成立.进而有$\mathrm{ht}(\mathfrak{M}/t^{-1}R)=\mathrm{ht}(m)$.所以如果存在包含$I$的极大理想$m$使得$\dim A=\mathrm{ht}(m)$(局部环条件保证这一点),那么有$\dim G=\dim R/t^{-1}R\ge\mathrm{ht}(\mathfrak{M}/t^{-1}R)=\mathrm{ht}(m)=\dim A$.这就得到$\dim G=\dim A$.
    \end{proof}
\end{enumerate}
\newpage
\section{正则序列}
\subsection{正则序列与Koszul复形}

正则序列.设$M$为$A$模,一个元$a\in A$称为$M$正则的,如果$a$不是$M$的零因子,换句话讲左乘$a$是$M$上的单射.$A$上的一个序列$\{a_1,a_2,\cdots,a_n\}$称为$M$正则序列,如果$a_1$是$M$正则的,$a_i$是$M/(a_1,a_2,\cdots,_{i-1})M$正则的,这里$2\le i\le n$,并且$M/(a_1,a_2,\cdots,a_n)M\not=0$.
\begin{enumerate}
	\item 注意,一个$M$正则序列改变次序后未必是$M$正则序列.例如设$k$是域,取$A=k[X,Y,Z]$,取$a_1=X(Y-1)$,$a_2=Y$,$a_3=Z(Y-1)$.那么$\{a_1,a_2,a_3\}$是$A$正则序列,但是$\{a_1,a_3,a_2\}$不是,因为$a_3=Z(Y-1)$是$k[X,Y,Z]/X(Y-1)$的零因子.
	\item 设$M$为$A$模,若$\{a_1,a_2,\cdots,a_n\}$是$M$正则序列,任取正整数$\mu_1,\mu_2,\cdots\mu_n$,那么$\{a_1^{\mu_1},a_2^{\mu_2},\cdots,a_n^{\mu_n}\}$也是$M$正则序列.
	\begin{proof}
		
		按照归纳法,只需证明如果$\{a_1,\cdots,a_n\}$是$M$正则序列,那么对正整数$v$,有$\{a_1^v,a_2,\cdots,a_n\}$是$M$正则序列.我们先断言如果$\{b_1,\cdots,b_n\}$是$M$正则序列,并且有$\xi_i\in M$使得$\sum_ib_i\xi_i=0$,那么每个$\xi_i\in(b_1,\cdots,b_n)M$.为此我们对$n$归纳.如果$n=1$,有$b_1\xi_1=0$,因为$b_1$是$M$正则元,导致$\xi_1=0\in b_1M$.下面设$n>1$,因为$b_n$是$M/(b_1,\cdots,b_{n-1})M$正则的,所以$n=1$的情况得到$\xi_n\in(b_1,\cdots,b_{n-1})M$,所以有$\xi_n=b_1\eta_1+\cdots+b_{n-1}\eta_{n-1}$,其中$\eta_i\in M$.那么有$b_1(\xi_1+b_n\eta_1)+\cdots+b_{n-1}(\xi_{n-1}+b_n\eta_{n-1})=0$.按照归纳假设,得到每个$\xi_i+b_n\eta_i\in(b_1,\cdots,b_{n-1})M$,于是得到每个$\xi_i\in(b_1,\cdots,b_n)M$证明了我们的断言.
		
		\qquad
		
		下面设$v>1$,我们对$v$归纳证明$\{a_1^v,a_2,\cdots,a_n\}$仍然是$M$正则序列.因为$a_1$是$M$正则元,所以$a_1^v$也是$M$正则元.假设存在某个$i>1$,存在$\omega\in M$和$\xi_j\in M$使得$a_i\omega=a_1^v\xi_1+a_2\xi_2+\cdots+a_{i-1}\xi_{i-1}$.按照归纳假设有$\{a_1^{v-1},a_2,\cdots,a_n\}$是$M$正则序列,所以按照我们的断言就得到$\omega\in(a_1^{v-1},a_2,\cdots,a_{i-1})M$,于是存在$\eta_1,\cdots,\eta_{i-1}\in M$使得$\omega=a_1^{v-1}\eta_1+a_2\eta_2+\cdots+a_{i-1}\eta_{i-1}$.于是有$0=a_1^{v-1}(a_1\xi_1-a_i\eta_1)+a_2(\xi_2-a_i\eta_2)+\cdots+a_{i-1}(\xi_{i-1}-a_i\eta_{i-1})$.再用一次断言得到$a_1\xi_1-a_i\eta_1\in(a_1^{v-1},a_2,\cdots,a_{i-1})M$,所以$a_i\eta_1\in(a_1,\cdots,a_{i-1})M$.再用断言的$n=1$的情况,又得到$\eta_1\in(a_1,\cdots,a_{i-1})M$,这就得到$\omega\in(a_1^v,a_2,\cdots,a_{i-1})M$.于是$a_i$是$M/(a_1^v,a_2,\cdots,a_{i-1})M$的正则元,证毕.
	\end{proof}
\end{enumerate}

拟正则序列.设$a_1,a_2,\cdots,a_n\in A$,记理想$I=(a_1,a_2,\cdots,a_n)$满足$M/IM\not=0$,称$\{a_1,a_2,\cdots,a_n\}$是拟正则序列,如果它满足如下等价描述中的任一条:
\begin{enumerate}
	\item 对任意自然数$v$和任意$v$次齐次元$F\in M[X_1,X_2,\cdots,X_n]=M\otimes_AA[X_1,X_2,\cdots,X_n]$(失去乘法),从$F(a_1,a_2,\cdots,a_n)\in I^{v+1}M$推出$F\in IM[X_1,X_2,\cdots,X_n]$.
	\item 对任意自然数$v$和任意$v$次齐次元$F\in M[X_1,X_2,\cdots,X_n]=M\otimes_AA[X_1,X_2,\cdots,X_n]$,从$F(a_1,a_2,\cdots,a_n)=0$推出$F\in IM[X_1,X_2,\cdots,X_n]$.
\end{enumerate}
\begin{proof}
	
	1推2平凡.现在假设2成立,若$F(a_1,a_2,\cdots,a_n)\in I^{v+1}M$,那么存在$v+1$次齐次多项式$G\in M[X_1,X_2,\cdots,X_n]$满足$G(a_1,a_2,\cdots,a_n)=F(a_1,a_2,\cdots,a_n)$.我们把$G$写作$G(X_1,X_2,\cdots,X_n)=\sum_{i=1}^nX_iG_i(X_1,X_2,\cdots,X_n)$,这里每个$G_i$都是$M[X_1,X_2,\cdots,X_n]$中的$v$次齐次多项式.取$F'=F-\sum_{i=1}^na_iG_i$,那么$F'(a_1,a_2,\cdots,a_n)=0$,从条件得到$F'(X_1,X_2,\cdots,X_n)\in IM[X_1,X_2,\cdots,X_n]$.这就得到$F\in IM[X_1,X_2,\cdots,X_n]$.
\end{proof}

另一个等价描述.设$M$为$A$模,取$a_1,a_2,\cdots,a_n\in A$,记$I=(a_1,a_2,\cdots,a_n)$,取$\mathrm{gr}^I(M)=\oplus_{m\ge0}I^mM/I^{m+1}M$.我们可以构造$M[X_1,X_2,\cdots,X_n]\to\mathrm{gr}^I(M)$的(交换群)同态为把每个$m$次齐次元$F(X_1,X_2,\cdots,X_n)$映射为$F(a_1,a_2,\cdots,a_n)$在$I^mM/I^{m+1}M$中的像.这是一个满射,并且$IM[X_1,X_2,\cdots,X_n]$必然落在这个同态的核中,于是诱导了一个交换群同态$\varphi:(M/IM)[X_1,X_2,\cdots,X_n]\to\mathrm{gr}^I(M)$.那么按照定义,序列$\{a_1,a_2,\cdots,a_n\}$是拟正则序列当且仅当$\varphi$是单射,当且仅当$\varphi$是同构.

设$M$是$A$模,$I=(a_1,a_2,\cdots,a_n)$是理想
\begin{enumerate}
	\item 若$\{a_1,a_2,\cdots,a_n\}$是$M$拟正则序列,若$x\in A$满足$(IM:x)=IM$,那么对每个正整数$v$总有$(I^vM:x)=I^vM$.
	\begin{proof}
		
		对$v$归纳.$v=1$没什么需要证的.下设$v>1$,一方面必然有$I^vM\subset(I^vM:x)$.现在任取$\xi\in(I^vM:x)$,于是$x\xi\in I^vM\subseteq I^{v-1}M$.于是归纳假设说明$\xi\in I^{v-1}M$.于是存在$v-1$次的$M[X_1,X_2,\cdots,X_n]$中的齐次元$F$使得$\xi=F(a_1,a_2,\cdots,a_n)$.于是$x\xi=xF(a_1,a_2,\cdots,a_n)\in I^vM$.按照拟正则序列的定义,得到$xF(X_1,X_2,\cdots,X_n)\in IM$.于是$F(X_1,X_2,\cdots,X_n)\in(IM:x)=IM$.于是$\xi=F(a_1,a_2,\cdots,a_n)\in I^vM$.
	\end{proof}
	\item 若$\{a_1,a_2,\cdots,a_n\}$是$M$正则序列,那么它是$M$拟正则序列.
	\begin{proof}
		
		对$n$归纳.$n=1$的情况,要证明$M$正则元$a_1$是拟正则的,需要验证的是如果$M[X_1]$中的$v$次齐次元$F(X_1)$满足$F(a_1)=0$,要验证$F\in a_1M[X_1]$.整理一下条件就是$F(X_1)=mX_1^v$,$m\in M$满足$a_1^vm=0$.但是按照$a_1$不是$M$的零因子,就得到$m=0$,于是此时$F=0\in a_1M[X_1]$.
		
		现在设$n>1$,假设命题对$\le n-1$均成立.给定$M$正则序列$\{a_1,a_2,\cdots,a_n\}$,按照归纳假设,已经得到$\{a_1,a_2,\cdots,a_{n-1}\}$是拟正则序列.现在任取$M[X_1,X_2,\cdots,X_n]$中的$v$次齐次元$F$,设它满足$F(a_1,a_2,\cdots,a_n)=0$,需要验证的是$F[X_1,X_2,\cdots,X_n]\in IM[X_1,X_2,\cdots,X_n]$.我们设$F(X_1,X_2,\cdots,X_n)=G(X_1,X_2,\cdots,X_{n-1})+X_nH(X_1,X_2,\cdots,X_n)$.于是这里$G$是$v$次齐次元,$H$是$v-1$次齐次元.于是有$a_nH(a_1,a_2,\cdots,a_n)=-G(a_1,a_2,\cdots,a_{n-1})\in I_1^vM$.这里$I_1=(a_1,a_2,\cdots,a_{n-1})$.于是有$H(a_1,a_2,\cdots,a_n)\in(I_1^vM:a_n)$.现在我们断言$(I_1M:a_n)=I_1M$:右侧包含于左侧平凡,左侧包含于右侧是因为,任取左侧中的元$m$,那么$a_nm\in I_1M$,在$\mod I_1M$下$a_n$不能是零因子,于是导致在模去$I_1M$后$m=0$,此即$m\in I_1M$.
		
		于是按照归纳假设,以及第一条,得到$(I_1^vM:a_n)=I_1^vM$.于是$H(a_1,a_2,\cdots,a_n)\in I_1^vM\subseteq I^vM$.这里对$v$归纳可得$H(X_1,X_2,\cdots,X_n)\in IM[X_1,X_2,\cdots,X_n]$.按照$H(a_1,a_2,\cdots,a_n)\in I_1^vM$,于是可取$v$次齐次元$h\in M[X_1,X_2,\cdots,X_{n-1}]$使得$H(a_1,a_2,\cdots,a_n)=h(a_1,a_2,\cdots,a_{n-1})$.现在取$g(X_1,X_2,\cdots,X_{n-1})=G(X_1,X_2,\cdots,X_{n-1})-a_nh(X_1,X_2,\cdots,X_{n-1})$,这是一个$v$次齐次元,并且$g(a_1,a_2,\cdots,a_n)=0$,于是归纳假设得到$g(X_1,X_2,\cdots,X_{n-1})\in IM[X_1,X_2,\cdots,X_{n-1}]$.这就得到$G(X_1,X_2,\cdots,X_n)$的系数都落在$IM$中.
	\end{proof}
\end{enumerate}

按照定义,拟正则序列的定义不依赖于序列的排列,而我们给出过反例说明正则序列依赖于序列的排列.于是一般来讲拟正则序列不会是正则序列.一个自然的问题是添加什么条件能够保证拟正则序列成为正则序列.
\begin{enumerate}
	\item 先来看一个元的情况.如果$\{a\}$是拟正则的,要想它是正则的,需要满足对任意的$\xi\in M$,从$a\xi=0$推出$\xi=0$.拟正则性说明$a\xi=0$推出$F=\xi X$落在$IM[X]$中,其中$I=(a)$,于是$\xi=a\xi'$.于是$\xi'a^2=0$,而拟正则性又推出$F=\xi'X^2$落在$IM[X]$中,于是$\xi=a\xi'\in I^2M$中.归纳得到如果$\{a\}$是拟正则的,并且$a\xi=0$,那么有$\xi\in\cap_{n\ge1}(a)^nM$.于是为了$\{a\}$是正则序列,可以要求$\cap_{n\ge1}a^nM=\{0\}$.这个条件即$(a)-$adic可分性.
	\item $M$是非零$A$模,取$a_1,a_2,\cdots,a_n\in A$,记$I=(a_1,a_2,\cdots,a_n)$满足
	$$M,M/(a_1M),\cdots,M/(a_1,a_2,\cdots,a_{n-1})M$$
	
	都是$I-$adic可分的.那么如果$\{a_1,a_2,\cdots,a_n\}$是拟正则序列,就有它是正则序列.
	\begin{proof}
		
		设$\{a_1,a_2,\cdots,a_n\}$是满足条件的拟正则序列.按照上一段的讨论得到$a_1$是$M$正则的(仅仅把上一段中$a^nM$都改为$I^nM$).于是下面仅需证明$\{a_2,a_3,\cdots,a_n\}$是$M/a_1M=M_1$的正则序列.对$n$做归纳假设,于是仅需验证$\{a_2,a_3,\cdots,a_n\}$是拟正则序列.也即任取$M_1[X_2,X_3,\cdots,X_n]$中的$w$次齐次元$f$,满足$f(a_2,a_3,\cdots,a_n)=0$,证明$f\in IM_1[X_2,X_3,\cdots,X_n]$.
		
		记$f$在$M[X_2,X_3,\cdots,X_n]$中的代表元是$F$,这里$F$依然是$w$次的齐次元.于是$F(a_2,a_3,\cdots,a_n)\in a_1M$.于是可记$F(a_2,a_3,\cdots,a_n)=a_1\omega$,这里$\omega\in M$.我们接下来的目标是期望$\omega$可以表示为$G(a_1,a_2,\cdots,a_n)$,其中$G$是$w-1$次齐次多项式,这等价于讲$\omega\in I^{w-1}M$.一旦这件事得证,那么$G_0=X_1G$就是$w$次齐次多项式,导致$F(a_2,a_3,\cdots,a_n)=a_1\omega$这个等式就变成$F(X_2,\cdots,X_n)-G_0(X_1,X_2,\cdots,X_n)$在$(a_1,a_2,\cdots,a_n)$处取值为零.于是按照$a_1,a_2,\cdots,a_n$是$M$的拟正则序列,得到$F-G_0\in IM[X_1,X_2,\cdots,X_n]$.但是这里$G_0$涉及到的单项式全部包含$X_1$,而$F$涉及到的单项式全部不包含$X_1$.于是得到$F\in IM[X_2,X_3,\cdots,X_n]$.于是$f\in IM_1[X_2,X_3,\cdots,X_n]$.
		
		最后我们证明上一段中断言的$\omega\in I^{w-1}M$.我们设$\omega\in I^mM,m<w-1$,现在证明$\omega\in I^{m+1}M$,于是从$\omega\in I^0M=M$,反复运用这个结论就得到$\omega\in I^{w-1}M$.现在记$\omega=G(a_1,a_2,\cdots,a_n)$,其中$G$是一个$M[X_1,X_2,\cdots,X_n]$中的$m$次齐次多项式.记$G_0=X_1G$,那么$G_0(a_1,a_2,\cdots,a_n)=a_1\omega=F(a_2,a_3,\cdots,a_n)\in I^wM$,于是从$\{a_1,a_2,\cdots,a_n\}$的拟正则性得到$G_0\in IM[X_1,X_2,\cdots,X_n]$,于是$G\in IM[X_1,X_2,\cdots,X_n]$.换句话讲$G$的系数仍然可以提出$I$中的元,于是得到$\omega\in I^{m+1}M$.注意这件事至多可以做到$m=w-1$.
	\end{proof}
	\item 注意如下几个条件都能推出上述定理中$I-$adic可分性这个条件.于是在下述任一条件下,$\{a_1,a_2,\cdots,a_n\}$是拟正则序列推出它是正则序列.另外还说明满足如下任一条件的正则序列是不依赖于序列排列的.
	\begin{enumerate}
		\item $A$是诺特环,$I\subset\mathrm{Rad}(A)$,$M$是有限$A$模,那么$M$上满足$I-$adic可分性.事实上取$M'=\cap_{n\ge1}a^nM$,它是有限$A$模并且$IM'=M'$,由NAK引理得$M'=0$.
		\item 上一条特例.设$(A,m)$是诺特局部环,$M$是有限$A$模,取理想$I\subseteq m$,那么$M$是$I-$adic可分的.
		\item 如果$A$是分次环,$M$是分次$A$模,设$a_1,a_2,\cdots,a_n$是正次的齐次元,记$I=(a_1,a_2,\cdots,a_n)$,那么$M$总是$I-$adic可分的.事实上,设$a_i$中次数最低的是$N>0$次,那么有$I^nM\subset\oplus_{m\ge nN}M_m$,而后者对$n$取交只有零元.
	\end{enumerate} 
\end{enumerate}

回顾复形的张量积.给定两个环$R$上的复形$K_*$和$L_*$.它们的张量积$K_*\otimes L_*$定义为一个$R$上的复形.满足$(K_*\otimes L_*)_n=\oplus_{p+q=n}(K_p\otimes L_q)$.边界算子$\partial$定义为,它把$K_p\otimes L_q$中的元素$x\otimes y$,映射为$(K_*\otimes L_*)_{n-1}$中的$\partial(x)\otimes y+(-1)^px\otimes\partial(y)$.

Koszul复形.设$A$为环,取$x_1,x_2,\cdots,x_n\in A$,定义关于$A$和$\{x_1,x_2,\cdots,x_n\}$的Koszul复形如下,记作$K_*(x_1,x_2,\cdots,x_n)$或者简单记作$K_*(\underline{x})$.
\begin{enumerate}
	\item 约定$K_p=0,p<0$或$p>n$;约定$K_0=A$;约定$K_p=\wedge^p(A^n)$,如果取$A^n$的标准基$\{e_1,e_2,\cdots,e_n\}$,那么形式元素$\{e_{i_1}\wedge e_{i_2}\wedge\cdots\wedge e_{i_p}\mid 1\le i_1<i_2<\cdots<i_p\le n\}$构成了$K_p$的一组基,于是$K_p$是秩为$\left(\begin{array}{c} n\\p\end{array}\right)$的自由$A$模.
	\item 定义边界算子$d_p:K_p\to K_{p-1}$,首先对于$d_1$约定它把$e_i\mapsto x_i,\forall i$.对于$p>1$,定义$e_{i_1}\wedge\cdots\wedge e_{i_p}\mapsto\sum_j(-1)^{j-1}x_{i_j}e_{i_1}\wedge\cdots\wedge\widehat{e_{i_j}}\wedge\cdots\wedge e_{i_p}$.容易验证$d_{p-1}\circ d_p=0$.
	\item 一些记号.设$M$为$A$模,记$K_*(\underline{x},M)=K_*(\underline{x})\otimes_AM$;对$A$模的复形$C_*$,记$C_*(\underline{x})=C_*\otimes K_*(\underline{x})$;记同调$H_p(\underline{x},M)=H_p(K_*(\underline{x},M))$.
\end{enumerate}

一些性质.
\begin{enumerate}
	\item 最简单的情况,取$x\in A$,那么$K_*(x)$即如下复形,它唯一的非零映射是$A$上的数乘$x$的模同态.
	$$\xymatrix{\cdots\ar[r]&0\ar[r]&A\ar[r]^{\cdot x}&A\ar[r]&0\ar[r]&\cdots}$$
	\item 设$x_1,x_2\in A$,那么总有$K_*(x_1)\otimes K_*(x_2)\cong K_*(x_1,x_2)$.
	$$\xymatrix{0\ar[r]&Ae\otimes A\xi\ar[r]\ar[d]&(Ae\otimes A)\oplus(A\otimes A\xi)\ar[r]\ar[d]&A\otimes A=A\ar[r]\ar[d]&0\\0\ar[r]&A\alpha_1\wedge A\alpha_2\ar[u]\ar[r]&A\alpha_1\oplus A\alpha_2\ar[r]\ar[u]&A\ar[r]\ar[u]&0}$$
	\item 更一般地,$K_*(x_1)\otimes K_*(x_2)\otimes\cdots\otimes K_*(x_n)=K_*(x_1,x_2,\cdots,x_n)$.
	\item 
	\begin{align*}
	K_*(\underline{x})&:A(e_1\wedge\cdots\wedge e_n)\to\cdots\to\oplus_{i=1}^nAe_i\to A\to0\\K_*(\underline{x})\otimes M&:M(e_1\wedge\cdots\wedge e_n)\to\cdots\to\oplus_{i=1}^nMe_i\to M\to0
	\end{align*}
	
	于是$H_0(\underline{x},M)=M/(x_1,x_2,\cdots,x_n)M$;$H_n(\underline{x},M)=\{m\in M\mid x_im=0,\forall 1\le i\le n\}$.
	\item 设$(C_*,d_*)$是一个$A$模的复形,取$x\in A$.考虑$C_*(x)=C_*\otimes K_*(x)$.其中$C_*(x)_n=\oplus_{p+q=n}C_p\otimes K_q(x)=C_n\oplus C_{n-1}$.边界算子$d_n:C_*(x)_n\to C_*(x)_{n-1}$为把$(\xi,\eta)\in C_n\oplus C_{n-1}$映射为$(d\xi+(-1)^{n-1}x\eta,d\eta)$.考虑投影映射$C_n(x)=C_n\oplus C_{n-1}\to C_{n-1}$,这得到如下交换图表:
	$$\xymatrix{C_*:\ar[r]&C_{n+1}\ar[r]\ar[d]&C_n\ar[r]\ar[d]&C_{n-1}\ar[r]\ar[d]&\cdots\\C_*(x):\ar[r]&C_{n+1}(x)\ar[r]\ar[d]&C_n(x)\ar[r]\ar[d]&C_{n-1}(x)\ar[r]\ar[d]&\cdots\\C_*(-1):\ar[r]&C_n\ar[r]&C_{n-1}\ar[r]&C_{n-2}\ar[r]&\cdots}$$
	
	这个交换图表的每一列实际上是短正合列,于是这实际上是三个复形的短正合列$0\to C_*\to C_*(x)\to C_*(-1)\to0$.它诱导了如下长正合列:
	$$\xymatrix{\cdots\ar[r]&H_p(C_*)\ar[r]&H_p(C_*(x))\ar[r]&H_{p-1}(C_*)=H_p(C_*(-1))\ar[r]^{\qquad\alpha_{p-1}}&H_{p-1}(C_*)\ar[r]&H_{p-1}(C_*(x))}$$
	
	这里的$\alpha_{p-1}$实际上就是数乘$(-1)^{p-1}x$.另外对任意$p$总有$xH_p(C_*(x))=0$.事实上这等价于讲$x\ker d_p\subset\mathrm{im}d_{p+1}$.为此任取$(\xi,\eta)\in\ker d_p$,那么有$d\xi+(-1)^{p-1}x\eta=0$和$d\eta=0$.于是我们取$(0,(-1)^{p}\xi)$,就得到$d_{n+1}(0,(-1)^{p-1}\xi)=(x\xi,x\eta)$.
	\item 更一般的,设$x_1,x_2,\cdots,x_n\in A$,$M$为$A$模,那么总有$(x_1,x_2,\cdots,x_n)H_p(x_1,x_2,\cdots,x_n,M)=0$.
\end{enumerate}

和正则序列的关系.
\begin{enumerate}
	\item $M$为$A$模,记$x_1,x_2,\cdots,x_n\in A$是$M$正则序列,那么定义说明$H_0(\underline{x},M)=M/(x_1,x_2,\cdots,x_n)M\not=0$.另外对任意的$p\ge1$总有$H_p(\underline{x},M)=0$.
	\begin{proof}
		
		$n=1$的时候,$H_1(x,M)=\{\xi\in M\mid x\xi=0\}$,从$x$是$M$正则的定义得到$H_1(x,M)=0$.下面设$n\ge2$,假设对$n-1$的情况命题成立,那么有$K_*(x_1,x_2,\cdots,x_n,M)\cong K_*(x_1,x_2,\cdots,x_{n-1},M)\otimes K_*(x_n)=C_*(x_n)$,其中$C_*=K_*(x_1,x_2,\cdots,x_{n-1},M)$.按照我们之前给出的复形的短正合列:
		$$\xymatrix{0\ar[r] K_*(x_1,x_2,\cdots,x_{n-1},M)\ar[r] K_*(x_1,x_2,\cdots,x_{n-1},M)(x_n)\ar[r] K_*(x_1,x_2,\cdots,x_{n-1},M)(-1)\ar[r]&0}$$
		
		它诱导了长正合列.考虑:
		$$\xymatrix{0=H_p(x_1,\cdots,x_{n-1},M)\ar[r]&H_p(\underline{x},M)\ar[r]&H_{p-1}(x_1,\cdots,x_{n-1},M)=0}$$
		
		得到$p\ge2$的时候$H_p(\underline{x},M)=0$.再考虑:
		$$\xymatrix{0=H_1(x_1,x_2,\cdots,x_{n-1},M)\ar[r]&H_1(\underline{x},M)\ar[r]&H_0(x_1,x_2,\cdots,x_{n-1},M)\ar[r]& H_0(x_1,x_2,\cdots,x_{n-1},M)}$$
		
		其中最后一个同态事实上就是$M/(x_1,x_2,\cdots,x_{n-1})M$上的数乘$x_n$的映射,按照正则序列的定义得到它是单射,于是上述图表的正合性得到$H_1(\underline{x},M)=0$.
	\end{proof}
	\item 上一条的逆命题不成立,因为比方说这个定义没有依赖于$x_i$的排列次序.但是如果附加如下条件的任意一个,那么从$H_1(\underline{x},M)=0$可以推出$\{x_1,x_2,\cdots,x_n\}$是$M$正则序列.
	\begin{itemize}
		\item $(A,m)$是诺特局部环,$M$是有限$A$模,$\{x_1,x_2,\cdots,x_n\}\subseteq m$.
		\item $A$是诺特环,$M$是有限$A$模,$(x_1,x_2,\cdots,x_n)\subset\mathrm{Rad}(A)$.
		\item $A$是分次环,$M$是有限分次$A$模,$x_1,x_2,\cdots,x_n$都是次数为正的齐次元.
	\end{itemize}
	\begin{proof}
		
		$n=1$时候,如果$H_1(x,M)=\{\xi\in M\mid x\xi=0\}=0$,
		
		这得到$x$是$M$正则的.现在假设$n\ge2$,并且$n-1$的情况命题成立.
		那么从$H_1(\underline{x},M)=0$,我们期望推出$H_1(x_1,x_2,\cdots,x_{n-1},M)=0$.这就从归纳假设得到$\{x_1,x_2,\cdots,x_{n-1}\}$是$M$正则序列.
		最后仅需验证$x_n$是$M/(x_1,x_2,\cdots,x_{n-1})M$正则的.为此从长正合列得到
		$0=H_1(\underline{x},M)\to H_0(x_1,x_2,\cdots,x_{n-1},M)\to H_0(x_1,x_2,\cdots,x_{n-1},M)$.这里最后一个同态是单的,但是我们之前说明这个最后的同态就是数乘$x_n$,于是$x_n$是
		$M/(x_1,x_2,\cdots,x_{n-1})M=H_0(x_1,x_2,\cdots,x_{n-1},M)$正则的,完成证明.
		
		最后我们说明三个额外条件的任意一个可以推出刚刚所提的期望证明的命题.按照复形的短正合列诱导的长正合列,我们得到:
		$$H_1(x_1,x_2,\cdots,x_{n-1},M)\to H_1(x_1,x_2,\cdots,x_{n-1},M)\to H_1(\underline{x},M)=0$$
		
		我们之前说过这里第一个映射是数乘$-x_n$,正合性得到它是满射.
		
	    在条件1和2下,记$H_1(x_1,x_2,\cdots,x_{n-1},M)=N$,那么$N$是有限$A$模,并且$x_n\in\mathrm{Rad}(A)$,满射说明$x_nN=N$,于是NAK引理得到$N=0$.在条件3下,从$\deg x_n>0$和$x_nN=N$只能得到$N=0$.
	\end{proof}
\end{enumerate}

关于投射维数的补充.设$M$为$A$模,设$d\in\mathbb{N}$,若$\mathrm{Ext}_A^(M,N)=0$对任意$i\ge d+1$和任意$A$模$N$成立,就称$M$的投射维数$\le d$.如果还能取到$A$模$N_0$使得$\mathrm{Ext}_A^d(M,N_0)\not=0$,就称$M$的投射维数为$d$,记作$\mathrm{proj.dim}(M)=d$.
\begin{enumerate}
	\item 一个同调代数引理.对$A$模$M$,取投射模$P$满足存在满同态$P\to M$,记它的核为$M'$,于是有短正合列$0\to M'\to P\to M\to0$.这诱导了长正合列:
	$$0\to\mathrm{Hom}(M,N)\to\mathrm{Hom}(P,N)\to\mathrm{Hom}(M',N)\to\mathrm{Ext}_A^1(M,N)\to\mathrm{Ext}_A^1(P,N)\to\mathrm{Ext}_A^1(M',N)\to\cdots$$
	
	按照$\mathrm{Ext}_A^i(P,N)=0,\forall i\ge1$,得到$\mathrm{Ext}^i(M',N)=\mathrm{Ext}^{i+1}(M,N),\forall i\ge1$.
	
	更一般的,如果有正合列$0\to M'\to P_s\to\cdots\to P_0\to M\to0$,其中$P_i$都是投射模,那么总有$\mathrm{Ext}^i(M',N)=\mathrm{Ext}^{i+s+1}(M,N),\forall i\ge1$成立.
	\item $M$为$A$模,那么$\mathrm{proj.dim}(M)\le d$当且仅当$M$存在投射预解$P_*\to M$满足$P_i=0,\forall i\ge d$.
	\begin{proof}
		
		充分性是平凡的,对这样的投射预解作用$\mathrm{Hom}$函子,取同调得到$\mathrm{Ext}_A^i(M,N)=0$,$\forall i\ge d+1$和$\forall A$模$N$.
		
		必要性.任取$M$的投射预解$\to P_{d+1}\to P_d\to\cdots\to P_0\to M\to0$,记$P_d\to P_{d-1}$为$\alpha$,那么有正合列$0\to\mathrm{im}\alpha\to P_{d-1}\to\cdots\to P_0\to M\to0$.那么上面引理得到$\mathrm{Ext}^i(\mathrm{im}\alpha,N)\cong\mathrm{Ext}^{i+d}(M,N),\forall i\ge1$.按照条件对任意$i\ge1$和任意$A$模$N$有$\mathrm{Ext}^i(\mathrm{im}\alpha,N)=0$.这得到$\mathrm{im}\alpha$是投射模,于是上面给出的正合列已经是$M$的投射预解,得证.
	\end{proof}
	\item 上一命题中还等价于$\mathrm{Ext}^{d+1}_A(M,N)=0$对任意$A$模$N$成立.
\end{enumerate}

接下来给出$M$正则序列和$\mathrm{Ext}$的联系.首先对理想$I\subseteq A$,和元$a_1,a_2,\cdots,a_r\in I$,称$\{a_1,a_2,\cdots,a_r\}$是$I$中的极大$M$正则序列,如果它本身是$M$正则序列,并且对任意的$b\in I$,都有$\{a_1,a_2,\cdots,a_n,b\}$不是$M$正则序列.
\begin{enumerate}
	\item 如果$A$是诺特环,那么对任意理想$I$和其中的任意一组$M$正则序列,总可以延拓为$I$中的极大$M$正则序列.事实上如果$\{a_1,a_2,\cdots,a_r\}$是落在$I$中的$M$正则序列,那么$a_r$是$M/(a_1,a_2,\cdots,a_{r-1})M$上的单射,特别的有$(a_1,a_2,\cdots,a_{r-1})M\subsetneqq(a_1,a_2,\cdots,a_r)M$(否则这个映射是零映射,不会是单的).于是必然有$(a_1,a_2,\cdots,a_{r-1})\subsetneqq(a_1,a_2,\cdots,a_r)$.按照$A$是诺特环,得到$I$是有限生成的,于是延长正则序列的长度的操作不会任意次的进行下去.
	\item 若$A$诺特,$M$是有限$A$模,$I\subseteq A$是理想,满足$IM\not=M$,给定正整数$n$,如下命题是等价的:
	\begin{enumerate}
		\item $\mathrm{Ext}_A^i(N,M)=0$对任意$i<n$和任意满足$\mathrm{Supp}(N)\subseteq V(I)$的有限$A$模$N$成立.
		\item $\mathrm{Ext}_A^i(A/I,M)=0$对任意$i<n$成立.
		\item 存在某个满足$\mathrm{Supp}(N)=V(I)$的有限$A$模$N$,使得$\mathrm{Ext}_A^i(N,M)=0$对任意$i<n$成立.
		\item 存在$I$中的长度为$n$的$M$正则序列.
	\end{enumerate}
	\begin{proof}
		
		(a)$\Rightarrow$(b)$\Rightarrow$(c)是平凡的.先证明(c)$\Rightarrow$(d),首先说明$I$中存在$M$正则元素.若否,那么$I$完全由$M$的零因子构成,于是$I$落在$M$的全部伴随素理想的并中.按照$M$是诺特环上的有限模,得到$M$的伴随素理想集合$\mathrm{Ass}(M)$是有限集,于是$I$包含在某个伴随素理想$p$中.按照定义存在单的模同态$A/P\to M$.局部化的正合性得到存在单同态$\kappa(p)=(A/p)_p\to M_p$.这说明$\mathrm{Hom}_{A_p}(\kappa(p),M_p)\not=0$.按照条件$p\in V(I)=\mathrm{Supp}(N)$,于是$N_p\not=0$.于是NAK引理得到$N_p\otimes_{A_p}\kappa(p)\not=0$(如果$A$局部环,$M,N$是有限$A$模,那么$M\otimes_AN=0$当且仅当$M$或者$N$是零).于是$\mathrm{Hom}_{\kappa(p)}(N_p\otimes_{A_p}\kappa(p),\kappa(p))\not=0$(因为它是非零线性空间$N_p\otimes_{A_p}\kappa(p)$的对偶空间).综上我们构造了这样一个非平凡的同态$N_p\to N_p\otimes_{A_p}\kappa(p)=N_p/(pA_p)N_p\to\kappa(p)\to M_p$.于是得到$\mathrm{Hom}_{A_p}(N_p,M_p)\not=0$.在诺特条件下有$\mathrm{Hom}_{A_p}(N_p,M_p)\cong(\mathrm{Hom}_A(N,M))_p$.这和3的条件矛盾,于是$I$中存在$M$正则元.
		
		\qquad
		
		于是$n=1$得证.下面设$n\ge2$,取$a\in I$是一个$M$正则元,考虑短正合列$0\to M\to M\to M'\to0$.这里$M\to M$取做数乘$a$的模同态,这里$M'=M/aM$.那么这个短正合列诱导了长正合列$\cdots\to\mathrm{Ext}^i(N,M)\to\mathrm{Ext}^i(N,M)\to\mathrm{Ext}^i(N,M')\to\mathrm{Ext}^{i+1}(N,M)\to\cdots$.那么有$\mathrm{Ext}^i(N,M')=0$对任意$i<n$成立.于是归纳假设知$M'=M/aM$中存在长度为$n-1$的$M'$正则序列$\{a_1,a_2,\cdots,a_{n-1}\}$,于是$\{a,a_1,a_2,\cdots,a_{n-1}\}$是$M$正则序列.
		
		\qquad
		
		最后证明(d)$\Rightarrow$(a),对$n$归纳,设$I$中有$M$正则序列$\{a_1,a_2,\cdots,a_n\}$.记$M'=M/a_1M$,那么有短正合列$0\to M\to M\to M'\to0$,这里$M\to M$是数乘$a_1$的映射.那么$M'$中存在长度为$n-1$的正则序列.按照归纳假设,有$\mathrm{Ext}^i(N,M')=0$对任意的$i<n-1$和任意满足条件的$N$成立(对$n=1$这是平凡的).于是对任意$i<n$有长正合列中的$f:\mathrm{Ext}^i(N,M)\to\mathrm{Ext}^i(N,M)$的由数乘$a_1$诱导的同态是单的.下面按照条件有$V(\mathrm{Ann}(N))=\mathrm{Supp}(N)\subseteq V(I)$,于是$a_1\in I\subset\sqrt{\mathrm{Ann}(N)}$.于是有某个次幂$a_i^uN=0$.但是$f$就是数乘$a_1$的映射(写出同调的定义).于是$f$的某个次幂是零映射,结合它是单射,得到$\mathrm{Ext}^i(N,M)=0,\forall i<n$.
	\end{proof}
	\item $A$是诺特环,$I$是理想,$M$是有限$A$模,满足$IM\not=M$,那么$I$中极大的$M$正则序列的长度$r$恰好就是$\inf\{i\in\mathbb{N}\mid \mathrm{Ext}^i(A/I,M)\not=0\}$.这个命题还可以把$A/I$替换为任意的满足$\mathrm{Supp}(N)=V(I)$的有限$A$模$N$.
	\begin{proof}
		
		按照上一定理,有$\mathrm{Ext}^i_A(N,M)=0$对任意$i<r$成立.于是我们仅需验证$\mathrm{Ext}^r_A(N,M)\not=0$.取$M_i=M/(a_1,a_2,\cdots,a_i)M$,那么有短正合列$0\to M_{i-1}\to M_{i-1}\to M_i\to0$,这里$M_{i-1}\to M_{i-1}$取为数乘$a_i$的模同态.这诱导了正合列$\mathrm{Ext}^j_A(N,M_{i-1})\to\mathrm{Ext}^j_A(N,M_i)\to\mathrm{Ext}^{j+1}_A(N,M_{i-1})$.由于$a_i,a_{i+1},\cdots,a_r$是$M_{i-1}$正则序列,于是上一定理得到$\mathrm{Ext}_A^j(N,M_{i-1})$在$j<r-i+1$即$j\le r-i$时为零.于是$\mathrm{Ext}^{r-i}_A(N,M_i)\to\mathrm{Ext}^{r-i+1}_A(N,M_{i-1})$是单射对任意$i$成立.于是得到$\mathrm{Hom}_A(N,M_r)\to\mathrm{Ext}_A^r(N,M)$是单射.由于$\{a_1,a_2,\cdots,a_n\}$是$I$中极大的$M$正则序列,于是$I$中不含$M_r$的零因子,于是上一定理得到此时$\mathrm{Hom}_A(N,M_r)\not=0$,于是单射得到$\mathrm{Ext}^r_A(N,M)\not=0$,得证.
	\end{proof}
\end{enumerate}

模的深度(depth).给定诺特环$A$,理想$I$,有限$A$模$M$,
\begin{enumerate}
	\item 如果$IM\not=M$,称$\inf\{i\in\mathbb{N}\mid\mathrm{Ext}^i_A(A/I,M)\not=0\}$为$M$的$I$深度,此即$I$中所含的$M$正则序列的长度的极大值.记作$\mathrm{depth}(I,M)$.
	\item 若$(A,m)$是局部诺特环,$I=m$,称$\mathrm{depth}(m,M)$为$M$的深度,记作$\mathrm{depth}(M)$.
	\item 约定$IM=M$时$\mathrm{depth}(I,M)=\infty$.
\end{enumerate}

下面给出一些性质.
\begin{enumerate}
	\item 设$A$是诺特环,$M$是有限$A$模.
	\begin{enumerate}
		\item 如果$I,J$均为$A$的理想,满足$\sqrt{I}=\sqrt{J}$,那么$\mathrm{depth}(I,M)=\mathrm{depth}(J,M)$.
		\item 记$A'=A/\mathrm{Ann}(M)$,记$I'=IA'$,那么$\{a_1,a_2,\cdots,a_n\}\subseteq I$是$M$正则序列当且仅当$\{a_1',a_2',\cdots,a_n'\}\subseteq I'$是$M$正则序列.
	\end{enumerate}
	\item 设$A$诺特,记$I=(y_1,y_2,\cdots,y_n)$,$M$是有限$A$模,$IM\not=M$,记$q=\sup\{i\mid H_i(\underline{y},M)\}$,那么$\mathrm{depth}(I,M)=n-q$.
	\begin{proof}
		
		设$I$中极大$M$正则序列为$\{x_1,x_2,\cdots,x_r\}$,于是$\mathrm{depth}(I,M)=r$.对$r$归纳,若$r=0$,则$I$中没有$M$的正则元,那么$I$完全由$M$的零因子构成,我们之前说过此时有$I$落在某个伴随素理想$p\in\mathrm{Ass}(M)$中.我们之前还给出过$H_n(\underline{y},M)=\{\xi\in M\mid y_i\xi=0,\forall i\}$.按照伴随素理想定义有$\xi\in M$使得$\mathrm{Ann}(\xi)=p\supset I$.于是$H_n(\underline{y},M)\not=0$.另外$i>n$的时候总有$H_i(\underline{y},M)=0$.于是此时公式成立.
		
		下面设$r\ge1$.考虑短正合列$0\to M\to M\to M_1\to0$,这里$M\to M$是数乘$x_1$的模同态,$M_1=M/x_1M$.于是有复形的短正合列$0\to K_*(\underline{y},M)\to K_*(\underline{y},M)\to K_*(\underline{y},M_1)\to0$.这诱导了正合列$H_{i+1}(\underline{y},M_1)\to H_i(\underline{y},M)\to H_i(\underline{y},M)\to H_i(\underline{y},M_1)\to H_{i-1}(\underline{y},M)\to$.按照归纳假设,有$\mathrm{depth}(I,M_1)=r-1$,于是$\sup\{i\mid H_i(\underline{y},M_i)\not=0\}=n-r+1$.于是当$i+1\ge(n-r+1)+1$时有$H_{i+1}(\underline{y},M_1)=0$.于是$H_{n-r+1}(\underline{y},M)\to H_{n-r}(\underline{y},M)$是单的.于是$H_{n-r}(\underline{y},M)\not=0$,于是$\mathrm{depth}(I,M)=r$.完成归纳.
	\end{proof}
	\item 设$A$是诺特环,$I=(y_1,y_2,\cdots,y_n)$是理想,$M$是$A$模,满足$IM\not=M$,若如下条件之一成立,那么$\{y_1,y_2,\cdots,y_n\}$是$M$正则序列当且仅当$\mathrm{depth}(I,M)=n$.
	\begin{enumerate}
		\item $A$诺特并且$I\subset\mathrm{Rad}(A)$,$M$是有限$A$模.
		\item $A$是分次诺特环,$M$是有限的分次$A$模,并且$y_1,y_2,\cdots,y_n$都是次数为正的齐次元.
	\end{enumerate}
	\begin{proof}
		
		按照上一定理,$\mathrm{depth}(I,M)=n$当且仅当$q=0$,当且仅当$H_i(\underline{y},M)=0,\forall i>0$.我们之前证明过这在1和2的条件下等价于讲$\underline{y}$是$M$正则序列.
	\end{proof}
    \item 局部诺特环$(A,m)$上的有限模$M$的深度为0当且仅当$m\in\mathrm{Ass}_A(M)$.
    \begin{proof}
    	
    	充分性是直接的,此时$m$中每个元都是零因子,必然不存在正则元.必要性,深度0意味着不存在正则元,于是$m$中每个元都零化$M$中某个元,按照$\mathrm{Ann}(x),x\in M$的极大元必然是伴随素理想,说明全部伴随素理想的并就是整个$m$,但是诺特环上有限模的伴随素理想只有有限个,于是$p_1\cup p_2\cup\cdots\cup p_r=m$,按照prime avoidance引理,得到$m$就是某个$p_i$.
    \end{proof}
\end{enumerate}
\newpage
\subsection{Cohen-Macaulay环}

首先我们解释模去正则序列后模的维数恰好减去正则元的个数.若$(A,m)$是诺特局部环,$M$是有限$A$模,$a_1,a_2,\cdots,a_r\in m$是$M$正则序列,记$M'=M/(a_1,a_2,\cdots,a_r)M$,那么有$\dim(M)=\dim(M')+r$.
\begin{proof}
	
	归结为$r=1$的情况.如果$a$是$M$正则元,$M'=M/aM$,在有限模条件下$\dim M=\dim A/\mathrm{Ann}(M)$,以及$\dim M'=\dim A/(\mathrm{Ann}(M)+(a))$,按照$a$是正则元,它不会落在$\mathrm{Ann}(M)$的极小素理想中(因为极小素理想都是伴随素理想,其中的元肯定不是正则元),这说明$V(\mathrm{Ann}(M)+(a))$的素理想链必然可以补上$\mathrm{Ann}(M)$的某个极小素理想构成$V(\mathrm{Ann}(M))$的素理想链,因此$\dim M'\le\dim M-1$.
	
	我们接下来证明对于一般的诺特局部环$(A,m)$和一般的元$a\in m$,总有不等式$\dim A/(a)\ge\dim A-1$,这就得到另一侧的不等式(通过把$A$替换为$A/\mathrm{Ann}(M)$).我们只要证明对$A$中的严格包含的素理想链$p_0\subseteq p_1\subset\cdots\subseteq p_n=m$,总存在$A$中的严格包含素理想链$p_0'\subseteq p_1'\subset\cdots\subseteq p_n'=m$.这个不等式说明总有$n\le\dim A/(a)+1$,取上确界就得到要证的$\dim A\le\dim A/(a)+1$.
	
	对$n$归纳.首先$n=1$的时候自动是满足要求的素理想链.下面设$n\ge2$.倘若$a\in p_{n-1}$,做$p_{n-1}$处的局部化,这得到了长度为$n-1$的$A_{p_{n-1}}$中的严格包含素理想链,按照归纳假设,可以找到长度$n-1$的$A_{p_{n-1}}$中的素理想链,满足$a/1$落在要求的素理想中,将素理想链回拉至$A$上,这就得到满足要求的素理想链.再设$a\not\in p_{n-1}$,取$p_{n-2}+aA$的极小素理想$p_{n-1}'$,按照Krull主理想定理,有$p_{n-1}'/p_{n-2}$是$A/p_{n-2}$的高度1的素理想,于是特别的$p_{n-1}'\not=m$,于是$p_{n-2}\subseteq p_{n-1}'\subseteq m$是严格包含的长度2的素理想链,满足上一情况的要求,归纳得证.
\end{proof}

设$(A,m)$是局部诺特环,设$M,N$均为非零的有限$A$模,记$k=\mathrm{depth}(M)$,也即$M$的极大正则序列的长度,注意局部环的情况下正则序列必然落在$m$中.再设$r=\dim N=\dim(\mathrm{Supp}(N))$.那么$\mathrm{Ext}^i(N,M)=0,\forall i<k-r$.
\begin{proof}
	
	对于$r=0$,此时$\mathrm{Supp}(N)=\{m\}$.我们之前证明过存在$m$中长度为$k$的$M$正则序列当且仅当对任意支集落在$V(m)=\{m\}$的模$N$总有$\mathrm{Ext}^i(N,M)=0$,$\forall i<n$.此即$r=0$结论成立.
	
	下面假设$r>0$,对于诺特环上有限模$N$,总存在滤过$N=N_0\supset N_1\supset\cdots\supset N_n=0$,满足每个$N_i/N_{i+1}\cong A/p_i$,其中$p_i\in\mathrm{Spec}(A)$.那么这里的素理想列$p_i$包含了$N$的极小素理想,于是有$\mathrm{Supp}(N)=\cup_{i=1}^nV(p_i)$,并且$\dim N=\max_i\{\dim A/p_i\}$.特别地,$\dim A/p_i\le\dim N$.下面考虑$A$模的短正合列$0\to N_{i+1}\to N_i\to A/p_i\to0$.它诱导了同调的正合列$\mathrm{Ext}^j(A/p_i,M)\to\mathrm{Ext}^j(N_i,M)\to\mathrm{Ext}^j(N_{i+1},M)$.倘若我们可以证明命题对$A/p_i$型的模成立,也即$\mathrm{Ext}^j(A/p_i,M)=0$对$j<k-\dim A/p_i$成立.那么对全部$i$统一就有$\mathrm{Ext}^j(A/p_i,M)=0$对$j<k-r$成立.接下来从$N_0=0$得到$\mathrm{Ext}^j(N_1,M)=0$对$j<k-r$成立.继续的$\mathrm{Ext}^j(N_2,M)=0$对$j<k-r$成立.最后得到$\mathrm{Ext}^i(N,M)=0$,$\forall i<k-r$成立.
	
	于是我们只需证明$N=A/p$的情况,其中$p\in\mathrm{Spec}(A)$.如果$p=m$,此为$r=0$的情况已经证明过了.于是设$p\subsetneqq m$,于是可取$x\in m\backslash p$.考虑如下短正合列$0\to N\to N\to N'\to0$.其中$N\to N$取为数乘$x$的映射,并且$M'=M/xM$.此时有$\dim N'+1=\dim N$:按照$\dim N=\dim V(p)$和$\dim N'=\dim V(p+(x))$.按照$V(p+(x))$的极大长度的严格包含的素理想链再添加$p$得到$V(p)$中的素理想链,于是$\dim N'+1\le\dim N$(这里实际上可说明$\dim N'+1=\dim N$).下面考虑诱导的长正合列的一部分$\mathrm{Ext}^j(N',M)\to\mathrm{Ext}^j(N,M)\to\mathrm{Ext}^j(N,M)\to\mathrm{Ext}^{j+1}(N',M)$.这里按照归纳假设有$j<k-r$时有两端的模都是零模,这里$r=\dim N$.于是在$j<k-r$的时候有$x\mathrm{Ext}^j(N,M)=\mathrm{Ext}^j(N,M)$成立.这里$x\in m$,于是只需说明$\mathrm{Ext}^j(N,M)$有限生成,按照NAK引理就得到$\mathrm{Ext}^j(N,M)=0$,$\forall j<k-r$.
	
	引理:$A$诺特,$M,N$是有限$A$模,那么$\mathrm{Ext}^j(M,N)$总是有限生成模.这个只要把$N$的投射预解取为有限生成的自由预解即可.
\end{proof}

推论.若$A$是诺特局部环,$M\not=0$是有限$A$模,取$p\in\mathrm{Ass}(M)$,那么$\dim A/p\ge\mathrm{depth}(M)$.特别地,总有$\dim M\ge\mathrm{depth}(M)$.
\begin{proof}
	
	按照条件有单射$A/p\to M$,于是$\mathrm{Hom}_A(A/p,M)\not=0$,上一定理结论是$\mathrm{Ext}^i(A/p,M)=0$对每个$i<\mathrm{depth}(M)-\dim(A/p)$成立.这就导致$\mathrm{depth}(M)-\dim(A/p)\le0$.
\end{proof}

CM模和CM环.设$(A,m)$是诺特局部环,设$M$是有限$A$模,称$M$是Cohen-Macaulay环,或者简称CM环,如果$M=0$或者$M\not=0$并且$\dim M=\mathrm{depth}(M)$.如果$A$作为自身模是CM模,就称$A$是CM环.下面设$(A,m)$是诺特局部环,
\begin{enumerate}
	\item $\dim A=0$时,此即阿廷环,此时环总是CM环.因为深度不超过维数,如果维数零此时只能取等.
	\item $\dim A=1$时,如果$A$既约,那么$A$是CM环.事实上此时有$\mathrm{depth}(A)\le\dim A=1$.而$\mathrm{depth}(A)=0$等价于极大理想$m$不含$A$的零因子,由此得到$A$是既约环:如果可取非平凡幂零元$x\in m$,设最小的正整数$n$使得$x^n=0$,导致$x\in m$是$A$正则元,这矛盾.
	\item $\dim A=2$,并且$A$是正规整环,那么$A$是CM环.
	\begin{proof}
		
		我们要证明的是$\mathrm{depth}(A)=2$.首先$m\not=0$否则此时维数零,于是可取$f\in m-\{0\}$,整环条件就得到$f$是$A$正则元.下面只需找到$A/fA$的一个正则元,就证明了$\mathrm{depth}(A)\ge2$,又有$\mathrm{depth}(A)\le\dim A=2$,就得到$A$是CM环.
		
		倘若能证明$m\not\in\mathrm{Ass}_A(A/fA)$,那么$fA$的极小素理想的高度都是1,设全部极小素理想为$\{p_1,p_2,\cdots,p_s\}$,那么$m-\cup_{i=1}^sp_i$非空,其中任意元$x$都是$A/fA$的正则元,因为模的零因子必然落在模的某个极小素理想中.
		
		假设$m\in\mathrm{Ass}_A(A/fA)$,那么按照伴随素理想的定义,存在$g\in A\backslash fA$,使得$mg\subseteq fA$,于是在$\mathrm{Frac}(A)$中有$m\frac{g}{f}\in A$.倘若$m\frac{g}{f}=A$,那么有$x\in m$使得$x\frac{g}{f}=1$,于是$m\subseteq x\frac{g}{f}m\subseteq Ax$.但是按照主理想定理有$\mathrm{ht}(m)\le1$,这和$\mathrm{ht}(m)=2$矛盾.因而必然有$m\frac{g}{f}\subseteq m$,于是$\frac{g}{f}$是$A$上整元,按照正规条件得到$\frac{g}{f}\in A$,这就和$g\in A\backslash fA$矛盾.
	\end{proof}
\end{enumerate}

设$(A,m)$是诺特局部环,$M$是有限$A$模,
\begin{enumerate}
	\item 若$M$是CM模,那么$\forall p\in\mathrm{Ass}(M)$,总有$\dim A/p=\dim M=\mathrm{depth}(M)$.换句话讲,$M$支集上的所有不可约分支的维数都相同.特别地,CM模没有嵌入素理想.
	\begin{proof}
		
		一方面我们总有$\dim M\ge\dim A/p\ge\mathrm{depth}(M)$.而$CM$条件是指$\dim M=\mathrm{depth}(M)$.于是$\forall p\in\mathrm{Ass}(M)$总有$\dim A/p=\dim M$,并且$p$是极小素理想(否则嵌入素理想相比极小素理想,高度必然变大).
	\end{proof}
	\item 若$a_1,a_2,\cdots,a_r\in m$是$M$正则序列,记$M'=M/(a_1,a_2,\cdots,a_r)M$,那么$M$是CM模当且仅当$M'$是CM模.
	\begin{proof}
		
		我们仅需验证$\dim M=\dim M'+r$以及$\mathrm{depth}(M)=\mathrm{depth}(M')+r$.后者是平凡的,因为深度的定义等价于极大正则序列的长度.前者我们在开篇证明过.
	\end{proof}
	\item 若$M$是CM模,取$p\in\mathrm{Spec}(A)$,那么$M_p$是CM的$A_p$模,另外若$M_p\not=0$(即$p$在$M$的支集中),那么有$\mathrm{depth}(pA_p,M_p)=\mathrm{depth}(p,M)$,它记作$\mathrm{depth}(M_p)$.
	\begin{proof}
		
		不妨设$M_p\not=0$,否则零模我们只是约定了为CM模.首先总有$\dim M_p\ge\mathrm{depth}(M_p)$.如果$a_1,a_2,\cdots,a_r\in p$是$M$正则序列,按照$M_p=M\otimes_AA_p$是平坦$A$模.于是$a_1',a_2',\cdots,a_r'\in pA_p$是$M_p$的正则序列.这说明了$\mathrm{depth}(pA_p,M_p)\ge\mathrm{depth}(p,M)$.于是我们仅需验证$M_p\not=0$时有$\dim M_p=\mathrm{depth}(p,M)$.
		
		对$\mathrm{depth}(p,M)$归纳.首先如果它为零.那么$p$完全由$M$的零因子构成.而我们知道零因子必然落在某个伴随素理想里,于是有$p\subset\cup_{q\in\mathrm{Ass}(M)}p$.这导致$p\subset$某个伴随素理想$q$.但是由1知$M$不存在嵌入素理想,它的伴随素理想全都是支集中的极小素理想,而$p$按照定义取的支集中的元,于是$p$自身恰好是一个极小素理想,于是$\mathrm{Supp}(M_p)=\{pA_p\}$,于是$\dim M_p=0$.
		
		下设$\mathrm{depth}(p,M)>0$.可取$a\in p$是$M$正则元,考虑短正合列$0\to M\to M\to M'=M/aM\to0$.这里$M\to M$是数乘$a$的映射.由2知$M'$是CM模,并且$\mathrm{depth}(M')=\mathrm{depth}(M)-1$.按照局部化的平坦性,$a$也是$M_p$中的正则元.按照归纳假设,有$\dim M'_p=\mathrm{depth}(p,M')$,并且$M'_p$也是CM模.于是$\dim M_p=\dim M'_p+1=\mathrm{depth}(p,M')+1=\mathrm{depth}(p,M)$.完成归纳.
	\end{proof}
\end{enumerate}

设$(A,m)$是诺特局部环,并且是CM环.
\begin{enumerate}
	\item 取$a_1,a_2,\cdots,a_r\in m$,那么如下条件两两等价:
	\begin{enumerate}
		\item $a_1,a_2,\cdots,a_r$是$A$正则序列.
		\item $\mathrm{ht}(a_1,a_2,\cdots,a_i)=i,\forall 1\le i\le r$.
		\item $\mathrm{ht}(a_1,a_2,\cdots,a_r)=r$.
		\item $a_1,a_2,\cdots,a_r$是$A$上参数系统的一部分.
	\end{enumerate}

    整理一下,如果$A$只是诺特局部环,如果它存在一组参数系统恰好是正则序列,按照定义就有$A$是CM环,并且此时$A$的参数系统等价于极大正则序列.
	\begin{proof}
		
		我们来证明$(a)\Rightarrow(b)\Rightarrow(c)\Rightarrow(d)\Rightarrow(a)$,这里前三个箭头没有用到CM条件.
		
		$(a)\Rightarrow(b)$.由Krull主理想定理,得到$\mathrm{ht}(a_1,a_2,\cdots,a_i)\le i$.下面证明$\mathrm{ht}(a_1,a_2,\cdots,a_{i-1})<\mathrm{ht}(a_1,a_2,\cdots,a_i),\forall i$.事实上按照$a_i$是$A/(a_1,a_2,\cdots,a_{i-1})$的正则元,得到$a_i$不在$(a_1,a_2,\cdots,a_{i-1})$的任一极小素理想中(零因子必然落在某个极小素理想中).于是包含$(a_1,a_2,\cdots,a_i)$的素理想必然不是$(a_1,a_2,\cdots,a_{i-1})$的极小素理想,于是高度有严格小于关系.最后从$0<\mathrm{ht}(a_1)<\mathrm{ht}(a_1,a_2)<\cdots<\mathrm{ht}(a_1,a_2,\cdots,a_r)=r$.这只能得到$\mathrm{ht}(a_1,a_2,\cdots,a_i)=i$.
		
		$(b)\Rightarrow(c)$是平凡的.下面证明$(c)\Rightarrow(d)$.首先有$\dim A\ge\mathrm{ht}(a_1,a_2,\cdots,a_r)=r$.倘若$\dim A=r$,那么$m$是包含$(a_1,a_2,\cdots,a_r)$的极小素理想,于是$A/(a_1,a_2,\cdots,a_r)$是阿廷环,于是$a_1,a_2,\cdots,a_r$是参数系统.倘若$\dim A>r$,那么$m$不会是$(a_1,a_2,\cdots,a_r)$的极小素理想,于是可取$a_{r+1}\in m$,满足$a_{r+1}$不在$(a_1,a_2,\cdots,a_r)$的任一极小素理想中,于是$\mathrm{ht}(a_1,a_2,\cdots,a_{r+1})>r$,结合主理想定理就得到$\mathrm{ht}(a_1,a_2,\cdots,a_{r+1})=r+1$.按照诺特局部环的维数有限,做有限步操作就使得$\{a_1,a_2,\cdots,a_r\}$延拓为一组参数系统,归结为$\dim A=r$的情况.
		
		$(d)\Rightarrow(a)$.这一条要用到CM条件.我们解释过在诺特局部环和有限模的条件下,正则序列不依赖于排序.于是只需证明如果$a_1,a_2,\cdots,a_r$是参数系统,那么它是$A$正则序列.对$r$归纳,先证明$a_1$是$A$正则元.在CM条件下等价于讲证明$a_1$不在$A$支集的任一元中.CM条件下有$p\in\mathrm{Ass}(A)$总满足$\dim A/p=\dim A$,而$\dim A/a_1A=\dim A-1$.倘若$a_1$在某个$p\in\mathrm{Ass}(A)$中,那么$p$是$a_1$的极小素理想,导致$\dim A/a_1A=\dim V(p)$.但是$\dim A/a_1A=\dim A-1$,而$\dim V(p)=\dim A/p=\dim A$,这就矛盾.这完成了$r=1$的证明.下面取$A'=A/a_1A$,那么$A'$是CM环,并且$a_2',a_3',\cdots,a_r'$是$A'$的参数系统,按照归纳假设它们是$A'$的正则序列.这就说明$a_1,a_2,\cdots,a_r$是$A$正则序列,完成归纳.
	\end{proof}
	\item 如果$I\subsetneqq A$是真理想,那么$\mathrm{ht}(I)=\mathrm{depth}(I,A)$.并且有$\mathrm{ht}(I)+\dim A/I=\dim A$.
	\begin{proof}
		
		记$r=\mathrm{ht}(I)$,我们断言存在$a_1,a_2,\cdots,a_r\in I$,使得$\mathrm{ht}(a_1,a_2,\cdots,a_i)=i,\forall 1\le i\le r$.这个事实结合上一条命题得到$\{a_1,a_2,\cdots,a_r\}$是$A$正则序列,于是$\mathrm{depth}(I,A)\ge r$.首先$r=0$的情况平凡,设$r>0$,那么$I$不包含于$A$的任一极小素理想中,于是可取$a_1\in I$使得$\mathrm{ht}(a_1)=1$,于是$a_1$的任一极小素理想的高度都是1.倘若$\mathrm{ht}(I)>1$,则$I$不含于$a_1$的任一极小素理想之中,于是可取$a_2\in I$不在$(a_1)$的全部极小素理想的并中,此时有$\mathrm{ht}(a_1,a_2)=2$,倘若$\mathrm{ht}(I)>2$,以此类推得到结论.
		
		反过来取$b_1,b_2,\cdots,b_s\in I$是$A$正则序列,则上一定理得到$\mathrm{ht}(I)\ge\mathrm{ht}(b_1,b_2,\cdots,b_s)=s$.于是取上确界得到$\mathrm{ht}(I)\ge\mathrm{depth}(I,A)$.于是就有$\mathrm{ht}(I)=\mathrm{depth}(I,A)$.
		
		下面证明第二个等式.按照定义$\mathrm{ht}(I)=\inf\{\mathrm{ht}(p)\mid I\subseteq p,p\in\mathrm{Spec}(A)\}$.以及$\dim A/I=\sup\{\dim A/p\mid I\subseteq p,p\in\mathrm{Spec}(A)\}$.于是问题归结为对素理想$p$证明该等式成立:$\mathrm{ht}(p)+\dim A/p=\dim A$.
		
		记$r=\mathrm{ht}(p)=\dim A_p$,$n=\dim A$.我们之前证明过从$A$是CM环得到$A_p$也是CM环,并且有$\dim A_p=\mathrm{depth}(p,A)$.于是可取$a_1,a_2,\cdots,a_r\in p$是$A$正则序列.于是$A/(a_1,a_2,\cdots,a_r)$也是CM环,并且维数$n-r$.从$\mathrm{ht}(p)=r$得到$p$是$(a_1,a_2,\cdots,a_r)$的极小素理想,于是$p\in\mathrm{Ass}(A/(a_1,a_2,\cdots,a_r))$.我们证明过CM环的不可约分支的维数都相同,并且就是环的维数,于是$\dim A/p=\dim A/(a_1,a_2,\cdots,a_r)=n-r$.完成证明.
	\end{proof}
	\item CM(局部)环一定是catenary环.
	\begin{proof}
		
		我们只需证明如果$A$是CM环,对任意理想$q\subseteq p$,总有$\mathrm{ht}(p/q)=\mathrm{ht}(p)-\mathrm{ht}(q)$.(这个事实可以推出环是catenary的,但是反过来不成立).但是从$A$是CM环得到$A_p$是CM环,于是上个定理得到$\mathrm{ht}(qA_p)+\dim A_p/qA_p=\dim A_p$.这里$\mathrm{ht}(qA_p)=\mathrm{ht}(p)$,$\dim A_p/qA_p=\mathrm{ht}(p/q)$,$\dim A_p=\mathrm{ht}(p)$.这就得证.
	\end{proof}
\end{enumerate}

CM条件和完备化.
\begin{enumerate}
	\item 引理.设$(A,m)$是诺特局部环,设$M,N$是有限$A$模,那么有$\mathrm{Ext}_A^i(N,M)\otimes_A\widehat{A}\cong\mathrm{Ext}_{\widehat{A}}^i(N\otimes_A\widehat{A},M\otimes_A\widehat{A})$.
	\begin{proof}
		
		取$N$的有限自由预解$L_*\to N$.那么$L_*\otimes_A\widehat{A}\to N\otimes_A\widehat{A}$是$\widehat{A}$模$N\otimes_A\widehat{A}$的有限自由预解.于是$\mathrm{Ext}_A^i(N,M)\otimes_A\widehat{A}=H^i(\mathrm{Hom}_A(L_*,M))\otimes_A\widehat{A}$.而同调函子和正合函子可交换,得到它同构于$H^i(\mathrm{Hom}_A(L_*,M)\otimes_A\widehat{A})=H^i(\mathrm{Hom}_{\widehat{A}}(L_*\otimes\widehat{A},M\otimes\widehat{A}))=\mathrm{Ext}_{\widehat{A}}^i(N\otimes_A\widehat{A},M\otimes_A\widehat{A})$.
	\end{proof}
	\item 设$(A,m)$是诺特局部环,那么$A$是CM环当且仅当$\widehat{A}$是CM环.
	\begin{proof}
		
		按照CM环的定义,需要证明$\dim A=\dim\widehat{A}$和$\mathrm{depth}(A)=\mathrm{depth}(\widehat{A})$.前者用维数论可以得出,例如考虑Samuel函数.后者因为,考虑同调定义$\mathrm{depth}(A)=\inf\{i\mid\mathrm{Ext}_A^i(A/m,A)\not=0\}$.于是只需证明$\mathrm{Ext}_A^i(A/m,A)\otimes_A\widehat{A}\cong\mathrm{Ext}_{\widehat{A}}^i(\widehat{A}/m\widehat{A},\widehat{A})$.而这就是上面引理.
	\end{proof}
\end{enumerate}

我们只定义了诺特局部环情况下的CM环.对于一般的诺特环,称它为CM环,如果对每个$p\in\mathrm{Spec}(A)$总有$A_p$满足$\dim A_p=\mathrm{depth}(A_p)$.我们解释过CM局部环的局部化还是CM局部环,所以这个定义等价于讲对每个极大理想$m$都有$A_m$是CM局部环.
\begin{enumerate}
	\item 设$A$是诺特环,一个真理想$I\subsetneqq A$称为非混合的(unmixed),如果它的素除子的高度都相同.称诺特环$A$满足非混合定理(unmixedness theorem),如果对每个自然数$r\ge0$,每个高度$r$并且被$r$个元生成的理想$I$都是非混合的.
	\item 设$A$是诺特环,那么$A$是CM环当且仅当它满足非混合定理.
	\begin{proof}
		
		先设$A$是CM环,取高度$r$被$r$个元生成的真理想$I=(a_1,\cdots,a_r)\subsetneqq A$.那么特别的$I$的每个极小素理想的高度都是$r$,假设$I$存在嵌入素理想$\mathfrak{p}$,那么$IA_{\mathfrak{p}}$的高度依旧是$r$,并且$\mathfrak{p}A_{\mathfrak{p}}$还是它的嵌入素理想.所以不妨设$A$本身是局部CM环.我们之前解释过$\{a_1,\cdots,a_r\}$是$A$正则序列,那么$A/(a_1,\cdots,a_r)$仍然是CM局部环,但是我们解释过CM模没有嵌入素理想,导致$I$没有嵌入素理想.
		
		\qquad
		
		反过来设$A$满足非混合定理,任取高度$r$的素理想$\mathfrak{p}$,可取$a_1,\cdots,a_r\in\mathfrak{p}$,使得每个$\mathrm{ht}(a_1,\cdots,a_i)=i$.按照非混合条件,$(a_1,\cdots,a_i)$的素除子的高度都是$i$,所以它不能包含$a_{i+1}$,所以$a_{i+1}$是$A/(a_1,\cdots,a_i)$的正则元,所以$\{a_1,\cdots,a_r\}$是$A$正则序列,所以有$\mathrm{depth}(A_{\mathfrak{p}})\ge\dim(A_{\mathfrak{p}})$,但是另一个方向的不等式总成立,于是$A_{\mathfrak{p}}$是局部CM环,于是$A$是CM环.
	\end{proof}	
	\item 若$A$是诺特CM环,那么$A[X_1,X_2,\cdots,X_n]$总是CM环.
	\begin{proof}
		
		按照归纳法,归结为证明$n=1$的情况.记$B=A[X]$,记$\mathfrak{M}\subseteq B$是极大理想,记$\mathfrak{m}=\mathfrak{M}\cap A$,那么$B_{\mathfrak{M}}$仍然是$A_{\mathfrak{m}}[X]$的局部化,所以用$A_{\mathfrak{m}}$替代$A$,我们可不妨设$(A,\mathfrak{m})$是诺特局部CM环.下面证明$B_{\mathfrak{M}}$是CM的.记$k=A/\mathfrak{m}$,那么$B/\mathfrak{m}B\cong k[X]$,所以$\mathfrak{M}/\mathfrak{m}B$是$k[X]$的被首一不可约多项式$\varphi(X)$生成的主理想.取$f(X)\in A[X]$是$\varphi(X)$的提升.那么$\mathfrak{M}=(\mathfrak{m},f(X))$.取$A$的参数系统$\{a_1,\cdots,a_n\}$,它也是$A$的正则序列(因为$A$是CM的).于是$\{a_1,\cdots,a_n,f\}$构成了$B_{\mathfrak{M}}$的参数系统.因为$B$在$A$上平坦,说明$\{a_1,\cdots,a_n\}$也是$B$正则序列.取$A'=A/(a_1,\cdots,a_n)$,那么$f$在$A'[X]$中的像是正则元(因为首一),所以$\{a_1,\cdots,a_n,f\}$是$B$正则序列.于是我们得到$\mathrm{depth}(B_{\mathfrak{M}})\ge\mathrm{depth}(\mathfrak{M},B)\ge n+1=\dim(B_{\mathfrak{M}})$,于是$B_{\mathfrak{M}}$是CM环.
	\end{proof}
	\item 按照上一条一样的做法,我们可以证明如果$A$是CM局部环,则$A[[X]]$也是CM的.但是如果$A$是未必局部的CM环,由于此时$A[[X]]\otimes_AA_{\mathfrak{p}}\not=A_{\mathfrak{p}}[[X]]$,就不能和上述证明一样把问题归结为$A$局部的情况.不过对于未必局部的诺特CM环$A$,这个结论倒是仍然成立,后文会给出证明.
	\item 正则局部环是CM局部环.
	\begin{proof}
		
		设$(A,\mathfrak{m})$是$n$维正则局部环,设$x_1,\cdots,x_n\in A$是正则参数系统.那么我们证明过$A/(x_1,\cdots,x_i)$是$n-i$维的正则局部环,进而它们都是整环,所以有严格递增的素理想链$(x_1)\subsetneqq(x_1,x_2)\subsetneqq\cdots\subsetneqq(x_1,\cdots,x_n)$.这说明$\{x_1,\cdots,x_n\}$是$A$正则序列,这导致$A$是CM环.
	\end{proof}
    \item CM环总是catenary的.因为如果有素理想$q\subseteq p$,只要考虑$A_p$这个局部CM环就归结为之前的局部情况.
	\item 如果$A$是CM环,那么每个商环$A/I$都是universally catenary的.特别的,域上有限生成代数总是universally catenary的.
	\begin{proof}
		
		设$R$是CM环,设$A=R/I$,我们要证明$A[X_1,\cdots,X_n]$是catenary的.但是它是CM环$R[X_1,\cdots,X_n]$的商,所以是CM环.
	\end{proof}
\end{enumerate}

参数理想.设$(A,m)$是诺特局部环,一个理想$I$称为参数理想,如果它由一组参数系统生成.例如对于诺特局部环,它的极大理想是参数理想当且仅当它是局部正则环.设$(A,m)$是诺特局部环,如下三个条件两两等价:
\begin{enumerate}
	\item $A$是CM环.
	\item 对每个参数理想$q$,总有$l(A/q)=e(q)$.
	\item 存在一个参数理想$q$,使得$l(A/q)=e(q)$.
\end{enumerate}
\begin{proof}
	
	1推2,任取参数理想$q$,设参数系统$x_1,x_2,\cdots,x_d$生成了理想$q$.按照CM条件,这组参数系统也就是$A$正则序列,再按照条件是诺特局部环上的有限模($A$自身),于是这组$A$正则序列也是拟正则序列.考虑同态$A[X_1,X_2,\cdots,X_d]\to\mathrm{gr}^q(A)=\oplus_{n\ge0}q^n/q^{n+1}$为$X_i\mapsto x_i'\in q/q^2$.这个同态的核即系数均落在$q$上的多项式,于是得到同构$(A/q)[X_1,X_2,\cdots,X_d]\cong\mathrm{gr}^q(A)$.计算Samuel函数$\chi_A^q=l(A/q^{n+1})=\sum_{i=0}^nl(q^i/q^{i+1})=l(A/q)\left(\begin{array}{c}n+d\\n\end{array}\right)=\frac{l(A/q)}{d!}n^d+\cdots$.于是重数定义得到$e(q)=l(A/q)$.
	
	2推3平凡.下面证明3推1.设参数理想$q$由参数系统$x_1,x_2,\cdots,x_d$生成,这里$d=\dim A$.只需验证$x_1,x_2,\cdots,x_d$是$A$正则序列,结合深度不超过维数,就说明此时环是CM环.而在条件下也等价于验证$x_1,x_2,\cdots,x_d$是$A$的拟正则序列.也即验证$\theta:A[X_1,X_2,\cdots,X_d]\to\mathrm{gr}^q(A)=\oplus_{n\ge0}q^n/q^{n+1}$为$X_i\mapsto x_i'\in q/q^2$的核为$q[X_1,X_2,\cdots,X_n]$.但是明显有$q[X_1,X_2,\cdots,X_d]\subset\ker\theta$,于是诱导了$\varphi:(A/q)[X_1,X_2,\cdots,X_d]\to\mathrm{gr}^q(A)$.需要验证它的核是零.
	
	设$\ker\varphi=I$,那么$(A/q)[X_1,X_2,\cdots,X_d]/I\cong\mathrm{gr}^q(A)$.记$(A/q)[X_1,X_2,\cdots,X_d]$和$(A/q)[X_1,X_2,\cdots,X_d]/I$的Hilbert函数为$\varphi_1$和$\varphi_2$.那么$l(q^n/q^{n+1})=\varphi_1(n)-\varphi_2(n)$,并且$\varphi_1(n)=l(A/q)\left(\begin{array}{c}n+d-1\\d-1\end{array}\right)=\frac{l(A/q)}{(d-1)!}n^{d-1}+\cdots$.而$l(q^n/q^{n+1})=\chi_A^q(n)=\chi_A^q(n-1)=\frac{e(q)}{(d-1)!}n^{d-1}+\cdots$,于是$\varphi_2(n)$的次数$\le d-2$.倘若$I\not=(0)$,下推矛盾.
	
	取非零齐次元$f\in I\subset(A/q)[X_1,X_2,\cdots,X_d]$,由于存在$m$某个次幂零化$f$,于是不妨设$mf=0$,否则可以用某个次数极大的$m^{v-1}f$代替$f$.现在取同态$\theta:k[X_1,X_2,\cdots,X_d]=B/mB\to Bf$为$b\mapsto bf$,这里$k=A/m$.记$\delta=\deg f$,对$M=\oplus_{n\ge0}M_n$,记$M(\delta)$是这样的分次模,满足$M(\delta)_i=M_{i-\delta}$.现在同态$\theta$是同构.按照$Bf\subseteq I$得到$\varphi_2(n)\ge\varphi_{Bf}(n)$.这导致$\varphi_{Bf}(n)=\varphi_{k[X_1,X_2,\cdots,X_d](\delta)}(n)=\varphi_{k[X_1,X_2,\cdots,X_d]}(n-\delta)=\left(\begin{array}{c}n-\delta+d-1\\d-1\end{array}\right)$.导致$\deg\varphi_2(n)\ge d-1$,这矛盾.
\end{proof}
\newpage
\subsection{Gorenstein环}

首先是一些同调代数知识.
\begin{enumerate}
	\item 设$A$是环,$M$是$A$模,对$n\in\mathbb{Z}_{\ge0}$,那么如下两个条件等价:
	\begin{enumerate}
		\item $M$的内射维数不超过$n$.
		\item 对任意理想$I$,有$\mathrm{Ext}_A^{n+1}(A/I,M)=0$.
	\end{enumerate}
	
	另外如果$A$是诺特环,还等价于对任意素理想$p$,总有$\mathrm{Ext}_A^{n+1}(A/p,M)=0$.
	\begin{proof}
		
		1推2,内射维数不超过$n$的定义就是对任意$i>n$和任意模$N$总有$\mathrm{Ext}_A^i(N,M)=0$.2推1,对于$n=0$,条件是对每个理想$I$总有$\mathrm{Ext}_A^1(A/I,M)=0$.取短正合列$0\to I\to A\to A/I\to0$,诱导了长正合列$0\to\mathrm{Hom}(A/I,M)\to\mathrm{Hom}(A,M)\to\mathrm{Hom}(I,M)\to\mathrm{Ext}_A^1(A/I,M)=0$,据此得到$M$是内射模(Baer准则),于是$M$的内射维数是零.
		
		下设$n>0$,取$M$的内射预解$0\to M\to I^0\to\cdots$,记其中$I^{n-1}\to I^n$为$\alpha$,于是有$0\to M\to I^0\to\cdots\to I^{n-1}\to C\to0$,其中$C=\ker\alpha$.这得到$\mathrm{Ext}_A^i(N,M)\cong\mathrm{Ext}_A^{i-n}(A/I,C)$对$i\ge n+1$成立.特别的,有$\mathrm{Ext}^{n+1}(A/I,M)\cong\mathrm{Ext}^1(A/I,C)=0$.于是$C$是内射模,于是上面$0\to M\to I^0\to\cdots\to I^{n-1}\to C\to0$已经是$M$的内射预解,于是$\mathrm{inj.dim}(M)\le n$.
		
		最后在诺特条件下,我们证明如果对任意素理想$p$,总有$\mathrm{Ext}_A^{n+1}(A/p,M)=0$,那么对任意理想$I$有$\mathrm{Ext}_A^{n+1}(A/I,M)=0$.事实上我们可以证明对任意有限$A$模$N$总有$\mathrm{Ext}_A^{n+1}(N,M)=0$.因为这个条件下$N$存在滤过,使得相邻两项的商都同构于$A/p$.于是这个条件得到$\mathrm{Ext}^{n+1}(N,M)=0$.
	\end{proof}
	\item 设$A$是环,$M$和$N$是$A$模,设$x\in A$是$A$正则元也是$M$正则元,并且$xN=0$.记$A'=A/xA$和$M'=M/xM$,那么:
	\begin{enumerate}
		\item $\mathrm{Ext}_A^{n+1}(N,M)=\mathrm{Ext}_{A'}^n(N,M')$.
		\item $\mathrm{Ext}_A^n(M,N)=\mathrm{Ext}_{A'}^n(M',N)$.
		\item $\mathrm{Tor}_n^A(M,N)=\mathrm{Tor}_n^{A'}(M',N)$.
	\end{enumerate}
	\begin{proof}
		
		先证第一个等式.对$n=-1$的情况,此即$\mathrm{Hom}_A(N,M)=0$.任取模同态$f:N\to M$,那么$f(n)=0$等价于$xf(n)=0$等价于$f(xn)=0$,这总成立,于是$f$总是零同态.对$n$的一般情况见讲义,用到泛同调函子.
		
		证明第二个等式.取$M$的自由预解$L_*\to M$.记$L_n'=L_n/xL_n$.考虑复形的短正合列$0\to(L_*,M)\to(L_*,M)\to(L'_*,M')\to0$,这可得到$L'_*\to M'$是$M'$的自由预解.于是$H^n(\mathrm{Hom}_A(L_*,N))=H^n(\mathrm{Hom}_{A'}(L'_*,N))$.取同调得到结果.第三个等式是类似的.
	\end{proof}
	\item 设$(A,m,k)$是诺特局部环,设$M$是有限$A$模,设$p\in\mathrm{Spec}(A)$满足$\mathrm{ht}(m/p)=1$,那么从$\mathrm{Ext}^{i+1}_A(k,M)=0$可推出$\mathrm{Ext}_A^i(A/p,M)=0$.
	\begin{proof}
		
		取$x\in m\backslash p$,那么数乘$x$是$A/p\to A/p$的单同态,这得到短正合列$0\to A/p\to A/p\to A/(p+Ax)\to0$.于是有$\mathrm{Ext}_A^i(A/p,M)\to\mathrm{Ext}_A^i(A/p,M)\to\mathrm{Ext}_A^{i+1}(A/(p+Ax),M)$.这里第一个映射是数乘$x$诱导的.倘若可证明$\mathrm{Ext}^{i+1}_A(A/(p+Ax),M)=0$,按照NAK引理就得到结论.
		
		按照$\mathrm{ht}(m/p)=1$得到$\mathrm{Supp}(A/(p+Ax))=\{m\}$,于是$A/(p+Ax)$存在滤过使得相邻两项的商总同构于$A/m=k$.于是从$\mathrm{Ext}_A^{i+1}(k,M)=0$得到$\mathrm{Ext}_A^{i+1}(A/(p+Ax),M)=0$.
	\end{proof}
	\item 引理.如果$A$诺特,$N$是有限$A$模,那么$(\mathrm{Ext}_A^i(N,M))_p=\mathrm{Ext}_{A_p}^i(N_p,M_p)$.更一般地,$A$如果是一般环,$N$为有限表示模,那么$(\mathrm{Ext}_A^i(N,M))_p=\mathrm{Ext}_{A_p}^i(N_p,M_p)$.
	\item 推论.设$(A,m,k)$是诺特局部环,$M$是有限$A$模,对$p\in\mathrm{Spec}(A)$,并且$\mathrm{ht}(m/p)=d$,那么从$\mathrm{Ext}_A^{i+d}(k,M)=0$推出$\mathrm{Ext}_{A_p}^i(\kappa(p),M_p)$.
	\begin{proof}
		
		取$m=p_0\supset p_1\supset\cdots\supset p_d=p$,满足$\mathrm{ht}(p_i/p_{i+1})=1,\forall i$.反复用上两条命题得到结论.
	\end{proof}
\end{enumerate}

设$(A,m,k)$是$n$维诺特局部环,如下八个结论互相等价,在条件成立下称局部环$A$是Gorenstein局部环.一个诺特环$A$称为Gorensteion环,如果它在每个极大理想处的局部化都是Gorenstein局部环.
\begin{enumerate}
	\item $A$的内射维数有限.
	\item $A$的内射维数恰好是它的维数$n$.
	\item $\mathrm{Ext}_A^i(k,A)=\left\{\begin{array}{cc}k=A/m&i=n\\0&i\not=n\end{array}\right.$
	\item 存在$i>n$使得$\mathrm{Ext}_A^i(k,A)=0$.
	\item $\mathrm{Ext}_A^i(k,A)=\left\{\begin{array}{cc}k=A/m&i=n\\0&i<n\end{array}\right.$
	\item $A$是CM环并且$\mathrm{Ext}_A^n(k,A)\cong k$.
	\item $A$是CM环并且$A$的每个参数理想都是不可约的.这里一个理想$I$不可约是指,从$I=J\cap J'$总能推出$I=J$或$I=J'$.
	\item $A$是CM环并且存在不可约的参数理想.
\end{enumerate}
\begin{proof}
	
	(1)$\Leftrightarrow$(2).设$A$的内射维数有限,记作$r$.我们对$r$归纳证明(2)成立.设$\mathfrak{p}$是一个极小素理想,使得$\mathrm{ht}(\mathfrak{m}/\mathfrak{p})=\dim A=n$.于是特别的有$\mathfrak{p}A_{\mathfrak{p}}\in\mathrm{Ass}(A_{\mathfrak{p}})$,按照定义就有单射$\kappa(\mathfrak{p})=A_{\mathfrak{p}}/\mathfrak{p}A_{\mathfrak{p}}\subseteq A_{\mathfrak{p}}$.所以$\mathrm{Hom}_{A_{\mathfrak{p}}}(\kappa(\mathfrak{p}),A_{\mathfrak{p}})\not=0$.按照我们前面补充的同调代数定理,就有$\mathrm{Ext}_A^n(k,A)\not=0$.于是有$r\ge n$.如果$r=0$,必然有$n=0$,没什么需要证的.如果$r>0$,那么存在一个素理想$\mathfrak{p}\subseteq A$使得$\mathrm{Ext}^r(A/\mathfrak{p},A)\not=0$,我们断言$\mathfrak{p}=\mathfrak{m}$.如果这不成立,可取$x\in\mathfrak{m}-\mathfrak{p}$,考虑数乘$x$诱导的短正合列$0\to A/\mathfrak{p}\to A/\mathfrak{p}\to A/(\mathfrak{p}+Ax)\to0$.考虑它诱导的长正合列,按照内射维数$r$,就有$i>r$时$\mathrm{Ext}^i_A(-,A)=0$,所以诱导了数乘$x$的满射$\mathrm{Ext}_A^r(A/\mathfrak{p},A)\to\mathrm{Ext}_A^r(A/\mathfrak{p},A)$,但是NAK引理就说明$\mathrm{Ext}_A^r(A/\mathfrak{p},A)=0$,这个矛盾说明$\mathfrak{m}=\mathfrak{p}$.于是有$\mathrm{Ext}_A^r(k,A)\not=0$.另外我们断言$\mathfrak{m}\not\in\mathrm{Ass}(A)$:否则存在单射$k\subseteq A$,就诱导了满射$0=\mathrm{Ext}_A^r(A,A)\to\mathrm{Ext}_A^r(k,A)$依旧矛盾.于是$\mathfrak{m}$不是伴随素理想说明它包含正则元$x$,记$A'=A/xA$,那么有$\mathrm{Ext}_{A'}^i(N,A')=\mathrm{Ext}_A^{i+1}(N,A)$.于是$A'$的内射维数是$r-1$,维数是$n-1$,归纳假设得到$n-1=r-1$,完成归纳.最后(2)推(1)平凡.
	
	\qquad
	
	(2)$\Rightarrow$(3).如果$n=\dim A=0$,那么$\mathfrak{m}\in\mathrm{Ass}(A)$,于是存在$A$模的单射$k\subseteq A$,于是$\mathrm{Hom}_A(k,A)\not=0$.按照(2)此时内射维数也是0,于是有$A$模的满同态$A=\mathrm{Hom}_A(A,A)\to\mathrm{Hom}_A(k,A)$,于是$\mathrm{Hom}_A(k,A)$是循环模(单个元生成的),它必然是商模$A/\mathfrak{p}$,但是$\mathrm{Hom}_A(k,A)$有被$\mathfrak{m}$零化,所以它必须同构于$A/\mathfrak{m}=k$.另外$A$的内射维数0导致$\mathrm{Ext}^i_A(k,A)=0,\forall i\ge1$.这就证明了$n=0$的情况.下面设$n=\dim A>0$,和上一段证明一样,我们可以得到$\mathfrak{m}$包含了一个正则元$x$,记$A'=A/xA$,和上一段一样有$A'$的内射维数是$n-1$,按照$\mathrm{Ext}_A^i(k,A)\cong\mathrm{Ext}_A^{i-1}(k,A')$,归纳法就得证.
	
	\qquad
	
	(3)$\Rightarrow$(4)是平凡的.下面证(4)$\Rightarrow$(1).我们对$n=\dim A$归纳.设存在$i>n$使得$\mathrm{Ext}_A^i(k,A)=0$.如果$n=0$,那么$\mathfrak{m}$是$A$的唯一素理想,按照我们之前补充的同调代数引理,从$\mathrm{Ext}_A^i(A/\mathfrak{m},A)=0$就得到$A$的内射维数$\le i-1$从而有限.下设$n=\dim A>0$.任取素理想$\mathfrak{p}\subsetneqq\mathfrak{m}$,记$d=\mathrm{ht}(\mathfrak{m}/\mathfrak{p})$,按照我们补充的同调代数引理有$\mathrm{Ext}^{i-d}_{A_{\mathfrak{p}}}(\kappa(\mathfrak{p}),A_{\mathfrak{p}})=0$.这里$\dim(A_{\mathfrak{p}})\le n-d<i-d$,那么按照归纳假设得到$A_{\mathfrak{p}}$的内射维数有限.并且对每个有限$A$模$M$,按照$i>n>\dim A_{\mathfrak{p}}=\mathrm{inj.dim}A_{\mathfrak{p}}$,就有$(\mathrm{Ext}_A^i(M,A))_{\mathfrak{p}}\cong\mathrm{Ext}_{A_{\mathfrak{p}}}^i(M_{\mathfrak{p}},A_{\mathfrak{p}})=0$.于是$\mathfrak{p}\not\in\mathrm{Supp}_A(\mathrm{Ext}_A^i(M,A))$.所以这个支集包含在$\{\mathfrak{m}\}$中.所以$\mathrm{Ext}_A^i(M,A)$是有限长度$A$模.我们再断言$\mathrm{Ext}_A^i(A/\mathfrak{p},A)=0$对任意素理想$\mathfrak{p}$成立,一旦这得证,按照我们补充的同调代数引理就得到$A$的内射维数有限.假设这个断言不成立,取不满足这个等式的素理想的极大元,记作$\mathfrak{p}$.那么它不能是唯一极大理想,即$\mathfrak{p}\subsetneqq\mathfrak{m}$.取$x\in\mathfrak{m}-\mathfrak{p}$,考虑数乘$x$诱导的短正合列$0\to A/\mathfrak{p}\to A/\mathfrak{p}\to A/(\mathfrak{p}+Ax)\to0$.取$A$模$A/(\mathfrak{p}+Ax)$的合成链,设合成因子为$A/\mathfrak{p}_j,1\le j\le m$,因为合成因子是$A/(\mathfrak{p}+Ax)$的子模商,所以$\mathfrak{p}+Ax\subseteq$每个$\mathfrak{p}_j$,所以$\mathfrak{p}\subsetneqq$每个$\mathfrak{p}_j$.按照我们选取的$\mathfrak{p}$是不满足断言的素理想的极大元,说明这些$\mathfrak{p}_j$都要满足$\mathrm{Ext}_A^i(A/\mathfrak{p}_j,A)=0$.进而有$\mathrm{Ext}_A^i(A/(\mathfrak{p}+Ax),A)=0$.于是之前短正合列诱导的长正合列就得到$0=\mathrm{Ext}^i(A/(\mathfrak{p}+Ax),A)\to\mathrm{Ext}_A^i(A/\mathfrak{p},A)\to\mathrm{Ext}_A^i(A/\mathfrak{p},A)$.也即数乘$x$是$\mathrm{Ext}_A^i(A/\mathfrak{p},A)$上的单射,但是这个$A$模是有限长度的,所以单射也是同构,那么按照NAK引理,从$x\in\mathfrak{m}$得到$\mathrm{Ext}_A^i(A/\mathfrak{p},A)=0$,这个矛盾说明断言成立,完成证明.
	
	\qquad
	
	(3)$\Rightarrow$(5)是平凡的.(5)$\Leftrightarrow$(6):我们知道CM环等价于讲$\mathrm{depth}(A)=n$,也等价于讲对每个$i<n$有$\mathrm{Ext}_A^i(k,A)=0$.
	
	\qquad
	
	(6)$\Rightarrow$(7).因为$A$是CM环,它的参数系统$\{x_1,\cdots,x_n\}$就是它的极大正则序列.记$B=A/(x_1,\cdots,x_n)A$,那么有$\mathrm{Hom}_B(k,B)\cong\mathrm{Ext}_{A/(x_1,\cdots,x_{n-1})}^1(k,A/(x_1,\cdots,x_{n-1}))\cong\cdots\cong\mathrm{Ext}_A^n(k,A)\cong k$.这里$B$是阿廷环,我们断言$B$只有唯一的非零极小素理想:设$I_0\subseteq B$是非零极小素理想,考虑理想$mI_0$,它要么是0理想要么是$I_0$,但是从$\mathfrak{m}I_0=I_0$按照NAK引理得到$I_0=0$矛盾,所以只能有$\mathfrak{m}I_0=0$,所以$I_0$是$k=B/\mathfrak{m}$上的线性空间,极小性导致它的维数必须是1,所以有$B$模同构$k\cong I_0$,换句话讲存在$f\in\mathrm{Hom}_B(k,B)$使得$I_0=\mathrm{im}f$.假设有另一个$f_0\in\mathrm{Hom}_B(k,B)$,按照$\mathrm{Hom}_B(k,B)\cong k$,就可找到$0\not=\lambda\in k$使得$f(x)=f_0(\lambda x),\forall x\in k$,于是有$\mathrm{im}f=\mathrm{im}f_0$,于是$I_0$是唯一的.接下来我们断言$B$的零理想是不可约的,因为如果$(0)=I\cap J$,其中$I,J$非零,那么理应有$I\cap J\supseteq I_0$,这个矛盾说明$I=0$或$J=0$.然后从$B$的零理想是不可约的就得到$A$的理想$(x_1,\cdots,x_n)$是不可约的.
	
	\qquad
	
	(7)$\Rightarrow$(8)是平凡的,最后我们证明(8)$\Rightarrow$(3):因为$A$是CM环,就有$\mathrm{Ext}_A^i(k,A)=0,\forall i<n$.设$\mathfrak{q}$是一个不可约参数理想,记$B=A/\mathfrak{q}$,那么有$\mathrm{Ext}_A^{n+i}(k,A)=\mathrm{Ext}_B^i(k,B),\forall i$.所以问题归结为设$B$是零维诺特局部环(阿廷局部环),设$(0)$是不可约理想,证明$\mathrm{Hom}_B(k,B)\cong k$和$\mathrm{Ext}_B^i(k,B)=0,\forall i>0$.前一个等式是容易的:对于阿廷局部环,它唯一的极大理想是伴随素理想,所以有嵌入$k\subseteq B$,所以至少有$\mathrm{Hom}_B(k,B)\not=0$.任取非零的$f,g\in\mathrm{Hom}_B(k,B)$,那么$\mathrm{im}f$和$\mathrm{im}g$在$k$上都是1维的,按照零理想不可约,就有$\mathrm{im}f\cap\mathrm{im}(g)\not=(0)$,于是只能有$\mathrm{im}f=\mathrm{im}g$.所以有$\alpha\in k$使得$f(1)=g(\alpha)$.那么有$f=\alpha g$,也即$\mathrm{Hom}_B(k,B)\cong k$.接下来证明第二个等式$\mathrm{Ext}_B^i(k,B)=0,\forall i>0$.这等价于$B$是内射$B$模,也等价于对$i=1$成立$\mathrm{Ext}_B^1(k,B)=0$.我们取$B$模$B$的合成链$0=N_0\subseteq N_1\subseteq\cdots\subseteq N_r=B$.那么合成因子只能有$N_{i+1}/N_i\cong k$.于是对每个$0\le i<r$,就有正合列$0\to\mathrm{Hom}_B(N_{i+1}/N_i,B)\to\mathrm{Hom}_B(N_{i+1},B)\to\mathrm{Hom}_B(N_i,B)\to\mathrm{Ext}_B^1(N_{i+1}/N_i,B)$.按照$\mathrm{Hom}_B(k,B)\cong k$,就得到$l(\mathrm{Hom}_B(N_{i+1},B))\le l(\mathrm{Hom}_B(N_i,B))+1$,归纳得到$l(\mathrm{Hom}_B(N_i,B))\le i$,但是$i=r$得到$l(\mathrm{Hom}_B(B,B))=r=l_B(B)$,所以商模这些不等式都取等,所以每个映射$\delta_j:\mathrm{Hom}_B(N_j,B)\to\mathrm{Ext}_B^1(k,B)$都是零.特别的$\delta_{r-1}$诱导了正合列$\mathrm{Hom}_B(N_{r-1},B)\to\mathrm{Ext}_B^1(B/N_{r-1},B)\to\mathrm{Ext}_B^1(B,B)=0$.所以这里$\mathrm{Ext}_B^1(B/N_{r-1},B)=\mathrm{Ext}_B^1(k,B)=0$,完成证明.
\end{proof}
\begin{enumerate}
	\item 如果$(A,m,k)$是零维诺特环,那么它的参数理想是零理想.所以此时$A$是Gorenstein环等价于讲$(0)$是不可约的.这也等价于讲$A$的所有非零极小理想的和$\mathrm{soc}(A)$(一般的,对$R$模$M$,记$\mathrm{soc}(M)$表示$M$的所有单子模的和)在$k$上是一维的.例如$F$是域,那么$F[X,Y]/(X^2,Y^2)$是Gorenstein环,而$F[X,Y]/(X^2,XY,Y^2)$不是.因为前者的$\mathrm{soc}$是$(xy)$,后者是$(x,y)$.【】
	\item 如果$(A,m)$是正则局部环,我们解释过它是CM环.另外$m$已经是参数理想,并且它明显是不可约的,所以我们证明了正则局部环总是Gorenstein环.
	\item 设$(A,m,k)$是诺特局部环,如果$x\in m$是$A$正则元,那么$A$是Gorenstein环当且仅当$A'=A/xA$是Gorenstein环.
	\begin{proof}
		
		因为$x$是$A$正则元,有$\dim A'=\dim A-1$,另外我们之前解释过对每个$i$都有$\mathrm{Ext}_A^i(k,A')\cong\mathrm{Ext}_A^{i+1}(k,A)$,得到结论.
	\end{proof}
    \item 如果$A$是Gorenstein局部环,对每个素理想$\mathfrak{p}$,都有$A_{\mathfrak{p}}$也是Gorenstein局部环.于是一个诺特环$A$是Gorenstein环也等价于讲它在每个极大理想处的局部化都是Gorenstein局部环.
    \begin{proof}
    	
    	这件事是因为如果$N$是内射$A$模,那么$N_{\mathfrak{p}}$是内射$A_{\mathfrak{p}}$模.所以取$A$模的内射预解做局部化就得到$A_{\mathfrak{p}}$模的内射预解,所以如果$A$的内射维数有限,就得到它局部化的内射维数也有限.
    \end{proof}
    \item 设$A$是诺特局部环,设$\widehat{A}$是它关于极大理想的完备化,那么$A$是Gorenstein局部环当且仅当$\widehat{A}$是Gorenstein环.
    \begin{proof}
    	
    	我们知道$\dim A=\dim\widehat{A}$,另外有$\mathrm{Ext}_A^i(k,A)\otimes_A\widehat{A}\cong\mathrm{Ext}_{\widehat{A}}^i(k,\widehat{A})$,结合$\widehat{A}$在$A$上忠实平坦,按照Gorenstein环等价定义第四条就得证.
    \end{proof}
\end{enumerate}

极小内射预解与Gorenstein环.设$M$是$A$模,取$M$的内射包$I^0$,记$K^1=I^0/M$,再取$K^1$的内射包$I^1$,记$K^2=I^1/K^1$,归纳构造下去,我们得到$M$的如下简化内射预解,它称为$M$的极小内射预解.如果$A$是诺特环,对素理想$\mathfrak{p}$,记$\mu_i(\mathfrak{p},M)$表示$I^i$分解成不可分解内射模直和时同构于$E(A/\mathfrak{p})$的分量个数(见内射模的Matlis理论,这个数字只依赖于$I^i$和$\mathfrak{p}$),于是我们可记$I^i=\oplus_{\mathfrak{p}\in\mathrm{Spec}A}\mu_i(\mathfrak{p},M)E(A/\mathfrak{p})$.
$$I^{\bullet}=\left(\xymatrix{0\ar[r]&I^0\ar[r]&I^1\ar[r]&\cdots}\right)$$
\begin{enumerate}
	\item 设$A$是诺特环,$S\subseteq A$是乘性闭子集,$M$是$A$模,如果$I^{\bullet}$是$M$的极小内射预解,那么$S^{-1}I^{\bullet}$是$S^{-1}A$模$S^{-1}M$的极小内射预解,另外如果素理想$\mathfrak{p}\cap S=\emptyset$,则有$\mu_i(\mathfrak{p},M)=\mu_i(S^{-1}\mathfrak{p},S^{-1}M)$.这件事是因为内射包在局部化下仍然是内射包(而这件事又是因为无论本性扩张还是内射模都在局部化下不变).
	\item 设$A$是诺特环,设$M$是$A$模,设$\mathfrak{p}$是素理想,那么有:
	$$\mu_i(\mathfrak{p},M)=\dim_{\kappa(\mathfrak{p})}\mathrm{Ext}_{A_{\mathfrak{p}}}^i(\kappa(\mathfrak{p}),M_{\mathfrak{p}})=\dim_{\kappa(\mathfrak{p})}\mathrm{Ext}_A^i(A/\mathfrak{p},M)_{\mathfrak{p}}$$
	
	特别的,如果$M$是有限$A$模,那么$\mu_i(\mathfrak{p},M)<\infty$.
	\begin{proof}
		
		首先因为$A$诺特,对每个有限$A$模$N$,典范映射$\mathrm{Ext}^i_A(N,M)_{\mathfrak{p}}\to\mathrm{Ext}_{A_{\mathfrak{p}}}^i(N_{\mathfrak{p}},M_{\mathfrak{p}})$是同构.结合上一条,说明不妨设$(A,\mathfrak{p})$是局部环.设$I^{\bullet}$是$M$的极小内射预解,于是$\mathrm{Ext}_A^{\bullet}(\kappa,M)$是如下复形的上同调:
		$$\xymatrix{\cdots\ar[r]&\mathrm{Hom}_A(\kappa,I^{i-1})\ar[r]&\mathrm{Hom}_A(\kappa,I^i)\ar[r]&\mathrm{Hom}_A(\kappa,I^{i+1})\ar[r]&\cdots}$$
		
		约定$I^{-1}=M$,把$\mathrm{Hom}_A(\kappa,I^i)$等同于$T^i=\{x\in I^i\mid\mathfrak{p}x=0\}\subseteq I^i$.按照极小内射预解的定义,有$I^i$是$\mathrm{d}(I^{i-1})$的本性扩张$\forall i\ge0$.于是对$x\in T^i$,有$I^i$的子模$Ax\cong\kappa$和$\mathrm{d}(I^{i-1})$有交,也即存在$u\in A^*$使得$ux\in\mathrm{d}(I^{i-1})$.于是$x\in\mathrm{d}(I^{i-1})$.于是$T^i\subseteq\mathrm{d}(T^{i-1})$.于是$\mathrm{d}(T^i)\subseteq\mathrm{d}^2(I^{i-1})=\{0\}$.于是上述复形的微分是零,于是$\mathrm{Ext}_A^i(\kappa,M)=T^i$,并且有$\dim_{\kappa}T^i=\dim_{\kappa}\mathrm{Hom}_A(\kappa,I^i)=\mu_i(\mathfrak{p},M)$(最后一个等式见内射模的Matlis理论).
	\end{proof}
    \item 推论.设$A$是诺特环,那么$A$是Gorenstein环当且仅当$A$的一个极小内射预解$I^{\bullet}$满足$I^i=\oplus_{\mathrm{ht}(\mathfrak{p})=i}E(A/\mathfrak{p})$.或者等价的讲,当且仅当对任意$\mathfrak{p}\in\mathrm{Spec}A$有$\mu_i(\mathfrak{p},A)=\delta_{i,\mathrm{ht}(\mathfrak{p})}$.
    \begin{proof}
    	
    	按照定义,$A$是Gorenstein环当且仅当对每个素理想$\mathfrak{p}$有$A_{\mathfrak{p}}$是Gorenstein环.当且仅当对每个素理想$\mathfrak{p}$有:
    	$$\mathrm{Ext}_{A_{\mathfrak{p}}}^i(\kappa(\mathfrak{p}),A_{\mathfrak{p}})\cong\mathrm{Ext}_A^i(\kappa(\mathfrak{p}),A)_{\mathfrak{p}}=\left\{\begin{array}{cc}0&i\not=\mathrm{ht}(\mathfrak{p})\\\kappa(\mathfrak{p})&i=\mathrm{ht}(\mathfrak{p})\end{array}\right.$$
    	
    	按照上一条,这等价于讲$\mu_i(\mathfrak{p},A)=\delta_{i,\mathrm{ht}(\mathfrak{p})}$.
    \end{proof}
    \item 设$(A,\mathfrak{m})$是诺特局部环,设$M\not=0$是有限$A$模.设$M$具有有限的内射维数,那么$\mathrm{inj.dim}(M)=\mathrm{depth}(A)$.
    \begin{proof}
    	
    	设$M$的内射维数是$r<\infty$.于是$\mathrm{Ext}_A^i(-,M)=0,\forall i>r$成立.于是$\mathrm{Ext}_A^r(-,M)$是右正合函子.设$\mathfrak{p}$是异于$\mathfrak{m}$的素理想,取$x\in\mathfrak{m}-\mathfrak{p}$.那么数乘$x$是$A/\mathfrak{p}$上的单射,于是诱导的$\mathrm{Ext}_A^r(A/\mathfrak{p},M)\to\mathrm{Ext}_A^r(A/\mathfrak{p},M)$是满射.因为$M$是有限$A$模,所以$\mathrm{Ext}_A^r(A/\mathfrak{p},M)$也是有限模.按照NAK引理得到$\mathrm{Ext}_A^r(A/\mathfrak{p},M)=0$对任意和$\mathfrak{m}$不同的素理想$\mathfrak{p}$成立.但是我们之前解释过$\mathrm{inj.dim}(M)=r$得到$\mathrm{Ext}_A^r(A/\mathfrak{p},M)$在$\mathfrak{p}$取遍素理想时不能全是零.这迫使$\mathrm{Ext}_A^r(\kappa,M)\not=0$.但是我们还解释过$\mathrm{depth}(M)=\inf\{i\mid\mathrm{Ext}_A^i(\kappa,M)\not=0\}$,于是$r\ge\mathrm{depth}(M)$.
    	
    	\qquad
    	
    	下面设$t=\mathrm{depth}(A)$,设$x_1,\cdots,x_t\in\mathfrak{m}$是极大$A$正则序列,记$N=A/(xX_1,\cdots,x_t)A$.此时$N$没有正则元,也即$\mathfrak{m}$中每个元都是$N$的零因子.于是$\mathfrak{m}\in\mathrm{Ass}(N)$.于是有单射$\kappa\to N$.由于$\mathrm{Ext}_A^r(-,M)$是右正合的,并且$\mathrm{Ext}_A^r(\kappa,M)\not=0$,得到$\mathrm{Ext}_A^r(N,M)\not=0$.另一方面因为$x_1,\cdots,x_t$是$A$正则序列,所以Koszul复形$K_{\bullet}(x_1,\cdots,x_t)$是$N$的长度$t$的自由预解,于是$\mathrm{Ext}_A^i(N,-)=0,\forall i>r$和$\mathrm{Ext}_A^t(N,M)=M/(x_1,\cdots,x_t)M$.这里$M/(x_1,\cdots,x_t)M\not=0$,否则的话NAK引理说明$M=0$和条件矛盾.所以$\mathrm{proj.dim}(N)=t$,所以$\mathrm{Ext}_A^r(N,M)=0$导致$t\ge r$.另外$r=\mathrm{inj.dim}(M)$说明从$\mathrm{Ext}_A^t(N,M)\not=0$得到$t\le r$,综上有$t=r$.
    \end{proof}
    \item 设$(A,\mathfrak{m},\kappa)$是诺特局部环,设$A$是CM环,那么$A$存在一个内射维数有限(按照上一条,内射维数就必须是$\dim A=\mathrm{depth}(A)$)的非零有限$A$模.
    \begin{proof}
    	
    	取$A$的极大正则序列$\{x_1,\cdots,x_d\in\mathfrak{m}\}$,按照CM条件这也是参数系统,记$B=A/(x_1,\cdots,x_d)A$,那么$B$是有限长度的.记$E=E_A(\kappa)$.那么$M=B'=\mathrm{Hom}_A(B,E)$也是有限长度的,特别的它是有限生成的.我们就断言$M$的内射维数$\le d$.考虑Koszul复形$K_{\bullet}(x_1,\cdots,x_d)$是$A$模$B$的自由预解.作用正合函子$\mathrm{Hom}_A(-,E)$(因为$E$是内射模)后仍然是正合的,它是$M$的一个长度$d$的内射预解,所以$\mathrm{inj.dim}(M)\le d$.
    \end{proof}
    \item 上一条的逆命题也是成立的:如果$(A,\mathfrak{m},\kappa)$是诺特局部环,如果$A$存在内射维数有限的非零有限模$M$,那么$A$是CM环.
\end{enumerate}
\newpage
\subsection{正则环}

极小自由预解.设$(A,\mathfrak{m},k)$是局部环.
\begin{itemize}
	\item 考虑$\textbf{A-Mod}\to\textbf{k-Mod}$的函子为,把$A$模$M$映射为$\overline{M}=M/\mathfrak{m}M$,把$A$模同态$f:M\to N$映射为它诱导的唯一同态$\overline{f}:M/\mathfrak{m}M\to N/\mathfrak{m}N$.那么如果$f:M\to N$是有限自由$A$模之间的$A$模同态,那么$f$是同构当且仅当$\overline{f}$是同构.
	\item 设$M$是$A$模,它的一个有限秩自由预解$\xymatrix{(L_{\bullet},\mathrm{d}_{\bullet})\ar[r]^{\varepsilon}&M\ar[r]&0}$称为极小自由预解(minimal free resolution),如果这里每个$\overline{\mathrm{d}_i}=0,i\ge1$.
	$$L_{\bullet}=\xymatrix{\cdots\ar[r]&L_i\ar[r]^{\mathrm{d}_i}&\cdots\ar[r]&L_1\ar[r]^{\mathrm{d}_1}&L_0\ar[r]&0}$$
\end{itemize}
\begin{enumerate}
	\item 设$(A,\mathfrak{m},k)$是局部环,设$M$是有限$A$模.如果$\xymatrix{L_{\bullet}\ar[r]^{\varepsilon}&M}$是极小自由预解,记$K_i=\mathrm{im}(d_{i+1})\subseteq L_i$,那么有短正合列$\xymatrix{0\ar[r]&K_{i+1}\ar[r]&L_{i+1}\ar[r]&K_i\ar[r]&0}$,那么$\overline{L_{i+1}}\to\overline{K_i}$都是满射,从$\overline{d_{i+1}}=0$就得到$\overline{K_i}\to\overline{L_i}$都是零.这迫使$\overline{L_{i+1}}\to\overline{K_i}$是同构:它已经是满射了,任取$x\in L_{i+1}$使得$\overline{d_{i+1}}(\overline{x})=0$,那么$d_{i+1}(x)\in\mathfrak{m}K_i$,也即$d_{i+1}(x)=\sum_km_kd_{i+1}(x_k)$,其中$m_k\in\mathfrak{m}$和$x_k\in L_{i+1}$,于是$x-\sum_km_kx_k\in\ker d_{i+1}=\mathrm{im}d_{i+2}=K_{i+1}$,于是在$\mathrm{mod}\mathfrak{m}$下就有$\overline{x}\in\overline{K_{i+1}}$,结合$\overline{K_{i+1}}\to\overline{L_{i+1}}$是零得到$\overline{x}=0$.类似的可说明有$\overline{\varepsilon}$是同构.
	\item 设$(A,\mathfrak{m},k)$是诺特局部环,设$M$是有限$A$模,那么$M$的极小自由预解可以这样构造:取$\varepsilon:L_0\to M$是$A$模的满同态,其中$L_0$是有限自由模,并且$\overline{\varepsilon}$是同构(比方说先取$\overline{M}$作为$k$线性空间的一组基).设$K_0=\ker\varepsilon$,取$L_1\to K_0$是$A$模同态,使得$L_1$有限自由(这里有限用到了$A$是诺特的)并且诱导的$\overline{L_1}\to\overline{K_0}$是同构.考虑$L_1\to K_0$和包含映射$K_0\to L_0$的复合为$d_1$,那么$\overline{d_1}=0$.并且有正合列$\xymatrix{L_1\ar[r]^{d_1}&L_0\ar[r]^{\varepsilon}&M\ar[r]&0}$.继续归纳构造下去即可.
	\item 设$(A,\mathfrak{m},k)$是局部环,设$M$是有限$A$模.$M$的任意两个极小自由预解作为复形都是同构的.如果$L_{\bullet}\to M$和$L'_{\bullet}\to M$是两个极小自由预解,定义$K_i'=\mathrm{im}d_{i+1}'$.由于$L_{\bullet}$是$M$的自由预解,于是存在复形的同态$f_{\bullet}:L_{\bullet}\to L_{\bullet}'$延拓了$M$上的恒等映射,我们要证明所有$f_i$是同构,等价于证明所有$\overline{f_i}$是同构.下面设$f_i:L_i\to L_i'$诱导的$K_i\to K_i'$是$g_i$.首先$\overline{f_0}$是同构因为有如下交换图表:
	$$\xymatrix{\overline{L_0}\ar[rr]^{\overline{\varepsilon}}_{\cong}\ar[d]_{\overline{f_0}}&&\overline{M}\ar@{=}[d]\\\overline{L_0}'\ar[rr]^{\overline{\varepsilon'}}_{\cong}&&M}$$
	
	于是$f_0:L_0\to L_0'$是同构,它诱导的$g_0:K_0\to K_0'$也是同构,再按照$\overline{L_1}\cong\overline{K_0}$和$\overline{L_1}'\cong\overline{K_0}'$,得到$\overline{f_1}$是同构,归纳下去得到$L_{\bullet}$和$L_{\bullet}'$是同构的复形.
	\item 例子.设$(A,\mathfrak{m})$是局部环,设$x_1,\cdots,x_n\in\mathfrak{m}$是$A$序列,设$K_{\bullet}(\underline{x})$是Koszul复形,那么它提供了$A/(x_1,\cdots,x_n)$的有限自由预解,并且这些微分映射在$\mathrm{mod}\mathfrak{m}$下都是零,所以它是$A/(x_1,\cdots,x_n)$的极小自由预解.
	\item 引理.设$(A,\mathfrak{m},k)$是局部环,设$M$是有限$A$模,设$L_{\bullet}$是$M$的极小自由预解,那么:
	\begin{enumerate}
		\item $\mathrm{Tor}_i^A(M,k)\cong\overline{L_i}$和$\dim_k\mathrm{Tor}_i^A(M,k)=\mathrm{rank}(L_i),\forall i$.
		\item $\mathrm{proj.dim}_A(M)=\sup\{i\mid\mathrm{Tor}_i^A(M,k)\not=0\}\le\mathrm{proj.dim}_A(k)$.
		\item 如果$M\not=0$且$\mathrm{Proj.dim}(M)=r<\infty$,那么对每个有限$A$模$N\not=0$都有$\mathrm{Ext}_A^r(M,N)\not=0$.
	\end{enumerate}
    \begin{proof}
    	
    	(a)就是用$M$的极小自由预解计算$\mathrm{Tor}$.(b)的不等式就是投射维数的定义,按照投射维数的定义还有$\mathrm{proj.dim}_A(M)\ge\sup\{i\mid\mathrm{Tor}_i^A(M,k)\not=0\}=r$.另一方面如果$r\le s$,那么对$i>s$从(a)得到$L_i=0$,于是$L_{\bullet}$是长度$\le s$的$M$的自由预解,于是$\mathrm{proj.dim}(M)\le s$,于是得到另一侧的不等式.最后证明(c),按照(b)有$L_{r+1}=0$和$L_r\not=0$,于是$\mathrm{Ext}_A^r(M,N)$就是映射$\mathrm{d}_r^*:\mathrm{Hom}_A(L_{r-1},N)\to\mathrm{Hom}_A(L_r,N)$的余核.由于$\overline{d_r}=0$,导致$\mathrm{d_r}(L_r)\subseteq\mathfrak{m}L_{r-1}$,进而有$\mathrm{im}(d_r^*)\subseteq\mathfrak{m}\mathrm{Hom}_A(L_r,N)$,于是NAK引理导致$\mathrm{Ext}_A^r(M,N)=0$当且仅当$\mathrm{Hom}_A(L_r,N)=0$,但是$L_r$是自由模,$N\not=0$的时候$\mathrm{Hom}_A(L_r,N)\not=0$.
    \end{proof}
    \item 推论.设$(A,\mathfrak{m},k)$是局部环,设$M$是有限$A$模,设$i\ge0$是整数,如果$\mathrm{Tor}_i^A(M,k)=0$,那么$L_i=0$,于是对$j\ge i$总有$\mathrm{Tor}_j^A(M,k)=0$.特别的,$M$的极小自由预解的长度不能超过$M$的投射维数.
    \item 上述推论是如下刚性猜想(Rigidity conjecture)的特殊情况:设$A$是诺特环,设$M,N$是有限$A$模,如果$\mathrm{proj.dim}_A(M)<\infty$,如果$\mathrm{Tor}_i^A(M,N)=0$,那么$\mathrm{Tor}_j^A(M,N)=0,\forall j\ge i$.这个命题对$A$是正则环已经得证,一般情况还未解决.
    \item (Auslander,Buchsbaum)设$(A,\mathfrak{m},k)$是局部环,设$M\not=0$是有限$A$模,设$\mathrm{proj.dim}(M)<\infty$,那么:
    $$\mathrm{proj.dim}_A(M)+\mathrm{depth}(M)=\mathrm{depth}(A)$$
    \begin{proof}
    	
    	设$h=\mathrm{proj.dim}(M)$,我们对$h$做归纳.如果$h=0$,那么$M$是局部环$A$上的有限投射模,所以它是有限自由模,我们解释过$M$的深度就是最小的使得$\mathrm{Ext}_A^i(k,M)\cong\mathrm{Ext}_A^i(k,A)^n\not=0$的指标$i$,于是$\mathrm{depth}(A)=\mathrm{depth}(M)$.如果$h=1$,那么$M$的极小自由预解的长度不超过1,设为:
    	$$\xymatrix{0\ar[r]&A^m\ar[r]^{\varphi}&A^n\ar[r]^{\varepsilon}&M\ar[r]&0}$$
    	
    	由于$\overline{\varphi}=0$,如果把$\varphi$视为矩阵,它的项就都落在$\mathfrak{m}$中.这个短正合列诱导了如下长正合列:
    	$$\xymatrix{\cdots\ar[r]&\mathrm{Ext}_A^i(k,A^m)\ar[r]^{\varphi_*}&\mathrm{Ext}_A^i(k,A^n)\ar[r]^{\varepsilon_*}&\mathrm{Ext}_A^i(k,M)\ar[r]&\cdots}$$
    	
    	把$\mathrm{Ext}_A^i(k,A^m)$典范的视为$\mathrm{Ext}_A^i(k,A)^m$,把$\mathrm{Ext}_A^i(k,A^n)$典范的视为$\mathrm{Ext}_A^i(k,A)^n$.于是$\varphi_*$视为矩阵是和$\varphi$相同的,它的项都在$\mathfrak{m}$中,于是$\varphi_*=0$,于是对每个整数$i$就有如下短正合列:
    	$$\xymatrix{0\ar[r]&\mathrm{Ext}_A^i(k,A)^n\ar[r]&\mathrm{Ext}_A^i(k,M)\ar[r]&\mathrm{Ext}_A^{i+1}(k,A)^m\ar[r]&0}$$
    	
    	于是按照$\mathrm{depth}(M)=\inf\{i\mid\mathrm{Ext}_A^i(k,M)\not=0\}$,就得到$\mathrm{depth}(M)=\mathrm{depth}(A)-1$.最后设$h>1$,取定一个短正合列$0\to M'\to A^n\to M\to0$,取长正合列得到$\mathrm{proj.dim}(M')=h-1$,于是按照归纳假设有$d'=\mathrm{depth}(M')=\mathrm{depth}(A)-h+1<\mathrm{depth}(A)$.取长正合列得到$\mathrm{Ext}_A^{i-1}(k,M)\cong\mathrm{Ext}_A^i(k,M')$,于是当$i\le d'-1$时$\mathrm{Ext}_A^{i-1}(k,M)=0$,并且$\mathrm{Ext}_A^{d'-1}(k,M)\cong\mathrm{Ext}_A^{d'}(k,M)\not=0$,于是$\mathrm{depth}(M)=\mathrm{depth}(M')-1$,完成归纳.
    \end{proof}
\end{enumerate}

整体维数和正则环.
\begin{enumerate}
	\item 设$A$是环,$n\ge0$是整数,如下条件互相等价:
	\begin{enumerate}
		\item $\mathrm{proj.dim}(M)\le n$对任意$A$模$M$成立.
		\item $\mathrm{proj.dim}(M)\le n$对任意有限$A$模$M$成立.
		\item $\mathrm{inj.dim}(N)\le n$对任意$A$模$N$成立.
		\item $\mathrm{Ext}_A^{n+1}(M,N)=0$对任意$A$模$M,N$成立.
	\end{enumerate}
    \item 环$A$的整体维数定义为当$M$跑遍$A$模时$\mathrm{proj.dim}(M)$的上确界,也等价于$M$跑遍有限$A$模时它投射维数的上确界,记作$\mathrm{gl.dim}(A)$.另外我们之前解释过如果$(A,\mathfrak{m},k)$是局部环,那么$A$的整体维数就是$k$(作为$A$模)的投射维数.
    \item 设$(A,\mathfrak{m},k)$是$n$维诺特局部环,那么如下条件互相等价:
    \begin{enumerate}
    	\item $A$是正则的.
    	\item $\mathrm{gl.dim}(A)=\dim(A)$.
    	\item $\mathrm{gl.dim}(A)<\infty$
    \end{enumerate}
    \begin{proof}
    	
    	(a)$\Rightarrow$(b):设$\mathfrak{m}=(x_1,\cdots,x_n)$,其中$x_1,\cdots,x_n$是$A$正则序列,那么Koszul复形$K_{\bullet}(x_1,\cdots,x_n)$是$k=A/(x_1,\cdots,x_n)$的长度为$n$的极小自由预解,我们解释过极小自由预解的长度是不变的,于是有$\mathrm{gl.dim}(A)=\mathrm{proj.dim}(k)=n$.
    	
    	\qquad
    	
    	(b)$\Rightarrow$(c)是平凡的,下面证明(c)$\Rightarrow$(a):先设$r=\mathrm{gl.dim}(A)<\infty$,设$A$的嵌入维数是$s$,此为$k$线性空间$\mathfrak{m}/\mathfrak{m}^2$的维数.我们来对$s$归纳,如果$s=0$,NAK引理导致$\mathfrak{m}=0$,此时$A=k$是域,它当然是正则的.下面设$s\ge1$,于是$\mathfrak{m}\not=0$,NAK引理导致$\mathfrak{m}^2\subsetneqq\mathfrak{m}$.我们断言$\mathfrak{m}\not\in\mathrm{Ass}(A)$:如果存在$0\not=a\in A$,使得$\mathfrak{m}a=0$,考虑$A$模$k$的极小自由预解如下,由于整体维数有限得到这个预解长度有限:
    	$$\xymatrix{0\ar[r]&L_r\ar[r]&L_{r-1}\ar[r]&\cdots\ar[r]&L_0\ar[r]&k\ar[r]&0}$$
    	
    	按照$L_r\subseteq\mathfrak{m}L_{r-1}$,得到$aL_r=0$,但是$L_r$是自由模,它不能被一个非零元$a$零化,这个矛盾证明了我们的断言.于是我们可以选取一个元$x\in\mathfrak{m}-\left(\mathfrak{m}^2\cup\left(\cup_{\mathfrak{p}\in\mathrm{Ass}(A)}\mathfrak{p}\right)\right)$.于是特别的$x$是$A$正则元.记$B=A/xA$,对任意$B$模$N$有$0=\mathrm{Ext}_A^{r+1}(\mathfrak{m},N)\cong\mathrm{Ext}_B^{r+1}(\mathfrak{m}/x\mathfrak{m},N)$,于是$\mathrm{proj.dim}_B(\mathfrak{m}/x\mathfrak{m})\le r$.
    	
    	\qquad
    	
    	我们接下来断言典范同态$\mathfrak{m}/x\mathfrak{m}\to\mathfrak{m}/xA$分裂,由于$x\not\in\mathfrak{m}^2$,我们可以取$\mathfrak{m}$的一组生成元$x_1=x,x_2,\cdots,x_s$,这里$s$是$A$的嵌入维数.记$\mathfrak{b}=(x_2,\cdots,x_s)$,如果$ax\in\mathfrak{b}\cap xA$,那么它在$\mathfrak{m}/\mathfrak{m}^2$的像$\overline{a}\overline{x}$就必须是零,所以$a\in\mathfrak{m}$,所以$\mathfrak{b}\cap xA\subseteq x\mathfrak{m}$,于是我们得到如下复合为恒等的映射链:
    	$$\mathfrak{m}/xA\cong(\mathfrak{b}+xA)/xA\cong\mathfrak{b}/(\mathfrak{b}\cap xA)\to\mathfrak{b}/x\mathfrak{m}\to\mathfrak{m}/x\mathfrak{m}\to\mathfrak{m}/xA$$
    	
    	于是我们的断言是成立的,于是$\mathfrak{m}/xA$是$\mathfrak{m}/x\mathfrak{m}$的直和项,于是$\mathrm{proj.dim}_B(\mathfrak{m}/xA)\le\mathrm{proj.dim}_B(\mathfrak{m}/x\mathfrak{m})\le r$.考虑短正合列$0\to\mathfrak{m}/xA\to B\to k\to0$,得到$\mathrm{gl.dim}(B)=\mathrm{proj.dim}(k)\le r+1$有限.另外$B$的嵌入维数是$s-1$,归纳假设导致$B$是正则局部环,按照$x$是$A$正则元,于是$\dim A=\dim B+1=s$,完成归纳.
    \end{proof}
    \item 推论.如果$A$是正则局部环,任取$\mathfrak{p}\in\mathrm{Spec}A$,那么$A_{\mathfrak{p}}$也是正则局部环.
    \begin{proof}
    	
    	我们有$\mathrm{proj.dim}_A(A/\mathfrak{p})\le\mathrm{gl.dim}(A)<\infty$,于是$A/\mathfrak{p}$作为$A$模具有有限长度的投射预解$P_{\bullet}$,于是$P_{\bullet}\otimes_AA_{\mathfrak{p}}$是$(A/\mathfrak{p})\otimes_AA_{\mathfrak{p}}\cong\kappa(\mathfrak{p})$的有限长度投射预解,于是有$\mathrm{gl.dim}(A_{\mathfrak{p}})=\mathrm{proj.dim}_{A_{\mathfrak{p}}}(\kappa(\mathfrak{p}))<\infty$,于是$A_{\mathfrak{p}}$是正则局部环.
    \end{proof}
\end{enumerate}

正则环.一个未必局部的诺特环$A$称为正则环,如果它在每个素理想处的局部化都是正则局部环,等价的讲,它在每个极大理想处的局部化是正则局部环.
\begin{enumerate}
	\item 例如戴德金整环是正则环.
	\item 正则环都是正规环,即它在每个素理想处的局部化都是正规整环.
	\begin{proof}
		
		问题是局部的,只需证明正则局部环是正规整环.我们之前证明过正则局部环都是整环,还证明过一个诺特整环$A$是正规的当且仅当对每个高度1素理想$\mathfrak{p}$有$A_{\mathfrak{p}}$是DVR,并且对每个非零非单位$x$有$A/xA$的伴随素理想的高度都是1.所以问题归结为证明正则局部环满足这两件事.第一件事是因为DVR等价于维数为1的诺特局部环,并且极大理想是主理想.我们之前证明了$A_{\mathfrak{p}}$还是正则局部环,这里它是一维的,所以极大理想是主理想.下面证明第二件事,任取$0\not=x\in A$是非单位元,任取$\mathfrak{p}\in\mathrm{Ass}(A/xA)$,我们要证明$\mathfrak{p}$的高度是1,有$\mathfrak{p}A_{\mathfrak{p}}\in\mathrm{Ass}_{A_{\mathfrak{p}}}(A_{\mathfrak{p}}/xA_{\mathfrak{p}})$,并且局部化后这个素理想的高度还是1.所以通过把$A$替换为$A_{\mathfrak{p}}$,可不妨设$\mathfrak{p}$是$A$的高度1的极大理想.由于$A$是正则局部环,它是CM局部环,因为$A$是整环所以$x$是正则元,所以$A/xA$也是CM模,我们解释过此时对每个$A/xA$的伴随素理想$\mathfrak{p}$有$0=\dim(A/\mathfrak{p})=\dim_A(A/xA)=\dim A-1$,于是$\mathrm{ht}(\mathfrak{p})=\dim A=1$.
	\end{proof}
    \item 如果$A$是正则环,那么$A[X]$和$A[[X]]$也是正则环.
    \begin{proof}
    	
    	首先明显有$A[X]$和$A[[X]]$是诺特环.任取$A[X]$的极大理想$\mathfrak{p}$,记$\mathfrak{m}=\mathfrak{p}\cap A$,那么$A-\mathfrak{m}$包含在$\mathfrak{p}$的补集中,所以对$A[X]$先对$A-\mathfrak{m}$做局部化,也即$A_{\mathfrak{m}}[X]$,再对$\mathfrak{p}$的补集的像集作为乘性闭子集做局部化,得到$(A[X])_{\mathfrak{p}}$.换句话讲$(A[X])_{\mathfrak{p}}$是$A_{\mathfrak{m}}[X]$的局部化.所以用$A_{\mathfrak{m}}$替换$A$,我们不妨设$(A,\mathfrak{m},k)$是正则局部环.
    	
    	\qquad
    	
    	我们断言$A[X]$的极大理想$\mathfrak{p}$可以表示为$(\mathfrak{m}[X],f(X))$,其中$\overline{f}(X)$是$k[X]$中的不可约多项式:一方面$A[X]/(\mathfrak{m}[X],f(X))\cong k[X]/(\overline{f}(X))$是域,说明$(\mathfrak{m}[X],f(X))$的确是$A[X]$的极大理想;另一方面先说明$\mathfrak{m}\subseteq\mathfrak{p}$,否则的话就有$\mathfrak{m}[X]+\mathfrak{p}=A[X]$,于是存在$\mathfrak{p}$中的$g(X)$和$\mathfrak{m}[X]$中的$f(X)$使得$f(X)+g(X)=1$矛盾.于是$\mathfrak{p}$在$A[X]/\mathfrak{m}[X]\cong k[X]$中的像$\overline{\mathfrak{p}}$是$(\overline{f}(X))$,于是$\mathfrak{p}=(\mathfrak{m}[X],f(X))$.
    	
    	\qquad
    	
    	由于$A\to A[X]$是平坦映射,这个环扩张满足下降条件,此时有如下公式:$\mathrm{ht}(\mathfrak{p})=\mathrm{ht}(\mathfrak{m})+\dim(A[X]_{\mathfrak{p}}/\mathfrak{m}A[X]_{\mathfrak{p}})=\dim A+1$,这里$A[X]_{\mathfrak{p}}/\mathfrak{m}A[X]_{\mathfrak{p}}=A[X]\otimes_A\kappa(\mathfrak{m})=k[X]$是一维的.另一方面$\mathfrak{p}=(\mathfrak{m}[X],f(X))$被$\dim A+1$个元素生成,于是$A[X]_{\mathfrak{p}}$是正则局部环.
    	
    	\qquad
    	
    	下面设$B=A[[X]]$,设$\mathfrak{M}$是$B$的极大理想,那么$\mathfrak{m}=\mathfrak{M}\cap A$是$A$的极大理想,并且$\mathfrak{M}=(\mathfrak{m},X)$.我们有:
    	$$\widehat{B}_{\mathfrak{M}}\cong\lim\limits_{\substack{\leftarrow\\n}}B/\mathfrak{M}^n=\lim\limits_{\substack{\leftarrow\\n}}A[[X]]/(\mathfrak{m},X)^n\cong\widehat{A}_{\mathfrak{m}}[[X]]$$
    	
    	由于$A_{\mathfrak{m}}$是正则局部环,于是$\widehat{A}_{\mathfrak{m}}$是正则局部环,于是$\mathfrak{m}\widehat{A}_{\mathfrak{m}}$被$r=\dim\widehat{A}_{\mathfrak{m}}$个元生成,于是极大理想$(\mathfrak{m}\widehat{A}_{\mathfrak{m}},X)$被$r+1$个元生成,结合$\dim(\widehat{A}_{\mathfrak{m}}[[X]])=\dim\widehat{A}_{\mathfrak{m}}+1$得到$\widehat{A}_{\mathfrak{m}}[[X]]$是正则环.于是$\widehat{B}_{\mathfrak{M}}$是正则环.最后取完备化不改变环的维数和嵌入维数,所以一个环是正则环等价于取完备化是正则环,这就得到$B$是正则环.
    \end{proof}
\end{enumerate}

有限自由预解.$A$模$M$的有限自由预解(finite free resolution),简记FFR,是指如下形式的有限长度正合列,其中每个$F_i$都是有限自由$A$模.这个FFR的欧拉数(Euler number)是$\chi(M)=\sum_{i=0}^n(-1)^i\mathrm{rank}_A(F_i)$.按照Schanuel引理,这个数不依赖于FFR的选取.
$$\xymatrix{0\ar[r]&F_n\ar[r]&\cdots\ar[r]&F_1\ar[r]&F_0\ar[r]&M\ar[r]&0}$$
\begin{enumerate}
	\item 设$M$是$A$模,如果$M$具有FFR,由于局部化是正合函子,对素理想$\mathfrak{p}$,就有$A_{\mathfrak{p}}$模$M_{\mathfrak{p}}$也具有FFR,并且有$\chi_A(M)=\chi_{A_{\mathfrak{p}}}(M_{\mathfrak{p}})$.另外如果$M$本身是有限秩自由$A$模,那么$M$具有FFR并且$\chi(M)=\mathrm{rank}(M)$.
	\item 设$(A,\mathfrak{m})$是局部化,对每个有限子集$E\subseteq\mathfrak{m}$,都存在$0\not=y\in A$使得$yE=0$,那么具有FFR的$A$模只有有限自由模.另外如果$A$还是诺特的,选取$\mathfrak{m}$的有限生成元集,那么这里对$\mathfrak{m}$的要求等价于讲$\mathfrak{m}\in\mathrm{Ass}(A)$,也即$\mathrm{depth}(A)=0$.
	\begin{proof}
		
		先设$A$是诺特局部环,如果$A$模$M$具有FFR,那么$M$是有限模并且具有有限投射维数,我们证明过此时有$\mathrm{proj.dim}_A(M)+\mathrm{depth}(M)=\mathrm{depth}(A)=0$,于是只能有$M$的投射维数是零,也即它是有限投射模,但是局部环上有限投射模是有限自由模.对于一般情况,先取$M$的FFR为:
		$$\xymatrix{0\ar[r]&F_n\ar[r]&\cdots\ar[r]&F_1\ar[r]&F_0\ar[r]&M\ar[r]&0}$$
		
		记$N=\mathrm{coker}(F_n\to F_{n-1})$,那么$0\to F_n\to F_{n-1}\to N\to0$是$N$的FFR,所以一旦我们证明了命题对$n=1$成立,那么$N$本身也是有限自由模,并且有$0\to N\to F_{n-2}\to\cdots\to F_1\to0$是$M$的长度少1的FFR,归纳法就归结为$n=1$的情况.此时$M$的投射维数$\le1$,并且$M$是有限$A$模.记函子$M\mapsto\overline{M}=M/\mathfrak{m}M$,可取有限自由模$L_0$和满同态$L_0\to M$,使得它诱导的$\overline{L_0}\to\overline{M}$是同构.记$L_1=\ker(L_0\to M)$,因为$M$的投射维数不超过1,所以这里$L_1$是$A$上投射模,于是它是自由模(此时还不能说明它是有限秩的).按照Schanuel引理得到$A$模的同构$L_0\oplus F_0\cong L_0\oplus F_1$,于是$L_1$是有限自由的.另外按照$\overline{L_0}\cong\overline{M}$,有$\overline{L_1}\to\overline{L_0}$是平凡映射,于是$L_1\to L_0$的像落在$\mathfrak{m}L_0$中.选取$L_1$的有限生成元集,它表示为$\mathfrak{m}L_0$的形式只涉及到$\mathfrak{m}$中有限个元,构成的集合记作$E\subseteq\mathfrak{m}$,按照条件可以选取$0\not=y\in A$使得$yE=0$.于是有$yL_1=0$,但是$L_1$是有限$A$模,NAK引理得到$L_1=0$,于是$M\cong L_0$是有限自由的.
	\end{proof}
    \item 设$M$是$A$模,具有FFR,那么$\chi(M)\ge0$.
    \begin{proof}
    	
    	取$A$的极小素理想$\mathfrak{p}$,那么$M_{\mathfrak{p}}$是具有FFR的$A_{\mathfrak{p}}$模,并且$\chi_A(M)=\chi_{A_{\mathfrak{p}}}(M_{\mathfrak{p}})$.于是我们不妨设$(A,\mathfrak{p})$是局部环,并且$\mathfrak{p}=\mathrm{nil}(A)$是幂零根.我们断言上一条的条件是满足的,一旦这成立,这里$M$只能是有限自由模,所以当然有$\chi_A(M)\ge0$.下面证明断言:如果$x_1,\cdots,x_r\in\mathfrak{p}$是有限个元,我们来对$r$归纳证明可以找到$0\not=y\in A$满足$yx_i=0,\forall 1\le i\le r$.对于$r=1$是因为$x_1\in\mathfrak{p}$是幂零元,如果命题对$r-1$成立,那么存在$0\not=z\in A$使得$zx_i=0,\forall 1\le i\le r-1$,因为$x_r$是幂零的,可以找整数$i$使得$zx_r^i\not=0$但$zx_r^{i+1}=0$.取$y=zx_r^i$就满足$y\not=0$并且$yx_i=0,\forall 1\le i\le r$.
    \end{proof}
    \item (Auslander-Buchsbaum).设$A$是诺特环,设$M$是$A$模具有FFR,那么如下条件互相等价:
    \begin{enumerate}
    	\item $\mathrm{Ann}(M)\not=0$.
    	\item $\chi_A(M)=0$.
    	\item $\mathrm{Ann}(M)$包含了某个$A$正则元.
    \end{enumerate}
    \begin{proof}
    	
    	(a)$\Rightarrow$(b):假设$\chi(M)>0$,那么对每个素理想$\mathfrak{p}\in\mathrm{Ass}(A)\not=\emptyset$,都有$\chi_{A_{\mathfrak{p}}}(M_{\mathfrak{p}})>0$,特别的$M_{\mathfrak{p}}\not=0$.另外从$\mathfrak{p}A_{\mathfrak{p}}\in\mathrm{Ass}_{A_{\mathfrak{p}}}(A_{\mathfrak{p}})$,按照之前的命题就有$M_{\mathfrak{p}}$是自由$A_{\mathfrak{p}}$模.记$I=\mathrm{Ann}(M)$,就有$0=\mathrm{Ann}_{A_{\mathfrak{p}}}(M_{\mathfrak{p}})=I_{\mathfrak{p}}$.记$J=\mathrm{Ann}_A(I)$,那么$\mathfrak{p}\not\in\mathrm{Supp}_A(I)=V(\mathrm{Ann}(J))$,于是$J\not\subseteq\mathfrak{p}$对任意$\mathfrak{p}\in\mathrm{Ass}(A)$成立,但是伴随素理想的并是全部正则元,于是$J$包含了某个$A$正则元,于是从$JI=0$得到$I=0$,这和条件矛盾.
    	
    	\qquad
    	
    	(b)$\Rightarrow$(c):如果$\chi_A(M)=0$,那么对每个伴随素理想$\mathfrak{p}$有$\chi_{A_{\mathfrak{p}}}(M_{\mathfrak{p}})=0$.有$\mathfrak{p}A_{\mathfrak{p}}$是$A_{\mathfrak{p}}$的伴随素理想,于是有限模$M_{\mathfrak{p}}\not=0$满足前面定理的条件,就导致$M_{\mathfrak{p}}$是有限自由模,于是从$\chi_{A_{\mathfrak{p}}}(M_{\mathfrak{p}})=0$得到$M_{\mathfrak{p}}=0$.此即$\mathrm{Ann}(M)\not\subseteq\mathfrak{p}$对任意伴随素理想$\mathfrak{p}$成立.于是$\mathrm{Ann}(M)$包含某个$A$正则元.
    \end{proof}
    \item (Vascibcelos).设$(A,\mathfrak{m})$是诺特局部环,设$I\subsetneqq A$是真理想,设$I$的投射维数有限(例如如果$A$是正则环),那么如下两个条件等价:
    \begin{enumerate}
    	\item $I$被$A$正则序列生成.
    	\item $I/I^2$是自由$A/I$模.
    \end{enumerate}
    \begin{proof}
    	
    	(a)$\Rightarrow$(b):设$I$被$A$正则序列$\{x_1,\cdots,x_r\}$生成,那么典范的分次$A/I$代数映射$(A/I)[X_1,\cdots,X_r]\to\mathrm{gr}_A^I=\oplus_{n\ge0}I^n/I^{n+1}$,$X_i\mapsto\overline{x_i}$是同构(我们之前解释过拟正则序列就保证它是同构).特别的有$I/I^2$是自由$A/I$模.
    	
    	\qquad
    	
    	(b)$\Rightarrow$(a):不妨设$I\not=0$,否则约定空集生成的理想是零理想.因为$I$具有有限投射维数,于是$A/I$作为$A$模也具有有限投射维数.因为$A$是局部环,于是投射$A$模都是自由模,于是$A$模$A/I$具有FFR.由于$\mathrm{Ann}(A/I)=I\not=0$,上一条的证明里得到$I$不包含在每个伴随素理想里,于是$I$不包含在所有$A$的伴随素理想的并以及$I\mathfrak{m}$中(因为不能有$I=I\mathfrak{m}$),于是可取$x\in I$,但$x$不在$A$的每个伴随素理想里,也不在$I\mathfrak{m}$里.那么$x$是$A$正则的,并且$x$在$I/I^2$中的像$\overline{x}$是$A/I$自由模$I/I^2$的基的一部分,因为NAK引理的推论说明$A/\mathfrak{m}$模$I/I\mathfrak{m}$的基可以提升为$A/I$模$I/I^2$的基.
    	
    	\qquad
    	
    	下面设$B=A/xA$,我们断言$I^*=I/xA$是投射维数有限的$B$模:因为$x$是$A$正则的,对每个整数$i$就有$\mathrm{Ext}_A^i(I,N)\cong\mathrm{Ext}_B^i(I/xI,N)$.于是从$\mathrm{proj.dim}_A(I)<\infty$得到$\mathrm{proj.dim}_B(I/xI)<\infty$.和我们在证整体维数有限推正则环中一样,有$I^*=I/xA$是$I/xI$的直和项.于是$\mathrm{proj.dim}_B(I^*)\le\mathrm{proj.dim}_B(I/xI)<\infty$.按照$I^*/(I^*)^2\cong I/(I^2+Ax)\cong(I/I^2)/((A/I)x)$,其中$I/I^2$是$A/I$自由模,并且$x$在$I/I^2$上的像能张成一组基,于是$(I/I^2)/((A/I)x)$是自由$A/I=B/I^*$模.并且$\mathrm{rank}_{A/I}(I/I^2)=\mathrm{rank}_{B/I^*}(I^*/(I^*)^2)-1$,所以只要对$\mathrm{rank}_{A/I}(I/I^2)$归纳得证.
    \end{proof}
    \item 设$(A,\mathfrak{m})$是正则局部环,设$x_1,\cdots,x_n\in\mathfrak{m}$,只要$a_i\in A$满足$\sum_ia_ix_i=0$,就有$a_i\in(x_1,\cdots,x_n)$对任意$i$成立.我们断言$x_1,\cdots,x_n$是$A$正则序列.
    \begin{proof}
    	
    	设$I=(x_1,\cdots,x_n)\subseteq A$是真理想,条件等价于讲$x_1,\cdots,x_n$在$I/I^2$中的像$\overline{x_1},\cdots,\overline{x_n}$是$A/I$模$I/I^2$的一组基.由于$A$是正则局部环,于是$A$具有有限整体维数,于是$I$的投射维数有限,上一条就说明$I$被一组$A$正则序列生成,记作$\{y_1,\cdots,y_r\}$.于是$I/I^2$作为$A/I$自由模的秩为$r$,于是$r=n$.如果记$x_i=\sum_ja_{ij}y_j$,那么$\left(a_{ij}\right)$是$A$上的矩阵,把它放在$A/I$上会是一个可逆矩阵,于是$\det\left(a_{ij}\right)\not\in\mathfrak{m}$,于是$\left(a_{ij}\right)$是$A$上的可逆矩阵.那么$\{x_1,\cdots,x_r\}$是拟正则序列【】,而在条件下正则序列和拟正则序列是相同的,于是$\{x_1,\cdots,x_r\}$是正则序列.
    \end{proof}
    \item Vascibcelos定理还可以提供诺特局部环上整体维数有限$\Rightarrow$正则局部环的另一个证明:设$(A,\mathfrak{m})$是整体维数有限的诺特局部环,那么$\mathrm{proj.dim}(\mathfrak{m})<\infty$.这里$\mathfrak{m}/\mathfrak{m}^2$是自由$A/\mathfrak{m}$模,按照Vasconcelos定理就有$\mathfrak{m}$被$A$正则序列生成,于是得到:
    $$\mathrm{emb.dim}(A)\le\mathrm{depth}(A)\le\dim(A)\le\mathrm{emb.dim}(A)$$
    
    于是$A$的嵌入维数$\mathrm{emb.dim}(A)=\dim A$,也即$A$是正则的.
\end{enumerate}
\newpage
\subsection{UFD}

稳定自由.设$M$是$A$模,它称为稳定自由的(stably free),如果存在有限自由$A$模$F,F'$,使得$M\oplus F\cong F'$.于是特别的稳定自由$A$模都是有限模.
\begin{enumerate}
	\item 例如如果$A$是具有FFR的投射模,那么$A$是稳定自由的.
	\item 设$A$是整环,设$I\subseteq A$是理想,并且是稳定自由的,那么$I$是主理想.
	\begin{proof}
		
		不妨设$I\not=0$,设$F,F'$是两个有限自由$A$模满足$I\oplus F\cong F'$.记$K=\mathrm{Frac}(A)$,对素理想$(0)$做局部化,按照$I_{(0)}\subseteq A_{(0)}=K$,得到$I_{(0)}$作为$K$模是一维的,于是我们有$\mathrm{rank}_A(F)+1=\mathrm{rank}_A(F')$.于是存在正整数$n$使得有同构$\varphi:A^{n+1}\cong I\oplus A^n$.固定$A^{n+1}$的一组基$e_0,\cdots,e_n$,把$I\oplus A^n$视为$A\oplus A^n$的子模,取$A$的基$\{f_0\}$和$A^n$的一组基$\{f_1,\cdots,f_n\}$.我们记$\varphi(e_j)=\sum_{i=0}^na_{ij}f_i$,其中$a_{ij}\in A$.记$M=\left(a_{ij}\right)$,那么有$\varphi(e_0,\cdots,e_n)=(f_0,\cdots,f_n)A$.记$M$的$(0,j)$元的代数余子式为$d_j$,记$\det M=d$,那么有$\sum_{j=0}^na_{0j}d_j=d$和$\sum_{j=0}^na_{ij}d_j=0,\forall i\not=0$.再记$e_0'=\sum_jd_je_j$,那么有$\varphi(e_0')=\sum_{j=0}^n\left(d_j\sum_{i=0}^na_{ij}e_i\right)=df_0$.又因为$f_1,\cdots,f_n$落在$\mathrm{im}\varphi$中,于是可取$e_1',\cdots,e_n'\in A^{n+1}$,使得$\varphi(e_i')=f_i,\forall i\ge1$.于是我们有$n+1$阶矩阵$N$使得$(e_0',\cdots,e_n')=(e_0,\cdots,e_n)N$.另一方面$\varphi(e_0',\cdots,e_n')=(f_0,\cdots,f_n)\mathrm{diag}(d,1,\cdots,1)$,这得到$MN=\mathrm{diag}(d,1,\cdots,1)$.但是$d=\det M$,结合$A$是整环,就得到$\det N=1$,于是$N$是可逆矩阵,于是$\{e_0',\cdots,e_n'\}$是$A^{n+1}$的一组基,并且$\mathrm{im}\varphi=Adf_0\oplus A^n$,于是$I=df_0A$是主理想.
	\end{proof}
\end{enumerate}

正则环和UFD.
\begin{enumerate}
	\item 正则局部环总是UFD.
	\begin{proof}
		
		设$(A,\mathfrak{m})$是正则局部化,我们对$\dim A$归纳证明$A$是UFD.首先$\dim A=0,1$的时候$A$是域或者DVR,此时$A$都是UFD.下面设$\dim A\ge2$,任取$f\in\mathfrak{m}-\mathfrak{m}^2$,由于$A/fA$仍然是正则局部环,于是特别的$A/fA$是整环,于是$fA$是素理想,即$f$是素元.我们之前解释过对于诺特整环,如果乘性闭子集$S$被一些素元生成,如果$S^{-1}A$是UFD,那么可以得到$A$是UFD.所以这里归结为证明$A_f$是UFD.
		
		\qquad
		
		设$\mathfrak{P}$是$A_f$的高度1素理想,设$\mathfrak{p}=\mathfrak{P}\cap A$,于是$\mathfrak{P}=\mathfrak{p}A_f$.由于$A$是正则局部环,它的整体维数有限,于是$\mathfrak{p}$的投射维数有限,于是$\mathfrak{p}$作为$A$模的极小自由预解是有限长度的,于是$\mathfrak{p}$具有FFR.按照局部化是正合的,就有$\mathfrak{P}$作为$A_f$模也具有FFR.如果我们能证明$\mathfrak{P}$是$A_f$投射模,那么$\mathfrak{P}$作为$A_f$模就是稳定自由的,前面定理就说明$\mathfrak{P}$是主理想,于是$A_f$是UFD.
		
		\qquad
		
		诺特环上投射模是一个局部性质.任取$A_f$的素理想$\mathfrak{Q}$,记$\mathfrak{q}=\mathfrak{Q}\cap A$,由于$f\in\mathfrak{m}$但$f\not\in\mathfrak{q}$,于是$\mathrm{ht}(\mathfrak{Q})=\mathrm{ht}(\mathfrak{q})<\mathrm{ht}(\mathfrak{m})$.于是$(A_f)_{\mathfrak{Q}}\cong A_{\mathfrak{q}}$是一个维数$<\dim A$的正则局部环,归纳假设说明它是UFD.于是$\mathfrak{P}_{\mathfrak{Q}}$要么是$(A_f)_{\mathfrak{Q}}$本身,要么是高度1的素理想,后者情况下它是主理想,所以作为模也是自由模,于是无论如何$\mathfrak{P}_{\mathfrak{Q}}$是自由$(A_f)_{\mathfrak{Q}}$模,于是$\mathfrak{P}$是投射$A_f$模.
	\end{proof}
    \item 推论.设$A$是诺特整环,如果每个有限$A$模都具有FFR,则$A$是UFD.
    \begin{proof}
    	
    	先说明$A$是正则环.任取素理想$\mathfrak{p}$,那么$A$模$A/\mathfrak{p}$具有FFR,于是$A_{\mathfrak{p}}$模$(A/\mathfrak{p})_{\mathfrak{p}}=\kappa(\mathfrak{p})$也具有FFR.于是$\kappa(\mathfrak{p})$的投射维数有限,于是$A_{\mathfrak{p}}$的整体维数有限,于是$A_{\mathfrak{p}}$是正则局部环,于是$A$是正则环.
    	
    	\qquad
    	
    	下面证明$A$是UFD,任取高度1素理想$\mathfrak{p}$,如果我们证明$\mathfrak{p}$作为$A$模是投射的,则$\mathfrak{p}$是稳定自由模,于是前面定理说明$\mathfrak{p}$是主理想,于是$A$就是UFD.但是投射模是局部性质,对$A$的极大理想$\mathfrak{m}$,有$A_{\mathfrak{m}}$是UFD.那么$\mathfrak{p}_{\mathfrak{m}}\subseteq A_{\mathfrak{m}}$要么是单位理想要么是高度1素理想,但是$A_{\mathfrak{m}}$是UFD导致高度1素理想是主的,于是此时它作为$A_{\mathfrak{m}}$模也是自由模,于是无论如何$\mathfrak{p}_{\mathfrak{m}}$都是自由$A_{\mathfrak{m}}$模,于是$\mathfrak{p}$是投射$A$模.
    \end{proof}
    \item 推论.设$A$是正则环,设$u,v\in A$,那么$uA\cap vA$总是投射$A$模.
    \begin{proof}
    	
    	投射模是一个局部性质,任取$A$的极大理想$\mathfrak{m}$,我们有$(uA\cap vA)_{\mathfrak{m}}=uA_{\mathfrak{m}}\cap vA_{\mathfrak{m}}$.由于$A_{\mathfrak{m}}$是正则局部环,它是UFD,于是两个元的最小公倍数总存在,也即$uA_{\mathfrak{m}}\cap vA_{\mathfrak{m}}$是主理想,于是它是自由$A_{\mathfrak{m}}$模,于是$uA\cap vA$是投射$A$模.
    \end{proof}
    \item 高斯引理.设$A$是UFD,其上的一个多项式$f(X)$称为本原的,如果它的系数的最大公约数是1.高斯引理是说两个本原多项式的乘积总是本原的.
    \item 引理.设$A$是UFD,设$I\subseteq A$是理想,作为$A$模是投射的,那么$I$是主理想.
    \begin{proof}
    	
    	我们解释过整环上分式理想是投射模等价于它可逆,于是这里$I$是可逆的,它的逆是$I^{-1}=\{x\in K=\mathrm{Frac}(A)\mid xI\subseteq A\}$,满足$I^{-1}I=A$.于是存在有限个$a_i\in I$和$u_i\in I^{-1}\subseteq K$使得$\sum_ia_iu_i=1$.特别的从$I\subseteq u_i^{-1}A$得到$I\subseteq\cap_iu_i^{-1}A$.另一方面如果$x\in\cap_iu_i^{-1}A$,那么$u_ix\in A$,于是$x=\sum_ia_iu_ix\in I$,于是$I=\cap_ia_iA$.按照$A$是UFD,我们解释过它的一族主分式理想的交还是主的,这就说明$I$是主的.
    \end{proof}
    \item 设$A$是UFD,那么$A[X]$也是UFD.如果$A$还是正则环,那么$A[[X]]$也是UFD.
    \begin{proof}
    	
    	第一件事之前证过了.下面假设$A$是正则的UFD,记$B=A[[X]]$,按照正则环的定义$A$是诺特的,于是$B$也是诺特环,于是特别的它对主理想满足升链条件.我们之前解释过对整环$A$,如果它对主理想满足升链条件,并且任意两个元存在最小公倍数,那么$A$是UFD.于是问题归结为证明对非零元$u,v\in B$,有$I=uB\cap vB$是主理想.记$I=X^rJ$,其中理想$J\not\subseteq(X)$.我们之前解释过$A$是正则环导致$B$是正则环,还解释过对正则环$B$有$I=uB\cap vB$是投射$B$模.进而$J$是投射$B$模.又因为$A\cong B/(X)$是投射$B$模,于是$J/XJ\cong J\otimes_BB/(X)\cong J\otimes_BA$是投射$A$模.因为理想$J$是投射$B$模,我们解释过它在素理想的局部化是主理想.下面断言$J$在$B$中的每个伴随素理想$\mathfrak{p}$的高度都是1:由于$J_{\mathfrak{p}}$是主理想,满足$\mathfrak{p}B_{\mathfrak{p}}\in\mathrm{Ass}_{B_{\mathfrak{p}}}(B_{\mathfrak{p}}/JB_{\mathfrak{p}})$,主理想定理导致$\mathfrak{p}$的高度是1.接下来主理想定理还说明$(X)\subseteq B$是高度1的素理想,由于$J\not\subseteq(X)$,导致$X\in B$不在$J$的任何伴随素理想中.于是$X$是$B/J$正则元,于是$(J:X)=J$,于是$J\cap(X)=XJ$.于是$J/XJ$可视为$B/(X)\cong A$的理想,由于$A$是UFD,得到投射$A$模$J/XJ$是主理想,于是$J=XJ+fB$,$f\in J$.又因为$X\in\mathrm{rad}(B)$,NAK引理导致$J=fB$,于是$I=(X^rf)$,得证.
    \end{proof}
\end{enumerate}

UFD和Krull环.
\begin{enumerate}
	\item 设$A$是整环,商域记作$K$,回顾$A$称为Krull环如果如下两个条件成立:
	\begin{enumerate}
		\item 对每个高度1素理想$\mathfrak{p}$有$A_{\mathfrak{p}}$是DVR.
		\item 对每个$0\not=a\in A$,主理想$aA$是有限个高度1准素理想的交.
	\end{enumerate}
    
    我们断言此时对$0\not=a\in A$,有$a$只包含在有限个高度1素理想中,并且$aA=\cap_{\mathfrak{p},\mathrm{ht}(\mathfrak{p})=1}(aA_{\mathfrak{p}}\cap A)$.
    \begin{proof}
    	
    	先记$aA=\mathfrak{q}_1\cap\cdots\cap\mathfrak{q}_r$,其中$\mathfrak{q}_i$是$\mathfrak{p}_i$准素理想,并且$\mathfrak{p}_i$是两两不同的高度1的.任取包含了$a$的高度1素理想$\mathfrak{p}$,从$\mathfrak{q}_1\cap\cdots\cap\mathfrak{q}_r\subseteq\mathfrak{p}$得到某个$\mathfrak{q}_i\subseteq\mathfrak{p}$,于是$\mathfrak{p}_i\subseteq\mathfrak{p}$,但是它们都是高度1的,得到$\mathfrak{p}=\mathfrak{p}_i$.
    	
    	\qquad
    	
    	一般的如果$B$是平坦$A$代数,对$A$的理想$I_1,I_2$就有$(I_1\cap I_2)B=(I_1B)\cap(I_2B)$,所以这里$aA_{\mathfrak{p}_i}=\left(\cap_j\mathfrak{q}_j\right)A_{\mathfrak{p}_i}=\cap_j\mathfrak{q}_jA_{\mathfrak{p}_i}=\mathfrak{q}_iA_{\mathfrak{p}_i}$.于是$\mathfrak{q}_i=\mathfrak{q}_iA_{\mathfrak{p}_i}\cap A=aA_{\mathfrak{p}_i}\cap A$.取交得到$aA=\cap_{\mathfrak{p},\mathrm{ht}(\mathfrak{p})=1}(aA_{\mathfrak{p}}\cap A)$.
    \end{proof}
    \item UFD是Krull环.
    \begin{proof}
    	
    	设$A$是UFD,我们来验证Krull环的两个条件.任取$A$的高度1的素理想$\mathfrak{p}$,那么它是主理想,记作$\mathfrak{p}=(f)$,于是$A_{\mathfrak{p}}$是1维局部整环,它的非零素理想只有唯一的极大理想,并且是主理想,于是$A_{\mathfrak{p}}$是诺特环(如果全部素理想是有限生成的,就得到全部理想是有限生成的).于是$A_{\mathfrak{p}}$是诺特局部环,维数大于0,并且极大理想是主理想,于是$A_{\mathfrak{p}}$是DVR.
    	
    	\qquad
    	
    	下面任取$0\not=a\in A$,不妨设它非单位(约定了空集的交是单位理想),我们要证明$aA$是有限个高度1的准素理想的交.记$a=up_1^{e_1}\cdots p_r^{e_r}$,其中$u$是单位,$p_i$是两两不同的素元.那么$aA=\cap_{i=1}^rp_i^{e_i}A$.另外如果$p$是素元,那么$p^eA$是高度1的准素理想:任取$\mathfrak{q}\in\mathrm{Ass}_A(A/p^eA)$,有$x\in A-p^eA$使得$(p^eA:x)=\mathfrak{q}$.由于$A$是UFD,因为$x\not\in p^eA$,于是$p^eA\subseteq\mathfrak{q}\subseteq pA$.于是$\mathfrak{q}=pA$,于是$p^eA$是高度1的准素理想.
    \end{proof}
\end{enumerate}

除子.设$A$是Krull环.
\begin{enumerate}
	\item 记$\mathscr{P}(A)$表示$A$的全部高度1的素理想.记$D(A)$是由$\mathscr{P}(A)$生成的自由阿贝尔群,它的元素称为除子.
	\item 设$K=\mathrm{Frac}(A)$,任取$0\not=a\in K$,它生成的除子是指$\mathrm{div}(a)=\sum_{\mathfrak{p}\in\mathscr{P}(A)}v_{\mathfrak{p}}(a)\mathfrak{p}$.其中$v_{\mathfrak{p}}:K^*\to\mathbb{Z}$是$A_{\mathfrak{p}}$上的规范离散赋值.那么对几乎所有的高度1素理想$\mathfrak{p}$有$v_{\mathfrak{p}}(a)=0$.并且$\mathrm{div}(ab)=\mathrm{div}(a)+\mathrm{div}(b)$.称可以表示为$\mathrm{div}(a)$的除子为主除子,全部主除子构成$D(A)$的子群,记作$F(A)$.
	\item 称$C(A)=D(A)/F(A)$是$A$的除子类群.
	\item 设$A$是Krull环,那么$A$是UFD当且仅当$C(A)=0$.所以除子类群度量Krull环和UFD的差距.
	\begin{proof}
		
		一方面如果$A$是UFD,那么它的高度1素理想都是主理想,于是每个高度1素理想本身都是主除子,所以$C(A)=0$.反过来设$C(A)=0$.我们来验证之前给出的整环是DVR的三个条件.
		\begin{itemize}
			\item $A$对主理想满足升链条件.取主理想升链$(a_1)\subseteq(a_2)\subseteq\cdots$,不妨设$a_1\not=0$.对每个高度1的素理想$\mathfrak{p}$,就有$v_{\mathfrak{p}}(a_1)\ge v_{\mathfrak{p}}(a_2)\ge\cdots\ge0$.于是当$i$足够大时有$\mathrm{div}(a_i)=\mathrm{div}(a_{i+1})$,于是$\mathrm{div}(a_i/a_{i+1})=0$,也即$a_i/a_{i+1}\in A-\{0\}$在$A_{\mathfrak{p}}$中可逆对任意高度1素理想$\mathfrak{p}$成立.再按照$\mathfrak{p}$取遍高度1素理想时$\cap A_{\mathfrak{p}}=A$,于是$a_i/a_{i+1}$和$a_{i+1}/a_i$都在$A$中,于是$a_i$和$a_{i+1}$相伴,也即$(a_i)=(a_{i+1})$.
			\item 对每个非零非单位$a\in A$,它都包含在某个高度1的素理想中.由于$A$是Krull环,有$aA=\mathfrak{q}_1\cap\cdots\cap\mathfrak{q}_r$,其中$\mathfrak{q}_i$是$\mathfrak{p}_i$准素理想,并且$\mathfrak{p}_i$是高度1的.于是$a\in\mathfrak{q}_i\subseteq\mathfrak{p}_i$.
			\item $A$的高度1素理想$\mathfrak{p}$总是主理想.由于$C(A)=0$,存在$a\in A$使得$\mathfrak{p}=\mathrm{div}(a)$.于是$v_{\mathfrak{p}}(a)=1$,对任意其余的高度1素理想$\mathfrak{q}$都有$v_{\mathfrak{q}}(a)=0$.于是$aA=\cap_{q\in\mathscr{P}(A)}(aA_{\mathfrak{q}}\cap Q)=\mathfrak{p}A_{\mathfrak{p}}\cap A=\mathfrak{p}$,于是$\mathfrak{p}$是主理想.
		\end{itemize}
	\end{proof}
\end{enumerate}

Picard群.设$A$是环.
\begin{itemize}
	\item 一个有限投射$A$模称为秩$r$的,如果对每个素理想$\mathfrak{p}$有$A_{\mathfrak{p}}$模$M_{\mathfrak{p}}$是秩$r$自由模.
	\item 全体秩1有限投射模的同构类构成的集合记作$\mathrm{Pic}(A)$,其上有张量积作为群运算,即两个秩1有限投射模的张量积还是秩1有限投射模,并且$A$模$A$所在的同构类是乘法幺元.秩1有限投射模$P$的逆元是$\mathrm{Hom}_A(P,A)$,因为有:
	$$P\otimes_A\mathrm{Hom}_A(P,A)\cong A$$
\end{itemize}
\begin{enumerate}
	\item 如果$A$是整环,$K$是商域,如果$I$是投射分式理想(等价于可逆分式理想),我们之前解释过它在素理想处的局部化都是秩1自由的,于是$A$的可逆分式理想总是秩1投射模.全体可逆分式理想构成的群记作$F(A)$,全体主分式理想构成的子群记作$P(A)$,那么$F(A)\to\mathrm{Pic}(A)$诱导了同态$F(A)/P(A)\to\mathrm{Pic}(A)$,我们断言这是同构.
	\begin{proof}
		
		如果$I,J$是可逆分式理想,投射模是平坦的,平坦模有理想准则,于是$IJ\cong I\otimes_AJ$,所以$F(A)\to\mathrm{Pic}(A)$是同态.下面证满射,如果$M$是秩1投射模,那么$M_{(0)}$是秩1$K=A_{(0)}$自由模.于是$M\subseteq M_{(0)}$作为$A$模同构于一个分式理想,于是$\varphi$是满同态.最后要证明$F(A)\to\mathrm{Pic}(A)$的核是$P(A)$,这是因为一个可逆分式理想$I$如果作为$A$模同构于$A$,那么$I$是主理想.
	\end{proof}
    \item 设$A$是Krull环,商域记作$K$,设$I$是分式理想,对每个高度1素理想$\mathfrak{p}$,定义$v_{\mathfrak{p}}(I)=\min\{v_{\mathfrak{p}}(x)\mid x\in I\}$.特别的如果$I$是可逆分式理想,那么$IA_{\mathfrak{p}}=aA_{\mathfrak{p}}$,那么$v_{\mathfrak{p}}(I)=v_{\mathfrak{p}}(a)$.这里我们断言对任意分式理想$I$,对几乎所有的高度1素理想$\mathfrak{p}$,都有$v_{\mathfrak{p}}(I)=0$.于是$\sum_{\mathrm{ht}(\mathfrak{p})=1}v_{\mathfrak{p}}(I)\mathfrak{p}$是一个除子,记作$\mathrm{div}(I)$.
    \begin{proof}
    	
    	设$0\not=a\in A$满足$aI\subseteq A$,那么$v_{\mathfrak{p}}(I)\ge v_{\mathfrak{p}}(a^{-1})$,并且这里$v_{\mathfrak{p}}(a^{-1})=0$对几乎所有的高度1素理想$\mathfrak{p}$成立.另外取$0\not=x\in I$,那么$v_{\mathfrak{p}}(x)=0$也对几乎所有高度1素理想$\mathfrak{p}$成立.所以对几乎所有的高度1素理想$\mathfrak{p}$有$v_{\mathfrak{p}}(I)\le v_{\mathfrak{p}}(x)=0$.于是对几乎所有的$\mathfrak{p}$有$v_{\mathfrak{p}}(I)=0$.
    \end{proof}
    \item 如果$I=aA$是主理想,就有$\mathrm{div}(I)=\mathrm{div}(a)$.如果$I_1,I_2$是两个分式理想,有$\mathrm{div}(I_1I_2)=\mathrm{div}(I_1)+\mathrm{div}(I_2)$.另外有$\mathrm{div}(A)=0$.另外如果$I$是可逆分式理想,那么有$\mathrm{div}(I^{-1})=-\mathrm{div}(I)$.
    \item 于是$\mathrm{div}:\mathrm{Pic}(A)\to C(A)$是群同态.我们断言它总是单的,并且如果$A$是正则环,则这个同态是同构.
    \begin{proof}
    	
    	如果可逆分式理想$I$满足$\mathrm{div}(I)=0$,那么对每个高度1素理想$\mathfrak{p}$有$v_{\mathfrak{p}}(x)=0$,于是特别的对每个$x\in I$和每个高度1素理想$\mathfrak{p}$都有$v_{\mathfrak{p}}(x)\ge0$,这说明$x\in A$,于是$I\subseteq A$.于是$A\subseteq I^{-1}$.同理从$\mathrm{div}(I^{-1})=0$得到$I^{-1}\subseteq A$,这就导致$I=A$.于是$I$同构于主分式理想,于是$\mathrm{div}$是单射.
    	
    	\qquad
    	
    	如果$A$是正则环,我们断言它的每个高度1素理想$\mathfrak{p}$都是可逆的,这就得到$\mathrm{div}$是满射.下面证明断言:按照$A$诺特有$\mathfrak{p}$是有限生成的,另外如果$\mathfrak{m}$是$A$的一个极大理想,如果$\mathfrak{p}\not\subseteq\mathfrak{m}$,那么$\mathfrak{p}A_{\mathfrak{m}}$是单位理想;如果$\mathfrak{p}\subseteq\mathfrak{m}$,按照$A_{\mathfrak{m}}$是UFD得到$\mathfrak{p}A_{\mathfrak{m}}$是主理想,于是无论如何有$\mathfrak{p}$是$A$的可逆分式理想.
    \end{proof}
    \item (Grothendieck).设$R$是正则局部化,设$\mathfrak{P}$是被$R$正则序列生成的素理想,设$A=R/\mathfrak{P}$,如果对$A$的每个高度$\le3$的素理想$\mathfrak{p}$,都有$A_{\mathfrak{p}}$是UFD,那么有$A$是UFD.
    \item 设$A$是Krull环,那么有除子群的同构$C(A)=C(A[X])$;如果$A$是正则环,那么有除子群的同构$C(A)=C(A[[X]])$.
\end{enumerate}
\newpage
\subsection{完全交环}

\begin{enumerate}
	\item 设$(A,\mathfrak{m},k)$是诺特局部环,设$A$的嵌入维数是$n$,设$x_1,\cdots,x_n\in\mathfrak{m}$是极小生成元集,记Koszul复形$E_{\bullet}=K_{\bullet}(x_1,\cdots,x_n)$.那么$E_{\bullet}$(在非典范同构的意义下)是唯一被$A$确定的.我们记$\varepsilon_i(A)=\dim_k\mathrm{H}_i(E_{\bullet})$,这是$A$的不变量.
	\begin{proof}
		
		如果$x_1',\cdots,x_n'$是另一组$\mathfrak{m}$的极小生成元集,那么存在$A$上的$n$阶可逆矩阵$\left(a_{ij}\right)$,使得$x_i'=\sum_ja_{ij}x_j$.这里$K_{\bullet}(x_1,\cdots,x_n)$可以视为$A^n$上的外代数,如果取$A^n$的一组基$e_1,\cdots,e_n$,那么微分映射满足$\mathrm{d}(e_i)=x_i$.类似的$K_{\bullet}(x_1',\cdots,x_n')$可视为$A^n$上的外代数,如果取一组基$e_1',\cdots,e_n'$,微分映射满足$\mathrm{d}(e_i')=x_i'$.下面构造$f:A^n\to A^n$为$f(e_i')=\sum_ka_{ij}e_j$,它诱导了外代数之间的同态,那么$f$和微分映射可交换,因为$\mathrm{d}f(e_i')=\sum_ja_{ij}x_j=x_i'=f\mathrm{d}(e_i')$.于是$f$是复形的同构.
	\end{proof}
	\item 设$(A,\mathfrak{m},k)$是诺特局部环,设$x_1,\cdots,x_n\in\mathfrak{m}$是极小生成元集,设$E_{\bullet}$是Koszul复形,记$\varepsilon_i=\varepsilon_i(A)$.
	\begin{itemize}
		\item 因为$\mathrm{H}_0(E_{\bullet})\cong A/(x_1,\cdots,x_n)=k$,于是$\varepsilon_0=1$.
		\item 如果$A$是正则局部环,那么$x_1,\cdots,x_n$就是正则序列,并且此时有$\mathrm{H}_i(E_{\bullet})=0,\forall i\ge1$,于是此时$\varepsilon_i=0,\forall i\ge1$.
		\item 反过来如果$\varepsilon_1=0$,我们之前解释过就有$x_1,\cdots,x_n$是$A$正则序列,于是$n\le\mathrm{depth}(A)\le\dim A\le\mathrm{emb.dim}(A)=n$,于是$A$是正则局部环.
	\end{itemize}
	\item 设$(A,\mathfrak{m},k)$是诺特局部环,那么如下两个条件互相等价:
	\begin{enumerate}
		\item $A$是某个正则局部环的商.
		\item 存在正则局部环$(R,\mathfrak{n},k)$,和$R$的一个理想$\mathfrak{a}$满足$\mathfrak{a}\subseteq\mathfrak{n}^2$,使得$A=R/\mathfrak{a}$.相比上一条,这里我们额外要求了$\mathfrak{a}\subseteq\mathfrak{n}^2$,这个条件等价于讲$A$和$R$的嵌入维数是相同的.
	\end{enumerate}
    \begin{proof}
    	
    	只需证明(a)$\Rightarrow$(b).设$A=R/\mathfrak{a}$,其中$(R,\mathfrak{n},k)$是正则局部环.如果$\mathfrak{a}\not\subseteq\mathfrak{n}^2$,取$x\in\mathfrak{a}-\mathfrak{n}^2$,那么$R'=R/xR$也是正则局部环,并且$A$也是$R'$的商.所以适当模去有限个这样的$x$后导致$\mathfrak{a}\subseteq\mathfrak{n}^2$,所以我们不妨设起初就有这个包含关系.
    	
    	\qquad
    	
    	如果有这个包含关系,那么$\mathfrak{m}=\mathfrak{n}/\mathfrak{a}$,于是有$\mathfrak{m}/\mathfrak{m}^2\cong\mathfrak{n}/(\mathfrak{n}^2+\mathfrak{a})=\mathfrak{n}/\mathfrak{n}^2$,于是嵌入维数相同.反过来如果嵌入维数相同,那么明显有$\mathfrak{n}^2+\mathfrak{a}=\mathfrak{n}^2$,也即$\mathfrak{a}\subseteq\mathfrak{n}^2$.
    \end{proof}
    \item 如果$A$是诺特局部环,一般来讲$A$未必是某个正则局部环的商,但是我们在后文会证明完备的情况下总是某个完备正则局部环的商.
    \item 设$(A,\mathfrak{m},k)$是诺特局部环,设$\widehat{A}$是完备化.
    \begin{enumerate}
    	\item 对每个$i\ge0$有$\varepsilon_i(A)=\varepsilon_i(\widehat{A})$.
    	\item $\varepsilon_1(A)\ge\mathrm{emb.dim}(A)-\dim A$.
    	\item 如果$A=R/\mathfrak{a}$,其中$R$是正则局部环,那么$\mu(\mathfrak{a})=\dim R-\mathrm{emb.dim}(A)+\varepsilon_1(A)$,其中$\mu(M)$表示$A$模$M$的极小生成元集的元素个数.
    \end{enumerate}
    \begin{proof}
    	
    	(a):取$\mathfrak{m}$的一组极小生成元集,那么它也是$\mathfrak{m}\widehat{A}$的极小生成元集.因为$\widehat{A}$是平坦$A$代数,如果记$E_{\bullet}$是这组极小生成元集的Koszul复形,那么$\mathrm{H}_p(E_{\bullet})\otimes_A\widehat{A}=\mathrm{H}_p(E_{\bullet}\otimes\widehat{A})$.于是$\varepsilon_p(A)=\varepsilon_p(\widehat{A})$.
    	
    	\qquad
    	
    	(b):按照(a),不妨设$A$本身是完备的,那么$A$可以表示为某个正则局部环$(R,\mathfrak{n},k)$的商$R/\mathfrak{a}$,并且可以约定$\mathfrak{a}\subseteq\mathfrak{n}^2$.选取$\mathfrak{n}$的极小生成元集$\{\xi_1,\cdots,\xi_n\}$,由于$\mathfrak{m}/\mathfrak{m}^2\cong\mathfrak{n}/\mathfrak{n}^2$,说明$\xi$在$\mathfrak{m}=\mathfrak{n}/\mathfrak{a}$中的像$x_i$构成了$\mathfrak{m}$的极小生成元集.记$L_{\bullet}=K_{\bullet}(\xi_1,\cdots,\xi_n)$,它是$0\to L_n\to\cdots\to L_0\to0$.并且典范映射$L_0\to k$诱导了复形的拟同构$L_{\bullet}\to k$.于是$L_{\bullet}$是$k$的自由预解.另外有$E_{\bullet}=K_{\bullet}(x_1,\cdots,x_n)\cong L_{\bullet}\otimes_RA$.于是$\mathrm{H}_i(E_{\bullet})=\mathrm{Tor}_i^R(k,A),\forall i\ge0$.对$0\to\mathfrak{a}\to R\to A\to0$取长正合列,得到$\mathrm{Tor}_i^R(k,A)\cong\mathrm{Tor}_{i-1}^R(k,\mathfrak{a}),\forall i\ge1$.特别的有$\mathrm{Tor}_1^R(k,A)\cong k\otimes_R\mathfrak{a}\cong\mathfrak{a}/\mathfrak{n}\mathfrak{a}$.于是$\varepsilon_1(A)=\dim_k(\mathrm{H}_1(E_{\bullet}))=\dim_k\mathfrak{a}/\mathfrak{n}\mathfrak{a}$就是$\mathfrak{a}$的极小生成元集的元素个数,也即$\mu(\mathfrak{a})$.按照$R$是CM环,并且$\mathfrak{a}$是真理想,得到$\dim R-\dim A=\mathrm{ht}(\mathfrak{a})$.按照Krull定理得到$\mathrm{ht}(\mathfrak{a})\le\mu(\mathfrak{a})$,综上有:
    	$$\mathrm{emb.dim}(A)-\dim A=\dim R-\dim A=\mathrm{ht}(\mathfrak{a})\le\mu(\mathfrak{a})=\varepsilon_1(A)$$
    	
    	(c):如果$\mathfrak{a}\not\subseteq\mathfrak{n}^2$,选取$\mathfrak{a}-\mathfrak{n}^2$中的元$x$,那么$R/xR$也是正则局部环,并且$A$也是它的商,所以做有限次这样的操作可以保证$\mathfrak{a}\subseteq\mathfrak{n}^2$.此时有$\mathrm{emb.dim}(R)=\mathrm{emb.dim}(A)$,于是要证的等式归结为$\mu(\mathfrak{a})=\varepsilon_1(A)$,而这在(b)中已经证明过了.
    \end{proof}
\end{enumerate}

完全交环.一个诺特局部环称为完全交环(complete intersection ring),简记作c.i.环,如果满足$\varepsilon_1(A)=\mathrm{emb.dim}(A)-\dim A$.
\begin{enumerate}
	\item 设$A$是诺特局部环.
	\begin{enumerate}
		\item $A$是c.i.环$\Leftrightarrow\widehat{A}$是c.i.环.
		\item 如果$A$是c.i.环,设$R$是正则局部环,使得$A=R/\mathfrak{a}$,那么$\mathfrak{a}$被一个$R$正则序列生成.反过来如果$\mathfrak{a}$被一个正则序列生成,那么$R/\mathfrak{a}$的确是c.i.环.
		\item $A$是c.i.环当且仅当$\widehat{A}$是某个完备的正则局部环$R$模去一个由$R$正则序列生成的理想.
	\end{enumerate}
    \begin{proof}
    	
    	(a):这里$\varepsilon_1(A)$,$\dim A$,$\mathrm{emb.dim}(A)$在完备化下都是不变的.
    	
    	\qquad
    	
    	(b):设$A=R/\mathfrak{a}$,其中$R$是正则局部环,之前我们证明过有$\mathrm{emb.dim}(A)-\dim A=\mathrm{ht}(\mathfrak{a})$,所以此时$A$是c.i.环当且仅当$\mathrm{ht}(\mathfrak{a})=\mu(\mathfrak{a})=\varepsilon_1(A)$.因为$R$是正则局部环,所以它是CM环,所以$\mathrm{ht}(\mathfrak{a})=\mu(\mathfrak{a})$等价于讲$\mathfrak{a}\subseteq R$被一个$R$序列生成.
    	
    	\qquad
    	
    	(c):这由(a)和(b),以及我们承诺后文证明的命题完备诺特局部环总可以表示为完备正则局部环的商,得到.
    \end{proof}
    \item c.i.环总是Gorenstein环.
    \begin{proof}
    	
    	我们解释过$A$是Gorenstein环当且仅当$\widehat{A}$是Gorenstein环,于是我们不妨设$A$是完备的c.i.环.于是可记$A=R/\mathfrak{a}$,其中$R$是正则局部环,并且$\mathfrak{a}$可被一组$R$正则序列生成.但是正则局部环是Gorenstein环,并且Gorenstein环模去正则序列生成的理想还是Gorenstein环,这就得到$A$是Gorenstein的.
    \end{proof}
    \item 整理一下我们有:
    $$\{\textbf{正则局部环}\}\subseteq\{\textbf{c.i.环}\}\subseteq\{\textbf{Gorenstein环}\}\subseteq\{\textbf{CM环}\}$$
    \item (Avramov).设$A$是c.i.环,对素理想$\mathfrak{p}$,总有$A_{\mathfrak{p}}$也是c.i.环.这里我们只证明$A$完备的情况,一般情况比较复杂.
    \begin{proof}
    	
    	在$A$完备的情况下,可设$A=R/\mathfrak{a}$,其中$R$是正则局部环,$\mathfrak{a}$可被一组$R$正则序列$\{x_1,\cdots,x_r\}$生成.任取素理想$\mathfrak{p}\subseteq\mathrm{Spec}A\subseteq\mathrm{Spec}R$.那么有$A_{\mathfrak{p}}=R_{\mathfrak{p}}/(x_1,\cdots,x_r)R_{\mathfrak{p}}$.由于$R_{\mathfrak{p}}$是平坦$R$代数,于是$\{x_1,\cdots,x_r\}$也是$R_{\mathfrak{p}}$的正则序列.于是$A_{\mathfrak{p}}=R_{\mathfrak{p}}/(x_1,\cdots,x_r)R_{\mathfrak{p}}$也是c.i.环.
    \end{proof}
\end{enumerate}
\newpage
\section{平坦性}
\subsection{局部平坦准则}
\begin{enumerate}
	\item 设$A$是环,$B$诺特$A$代数,$M$是有限$B$模,设$J$是$B$的理想,记$M_n=M/J^{n+1}M,n\ge0$,如果$M_n,n\ge0$都是平坦$A$模,那么对任意有限生成$A$理想$I$有$\ker(I\otimes_AM\to M)\subseteq_nJ^n(I\otimes_AM)$.于是特别的如果$J\subseteq\mathrm{Rad}(B)$,按照平坦的理想准则就有$M$在$A$上平坦.
	\begin{proof}
		
		对每个自然数$n$,我们有交换图表:
		$$\xymatrix{I\otimes_AM\ar[rr]\ar[d]&&M\ar[d]\\I\otimes_AM_n\ar[rr]&&M_n}$$
		
		由于$M_n$在$A$上平坦,有下行映射是单射,于是$\ker(I\otimes_AM)\subseteq\ker(I\otimes_AM\to I\otimes_AM_n)=J^n(I\otimes_AM)$对任意$n$成立.这得到包含关系.再设$J\subseteq\mathrm{Rad}(B)$,由于$I$是有限$A$模,$M$是有限$B$模,得到$I\otimes_AM$是有限$B$模,于是NAK引理说明$I\otimes_AM$是$J$-adic可分模(这是在说$\cap_{n\ge0}J^nM'=0$).于是$\cap_nJ^n(I\otimes_AM)=0$,于是$I\otimes_AM\to M$总是单射,平坦性的理想准则说明$M$在$A$上平坦.
	\end{proof}
    \item 设$A$是环,$B$是诺特$A$代数,$M$是有限$B$模,设$b\in\mathrm{Rad}(B)$是一个$M$正则元,如果$M/bM$是平坦$A$模,则$M$也是平坦$A$模.
    \begin{proof}
    	
    	对每个$i>0$,我们有$A$模的短正合列$0\to M/b^iM\to M/b^{i+1}M\to M/bM\to0$,这里第一个非平凡同态取为数乘$b$.我们知道一个$A$模短正合列如果两边的模都是平坦模,那么中间的模也是平坦$A$模,结合条件$M/bM$是平坦$A$模,归纳得到每个$M/b^iM$都是平坦$A$模,于是上一个定理得到$M$是平坦$A$模.
    \end{proof}
    \item $I$-adic理想可分模.设$A$是环,$I$是理想,$A$模$M$称为$I$-adic理想可分的,如果对每个有限生成$A$理想$J$,总有$M'=J\otimes_AM$是$I$-adic拓扑可分的,也即$\cap_{n\ge0}I^nM'=0$.例如如果$B$是诺特$A$代数,$A$中理想$I$满足$IB\subset\mathrm{Rad}(B)$,那么每个有限$B$模$M$作为$A$模总是$I$-adic理想可分的.
    \begin{proof}
    	
    	需要验证的是对每个有限生成$A$理想$J$,$M'=J\otimes_AM$总是$I$-adic拓扑可分的.但是按照$M'$作为$B$模有限生成,以及$IB\subset\mathrm{Rad}(B)$,按照NAK引理就得到$\cap_{n\ge0}I^nM'=0$.
    \end{proof}
\end{enumerate}

我们接下来要给出的是所谓的局部平坦性准则.
\begin{enumerate}
	\item 设$A$是环,$I$是$A$的理想,$M$是$A$模.对每个$n\ge0$,记$A_n=A/I^{n+1}$,$M_n=M/I^{n+1}M$.记$\mathrm{gr}^I(A)=\oplus_{n\ge0}I^n/I^{n+1}$和$\mathrm{gr}^I(M)=\oplus_{n\ge0}I^nM/I^{n+1}M$.对每个$n\ge0$存在典范的同态$\gamma_n:(I^n/I^{n+1})\otimes_{A_0}M_0\to I^nM/I^{n+1}M$.于是这些同态诱导了$\mathrm{gr}^I(A)$模同态$\gamma:\mathrm{gr}^I(A)\otimes_{A_0}M_0\to\mathrm{gr}^I(M)$.
	\item 局部平坦性准则.在上述记号下,如果$I$是幂零理想,或者$A$诺特并且$M$是$I$-adic理想可分的,那么如下条件都是互相等价的:(这里a推b,b和c和d等价,d推e,e和f等价,e推g是不需要本条所加的额外条件的)
	\begin{enumerate}
		\item $M$是平坦$A$模.
		\item $\mathrm{Tor}_1^A(N,M)=0$对每个$A_0$模$N$均成立.
		\item $M_0$是平坦$A_0$模,并且典范同态$I\otimes_AM\to IM$是同构.
		\item $M_0$是平坦$A_0$模,并且$\mathrm{Tor}_1^A(A_0,M)=0$.
		\item $M_0$是平坦$A_0$模,并且每个$n\ge0$都有$\gamma_n$是同构.
		\item $M_0$是平坦$A_0$模,并且$\gamma$是同构.
		\item 对每个$n\ge0$都有$M_n$是平坦$A_n$模.
	\end{enumerate}
	\begin{proof}
		
		a推b是平凡的,因为$A_0$模总是$A$模.下面证明b推c,设$N$是$A_0$模,那么有张量积的同构$N\otimes_AM\cong(N\otimes_{A_0}A_0)\otimes_AM\cong N\otimes_{A_0}M_0$.于是对每个$A_0$模的短正合列$0\to N_1\to N_2\to N_3\to0$,它总诱导了长正合列$0=\mathrm{Tor}_1^A(N_3,M)=0\to N_1\otimes_{A_0}M_0\to N_1\otimes_{A_0}M_0\to N_2\otimes_{A_0}M_0\to N_3\otimes_{A_0}M_0\to0$.于是$-\otimes_{A_0}M_0$是正合函子,于是$M_0$是$A_0$平坦模.下面验证第二个结论,从短正合列$0\to I\to A\to A_0\to0$得到长正合列$0=\mathrm{Tor}_1^A(A_0,M)\to I\otimes_AM\to M\to M_0\to0$,这说明典范的同态$I\otimes_AM\to IM$也是单的,它是满的是总成立的,于是它是同构.
		
		c和d等价是直接的,只要考虑$0\to I\to A\to A_0\to0$诱导的长正合列即可.下面证明d推b,从而得到b和c和d的等价性.任取$A_0$模$N$,取自由$A_0$模$F_0$使得存在短正合列$0\to R\to F_0\to N\to0$,于是诱导了长正合列$0=\mathrm{Tor}_1^A(F_0,M)\to\mathrm{Tor}_1^A(N,M)\to R\otimes_{A_0}M_0\to F_0\otimes_{A_0}M_0$.按照$M_0$是平坦$A_0$模,这里最后一个箭头是单射,于是只能有$\mathrm{Tor}_1^A(N,M)=0$.
		
		下面证明b和c和d推e,从b得到$\mathrm{Tor}_1^A(I/I^2,M)=0$,于是从短正合列$0\to I^2\to I\to I/I^2\to0$得到$0\to I^2\otimes M\to I\otimes M\to(I/I^2)\otimes M\to0$正合.这说明同构$I\otimes M\cong IM$诱导了典范同构$I^2\otimes M\cong I^2M$以及$(I/I^2)\otimes M\cong IM/I^2M$.
		
		这里前一个同构是因为,首先总有典范的满同态$I^2\otimes M\to I^2M$,它是典范同态$I\otimes M\to IM$所诱导的.倘若$I^2\otimes M\to I^2M$的核中有元$x$,那么它同样是$I\otimes M\to IM$的核中的元,于是$x$在$I\otimes M$中是零.但是这里有一个问题,一般情况下可能存在$I^2\otimes M$中的非零元在$I\otimes M$中为零.但是这里条件要求的$I^2\otimes M\to I\otimes M$是单的保障了这一情况不会发生,因而此时$I^2\otimes M$是典范同构于$I^2M$的.第二个同构可以按照短五引理直接得到.
		
		继续对$0\to I^{n+1}\to I^n\to I^n/I^{n+1}\to0$做同样的事,就得到每个$\gamma_n:(I^n/I^{n+1})\otimes_{A_0}M_0\to I^nM/I^{n+1}M$都是同构.另外这里结论的全部$\gamma_n$是同构自然等价于分次模同态$\gamma$是同构,即e和f的等价性是直接的.
		
		下面证明e推g.我们需要验证$M_n$都是$A_n$平坦模,为此需要对每个$A_n$模$N$验证$\mathrm{Tor}_1^{A_n}(N,M_n)=0$.为此我们固定$n$,对每个$i\le n$考虑如下交换图:
		$$\xymatrix{&(I^{i+1}/I^{n+1})\otimes M\ar[r]\ar[d]_{\alpha_{n+1}}&(I^{i+1}/I^n)\otimes M\ar[r]\ar[d]_{\alpha_i}&(I^i/I^{i+1})\otimes M\ar[r]\ar[d]_{\gamma_i}&0\\0\ar[r]&I^{i+1}M/I^{n+1}M\ar[r]&I^iM/I^{n+1}M\ar[r]&I^iM/I^{i+1}M\ar[r]&0}$$
		
		按照$\alpha_{n+1}$是同构(此时是0到0的同构)以及$\gamma_n$是同构,短五引理得到$\alpha_n$是同构,结合全部$\gamma_i$是同构,反复运用短五引理得到每个$\alpha_i,0\le i\le n+1$都是同构.特别的,我们有$\alpha_1:(I/I^{n+1})\otimes_AM\cong IM_n$是同构.于是条件c被$A_n,M_n$和理想$I/I^{n+1}$满足,按照b和c的等价性,有$\mathrm{Tor}_1^{A_n}(N,M_n)=0$对每个$A_0$模$N$成立.现在归纳假设对每个$A_{k-1}$模$N$有$\mathrm{Tor}_1^{A_{k-1}}(N,M_{k-1})=0$.任取$A_k$模$N$,那么$IN$和$N/IN$都是$A_{k-1}$模,从短正合列$0\to IN\to N\to N/IN\to0$得到$\mathrm{Tor}_1^{A_n}(N,M_n)=0$.归纳得到对每个$A_n$模$N$都有该式成立,于是$M_n$是平坦$A_n$模.
		
		最后我们来证明在额外添加的两个条件的任一下,能得到g推a.首先假设$I$是幂零理想,那么对于足够大的$n$,就有$A=A_n$和$M=M_n$,于是此时自然$M$是平坦$A$模.下面假设$M$是$I$-adic理想可分的.我们只需验证平坦性的理想准则的条件:对每个有限生成$A$理想$J$,典范同态$u:J\otimes M\to JM$是同构.条件说明$\cap_{n\ge0}I^n(J\otimes M)=0$,于是只需验证对每个$n>0$总有$\ker u\subseteq I^n(J\otimes M)$.
		
		对每个$n>0$,由Artin-Rees引理说明对足够大的$k>n$总有$I^k\cap J\subseteq I^nJ$.考虑典范的同态:
		$$\xymatrix{J\otimes M\ar[r]^f&(J/I^k\cap J)\otimes M\ar[r]^g&(J/I^nJ)\otimes M=J\otimes M/I^n(J\otimes M)}$$
		
		按照$M_{k-1}$是$A_{k-1}=A/I^k$平坦模,映射$(J/I^k\cap J)\otimes_AM=(J/I^k\cap J)\otimes_{A_{k-1}}M_{k-1}\to M_{k-1}$是单射,我们有如下交换图:
		$$\xymatrix{J\otimes M\ar[rr]^f\ar[d]&&(J/I^k\cap J)\otimes M\ar[d]\\M\ar[rr]&&M_{k-1}}$$
		
		于是$\ker u\subset\ker(J\otimes M\to M\to M_{k-1})=\ker(gf)=I^n(J\otimes M)$.完成证明.
	\end{proof}
	\item 一个常用的特殊情况是$(A,m)$为诺特局部环,此时额外条件的第二个必然满足.另外$A_0$是域,于是$M_0$自动是$A_0$平坦模,并且此时要求$M_n$是平坦$A_n$模等价于要求$M_n$是自由$A_n$模.
\end{enumerate}

诺特局部环上完备化的平坦性.设$(A,m)$和$(B,n)$都是诺特局部环,记$\widehat{A}$和$\widehat{B}$分别为它们的完备化,设$B$经一个局部环同态作为$A$代数.
\begin{enumerate}
	\item 对每个有限$B$模$M$,记$\widehat{M}=M\otimes_B\widehat{B}$,那么$M$是平坦$A$模当且仅$\widehat{M}$是平坦$A$模,当且仅当$\widehat{M}$是平坦$\widehat{A}$模.
	\begin{proof}
		
		第一个等价关系:$M$是平坦$A$模等价于$M\otimes_A-$是正合函子,按照$\widehat{B}$是忠实平坦$B$模,这等价于$\widehat{B}\otimes_B(M\otimes_A-$是正合函子,于是$\widehat{M}=M\otimes_B\widehat{B}$是平坦$A$模.
		
		第二个等价关系:此时条件满足局部平坦准则,要证等价的两侧都等价于$\widehat{M}/m^n\widehat{M}$是平坦$A/m^n$模,$\forall n\ge0$.
	\end{proof}
	\item 同样设$M$是有限$B$模,设$M$的$mB$-adic完备化为$M^*$,那么$M$是平坦$A$模当且仅当$M^*$是平坦$A$模,当且仅当$M^*$是平坦$\widehat{A}$模.
	\begin{proof}
		
		条件满足局部平坦准则的条件,要证互相等价的三个命题均等价于$M/m^nM$是平坦$A/m^n$模.
	\end{proof}
\end{enumerate}

设$(A,m,k)$和$(B,n,k')$都是局部诺特环,设$A\to B$是局部映射,设$u:M\to N$是有限$B$模之间的同态,如果$N$是平坦$A$模,那么如下条件等价:
\begin{enumerate}
	\item $u$是单射并且$\mathrm{coker}u=N/u(M)$在$A$上平坦.
	\item $\overline{u}:M\otimes_Ak\to N\otimes_Ak$是单射.
\end{enumerate}
\begin{proof}
	
	先证明1推2,按照$u$是单射,有短正合列$\xymatrix{0\ar[r]&M\ar[r]^u&N\ar[r]&N/u(M)\ar[r]&0}$,它诱导了长正合列$\mathrm{Tor}_1^A(N/u(M),k)\to M\otimes_Ak\to N\otimes_Ak\to N/u(M)\otimes_Ak\to0$.按照$N/u(M)$是平坦$A$模得到$\mathrm{Tor}_1^A(N/u(M),k)=0$,于是诱导的$\overline{u}$是单射.
	
	证明2推1,先证明$u$是单射.设$x\in M$满足$u(x)=0$,那么$\overline{u}(\overline{x})=0$,于是$\overline{x}=0$.于是$x\in mM$.下面假设$x\in m^nM$,我们证明$x\in m^{n+1}M$.设局部环$A$上的模$m^n$的一个极小生成元集为$\{a_1,a_2,\cdots,a_r\}$,记$x=\sum a_iy_i$,其中$y_i\in M$.于是条件有$\sum a_iu(y_i)=0$.按照$N$是平坦$A$模,按照下面引理得到$c_{ij}\in A$和$Z_j\in N$满足$\sum a_ic_{ij}=0,\forall j$以及$u(y_i)=\sum_jc_{ij}z_j,\forall i$如果如果某个$c_{ij}$不在极大理想$m$里,那么它是$A$中可逆元,会导致$\{a_1,a_2,\cdots,a_r\}$不是极小生成元集.于是全体$c_{ij}\in m$.于是$u(y_i)\in mN$.于是$\overline{u}(\overline{y_i})=0$,于是$\overline{y_i}=0$,于是$y_i\in mM$,这导致$x\in m^{n+1}M$.于是$x\in\cap_{n\ge1}m^nM$.由NAK引理得到$\cap_{n\ge1}m^nM=\{0\}$.这就说明了$u$是单射.
	
	最后说明$N/u(M)$是平坦$A$模.按照局部平坦准则仅需说明$\mathrm{Tor}_1^A(k,N/u(M))=0$.按照$u$是单射得到短正合列$0\to M\to N\to N/u(M)\to0$.这诱导了如下长正合列,按照$\overline{u}$是单射得到$\alpha$是零映射,按照$N$是平坦$A$模得到$\mathrm{Tor}_1^A(k,N)=0$,于是$\mathrm{Tor}_1^A(k,N/u(M))=0$,得证.
	$$\xymatrix{\mathrm{Tor}_1^A(k,N)\ar[r]&\mathrm{Tor}_1^A(k,N/u(M))\ar[r]^{\alpha}&k\otimes_AM\ar[r]^{\overline{u}}&k\otimes_AN\ar[r]&k\otimes_AN/u(M)\ar[r]&0}$$
\end{proof}

推论.设$(A,m,k)$和$(B,n,k')$都是局部诺特环,设$A\to B$是局部映射.设$M$是有限$B$模,记$\overline{B}=B\otimes_Ak=B/mB$.对$x_1,x_2,\cdots,x_n\in n$,记它们在$\overline{B}$中的像为$\overline{x_i}$.如下两个条件等价:
\begin{enumerate}
	\item $x_1,x_2,\cdots,x_n$是$M$正则序列,并且$M_n=M/(x_1,x_2,\cdots,x_n)M$是平坦$A$模.
	\item $\overline{x_1},\overline{x_2},\cdots,\overline{x_n}$是$M\otimes_A k$正则序列并且$M$是平坦$A$模.
\end{enumerate}
\begin{proof}
	
	2推1,先证明$x_1$是$M$正则序列,并且$M_1=M/x_1M$是平坦$A$模.考虑数乘$x_1$的模同态$u:M\to M$,有$\overline{u}:M\otimes_A k\to M\otimes_A k$是单射,于是按照上一个定理,得到$u$是单射(也即$x_1$是正则$M$元),并且$M/u(M)=M/x_1M$是平坦$A$模.
	
	假设证明了$x_1,x_2,\cdots,x_r$是$M$正则序列,并且$M_r=M/(x_1,x_2,\cdots,x_r)M$是$A$平坦模,下面构造$u:M_r\to M_r$是数乘$x_{r+1}$,同样的$\overline{u}$是单射,按照上一个定理得到$u$是单射(也即$x_{r+1}$是$M_r$正则元,也即$x_1,x_2,\cdots,x_{r+1}$是$M$正则序列),并且$M_{r+1}=M/(x_1,x_2,\cdots,x_{r+1})M$是$A$平坦模.归纳得到1成立.
	
	1推2,我们的思路依然是和2推1一样由上一个定理得到$\overline{x_1},\overline{x_2},\cdots,\overline{x_n}$是$M\otimes_Ak$正则序列.但是为了归纳可以进行下去,需要说明每个$M_i=M/(x_1,x_2,\cdots,x_i)M$是平坦$A$模.为此我们用本节的第二个定理.按照$x_{n}$是$M_{n-1}$正则元,从$M_n=M_{n-1}/x_nM_{n-1}$是平坦$A$模得到$M_{n-1}$是平坦$A$模,反复归纳下去得到每个$M_i$都是平坦$A$模.
\end{proof}

设$A$是诺特环,$B$是诺特的$A$代数,$M$是有限$B$模,设$b\in B$.设$M$在$A$上平坦,并且对每个$B$的极大理想$P$都有$b$是$M/(P\cap A)M$正则元(尽管$A\to B$未必是单射或者说环扩张,我们依旧用$\mathfrak{P}\cap A$表示$\mathfrak{P}$在$A$中的原像),那么有$b$是$M$正则元并且$M/bM$是平坦$A$模.
\begin{proof}
	
	为证明$b$是正则元,记$M$上数乘$b$的映射的核为$K$,需要验证$K=0$,等价于对每个极大理想$P$有$K_P=0$.于是$b$是$M$正则元当且仅$b$是$M_P$正则元对任意极大理想$P$成立.(这实际上证明了正则元是一个局部性质).按照平坦性也是一个局部性质,于是对每个$B$的极大理想$P$,可把$B$替换为$B_P$,把$A$替换为$A_{P\cap A}$,把$M$替换为$M_P$.
	
	于是条件为$b$是$M/mM=M\otimes_Ak$正则元,并且$M$是平坦$A$模,按照上一定理的2推1,得到$b$是$M$正则元,并且$M/bM$是平坦$A$模.
\end{proof}

推论.设$A$是诺特环,设$B=A[X_1,X_2,\cdots,X_n]$,设$f(X)\in B$.设由$A$的系数生成的$A$的理想是单位理想,那么$f$是$B$的非零因子,并且$B/fB$是平坦$A$模.在形式幂级数的情况下同样的结论成立.
\begin{proof}
	
	我们只需验证上一定理的条件.首先$A[X]$和$A[[X]]$都是平坦$A$代数(本质上讲,$A[X]$作为$A$模就是可数个$A$的直和,自由模自然是平坦的.而$A[[X]]$作为$A$模是可数个$A$的直积,我们证明过诺特环上平坦模的直积总是平坦的.),需要说明对每个$B$中极大理想$p$,$f$分别在$A[X_i]\otimes_Ak=(A/p)[X_i]$和$A[[X_i]]\otimes_Ak=(A/p)$(注意一般未必有$A[[X]]\otimes_AB\cong B[[X]]$,但是总有$A[[X]]\otimes_AA/I\cong (A/I)[[X]]$)上正则元,但是这两个环都是整环,于是仅需说明$f$在其上不会是零元.但是是零元意味着所有系数都落在$p$中,按照条件导致$p$是单位理想,这矛盾.
\end{proof}

分次模上的平坦性.
\begin{enumerate}
	\item 设$R=\oplus_{g\in G}R_g$是$G$分次环,这里$G$是一个阿贝尔半群,设$M=\oplus_{g\in G}M_g$是分次$R$代数.那么如下命题互相等价:
	\begin{enumerate}
		\item $M$是平坦$R$模.
		\item 如果$\cdots\to N\to N'\to N''\to\cdots$是分次$R$模和保次数的$R$模同态构成的正合列,那么张量$-\otimes_RM$后仍然是正合的.
		\item 对$R$的任意有限生成齐次理想$I$有$\mathrm{Tor}_1^R(M,R/I)=0$.
	\end{enumerate}
    \begin{proof}
    	
    	(a)推(b)推(c)是容易的.下面证明(c)推(a):为了证明$M$是平坦$R$模,我们要证明对任意的等式$Ax=(a_1,\cdots,a_s)(x_1,\cdots,x_s)^t=\sum_ia_ix_i=0$,其中$a_i\in R$和$x_i\in M$,存在$A$上的一个$n\times s$的矩阵$(b_{ij})$和$y=(y_1,\cdots,y_s)^t\in M^s$,使得$AB=0$和$x=By$.
    	
    	\qquad
    	
    	我们记$a_i=\sum_{g\in G}a_{i,g}$和$x_i=\sum_{g\in G}x_{i,g}$是齐次分解,那么有$0=\sum_ia_ix_i=\sum_{g\in G}\sum_{i,h}a_{i,g-h}x_{i,h}\in M$.于是有$0=\sum_{i,h}a_{i,g-h}x_{i,h}\in M_g,\forall g\in G$.从(c)得到存在$b_{i,h,j}\in R$和$y_j\in M$使得$x_{i,h}=\sum_jb_{i,h,j}y_j$和$0=\sum_{i,h}a_{i,g-h}b_{i,h,j}$.于是$x_i=\sum_hx_{i,h}=\sum_j\left(\sum_hb_{i,h,j}\right)y_j$.记$b_i^{(j)}=\sum_hb_{i,h,j}$,就有:
    	$$\sum_ia_ib_i^{(j)}=\sum_ia_i\sum_hb_{i,h,j}=\sum_h\sum_ia_ib_{i,h,j}=\sum_{i,h,g}a_{i,g-h}b_{i,h,j}=0$$
    	$$\sum_jb_i^{(j)}y_j=\sum_h\sum_jb_{i,h,j}y_j=\sum_hx_{i,h}=x_i$$
    \end{proof}
    \item 设$R$是诺特$G$分次环,$M$是$G$分次模,设$I\subseteq R$是未必齐次的理想,那么如下条件:
    \begin{enumerate}
    	\item 对任意有限生成齐次理想$J\subseteq R$,有$J\otimes_RM$是$I$-adic可分的.
    	\item $M/IM$在$R/I$上平坦.
    	\item $\mathrm{Tor}_1^R(R/I,M)=0$.
    \end{enumerate}
 
    那么$M$在$R$上平坦.
    \item 设$A=\oplus_{n\in\mathbb{N}}$和$B=\oplus_{n\in\mathbb{N}}B_n$是分次诺特环,设$A_0$和$B_0$分别是以$\mathfrak{m}$和$\mathfrak{n}$为极大理想的局部环,设$\mathfrak{M}=\mathfrak{m}+\sum_{i\ge1}A_i$和$\mathfrak{N}=\mathfrak{n}+\sum_{i\ge1}B_i$,那么它们分别是$A$和$B$的极大理想.设$f:A\to B$是保次数的局部同态(局部同态指的是$f(\mathfrak{m})\subseteq\mathfrak{n}$),那么如下命题互相等价:
    \begin{enumerate}
    	\item $B$在$A$上平坦.
    	\item $B_{\mathfrak{N}}$在$A$上平坦.
    	\item $B_{\mathfrak{N}}$在$A_{\mathfrak{M}}$上平坦.
    \end{enumerate}
    \begin{proof}
    	
    	(a)推(b)因为$B_{\mathfrak{N}}$在$B$上平坦,(b)推(c)因为对$A_{\mathfrak{M}}$模$M$,有$B_{\mathfrak{N}}\otimes_AM=B_{\mathfrak{N}}\otimes_{A_{\mathfrak{M}}}(A_{\mathfrak{M}}\otimes_AM)=B_{\mathfrak{N}}\otimes_{A_{\mathfrak{M}}}M$.(c)推(b)是因为对任意$A$模$N$,有$B_{\mathscr{N}}\otimes_AN=B_{\mathfrak{N}}\otimes_{A_{\mathfrak{M}}}\left(A_{\mathfrak{M}}\otimes_AN\right)$.
    	
    	\qquad
    	
    	最后证明(b)推(a):为此我们验证上一条中的三个条件.第一个条件要验证对任意$R$的有限生成齐次理想$J$,有$J\otimes_AB$是$I=\mathfrak{M}$-adic可分的,对此我们证明更一般的结论:如果$M=\oplus_nM_n$是有限分次$B$模,我们来证明$M$是$\mathfrak{M}$-adic可分的,也即$\cap_n\mathfrak{M}^nM=0$.为此任取$x\in\cap_n\mathfrak{M}^nM$,不妨设$x$是$q$次齐次元,因为我们总可以把$x$替换为它的齐次分支.因为$A_iM_q\subseteq M_{q+i},\forall i$,于是$x\in\cap_n\mathfrak{m}^nM_q$.但是$B$是诺特环,$M$是有限$B$模,导致$M_q$是诺特局部环$B_0$上的有限模,于是$\cap_n\mathfrak{m}^nM_q=0$,这说明$x=0$,验证了第一个条件.
    	
    	\qquad
    	
    	由于$A/\mathfrak{M}=A_0/\mathfrak{m}$是域,所以第二个条件是显然成立的.第三个条件是要验证$\mathrm{Tor}_1^A(B,A/\mathfrak{M})=0$,或者等价的讲,验证自然映射$i:\mathfrak{M}\otimes_AB\to B$是单射.这是有限分次$B$模之间的零次同态,它的核$K$也是有限分次$B$模.再记$I=\mathrm{Ann}_B(K)$,它是$B$的齐次理想(齐次$B$模$M$的零化子总是齐次理想:如果$b\in\mathrm{Ann}_B(M)$,记齐次分解$b=\sum_nb_n$,那么$bM_m=0$导致$b_nM_m=0,\forall m,n$,导致$b_n\in\mathrm{Ann}_B(M)$).那么如果$K\not=0$,等价于$I$是$A$的真理想,于是$I\subseteq\mathfrak{N}$.于是$K_{\mathfrak{N}}\not=0$,但是(b)要求$B_{\mathfrak{N}}$在$A$上平坦,导致$i\otimes1_{B_{\mathfrak{N}}}:M\otimes_AB_{\mathfrak{N}}\to B_{\mathfrak{N}}$是单射,它的核$K_{\mathfrak{N}}$就理应是零.这个矛盾说明第三个条件成立.综上$B$在$A$上平坦.
    \end{proof}
\end{enumerate}
\newpage
\subsection{平坦性和纤维}

平坦和维数.
\begin{enumerate}
	\item 首先回顾维数不等式和等式,如果$(A,m)$和$(B,n)$都是诺特局部环,设$\varphi:A\to B$是局部映射,记$\varphi$在$m$处的纤维环$F=B\otimes_Ak(m)=B/mB$,如果$B$是平坦$A$模,那么有维数等式$\dim B=\dim A+\dim F$.
	\item 我们第一个定理是在某种条件下上述命题的逆命题.设$(A,m)$和$(B,n)$是局部诺特环,设$\varphi:A\to B$是局部映射,设$F$是$\varphi$在$m$处的纤维环.如果$A$是正则局部环,$B$是CM环,那么从维数等式$\dim B=\dim A+\dim F$可推出$B$在$A$上平坦.
	\begin{proof}
		
		对$\dim A$做归纳.如果$\dim A=0$,那么$A$是域(正则局部环是整环),此时$A$上的模总是平坦的.下面设$\dim A>0$,于是特别的$m/m^2\not=0$,于是可取$x\in m-m^2$,记$A'=A/xA$和$B'=B/xB$.我们之前证明过正则局部环$(A,m)$上的参数系统恰好就是在$m/m^2$中构成$k=A/m$基的一组元,于是特别的$x$必然是某个参数系统的子集,于是$\dim A'=\dim A-1$.并且$\varphi$诱导的$A'\to B'$的映射$\varphi'$在$m'$处的纤维环$F'$同构于$F$(见下一条).于是从维数不等式得到:$\dim B'\le\dim A'+\dim F'=\dim A-1+\dim F=\dim B-1$.另外对于一般的局部环$(R,m)$,任取$x\in m$,那么总有$\dim R/(x)\ge\dim R-1$(取$R/(x)$的一组参数系统,它们并上$x$的这组元被$B$模去是有限长度的,于是有这个不等式).于是特别的这里有$\dim B'\ge\dim B-1$(这里用到了$\varphi$是局部映射).综上得到$\dim B'=\dim B-1$.于是按照CM环上参数系统和极大正则序列等价的那个定理的一部分,得到$x$(或者说实际上是$\varphi(x)$)是$B$正则元.于是按照我们知道证明的一个模是CM模当且仅当它模去某个正则序列是CM模,这就得到$B'$也是CM环.另外$A'$是维数$\dim A-1$的正则局部环,于是按照归纳假设,得到$B'$是平坦$A'$模.
		
		于是得到$\mathrm{Tor}_1^{A'}(A/m,B')=0$.现在按照$x$是$A$正则元也是$B$正则元,并且$x$将$k=A/m$零化,得到$\mathrm{Tor}_1^{A}(A/m,B)=\mathrm{Tor}_1^{A'}(A/m,B')=0$.(一般的,如果$M,N$都是$A$模,$x\in A$是$A$正则元也是$M$正则元,并且$xN=0$,那么就有$\mathrm{Tor}_n^A(M,N)=\mathrm{Tor}_n^{A'}(M',N)$,其中$A'=A/xA$,$M'=M/xM$).于是按照局部平坦性准则,得到$B$在$A$上平坦.
	\end{proof}
    \item 如果$\varphi':A'\to B'$是环同态,$I$是$A'$的理想,设$A=A'/I$和$B=B'/IB'$,记$\varphi'$诱导的$A\to B$为$\varphi$,那么对任意$p'\in V(I)\subset\mathrm{Spec}(A')$,记$p=p'/I$,那么$\varphi'$在$p'$处的纤维环同构于$\varphi$在$p$处的纤维环(中间的同构直接构造):
    $$B'\otimes_{A'}\kappa(p')=B'\otimes_{A'}(A'/p')_{p'}\cong B\otimes_A(A/p)_p=B\otimes_A\kappa(p)$$
    \item 设$k$是域,设$X,Y$是$k$上的不可约概形,设$f:Y\to X$是$k$概形态射,设$\dim X=n$,$\dim Y=m$,设如下条件成立,那么$f$是平坦的.
    \begin{enumerate}
    	\item $X$是正则概形.
    	\item $Y$是CM的.
    	\item $f$把$Y$的闭点映射为$X$的闭点.
    	\item 对任意闭点$x\in X$,有纤维$f^{-1}(x)$要么是空集要么是$m-n$维的.
    \end{enumerate}
    \begin{proof}
    	
    	设$y\in Y$是闭点,设$x=f(y)$,设$A=\mathscr{O}_{X,x}$和$B=\mathscr{O}_{Y,y}$,那么$\dim A=n$和$\dim B=m$.此时满足维数等式$\dim B=\dim A+\dim F$,这里$F=f^{-1}(x)$.于是$A$在$B$上平坦.
    \end{proof}
\end{enumerate}

平坦和深度.
\begin{enumerate}
	\item 设$(A,m,k)$和$(B,n,k')$是诺特局部环,设$\varphi:A\to B$是局部映射,设$M$是有限$A$模,$N$是有限$B$模,设$N$在$A$上平坦,那么有等式:
	$$\mathrm{depth}_B(M\otimes_AN)=\mathrm{depth}_A(M)+\mathrm{depth}_B(N/mN)$$
	\begin{proof}
		
		取$A$模$M$的极大长度的正则序列为$x_1,x_2,\cdots,x_r\in m$,取$B$模$N/mN$的极大长度的正则序列为$y_1,y_2,\cdots,y_s\in n$.记$x_i$在$B$中的像是$x_i'$.我们只需证明$x_1',x_2',\cdots,x_r',y_1,\cdots,y_s$是$B$模$M\otimes_AN$的极大长度的正则序列.
		
		按照$N$是平坦$A$模,从$x_1,x_2,\cdots,x_r$是$M$正则序列得到$x_1',x_2',\cdots,x_r'$是$M\otimes_AN$正则序列,下面记$M_r=M/(x_1,x_2,\cdots,x_r)M$.那么$M_r$是深度为0的诺特环上的有限模,于是必然有$m\in\mathrm{Ass}_A(M_r)$.
		
		按照张量和商可交换得到$(M\otimes_AN)/\sum_{1\le i\le r}x_i'(M\otimes_AN)\cong M_r\otimes_AN$.按照上一节的一个定理,从$y_1$是$N/mN=N\otimes_Ak$上正则元得到$y_1$是$N$正则元并且$N_1=N/y_1N$是平坦$A$模.于是短正合列$0\to N\to N\to N_1\to0$是纯短正合列,它张量任意模都是正合的,于是有短正合列$0\to M_r\otimes N\to M_r\otimes N\to M_r\otimes N_1\to0$.归纳操作下去得到$y_1,y_2,\cdots,y_s$是$M_r\otimes_AN$正则序列.这就说明$x_1',x_2',\cdots,x_r',y_1,\cdots,y_s$是$M\otimes_AN$的的正则序列,最后为了说明它是极大长度的.考虑商$(M\otimes_AN)/(\sum_ix_i'(M\otimes_AN)+\sum_jy_j(M\otimes_AN))=M_r\otimes_AN_s$,只需说明它的深度为0.也即$n\in\mathrm{Ass}_B(M_r\otimes_AN_s)$.但是按照深度0得到$m\in\mathrm{Ass}_A(M_r)$和$n\in\mathrm{Ass}_B(N_S/mN_s)$,于是从$\mathrm{Ass}_B(M\otimes_AN)=\cup_{\mathfrak{p}\in\mathrm{Ass}_A(M)}\mathrm{Ass}_B(N/\mathfrak{p}N)$得到结论.
	\end{proof}
    \item 推论.设$(A,m,k)$和$(B,n,k')$是诺特局部环,设$\varphi:A\to B$是局部映射,设$F=B/mB$是纤维环,设$B$在$A$上平坦,那么:
    \begin{enumerate}
    	\item $\mathrm{depth}(B)=\mathrm{depth}(A)+\mathrm{depth}(F)$.
    	\item $B$是CM环当且仅当$A$和$F$都是CM环.
    \end{enumerate}
    \begin{proof}
    	
    	在上述定理中取$M=A$和$N=B$直接得到第一条.在条件下有维数等式$\dim B=\dim A+\dim F$,于是$\dim B-\mathrm{depth}(B)=(\dim A-\mathrm{depth}(A))+(\dim F-\mathrm{depth}(F))$.按照深度不超过维数,说明$B$是CM环当且仅当左侧为0,当且仅当右侧为0,当且仅当$A$和$F$都是CM环.
    \end{proof}
    \item 这里我们给出之前承诺的定理证明:如果$A$是CM环,未必是局部环,那么$A[[X]]$也是CM环.归纳得到如果$A$是CM环,那么$A[[X_1,X_2,\cdots,X_n]]$总是CM环.
    \begin{proof}
    	
    	设$B=A[[X]]$,取$B$的极大理想$\mathfrak{M}$,我们要证明$B_{\mathfrak{M}}$是CM局部环.为此取$\mathfrak{m}=\mathfrak{M}\cap A$,这是$A$的极大理想,并且有$\mathfrak{M}=(\mathfrak{m},X)$.环同态$A_{\mathfrak{m}}\to B_{\mathfrak{M}}$的纤维环是$B\otimes_A\kappa(\mathfrak{m})=(A/\mathfrak{m})[[X]]$这是一个正则局部环,于是它是CM局部环,结合$A_{\mathfrak{m}}$是CM局部环得到$B_{\mathfrak{M}}$是CM局部环,于是$B=A[[X]]$是CM环.
    \end{proof}
    \item 设$A$是CM环包含了一个域$k$,设$K/k$是有限生成域扩张,那么$A\otimes_kK$也是CM环.【】
\end{enumerate}

平坦性和Gorenstein环.
\begin{enumerate}
	\item 设$\varphi:A\to B$是诺特局部环之间的局部映射,设$A$的极大理想为$\mathfrak{m}$,设$\varphi$在$\mathfrak{m}$处的纤维环为$F=\kappa(\mathfrak{m})\otimes_AB=B/\mathfrak{m}B$,设$B$在$A$上平坦,那么$B$是Gorenstein环当且仅当$A$和$F$都是Gorenstein环.
	\begin{proof}
		
		按照Gorenstein环必然是CM环,并且在条件下上面推论说明了$B$是CM环当且仅当$A$和$F$都是CM环,于是我们不妨就假定$A,B,F$都是CM环.记$\dim A=r$,$\dim F=s$,按照条件下的维数等式就得到$\dim B=r+s$.取$x_1,x_2,\cdots,x_r$是$A$的极大正则序列,按照CM条件它也是$A$的一组参数系统.再记$y_1,y_2,\cdots,y_s$是$B$的子集,满足它在$F=B/\mathfrak{m}B$中是一组极大正则序列,也就是$F$的一组参数系统.于是$x_1,x_2,\cdots,x_r,y_1,y_2,\cdots,y_s$是$B$的正则序列,于是CM条件导致它也是$B$的一组参数系统.
		
		\qquad
		
		取$A'=A/(x_i)A$和$B'=B/(x_i,y_j)B$,此时纤维环$F'$同构于$F$模去它的极大正则序列$\{y_1',y_2',\cdots,y_s\}$.我们知道如果$x$是$R$的正则元,那么$R$是Gorenstein环当且仅当$R/(x)$是Gorenstein环.于是我们不妨设$A,B,F$的维数都是0,否则以$A',B',F'$代替它们.这里还需要说明下$B'$在$A'$上平坦:按照归纳法只需说明$B/(y_1)$是$A$平坦模.由$B$是$A$平坦模,按照局部平坦准则,得到$\mathrm{Tor}_1^A(A/\mathfrak{m},B)=0$.同样按照局部平坦准则,只需说明$\mathrm{Tor}_1^A(A/\mathfrak{m},B/(y_1))=0$.为此考虑短正合列$0\to B\to B\to B/y_1B\to0$,诱导了长正合列$0=\mathrm{Tor}_1^A(A/\mathfrak{m},B)\to\mathrm{Tor}_1^A(A/\mathfrak{m},B/y_1B)\to A/\mathfrak{m}\otimes_AB\to A/\mathfrak{m}\otimes_AB$.只需验证数乘$y_1$的映射$A/\mathfrak{m}\otimes_AB\to A/\mathfrak{m}\otimes_AB$是单射,此即$B/\mathfrak{m}B\to B/\mathfrak{m}B$的映射,但是$y_1$是$F=B/\mathfrak{m}B$正则元,得证.
		
		\qquad
		
		我们现在设$A,B$都是维数零的CM环,一个零维局部环$(R,\mathfrak{m})$是Gorenstein环当且仅当$\mathrm{Hom}_R(R/\mathfrak{m},R)=(0:\mathfrak{m})_R\cong R/\mathfrak{m}$(前一个等号总成立).记$B$的极大理想是$\mathfrak{n}$,那么$F$的极大理想是$\mathfrak{n}'=\mathfrak{n}/\mathfrak{m}B$.记$I=(0:\mathfrak{m})_A$.因为$A$是诺特的,所以$I$是一个有限维$\kappa(\mathfrak{m})$线性空间.于是有$I\cong\kappa(\mathfrak{m})^t$,其中$t\ge1$是一个整数.由于$B$是$A$平坦的,此时一般的对$A$的理想$I_1,I_2$有$(I_1:I_2)B=(I_1B:I_2B)$.于是$(0:\mathfrak{m}B)_B=(0:\mathfrak{m})_AB=IB\cong I\otimes_AB=\kappa(\mathfrak{m})^t\otimes_AB=(B/\mathfrak{m}B)^t=F^t$.另外我们断言有$(0:\mathfrak{n})_B=(0:\mathfrak{n})_{IB}$:一方面$(0:\mathfrak{n})_{IB}\subseteq(0:\mathfrak{n})_B$,另一方面有$x\in(0:\mathfrak{n})_B\subseteq(0:\mathfrak{m}B)_B=IB$,于是$x\in(0:\mathfrak{n})_{IB}$,完成断言的证明.于是我们有:
		$$(0:\mathfrak{n})_B=(0:\mathfrak{n})_{IB}\cong(0:\mathfrak{n})_{F^t}=(0:\mathfrak{n})_F^t=(0:\mathfrak{n}')_F^t$$
		
		于是$B$是Gorenstein环等价于$(0:\mathfrak{n})$是$\kappa(\mathfrak{n}')=\kappa(\mathfrak{n})$上的一维线性空间,这等价于$t=1$并且$(0:\mathfrak{n}')_F$是$\kappa(\mathfrak{n}')=\kappa(\mathfrak{n})$上的一维线性空间,也即$A$和$F$都是Gorenstein环.
	\end{proof}
    \item 若$A$是Gorenstein环,那么$A[X]$和$A[[X]]$也都是Gorenstein环.归纳得到如果$A$是Gorenstein环,那么$A[X_1,\cdots,X_n]$和$A[[X_1,\cdots,X_n]]$总是Gorenstein环.
    \begin{proof}
    	
    	记$B=A[X]$,取$B$的极大理想$\mathfrak{M}$,记$\mathfrak{p}=\mathfrak{M}\cap A$,那么$B_{\mathfrak{M}}$是$B\otimes_AA_{\mathfrak{p}}\cong A_{\mathfrak{p}}[X]$的局部化,此时环同态$A_{\mathfrak{p}}\to B_{\mathfrak{M}}$的纤维环是$B_{\mathfrak{M}}\otimes_{A_{\mathfrak{p}}}\kappa(\mathfrak{p})$,它是$\kappa(\mathfrak{p})[X]$的局部化,所以是正则局部环,结合$A_{\mathfrak{p}}$是Gorenstein环得到$B_{\mathfrak{M}}$是Gorenstein局部环,于是$B$是Gorenstein环.
    	
    	\qquad
    	
    	再考虑$C=A[[X]]$,记$\mathfrak{M}$是$C$的一个极大理想,那么$\mathfrak{m}=\mathfrak{M}\cap A$是$A$的极大理想,并且有$\mathfrak{M}=(\mathfrak{m},X)$.环同态$A_{\mathfrak{m}}\to C_{\mathfrak{M}}$的纤维环是$C_{\mathfrak{M}}\otimes_{A_{\mathfrak{p}}}\kappa(\mathfrak{M})=\kappa(\mathfrak{p})[[X]]$是正则局部环,结合$A_{\mathfrak{p}}$是Gorenstein环得到$C_{\mathfrak{M}}$是Gorenstein环,于是$C=A[[X]]$是Gorenstein环.
    \end{proof}
    \item 设$A$是Gorenstein环,并且包含了一个域$k$,对$k$的任意有限生成扩张$K$,基变换$A\otimes_kK$总是Gorenstein环.
    \begin{proof}
    	
    	归结为$k\subseteq K$是单扩张的情况.设$K=k(x)$.倘若$x$是$k$上超越元,那么$A\otimes_kK$是$A\otimes_kk[X]=A[X]$的分式化.上一条证明了$A[X]$总是Gorenstein环.我们之前证明过Gorenstein环的分式化总是Gorenstein环(如果$N$是$A$内射模,那么$S^{-1}N$是$S^{-1}A$内射模,于是从$A$的有限长度内射预解式得到$S^{-1}A$的有限长度的内射预解式,于是内射维数有限,于是Gorenstein).这就得证.
    	
    	再设$x$是$k$上代数元,那么$K\cong k[X]/(f(X))$,其中$f(X)$是$k$上首一多项式.那么有$A\otimes_kK\cong A[X]/(f(X))$.这里$f(x)$必然是$A[X]$的非零因子(正则元),我们证明过Gorenstein环模去正则元必然还是Gorenstein环,这就得证.
    \end{proof}
\end{enumerate}

平坦性和完全交环.借助Andr\'e同调,我们可以证明完全交环也有这两个结论:
\begin{enumerate}
	\item 设$A$是完全交环,那么$A[X_1,\cdots,X_n]$和$A[[X_1,\cdots,X_n]]$都是完全交环.
	\item 设$A$是完全交环,包含一个域$k$,设$K/k$是有限生成域扩张,那么基变换$A\otimes_kK$也是完全交环.
\end{enumerate}

平坦性和正则环.
\begin{enumerate}
	\item 设$(A,\mathfrak{m},\kappa)$和$(B,\mathfrak{m},\kappa')$都是诺特局部环,设$A\to B$是局部映射,取$F=B/\mathfrak{m}B$,设$B$在$A$上平坦.
	\begin{enumerate}
		\item 若$B$是正则局部环那么$A$也是正则局部环.
		\item 若$A$和$F$都是正则局部环,那么$B$也是正则局部环.
	\end{enumerate}
	\begin{proof}
		
		(a):首先Tor函子满足平坦积变换公式:如果$T$是平坦$R$代数,对任意$R$模$M,N$总有$\mathrm{Tor}_i^R(M,N)\otimes_RT\cong\mathrm{Tor}^T(M\otimes_RT,N\otimes_RT)$.特别的,这里就有$\mathrm{Tor}_i^A(\kappa,\kappa)\otimes_AB\cong\mathrm{Tor}_i^B(B\otimes_A\kappa,B\otimes_A\kappa)$.这里$B$是正则局部环的一个等价描述是整体维数恰好是$B$的维数,于是特别的总有$i>\dim B$的时候上述等式右侧总取零.按照$B$是忠实平坦$A$代数,得到$i$足够大的时候总有$\mathrm{Tor}_i^A(k,k)=0$.于是$A$的整体维数有限,也即它是正则局部环.
		
		\qquad
		
		(b):设$r=\dim A$,$s=\dim F$,取$A$的正则参数系统$x_1,\cdots,x_r$,取$y_1,\cdots,y_s\in\mathfrak{n}$使得它构成$F$的正则序列.因为$B$在$A$上忠实平坦,得到$A\to B$是单射.于是$\{x_1,\cdots,x_r,y_1,\cdots,y_s\}$生成了整个$\mathfrak{n}$,但是$\dim B=r+s$,这说明$B$是正则局部环.
	\end{proof}
    \item 在上一条的条件下,如果$B$是正则局部环,未必会有$F$是正则环.例如取$k$是域,取$X$是未定元,取$B=k[X]_{(X)}$和$A=k[X^2]_{(X^2)}$,取包含映射$A\to B$,那么$F=B/x^2B=k[X]/(X^2)$是局部环,它有幂零元所以不是正则局部环.另外$B$在$A$上平坦是因为满足维数等式$\dim B=\dim A+\dim F$.
\end{enumerate}

Serre条件.对每个$n\ge0$,对诺特环$A$,记$(R_n)$条件表示$\forall$高度不超过$n$的素理想$p$,总有$A_p$是正则局部环;记$(S_n)$条件表示对任意$p\in\mathrm{Spec}(A)$总有$\mathrm{depth}(A_p)\ge\min\{\mathrm{ht}(p),n\}$.先给出一些基本观察:
\begin{enumerate}
	\item 条件$(S_0)$总成立;如果$0\le n\le m$,那么$(S_m)$推出$(S_n)$,$(R_m)$推出$(R_n)$;所有条件$(R_n)$成立等价于讲$A$是正则环.
	\item 条件$(S_n)$等价于讲只要$\mathrm{depth}(A_p)<n$,那么$A_p$是CM环.于是特别的,所有条件$(S_n)$成立等价于讲$A$是CM环.
	\begin{proof}
		
		必要性,如果$\mathrm{depth}(A_p)<n$,那么只能有$\min\{\dim A_p,n\}=\dim A_p$,此时$\mathrm{depth}(A_p)\ge\dim A_p$,但是另一侧的不等式是总成立的,于是此时$A_p$是CM环.
		
		充分性,如果$\mathrm{depth}(A_p)<n$,那么$\min\{\dim M_p,n\}<n$,导致$\mathrm{depth}(A_p)=\dim A_p<n$,于是此时$S_n$条件的这个不等式实际上是等式.如果$\mathrm{depth}(A_p)\ge n$,那么$\dim A_p\ge n$,此时不等式总成立.
	\end{proof}
	\item 条件$(S_1)$等价于讲$A$没有嵌入素理想.
	\begin{proof}
		
		必要性,假设$A$有嵌入素理想$p$,那么$pA_p$是$A_p$的伴随素理想,一个局部环上有限模$M$深度为0当且仅当极大理想是$M$的伴随素理想,于是特别的$\mathrm{depth}(A_p)=0$,但是这按照$S_1$条件得到$A_p$是CM环,特别的它维数0,导致$p$是极小素理想,矛盾.
		
		充分性,取素理想$p$使得$A_p$的深度为零,于是$pA_p$是$A_p$的伴随素理想,于是$p$是$A$的伴随素理想,按照条件$p$只能是极小素理想,于是$A_p$的维数也是0,于是$A_p$是CM环.
	\end{proof}
	\item 设$A$是整环,那么$(S_2)$条件等价于讲$A$的每个非零主理想的伴随素理想的高度为1.
	\begin{proof}
		
		必要性,任取$x\in A$非零非单位,它是一个正则元,按照Krull主理想定理,$Ax$的极小素理想的高度为1.我们接下来仅需验证$A/xA$没有嵌入素理想,这等价于验证$A/xA$满足$(S_1)$.任取素理想$p$使得$(A/xA)_p=A_p/xA_p$的深度为零,那么$A_p$的深度为1,按照$S_2$条件得到$A_p$是CM环,于是模去正则元后的商环$A_p/xA_p$也是CM环,于是$A/xA$没有嵌入素理想.
		
		充分性,条件说明$A/xA$没有嵌入素理想,于是$A/xA$满足$S_1$条件.下面任取$A$的素理想$p$使得$\mathrm{depth}(A_p)\le1$,那么$\mathrm{depth}(A_p/xA_p)=0$,于是$S_1$条件说明$A_p/xA_p$是CM环,于是$A_p$是CM环,于是$A$满足$S_2$.
	\end{proof}
	\item 条件$(R_n)$和$(S_n)$都可以传递给局部化.
	\begin{proof}
		
		设$A$满足$(R_n)$,任取素理想$p$,需要验证$A_p$满足$(R_n)$.任取$A_p$的素理想$qA_p$满足$\mathrm{ht}(qA_p)\le n$,那么$\mathrm{ht}(q)\le n$.于是$A_q=(A_p)_{qA_p}$是正则环.
		
		设$A$满足$(S_n)$,任取素理想$p$,需要验证$A_p$满足$(S_n)$.任取$A_p$的素理想$qA_p$使得$\mathrm{depth}((A_p)_{qA_p})=\mathrm{depth}(A_q)\le n$,于是$A_q$是CM环.
	\end{proof}
    \item 一个诺特环$A$是既约环(没有非平凡幂零元)当且仅当它满足$(R_0)$和$(S_1)$.
    \begin{proof}
    	
    	必要性.设$A$是既约环.任取极小素理想$p$,那么$A_p$是零维既约环,但是零维既约环只有唯一的素理想,它也就是幂零根,按照既约性得到它是域,这说明满足$(R_0)$.另外按照既约条件知极小素理想的交为零,这也是零理想的准素分解,于是所有伴随素理想都是极小素理想,于是$(S_1)$满足.
    	
    	充分性.假设诺特环$A$满足$(R_0)$和$(S_1)$.取零理想的不可缩短的准素分解$0=\cap_iq_i$,记$p_i=\sqrt{q_i}$,按照$(S_1)$条件知$p_i$都是$A$的极小素理想,按照$(R_0)$知每个局部化$A_{p_i}$都是域.于是$p_iR_{p_i}=0$,于是$q_iR_{p_i}\subseteq p_iR_{p_i}=0$,于是$q_i=p_i$,于是$0=\cap_ip_i$,于是所有素理想的交是零理想,于是$A$是既约环.
    \end{proof}
    \item 一个诺特环是正规环(即在每个素理想的局部化是正规整环)当且仅当它满足$(R_1)$和$(S_2)$.
    \begin{proof}
    	
    	因为我们说明过$(R_n)$和$(S_n)$都传递给局部化,于是不妨设$A$是局部环.必要性,我们证明过一维正规诺特局部环是DVR,也即一维局部正则环,此即$(R_1)$条件.我们还证过正规诺特整环的非零主理想的伴随素理想的高度都为1,此即$(S_2)$条件.
    	
    	充分性.首先$A$满足$(R_0)$和$(S_1)$,于是它是既约环,并且可取零理想的最短准素分解为$0=p_1\cap p_2\cap\cdots\cap p_r$.其中$p_i$是$A$的全部极小素理想.那么有$A\cong A/(0)\cong A/p_1\times A/p_2\times\cdots\times A/p_r$.记$K_i=\mathrm{Frac}(A/p_i)$,那么$A$的全商域(total ring of fractions)即$K=K_1\times K_2\times\cdots\times K_r$.
    	
    	先证明$A$在$K$中正规.取$a/b\in K$,其中$a,b\in A$并且$b$是$A$正则元,满足存在$A$中的元$c_1,c_2,\cdots,c_n$使得$(a/b)^n+c_1(a/b)^{n-1}+\cdots+c_n=0$.于是得到$a^n+\sum_{i=1}^nc_ia^{n-i}b^i=0$.取$\mathrm{Spec}(A)$中高度为1的素理想$p$,按照$(R_1)$条件有$A_p$是一维局部正则环,于是正规,于是$a_p\in b_pA_p$,其中$a_p$和$b_p$表示$a,b$在$A_p$中的像.按照$b$是$A$正则元,于是$(S_2)$说明$A/bA$的全部伴随素理想的高度为1,于是有最短准素分解$bA=q_1\cap q_2\cap\cdots\cap q_m$,记$p_i=\sqrt{q_i}$,那么$a\in bA_{p_i}\cap A=q_i$.取交得到$a\in bA$,于是$a/b\in A$,于是$A$在$K$中正规.
    	
    	记第$i$分量为$1_{K_i}$,其余分量取零的元为$e_i$,那么$e_i$满足$e_i^2-e_i=0$,于是全部$e_i$在$A$上整,于是$e_i\in A$,于是从$\sum e_i=1_A$和$e_ie_j=0,\forall i\not=j$得到$A=Ae_1\times Ae_2\times\cdots\times Ae_r$(这是同构,其中满是因为$a_1e_1+a_2e_2+\cdots+a_re_r$映射过去是$(a_1e_1,a_2e_2,\cdots,a_re_r)$).最后按照$A$是局部环,说明$r=1$,否则可构造多个不同的极大理想,于是$A$是正规整环.
    \end{proof}
    \item 设$(A,\mathfrak{m})$和$(B,\mathfrak{n})$是诺特局部环,$A\to B$是一个局部映射,设$B$在$A$上平坦,设$i\ge0$是给定自然数,那么:
    \begin{enumerate}
    	\item 如果$B$满足$(R_i)$,那么$A$也满足$(R_i)$.
    	\item 如果$A$和每个纤维环$B\otimes_A\kappa(\mathfrak{p}),\mathfrak{p}\in\mathrm{Spec}(A)$都满足$(R_i)$,那么$B$满足$(R_i)$.
    	\item 上两条中把$(R_i)$改成$(S_i)$同样成立.
    \end{enumerate}
    \begin{proof}
    	
    	第一条.取素理想$p\in\mathrm{Spec}(A)$的高度$\le i$,局部环之间的平坦映射一定是忠实平坦的,此时环扩张满足提升条件,我们在它的所有提升素理想中取一个极小元$\mathfrak{P}$,那么有$\mathrm{ht}(\mathfrak{P}/\mathfrak{p}B)$.于是$B_{\mathfrak{P}}/\mathfrak{p}B_{\mathfrak{P}}$的维数为零,于是按照维数等式,有$\mathrm{ht}(\mathfrak{P})=\mathrm{ht}(\mathfrak{p})\le i$.于是$B_{\mathfrak{P}}$是正则的,这说明$A_{\mathfrak{p}}$也是正则的.另外按照本节的定理有深度等式$\mathrm{depth}(B_{\mathfrak{P}})=\mathrm{depth}(A_{\mathfrak{p}})$(因为原本右侧要加的纤维环的深度$\mathrm{depth}(B_{\mathfrak{P}}/\mathfrak{p}B_{\mathfrak{P}})\le\dim(B_{\mathfrak{P}}/\mathfrak{p}B_{\mathfrak{P}})=0$),于是如果$B$满足$(S_i)$条件,那么如果$\mathfrak{p}\in\mathrm{Spec}(A)$满足$\mathrm{depth}(A_{\mathfrak{p}})<i$,从$B_{\mathfrak{P}}$是CM环得到$A_\mathfrak{p}$是CM环,于是$A$满足$(S_i)$.
    	
    	第二条.取$\mathfrak{P}\in\mathrm{Spec}(B)$,记$\mathfrak{p}=\mathfrak{P}\cap A$,如果$\mathrm{ht}(\mathfrak{P})\le i$,那么按照平坦下降定理,有$\mathrm{ht}(\mathfrak{p})\le i$和$\mathrm{ht}(\mathfrak{P}/\mathfrak{p}B)\le i$,于是条件保证了$A_{\mathfrak{p}}$和$B_{\mathfrak{P}}/\mathfrak{p}B_{\mathfrak{P}}$都是正则环,于是$B_{\mathfrak{P}}$是正则局部环.对于$(S_i)$条件,按照本节证明的$\mathrm{depth}(B_{\mathfrak{P}})=\mathrm{depth}(A_{\mathfrak{p}})+\mathrm{depth}(B_{\mathfrak{P}}/\mathfrak{p}B_{\mathfrak{P}})$,说明如果$\mathrm{depth}(B_{\mathfrak{P}})<i$,则$\mathrm{depth}(A_{\mathfrak{p}})$和$\mathrm{depth}(B_{\mathfrak{P}}/\mathfrak{p}B_{\mathfrak{P}})$都小于$i$,于是$A_{\mathfrak{p}}$和$B_{\mathfrak{P}}/\mathfrak{p}B_{\mathfrak{P}}$都是CM环,于是$B_{\mathfrak{P}}$是CM环,得到$B$满足$(S_i)$条件.
    \end{proof}
    \item 推论.设$(A,\mathfrak{m})$和$(B,\mathfrak{n})$是诺特局部环,$A\to B$是一个局部映射,$B$在$A$上平坦,那么:
    \begin{enumerate}
    	\item 若$B$是既约环,那么$A$也是既约环.
    	\item 若$B$是正规环,那么$A$也是正规环.
    	\item 若$A$和全部纤维环$B\otimes_A\kappa(\mathfrak{p}),\mathfrak{p}\in\mathrm{Spec}(A)$都是既约环,那么$B$也是既约环.
    	\item 若$A$和全部纤维环$B\otimes_A\kappa(\mathfrak{p}),\mathfrak{p}\in\mathrm{Spec}(A)$都是正规环,那么$B$也是正规环.
    \end{enumerate}
\end{enumerate}
\newpage
\subsection{Nagata准则}

局部自由性.
\begin{enumerate}
	\item 设$A$是环,$M$是有限$A$模.
	\begin{enumerate}
		\item 对任意自然数$r$,用$U_r$表示那些使得$M_{\mathfrak{p}}$在$A_{\mathfrak{p}}$上被$r$个元生成的素理想$\mathfrak{p}$构成的集合,那么$U_r$是$\mathrm{Spec}A$的开子集.在证明中我们还会证明,如果$\omega_1,\cdots,\omega_r\in M$生成了$M_{\mathfrak{p}}$,那么存在$\mathfrak{p}$的开邻域$V$,使得$\omega_1,\cdots,\omega_r$生成了每个$M_{\mathfrak{q}}$.
		\item 如果$M$是有限表示模,用$U$表示那些使得$M_{\mathfrak{p}}$是自由$A_{\mathfrak{p}}$模的素理想$\mathfrak{p}$构成的集合,那么$U$是$\mathrm{Spec}A$的开子集.在证明中我们还会证明,在条件下,如果素理想$\mathfrak{p}$满足$M_{\mathfrak{p}}$是$A_{\mathfrak{p}}$自由模,那么存在$a\not\in \mathfrak{p}$,使得$M_a$是自由$A_a$模.
	\end{enumerate}
	\begin{proof}
		
		(a):先设$M_{\mathfrak{p}}=A_{\mathfrak{p}}\omega_1+\cdots+A_{\mathfrak{p}}\omega_r$,其中$\omega_i=m_i/s_i\in M_{\mathfrak{p}}$,并且$m_i\in M$,$s_i\in A-\mathfrak{p}$.那么有$M_{\mathfrak{p}}=A_{\mathfrak{p}}m_1+\cdots+A_{\mathfrak{p}}m_r$.定义$A$模同态$\varphi:A^r\to M$为$(a_1,\cdots,a_r)\mapsto\sum_ia_im_i$,它的余核记作$C$,于是有正合列$A^r\to M\to C\to0$.对素理想$\mathfrak{q}$做局部化得到正合列$A_{\mathfrak{q}}^r\to M_{\mathfrak{q}}\to C_{\mathfrak{q}}\to0$.如果$\mathfrak{p}=\mathfrak{q}$,此时$C_{\mathfrak{q}}=0$.由于$C$是$M$的商,导致$C$是有限$A$模,所以$\mathrm{Supp}(C)$在素谱中是闭子集.于是存在$\mathfrak{p}$的开邻域$V$使得$C_{\mathfrak{q}}=0,\forall\mathfrak{q}\in V$.于是$V\subseteq U_r$,这说明$U_r$是开集.
		
		\qquad
		
		(b):设$M_{\mathfrak{p}}$是自由$A_{\mathfrak{p}}$模,设$\omega_1,\cdots,\omega_r$是一组基,我们可以不妨设$\omega_i\in M$.按照上一条,可取$\mathfrak{p}$的开邻域$D(a)$,使得对任意$\mathfrak{q}\in D(a)$都有$M_{\mathfrak{q}}$被$\omega_1,\cdots,\omega_r$生成.于是用$A_a$和$M_a$分别替换$A$和$M$(并且一旦证明此时$U$是开集,就有原本的$U$是开集),我们可以不妨设$\omega_1,\cdots,\omega_r\in M$能够生成任意一个$M_{\mathfrak{q}}$,其中$\mathfrak{q}$是任意素理想.也即$M/\sum A\omega_i$在任意素理想的局部化都是零,这导致它本身是零模,也即$M=\sum_iA\omega_i$.构造$A^r\to M$为$(a_1,\cdots,a_r)\mapsto\sum_ia_i\omega_i$,记核为$K$,我们有短正合列$0\to K\to A^r\to M\to0$,并且满足$K_{\mathfrak{p}}=0$.用(a)的$r=0$的情况,说明存在$\mathfrak{p}$的开邻域$V$,使得每个$\mathfrak{q}\in V$都满足$K_{\mathfrak{q}}=0$,也即$M_{\mathfrak{q}}$是自由$A_{\mathfrak{q}}$模.这说明$U$是开集.
	\end{proof}
	\item 推论.设$M$是诺特环$A$上的有限模,如果素理想$\mathfrak{p}$满足$M_{\mathfrak{p}}$是自由$A_{\mathfrak{p}}$模,那么存在一个$a\not\in\mathfrak{p}$使得$M_a$是自由$A_a$模.
	\item 推论.设$A$是诺特整环,设$M$是有限$A$模,那么存在$0\not=a\in A$使得$M_a$是自由$A_a$模.因为零理想的局部化是域.
    \item 引理1.设$A$是环,$E$是$A$模,存在$E$的滤过$0=E_0\subseteq E_1\subseteq\cdots$,$E=\cup_iE_i$,使得合成因子(相邻两项的商)都是自由$A$模,那么$E$是自由$A$模.
    \item 引理2.设$B$是诺特环,设$C$是有限生成$B$代数,设$E$是有限$C$模,设$F\subseteq E$是一个有限$B$子群,那么存在$E$的$B$子模滤过$F=E_0\subseteq E_1\subseteq\cdots$(按照$B$是诺特的,滤过总是有限长度的),满足$E=\cup_iE_i$,并且相邻两项的商都是有限$B$模.
    \begin{proof}
    	
    	取$F'\subseteq E$是有限$B$模,使得它包含$F$,也包含$E$作为$C$模的一组有限生成元集.如果我们找到以$F'$为起点的满足结论的滤过链,把它补上$F\subseteq F'$,此时$F'/F$是有限$B$模,并且补充的滤过就是以$F$为起点的满足结论的滤过链.于是我们不妨设$CF=E$,否则可以用$F'$替代$F$.设$B$在$C$中的像是$B'$,可记$C=B'[x_1,\cdots,x_h]$.我们来对$h$归纳.如果$h=0$,那么$E$本身就是有限$B$模并且包含$F$,直接取滤过$F=E_0\subseteq E_1=E$即可.
    	
    	\qquad
    	
    	下面设$h>0$,记$C'=B'[x_1,\cdots,x_{h-1}]\subseteq C$.按照归纳假设,有限$C'$模$C'F$存在满足结论的以$F$为起点的滤过链.一旦我们能证明$E$存在以$C'F$为起点的$C'$子模构成的滤过,使得相邻两项的商是有限$C'$模.比方说这个滤过有相邻两项是$M_1\subseteq M_2$,这里$M_2$是有限$C'$模,【】反复用$h-1$的情况,$C'$子模的滤过
    \end{proof}
    \item 设$A$是诺特整环,设$R$是有限生成$A$代数,设$S$是有限生成$S$代数,设$E$是有限$S$模,设$M\subseteq E$是有限$R$模,设$N\subseteq E$是有限$A$模.记$D=E/(M+N)$,这是一个$A$模.我们断言存在$0\not=a\in A$满足$D_a$是$A_a$自由模.【】
    \item 推论.设$A$是诺特整环,设$B$是有限生成$A$代数,设$M$是有限$B$模,那么存在$a\in A$使得$M_a$是自由$A_a$模.
\end{enumerate}

拓扑Nagata准则和平坦性.
\begin{enumerate}
	\item 拓扑Nagata准则.设$A$是诺特环,设$U\subseteq\mathrm{Spec}A$是一个子集,那么$U$是开集当且仅当如下两个条件成立:
	\begin{enumerate}
		\item $U$在一般化下不变,换句话讲如果素理想$\mathfrak{p}\subseteq\mathfrak{q}$,如果$\mathfrak{q}\in U$,那么有$\mathfrak{p}\in U$.
		\item 如果$\mathfrak{p}\in U$,那么$U$包含了$V(\mathfrak{p})=\overline{\{\mathfrak{p}\}}$的某个非空开子集.
	\end{enumerate}
    \begin{proof}
    	
    	如果$U$本身是开集,明显这两个条件成立.反过来我们设这两个条件成立.记$\overline{\mathrm{Spec}A-U}=V(I)$.设$\mathfrak{p}_1,\cdots,\mathfrak{p}_r$是$I$的全部极小素理想,记$V_i=V(\mathfrak{p}_i)$,这是$V(I)$的全部不可约分支.我们只需证明$\mathrm{Spec}A-U=\cup_{1\le i\le r}V_i$.如果$\mathfrak{p}_i\in U$,按照(b)知$U$包含了$V_i$的某个非空开子集$W\cap V_i$,其中$W$是$\mathrm{Spec}A$的开子集.于是$\mathrm{Spec}A-U\subseteq(V_i-W)\cup\left(\cup_{j\not=i}V_j\right)$.但是右侧是闭集,左侧取闭包依旧在右侧中,导致$\cup_{j=1}^rV_j\subseteq(V_i-W)\cup\left(\cup_{j\not=i}V_j\right)$.但是按照$V_i-W\not=V_i$,它不包含$\mathfrak{p}_i$.这个矛盾说明每个$\mathfrak{p}_i\not\in U$.但是按照(a),如果任取$V(I)\cap U$中的元$\mathfrak{q}$,都存在某个$\mathfrak{p}_i$是$\mathfrak{q}$的一般化,这迫使$V(I)\cap U$是空集,也即$\mathrm{Spec}A-U=\cup_iV_i$.
    \end{proof}
    \item 设$A$是诺特环,设$B$是有限生成$A$代数,设$M$是有限$B$模,设$U=\{\mathfrak{P}\in\mathrm{Spec}B\mid M_{\mathfrak{P}}\text{在}A\text{上平坦}\}$,那么$U\subseteq\mathrm{Spec}B$是开子集.
    \begin{proof}
    	
    	我们来验证拓扑Nagata准则的两个条件.第一条:如果$\mathfrak{P}\subseteq\mathfrak{Q}$是$B$的两个素理想,并且$M_{\mathfrak{Q}}$是平坦$A$模,按照$-\otimes_AM_{\mathfrak{P}}=(-\otimes_AM_{\mathfrak{Q}})\otimes_{B_{\mathfrak{Q}}}B_{\mathfrak{P}}$得到$M_{\mathfrak{P}}$是平坦$A$模.
    	
    	\qquad
    	
    	第二条:设$\mathfrak{P}\in U$,记$\mathfrak{p}=\mathfrak{P}\cap A$,记$A'=A/\mathfrak{p}$.任取$\mathfrak{Q}\in V(\mathfrak{P})$,那么有$\mathfrak{p}\subseteq\mathfrak{Q}\cap A$,于是NAK引理说明$M_{\mathfrak{Q}}$是$\mathfrak{p}$-adic可分模.于是按照局部平坦准则,$M_{\mathfrak{Q}}$在$A$上平坦当且仅当$M_{\mathfrak{Q}}/\mathfrak{p}M_{\mathfrak{Q}}=(M/\mathfrak{p}M)_{\mathfrak{Q}/\mathfrak{p}B}$在$A'=A/\mathfrak{p}$上平坦,并且有$0=\mathrm{Tor}_1^A(M_{\mathfrak{Q}},A')\cong\mathrm{Tor}_1^A(M,A')\otimes_BB_{\mathfrak{Q}}\cong\mathrm{Tor}_1^A(M,A')_{\mathfrak{Q}}$.
    	
    	\qquad
    	
    	按照一般自由性,存在$a\in A-\mathfrak{p}$,如果记$a$在$B'=B/\mathfrak{p}B$中的像是$a'$,那么$M_a/\mathfrak{p}M_a=(M/\mathfrak{p}M)_{a'}$在$A'_{a'}$上自由.于是如果$\mathfrak{Q}\not\in V(aB)$,就有$M_{\mathfrak{Q}}/\mathfrak{p}M_{\mathfrak{Q}}$在$A'_{a'}$上平坦,于是它也在$A'$上平坦.另一方面因为$A'$是有限$A$模,可以取它的有限自由预解来求$\mathrm{Tor}$,结合$M$是有限$B$模,就得到$\mathrm{Tor}_1^A(M,A')$是有限$B$模.因为它在$\mathfrak{P}$的局部化是零,所以存在$\mathfrak{P}$在$\mathrm{Spec}B$中的开邻域$W$,使得$\mathfrak{Q}\in W$总满足$\mathrm{Tor}_1^A(M,A')_{\mathfrak{Q}}=0$(我们之前证明过,对环$A$上的有限模$M$,用$U_r$表示使得$M_{\mathfrak{p}}$在$A_{\mathfrak{p}}$上被$r$个元生成的素理想$\mathfrak{p}$构成的集合,那么$U_r$总是开集,这里就是$U_0$是开集).于是$U$包含了$V(\mathfrak{P})$的开子集$W\cap V(\mathfrak{P})\cap V(aB)^c$.
    \end{proof}
    \item 设$A$是诺特环,设$B$是有限生成$A$代数,并且$A\to B$满足下降条件(比方说,$B$在$A$上平坦),那么诱导的$f:\mathrm{Spec}B\to\mathrm{Spec}A$是开映射.
    \begin{proof}
    	
    	设$U\subseteq\mathrm{Spec}B$是开集,那么$f(U)$是一个可构造集,它是开集当且仅当在一般化下不变.也即如果$\mathfrak{q}_1\in f(U)$,也即$\mathfrak{q}_1\cap A=\mathfrak{p}_1\in U$,并且$\mathfrak{q}_2\subseteq\mathfrak{q}_1$,那么存在$\mathfrak{p}_2\in U$使得$\mathfrak{p}_2=\mathfrak{q}_2\cap A$,但是此即下降条件.
    \end{proof}
\end{enumerate}

open loci.
\begin{itemize}
	\item 设P是一个关于局部环的性质,对交换环$A$,我们用$\mathrm{P}(A)$表示使得$A_{\mathfrak{p}}$具有性质P的素理想$\mathfrak{p}$构成的集合.例如P是正则局部环,完全交环,Gorenstein环,CM环时我们分别用$\mathrm{Reg}(A)$,$\mathrm{CI}(A)$,$\mathrm{Gor}(A)$和$\mathrm{CM}(A)$表示$\mathrm{P}(A)$.
	\item 设P是局部环上四个性质正则局部环,完全交环,Gorenstein环,CM环之一,称性质P满足Nagata准则,如果对诺特环$A$,如果对任意素理想$\mathfrak{p}\in\mathrm{Spec}A$,有$\mathrm{P}(A/\mathfrak{p})$包含了$\mathrm{Spec}(A/\mathfrak{p})$的某个开子集,那么有$\mathrm{P}(A)$是$\mathrm{Spec}A$的开子集.
\end{itemize}
\begin{enumerate}
	\item P=正则局部环满足Nagata准则.换句话讲,如果$A$是诺特环,如果对每个素理想$\mathfrak{p}$,都有$\mathrm{Reg}(A/\mathfrak{p})$包含了$\mathrm{Spec}A/\mathfrak{p}$的某个开子集,那么$\mathrm{Reg}(A)$是$\mathrm{Spec}A$的开子集.
	\begin{proof}
		
		设$A$是诺特环,记$U=\mathrm{Reg}(A)$.因为正则局部环在某个素理想的局部化还是正则局部环,说明$U$满足拓扑Nagata准则的第一个条件.下面验证第二个条件,即如果素理想$\mathfrak{p}$使得$A_{\mathfrak{p}}$是正则局部环,我们要找$V(\mathfrak{p})$的一个非空开子集包含在$U$里.
		
		\qquad
		
		因为$A_{\mathfrak{p}}$是正则局部环,可取$x_1,x_2,\cdots,x_n\in\mathfrak{p}$使得它在$A_{\mathfrak{p}}$中的像构成正则参数系统,其中$n=\mathrm{ht}(\mathfrak{p})$.我们之前解释过此时存在$\mathfrak{p}$在素谱中的开邻域$W$使得对任意$\mathfrak{q}\in W$都有$\mathfrak{p}A_{\mathfrak{q}}=(x_1,\cdots,x_n)A_{\mathfrak{q}}$.另一方面,设$\mathrm{Reg}(A/\mathfrak{p})$包含了$\mathrm{Spec}A/\mathfrak{p}$的开子集$W'$.那么对$\mathfrak{q}\in W'\subseteq V(\mathfrak{p})\subseteq\mathrm{Spec}A$,我们有$A_{\mathfrak{q}}/\mathfrak{p}A_{\mathfrak{q}}=(A/\mathfrak{p})_{\mathfrak{q}/\mathfrak{p}}$是$r=\mathrm{ht}(\mathfrak{q}/\mathfrak{p})$维的正则局部环,于是存在$y_1,\cdots,y_r\in A_{\mathfrak{q}}$,使得它在$A_{\mathfrak{q}}/\mathfrak{p}A_{\mathfrak{q}}$中构成正则参数系统.于是如果$\mathfrak{q}\in W\cap W'$,那么$\mathfrak{q}A_{\mathfrak{q}}=(x_1,\cdots,x_n,y_1,\cdots,y_s)A_{\mathfrak{q}}$.按照$\dim(A_{\mathfrak{q}})=\mathrm{ht}(\mathfrak{q})\ge\mathrm{ht}(\mathfrak{p})+\mathrm{ht}(\mathfrak{q}/\mathfrak{p})=n+r$,得到$x_1,\cdots,x_n,y_1,\cdots,y_s$构成$A_{\mathfrak{q}}$的正则参数系统,于是$A_{\mathfrak{q}}$是正则局部环,于是$W\cap W'\subseteq U$,而且$W\cap W'$是$V(\mathfrak{p})$的非空开子集(因为包含$\mathfrak{p}$),于是$A$满足拓扑Nagata准则的第二个条件.
	\end{proof}
    \item 推论.设$k$是代数闭域,设$A$是有限生成$k$代数,那么$\mathrm{Reg}(A)$是开集.
    \begin{proof}
    	
    	按照Nagata准则,归结为证明如果$A$还是一个整环,那么$\mathrm{Reg}(A)$包含了$\mathrm{Spec}A$的某个非空开集.我们知道代数闭域上正则和光滑是等价的,并且此时$A$总存在一个光滑闭点(比方说,域上几何既约局部有限型概形总存在正则闭点),于是此时该点附近的有限型仿射开覆盖的雅各比矩阵的秩是$n-d$,也即存在一个$n-d$阶子式在该点的剩余域里是非零的.于是存在该点的开邻域,使得这个$n-d$阶子式处处非零(因为一个数在剩余域中为零等价于落在对应的素理想里,这在素谱里是一个闭集).这就说明了$\mathrm{Reg}(A)$包含了一个非空开集.
    \end{proof}
    \item 更一般的,如果$R$是诺特环,满足如下三个条件的任一,那么对任意有限生成$R$代数$A$,都有$\mathrm{Reg}(A)$是$\mathrm{Spec}A$的开子集.
    \begin{enumerate}
    	\item $R$是特征为零的戴德金整环.
    	\item $R$是维数$\le1$的诺特半局部环.
    	\item $R$是完备诺特局部环.
    \end{enumerate}
    \item 引理.设$A$是诺特环,$I\subseteq A$是理想,存在正整数$s$使得$I^s=0$,并且对每个$0\le i\le s-1$都有$I^i/I^{i+1}$是有限自由$A/I$模,那么$x_1,\cdots,x_n\in A$是$A$正则序列当且仅当它是$A/I$正则序列.
    \begin{proof}
    	
    	我们先断言$\mathrm{Ass}_A(A)=\mathrm{Ass}_A(A/I^i)$对任意$i$成立.对每个$i$考虑如下短正合列:
    	$$\xymatrix{0\ar[r]&I^i/I^{i+1}\ar[r]&A/I^{i+1}\ar[r]&A/I^i\ar[r]&0}$$
    	
    	因为$I^i/I^{i+1}$都是有限自由$A/I$模,于是$\mathrm{Ass}_A(A/I)=\mathrm{Ass}_A(I^i/I^{i+1})\subseteq\mathrm{Ass}_A(A/I^{i+1})$.特别的,取$i=s-1$得到$\mathrm{Ass}_A(A/I)\subseteq\mathrm{Ass}_A(A)$.另一方面从$i=s-1$的短正合列得到$\mathrm{Ass}_A(A)\subseteq\mathrm{Ass}_A(A/I)\cup\mathrm{Ass}_A(A/I^{s-1})$.从$i=s-2$的短正合列得到$\mathrm{Ass}_A(A/I^{s-1})\subseteq\mathrm{Ass}_A(A/I)\cup\mathrm{Ass}_A(A/I^{s-2})$,归纳下去最后得到$\mathrm{Ass}_A(A)\subseteq\mathrm{Ass}_A(A/I)$,于是这些包含关系都是取等的,于是特别的$\mathrm{Ass}_A(A/I^{i+1})\subseteq\mathrm{Ass}_A(A/I)$.这得到$\mathrm{Ass}_A(A)=\mathrm{Ass}_A(A/I^i),\forall i$.
    	
    	\qquad
    	
    	由于所有伴随素理想的并就是所有零因子,于是$x_1\in A$是$A$正则元当且仅当$x_1$是$A/I^i$正则元$\forall i\ge1$.设$x_1$是$A$正则元,记$A'=A/x_1A$,记$B=A/I$,记$B'=B/x_1B$,记$I'=I/x_1I$,记$I'=(x_1A+I)/x_1A\cong I/(x_1A\cap I)=I/x_1I$,这里最后一个等式是因为$x_1$是$A/I$正则元.于是$B'=A'/I'$.再记$(I^i)'=(x_1A+I^i/xA)$,那么有$(I^i)'/(I^{i+1})'=(x_1A+I^i)/(x_1A+I^{i+1})\cong I^i/(I^i\cap(x_1A+I^{i+1}))\cong I^i/(x_1I^i+I^{i+1})$.这里最后一个等式是因为$x_1$是$A/I^i$正则元.于是$(I^i)'/(I^{i+1})'\cong\mathrm{coker}(\xymatrix{I^i/I^{i+1}\ar[r]^{x_1}&I^i/I^{i+1}})$是一个自由$B'=A'/I'$模.于是依旧按照第一段的讨论,有$x_2\in A$是$A'$正则元当且仅当$x_2$是$B'=A'/I'$正则元.归纳操作下去得到结论.
    \end{proof}
    \item P=CM环满足Nagata准则.换句话讲,如果$A$是诺特环,如果对每个素理想$\mathfrak{p}$,都有$\mathrm{CM}(A/\mathfrak{p})$包含了$\mathrm{Spec}A/\mathfrak{p}$的某个开子集,那么$\mathrm{CM}(A)$是$\mathrm{Spec}A$的开子集.
    \begin{proof}
    	
    	设$A$是诺特环,记$U=\mathrm{CM}(A)$.因为CM局部环在一个素理想处的局部化还是CM环,于是$U$满足拓扑Nagata准则的第一个条件.下面验证第二个条件.设$\mathfrak{p}\in U$,设$n=\mathrm{ht}(\mathfrak{p})$.我们要证明$U$包含了$V(\mathfrak{p})$的某个非空开子集.
    	
    	\qquad
    	
    	因为$A_{\mathfrak{p}}$是CM环,可取$y_1,\cdots,y_n\in\mathfrak{p}$使得它们在$A_{\mathfrak{p}}$中是正则序列.因为$\ker(\xymatrix{A\ar[r]^{y_1}&A})_{\mathfrak{p}}\cong\ker(\xymatrix{A_{\mathfrak{p}}\ar[r]^{y_1}&A_{\mathfrak{p}}})=0$.于是可取$a\not\in\mathfrak{p}$,使得$\ker(\xymatrix{A\ar[r]^{y_1}&A})_a\cong\ker(\xymatrix{A_a\ar[r]^{y_1}&A_a})=0$.于是$y_1$是$A_a$正则元.明显有$\mathrm{CM}(A_a)=\mathrm{CM}(A)\cap D(a)$.于是只需证明$\mathrm{CM}(A_a)$包含了$V(\mathfrak{p})$的某个非空开子集.综上我们不妨设$y_1$是$A$正则元,否则可以用$A_a$替换$A$.归纳的操作下去,我们不妨设$y_1,\cdots,y_n$是$A$正则序列.
    	
    	\qquad
    	
    	下面断言$\mathfrak{p}$是$I=(y_1,\cdots,y_n)A$的极小素理想:设$\mathfrak{q}$是包含在$\mathfrak{p}$里的$I$的极小素理想,那么Krull高度定理说明$\mathrm{ht}(\mathfrak{q})\le n$.但是由于$y_1,\cdots,y_n$是$A$正则序列,它们也是$A_{\mathfrak{q}}$正则序列,于是$0\le\dim(A_{\mathfrak{q}}/(y_1,\cdots,y_n)A_{\mathfrak{q}})=\dim(A_{\mathfrak{q}})-n$.这迫使$\mathrm{ht}(\mathfrak{q})=\dim(A_{\mathfrak{q}})=n$,也即$\mathfrak{p}=\mathfrak{q}$.
    	
    	\qquad
    	
    	于是如果适当把$A$替换为更小的主开集上(比方说,对包含$\mathfrak{p}$的素理想$\mathfrak{q}$,取$\mathfrak{q}-\mathfrak{p}$中的元取主开集,对$I$的其它极小素理想$\mathfrak{q}$,也取$\mathfrak{q}-\mathfrak{p}$中的元取主开集),我们不妨设$\mathfrak{p}$是$A/(y_1,\cdots,y_n)A$的唯一伴随素理想,换句话讲$I=(y_1,\cdots,y_n)A$是$\mathfrak{p}$准素理想.
    	
    	\qquad
    	
    	下面设$\mathfrak{q}\in V(\mathfrak{p})=\mathrm{Spec}(A/\mathfrak{p})=\mathrm{Spec}(A/I)$.那么我们解释过$A_{\mathfrak{q}}$是CM环当且仅当模去正则序列也是CM环,也即$A_{\mathfrak{q}}/IA_{\mathfrak{q}}$是CM的.于是不妨用$A/I$替换$A$,归结为设$(0)$是$\mathfrak{p}$准素理想,并且此时$\mathfrak{p}$是$A$的唯一极小素理想.那么诺特条件保证存在某个正整数$r$使得$\mathfrak{p}^r=0$.考虑链$0=\mathfrak{p}^r\subseteq\mathfrak{p}^{r-1}\subseteq\cdots\subseteq\mathfrak{p}\subseteq A$.这里每个$\mathfrak{p}^i/\mathfrak{p}^{i+1}$都是有限$A/\mathfrak{p}$模,这里$A/\mathfrak{p}$是整环.于是再限制在更小的主开集上可以约定每个$\mathfrak{p}^i/\mathfrak{p}^{i+1}$都是$A/\mathfrak{p}$上的自由模.此时如果$x_1,\cdots,x_m\in A$是$A/\mathfrak{p}$正则序列,那么它也是$A$正则序列(上面引理).于是有$\mathrm{depth}(A_{\mathfrak{q}})\ge\mathrm{depth}(A_{\mathfrak{q}}/\mathfrak{p}A_{\mathfrak{q}})$对任意$\mathfrak{q}\in V(\mathfrak{p})=\mathrm{Spec}A$成立.
    	
    	\qquad
    	
    	但是按照Nagata准则的条件,这里$\mathrm{CM}(A/\mathfrak{p})$包含了$\mathrm{Spec}A/\mathfrak{p}$的某个开子集,所以适当传递给某个包含在这个开集中的主开集,我们可设$A/\mathfrak{p}$本身就是CM环.于是对任意$\mathfrak{q}\in V(\mathfrak{p})=\mathrm{Spec}(A)$,就有$A_{\mathfrak{q}}/\mathfrak{p}A_{\mathfrak{q}}$是CM的,于是上述等式得到$\mathrm{depth}(A_{\mathfrak{q}})\ge\mathrm{depth}(A_{\mathfrak{q}}/\mathfrak{p}A_{\mathfrak{q}})=\dim(A_{\mathfrak{q}}/\mathfrak{p}A_{\mathfrak{q}})=\dim(A_{\mathfrak{q}})$,这说明$A_{\mathfrak{q}}$是CM环,于是$U=V(\mathfrak{p})=\mathrm{Spec}A$.完成拓扑Nagata准则第二条的验证.
    \end{proof}
    \item 设$A$是诺特环,设$I$是理想,记$B=A/I$,记$Y=V(I)\subseteq\mathrm{Spec}A$,设$M$是有限$A$模.我们称$M$是沿$Y$正规平坦的(normally flat),如果$B$模$\mathrm{gr}_I(M)=\oplus_{i\ge0}I^iM/I^{i+1}M$是平坦的.上一条的证明中我们得到了:如果$I\subseteq A$是幂零的,如果$A$沿着$Y$是正规平坦的,那么$A/I$正则序列也是$A$正则序列.
    \item 推论.设$A$是诺特环,并且是某个CM环的商(例如$A$是域$k$上的有限生成代数,我们解释过$k[X_1,\cdots,X_n]$总是CM环),那么$\mathrm{CM}(A)$是$\mathrm{Spec}A$的开子集.
    \begin{proof}
    	
    	按照CM条件满足Nagata准则,归结为证明如果$A$是某个CM环$R$的商并且$A$是整环,那么$\mathrm{CM}(A)$包含了$\mathrm{Spec}A$的某个开子集.记$A=R/\mathfrak{p}$,其中$\mathfrak{p}\in\mathrm{Spec}R$.因为$R$是CM环,可取$R_{\mathfrak{p}}$正则序列$x_1,\cdots,x_n\in\mathfrak{p}$,其中$n=\mathrm{ht}(\mathfrak{p})$.适当限制在主开集上,我们不妨设$x_1,\cdots,x_n$也是$R$正则序列,并且$I=(x_1,\cdots,x_n)R$是$\mathfrak{p}$准素理想.特别的此时$R/I$也是CM环,于是用$R/I$替换$R$,我们不妨设$\mathrm{ht}(\mathfrak{p})=0$,并且$(0)$是$\mathfrak{p}$准素理想.于是$\mathfrak{p}$是幂零理想,记正整数$s$使得$\mathfrak{p}^s=0$.和上面证明一样,适当限制在更小的主开集上可以保证$\mathfrak{p}^i/\mathfrak{p}^{i+1}$都是$R/\mathfrak{p}$有限自由模.取$\mathfrak{q}\in V(\mathfrak{p})=\mathrm{Spec}(R)$.按照上面引理,有$R_{\mathfrak{q}}$和$R_{\mathfrak{q}}/\mathfrak{p}R_{\mathfrak{q}}$具有相同深度,于是$R_{\mathfrak{q}}$是CM环得到$A_{\mathfrak{q}/\mathfrak{p}}=R_{\mathfrak{q}}/\mathfrak{p}R_{\mathfrak{q}}$也是CM环.于是此时$A$本身就是CM环,于是$U=\mathrm{CM}(A)=\mathrm{Spec}A$是开集.
    \end{proof}
    \item P=Gorenstein环满足Nagata准则.换句话讲,如果$A$是诺特环,如果对每个素理想$\mathfrak{p}$,都有$\mathrm{Gor}(A/\mathfrak{p})$包含了$\mathrm{Spec}A/\mathfrak{p}$的某个开子集,那么$\mathrm{Gor}(A)$是$\mathrm{Spec}A$的开子集.
    \begin{proof}
    	
    	设$A$是诺特环,设$U=\mathrm{Gor}(A)$.依旧按照Gorenstein局部环在素理想处的局部化还是Gorenstein环,于是$U$满足拓扑Nagata准则的第一条.下面设$\mathfrak{p}\in U$,记$n=\mathrm{ht}(\mathfrak{p})$.因为$A_{\mathfrak{p}}$是CM环,可取$x_1,\cdots,x_n\in\mathfrak{p}$是$A_{\mathfrak{p}}$正则序列.依旧适当约化到更小的主开集上,可设$x_1,\cdots,x_n$也是$A$正则序列.于是对包含$(x_1,\cdots,x_n)$的素理想$\mathfrak{q}$,我们有$A_{\mathfrak{q}}$是Gorenstein环当且仅当$A_{\mathfrak{q}}/(x_1,\cdots,x_n)A_{\mathfrak{q}}$是Gorenstein环.于是把$A$替换为$A/(x_1,\cdots,x_n)A$,我们可设$\mathrm{ht}(\mathfrak{p})=0$,并且可设$\mathfrak{p}$是$A$的唯一极小素理想.
    	
    	\qquad
    	
    	因为$A_{\mathfrak{p}}$是零维Gorenstein局部环,于是$\mathrm{Ext}_A^1(A/\mathfrak{p},A)_{\mathfrak{p}}=\mathrm{Ext}_{A_{\mathfrak{p}}}^1(\kappa(\mathfrak{p}),A_{\mathfrak{p}})=0$和$\mathrm{Hom}_A(A/\mathfrak{p},A)_{\mathfrak{p}}\cong\mathrm{Hom}_{A_{\mathfrak{p}}}(\kappa(\mathfrak{p}),A_{\mathfrak{p}})\cong\kappa(\mathfrak{p})\cong(A/\mathfrak{p})_{\mathfrak{p}}$.因为$\mathrm{Ext}_A^1(A/\mathfrak{p},A)$和$\mathrm{Hom}_A(A/\mathfrak{p},A)$都是有限$A$模,约化到主开集上可以要求有$\mathrm{Ext}_A^1(A/\mathfrak{p},A)=0$和$\mathrm{Hom}_A(A/\mathfrak{p},A)\cong A/\mathfrak{p}$.这里$\mathfrak{p}$是唯一的极小素理想,所以存在正整数$r$使得$\mathfrak{p}^r=0$.和之前的证明一样,我们适当选取主开集,可以归结为设$\mathfrak{p}^i/\mathfrak{p}^{i+1}$都是有限自由$A/\mathfrak{p}$模.考虑短正合列$0\to\mathfrak{p}^i/\mathfrak{p}^{i+1}\to\mathfrak{p}/\mathfrak{p}^{i+1}\to\mathfrak{p}/\mathfrak{p}^i\to0$,因为每个$\mathfrak{p}^i/\mathfrak{p}^{i+1}$都是自由$A/\mathfrak{p}$模,于是$\mathrm{Ext}_A^i(\mathfrak{p}^i/\mathfrak{p}^{i+1},A)=0$.于是上面短正合列归纳得到$0=\mathrm{Ext}_A^1(\mathfrak{p}/\mathfrak{p}^2,A)=\mathrm{Ext}_A^1(\mathfrak{p}/\mathfrak{p}^3,A)=\cdots=\mathrm{Ext}_A^1(\mathfrak{p},A)$.进而从短正合列$0\to\mathfrak{p}\to A\to A/\mathfrak{p}\to0$诱导的长正合列得到$\mathrm{Ext}_A^2(A/\mathfrak{p},A)=0$.归纳操作下去得到$\mathrm{Ext}_A^i(A/\mathfrak{p},A)=0,\forall i\ge1$.
    	
    	\qquad
    	
    	设$0\to A\to I^*$是$A$的自由预解,那么$\mathrm{Ext}_A^i(A/\mathfrak{p},A)=\mathrm{H}^i(\mathrm{Hom}_A(A/\mathfrak{p},I^*))$.这说明复形$\mathrm{Hom}_A(A/\mathfrak{p},I^*)$提供了$\mathrm{H}^0(\mathrm{Hom}_A(A/\mathfrak{p},I^*))=\mathrm{Hom}_A(A/\mathfrak{p},A)\cong A/\mathfrak{p}$的自由预解.另一方面如果$I$是内射$A$模,那么$\mathrm{Hom}_A(A/\mathfrak{p},I)$是内射$A/\mathfrak{p}$模.于是$\mathrm{Hom}_A(A/\mathfrak{p},I^*)$是$A/\mathfrak{p}$的作为$A/\mathfrak{p}$模的内射预解.做局部化不改变正合性,如果取$\mathfrak{q}\in V(\mathfrak{p})$,取$I^*$是$A_{\mathfrak{q}}$的内射预解,那么$\mathrm{Hom}_{A_{\mathfrak{q}}}(A_{\mathfrak{q}}/\mathfrak{p}A_{\mathfrak{q}},I^*)$就是$A_{\mathfrak{q}}/\mathfrak{p}A_{\mathfrak{q}}$作为自身模的内射预解.于是我们有:
    	\begin{align*}
    		\mathrm{Ext}_{A_{\mathfrak{q}}/\mathfrak{p}A_{\mathfrak{q}}}^1(\kappa(\mathfrak{q}),A_{\mathfrak{q}}/\mathfrak{p}A_{\mathfrak{q}})&\cong\mathrm{H}^i\left(\mathrm{Hom}_{A_{\mathfrak{q}}/\mathfrak{p}A_{\mathfrak{q}}}(\kappa(\mathfrak{q}),\mathrm{Hom}_{A_{\mathfrak{q}}}(A_{\mathfrak{q}}/\mathfrak{p}A_{\mathfrak{q}},I^*))\right)\\&\cong\mathrm{H}^i\left(\mathrm{Hom}_{A_{\mathfrak{q}}}(\kappa(\mathfrak{q}),I^*)\right)\\&\cong\mathrm{Ext}_{A_{\mathfrak{q}}}^i(\kappa(\mathfrak{q}),A_{\mathfrak{q}})
    	\end{align*}
    	
    	于是$A_{\mathfrak{q}}$是Gorenstein环当且仅当$A_{\mathfrak{q}}/\mathfrak{p}A_{\mathfrak{q}}$是Gorenstein环,换句话讲$\mathrm{Gor}(A/\mathfrak{p})=\mathrm{Gor}(A)\cap V(\mathfrak{p})$.最后Nagata准则告诉我们$\mathrm{Gor}(A/\mathfrak{p})$包含了$V(\mathfrak{p})$的非空开子集,这验证了拓扑Nagata准则的第二个条件.
    \end{proof}
    \item 设$A$是诺特环,并且是某个Gorenstein环的商(比方说$A$是域上的有限生成代数),那么$\mathrm{Gor}(A)$是$\mathrm{Spec}A$的开子集.
    \begin{proof}
    	
    	按照Gorenstein条件满足Nagata准则,归结为设$A$是诺特整环,并且是某个Gorenstein环$R$的商,只需证明$\mathrm{Gor}(A)$包含了$\mathrm{Spec}A$的某个非空开集.可记$A=R/\mathfrak{p}$,其中$\mathfrak{p}\in\mathrm{Spec}R$,设这个素理想的高度为$n$.那么$R_{\mathfrak{p}}$是CM环,和之前的证明一样,适当选取主开集可以约定存在$x_1,\cdots,x_n\in\mathfrak{p}$是$A$正则序列,并且理想$I=(x_1,\cdots,x_n)R$是$\mathfrak{p}$准素理想.此时$R/I$依旧是Gorenstein环.所以用$R/I$替换$R$,可以不妨设$\mathfrak{p}\in\mathrm{Spec}R$是唯一的极小素理想.于是存在正整数$s$使得$\mathfrak{p}^s=0$.和上面证明一样适当选取主开集可以要求$\mathrm{Ext}_R^1(R/\mathfrak{p},R)=0$和$\mathrm{Hom}_R(R/\mathfrak{p},R)\cong R/\mathfrak{p}$,并且$\mathfrak{p}^i/\mathfrak{p}^{i+1}$都是$R/\mathfrak{p}$上的有限自由模.上一条的证明还说明了对$\mathfrak{q}\in V(\mathfrak{p})=\mathrm{Spec}R$,有$R_{\mathfrak{q}}$是Gorenstein环当且仅当$A_{\mathfrak{q}/\mathfrak{p}}\cong R_{\mathfrak{q}}/\mathfrak{p}R_{\mathfrak{q}}$是Gorenstein环.这导致$A=R/\mathfrak{p}$是Gorenstein环,于是$\mathrm{Gor}(A)=\mathrm{Spec}A=V(\mathfrak{p})$,得证.
    \end{proof}
\end{enumerate}
\newpage
\section{优等环}
\subsection{形式等维数环和形式catenary环}

诺特局部环$(A,\mathfrak{m})$称为形式等维数环(拟非混合环,quais-unmixed ring),如果它的$\mathfrak{m}$-进制完备化$\widehat{A}$是等维数的.
\begin{enumerate}
	\item 设$A$是诺特局部环,那么它是形式等维数环当且仅当$\widehat{A}$的每个极小素理想的余高度是$\dim A$,等价于对$A$的每个极小素理想$\mathfrak{p}$有$\widehat{A}/\mathfrak{p}\widehat{A}$是等维数$\dim A$的.特别的此时$A$是等维数的.
	\begin{proof}
		
		任取$\widehat{A}$的极小素理想$\mathfrak{q}$,那么$\mathfrak{q}\cap A$包含了$A$的某个极小素理想$\mathfrak{p}_i$,于是$\mathfrak{p}_i\widehat{A}\subseteq\mathfrak{q}$.于是$\widehat{A}$的极大长度的素理想链对应于某个$\widehat{A}/\mathfrak{p}_i\widehat{A}$的极大长度的素理想链.但是有$\widehat{A}/\mathfrak{p}_i\widehat{A}=A/\mathfrak{p}_i\otimes_A\widehat{A}$,于是$\dim\widehat{A}/\mathfrak{p}_i\widehat{A}=\dim A/\mathfrak{p}_i$.
	\end{proof}
	\item 设$\varphi:(A,\mathfrak{m})\to A'$是诺特局部环之间的平坦的局部同态,设$A'$是等维数和catenary的,那么有:
	\begin{enumerate}[(1)]
		\item $A$也是等维数和catenary的.
		\item 如果$\mathfrak{m}A'$是$A'$的定义理想,那么理想$\mathfrak{a}\subseteq A$满足$A'/\mathfrak{a}A'$是等维数的当且仅当$A/\mathfrak{a}A$是等维数的.特别的对任意素理想$\mathfrak{p}\in\mathrm{Spec}A$有$A'/\mathfrak{p}A'$是等维数的.
	\end{enumerate}
	\begin{proof}
		
		【】
		
	\end{proof}
\end{enumerate}
\subsection{优等环【】}

一个诺特环$A$称为优等环(excellent ring),如果它满足如下条件:
\begin{itemize}
	\item $A$是泛catenary的.
	\item 对任意素理想$\mathfrak{p}$,有$A_{\mathfrak{p}}$的形式纤维(此即$\mathrm{Spec}\widehat{A_{\mathfrak{p}}}\to\mathrm{Spec}A_{\mathfrak{p}}$的纤维)总是几何正则的.
	\item 对$A$的任意整商环$B$,记$K=\mathrm{Frac}(B)$,对任意有限根式扩张$K'/K$,都存在有限$B$子代数$B'\subseteq K'$,它以$K'$为商域,并且$\mathrm{Spec}B'$的正则点集包含了一个非空开集.
\end{itemize}
\begin{enumerate}
	\item 对诺特局部环,它是优等环当且仅当它是泛catenary的并且形式纤维都是几何正则的.
	\item 如果$A$是优等环,那么$A$的分式化和有限型代数也是优等环.
	\item 完备诺特局部环和特征零的戴德金整环是优等环.
	\item 如果$A$是优等环,那么$X=\mathrm{Spec}A$的全部正则点$\mathrm{Reg}(X)$,全部正规点$\mathrm{Nor}(X)$,固定$n$时全部满足Serre条件$(R_n)$的点$\mathrm{U}_{(\mathrm{R}_n)}(X)$都构成开集.
	\item 设$A$是优等环,设$I$是理想,记$I$-adic完备化为$\widehat{A}$,那么典范态射$f:X'=\mathrm{Spec}\widehat{A}\to X=\mathrm{Spec}A$是正则的,也即平坦和具有几何正则纤维.并且总有:
	$$\mathrm{Reg}(X')=f^{-1}(\mathrm{Reg}(X))$$
	$$\mathrm{Nor}(X')=f^{-1}(\mathrm{Nor}(X))$$
	$$\mathrm{U}_{(\mathrm{R}_n)}(X')=f^{-1}(\mathrm{U}_{(\mathrm{R}_n)}(X))$$
	\item 优等环$A$总是泛日本环.特别的,如果$A$是整优等环,那么它在函数域任意有限扩张中的正规化都是有限$A$代数.
\end{enumerate}


\newpage
\section{完备性}
\subsection{$I$-光滑性}

\newpage
\subsection{完备局部环的结构定理}

\newpage
\subsection{和导数的联系}


\newpage
\subsection{素理想链}

\newpage
\subsection{形式纤维}


\newpage
\subsection{其它应用}







