\chapter{群论}
\section{群范畴}

\begin{introduction}
	\item 基本概念
	\item 循环群
	\item 有限生成交换群的结构定理
	\item Krull-Schmidt定理
\end{introduction}

\subsection{基本概念}

集合$A$上\textbf{二元运算}是指映射$A\times A\to A$,把二元运算的结果记作$(a,b)$,称它具有\textbf{结合律},如果对任意集合中的$a,b,c$满足$((a,b),c)=(a,(b,c))$.称集合是\textbf{半群},如果它具备一个满足集合律的二元运算.半群上的二元运算$(a,b)$通常简记作$ab$,半群上一个元$e$称为\textbf{幺元},如果对于任何元素$a$有$ae=ea=a$,幺元如果存在那么唯一,具备幺元的半群称为\textbf{幺半群},在幺半群中给定一个元素$a$,如果存在一个元$b$满足$ab=ba=e$,就称$b$为$a$的逆元,逆元未必总是存在的,倘若幺半群中每个元素都有逆,就称为\textbf{群}.称群是\textbf{交换群}如果满足交换律$ab=ba$.一个群的子集如果对限制的二元运算封闭并且自身构成一个群,那么我们称它为原来群的\textbf{子群}.

\begin{enumerate}
	\item 群可以看作是只有一个对象的群胚.
	\item 承认结合律的同时便可以不依赖运算次序的定义$n$个元素的积
	$$a_1\circ a_2\circ\cdots\circ a_n$$
	\item 幺元存在则唯一,左右逆元均存在则他们相同,双侧逆元存在则唯一.
	\item 左乘或右乘某个元素是群上的一个双射.
	\item 群上的二元运算通常记作乘法$ab$或者加法$a+b$,在加法的情况下通常是交换群,除非加以说明.
	\item 若群$G$的子集上任意两个元素$a,b$有$ab^{-1}$属于该子集,那么这个子集是子群.
	\item 对于交换群,我们可以把$n$个$a$相加简单记作$na$,将$b$的逆记作$-b$,那么容易验证$n(-a)=-na$.事实上这相当于将整数集$\mathbb{Z}$作用于任一交换群上,也就是任意交换群自然具备一个$\mathbb{Z}$-模结构
	\item 群的子集是子群,当且仅当包含映射是群同态.
	\item 一族子群的交仍为子群,具备保交的结构我们都可以定义生成.即把一个子集生成的子群定义为全部包含这个子集的子群的交.能够被有限子集生成的群称为\textbf{有限生成群}.实际上子集$S$生成的子群可以表示为$\{s_1^{e_1}s_2^{e_2}\cdots s_n^{e_n}\mid n\in\mathbb{N},s_i\in S,e_i=-1,1\}$,其中$n=0$时空集的乘积约定为幺元.
\end{enumerate}

这里给出几个群的例子.
\begin{enumerate}
	\item 对于正整数$n$,在全体整数$\mathbb{Z}$上定义一个等价关系,两个整数等价当且仅当它们的差被$n$整除,由此得到的等价类一共有$n$个,它们分别是除以$n$的余数为$k$的那些整数构成的等价类,记$k$所在的等价类为$[k]$,那么对任意$k$都存在一个$0\le j\le n-1$使得$[k]=[j]$.在等价类上可以定义加法$[i]+[j]=[i+j]$,并由此构成一个交换群.称它为\textbf{模$n$加法群},记作$\mathbb{Z}/n$.它恰好可以由$[1]$生成,因为$n[1]=[n]$.在$\mathbb{Z}/n$中能够作为单个生成元的恰好是那些$[k],0\le k\le n-1$使得$k$和$n$互素.这些元素在乘法下构成一个群.事实上这是$\mathbb{Z}/n$作为环的单位群$(\mathbb{Z}/n)^*$.
	\item 对称群.集合$S$上全体双射在复合意义下构成群,这个群称为集合$S$上的对称群,它的元素称为置换.对称群本质上讲只依赖于集合的势,对于集合中具体元素的特性是无关的,这实际是群的同构的概念.如果集合的势有限,就称为$n$元集合上的对称群,记作$S_n$.对称群的子群称为置换群.事实上对于任意范畴$\Omega$的任意对象$A$,$\mathrm{Hom}(A,A)$的全体同构构成一个群,记作$\mathrm{Aut}_{\Omega}(A)$.它是一个群,称为对象的自同构群,对称群即集合范畴上集合对象的自同构群.
	\item 给定域$F$,全体$n$阶的域$\mathbb{F}$上可逆矩阵构成了一个群,称为一般线性群,记作$\mathrm{GL}_n(\mathbb{F})$.它的全部行列式为1的矩阵构成的子集是一个子群,称为特殊线性群,记作$\mathrm{SL}_n(\mathbb{F})$.
	\item 一个具体的例子,二面体群$D_n$.设$n\ge3$,考虑平面直角坐标系,以点$(1,0)$为一个端点,以原点为中心做端点标号的正$n$边形,不妨以$(0,1)$端点开始逆时针标记为$1,2,\cdots,n$.设$y$是将该正$n$边形以原点为中心逆时针旋转$\frac{2\pi}{n}$的角度,此时新的图形和原本的图形重合.再设$x$是把图形经$x$轴做对称的变换.那么有$x^2=y^n=e$.就把$x,y$生成的群称为$n$次二面体群,记作$D_n$.
	
	我们现在讨论它的阶数.$yx$变换后从$(1,0)$端点开始逆时针的标记是$2,1,n,n-1,\cdots,3$;而$xy^{-1}$变换后从$(1,0)$端点开始逆时针的标记仍然是$2,1,n,\cdots,3$.于是得到$yx=xy^{-1}$.这导致$x,y$生成的群中的任意元$x^{e_1}y^{f_1}x^{e_2}y^{f_2}\cdots$总可以经上述等式整理为$x^iy^j$,其中$i=0,1$和$y=0,1,\cdots,n-1$.验证它们是不同的元,于是$D_n$中有$2n$个元.
\end{enumerate}

元素的阶数.对群中元素$a$,称它具有有限阶,如果存在有限的次数使得$a^n=e$,此时把满足该式的最小正整数次数称为$a$的阶,记作$|a|$.若不存在这样的$n$就称元素的阶是无限的.

\begin{enumerate}
	\item 倘若有$a^N=e$,那么有$|a|\mid N$.这个只要反证用带余除法.
	\item 若$g$具有有限阶,那么对于任意正整数$m$,$g^m$具有有限阶,并且满足:
	$$|g^m|=\frac{|g|}{\gcd(m,|g|)}$$
	\item 倘若$ab=ba$,那么$|ab|$整除$|a|$和$|b|$的最大公约数,特别的如果$|a|$和$|b|$互素那么$|ab|$是它们乘积.
	\item 有限群中元素的阶数必然有限.任取元素$g$,那么序列$1,g,g^2,g^3,\cdots$必然存在两个元相同,否则和群的元素个数有限矛盾.于是有$g^n=g^m,n\not=m$,于是得到$g^{n-m}=e$.另外拉格朗日定理后我们会看到此时元素的阶数总会是群的阶数的约数.
\end{enumerate}


对两个群$G,H$,称一个映射$f:G\to H$是群同态,如果满足$\forall a,b\in G$有$f(ab)=f(a)f(b)$.
\begin{enumerate}
	\item 群同态保群的幺元和逆元.
	\item 若$f:G\to H$是群同态,如果元素$a\in G$具有有限阶,那么$|f(a)|$整除$|a|$.特别的群同构保有限阶元素的阶数.这在某种意义上讲具有逆命题:两个有限阶交换群同构当且仅当它们具有相同个数的任意阶元.
	\item 称群同态在幺元处的纤维为\textbf{核}.那么群同态的核与象均为子群.
\end{enumerate}

\textbf{群范畴}.以群作为对象,以群同态作为态射,则构成了一个范畴,称为群范畴,记作\textbf{Grp}.如果把对象类限制到交换群,这样得到的完全子范畴称为\textbf{交换群范畴},记作\textbf{Ab}.下面给出群范畴和交换群范畴中的一些性质,包括一些特殊指标范畴上的极限,以及单满态射的描述,还有自由对象等.
\begin{enumerate}
	\item
	首先是一些简单的内容.平凡群是群范畴和交换群范畴中的零对象;一个群同态如果是双射,那么它的逆映射自动是群同态.这个时候就称这个群同态为群同构,这也就是群范畴和交换群范畴中的同构.称两个群$G,H$是同构的,如果它们之间存在群同构.
	\item
	交换化函子.我们先来定义导群.对群$G$,定义一个新范畴,它的对象是全体$(f,A)$,其中$A$是一个交换群,$f:G\to A$是一个群同态,定义从$(f,A)$到$(g,B)$的态射是从$A$到$B$的群同态$\phi$满足下面图标交换:
	$$\xymatrix{
		&G\ar[dl]_f\ar[dr]^g&\\
		A\ar[rr]^{\phi}&&B
	}$$
	
	粗略的讲,这个终对象可以理解为最接近$G$的交换群.这个初对象总是存在的.对群$G$中元素$g,h$,定义换位子$[g,h]=g^{-1}h^{-1}gh$.换位子具有一些简单性质,$[e,e]=e$,$[g,h]^{-1}=[h,g]$,对任意群同态$f:G\to H$,有$f([g,h])=[f(g),f(h)]$.但是,两个换位子的乘积未必是一个换位子,例如集合$\{a,b,c,d\}$上的自由群$F$,则$[a,b][c,d]$不会是一个换位子.(阶数最小的换位子不构成子群的群是一个96阶群)
	
	群$G$的所有换位子生成的子群记作$G'$或者$[G,G]$,称为$G$的换位子群或者(一阶)导群.那么$G'$中的元具有的形式是$[g_1,h_1][g_2,h_2]\cdots[g_n,h_n]$.按照$f([g,h])=[f(g),f(h)]$得到换位子群是一个特征子群.另外$G/G'$是一个交换群.存在$G\to G/G'$的典范商同态$\pi$.对任意的交换群$A$和任意的群同态$f:G\to A$.那么满足图表交换的群同态$\phi:G/G'\to A$需要满足$\phi(gG')=f(g)$,先证明定义的良性,当$gG'=hG'$时,$h^{-1}g\in G'$,于是从$A$是交换群得到$f(h^{-1}g)=e_A$,于是$f(g)=f(h)$.于是这是唯一能使图表交换的群同态$\phi:G/G'\to A$.即$(\pi,G/G')$是上述范畴的初对象.泛映射性质转化为群论语言就是在说,$G/H$是交换群当且仅当$H$包含了导群$G'$.
	
	给定群$G$,商群$G/G'$称为$G$的交换化,记作$G^{ab}$.定义交换化函子$F:\textbf{Grp}\to\textbf{Ab}$为$F(G)=G^{ab}$,对群同态$f:G\to H$,定义$F(f):G/G'\to H/H'$为$gG'\mapsto f(g)H'$,这个同态的良性由$f([G,G])\subset[H,H]$所保证.事实上这个交换化函子左伴随于嵌入函子$\textbf{Ab}\to\textbf{Grp}$.这使得范畴$\textbf{Ab}$是$\textbf{Grp}$的反射子范畴.
	
	按照嵌入函子$\textbf{Ab}\to\textbf{Grp}$是右伴随函子,说明交换群范畴中的极限也就是群范畴中涉及到交换群的函子的极限.因此,对于函子的极限,交换群范畴和群范畴是一致的,但是余极限往往不同.
	
	交换化函子不仅是一个函子,它还满足自然性,即存在从$\textbf{Grp}$上恒等函子到交换化函子的自然变换,这里自然变换取的就是典范映射$\phi_G:G\to G^{ab}$.即对任意群同态$f:G\to H$,记诱导的交换群同态$G^{ab}\to H^{ab}$为$f'$,那么有交换图:
	$$\xymatrix{G\ar[r]^{\phi_G}\ar[d]_f&G^{ab}\ar[d]^{f'}\\
		H\ar[r]_{\phi_H}&H^{ab}}$$
	
	\item
	积对象和余积对象.先抛开范畴论,我们介绍群族的直积,内直和与外直积.考虑一族群$G_i,i\in I$,定义它的直积是一个群$G$,作为集合它是这族群的笛卡尔积$G=\prod_ {i\in I}G_i$,即是由$I$到$G_i$无交并的映射$f$,使得每个$i\in I$有$f(i)\in G_i$,定义两个映射$f,g$的二元运算是$fg:I\to\coprod_ {i\in I}G_i$使得$fg(i)=f(i)g(i)\in G_i$.存在从直积到分量的典范群同态,即投影映射$\pi_i:\prod_{i\in I}G_i\to G_i$.
	
	外直和是外直积的子群.定义为$\prod_{i\in I}G_i$中全体只在至多有限个$i\in I$处$f(i)$不取$G_i$中幺元的那些$f$构成的子群.外直和记作$\sum_{i\in I}^{out}G_i$.那么外直和实际上是外直积的一个正规子群.
	
	内直和.给定群$G$,取一族正规子群$\{N_i:i\in I\}$,满足子集$\cup_{i\in I}N_i$生成了整个群$G$,并且对任意$k\in I$有$N_k\cap<\cup_{i\not=k}N_i>=<e>$,并且对任意$i\not=j$有$N_i$中的元和$N_j$中的元可交换,那么就称$G$是正规子群族$\{N_i:i\in I\}$的内直和.等价的,$G$是正规子群族$\{N_i:i\in I\}$的内直和,当且仅当,对$G$的每个非单位$a$,存在唯一的一组有限指标$\{i_1,\cdots,i_n\}$,使得$a$可以唯一表示为$a_{i_1}a_{i_2}\cdots a_{i_n}$,其中$e\not=a_{i_k}\in N_{i_k}$.
	
	内直和与外直和实际上仅仅是元素表示上具有差异,它们实际上是同构的.即一个外直和$\prod_{i\in I}G_i$实际上也是正规子群族$\{G_i:i\in I\}$的内直和.于是在同构意义下我们可以把外直和与内直和统称为直和,记作$\sum_{i\in I}G_i$.
	
	从范畴角度看,群范畴总存在积对象,它就是群族的直积;交换群范畴也总存在积对象,也是群族的直积.对于余积,交换群范畴的余积就是这族群的直和,但是描述群范畴的余积稍微复杂一些,它称为群的自由积,这个内容需要了解自由群之后理解:
	
	自由积.给定一族群$G_i$,它们的自由积$\ast_iG_i$定义为全体“简化字”,即有限长度的$g_1g_2\cdots g_m$,其中$m\ge0$.每个$g_i$属于某个群$G_j$.并且相邻的两个$g_i$不属于同一个群.两个字的乘积$(g_1\cdots g_m)(h_1\cdots h_n)$约定为字$g_1\cdots g_mh_1\cdots h_n$的简化字,即如果$g_m$和$h_1$不在同一个群中,就约定为$g_1\cdots g_mh_1\cdots h_n$,否则,就对$g_1\cdots (g_mh_1)\cdots h_n$继续做简化直到成为简化字.约定空字为幺元,验证群公理当中,唯一困难的是验证结合律,为此可以利用下文在自由群那里同样的处理来验证.
	\item
	商对象.首先给出群$G$上的一个等价关系$\sim$,记等价类的集合是$G/\sim$,它称为关于$\sim$的商集.记$g\in G$所在的等价类为$[g]$.我们期望商集本身具备一个群结构,并且从$G$到商群的典范映射$\pi:g\mapsto [g]$是群同态.这等价于说期望$[a]\times[b]=[ab]$,这又等价于在讲:
	$$\forall g\in G,a\sim a'\Rightarrow ag\sim a'g$$
	$$\forall g\in G,a\sim a'\Rightarrow ga\sim ga'$$
	
	分别考虑这两个条件,对于满足第一条的划分,幺元所在的等价类构成一个子群$H$,并且两个元$a,b$在同一个等价类当且仅当$a^{-1}b\in H$,于是全体等价类即$aH={ah,h\in H},\forall a\in G$,称这样的划分为关于子群$H$的左陪集划分,每个等价类是$H$的左陪集.同理对于满足第二条的划分即右陪集划分.容易看出对一个子群左右陪集的划分完全被这个子群刻画.同时满足两个条件等价于说对于子集$H$的左右陪集划分是吻合的,也即$gH=Hg,\forall g\in G$.因为如果存在$g_1H=Hg_2$,那么$g_1Hg_2^{-1}=H$,右侧有幺元,于是存在$h\in H$使得$g_1hg_2^{-1}=e$,导致$h=g_1^{-1}g_2$,于是$g_1H=g_2H=Hg_2$.这说明子群$H$还要满足一个特殊的条件,即$\forall g\in H$有$gHg^{-1}=H$,称这样的子群为\textbf{正规子群}.于是每个群$G$的商群和群$G$的正规子群一一对应.我们把等价关系就记作对应的正规子群$H$,用$G/H$表示一个商群,它的元素简记作$\{aH,a\in G\}$,并且作为群的二元运算为:$(aH) (bH)=abH$,其中$eH=H$是幺元,$aH$的逆是$a^{-1}H$.
	
	这里补充一些正规子群的性质.首先定义是,群$G$的一个子群$N$满足$\forall g\in G$有$gNg^{-1}=N$.但是事实上倘若我们证明了$\forall g\in G$有$gNg^{-1}\subset N$,把$g$换为$g^{-1}$就得到$g^{-1}Ng\subset N$,也就是$N\subset gNg^{-1}$,也即$N=gNg^{-1}$;按照定义,交换群的任何子群均为正规子群;另外,按照正规子群的交仍然是正规子群,就可以类似子集生成的子群来定义子集生成的正规子群,它就是全体包含这个子集的正规子群的交.最后,一个子群是正规子群当且仅当它是若干共轭类的并.
	
	陪集分解可以告诉我们更多的关于子群的信息.首先对于一个子群的全体左/右陪集的势是相同的.因为如果每个左/右陪集类取一个代表,得到的集合称为左/右陪集代表集,那么任意一个左陪集代表集全部取逆便得到一个右陪集代表集.这个势称为子群的指数,记作$[G:H]$.另外按照子群的陪集分解,可以得到一个势关系$|H|[G:H]=|G|$,这称为\textbf{Lagrange定理},即对于有限群,子群的阶数总是群的阶数的因数.注意到每个元素的阶数就是它生成的子群的阶数,由此得出有限群上元素的阶数整除群的阶数.特别的对于素数$p$考虑$(Z/p)^*$,这是一个$p-1$阶群,于是对于任意整数$a$有$a^p=a(\mod p)$,这称为\textbf{Fermat小定理}.另外子群的指数是乘性的,即若存在群包含$N\subset H\subset G$,那么:$[G:H]=[G:N][N:H]$.
	
	我们并不能从Langrange定理上奢求太多,并不是每个有限群阶数的因子都存在以它为阶数的子群.但是这对素数幂是成立的.它是Sylow理论的一部分.
	
	\item
	核与余核.注意群范畴和交换群范畴存在零对象.若$p:G\to H$是一个群同态,定义一个新范畴,它的对象是对$(A,f)$,其中$A$是一个群,$f$是从$A$到$G$的群同态,满足$p\circ f$是映射为幺元的恒等映射,从对象$(A,f)$ 到$(B,g)$的态射$h$是从$A$ 到$B$的群同态使得$h\circ f=g$,这个范畴的终对象称为态射核.这个定义吻合于之前对核的定义$f^{-1}(e_H)$.核一般记作$\ker f$.
	$$\xymatrix{A\ar[dr]_{\exists_!\overline{f}}\ar[r]_f\ar@/^1pc/[rr]^0&G\ar[r]_p&H\\
		&\ker p\ar[u]&}$$
	
	余核.对群同态$p:G\to H$,定义新范畴,对象为全体对$(X,f)$,其中$X$是群,$f$是从$X$到$G$的群同态使得$p\circ f$是恒等于幺元的群同态,从$(X,f)$到$(Y,g)$的态射是从$X$到$Y$的群同态使得$h\circ f=g$,$p$的余核定义为该范畴的终对象.在群范畴中,态射的余核是像空间模掉像生成的正规子群,在交换群范畴中子群总是正规的,所以此时生成的正规子群就是自身.余核一般记作$\mathrm{coker}f$.
	$$\xymatrix{X\ar[r]_f\ar@/^1pc/[rr]^0&G\ar[d]^{\pi}\ar[r]_p&H\\
		&\mathrm{coker}p\ar[ur]_{\exists_!\overline{f}}&}$$
	
	在群范畴和交换群范畴中,态射是(集合意义上的)单射等价于它是monic,等价于它的核平凡.
	\begin{proof}
		如果群同态$\phi$是monic,那么考虑从$\ker\phi$到$G$的两个态射,一个是包含映射$i$,一个是0映射$e$,它们和$\phi$的复合都是平凡映射,按照$\phi$是mono得到二者相同,于是得到$\ker\phi=\{e\}$.现在假设$\ker\phi=\{e\}$,那么如果存在$a,b$使得$\phi(a)=\phi(b)$,就有$\phi(ab^{-1})=e$,于是$ab^{-1}=e$,于是$a=b$,于是$\phi$是单射.最后,如果$\phi$是单射,那么它自然是monic,得证.
	\end{proof}
	
	这个对偶命题在交换群范畴中是成立的:一个交换群同态是epic等价于它是满映射等价于余核平凡.
	\begin{proof}
		
		首先如果群同态$\phi:G\to G'$是epic,那么考虑$G'\to\mathrm{coker}\phi$的两个态射,一个是到商对象的典范映射$\pi$,另一个是平凡映射,那么它们和$\phi$的复合都是平凡同态,于是导致二者相同,于是余核是平凡的.现在如果余核平凡,按照交换群范畴中子群总是正规子群,立刻得到$G'=\mathrm{im}\phi$,即同态是满的.最后如果同态是满的,直接得到它是epic.
	\end{proof}
	
	然而在群范畴中,情况是不同的.一个群同态是epic等价于它是满的,并且二者都可以推出余核为0.但是反过来,余核为0是推不出前者的.例如考虑3元对称群$S_3$中的子群$H=\{e,(12)\}$这是一个二阶子群,那么从$H$到$S_3$的包含映射$\phi$不是满射,但是$H$生成的正规子群却是整个$S_3$,导致余核平凡.现在,从满射推出epic和从满射推出余核为0都是直接的.困难的情况是从epi推出满射:
	\begin{proof}
		
		设$f:H\to G$为epic的群同态.考虑$G$的子群$L=f(H)$,满射等价于证明$L$的指数为1.如果$L$的指数为2,那么$L$是$G$的正规子群,于是有商群$G/L$.现在考虑两个映射$\alpha,\beta:G\to G/L$,其中$\alpha$为恒等于幺元的群同态,而$\beta$是商群的典范群同态,按照$\alpha\circ f=\beta\circ f$,得到$\alpha=\beta$,这导致$G=L$和$L$的指数为2矛盾.
		
		现在假设$L$的指数大于等于3.于是可以找到$G$中两个元$u,v$,满足$L,uL,vL$是两两不交的三个左陪集.现在构造$G$上的一个双射$\sigma$为,在$uL$上把$ul\mapsto vl$,在$vL$上把$vl\mapsto ul$,在$G\backslash (uL\cup vL)$上为恒等.记$G$上的对称群为$S(G)$,考虑$G$在自身上的右平移作用,即群同态$\alpha:G\to S(G)$为$g\mapsto\alpha_g$,其中$\alpha_g:G\to G$为$x\mapsto xg$.再构造$\beta:G\to S(G)$为群同态$g\mapsto\sigma^{-1}\circ\alpha_g\sigma$.那么有$(\beta\circ f)(x)(g)=\sigma^{-1}(\sigma(g)f(x))$.于是分$g\in uL,vL,G\backslash(uL\cup vL)$得到$(\beta\circ f)(x)(g)=gf(x)$.即$\beta\circ f=\alpha\circ f$.于是按照$f$是epic得到$\alpha=\beta$.但是只要取$x\in G\backslash(L\cup v^{-1}uL)$得到$\beta(x)(u)=vx$,$\alpha(x)(u)=ux$,$vx\not=ux$,矛盾.综上$L$的指数只能为1,也即$f$是满射.
	\end{proof}
	\item
	等化子和余等化子.在群范畴和交换群范畴中,等化子是一致的,即给定两个群同态$f,g:G\to H$,无论$G,H$是任意的群还是限制为交换群,那么$\mathrm{Eq}(f,g)$是$G$的子群$\{a\in G\mid f(a)=g(a)\}$.对于余等化子,在群范畴中,两个群同态$f,g:G\to H$的余等化子是$Y$的商群$Y/N$,其中$N$是$S=\{f(x)g^{-1}(x)\mid x\in X\}$在$Y$中生成的正规子群.在交换群范畴中,这里的$S$已经是一个正规子群,于是在加号记法下,余等化子定义为$Y/\mathrm{im}(f-g)$.
	
	至此我们得到群范畴和交换群范畴都是完备和余完备范畴,也即任意函子的极限和余极限都是存在的.
	
	\item \textbf{核$\Leftrightarrow$正规性},\textbf{商$\Leftrightarrow$满射像}.给定群同态$f:G\to H$,容易验证核是$G$的正规子群.反过来,给定群$G$的正规子群$N$,那么$N$是商群的典范同态$G\to G/N$的核.于是核等价于正规子群.另外,一个商$G/N$必然是典范同态$G\to G/N$的满射像.反过来,一个像必然可以表示为商,这遍是群的\textbf{第一同构定理}:给定群同态$f:G\to H$,我们断言$\mathrm{im}f\cong G/\ker f$.为此只需直接构造$\phi:G/\ker f\to\mathrm{im}f$为$g+\ker f\mapsto f(g)$.先验证定义的良性,即如果$g_1+\ker f=g_2+\ker f$,那么$f(g_1)=f(g_2)$,这只要注意到$g_1-g_2\in\ker f$.接下来验证$\phi$是群同态并且是双射即可.
	
	关于第一同构定理我们还可以讲的更多.给定群同态$f:G\to H$,它可以分解为如下三个群同态的复合,其中$\pi$是满同态,$\overline{f}$是同构,$l$是单同态.$$\xymatrix{
		G\ar@/^2pc/[rrr]^f \ar[r]^{\pi} & G/\ker f \ar[r]^{\overline{f}} &\mathrm{im}f \ar[r]^{l} & H}$$
	
	\textbf{第二同构定理}.给定群$G$,给定两个子群$S,N$,那么$SN=\{sn,s\in S,n\in N\}$是子群当且仅当有$SN=NS$.特别的,如果$N$是正规子群,那么总有$SN=NS$,于是此时$SN$是子群.另外这个情况下$N$是$SN$的正规子群,$S\cap N$是$S$的正规子群,第二同构定理是说此时有同构$\frac{SN}{N}\cong\frac{S}{S\cap N}$.
	$$\xymatrix{&&N\ar[dr]&&\\ {e}\ar[r]&S\cap N\ar[ur]\ar[dr]&&SN\ar[r]&G \\ &&S\ar[ur]&&}$$
	
	严格说第二同构定理不需要$N$必须是正规子群,只要$S$包含于$N$作为$G$的子群的正规化子中,就保证$N$是$SN$的正规子群,并且$S\cap N$是$S$的正规子群.第二同构定理定理也称为菱形定理或者平行四边形定理,名字源自于上面这个图.
	
	\textbf{第三同构定理}.给定群$G$,给定正规子群$N$,那么对任意满足$N\subset K\subset G$的子群$K$,有$N$是$K$的正规子群,此时$K/N$是$G/N$的子群.反过来对任意的$G/N$的子群必然可以表示为$K/N$的形式,其中$K$是满足$N\subset K\subset G$的子群.另外,如果上述$K$改成正规子群,结论全部成立.最后如果$N\subset K\subset G$,其中$N,K$都是$G$的正规子群,那么有同构$\frac{G/N}{K/N}\cong\frac{K}{N}$.
	
	第三同构定理蕴含了所谓的\textbf{子群对应定理}或者叫格定理.给定群$G$的正规子群$N$,那么存在从$G$的包含$N$的子群到$G/N$的子群的保序的一一对应,即$K\mapsto\frac{K}{N}$.另外这个对应限制成正规子群也是一个保序的一一对应.
	
	\item 自由对象.下面构造群范畴和交换群范畴上的自由对象.先来看群范畴的情况.
	
	对任意非空集合$X$,取一个与其无交的集合$X'$使得$|X|=|X'|$,任取一个双射,记$x\in X$对应于$x^{-1}\in X'$,再定义$X\cup X'$到自身的双射,将$x\in X$映射到$x^{-1}\in X'$,将$x^{-1}\in X'$映射到$x\in X$,无论$a\in X$或$X'$,记它对应的为$a^{-1}$,那么就有$(a^{-1})^{-1}=a$.现在取和$X\cup X'$ 无交的一个元素记作$1$.记$Y=X\cup X'\cup \{1\}$,称$X$上的一个字(word)是指$Y$上的一个序列:
	$$\left(a_1,a_2,\cdots\right),a_i\in Y,\exists k\in N^+,\forall i\ge k,a_i=1$$
	
	按照定义,我们可以无歧义的把字写作有限积:
	$$\left(a_1,a_2,\cdots\right)=a_1a_2\cdots a_k,a_{k+n}=1,\forall n\ge1$$
	
	注意这里写法是不唯一的,现在定义简化字是指在字的基础上,要求相邻的$a_i,a_ {i+1}$不会是$X\cup X'$中相对应的元素对,并且倘若$a_i=1$,那么$a_ {i+n}=1,\forall n\ge0$,这样对每一个字就存在一个最小的$a_{i+1}=1$,从而可以把字写作有限积$a_1\cdots a_i$.
	
	约定两个简化字$a_1a_2\cdots a_s$和$b_1b_2\cdots b_t$相等当且仅当$s=t,a_i=b_i\in Y$.
	
	将全体$X$上简化字$\cup \{1\}$的集合记作$F(X)$,有时在指名集合的前提下会简写作$F$,下面来构造集合上的二元运算.
	
	$1$作为幺元,它左右乘以任何简化字都是不变的.对任意简化字$a_1\cdots a_s$和$b_1\cdots b_t$,定义他们的积是$a_1\cdots a_sb_1\cdots b_t$, 但是这个字未必是简化的,倘若$a_s,b_1$是$X\cup X'$中相对应的元素,那么应该划去这两个元素,继续判断$a_{s-1}$和$b_2$,这样经过有限步后会得到一个简化字或者是$1$,并且这个结果是唯一的.
	
	现在任意简化字是有逆的,即$(a_1\cdots a_r)^{-1}=(a_r^{-1}\cdots a_1^{-1})$. 需要验证的群公理中还剩下直接验证会相当繁琐的结合律.这里我们巧妙的通过把$F$嵌入到一个群里,从群的结合律得到$F$上的结合律.
	
	对每个$a\in Y$,定义$F$到自身的映射$L_a$,它把每一个简化字$(a_1\cdots a_r)$映射为$(aa_1\cdots a_r)$,但是这个字未必是简化的,于是我们约定倘若$a$和$a_1$是相对应的,那么映射为$(a_2\cdots a_r)$,如果$r=1$则映射为$1$.
	
	现在容易验证$L_1$是$F$上恒等映射,并且$L_aL_{a'}=L_{aa'},a,a'\in Y$,这两条说明了每一个$L_a$是可逆的,并且逆映射就是$L_{a^{-1}}$.即每一个$L_a$就是$F$上的置换,下面把$L$延拓为$F$到$F$上对称群$S(F)$的映射:
	$$L:a_1a_2\cdots a_r\to L_{a_1}L_{a_2}\cdots L_{a_r}$$
	
	上一段的两条性质保证了$w_1,w_2\in F$,则$L_{w_1w_2}=L_{w_1}L_{w_2}$,现在如果$L_w=1_F$是$F$上的恒等映射,它把$1$映射为了$w$,于是$w=1$,这说明$L$是从$F$到$F$上对称群的单射.于是对任意$w_i\in F,i=1,2,3$有:
	$$\left(L_{w_1}L_{w_2}\right)L_{w_3}=L_{w_1}\left(L_{w_2}L_{w_3}\right)\Rightarrow
	L_{(w_1w_2)w_3}=L_{w_1(w_2w_3)}\Rightarrow (w_1w_2)w_3=w_1(w_2w_3)$$
	
	至此完成了自由群的构造,它就是群范畴中的自由对象.
	
	按照自由群的泛映射性质,对任意群,可以取它的一个生成元集,注意这总是能取到的,因为我们可以取它整个凭借集合.那么群是这个集合上自由群的满射像,也就是说,任意群同构于自由群的商,由此可以得出一种对群的刻画,称为\textbf{群的表现},它包含一个生成元集$A$和一个关系集$B$,记$F(A)$是集合上的自由对象,关系作为自由群上的words可以生成一个正规子群$R$,如果$G\sim\frac{F(A)}{R}$,就称$(A,B)$是群$G$的表现.注意到按照构造,所谓的关系集就是等于幺元的字串,这也就是关系的含义.特别的自由群的表现的关系集为空,这也正是称为"自由"的原因.
	
	和自由交换群情况不同的,自由群的子群自由这个事实并不平凡,它需要借助代数拓扑或者几何群论.另外自由群的子群的"秩"可以变得更大,例如二元集合上的自由群的交换子群是可数集合上的自由群.
	
	另外对于群上的自由对象$(F(X),i)$,这里的$i$必然是单射,否则如果$i(a)=i(b)$,可构造$j:X\to G$的集合映射满足$j(a)\not=j(b)$,于是此时泛映射性质的交换图总不能成立.
	
	下面构造交换群范畴上的自由对象.交换群$F$的基是指满足如下条件的子集$X$:
	\begin{enumerate}
		\item $F$由子集$X$生成.
		\item 若对不同的$x_1,x_2,\cdots,x_r\in X$,有
		$$n_1x_1+n_2x_2+\cdots+n_rx_r=0,n_i\in Z\Rightarrow n_1=n_2=\cdots=n_r=0$$
	\end{enumerate}
	
	注意交换群上是未必存在基的,但是倘若存在,就称这个交换群是自由交换群,那么交换群是自由交换群当它满足如下等价的任一条件:
	\begin{enumerate}
		\item 存在非空基.
		\item 同构于一族$\mathbb{Z}$的直和.
		\item 是交换群范畴上的自由对象.
		\item 是由生成元$X$和关系集$Y=\{aba^{-1}b^{-1}:a,b\in X\}$生成的群.
	\end{enumerate}
	
	自由交换群上的不同基具有相同的势.
	\begin{proof}
		
		如果自由交换群$F$存在一个基$X$是有限子集,那么$F$同构于$|X|$个$Z$的直和,于是$F/2F$同构于$|X|$个$Z/2$的直和,于是$F/2F$的阶数是$2^{|X|}$.这导致$F$上任意的基必然是有限集,并且元素个数恰好为$|X|$.现在假设$F$只存在无限集合的基,任取一个基$X$.按照$F$的每个元素可以表示为$X$中元素的$\mathbb{Z}$线性组合,就有$|F|=|X|$.这导致任意两个基的势相同,完成证明.
	\end{proof}
	
	同样按照自由对象的泛映射性质,任一交换群同构于某自由交换群满射象.即取交换群$G$的一个生成元集合$X$,记$X$生成的自由交换群为$F$,那么按照自由对象的泛映射性质,存在$F$到$G$的满同态.
	
	和群的情况一样,集合$X$到自由群$F(X)$的典范映射一定是单射,否则如果$i(a)=i(b)$,构造一个集合映射$j:X\to G$使得$j(a)\not=j(b)$就会使得泛映射定义的交换图不能成立.
	
	自由交换群子群的结构定理:$F$是秩为$n$的自由$abel$群,则对$F$任一子群$G$,存在$F$的一组基$\{x_1,x_2,\cdots,x_n\}$及$r\le n$个正整数$d_1\mid d_2\mid\cdots\mid d_r$,使得$G$是以$\{d_1x_1,d_2x_2,\cdots,d_rx_r\}$为基的自由$abel$群.
	\begin{proof}
		
		对$n$归纳.如果$n=1$,那么$F=<x_1>\sim Z$,于是按照循环群结构,立刻得到$G=<d_1x_1>$.现在我们假设命题对所有阶数小于$n$的自由交换群成立.不妨设$G$非0,我们记$S$是全体这样的整数$s$构成的集合:存在$F$的一组基$\{y_1,\cdots,y_n\}$使得存在$G$中一个元的表达式为$sy_1+k_2y_2+\cdots+k_ny_n$.按照$G$非0,我们看到$S$非空.于是$S$包含了一个最小的正整数$d_1$,对应于一组基$\{y_1,\cdots,y_n\}$,使得存在$v\in G$有$v=d_1y_1+k_2y_2+\cdots+k_ny_n$.现在我们做带余除法$k_i=d_1q_i+r_i$,于是得到$v=d_1(y_1+q_2y_2+\cdots+q_ny_n)+r_2y_2+\cdots+r_ny_n$.记$y_1+q_2y_2+\cdots+q_ny_n$,那么有$\{x_1,y_2,\cdots,y_n\}$是一组基.于是按照$d_1$的最小性得到$r_2=r_3=\cdots=r_n=0$.于是得到$d_1x_1=v\in G$.现在取$H=<y_2,\cdots,y_n>$.这是一个秩为$n-1$的自由交换群.并且有$F=<x_1>\oplus H$.另外有$G=<v>\oplus(G\cap H)=<d_1x_1>\oplus(G\cap H)$.这是因为,按照$\{x_1,y_2,\cdots,y_n\}$是$F$的一组基,导致$<v>\cap(G\cap H)=0$.现在如果$u=t_1x_1+t_2y_2+\cdots+t_ny_n\in G$.那么做带余除法$t_1=d_1q_1+r_1,0\le r_1<d_1$.那么得到$u-q_1v=r_1x_1+t_2y_2+\cdots+t_ny_n\in G$,按照$d_1$的极小性,得到$r_1=0$.于是$u-q_1v\in G\cap H$.于是我们看到$G=<v>+(G\cap H)$.于是这是直和.现在如果$G\cap H=0$,此时$G=<d_1x_1>$,那么命题成立.如果$G\cap H$非0.那么按照归纳假设,存在$H$的基$\{x_2,\cdots,x_n\}$,和一族$d_2\mid d_3\mid\cdots\mid d_n$使得$\{d_2x_2,\cdots,d_rx_r\}$.于是我们得到了$F$的一组基$\{x_1,\cdots,x_n\}$使得$\{d_1x_1,\cdots,d_rx_r\}$是$G$的一组基.最后我们需要证明$d_1\mid d_2$完成证明.做带余除法$d_2=qd_1+r$.那么得到$r_0x_2+d_1(x_1+qx_2)=d_1x_1+d_2x_2\in G$,注意到$\{rx_2,x_1+qx_2,x_3,\cdots,x_n\}$是一组基,按照$d_1$的极小性,得到$r=0$.于是$d_1\mid d_2$.
	\end{proof}
\end{enumerate}

$\prod_{n\ge1}\mathbb{Z}$不是自由交换群,注意这个群的元素可以理解为$\mathbb{Z}$上的序列.按照PID上自由模的子模自由,只要找一个它的子模不是自由模就得证.
\begin{proof}

设$f(n)$是$n$唯一分解中2的次数,设满足$\lim_{n\to\infty}f(a_n)=\infty$的序列$\{a_n\}$构成的子模为$N$.例如$a_n=2^n$在$N$中,$a_n=n!$在$N$中,终端为零的序列也在$N$中,但是$a_n=n$不在$N$中.下面只需说明$N$不是自由交换群.

取$e_i\in N$表示第$i$分量取1其余取零的序列.那么$\{e_i\}$不能线性生成整个$N$,因为它线性生成的必然是终端为零的序列.但是注意到$N/2N$只有有限个分量不取零,它作为$\mathbb{Z}/2$模以$\{e_i+2N\}$为基.现在假设$N$是自由交换群,取一组基$\{a_j,j\in J\}$,于是$N/2N$作为$\mathbb{Z}/2$模以$\{a_j+2N,j\in J\}$为基.于是$J$是可数基,$N$是可数秩的自由交换群,这说明$N$是一个可数集.现在我们构造$\prod_{n\ge1}\mathbb{Z}\to N$的单射,$\{a_n\}\mapsto\{b_n=2^na_n\}$,从而说明$N$是不可数集,推出矛盾.
\end{proof}
\newpage
\subsection{循环群}

这一节我们探究最简单的群——循环群.称能被单个元生成的群为\textbf{循环群}.那么我们已经看到了$\mathbb{Z}$和$\mathbb{Z}/n$都是循环群.事实上它们就是循环群在同构意义下的完全刻画,即无限阶循环群同构于$\mathbb{Z}$,有限阶循环群同构于$\mathbb{Z}/n$.
\begin{proof}
	
	对循环群$G$,取单个生成元$a$,那么存在从$\mathbb{Z}\to G$的群同态为$n\mapsto a^n$.于是这个群同态是满的,按照第一同态定理,存在一个$\mathbb{Z}$的正规子群$H$使得$\mathbb{Z}/H\cong G$.注意到$\mathbb{Z}$是交换群,于是它的子群全是正规子群.现在来描述$\mathbb{Z}$的全部子群.取$H$中出现的最小的正整数$n$,按照正整数集是良序的,只要$H$不是平凡子群,就能找到这样的$n$,我们断言$H$是由$n$的全部倍数构成的子群,事实上倘若存在一个不是$n$的倍数$m\in H$,那么考虑$m,n$的正的最大公约数$d$,必然有$d<n$,但是按照Bezout定理存在整数$p,q$使得$pn+qm=d$,于是$d\in H$,这和$n$的最小性矛盾!于是必然有$H$由$n$的倍数构成,记作$(n)$,于是$G\cong\mathbb{Z}/n$.
\end{proof}

另外,素数$p$阶群必然是$\mathbb{Z}/p$.事实上任取一个非幺元$a$,那么$a$的阶数整除群的阶数$p$,但是按照$p$是素数,这只能有$a$的阶数是$p$,于是群是循环群.

我们有如下\textbf{循环准则}:一个有限群是循环群当且仅当对每个群阶数的因子$d$,方程$x^d=e$在群中至多有$d$个解.由此可以得出域上有限阶乘法子群一定是循环群,特别的有限域的乘法群均为循环群.
\begin{proof}
	
	设$G$中全部元素的阶为$d_1,\cdots,d_t$.那么按照Lagrange定理,这些$d_i$都是$|G|$的因数.我们断言$G$中阶数为$d_i$的元素个数恰好有$\phi(d_i)$个,事实上如果$a,b\in G$都是阶数$d_i$的,那么按照条件必然有$b\in(a)$,也就是存在一个和$d_i$互素的$k$使得$b=a^k$,但是这样和$d_i$互素的$k$一共有$\phi(d_i)$个,于是阶数$d_i$的恰好有$\phi(d_i)$个元.现在按照初等数论的知识,有$\sum_{m\mid n}\phi(m)=|G|$.另外按照$d_i$遍历了全部元的阶数,于是$\sum_ {i=1}^{t}\phi(d_i)=|G|$.但是$\{d_1,\cdots,d_t\}\subset\{m>0:m\mid n\}$.于是这两个集合必然相同.导致存在某个$d_i=|G|$,于是$G$是循环群.
\end{proof}

自同构群.给定一个群$G$,那么群上的全体自同构按照复合这个二元运算构成了一个群,记作$\mathrm{Aut}(G)$.那么关于一个元的共轭是自同构,即对任意$g\in G$,有自同构$\tau_g:x\mapsto x^g$,这个自同构称为关于$g$的内自同构.把不是内自同构的自同构称为外自同构.全体内自同构构成自同构群的子群,记作$\mathrm{Inn}(G)$.注意自同构群一般不会是交换群,因为映射的复合通常不满足交换律.

$\mathrm{Inn}(G)$实际上是$\mathrm{Aut}(G)$的正规子群.任取$\mathrm{Aut}(G)$中的元$f$,那么$f\circ\tau_g\circ f^{-1}=\tau_{f(g)}$.另外存在$G\to\mathrm{Inn}(G)$的满同态$g\mapsto\tau_g$,这个同态的核就是$C(G)=\{g\in G\mid\forall h\in G,gh=hg\}$称为群$G$的中心,于是有同构$G/C(G)\cong\mathrm{Inn}(G)$.把商$G/\mathrm{Inn}(G)$称为$G$的外自同构群.

这里给出一些简单群的自同构群的计算.$\mathrm{Aut}(S_3)=S_3$.注意到$S_3$包含了3个对换.那么$S_3$的每个自同构在这三个对换上的作用必然是置换.于是这得到了从$\mathrm{Aut}(S_3)$到$S_3$的同态.按照这三个对换是$S_3$的生成元,于是核必然平凡,最后可以验证$\mathrm{Aut}(S_3)$中含有3阶元和2阶元,就得出它是满射.同理可以证明$A_4$的4阶子群$V$的自同构群也是$S_3$.于是不同构的群可以有相同的自同构群.另外还有$\mathrm{Aut}((\mathbb{Z}/p)^n)\cong\mathrm{GL}(n,p)$.因为$(\mathbb{Z}/p)^n$是$\mathbb{Z}_p$上的$n$维线性空间,导致自同构等价于线性同构.

我们接下来给出循环群的自同构群$\mathrm{Aut}(\mathbb{Z}/n)$的描述.首先,$\mathrm{Aut}(\mathbb{Z}/n)\cong U(\mathbb{Z}/n)$.这里$U(\mathbb{Z}/n)$表示的是$\mathbb{Z}/n$作为环的单位群.取$\mathbb{Z}/n$一个生成元$a$,任取$\mathbb{Z}/n$上一个自同构$f$,那么$f(a)=a^k$.并且$a^k$必须是$\mathbb{Z}/n$的一个生成元,这导致$(k,n)=1$.接下来只需验证$\mathrm{Aut}(G)\to U(\mathbb{Z}/n)$是一个同构.另外按照初等数论中的知识,$U(\mathbb{Z}/n)$中元素个数就是$n$在欧拉函数下的值$\phi(n)$,即$1$到$n$中全部和$n$互素的元素个数.

$\mathrm{Aut}(\mathbb{Z}/2)$和$\mathrm{Aut}(\mathbb{Z}/4)$分别是一阶群和二阶群,于是他们必然同构$1$和$\mathbb{Z}/2$.现在假设$m\ge3$,我们断言$\mathrm{Aut}(\mathbb{Z}/2^m)=U(\mathbb{Z}/2^m)=\langle[-1],[5]\rangle\cong \mathbb{Z}/2\times \mathbb{Z}/2^{m-2}$.
\begin{proof}
	
	首先按照欧拉函数的公式,有$|U(\mathbb{Z}/2^m)|=\phi(2^m)=2^{m-1}$.归纳可得$5^{2^{m-3}}\equiv1+2^{m-1}(\mod 2^m)$.这导致$[5]$具有阶数$2^s$,这里$s\ge m-2$.另外$[-1]$具有阶数2,我们断言$\langle[5]\rangle\cap\langle[-1]\rangle={0}$.若否,存在某个$t$使得$5^t\equiv -1(\mod 2^m)$,按照$m\ge3$,得到$5^t\equiv -1(\mod 4)$.但是这明显矛盾.于是这两个循环群的直积是$U(\mathbb{Z}/2^m)$的子群,接下来计算阶数,得到二者相同,于是这个群恰好就是$U(\mathbb{Z}/2^m)$.
\end{proof}

如果$p$是一个奇素数,那么$\mathrm{Aut}(\mathbb{Z}/p^n)\cong \mathbb{Z}/(p-1)p^{n-1}$是$\phi(p^n)$阶的循环群.
\begin{proof}
	
	$n=1$时候,按照$\mathbb{Z}/p$是域,并且它的乘法群是循环群,可以取一个乘法生成元$a$,这在初等数论里叫$p$的原根,即$U(\mathbb{Z}/p)$是$p-1$阶循环群.
	
	现在假设$n\ge2$,把$U(\mathbb{Z}/p^n)$记作$G$.取$B=\{[b]\in G\mid b\equiv1(\mod p)\}$,这是$G$的一个子群.现在按照每一个$b$可以唯一的表示维$b=a_0+a_1p+\cdots+a_{n-1}p^{n-1}$,其中$0\le a_i<p$.那么$B$中的元恰好就是这个唯一分解中$a_0=1$的元,于是$|B|=p^{n-1}$.按照有限生成交换群的结构定理,存在$G$的一个$p-1$阶子群$A$,满足$G=A\oplus B$.如果可以证明$A$和$B$都是循环群,按照他们的阶数互素,立马会得到$G$是循环群.
	
	先来构成$f:G\to U(\mathbb{Z}/p)$为把$\mod p^n$下的元$a$映射到$\mod p$下的元.这是一个满同态,并且核就是$B$.于是$G/B\cong U(\mathbb{Z}/p)\cong \mathbb{Z}/(p-1)$.但是$G/B\sim A$,得到$A$是循环群.
	
	最后证明$B$是循环群,为此来证明$1+p$是一个生成元.归纳证明对$m\ge0$有$(1+p)^ {p^m}\equiv1(\mod p^{m+1})$和$(1+p)^{p^m}\not\equiv1(\mod p^{m+2})$.这会得到$1+p$在$\mod p^n$下具有阶数$p^{n-1}$,于是得证.
\end{proof}

现在来分解$\mathrm{\mathbb{Z}/n}=U(\mathbb{Z}/n)$.首先注意到如果有环上的直和$R=R_1\times\cdots\times R_t$,那么有单位群的直和分解$U(R)=U(R_1)\times\cdots\times U(R_t)$.于是$\mathrm{\mathbb{Z}/n}\cong\prod_i\mathrm{Aut}(\mathbb{Z}/p_i^{e_i})$.这里唯一分解是$n=\prod_ip_i^{e_i}$.

在初等数论中,$\mod n$下的乘法群如果是循环群,就称乘法生成元为原根.按照上一段的分解,就可以得到$\mod n$下原根存在的全部情况:$n=1,2,p^k,2p^k$.
\newpage
\subsection{有限生成交换群的结构定理}

交换性这个条件如此之强,强到我们可以完全刻画有限生成交换群.

按照有限生成交换群必然是有限秩自由交换群的商,并且我们已经得到了有限秩自由交换群的子群的结构,于是可以立刻得到两个有限生成交换群的结构定理,其中第二条只要注意到,如果$m,n$互素,那么$\mathbb{Z} _{mn}\sim\mathbb{Z}_m\oplus\mathbb{Z}_n$:
\begin{enumerate}
	\item 任一有限生成交换群同构于有限个循环群直和,若有限循环群存在,则它们可以存在一个阶的排列使得:$m_1\mid m_2\mid\cdots\mid m_t,m_1>1$.满足条件的这组阶数的排列称为群的\textbf{不变因子组}.
	\item 任一有限生成交换群同构于有限个循环群直和,其中有限循环群可以都取为素数幂.满足条件的这组素数幂的排列称为\textbf{初等因子组}.
\end{enumerate}

这里我们给出结构定理存在性的第二个证明.
\begin{proof}
	
	我们主要依赖如下结论:有限交换群的最大阶元生成的循环群是它的一个直和项.首先说明这如何证明定理.它基于如下事实:一个交换群的最大阶元的阶数被每个有限阶元的阶数整除.对于有限交换群,我们自然可以按照上一段结论反复取出是循环群的直和项,并且因为有限生成,这将在有限次后终止,而这个事实保证了先取出来的直和项的阶数总是被后取的阶数整除,这便得到了结构定理1.现在我们来证明本段给出的命题,倘若不成立,记$|a|=p^ns$是一个最大阶数元,并且存在$|b|=p^mt$使得$p$不整除$m,n$而且$m>n$,那么有$|a^{p^n}|=\frac{p^ns}{(p^ns,p^n)}=s$, 并且$|b^t|=\frac{p^mt} {(p^mt,t)}=p^m$,这说明两个次幂的阶数是互素的,结合它们可交换,得出$a^ {p^n}b^t$的阶数是$p^ms$,这和约定的$a$是最大阶数元矛盾.
	
	现在我们着手证明上述关于直和项的结论.记$G$是有限交换群.第一步,我们证明$G$是它全部Sylow子群的直和.这使得我们只需处理单个交换p群.首先$G$是交换群告诉我们对于每个素数$p$,它的Sylow $p$子群正规,于是是唯一的(这是Sylow第二定理).我们记$G$ 的全部非平凡Sylow 子群是$G_{p_1},G_{p_2},\cdots,G_{p_r}$, 其中$p_i$ 是两两不同的素数.于是有$\left(\oplus_{i\not=j}G_{p_i}\right)\oplus G_{p_j}=0$,并且他们阶数的乘积是$G$的阶数,前者说明从$\oplus G_{p_i}$到$G$的自然同态是单射,后者说明同样是满射,于是同构.注意到,事实上我们证明了一个有限群的全部Sylow子群都是正规的,那么它同构于它全部Sylow子群的直和.
	
	在这里我们插入一个和证明无关的内容,同构于自身全部Sylow子群的直和是幂零群的一个等价刻画,第一步事实上相对复杂的说明了交换群都是幂零群,之所以说相对复杂因为我们有幂零群其他等价刻画更直接的给出交换群是幂零群的结果,事实上这便是"中心升链长度有限".
	
	第二步,倘若$a_i$是$G_{p_i}$中的阶数最大元,那么$\sum a_i$是$\oplus G_{p_i}$ 中阶数最大元,这只需注意到阶数互素的两个元可交换那么乘积的阶数是他们阶数的乘积.第三步,从第二步我们看出只需证明$G$是有限交换$p$群的顷刻间.对于$G$的一个最大阶数元$a$,那么$G$关于$\langle a\rangle$的每个陪集中存在唯一一个代表$d+\langle a\rangle$使得$\langle d\rangle\cap\langle a\rangle=\{0\}$.这是一个不太直观的结果.对任一陪集$g+\langle a\rangle$,它在商群中的阶数是一个$p$次幂$s$,那么$sg\in \langle a\rangle$,记$sg=np^{\alpha}a$,其中$(n,p)=1$,那么$\langle na\rangle=\langle a\rangle$,有$|g|/s=|a|/p^{\alpha}$,于是由$|g|$最大知$s|p^{\alpha}$,记$p^{\alpha}=sm$,那么$sg=smna$,于是取$d=g-mna$满足条件.最后,我们直接构造$\langle a\rangle\oplus G/\langle a\rangle$到$G$的同构,即$(ax,g+\langle a\rangle)\mapsto ax+d$,其中$d$是第三步中在$g+\langle a\rangle$中满足条件的唯一代表.
\end{proof}

事实上这两个结构定理的还有唯一性,即,给定有限生成交换群,则无限循环群的个数是固定的,并且初等因子组和不变因子组都是固定的.

先来说明无限循环群的部分是固定的.称交换群$G$上$a$是挠元,如果存在某个非0整数$n$使得$na=0$,于是全体挠元构成了子群,记作$tG$.如果$tG=0$,就称$G$是无挠交换群,如果$tG=G$,就称$G$是挠交换群.那么$tG$就是$G$在结构定理中的那些有限循环群的直和,而$G/tG$是那些无限循环群的直和.注意到同构的两个群$G\sim H$有$tG\cong tH$并且$G/tG\cong H/tH$.于是这得到一个有限生成交换群的上述分解中,无限循环群的个数是固定的,这个数就称为有限生成交换群的秩.

现在只要证明挠子群部分的唯一性.先说明初等因子组的唯一性可以推出不变因子组的唯一性.如果给定不变因子组,把每个不变因子分解为素数幂,就得到了初等因子组.对于初等因子组,把相同一个素数出现的次幂按照降列排列,依次对应相乘不同素数底的第$i$大的素数幂,就得到了第$i$个不变因子.这就导致初等因子组对应到不变因子组是一一对应的.

现在来证明初等因子组的唯一性.这依赖于一些特殊的群上算子.我们需要证明如果$tG$可以写作若干素数幂阶循环群的直和.那么同一个素数幂阶循环群的个数是固定的.为此先注意到,对素数$p$设$G(p)$是$G$中元素阶数是素数幂的元构成的子群,那么有:$(\oplus_{1\le i\le r}G_i)(p)\sim\oplus_{1\le i\le r}G_i(p)$.并且注意到如果素数$q\not=p$,那么$\mathbb{Z}_{q^r}(p)=0$.于是如果$tG$写成两种素数幂阶循环群的直和,作用这个算子,得到两组同一个素数$p$为底的素数幂阶循环群的直和是同构的.现在注意到当$m<n$时$p^m\mathbb{Z}_{p^n}\sim \mathbb{Z}_{p^{n-m}}$.反复运用这个算子,就能得到每个素数幂阶群的个数是相同的.
\newpage
\subsection{Krull-Schmidt定理}

我们已经给出有限生成交换群的结构定理.现在一个自然的问题是,什么时候可以将一个未必交换的群分解为若干真子群的直和,并且这个直和的每一项不可再分解为更小的子群的直和?另外如果这种分解存在,那么是否唯一?Krull-Schmidt定理断言,在很大一类群(包含了全部有限群!)上这是成立的.这个定理的名字和发现者实际上是不符的.这个命题最早在1909年被J.M.H Wedderburn提出,但是他本人的证明是错误的.第一个针对有限群的正确的证明是于1911年被R.Remak给出.随后1912年W.Krull简化了证明.Krull于1925年把结论推广到模上.

称一个群$G$是\textbf{不可分解群},如果$G$非平凡,并且如果$G=H\times K$就有$H=1$或$K=1$.允许群成为若干不可分解子群的直和的条件是\textbf{链条件}.给定群$G$,称它满足acc条件,即\textbf{升链条件},如果每个$G$的正规子群的升链必然终止,也就是说如果有正规子群链$K_1\le K_2\le\cdots$,那么存在某个$t$开始有$K_t=K_{t+1}=\cdots$.对偶的有群的dcc条件,即\textbf{降链条件}.

整数作为加性群,满足升链条件但是不满足降链条件.固定素数$p$,考虑全部既约形式的分母是$p$幂的全体有理数,它们和0构成一个加法群$\mathbb{Z}(p^{\infty})$,这个群满足降链条件但是不满足升链条件.有理数在加法下的群既不满足升链条件也不满足降链条件.最后,有限群必然总是同时满足两个条件.

设$H$是$G$的正规子群,如果$G$同时满足两个链条件,那么$H$和$G/H$必然满足两个链条件.反过来也是成立的:群$G$的一个正规子群$H$,如果$H$和$G/H$都满足两个链条件,那么$G$满足两个链条件.特别的,我们看到$K,H$满足两个链条件,当且仅当$K\times H$满足两个链条件.
\begin{proof}
	
	引理:Dedekind法则,如果$H,K,L$是$G$的子群,其中$H$是$L$的正规子群,那么有$HK\cap L=H(K\cap L)$.注意这个引理中没有要求$HK$和$H(K\cap L)$是子群,仅仅是子集.
	
	取$G$的一个正规子群降链$G_1\ge G_2\ge\cdots$,那么$H\cap G_1\ge H\cap G_2\ge\cdots$是$H$上一个正规子群的降链,并且$HG_1/H\ge HG_2/H\ge\cdots$是$G/H$上的一个正规降链,按照条件有某个$l$使得$H\cap G_l=H\cap G_{l+1}=\cdots$和$HG_l=HG_{l+1}=\cdots$.那么对$i\ge l$,按照引理有$G_i=G_iH\cap G_i=G_{i+1}H\cap G_i=G_{i+1}(H\cap G_i)=G_{i+1}(H\cap G_{i+1})\le G_{i+1}$.这说明当$i\ge l$时$G_i$稳定.类似可以证明升链情况.
\end{proof}

群$G$满足任一链条件,那么$G$是有限个不可分解群的直积.即,如果一个群$G$不能分解为有限个不可分解群的直积,那么它两个链条件都不能满足.
\begin{proof}
	
	如果$G$不能分解为有限个不可分解群的直积,那么$G$本身不能是不可分解群,于是存在$G=U\times V$,其中$U,V$非平凡,并且,$U$和$V$中会有一个是不能分解为有限个不可分解群直积的,称这个直和项为$G$的不可分解分支.由此可以构造严格降链,即取$G=H_0$,如果已经构造了$H_n$,接下来取$H_{n+1}$是$H_n$的不可分解分支.于是得到了一列严格递减的无穷正规子群链$G=H_0\ge H_1\ge\cdots$.现在记$H_{i-1}=H_i\times K_i$,那么每个$K_i$都是$G$的直和项,由此构造$K_1\le K_1\times K_2\le\cdots$,这是严格递增的无穷正规子群链.于是$G$不满足任一链条件.
\end{proof}

\textbf{krull-Schmidt定理}.如果$G$满足两个链条件,那么对任一两种分解为不可分解群的直和$G=H_1\times\cdots\times H_s=K_1\times\cdots\times K_t$,有$s=t$,并且存在一种重排记号使得$H_i\sim K_i$.事实上,甚至存在一种重排记号使得对任意的$1\le r\le s$,有$G=H_1\times\cdots\times H_r\times K_{r+1}\times\cdots\times K_s$.【】
\newpage
\section{Sylow理论}
\subsection{群的作用}

群$G$在范畴中对象$A$的表示(作用)是指从$G$到$\mathrm{Aut}(A)$的群同态.表示的核是$G$的正规子群,其中的元素恰好就是在$A$上是恒等映射的元素,如果核是平凡的,即只有$G$中的幺元才能在$A$上是恒等作用,就称这个表示是忠实的.

从一种更高的观点看,我们关注群的主要原因是它可以作用在对象上,这是获取代数或几何对象的性质的一种重要手段,另外,将群作用在某些对象上也是获得群本身的信息的最佳办法.

对集合范畴的情况.自同构群即集合上的对称群,那么群作用在集合上等价于在要求存在一个映射$\alpha:G\times A\to A$;$(g,a)=ga$,使得$\forall a\in A,g_1,g_2\in G$,有$ea=a,(g_1g_2)a=g_1(g_2a)$.通常称集合$A$中的元素为点.当约定了集合$S$上的一个群$G$的作用时,就称$S$是一个$G$-set,也即等价于约定了一个映射$G\to S(A)$.

通常把上述定义称为左作用,左右作用的区别仅仅在于作用的顺序不同,左作用约定$g_1g_2,\cdots,g_n$作用到点$x$上是从右至左,即先作用$g_n$,然后把$g_{n-1}$作用到结果,再作用$g_{n-2}$,一直到$g_1$.而右作用约定先作用$g_1$,一直到$g_n$.对于每个群可以定义它的反群,即作为集合是不变的,但是两个元素的新的乘法$a\circ b$定义为$ba$,于是右作用一个群相当于左作用它的反群.

关于作用的记号.无论左作用还是右作用均可以用左侧记号$gx$和右侧记号$xg$,左作用运用记号$gx$时作用顺序和记号顺序是符合的,即$g_1g_2x=g_1(g_2x)$,但是用右侧记号就稍微有点别扭:$g_2g_1x=g_1(g_2x)$.于是习惯上对左作用采取左侧记号.同理右作用习惯上采取右侧记号,若用左侧记号就会稍微别扭:$xg_1g_2=(xg_2)g_1$.

那么群作用有什么用处呢?首先,他提供了一些"好"的核,即如果我们把一个群作用在某个特定的集合$A$上,那么相当于得到了这个群到$\mathrm{Aut}(A)$的同态,于是就得到了一个核.得到好的核的最基本意义是我们可以计数,即得到数量关系.

如果把$G$作用在了某个集合$S$上,那么对$s\in S$,记$G_s=\{g\in G\mid gs=s\}$为点$s$的稳定子,它是群$G$的子群.稳定子有两个有用事实,如果存在$g\in G$使得$gs=t$,$s,t\in S$,那么有$G_s^g=gG_sg^{-1}=G_t$;另外,表示的核就是全部稳定子的交.

下面给出一些具体的群作用.
\begin{enumerate}
	\item 每个群可以作用于自身的凭借集合上,例如$\alpha(g,h)=gh$.这个群作用称为群自身上的左正则作用,或者左平移作用.每个点$h\in G$的稳定子就是{e}.于是这个群作用总是忠实的,也可以从$gh=h,\forall h\in G$当且仅当$g=e$直接得到.于是按照同构定理得到$G$可以嵌入到$S(G)$中,这就是\textbf{Cayley定理}:任何群都同构于对称群的子群.
	\item 将群$G$通过$\alpha(g,h)=h^g=ghg^{-1}$作用于$G$的凭借集合上,这称为子群$H$在$G$上的左共轭作用.点$h\in G$的稳定子是$C_G(h)=\{g\in G\mid gh=hg\}$,它称为$h$的中心化子.这个作用的核是$Z(G)=\cap_{h\in G}C_G(h)=\{g\in G\mid gh=hg\forall h\in G\}$,也就是$G$的中心.它占据群的大小度量了一个群的交换性,当中心占据整个群的时候这个群就是交换群.另外我们有如下等价描述:$G/Z(G)$是循环群;$G/Z(G)$是平凡群;$G$是交换群.
	\item 群$G$还可以作用于$G$的某些子集族上.例如考虑群$G$的全部子集构成的集合$S$.把$G$正则的作用在$S$上,即$gX=\{gx\mid x\in X\}\in S$.
	
	一个有用的例子是取子集族为一个子群$H$的全体左陪集构成的集合$S$,那么可以将$G$左平移作用在$S$上,即$g(xH)=gxH$.点$xH$的稳定子就是满足$gxH=xH$的$g$构成的子群.这也就子群$H^x=xHx^{-1}$.于是这个表示的核就是$\cap_{x\in G}H^x$,它是$H$的正规子群,称为$H$在$G$中的核,记作$\mathrm{core}_G(H)$.事实上它就是包含在$H$中的最大的(这里最大是指它包含了全部其他这样的正规子群)正规子群,因为如果$N$是一个包含在$H$中的正规子群,那么$N=N^x\subset H^x,\forall x\in G$.
	
	如果上述例子中子群$H$的指数是$n$,那么$G/\mathrm{core}_G(H)$同构于$S_n$的子群.这说明,如果$H$的指数是有限数$n$,那么存在$G$的正规子群$N$满足$[G:N]$整除$n!$,事实上这里$N$可以取包含在$H$中的最大正规子群.特别的,\textbf{如果$G$是有限单群,并且存在一个指数为$n>1$的真子群,那么$G$的阶数整除$n!$}.
	
	按照最后这个结果我们可以得到一个经典结论:如果$H$是有限群$G$的指数$p$的子群,并且$p$是$|G|$的最小素因子,那么$H$是$G$的正规子群.
	\item 也可以把$G$共轭作用在$G$的全部子集$S$上,即对$X\in S$,记$gX=\{x^g=gxg^{-1}\mid x\in X\}$.按照子群的共轭总是子群,也可以把$S$取为$G$的全体子群构成的集合$S'$.子集$S$的稳定子是$N_G(S)=\{g\in G\mid gSg^{-1}=S\}$,称为子集$S$的正规化子.当$S$是子群的时候,正规化子正规化子占据群的大小度量了它和正规子群的差距,如果子群的正规化子是整个群,那么$X$是正规子群.
	\item 注意到$H$是$N_G(H)$的正规子群,即$g\in N_G(H)$就有$gNg^{-1}=N$,这导致$N_G(H)$可以左共轭作用于$H$上.这个作用的核是$C_G(H)=\{g\in G\mid gh=hg,\forall h\in H\}$,称为子群$H$的中心化子.于是$C_G(H)\triangleleft N_G(H)$,这个正规子群结论称为\textbf{N/C定理}.特别的,如果取$H=G$,就得到$Z(G)$是$G$的正规子群,事实上我们证明过$G/Z(G)\cong\mathrm{Inn}(G)$.最后注意一点,子群$H$总包含于正规化子中,但是它未必包含于自身的中心化子中,而$H\subset C_G(H)$当且仅当$H$自身是一个交换群.
\end{enumerate}

这一段我们证明,如果$G$的阶数是$2n$,其中$n$是奇数,那么$G$必有指数为2的正规子群.考虑$G$的左正则表示,即同态$\phi:G\to S_{2n}$,这是单射.而$G$中必有2阶元$g$,因为阶数大于2的元和自身的逆元成对出现,导致一个偶数阶群的阶数等于2的元的个数是奇数.现在对任意$a\in G$,有$\phi(g)(a)\not=a,\phi^2(g)(a)=a$,这导致置换$\phi(g)$由若干个不交的对换$(a,\phi(g)a)$之积构成,但是$G$有$2n$个元,导致$\phi(g)$是$n$个对换之积,于是$\phi(g)$是奇置换,于是$\phi(G)$包含了奇置换,于是考虑$\phi(G)\to Z_2$的典范映射,把奇置换映射为1,把偶置换映射为0,这得到了全体$\phi(G)$中的偶置换构成了一个指数为2的正规子群.

为了通过群作用来计数,需要定义作用的轨道.如果$G$作用在集合$A$上,称一个点$a\in A$的轨道为$O_a=\{ga\mid g\in G\}$.那么轨道是两两不交的,并且同一个轨道中的两个点$s,t$,必然存在一个$g\in G$使得$gs=t$.并且按照每个点必然属于某个轨道,这说明轨道构造了集合$A$上的一个划分,称为群作用的轨道分解.于是$|A|$就是全部轨道长度和.当作用只有一个轨道的时候,称这个群在集合上的作用为可迁的,也即对于任意集合中的元素$a,b$,存在群中的一个元将$a$映射为$b$.于是如果$A$是$G$-set,每个轨道都是一个可迁$G$-set,这说明只需探究可迁$G$-set的结构,便可得到一般$G$-set的结构.

例如,如果子群$H$左正则作用在群$G$上,那么轨道分解就是关于子群$H$的右陪集分解.如果$G$共轭作用于自身上,那么轨道分解就是元素的共轭类分解.

关于轨道个数,有如下计数公式:设$A$是一个有限$G$-set,如果记$A$中被$g\in G$固定的点的个数有$F(g)$个,那么轨道个数就是$\frac{1}{|G|}\sum_{g\in G}F(g)$.事实上在和$\sum_{g\in G}F(g)$中点$x\in A$被计算了$[G:G_x]|G_x|=|G|$次.

在给出计数关系之前,先定义$G$-set范畴.对任意群$G$,定义畴$G$-set范畴的对象是对$(\rho,A)$,其中$A$是一个集合,$\rho$是$G\times A$到$A$定义作用的映射,定义从对象$(\rho,A)$ 到$(\varrho,B)$的态射是从$A$到$B$的集合映射$f$,满足$f(ga)=gf(a),\forall g\in G,a\in A$,或者等价说满足如下图表交换.如果$f$甚至是一个集合双射,就称这两个$G$-set是同构的.
$$\xymatrix{
	G\times A\ar[rr]^{1_G\times f}\ar[d]^{\rho}&&G\times B\ar[d]^{\varrho}\\
	A\ar[rr]^{f}&&B
}$$

轨道和计数之间的联系在于\textbf{基本计数原理}:
\begin{enumerate}
	\item 若$G$作用于集合$S$上,任取一个轨道$O_s$,记$s$的稳定子为$G_s$,那么轨道长度为$|O_s|=[G:G_s]$.
	\item 由此可以得出轨道长度总是整除群的阶数的.
	\item 另外,从轨道分解就得到计数公式$|S|=|S_0|+\sum_{a}[G:G_a]$.其中集合$S_0$是全体自身构成单一轨道的元构成的集合,也等价于说$S_0=\{s\in S\mid gs=s,\forall g\in G\}$,它称为作用的固定子集.
\end{enumerate}

本质上讲,这个基本计数原理就是在说一个$G$-set同构:$G$的每个可迁的作用作为$G$-sets的对象同构于左平移作用在某个子群的左陪集上,这里的子群可以取集合上任意一个点的稳定子.设可迁作用的集合为$A$,任取一个点$a_0$的稳定子记作$H$,那么存在$A$到$G/H$(这个记号这里表示$H$的全体左陪集)的双射$\phi$为$a\mapsto gH$,其中$g$满足$ga_0=a$.先验证这个定义的良性,$g_1a_0=g_2a_0=a$,等价于$g_2^{-1}g_1a_0=a_0$,也等价于$g_2^{-1}g_1\in H$,即$g_1H=g_2H$.上述等价性也得到这个映射是单射,而满射因为可迁条件.最后验证$g'\phi(a)=g'gH$,其中$ga_0=a$,于是$g'ga_0=g'a$,于是$g'gH=\phi(g'a)$,就说明这是$G$-set同构.

按照基本计数原理,立刻得到$x\in G$的共轭类中元素个数为$[G:C_G(x)]$;子群$H\subset G$的不同共轭子群个数为$[G:N_G(H)]$.

一个比较有价值的例子是群自身上左共轭作用的轨道公式,它称为类方程,即$|G|=|C(G)|+\sum[G:C_G(a)]$.类方程可以提供给我们群的一些粗略的信息.例如一个六阶群,它的类方程首先可以是6=6,这意味着所有元都是中心元,即群是交换群.现在假定群不是交换群,那么中心阶数不是6 并且是6的因子,结合$G/Z(G)$循环当且仅当它平凡,以及素数阶群都是循环群,说明6除以这个数字不能是素数.于是中心阶数只能是1,下面除去中心阶数以外的数字必然是6的因子并且大于1,这只有一种选择,即6=1+2+3.事实上,这对应着六阶群$S_3$.另外,正规子群必然是共轭类的并,这可以给出一种粗略验证群不含有某阶正规子群的做法.

$p$是素数,$S_p$的每个在$\{1,2,\cdots,p\}$上可迁的子群的阶数必然被$p$整除.因为可迁作用下,集合的元素个数整除群的阶数.在计算多项式的Galois群那里我们会遇到这个命题.反过来,如果$p$是素数,$G$是$S_p$的子群,并且诱导的在$\{1,2,\cdots,p\}$上的作用是可迁的,那么对$G$的任意非平凡正规子群$N$,它诱导的在$\{1,2,\cdots,p\}$上的作用是自动可迁的!
\begin{proof}
	
	首先我们证明,如果$G$在$X$上的作用是可迁的,任取$G$的正规子群$N$,那么诱导的$N$在$X$上的作用,所有轨道的长度是相同的.任取$x,y\in X$,按照可迁条件,可以找到$g\in G$满足$y=gx$.于是得到轨道$Nx$和$Ny=Ngx=gNx$之间的双射为$t\mapsto gt$.
	
	现在回到原题.$N$作用在$\{1,2,\cdots,p\}$,那么全部轨道的长度相同,记作$m$.由于要求了$N$非平凡,并且是$S_p$的子群,所以$m>1$.于是计数得到$p=mn$,这里$n$是轨道个数,那么这导致$m=p,n=1$,于是群作用是可迁的.
	
\end{proof}

给定群$G$的两个G-Sets对象$X,Y$,如果存在G-Sets态射$f:X\to Y$,并且$Y$上的群作用是可迁的,那么有$|Y|\mid |X|$.证明,首先我们把$X$分解为$\cup_{y\in Y}f^{-1}(y)$.注意到这是一个无交并.我们断言对任意$y,y'\in Y$有$|f^{-1}(y)|=|f^{-1}(y')|$,这就说明了要证的命题.为此,按照可迁条件,存在$g_0\in G$满足$y'=g_0y$.由此可以得到映射$f^{-1}(y)\to f^{-1}(y')$为$x\mapsto g_0x$和$f^{-1}(y')\to f^{-1}(y)$的映射$x\mapsto g_0^{-1}x$,它们互为逆映射,完成证明.

Frattini定理.如果$G$作用在集合$X$上,$H$是$G$的子群,那么如下两个条件等价:$H$在$X$上诱导的作用是可迁的;$G$在$X$上的作用是可迁的,并且$G=HG_x$,这里$G_x$是$x$的稳定子群.当条件成立时,$G=HG_x$对任意的$x\in X$成立.
\begin{proof}
	
	如果$H$在$X$上的作用是可迁的,那么$G$自然也在$X$上可迁.固定任意的$x\in X$,对每个$g\in G$,存在$h\in H$满足$gx=hx$,于是有$h^{-1}g\in G_x$.于是$g\in HG_x$,导致$G=HG_x$.反过来,如果$G$在$X$上可迁并且$G=HG_x$,那么有$X=Gx=HG_xx=Hx$,于是$H$在$X$上可迁.
	
\end{proof}
\newpage
\subsection{Sylow定理}

本节介绍Sylow定理.Sylow定理是理解有限群的最基本工具.Sylow子群存在性早在1872年就给出了,我们这里将要给出的H.Wielandt的证明在1959年才发表,并且证明所运用技巧看起来不会有其他的方面应用,而且是非构造性的证明,但是这个证明更加精短和优美.

$p$群.对素数$p$,称有限群是$p$群如果它的阶数是个素数幂.利用下面的Cauchy定理,即有限群阶数的每个素因子,都存在一个该阶数的元,可以说明一个有限群是$p$群当且仅当它的每个元的阶都是同一个素数$p$的次幂.这也用来定义无限阶的$p$群.

关于$p$群的几个性质,它们将会和幂零群产生联系.
\begin{enumerate}
	\item $p$群的中心非平凡.这个可以直接考虑类方程$|G|=|C(G)|+\sum[G:C_G(x_i)]$,其中后面的和式每一项都被$p$整除,导致$p\mid|C(G)|$,而中心$C(G)$至少包含幺元,于是得到$C(G)$至少还有$p-1$个元.
	\item 事实上一个更强的命题是成立的:给定有限$p$群$P$,任取一个非平凡正规子群$N$,那么$N\cap C(P)$是非平凡的.
	\begin{proof}
		
		$N$是正规子群,于是可以让$P$共轭作用在$N$上.于是$N\cap C(P)$恰好就说那些所在轨道只有自身的点.按照基本计数原理,每个长度非1的轨道的长度是$p$的次幂,于是全部轨道长度1的那些点的个数被$p$整除.但是既然幺元是轨道长度1的,于是这些点的集合是非空的,于是$N\cap Z(P)$是非平凡的.
	\end{proof}
	\item 如果$H$是有限$p$群$G$的真子群,那么$N_G(H)$严格包含$H$.
	\begin{proof}
		
		记$S$为$H$在$G$中的全部左陪集构成的集合,那么$|S|=[G:H]$.将$H$左平移作用于$S$上,那么由于$G$是$p$群,从轨道分解就得到$|S_0|\equiv|S|=\equiv0(\mod p)$,其中$S_0=\{gH\in S\mid\forall h\in H,hgH=gH\}$是固定子集,那么$gH\in S_0$当且仅当$g\in N_G(H)$.于是$|S_0|=[N_G(H):H]$.于是$|S_0|$是$p$的倍数,导致$[N_G(H):H]>1$,实际上它甚至$\ge p$,这就说明$N_G(H)$严格包含$H$.
		
	\end{proof}
\end{enumerate}

\textbf{Cauchy定理}:对于有限群阶数的素因子$p$,群中总存在$p$阶元.
\begin{proof}
	
	考虑集合$S=\{(a_1,a_2,\cdots,a_p)\mid a_i\in G,a_1a_2\cdots a_p=e\}$,那么$|S|=n^{p-1}$.现在构造$\mathbb{Z}/p$在$S$上的作用,即$k(a_1,a_2,\cdots,a_p)=(a_{k+1},a_{k+2},\cdots,a_{k+p})$.从$p\mid n$得到$|S|\equiv0(\mod p)$,考虑作用的类方程,得到固定点集$S_0=\{s\in S\mid gs=s\forall g\in G\}$和$S$的元素个数在$\mod p$下相同,而$S_0$中的点是形如$(a,a,\cdots,a)$的元,也即$a$满足$a^p=e$,注意到$e\in S_0$,于是$|S_0|$是$p$的倍数还不是0,于是必然存在非幺元的$a$满足$a^p=e$,完成证明.
\end{proof}

给定一个有限群$G$,给定一个素数$p$,称子群$H$是Sylow-p子群,如果$H$的次数是一个$p$次幂,并且他的指数和$p$互素,即如果把$|G|$写作$p^nm,p\not\mid m$,那么Sylow-p子群就是阶数为$p^n$的子群.拉格朗日定理告诉我们子群的阶一定整除群的阶,于是可以说Sylow-p子群是这样的子群,它是被拉格朗日定理所限制的最大的$p$群.有两种平凡的现象,一种是当$|G|$中不含素因子$p$,那么这时候Sylow-p子群就是平凡群$<e>$,另一种是如果$|G|$自身就是一个$p$群,那么Sylow-p子群就是它自身.这两种情况下Sylow-p子群是必然存在的.而\textbf{Sylow存在定理(Sylow-E定理)}告诉我们对任意有限群$G$,对任意素因子$p$,Sylow-p子群总是存在的.它是某种程度上的拉格朗日定理的逆命题.
\begin{proof}
	
	设$|G|=p^nm$,其中$p\not\mid m$.设$S$是全部$G$的$p^n$元子集的集合,那么$|S|=\left(\begin{array}{c}
	p^nm\\
	p^n\end{array}\right)$.在$F_p$里$(1+x)^{p^nm}=(1+x^{p^n})^m$,考虑两边$x^{p^n}$的系数,得到这个数在$\mod p$下这个数是$m\not=0$.
	
	把$G$右乘作用在$S$上,那么至少存在一个轨道的元素个数不被$p$整除,取这个轨道中的一个元素$X$,考虑它的不变子群$G_X$,基本计数原理说明轨道中元素个数为$|G|/|G_X|$,但是因为$p$不整除这个商,所以必然有$p^n$整除$G_X$,于是$p^n\le|G_X|$.又因为$G_X$中的元素在右乘下固定$X$,于是$\forall x\in X$有$xG_x\subset X$,于是$|G_X|=|xG_X|\le |X|=p^n$.于是$p^n=|G_X|$,得证.
\end{proof}

这里补充一下特征子群的概念.给定群$G$的一个子群$H$,如果对于$G$的任意一个自同构$f$都有$f(H)\subset H$,那么就称$H$是特征子群.注意到对子群$H$,我们只要验证对任意$G$的自同构$\phi$有$\phi(H)\subset H$,那么有$\phi^ {-1}(H)\subset H$,于是$\phi(H)=H$,就得到$H$是特征子群.正规子群也就是对每个$G$的内自同构(也就是关于某个元共轭)$f$都有$f(H)\subset H$.于是特征子群自然是正规子群.另外,如果$K\subset N\subset G$,并且$N$是$G$的正规子群,$K$是$N$的特征子群,那么必然有$K$是$G$的正规子群.这里我们指出,寻找一般群的全部自同构是十分困难的,所以在具体例子里很难直接从全部自同构这一角度证明一个子群是特征子群.

尽管如此,我们容易看到一些特别的子群都是特征子群,例如中心,导群,还有对于每一个固定的素数$p$,全体Sylow-p子群的交$O_p(G)$.

现在继续给出Sylow子群的第二个性质.首先证明,如果给任意一个有限群$G$的$p$子群$P$,那么对任意$S\in\mathrm{syl}_p(G)$(这个记号表示$G$的全体sylow-$p$子群构成的集合),存在一个元$g\in G$使得$P\subset S^g=g^{-1}Sg$.
\begin{proof}
	
	记$S$的全体右陪集构成的集合为$\Omega$,那么$|\Omega|=[G:S]$,它不被$p$整除.现在把$P$右乘作用在$\Omega$上,按照轨道分解说明必然存在一个轨道$O$的长度不被$p$整除.我们知道$|O|$是$P$内某个子群的指数,但是这说明它的阶数是$p$的次幂,于是$|O|=1$,不妨记陪集$Sg$以自身为一个轨道,于是对每个$u\in P$有$Sgu=Sg$,也就是说$u\in g^{-1}Sg,\forall u\in P$,于是$P\subset g^{-1}Sg$.
\end{proof}

据此我们可以得到\textbf{Sylow共轭定理(Sylow-C定理)}:任意两个Sylow-p子群$S,T$是共轭的.另外每个$p$子群都包含于某个Sylow-p子群中.共轭定理还告诉我们,一个有限群$G$恰好存在唯一一个Sylow-$p$子群等价于说存在一个正规的Sylow-$p$子群.另外按照自同构保阶数,在恰好存在唯一一个Sylow-$p$子群的时候它也是特征子群,即Sylow-p子群正规等价于特征等价于唯一.

现在如果给定两个不同的Sylow-p子群$P,P'$,为了寻找$g$使得$P'=P^g$,按照上述证明,就要把$P'$右乘作用在$P$的全体右陪集上,然后找到一个自己是一个轨道的元$Pg$,那么就有$P'=P^g$.另外,如果我们确定了一个这样的$g$,那么全体可以把$P$共轭为$P'$的元构成的集合是$N_G(P)g$.

关于Sylow-C定理的一个频繁应用是所谓的\textbf{Frattini命题},即如果$N$是$G$的有限正规子群,并且如果存在$N$的一个Sylow-p子群$P$,那么有$G=N_G(P)N$.对任意的$g\in G$有$P^g\subset N^g\subset N$,于是$P^g$同样是$N$的Sylow-p子群,这结合Sylow-C定理(运用在$N$上!)说明存在一个$n\in N$使得$P^{gn}=P$,于是$gn\in N_G(P)$,于是$g\in N_G(P)N$,结合$g$的任意性得证.

最后一个定理是对Sylow子群个数的描述.即\textbf{Sylow个数定理}.首先共轭定理告诉我们全部Sylow-p子群就是一个子群的共轭类,于是它是一个群作用下的一个轨道,这说明全部Sylow-p子群的个数$n_p(G)$必然整除$|G|$.另外如果考虑全体Sylow-p子群构成的集合$\mathrm{Syl}_p(G)$,让其中一个$P$共轭作用在$S$上,那么固定子集$S_0$中的Sylow-p子群恰好就是满足$xQx^{-1}=Q,\forall x\in P$的子群.而这等价于说$P<N_G(Q)$,但是$P,Q$都是$N_G(Q)$的Sylow-p子群,所以它们共轭,但是$Q$是$N_G(Q)$的正规子群,这告诉我们$P=Q$,于是$S_0$中只有一个元,于是$|S|\equiv1(modp)$.

套用存在性定理的证明也可以证明Sylow个数定理,事实上我们可以证明,只要$p^b$整除有限群$G$的阶数,那么$G$的全部阶数为$p^b$的子群的个数是$\mod p$下余1的.为此,设$|G|=p^am$,其中$p\not\mid m$,那么$b\le a$,并且$G$的全部元素个数为$p^b$的子集构成的集合$S$的势是$\left(\begin{array}{c}
p^am\\
p^b\end{array}\right)$.现在让$G$右乘作用在集合$S$上,任取一个轨道$O$,任取$T\in O$,任取$x\in T$,那么$Tx^{-1}$包含幺元$e$,也就是说每个轨道$O$都存在一个元$A$,使得$A$包含幺元$e$.现在考虑$A$的稳定子$H$,如果$h\in H$,那么有$Ah=A$,于是$1\cdot h\subset A$,于是$H\subset A$.这就分为两种情况,倘若$A=H$,那么轨道$S$实际上就是子群$H$的全体右陪集构成的集合,于是轨道长度$|O|=|G|/|H|=p^{a-b}m$,并且$A$是$O$中唯一一个是子群的子集.倘若$A\not=H$,那么$|A|>|H|$,于是$|O|>p^{a-b}m$.按照$|O|$整除$p^am$,于是倘若$|O|=p^xt_0$,其中$p\not\mid t_0$,$x\le a,t_0\mid t$.那么$p^xt_0>p^{a-b}t\ge p^{a-b}t_0$导致$x\ge a-b+1$,也就是$p^{a-b+1}\mid |O|$.于是这时候不存在$G$的子群在轨道$O$中.于是如果$p^b$阶子群个数是$k$,那么$k$就是那些具有长度$p^{a-b}t$的轨道的个数.而不包含子群的轨道$O$轨道长度被$p^{a-b+1}$整除,这就导致$|S|=kp^{a-b}m+lp^{a-b+1}$,于是得到$|S|/p^{a-b}=km+lp\equiv km(\mod p)$,由于$p$不整除$m$,于是$m$存在$\mod p$下的逆$u$,导致$k\equiv |S|u/p^{a-b}\equiv\left(\begin{array}{c}
p^am\\
p^b\end{array}\right)u/p^{a-b}(\mod p)$.

我们可以直接用初等数论的方法证明最后一式在$\mod p$下为1.不过按照如下办法可以回避它:注意到$k(\mod p)$仅为$|G|$和$p^b$的函数,于是对全部阶数$p^am$的群都成立,因此我们可以直接取$G$为$p^am$阶循环群,于是$k\equiv1(\mod p)$,因为循环群的某阶子群恰为1个.
\newpage
\subsection{对称群}

有了群作用工具我们来探究对称群.称集合$X$上全体双射以复合作为二元运算得到的群为$X$上的对称群,记作$S(X)$.那么如果$X$和$Y$等势,就有$S(X)\cong S(Y)$.通常把有限集合上的对称群记中$S_n$,这里$n$是集合中的元素个数.

称置换$\sigma_1,\cdots,\sigma_s$是不交的,如果对任意$a\in S$若$\sigma_i(a)\not=a$,则$\forall j\not=i$有$\sigma_j(a)=a$.不交的置换是可交换的.

轮换和可迁作用.给定一个$n$元集合$S$,和$S_n$中的一个置换$\sigma$,如果$\langle\sigma\rangle$在$S$上的作用是可迁的,那么任取一个元$a_1\in S$,记$a_{i+1}=\sigma(a_i)$,那么必然存在一个$r$使得$a_{r+1}=a_1$.倘若$r<n$,那么取$S-\{a_1,\cdots,a_r\}$的一个元$b$,不存在$<\sigma>$中的元把$a_1$映射为$b$,这和可迁性矛盾.于是$r=n$.也就是说,对于置换$\sigma$在集合上的可迁作用,总存在一个$S$中元素的排列$a_1,\cdots,a_n$,使得置换可以表示为$\left(a_1a_2\cdots a_n\right)$,这个符号表示的是把$a_i$映射到$a_ {i+1},i=1,\cdots,r-1$,$a_r$映射到$a_1$.对于互不相同的$a_1,\cdots,a_r\in S$,把$a_i$映射像记作$a_ {i+1},i=1,\cdots,r-1$,$a_r$映射到$a_1$,在集合上其余元素不变的置换称为一个轮换,称$r$为轮换的长度,记作$\left(a_1a_2\cdots a_r\right)$长度为2的轮换叫做对换.那么也就是说,轮换就是生成的置换群能在在$S$某个子集上是可迁作用的置换.

置换的型.按照轨道分解,和每个轨道上作用是可迁的,以及不交的置换可交换,得到任一置换可以在不计顺序的意义下唯一的写作不交的有限个轮换之积.把它全部轮换的长度从大到小排列称为置换的型.例如$S_8$中(12345)(67)的型是[5,2,1].我们断言两个置换是共轭的当且仅当它们具有相同的型.
\begin{proof}
	
	如果$\beta=\gamma_1\cdots\gamma_t$.其中每个$\gamma_i$都是轮换,那么对任意置换$\alpha$,有$\beta^ {\alpha}=\gamma_1^{\alpha}\cdots\gamma_t^{\alpha}$.而$(i_1,\cdots,i_s)^{\alpha}=(\alpha(i_1),\cdots,\alpha(i_s))$.于是共轭不改变置换的型.反过来,如果$\alpha=\gamma_1\cdots \gamma_t$,$\beta=\delta_1\cdots\delta_t$,其中$\gamma_i$和$\delta_i$是具有相同长度的轮换.如果$\gamma_i= (i_1,\cdots,i_s)$,$\delta_i=(j_1,\cdots,j_s)$.取$\eta$为把$i_u$映射到$j_u$的置换,那么$\alpha^{\eta}=\beta$.
\end{proof}

于是$S_n$的共轭类个数就等于$n$的无序分拆个数.由此可以快速计算小阶对称群的类方程.以$S_5$为示范,首先5具有七个无序分拆,它们是:
$$5=4+1=3+2=3+1+1=2+2+1=2+1+1+1=1+1+1+1+1$$

我们来求5对应的共轭类的元素个数,它相当于五个位置的排列但是要除去轮换,或者说是一个圆排列,这得到$\frac{5!}{5}=24$个,对于4+1对应的共轭类,是$\frac{5\times 4\times 3\times 2}{4}\times\frac{1}{1}=30$,而对于3+1+1是$\frac{5\times 4\times 3}{3}\times\frac{2}{1}\times\frac{1}{1}\times\frac{1}{2!}=20$,注意到这里无序分拆中有两个1,所以需要除以$2!$除掉他们之间的差异,据此就可以得到$S_5$的类方程是:
$$120=1+10+15+20+20+30+24$$

按照正规子群是共轭类的并这个性质,从类方程能给出一个判断不存在某阶数正规子群的粗糙做法,例如这里通过把共轭类相加,普查得出$S_5$不存在30阶正规子群.注意,按照$60=1+15+20+24$只能说明可能存在60阶子群,并未给出证明.还要注意一点是,尽管我们将会在下面看到$S_5$的确存在60阶正规子群即交错群$A_5$,但这不代表上式即$A_5$的类方程,事实上子群的共轭类会出现分裂为多个共轭类的情况,对称群的共轭类要么在交错群中不变要么等分为二.例如$A_5$的类方程中,上式中的24需要分解为两个12之和.

这里给出对称群$S_n$的一些生成元集.
\begin{enumerate}
	\item 每个置换可以写作轮换乘积,而每个置换可以写作有限个对换之积:$(a_1,a_2,\cdots,a_n)=(a_1,a_2)(a_1,a_3)\cdots(a_1,a_n)$.于是全部对换生成整个对称群.
	\item 另外按照$\{(12),(23),\cdots,(n-1,n)\}$可以生成任意一个对换,于是这个集合也是对称群的生成元集.
	\item 再有$\{(12),(123\cdots n)\}$可以生成每个$(k,k+1)$(计算长度为n的轮换共轭于(12)得到的元),于是这个集合也是对称群的生成元集.
\end{enumerate}

每个置换可以写作对换之积,但是这样的写法不是唯一的,不过任一置换写作对换乘积的对换个数的奇偶性是不变的.这只需要考虑$S_n$的下述函数$\Delta$.$\Delta$是置换上的乘性函数,并且对换的值是$-1$,于是倘若一个置换能同时写作奇数个和偶数个对换乘积,它在$\Delta$下的值会不唯一,这矛盾.称写为奇数/偶数个对换的置换是奇置换/偶置换.
$$\Delta(\sigma)=\mathrm{sgn}\prod_{1\le i<j\le n}(\sigma(i)-\sigma(j))$$

型可以用来判断一个置换的奇偶性:非幺元的置换是偶置换当且仅当它的型中偶长度的有偶数个的置换是偶置换.这是因为一个长度为$r$的轮换可以写作$r-1$个对换之积.

事实上在得出$\Delta$把奇置换映射为$-1$,偶置换映射为1之后,就得到$\Delta$实际上是从$S_n$到${-1,1}$乘法群的满的($n\ge2$)群同态,它的核由全部偶置换构成(幺元是偶置换).于是从同构定理得到全体偶置换构成对称群的指数为2的子群,注意指数为2的子群必然是正规子群,这个群称为$n$元集合上的交错群,记作$A_n$.它是全体$S_n$中偶置换构成的群.

现在考虑$A_n$中的共轭类.对于偶置换$\sigma$,记$[\sigma]_A$和$[\sigma]_S$分别为该偶置换在$A_n$和$S_n$上的共轭类,那么会出现两种情况,要么$[\sigma]_A=[\sigma]_S$,那么前者是后者个数的一半,是哪种取决于$\sigma$在$S_n$中的中心化子包含于$A_n$还是并不包含.
\begin{proof}
	
	取偶置换$\sigma$,那么有$C_{A_n}(\sigma)=A_n\cap C_{S_n}(\sigma)$.现在如果$C_{S_n}(\sigma)\subset A_n$,那么$C_ {A_n}(\sigma)=C_{S_n}(\sigma)$,导致$[S_n:C_ {S_1}(\sigma)]=2[A_n:C_{A_n}(\sigma)]$,于是这时候共轭类要一分为二.如果$C_{S_n}(\sigma)\not\subset A_n$,注意到$A_nC_ {S_n}(\sigma)$是$S_n$的子群,并且真包含了$A_n$,按照$A_n$是指数2的子群,不可能存在更大的真子群,于是$A_nC_{S_n}(\sigma)=S_n$.于是$[A_n:C_ {A_n}(\sigma)]=[A_n:A_n\cap C_{S_n}(\sigma)]=[A_nC_{S_n}(\sigma):C_{S_n}(\sigma)]=[S_n:C_{S_n}(\sigma)]$.于是此时共轭类不分裂.
\end{proof}

$S_n$中的一个偶置换的共轭类分裂为$A_n$中两个共轭类当且仅当这个置换的型是由不同奇数构成的.即$C_{S_n}(\sigma)\subset A_n$当且仅当$\sigma$的型由不同奇数构成.
\begin{proof}
	
	记$\sigma=\sigma_1\cdots\sigma_t$,其中$\sigma_i$的长度两两不同的奇数,那么$\sigma^ {\gamma}=\sigma_1^{\gamma}\cdots\sigma_t^{\gamma}$.如果$\sigma$和$\gamma$可交换,也就是说$\sigma^{\gamma}=\sigma$,必然有对每个$\gamma_i= (i_1,\cdots,i_u)$,有$(i_1,\cdots,i_u)=(\gamma(i_1),\cdots,\gamma(i_u))$.也就是说必然有$\gamma$限制在这个轨道上是$(i_1,\cdots,i_u)^{r}$.这是一个偶置换的$r$次复合,于是它是偶置换,于是$\gamma\in A_n$.
	
	反过来,假设$\sigma$存在一个轮换是奇数长度,或者轮换长度都是奇数但是存在两个相同奇数长度的轮换.在第一种情况下,就取$\gamma$是这个偶数长度的轮换,那么$\gamma$和$\sigma$交换,导致$C_{S_n}(\sigma)\not\subset A_n$.在第二种情况下,考虑分解中两个长度相同的轮换$(a_1,\cdots,a_u)$和$(b_1,\cdots,b_u)$.取奇置换$\gamma= (a_1b_1)\cdots(a_ub_u)$,那么它和$\sigma$交换,同样导致$C_ {S_n}(\sigma)\not\subset A_n$,证毕.
\end{proof}

例如考察$A_5$,它的型包括$5,3+1+1,2+2+1,1+1+1+1+1$,其中$5$需要分裂为2, 就得到了$A_5$的类方程是:$60=1+15+20+12+12$.另外由此可以立马得出它不存在非平凡正规子群.即$A_5$是单群.

事实上我们可以证明$n\ge5$时$A_n$都是单群.
\begin{proof}
	
	$S_n$的全部3轮换生成了$A_n$.事实上偶置换总可以写作偶数个对换之积,于是只要说明形如$(ab)(cd),a\not=b,c\not=d$总可以写作3轮换之积.如果$(ab)=(cd)$,那么没什么可证的;如果$\{a,b\}$和$\{c,d\}$的交恰好有一个元,不妨设$c=a$,那么$(ab) (ad)=(adb)$.如果$\{a,b\}$和$\{c,d\}$不交,那么有$(ab)(cd)=(adc)(abc)$.
	
	如果$n\ge5$,有$A_n$的正规子群包含一个3轮换,那么必然包含全部的3轮换.这只要证明全部3轮换构成了$A_n$中的一个共轭类,因为正规子群是共轭类的并,但是3轮换的型是$[3,1,1,\cdots]$,于是不在$A_n$中分裂.
	
	$A_6$是单群.取$A_6$的一个非平凡正规子群$H$,取$H$中的非幺元$\alpha$.如果$\alpha$固定了某个$i$,取$F=\{\beta\in A_6\mid \beta(i)=i\}$.于是$F\sim A_5$,并且$\alpha\in H\cap F$.但是按照第二同态定理,$H\cap F$是$F$的正规子群,于是从$F$单群得到$H\cap F=F$,于是$F\subset H$,于是$H$包含(至少)一个3轮换,于是得证.现在假设$H$中的一个非幺元$\alpha$不固定每个$1\le i\le6$.那么$\alpha=(12)(3456)$或者$(123)(456)$,对第一种情况$\alpha^2$固定1和2并且非幺元.在第二种情况,取$\beta=(234)$,那么$\alpha\beta\alpha^{-1}\beta^{-1}$不是幺元固定了1.
	
	$A_n,n\ge7$是单群.取非平凡的正规子群$N$,取它的非幺元$\beta$,那么存在某个$i$使得$\beta$不固定$i$.取$\alpha\in A_n$是一个3轮换,并且固定了$i$,于是$\alpha$和$\beta$不交换,于是它们的交换子$\gamma=\beta^ {\alpha}\beta^{-1}$不是幺元.另外按照$N$的正规性这个交换子在$N$中,于是它是两个3轮换的乘积$(\beta^{\alpha})\beta^{-1}$.于是它移动了至多六个元,记作$i_1,\cdots,i_6$.考虑这六个元上全部置换构成的子群$F$,那么$F\sim A_6$.并且$H\cap F$是$F$的正规子群,按照$A_6$单群得到$H\cap F=F$,于是$F\subset H$,于是$H$包含了(至少)一个3轮换,于是$H=A_n$.
\end{proof}

对称群的最后一个内容是求出$S_n$的自同构群.如果一个群中心平凡,并且没有外自同构,那么会得到$G\cong\mathrm{Aut}(G)$.我们称满足中心平凡并且没有外自同构的群为完备群.我们要给出的结论是,$S_n$在$n\not=2,6$时都是完备群.$S_2$中心不平凡,于是自然不是完备群,但是$S_2$的自同构群自然是{e}.$S_6$的情况会复杂一些,我们会看到$S_6$上存在外自同构,它的外自同构群是$\mathbb{Z}/2$.

先来说明,$n\ge3$时$S_n$的中心总是平凡的.
\begin{proof}
	
	取$S_n$的一个非幺元$\sigma$,那么它会移动至少一个元,记$\sigma(i)=k\not=i$.那么它必然也移动了$k$,再记$\sigma(k)=j$.这时就出现两种可能.
	
	如果$i=j$.也就是说在$\sigma$的不交轮换的分解中,存在一个对换$(ik)$,即$\sigma=(ik)\tau=\tau(ik)$,这里$\tau$在$i,k$上都是固定.按照$n\ge3$,可以再取一个集合中的元$l$和$i,k$都不同,那么考虑$(lk)$,它和$\sigma$不交因为$(lk)\sigma$把$i$映射为$l$,而$\sigma(lk)$把$i$映射为$k$.
	
	假设$i\not=j$.设$\sigma$的不交的轮换的分解为$(ikj\cdots)\tau$.那么$(ik)$和$\sigma$是不交的,因为$\sigma(ik)$把$i$映射到$j$,但是$(ik)\sigma$却固定了$i$.
\end{proof}

$S_n$内自同构的刻画.$S_n,n\ge3$上的一个自同构$\phi$,它是内自同构当且仅当它把每个对换映射为对换.
\begin{proof}
	
	一方面共轭的置换具有相同的型,于是内自同构不会改变置换的型,于是内自同构把对换映射为对换.
	
	现在需要证明把对换映射为对换的自同构$\sigma$必然是内自同构.先任取三个不同的点$a,b,c$,那么$\sigma(ab)$和$\sigma(ac)$都是对换.我们断言这两个对换恰好有一个公共元,若否,因为它们不会相同,于是它们必然不交,这导致它们可交换,但是作用自同构的逆,得到$(ab)$和$(ac)$可交换,这明显矛盾.于是对不同的$a,b,c$存在不同的$a',b',c'$使得$\sigma(ab)= (a'b')$,$\sigma(bc)=(b'c')$,$\sigma(ca)=(c'a')$.
	
	接下来证明,对任意的$d$,有某个$d'$使得$\sigma(ad)=(a'd')$.不妨设$d\not=b,c$否则已经成立.那么$(ad)$和$(bc)$可交换,导致$\sigma(ad)$和$(b'c')$可交换.于是$\sigma(ad)$固定$b'$和$c'$.但是$(ab)$和$(ad)$不交换得到$(a'b')$和$\sigma(ad)$不交换,这导致$a'$或者$b'$在$\sigma(ad)$中,于是迫使$a'$出现在$\sigma(ad)$中.于是可以定义一个映射$g(a)=a'$,它是唯一的公共出现在每个$\sigma(ax)$中的元$a'$.于是$g$定义了一个$S$上的双射,有$\sigma(ab)=(g(a)g(b))$,于是$\sigma$是被置换$g$共轭的内自同构.
\end{proof}

如果$n\not=2,6$,那么$S_n$是完备群.
\begin{proof}
	
	设$n\ge3$,定义$T_k$是$S_n$这样的子集,如果一个置换分解为不交轮换积时,它恰好是$k$个两两不交的对换的乘积,那么它属于$T_k$.于是$T_k$是共轭类.于是如果一个置换的阶是2,那么它必然属于某个$T_k$.我们知道一个自同构把共轭类映射为共轭类,于是对$\theta\in\mathrm{Aut}(S_n)$,有$\theta(T_1)=T_k$.
	
	我们断言$n\not=6$时有$|T_k|\not=|T_1|$,$\forall k\not=1$.一旦证明这一点,就导致$\theta(T_1)=T_1$,于是从上一个结论得到$\theta$总是内自同构.
	
	首先$|T_1|=\frac{n(n-1)}{2}$.而$k$个有序的两两不交的对换一共有$\frac{n(n-1)} {2}\times\frac{(n-2)(n-3)}{2}\times\cdots\times\frac{(n-2k+2)(n-2k+1)}{2}$.接下来由于不交对换是可交换的,于是要除以$k!$,就得到$|T_k|=\frac{n(n-1)\cdots(n-2k+1)}{k!2^k}$.倘若存在等式$|T_k|=|T_1|$,那么有$(n-2)(n-3)\cdots(n-2k+1)=k!2^{k-1}$.于是$n\ge2k$.于是左侧$\ge(2k-2)!$而当$k\ge4$时有$(2k-2)!>k!2^{k-1}$.于是只能有$k=2,3$.当$k=2$时右侧是4,此时等式必然不会成立.当$k=3$时,有$n\ge6$,如果$n>6$,那么左侧$\ge 5!=120>$右侧的24.于是唯一可能相等的情况是$n=6$.
\end{proof}

至此我们得到了对$n\not=2,6$有$\mathrm{Aut}(S_n)\sim S_n$.另外$\mathrm{Aut}(S_2)=1$.最后来处理$S_6$.

首先按照上述证明,如果记$T_k$为型是$k$个不交对换乘积的置换构成的集合.那么对于$S_6$,有$|T_1|=|T_3|$,并且$|T_1|\not=|T_k|,k\not=1,3$,这导致如果$S_6$上存在外自同构,那么它把对换映射为三个不交对换的乘积.

先来说明$S_6$至少存在一个外自同构.
\begin{proof}
	
	先说明$S_6$存在一个阶数为120的可迁子群,并且它不包含对换.【】
	
	记上述可迁子群为$K$,把$K$的全体左陪集构成的集合记作$Y$,现在把$S_6$左平移作用到$Y$上,就得到一个满同态$\theta:S_6\to S_Y$.那么$\ker\theta\subset K$是$S_6$的正规子群.但是$S_6$的唯一非平凡正规子群是$A_6$,于是$\ker\theta={e}$,于是$\theta$必然是一个双射.于是$\theta$本身也在$\mathrm{Aut}(S_6)$中.现在说明$\theta$不是内自同构.假设它是,那么它应该保置换的型,特别的,$\theta(1,2):aK\mapsto(1,2)a_iK,1\le i\le6$是$Y$上的一个对换,其中$a_i$满足$Y=\{a_1K,\cdots,a_6K\}$.也就是说$\theta(1,2)$至少固定了$1\le i\le6$中的四个.但是如果$\theta(1,2)$固定了哪怕一个陪集,就得到$a_i^{-1}(1,2)a_i\in K$,这就和$K$不含对换矛盾.
\end{proof}

接下来证明$S_6$的外自同构群是$\mathbb{Z}/2$.
\begin{proof}
	
	任取$S_6$的两个外自同构$\phi,\psi$,那么我们已经证明了$\phi,\psi$都把$T_1$映射到$T_3$,把$T_3$映射到$T_1$.于是$\phi\circ\psi^{-1}$把$T_1$映射到$T_1$,于是这是一个内自同构.于是$\mathrm{Aut}(S_6)/\mathrm{Inn}(S_6)$是二阶的.
\end{proof}
\newpage
\section{次正规列}
\subsection{幂零群}

$p$群在有限群论中无处不在,这不仅仅是因为Sylow定理是有限群论的基本工具.还有一个原因是,相比非$p$群,$p$群的个数非常之多.$2^7$阶群在同构意义下有有2,328个;$2^8$阶群在同构意义下有56,092个;$2^9$阶群在同构意义下有10,494,213个;而$2^{10}$阶群在同构意义下有49,487,365,422个.在同构意义下,小于2000阶的群中,$2^{10}$阶群的个数已经占据了$99.15\%$!

有限$p$群实际上是一类特殊的\textbf{幂零群}.给定一个群$G$,它的正规列是指$G$的一个正规子群的有限集$\{N_i,0\le i\le r\}$,满足$1=N_0\subset N_1\subset\cdots\subset N_r=G$.称正规列是中心列,如果还满足$N_i/N_{i-1}\subset Z(G/N_{i-1}),\forall 1\le i\le r$.如果群$G$有中心列,就称它为幂零群.我们约定平凡群不是幂零群.

那么首先,幂零群的子群和商群都是幂零群.
\begin{proof}
	
	设$1=N_0\subset N_1\subset\cdots\subset N_r=G$是$G$的一个中心列,对$G$的任意子群$H$,取$H_i=N_i\cap H$,那么$H_i$都是$H$的正规子群(第二同构定理).于是得到$H$的正规列$1=H_0\subset H_1\subset\cdots\subset H_r=H$.现在需要验证$H_i/H_{i-1}\subset Z(H/H_{i-1})$,这等价于讲$\forall h\in H_i,g\in H$有$hgH_{i-1}=ghH_{i-1}$,即$h^{-1}g^{-1}hg\in H_{i-1}=N_{i-1}\cap H$,但是由于$h\in N_i$和$N_i/N_{i-1}\subset Z(G/N_{i-1})$得到$h^{-1}g^{-1}hg\in N_{i-1},\forall g\in G$,于是$h^{-1}g^{-1}hg\in N_{i-1}\cap H$.即子群$H$有中心列.
	
	对$G$的任意商群$G/H$,按照子群对应定理,每个$N_i/H$是$G/H$的正规子群,于是得到$G/H$的正规列$1=N_0/H\subset N_1/H\subset\cdots\subset N_r/H=G/H$.现在需要验证的是$\frac{N_i/H}{N_{i-1}/H}\subset Z(\frac{G/H}{N_{i-1}/N})$,这等价于$N_i/N_{i-1}\subset Z(G/N_{i-1})$.即$G/H$有中心列.
	
\end{proof}

幂零群的中心非平凡.按照定义,中心列中的$N_1$需要满足$N_1\subset Z(G)$,倘若中心列平凡,就导致不会存在中心列.

给定群$G$,我们来定义它的上中心链.取$Z_0=1$,取$Z_1=Z(G)$.那么$Z_1$是正规子群.$Z(G/Z_1)$是$G/Z_1$的正规子群,于是按照对应定理唯一存在$G$的正规子群$Z_2$满足$Z_2/Z_1=Z(G/Z_1)$.归纳构造下去,记$Z_n$是$G$的唯一的正规子群使得$Z_n/Z_{n-1}=Z(G/Z_{n-1})$.由此得到的正规子群列$1=Z_0\subset Z_1\subset\cdots$称为$G$的上中心链.那么一般来讲上中心链是未必终止于$G$的,即未必存在一个$r$使得$Z_r=G$.有可能所有$Z_i$都是真子群,还有可能$Z_i$在足够大的时候终止,但是终止于$G$的一个真子群.我们断言有限群$G$是幂零群有如下等价描述:
\begin{enumerate}
	\item $G$非平凡,并且每个非平凡的$G$的商都有非平凡中心.
	\item $G$非平凡,并且上中心列终止于$G$,即存在$r$使得$Z_r=G$.
\end{enumerate}
\begin{proof}
	
	如果$G$是幂零群,那么$G$的每个非平凡的商都是幂零群,于是中心是非平凡的.如果有限群$G$满足条件1,从$Z_{i+1}/Z_i=Z(G/Z_i)$得到$Z_{i+1}$严格包含$Z_i$,按照群是有限群,一个严格子群升链必然终止于$G$.最后如果有限群$G$满足条件2,此时上中心列本身就是一个中心列,于是群是幂零群.
\end{proof}

对有限$p$群,它的非平凡商必然也是一个有限$p$群,按照有限$p$群中心总是非平凡的,立刻得到有限$p$群都是幂零群.

上述对幂零群的等价描述中第二条对无限群也成立,即一个可以无限阶的群是幂零群当且仅当它非平凡并且的上中心列终止于$G$.这个结论用到如下定理:对于幂零群,上中心列包含了每一个中心链.即,如果未必有限阶的幂零群$G$有中心列$1=N_0\subset N_\subset\cdots\subset N_r=G$.记上中心列为$1=Z_0\subset Z_1\subset\cdots$,那么每个$N_i\subset Z_i$,特别的,有$Z_r=G$.
\begin{proof}
	
	对$i$归纳,当$i=0$时有$Z_0=N_0=1$.现在设$i>0$,归纳假设$N_{i-1}\subset Z_{i-1}$.那么存在典范的满同态$\theta:G/N_{i-1}\to G/Z_{i-1}$,并且它把中心元映射为中心元,于是得到$N_i\subset Z_i$.
\end{proof}

现在任取一个幂零群$G$,记最小的满足$Z_r=G$的$r$为$G$的幂零类.于是,幂零类1等价于说群是非平凡交换群,幂零类2等价于要求群$G$是$G/Z(G)$交换的非交换群.另外,按照上述定理,幂零类给出了中心链长度的下界.

幂零群的一个重要性质是\textbf{正规化子性质}.即,如果$G$是幂零群,对真子群$H$,有$H$真包含于$N_G(H)$.
\begin{proof}
	
	取中心列$\{N_i\mid 0\le i\le r\}$.按照$N_0=1\subset H$,并且$N_r=G\not\subset H$,可以取一个最大的$0\le k<r$使得$N_k\subset H$但是$N_{k+1}\not\subset H$.我们来证明$N_{k+1}\subset N_G(H)$,由此原命题得证.
	
	取$\overline{G}=G/N_k$,那么$\overline{N_{k+1}}=N_{k+1}/N_k\subset Z(\overline{G})\subset N_{\overline{G}}(\overline{H})=\overline{N_G(H)}$.于是按照$N_k\subset N_G(H)$得到$N_{k+1}\subset N_G(H)$得证.
\end{proof}

对$p^a$阶的$p$群$P$,对任意阶数$p^b,0\le b\le a$.存在$P$的$p^b$阶正规子群.事实上若$P$是有限$p$群,取真正规子群$N\subset M$,那么存在正规子群$L$包含$N$,并且满足$[L:N]=p$.
\begin{proof}
	
	取$P'=P/N$,记$M'=M/N$,那么$M'$是$P'$的非平凡正规子群.注意到$Z(P')\cap M'$是非平凡的,于是存在一个阶数$p$的元,于是它生成了$p$阶正规子群,记作$L'$,那么$L'$是$P'$的正规子群,就有$L$是$P$的正规子群,并且$[L:N]=p$.
\end{proof}

现在我们利用Sylow理论来给出有限幂零群的等价描述:
\begin{enumerate}
	\item $G$是有限幂零群.
	\item $G$是满足正规子群性质的非平凡有限群,即对每个真子群$H$有$N_G(H)$真包含了$H$.
	\item $G$非平凡,并且所有极大子群(全部真子群在包含序下的极大元)都是正规的.
	\item $G$非平凡,并且所有Sylow子群是正规的.
	\item $G$非平凡,并且是全部Sylow子群的直积.
\end{enumerate}
\begin{proof}
	
	1推2已经得证,2推3也是直接的.现在假设3成立,取一个非平凡Sylow-$p$子群$P$,如果$N_G(P)$是真子群,那么它包含在某个极大子群$M$中,按照条件有$M$是正规子群.于是由Frattini命题得到$G=N_G(P)M\subset M$,这矛盾.于是$P$是正规的.
	
	从4推5只要注意到如果有限群$G$的一组有限个正规子群,并且它们的两两阶数互素,那么这些正规子群的直积是$G$的一个子群,特别的如果阶数相同,那么这些正规子群的直积就是$G$.
	
	最后从5推1,当5成立时自然有4成立,于是4对每个$G$的商都成立.按照一个直积的中心是每一分支中心的直积,结合有限$p$群中心非平凡,得到$G$的每个非平凡商都有非平凡中心,于是得到$G$是幂零群.
\end{proof}

记有限群$G$的全部$\mathrm{Syl}_P(G)$中子群的交为$O_p(G)$,那么按照Sylow共轭定理得到它是唯一的极大正规$p$子群,即它包含了每一个正规$p$子群.定义$G$的\textbf{Fitting子群}是全部$O_p(G)$的直积,其中$p$取遍$|G|$的全部素因子.Fitting子群记作$F(G)$.由于Fitting子群是两两阶数互素的特征子群的直和,于是它是特征子群.

有限群$G$的Fitting子群是正规幂零子群,并且每个正规幂零子群都包含在Fitting子群中,即Fitting子群是极大的正规幂零子群.
\begin{proof}
	
	按照$F(G)$是有限个不同$p$的$p$群的直积,于是立刻得到它是幂零群.现在假设$N$是$G$的幂零正规子群,如果$P\in\mathrm{Syl}_p(N)$,那么$P$是$N$的正规子群,于是它是$N$的特征子群,于是它在$G$中是正规子群.而$P\subset O_p(G)\subset F(G)$.按照$N$是自身Sylow-$p$子群的直积,每一项都包含在$F(G)$对应直和项中,于是$N\subset F(G)$.
\end{proof}

如果$K,L$是有限群$G$的幂零正规子群,那么$KL$是幂零的.事实上按照$K,L\subset F(G)$得到$KL\subset F(G)$,而幂零群的子群幂零.

一个无限幂零群的例子.给定域$k$,考虑全体对角元为1的上三角矩阵构成的$\mathrm{GL}_n(k)$的子群$G$,我们断言$G$是一个幂零群.
\begin{proof}
	
	我们直接求它的中心升链.首先$Z_0=1$,$Z_1=Z(G)$中的元即和全部对角元为1的上三角可逆矩阵可交换的$G$中的元,这只能是数量矩阵,于是$Z_1=\{\lambda E_n\mid \lambda\in k\}$.按照定义$Z_2/Z_1$是$G/Z_1$的中心,于是
\end{proof}


\newpage
\subsection{非交换单群}

我们接下来讨论单群.单群就是没有非平凡正规子群的群.那么有限交换单群就是素数阶循环群.但是非交换单群非常罕见.阶数小于1000的非交换单群在同构意义下只有五个,它们的阶数分别是60,168,360,504,660.每个阶数只有一个非交换单群.的确存在同一个阶数的两个不同构的单群,但是这样的阶数的最小值已经是$8!/2=20160$.没有一个阶数是三个两两不同构单群的相同阶数.

本节我们提及单群都指有限单群.的确存在无限非交换单群,一个典型的例子是$A_{\infty}$,记$A_n$为$n$次交错群,则$A_{\infty}$的定义是典范同态$f_i$构成的下述图表的极限.它实际上就是可数集合$S=\{a_n,n\ge1\}$上全体只变动其中有限个元的偶置换构成的群.换句话说,如果$A_n$记作$S$的子集$\{a_1,\cdots,a_n\}$上的交错群,那么$A_{\infty}\cup_ {n\ge1}A_n$.
$$\xymatrix{A_1\ar@{^{(}->}[r]&A_2\ar@{^{(}->}[r]&\cdots\ar@{^{(}->}[r]&A_n\ar@{^{(}->}[r]&\cdots}$$

为证明$A_{\infty}$是单群,只要证明这样一个结论,如果存在一列单群$\{G_n\}_{n\ge1}$,满足每个$G_n$都是$G_{n+1}$的子群,那么$\cup_{n\ge1}G_n$是单群.
\begin{proof}
	
	容易验证$G=\cup_{n\ge1}G_n$是群.为了验证$G$是单群,这里提供一个新的思路,群$G$是单群当且仅当对任意一个非平凡的群同态$f:G\to H$,它必然是单的.现在倘若存在$f:G\to H$不是单的,也就是存在$g_1\not=g_2\in G$使得$f(g_1)=f(g_2)$.存在一个足够大的正整数$N$使得$n\ge N$的时候$g_1,g_2\in G_n$.而$f$限制在子群$G_n$上有$f(g_1)=f(g_2),g_1\not=g_2$,这就和$G_n$是单群矛盾.
	
\end{proof}

罕见性可以理解为我们乐于寻找非交换单群的理由之一.一个更严谨的解释是,全部有限群可以从单群上构造出来.给定一个非平凡的有限群$G$,如果$G$本身不是单群,按照有限性,$G$存在至少一个极大的正规真子群$N$,那么此时$G/N$是单群.现在如果$N$不是单群,就可以重复操作取$N$的一个极大的正规真子群.按照群$G$是有限的,这个操作会在有限步后终止.于是得到一列子群列$1=N_0\triangleleft N_1\triangleleft\cdots\triangleleft N_r=G$.其中每个$N_i$是$N_ {i+1}$的正规子群,并且$N_{i+1}/N_i$是单群.这样的子群列称为群$G$的合成列,全部单群$N_{i+1}/N_i$称为合成因子.\textbf{Jordan-Holder}定理告诉我们,尽管同一个群可能存在多个合成列,但是合成因子在计重数和同构的意义下是固定的.于是有限群唯一的对应于这些计重数意义下的单群集.那么,如何从一个单群列构造出原始的群呢?这就是所谓的群的延拓问题.如果已经构造了群$N_i$,那么接下来需要找的$N_{i+1}$是这样的群,它以$N_i$为正规子群,并且$N_{i+1}/N_i$是我们指定的单群.给定两个有限群$A,B$,称它们的延拓就是一个短正合列$1\to A\to C\to B\to1$,也就是$C$以$A$(的某个同构)为正规子群,并且$C/A$同构于群$B$.于是一旦解决延拓问题,那么就可以从单群列构造出有限群.

寻找非交换单群可以从两个方向实现:直接构造非交换单群;寻找单群不能满足的性质.第二个方向的一个重要的结果是在20世纪早期,Burnside证明了非交换单群的阶数必然是至少三个不同素数的幂的乘积,也就是Burnside的$p^aq^b$定理.Burnside还注意到全部非交换单群已有的例子都是偶数阶群,于是他猜测所有奇数阶单群都是循环素数阶群.这个猜测可以等价描述为奇数阶群都可解,最终于二十世纪六十年代早期被W.Feit和J.G.Thompson证明,这个证明长达250页!在这之后陆续出现很多单群分类定理.最后于本世纪初,有限单群的完全分类被彻底解决.每一个非交换单群是如下三类单群之一:
\begin{enumerate}
	\item 交错群$A_n,n\ge5$.
	\item $\mathrm{PSL}(n,q)$.
	\item 26种不属于上述两种情况的非交换单群,称为sporadic单群.
\end{enumerate}

这里我们解释第二个情况.这里的$\mathrm{PSL}(n,q)$是投射特殊线性群.其中$q$是一个素数幂,而$n$是$\ge2$的整数.取$F$为阶数$q$的有限域,那么一般线性群$\mathrm{GL}(n,q)$表示的是全体$F$上$n$阶可逆矩阵构成的群.特殊线性群$\mathrm{SL}(n,q)$表示的是全体行列式1的$\mathrm{GL}(n,q)$的子群.那么特殊线性群的中心$Z$是全体行列式1的系数矩阵.记$\mathrm{PSL}(n,q)$为商群$\mathrm{SL}(n,q)/Z$.事实上在除去$n=2$且$q=2,3$的情况,投射特殊线性群总是单群.并且阶数小于1000的五个非交换单群事实上就是$\mathrm{PSL}(2,q)$, $q=5,7,8,9,11$.对应的阶数依次为60,168,504,360,660.另外同一个单群可以存在不同的描述,例如唯一的60阶非交换单群可以作为$\mathrm{PSL}(2,5)$,但是他也同构于$\mathrm{PSL}(2,4)$和$A_5$.168阶单群还可以表示为$\mathrm{PSL}(3,2)$.360阶单群还可以表示为$A_6$.

现在开始我们给出单群不能满足的一些简单性质.
\begin{enumerate}
	\item 如果$|G|=pq$,其中$q<p$是两个不同的素数,那么$G$存在正规的Sylow-$p$子群.并且当$q\not\mid p-1$时候有$G$是循环群.当$q\mid p-1$时在同构意义下恰好存在两个群,一个是循环群$Z/pq$,一个是非交换群$K$,它由$c,d$生成,其中$c$阶数$p$,$d$阶数$q$,并且满足$dc=c^sd$,这里$s\not\equiv1(\mod p)$并且$s^q\equiv1(\mod p)$.特别的,如果$p$是奇素数,那么$2p$阶群在同构意义下恰好有两个,一个是循环群$Z/2p$,一个是二面体群$D_p$.综上,$pq$阶群必然不是单群.
	\begin{proof}
		
		给定一个$pq$阶的群$G$,如果$n_p>1$,按照Sylow计数定理,有$n_p\equiv1(\mod p)$,于是如果$n_p>1$,有$n_p>p>q$,但是$n_p$要整除$q$,这矛盾.于是$n_p=1$,即Sylow-$p$子群恰有1个.
		
		现在按照Cauchy定理,存在一个阶为$p$的元$a$,也存在一个阶为$q$的元$b$.于是$S=\langle a\rangle$是一个正规子群.现在$G/S$是$q$阶群,于是对每个$G$中元必然可以写作$b^ia^j$,并且$G=\langle a,b\rangle$.现在考虑Sylow-$q$子群的个数$n_q$,它是$kq+1$型并且必然整除$p$,于是$n_q=1$或$p$.
		
		倘若$n_q=1$,那么$\langle b\rangle$也是唯一的Sylow-$q$子群.于是$G\sim \langle a\rangle\oplus\langle b\rangle$,于是此时$G$是循环群$Z/pq$.
		
		现在假设$n_q>1$,那么此时必然有$q\mid p-1$.此时存在一个$r\not\equiv1(\mod p)$,使得$bab^{-1}=a^r$,否则必然有$G$是交换群,那么Sylow-$q$子群理应只有一个.现在归纳得到$b^jab^ {-j}=a^{r^j}$,特别的取$j=q$,得到$a=a^{r^j}$.于是得到$r^q\equiv1(\mod p)$.
		
		最后我们需要验证这时候$G$同构于$K$.注意到$x^q\equiv1(\mod p)$恰好有$q$个$\mod p$下的不同根.并且如果$r$是一个根,$k$是最小的满足$r^k\equiv1(\mod p)$的正整数,那么$k\mid q$,这导致$k=q$.于是$1,r,\cdots,r^{q-1}$是全部不同的根.于是存在某个$1\le t\le q-1$使得$s\equiv r^t(\mod p)$.取$b_1=b^t$,那么$b_1$的阶仍然是$q$.并且$G=<a,b_1>$.此时$G$的每个元可以表示为$b_1^ia^j$,其中$|a|=p,|b_1|=q$.并且有$b_1ab_1^{-1}=b^tab^{-t}=a^{r^t}=a^s$.那么从$a\mapsto c,b_1\mapsto d$的映射是$G$到$K$的同构,得证.
	\end{proof}
	\item 如果$|G|=p^2q$,其中$p,q$是不同素数,那么$G$要么有一个正规Sylow-$p$子群,要么有一个正规Sylow-$q$子群.于是此时群必然不是单群.
	\begin{proof}
		
		若否,有$n_p>1,n_q>1$,那么必然有$n_p>p,n_q>q$.注意到$n_p$整除$q$,并且$n_q$整除$p^2$.那么$n_p=q$.如果$n_q=p$,那么有$n_p>p=n_q>q=n_p$,这矛盾.于是必然有$n_q=p^2$.即$G$有$p^2$个不同的$q$阶群.但是两个不同的$q$阶群的交只能是幺元,这导致这$p^2$个不同的$q$阶群提供了$p^2(q-1)$个$q$阶元.于是$G$中的阶数不为$q$的元的个数是$p^2$个.于是任取$G$的一个Sylow-$p$子群,它恰好就是全部剩余的元,这导致$G$只有一个Sylow-$p$子群,矛盾.
	\end{proof}
	\item 类似上面证明考虑阶数,还可以得到如果$|G|$是三个不同素数的乘积,那么$G$不是单群.
	\item 如果$a>0$是整数,那么$p^aq$阶群必然不是单群,这里$p,q$是两个不同的素数.
	\begin{proof}
		
		不妨设$n_p>1$,于是得到$n_p=q$.现在取两个不同的Sylow-$p$子群$S,T$,使得$|S\cap T|$的阶数是最大的,记这时$D=S\cap T$.如果$D=1$,那么任意两个不同的Sylow-$p$子群的交是平凡的,这导致全部阶数为$p$次幂的非幺元的元的个数为$q(p^a-1)$.于是阶数不是$p$正整数次幂的元的个数是$q$.这导致Sylow-$q$子群恰好有一个,于是群不是单群.
		
		现在假设$|D|>1$,取$N=N_G(D)$,按照$p$群上一个真子群的正规化子严格包含这个真子群,有$N\cap S$真包含$D$,$N\cap T$也真包含$D$.
		
		现在我们断言$N$不是$p$群,否则它包含在某个Sylow-$p$群$R$中,但是$R\cap S\supset N\cap S$是真包含$D$的,这和$D$的极大性矛盾.这就导致$q$整除$|N|$,于是如果取$Q\in\mathrm{Syl}_q(N)$,得到$|Q|=q$,那么$S\cap Q=1$,结合$|SQ|=|G| $,得到$SQ=G$.于是对任意的$g\in G$,可以写作$g=xy$,其中$x\in S,y\in Q$.那么得到$S^g=S^{xy}=S^y\supset D^y=D$,其中最后一个等式是因为$y\in Q\subset N=N_G(D)$.于是$D$包含在$G$中$S$的每个共轭类上中.于是$D\subset O_p(G)$.这导致$O_p(G)$是非平凡的真正规子群,于是$G$仍然不是单群.
	\end{proof}
	\item 有限群$G$,取$\mathrm{Syl}_p(G)$的两个元的交的极小元$D=S\cap T$.我们断言$O_p(G)$是唯一的同时在$S,T$中正规的极大子群.特别的,当$P$是交换群时,必然有$\mathrm{Syl}_p(G)$中两个元的交就是$O_p(G)$.证明,取$K\subset D$,是在$S,T$中都正规的子群,我们需要证明$K$包含在每个Sylow-$p$子群中.取$N=N_G(K)$,那么$S\subset N$,于是$S\in\mathrm{Syl}_p(N)$.任取一个$G$的Sylow-$p$子群$P$,那么$P\cap N$是$N$的一个$p$子群.于是$P\cap N$包含在了$N$的某个Sylow-$p$子群中,也就是存在$x\in N$使得$P\cap N\subset S^x$.按照$T\subset N$得到$T^x\subset N$,于是得到$P\cap T^x=P\cap N\cap T^x\subset S^x\cap T^x=D^x$.于是得到$P^{x^{-1}}\cap T=(P\cap T^x)^{x^{-1}}\subset D$.但是$P^{x^{-1}}$和$T$都是$G$的Sylow-$p$子群,交包含在极小元$D$中,于是$P^{x^{-1}}\cap T=D$.于是$K\subset D\subset P^{x^{-1}}$.于是$K^x=K\subset P$.
	
	若有限群$G$有交换的Sylow-$p$子群$P$,那么有$[G:O_p(G)]\le[G:P]^2$.事实上任取两个Sylow-$p$子群$S,T$使得$S\cap T=O_p(G)$,立刻得到$|G|\ge|ST|=\frac{|S||T|}{|S\cap T|}=\frac{|P|^2}{|O_p(G)|}$,整理得证.
	
	据此我们可以得到一个单群不能满足的性质:有限群$G$如果阶数不是素数$p$,有一个非平凡的交换Sylow-$p$子群$P$,假设$|P|>\sqrt{|G|}$,那么$O_p(G)>1$,于是它是一个非平凡的真正规子群,导致$G$不是单群.
\end{enumerate}
\newpage
\subsection{Jordan-Holder定理}

现在继续讨论单群构建群的思路.这主要包含两个内容:处理合成列,以及处理群的延拓问题.对于第一个内容,我们需要定义一种特殊的子群列,它比正规列的要求更加广泛.称一个子群列$1=N_0\subset N_1\subset\cdots\subset N_r=G$是次正规列,如果满足每个$N_i$都是$N_{i+1}$中的正规子群,其中$0\le i\le n-1$.那么正规列是特殊的次正规列.

和次正规列相关的概念是次正规子群.称群$G$的子群$S$是次正规子群,如果存在$G$的一个有限子群集$\{H_i,0\le i\le r\}$,满足$S=H_0\triangleleft H_1\triangleleft\cdots\triangleleft H_r=G$.这里记号$H\triangleleft G$表示$H$是$G$的正规子群.次正规子群记作$S\triangleleft\triangleleft G$.我们知道正规子群这个性质不具有传递性,即$K\triangleleft H$和$H\triangleleft G$不能推出$K\triangleleft G$.但是次正规子群满足这个性质.

如果$S\triangleleft\triangleleft G$,那么可能存在多个子群列实现次正规性.把其中的最小长度称为子群$S$的次正规长度.这里子群列的长度约定为包含号的个数,也即上述定义中的$r$.那么次正规长度0的子群只有$G$,次正规长度1的子群就是正规子群,次正规长度大于等于2的子群就是那些不是正规子群的次正规子群.

那么首先,次正规性和幂零群有如下联系:一个有限群是幂零群当且仅当它非平凡,并且每个子群都是次正规子群.
\begin{proof}
	
	充分性.我们曾经证明过一个群是幂零群等价于它满足正规化子性质,即对每个真子群$H$有$H$真包含于$N_G(H)$.现在任取真子群$H\subset G$,它是次正规子群,于是有子群列$H=H_0\triangleleft H_1\triangleleft\cdots\triangleleft H_r=G,r>0$.不妨设$H_0$真包含于$H_1$中,那么$H\subset H_1\subset N_G(H)$,于是$G$满足正规化子性质,于是$G$是幂零群.
	
	必要性.假设$G$是幂零群,那么$G$满足正规化子性质.对任意子群$H$,对$[G:H]$归纳证明$H$是次正规子群.如果指数是1,这没什么可证的.现在设$H$是真子群,那么$H$真包含于$N_G(H)$中,于是$N_G(H)$的指数严格小于$H$的指数,由归纳假设有$N_G(H)$是次正规子群,导致$H$是次正规子群.
	
\end{proof}

次正规性的一些性质.
\begin{enumerate}
	\item 和正规子群类似,如果$S$是$G$的次正规子群,$K$是$G$的任意子群,那么$S\cap K\triangleleft\triangleleft K$.特别的,如果$S,T$都是次正规子群,那么$S\cap T$是$T$的次正规子群,按照次正规性的传递性得到$S\cap T$是$G$的次正规子群.
	\item 如果$S,T$都是正规子群,那么$S,T$生成的子群$\langle S,T\rangle$就是$ST$,并且这也是个正规子群.这个结论对于次正规子群仍然成立.【】
	\item 我们知道两个子群$H,K$满足$HK$是子群当且仅当$HK=KH$.倘若$H,K$中有一个是正规子群,那么可交换性总是成立的.一个子群如果和所有子群可交换,就称它是准正规子群.于是可以说正规子群必然是准正规子群.而事实上准正规子群必然是次正规子群.这个结论甚至可以说的更强:如果有限群的子群$S$和自己的全部共轭子群都可交换,那么$S$就是准正规子群.【】
\end{enumerate}

现在继续讨论群的次正规列.对次正规列$1=H_0\subset H_1\subset\cdots\subset H_r=G$,称它的长度为真包含符号的个数,称商群列$\{H_{i+1}/H_i,0\le i\le r-1\}$为这个次正规列的次正规因子.称次正规列$1=H_0\subset H_1\subset\cdots\subset H_m=G$是$1=G_0\subset G_1\subset\cdots\subset G_n=G$的加细,如果$G_0,\cdots,G_n$是$H_0,\cdots,H_m$的子列.如果一个加细仅仅是添加了若干重复的子群,例如$H_i\subset H_{i+1}$加细为$H_i\subset H_i\subset H_{i+1}$,就称是平凡加细,否则称为真加细.

称一个次正规列是极大的,如果它不存在真加细.这个极大性还可以用另一种方式描述,如果群$G$的一个次正规列$\{H_i,0\le i\le r\}$已经是极大的,那么考虑任意一对相邻的子群$H_i\triangleleft H_{i+1}$,如果$H_i\not=H_{i+1}$,那么$H_i$必须是$H_{i+1}$的极大正规子群,这等价于讲$H_{i+1}/H_i$是单群.否则,按照子群对应定理,就有一个$H_{i+1}$的正规子群$K$满足$H_i\subsetneqq K\subsetneqq H_{i+1}$,并且$H_i\triangleleft K$,$K\triangleleft H_{i+1}$.

群$G$的极大次正规列就称为群的\textbf{合成列},于是$1=H_0\subset H_1\subset\cdots\subset H_r=G$是$G$的合成列当且仅当,对每个$0\le i\le r-1$,要么$H_i$是$H_{i+1}$的极大正规子群,要么$H_i=H_{i+1}$,这也等价于讲,对每个$0\le i\le r-1$,$H_{i+1}/H_i$要么是平凡群要么是单群.合成列的次正规因子就称为合成因子.

注意一个群有次正规列不代表它一定有合成列.两个例子:对于有限群,必然总存在合成列,这是因为反复抽取极大正规子群必然在有限步后终止于1,另外按照阶数有限,还可以对每个次正规列反复做真加细,也会在有限步后终止.另一个例子是对于交换群,它的子群都是正规子群,这时候群有合成列当且仅当群是有限群.

称群$G$的两个次正规列是等价的,如果存在非平凡因子之间的双射.\textbf{Jordan-Holder}定理便是说,一个群的两个合成列总是等价的.
\begin{proof}
	
	引理1:Zassenhaus引理,如果$A\triangleleft A',B\triangleleft B'$是四个$G$的子群,那么有$A(A'\cap B)\triangleleft A(A'\cap B')$和$B(B'\cap A)\triangleleft B(B'\cap A')$.并且它们的商是同构的.
	
	按照$A\triangleleft A'$得到$A\triangleleft A'\cap B'$.于是$A\cap B'=A\cap(A'\cap B')\triangleleft A'\cap B'$.同理有$A'\cap B\triangleleft A'\cap B'$.于是$D=(A'\cap B)(A\cap B')$是$A'\cap B'$的正规子群.现在取$x\in B(B'\cap A)$,那么$x=bc,b\in B,c\in B'\cap A'$.定义$f:B(B'\cap A')\to(A'\cap B')/D$为$bc\mapsto cD$.验证定义良性,并且满,并且核为$B(B'\cap A)$.就得到同构$\frac{B(B'\cap A')}{B(B'\cap A)}\cong\frac{B'\cap A'}{D}$,对偶的得到$\frac{B(B'\cap A')}{A(A'\cap B)}\cong\frac{B'\cap A'}{D}$,这就得证.
	
	引理2:Schreier加细定理,同一个群的任意两个正规列,存在它们分别的加细是等价的.
	
	取两个正规列$G=G_0\ge G_1\ge\cdots\ge G_n=1$和$G=H_0\ge H_1\ge\cdots\ge H_m=1$.取$G_{i,j}=G_{i+1}(G_i\cap H_j)$,那么有$G_ {i,j+1}\triangleleft G_{i,j}$.注意到$G_ {i,0}=G_i$,$G_{i,m}=G_{i+1}$.于是得到:
	$$G_{0,0}\ge G_{0,1}\ge\cdots\ge G_{0,m}\ge G_{1,0}\ge\cdots\ge G_{n-1,0}\ge\cdots\ge G_{n-1,m}=1$$
	$$H_{0,0}\ge H_{0,1}\ge\cdots\ge H_{0,m}\ge H_{1,0}\ge\cdots\ge H_{n-1,0}\ge\cdots\ge H_{n-1,m}=1$$
	
	按照引理1有$G_{i,j}/G_{i,j+1}\sim H_{i,j}/H_{i+1,j}$.于是两个加细等价.
	
	现在证明Jordan-Holder定理.给定群$G$的两个合成列,那么按照引理2得到它们有加细是等价的.但是合成列只有平凡加细,于是两个合成列的因子存在双射.
	
\end{proof}

现在我们有了Jordan-Holder定理,可以用长度概念描述合成列.给定一个群$G$,它可能会有很多次正规列,一个次正规列中的真包含号的个数称为这个列的长度,那么一个群的次正规列的长度集合可能有界,也可能无上界.那么一个群有合成列当且仅当这个长度集合有上确界,按照Jordan-Holder定理,长度为这个上确界的次正规列必然是合成列.一个群如果有合成列,就把合成列的长度称为群的长度,记作$l(G)$.

给定群$G$和正规子群$N$,那么$G$有合成列当且仅当$N$和$G/N$都具有合成列,此时$l(G)=l(N)+l(G/N)$,并且$G$的合成因子就是$N$的合成因子和$G/N$的合成因子在计重数意义下的并.
\begin{proof}
	
	按照同态定理,如果$G/N$具有合成列,那么合成列的项对应于$G$的包含$N$的子群,并且具有同构的合成因子.于是如果$N$和$G/N$都有合成列,可以凑出一个$G$的合成列并且合成因子和之前两个合成列的合成因子的并相同.
	
	反过来,如果$1=G_0\subset G_1\subset\cdots\subset G_n=G$是$G$的合成列.得到$N$的子群列$1=N\cap G_0\subset N\cap G_1\subset\cdots\subset N\cap G=G$.并且每个$G_i\cap N$在$G_{i+1}\cap N$中正规,我们断言$\frac{G_{i+1}\cap N}{G_i\cap N}$要么平凡要么是单群.并且后者情况下同构于$G_{i+1}/G_i$.为此只要构造$G_i\cap N\hookrightarrow G_i\twoheadrightarrow\frac{G_i}{G_{i+1}}$,它的核是$G_{i+1}\cap N$,于是$\frac{G_i\cap N}{G_{i+1}\cap N}$是$\frac{G_i}{G_{i+1}}$子群,按照$N$正规得到这是一个正规子群.于是按照$G_i/G_{i+1}$是单群说明前者要么平凡要么就是$G_i/G_{i+1}$自身.
	
	接下来处理$G/N$.注意到$1=\frac{G_0N} {N}\subset\frac{G_1N}{N}\subset\cdots\subset\frac{G_nN}{N}=G/N$.每个$\frac{G_iN}{N}$是$\frac{G_{i+1}N}{N}$的正规子群.并且类似可以证明存在$G_{i+1}/G_i$到$(G_{i+1}N)/(G_iN)$的群同态,于是次正规因子要么平凡要么是单群,于是$G/N$有合成列.
\end{proof}

注意Jordan-Holder定理只说明了同一个群对应固定的一列合成因子,为了使单群构建群的思路成立,还需要两个群的合成因子相同则它们同构.C.Jordan在1868年证明了这对有限群成立,之后O.Holder于1889年证明了这对任意群成立.

从有限群在同构意义下唯一对应一组合成因子,还可以得到算术基本定理.考虑正整数$n$,记$n$唯一分解为$n=p1p2\cdots p_r$.其中$p_i$是可能相同的素数.记$G=\mathbb{Z}/n=\langle x\rangle$,那么按照素数阶循环群是单群,得到$1\subset(x^{p1p2\cdots p_r})\subset(x^{p1p2\cdots p_{r-1}})\subset\cdots\subset(x^1)=G$是以列$\{p_r,p_{r-1},\cdots,p_1\}$为合成列的群,于是$n$唯一的被计重数意义的这些素数所决定.
\newpage
\subsection{可解群}

本节介绍可解群.称一个有限群是可解列,如果它存在一个次正规列,它的所有次正规因子都是交换群.那么首先,按照Jordan-Holder定理,可以直接得到如下条件都是可解群的等价描述:
\begin{enumerate}
	\item $G$的所有合成因子都是素数阶循环群.
	\item $G$存在次正规列,它的所有次正规因子都是交换群.
\end{enumerate}

$G$是可解群则所有商和所有子群都是可解群,如果$G$有正规子群$N$,那么$G$是可解群当且仅当$N$和$G/N$是可解群.这从$G$的合成因子就是$N$和$G/N$的合成因子在计重数意义下的并直接得出.特别的,可解群的有限直和是可解群.

$p$群是可解群.这结合可解群有限直和是可解群,得到有限幂零群都是可解群.
\begin{proof}
	
	对$|G|=p^n$的$n$归纳.当$n=1$时结论是平凡的.现在假设$n>1$,那么$G$的中心$Z(G)$非平凡,而$G/Z(G)$是一个次数严格小于$p^n$的$p$群,于是归纳假设得到$G/Z(G)$是可解群.又有$Z(G)$交换群,合成因子必然都交换,于是是可解群,就得到$G$是可解群,完成归纳.
\end{proof}

$S_n,n\ge 5$不可解.由于$A_n,n\ge5$是非循环群的单群,于是$S_n\supsetneq A_n\supsetneq\langle e\rangle$是$S_n$的合成列,于是它不可解.

尽管可解群的定义中用的是次正规列.存在一个和幂零群的上中心列类似的正规列,这种正规列完全描述了可解性.它就是高阶导群构成的导出列.我们曾介绍过导群的概念,并且强调过导群的泛映射性质转化为群论语言就是在说,$G/H$是交换群当且仅当$H$包含了导群$G'$.

记$G'$的导群是$G''$,称为$G$的二阶导群,归纳定义$n$阶导群$G^{(n)}$是$n-1$阶导群的导群.按照导群总是特征子群,以及特征子群的传递性,得到$G^{(n)}$都是$G$的特征子群,这就得到了一个子群列,其中每个子群都是正规子群:
$$G=G^{(0)}\supset G^{(1)}\supset G^{(2)}\supset\cdots$$

如果上述子群列在有限项终止于1,就称这个正规列为导出列.$G$是可解群就等价于$G$有导出列.
\begin{proof}
	
	一方面,如果群$G$有导出列,那么导出列本身就是一个(次)正规列.它的正规因子是$G^{(i)}/G^{(i+1)}$,按照导群的性质,这个商是交换群,于是$G$是可解群.
	
	另一方面,如果$G$是可解群.任取一个正规因子都是交换群的正规列$G=G_0\ge G_1\ge\cdots\ge G_n=1$.我们断言$i$阶导群$G^{(i)}$包含于$G_i$,这就说明高阶导群在有限项后终止于1.对$i$归纳,由于$G/G_1$是正规因子,它是交换群,于是$G'\subset G_1$.现在假设$G^{(i)}\subset G_i$.那么有$G^{(i+1)}=(G^{(i)})'\subset (G_i)'$.由于$G_i/G_{i+1}$是交换群,说明$(G_i)'\subset G_{i+1}$,这就得到$G^{(i+1)}\subset G_{i+1}$,完成归纳.
\end{proof}

可解群不总是幂零群.按照幂零群的中心非平凡,于是反例可以考虑中心平凡的并且满足$G=G'$的群,例如$S_3$.

可解群的极小正规子群.称有限群$G$的极小正规子群是一个非${e}$的正规子群$M$,并且$M$没有除了自身的子群也是$G$的正规子群.

那么极小正规子群满足一个称为特征单的性质.群$G$称为特征单的,如果$G$的特征子群只能是${e}$和$G$.这个性质是直接的,因为极小正规子群的特征子群必然是原群的正规子群.有限特征单群具有描述:是一组同构单群的直积,并且这个单群可以取为极小正规子群.
\begin{proof}
	
	设$G$是有限的特征单群.设$S$是$G$的极小正规子群.考虑集合$\mathscr{D}=\{N\triangleleft G\mid N=S_1\times\cdots S_k,\text{其中}S_i\text{都是}G\text{的同构于}S\text{的极小正规子群}\}$.那么$S$已经在$\mathscr{D}$中,于是$\mathscr{D}$是一个非空的集合.按照$G$是有限群说明$\mathscr{D}$是有限集,于是可以取到它在包含序下的极大元$N$.
	
	我们断言$N=G$.假设$N\subsetneqq G$,那么$N$不会是一个特征子群,于是存在$G$上的自同构$\phi$使得$\phi(N)\not\subset N$.记$N=S_1\times S_2\times\cdots\times S_k$,其中每个$S_i$都是同构于$S$的极小正规子群.于是至少存在一个$S_i$满足$\phi(S_i)\not\subset N$.由于$\phi$是$G$的自同构,说明$\phi(S_i)$也是一个同构于$S$的极小正规子群.现在$N\cap\phi(S_i)$是$G$的正规子群,并包含于$\phi(S_i)$中,于是按照极小性得到$N\cap\phi(S_i)={e}$.这就导致$N\cdot\phi(S_i)=S_1\times S_2\times\cdots\times S_k\times\phi(S_i)$是内直和,并且是$G$的正规子群,这就和$N$在$\mathscr{D}$中的极大性矛盾.于是$N=G$.
\end{proof}

给定素数$p$,称群$G$是初等交换$p$群,如果它是交换群并且满足$x^p=1,\forall x\in G$.这等价于讲$G\cong (\mathbb{Z}/p)^r$再结合可解群是合成因子都是素数阶循环群的群,就得到:有限可解群的极小正规子群是一个初等交换$p$群.

一个无限可解群的例子.给定域$k$,考虑由全体上三角矩阵构成的$\mathrm{GL}_n(k)$的子群$G$,那么$G$是一个可解群.
\begin{proof}
	
	先来证明$G$的一节导群(此即换位子群,即全体换位子$[g,h]=ghg^{-1}h^{-1}$生成的$G$的子群)中的矩阵$A=(a_{ij})$总满足$a_{i,i+1}=0,\forall 1\le i\le n-1$.为此任取两个上三角可逆矩阵$A=(a_{ij})$和$B=(b_{ij})$.记$A^{-1}=(a^{ij})$和$B^{-1}=(b^{ij})$,那么有$\sum_{k=1}^na_{ik}a^{kj}=\sum_{k=1}^na^{ik}a_{kj}=\delta_{ij}$.于是$AB$的$(i,j)$项是$\sum_{k_1=1}^na_{i,k_1}b_{k_1,j}$,$ABA^{-1}$的$(i,j)$项是$\sum_{k_2=1}^n\sum_{k_1=1}^na_{i,k_1}b_{k_1,k_2}a^{k_2,j}$,于是$ABA^{-1}B^{-1}$的$(i,i+1)$项是$\sum_{k_3=1}^n\sum_{k_2=1}^n\sum_{k_1=1}^na_{i,k_1}b_{k_1,k_2}a^{k_2,k_3}b^{k_3,i+1}$.按照矩阵均为上三角的,于是不妨设$i+1\ge k_3\ge k_2\ge k_1\ge i$,于是这个和式实际上是:
	$$a_{i,i}b_{i,i}a^{i,i}b^{i,i+1}+a_{i,i}b_{i,i}a^{i,i+1}b^{i+1,i+1}+a_{i,i}b_{i,i+1}a^{i+1,i+1}b^{i+1,i+1}+a_{i,i+1}b_{i+1,i+1}a^{i+1,i+1}b^{i+1,i+1}$$
	
	按照矩阵都是上三角矩阵,得到$a_{i,i}a^{i,i}=b_{i,i}b^{i,i}=1$,以及$a_{i,i}a^{i,i+1}+a_{i,i+1}a^{i+1,i+1}=0$.于是上式就等于
	$$b_{i,i}b^{i,i+1}+a_{i,i}b_{i,i}a^{i,i+1}b^{i+1,i+1}+a_{i,i}b_{i,i+1}a^{i+1,i+1}b^{i+1,i+1}+a_{i,i+1}a^{i+1,i+1}$$
\end{proof}





\newpage
\subsection{Hall定理}

在有限可解群上,Sylow定理可以推广为Hall定理.有限群$G$的子群$H$称为Hall子群,如果$|H|$和$[G:H]$互素.当$H$满足这个条件时,把$|H|$的所以素因子构成的集合记作$\pi$,称$H$是一个Hall-$\pi$子群.特别的,如果$\pi$取一个整除$|G|$的单个素数构成的集合,那么Hall-$\pi$子群就是Sylow子群.

考虑交错群$A_5$,$|A_5|=2^2\cdot3\cdot5$.那么Hall-$\{2,3\}$子群的阶数是12,于是$A_4$是$A_5$的一个Hall-$\{2,3\}$子群.现在考虑Hall-$\{2,5\}$子群和Hall-$\{3,5\}$子群,它们分别具有阶数20和15,分别具有指数3和4.倘若存在子群$H$是这两个阶数中的一个,那么把$A_5$左平移作用到$H$全部左陪集上,就得到了群作用$\rho:A_5\to S_r$,其中$r$为3或4.$\rho$不会是平凡作用,另外$|S_4|=24<60$,说明这是个满同态,于是核非平凡,是一个非平凡的极大理想,这和$A_5$是单群矛盾.

于是对于有限群,当素数集$\pi$整除$|G|$的时候,Hall-$\pi$子群可能存在也可能不存在.而对于有限可解群,$\pi$如果整除$|G|$,那么Hall-$\pi$子群总是存在的:

Hall定理.$G$是有限可解群,$\pi$是整除$|G|$的一个素数的集合,那么:
\begin{enumerate}
	\item $G$存在Hall-$\pi$子群.
	\item $G$的任意两个Hall-$\pi$子群是共轭的.
	\item 任意的$G$的$\pi$子群,即阶数唯一分解后出现的素数都在$\pi$中的子群,必然包含于某个Hall-$\pi$子群中.
\end{enumerate}

下述证明用到两个我们证明过的定理.回顾Frattini命题是指,对有限群$G$,任取正规子群$N$,设$G$的一个Sylow子群为$P$,那么有$G=N_G(P)N$.另外回顾我们证明过有限可解群的极小正规子群是初等交换$p$群.

\begin{proof}
	
	我们的证明思路和Sylow定理一样,先证明存在性定理,接下来通过证明每个$\pi$子群都在某个Hall-$\pi$子群的共轭之中,得到第三个命题.最后在这个命题中把$\pi$子群取为Hall-$\pi$子群,就得到两个同阶Hall-$\pi$子群都是共轭的.即我们需要证明如下两个命题:
	\begin{enumerate}
		\item $G$有Hall-$\pi$子群$H$.
		\item 如果$L$是$G$的$\pi$子群,那么$L$包含于$H$的某个共轭中.
	\end{enumerate}
	
	我们通过对$|G|$归纳来证明这两个命题.对$|G|=1$两个结论都是平凡的.现在假设$|G|>1$,并且对阶数小于$|G|$的所有有限可解群两个命题均成立.设$|G|=mn$,其中$m$唯一分解中只有$\pi$中素数,$n$唯一分解中没有$\pi$中素数.于是一个Hall-$\pi$子群就是阶数为$m$的子群.于是不妨设$m>1$,否则结论也是平凡的.
	
	现在取$G$的一个极小正规子群$M$,那么它是初等交换$p$群,现在考虑如下两种情况:$p\in\pi$;不存在极小正规子群是初等交换$p$群,其中$p\in\pi$.
	
	情况1.此时记$|M|=p^{\alpha}$.那么有$|G/M|=mn/p^{\alpha}=m_1n$.按照归纳假设,$G/M$满足两个命题.于是存在$G/M$的子群$H/M$的阶数是$m_1$,于是$H$就是$G$的一个Hall-$\pi$子群.现在任取$G$的$\pi$子群$L$,按照第二同构定理,$LM/M\cong L/L\cap M$是$G/M$的$\pi$子群,于是存在$H/M$的共轭包含了$LM/M$,\\
	即$LM/M\subset(H/M)^{xM}=H^x/M$,这就导致$L\subset LM\subset H^x$.
	
	情况2.取极小正规子群$M$满足$|M|=q^{\beta}$,其中素数$q\not\in\pi$.那么$|G/M|=mn/q^{\beta}=mn_1$,其中$n=n_1q^{\beta}$.现在分$n_1=1$和$n_1>1$两种情况.
	
	情况2a.如果$n_1>1$,按照归纳假设$G/M$有Hall-$\pi$子群$K/M$,其中$K$是包含$M$的子群,并且$|K/M|=m$.于是$|K|=mn/n_1<mn$.再按照归纳假设,$K$具有Hall-$\pi$子群$H$,并且$|H|=m$,于是$H$就是$G$的一个Hall-$\pi$子群.再任取$G$的$\pi$子群$L$,那么$LM/M\cong L/M\cap L$是$G/M$的$\pi$子群,于是按照归纳假设有$LM/M\subset(K/M)^{xM}=K^x/M$,于是$L\subset LM\subset K^x$,导致$L^{x^{-1}}\subset K$,而$L^{x^{-1}}$是$K$的$\pi$子群于是归纳假设得到$L^{x^{-1}}\subset H^y$,于是$L\subset H^{yx}$.
	
	情况2b.如果$n_1=1$,此时$|G|=mq^{\beta}$.并且极小正规子群$M$只能是初等交换$q$群.现在$|G/M|=m>1$,取$G/M$的极小正规子群$N/M$,那么$N/M$是初等交换$p$群,其中$p\in\pi$.记$|N/M|=p^{\alpha}$,现在$N\triangleleft G$,并且有$|N|=p^{\alpha}q^{\beta}$.取$N$的Sylow-$p$子群$P$,按照Frattini命题,得到$G=N_G(P)N$.又从$N=PM$,得到$G=N_G(P)PM=N_G(P)M$.
	
	现在考虑$J=N_G(P)\cap M$,由于$M$是交换群,得到$J\triangleleft M$,另外$M\triangleleft G$,并且$J\triangleleft N_G(P)$,就得到$J\triangleleft N_G(P)M=G$.但是由于$M$已经是极小正规子群,说明$J={e}$或者$J=M$.
	
	如果$J=M$,等价于$M\subset N_G(P)$,于是$G=N_G(P)M=N_G(P)$,于是$P$是$G$的正规$p$子群,于是存在$P$的某个子群是$G$的极小正规子群,这个群还是初等交换$p$群,其中$p\in\pi$,这和我们预先约定的不存在$G$的极小正规子群是初等交换$p$群,其中$p\in\pi$相矛盾.
	
	于是只能有$J={e}$,即$N_G(P)\cap M={e}$.于是$mq^{\beta}=|G|=|N_G(P)|\cdot|M|$,导致$|N_G(P)|=m$,于是$H=N_G(P)$就是我们要找的Hall-$\pi$子群.这就完成了存在性部分的归纳证明.
	
	接下来任取$G$的$\pi$子群$L$,那么从$G=HM$得到$LM=LM\cap G=LM\cap HM=(LM\cap H)M$.而$LM\cap H$是$\pi$子群,并且有$[LM:LM\cap H]=|M|$.于是$LM\cap H$是$LM$的Hall-$\pi$子群.如果$LM$是$G$的真子群,按照归纳假设,$L$作为$LM$的$\pi$子群,存在$LM\cap H$在$LM$中的共轭包含了$L$,即$L\subset(LM\cap H)^x\subset H^x$其中$x$是$LM$中的某个元.这就完成归纳.
	
	最后只剩下情况$LM=G$.按照$L$和$M$的阶数互素,得到$L\cap M={e}$.于是$|G|=|L|\cdot|M|$,于是$|L|=m$.再按照$M\subset N$,这里$N$是满足$N/M$是$G/M$极小正规子群的子群.于是$G=LN$,那么$|G|=|LN|=\frac{|L|\cdot|N|}{|L\cap N|}$.于是$|L\cap N|=p^{\alpha}$.即$L\cap N$是$N$的一个Sylow-$p$子群,按照Sylow定理,它共轭于$P$,于是得到$L\cap N=P^x$其中$x$是某个$N$中元,再从$L\cap N\triangleleft L$,得到$L\subset N_G(L\cap N)=N_G(P^x)=N_G(P)^x=H^x$.于是$L$包含在$H$的某个共轭中,至此完成证明.
\end{proof}

Hall子群存在定理的逆命题同样成立,即一个有限群,如果对任意整除阶数的一组素数构成的集合$\pi$,总有Hall-$\pi$子群存在,那么群是可解群.这个逆命题能够说明Burnside定理成立(但是我们在下面证明中是用到Burnside定理的),即如果群的阶数是$p^aq^b$,其中$p,q$是不同素数幂,那么群是可解群,因为此时非平凡的Hall子群就是Sylow子群,这是总存在的.并且Burnside定理结合可解群是单群当且仅当它是素数阶循环群,说明$p^aq^b$阶群必然不是单群,也即有限阶非交换单群至少存在三个不同素数.另外逆命题还说明了Hall子群存在是有限可解群的一个等价描述.

对有限群$G$,取素数$p$整除$|G|$,那么就有$|G|=p^{a}m$,称$G$的阶数$m$的Hall子群为$G$的$p$补子群.称为补群的理由是,取有限群$G$的Sylow-$p$子群$P$,那么$|G|=p^am$,其中$p$不整除$m$.那么一个$p$补子群就是阶数为$m$的Hall子群$H$.按照$P$和$H$的阶数互素,立刻得到$|G|=|H|\cdot|P|$,于是$G=HP$并且$H\cap P={e}$.

现在设$G$是有限可解群,记$|G|=p_1^{n_1}\cdots p_k^{n_k}$.其中$p_i$是两两不同的素数,并且$n_i>0$.按照Hall定理,可取$Q_i$为$p_i$补子群.那么$|Q_i|=|G|/p_i^{n_i}$,并且$[G:Q_i]=p_i^{n_i}$.

我们断言$Q_1\cap Q_2\cap\cdots\cap Q_t$是Hall-$\{p_{t+1},\cdots,p_k\}$子群.
\begin{proof}
	
	证明中我们会用到:如果$[G:H]$和$[G:K]$互素,那么$[G:H\cap K]=[G:H][G:K]$.
	
	对$t=1$自然是成立的,现在假设对$t$成立,即$H=\cap_{1\le i\le t}Q_i$是Hall-$\{p_{t+1},\cdots,p_k\}$子群.那么$|H|=\prod_{t+1\le i\le k}p_i^{n_i}$,并且$[G:H]=\prod_{1\le i\le t}p_i^{n_i}$.按照$H$和$Q_{t+1}$有互素的指数,于是$[G:H\cap Q_{t+1}]=\prod_{1\le i\le t+1}p_i^{n_i}$.于是$|H\cap Q_{t+1}|=\prod_{t+2\le i\le k}p_i^{n_i}$,导致$H\cap Q_{t+1}=Q_1\cap\cdots\cap Q_{t+1}$是Hall-$\{p_{t+2},\cdots,p_k\}$子群.
\end{proof}

特别的,$P_i=\cap_{j\not=i}Q_j$是$G$的Sylow-$p_i$子群.任取$i\not=j$,那么$P_i,P_j$满足交是1,于是$|P_iP_j|=|P_jP_i|=p_i^{n_i}p_j^{n_j}$,另外$\cap_{t\not=i,j}Q_t$包含了$P_i$和$_j$,并且阶数和$P_iP_j$与$P_jP_i$相同,这就说明$P_iP_j=P_jP_i=\cap_{t\not=i,j}Q_t$.于是这些$P_i$满足了两两可交换.

给定有限群$G$,设$|G|$的全部不同素因子为$p_1,p_2,\cdots,p_k$.称$G$的一个Sylow系统为子群族$\{Q_1,$$Q_2,\cdots,Q_k\}$,其中$Q_i$是一个$p_i$补子群.称一组Sylow基是子群族$\{P_1,P_2,\cdots,P_k\}$,其中$P_i$是Sylow-$p_i$子群,并且$P_i$两两可交换.

那么我们已经证明了,任意有限可解群具有Sylow系统和Sylow基.Sylow基的意义在于,由于两个子群的乘积是子群当且仅当它们可交换,所以Sylow基的可交换条件保证了$P_{i_1}P_{i_2}\cdots P_{i_s}$总是一个子群,并且它的阶数就是$p_{i_1}^{n_{i_1}}p_{i_2}^{n_{i_2}}\cdots p_{i_s}^{n_{i_s}}$.这就是一个Hall-$\{i_1,i_2,\cdots,i_s\}$子群.

有限可解群上不同Sylow基之间是共轭关系.即如果$G$是有限可解群,那么任取两组Sylow基$\{P_1,P_2,\cdots,P_k\}$和$\{R_1,R_2,\cdots,R_k\}$,那么存在同一个元$x\in G$,使得$R_i=P_i^x,\forall 1\le i\le k$成立.【】

这里我们证明一个比Hall子群存在定理的逆命题更强的结论:一个有限群,如果对任意阶数的素因子$p$,存在$p$补子群,那么$G$是可解群.
\begin{proof}
	
	假设存在有限阶非可解群满足这个命题,取$G$是阶数最小的一个例子.如果$G$有非平凡的正规子群$N$,再设$H$是$G$的一个$p$补子群,那么$H\cap N$是$N$的一个$p$补子群,并且$HN/N\cong H/H\cap N$是$G/N$的$p$补子群.它们都满足命题条件,由于$N$和$G/N$的阶数严格小于$|G|$,于是它们都是可解群,这导致$G$可解,矛盾.
	
	于是得到了$G$是单群.设$|G|=p_1^{e1}\cdots p_n^{e_n}$.其中$p_i$是两两不同 素数,并且$e_i>0,\forall i$.对每个$i$,按照条件可设$G$存在$p_i$补子群$H_i$.那么每个$H_i$的指数就是$p_i^{e_i}$.取$D=H_3\cap\cdots\cap H_n$,那么$[G:D]=\prod_{3\le i\le n}p_i^{e_i}$,于是$|D|=p_1^{e_1}p_2^{e_2}$.按照Burnside定理,这是一个可解群.取$D$的极小正规子群$N$,那么$N$是一个初等交换$p$群,不妨设$p=p_1$,于是$[G:D\cap H_2]=\prod_{2\le i\le n}p_i^{e_i}$.于是$|D\cap H_2|=p_1^{e_1}$.于是$D\cap H_2$是$D$的Sylow-$p_1$子群.于是有$N\subset D\cap H_2\subset H_2$.类似有$|D\cap H_1|=p_2^{e_2}$,而$H_2$和$D\cap H_1$的阶数互素,导致$G=H_2(D\cap H_1)$.于是对$g\in G$有$g=hd$,其中$h\in H_2$,$d\in D\cap H_1$.对$x\in N$,有$x^g=hdxd^{-1}h^{-1}=y^h$,其中$y=x^d\in N$,于是$y^h\in H_2$,导致$H$在$G$中生成的正规子群在$H_2$中,因为$H_2$真包含于$G$,就得到了$G$的一个真正规子群,和$G$是单群矛盾.
\end{proof}
\newpage
\section{拓扑群}
\subsection{基本内容}

一个群$G$称为拓扑群,如果它是拓扑空间范畴中的群对象,此即它同时是拓扑空间和群,并且赋予的拓扑使得$(g,h)\mapsto gh$是$G\times G\to G$的连续映射,并且$g\mapsto g^{-1}$是$G\to G$的连续映射.
\begin{enumerate}
	\item 一些简单性质.
	\begin{enumerate}[(1)]
		\item 任取$g\in G$,那么无论左乘$g$还是右乘$g$都是$G$上的自同胚.于是$U\subset G$是开子集当且仅当$gU$是开子集,当且仅当$Ug$是开子集,当且仅当$U^{-1}$是开子集.
		\item 如果$K_1,K_2$是$G$的两个紧子集,那么$K_1K_2$也是紧子集.
		\begin{proof}
			
			$K_1\times K_2$是$G\times G$的紧子集,而$K_1K_2$是该紧子集在$G\times G\to G$下的连续像,于是是紧的.
		\end{proof}
		\item 对称开邻域.拓扑群$G$的子集$S$称为对称的,如果$S=S^{-1}$.我们断言$e$的任意开邻域$U$总包含了一个对称开邻域$V$使得$VV\subset U$.
		\begin{proof}
			
			按照典范映射$\phi:U\times U\to G$,$(g,h)\mapsto gh$是连续映射,导致$\phi^{-1}(U)$是$U\times U$的开子集,并且这个原像集包含了点$(e,e)$.于是存在$e$的开邻域$V_1,V_2\subset U$使得$(e,e)\in V_1\times V_2\subset\phi^{-1}(U)$,换句话讲$V_1V_2\subset U$.再记$V_3=V_1\cap V_2$,那么$V_3$是$e$的开邻域,并且$V_3V_3\subset U$.最后取$V=V_3\cap V_3^{-1}$满足要求.
		\end{proof}
	\end{enumerate}
    \item 拓扑群的子群.
	\begin{enumerate}[(1)]
		\item 拓扑群$G$的开子群都是闭子群,拓扑群$G$的有限指数闭子群都是开子群.一个子群$H\le G$是开子群当且仅当$H$的内点非空.另外如果$U$是$e$的对称开邻域,那么$H=\cup_{n\ge1}U^n$是$G$的开子群.
		\begin{proof}
			
			设$H$是拓扑群$G$的开子群,那么每个陪集$gH$都是开子集,于是$Y=\cup_{g\in G-H}gH$是开子集,它的补集$H$是闭子集.第二个命题证明类似.
		\end{proof}
		\item 拓扑群$G$的子群$H$的拓扑闭包$\overline{H}$也是$G$的子群.
		\begin{proof}
			
			任取$g,h\in\overline{H}$,任取$gh$的开邻域$U$,记典范映射$\phi:G\times G\to G$,那么$\phi^{-1}(U)$是$G\times G$的包含$(g,h)$的开子集.于是可取$g$和$h$的开邻域$V_1,V_2$使得$V_1\times V_2\subset\phi^{-1}(U)$.按照闭包的定义,可取$x\in V_1\cap H$和$y\in V_2\cap H$.于是$xy\in U\cap H$.于是我们证明了$gh\in\overline{H}$.
			
			再记$\tau:G\to G$为$g\mapsto g^{-1}$,任取$h\in\overline{H}$和$h^{-1}$的开邻域$W$,那么$\tau^{-1}(W)=W^{-1}$是包含$h$的开邻域,按照闭包定义有$g\in H\cap W^{-1}$,于是$g^{-1}\in H\cap W$,于是$g^{-1}\in\overline{H}$.
		\end{proof}
	\end{enumerate}
    \item 分离公理.设$G$是拓扑群.
    \begin{enumerate}[(1)]
    	\item $T_0$和$T_1$等价.按照齐次性,$T_1$等价于至少有一个点是闭点.
    	\begin{proof}
    		
    		设$x\not=y\in G$,不妨设有$x$的开邻域$U$满足$y\not\in U$.那么$xyU^{-1}$是$y$的开邻域,并且不包含$x$.所以固定点$x$,任取$y\not=x$,那么总可以找到$y$的开邻域不包含$x$,对$y\not=x$的点取并就得到$G-\{x\}$是开集,于是$\{x\}$是闭集.
    	\end{proof}
    	\item $T_1$和$T_2$等价.
    	\begin{proof}
    		
    		任取$g\not=h\in G$,于是可选取$e$的开邻域$U$使得$gh^{-1}\not\in U$.可选取1的对称开邻域$V$使得$VV\subset U$.那么$Vg$是包含$g$的开邻域,$Vh$是包含$h$的开邻域,我们断言有$Vg\cap Vh=\emptyset$,否则存在$v_1,v_2\in V$使得$v_1g=v_2h$,导致$gh^{-1}=v_1^{-1}v_2\in U$矛盾.
    	\end{proof}
    	\item $T_1$和$T_3$等价,这里$T_3$是指$T_1$且正则,我们只需证明拓扑群总满足正则条件,即任取点$x$的开邻域$U$,那么存在$x$的开邻域$V$使得$x\in V\subseteq\overline{V}\subseteq U$.
    	\begin{proof}
    		
    		按照齐次性,只需证$x=e$的情况即可.此时$U$是$e$的开邻域,可取$e$的开邻域$V$使得$VV\subseteq U$,我们断言$\overline{V}\subseteq VV$从而完成证明.任取$x\in\overline{V}$,那么$xV^{-1}\cap V$非空,此即$x\in VV$,于是$\overline{V}\subseteq VV\subseteq U$.
    	\end{proof}
    	\item 其实$T_0$甚至等价于完全正则$(T_{3,5})$的.
    \end{enumerate}
    \item 连通分支.设$G$是拓扑群,设$G^0$是幺元所在的连通分支,那么$G^0$是$G$的正规子群,并且$G$关于$G^0$的陪集分解就是$G$的连通分支分解.
    \begin{proof}
    	
    	$G_0$是$G$的正规子群:任取$x\in G_0$,那么$x^{-1}G_0$也是一个包含$x$的连通子集,于是$x^{-1}G_0\subseteq G_0$,也即$G_0^{-1}\subseteq G_0$.于是$xG_0$也是包含$e$的连通子集,进而有$xG_0\subseteq G_0$,也即$G_0G_0\subseteq G_0$,这说明$G_0$是子群.同样的任取$y\in G$有$yG_0y^{-1}$是$e$的连通子集,导致$yG_0y^{-1}\subseteq G_0$,于是$G_0$是正规子群.
    \end{proof}
    \item 商空间.设$G$是拓扑群,设$H\le G$是子群,记$G/H$表示$H$全体左陪集构成的集合,那么$G$诱导了$G/H$上的拓扑,也即最细拓扑使得典范映射$\rho:G\to G/H$是连续的.子集$U\subseteq G/H$是开集当且仅当$\rho^{-1}(U)$是$G$的开集.
    \begin{enumerate}[(1)]
    	\item 典范映射$\rho:G\to G/H$是开映射.
    	\begin{proof}
    		
    		任取开集$V\subseteq G$,要证明$\rho(V)$是开集,等价于$\rho^{-1}(\rho(V))$是$G$的开集.但是$\rho^{-1}(\rho(V))=VH=\cup_{h\in H}Vh$是开集的并.
    	\end{proof}
    	\item $G/H$是$T_1$的当且仅当$H$是闭集.
    	\begin{proof}
    		
    		因为$G/H$是$T_1$空间当且仅当每个$gH$是闭点,当且仅当$\rho^{-1}(gH)=gH$都是$G$的闭集,当且仅当$H$是闭子群.
    	\end{proof}
    	\item $G/H$是离散空间当且仅当$H$是开集.另外如果$G$是紧集,那么$H$是开集当且仅当$G/H$是有限集.
    	\begin{proof}
    		
    		$G/H$是离散空间当且仅当每个点$gH$都是单点开集,当且仅当每个$\rho^{-1}(gH)=gH$是$G$的开集,当且仅当$H$是开子群.另外如果$G$是紧集,那么$H$是开集当且仅当$G/H$是离散空间,当且仅当$\{gH\}$是$G$的无交开集,并是整个$G$,紧性就导致它是有限集.
    	\end{proof}
    	\item 如果$H$是$G$的正规子群,那么$G/H$的拓扑使得它构成拓扑群.
    	\begin{proof}
    		
    		就是验证二元乘法和取逆映射都是连续的,以二元乘法为例.设$g_1H,g_2H\in G/H$,任取$g_1g_2H\in G/H$的开邻域$U$,我们的目标是找$g_1H\in G/H$和$g_2H\in G/H$的开邻域$W_1,W_2$,使得$W_1W_2\subseteq U$.我们有$\rho^{-1}(U)$是$g_1g_2\in G$的开邻域,所以按照$G$上二元乘法的连续性,有$g_1$的开邻域$V_1$和$g_2$的开邻域$V_2$,使得$V_1V_2\subseteq\rho^{-1}(U)$.取$W_1=\rho(V_1)=V_1H\in G/H$和$W_2=\rho(V_2)=V_2H\in G/H$,它们分别是$g_1H$和$g_2H$的开邻域,并且有$W_1W_2=V_1V_2H\subseteq U$.
    	\end{proof}
    	\item 设$\{e\}\subseteq G$的闭包是$H$,那么$H$是$G$的正规子群,并且$G/H$是Hausdorff的.
    	\begin{proof}
    		
    		我们解释过子群的闭包还是子群,所以这里$H$是闭子群.另外任取$g\in G$,那么$gHg^{-1}$也是包含$\{e\}$的闭集,按照闭包的定义有$H\subseteq gHg^{-1}$,也即$H=gHg^{-1},\forall g\in G$,也即$H$是正规子群.最后我们解释过$H$是闭子群时$G/H$是Hausdorff的.
    	\end{proof}
    	\item 设$G$是Hausdorff拓扑群,设$H\le G$是紧子群,那么$\rho:G\to G/H$是闭映射.这里$H$紧性是必须的,取$G=(\mathbb{R}^2,+)$,取子群$H=\{(0,y)\mid y\in\mathbb{R}\}$,那么$G/H\cong\mathbb{R}$,商映射是$\rho(x,y)=x$.取闭集$X=\{(x,y)\in\mathbb{R}^2\mid xy=1\}$,它在商群中的像是$\rho(X)=\mathbb{R}^*$不是闭集.
    	\begin{proof}
    		
    		对于Hausdorff拓扑群$G$我们有如下事实:一个闭子集和一个紧子集的乘积总是闭的.任取$G$的闭子集$X$,它在$G/H$中的像是$XH$,要证它是$G/H$的闭集,只需验证$\rho^{-1}(XH)=XH$是$G$的闭子集,而这由我们指出的命题是成立的.
    	\end{proof}
    \end{enumerate}
    \item 局部紧群.一个拓扑群称为局部紧群,如果它的每个点都存在紧邻域,并且是Hausdorff的.设$G$是Hausdorff拓扑群,设$H\le G$是局部紧的子群,那么$H$是闭子群.特别的这说明$G$的离散子群都是闭的.这件事纯粹是拓扑的:设$X$是Hausdorff空间,设$Y\subseteq X$是局部紧子集,证明$Y$是局部闭子集.
    \begin{proof}
    	
    	任取$x\in Y$,存在$x$在$Y$中的紧子集$K_x$,于是存在$x$在$X$中的开邻域$U_x$使得$U_x\cap Y\subseteq K_x$.取$U=\cup_{x\in Y}U_x$,我们断言$U\cap\overline{Y}=Y$.
    	
    	\qquad
    	
    	$Y\subseteq U\cap\overline{Y}$是平凡的.证明另一侧包含归结于证明对每个$x\in Y$有$U_x\cap\overline{Y}\subseteq Y$,为此我们证明$U_x\cap\overline{Y}\subseteq K_x$即可.
    	
    	\qquad
    	
    	任取$z\in U_x\cap\overline{Y}$,任取$z$在$X$中的开邻域$V$,那么$V\cap U_x\cap Y$非空,进而有$V\cap K_x$非空,于是$z\in\overline{K_x}=K_x$,也即$U_x\cap\overline{Y}\subseteq K_x$.
    \end{proof}
\end{enumerate}
\newpage
\subsection{Haar测度}

设$X$是拓扑空间,它的全部开集生成的最小的$\Sigma$代数记作$\mathscr{B}$,其中的集合称为Borel集,
\begin{itemize}
	\item $X$上的一个Borel测度$\mu$是指定义在$\mathscr{B}$上的正测度.
	\item $X$上的一个Borel测度$\mu$称为在可测集$E$上外正则,如果$\mu(E)=\inf\{\mu(U)\mid E\subseteq U,U\text{是开集}\}$;称$\mu$在可测集$E$上内正则,如果$\mu(E)=\sup\{\mu(K)\mid K\subseteq E,K\text{是紧集}\}$.
	\item 称一个Borel测度是Radon测度,如果它在紧集上测度有限,在所有Borel集合上外正则,在所有开集上内正则(进而有它在$\sigma$有限集合,也即$\mu$测度有限集的可数并,上内正则).
	\item 设$G$是局部紧拓扑群,它的左或者右Haar测度指的是$G$上的非零Radon测度$\mu$,满足左平移不变或者右平移不变.这里左平移不变指的是对任意Borel集$E$,有$\mu(gE)=\mu(E),\forall g\in G$.如果同时满足左右平移不变就称它是双边Haar测度.
\end{itemize}

拓扑群上函数的一致连续性.
\begin{enumerate}
	\item 设$f$是拓扑群$G$上的实或复值函数,对元素$h\in G$,定义$f$的左变换$L_hf$和右变换$R_hf$为$L_hf(g)=f(h^{-1}g)$和$R_hf(g)=f(gh)$.称连续函数$f$是左一致连续的,如果对任意$\varepsilon>0$,存在$e\in G$的开邻域$V$,使得对任意$h\in V$都有$|L_hf-f|_u<\varepsilon$,其中$|f|_u=\max\{|f(g)|,g\in G\}$.类似定义右一致连续.
	\item 记$\mathscr{C}_c(G)$表示$G$上所有具有紧支集(这里我们定义一个连续函数$f:X\to\mathbb{R}$的支集$\mathrm{Supp}f$是它的闭支集$\overline{\{x\in X\mid f(x)\not=0\}}$,称$f$具有紧支集是指它的闭支集是紧的)的连续函数.如果$f\in\mathscr{C}_c(G)$,那么$f$同时是左和右一致连续的.
	\begin{proof}
		
		我们来证明右一致连续性.记$K=\mathrm{Supp}(f)$,固定$\varepsilon>0$,对每个$g\in K$,存在$e\in G$的一个开邻域$U_g$,使得$h\in U_g$时有$|f(gh)-f(g)|<\varepsilon$.取$e$的对称开邻域$V_g$使得$V_gV_g\subseteq U_g$.明显的$\{gV_g,g\in K\}$覆盖了整个$K$,所以可取有限子覆盖$\{g_1V_1,\cdots,g_nV_n\}$,这里把$V_{g_i}$简记作$V_i$.取$V=\cap_{j=1}^nV_j$,这是$e$的一个对称开邻域.任取$g\in K$,那么可设$g\in g_jV_j$,任取$h\in V$,那么有$|f(gh)-f(g)|\le|f(gh)-f(g_j)|+|f(g_j)-f(g)|$.由于$g_j^{-1}g$和$g_j^{-1}gh$都在$U_j$中(用到了$V_gV_g\subseteq U_g$),所以这两个绝对值都被$\varepsilon$控制.至此我们证明了$f$在$K$上是右一致连续的.下面如果$g\not\in K$,如果$f(gh)\not=0$,那么$gh$在某个$g_jV_j$中,于是$|f(gh)-f(g)|=|f(gh)|=|f(gh)-g(g_j)|+|f(g_j)|$.按照$g_j^{-1}gh\in V_j\subseteq U_j$得到$|f(gh)-g(g_j)|<\varepsilon$,另外$g_j^{-1}g=g_j^{-1}ghh^{-1}\in V_jV_j\subseteq U_{g_j}$,于是$|f(g_j)-f(g)|<\varepsilon$,综上得到$h\in V$时有$|f(gh)|<2\varepsilon$,完成证明.
	\end{proof}
\end{enumerate}

设$G$是局部紧拓扑群,设$\mu$是一个非零的Radon测度.
\begin{enumerate}
	\item $\mu$是$G$上的左Haar测度,当且仅当$\widetilde{\mu}:E\mapsto\mu(E^{-1})$是$G$上的右Haar测度.
	\begin{proof}
		
		对拓扑群$G$,子集$E$是Borel集当且仅当$E^{-1}$是Borel集.于是$\widetilde{\mu}(E)=\widetilde{\mu}(Eg),\forall g\in G$当且仅当$\mu(E^{-1})=\mu(g^{-1}E^{-1}),\forall g\in G$.
	\end{proof}
    \item $\mu$是左Haar测度当且仅当对任意$\mathscr{C}_c^+(G)=\{f\in\mathscr{C}_c(G)\mid,f\ge0,|f|_u>0\}$和任意$g\in G$有$\int_GL_gf\mathrm{d}\mu=\int_Gf\mathrm{d}\mu$.
    \begin{proof}
    	
    	如果$\mu$是左Haar测度,那么$\mu(gE)=\mu(E),\forall g\in G$,于是如果$f$是简单函数(此即$G$上可测集的特征函数的线性组合),上述积分不等式总成立.对一般的连续函数$f$只要对简单函数取极限即可,这完成必要性证明.反过来对开集$U$有$\mu(U)=\sup\{\int_Gf\mathrm{d}\mu\mid f\in\mathscr{C}_c(G),|f|_u\le1,\mathrm{Supp}f\subseteq U\}$.所以如果积分是左变换不变的,就得到$\mu(gU)=\mu(U)$.再按照Radon测度的外正则性,说明$\mu(sE)=\mu(E)$对任意Borel集$E$成立.
    \end{proof}
    \item 如果$\mu$是左Haar测度,那么$\mu$在任意非空开集上测度为正,并且对任意$f\in\mathscr{C}_c^+$有$\int_Gf\mathrm{d}\mu>0$.
    \begin{proof}
    	
    	按照$\mu$是非零的,按照内正则性,至少存在一个紧集$K$满足$\mu(K)>0$.任取非空开集$U$,有$\{gU\}$构成了$K$的开覆盖,所以可取有限子覆盖,所以必须有$\mu(U)>0$,否则$\mu(gU)\equiv0$导致$\mu(K)=0$矛盾.下面设$f\in\mathscr{C}_c^+$,至少有一个点$g\in G$满足$f(g)>0$,取$g$的开邻域$U$使得$\forall h\in U$有$f(h)>f(g)/2$,那么$\int_Gf\mathrm{d}\mu\ge f(g)\mu(U)/2>0$.
    \end{proof}
    \item 设$\mu$是左Haar测度,那么$\mu(G)$有限当且仅当$G$是紧的.
    \begin{proof}
    	
    	如果$G$是紧的,Radon测度定义就要求$\mu(G)$有限.反过来如果$G$不是紧的,设$K$是$e$的紧邻域(这是指一个紧子集,并且包含了$e$的某个开邻域),那么$K$的有限个左平移不会覆盖整个$G$,否则有限个紧集并是紧的.于是可以取一个无穷序列$\{s_j\}\subseteq G$,满足$s_n\not\in\cup_{j<n}s_jK$.取$e$的对称开邻域$U$使得$UU\subseteq K$,我们断言$s_jU,j\ge1$是两两不交的,这件事结合$\mu(U)>0$就得到$\mu(G)=+\infty$.假设存在指标$i<j$使得$s_iu=s_jv$,其中$u,v\in U$,那么$s_j=s_iuv^{-1}\in s_iUU\subseteq s_iK$,这和我们的构造矛盾.
    \end{proof}
\end{enumerate}

局部紧群上总存在左(右)Haar测度,并且在相差一个倍数的意义下唯一.
\begin{enumerate}
	\item Haar覆盖数.设非空开集$U=\{s\in G\mid\varphi(s)>|\varphi|_u/2\}$,那么$U$的有限个左平移覆盖了$\mathrm{Supp}(f)$,记作$s_1U,\cdots,s_nU$.我们断言有$f\le\frac{2|f|_u}{|\varphi|_u}\sum_{j=1}^nL_{s_j}\varphi$.这件事说明$f$可以被有限个$\varphi$的左变换的正系数线性组合控制.我们定义$f$关于$\varphi$的Haar覆盖数为:
	$$(f:\varphi)=\inf\{\sum_{j=1}^nc_j\mid c_1,\cdots,c_n>0,s_1,\cdots,s_n\in G,f\le\sum_{j=1}c_jL_{s_j}\}$$
	\begin{proof}
		
		因为任取$s\in\mathrm{Supp}f$,可设$s\in s_tU$,那么$s_t^{-1}s\in U$,于是$\frac{2|f|_u}{|\varphi|_u}\sum_{j=1}^nL_{s_j}\varphi(s)\ge\frac{2|f|_u}{|\varphi|_u}\varphi(s_t^{-1}s)\ge|f|_u\ge f(s)$.
	\end{proof}
    \item Haar覆盖数的基本性质.如下函数都约定在$\mathscr{C}_c^+(G)$中.
    \begin{enumerate}
    	\item $(f:\varphi)=(L_sf:\varphi),\forall s\in G$.
    	\item $(f_1+f_2:\varphi)\le(f_1:\varphi)+(f_2:\varphi)$.
    	\item $(cf:\varphi)=c(f:\varphi),\forall c>0$.
    	\item 如果$f_1\le f_2$,有$(f_1:\varphi)\le(f_2:\varphi)$.
    	\item $(f:\varphi)\ge|f|_u/|\varphi|_u$.这件事说明覆盖数总是正的.
    	\begin{proof}
    		
    		任取控制$f$的线性组合$f(s)\le\sum c_j\varphi(s_j^{-1}s)$,那么有$f(s)\le(\sum c_j)|\varphi|_u$,对$f(s)$取上确界得到$\sum c_j\ge|f|_u/|\varphi|_u$,再对$\sum c_j$取下确界得到$(f:\varphi)\ge|f|_u/|\varphi|_u$.
    	\end{proof}
        \item $(f_1:\varphi)\le(f_1:f_0)(f_0:\varphi)$.
        \begin{proof}
        	
        	如果$f_1\le\sum c_jL_{s_j}f_0$和$f_0\le\sum d_kL_{t_k}\varphi$,那么有$f_1\le\sum c_jd_kL_{s_jt_k}\varphi$,再取下确界即可.
        \end{proof}
    \end{enumerate}
    \item 固定一个$f_0\in\mathscr{C}_c^+(G)$,记$I_{\varphi}(f)=\frac{(f:\varphi)}{(f_0:\varphi)}$.我们有$(f_0:f)^{-1}\le I_{\varphi}(f)\le(f:f_0)$.我们断言$I_{\varphi}$是几乎线性的:任取$f_1,f_2\in\mathscr{C}_c^+(G)$,任取$\varepsilon>0$,那么存在$e$的开邻域$V$,使得只要$\varphi$的支集落在$V$中,就有:
    $$I_{\varphi}(f_1)+I_{\varphi}(f_2)\le I_{\varphi}(f_1+f_2)+\varepsilon$$
    \begin{proof}
    	
    	局部紧Hausdorff空间上有Urysohn引理,于是可取$g\in\mathscr{C}_c^+(G)$,使得它在$\mathrm{Supp}(f_1+f_2)=\mathrm{Supp}f_1\cup\mathrm{Supp}f_2$(闭包和有限并可交换)上恒取1.设$\delta>0$,取$h=f_1+f_2+\delta g$,这是一个连续函数.取$h_i=f_i/h,i=1,2$,约定$h_i$在$\mathrm{Supp}f_i$以外恒取零.那么$h_i\in\mathscr{C}_c^+$.按照我们证明的$\mathscr{C}_c$具有左和右一致连续性,可取$e$的开邻域$U_{\delta}$,使得只要$t^{-1}s\in U_{\delta}$,就有$|h_i(s)-h_i(t)|<\delta,i=1,2$.
    	
    	\qquad
    	
    	设$\mathrm{Supp}\varphi\subseteq U$,并设$h\le\sum_jc_jL_{s_j}\varphi$,那么$f_i(s)=h(s)h_i(s)\le\sum_jc_j\varphi(s_j^{-1}s)h_i(s)\le\sum_jc_j\varphi(s_j^{-1}s)(h_i(s_j)+\delta)$.于是得到$(f_i:\varphi)\le\sum_jc_j(h_i(s_j)+\delta)$.按照$h_1+h_2\le1$,得到$(f_1:\varphi)+(f_2:\varphi)\le(1+2\delta)\sum_jc_j$,取下确界得到$(f_1:\varphi)+(f_2:\varphi)\le(1+2\delta)(h:\varphi)$,于是有:
    	\begin{align*}
    		I_{\varphi}(f_1)+I_{\varphi}(f_2)&\le(1+2\delta)I_{\varphi}(h)\\&\le(1+2\delta)\left(I_{\varphi}(f_1+f_2)+\delta I_{\varphi}(g)\right)\\&=I_{\varphi}(f_1+f_2)+2\delta\left(I_{\varphi}(f_1+f_2)+\delta I_{\varphi}(g)\right)
    	\end{align*}
    
        我们之前给出过$I_{\varphi}(f)\le(f:f_0)$所以$I_{\varphi}(f_1+f_2)$和$I_{\varphi}(g)$的上界不依赖$\varphi$,所以适当选取足够小的$\delta$,取$V=U_{\delta}$,那么当$\varphi$的支集落在$V$中时就有$I_{\varphi}(f_1)+I_{\varphi}(f_2)\le I_{\varphi}(f_1+f_2)+\varepsilon$.
    \end{proof}
    \item 记$X=\prod_{f\in\mathscr{C}_c^+(G)}[(f_0:f)^{-1},(f,f_0)]$,那么$X$中的元素可视为$\mathscr{C}_c^+(G)$上的实值函数,所有$I_{\varphi}$都可以视为$X$的元素.对$e\in G$的开邻域$U$,记$K_U$表示$F_U=\{I_{\varphi}\mid\mathrm{Supp}\varphi\subseteq U\}$在$X$中的闭包.那么$\{K_U\}$满足有限交性质,因为$F_{\cap_{j=1}^nU_j}=\cap_{j=1}^nF_{U_j}$和$\overline{A\cap B}\subseteq\overline{A}\cap\overline{B}$得到$K_{\cap_{j=1}^nU_j}\subseteq\cap_{j=1}^nK_{U_j}$.按照Urysohn引理得到$F_U$总是非空的,所以$K_{\cap_{j=1}^nU_j}$已经非空.按照$X$是紧的(紧空间的积拓扑总是紧的),所以当$U$取遍$e$的开邻域时$\cap K_U$是非空的,任取一个元记作$I$,那么它落在每个$F_U$的闭包中.于是任取$e$的开邻域$U$,任取有限个$f_1,\cdots,f_n\in\mathscr{C}_c^+(G)$,任取$\varepsilon>0$,那么存在$\varphi\in\mathscr{C}_c^+(G)$,满足$\mathrm{Supp}\varphi\subseteq U$,并且有$|I(f_j)-I_{\varphi}(f_j)|<\varepsilon,1\le j\le n$.
    \item $I$是$\mathbb{R}^+$线性的,即对$c_1,c_2>0$和$f_1,f_2\in\mathscr{C}_c^+(G)$,有$I(c_1f_1+c_2f_2)=c_1I(f_1)+c_2I(f_2)$.并且它是左平移不变的,即$I(f)=I(L_gf),\forall g\in G$.
    \begin{proof}
    	
    	对$f\in\mathscr{C}_c^+(G)$和$c>0$,考虑$f_1=f$和$f_2=cf$用上一条结论,说明对任意$\varepsilon>0$,存在$e$的开邻域$U$,使得$\varphi\in\mathscr{C}_c^+$如果满足$\mathrm{Supp}\varphi\subseteq U$,就有$|I(f)-I_{\varphi}(f)|<\varepsilon$和$|I(cf)-I_{\varepsilon}(cf)|<\varepsilon$.我们有$I_{\varepsilon}(cf)=cI_{\varepsilon}(f)$,所以$|I(cf)-cI(f)|\le|I(cf)-I_{\varepsilon}(cf)|+c|I_{\varepsilon}(f)-I(f)|<(1+c)\varepsilon$.让$\varepsilon\to0$就得到$I(cf)=cI(f)$.
    	
    	\qquad
    	
    	再取$f_1,f_2\in\mathscr{C}_c^+$,对任意$\varepsilon>0$,我们可以选取$e$的开邻域$U$,使得如果$\varphi\in\mathscr{C}_c^+$满足$\mathrm{Supp}\varphi\subseteq U$,就有$I_{\varphi}(f_1)+I_{\varphi}(f_2)\le I_{\varphi}(f_1+f_2)+\varepsilon$,$|I(f_i)-I_{\varphi}(f_i)|<\varepsilon,i=1,2$,$|I(f_1+f_2)-I_{\varepsilon}(f_1+f_2)|<\varepsilon$,做一些放缩得到$I(f_1+f_2)=I(f_1)+I(f_2)$.
    	
    	\qquad
    	
    	最后左平移不变是因为$I_{\varphi}(f)=I_{\varphi}(L_gf)$,依旧按照上一条做放缩得到结论.
    \end{proof}
    \item Haar测度的存在性.设$f\in\mathscr{C}_c$,那么可记$f=f^+-f^-$,其中$f^+,f^-\in\mathscr{C}_c^+$,定义$I(f)=I(f^+)-I(f^-)$,这个定义良性.这就把$I$的定义域延拓到整个$\mathscr{C}_c$上.再按照Riesz表示定理:设$X$是局部紧Hausdorff空间,如果$\sigma$是$\mathscr{C}_c(X)$上的线性泛函,那么存在$X$上的Radon测度$\mu$,使得$\sigma(f)=\int_Xf\mathrm{d}\mu$.于是我们得到了$G$上的Radon测度$\mu$,使得$I(f)=\int_Gf\mathrm{d}\mu$.另外我们解释过$I$是左平移不变的,所以这个积分是左平移不变的,所以$\mu$是左平移不变的.
    \item 设$G$上两个Haar测度为$\mu$和$\nu$,记$I(f)=\int_Gf\mathrm{d}\mu$和$J(f)=\int_Gf\mathrm{d}\nu$,Haar测度的唯一性是要证明$I(f)/J(f)$的值不依赖于$f\in\mathscr{C}_c^+$的选取.我们的思路是构造一个$h\in\mathscr{C}_c^+$使得$I(f)/J(f)$和$I(g)/J(g)$都距离$I(h)/J(h)$任意小.
    \item 设$K$是$e\in G$的紧邻域(此即一个紧集,包含了$e$的某个开邻域),那么可取$e$的对称开邻域$U$包含在$K$中,并且这个开邻域在$G$中的闭包$K_0$也在$K$中(Hausdorff空间的紧子集是闭的,所以$K$是闭的,所以$U$的闭包在$K$中),并且$K_0$是紧的(因为紧集的闭子集紧),并且$K_0$是对称的(因为对称开集的闭包还是对称的).定义$K_f=\mathrm{Supp}(f)K_0\cup K_0\mathrm{Supp}(f)$和$K_g=\mathrm{Supp}(g)K_0\cup K_0\mathrm{Supp}(g)$,我们之前解释过$G$上紧集和紧集的乘积是紧的,于是这里$K_f$和$K_g$都是紧集.对$t\in K_0$,定义$\gamma_tf(s)=f(st)-f(ts)$,换句话讲$\gamma_tf=R_tf-L_{t^{-1}}f$.类似定义$\gamma_tg$.那么有$\mathrm{Supp}(\gamma_tf)\subseteq\mathrm{Supp}(f)t^{-1}\cup t^{-1}\mathrm{Supp}(f)\subseteq K_f$.类似的有$\mathrm{Supp}(\gamma_tg)\subseteq K_g$.另外$G$的中心$\mathrm{Z}(G)$满足$\gamma_tf(\mathrm{Z}(G))=0$和$\gamma_tg(\mathrm{Z}(G))=0$.按照左右一致连续性,对任意$\varepsilon>0$,可取$K_0$包含的$e$的开邻域$U_0$,使得对$s\in G$和$t\in U_0$有$|\gamma_tf(s)|<\varepsilon/2$和$|\gamma_tg(s)|<\varepsilon/2$.取$K_0$包含的$e$的对称开邻域$U_1$,使得$U_1$的闭包$K_1$是紧对称并且在$K_0$中,按照连续性,对$s\in G$和$t\in K_1$就有$|\gamma_tf(s)|<\varepsilon$和$|\gamma_tg(s)|<\varepsilon$.
    \item 下面构造$h$.我们可以取$e$的紧邻域$K_2$使得$K_2$落在$K_1$的内点集$K_1^{\circ}$中(比方说,按照$G$是正规的,可取$e$的开邻域$U_2$满足$e\in U_2\subseteq\overline{U_2}\subseteq K_1^{\circ}$,那么$K_2=\overline{U_2}$满足条件),于是按照Urysohn引理可取连续映射$\widetilde{h}\in\mathscr{C}_c^+$,使得它在$K_2$上恒取1,在$K_1$以外恒取零.再取$h\in\mathscr{C}_c^+$为$h(s)=\widetilde{h}(s)+\widetilde{h}(s^{-1})$.那么$\mathrm{Supp}(h)\subseteq K_1$.并且$h$满足$h(s)=h(s^{-1})$,可以称为偶函数.
    \item Haar测度的唯一性.我们有:
    \begin{align*}
    	I(f)J(h)&=\iint_Gf(s)h(t)\mathrm{d}\mu_s\mathrm{d}\nu_t\\&=\iint_Gf(ts)h(t)\mathrm{d}\mu_s\mathrm{d}\nu_t
    \end{align*}
    
    按照$h$是偶函数,得到:
    \begin{align*}
    	I(h)J(f)&=\iint_Gh(s)f(t)\mathrm{d}\mu_s\mathrm{d}\nu_t\\&=\iint_Gh(t^{-1}s)f(t)\mathrm{d}\mu_s\mathrm{d}nu_t\\&=\iint_Gh(s^{-1}t)f(t)\mathrm{d}\mu_s\mathrm{d}nu_t\\&=\iint_Gh(t)f(st)\mathrm{d}\mu_s\mathrm{d}nu_t
    \end{align*}
    
    相减,按照$\mathrm{Supp}(h)\subseteq K_1$,可不妨设$t\in K_1$,我们解释了此时有$|\gamma_tf(s)|<\varepsilon,\forall s\in G$,另外我们解释了$\gamma_tf$的支集落在$K_f$里,于是有:
    \begin{align*}
    	\left|I(h)J(f)-I(f)J(h)\right|&=\left|\iint_Gh(t)\left(f(st)-f(ts)\right)\mathrm{d}\mu_s\mathrm{d}nu_t\right|\\&=\left|\iint_Gh(t)\gamma_tf(s)\mathrm{d}\mu_s\mathrm{d}nu_t\right|\\&\le\varepsilon\mu(K_f)J(h)
    \end{align*}
    
    于是得到:
    $$\left|\frac{I(h)}{J(h)}-\frac{I(f)}{J(f)}\right|\le\frac{\varepsilon\mu(K_f)}{J(f)}$$
    
    同理有:
    $$\left|\frac{I(h)}{J(h)}-\frac{I(g)}{J(g)}\right|\le\frac{\varepsilon\mu(K_g)}{J(f)}$$
    
    最后让预先取定的$\varepsilon$足够小,就得到$I(f)/J(f)=I(g)/J(g)$.证明了Haar测度的唯一性.
\end{enumerate}
\newpage
\subsection{射影有限群}

射影系统和射影极限.
\begin{itemize}
	\item 非空指标集$I$上的预序(preorder)是指二元关系$\le$,满足自反性($i\le i$)和传递性($i\le j$和$j\le k$推出$i\le k$),但可以不满足对称性(即不要求$i\le j$和$j\le i$推出$i=j$).称$I$是有向预序集,如果还满足$\forall i,j\in I$有$k\in I$使得$i\le k$和$j\le k$.
	\item 群的射影系统(projective system)也称为逆向系统(inverse system),指的是以有向预序集$I$作为指标集的一族群$\{G_i\mid i\in I\}$.并且如果$i\le j$,就取定一个群同态$\varphi_{ij}:G_j\to G_i$.满足$\forall i\in I$有$\varphi_{ii}=\mathrm{id}_{G_i}$,并且如果$i\le j\le k$,那么$\varphi_{ij}\circ\varphi_{jk}=\varphi_{ik}$.射影系统记作$(G_i,\varphi_{ij})_{i\in I}$.
	\item 如果$(G_i,\varphi_{ij})_{i\in I}$和$(G_s',\varphi_{st}')_{s\in I'}$是两个射影系统,设$\alpha:I'\to I$是保序映射,对每个指标$s\in I'$取一个群同态$\alpha(s):G_{\alpha(s)}\to G_{s}'$,满足对$s\le t$总有如下图表交换.我们就称$(\alpha,\alpha(s),s\in I')$是$(G_i,i\in I)\to(G_s',s\in I')$的同态.
	$$\xymatrix{G_{\alpha(t)}\ar[rr]^{\alpha(t)}\ar[d]_{\varphi_{\alpha(s)\alpha(t)}}&&G_t'\ar[d]^{\varphi'_{st}}\\G_{\alpha(s)}\ar[rr]^{\alpha(s)}&&G_s'}$$
	\item 射影系统$(G_i,\varphi_{ij})$的射影极限(projective limit)也称为逆向极限(inverse limit),定义为群:
	$$\lim\limits_{\leftarrow}G_i=\{(g_i)\in\prod_{i\in I}G_i\mid\forall i\le j,\varphi_{ij}(g_j)=g_i\}$$
\end{itemize}
\begin{enumerate}
	\item 射影极限有如下泛映射性质:如果$H$是一个群,对每个指标$i\in I$取了一个群同态$\psi_i:H\to G_i$,使得对任意$i\le j$有如下交换图表:
	$$\xymatrix{&H\ar[dl]_{\psi_j}\ar[dr]^{\psi_i}&\\G_j\ar[rr]_{\varphi_{ij}}&&G_i}$$
	
	那么存在唯一的群同态$\psi:H\to\lim\limits_{\leftarrow}G_i$使得对任意$i\in I$有如下交换图表:
	$$\xymatrix{H\ar[rr]^{\psi}\ar[drr]_{\psi_i}&&\lim\limits_{\leftarrow}G_i\ar[d]^{\pi_i}\\&&G_i}$$
	\item 射影系统的同态诱导了射影极限的群同态.如果$(\alpha,\alpha(s),s\in I')$是$(G_i,i\in I)\to(G_s',s\in I')$的同态,首先它诱导了$G=\prod_{i\in I}G_i\to G'=\prod_{s\in I'}G_s'$的同态为把$x\in G_{\sigma(s)}$映射到$\alpha(s)(x)\in G_s$,而这个群同态可限制为$\lim\limits_{\leftarrow}G_i\to\lim\limits_{\leftarrow}G_s'$.
	\item 射影系统$(G_i,\varphi_{ij})$无论放在集合范畴,放在群范畴,放在拓扑群范畴,我们构造的$\lim\limits_{\leftarrow}G_i$都是相应范畴中的射影极限.在拓扑群范畴的情况下,$\lim\limits_{\leftarrow}G_i$的拓扑取为$\prod_{i\in I}G_i$上积拓扑的子空间拓扑.
	\item 如果$\{\alpha_i\}:\{G_i',g_{ij}'\}\to\{G_i,g_{ij}\}$和$\{\beta_i\}:\{G_i,g_{ij}\}\to\{G_i'',g_{ij}''\}$是正向系统的两个态射,如果每个$i\in I$有$\xymatrix{G_i'\ar[r]^{\alpha_i}&G_i\ar[r]^{\beta_i}&G_i''}$是正合列,那么诱导的逆向极限之间的同态也是正合的:
	$$\xymatrix{G'=\lim\limits_{\leftarrow}G_i'\ar[rr]^{\alpha}&&G=\lim\limits_{\leftarrow}G_i\ar[rr]^{\beta}&&G''=\lim\limits_{\leftarrow}G_i''}$$
	\begin{proof}
		
		考虑如下交换图表:
		$$\xymatrix{G_i'\ar[rr]^{\alpha_i}\ar[d]^{g_i'}&&G_i\ar[rr]^{\beta_i}\ar[d]^{g_i}&&G_i''\ar[d]^{g_i''}\\G'\ar[rr]^{\alpha}&&G\ar[rr]^{\beta}&&G''}$$
		
		任取$x\in G$使得$\beta(x)=1$,那么存在指标$i$使得$x_i\in G_i$,满足$g_i(x_i)=x$,于是有$g''_i\beta_i(x_i)=\beta g_i(x_i)=\beta(x)=1$,于是存在指标$j\ge i$使得在$G_j''$中有$\beta_i(x_i)=1$.于是不妨设$\beta_i(x_i)=1$,于是存在$y_i\in G_i'$使得$\alpha_i(y_i)=x_i$,取$y=g'_i(y_i)$得到$\alpha(y)=x$.
	\end{proof}
	\item 对于逆向系统的情况,需要添加条件它们是紧群构成的逆向系统,此时逆向系统之间的正合列诱导了逆向极限之间的正合列.
\end{enumerate}

射影有限群.一个拓扑群称为射影有限群(profinite group),如果它作为拓扑群同构于有限群构成的射影系统的极限,这里拓扑取为有限群上赋予离散拓扑后,积空间上积拓扑的子空间拓扑.
\begin{enumerate}
	\item 群$G$是射影有限群,那么它是紧Hausdorff拓扑群,使得某些正规子群构成了幺元1的开邻域基.
	\begin{proof}
		
		如果记$\{G_i,f_{ij}:G_j\to G_i\}$是一个逆向系统,它的极限可表示为$G=\{(g_i)_{i\in I}\in\prod_{i\in I}G_i\mid f_{ij}(g_j)=g_i,i\le j\}$.Hausdorff性是容易的:离散空间是Hausdorff空间,Hausdorff空间的积空间仍然是Hausdorff的,Hausdorff空间的子空间总是Hausdorff的.
		
		验证紧性.首先有限离散空间必然是紧空间,紧空间的积拓扑仍然是紧空间,紧空间的闭子集是紧空间.于是问题归结为证明$G$作为$\prod_{i\in I}G_i$子空间是闭的.记$G_{ij}=\{(g_k)_{k\in I}\in\prod_kG_k\mid f_{ij}(g_j)=g_i\}$,那么$G=\cap_{i\le j}G_{ij}$,于是问题归结为证明$G_{ij}$都是闭子集.记$p_i$为分量$i$的投影映射$\prod_{k\in I}G_k\to G_i$,记$f=f_{ij}\circ p_j$.那么$g=p_i$和$f$定义的$\{x\in\prod_kG_i\mid f(x)=g(x)\}=G_{ij}$,按照Hausdorff条件,有$G_{ij}$是闭子集.
		
		验证开邻域基.记$U_S=\prod_{i\not\in S}G_i\times\prod_{i\in S}N_i$,其中$S$跑遍$I$的有限子集,$N_i$跑遍$G_i$的正规子群.那么$U_S$是$\prod_{i\in I}G_i$的由正规子群构成的1的开邻域基.而$U_S\cap\lim\limits_{\leftarrow}G_i$是$\lim\limits_{\leftarrow}G_i$的正规子群,于是这构成了$\lim\limits_{\leftarrow}G_i$的由正规子群构成的1的开邻域基.
	\end{proof}
	\item 反过来如果$G$是紧Hausdorff拓扑群,满足开正规子群构成了幺元1的开邻域基,那么有拓扑群同构$G\cong\lim\limits_{\substack{\leftarrow\\N}}G/N$,其中$N$跑遍开正规子群,我们解释过紧拓扑群商掉开正规子群得到有限群,于是这里$G$是射影有限群.
	\begin{proof}
		
		设$\{N_i\mid i\in I\}$是一些构成1的开邻域基的开正规子群.赋予$I$以偏序,定义$i\le j$如果$N_j\subseteq N_i$.记$G_i=G/N_i$,如果$i\le j$,记$g_{ij}:G_j\to G_i$是典范映射.那么$\{G_i,g_{ij}\}$构成逆向系统.这里$G_i$是有限群因为按照定义$N_i$的全体陪集是$G$的不交开子集,而$G$的紧性导致这样开子集的不交并必然是有限的,于是$[G:N_i]$有限.
		
		对每个$i\in I$有典范商映射$G\to G_i$,按照逆向极限的泛性质,这得到了一个典范映射$f:G\to\lim\limits_{\substack{\leftarrow\\i\in I}}G_i$,满足$g\mapsto\prod_{i\in I}g_i$,使得$g_i=g(\mathrm{mod}N_i)$.它的核是$\cap_{i\in I}N_i$,按照$\{N_i\}$是$1$的开邻域基,结合Hausdorff条件,说明$\cap_{i\in I}N_i=\{1\}$,这得到$f$是单射.
		
		对$I$的有限子集$S$,记$U_S=\prod_{i\not\in S}G_i\times\prod_{i\in S}\{1_{G_i}\}$,这是$\prod_{i\in I}G_i$的基元素.我们有$f^{-1}(U_S\cap\lim\limits_{\leftarrow}G_i)=\cap_{i\in S}N_i$,这表明$f$是连续映射.
		
		按照$G$是紧空间,它的闭子集都是紧集,紧集的连续像是紧集,而$\lim\limits_{\leftarrow}G_i$是Hausdorff空间,它的紧子集都是闭子集,这说明$f$是闭映射.于是一旦我们证明$f$是满射,它就是同构和同胚.
		
		只需证明$f(G)$是$\lim\limits_{\leftarrow}G_i$的稠密子集.任取$x=(x_i)_{i\in I}\in\lim\limits_{\leftarrow}G_i$,那么$x(U_S\cap\lim\limits_{\leftarrow}G_i)$是$x$的开邻域基,记$N_k=\cap_{i\in S}N_i$,设$y\in G$使得它在典范映射$G\to G/N_k$下的像是$x$的指标$k$的分量$x_k$,这导致$y(\mathrm{mod}N_i)=x_i,\forall i\in S$,导致$f(y)\in x(U_S\cap \lim\limits_{\leftarrow}G_i)$,这说明$f(G)$在$\lim\limits_{\leftarrow}G_i$中稠密.
	\end{proof}
    \item 一个拓扑群是射影有限群当且仅当它是紧致全不连通空间.这里全不连通是指空间的连通分支是所有单点集.另外注意拓扑群上全不连通能推出Hausdorff:全不连通说明每个点都是单点集,但是拓扑群上$T_1$是等价于Hausdorff的.
    \begin{proof}
    	
    	必要性,我们要证明射影有限群$G$是全不连通的.按照齐次性,等价于证明$e$所在的连通分支$G_0=\{e\}$.我们知道$G_0$是$G$的正规子群.接下来证明$G_0$包含在$G$的每个开子群$U$中:$U\cap G_0$是$G_0$的非空开子集,记$V=\cup_{x\in G_0-U}x(U\cap G_0)$,按照$G_0$是子群,$V$是$G_0$中的开集,并且和$U$无交,并且$U\cap V=\emptyset$.于是连通集$G_0$写成了两个不交开集$U\cap G_0$和$V$的并,这迫使$V=\emptyset$,也即$G_0\subseteq U$.
    	
    	\qquad
    	
    	证明$G_0=\{e\}$:任取$e\not=y=(y_i)\in G$,那么存在某个指标$i_0$不为$G_{i_0}$的幺元,考虑$U_0=\pi_{i_0}^{-1}(e_{i_0})$,因为$G_{i_0}$上是离散拓扑,这里$U_0$就是$G$的开正规子群,导致$G_0\subseteq U_0$,导致$y\not\in G_0$.于是$G_0=\{e\}$,也即$G$是全不连通的.
    	
    	\qquad
    	
    	充分性只要证明紧Hausdorff全不连通拓扑群上$e$存在由开正规子群构成的开邻域基.我们先来证明Hausdorff紧空间$X$上一个点$t$所在的连通分支$C_t$是包含$t$的所有紧致开集的交.考虑$A=\cap E$,其中$E$跑遍$X$的紧致开集,那么首先$C_t\in A$,因为$C_t$是连通的导致它必须在$E$和$E^c$中的一个,但是$t\in E$所以$C_t\subseteq A$.下面只需证明$A$是连通的.首先它是若干闭集的交,所以是闭集.假设$A$可以表示为两个闭子集$B_1,B_2$的无交并,那么$B_1,B_2$也是$X$的闭子集,所以按照$X$是正规的,可取$X$的不交开集$U_1,U_2$分别包含$B_1,B_2$.于是$A\subseteq U_1\cup U_2$,于是$U_1^c\cap U_2^c\subseteq\cup E^c$,这里$U_1^c\cap U_2^c$是$X$的闭子集,所以是紧的,所以可取有限个包含$t$的开闭集$E_1,\cdots,E_n$使得$U_1^c\cap U_2^c\subseteq\cup E_i^c$,所以$K=\cap E_i\subseteq U_1\cup U_2$.那么$K$也是开闭集,设$t\in U_1,t\not\in U_2$,这里$K\cap U_1=K\cap U_2^c$也是紧致开集($U_2^c$是紧的),所以按理说它包含整个$A$,这迫使$B_2$是空集,于是$A$是连通的.
    	
    	\qquad
    	
    	下面证明$e$的每个开邻域都包含了$e$的一个紧致的开邻域$W$.设$e$的所有紧致的开邻域构成的集合为$\mathscr{U}$,那么$C_e=\cap_{K\in\mathscr{U}}K$,按照$G$是全不连通的,有$C_e=\{e\}$.再设$U$是$e$的任意开邻域,那么$G-U$是闭的,于是由于$G$紧得到$G-U$紧.从$C_e=\{e\}$得到$\{K^c,K\in\mathscr{U}\}$是$G-U$的开覆盖,所以可取有限子覆盖,于是有有限个$e$的紧致开邻域$K_1,\cdots,K_r$满足$K=\cap_{j=1}^rK_j\subseteq U$,这就找到了紧致开邻域$K$.
    	
    	\qquad
    	
    	我们再证明$e$的紧致开邻域$W$包含了$e$的一个对称开邻域$V$,使得$WV\subseteq W$.考虑群的二元运算的限制映射$\mu:W\times W\to G$,这是连续的,因为$\mu(w,e)\in W,\forall w\in W$,所以可找$w\in W$的开邻域$U_w\subseteq W$以及$e$的对称开邻域$V_w\subseteq W$,使得$U_w\times V_w\subseteq\mu^{-1}(W)$.那么$\{U_w,w\in W\}$构成了$W$的开覆盖,可取有限子覆盖$\{U_1,\cdots,U_r\}$,设对应的$V$是$\{V_1,\cdots,V_r\}$.取$V=\cap_{j=1}^rV_j$,按照构造有$WV\subseteq W$,并且有$V^n\subseteq W,\forall n\ge0$.
    	
    	\qquad
    	
    	最后我们取$O=\cup_{n\ge1}V^n$,这是一个开子群.按照$G$是紧的,有$G/O$是有限离散空间.于是可设全部左陪集代表元$x_1,\cdots,x_s$,于是$O$在$G$中的共轭只有有限个$\{x_jOx_j^{-1},j=1,\cdots,s\}$.取$N=\cap_{j=1}^sx_jOx_j^{-1}$是$G$的开正规子群,并且$N\subseteq O\subseteq W\subseteq U$.得证.
    \end{proof}
	\item 设$G$是射影有限群,设$H$是$G$的子群,那么如下三个条件互相等价:
	\begin{enumerate}
		\item $H$是闭子群.
		\item $H$是射影有限群.
		\item $H$是一族开子群的交.
	\end{enumerate}
	
	并且在条件成立时有:
	\begin{itemize}
		\item 当$U$取遍$G$的开正规子群时有如下典范同构(我们解释过$G$是紧拓扑群时商掉开正规子群是有限群).
		$$H\cong\lim\limits_{\leftarrow}H/H\cap U$$
		\item 如果$H$是正规子群,那么$G/H$也是射影有限群,并且当$U$取遍$G$的开正规子群时有如下典范同构:
		$$G/H\cong\lim\limits_{\leftarrow}G/UH$$
	\end{itemize}
	\begin{proof}
		
		(a)$\Rightarrow$(b).先设$H$是闭子群,那么$H$是紧Hausdorff拓扑群,我们来证明如果$U$取遍$G$的开正规子群时有$H\cong\lim\limits_{\leftarrow}H/H\cap U$,此时$H\cap U$是$H$的开正规子群,我们解释过紧拓扑群商掉开正规子群是有限群,所以这个同构就得到$H$是射影有限群:$V$是$H$的开正规子群,那么存在$e\in G$的某个开邻域$O$使得$V=O\cap H$.这里$O$要包含一个开正规子群$U$,那么$U\cap H\subseteq V$.于是$\{U\cap H\}$当$U$跑遍开正规子群时,它在$H$的全体开正规子群中是共尾的,于是取逆向极限是典范同构的.
		
		\qquad
		
		(b)$\Rightarrow$(a)就是因为Hausdorff空间的紧子集是闭的.(c)$\Rightarrow$(a)就是因为开子集都是闭的,闭集的交当然是闭的.下面证明(a)$\Rightarrow$(c):设$\mathscr{N}$表示$G$的全体开正规子群构成的集合,对每个$N\in\mathscr{N}$,有$NH$是$G$的开子群,并且$[G:N]$是有限的,于是$[G:NH]$也是有限的.我们有$H\subseteq\cap_{N\in\mathscr{N}}NH$,只需证明另一侧包含关系.任取$x\in\cap_{N\in\mathscr{N}}NH$,任取$x$的开邻域$U$,那么$Ux^{-1}$是$e$的开邻域,我们解释过可以找到开正规子群$N_0$使得$N_0\subseteq Ux^{-1}$,按照$x$的选取有$x\in N_0H$,可记$x=n_0h$,那么$N_0x=N_0h$,于是$h\in N_0x\subseteq U$,这说明$x$的每个开邻域都和$H$有交,于是$x\in\overline{H}=H$,这得证.
		
		\qquad
		
		最后设$H$是闭正规子群,我们来证明$G/H$是射影有限群,首先$G$是紧一定能得到$G/H$是紧的,因为任取$G/H$的开覆盖$\{U_i\}$,回拉到$G$上也得到一个开覆盖$\{U_iH\}$,所以$G$的紧性说明$\{U_iH\}$有有限子覆盖,就导致$\{U_i\}$也有有限子覆盖.下面证明$G/H$是全不连通的:设幺元$e'\in G/H$所在的连通分支是$S$,我们解释过它是闭正规子群,记典范商同态$\rho:G\to G/H$,那么$M=\rho^{-1}(S)$是$G$的一个闭正规子群,不妨设$H,G/H$是非平凡的,倘若$M\not=H$,那么$X=\cup_{mH\not=H,m\in M}mH$的元素多于2个,所以按照$G$是全不连通的,有$X$能表示为两个非空的$X$的开闭集$F_1,F_2$的无交并,那么可设有$G$的闭集$E_1,E_2$使得$F_1=E_1\cap X$和$F_2=E_2\cap X$,按照$F_1=M\cap H^c\cap E_1=M\cap H^c\cap E_2^c$得到$F_1$是$M$中的开集,类似的$F_2$是$M$中的开集,于是$F_1\cup H$是$M$的闭集,$F_2$作为它在$M$中的【】
		
		
		于是$F_1\cup H$和$F_2$是$M$的两个非空的不交闭集,并且并是整个$M$,但是$M$本身是闭的,所以$F_1\cup H$和$F_2$也是$G$的闭集.由于$H$是紧的,有$\rho:G\to G/H$是闭映射,所以$\rho(F_1\cup H),\rho(F_2)$是$S$的非空不交闭子集,并且并是整个$S$,但是这和$S$连通矛盾.所以$M=H$.最后按照$U$取遍$G$的开正规子群时$\{UH/H\}$是$G/H$的共尾开正规子群族,就得到$G/H\cong\lim\limits_{\leftarrow}G/UH$.
	\end{proof}
    \item 设$G$是任意一个群,设$\{H_i,i\in I\}$是一个正规子群族,定义预序是$i\le j$为$H_j\subseteq H_i$,如果它是有向预序集,对$i\le j$取典范映射$G/H_j\to G/H_i$,得到一个射影系统,它的射影极限记作$L=\lim\limits_{\leftarrow}G/H_i$.那么存在同态$G\to L$为$g\mapsto(gH_i)$.它的核是$G_0=\cap H_i$.
\end{enumerate}

射影有限群的例子.
\begin{enumerate}
	\item 考虑可能无限的Galois扩张$k\subseteq F$,记Galois群为$G$,取这个扩张的有限Galois子扩张$k\subseteq K$,那么$G$恰好是射影有限群$\lim\limits_{\leftarrow}G(K/k)$.
	\item 考虑一个完备离散赋值环$A$,取极大理想$p$,那么$A=\lim\limits_{\leftarrow}A/p^n$,有$A^*=\lim\limits_{\leftarrow}A^*/U^{(n)}$.
	\item 考虑$\widehat{\mathbb{Z}}=\lim\limits_{\leftarrow}\mathbb{Z}/n\mathbb{Z}$,这里$n$取遍全体正整数,约定$m\mid n$时取典范映射$\mathbb{Z}/n\mathbb{Z}\to\mathbb{Z}/m\mathbb{Z}$.这个环称为Pr\"ufer环,它同构于$\prod_p\mathbb{Z}_p$.存在典范的嵌入$\mathbb{Z}\to\widehat{\mathbb{Z}}$为$a\mapsto(a,a,\cdots)$.
	\item 取$k=\mathbb{F}_q$是有限域,取$n$次扩张$F=\mathbb{F}_{q^n}$,那么Galois群可以描述为由Frobenius映射$x\mapsto x^q$生成的$n$阶循环群$\mathbb{Z}/n\mathbb{Z}$,于是$k\subset\overline{k}=\overline{F_q}$的Galois群就是$\widehat{\mathbb{Z}}$.
	\item $\mathbb{Z}_p$和$\widehat{\mathbb{Z}}$都是所谓的射影循环群:它们拓扑上都是被单点生成的,即$G=\overline{\{\sigma^i\}}$.
	\item 取$A$是一个交换挠群(即每个元的阶数都有限),那么$A$是它全部有限循环子群的并,也即全部有限循环子群的正向极限$A=\lim\limits_{\rightarrow}A_i$,作用逆变函子$A^v=\mathrm{Hom}(A,\mathbb{Q}/\mathbb{Z})$后得到$A^v=\lim\limits_{\leftarrow}A_i^v$是一个射影有限群.例如取$A=\mathbb{Q}/\mathbb{Z}=\cup_{n\ge1}(1/n)\mathbb{Z}/\mathbb{Z}$,得到$A^v=\widehat{\mathbb{Z}}$.
	\item 射影完备化.设$G$是群,设$N$跑遍全部有限指标的正规子群,那么$\widehat{G}=\lim\limits_{\leftarrow}G/N$称为$G$的射影完备化.例如$\mathbb{Z}$的射影完备化就是我们提到的Pr\"ufer群$\widehat{\mathbb{Z}}$.
\end{enumerate}

射影有限群的阶数.
\begin{enumerate}
	\item 一个超自然数(supernatural number)指的是一个形式乘积$\prod_pp^{n_p}$,其中$p$取遍素数,$n_p$取自然数或者$\infty$.设$a$是一个超自然数,用$v_p(a)$表示分解中的$n_p$.如果$a,b$是两个超自然数,满足对任意素数$p$有$n_p(a)\le n_p(b)$,就称$a$整除$b$,记作$a\mid b$,这也等价于讲存在第三个超自然数$c$使得$ac=b$.自然数明显可以视为超自然数.另外给定一族超自然数$\{a_i,i\in I\}$,我们可以定义它们的最小公倍数和最大公约数都是超自然数:
	$$\mathrm{lcm}(a_i,i\in I)=\prod_pp^{\sup\{v_p(a_i),i\in I\}},\gcd(a,b)=\prod_pp^{\inf\{v_p(a_i),i\in I\}}$$
	\item 设$G$是射影有限群,设$H$是闭子群,定义$[G:H]=\mathrm{lcm}_{N\in\mathscr{N}}[G/N:HN/N]$,其中$\mathscr{N}$是$G$的所有开正规子群构成的集合,那么紧性保证了$G/N$是有限群,所以$[G/H:HN/N]$是自然数,所以这个最小公倍数定义良性,是一个超自然数.我们定义$[G:\{e\}]$为射影有限群$G$的阶数,记作$|G|$.另外如果$G$本身是有限群,那么它是离散的,所以这里定义的阶数和它的常义阶数是吻合的.
	\item 设$G$是射影有限群,设$H\subseteq K$是闭子群,那么$[G:K]=[G:H][H:K]$.
	\begin{proof}
		
		设$\mathscr{N}$是$G$的全部开正规子群,任取$N\in\mathscr{N}$,那么有$[G/N:K/K\cap N]=[G/N:H/H\cap N][H/H\cap N:K/K\cap N]$.明显的对两组超自然数$\{a_i\}$和$\{b_i\}$,有$\mathrm{lcm}_ia_ib_i=\mathrm{lcm}ia_i\mathrm{lcm}_ib_i$.而这里$[G/N:K/K\cap N],N\in\mathscr{N}$取最小公倍数就是$[G:K]$,$[G/N:H/H\cap N],N\in\mathscr{N}$取最小公倍数是$[G:H]$.所以问题归结为证明$[H/H\cap N:K/K\cap N],N\in\mathscr{N}$取最小公倍数是$[H:K]$.任取$H$的开正规子群$M$,那么存在$G$的开集$U$使得$M=H\cap U$,我们之前解释过可在$U$中取$G$的一个开正规子群$N$,那么有$[H/M:K/K\cap M]\mid[H/H\cap N:K/K\cap N]$,于是计算$[H:K]$可以取$H$的形如$H\cap N$的开正规子群取最小公倍数,其中$N$是$G$的开正规子群,这就得证.
	\end{proof}
    \item 上述证明还说明,计算$[G:H]$时,可取$\mathscr{N}$的共尾子集$\mathscr{M}$(此即对任意$M\in\mathscr{M}$,可取$N\in\mathscr{N}$满足$N\subseteq M$):
    $$\mathrm{lcm}_{N\in\mathscr{N}}[G/N:HN/N]=\mathrm{lcm}_{M\in\mathscr{M}}[G/M:HM/M]$$
    \item 例如$\mathbb{Z}_p=\lim\limits_{\substack{\leftarrow\\n\ge1}}\mathbb{Z}/n\mathbb{Z}$的阶数是$p^{\infty}$.而$\widehat{\mathbb{Z}}=\lim\limits_{\substack{\leftarrow\\n\ge1}}\mathbb{Z}/n\mathbb{Z}$的阶数是$\prod_pp^{\infty}$,其中$p$取遍素数.
\end{enumerate}

射影$p$群.固定素数$p$,如果拓扑群同构于有限$p$群的射影极限,则称为射影$p$群.设$G$是射影有限群,固定素数$p$,称$G$的极大射影$p$群为$G$的射影Sylow-$p$子群.
\begin{enumerate}
	\item 一个射影有限群$G$是射影$p$群当且仅当它的阶数是$p$的次幂.
	\begin{proof}
		
		充分性,此时每个$G/N$都是有限$p$群,而且我们解释过$G/N$的射影极限就是$G$,于是$G$是射影$p$群.必要性,如果$G$是射影系统$\{P_i,i\in I\}$的极限,其中$P_i$都是$p$群.考虑所有形如$M=(\prod Q_i)\cap G$的开正规子群,其中除了有限个指标以外$Q_i=P_i$,其余这有限个指标有$Q_i=\{e_i\}$.按照积拓扑的性质,这些$M$在全体开正规子群中共尾,并且$G/M$都是有限$p$群,所以$[G:M]$取最小公倍数得到的是$G$的阶数,它是$p$的次幂.
	\end{proof}
    \item 设$G$是射影有限群,设$p$是素数,类似于Sylow定理,我们有:
    \begin{enumerate}
    	\item 射影Sylow-$p$子群总存在.
    	\begin{proof}
    		
    		记$\mathscr{N}$表示$G$的全部开正规子群.对$N\in\mathscr{N}$,记$\mathscr{P}(N)$表示$G/N$的所有Sylow-$p$子群.那么$$\mathscr{P}(N)$$总是有限的和非空的.如果$N\subseteq M$是两个开正规子群,那么有满同态$\varphi_{M,N}:G/N\to G/M$.它把$G/N$的Sylow-$p$子群映射成$G/M$的Sylow-$p$子群,于是$\varphi_{M,N}$诱导了映射$\mathscr{P}(N)\to\mathscr{P}(M)$.于是$(\mathscr{P}(N),\varphi_{M,N})$是集合上的射影系统,它的极限是非空的,任取一个元素,换句话讲我们有射影系统$(P_N,\varphi_{M,N})$,其中$P_N\subseteq G/N$是Sylow-$p$子群.这个射影系统的极限记作$P$,它是一个射影$p$群,由于每个$P_N\subseteq G/N$是单射,射影系统的单同态诱导了极限的单同态,于是$P$是$G$的子群.我们只需证明它是极大的射影$p$群.
    		
    		\qquad
    		
    		假设$Q$是另一个射影$p$群,并且包含了$P$,那么$QN/N$是$G/N$的包含$PN/N=P_N$的子群,所以$QN/N=PN/N$对任意开正规子群$N$成立.因为当$N$取遍开正规子群时有$G\to\lim\limits_{\leftarrow}G/N$,$x\mapsto(xN)$是同构,所以如果$xN=yN,\forall$开正规子群$N$,那么$x=y$.于是从$QN/N=PN/N$就得到$Q=P$.得证.
    	\end{proof}
    	\item 任意两个射影Sylow-$p$子群是共轭的.
    	\begin{proof}
    		
    		设$P,Q$是两个射影Sylow-$p$子群,任取开正规子群$N$,记$P_N=PN/N$和$Q_N=QN/N$和$Y_N=\{y_N\in G/N\mid y_NP_Ny_N^{-1}=Q_N\}$.那么按照Sylow定理,这里$Y_N$非空.按照$G/N$是有限群,有$Y_N$有限.那么当$N$跑遍全部开正规子群时,$Y_N$连同典范映射$G/N\to G/M$构成一个射影系统.它的极限$Y$是$G$的非空子集,任取$y\in Y$,那么$yPy^{-1}$和$Q$在每个$G/N$的像相同,于是$yPy^{-1}=Q$.
    	\end{proof}
    	\item 如果$P$是射影Sylow-$p$子群,那么$[G:P]$和$p$互素的,即超自然数$[G:P]$中$p$的次幂是1.
    	\begin{proof}
    		
    		按照定义有$[G:P]=\mathrm{lcm}_{N\in\mathscr{N}}[G/N:PN/N]$,这里$PN/N$是$G/N$的Sylow-$p$子群,所以这些指数$[G/N:PN/N]$都和$p$互素,取最小公倍数也就和$p$互素.
    	\end{proof}
    	\item 特别的,(a)和(c)说明$G$的射影Sylow-$p$子群非平凡当且仅当$p$整除超自然数$|G|$.
    \end{enumerate}
    \item 推论.如果$G$是交换射影有限群.
    \begin{enumerate}
    	\item 对每个素数$p$,$G$有唯一的射影Sylow-$p$子群.
    	\item 如果$p,q$是不同的素数,设对应的射影Sylow-$p$子群是$P,Q$,那么$P\cap Q$是平凡的.
    	\item $G$同构于它所有射影Sylow子群的直积.
    	\begin{proof}
    		
    		任取$G$的开正规子群$N$,任取$G$的射影Sylow-$p$群$P$,那么$PN/N$是$G/N$的Sylow-$p$子群,按照$G/N$是有限群,那么有$\prod_PPN/N\cong G/N$.这个直积只有有限项不为平凡群.当$N$跑遍$G$的开正规子群时,形如$P\cap N$的$P$的子群构成了$P$的开正规子群族的共尾子集,所以有典范同构$\lim\limits_{\substack{\leftarrow//N}}P/P\cap N\cong P$.于是有$G=\lim\limits_{\leftarrow}G/N=\lim\limits_{\leftarrow}\prod_P PN/N=\prod_P\lim\limits_{\leftarrow}P/N\cap P=\prod_PP$.
    	\end{proof}
    \end{enumerate}
    \item 例如考虑阿贝尔射影有限群$\widehat{\mathbb{Z}}=\lim\limits_{\leftarrow}\mathbb{Z}/n\mathbb{Z}$,给定素数$p$,那么它的射影Sylow-$p$子群就是$\mathbb{Z}_p$:设$\mathbb{Z}/n\mathbb{Z}$的Sylow-$p$子群是$P_n$,那么$\lim\limits_{\leftarrow}P_n=\lim\limits_{\leftarrow}\mathbb{Z}/p^{v_p(n)}\mathbb{Z}=\lim\limits_{\leftarrow}\mathbb{Z}/p^m\mathbb{Z}=\mathbb{Z}_p$.进而得到$\widehat{\mathbb{Z}}=\prod_p\mathbb{Z}_p$.
    \item 自由射影$p$群.设$I$是集合,设$L(I)$是集合$I$上的自由群.考虑全体满足如下两个条件的正规子群$M\subseteq L(I)$:
    \begin{enumerate}[(a)]
    	\item $L(I)/M$是有限$p$群.
    	\item $M$包含了几乎全部$I$,也即$I-M\cap I$是有限集合.
    \end{enumerate}
    定义$I$上的自由射影$p$群为$F(I)=\varprojlim L(I)/M$.它满足如下泛性质:对任意射影$p$群$G$,同态$F(I)\to G$一一对应于$I\to G$的映射$\{g_i\}_{i\in I}$,满足这些元在$G$的余有限filter上是取于幺元的.当$I$是有限集合时这个条件自动去掉.
\end{enumerate}




