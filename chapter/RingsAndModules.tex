\chapter{环与模论}
\section{环范畴}
\subsection{基本概念}

环和模都是在交换群上添加一种运算结构.把交换群的二元运算记作加法,加法幺元记作0,称为0元.称一个交换群$R$是\textbf{环(ring)},如果它具备第二个二元运算,记作乘法,满足结合律,即$\forall a,b,c\in R$,有$(ab)c=a(bc)$,并且乘法和加法是兼容的,即满足分配律$a(b+c)=ab+ac,(a+b)c=ac+bc$.如果乘法存在幺元$1$,即$r1=1r=1,\forall r\in R$,就称为\textbf{含幺环}.含幺环$R$上的幺元通常记作$1_R$,在不引起歧义的前提下直接记作1.如果环不存在幺元,往往称为\textbf{rng},去掉字母$i$,代表着不含幺.例如全体偶整数$2\mathbb{Z}$在加法和乘法下成为一个rng.若不加以强调,我们指的环总是含幺环.如果环的乘法满足交换律,即$ab=ba$,则称为\textbf{交换环}.
\begin{enumerate}
	\item $0r=r0=0,\forall r\in R$;$(-1)r=-r,\forall r\in R$.
	\item 存在一元环,此时唯一的元既是0元也是幺元,它称为\textbf{0环}.一个环如果0元和幺元相等那么必然是0环.
	\item 环中一个元$a$称为左或右\textbf{零因子},如果存在环中的非0元$b$或$c$使得$ab=0$或$ca=0$.在交换环中左右之间没有差异.注意到0元总是零因子,称为平凡零因子.一个元$a$不是左/右零因子等价于说环上左/右乘元$a$是环自身上的单射.一个交换环如果没有非平凡零因子,就称为\textbf{整环},即如果$ab=0$,那么$a,b$中必然有一个是0元.
	\item 环中一个元$a$称为左或右\textbf{单位},如果存在环中的元$b$或$c$使得$ab=1$或$ca=1$.在交换环中左右之间没有差异.容易看出一个元$a$是左/右单位等价于说环上左/右乘元$a$是环到自身的满射.并且一个左/右单位必然不是右/左零因子.一个元如果同时是左右单位,那么它的所有左逆和右逆都是唯一的一个,这称为它的(双侧)逆,这个元称为(双侧)单位.环上全体双侧单位构成一个群,称为环的\textbf{单位群}.如果每个非0元都是双侧单位,就称环是\textbf{除环}.
	\item 称一个环是\textbf{域},如果它交换并且每个非0元都是单位.域既是整环又是除环.
	\item 有限整环是域.在交换环上,任意一个非0元是非零因子等价于说环上乘以任意非0元是一个单射,有限集合上的单射等价于满射,于是这个环上乘以任意非0元都是满射,于是所有非0元都是单位.
	\item 事实上有限除环同样是域,这称为\textbf{Wedderburn 定理}.但是相比上一个结论证明会麻烦很多.
	\begin{proof}
		
		第一步,说明除环的中心是域.给定除环$R$,取一个中心元$a$,那么对每个$r\in R$,有$ar=ra$,于是得到$ra^{-1}=a^{-1}r$,这导致$a^{-1}$也在中心,再结合如果$a,b$在中心那么$ab$也是,这就说明了$R$的中心是域.
		
		第二步,计算$R$的阶数.首先$R$的中心是有限域,于是它的阶数是一个素数幂$q$,下面,$R$作为加法群,是中心$C$上的线性空间.这就导致$|R|=q^n$.
		
		第三步,求出$R$上乘法群的类方程.首先任取$R$中非0元$r$,记$r$在乘法群$R^*$中的中心化子并上0为$K$.那么$K$是一个包含了$C$的$R$的子环,于是它也是$C$上的线性空间,于是$K$的阶数是某个$q^d$,这里$d\le n$,于是,除去0,$r$在乘法群中的中心化子的阶数是$q^d-1$,于是所在共轭类的元素个数是$\frac{q^n-1}{q^d-1}$.结合乘法群的中心就是$C$扣去0元,这就得到乘法群的类方程是$q^n-1=(q-1)+\sum_{i=1}^{r}\frac{q^n-1}{q^{d_i}-1}$,其中每个$d_i<n$.
		
		第四步,利用类方程证明$n=1$,从而$R=C$是域完成证明.首先,我们断言分圆多项式$\Omega_n(x)$满足$\Omega_n(q)\mid\frac{q^n-1}{q^d-1}$,其中$d\mid n$,$d<n$.按照$q^n-1=\prod_{1\le d\mid n}\Omega_d(q)$,$q^d-1=\prod_{1\le i\mid d}\Omega_i(q)$,得到$\frac{q^n-1}{q^d-1}=\prod_{1\le i\mid n,i\not\mid d}\Omega_i(q)$,特别的,$\Omega_n(q)\mid\frac{q^n-1}{q^d-1}$.于是按照类方程,得到$\Omega_n(q)\mid q-1$,于是$\Omega_n(q)\le q-1$.但是$\Omega_n(q)$是$\varphi(n)$个$q-\xi$的乘积,这里$\xi$是一个本原$n$次单位根,导致$|q-\xi|\ge|q|-|\xi|=q-1$,于是$\Omega_n(q)\ge q-1$,于是满足整除条件只能有$n=1$,完成证明.
	\end{proof}
	\item 称环$R$的子集$S$是\textbf{子环},如果它在环的乘法加法下封闭,并且是一个环,具有公共的0元和幺元.
	\item 环的中心.给定环$R$,称中心元是指和所有元都可乘法交换的元.那么全体中心元构成了环的一个交换子环,称为环的中心.给定$R$中一个元$x$,称$x$的中心是$R$中全部和$x$可乘法交换的元,那么元素的中心也是一个子环.并且环的中心就是全体元素的中心的交.
\end{enumerate}

接下来给出环的一些例子.
\begin{enumerate}
	\item
	交换环$\mathbb{Z}/n$.我们知道循环群$\mathbb{Z}/n$是最简单的群.事实上还可以给它定义乘法为$[s]\cdot[t]=[st]$使得它成为一个交换环.
	
	如果$n$不是素数,可记$n=ab,a,b\not=0,1$,那么$[a]\cdot[b]=[ab]=0$,这说明此时$\mathbb{Z}/n$不是整环.但是当$n=p$是素数的情况,按照初等数论中的Bezout定理得到对$1\le i\le p-1$,有$(i,p)=1$,于是存在整数$s,t$使得$si+pt=1$,于是在$\mod p$下得到$[i]\cdot[s]=1$,于是$\mathbb{Z}/p$是一个域.
	
	$\mathbb{Z}/p$不是仅有的有限域,我们在域论中会看到$\mathbb{Z}/p$的$n$次直和的这个集合上在同构意义下存在唯一一种对乘法的约定使得它构成域.并且事实上这完全刻画了全部有限域.另外这里要强调$p^n$阶域的乘法并不是简单的$\mathbb{Z}/p$的$n$次直和对应分量的乘法.事实上两个非平凡环的直积必然不会是整环,这是因为$(1_A,0)\times(0,1_B)=(0,0)$,于是它远远达不到域的条件.
	\item \textbf{多项式环}.设$R$是环,其上多项式环$R[x]$作为集合是$R$上全体终端为0的序列:
	$$R[x]=\{(r_1,r_2,\cdots),r_i\in R,\exists N\in N^+,r_{N+i}=0,i\in N^+\}$$
	
	$R[x]$上定义加法和乘法为:
	$$(a_1,a_2,\cdots)+(b_1,b_2,\cdots)=(a_1+b_1,a_2+b_2,\cdots)$$
	$$(a_1,a_2,\cdots)\times(b_1,b_2,\cdots)=(c_1,c_2,\cdots),c_n=\sum_{k=0}^n a_kb_{n-k}$$
	
	那么这些运算结果都是终端为0的序列.记$(0,1_R,0,\cdots)=x$,称为未定元,未定元是$R[x]$的中心元.第$n+1$分量为$1_R$,其余为0的元就是$x^n$,于是每个多项式可以唯一的表示做:$a_nx^n+a_{n-1}x^{n-1}+\cdots+a_1x+a_0,a_n\not=0$.称这里的$a_i$为多项式的系数,称$a_n$是首系数,$a_0$是常数项.另外$R$交换当且仅当$R[x]$是交换环.最后,存在从$R$到$R[x]$的典范映射$r\mapsto(r,0,0,\cdots)$.
	
	称环上一个元$a$是\textbf{幂零元},如果它的某个次幂是0,非0的幂零元自然不会出现在整环中.下面我们对交换的多项式环中的单位,幂零元和零因子进行描述.
	\begin{enumerate}
		\item 单位.一个多项式是单位当且仅当它的常数项是单位并且其余系数项都是幂零元.
		\begin{proof}
			先来证明,如果$x$是交换环上幂零元,那么$1-x$是单位.事实上如果$x^n=0$,得到$(1-x)(1+x+x^2+\cdots+x^{n-1})=1$.这个事实还告诉我们,如果$x$是幂零元,$u$是单位,那么$x+u$是单位.
			
			接下来说明,幂零元的和仍然是幂零元.事实上如果有$x^m=y^n=0$,那么$(x+y)^{m+n}$的展开中,每一项要么含有$x^m$要么含有$y^n$,于是恒为0.由此说明如果$a_1,\cdots,a_n$是幂零元,那么$a_1x+a_2x^2+\cdots+a_nx^n$是幂零元,于是按照上一段得到,如果$a_0$是单位那么$a_0+a_1x+\cdots+a_nx^n$是单位.
			
			反过来,取定交换环$R$上的多项式$f=a_0+a_1x+\cdots+a_nx^n$,设它是单位,设逆元为$g=b_0+b_1x+\cdots+b_mx^m$,那么按照$fg=1$,得到$a_0b_0=1$,于是$a_0$是单位.为了证明$a_1,\cdots,a_n$都是幂零元,只要说明$a_n$是幂零元,据此得到$-a_nx^n$是幂零元,于是$f-a_nx^n=a_0+a_1x+\cdots+a_{n-1}x^{n-1}$也是单位,这就可以归纳说明$a_1,\cdots,a_n$都是幂零元.
			
			考虑$fg=1$的最高次项,得到$a_nb_m=0$,考虑第二高次项,得到$a_nb_{m-1}+a_{n-1}b_m=0$,两边乘以$a_n$得到$a_n^2b_{m-1}=0$,归纳得到$a_n^{r+1}b_{m-r}=0$,于是有$a_n^{m+1}b_0=0$,按照$b_0$是单位就得到$a_n^{m+1}=0$.
		\end{proof}
		\item 幂零元.$f=a_0+a_1x+\cdots+a_nx^n$是幂零元当且仅当$a_0,a_1,\cdots,a_n$都是幂零元.
		\begin{proof}
			一方面按照幂零元的和是幂零元,说明$a_0,a_1,\cdots,a_n$是幂零元的时候$f$是幂零元.另一方面,如果$f$是幂零元,那么$a_0$本身必然是幂零元.按照上一个证明中我们所用的幂零元加单位是单位,于是$f+1$是单位,于是再利用上一段结论得到$a_1,\cdots,a_n$都是幂零元.
		\end{proof}
		\item 零因子.$R$上多项式$f=a_0+a_1x+\cdots+a_nx^n$是左零因子,当且仅当存在$R$中非0元$b$零化了$f(x)$的全部系数.
		\begin{proof}
			设$g=b_0+b_1x+\cdots+b_mx^m$是次数最小的非零多项式满足$fg=0$,于是得到$a_nb_m=0$,那么我们断言$a_ng=0$,否则$a_ng$是一个次数比$g$更小的非零多项式满足$a_ngf=0$.于是$a_n$零化了所有$b_i$.于是有$f_1=a_0+a_1x+\cdots+a_{n-1}x^{n-1}$满足$f_1g=0$,那么$a_{n-1}b_m=0$,我们断言$a_{n-1}g=0$,否则$a_{n-1}g$是一个次数比$g$更小的非零多项式满足$a_{n-1}gf=0$,归纳下去,我们看到总有$a_ib_j=0,\forall 0\le i\le n,0\le j\le m$.于是$b_i$中的非零元就零化了$f$.
		\end{proof}
	\end{enumerate}
	\item \textbf{多元多项式环}.$R$是一个环,其上多项式环$R[x_1,x_2,\cdots,x_n]$作为集合是全体$\mathbb{N}^n\to R$的映射的子集,要求只在有限个点处不为0.在其上定义加法和乘法为:
	$$(f+g)(u)=f(u)+g(u);(fg)(u)=\sum_{i+j=u,i,j\in N^n}f(i)g(j)$$
	
	这构成一个环.记在$\varepsilon_i=(0,0,\cdots,1,\cdots,0)\in N^n$取$1_R$,其余$\varepsilon_j$取0的$R[x_1,\cdots,x_n]$中元素为$x_i$,称$x_1,x_2,\cdots,x_n$为未定元.于是多元多项式环的每个元可以唯一的表示做我们熟知的形式$\sum_{i_1,i_2,\cdots,i_n}a_{i_1,i_2,\cdots,i_n}x_1^{i_1}x_2^{i_2}\cdots x_n^{i_n}$.
	
	现在以幺半群环的观点描述多项式环.一个幺半群是指集合上赋予一个二元运算,满足结合律并且存在幺元,这个二元运算记作乘法.现在取一个幺半群$M$,再取一个环$R$,幺半群环$R[M]$作为交换群是直和$R^{\oplus M}$,既加法是:
	$$\sum_{m\in M}a_mm+\sum_{m\in M}b_mm=\sum_{m\in M}(a_m+b_m)m$$
	
	约定乘法为:
	$$\left(\sum_{m\in M}a_mm\right)\left(\sum_{m\in M}b_mm\right)=\sum_{m\in M}\sum_{st=m}(a_sb_t)m$$
	
	于是多元多项式环$R[x_1,\cdots,x_n]$可以看作幺半群环$R[M]$,这里$M$是由$\{x_1,\cdots,x_n\}$生成的自由幺半群.单元多项式环可以理解为幺半群环$R[\mathbb{N}]$.特别的,幺半群也可以直接取为群,这时称为群环.
	\item \textbf{形式幂级数环}.$R$是一个环,其上形式幂级数环$R[[x]]$作为集合是$R$上全体序列,即不再约定终端于0:
	$$R[[x]]=\{(r_1,r_2,\cdots),r_i\in R\}$$
	
	在其上定义加法和乘法为:
	$$(a_1,a_2,\cdots)+(b_1,b_2,\cdots)=(a_1+b_1,a_2+b_2,\cdots)$$
	$$(a_1,a_2,\cdots)\times(b_1,b_2,\cdots)=(c_1,c_2,\cdots),c_n=\sum_{k=0}^n a_kb_{n-k}$$
	
	这构成一个环.记$(0,1,0,\cdots)=x$,称为未定元,于是每个元可以唯一的表示做:$\cdots+a_nx^n+a_{n-1}x^{n-1}+\cdots+a_1x+a_0$.多项式环$R[x]$是它的子环.$R$的交换性,整环性都可以传递给$R[[x]]$.
	
	交换的形式幂级数环上单位和幂零元的描述和多项式环上很不同.
	\begin{enumerate}
		\item 形式幂级数环上一个元$\sum_{n\ge0}a_nx^n$是单位当且仅当$a_0$是$R$中单位,这是因为可以反复构造$b_n$使得$g=\sum_{n\ge0}b_nx^n$是逆,即$b_0=a_0^{-1}$,$b_1=-b_0a_1a_0^{-1}$,$\cdots$.
		
		\item 对于幂零元,$\sum_{n\ge0}a_nx^n$是幂零元可推出$a_n$都是幂零元,这个只要归纳即可.但是反过来的命题是错误的.在多项式的情况下,我们是利用$a_k$都是幂零元,如果设$a_k$的幂零指数是$e_k$,证明了$e=\sum e_k$满足$f^e=0$.但是在形式幂级数环的情况下,这个$\sum e_k$可以不再是有限数.一个具体的反例是,考虑$R=\mathbb{F}_2(t,t^{1/2},t^{1/3},\cdots)/(t)$,取$a_n=t^{1/n}$,取$f=\sum_{n\ge1}a_nx^n$,那么对任意正整数$k$,由于特征2有$f^{2^k}=\sum_{n\ge1}t^{2^k/n}x^n$总不是零多项式.这就保证了必然不存在正整数$m$满足$f^m=0$.事实上在特征大于0的交换环$R$上,一个形式幂级数$f=\sum_{n\ge0}a_nx^n$是幂零元当且仅当$a_n$的幂零指数$e_n$构成的正整数点列$\{e_n\}$是有界的.另外如果$R$是诺特环,也可以得到$a_n$都是幂零元推出$f$是幂零元.
	\end{enumerate}
\end{enumerate}

理想.一个环$R$上的交换群子群$I$称为一个左理想,如果对任意$r\in R,a\in I$有$ra\in I$.称为右理想,如果对任意$r\in R,a\in I$有$ar\in I$,称为双侧理想如果它同时是左理想和右理想.在交换环的情况下左右理想是没有差异的.在不强调左右的情况下我们提及理想指双侧理想.

{0}是环上最简单的理想,称为零理想.另外按照定义环也是自身的理想,称为单位理想.事实上如果一个理想含有单位,那么它是单位理想.除此情况外,理想一般都不是子环,而是rng.反过来子环一般也未必是理想.由于域上非0元都是单位,于是域中的理想只能是0理想和单位理想.如果理想不是零理想也不是单位理想,就称它是真理想.

例子.考虑某个除环上的$n$阶矩阵构成的环$R$,记全体第$k$列为0的矩阵构成的子集为$I_k$,这是一个左理想,但不是右理想.对偶的可以构造右理想不是左理想.尽管这个环$R$有真左理想和真右理想,但是它没有真(双侧)理想.【】

尽管理想一般不是环的子结构,但是我们会经常考虑到它,这主要有两个原因.首先,我们会看到理想对应着环同态的核,它扮演的角色相当于群中的正规子群.另一个原因是,理想的确是环的一种子结构,它就是环作为自身模的子模.在介绍模后我们会逐渐习惯用环上的模反映环的性质这一思想.

我们已经给出一些环的实例和性质,现在是引出环范畴的时候了.环涉及了三个不同的范畴,全体rng作为对象,态射定义为映射$f:R\to S$,满足$f(a+b)=f(a)+f(b)$,这保证了$f$是交换群之间的群同态,以及$f(ab)=f(a)f(b)$,这保证了$f$是乘法半群之间的同态.这个范畴记作\textbf{Rng}.如果以全体环作为对象,那么此时环相比rng还具备乘法幺元,此时态射的定义还要满足保幺元$f(1_R)=1_S$,这样的环之间的映射就称为环同态.把对象限制为交换环的\textbf{Rng}的完全子范畴记作\textbf{CRing}.环在失去乘法幺元的同时会失去很多有趣的性质,这使得我们把注意力聚集在有幺元的环.注意,按照定义,0环是有乘法幺元的.

环同态保单位,即单位在环同态下的像也是单位.但是在环同态下零因子/非零因子并不仍为零因子/非零因子.环范畴中的态射要求了保幺元,这个要求实际上比较强,它导致了并不是对任意两个环之间总存在环同态.另外和群的情况一样,一个环同态如果是双射,那么它的逆映射是自动为环同态的,此时环同态就是同构.

\begin{enumerate}
	\item 0环是环范畴中的终对象,环范畴的初对象是整数环$Z$.即从$\mathbb{Z}$到任意一个环$A$存在唯一的环同态$n\mapsto n1_A$.这个性质允许我们定义含幺环上的特征.对含幺环$R$,唯一的$\mathbb{Z}\to R$的核是$\mathbb{Z}$的理想,这个理想可以唯一的表示为$(n)$,其中$n$是非负整数,那么就称$R$的特征是$n$.环$R$的特征的等价描述是,它是最小的非负整数$n$使得$n1_R=0$.那么如果$R$非0环并且是没有非平凡零因子的环,则特征$n$必须是素数,否则就有$0=n1_R=(n_11_R)(n_21_R)$.对每个非负整数$n$都存在以这个数为特征的环,例如$\mathbb{Z}/n$.
	
	这个性质可以说明的确存在两个含幺环之间不会存在环同态.事实上如果$f:R\to S$是环同态,如果$n>0$为$R$的特征,那么$0=f(n1_R)=nf(1_R)=n1_S$,于是$S$的特征必须被$R$的特征整除.
	
	\item 商对象.按照之前在集合范畴和群范畴的做法,需要在环上做一个划分,并且让等价类构成的集合也即商具备一个环结构,使得从环到商的自然映射构成一个环同态.那么,部分工作我们已经在群的情况完成了,因为环也是交换群,环同态同样是交换群同态,这说明商的形式必然是$R/I$的形式,这里$I$是$R$的一个(必然交换的)子群,自然映射是$r\mapsto r+I$.现在需要让自然映射是一个环同态,这意味着$(a+I)(b+I)=ab+I,\forall a,b\in R$.现在需要让这个乘法定义是良性的,即如果$a_1+I=a_2+I,b_1+I=b_2+I$,那么应该有$a_1b_1+I=a_2b_2+I$,如果取$b_1=b_2=r$, 得到$\forall a\in I,r\in R$有$ar\in I$,同理取$a_1=a_2$得到$\forall a\in I,r\in R$有$ra\in I$,这告诉我们需要$I$是一个(双侧)理想.反过来,如果取$I$是理想,那么这个商的确构成了一个环,$1_R+I$是它的乘法幺元.于是环的商和它的理想一一对应.如果环$R$交换,那么它的商环也是交换的.
	
	商环的泛映射性质.对一个环$R$,考虑它关于理想$I$的陪集划分$\sim$.现在定义一个范畴,它的对象是全体对$(A,f)$,其中$A$是环,$f$是从$R$到$A$的环同态使得$f$在等价类上具有相同的取值,注意到这一点告诉我们$I\subset\ker f$.从$(A,f)$到$(B,g)$的态射是从$A$到$B$的环同态$\varphi$,使得如下图表交换.那么$R$关于$I$的商环就是这个范畴的初对象.
	$$\xymatrix{
		A\ar[rr]^{\varphi}&&B\\
		&R\ar[ul]^{f}\ar[ur]_{g}&
	}$$
	\item 补充多项式环的泛映射性质.给定环$R$,定义新范畴,对象是$n+2$串$(j,S,s_1,\cdots,s_n)$,其中$s_1,\cdots,s_n$是$S$中可以重复的$n$个中心元,$S$是一个环,$j:R\to S$是环同态.定义从$(j,S,s_1,\cdots,s_n)$到$(k,T,t_1,\cdots,t_n)$的态射是一个环同态$\varphi:S\to T$,满足$\varphi\circ j=k$,并且$\varphi(s_i)=t_i,\forall 1\le i\le n$.那么$(l,R[x_1,\cdots,x_n],x_1,\cdots,x_n)$就是这个范畴中的一个初对象.事实上,任取$(j,S,s_1,\cdots,s_n)$,任取$f=\sum_J a_Jx^J$,我们定义$\varphi:R[x_1,\cdots,x_n]$到$T$的环同态为$\varphi(f)=\sum_J k(a_J)s^J$.并且这是在满足$\varphi$限制在$R$上是$k$和把每个$x_i$映射到$s_i$的意义下是唯一的.
	
	利用泛映射性质可以证明,给定环$R$,存在同构$R[x_1,\cdots,x_k][x_{k+1},\cdots,x_n]\cong R[x_1,\cdots,x_n]\cong R[x_{k+1},\cdots,x_n][x_1,\cdots,x_k]$.我们来证明第一个同构.首先给定环同态$\varphi:R\to S$,取定$s_1,\cdots,s_n\in S$.那么按照泛映射性质,存在环同态$\varphi':R[x_1,\cdots,x_k]\to S$,满足$\varphi'$在$R$上的限制就是$\varphi$,并且$\varphi'(x_i)=s_i,1\le i\le k$.再利用泛映射性质,得到$\varphi'':R[x_1,\cdots,x_k][x_{k+1},\cdots,x_n]\to S$,满足$\varphi''$在$R[x_1,\cdots,x_k]$上的限制就是$\varphi'$,并且$\varphi''(x_i)=s_i,\forall 1\le i\le n$.证明这样的延拓是唯一的,于是按照泛映射性质,得到同构$R[x_1,\cdots,x_k][x_{k+1},\cdots,x_n]\simeq R[x_1,\cdots,x_n]$.
	\item 称环同态在0元的原像为核,记作$\ker f=\{x\in R\mid f(x)=0\}$.那么核总是环的理想.另外$R$的理想$I$必然是商同态$R\to R/I$的核,于是理想与核是等价的概念.另外商和环同态的满射像也是等价的概念.一方面,商$R/I$是典范同态$R\to R/I$的满射像.为了证明满射像同构于商.就需要环的同构定理.
	
	\textbf{环的第一同构定理}:如果$\varphi:R\to S$是环同态,那么存在同构$\mathrm{im}\varphi\cong R/\ker\varphi$.这个定理保证了满射像总可以表示为商的形式.类似群的情况,有如下标准分解:若$f:R\to S$是一个环同态,那么:
	$$\xymatrix{
		R\ar[r]_{\pi}\ar@/^2pc/[rrr]^{f}&R/\ker f\ar[r]_{\sim}&\mathrm{im}f\ar[r]_l&S
	}$$
	
	\textbf{环的第三同构定理}:设$I$是$R$的理想,那么商环$R/I$的理想具有形式$J/I$,其中$J$是包含理想$I$的$R$的理想,并且有环同构$\frac{R/J}{J/I}\cong R/I$.这个定理可以得到\textbf{理想对应定理}或者叫理想的格定理:如果$I$是$R$的理想,那么存在$R$的包含理想$I$的理想和$R/I$的理想之间的保序一一对应$J\mapsto J/I$.
	
	描述\textbf{环的第二同构定理}需要涉及rng,一个更合理的处理它的地方是在模,这里我们只能描述为rng之间的一个同构:如果$I,J$是环$R$的理想,那么存在rng的同构$I/I\cap J\cong (I+J)/J$.
	\item 环范畴中的积存在,它就是相应交换群的直和上,乘法定义为相应分量中的元素乘法.环范畴上的余积也存在,但是描述起来很复杂,首先它并不是环的直和,另外若干非0环的余积甚至可以是一个0环.在它的(full)子范畴交换环范畴中,两个交换环的余积是它们(作为$\mathbb{Z}$代数)的张量积.这个内容将在张量积部分证明.
	\item 等化子和余等化子.两个环同态$f,g:R\to S$的等化子和群的情况一样,它就是$R$的子环$\{r\in R\mid f(r)=g(r)\}$.这两个环同态的余等化子是$S$商去由全体$\{f(r)-g(r)\mid r\in R\}$生成的理想的环.
	
	于是环范畴和交换环范畴都是完备范畴和余完备范畴.
	\item 核与余核.由于环范畴和交换环范畴上没有零对象,导致核与余核没法严格定义.在环范畴中等同于核与余核的概念是核对(kernel pair)与余核对(cokernel pair).一个环同态$f:R\to S$的核对定义为两个$f$的纤维积,这吻合于我们之前的定义$\ker f=\{r\in R\mid f(r)=0_S\}$.对偶的$f$的余核定义为两个$f$的纤维和.
	\item 在环范畴中,环同态是monic等价于核对是${0}$等价于环同态作为集合是单射.
	\begin{proof}
		
		环同态也是交换群同态,于是核为${0}$是等价于它是单射的.另外单射总能推出态射是monic.最后来证明从态射monic能推出核平凡.设$\varphi:R\to S$是环同态,任取$r\in\ker\varphi$,按照多项式环的泛映射性质,可以定义环同态$\mu_r:\mathbb{Z}[x]\to R$为$f(x)\mapsto f(r)$.那么有$\varphi\circ\mu_r=\varphi\circ\mu_0$,导致$\mu_r=\mu_0$,于是$r=\mu_r(x)=\mu_0(x)=0$,就得到$\ker\varphi={0}$.
	\end{proof}
	
	但是epi和满射之间并不是等价的,环同态中存在epi但不是满射,例子可以考虑嵌入映射$l:\mathbb{Z}\to\mathbb{Q}$.如果有环同态$f,g:\mathbb{Q}\to R$,满足$f\circ l=g\circ l$,按照分式化的泛映射性质,就得到$f(m/n)=f(m)/f(n)=g(m)/g(n)=g(m/n)$.于是$l$总是epi,但是它明显不是满射.
\end{enumerate}

环和它的交换群结构的自同态环的联系.如果$G$是群,$H$是交换群,那么$\mathrm{Hom}_{Grp}(G,H)$按照加法和复合作为乘法构成一个环.特别的对于一个交换群$G$,$\mathrm{End}(G)=\mathrm{Hom}_{Grp}(G,G)$总是一个环,称为$G$的自同态环.环中一个元左乘是它作为交换群的一个自同态,事实上这得到了从$R$到$\rm{End}_{Ab}(R)$的群同态$\varphi$.注意到这个同态是单的,因为对$r,s\in R$,如果像是相同的自同态,那么在$1_R$上的作用结果也是相同的,这导致$r=s$.这个群同态$\varphi$几乎是一个环同态,因为在乘法上它满足的是$\varphi(rs)=\varphi(s)\circ\varphi(r)$.如果$R$本身是交换环,那么这个单的群同态实际上就是一个单的环同态了.

$\mathbb{Z}$作为交换群,存在多种乘法使得它构成一个rng.注意到交换群$(\mathbb{Z},+)$和$(n\mathbb{Z},+)$是同构的,据此可以把$\mathbb{Z}$上标准的环结构传递给$(n\mathbb{Z},+)$,即$\forall a,b\in \mathbb{Z}$,定义$a\circ b=nab$.这得到了一个rng结构,对于选取不同的$n$它们是不同构的.但是$\mathbb{Z}$上仅有一种乘法定义使得它是含幺环.

首先,交换群$\mathbb{Z}$上的自同态环就是环$\mathbb{Z}$赋予常义的乘法结构.
\begin{proof}
	
	考虑映射$\varphi:\mathrm{End}_{Ab}(\mathbb{Z})\to\mathbb{Z}$为$\alpha\mapsto\alpha(1)$.那么这是一个环同态.再构造它的逆(环)同态为,对每个$a\in\mathbb{Z}$,定义$\psi(a)$是$\mathbb{Z}$上的自同态$n\mapsto an$.于是得到$\varphi$是一个环同构.
\end{proof}

$\mathbb{Z}$在同构意义下只存在一种乘法结构使得它成为一个含幺交换环.
\begin{proof}
	
	取$R$为一个交换环,它的加法群结构是$(\mathbb{Z},+)$,那么存在$R$到$\mathrm{End}_{Ab}(R)$的单的环同态.后者同构于$\mathbb{Z}$赋予常义的乘法结构构成的环.这个单同态和同构的复合就是映射$r\mapsto\left(\varphi(r):s\mapsto rs\right)\mapsto r1=r$.于是这个复合是满射.于是存在环同构$R\cong\mathbb{Z}$.
\end{proof}

中国剩余定理.如果$I_1,\cdots,I_k$是交换环$R$的理想,并且两两互素,那么典范同态$\varphi:R\to R/I_1\times\cdots\times R/I_k$是满同态,它的核是$I_1I_2\cdots I_k=I_1\cap I_2\cap\cdots\cap I_k$,于是有同构:
$$R/I_1I_2\cdots I_k\cong R/I_1\times R/I_2\times\cdots\times R/I_k$$
\begin{proof}
	
	引理1:如果$I_1,\cdots,I_k$是交换环$R$的两两互素的理想,那么$I_1I_2\cdots I_k=I_1\cap I_2\cap\cdots\cap I_k$.
	
	问题可以归结为证明$k=2$.事实上如果$I,J$互素,那么一方面$IJ\subset I\cap J$,另一方面$I\cap J=(I\cap J)(I+J)\subset IJ$.
	
	引理2:如果$I_1,\cdots,I_k$是交换环$R$的理想,满足$I_i+I_k=(1),i=1,2,\cdots,k-1$,那么有$I_1I_2\cdots I_{k-1}+I_k=(1)$.
	
	按照条件,对每个$1\le i\le k-1$,存在$a_i\in I_k$使得$1-a_i\in I_i$.于是$\prod_{1\le i\le k-1}(1-a_i)\in I_1\cdots I_{k-1}$.而展开乘积得到$1-\prod_{1\le i\le k-1}(1-a_i)\in I_k$.于是得证.
	
	现在证明原定理.首先$\varphi$的核是$I_1\cap I_2\cap\cdots\cap I_k$,按照引理1这就是$I_1I_2\cdots I_k$.于是只需证明$\varphi$是满同态.对$k$归纳,当$k=1$时结论是平凡的.假设$k>1$,并且对$<k$的所有情况都成立.于是有同构$R/I_1I_2\cdots I_{k-1}\cong R/I_1\times\cdots\times R/I_{k-1}$.现在设典范同态$\varphi:R\to R/I_1I_2\cdots I_{k-1}\times R/I_k$.由引理2知$I_1\cdots I_{k-1}$和$I_k$互素,于是证明满射性归结为证明$k=2$的情况.
	
	设$I,J$是$R$的互素的理想,记$\varphi:R\to R/I\times R/J$.任取$r_I,r_J\in R$,现在证明存在$r\in R$使得$\varphi(r)=(r_I+I,r_J+J)$.从$I+J=(1)$得到$a\in I,b\in J$使得$a+b=1$,取$r=ar_J+br_I$,那么有$r+I=r_I+I$,$r+J=r_J+J$.于是$\varphi$是满同态.
\end{proof}
\newpage
\subsection{素理想和极大理想}

本节里,当提及理想的时候,它可能指左理想,右理想或者双侧理想.对于双侧理想我们不简称为理想,而是严格称作双侧理想.称一族理想是相同型的,如果它们同为左理想或者同为右理想或者同为双侧理想.另外环总指含幺元的rng,交换环总指交换并且含幺元的rng.

理想的和.给定两个理想$I,J$,定义它们的和为$R$的子集$I+J=\{i+j\mid i\in I,j\in J\}$.这个子集未必是理想,如果$I,J$具有相同型,那么和也是相应型的理想.理想的和满足结合律,即对理想$I,J,K$,总有子集的等式$(I+J)+K=I+(J+K)$.这使得我们可以定义有限个理想的和,即$\sum_{1\le k\le n} I_k=\{\sum_{1\le k\le i}i_k\mid i_k\in I_k\}$.这个子集一般也未必是理想,当全部$I_k$具有相同型的时候,和也是相同型的理想.类似的还可以定义一族理想$\{I_s,s\in S\}$的和$\sum_{s\in S}=\{\sum_{s\in S_0\subset_{\mathrm{Fin}}S}i_s\mid i_s\in I_s\}$即为$\cup_{s\in S}I_s$中有限个元的和构成的子集,那么理想族的和也未必是理想,如果所有$I_k$同为相同型的理想,那么和也是相应型的理想.同型的理想族的和就是包含这族理想的最小的同型理想.

理想的积.给定两个理想$I,J$,定义它们的积是$R$的子集$IJ=\{\sum ij\mid i\in I,j\in J\}$.在非交换环中积是有序的,即$IJ$和$JI$可能不相等.理想的积未必是理想,如果$I$是左理想,那么$IJ$也是左理想.如果$J$是右理想,那么$IJ$是右理想.理想的积满足结合律,即对理想$I,J,K$,有$(IJ)K=I(JK)$,这使得我们可以定义有限个理想的乘积,即$I_1I_2\cdots I_r=\{\sum i_1i_2\cdots i_r\mid i_s\in I_s,1\le s\le r\}$.这个子集也未必是理想,如果$I_1$是左理想,那么$\prod I_s$是左理想,如果$I_r$是右理想,那么$\prod I_s$是右理想.另外注意到一般来讲理想的积$IJ$未必包含于$I$或$J$中,但是对于双侧理想$I,J$有$IJ\subset I\cap J$.最后理想的和与积满足分配律,即$B\sum_i A_i=\sum_i BA_i$和$(\sum_iA_i)B=\sum_iA_iB$

一族同型理想的交仍然为同型的理想,环的一个子集生成的某型的理想约定为全体包含这个子集的这个型的理想的交.这也就是包含这个子集的最小的这个型的理想.我们先来表述单个元生成的理想,由单个元生成的理想称为主理想.
\begin{enumerate}
	\item 在rng中,可能不含幺元使得单个元生成的理想的表述比较复杂.$a$生成的左理想就是$\{ra+na\mid r\in R,n\in\mathbb{Z}\}$.生成的右理想为$\{ar+na\mid r\in R,n\in\mathbb{Z}\}$.$a$生成的双侧理想中元的一般形式是$ra+as+na+\sum r_ias_i$其中$r,s,r_i,s_i\in R,n\in\mathbb{Z}$.
	\item 在交换rng中,左右理想和双侧理想是一致的,所以可以统一表述为$\{ra+na\mid r\in R,n\in\mathbb{Z}\}$.
	\item 在环中,此时存在幺元,于是$na$可以表示为$(n1_R)a$,于是$a$生成的左理想是$Ra=\{ra\mid r\in R\}$,右理想是$aR=\{ar\mid r\in R\}$,双侧理想是$RaR=\{r_1ar_2\mid r_1,r_2\in R\}$.
	\item 在交换环中,左右理想和双侧理想全部一致,此时$a$生成的理想就是$Ra$.
\end{enumerate}

理想的和就是包含这族理想的最小理想.于是由子集$\{a_i\}_{i\in I}$生成的某型的理想就是和$\sum_{i\in I}(a_i)$,这里$(a_i)$表示$a_i$生成的该型的理想.

素理想.设$R$是rng,它的一个双侧理想$P$称为素理想,如果它不是$R$,并且对任意双侧理想$A,B$满足$AB\subset P$,可以推出$A,B$中至少一个包含于$P$中.素理想具有如下等价描述:
\begin{enumerate}
	\item $P$是素理想.
	\item 双侧理想$P$不是整个rng,满足若$r,s\in R$满足$rRs\subset P$,那么$r,s$中至少一个属于$P$.
	\item 双侧理想$P$不是整个rng,满足若$(r),(s)$是主的双侧理想,那么$(r)(s)\subset P$推出$(r),(s)$中至少一个包含于$P$.
	\item 双侧理想$P$不是整个rng,满足若$A,B$是左理想,满足$AB\subset P$,那么$A,B$中至少一个包含于$P$.
	\item 双侧理想$P$不是整个rng,满足若$A,B$是右理想,满足$AB\subset P$,那么$A,B$中至少一个包含于$P$.
	\item 双侧理想$P$不是整个rng,并且$P$在$R$中的补集$S$满足,对任意$a,b\in S$,存在$R$中的$r$使得$arb\in S$.
\end{enumerate}
\begin{proof}
	
	1推2,如果$P$是素理想,设$rRs\subset P$,那么有$(RrR)(RsR)\subset P$,于是得到$RrR$和$RsR$中至少一个包含于$P$,不妨设$RrR\subset P$,那么有$(r)(r)(r)\subset RrR\subset P$,于是$(r)\subset P$,于是$r\in P$.2推3只需注意到$rRs\subset(r)(s)$.3推4,如果右理想$A,B$满足$AB\subset P$,假设存在$A$中的元$a\not\in P$,任取$B$中的元$b$,那么有$aRb\subset AB\subset P$,于是$b\in P$,于是$B\subset P$.同理3推出5.接下来4推1,任取双侧理想$A,B$满足$AB\subset P$,如果存在$A$中的元$a$不属于$P$,任取$B$中的元$b$,记$a$生成的左理想为$A_1$,$b$生成的左理想为$B_1$,那么$A_1B_1\subset AB\subset P$,于是$B_1\subset P$,于是$b\in P$,于是$B\subset P$.同理得到5推出1.最后2和6是明显等价的.完成证明.
	
\end{proof}

添加条件,素理想还有如下两个等价描述:
\begin{enumerate}
	\item 在交换rng上,双侧理想$P$是素理想当且仅当$P$不是整个环,并且满足$a,b\in R$满足$ab\in P$,那么$a,b$中至少一个元属于$P$.
	
	在rng上充分性成立,因为如果$A,B$是双侧理想使得$AB\subset P$,假设$A\not\subset P$,那么存在$a\in A$使得$a\not\in P$,那么对任意$b\in B$,有$ab\in P$,于是$b\in P$,于是$B\subset P$,于是$P$是素理想.
	
	添加交换条件下必要性成立.如果$P$是素理想,如果$a,b\in R$使得$ab\in P$,那么$(ab)\subset P$,从交换性得到$(a)(b)\subset (ab)$,于是$(a),(b)$中至少一个包含于$P$,于是$a,b$中至少一个属于$P$.
	
	这里交换性条件是必须的,考虑除环上的$n$阶方阵构成的环$S$,那么这个环有非平凡零因子,于是会存在均不为0的元$a,b$满足$ab=0$.我们证明过$S$是单环,即不存在非平凡的双侧理想,于是0理想是一个素理想,这就导致命题不成立.
	\item 在交换环中,双侧理想$P$是素理想当且仅当$P$不是整个环,并且$R/P$是整环.
	\begin{proof}
		
		如果$P$是素理想,并且$R/P$中有$(a+P)(b+P)=P$,这等价于$ab\in P$,于是得到$a,b$中至少一个属于$P$,于是$R/P$是整环.反过来如果$R/P$是整环,那么$ab\in P$得到$(a+P)(b+P)=P$,于是$a+P,b+P$中至少一个是$R/P$中的零元$P$,于是$a,b$中至少一个属于$P$.
	\end{proof}
\end{enumerate}

极大理想.素理想只对双侧理想定义,而极大理想是对任意型的理想定义的.设$R$是rng,一个理想$M$称为某个型的极大理想,如果$M\not=R$,并且对任意和$M$同型的理想$N$,如果满足$M\subset N\subset R$,那么有$N=M$或$N=R$.换句话说,指定理想的某个型,记全体这个型的真理想构成的集合为$S$,这个集合上赋予包含序,那么极大元就称为这个型的极大理想.

\textbf{Krull定理}.在环中每一个型的极大理想总是存在的.利用Zorn引理甚至能证明更强的结论成立:指定一个理想,那么包含这个理想的同型的极大理想必然存在.
\begin{proof}
	
	零理想是每一个型的理想,所以只需证明更强的命题,就能说明第一个命题成立.给定理想$I$,把$R$中所有和$I$同型并且包含$I$的理想构成的集合记作$S$.那么$S$上可以赋予包含序,即$B_1,B_2\in S$,$B_1\le B_2$当且仅当$B_1\subset B_2$.现在取链族$\{C_i\mid i\in I\}\subset S$,这里链族是指,对任意$i,j\in I $,有$C_i\le C_j$或者$C_j\le C_i$中一个成立.换句话说链就是偏序集的全序子集.取$C=\cup_{i\in I}C_i$,我们断言$C$是一个和$I$同型的理想.
	
	设$a,b\in C$,那么存在$i,j\in I$使得$a\in C_i,b\in C_j$,不妨设$C_j\subset C_i$,那么$a,b\in C_i$,于是$a-b\in C_i$,这说明$C$是一个交换群.接下来无论$I$是哪种型的理想,都有相应的$ra,ar,ras\in C$,于是$C$是一个包含$I$的理想.另外,由于每一个$C_i$都不含幺元,所以$C$不含幺元.于是$C$就是这个链的上界.由Zorn引理,就得到$S$存在极大元.
	
\end{proof}

在rng中极大理想未必存在,在上述证明里,不含幺元的时候我们可能无法证明上界$C$不是整个rng.例如$2\mathbb{Z}/4\mathbb{Z}$.事实上一个交换rng没有极大理想当且仅当它的Jacobson根是自身,并且对每个素数$p$,rng满足$R^2+pR=R$.

极大理想具有如下等价描述.
\begin{enumerate}
	\item 环$R$的左理想$M$是极大左理想当且仅当$M\not=R$,并且$R/M$是一个单左$R$模,即没有非平凡的真子模.
	\item 环$R$的右理想$M$是极大左理想当且仅当$M\not=R$,并且$R/M$是一个单右$R$模,即没有非平凡的真子模.
	\item 环$R$的双侧理想$M$是极大双侧理想当且仅当$M\not=R$,并且$R/M$是一个单环,即没有非平凡的真双侧理想.
	\item 如果$R$是交换rng,那么$M$是极大理想当且仅当$M\not=R$并且$R/M$要么是域要么是$\mathbb{Z}/p$上的乘法约定为乘积0元的rng.
	\item 如果$R$是交换环,那么$M$是极大理想当且仅当$M\not=R$并且$R/M$是域.
\end{enumerate}

即便在交换环上,素理想也未必是极大理想,例如$\mathbb{Z}$中的零理想.在rng中极大理想未必是素理想,例如$2\mathbb{Z}$中,$4\mathbb{Z}$是极大理想但是不是素理想.不过在交换环中,所有极大理想总是素理想.
\begin{proof}
	
	设$M$是交换环$R$的极大理想,假设$M$不是素理想,也就是存在$a,b$使得$ab\in M$,但$a,b\not\in M$,那么$(a)+M$和$(b)+M$是真包含了$M$的理想,按照极大性得到$M+(a)=M+(b)=R$,于是$R=R^2=((a)+M)((b)+M)=(ab)+M=M$,这和$M\not=R$矛盾.
	
\end{proof}

环上的根.粗略的讲,环上的根是指由一些性质不好的元构成的理想.给定一个根性质,就可以定义出环上的一种根.研究根性质的理论是挠理论.这里我们介绍三种常用的根.
\begin{enumerate}
	\item \textbf{幂零根}.幂零元是指某个次幂为0元的元,在一般环上幂零元未必构成一个理想,但是在交换环上是理想,这个理想称为幂零根.交换环上幂零根实际上就是全部素理想的交.
	\begin{proof}
		
		一方面,取一个幂零元$a$,即存在某个正整数$n$满足$a^n=0$,任取素理想$P$,按照素理想的定义,从$a^n=0\in P$推出$a\in P$.反过来,我们证明一个非幂零元$a$,必然存在一个素理想不包含它.对交换环上一个乘性闭子集,它含1不含0,那么全体和它不交的理想中必然存在极大元,并且这个极大元是一个素理想.这里只要取乘性闭子集为$\{1,a,a^2,\cdots\}$就得证.
	\end{proof}
	\item \textbf{Jacobson根}.交换环上全体极大理想的交同样是一个值得提及的理想,它称为交换环的Jacobson根,那么交换环上Jacobson根必然在幂零根中.这个根中的元具有如下刻画:$x\in R$在Jcaobson根中当且仅当,对任意$y\in R$有$1+xy$是单位.
	\begin{proof}
		
		一方面如果$x$落在全部极大理想的交中,倘若存在一个$y$使得$1+xy$不是单位,那么存在一个极大理想$m$包含了$(1+xy)$,于是从$x\in m$推出了$1\in m$这矛盾.另一方面,如果$x$满足对任意$y$有$1+xy$是单位,假设存在一个极大理想$m$不包含$x$,那么有$(x)+m=R$,于是存在$m$中的元$u$和$R$中的元$y$满足$u+xy=1$,导致$u=1-xy$不是单位,这矛盾.
	\end{proof}
	\item \textbf{理想的根}.交换环上给定理想$I$,它的根定义为$\sqrt{I}=\{x\in R\mid \exists n>0,x^n\in I\}$.考虑商环$R/I$,它的全部幂零元构成的理想就是$\sqrt{I}/I$,于是按照理想对应定理,得到理想的根$\sqrt{I}$实际上就是包含$I$的全部素理想的交.
	
	根满足如下基本性质:$I\subset\sqrt{I}$;$\sqrt{\sqrt{I}}=\sqrt{I}$;$\sqrt{IJ}=\sqrt{I\cap J}=\sqrt{I}\cap\sqrt{J}$;$\sqrt{I}$是单位理想当且仅当$I$是单位理想;$\sqrt{I+J}=\sqrt{\sqrt{I}+\sqrt{J}}$;若$p$是素理想,对每个正整数$n$有$\sqrt{p^n}=p$.
	
	如果一个理想等于自己的根,就称它为根理想.那么按照根理想$\sqrt{I}$就是包含$I$的全部素理想的交,说明素理想总是根理想.另外根理想还具有另一个等价描述,这涉及到环的简化(reduced)性.称一个环是简化环如果它没有非平凡的幂零元,即从$a^m=0$推出$a=0$.那么商环$R/I$是简化的当且仅当$I$是根理想:一方面如果$I$是根理想,从$(a+I)^m=0$推出$a^m\in I$推出$a\in\sqrt{I}=I$;另一方面如果$R/I$没有非平凡幂零元,那么从$a^m\in I$得到$R/I$中有$(a+I)^m=0$,于是$a\in I$.
\end{enumerate}

如果$R$是交换环,那么$R[x]$的幂零根和Jacobson根是一致的.只需说明Jacobson根包含于幂零根,任取Jacobson根中的元$f(x)$,那么对任意多项式$g(x)\in R[x]$,有$1-g(x)f(x)$总是$R[x]$中的单位,特别的有$1-xf(x)$是$R[x]$中的单位,这意味着$f(x)$的系数项均为$R$中的幂零元,从而$f(x)$本身是$R[x]$中的幂零元.

在幂零根的性质中得到过全体素理想的交就是环的幂零元集,这里来说明环的零因子集是某些素理想的并.为此只需说明全体由零因子构成的理想的极大元存在,并且总是素理想.其中极大元的存在性依然运用Zorn引理,下面说明极大元$P$是素理想,若否,则存在环中两个元$a,b\not\in P$但满足$ab\in P$,那么$P$真包含于$P+(a)$和$P+(b)$,极大性说明它们都不是全由零因子构成的,因而存在非零因子$x_1=p_1+r_1a$和$x_2=p_2+r_2b$,但是从$(P+(a))(P+(b))\subset P$得到$x_1x_2$是零因子,于是存在环中的非零元$r$满足$rx_1x_2=0$,倘若$rx_1=0$,则$x_1$是零因子,倘若$rx_1\not=0$,则$x_2$是零因子,均矛盾.

素理想avoidance定理.如果$R$是交换环,并且$R$有一个子集$A$在加法和乘法下封闭,如果$\{P_i,1\le i\le n\}$是至多有两个不是素理想的理想集,那么从$A\subset\cup_{1\le i\le n}P_i$可以推出$A$包含在某个$P_i$中.或者等价的说,如果对每个$i$有$A\not\subset P_i$,那么存在$x\in A$不在$\cup_{1\le i\le n}P_i$中.
\begin{proof}
	
	对$n$归纳.$n=1$是平凡的,设$n\ge2$并且$P_3,\cdots,P_n$都是素理想,按照归纳假设,对每个$1\le i\le n$存在一个$x_i\in A$使得$x_i\not\in P_j,j\not=i$.不妨设$x_i\in P_i$,否则已经得证.如果$n=2$,那么明显$x_1+x_2$是不在$P_1\cup P_2$中的$A$中的元素.如果$n\ge3$,考虑$x=x_1\cdots x_{n-1}+x_n$,它不在每个$P_i$中.
\end{proof}

关于直积的理想.给定两个交换环$R_1,R_2$,考虑直积$R=R_1\times R_2$.任取它的理想$I$,那么分别以$(1,0)$和$(0,1)$乘以$I$中的元说明它必然具有形式$I_1\times I_2$,其中$I_j$是$R_j$的理想,反过来也是成立的.另外$R_1\times R_2\times\cdots\times R_n$的素理想恰好具有形式$R_1\times\cdots\times P_t\times\cdots\times R_n$,其中$P_t$是$R_t$的素理想.

局部环.称恰有一个极大理想的交换环为局部环.例如在素理想处的局部化是一个局部环.局部环具有两个等价描述:
\begin{enumerate}
	\item 全部非单位元构成了一个极大理想.
	\item 全部非单位元包含于某个非单位理想中.
\end{enumerate}
\begin{proof}
	
	如果环是局部环,任取一个非单位元$a$,那么$(a)$包含于某个极大理想中,于是局部环的情况下$(a)$在唯一的极大理想$M$中,也即所有非单位元都包含在极大理想中.环中的元要么是单位元要么是非单位元.$M\not=R$说明$M$不含单位元,于是此时全体非单位元构成了唯一的极大理想.
	
	条件1推条件2是直接的.最后假设全部非单位元包含于某个非单位理想$I$中,那么$I$不能含单位元,否则$I=R$.于是此时$I$恰好是由全体非单位元构成的.任取一个非单位理想$J$,则$J$的所有元都是非单位元,于是$J\subset I$,于是$I$是唯一的极大理想.
\end{proof}

设$R$是交换环,我们来讨论下$R$上多项式环和形式幂级数环的极大理想和素理想.
\begin{enumerate}
	\item 先考虑形式幂级数环.我们证明过一个形式幂级数是单位当且仅当它的常数项是$R$中单位.按照$R[[x,y]]\cong R[[x]][[y]]$,就说明$S=R[[x_1,x_2,\cdots,x_n]]$中的元是单位当且仅当常数项是单位.于是对每个$g\in(x_1,x_2,\cdots,x_n)$,有$1+gh$总是单位,这说明$(x_1,x_2,\cdots,x_n)$包含于$S$的Jacobson根中.
	\item 从$R[[x_1,x_2,\cdots,x_n]]/(x_1,x_2,\cdots,x_n)\cong R$说明,如果$R$是域,那么$S$是以$(x_1,x_2,\cdots,x_n)$为唯一极大理想的局部环.如果$R$不是域,那么$S$的每个极大理想会具有形式$mS+(x_1,x_2,\cdots,x_n)$,其中$m$是$R$的极大理想.特别的,此时$S$中的极大理想$m'$总满足$m'\cap R=m$是$R$的极大理想.
	\item 在多项式环上情况会麻烦得多.首先$R[x]$上存在不包含$x$的极大理想.例如$x-1$是非单位元,于是存在极大理想$m$包含它,倘若$x\in m$,这导致$1\in m$矛盾.另外$R[x]$的极大理想$m'$的原像$m'\cap R$也未必是$R$的极大理想.
	\item 从$R[x]/I[x]\cong(R/I)[x]$和$R[[x]]/I[[x]]\cong(R/I)[[x]]$说明如果$P$是$R$的素理想,那么$P[x]$和$P[[x]]$分别是$R[x]$和$R[[x]]$的素理想.
	\item 最后我们注意下$I[[x]]$和$IR[[x]]$的区别.前者是指这样的形式幂级数,它的系数都在$I$中,而后者是指可以表示为$\sum a_if_i$的元,其中$a_i\in I$,$f_i\in R[[x]]$.如果$I$是有限生成的,那么二者是相同的.但是如果$I$非有限生成,可能会初学$IR[[x]]$严格包含于$I[[x]]$的情况.另外对于多项式环这个问题是不存在的,总有$I[x]=IR[x]$.
\end{enumerate}
\newpage
\subsection{整环中的元素分解}

称一个整环$R$中元$a$整除元$b$,如果存在一个元$c$使得$b=ac$,记作$a\mid b$,按照理想语言,这等价于说$(b)\subset(a)$.称两个元$a,b$相伴,如果它们分别整除对方,即$a\mid b,b\mid a$.按照整环没有非平凡零因子,这等价于说$a$和$b$相差一个单位.按照理想语言这等价于$(a)=(b)$.

倘若去掉整环的条件,只约定环是交换环,那么上述等价命题是失效的,例如$\mathbb{Z}/6$, 有$[2]$和$[4]$互相整除,但是它们不相差一个单位的乘积,这说明非整环上元素的整除性复杂得多,这也导致我们把注意力聚集在整环上的整除性.

整数环$\mathbb{Z}$是整除性质的原型,并且我们有最基本的元素,即素数.但是在一般整环上实际存在两种不同的基本的整除性元素,它们抽象为不可约元和素元.只不过在$\mathbb{Z}$的情况它们恰好吻合.
\begin{itemize}
	\item 称整环$R$上一个非零非单位元$p$是素元,如果对任意$a,b\in R$,从$p\mid ab$能推出$p\mid a$或者$p\mid b$,这等价于讲$(p)$是素理想.
	\item 称整环$R$上一个非零非单位元$p$是不可约元,如果从$p=ab$总能推出$a,b$当中有一个是单位.这等价于讲$(p)$在全体主理想集合上是极大元.
	\item 称整环$R$的非零非单位$a$是可分解的,如果$a$可以表示为有限个不可约元的乘积.称整环$R$是可分解的,如果它的每个非零非单位都是可分解的.称整环$R$是唯一分解的(UFD),如果它的每个非零非单位都可以表示为有限个素元的乘积.
\end{itemize}
\begin{enumerate}
	\item 素元的相伴元也是素元,不可约元的相伴元也是不可约元.
	\item 整环上素元总是不可约元.
	\begin{proof}
		
		如果$p=ab$,那么$p\mid ab$,按照$p$是素元,不妨设$p\mid a$,那么存在$c\in R$使得$a=cp$,于是$p=bcp$.由于整环没有非平凡零因子,于是$b$是一个单位,于是$p$是不可约元.
	\end{proof}
    \item 如果整环的每个理想都是主理想,就称它是一个主理想整环或PID.例如$\mathbb{Z}$是一个PID.$\mathbb{Z}$的理想首先是一个子群,而$\mathbb{Z}$的子群只有$n\mathbb{Z}$,并且它是理想,所以$\mathbb{Z}$的理想就是$(n),n\in \mathbb{Z}$,于是$\mathbb{Z}$是PID.
    \item 如果整环$R$的一个非零非单位可以分解为有限个素元的乘积,那么它的这种分解在相伴意义下是唯一的,换句话讲,如果$a=p_1\cdots p_n=q_1\cdots q_m$,其中$p_i,q_i$都是素元,那么$m=n$,并且存在$q_1,\cdots,q_m$的一个重新排序,使得对每个$i$有$p_i$和$q_i$是相伴的.
    \begin{proof}
    	
    	$a=p_1p_2\cdots p_n=q_1q_2\cdots q_m$,考虑素元$p_1$,从$p_1\mid q_1q_2\cdots q_m$就得到$p_1\mid$某个$q_i$,但是按照$q_i$也是素元就得到$p_1$和$q_i$相伴,按照整环没有非平凡零因子,在相差一个单位的意义下我们可以约去等式两侧的$p_1$和$q_i$,归纳操作下去得到结论.
    \end{proof}
    \item 设$R$是整环,设$a\in R$是非零非单位,那么$a$是可分解的当且仅当所有包含$a$的主理想构成的集合满足升链条件(acc),即如果存在一列$(a)\subset (r_1)\subset (r_2)\subset\cdots$,那么存在一个正整数$N$使得$\forall i\in\mathbb{N}^+$有$(r_N)=(r_{N+i})$.进而一个整环是可分解的当且仅当它的主理想构成的集合总满足升链条件.
    \begin{proof}
    	
    	充分性,若$a$不是可分解元,那么$a$不会是不可约元,这说明存在$a=st$ 使得$s,t$都不是单位,那么它们当中必然存在一个元不是可分解元,否则乘积必然是可分解元,不妨设为$r_1$, 于是$(a)\subsetneq(r_1)$,继续对$r_1$重复操作,就得到了一个无限长的严格升链,这和升链条件矛盾.
    	
    	\qquad
    	
    	必要性,取主理想的升链$(a)\subseteq(a_1)\subseteq\cdots$.设$a=p_1^{e_1}\cdots p_n^{e_n}$,这里$p_i$两两不相伴,对每个指标$i$就有$a_i=u_ip_1^{r_{i1}}\cdots p_n^{r_{in}}$,其中$0\le r_{ij}\le e_j$,$1\le j\le n$.按照$(a_i)\subseteq(a_{i+1})$,得到$(r_{ij})_j$是自然数的降链,于是当$i$足够大时总有$(a_i)=(a_{i+1})$.
    \end{proof}
    \item 于是一个整环$R$是唯一分解的当且仅当它满足如下两个条件:
    \begin{enumerate}
    	\item $R$的全体主理想构成的集合上满足升链条件.
    	\item $R$的每个不可约元都是素元.
    \end{enumerate}
    \begin{proof}
    	
    	我们来证明UFD上不可约元$p$都是素元.取$p$的素元乘积分解$p=p_1\cdots p_n$,按照不可约的定义有$n=1$(否则右侧某个素元是单位),于是$p$是素元.
    \end{proof}
    \item 设$A$是整环,那么$A$是UFD当且仅当如下三个条件成立:
    \begin{enumerate}
    	\item $A$对主理想满足升链条件.
    	\item 每个非零非单位元$a$都包含在某个高度1的素理想中.
    	\item 每个高度1素理想都是主理想.
    \end{enumerate}
    
    特别的,如果$A$是诺特整环,那么$A$是UFD当且仅当$A$的高度1素理想都是主的(当然,UFD未必是诺特的).
    \begin{proof}
    	
    	先设$A$是UFD,我们解释了(a)成立.接下来如果$a$是非零非单位,取它唯一分解中的素元$p$,那么$(a)\subseteq(p)$.下面说明$(p)$的高度为1,假设有素理想$\mathfrak{p}\subsetneqq(p)$,任取$a\in\mathfrak{p}$,那么$a=pb$对某个$b\in A$.从$p\not\in\mathfrak{p}$得到$b\in\mathfrak{p}$,所以$b=pc$对某个$c\in A$.同理有$c\in\mathfrak{p}$,归纳下去得到对每个正整数$n$都有$p^n\mid a$,但是按照$A$是UFD,这迫使$a=0$,也即$\mathfrak{p}=0$,即$(p)$的高度为1,这证明了(b).最后设$\mathfrak{p}$是高度1的素理想,任取$\mathfrak{p}$中的非零元$a$,那么它一定非单位,取唯一分解$a=\pi_1\cdots\pi_n$,那么至少某个$\pi_i\in\mathfrak{p}$,但是$(\pi_i)$已经高度1了,所以$(\pi_i)=\mathfrak{p}$是主理想.
    	
    	\qquad
    	
    	反过来,如果整环$A$满足这三个条件.(a)保证了$A$的每个非零非单位都可以写作有限个不可约元的乘积,下面只需证明$A$的不可约元都是素元.设$a$是不可约元,按照(b)可以找到高度1素理想$\mathfrak{p}$包含了$a$,按照(c)这个$\mathfrak{p}$是主理想.但是$a$是不可约元,它生成的主理想已经在主理想集合上极大,于是$\mathfrak{p}=(a)$,于是$a$是素元.
    	
    	\qquad
    	
    	如果$A$是诺特整环,这里条件(a)因为诺特性成立,按照主理想定理,主理想的极小素理想的高度都是1,也即条件(b)总成立.
    \end{proof}
    \item 设$A$是诺特整环,设$S$是乘性闭子集.
    \begin{enumerate}
    	\item 如果$A$是UFD,那么$S^{-1}A$也是UFD.
    	\item 反过来如果$S\subseteq A$是由某些素元构成的集合$\Gamma$生成的乘性闭子集,如果$S^{-1}A$是UFD,那么$A$是UFD.
    \end{enumerate}
    \begin{proof}
    	
    	(a):诺特整环$S^{-1}A$的高度1素理想具有形式$S^{-1}\mathfrak{p}$,其中$\mathfrak{p}$是$A$的高度1素理想,按照$A$是UFD得到$\mathfrak{p}$是主理想,于是$S^{-1}\mathfrak{p}$也是主理想.于是$S^{-1}A$是UFD.
    	
    	\qquad
    	
    	(b):设$\mathfrak{p}$是$A$的高度1素理想,只需证明它是主理想.如果$\mathfrak{p}\cap S$非空,按照素理想定义有$\mathfrak{p}$包含了某个$\Gamma$中的元$\pi$,所以$\mathfrak{p}$包含了非零素理想$\pi A$,但是$\mathfrak{p}$的高度1,迫使$\mathfrak{p}=\pi A$.如果$\mathfrak{p}\cap S$是空集,那么$S^{-1}\mathfrak{p}$是$S^{-1}A$的高度1素理想,所以是主理想,可记$S^{-1}\mathfrak{p}=aS^{-1}A$,其中$a\in A\cap S^{-1}\mathfrak{p}=\mathfrak{p}$.把$a$取为所有满足这个等式的$a$生成的主理想中的极大元.那么$a$不被$\Gamma$中的素元整除.最后如果$x\in\mathfrak{p}$,记$x/1=ay/s$,其中$s\in S,y\in A$,那么$xs=ay$,但是$a$的唯一分解中不含$\Gamma$中的素元,而$s$的唯一分解中只含$\Gamma$中的素元,所以$y\in sA$,于是$x\in aA$,也即$\mathfrak{p}=aA$.
    \end{proof}
    \item PID是UFD.特别的,$\mathbb{Z}$是唯一分解整环,这是算术基本定理.另外PID上素元和不可约元一致这个结论等价于说PID上非0素理想和极大理想一致,也即PID的Krull维数是1.
    \begin{proof}
    	
    	先证明PID上主理想满足升链条件,考虑主理想链$(r_1)\subset (r_2)\subset\cdots$,那么$\cup_{i\ge1}(r_i)$是理想,于是它具有形式$(a)$,按照$a\in\cup_{i\ge1}(r_i)$,说明存在某个$t$使得$a\in (r_t)$,于是$(a)\subset (r_t)$,于是$(a)=(r_t)=(r_{t+1})=\cdots$.
    	
    	接下来证明PID上不可约元和素元是一致的.任取不可约元$a$,假设$a\mid bc$,不妨设$a\not\mid b$,那么$(a)$真包含于$(a,b)$中.按照不可约元生成的理想在全体真主理想中是极大元,结合条件PID,我们看到$(a,b)$是单位理想,于是存在环中的$r,s$使得$ra+sb=1$,于是得到$c=rac+sbc\subset (a)$,于是$a$是素元,完成证明.
    \end{proof}
\end{enumerate}

最大公约数和最小公倍数.设$A$是整环,设$a_1,a_2,\cdots,a_n\in A$.
\begin{itemize}
	\item 它们的最大公约数定义为最小的使得$(a_1,a_2,\cdots,a_n)\subset(d)$的$(d)$,这里最小是指如果$(e)$满足相同的条件,那么$(d)\subseteq(e)$.换句话讲$d$整除每个$a_i$,并且如果$e$是另一个整除每个$a_i$的元,那么$e$整除$d$.最大公约数记作$\mathrm{g.c.d}(a_1,\cdots,a_n)$.
	\item 它们的最小公倍数是最大的使得$(d)\subseteq$每个$(a_i)$的$(d)$,这里最大是指如果$(e)$满足相同的条件,那么$(e)\subseteq(d)$.换句话讲$d$被每个$a_i$整除,并且如果$e$是另一个被每个$a_i$整除的元,那么$d$整除$e$.最小公倍数记作$\mathrm{l.c.m}(a_1,\cdots,a_n)$.
\end{itemize}
\begin{enumerate}
	\item 最大公约数和最小公倍数都是定义为主理想,因为定义为数的时候可以差一个单位.在$\mathbb{Z}$上单位只有$\{pm1\}$,所以最大公约数或者最小公倍数如果写作正整数则是唯一的.
	\item PID上总存在最大公约数和最小公倍数.因为$(a_1,\cdots,a_n)$和$\cap_i(a_i)$都是主理想.
	\item 在一般整环中最大公约数是未必存在的,例如$\mathbb{Z}[x]$中全体一次项系数为0的多项式构成的子环,那么$x^5,x^6$不存在最大公约数.
	\item 设$A$是整环,设$a,b\in A$,那么$\mathrm{l.c.m}(a,b)$存在当且仅当$(a)\cap(b)$是主理想,并且此时$(a)\cap(b)$就是它们的最小公倍数.
	\item 如果$a,b$的最小公倍数存在,则它们的最大公约数存在.记$(ab)\subseteq(a)\cap(b)=(e)$,可记$de=ab$,那么$(d)$是$a,b$的最大公约数.
	\item 但是反过来两个元的最大公约数存在不保证它们的最小公倍数存在.例如设$A$是诺特整环,但不是UFD,于是存在不可约元$a$但不是素元,于是$(a)$不是素理想,那么存在$x,y$满足$xy\in(a)$但$x,y\not\in(a)$.由于$a$是不可约的,所以$a,x$的公约数只能有单位,于是$(a,x)=1$.但是如果$a,x$的最小公倍数存在,它就必须是$ax/1=ax$,于是理应有$(ax)=(a)\cap(x)$.但是$xy\in(a)\cap(x)$,$xy\not\in(ax)$,后者是因为会导致$y\in(a)$.这就矛盾.
	\item 设$A$是整环.
	\begin{enumerate}
		\item 如果$A$是UFD,那么$A$的任意一族主分式理想的交还是主的,特别的对任意$a,b\in A$都存在最小公倍数.
		\item 反过来如果主理想构成的集合上满足升链条件,那么$A$中任意两个元都存在最小公倍数.
	\end{enumerate}
    \begin{proof}
    	
    	(a):设$A$的所有高度1的素理想为$\{\mathfrak{p}_{\alpha}\mid\alpha\in J\}$,记$\mathfrak{p}_{\alpha}=(p_{\alpha})$,那么$\{p_{\alpha}\}$就是全部两两不相伴的素元.设$K=\mathrm{Frac}(A)$,任取一族$a_i\in K$,我们要证明$\cap_ia_iA$是主分式理想.不妨设$\cap_ia_iA\not=0$.任取$0\not=a\in\cap_ia_iA$,那么每个$a_i\not=0$.记$a=u\prod_{\alpha}p_{\alpha}^{r_{\alpha}}$,其中$r_{\alpha}$是整数.再把$a_i$分解为$u_o\prod_{\alpha}p_{\alpha}^{r(i,\alpha)}$,其中$r(i,\alpha)\in\mathbb{Z}$.由于$a\in a_iA$,得到$r_{\alpha}\ge r(i,\alpha)$.取$d=\prod_{\alpha}p_{\alpha}^{\max_i\{r(i,\alpha)\}}$,这个是个有限乘积,指数也是有限数,就有$\cap_{i\in I}a_iA=dA$.
    	
    	\qquad
    	
    	(b):要证明$A$是UFD,归结为证明$A$的不可约元都是素元.设$a$是不可约元,设$a\mid xy$,设$a\not\mid x$,那么$1$是$x,a$的最大公约数.由于任意两个元都存在最小公倍数,导致$(x)\cap(a)=(xa)$,但是$xy$落在左侧,迫使$xy\in(ax)$,导致$y\in(a)$,于是$a$是素元.
    \end{proof}
\end{enumerate}

欧氏整环.整环$R$上的欧氏赋值指的是一个映射$v:R-\{0\}\to\mathbb{N}$,满足对任意$a\in R$,对任意$0\not=b\in R$,要么$b\mid a$,要么存在$q,r\in R$使得$a=bq+r$并且$v(r)<v(b)$.称存在欧氏赋值的整环是欧氏整环(ED).
\begin{enumerate}
	\item ED是PID.
	\begin{proof}
		
		任取欧氏整环的一个非0理想$I$,按照良序性,可取$I$中赋值的最小元为$a$.我们断言$I=(a)$,若否,倘若存在$b\in I$,使得$b\not\mid a$,那么按照定义,存在$q,r\in R$ 使得$b=aq+r$,且$v(r)<v(a)$,由于$r=b-aq\in I$,这和$a$赋值的最小性矛盾.
	\end{proof}
    \item 域上多项式到它次数的映射是一个欧氏赋值.整数环到它绝对值的映射是一个欧氏整环.它们都是ED.
    \item 欧氏赋值提供给欧氏整环一种计算最大公约数的算法.对元$a,b$,如果$a,b$有整除关系,例如$a\mid b$,那么$(a,b)=(a)$.如果没有整除关系,可设$a=bq+r_1$那么$(a,b)=(b,r_1)$.再继续做$b=r_1q_2+r_2$,那么$(b,r_1)=(r_1,r_2)$,继续下去,注意到$v(b),v(r_1),v(r_2),\cdots$的赋值不断减小,于是最终会得到某个$r_N=0$,并且$r_{N-1}\not=0$.那么$r_{N-1}$就是$a,b$的最大公约数,因为$(a,b)=(b,r_1)=\cdots=(r_{N-2},r_{N-1})=r_{N-1}$.另外,如果我们把写出的带余除法反复回带,会得到$s,t$满足$as+bt=r_{N-1}=(a,b)$:
    $$a=bq_1+r_1$$
    $$b=r_1q_2+r_2$$
    $$r_1=r_2q_3+r_3$$
    $$\cdots$$
    $$r_{N-3}=r_{N-2}q_{N-1}+r_{N-1}$$
    $$r_{N-2}=r_{N-1}q_N$$
\end{enumerate}

接下来我们证明UFD的多项式环仍然是UFD,证明需要分式化手段和本原多项式的概念.注意到域上多项式环是欧式整环,于是它是UFD.我们将借助分式化的手段把一个UFD转化为它的商域,借助域的情况证明UFD的情况.下面引入本原多项式的概念.

设$I$是交换环$R$的理想,记$IR[x]=\{a_nx^n+a_ {n-1}x^{n-1}+\cdots+a_0,a_i\in I\}$,那么有$R[x]/IR[x]\cong (R/I)[x]$.另外,如果$I$是素理想,那么$R/I$是整环,于是$(R/I)[x]$是整环,于是$R[x]/IR[x]$是整环,于是$IR[x]$是素理想.

设$R$是交换环,$R[x]$中的一个元称为本原多项式,如果对每个主的素理想$P$,总有$f\not\in PR[x]$.于是$fg$是本原多项式等价于对任意主的素理想$P\subset R$有$fg\not\in PR[x]$,按照$PR[x]$是$R[x]$的素理想,这等价于$f,g\not\in PR[x]$.

在UFD上,本原多项式等价于说全部系数的最大公约数是单位.
\begin{proof}
	
	充分性,假设主的素理想$P=(p)$满足$f\in PR[x]$,那么系数生成的理想包含于$P$,于是最大公约数生成的理想包含于$P$,不会是单位.必要性,假设全部系数的最大公约数不是单位,那它是某个非单位的主理想$(n)$,取$n$唯一分解中的任一不可约元$p$,记$P=(p)$,那么多项式的所有系数都在$P$中,导致$f\in PR[x]$,矛盾.
\end{proof}

约定UFD上一个多项式$f$的全体系数的最大公约数$c(f)$,它不是唯一的,但是在相伴意义下唯一.那么UFD上一个多项式$f$是本原多项式等价于说$c(f)\sim1$.并且对UFD中每个多项式$f$总有$f=c(f)f'$,其中$f'$是一个本原多项式.

高斯引理:如果$f,g$是UFD上的多项式,那么有$c(fg)\sim c(f)c(g)$.
\begin{proof}
	
	设$f=c(f)f_1$,$g=c(g)g_1$,那么$f_1g_1$是本原多项式,并且$fg=c(f)c(g)f_1g_1$,而$c(f)c(g)f_1g_1$的系数的最大公约数就是$c(f)c(g)$,于是$c(fg)\sim c(f)c(g)$.
\end{proof}

如果$R$是UFD,设商域是$K$,那么$R[x]$中一个非常数的多项式是不可约多项式当且仅当它是本原的并且作为$K[x]$中多项式是不可约的.
\begin{proof}
	
	必要性.给定$R[x]$上的多项式$f$,先设$f$不可约,如果$c(f)$不是单位,那么$f=c(f)f'$已经导致$f$可约.另外,如果$f$在$K[x]$中分解为$f=gh$,那么存在$K$中的元$a,b$使得$g=ag',h=bh'$,并且我们不妨设$g',h'$都是本原的,否则可以提出系数的最大公约数放入$a,b$中.于是得到$f=abg'h'$,其中$g'h'$是本原多项式.假设$ab=p/q$,其中$p,q\in R$,那么有$qf=pg'h'$,按照高斯引理,得到$qc(f)$和$p$相伴,于是得到$q$整除$p$,于是$ab\in R$.这说明$f$在$R[x]$中可约.
	
	充分性.如果$f$是$R[x]$的本原多项式,并且作为$K[x]$中的多项式不可约,那么如果在$R[x]$中$f$可约,有$f=gh$,如果$g,h$有个常数,那么和本原多项式矛盾,如果$g,h$都不是常数,那么和在$K[x]$中不可约矛盾.得证.
\end{proof}

引理.如果$f,g\in R[x]$并且在$K[x]$中有$f\mid g$,且在$R$中有$c(f)\mid c(g)$,那么有在$R[x]$中$f\mid g$.
\begin{proof}
	
	按照$K$中$f\mid g$,得到$K[x]$中的元$h$满足$g=fh$,记$h=(a/b)h'$,其中$h'$是$R[x]$中本原多项式,$a,b\in R$,那么在$R[x]$中有$bg=afh'$,按照高斯引理,得到$R$中有$(ac(f))=(bc(g))$,于是从条件$(c(g))\subset (c(f))$得到$(bc(g))\subset(bc(f))$,于是$(ac(f))\subset (bc(f))$,于是按照$c(f)\not=0$,得到$R$中有$b\mid a$,于是$a=bc,c\in R$,于是$h=ch'\in R[x]$,于是$R[x]$中有$(g)\subset(f)$.
\end{proof}

现在记号同上,我们来证明$R[x]$也是UFD.
\begin{proof}
	
	只需验证UFD等价描述中的两条.首先我们验证$R[x]$中主理想的升链总是满足acc条件的,为此设$f_i\in R[x]$使得$(f_1)\subset(f_2)\subset\cdots$, 那么有在$K[x]$中$(f_1)\subset (f_2)\subset\cdots$,并且在$R$中有$(c(f_1))\subset (c(f_2))\subset\cdots$,按照$K[x],R$都是UFD告诉我们这两个升链是满足acc的,即存在一个正整数$N$,使得对于每个正整数$i$有在$K[x]$中$(f_N)= (f_{N+i})$, 在$R$中$(c(f_N))=(c(f_{N+i}))$, 按照引理,这说明在$R[x]$中有$(f_N)=(f_{N+i})$,即满足acc.
	
	下面证明$R[x]$中每个不可约元都是素元.取不可约元$f$,如果它是常数,那么它是$R$中的不可约元,按照$R$是UFD告诉我们$f$是$R$中素元,那么它同样是$R[x]$中素元.如果$f$不是常数,按照引理我们看到$f$也是$K[x]$中的不可约多项式.既然$K[x]$是一个PID,那么$(f)_K$(即$f$在$K$中生成的理想)是$K[x]$ 中的素理想,那么得到$R[x]\to K[x]\to\frac{K[x]}{(f)_K}$,我们断言这个映射的$ker$是$(f)$(即$f$在$R$中生成的理想),这只需再次运用引理.那么有从$\frac{R[x]} {(f)}$到$\frac{K[x]}{(f)_K}$ 的单射.但是后者是域,于是前者必然是整环,于是$(f)$是素理想,于是$f$是素元.
\end{proof}

对元数归纳得到:如果$R$是UFD,那么$R[x_1,x_2,\cdots,x_n]$是UFD.事实上这也说明了任意集合上的多项式环$R[A]$是UFD,因为每个具体多项式只涉及到有限个不定元的,这也给出了一个不是诺特整环的UFD的例子.

UFD上的形式幂级数环未必是一个UFD,但是这样的例子并不好构造.不过域上的形式幂级数环是UFD,这是因为它实际上甚至是一个欧氏整环.取欧式赋值是把形式幂级数映射为它系数不为0项的最低次数.对于两个形式幂级数$f,g$,如果$v(f)<v(g)$,那么取$q=0,r=f$得到$f=qg+r$满足条件,如果$v(f)\ge v(g)$那么有$g\mid f$.

接下来考察不可约多项式的一些性质.称元$a\in R$是多项式$f\in R[x]$的根,如果$f(a)=0$,按照单元多项式环是一个欧氏整环,对任一个多项式$f$,存在带余除法$f=(x-a)g+r$,这里$r$的次数要低于$(x-a)$的次数,于是$r$是一个常数.把$x=a$带入就是说$r=f(a)$.那么$a$是$f$的一个根等价于说,$(x-a)\mid f(x)$.如果正整数$r$满足$(x-a)^r\mid f(x),(x-a)^{r+1}\not\mid f(x)$,就称$a$是一个$r$重根.

关于根的个数我们有:一个整环上多项式至多有它次数个根,这里个数计重数.事实上一个$R$系数多项式在$R$中的根在计重数意义下的个数,一定小于等于它在商域$K$中的计重数意义下的个数,而域上多项式环必然是UFD,于是根的个数在计重数意义下不超过次数.于是,一个无限元的整环上的两个多项式相等,当且仅当它们的取值总是相同的.注意根的个数不超过次数这个结论在非整环中是未必成立的,例如$\mathbb{Z}/6$中$x^2+x$存在四个不同根.

下面我们给出几种判断不可约性的方法:
\begin{enumerate}
	\item 域上一个二次或三次多项式不可约当且仅当它没有根.
	\item 如果$R$是一个UFD,它的多项式$f=a_nx^n+\cdots+a_0$,那么它在商域$F$中的根$\frac{p}{q},(p,q)=1$必然有$p\mid a_0,q\mid a_n$.
	\item (Eisenstein判别法)如果$R$是一个交换环,$P$是一个素理想,多项式$f=a_nx^n+\cdots+a_0$,如果$a_n\not\in P,a_i\in P,0\le i\le n-1,a_0\not\in P^2$,那么$f$不是两个低于$n$次数多项式的乘积
\end{enumerate}
\begin{proof}
	
	这里证明第三个命题.假设有$f=gh$,其中$g,h$的次数都严格小于$f$的次数,设$g=b_0+b_1x+\cdots+b_dx^d$,$h=c_0+c_1x+\cdots+c_ex^e$.于是在$(R/P)[x]$下,有$f'=g'h'$.按照条件,$\mod P$下有$f=a_n'x^n$.按照$R/P$是整环,得到$g'=b_d'x^d,h'=c_e'x^e$,其中$d,e>0$.于是$b_0,c_0\in P$.但是这导致$a_0=b_0c_0\in P^2$,矛盾.
\end{proof}

这里我们给出一些常见域上的不可约多项式.注意域上的常数0和非0常数作为多项式,是0元和单位元,所以不涉及不可约性.不可约多项式必然次数大于等于1.

复数域上的不可约多项式只有1次的.这也等价于说复数域是代数闭域.这就是代数学基本定理.它有多种借助其他理论的整理,例如复分析,代数拓扑等.对于纯代数证明需要借助Galois理论和一点点群论知识,我们会在域论中给出这一证明.在这里我们先承认这一结论.

实数域上的不可约多项式为一次多项式和满足判别式小于0的二次多项式,即满足$b^2-4ac<0$的多项式$ax^2+bx+c$.特别的,奇数次实系数多项式必然有实根.
\begin{proof}
	
	域上一次多项式必然是不可约多项式,另外二次多项式不可约当且仅当没有根,于是判别式小于0保证了这个二次多项式是实不可约多项式.
	
	现在给定一个非常数的多项式$f$,我们来证明它总可以分解为上述多项式的乘积.不妨设多项式$f$是首系数为1的.现在放在复数域中,按照代数基本定理,它在计重数意义下有次数个复根.现在,对任意一个复根$z$,取共轭,结合系数都是实数,得到$z$的共轭$\overline{z}$也是多项式的根,这告诉我们实系数多项式$f$的非实的复根是成对出现的.于是可以设全部根为$z_i,\overline{z_i},i=1,2,\cdots,s$,$r_1,\cdots,r_t$,其中$r_i$是实数根,$z_i$是非实数的复根.于是,在复数域上多项式分解为$\prod_{i=1}^s(x-z_i) (x-\overline{z_i})\prod_{j=1}^{t}(x-r_j)$.现在注意到每个$(x-z_i)(x-\overline{z_i})=x^2-(z_i+\overline{z_i})x+z_i\overline{z_i}$是上述第二类实系数不可约多项式.这就得证.
\end{proof}

至于有理数域上的不可约多项式,情况复杂得多.按照之前给出的UFD上不可约多项式和它商域上不可约多项式直接的联系,对有理数域上不可约多项式的探究可以等价于探究$\mathbb{Z}$上的不可约多项式.关于这种复杂性我们可以从一个性质初步的体会:$\mathbb{Z}[x]$和$\mathbb{Q}[x]$上存在任意次数的不可约多项式.因为按照爱森斯坦判别法,对每个素数$p$和每个正整数$n$,多项式$x^n-p$都是不可约多项式.

习题.设$p$是一个素数,$k$是一个域,设$a\in k$,则$f(x)=x^p-a$在$k[x]$中不可约当且仅当它在$k$上没有根.
\begin{proof}
	
	只需证明充分性.取$k$关于$f(x)$的分裂域$F$,记$f(x)$在$F[x]$中分解为$f(x)=(x-a_1)(x-a_2)\cdots(x-a_p)$.假设$f(x)$在$k[x]$上可约,那么存在非常数的$k$中多项式$g,h$满足$f(x)=g(x)h(x)$.于是不妨设$g(x)$在$F$中分解为$g(x)=(x-a_1)\cdots(x-a_r)$,其中$r<p$.记$b=a_1a_2\cdots a_r$,那么$b\in k$,并且$b^p=a_1^pa_2^p\cdots a_r^p=a^r$.按照$p$和$r$互素,说明存在整数$x,y$满足$xp+yr=1$,那么有$(a^xb^y)^p=a^{xp}b^{yp}=a^{xp}a^{yr}=a$.但是$a,b\in k$,说明$k$上有$f(x)$的根$a^xb^y$,完成证明.
\end{proof}

设$R$是PID,多项式环$R[x]$上的素理想$P$,那么有$P=0$,或$P=(f)$,或$P$是一个极大理想.如果$P$是极大理想,那么要么$P=(f)$,其中$f$是素元,要么$P=(p,g)$,其中$p$是$R$中素元,并且$g\in R[x]$满足$g$在$R/p[x]$中的像是素元.
\begin{proof}
	
	不妨设$P$非0,任取非0不可约多项式$f_1\in P$,假设$P\not=(f_1)$.那么存在一个不可约多项式$f_2\in P-(f_1)$.取$R$的商域$K$,那么Gauss引理告诉我们$f_1,f_2$同样是$K[x]$中的不可约多项式.于是$f_1,f_2$在$K[x]$中是互素的,也就是说存在$R[x]$中的$g_1,g_2$和一个$c\in R$使得$c=g_1f_1+g_2f_2\in R\cap P$.于是素理想$R\cap P$非0,按照$R$是PID,就有一个不可约元$p\in R$使得$R\cap P=(p)$.现在取$k=R/p$,那么$k$是域.取$q=P/(p)\subset k[x]$,那么$k[x]/q=(R/p)[x]/(P/p)=R[x]/P$,右侧是整环,于是$q$是$k[x]$的素理想,记作$q=(g')$,也就是极大理想,于是得到$P$是$R[x]$的极大理想.取$g\in P$使得在$q$中的像为$g'$,就得到$P=(p,g)$.
\end{proof}
\newpage
\section{模范畴}
\subsection{基本概念}

设$M$是交换群,设$R$是环,称$M$具有一个$R$-模结构,如果约定了从$R$到$\mathrm{End}_{Ab}(M)$的一个环同态.这等价于存在$\theta:R\times M\to M$ 使得$\forall r,s\in R,m,n\in M$有:
\begin{enumerate}
	\item $\theta(r,m+n)=\theta(r,m)+\theta(r,n)$.
	\item $\theta(r+s,m)=\theta(r,m)+\theta(s,m)$.
	\item $\theta(rs,m)=\theta(r,\theta(s,m))$.
	\item $\theta(1,m)=m$.
\end{enumerate}

通常把这个映射$\theta(r,m)$直接写作$rm$.对这个映射最简单的观察是$\theta(0,m)=0,\theta(-1,m)=-m$.

和群作用一样,习惯上把作用写作左侧,约定作用的顺序是从右到左,即$(rs)m=r(sm)$,这时称为左模,如果把作用写在右侧,即$m(rs)=(mr)s$, 这相当于写在左侧时约定作用的顺序是从左到右,即$(rs)m=s(rm)$.类似群作用的情况,对给定的一个环$R$,可以这样定义一个新的环,它的凭借集合和加法结构不变,但是新的乘法$a\circ b$定义为$ba$,这个环称为原来环的反环.那么右作用一个环相当于左作用这个环的反环.当作用的环是交换环的时候,左右作用的差异消失.

双侧模.给定环$R,S$,一个交换群$M$被称为$(R,S)$双边模,如果它既是左$R$模又是右$S$模,并且两种数乘满足$r(ns)=(rn)s,\forall r\in R,s\in S,n\in N$.于是记号$rms$所指的元素是没有歧义的.

子模.一个左$R$模$M$的子模是它的一个交换子群$N$,满足$\forall r\in R,n\in N$有$rn\in N$.一族子模的交仍然是子模,由此可以定义模的子集生成的子模是包含这个子集的所有子模的交,它也就是包含这个子集的最小的子模.

给定环$R$,那么$R$作为交换群是$R$上的左模和右模,以及$(R,R)$双侧模,其中模结构定义为环$R$上的乘法.此时,左$R$模$R$的子模和$R$的左理想概念是一致的,右$R$模$R$的子模和$R$的右理想一致,$(R,R)$模$R$的子模和$R$的双侧理想是一致的.

给定的环$R$,定义左$R$-模范畴,它的对象是全体左$R$模,它的态射称为左$R$模同态,定义为映射$f:A\to B$满足$f(a_1+a_2)=f(a_1)+f(a_2),f(ra)=rf(a),\forall a_i\in A,r\in R$.前者约定它也是交换群的群同态,后者约定它保左$R$模结构.给定交换群$M$,我们知道约定左$R$模结构相当于给定一个环同态$R\to\mathrm{End}(M)$,此时对每个$a\in R$,左乘$a$这个映射是$M$上的自同态,记作$a_L$,那么称$\mathrm{End}(M)$中的一个元$\varphi$是$M$上的模同态就等价于约定$\varphi$和全部$a_L$可交换,此即$\varphi(ax)=a\varphi(x)$.
\begin{enumerate}
	\item 单元群的自同态环是零环,于是它是任意环上的左模和右模,它称为环上的0模.左$R$模范畴和右$R$模范畴上的零对象存在,就是0模.如同群和环的情况,一个模同态如果是双射,那么它的逆自动是一个模同态,即它是模同构.
	\item 交换群范畴和$\mathbb{Z}$-模范畴同构.我们曾给出过$\mathbb{Z}$是环范畴的初对象,这说明对于任意一个交换群$M$,总存在从$\mathbb{Z}$到$End_{Ab}(M)$的唯一的映射.于是每个交换群都具备唯一的$\mathbb{Z}$模结构.并且交换群同态等价于是$\mathbb{Z}$模同态.于是$\mathbb{Z}$-模范畴和交换群范畴是同构的范畴.
	\item 商对象.给定左$R$模$M$,取子模$N$,于是$N$也是交换群$M$的子群,考虑商群$M/N$.$M/N$可以自然的成为一个左$R$模,模结构为$r(m+N)=rm+N$.这个模就称为商模.其中典范映射为$\pi:M\to M/N$,即$m\mapsto m+N$.
	
	商模的泛映射性质.给定左$R$模$M$的子模$N$,对任意左$R$模同态$\varphi:M\to P$,满足$N\subset\ker\varphi$,那么存在唯一的$R$模同态$\varphi':M/N\to P$满足交换图:
	$$\xymatrix{
		M\ar[dr]_{\pi}\ar[rr]^{\varphi}&&P\\
		&M/N\ar[ur]^{\exists!\varphi'}&
	}$$
	\item 积和余积.给定一族左$R$模$\{M_i\}_{i\in I}$,作为交换群的积记作$\prod_{i\in I}M_i$,这个交换群上赋予左模结构为$r(m_i)=(rm_i)$就是积对象,也称为模族的直积.$\prod_{i\in I}M_i$中全部只在有限个分量上不取0元的元构成了一个子模,它就是余积,也称为模族的直和,记作$\oplus_{i\in I}M_i$.有限个模的直和与直积是同构的.
	
	直和项.称模$N$是模$M$的直和项,如果存在模$N'$使得$M\cong N\oplus N'$.那么模$M$的直和项必然同构于$M$的子模.但是反过来一般不成立,不过一旦子模$N$是$M$的直和项,那么必然有同构$M\cong N\oplus M/N$.为此考虑典范满同态$M=N\oplus N'\to M/N$,它的核是$N$,于是$N'=N\oplus N'/N\cong M/N$.
	\item 核与余核.模同态$f:M\to N$的核就是$\ker f=\{m\in M\mid f(m)=0\}$,余核就是$\mathrm{coker}f=N/\mathrm{im}f$.同群与环的情况一样,核等价于子模,满射像等价于商.其中满射像是商需要同构定理:
	\begin{enumerate}
		\item 第一同构定理.给定左$R$模同态$f:M\to N$,那么存在同构$M/\ker f\cong\mathrm{im}f$,为$m+\ker f\mapsto f(m)$.
		\item 第二同构定理.给定左$R$模$M$的两个子模$S,T$,那么存在同构$S/(S\cap T)\cong(S+T)/T$.
		\item 第三同构定理.给定左$R$模的子模链$T\subset S\subset M$,那么存在同构$(M/T)/(S/T)\cong M/S$.
		\item 对应定理.如果$T$是左$R$模$M$的子模,那么从$S$到$S/T$构成了$M$的包含$T$的子模,和$M/T$的子模之间的保序一一对应,即$S\mapsto S/T$.
	\end{enumerate}
	\item 单态射和满态射的等价描述.
	\begin{enumerate}
		\item $f$是单态射$\Leftrightarrow$$f$是单射$\Leftrightarrow$$\ker f$平凡.
		\item $f$是满态射$\Leftrightarrow$$f$是满射 $\Leftrightarrow$$\mathrm{coker}f$平凡.
	\end{enumerate}
	\item 等化子和余等化子.模范畴中等化子和余等化子总存在,对任意$f,g:M\to N$,等化子就是$M$的子模$M'=\{m\in M\mid f(m)=g(m)\}$,余等化子就是商$N/N'$,其中$N'$是$N$的子模$\{f(m)-g(m)\mid m\in M\}$.见纤维积与纤维和.
	\item 张量积作为交换$k$代数的余积.设$k$是交换环,设$A,B$是交换$k$代数,它们在交换$k$代数范畴中的余积为张量积$A\otimes_kB$.特别的,按照交换环即交换$\mathbb{Z}$代数,于是交换环范畴上的二元余积即张量积$A\otimes_{\mathbb{Z}}B$.
	\begin{proof}
		
		定义两个典范$k$代数同态为$\alpha:A\to A\otimes_kB$,$a\mapsto a\otimes1$和$\beta:B\to A\otimes_kB$,$b\mapsto 1\otimes b$.任取交换$k$代数$X$,考虑如下交换图,其中$f,g$是给定的$k$代数同态.需要验证的是存在$k$代数同态$\varphi:A\otimes_kB\to X$使得图表交换.
		$$\xymatrix{&A\ar[dl]_{\alpha}\ar[dr]^{f}&\\A\otimes_kB\ar[rr]^{\varphi}&&X\\&B\ar[ul]^{\beta}\ar[ur]_g&}$$
		
		构造二重$k$线性映射$\varphi^*:A\times B\to X$为$(a,b)\mapsto f(a)g(b)$.于是存在唯一的$k$模同态$\varphi:A\otimes_kB\to X$提升了$\varphi^*$,即$\varphi(a\otimes b)=f(a)g(b)$.还需要说明$\varphi$是代数同态,为此只需验证$\varphi((a\otimes b)(a'\otimes b'))=\varphi(a\otimes b)\varphi(a'\otimes b')$,这从交换性直接得出.于是我们找到了$k$代数同态$\varphi$满足上述交换图.最后还需要验证唯一性.设$\psi$是同样满足交换图的$k$代数同态,那么从$a\otimes b=(a\otimes1)(1\otimes b)=\alpha(a)\beta(b)$得到$\psi(a\otimes b)=\psi(\alpha(a)\beta(b))=f(a)g(b)=\varphi(a\otimes b)$,而$\{a\otimes b\mid a\in A,b\in B\}$生成了整个张量积,这就得到唯一性.
	\end{proof}
    \item 回拉和推出(即纤维积和纤维和).给定两个模同态$f:B\to A$和$g:C\to A$,它们的纤维积或者回拉定义为$(D,\alpha,\beta)$,其中$D$是一个左模,$\alpha:D\to C$,$\beta:D\to B$满足$g\circ\alpha=f\circ\beta$.它满足对任意另一组满足这个条件的$(X,\alpha',\beta')$,存在唯一的模同态$\varphi:X\to D$使得如下图表交换:
    $$\xymatrix{X\ar[drr]^{\alpha'}\ar[dr]^{\varphi}\ar[ddr]_{\beta'}&&\\&D\ar[r]^{\alpha}\ar[d]^{\beta}&C\ar[d]^g\\&B\ar[r]_f&A}$$
    
    模范畴上纤维积总是存在的,这里来直接构造.定义$D=\{(b,c)\in B\oplus C\mid f(b)=g(c)\}$,定义$\alpha,\beta$是$B\oplus C$的两个投影映射$\pi_B,\pi_C$在$D$上的限制.那么此时满足$g\circ\alpha=f\circ\beta$.假设存在$(X,\alpha',\beta')$满足相同的条件,定义$\varphi:X\to D$为$x\mapsto(\beta'(x),\alpha'(x))$,那么从$f\circ\beta'=g\circ\alpha'$说明$\varphi$的像的确落在$D$中.此时图表交换.最后唯一性是因为如果$\varphi(x)=(y,z)$,从$\alpha\circ\varphi=\alpha'$得到$y=\alpha'(x)$,同理$z=\beta'(x)$.
    
    纤维积的对偶概念为限维和.给定两个模同态$f:A\to B$和$g:A\to C$,它们的纤维和或者推出定义为$(D,\alpha,\beta)$,其中$\alpha:B\to D$,$\beta:C\to D$,满足$\beta\circ g=\alpha\circ f$.它们满足,如果$(Y,\alpha',\beta')$满足相同的条件,那么存在唯一的模同态$\varphi:D\to Y$使得如下图表交换:
    $$\xymatrix{A\ar[r]^g\ar[d]_f&C\ar[d]_{\beta}\ar[ddr]^{\beta'}&\\B\ar[r]^{\alpha}\ar[drr]_{\alpha'}&D\ar[dr]^{\varphi}&\\&&Y}$$
    
    模范畴上纤维和总是存在的,这里同样直接构造.定义$S=\{(f(a),-g(a))\in B\oplus C\mid a\in A\}$,那么$S$是$B\oplus C$的子模,定义$D=B\oplus C/S$,取$\alpha:B\to D$为$b\mapsto (b,0)+S$,取$\beta:C\to D$为$C\mapsto(0,c)+S$.那么有$\beta\circ g=\alpha\circ f$.现在假设$(Y,\alpha',\beta')$满足同样的条件,定义$\varphi:D\to X$为$(b,c)+S\mapsto\alpha'(b)+\beta'(c)$,那么这是模同态,并且使得上述图表交换.最后验证唯一性,假设模同态$\psi:D\to Y$同样使得图表交换,任取$c\in C$,那么$\psi((0,c)+S)=\psi\circ\beta(c)=\beta'(c)$,同理$\psi((b,c)+S)=\alpha'(b)$,于是$\psi((b,c)+S)=\alpha'(b)+\beta'(c)$,这就得到唯一性.
    
    注意如果$B=C$,此时纤维积即等化子,纤维和即余等化子.核与余核分别是纤维积与纤维和的一个特例.
    \item 按照模范畴上存在积,余积,等化子,余等化子,说明模范畴是完备范畴和余完备范畴,即小范畴到模范畴的函子的极限和余极限总存在.
	\item 逆向极限.模范畴上的一个逆向系统是指以偏序集$I$为指标集的一族(左)模$\{M_i\}$,满足对每个$I$上的关系$i\le j$,有指定的一个模同态$\psi_i^j:M_j\to M_i$,满足如下两件事:第一是总有$\psi_i^i=1_{M_i}$,第二是对任意满足$i\le j\le k$的指标,有交换图:
	$$\xymatrix{M_k\ar[dr]_{\psi_j^k}\ar[rr]^{\psi_i^k}&&M_i\\&M_j\ar[ur]_{\psi_i^j}&}$$
	
	逆向系统的逆向极限或者称投射极限是指一个模$\lim_{\leftarrow}M_i$,以及一族模同态$\alpha_i:\lim_{\leftarrow}M_i\to M_i,i\in I$,满足对任意$i\le j$有$\psi_i^j\circ\alpha_j=\alpha_i$.并且对任意另一个满足这个条件的模$X$和一族模同态$f_i:X\to M_i$,存在唯一的模同态$\varphi:X\to\lim_{\leftarrow}M_i$满足如下图表交换:
	$$\xymatrix{\lim_{\leftarrow}M_i\ar[dr]^{\alpha_i}\ar[ddr]_{\alpha_j}&&X\ar[ll]_{\varphi}\ar[dl]_{f_i}\ar[ddl]^{f_j}\\&M_i&\\&M_j\ar[u]^{\psi_i^j}&}$$
	
	左$R$模范畴是完备范畴,于是任意逆向系统存在逆向极限.这里给出具体构造.任取逆向系统$(M_i,\psi_i^j:M_j\to M_i)$,定义$L$为$\prod_i M_i$的子集,它的元素具有如下描述:写作$(m_i)\in\prod_iM_i$,满足对任意的$i\le j$,有$m_i=\psi_i^j(m_j)$.于是$L$是一个子模.取$\prod_iM_i$到$M_i$的投射映射在$L$上的限制为$p_i$.于是有$\psi_i^jp_j=p_i,\forall i\le j$.现在任取满足相同条件的模$X$和一族模同态$f_i:X\to M_i$.定义模同态$\varphi:X\to\prod_iM_i$为$x\mapsto(f_i(x))_{i\in I}$.于是从$\psi_i^j\circ f_j=f_i$说明像集落在$L$中,于是$\varphi$可视为$X\to L$的模同态.另外$\varphi$使得上述图表交换,即$p_i\circ\varphi(x)=f_i(x)$.最后容易验证$\varphi$的唯一性,这就说明$(L,p_i)$是逆向极限.
	
	\item 正向极限.模范畴上的一个正向系统是指以偏序集$I$为指标集的一族(左)模$\{M_i\}$,满足对每个$I$上的关系$o\le j$,有指定的一个模同态$\varphi_j^i:M_i\to M_j$.满足$\varphi_i^i=1_{M_i}$和$\varphi_k^j\circ\varphi_j^i=\varphi_k^i$.正向系统和逆向系统本质上的区别只是指定的模同态不同,逆向系统中$i\le j$时指定了模同态$\psi_i^j:M_j\to M_i$是从指标大的模映射到指标小的模.而正向系统中$i\le j$时指定的模同态$\varphi_j^i:M_i\to M_j$是从指标小的模映射到指标大的模.
	
	给定上述正向系统,它的正向极限或者归纳极限是指一个模$\lim_{\rightarrow}M_i$和一族模同态$\alpha_i:M_i\to\lim_{\rightarrow}M_i$满足$i\le j$时总有$\alpha_j\circ\varphi_j^i=\alpha_i$.并且如果模$X$和同态族$f_i:M_i\to X$同样满足上述条件,则存在唯一的模同态$\psi:\lim_{\rightarrow}M_i\to X$使得如下图表交换:
	$$\xymatrix{\lim_{\rightarrow}M_i\ar[rr]^{\psi}&&X\\&M_i\ar[ul]_{\alpha_i}\ar[ur]^{f_i}\ar[d]_{\varphi_j^i}&\\&M_j\ar[uul]^{\alpha_j}\ar[uur]_{f_j}&}$$
	
	左$R$模范畴是余完备范畴,于是任意正向系统存在正向极限.这里给出直接构造.任取正向极限$\{M_i,\varphi_j^i\}$.取典范映射$M_i\to\oplus_iM_i$为$\lambda_i$.定义$D=\oplus_iM_i/S$.这里$S$是由全体$\lambda_j\varphi_j^im_i-\lambda_im_i,m_i\in M,i\le j$生成的子模.定义映射$\alpha_i:M_i\to D$为$m_i\mapsto\lambda_i(m_i)+S$,那么按照$S$的构造说明$\alpha_j\circ\varphi_j^i=\alpha_i$.现在假设模$X$和$f_i:M_i\to X$同样满足上述性质,构造$\psi^*:\oplus_iM_i\to X$为$\sum_i\lambda_i(m_i)\mapsto\sum_if_i(m_i)$,现在验证$S\subset\ker\psi^*$,这会得到$\psi^*$诱导了商的模同态$\psi:D\to X$为$\sum_i\lambda_i(m_i)+S\mapsto\sum_if_i(m_i)$.而验证这件事只需验证$S$的每个生成元都在$\ker\psi^*$中,即$\psi^*(\lambda_j\varphi_j^im_i-\lambda_im_i)=f_j(\varphi_j^i(m_i))-f_i(m_i)=0$这是$f_i$的条件.容易验证此时$\psi$使得图表交换,并且是唯一的.这就得出$(D,\alpha_i)$是正向极限.
\end{enumerate}

关于正向系统和正向极限的几个补充.
\begin{enumerate}
	\item 正向极限没有逆向极限那么直观,不过在添加一个条件后正向极限具有更简单的描述.称偏序集$I$是有向集,如果对任意$i,j\in I$,存在一个$k\in I$使得$i\le k$和$j\le k$.
	\begin{enumerate}
		\item 倘若正向系统的指标集$I$是有向集,那么反复运用正向极限中$S$包含了$\lambda_j\varphi_j^im_i-\lambda_im_i$,会导致正向极限$D$中每个元可以表示为某个$\alpha_i(m_i)$.
		\item 在$D$中有$\alpha_i(m_i)=0$当且仅当存在某个$j\ge i$使得$\varphi_j^i(m_i)=0$.
		\begin{proof}
			
			充分性是直接的,因为$\alpha_i(m_i)=\alpha_j(\varphi_j^i(m_i))=0$.必要性,按照$\alpha_i(m_i)$在正向极限$\oplus_iM_i/S$中为零,说明$m_i$在$\oplus_iM_i$中的像$\lambda_im_i$可表示为一些$r(j,k,m_j)=\lambda_k\varphi^j_km_j-\lambda_jm_j$的线性组合$\lambda_im_i=\sum_ja_jr(j,k,m_j)=\sum_jr(j,k,a_jm_j)$.用$m_j$代替$a_jm_j$,得到$\lambda_im_i=\sum_jr(j,k,m_j)$.按照有向集条件,取指标$t$大于这个表达式中的每个指标,得到:
			\begin{align*}
		    \lambda_t\varphi^i_tm_i&=r(i,t,m_i)+\lambda_im_i=r(i,t,m_i)+\sum_jr(j,k,m_j)\\&=r(i,t,m_i)+\sum_j\left(r(j,t,m_j)+r(k,t,-\varphi_k^jm_j)\right)=\sum_lr(l,t,x_l)\\&=\lambda_t(\sum_l\varphi_t^lx_l)-\sum_l\lambda_lx_l
			\end{align*}
			
			于是当$l\not=t$时$\lambda_lx_l=0$,于是$x_l=0$,于是这个式子变成$\lambda_t\varphi_t^im_i=0$,于是$\varphi_t^i(m_i)=0$.
		\end{proof}
		\item 给定有向集上的正向系统$\{M_i,\varphi_j^i\}$,取全体$M_i$的无交并为集合$M$.现在定义$M$上的一个等价关系为,对$m_i\in M_i$和$m_j\in M_j$,它们等价当且仅当存在公共前继元,即存在某个$k\in I$满足$k\ge i$和$k\ge j$,且有$\varphi_k^i(m_i)=\varphi_k^j(m_j)$.自反性和对称性是直接的,现在说明传递性.假设$m_i\sim m_j$,即存在$p\ge i,j$使得$\varphi_p^i(m_i)=\varphi_p^j(m_j)$.又设$m_j\sim m_k$,即存在$q\ge j,k$使得$\varphi_q^j(m_j)=\varphi_q^k(m_k)$.由于$I$是有向集,可取$r\ge p,q$,于是图表交换性说明$\varphi_r^i(m_i)=\varphi_r^p\circ\varphi_p^i(m_i)=\varphi_r^p\circ\varphi_p^j(m_j)=\varphi_r^j(m_j)=\varphi_r^q\circ\varphi_q^j(m_j)=\varphi_r^q\circ\varphi_q^k(m_k)=\varphi_r^k(m_k)$.
		\item 现在取全体等价类构成的集合为$L$,我们断言$L$具有左$R$模结构.把$m_i\in M$所在的等价类记作$[m_i]$.定义交换群结构为$[m_i]+[m_j]=[\varphi_k^i(m_i)+\varphi_k^j(m_j)]$,其中$k\ge i,j$由有向集条件保证.定义模结构为$r[m_i]=[rm_i]$.这使得$L$成为一个$R$模.
		\item 最后验证$L$同构于我们之前构造的正向极限$D$.构造$f:L\to D$为$[m_i]\mapsto m_i+S$,按照$S$的构造知$f$定义良性,验证$f$是模同态.我们说明过$I$是有向集的时候$D$中每个元可以表示为$\alpha_i(m_i)$,这导致$f$是满同态.我们还证明过$0=f([m_i])=m_i+S$导致存在某个$j\ge i$使得$\varphi_j^i(m_i)=0$,这说明$m_i\sim0$,即$[m_i]=[0]$,于是$f$是单射,这就证明了$f$是同构.
	\end{enumerate}
	\item 正向系统的同态诱导了正向极限的同态.给定两个定义在统一指标集上的正向系统$A={A_i,\alpha_j^i}$,$B={B_i,\beta_j^i}$.从$A$到$B$的态射是指一个自然变换$r:A\to B$,即一族模同态${r_i:A_i\to B_i}_{i\in I}$,满足图表交换:
	$$\xymatrix{A_i\ar[r]^{r_i}\ar[d]_{\alpha_j^i}&B_i\ar[d]^{\beta_j^i}\\
		A_j\ar[r]_{r_j}&B_j}$$
	
	这样的态射诱导了模同态$r':\lim_{\rightarrow}A_i\to\lim_{\rightarrow}B_i$为$\sum\lambda_ia_i+S\mapsto\sum\mu_ir_ia_i+T$.其中$\lambda_i$和$\mu_i$分别是$A_i\to\oplus_iA_i$和$B_i\to\oplus_iB_i$的典范映射.
\end{enumerate}

关于正合性.
\begin{enumerate}
	\item 设有$A$模的逆向系统之间的短正合了$0\to(K_i)\to(M_i)\to(L_i)\to0$,那么它一般只诱导了逆向极限上的左正合性,最右边的同态一般不是满射:
	$$\xymatrix{0\ar[r]&\varprojlim K_i\ar[r]&\varprojlim M_i\ar[r]&\varprojlim L_i}$$
	\item 但是如果指标集是$\mathbb{N}$,并且$K_{i+1}\to K_i$总是满射,那么上述逆向系统的短正合列就诱导了逆向极限的短正合列:
	$$\xymatrix{0\ar[r]&\varprojlim K_i\ar[r]&\varprojlim M_i\ar[r]&\varprojlim L_i\ar[r]&0}$$
	\item 
\end{enumerate}


\newpage
\subsection{分式化}

分式化是从环和模上构造新的环和模的一种操作,粗略的讲它相当于约定了除法.例如从$\mathbb{Z}$到$\mathbb{Q}$的过程,从多项式环到有理函数域的过程.它们都是约定了某些元可以作为"分母",然后把元素扩充为分式$\frac{r}{s}$.按照我们以往熟悉的分式的乘法,这些能作为分母的元构成的子集需要满足乘性,即如果$s_1,s_2\in S$那么$s_1s_2\in S$.交换rng上满足这个性质的子集称为乘性闭子集.

交换环上一个理想是素理想等价于约定它的补集是乘性闭子集,注意这并不是说乘性闭子集的补集恰好是素理想,因为补集可能未必是理想.

我们期望把分式定义为为集合$R\times S$上的等价类,按照我们熟悉的整数上分式的性质,需要满足$\frac{ur} {us}=\frac{r}{s}=\frac{tr}{ts}$,也就是要约定$(r,s)\sim(tr,ts)\sim(ur,us),\forall t,u\in S$,这等价于说$(r_1,s_1)\sim(r_2,s_2)$只要存在$s',s''\in S$使得$(s'r_1,s's_1)= (s''r_2,s''s_2)$.即存在一个$s=s's''\in S$,使得$s(r_1s_2-r_2s_1)=0$,反过来,如果存在$s\in S$使得$s(r_1s_2-s_1r_2)=0$,那么取$s'=ss_2,s''=ss_1$,满足上述要求.综上我们给出$R\times S$上的一个等价关系:$(r_1,s_1)\sim(r_2,s_2)$如果存在一个$s\in S$使得$s(r_1s_2-r_2s_1)=0$.把$(r,s)$所在的等价类记作$\frac{r}{s}$.全体等价类构成的集合记作$S^{-1}R$,称为$R$关于$S$的分式化.
\begin{enumerate}
	\item 按照如下等式约定的加法和乘法是良性的,使得$S^{-1}R$构成了交换rng.
	$$\frac{r_1}{s_1}+\frac{r_2}{s_2}=\frac{r_1s_2+r_2s_1}{s_1s_2};\frac{r_1}{s_1}\times\frac{r_2}{s_2}=\frac{r_1r_2}{s_1s_2}$$
	\item $\{0/s,s\in S\}$构成了一个等价类,它就是$S^{-1}$的零元,$\{s/s,s\in S\}$同样构成了一个等价类,它是$S^{-1}R$的幺元.于是尽管交换rng$R$未必有幺元,但是分式化总是有幺元的.
	\item 如果乘性闭子集$S$包含了0元,那么$R\times S$所有元构成同一个等价类,此时分式化是0环.
	\item 对于交换环,一个乘性闭子集$S$如果不包含1,给它添加1仍然会是乘性闭子集,并且产生的分式化不会有本质区别:在不含1的时候我们以$rs/s,\forall s\in S$等价的代替了$r/1$.
	\item 如果环是非零环的整环,并且乘性闭子集不含零元,那么分式化是整环.
	\item 给定交换环$R$,它的全体非零因子构成了一个饱和乘性闭子集,此时的分式化称为$R$的全商环,并且此时典范映射是单射.
	\item 如果环是非零环的整环,并且乘性闭子集是全体非零元,那么分式化中每个非零元$a/s$都有逆元$s/a$,这说明此时分式化是域.这称为整环$R$的商域,通常记作$\mathrm{Frac}(R)$.整环到商域的典范环同态是一个单射.
\end{enumerate}

分式化的典范映射.$R\to S^{-1}R$的典范映射约定为$r\mapsto rs/s$,其中$s$可取$S$中任意元而不影响取值.如果约定乘性闭子集包含幺元1,此时典范环同态可以表示为$r\mapsto r/1$.
\begin{enumerate}
	\item 倘若$S$不含零因子,那么典范映射总是单同态.
	\item 典范映射是同构当且仅当乘性闭子集由环中单位构成.
	\item 给定交换环$R$上的两个乘性闭子集$S_1\subset S_2$,倘若它们均不含零因子,那么有$R\subset S_1^{-1}R\subset S_2^{-1}R$.
	\item 给定环$R$和乘性闭子集$S$,那么$R$到$S^{-1}R$的典范映射是同构当且仅当$S$由单位构成.
	\begin{proof}
		
		如果它是同构,那么对$s\in S$有$(s/1)(1/s)=1$,于是$s/1$在$S^{-1}R$中是单位,于是同构说明$s$是单位.反过来,如果$S$由单位构成,那么$(R,1_R)$满足局部化的泛映射性质,这说明是同构.
	\end{proof}
\end{enumerate}

关于$\mathbb{Q}$的全部子环.按照$\mathbb{Q}$的素子环是$\mathbb{Z}$,说明$\mathbb{Q}$的每个子环$R$满足$\mathbb{Z}\subset R\subset\mathbb{Q}$.其中$\mathbb{Q}$可视为$\mathbb{Z}$最大的分式化,即全体非零元作为乘性闭子集的分式化.这里我们断言每个中间环$R$都可以视为$\mathbb{Z}$的分式化:任取这样的$R$,设$S_0$是这样的素数$p$构成的集合,满足存在$R$中的一个既约分式使得分母被$p$整除.记$S_0$生成的乘性闭子集为$S$,那么$S$中的元恰好就是$S_0$中的素数的次幂的有限乘积,这里约不约定包含幺元不影响结果.现在我们断言$S^{-1}\mathbb{Z}=R$,事实上证明左侧包含于右侧只要证明如果$p\in S_0$那么有$1/p\in R$,这是因为按照定义有$r/ps\in R$,导致$r/p\in R$,其中$(p,r)=1$,于是存在整数$x,y$使得$xp+ry=1$,于是有$1/p=x+y(r/p)\in R$;证明右侧包含于左侧只要注意到$R$中的元的既约形式$a/b$,必然有$b$由$S_0$中若干素数的次幂乘积构成.

交换环上分式化的泛映射性质.给定交换环$R$和它的一个乘性闭子集$S$,记典范映射$\varphi:R\to S^{-1}R$.取未必交换的环$R'$和环同态$\psi:R\to R'$,那么每个$s\in S$有$\psi(s)$为$R'$中的单位,当且仅当存在提升映射$\rho:S^{-1}R$,换句话说存在环同态$\rho$满足如下图表交换.另外提升映射如果存在则是唯一的.
$$\xymatrix{R\ar[r]^{\varphi}\ar[dr]_{\psi}&S^{-1}R\ar[d]_{\rho}\\ &R'}$$
\begin{proof}
	
	假设存在提升同态$\rho$,那么$\psi(s)\rho(1/s)=\rho(s/1\cdot1/s)=1$,于是每个$\psi(s),s\in S$均为$R$中单位.这还说明了$\rho(r/s)=\rho(r/1)\rho(1/s)=\psi(r)\psi^{-1}(s)$,这说明了提升映射的唯一性.
	
	现在假设每个$\psi(s),s\in S$都是$R'$中的单位,构造$\rho:S^{-1}R\to R'$为$\rho(r/s)=\psi(r)\psi^{-1}(s)$.需要验证这个定义的良性,即如果有$r/s=r'/s'$,等价于存在$s_0\in S$使得$rs's_0=r'ss_0$,于是有$\psi(r)\psi^{-1}(s)=\psi(r')\psi^{-1}(s')$.最好验证下$\rho$是环同态,并且使图表交换即可.
\end{proof}

单点处的局部化.给定环$R$的一个元$t$,记乘性闭子集$S=\{t^n\mid n\ge0\}$,其中$t^0$约定为幺元.称$S^{-1}R$为$R$在$t$处的局部化,记作$R_t$.我们断言$R_t\cong R[X]/(1-tX)$.
\begin{proof}
	
	记$S=R[X]/(1-tX)$,取$\varphi$为$R$到$S$的典范映射$R\to R[X]\to R[X]/(1-tX)$,只需证明$(R,\varphi)$满足局部化的泛映射性质.
	
	取$x=X+(1-tX)\in S$,那么$1-x\varphi(t)=0$,这说明$\varphi(t)$是$S$中单位,于是每个$\varphi(t^n),n\ge0$都是单位.现在任取$\psi:R\to S'$满足$t$的像是单位,定义$\theta:R[X]\to S'$为$\theta$在$R$上的限制为$\psi$,并且$\theta(X)=\psi(t)^{-1}$,那么$\theta(1-ft)=0$,于是$\theta$诱导了一个$S\to S'$的同态$\theta'$,并且有$\psi=\theta'\circ\varphi$.另外这样的$\theta'$必然是唯一的.于是$S$就是$R$在$t$处的局部化.
\end{proof}

给定乘性闭子集$S$,取交换环$R$的子集$I$,$I^S=\{a\in R\mid\exists s\in S,as\in I\}$称为$I$的饱和化,如果$I=I^S$,就称$I$是饱和的.
\begin{enumerate}
	\item 如果$I$是理想,那么它的饱和化$I^S$是包含$I$的理想.
	\item 理想$I$的饱和化实际上就是$S^{-1}I$在典范映射下的原像.特别的从$R$到$S^{-1}R$的典范映射的核恰好就是$\{0\}^S$.
	\begin{proof}
		
		设$S^{-1}I$的原像集合为$J$,需要证明$I^S=J$.一方面任取$a\in I^S$,那么存在某个$s\in S$使得$as=b\in I$,于是有$a/1=b/s\in S^{-1}I$.另一方面任取$a\in J$,那么$a/1=b/s\in S^{-1}I$,其中$b\in I$和$s\in S$,那么存在$t\in S$使得$ast=bt\in I$,于是$a\in I^S$.
	\end{proof}
	\item 如果理想满足$I\subset J$,那么有$I^S\subset J^S$.
	\item 对理想$I$,总有$I^S$是饱和的,即$(I^S)^S=I^S$.
	\item 对理想$I$,有$\sqrt{I^S}=\sqrt{I}^S$.
	\item 对理想$I,J$,有$(I\cap J)^S=I^S\cap J^S$.
	\item 对理想$I$,有$I^S=R$当且仅当$S$和$I$有交.
\end{enumerate}

分式化的理想.给定环$R$和一个乘性闭子集$S$,取$R$的理想$I$,取典范映射$\varphi:R\to S^{-1}R$.那么$\varphi(I)$在$S^{-1}R$中生成的理想为$S^{-1}I=\{a/s\mid a\in I,s\in S\}$.反过来如果给出了$S^{-1}R$中的理想$J$,那么$\varphi^{-1}(J)$是$R$中的理想.于是我们得到了$R$的理想集和$S^{-1}R$的理想集之间的两个映射:$I\mapsto S^{-1}I$和$J\mapsto\varphi^{-1}(J)$.
\begin{enumerate}
	\item 首先是一些基本关系式,分式化和理想的和积交都是可交换的,即$S^{-1}(I+J)=S^{-1}I+S^{-1}J$和$S^{-1}(IJ)=(S^{-1}I)(S^{-1}J)$和$S^{-1}(I\cap J)=(S^{-1}I)\cap(S^{-1}J)$.
	\item $\varphi^{-1}(J)$总是$R$的饱和理想.事实上如果$a\in R$满足存在$s\in S$使得$as\in\varphi^{-1}(J)$,导致$as/1\in J$,按照$J$是$S^{-1}R$的理想得到$(1/s)(as/1)=a/1\in J$,导致$a\in\varphi^{-1}(J)$.
	\item 对$S^{-1}R$的任意理想$J$,有$S^{-1}(\varphi^{-1}(J))=J$.这一事实说明了$I\mapsto S^{-1}I$是理想集的满射,换句话说$S^{-1}R$的任一理想均可以表示为$S^{-1}I$,其中$I$是$R$的某个理想;并且$J\mapsto\varphi^{-1}(J)$是单射,换句话说从$\varphi^{-1}(J_1)=\varphi^{-1}(J_2)$得到$J_1=J_2$.
	\item 对$R$的任意理想$I$,有$\varphi^{-1}(S^{-1}I)=I^S$.这一事实说明了$I\mapsto S^{-1}I$未必是单射,可能存在$S^{-1}R$的理想$J$可以表示为多个$S^{-1}I$.另外这说明了$R$的饱和理想和$S^{-1}R$的理想之间按照上述两个映射是保序的一一对应的.
	\item 上两条说明了总有$S^{-1}I=S^{-1}(I^S)$.我们断言$R$的两个理想$I,J$满足$S^{-1}I=S^{-1}J$当且仅当$I^S=J^S$.这一事实说明尽管$S^{-1}R$的理想$J$表示为$S^{-1}I$时候不是唯一的,但是这样的$I$能取的(集合意义上)极大元是唯一的,它恰好就是$\varphi^{-1}(J)$,是一个饱和理想.
	\begin{proof}
	
	一方面如果$I^S=J^S$,那么有$S^{-1}I=S^{-1}(I^S)=S^{-1}(J^S)=S^{-1}J$;另一方面如果$I^S=J^S$,那么有$\varphi^{-1}(S^{-1}I)=\varphi^{-1}(S^{-1}J)$,从$J\mapsto\varphi^{-1}(J)$是单射就得到$S^{-1}I=S^{-1}J$.最后极大性是因为只要$S^{-1}I=J$,就有$I\subset I^S=\varphi^{-1}(J)$.
	\end{proof}
    \item 上述几条说明了$S^{-1}I=S^{-1}R$当且仅当$I^S=R$,当且仅当$I\cap S$是非空的.
    \item $R$的和$S$不交的素理想是饱和理想,换句话说$R$的饱和素理想恰好是和$S$不交的素理想.另外上述$R$的饱和理想和$S^{-1}R$的理想之间的一一对应限制在饱和素理想上会得到和$S^{-1}R$的保序一一对应,换句话说$R$的和$S$不交的素理想和$S^{-1}R$的素理想之间是按照上述两个映射保序一一对应的.
    \begin{proof}
    	
    	事实上一方面任取$S^{-1}R$的素理想$Q$,那么$\varphi^{-1}Q$是$R$的饱和素理想,另一方面任取$R$的和$S$无交的素理想$P$,只要说明$S^{-1}P$是$S^{-1}R$的素理想:
    	
    	如果$(a/s_1)(b/s_2)=ab/s_1s_2=c/s\in S^{-1}P$,其中$c\in P$,等价于存在$s_0\in S$使得$abss_0=cs_1s_2s_0\in P$,结合$ss_0\not\in P$导致$ab\in P$,这得到$a,b$至少一个$\in P$,于是$a/s_1,b/s_2$至少一个$\in S^{-1}P$.
    \end{proof}
\end{enumerate}

我们指出过素理想的补集是乘性闭子集,如果把乘性闭子集$S$就取为素理想$P$的补,此时分式化$S^{-1}R$称为在素理想$P$处的局部化,记作$R_P$.这时候理想和$S$不交这个条件等价于说理想包含于$P$.于是按照分式化的理想对应定理,$R_P$的真理想总可以表示为$S^{-1}I$,其中$I$是包含于$P$的理想.而$R_P$的素理想保序一一对应于$R$的包含于$P$的素理想.于是此时$S^{-1}P$是$R_P$的唯一极大理想,即局部化总是局部环.

从理想角度看,做商和做局部化可以说是互相对偶的概念.如果取素理想$P$,那么$R/P$的理想对应于包含着$P$的$R$的理想,而局部化$R_P$的理想对应于包含于$P$的$R$的理想.

饱和素理想总是存在的.给定交换环$R$的乘性闭子集$S$,它含1不含0,在所有和$S$不交的$R$的理想中,按照Zorn引理可以找到极大元,并且这样的极大元都是$R$的素理想.
\begin{proof}
	
	设$R$的全体和$S$不交的理想构成的集合为$A$,对$A$赋予包含序,取链为$\{I_j\}_{j\in J}$.记$I=\cup_{j\in J}I_j$,那么$I$是$R$的理想,并且和$S$不交,于是由Zorn引理得到$A$在包含序下有极大元.
	
	设$P$是这样一个满足条件的极大元.假设有$ab\in P$,倘若$a,b$都不属于$P$,那么$P$真包含于$(a)+P$和$(b)+P$,于是按照极大性得到这两个理想都和$S$有交,即存在$c_1,c_2\in P,s_1,s_2\in S,r_1,r_2\in R$使得$s_1=r_1a+c_1,s_2=r_2b+c_2$.于是$s_1s_2=(r_1a+c_1)(r_2b+c_2)\in P$,导致$P$和$S$相交,矛盾.
\end{proof}

一个乘性闭子集$S$称为饱和的,如果环中任意两个元只要满足$xy\in S$,就有$x,y\in S$.例如一个交换环的单位乘法群是饱和的乘性闭子集,再例如一个交换环的全部非零因子构成饱和的乘性闭子集.
\begin{enumerate}
	\item 饱和乘性闭子集必然含1,另外如果它含0,那么它是整个环.
	\item $S$是饱和乘性闭子集当且仅当$R-S$是若干素理想的并.注意这里当$S=R$时我们约定空集$R-S$是空集作为指标集的素理想集的并,当然也可以除去这一特殊情况.
	\begin{proof}
		
		一方面如果$R-S$是若干素理想$P_i$的并,那么$xy\in S$当且仅当$xy\not\in\cup P_i$,当且仅当$xy$不在每个$P_i$中,这推出$x$和$y$均不在每个$P_i$中,于是$x,y\in S$.
		
		另一方面如果$S$是饱和乘性闭子集,我们断言和$S$不交的全部素理想的并就是补集$R-S$.为此注意到按照饱和条件,$R-S$一旦包含某个元$a$,那么它包含整个$(a)$.另外我们知道和乘性闭子集不交的理想的极大元是素理想,这说明对每个$a\in R-S$,有$a$属于于某个和$S$不交的素理想,这就完成证明.
	\end{proof}
    \item 一族饱和乘性闭子集的交仍然是饱和的乘性闭子集,据此定义一个乘性闭子集的饱和化就是全部包含该乘性闭子集的饱和乘性闭子集的交,我们断言一个乘性闭子集$S$的饱和化$\overline{S}$恰好就是全部和它不交的素理想的并的补集.
    \begin{proof}
    	
    	设和$S$不交的全部素理想为$\{P_i\}$,设$\{P_i\}$并的补集是$T$,那么$S\subset T$,并且第二条说明$T$是一个饱和乘性闭子集,于是$\overline{S}\subset T$.如果这个包含关系是真包含,那么$\overline{S}$的补集中至少存在一个素理想和$S$有交,那么这个交中的点会同时落在$S$中和$R-\overline{S}\subset R-S$中,这矛盾.
    \end{proof}
\end{enumerate}

关于乘性闭子集的复合.
\begin{enumerate}
	\item 这一条是一个引理,如下几条更为常用的结论都是这一条的推论.设$S$是环$A$的乘性闭子集.设$B$是环,并且存在环同态$g:A\to B$和$h:B\to S^{-1}A$,它们的复合$f=h\circ g$是分式化的典范映射$A\to S^{-1}A$.另外对每个$b\in B$,存在$s\in S$使得$g(s)b\in g(A)$.那么有$S^{-1}A$可视为$B$的分式化:$S^{-1}A=g(S)^{-1}B=T^{-1}B$,这里$T=\{t\in B\mid h(t)\in U(S^{-1}A)\}$.
	\begin{proof}
		
		按照分式化的泛映射性质,$h$可分解为映射的复合$B\to T^{-1}B\to S^{-1}A$,这里把第二个映射记作$\alpha$.由于$g(S)\subset T$,说明$A\to B\to T^{-1}B$还可以分解为$A\to S^{-1}A\to T^{-1}B$,把这里的$S^{-1}A\to T^{-1}B$记作$\beta$,那么有:
		$$\alpha(\beta(a/s))=\alpha(g(a)/g(s))=hg(a)/hg(s)=f(a)/f(s)=a/s$$
		
		于是$\alpha\circ\beta$是$S^{-1}A$上的恒等映射,再按照条件对每个$b\in B$,存在$a\in A$和$s\in S$使得$bg(s)=g(a)$,于是$\beta(a/s)=g(a)/g(s)=b/1$,特别的,如果$t\in T$,取$u\in S^{-1}A$使得$t/1=\beta(u)$,就得到$u=\alpha\beta(u)=\alpha(t/1)=h(t)$,于是$u$是$S^{-1}A$中单位,于是$b/t=\beta(a/s)\beta(u^{-1})$,这说明$\beta$和$\alpha$互为逆映射,另外注意到$T$即$g(S)$的饱和化,于是$S^{-1}A=g(S)^{-1}B$,完成证明.
	\end{proof}
    \item 特别的,如果$B$是$A\subset S^{-1}A$的中间环,那么$B$是$A$的某个分式化.
    \item 特别的,如果$S,T$是$A$的两个乘性闭子集,满足$S\subset T$,把$T$在$S^{-1}A$中的像记作$T'$,那么有$(T')^{-1}(S^{-1}A)=(T')^{-1}A$.
    \item 特别的,如果$S\subset A$是乘性闭子集,而$p$是$A$的和$S$不交的素理想,那么有$S^{-1}A$在$S^{-1}p$处的分式化即$A_p$.
    \item 特别的,如果$A$有素理想$p\subset q$,那么$(A_q)_{pA_q}=A_p$.
\end{enumerate}

给定环$R$上两个乘性闭子集$S\subset T$,这诱导了两个分式化$S^{-1}R$与$T^{-1}R$,另外有典范的同态$S^{-1}R\to T^{-1}R$把$a/s$映射为自身.这个映射是同构当且仅当满足如下等价条件的任一:
\begin{enumerate}
	\item 对每个$t\in T$有$t/1$是$S^{-1}R$中单位.
	\item 对每个$t\in T$有$r\in R$使得$rt\in S$.
	\item $T$包含在$S$的饱和化中.
	\item 每个和$T$有交的素理想也和$S$有交.
\end{enumerate}
\begin{proof}
	
	如果映射是同构,那么它保单位,于是$t/1$在$T^{-1}R$中是单位得到$t/1$在$S^{-1}R$中是单位.
	
	如果$t/1$是单位,那么存在$t/1\cdot a/s=1/1$,也即存在$s'\in S$使得$tas'=ss'\in S$,于是$x=as'$满足$tx\in S$.
	
	如果第二条成立,按照$xt\in\overline{S}$,说明$x,t\in\overline{S}$,于是$T\subset\overline{S}$.
	
	如果$T$包含在$S$的饱和化中,则和$S$不交的素理想必然和$T$不交,换句话说和$T$有交的素理想必然和$S$有交.
	
	最后如果第四条成立,【】
\end{proof}

给定交换环$R$上的模$M$,给定乘性闭子集$S$,称$M$具有兼容的$S^{-1}R$模结构,如果$\forall r\in R$有$rm=(r/1)m,\forall m\in M$成立.那么$M$具有兼容的$S^{-1}R$模结构当且仅当对每个$s\in S$,有左乘$s$的$M$上映射$\mu_s$是交换群同构.并且兼容的$S^{-1}R$模结构一旦存在则唯一.
\begin{proof}
	
	一方面如果$M$具有兼容的$S^{-1}R$模结构,那么有$\mu_s\mu_{1/s}=\mu_{s/1}\mu_{1/s}=1$和$\mu_{1/s}\mu_s=1$,于是$\mu_s$是$M$上的交换群同构.另一方面如果每个$s\in S$都有$\mu_s$是同构,构造$\mu_{a/s}=\mu_a\mu_s^{-1}$,由此得到唯一的$\mu:S^{-1}R\to\mathrm{End}_{\mathbb{Z}}(M)$.这还可以从环的分式化的泛映射性质直接得到,注意到环的分式化的泛映射性质中可以把交换环替换为任意含幺环.
\end{proof}

分式模.对交换环$R$模$M$,和$R$的一个乘性闭子集$S$,约定含1不含0,定义$M\times S$上的等价关系为$(m,s)\sim(m',s')$为存在$s_0\in S$使得$s_0(s'm-sm')=0$.$(m,s)$所在的等价类记作$m/s$.把全体等价类构成的集合记作$S^{-1}M$,它称为$M$关于$S$的分式化.此时$S^{-1}M$同时具备$R$模结构和$S^{-1}R$模结构,即$r(m/s)=rm/s$和$(r/s')(m/s)=rm/ss'$.并且此时$S^{-1}R$模结构是兼容的.

分式模有如下性质:
\begin{enumerate}
	\item 分式模作为$S^{-1}R$模的兼容性说明每个$\mu_s:S^{-1}M\to S^{-1}M$是交换群同构,这里$s\in S$.
	\item 从$M\to S^{-1}M$的典范映射$m\mapsto m/1$是$R$模同态.
	\item 如果取乘性闭子集是素理想$p$的补集,称模$M$此时的分式化为关于$p$的局部化,记作$M_p$.给定环中的点$f$,取乘性闭子集是$\{f^m\mid m\ge0,f^0=1\}$,称模$M$此时的分式化为关于$f$的局部化,记作$M_f$.
	\item 分式模的泛映射性质.考虑$R$模$M$,给定$R$上一个乘性闭子集$S$,含1不含0,对任意的$S^{-1}R$模$N$,它自然具备$R$模结构.记典范同态$M\to S^{-1}M$为$f$,那么对任意$R$模同态$\varphi:M\to N$,存在唯一的$S^{-1}R$模同态$\varphi':S^{-1}M\to N$延拓了$\varphi$,即满足$\varphi'\circ f=\varphi$.并且分式化被这个性质完全刻画.
	\item $R$模$M$的分式化记作$S^{-1}M$,典范映射是同构当且仅当$M$具备兼容的$S^{-1}R$模结构,当且仅当每个$\mu_s,s\in S$是$M$上的交换群同构.
	\item 给定$R$的两个乘性闭子集$S\subset T$,给定$R$模$M$,记$T_0=\varphi(T)\subset S^{-1}R$,那么有$T^{-1}M=T_1^{-1}(S^{-1}M)$.
\end{enumerate}

和环的情况一样,借助饱和子模的概念可以完全从$M$的子模刻画$S^{-1}M$的子模.设$R$是交换环,$S$是乘性闭子集,取$R$模$M$,子模$N$的饱和化是指子模$N^S=\{m\in M\mid\exists s\in S, sm\in N\}$.称子模$N$是饱和的如果$N=N^S$.记典范同态$\varphi:M\to S^{-1}M$:
\begin{enumerate}
	\item 饱和化的一些基本性质:典范同态$M\to S^{-1}M$的核就是$\{0\}^S$;如果$M$的子模满足$N_1\subset N_2$,那么有$N_1^S\subset N_2^S$;饱和化总是饱和子模,即$(N^S)^S=N^S$.
	\item 对$M$的任意子模$N$,有$S^{-1}N=\{n/s\mid n\in N,s\in S\}$是由$\varphi(N)$生成的$S^{-1}M$的子模.这就得到了从$M$的子模到$S^{-1}M$子模的映射$N\mapsto S^{-1}N$;反过来$T\mapsto\varphi^{-1}(T)$是$S^{-1}M$子模到$M$子模的映射.
	\item 这两个映射的复合满足:$\varphi^{-1}(S^{-1}N)=N^S$和$S^{-1}(\varphi^{-1}T)=T$.这说明每个$S^{-1}M$的子模均可以表示为$S^{-1}N$的形式,这个表示尽管可能不唯一,即$T=S^{-1}N$这里的$N$可以取不同的$M$子模,但是其中包含序下极大的是唯一的一个,也即$\varphi^{-1}(T)$.
	\item 上一条就说明了$M$的全体饱和子模和$S^{-1}M$的子模是保序一一对应的.
	\item 特别的$S^{-1}M=\{0\}$当且仅当$\{0\}^S=M$,也即对$M$中任一元可以被$S$中的元零化.于是倘若$\mathrm{Ann}(M)\cap S$非空,可推出$S^{-1}M=0$.反过来一般不成立,但是如果$M$是有限生成模则是成立的.
\end{enumerate}

分式化函子.给定环$R$和一个乘性闭子集$S$,那么模同态$f:M\to N$诱导了$S^{-1}M\to S^{-1}N$的同态$S^{-1}f:m/s\mapsto f(m)/s$.$M\mapsto S^{-1}M$和$f\mapsto S^{-1}f$是$R$模范畴到$S^{-1}R$模范畴上的加性函子,称为关于$S$的分式化函子,记作$S^{-1}(\cdot)$.其中函子加性是指限制在Hom集上构成了交换群同态.另外注意到典范映射$\varphi$满足如下图表交换,这说明典范映射族$\{\varphi_M\}$是一个自然变换.
$$\xymatrix{M\ar[r]^{\varphi_M}\ar[d]^{f}&S^{-1}M\ar[d]^{S^{-1}f}\\ N\ar[r]^{\varphi_N}&S^{-1}N}$$

分式化函子是左伴随函子.取$S^{-1}R$模范畴到$R$模范畴中的函子为限制系数函子$F$,即把每个$S^{-1}R$模$M$对应为$R$模$M$,$S^{-1}R$模同态不变的视为$R$模同态.分式化函子是左伴随于这个函子的.于是分式化函子保全部余极限.
\begin{proof}

取$R$模$M$和$S^{-1}R$模$N$,那么$N=S^{-1}N$,对任意的$S^{-1}R$模同态$f:S^{-1}M\to N$,记$\tau_{M,N}(f)$为同态的复合$M\to S^{-1}M\to N$,这是一个$R$模同态.这里$\tau_{M,N}$实际上是$\mathrm{Hom}_{S^{-1}R}(S^{-1}M,N)\to\mathrm{Hom}_R(M,FN)$的同构,因为它的逆映射恰好就是$S^{-1}(\cdot)$.接下来验证对任意$S^{-1}R$模同态$f:N\to N'$和任意$R$模同态$g:M\to M'$,有如下两个图表交换,这就说明了伴随性:
$$\xymatrix{\mathrm{Hom}_{S^{-1}R}(S^{-1}M,N)\ar[d]_{f_*}\ar[r]^{\tau_{M,N}}&\mathrm{Hom}_R(M,FN)\ar[d]^{(Ff)_*}\\ \mathrm{Hom}_{S^{-1}R}(S^{-1}M,N')\ar[r]_{\tau_{M,N'}}&\mathrm{Hom}_R(M,FN')
}$$
$$\xymatrix{\mathrm{Hom}_{S^{-1}R}(S^{-1}M,N)\ar[d]_{(S^{-1}g)_*}\ar[r]^{\tau_{M,N}}&\mathrm{Hom}_R(M,FN)\ar[d]^{g_*}\\ \mathrm{Hom}_{S^{-1}R}(S^{-1}M',N)\ar[r]_{\tau_{M',N}}&\mathrm{Hom}_R(M',FN)
}$$
\end{proof}

分式化函子是正合函子.已经知道分式化函子是加性函子,从左伴随性得到它保余核,只需再说明它保核就得到它是正合函子.事实上任取$R$模同态$f:M\to N$,记核为$K=\ker f$,需要证明的是$\ker S^{-1}f=S^{-1}K$.一方面任取$S^{-1}K$中的元$a/s$,那么$S^{-1}f(a/s)=f(a)/s=0$.另一方面任取$a/s\in\ker S^{-1}f$,即$f(a)/s=0$,于是存在$s'\in S$使得$s'f(a)=0$,导致$s'a\in K$,于是$a/s=s'a/s's\in S^{-1}K$,完成证明.

分式化函子与一些算子的可交换性.
\begin{enumerate}
	\item 分式化和有限和与有限交可交换,即如果$N_1,N_2$是$M$的两个子模,那么$S^{-1}(N_1\cap N_2)=S^{-1}N_1\cap S^{-1}N_2$和$S^{-1}(N_1+N_2)=S^{-1}N_1+S^{-1}N_2$.但是分式化不和无限交运算可交换.取$k$是一个无限域,取$R=k[x]$,取乘性闭子集$S$是$R$的全体非零元,取$M_a,a\in k$是$x-a$生成的子模.那么有$S^{-1}\left(\cap_{a\in k} M_a\right)=0$,但是注意到$1\in\cap_{a\in k}S^{-1}M_a$.
	\item 分式化和商可交换.给定交换环$R$的乘性闭子集$S$,给定$R$的理想$I$,给定$R$模$M$,那么$S^{-1}(M/IM)=S^{-1}M/S^{-1}(IM)=S^{-1}M/IS^{-1}M$.特别的,取$M=R$得到$S^{-1}(R/I)=S^{-1}R/S^{-1}I$.特别的,取$R$的素理想$p$,那么$\mathrm{Frac}(R/p)=(R/p)_p=R_p/pR_p$,这称为$R$在素理想$p$处的剩余类域.特别的,这一条也得到了分式化是保余核的,即有$(N/\mathrm{im}\varphi)_P\cong N_P/ (\mathrm{im}\varphi_P)$,这一事实我们刚刚是从左伴随性得出的.
\end{enumerate}

局部性质.所谓的局部性质,通常是这样的陈述:一个模$M$具备一个性质,等价于对任意素理想$P$有$M_P$具备这个性质,又等价于对任意极大理想$m$有$M_m$具备这个性质.
\begin{enumerate}
	\item 模是零模这一性质是局部性质.记模$M$是零模当且仅当对每个素理想$P$有$M_P=0$,当且仅当对每个极大理想$m$有$M_m=0$.
	\begin{proof}
		
		假设对每个极大理想$m$均有$M_m=0$.任取$M$中的元$a$,它是零元等价于讲$\mathrm{Ann}(a)=R$.倘若$\mathrm{Ann}(a)$是真理想,那么它包含于某个极大理想$m$中,从$M_m=0$得到$a/1=0$,导致存在某个$s\in R-m$使得$sa=0$,导致$s\in\mathrm{Ann}(a)$,这矛盾.
	\end{proof}
	\item 单满同态是一个局部性质.即模同态$\varphi:M\to N$是单同态/满同态,等价于对每个素理想$P$,它诱导的模同态$\varphi_P:M_P\to N_P$都是单同态/满同态,也等价于对每个极大理想$m$,它诱导的模同态$\varphi_m:M_m\to N_m$都是单同态/满同态.
	\begin{proof}
		
		已经知道分式化函子保核,于是按照0模的局部性质,得到$\varphi$是单同态等价于$\ker\varphi=0$等价于全部$(\ker\varphi)_P=0$等价于全部$\ker\varphi_P=0$等价于全部$\ker\varphi_m=0$,即全部$\varphi_P$是单同态与全部$\varphi_m$是单同态.
		
		对偶的按照分式化函子保余核,于是按照0模的局部性质,得到$\varphi$是满同态等价于$N/\mathrm{im}\varphi=0$等价于全部$(N/\mathrm{im}\varphi)_P=0$等价于全部$N_p/\mathrm{im}\varphi_P$等价于全部$N_m/\mathrm{im}\varphi_m=0$,即全部$\varphi_P$是满同态核全部$\varphi_m$是满同态.
	\end{proof}
    \item 一个交换环称为简化环,如果它不存在非平凡的幂零元.那么简化性是局部性质,即一个交换环$R$是简化环当且仅当对每个素理想$P$有$R_P$是简化环,当且仅当对每个极大理想$m$有$R_m$是简化环.
    \begin{proof}
    	
    	假设$R$是简化环,任取素理想$P$,假设$R_P$中有幂零元$a/s$,其中$a\in R,s\in R-P$,于是有$(a/s)^n=0$,于是存在$R-P$中的元$s_0$使得$a^ns_0=0$,于是$(as_0)^n=0$,按照$R$简化就说明$as_0=0$,导致在$R_P$中有$a/s=0$.
    	
    	假设$R$存在非平凡的幂零元$a$,那么$I=\mathrm{Ann}(a)$是真理想,它包含于某个极大理想$m$中,考虑$R_m$,其中的元$a/1$是非零元,否则存在$s\in R-m$使得$sa=0$,和$\mathrm{Ann}(a)\cap m=\emptyset$矛盾.这说明$R_m$存在非平凡的幂零元.
    \end{proof}
    \item 无挠性是一个局部性质.给定整环$R$,那么$R$模$M$是无挠模当且仅当对每个素理想$P$有$M_P$是无挠$R_P$模,当且仅当对每个极大理想$m$有$M_m$是无挠$R_m$模.
    \begin{proof}
    	
    	首先取乘性闭子集$S$,先证明$\mathrm{Tor}(S^{-1}M)=S^{-1}(\mathrm{Tor}(M))$.首先右侧包含于左侧是直接的,现在任取左侧的一个元$m/s$,于是存在$r\in R$使得$rm/s=0$,也即存在$t\in S$使得$rtm=0$,于是$m\in\mathrm{Tor}(M)$,于是左侧包含于右侧.
    	
    	现在$M$是无挠模等价于$\mathrm{Tor}(M)=0$,于是按照零模是局部性质,以及分式化和$\mathrm{Tor}$可交换,就得到无挠性是一个局部性质.
    \end{proof}
    \item 一个环称为绝对平坦环,如果它的每个模都是平坦模.这是一个局部性质,即交换环$R$是绝对平坦环当且仅当对每个素理想$P$有$R_P$是绝对平坦环,当且仅当对每个极大理想$m$有$R_m$是绝对平坦环.
    \begin{proof}
    	
    	先来证明绝对平坦环$R$的分式化总是绝对平坦环.任取$x/s\in S^{-1}R$,那么存在$x$的广义逆$y$,即满足$x^2y=x$.于是$y/s$是$x/s$的广义逆,于是$S^{-1}R$是绝对平坦环.
    	
    	现在证明如果对每个极大理想$m$有$R_m$是绝对平坦环,那么$R$是绝对平坦环.后文会给出交换环是绝对平坦环的一个等价描述为对每个极大理想$m$有$R_m$必然是域,于是这里我们只需来说明$R_m$是域.假设可以取到$R_m$的一个非零非单位元$a/s$,那么存在某个$b/t\in R_m$,满足$(a/s)((a/s)(b/t)-1)=0$,那么$(a/s)(b/t)-1$不会是单位元,否则$(a/s)=0$和假设矛盾,于是它是唯一极大理想$m_m$中的元,于是$ab-st\in m$,而$ab\in m$导致$st\in m$,但是$s,t\not\in m$,这就和$m$是素理想矛盾.
    \end{proof}
\end{enumerate}

补充.
\begin{enumerate}
	\item 设$A$是整环,商域记作$K$,当$\mathfrak{m}$取遍$A$的极大理想时,有$A=\cap_{\mathfrak{m}}A_{\mathfrak{m}}$.
	\begin{proof}
		
	    对$x\in K$,记$I=\{a\in A\mid ax\in A\}$,这是$A$的理想,那么$x\in A_{\mathfrak{p}}$等价于讲$I\not\subseteq\mathfrak{p}$,于是$x\in A_{\mathfrak{m}}$对每个极大理想$\mathfrak{m}$成立等价于讲$I$不包含在每个极大理想中,于是$I$是单位理想,也即$x\in A$.
	\end{proof}
    \item 设$A$是环,$M$是有限$A$模,如果$M\otimes_A\kappa(\mathfrak{m})=0$对每个极大理想$\mathfrak{m}$成立,那么$M=0$.
    \begin{proof}
    	
    	我们有$M\otimes_A\kappa(\mathfrak{m})=M_{\mathfrak{m}}/\mathfrak{m}M_{\mathfrak{m}}$.NAK引理说明$M\otimes_A\kappa(\mathfrak{m})=0$等价于$M_{\mathfrak{m}}=0$.按照局部性质有$M=0$.
    \end{proof}
    \item 设$f:A\to B$是环同态,设$M$是$B$模,如果对每个素理想$\mathfrak{p}\in\mathrm{Spec}A$有$M\otimes_A\kappa(\mathfrak{p})=0$,那么$M=0$.
    \begin{proof}
    	
    	如果$M\not=0$,那么至少存在一个$B$的极大理想$\mathfrak{P}$使得$M_{\mathfrak{P}}\not=0$.记$\mathfrak{p}=\mathfrak{P}\cap A$(这个记号表示$f^{-1}(\mathfrak{P})$),于是按照NAK引理有$M_{\mathfrak{P}}/\mathfrak{P}M_{\mathfrak{P}}\not=0$.因为有满射$M_{\mathfrak{P}}/\mathfrak{p}M_{\mathfrak{P}}\to M_{\mathfrak{P}}/\mathfrak{P}M_{\mathfrak{P}}$,就得到$M_{\mathfrak{P}}/\mathfrak{p}M_{\mathfrak{P}}\not=0$.但是$M_{\mathfrak{P}}/\mathfrak{p}M_{\mathfrak{P}}$是$M\otimes_A\kappa(\mathfrak{p})$的分式化,这迫使$M\otimes_A\kappa(\mathfrak{p})\not=0$,这就矛盾.
    \end{proof}
\end{enumerate}
\newpage
\subsection{自由模}

取环$R$,给定集合$X$,在左$R$模范畴中的自由对象$F(X)$被如下图表描述的泛映射性质唯一决定:
$$\xymatrix{
	F(X)\ar[r]^{\varphi}&M\\
	X\ar[u]^{i}\ar[ur]_{j}&
}$$

模仿$\mathbb{Z}$模的情况,应该构造$F(X)$是$|X|$个左$R$模$R$的直和$R^{\oplus X}$.把典范映射$i:X\to R^{\oplus X}$取为对每个$x\in X$,$i(x)$是在$x$分量取$1_R$,其余分量取0的元.它满足自由对象的泛映射性质.

对任意左$R$模$M$,任取$M$的子集$X$,则按照泛映射性质存在唯一的模同态$\varphi:F(X)\to M$.称$X$是线性无关的,如果$\varphi$是单同态,否则称为线性相关的;称$X$是生成元集,如果$\varphi$是满同态.称$X$是$M$的一组基,如果$\varphi$是模同构.这等价于说,$R$模$M$的子集$X$是线性无关的,如果对于$X$的任意一个有限子集$a_1,\cdots,a_n$,从
$r_1a_1+\cdots+r_na_n=0,r_i\in R$能推出$r_1=\cdots=r_n=0$.称一个线性无关组是基,如果对任意$M$中的$m$,存在$I$的有限子集$a_1,\cdots,a_n$和一组系数$r_1,\cdots,r_n\in R$使得$m=\sum_{i=1}^nr_ia_i$.
\begin{enumerate}
	\item 若$M$的一个子集中有重复的元素,那么它必然是线性相关的,如果子集包含0元素,那么同样必然是线性相关的.
	\item 对任意模$M$,取$M$作为集合上的自由模,这个自由模到$F$是满同态.即任何一个模都是某个自由模的商.
	\item 当模$M$的基存在时,$M$同构于自由模,反过来对于自由模$(F(X),i)$,那么$i(X)$就是$F(X)$的一组基,即模是自由模当且仅当它存在一组基.
	
	总结一下至此我们得到的自由模的等价描述:$F$是环$R$上的模,则$F$是自由模如果它满足如下等价条件中的任一个:
	\begin{enumerate}
		\item $F$有非空基.
		\item $F$同构于一族$R$的直和
		\item $F$是$R$模范畴上的自由对象
	\end{enumerate}
\end{enumerate}

自由模的商和分式化.
\begin{enumerate}
	\item 给定环$R$,给定真理想$I$,如果$F$是一个自由$R$模,那么$F/IF$是自由$R/I$模,另外如果$X$是$F$的一组基,就有$\pi(X)$是$F/IF$的一组基,并且$|X|=|\pi(X)|$.
	\begin{proof}
		
		取$F$的一组基$X=\{x_i\mid i\in I\}$,如果$u+IF\in F/IF$,得到$u+IF$可以被$\pi(X)=\{x_i+IF\mid i\in I\}$线性表出.
		
		现在证明$\pi(X)$是线性无关组.如果$\sum_{i=1}^{m}r_kx_k+IF=0$,得到$\sum_k r_kx_k\in IF$,于是有$\sum_kr_kx_k=\sum_j s_ju_j$其中$s_j\in I u_j\in F$.按照每个$u_i$是$X$的线性组合,以及$X$是一组基,得到$\sum_kr_kx_k$是$I$系数的线性组合.于是$r_k\in I$,于是得到$\pi(X)$是线性无关组,导致它是一组基,于是$F/IF$是$R/I$自由模.
	\end{proof}
    \item 分式化.按照分式化与直和可交换,说明如果$R$是交换环,$S$是乘性闭子集,$M$是自由$R$模,那么$S^{-1}M$是自由$S^{-1}R$模.
\end{enumerate}

对$R$模$M$的任意已给定的线性无关组,总可以找到一个极大的线性无关组包含它.事实上该命题是等价于Zorn引理的.
\begin{proof}
	
	取定模$M$的一个线性无关组$X$,记全体包含$X$的线性无关组构成的集合为$S$.赋予$S$包含序,任取一个链$C=\{Y_i\}_{i\in I}$.考虑$Y=\cup_{i\in I}Y_i$,现在证明它是一个线性无关组,倘若存在$x_1,x_2,\cdots,x_n\in Y$,并且$R$中的元$r_1,r_2,\cdots,r_n$使得$\sum r_ix_i=0$.可取到一个$Y_t$包含了全部$x_i$,于是按照$Y_t$的线性无关性,得到$r_i\equiv0$.于是按照Zorn引理得到$S$中存在极大元.
\end{proof}

但是模上的线性无关组未必存在.这样的例子比较容易构造,称模$M$上的一个元是挠元,如果它自身构成一个线性相关集,等价于说存在环$R$中的非零元零化它.那么模中的0元自然是挠元,称为平凡的挠元.如果模的所有元都是挠元,那么此时模中不存在线性无关组.例如$\mathbb{Z}/4$的理想${[0],[2]}$作为$\mathbb{Z}/4$的子模不存在线性无关组.

倘若线性无关组存在,那么极大线性无关组存在,但是极大线性无关组也未必是基.例如交换群$\mathbb{Z}$作为$\mathbb{Z}$模,它的子集${2}$是极大线性无关组,但是不生成整个模.

对于除环$R$上的模,每个非0元都构成一个线性无关组,此时极大线性无关组就存在.我们断言这样的极大线性无关组必然是基.于是除环上的模总是自由模.
\begin{proof}
	
	任取除环$R$上模$M$的一个极大线性无关组$X$,假设$X$不能生成整个模,也即存在一个$m\in M$不被$X$线性表出,我们断言$X\cup\{m\}$仍然是一个线性无关组.倘若存在$X$中的一组元$x_1,x_2,\cdots,x_n$以及$R$中的元$r_0,r_1,\cdots,r_n$满足$r_0m+r_1x_1+\cdots+r_nx_n=0$.如果$r_0=0$,那么按照$X$的线性无关性得到$r_i\equiv0$.如果$r_0\not=0$,那么它是一个单位元,导致$m=-r_0^{-1}\sum r_ix_i$,这导致$m$可以被$X$线性表出,同样矛盾.
\end{proof}
\begin{enumerate}
	\item 除环上模的一个线性无关组必然包含在某个基中.因为线性无关组必然包含于某个极大线性无关组.
	\item 除环上模的一个生成元集必然包含了某组基.对于非0模,生成元基必然包含了非0元,于是这个生成元集至少包含了一个线性无关组.按照Zorn引理取这个生成元集包含的极大线性无关组,那么除环条件保证了这个极大线性无关组必然生成了生成元集的所有元,于是它是一组基.
\end{enumerate}

特别的,我们把域上的模称为线性空间,把模中元素称为向量,把域中元素称为系数.于是域上的模都是自由模,线性空间总存在一组基.线性代数便是处理域上模的理论.

设$R$是交换环,$F$是自由$R$模,设一组基为$X$,取$R$的极大理想$m$,那么$F/mF$是$R/m$上的线性空间,并且$X'=\{x_i+mF,x_i\in X\}$是一组基.首先$X'$生成整个$F/mF$是直接的,现在只要说明它是线性无关的,倘若存在一组$x_i\in X$和$r_i\in R$使得$\sum r_ix_i\in mF$,那么存在$s_i\in m$使得$\sum r_ix_i=\sum s_ix_i$,这迫使所有$r_i=s_i\in m$导致$\sum r_ix_i\in mF$,于是线性无关.

一个自然的问题是,给定环上自由模的两组基,它们的势是否相同?我们会看到的确存在环上的自由模存在不同势的基,但是这种情况不会出现在除环和交换环上:称一个环满足左/右维数不变性(Invariant Basis Number Property),如果它的所有左/右自由模上任意两组基具有相同的势.交换环和除环都满足维数不变性.当自由模的两组基的势总相同时,就把这个势称为自由模的秩.
\begin{enumerate}
	\item 设$R$是环,如果自由模$M$存在一组基$X=\{x_i\mid i\in I\}$是无穷集,那么$M$的任意基都是无穷集,并且势和$|I|$相同.
	\begin{proof}
		
		假如取了一组基$Y=\{y_j\mid j\in J\}$满足$|J|<|I|$.先来说明如果$I$是无穷集,那么$J$总是无限集.假设$J$是有限集,按照每个$y_j$被有限个$x_i$生成,那么存在有限子集$I_0\subset I$使得所有$y_j$都被$\{x_i\mid i\in I_0\}$生成,导致$\mathrm{span}(Y)\subset\mathrm{span}\{x_i\mid i\in I_0\}\subsetneqq M$.这就矛盾.
		
		下面证明$|X|=|Y|$.设$Y$的全体有限子集构成的集合为$S(Y)$,那么按照$Y$是无限集合得到$|S(Y)|=|Y|$.现在定义映射$f:X\to S(Y)$为,对每个$x_i\in X$,它可唯一的表示为$\sum r_jy_j$,其中$r_j\not=0$,就把$f(x_i)$定义为$\{y_j\}\in S(Y)$.那么$\mathrm{im}f$同样是无限集,否则导致$Y$中有限子集就生成了整个$M$.现在任取$T\in\mathrm{im}f$,我们断言$f^{-1}(T)$是有限集,一旦这一断言得证,按照$f^{-1}(T),T\in\mathrm{im}f$是$X$的划分,每一部分是有限集,就得到$|X|\le|\mathrm{im}f|\aleph_0=|\mathrm{im}f|\le|S(Y)|=|Y|$,同理会得到$|Y|\le|X|$,这就得证.
		
		最后来说明$f^{-1}(T)$是$X$的有限子集.设$T$生成的自由子模为$F_T$,那么$f^{-1}(T)\subset F'$.取$X$的有限子集$S$,使得有限集$T$中每个元被$S$线性生成,设$S$生成的自由子模是$F_S$,那么得到$F_T\subset F_S$.于是$f^{-1}(T)\subset F_S$,于是$f^{-1}(T)$中每个元都被$S$中的元线性表出,但是$f^{-1}(T)\in X$,这说明只能有$f^{-1}(T)\subset S$,于是为有限子集.
	\end{proof}
	\item 于是一个自由模两组基的势不同的情况,只能出现在基总是一个有限集的情况下.并且验证一个环是左/右维数不变性,只需验证对任意正整数$n,m$有左$R$模的同构$R^m\cong R^n$当且仅当$m=n$.
	
	遗憾的是,这种情况是存在的.一个经典的例子是,取$V$为域$k$上的可数维线性空间,考虑环$R=\mathrm{End}_k(V)$即$V$上自同态环.那么$R$作为自身左$R$模,有一组基含有1个元,即恒等映射.但是,$R$上事实上存在两个元素的基.我们取$V$上一组基为$\{e_1,e_2,\cdots\}$.取$R$中这样两个映射$f_1,f_2$:$f_1(e_ {2n})=e_n,f_1(e_{2n-1})=0$,$f_2(e_{2n})=0,f_2(e_{2n-1})=2_n$.那么$f_1,f_2$是线性无关的,事实上如果有$g_1,g_2\in R$使得$g_1f_1+g_2f_2=0$,那么作用在$e_{2n}$上得到$g_1(e_n)\equiv0$,导致$g_1=0$.再作用在$e_{2n-1}$上得到$g_2(e_n)\equiv0$,导致$g_2=0$.于是线性无关.并且,$R$中每个元$g$是它的线性组合,事实上取$g_1(e_n)=g(e_ {2n})$,$g_2(e_n)=g(e_{2n-1})$.我们看到有$g=g_1f_1+g_2f_2$.于是$R$作为模同构于$R^2$.
	
	另外注意模同构$R\cong R^2$导致$R\cong R^n$对任意正整数$n$成立.
	\item 按照线性代数的语言,$R$不满足维数不变性当且仅当存在存在$m\times n,m\not=n$的矩阵$A$和$n\times m$的矩阵$B$使得$AB=E_m$,$BA=E_n$.例如上述例子,$f_1,f_2$取做同上,取$g=1$对应的$g_1,g_2$,那么$\left(g_1,g_2\right)\left(\begin{array}{c} f_1\\ f_2\end{array}\right)=1$并且$\left(\begin{array}{c} f_1\\ f_2\end{array}\right)\left(g_1,g_2\right)=\left(\begin{array}{cc}1&0\\0&1\end{array}\right)$.
	
	另外,在这个等价描述下,我们可以直接得出环的右维数不变性和左维数不变性是要么同时成立要么同时不成立的.于是今后我们把左右维数不变性统称为维数不变性.
	
	借助Hom函子我们还可以给出第二个证明.如果$R$视为左$R$模,从$R^m\cong R^n$总能推出$m=n$,需要验证的是$R$视为右$R$模的时候从$R^m\cong R^n$推出$m=n$.为此注意到视为右$R$模的同构诱导出同构$\mathrm{Hom}_R(R^m,R)\cong\mathrm{Hom}_R(R^n,R)$,其中两个Hom集中第二个$R$视为$(R,R)$模,于是有左$R$模同构$\mathrm{Hom}_R(R^i,R)\cong R^i$,这就得到$m=n$.
	\item 有限环满足维数不变性,事实上这时候$R^n\cong R^m$推出$|R|^n=|R|^m$,于是必然有$m=n$.
	\item 除环满足维数不变性.
	\begin{proof}
		
		给定除环$D$上一个模$V$.设$V$不存在基是无限集.假设$V$存在两个有限的基$X=\{x_1,\cdots,x_n\}$, $Y=\{y_1,\cdots,y_m\}$,假设有$m>n$.按照$X,Y$都是基,得到$0\not=y_1=r_1x_1+\cdots+r_nx_n$.那么右侧必然存在某个系数$r_i$不为0,不妨重排$x_i$使得$r_1$不为0,得到$x_1=r_1^ {-1}y_1-r_1^{-1}r_{2}x_{2}-\cdots-r_1^{-1}r_nx_n$.于是$\{y_1,x_2,\cdots,x_n\}$是一组基.归纳替换下去,得到了$\{y_1,\cdots,y_n\}$是一组基,这和$Y$的线性无关性矛盾.
	\end{proof}
	\item 如果存在$R\to S$的环同态$f$,其中$S$不是0环并且满足维数不变性,则$R$也满足维数不变性.特别的,如果$S$的一个非0环的商满足维数不变性,那么$S$满足维数不变性.
	\begin{proof}
		
		如果$R$不满足维数不变性,即存在正整数$m\not=n$,使得存在两个矩阵$A:m\times n$和$B:n\times m$,有$AB=E_m,BA=E_n$.将$f$作用到这个矩阵等式中,得到$S$不满足维数不变性.
	\end{proof}
	\item 交换环总满足维数不变性.交换环上总存在极大理想,并且商去极大理想得到域,域上满足维数不变性,于是按照上一个定理得到交换环满足维数不变性.这里我们借助张量积给出这个结论的第二个证明.
	\begin{proof}
		
		假设交换化$R$满足$R^m\cong R^n$,设同构为$\varphi$,取$R$的一极大理想$m$,那么$\varphi$诱导了同构$1\otimes\varphi:(R/m)\otimes_RR^m\cong(R/m)\otimes_RR^n$(函子把同构映射为同构).这里同构号两侧即$R/m$上的$m$维线性空间和$n$维线性空间,并且这里的同构$\varphi$可视为$R/m$线性空间同构,这就导致$m=n$.
	\end{proof}
\end{enumerate}

秩条件,强秩条件.
\begin{enumerate}
	\item 称一个环满足(右)秩条件,如果对每个正整数$n$,对右自由模$R^n$上任意的生成元集,它的势$\ge n$.这个条件等价于说,如果存在(右)自由模的满同态$R^k\to R^n$,那么有$k\ge n$.
	\item 称一个环满足强(右)秩条件,如果对任意正整数$n$对右自由模$R^n$上任意线性无关组具有势$\le n$.等价的说,如果存在右自由模之间的单同态$R^m\to R^n$,那么$m\le n$.
\end{enumerate}

尽管这两个条件看起来是对偶的,但是事实上强(右)秩条件可以推出(右)秩条件.
\begin{proof}
	
	任取一个满同态$\alpha:R^k\to R^n$,取$R^n$一组基$\{x_1,\cdots,x_n\}$,的原像是$\{y_1,\cdots,y_n\}$,那么$\{y_1,\cdots,y_n\}$是$R^k$上线性无关的.于是我们可以构造$R^n\to R^k$的单同态为把$x_i$映射为$y_i$.这按照强秩条件得到$n\le k$,于是得证.
\end{proof}

另外这两个条件都可以推出维数不变性.事实上如果$R^n\sim R^m$,按照秩条件得到$n\le m$和$m\le n$,于是$m=n$.

维数不变性,秩条件,强秩条件的等价描述:
\begin{enumerate}
	\item 环$R$满足维数不变性等价于,若有$R$上$m\times n$矩阵$A$和$n\times m$矩阵$B$满足$AB=E_m$和$BA=E_n$,那么有$m=n$.
	\item 环$R$满足(右)秩条件等价于,若存在$n\times k$矩阵$A$和$k\times n$矩阵$B$满足$AB=E_n$,那么$n\le k$.
	\begin{proof}
		
		一方面如果存在这样的矩阵$A,B$但是$n>k$,定义$\alpha:R^k\to R^n$为左乘矩阵$A$,这是一个满同态,导致秩条件不成立.另一方面,如果秩条件不成立,可以找到满同态$\alpha:\alpha:R^k\to R^n$使得$k<n$,于是可以取一个单同态$\beta:R^n\to R^k$使得$\alpha\beta=1_{R^n}$,取基写作矩阵形式就得到$AB=E_n$.
	\end{proof}
	
	按照这个等价描述,右秩条件和左秩条件要么同时成立要么同时不成立.因此可统一称为秩条件.
	
	环$R$满足秩条件当且仅当对任意正整数$m,n$,从$F_m\cong F_n\oplus K$推出$m\ge n$.
	\begin{proof}
		
		一方面,如果环$R$满足这个新条件,那么对任意满同态$f:R^m\to R^n$,得到正合列$0\to K=\ker f\to R^m\to R^n\to0$,得到$R^m\cong R^n\oplus K$,于是$m\ge n$.
		
		另一方面,如果$R$满足秩条件,那么$R^m\cong R^n\oplus K$提供了一个满同态$f:R^m\to R^n$,于是$m\ge n$.
	\end{proof}
	\item 环$R$满足(右)强秩条件等价于,若$m>n$则任意$n\times m$的矩阵$A$总有非0的满足$Ax=0$的解.
	
	但是一般来讲左强秩条件和右强秩条件是不等价的.即便对于无非零因子的情况也不是等价的.
\end{enumerate}

诺特环满足(右)强秩条件.
\begin{proof}
	
	假设$R$是诺特环,$m>n$并且$R^m=R^n\oplus R^{m-n}$可以嵌入$R^n$,我们断言$R^n$不是诺特模.即只需证明如果$R$模$A,B\not=0$满足$A\oplus B$可以嵌入到$A$中,那么$A$不是诺特模.
	
	事实上这时存在$A$的子模$A_1\oplus B_1$使得$A_1\cong A$,$B_1\cong B$.于是$A\oplus B$又可以嵌入到$A_1$中,于是$A_1$有子模$A_2\oplus B_2$使得$A_2\cong A$,$B_2\cong B$.归纳构造下去,得到$A$的一个子模$B_1\oplus B_2\oplus\cdots$,其中$B_i\cong B\not=0$,于是$A$不是诺特模.
\end{proof}

交换环总满足秩条件与强秩条件.
\begin{proof}
	
	按照之前给出的等价描述,需要证明对$n\times m,m>n$的$R$上矩阵$A$,有$Ax=0$存在平凡解.考虑这个矩阵的$mn$个项在$\mathbb{Z}$上生成的$R$的子环$R_0$,按照Hilbert基定理$R_0$是诺特环.于是按照诺特环上的强秩条件,得到$Ax=0$存在$R_0$中的非平凡解.于是$R$上也存在非平凡解.
\end{proof}
\newpage
\subsection{有限生成和有限表出模}

定义.
\begin{enumerate}
	\item 有限生成模.如果模$M$可以表示为一个有限子集所生成的模,就称$M$是有限生成模.这个条件等价于讲存在如下正合列$A^n\to M\to0$.我们通常把有限生成模称为有限模.
	\item 自由表出.模$M$的一个自由表出是指正合列$G\to F\to M\to0$,其中$G,F$都是自由模.模的自由表出总是存在的,因为任意模都是自由模的商,于是可取自由模$F$使得有正合列$\xymatrix{F\ar[r]^g&M\ar[r]&0}$,但是$\ker g$作为模仍然是某个自由模$G$的商,也即有正合列$\xymatrix{G\ar[r]^f&\ker g\ar[r]&0}$.由此得到一个正合列$\xymatrix{G\ar[r]^f&F\ar[r]^g&M\ar[r]&0}$.
	\item 有限表出模.如果模$M$存在正合列$A^m\to A^n\to M\to0$,就称它是有限表出模.这个条件等价于讲$M$存在这样一个表示,它的生成元集有限,并且关系集也是有限的.
\end{enumerate}

短正合列准则.给定$A$模的短正合列$\xymatrix{0\ar[r]&M_1\ar[r]^f&M_2\ar[r]^g&M_3\ar[r]&0}$,
\begin{enumerate}
	\item 如果$M_2$是有限模,那么$M_3$是有限模.
	\item 如果$M_1$和$M_3$是有限模,那么$M_2$是有限模.
	\item 如果$M_1$和$M_3$是有限表出模,那么$M_2$是有限表出模.
	\item 如果$M_2$是有限表出模,$M_1$是有限模,那么$M_3$是有限表出模.
\end{enumerate}
\begin{proof}
	
	第一条.$M_3$即$M_2$的商模,于是取$M_2$的有限生成元集在$M_3$中的像集,它们生成了$M_3$.
	
	第二条.设$M_1$由$a_1,a_2,\cdots,a_m$生成,设$M_3$由$c_1,c_2,\cdots,c_n$生成,按照$g$是满射,存在$M_2$中的$b_i$使得$g(b_i)=c_i,1\le i\le n$.我们断言$\{f(a_1),f(a_2),\cdots,f(a_m),b_1,b_2,\cdots,b_n\}$这至多$m+n$个元生成.事实上任取$M_2$中元$b$,那么$g(b)\in C$,于是有$g(b)=\sum_{i=1}^nk_ic_i$,于是$b-\sum_{i=1}^nk_ib_i\in\ker g=\mathrm{im}f$,不妨设这个式子等于某个$f(a)$,那么有$a=\sum_{j=1}^m\lambda_ja_j$,于是得到$b=\sum_{j=1}^m\lambda_jf(a_j)+\sum_{i=1}^nk_ib_i$.
	
	第三条.上一条的证明说明存在如下交换图,这里两行都是短正合列,按照蛇形引理,得到短正合列$0\to\ker f\to\ker g\to\ker h\to0$.这个短正合列两端是有限生成模,于是$\ker g$也是有限生成的,也即$M_2$是有限表出模.
	$$\xymatrix{0\ar[r]&A^n\ar[r]\ar[d]_f&A^{m+n}\ar[r]\ar[d]_g&A^m\ar[r]\ar[d]_h&0\\0\ar[r]&M_1\ar[r]&M_2\ar[r]&M_3\ar[r]&0}$$
	
	第四条.$M_2$可视为$A^n$的商模,这里商调的子模$K$是有限生成的,于是$M_3$可视为$A^n$商去$K+M_1$,按照条件这也是有限生成的.
\end{proof}

一些性质.
\begin{enumerate}
	\item 有限模必然是有限表出模.
	\item 设$M$是有限表现模,对任意有限生成模到$M$的满同态,这个同态的核必然是有限生成的.于是定义中的存在一个表示使得生成元集和关系集均为有限的,可以改为任意一个生成元集有限的表示,它的关系集都可以取为有限集.
	\begin{proof}
		
		相当于给定短正合列$0\to K\to N\to M\to0$,其中$M$是有限表现模,$N$是有限生成模,证明$K$是有限生成模.按照$M$有限表现,可取自由模$A^m,A^n$使得有正合列$A^m\to A^n\to M\to0$,按照自由模都是投射模,可构造模同态使得得到如下交换图表:
		$$\xymatrix{
			\ar[r]&A^m\ar[r]\ar[d]^g&A^n\ar[r]\ar[d]^f&M\ar[r]\ar[d]^{1_M}&0\\
			0\ar[r]&K\ar[r]&N\ar[r]&M\ar[r]&0}$$
		
		按照蛇形引理得到$\mathrm{coker}g\cong\mathrm{coker}f$,按照$N$是有限模得到$\mathrm{coker}f$是有限模,于是$\mathrm{coker}g$是有限模,最后考虑短正合列$0\to g(A^m)\to K\to\mathrm{coker}g\to0$,按照$g(A^m)$和$\mathrm{coker}g$都是有限生成的,得到$K$是有限生成的.
	\end{proof}
\end{enumerate}

交换环上有限表示模的张量积是有限表示模.
\begin{proof}
	
	给定交换环$R$上的两个有限表示模$M,N$,那么存在正整数$m,n,s,t$,以及模同态$i,j,p,q$使得存在正合列:
	$$\xymatrix{R^m\ar[r]^i&R^n\ar[r]^j&M\ar[r]&0}$$
	$$\xymatrix{R^r\ar[r]^p&R^s\ar[r]^q&N\ar[r]&0}$$
	
	于是$\alpha\otimes\beta$是$R^n\otimes_RR^s\to M\otimes_RN$的模同态.现在取$R$模$S=(R^m\otimes_RR^s)\oplus(R^n\otimes_RR^r)$,取模同态$\varphi:S\to R^n\otimes_RR^s$为$(a\otimes b,c\otimes d)\mapsto i(a)\otimes b+c\otimes p(d)$,我们断言有$\mathrm{im}\varphi=\ker(j\otimes q)$,这就证明了$M\otimes_RN$是有限表示模.
	
	左侧包含于右侧是直接的,因为$(j\otimes q)\circ\varphi(a\otimes b+c\otimes d)=(j\otimes q)(i(a)\otimes b+c\otimes p(d))=0$.下面证明右侧包含于左侧.注意到$\mathrm{im}\varphi=\mathrm{im}i\otimes 1_{R^s}+\mathrm{im}1_{R^n}\otimes p$.任取$x\in\ker(j\otimes q)$,需要验证的是存在$R^m\otimes_R R^s$中的元$y$,使得$x-(i\otimes 1_{R^s})(y)\in\mathrm{im}(1_{R^n}\otimes p)$.张量函子的右正合性得到$\mathrm{im}(1_{R^n}\otimes p)=\ker(1_{R^n}\otimes q)$.于是需要验证的是$(1_{R^n}\otimes q)(x)=(1_{R^n}\otimes q)(i\otimes 1_{R^s})(y)=(i\otimes q)(y)$.
	
	按照$x$满足$(j\otimes q)x=0$,得到$(j\otimes 1_N)(1_{R^n}\otimes q)x=0$,于是有$(1_{R^n}\otimes q)x\in\ker(j\otimes 1_N)$.从张量函子的右正合性得到$\ker(j\otimes 1_N)=\mathrm{im}(i\otimes 1_N)$.再按照$q$是满射,得到$(1_{R^m}\otimes q)$是满射,于是就存在$R^m\otimes_RR^s$中的$y$满足$(i\otimes1_M)(1_{R^n}\otimes q)(y)=(1_{R^n}\otimes q)(x)$,得证.	
\end{proof}

Schanuel引理.
\begin{enumerate}
	\item 投射情况.给定两个短正合列$0\to K\to P\to M\to0$和$0\to K'\to P'\to M\to0$,其中$P$和$P'$都是投射模,那么有同构$K\oplus P'\cong K'\oplus P$.
	\begin{proof}
		
		按照$P$是投射模,于是存在$\beta:P\to P'$提升了$P\to M$,于是按照短五引理得到存在唯一的$\alpha:K\to K'$使得如下图表交换:
		$$\xymatrix{0\ar[r]&K\ar[r]^i\ar[d]^{\alpha}&P\ar[r]^{\pi}\ar[d]^{\beta}&M\ar[r]\ar[d]^{1_M}&0\\0\ar[r]&K'\ar[r]_{i'}&P'\ar[r]_{\pi'}&M\ar[r]&0}$$
		
		现在构造$\xymatrix{0\ar[r]&K\ar[r]^{\theta}&P\oplus K'\ar[r]^{\psi}&P'\ar[r]&0}$,其中$\theta:x\mapsto(ix,\alpha x)$,$\psi:(u,x')\mapsto\beta u-i'x'$.验证$\theta$单射是直接的,下面验证$\psi$是满射,任取$y'\in P'$,那么按照$\pi$是满射,存在$y\in P$满足$\pi(y)=\pi'(y')$,另外图表交换得到$\pi(y)=\pi'(\beta(y))$,于是$y'-\beta(y)\in\ker\pi'=\mathrm{im}i'$.这就得到$y'=(y'-\beta(y))+\beta(y)\in\mathrm{im}\psi$.
		
		最后验证$\mathrm{im}\theta=\ker\psi$.一方面$\psi(\theta(x))=\beta(ix)-i'(\alpha x)=0$说明$\mathrm{im}\theta\subset\ker\psi$.另一方面任取$(y,x')\in\ker\psi$,即$\beta y=i'x'$,于是$0=\pi'(\beta(y))=\pi(y)$,于是$y\in\ker\pi=\mathrm{im}i$,于是存在$x\in K$使得$ix=y$,于是有$i'(\alpha(x))=i'(x')$,按照$i'$是单射得到$\alpha(x)=x'$,即$(y,x')=\theta(x)$,也即$\ker\psi\subset\mathrm{im}\theta$.
	\end{proof}
	\item 这个引理允许我们给出有限表现模一个性质的另一证明,即给定有限表现模$M$,任取有限生成自由模$F$,取满同态$F\to M$,那么它的核$K$总是有限生成的.事实上按照$M$有限表现,可取有限生成自由模$F'$和满同态$F'\to M$使得核$K'$是有限生成模,于是按照Schanuel引理,存在同构$K\oplus F'\cong K'\oplus F$.现在$K'\oplus F$是一个有限生成模,而$K$是它的商,于是也是有限生成的.
\end{enumerate}

\newpage
\subsection{PID上有限生成模结构定理}

存在很多种方式能让一个模不成为自由模,其中最自然的一种方式是存在挠性.称环上模中一个元是挠元,如果它自身是一个线性相关集.这等价于说存在一个环中非0元和它相乘为0,那么模中的0元自然是挠元,称为平凡的挠元.记环$R$的模$M$中全体挠元构成的集合为$\mathrm{Tor}_R(M)$.称模$M$是无挠模,如果$\mathrm{Tor}_{R}(M)=\{0\}$,称模$M$是挠模如果$\mathrm{Tor}_R(M)=M$.在一般环上集合$\mathrm{Tor}_R(M)$未必是$M$的子模,例如环$\mathbb{Z}/6$上挠元不构成理想,但是当环是整环的时候它的确是一个子模.无挠模的子模或者直和均为无挠模.

整环上的自由模是无挠的.反过来整环上的无挠模未必是自由模.这里给出两个例子,一个例子是$\mathbb{Q}$作为$\mathbb{Z}$上的模是无挠模但不是自由模,不是自由模是因为$\mathbb{Q}$上任意多于一个元的集合都是线性相关的,而$\mathbb{Q}$又不会是$\mathbb{Z}$上的一维自由模(比方说前者是可除模,而一维自由模不是).

第二个例子.整环上一个非主的理想作为子模是无挠的但不是自由的,由于交换环满足强秩条件,即自由模的子自由模的秩不会变大,说明倘若这个理想作为子模是自由的,那么它的秩不会超过1,这只能是一个主理想.藏在这个例子背后的事实是PID的从模角度的描述:一个整环是PID当且仅当它的一维自由模(作为自身的模)的每个子模都是自由的.

循环模的无挠性可以完全刻画域:给定整环$R$,如果每个循环$R$模都是无挠模,那么$R$是域.
\begin{proof}
	
	如果整环$R$满足这个性质,任取$R$中非0元$c$,那么$M=R/(c)$是一个循环模.注意到对任意的$M$中的元$r+(c)$,总有$c(r+(c))=(c)$,于是有$M=\mathrm{Tor}(M)$.但是,按照条件有$\mathrm{Tor}(M)=\{0\}$.于是$M=0$,就得到$R$中有$(c)=(1)$,于是$c$是单位.于是$R$是域.
\end{proof}

我们已经看到整环上自由模的子模未必是自由的,并且域上自由模的子模总是自由的,因为域上的模总是自由的,于是有理由相信存在一种介于整环和域之间的条件,在这个条件下自由模的子模是自由的.事实上这个条件就是PID:

PID上自由模的子模结构定理.给定$R$是PID,给定$F$是$R$上有限生成自由模,取$F$的子模$M$,那么$M$是自由模,并且存在$F$的一组基$(x_1,x_2,\cdots,x_n)$,和一组非0的元$a_1,a_2,\cdots,a_m\in R$,其中$m\le n$,使得$(a_1x_1,a_2x_2,\cdots,a_mx_m)$是$M$的一组基,并且有$a_1\mid a_2\mid\cdots\mid a_m$.
\begin{proof}
	
	先来证明如下引理:设$R$是PID,设$F$是$R$上有限生成自由模,取$F$的非0子模$M$,那么存在$a\in R$,$x\in F$,$y\in M$,和子模$F'\subset F$和$M'\subset M$,满足$y=ax$,$M'=F'\cap M$,并且$F=(x)\oplus F'$和$M=(y)\oplus M'$.
	
	对每个$F\to R$的模同态$\varphi$,$\varphi(M)$是$R$的理想,按照$R$是PID所以是诺特的,于是可以在全体这种真理想中选取一个极大元,记作$\alpha(M)$,另外按照$M\not=0$得到$\alpha(M)\not=0$.现在把$\alpha(M)$记作$(a)$,其中$a$是$R$中非0元,既然$a\in\alpha(M)$,就存在一个$y\in M$使得$\alpha(y)=a$.
	
	我们断言对每个$F\to R$的同态$\varphi$,都有$a$整除$\varphi(y)$.事实上取$(a,\varphi(y))=(b)$,于是存在$R$中的元$r,s$满足$b=ra+s\varphi(y)$.现在取$\psi=r\alpha+s\varphi$,那么一方面$\alpha(M)=(a)\subset\psi(M)$,另一方面$b=ra+s\varphi(y)=(r\alpha+s\varphi)(y)=\psi(y)\subset\psi(M)$,于是$\alpha(M)\subset\psi(M)$,按照$\alpha$选取方式是极大的,于是$\alpha(M)=\psi(M)$,于是$(a)=(b)$,导致$a$整除$\varphi(y)$.
	
	现在取定$F$的一组基,此时有$F\simeq R^n$,记$y$在这组基下的表示是$(s_1,s_2,\cdots,s_n)$,于是每个$s_i$是$y$在投射同态下的像,于是按照刚刚所证的,就有每个$s_i$被$a$整除,于是有$R$中的$r_1,r_2,\cdots,r_n$使得$s_i=ar_i$,记$x=(r_1,r_2,\cdots,r_n)\in F$.于是$y=ax$.现在按照$a=\alpha(y)=\alpha(ax)=a\alpha(x)$,按照整环条件,得到$\alpha(x)=1$.
	
	最后,取$F'=\ker\alpha$,取$M'=F'\cap M$,只要验证最后两个直和就行了.第一个,对每个$z\in F$,有$z=\alpha(z)x+(z-\alpha(z)x)$,其中$z-\alpha(z)x\in\ker\alpha=F'$,于是$F=(x)+F'$,现在取$rx\in F'$,那么$\alpha(rx)=0$,得到$r=0$,于是这个和是直和.第二个,对$z\in M$,那么$a$整除$\alpha(z)$,因为$\alpha(z)\in\alpha(M)=(a)$.于是有$\alpha(z)=ca$,那么$\alpha(z)x=cax=cy$,把$z$按照上述分解,得到$z-\alpha(z)x=z-cy\in M\cap F'=M'$,于是得到直和$M=(y)\oplus M'$.
	
	取$F$的子模$M$,不妨设$M$非0,那么按照引理存在$y_1\in M$和子模$M_1\subset M$,满足$M=(y_1)\oplus M_1$,如果$M_1=0$就结束,若否可以继续按照引理找到$M_1$中的元$y_2$,满足存在子模$M_2\subset M_1$满足$M=(y_1)\oplus (y_2)\oplus M_2$.但是这个步骤不能超过$F$的秩$n$,否则可以找到线性无关基$\{y_1,y_2,\cdots\}$的个数超过了秩.于是得一个正整数$m\le n$,和若干个$y_i$满足$M=(y_1)\oplus (y_2)\oplus\cdots\oplus (y_m)$.
	
	最后证明整除性质.事实上,按照$(a_1)$是全体$\varphi(M)$在$\varphi$跑遍$F\to R$同态的极大元.取$F\to R$的同态$\varphi$满足$\varphi(x_1)=\varphi(x_2)=1$,那么$\varphi(y_1)=a_1$,得到$(a_1)\subset\varphi(M)$,按照极大性得到$\varphi(M)=(a_1)$,于是$a_2=\varphi(y_2)\in\varphi(M)=(a_1)$,就得到整除性.
\end{proof}

PID上有限生成无挠模$A$是自由模.这里有限生成是必须的,因为$\mathbb{Q}$是无挠$\mathbb{Z}$模,但它不是自由模,因为任意元素个数大于1的子集是线性相关的.
\begin{proof}
	
	不妨设$A\not=0$,取$A$的生成元集$X$,按照条件$X$可以取有限集.取$X$包含的极大的线性无关组为$S$,按照没有非平凡挠元知这是可以取到的.那么$S$生成的$A$的子模$F$自然是一个自由模.现在任取$y\in X-S$,按照线性无关组$S$的极大性,存在不全为0的系数满足$r_yy+r_1x_1+\cdots+r_kx_k=0$,其中$x_i\in S$.于是$r_yy=-\sum_{i=1}^{k}r_ix_i\in F$.注意到$r_y$非0,否则$S$线性相关.现在按照$X$是有限集,可以取$r=\prod_{y\in X-S}r_y$非0.于是$rX=\{rx\mid x\in X\}$包含在$F$中,于是$rA\subset F$.现在做同态$f:A\to A$为$a\mapsto ra$,那么像空间是$rA$,按照$A$无挠,得到$\ker f=0$,于是$A\cong\mathrm{im}f=rA\subset F$,于是从有限生成自由模的子模自由得证.
\end{proof}

PID上有限生成模的结构定理.设$R$是一个PID,取其上一个有限生成模$M$,记rank$M$表示$M$中极大的线性无关组的长度,那么有如下结构定理.其中第一个结构定理中的全体$(a_i^{r_{ij}})$称为初等因子,第二个结构定理中的全体$(a_i)$称为不变因子.分解中的初等因子组和不变因子组在每一项相差某个单位的意义下是唯一决定的.
\begin{enumerate}
	\item 存在不同的素理想$(q_1),(q_2),\cdots,(q_n)\in R$,正整数列$r_ {ij}$, 使得:
	$$M\cong R^{\mathrm{rank} M}\oplus\left(\oplus_{i,j}\frac{R}{(q_i^{r_{ij}})}\right)$$
	\item 存在非单位的理想$(a_m)\subset (a_{m-1})\subset\cdots\subset(a_1)$, 使得:
	$$M\cong R^{\mathrm{rank} M}\oplus\left(\oplus_{i}\frac{R}{(a_i)}\right)$$
\end{enumerate}
\begin{proof}
	
	给定PID上一个有限生成模$M$,那么存在一个正整数$n$,使得存在满同态$f:R^n\to M$,于是$M\cong R^n/\ker f$.按照$\ker f$是有限生成自由模$R^n$的子模,于是存在正整数$m\le n$,和$R$中依次整除的一组元$a_1\mid a_2\mid\cdots\mid a_m$,使得$\ker f=(a_1)\oplus (a_2)\oplus\cdots\oplus (a_m)$,于是得到有限生成自由模$M$同构于:
	$$R^{n-m}\oplus\frac{R}{(a_1)}\oplus\frac{R}{(a_2)}\oplus\cdots\oplus\frac{R}{(a_m)}$$
	
	如果把每个$a_i$做唯一分解$a_i=\prod_{j=1}^{s_i}p_{j}^{r_{ij}}$.按照中国剩余定理,得到每个$R/(a_i)$同构于$R/(p_1^{s_{i1}})\oplus R/(p_1^{s_{i2}})\oplus\cdots\oplus R/(p_1^{r_{i,s_i}})$.于是得到有限生成模同构于:
	$$R^{n-m}\oplus\left(\oplus_{1\le i\le m}\oplus_{1\le j\le s_i}\frac{R}{(p_i^{r_{ij}})}\right)$$
	
	至此完成了结构定理的存在性证明.下面证明唯一性.首先对一个模$M$,我们知道它可以分解为自身挠子模和一个无挠模的直和,即有$M=\mathrm{Tor}M\oplus M/\mathrm{Tor}M$.于是只要分别证明两部分的分解是唯一的.
	
	对于无挠部分,我们已经证明了PID上有限生成无挠模是自由模,于是这一部分是自由模,它的秩就是原模$M$的极大线性无关组的元素个数.于是这是唯一决定的.
	
	接下来讨论挠子模部分.为此,先说明不变因子组可以推出初等因子组.把初等因子组这样排列:把同一个不可约元的次幂排在一列上,要求次数自上而下是从大到小排列的,对每个出现在初等因子中的不可约元这样排一列,让全部最高次数项排成一行,把下方不够的项拿0次幂补充,这得到一个数表,把第$i$行相乘,就得到第$i$个不变因子.于是,只要说明第二种直和分解是唯一的,即初等因子组是唯一决定的,就得到不变因子组是唯一决定的.
	
	现在假设$M$的挠子模部分存在两种分解:$M'=\mathrm{Tor}M=\oplus_{i,j}\frac{R}{(p_j^{r_{ij}})}=\oplus_{ij}\frac{R}{((p_j')^{r_{ij}})}$.我们固定一个不可约元$p$,记$M[p^n]$表示的是被$p^n$零化的$M$中的元构成的子模.那么$M[p^n]/M[p^{n-1}]$是一个$R/(p)$线性空间,它的维数,恰好就是底数是这个不可约元$p$,次数不小于$n$的初等因子的个数.让$p$取遍在不变因子中出现过的不可约元,让$n$取遍正整数,于是两种分解是相同的.这就完成了唯一性.
\end{proof}

本节最后我们解释下这个结构定理具体是对哪类子模的分解.称一个子模是循环子模,如果这个子模可以被单个元生成.称一个模$M$是不可分解子模,如从模同构$M\simeq N\oplus N'$可以推出$N$或$N'$中存在一个零模.称一个模是单模,如果它不存在非平凡的子模.那么这三种模的关系是,单模必然是不可分解模,必然是循环模,反过来都不成立,并且循环模和不可分解模之间没有关系.在线性代数的空间理论中我们会看到这些关系的全部例子,除了一个,不可分解模未必是循环模,这只要注意到$R=K[x,y]$的理想$(x,y)$作为$R$的模是不可分解模,但不是循环模.本节的第一个结构定理,即初等因子对应的结构定理,实际上是把空间分解为不可分解模的直和.这也是说PID上有限生成模是完全分解模.注意到如果$r>1$,那么$R/(q^r)$,其中$(q)$是素理想,只能是不可分解模,它是单模只有$r=1$的情况才能保证.我们在后文中会看到能分解为单模直和的模称为半单模,运用在线性代数的空间理论中,可对角化矩阵对应于单模的直和,因此我们会把可对角化矩阵称为半单矩阵.
\newpage
\subsection{Hom函子和张量函子}

Hom函子.我们介绍过范畴$\mathscr{C}$上的Hom函子是$\mathscr{C}^{op}\times\mathscr{C}\to\textbf{Set}$的函子,即把对象对$(A,B)$映射为集合$\mathrm{Hom}(A,B)$,把态射$(\varphi,\psi),\varphi:A'\to A$,$\psi:B\to B'$映射为态射$\mathrm{Hom}(\varphi,\psi):\mathrm{Hom}(A,B)\to\mathrm{Hom}(A',B')$,即$(f:A\to B)\mapsto(\psi\circ f\circ\varphi:A'\to B')$.另外通常也会把Hom函子分拆为两个函子,即$\mathrm{Hom}(M,-)$是共变函子,$\mathrm{Hom}(-,N)$是逆变函子.

在模范畴中,Hom集合具备交换群结构,即$\mathrm{Hom}_R(M,N)$中的零元是把$M$中所有元映射为$0_N$的模同态,模同态$f:M\to N$的逆元为$-f:m\mapsto -f(m)$.此时两个函子$\mathrm{Hom}_R(M,-)$和$\mathrm{Hom}_R(-,N)$分别是左$R$模范畴到$\textbf{Ab}$的共变和逆变函子.
\begin{proof}
	
	先考虑$\mathrm{Hom}_R(M,-)$,它把左$R$模$N$映射为交换群$\mathrm{Hom}_R(M,N)$.现在任取交换群同态$f:N\to N'$,那么$\mathrm{Hom}_R(1_M,f)$是$\mathrm{Hom}_R(M,N)\to\mathrm{Hom}_R(M,N')$的映射,即把模同态$\varphi:M\to N$映射为模同态$f\circ\varphi\in\mathrm{Hom}_R(M,N')$.为说明它是共变函子需要验证两件事:$\mathrm{Hom}_R(1_M,1_N)$是$\mathrm{Hom}_R(M,N)$上的恒等映射;任取模同态$f:N\to N'$和$f':N'\to N''$,那么$\mathrm{Hom}_R(1_M,f'\circ f)=\mathrm{Hom}_R(1_M,f')\circ\mathrm{Hom}_R(1_M,f)$.对于前者,对每个模同态$\varphi:M\to N$,有$\mathrm{Hom}_R(1_M,1_N)(\varphi)=1_N\circ\varphi\circ1_M=\varphi$是恒等.对于后者,任取模同态$\varphi:M\to N$,那么$\mathrm{Hom}_R(1_M,f'\circ f)(\varphi)=f'\circ f\circ\varphi=f'\circ\mathrm{Hom}_R(1_M,f)(\varphi)=\mathrm{Hom}_R(1_M,f')\circ\mathrm{Hom}(1_M,f)(\varphi)$.于是$\mathrm{Hom}_R(M,-)$是共变的函子.
	
	再考虑函子$\mathrm{Hom}_R(-,N)$,它把左$R$模$M$映射为交换群$\mathrm{Hom}_R(M,N)$.现在任取交换群同态$f:M\to M'$,那么$\mathrm{Hom}_R(f,1_N)$是$\mathrm{Hom}_R(M',N)\to\mathrm{Hom}_R(M,N)$的映射,即把模同态$\varphi:M'\to N$映射为模同态$\varphi\circ f\in\mathrm{Hom}_R(M,N)$.同样为说明它是逆变函子需要验证两件事:$\mathrm{Hom}_R(1_M,1_N)$是$\mathrm{Hom}_R(M,N)$上的恒等映射;任取模同态$f:M'\to M$和模同态$f':M''\to M'$,那么$\mathrm{Hom}_R(f\circ f',1_N)=\mathrm{Hom}_R(f',1_N)\circ\mathrm{Hom}_R(f,1_N)$.对于前者,任取模同态$\varphi:M\to N$,那么$\mathrm{Hom}_R(1_M,1_N)(\varphi)=1_N\circ\varphi\circ1_M=\varphi$.对于后者,任取模同态$\varphi:M\to N$,那么有$\mathrm{f\circ f',1_N}(\varphi)=\varphi\circ f\circ f'=\mathrm{Hom}_R(f,1_N)(\varphi)\circ f'=\mathrm{Hom}_R(f',1_N)\circ\mathrm{Hom}_R(f,1_N)(\varphi)$.这就说明$\mathrm{Hom}_R(-,N)$是逆变的函子.
\end{proof}

加性函子.给定模范畴之间的共变或者逆变函子$F$,称它是加性函子,如果诱导的Hom集之间的映射是交换群同态,具体的讲,对于共变的$F$,它是加性函子当且仅当对每对对象$A,B$,诱导的映射$\mathrm{Hom}(A,B)\to\mathrm{Hom}(FA,FB)$,$f\mapsto Ff$是交换群同态,即$F(f+g)=Ff+Fg$.对于逆变的$F$,它是加性函子当且仅当对每对对象$A,B$,诱导的映射$\mathrm{Hom}(A,B)\to\mathrm{Hom}(FB,FA)$是交换群同态,即$F(f+g)=Ff+Fg$.那么模范畴上的两个Hom函子都是加性函子,注意到$\textbf{Ab}$实际上就是$\mathbb{Z}$模,因而它的Hom集合同样是交换群.为说明Hom函子是加性的,只要注意到任取$\varphi_1,\varphi_2\in\mathrm{Hom}_R(M,N)$,那么在映射复合有意义的前提下总有$f\circ(\varphi_1+\varphi_2)=f\circ\varphi_1+f\circ\varphi_2$和$(\varphi_1+\varphi_2)\circ g=\varphi_1\circ g+\varphi_2\circ g$.

有时Hom集合会具备某些模结构:
\begin{enumerate}
	\item 如果$M$是左$R$模,$N$是$(R,S)$模,那么$\mathrm{Hom}_R(M,N)$是右$S$模.即约定模同态$f:M\to N,s\in S$有$(fs)(m)=f(m)s$.
	\item 如果$M$是右$S$模,$N$是$(R,S)$模,那么$\mathrm{Hom}_S(M,N)$是左$R$模.即约定模同态$f:M\to N,r\in R$有$(rf)(m)=r(f(m))$.
	\item 如果$M$是$(R,S)$模,$N$是左$R$模,那么$\mathrm{Hom}_R(M,N)$是左$S$模.即约定模同态$f:M\to N,s\in S$有$(sf)(m)=f(ms)$.
	\item 如果$M$是$(R,S)$模,$N$是右$S$模,那么$\mathrm{Hom}_S(M,N)$是右$R$模.即约定模同态$f:M\to N,r\in R$有$(fr)(m)=f(rm)$.
	\item 最重要的情况,如果$R$是交换环,那么任意$R$模$M,N$有$\mathrm{Hom}_R(M,N)$是$R$模.此时两个Hom函子将会是$R$模范畴自身上的共变和逆变函子.
\end{enumerate}

关于模范畴之间的加性函子.
\begin{enumerate}
	\item 加性函子保零态射.这是因为加性函子保Hom集合的零元.这里零元恰好就算零态射.
	\item 加性函子保零对象.模$M$是零模当且仅当态射$1_M$是零态射:必要性是直接的,充分性是因为对每个$m\in M$有$0=(0_M)m=(1_M)m=m$.于是按照加性函子保零态射,就得到$1_{T(M)}=T(1_M)=T(0_M)=0_{T(M)}$,于是$T$保零对象.
\end{enumerate}

Hom函子的一些基本性质.
\begin{enumerate}
	\item 把$R$看作$(R,R)$模,那么$\mathrm{Hom}(R,A)$是左$R$模,此时它模同构于左$R$模$A$.事实上对每个$a\in A$,设它对应的模同态为$\varphi_a:r\mapsto ra$.那么$\varphi_{a+b}=\varphi_a+\varphi_b$和$\varphi_{ra}=r\varphi_a$说明这是$R$模同构.另外这个同构甚至是自然的.
	\item 直积直和与Hom函子满足如下同构关系,如果$R$未必交换那么下述是交换群同构,如果$R$交换那么下述是$R$模同构,并且两种情况下同构是自然的:
	$$\mathrm{Hom}_R(A,\prod_{i\in I}B_i)\cong\prod_{i\in I}\mathrm{Hom}_R(A,B_i)$$
	$$\mathrm{Hom}_R(\oplus_{i\in I}A_i,B)\cong\prod_{i\in I}\mathrm{Hom}_R(A_i,B)$$
	\begin{proof}
		
		给定$f:A\to\prod_{i\in I}B_i$,记$A\to\prod_{i\in I}B_i\to B_i$为$f_i$,那么取$(f_i)\in\prod_{i\in I}\mathrm{Hom}_R(A,B_i)$.考虑$f\mapsto (f_i)$.这是交换群同态.另外容易验证单射和满射.同理证明第二个.
	\end{proof}
    \item 特别的,模范畴上有限直和与有限直积同构的,此时就有:
    $$\mathrm{Hom}_R(M,N\oplus N')\cong\mathrm{Hom}_R(M,N)\oplus\mathrm{Hom}_R(M,N')$$
    $$\mathrm{Hom}_R(M\oplus M',N)\cong\mathrm{Hom}_R(M,N)\oplus\mathrm{Hom}_R(M',N)$$
    \item 其他几种直和直积和Hom函子不可交换的反例.
    \begin{enumerate}
    	\item 第一个反例是$\mathrm{Hom}_{\mathbb{Z}}(\prod_{n\ge1}\mathbb{Z},\mathbb{Z})\not\cong\prod_{n\ge1}\mathrm{Hom}_{\mathbb{Z}}(\mathbb{Z},\mathbb{Z})$.
    	
    	这可以从如下两个事实得出,第一个是同构式$\mathrm{Hom}_{\mathbb{Z}}(\prod_{n\ge1},\mathbb{Z})\cong\oplus_{n\ge1}\mathrm{Hom}_{\mathbb{Z}}(\mathbb{Z},\mathbb{Z})\cong\oplus_{n\ge1}\mathbb{Z}$.第二个是可数直积$\prod_{n\ge1}\mathbb{Z}$与可数直和$\oplus_{n\ge1}\mathbb{Z}$作为交换群不同构.
    	
    	首先给出第一件事的另一种等价描述,尽管我们证明不从此入手.$\prod_{n\ge1}\mathbb{Z}$视为交换群可以等同于形式幂级数环$\mathbb{Z}[[x]]$作为交换群,而直和$\oplus_{n\ge1}\mathbb{Z}$视为交换群等同于多项式环$\mathbb{Z}[x]$作为交换群.所以问题等价于$\mathrm{Hom}_{\mathbb{Z}}(\mathbb{Z}[[x]],\mathbb{Z})\cong\mathbb{Z}[x]$.另外第二条可说明$\mathrm{Hom}_{\mathbb{Z}}(\mathbb{Z}[x],\mathbb{Z})\cong\mathbb{Z}[[x]]$,于是多项式环和形式幂级数环可以存在某种对偶关系.
    	
    	第二件事只需注意到可数直和是一个可数集,而可数直积是一个不可数集,这必然不能是同构的.
    	\item 设$A=\oplus_{n\ge2}\mathbb{Z}/n$,那么有$\mathrm{Hom}_{\mathbb{Z}}(A,\oplus_{n\ge2}\mathbb{Z}/n)\not\cong\oplus_{n\ge2}\mathrm{Hom}_{\mathbb{Z}}(A,\mathbb{Z}/n)$.事实上右侧直和中每个元都具有有限阶数,但是$\mathrm{Hom(A,A)}$中考虑每个分量取恒等的同态,它的阶数无穷,这就说明二者不同构.
    	\item $\mathrm{Hom}_{\mathbb{Z}}(\prod_{n\ge2}\mathbb{Z}/n,\mathbb{Q})\not\cong\prod_{n\ge2}\mathrm{Hom}_{\mathbb{Z}}(\mathbb{Z}/n,\mathbb{Q})$.注意后者实际上就是$\{0\}$,于是只需说明前者存在非平凡元.注意$\prod_{n\ge2}\mathbb{Z}/n$存在无挠元$x=(1,1,\cdots)$,于是存在非平凡同态$\langle x\rangle\to\mathbb{Q}$例如$x\mapsto1$,按照$\mathbb{Q}$是内射交换群,这个同态可以提升为$\prod_{n\ge2}\mathbb{Z}/n\to\mathbb{Q}$,这就得到一个非平凡元.
    \end{enumerate}
\end{enumerate}

设$R$是交换环,$I\subset R$是理想,$M$是$R$模,则有同构$\mathrm{Hom}_R(R/I,M)\cong M[I]$,这里$M[I]=\{m\in M\mid \forall r\in I,rm=0\}$是$M$的子模.特别的这一事实说明$\mathrm{Hom}_{\mathbb{Z}}(\mathbb{Z}/n,\mathbb{Q})=\{0\}$,更说明对于有限交换群$G$有$\mathrm{Hom}_{\mathbb{Z}}(G,\mathbb{Q})=\{0\}$.
\begin{proof}
	
	考虑正合列$I\to R\to R/I\to0$,按照逆变函子$\mathrm{Hom}_R(-,M)$的左正合性,得到$0\to\mathrm{Hom}_R(R/I,M)\to\mathrm{Hom}_R(R,M)\to\mathrm{Hom}_R(I,M)$.其中典范映射$\mathrm{Hom}_R(R,M)\cong M$即把$f:R\to M$对应于$f(1)$.映射$\mathrm{Hom}_R(R,M)\to\mathrm{Hom}_R(I,M)$就是把映射$f:R\to M$限制到$I$上.它的核就是原本在$I$上恒取零的$R\to I$的映射,于是它在典范映射下的像就是对每个$r\in I$有$rm=0$的$m$构成的子集.按照正合性就得到$\mathrm{Hom}_R(R/I,M)\cong M[I]$.
\end{proof}

张量积.先来处理非交换环上模的张量积.给定环$R$,给定右$R$模$A$和左$R$模$B$,取一个交换群$G$,称$f:A\times B\to G$是双线性映射,如果指满足如下条件:
$$f(a+a',b)=f(a,b)+f(a',b)$$
$$f(a,b+b')=f(a,b)+f(a,b')$$
$$f(ar,b)=f(a,rb),r\in R$$

把右模$A$和左模$B$的张量积$A\otimes_RB$定义为如下泛映射对象:交换群$A\otimes_RB$以及一个$R$双线性映射:$h:A\times B\to A\otimes_RB$,满足对任意的交换群$G$和双线性映射$f:A\times B\to G$,存在唯一的交换群同态$f':A\otimes_RB\to G$提升了$f$,即满足:
$$\xymatrix{
	A\otimes_RB\ar[rr]^{\exists!f'}&&G\\
	A\times B\ar[u]^{h}\ar[urr]_{f}&&
}$$

按照泛映射定义,张量积存在则在同构意义下唯一.下面构造性的证明张量积的存在性.取以$A\times B$为基的自由交换群$F$,取$S$为$F$上全体如下三种形式的元素生成的子群:$$(a,b+b')-(a,b)-(a,b')$$
$$(a+a',b)-(a,b)-(a',b)$$
$$(ar,b)-(a,rb)$$

考虑交换群$F/S$,定义$h:A\times B\to A\otimes_RB$为$(a,b)\mapsto a\otimes b$.那么这是一个双线性映射.现在验证$(F/S,h)$满足张量积的泛映射性质.任取交换群$G$和双线性映射$f:A\times B\to G$.首先$h$可以分解为两个典范映射的复合$A\times B\to F\to F/S$.前者记作$i$,后者记作$j$,那么按照自由交换群的泛映射性质,存在唯一的$\varphi:F\to G$使得$\varphi\circ i=f$.按照$f$是双线性映射,说明$S\subset\ker\varphi$,于是按照商的泛映射性质,存在唯一的$f':F/S\to G$满足$f'\circ j=\varphi$.最后$f'$被$(a,b)$上的像决定,而这个像被$f$决定,于是提升$f'$自然是唯一的.这就说明了$(F/S,h)$是张量积,此时把$F/S$记作$A\otimes_RB$,把陪集$(a,b)+S$记作$a\otimes b$.
$$\xymatrix{F/S\ar[r]^{f'}&G\\
F\ar[u]^{j}\ar[ur]^{\varphi}\\
A\times B\ar[u]^{i}\ar[uur]_f}$$

张量积的函子性.张量积是从$\textbf{mod-R}\times\textbf{R-mod}$到\textbf{Ab}的函子.它把对象$(A,B)$映射为$A\otimes_RB$,把态射$(f,g)$映射为$f\otimes g$,即任一个$A,A'$是右$R$模,$B,B'$是左$R$模,$f:A\to A'$和$g:B\to B'$是两个模同态,那么定义交换群同态$f\otimes g:A\otimes_RB\to A'\otimes_RB'$为$a\otimes b\mapsto f(a)\otimes g(b)$.为了验证这的确诱导了模同态,先注意到$(f,g):A\times B\to A'\times B'$,$(a,b)\mapsto (f(a),g(b))$诱导了两个自由群之间的同态$F_1\to F_2$,其中$F_1$是$A\times B$上的自由交换群,$F_2$是$A'\times B'$上的自由交换群.如果记两个张量积定义中的关系集分别为$S_1,S_2$,那么有$(f,g)(S_1)\subset S_2$,这就导致了$(f,g)$诱导了群同态$f\times g:F_1/S_1\to F_2/S_2$.

另外这个定义下,两个这种同态复合为:$(f'\otimes g')(f\otimes g)=f'f\otimes g'g$.特别的,如果有模同构$A_1\cong A_2$和$B_1\cong B_2$,那么有交换群同构$A_1\otimes_RB_1\cong A_2\otimes_RB_2$.

如果固定一个右$R$模$A$,可定义从左$R$模范畴到交换群范畴的一个共变函子$F_A$,它把左$R$模$B$映射为交换群$A\otimes_RB$,把左模同态$g:B\to B'$映射为交换群同态$1_A\otimes g$.类似的取定一个左$R$模$B$,可定义从右$R$模范畴到交换群范畴的共变函子$F_B$.这两种函子分别记作$A\otimes_R-$和$-\otimes_RB$.这两种函子都是加性的.

如果模具备更多的结构,会使得张量积具备更多的结构.
\begin{enumerate}
	\item 如果$M$是右$R$模,$N$是$(R,S)$模,那么$M\otimes_RN$是右$S$模.即定义$(m\otimes n)s=m\otimes ns$.如果取$(R,S)$双侧模$N$,那么函子$-\otimes_SN$是右$R$模范畴到右$S$模范畴的加性函子.
	\item 如果$M$是$(R,S)$模,$N$是左$S$模,那么$M\otimes_RN$是左$R$模.即定义$r(m\otimes n)=rm\otimes n$.如果取$(R,S)$双侧模$M$,那么函子$M\otimes_R-$是左$S$模范畴到左$R$模范畴的加性函子.
\end{enumerate}

按照泛映射性质可得到张量积的若干性质:
\begin{enumerate}
	\item 恒等律.把$R$视为$(R,R)$模,设$N$是左$R$模,那么$R\otimes_RN$是左$R$模,并且自然模同构于$N$;如果$M$是右$R$模,那么$M\otimes_RR$是右$R$模,并且自然模同构于$M$.这两个自然同构取为$r\otimes n\mapsto rn$和$m\otimes r\mapsto mr$.
	\item 反交换律.对右模$M$和左模$N$,存在自然的交换群同构$\tau:M\otimes_RN\to N\otimes_{R^{op}}M$,$m\otimes n\mapsto n\otimes m$.这里自然是指如下交换图成立:
	$$\xymatrix{
		M\otimes_RN\ar[r]^{\tau}\ar[d]_{f\otimes g}&N\otimes_{R^{op}}M\ar[d]^{g\otimes f}\\
		M'\otimes_RN'\ar[r]_{\tau}&N'\otimes_{R^{op}}M'
	}$$
	\item 结合律.给定环$R,S$,给定右$R$模$A$,左$S$模$C$和一个$(R,S)$双侧模$B$,那么存在交换群同构:
	$$A\otimes_R(B\otimes_SC)\cong(A\otimes_RB)\otimes_SC$$
	\begin{proof}
		
		核心思路是验证上述等式两侧均是一个泛映射问题的解,于是唯一性得到二者同构.给定交换群$G$,定义三重线性映射$h:A\times B\times C\to G$为每个分量满足线性,并且对任意的$r\in R$和$s\in S$有$h(ar,b,c)=h(a,rb,c)$和$h(a,bs,c)=h(a,b,sc)$.考虑如下泛映射问题,其中$h,f'$是交换群同态,$T(A,B,C)$是一个交换群,$f$是三重线性映射:
		$$\xymatrix{T(A,B,C)\ar[r]^{f'}&G\\
		A\times B\times C\ar[u]^h\ar[ur]_f&}$$
	
	    先来验证$A\otimes_R(B\otimes_SC)$和典范映射$h:(a,b,c)\mapsto a\times(b\times c)$是该泛映射问题的解.固定每个$a\in A$,那么$f_a:B\times C\to G$,$(b,c)\mapsto f(a,b,c)$是$S$双线性映射,于是存在唯一的交换群同态$f_a':B\otimes_SC\to G$满足$b\times c\mapsto f(a,b,c)$.现在定义$\varphi:A\times(B\otimes_SC)\to G$是$(a,b\times c)\mapsto f_a'(b\times c)$.那么这是$R$双线性映射,于是存在唯一的群同态$A\otimes_R(B\otimes_SC)\to G$,满足$a\times(b\times c)\mapsto \varphi(a,b\times c)=f(a,b,c)$.
	
	    对偶的验证$(A\otimes_RB)\otimes_SC$和典范映射$k:(a,b,c)\mapsto(a\otimes b)\otimes c$同样是该泛映射问题的解,于是二者同构.
	\end{proof}
\end{enumerate}

张量函子和Hom函子的伴随性:
\begin{enumerate}
	\item 给定右$R$模$A$,$(R,S)$模$B$,右$S$模$C$,那么存在交换群的自然同构:
	$\tau_{A,B,C}:\mathrm{Hom}_S(A\otimes_RB,C)\to\mathrm{Hom}_R(A,\mathrm{Hom}(B,C))$,即把$f:A\otimes_RB\to C$,映射为$\tau(f)_a:b\mapsto f(a\otimes b)$,其中$a\in A,b\in B$.其中自然性是说是说,任意固定$A,B,C$中的两个,对第三个位置总是自然的,举例来讲,对$f:A\to A'$,满足交换群的交换图:
	$$\xymatrix{
		\mathrm{Hom}_S(A'\otimes_RB,C)\ar[r]^{\tau_{A',B,C}}\ar[d]_{(f\otimes 1_B)^*}&\mathrm{Hom}_R(A',\mathrm{Hom}(B,C))\ar[d]^{f^*}\\
		\mathrm{Hom}_S(A\otimes_RB,C)\ar[r]_{\tau_{A,B,C}}&\mathrm{Hom}_R(A,\mathrm{Hom}(B,C))
	}$$
    \begin{proof}
    	
    	先验证$\tau_{A,B,C}$是交换群同态.单射是因为如果$\tau(f)_a(b)=0$,$\forall a\in A,b\in B$,得到$f(a\otimes b)=0,\forall a\in A,b\in B$,于是$f$是零同态.最后说明$\tau_{A,B,C}$是满射,任取$R$模同态$F:A\to\mathrm{Hom}_S(B,C)$,那么$\varphi:A\times B\to C$,$(a,b)\mapsto F_a(b)$是$R$双线性映射,它存在唯一的提升$\varphi':A\otimes_RB\to C$,于是$\varphi'(a\times b)=\varphi(a,b)=F_a(b)$,于是$F=\tau(\varphi')$说明满射.
    \end{proof}
	\item 给定左$R$模$A$,$(S,R)$模$B$,左$S$模$C$,那么存在交换群的自然同构:
	$\tau:\mathrm{Hom}_S(B\otimes_RA,C)\to\mathrm{Hom}_R(A,\mathrm{Hom}(B,C))$,即把$f:B\otimes_RA\to C$,映射为$\tau(f)_a:b\mapsto f(b\otimes a)$,其中$a\in A,b\in B$.其中自然性是说是说,任意固定$A,B,C$中的两个,对第三个位置总是自然的.
	\item 特别的,对交换环$R$,张量函子$-\otimes_RB$左伴随于Hom函子$\mathrm{Hom}_R(B,-)$.
\end{enumerate}

张量积$M\otimes N$中的元素未必总能表示为$a\otimes b$的形式.但是按照定义它一定可以表示为若干$a_i\otimes b_i$的和.如果引入自由模,那么的确存在某种唯一表示.给定环$R$,如果$A$是一个右$R$模,$F$是一个自由左$R$模,记一组基为$Y$,那么$A\otimes_RF$中的每个元可以唯一的表示为$u=\sum_{i=1}^{n}a_i\otimes y_i$.其中$a_i\in A,y_i\in F$.特别的,如果有交换环$R$的模$A$和$B$都是自由模,分别以$X,Y$为基,那么$A\otimes_RB$是自由$R$模,它的基可以取$W=\{x\otimes y\mid x\in X,y\in Y\}$,它的势为$|X||Y|$.
\begin{proof}
	
	有同构$A\otimes_RF=A\otimes_R(\sum_{y\in Y}Ry)\cong\sum_{y\in Y}A\otimes_RRy\cong\sum_{y\in Y}A_y$.记这个同构为$\theta$,那么$\theta(a\otimes z)=\{u_y\}\in\sum A_y$使得$u_z=a$,$u_y=0,\forall y\not=z$.如果取$l_z:A_z\to\sum A_y$那么也就是说$\theta(a\otimes z)=l_z(a)$.现在对每个$v\in\sum A_y$都是有限和$v=l_{y_1}(a_1)+\cdots+l_{y_n}(a_n)=\theta(a_1\otimes y_1+\cdots+a_n\otimes y_n)$.其中$y_i$两两不同,而且$a_i$被唯一决定.于是唯一性可得.
\end{proof}

这里举例说明张量积中的元素未必可以表述为$a\otimes b$.考虑域$k$上的二维线性空间$V$,设一组基为$\{v_1,v_2\}$,上一个定理说明张量基$V\otimes_kV$是以$\{v_1\otimes v_1,v_1\otimes v_2,v_1\otimes v_1,v_2\otimes v_2\}$为基的四维线性空间.我们断言$v_1\otimes v_2+v_2\otimes v_1$不会表示为某个$w_1\otimes w_2$.否则记$w_1=av_1+bv_2$和$w_2=cv_1+dv_2$,展开$w_1\otimes w_2$得到关系式$ac=0=bd$和$ad=1=bc$,这在域中总不会成立.

交换环上的张量积.设$R$是交换环,取$R$模$A_1,A_2,\cdots,A_n,G$,称$f:A_1\times A_2\times\cdots\times A_n\to G$为$n$重线性映射,如果满足:
$$f(\cdots,a_i+a_i',\cdots)=f(\cdots,a_i,\cdots)+f(\cdots,a_i',\cdots),1\le i\le n$$
$$f(\cdots,ra_i,\cdots)=rf(\cdots,a_i,\cdots),r\in R,1\le i\le n$$

如果$R$是交换环,给定$n$个$R$模$M_i,1\le i\le n$,现在定义一个新范畴,它的对象是对$(T,g)$,其中$T$是一个$R$模,$g$是从$\oplus_{1\le i\le n}M_i$到$T$的多重线性映射.从对象$(T,g)$到$(T',g')$的态射定义为从$T$到$T'$的$R$模同态$h$,使得$h\circ g=g'$.我们定义$n$个$R$模$M_i,1\le i\le n$的张量积为这个新范畴的初对象,记作$\otimes_{1\le i\le n}M_i$.

交换环上张量积的存在性.我们直接给出构造.考虑$R$在集合$\oplus_{1\le i\le n} M_i$上的自由模$F$,它模去多重线性关系$K$得到的商模$F/K$就满足张量积的泛映射性质.
$$(\cdots,a_i+a_i',\cdots)-(\cdots,a_i,\cdots)-(\cdots,a_i',\cdots),1\le i\le n$$
$$(\cdots,ra_i,\cdots)-r(\cdots,a_i,\cdots),r\in R,1\le i\le n$$

按照泛映射性质,可得到交换环上张量积的若干性质:
\begin{enumerate}
	\item 恒等律:$R\otimes_RM\cong M$和$M\otimes_RR\cong M$,并且同构是自然的,换句话说张量函子$R\otimes_R-$和$R$模范畴上的恒等函子是自然同构的.
	\item 交换律:$M\otimes N\cong N\otimes M$,并且同构是自然的.换句话说张量函子$M\otimes_R-$和$-\otimes_RM$是自然同构的.
	\item 和(可无限)直和的分配律:
	$$\left(\oplus_iM_i\right)\otimes_R N\cong\oplus_i\left(M_i\otimes_R N\right)$$
	\item 注意一般来讲直积和张量积不是可交换的.例如取挠交换群$\mathbb{Z}/n$,有$\mathbb{Z}/n\otimes\mathbb{Q}={0},\forall n\ge1$.于是$\prod_{n\ge1}(\mathbb{Z}/n\otimes\mathbb{Q})={0}$.但是取$\prod_{n\ge1}\mathbb{Z}/n$中的分量都取$[1]$的元$x$,则$\langle x\rangle\otimes\mathbb{Q}\cong\mathbb{Z}\otimes\mathbb{Q}$不是零模.按照$\mathbb{Q}$是内射$\mathbb{Z}$模,于是$\langle x\rangle\otimes\mathbb{Q}\hookrightarrow(\prod_{n\ge1}\mathbb{Z}/n)\otimes\mathbb{Q}$是嵌入,于是后者不是0模.
\end{enumerate}

设$R$是交换环,$I\subset R$是理想,$M$是$R$模,则有同构$R/I\otimes_RM\cong M/IM$.
\begin{proof}
	
	考虑正合列$I\to R\to R/I\to0$,按照右正合性得到$I\otimes_RM\to R\otimes_RM\to R/I\otimes_RM\to0$.其中$R\otimes_RM\cong M$,自然同构就是$r\otimes m\mapsto rm$.在这个自然同构下$I\otimes_RM$的像是$IM$,于是按照正合性,得到$R/I\otimes_RM\cong M/IM$.
	
	这里我们还可以给出第二个证明,考虑双线性映射$R/I\times M\to M/IM$为$(r+I,m)\mapsto rm$,这个定义是良性的,它不依赖于$r+I$的表示的选取.于是按照张量积的泛映射性质,得到模同态$R/I\otimes_RM\to M/IM$.这自然是满的,但是直接验证单射比较麻烦,为此我们直接构造逆映射,记$M/IM\to R/I\otimes_RM$为$m+IM\mapsto1+I\otimes m$,这个映射的良性是由张量积的性质保证的,因为如果$m-m'=rm_0\in IM$,那么$1+I\otimes(m-m')=r(1+I)\otimes m_0=0$.最后验证上述两个映射互为逆映射,就得证.
\end{proof}

特别的,上一定理说明有交换群同构$\mathbb{Z}/n\otimes_{\mathbb{Z}}B\cong B/nB$;$\mathbb{Z}/n\otimes_{\mathbb{Z}}\mathbb{Z}/m\cong\mathbb{Z}/(m,n)$.对交换环$R$也有模同构$R/I\otimes_RR/J\cong R/(I+J)$.这提供了两个非零模的张量积为零的例子,只要$m,n$互素,就有$\mathbb{Z}/n\otimes_{\mathbb{Z}}\mathbb{Z}/m\cong0$,这一事实还可以由Bezout定理直接得到:只需说明$1\otimes1=0$,按照$(m,n)=1$得到$sm+nt=1$,于是$1\otimes1=(sm+nt)(1\otimes1)=t(n\otimes1)+s(m\otimes1)=0$.

另外,我们可以运用上一定理计算两个有限生成交换群的张量积.给定两个有限生成交换群$G,H$,那么按照结构定理,可记$G=\oplus_{1\le i\le n}G_i$和$H=\oplus_{1\le j\le m}H_j$,其中$G_i$和$H_j$都是循环群.那么按照直和与张量积可交换,这等价于计算$G_i\otimes_{\mathbb{Z}}H_j$.倘若$G_i,H_j$中存在无限循环群$\mathbb{Z}$,那么按照恒等率,张量积同构于另一个群;倘若$G_i,H_j$都是有限交换群,上一段给出了此时张量积的结果,这就保证计算出结果.

如果$R$是局部交换环,$M,N$是两个有限生成模,那么$M\otimes_RN=0$当且仅当$M,N$中某个为零.事实上记$R$的唯一极大理想是$m$,记$R/m=k$,那么从$M\otimes_RN$得到$M_k\otimes_kN_k=0$,其中$M_k=M/mM$,$N_k=N/mN$,它们都是有限秩的$k$上线性空间,于是张量积为零说明$M_k$和$N_k$中至少一个为零.最后按照$m$落在$R$的Jacobson根中,按照Nakayama引理,从$M/mM=0$推出$M=0$,完成证明.

可除群.我们来把$\mathbb{Q}$上的一种可除性做下整理.称一个交换群$D$是可除群,如果对每个非零自然数$n$总有$nD=D$.换句话说对每个自然数$n$和每个$d\in D$,存在$d'\in D$满足$nd'=d$.那么$\mathbb{Q}$是一个可除群.

如果$D$是一个可除群,$G$是一个挠交换群,即每个非零元的加法阶数都是有限的,那么有$G\otimes_{\mathbb{Z}}D=0$.这一事实解释了为什么非零有限交换群$G$不能具备$\mathbb{Q}$模结构,因为倘若具备,那么$G=G\otimes_{\mathbb{Q}}\mathbb{Q}$是$G\otimes_{\mathbb{Z}}\mathbb{Q}$的子群,后者是零群,导致$G$是零群,这矛盾.

给定交换群$D$,如果$D$具备乘法结构使得它成为环,那么乘法结构就是$D\times D\to D$的双线性映射,那么它会诱导了交换群群同态$D\otimes_{\mathbb{Z}}D\to D$,按照乘法幺元的存在性,这个同态应该是满的.如果$D$是非零的可除交换群,并且是挠群,那么上一段说明了$D\otimes_{\mathbb{Z}}D=0$,导致上述同态总不会是满的,这就矛盾.

给定交换环$R$,取定一个乘性闭子集$S$,含1不含0,两个共变函子$S^{-1}R\otimes-$和$S^{-1}(-)$是自然同构的.这里自然同构为$\varphi_M:S^{-1}R\otimes_RM\to S^{-1}M$,$r/s\otimes m\mapsto rm/s$.
\begin{proof}
	
	对每个模$M$,双线性映射$S^{-1}A\times M\to S^{-1}M$定义为$(r/s,m)\mapsto rm/s$.按照张量积的定义,这诱导了模同态$\varphi_M:S^{-1}A\otimes_RM\to S^{-1}M$,满足$\varphi_Mr/s\otimes m\mapsto rm/s$.这个同态明显是满的,下面只需说明它是单同态.
	
	任取$\sum_i(r_i/s_i)\otimes m_i\in S^{-1}A\otimes_RM$.如果记$s=\prod_is_i\in S$,记$t_i=\prod_{j\not=i}s_j$,那么有$\sum_i(a_i/s_i)\otimes m_i=(1/s)\otimes\sum_ia_it_im_i$.于是$S^{-1}A\otimes_RM$中每个元可以表示为$(1/s)\otimes m$.现在设$f((1/s)\otimes m)=0$,即$m/s=0$,于是存在$t\in S$使得$tm=0$.于是$(1/s)\otimes m=(1/st)\otimes tm=0$.这说明$\varphi_M$是单射.
	
	最后说明自然性,任取模同态$f:M\to N$,验证如下图表交换即可:
	$$\xymatrix{S^{-1}R\otimes_RM\ar[r]^{\varphi_M}\ar[d]_{1\otimes f}&S^{-1}M\ar[d]^{S^{-1}f}\\ S^{-1}R\otimes_RN\ar[r]_{\varphi_N}&S^{-1}N}$$
\end{proof}

关于$R[x]\otimes_R-$.设$R$是交换环,那么$R[x]$作为模相当于可数秩的自由模,一组基即$\{1,x,x^2,\cdots\}$.于是按照直和与张量积的可换性,得到$R[x]\otimes_RM\cong M^{\mathbb{Z}}$.记$M$系数的全体多项式构成的交换群为$M[x]$,那么实际上有$R[x]\otimes_RM\cong M[x]$,即把$(r_nx^n+\cdots+r_0)\otimes m$映射为$r_nmx^n+\cdots+r_0m$.此时$M[x]$具备$R[x]$模结构.

Hom函子和张量函子的正合性.
\begin{enumerate}
	\item 函子$\mathrm{Hom}_R(N,-)$是共变的左正合函子.即对任意的左$R$模的正合列$\xymatrix{0\ar[r]&A\ar[r]^f&B\ar[r]^f&C}$,有交换群的正合列:
	$$\xymatrix{0\ar[r]&\mathrm{Hom}_R(N,A)\ar[r]^{f'}&\mathrm{Hom}_R(N,B)\ar[r]^{g'}&\mathrm{Hom}_R(N,C)}$$
	\begin{proof}
		
		先给出第一个证明,老老实实的验证定义.记$f'=\mathrm{Hom}(1_N,f)$和$g'=\mathrm{Hom}(1_N,g)$.
		\begin{enumerate}
			\item $\ker f'=\{0\}$.任取$g:N\to A$是模同态,那么$\mathrm{Hom}(1_N,f)(g)=f\circ g$,按照$f$是单射,说明$f\circ g=0$推出$g=0$,于是它是单同态.
			\item $\mathrm{im}f'\subset\ker g'$.任取模同态$h:N\to A$,那么$g'\circ f'(h)=g\circ f\circ h=0$,这说明$g'\circ f'=0$,即$\mathrm{im}f'\subset\ker g'$.
			\item $\ker g'\subset\mathrm{im}f'$.任取模同态$h:N\to B$使得$g'(h)=0$,即$g\circ h=0$.那么对每个$n\in N$,有$h(n)\in\ker g=\mathrm{im}f$,于是存在某个$a\in A$使得$h(n)=f(a)$.按照$f$是单射,说明这里的$a$在$n$固定时是唯一的.这就构造了一个$N\to A$的映射$k:n\mapsto a$.现在验证下它是模同态,假设$h(n_i)=f(a_i),i=1,2$,那么$h(n_1+n_2)=f(a_1+a_2)$,于是$k(n_1+n_2)=k(n_1)+k(n_2)$;再任取$r\in R$和$h(n)=a$,那么$h(rn)=f(ra)$说明$k(rn)=ra=rk(n)$,这就说明$k$是模同态.最后按照$f'(k)=f\circ k=h$,就说明$\ker g'\subset\mathrm{im}f'$.
		\end{enumerate}
		
		第二个证明.已经说明过这是加性函子,于是它是左正合的等价于保核,即如果$f:A\to B$是模同态,那么$\mathrm{Hom}_R(N,\ker f)$是$f'=\mathrm{Hom}(1_N,f):\mathrm{Hom}_R(N,A)\to\mathrm{Hom}_R(N,B)$的核.这里$\mathrm{Hom}_R(N,\ker f)$表示的是$\mathrm{Hom}_R(N,A)$的子集,满足像集落在$\ker f$中.事实上模同态$g:N\to A$满足$\mathrm{im}g\subset\ker f$,当且仅当$f\circ g=0$,当且仅当$g\in\ker f'$.于是$\mathrm{Hom}_R(N,\ker f)=\ker f'$.
	\end{proof}
	\item 函子$\mathrm{Hom}_R(-,N)$是逆变的左正合函子.即对任意的左$R$模的正合列$\xymatrix{A\ar[r]^f&B\ar[r]^g&C\ar[r]&0}$,有交换群的正合列:
	$$\xymatrix{0\ar[r]&\mathrm{Hom}_R(C,N)\ar[r]^{g'}&\mathrm{Hom}_R(B,N)\ar[r]^{f'}&\mathrm{Hom}_R(A,N)}$$
	\begin{proof}
		
		设$f'=\mathrm{Hom}(f,1_N)$和$g'=\mathrm{Hom}(g,1_N)$.
		\begin{enumerate}
			\item $\ker g'=\{0\}$.倘若$h:C\to N$满足$g'(h)=h\circ g=0$,按照$g$是满射,就得到$h=0$.
			\item $\mathrm{im}g'\subset\ker f'$.任取模同态$h:C\to N$,那么$f'\circ g'(h)=h\circ g\circ f=0$,于是$f'\circ g'=0$,即$\mathrm{im}g'\subset\ker f'$.
			\item $\ker f'\subset\mathrm{im}g'$.任取$h\in\ker f'$,即$h\circ f=0$.对每个$c\in C$,按照$g$是满射,可取$b\in B$使得$g(b)=c$.现在定义一个映射$k:C\to N$为$k(c)=h(b)$.需要说明这个映射良性,假如还存在$b'\in B$满足$g(b')=c$,那么$b-b'\in\ker g=\mathrm{im}f$,于是存在$a\in A$使得$b-b'=f(a)$,那么$h(b)=h(b')+h(b-b')=h(b')+h\circ f()=h(b')$.再验证$k$是一个模同态,并且满足$g'(k)=k\circ g=h$.这就说明$\ker f'\subset\mathrm{im}g'$.
		\end{enumerate}
	\end{proof}
	\item 函子$M\otimes_R-$是共变的右正合函子.即对任意的左$R$模的正合列$\xymatrix{A\ar[r]^f&B\ar[r]^g&C\ar[r]&0}$,有交换群的正合列:
	$$\xymatrix{M\otimes_RA\ar[r]&M\otimes_RB\ar[r]&M\otimes_RC\ar[r]&0}$$
	函子$-\otimes_RN$是共变的右正合函子.即对任意的右$R$模的正合列$\xymatrix{A\ar[r]&B\ar[r]&C\ar[r]&0}$,有交换群的正合列:
	$$\xymatrix{A\otimes_RN\ar[r]&B\otimes_RN\ar[r]&C\otimes_RN\ar[r]&0}$$
	\begin{proof}
		
		以第一个断言为例,需要验证的内容有三个:
		\begin{enumerate}
			\item $\mathrm{im}(1_M\otimes g)=M\otimes_RC$.由于$g$是满同态,对$M\otimes_RC$的每个生成元$m\otimes c$,可记$g(b)=c$,于是它可以表示为$(1_M\otimes g)(m\otimes b)$,这就导致$1_M\otimes g$是满同态.
			\item $\mathrm{im}(1_M\otimes f)\subset\ker(1_M\otimes g)$.从$g\circ f=0$,得到$(1_M\otimes g)\circ(1_M\otimes f)=0$,于是$\mathrm{im}(1_M\otimes f)\subset\ker(1_M\otimes g)$.
			\item $\ker(1_M\otimes g)\subset\mathrm{im}(1_M\otimes f)$.设$f_*=1_M\otimes f,g_*=1_M\otimes g$.需要证明$\ker g_*\subset\mathrm{im}f_*$.记$\mathrm{im}f_*=E$,按照同态基本定理,$E\subset \ker g_*$诱导了模同态$M\otimes B/E\to M\otimes C$.即把$m\otimes b+E$映射为$m\otimes g(b)$.下面通过直接构造逆来说明它是模同构,这就得到$E=\mathrm{im}f_*=\ker g_*$.
			
			现在构造逆.先构造双线性映射$s:M\times C\to(M\otimes_RB)/E$为,对任意$(m,c)$,从$g$满射知存在$b\in B$使得$g(b)=c$,就定义$f(m,c)=m\otimes b+E$.需要说明这个定义良性,倘若还存在$b'\in B$使得$g(b')=c$,那么$b-b'\in\ker g=\mathrm{im}f$,于是存在$a\in A$使得$f(a)=b-b'$,那么$m\otimes(b-b')+E=(1_M\otimes f)(m\otimes a)+E=E$,这说明定义良性.验证它是双线性型,于是张量积的泛映射性质说明这诱导了模同态$s^*:M\otimes_RC\to(M\otimes_RB)/E$,验证它和之前的同态互为逆映射,这就完成证明.
		\end{enumerate}    
	\end{proof}
	\item 如果$R$是交换环,那么上述四条结论中交换群的正合列实际上是$R$模的正合列.换句话讲此时函子$\mathrm{Hom}_R(N,-)$是$R$模范畴上左正合共变函子;$\mathrm{hom}_R(-,N)$是$R$模范畴上的左正合逆变函子;$-\otimes_RN$和$N\otimes_R-$是$R$模范畴上的右正合共变函子.这一事实只要注意到当$R$交换的时候Hom集和张量集都是$R$模,另外$\mathrm{Hom}_R(f,g)$和$f\otimes_R g$都是$R$模同态.
\end{enumerate}

Hom函子的左正合性存在某种意义的逆命题:
\begin{enumerate}
	\item 给定左$R$模$A,B,C$,如果对任意的左$R$模$D$有交换群的短正合列:
	$$\xymatrix{0\ar[r]&\mathrm{Hom}_R(D,A)\ar[r]^{f_*}&\mathrm{Hom}_R(D,B)\ar[r]^{g_*}&\mathrm{Hom}_R(D,C)}$$
	
	则有左$R$模的短正合列$\xymatrix{0\ar[r]&A\ar[r]^f&B\ar[r]^g&C}$.
	\begin{proof}
		
		需要验证的有三件事.
		\begin{enumerate}
			\item $\ker f=\{0\}$.取$D=\ker f$,取$h:D\to A$是包含映射,那么$f_*(h)=f\circ h=0$,按照$f_*$是单射说明$h=0$,于是$\ker f=\{0\}$.
			\item $\mathrm{im}f\subset\ker g$.取$D=A$,取$h$是$A$上的恒等映射.那么$0=g_*\circ f_*(h)=g\circ f\circ h=g\circ f$,即$\mathrm{im}f\subset\ker g$.
			\item $\ker g\subset\mathrm{im}f$.取$D=\ker g$,取$h:D\to B$是包含映射.那么$g_*(h)=g\circ h=0$,于是$h\in\ker g_*=\mathrm{im}f_*$,于是存在模同态$k:D\to A$使得$f_*(k)=f\circ k=h$.任取$b\in\ker g$,那么$f\circ k(b)=h(b)=b$,这说明$\ker g\subset\mathrm{im}f$.
		\end{enumerate}
	\end{proof}
	\item 给定左$R$模$A,B,C$,如果对任意的左$R$模$D$有交换群的短正合列:
	$$\xymatrix{0\ar[r]&\mathrm{Hom}_R(C,D)\ar[r]^{g^*}&\mathrm{Hom}_R(B,D)\ar[r]^{f^*}&\mathrm{Hom}_R(A,D)}$$
	
	则有左$R$模的短正合列$\xymatrix{A\ar[r]^f&B\ar[r]^g&C\ar[r]&0}$.
	\begin{proof}
		
		同样要验证三件事.
		\begin{enumerate}
			\item $\mathrm{coker}g=\{0\}$,即$g$是满射.取$D=C/\mathrm{im}g$,取典范映射$h:C\to D$,那么$g^*(h)=h\circ g=0$,结合$g^*$是单射说明$h=0$,于是$D=0$.
			\item $\mathrm{im}f\subset\ker g$.取$D=C$,取$h$是$C$上的恒等映射,那么有$0=f^*\circ g^*(h)=h\circ g\circ f=g\circ f$,于是$\mathrm{im}f\subset\ker g$.
			\item $\ker g\subset\mathrm{im}f$.取$D=B/\mathrm{im}f$,取$h:B\to D$是典范映射,那么$f^*(h)=h\circ f=0$,于是$h\in\ker f^*=\mathrm{im}g^*$,于是存在模同态$k:C\to D$使得$g^*(k)=k\circ g=h$.于是对每个$b\in\ker g$,有$h(b)=0$,于是$b\in\mathrm{im}f$,得证.
		\end{enumerate}
	\end{proof}
\end{enumerate}

正合性的一些反例.我们始终考虑短正合列$\xymatrix{0\ar[r]&\mathbb{Z}\ar[r]^{\iota}&\mathbb{Q}\ar[r]^{\pi}&\mathbb{Q}/\mathbb{Z}\ar[r]&0}$.
\begin{enumerate}
	\item 将函子$\mathrm{Hom}_{\mathbb{Z}}(\mathbb{Z}/2,-)$作用其上,那么$\mathrm{Hom}_{\mathbb{Z}}(\mathbb{Z}/2,\mathbb{Q}/\mathbb{Z})$是非平凡的,比方说把$[1]$映射为$\frac{1}{2}+\mathbb{Z}$.但是我们说明过$\mathrm{Hom}_{\mathbb{Z}}(\mathbb{Z}/2,\mathbb{Q})=\{0\}$,这说明这个函子不是右正合的.
	\item 这回我们用逆变函子$\mathrm{Hom}_{\mathbb{Z}}(-,\mathbb{Z})$作用其上.注意到$\mathrm{Hom}_{\mathbb{Z}}(\mathbb{Q},\mathbb{Z})=\{0\}$,而$\mathrm{Hom}_{\mathbb{Z}}(\mathbb{Z},\mathbb{Z})=\mathbb{Z}$,导致这个函子不是右正合的.
	\item 将函子$\mathbb{Z}/2\otimes_{\mathbb{Z}}-$作用其上,那么$\mathbb{Z}/2\otimes\mathbb{Z}=\mathbb{Z}/2$,以及$\mathbb{Z}/2\otimes\mathbb{Q}=0$,这说明它不是左正合的.
\end{enumerate}

关于张量函子右正合性的第二个证明.我们注意到张量函子的右正合性的证明比Hom函子的左正合性的证明稍微复杂一点,实际上可以从伴随性将Hom函子的左正合性转化为张量函子的右正合性.
\begin{proof}
	
	给定右$R$模的正合列$\xymatrix{A\ar[r]^f&B\ar[r]^g&C\ar[r]&0}$,需要说明的是对任意左$R$模$M$,有交换群的正合列$\xymatrix{M\otimes_RA\ar[r]^{1_M\otimes f}&M\otimes_RB\ar[r]^{1_M\otimes g}&M\otimes_RC\ar[r]&0}$.按照Hom函子左正合性的逆命题,这等价于验证对任意交换群$N$,有正合列:
	$$\xymatrix{0\ar[r]&\mathrm{Hom}_{\mathbb{Z}}(M\otimes_RA,N)\ar[r]&\mathrm{Hom}_{\mathbb{Z}}(M\otimes_RB,N)\ar[r]&\mathrm{Hom}_{\mathbb{Z}}(M\otimes_RC,N)\ar[r]&0}$$
	
	按照Hom函子和张量函子的伴随性,如果记$i'=\mathrm{Hom}_{\mathbb{Z}}(i,N)$,其中$i=A,B,C$,那么有如下交换图,其中三个垂直的映射都是同构:
	$$\xymatrix{0\ar[r]&\mathrm{Hom}_{\mathbb{Z}}(M\otimes_RA,N)\ar[d]\ar[r]&\mathrm{Hom}_{\mathbb{Z}}(M\otimes_RB,N)\ar[d]\ar[r]&\mathrm{Hom}_{\mathbb{Z}}(M\otimes_RC,N)\ar[d]\ar[r]&0\\ 0\ar[r]&\mathrm{Hom}_R(M,A')\ar[r]&\mathrm{Hom}_R(M,B')\ar[r]&\mathrm{Hom}_R(M,C')\ar[r]&0}$$
	
	我们证明过在垂直的三个映射均为同构的情况下,两行中任一行是正合列得到另一行是正合列,于是仅需说明下行是正合列,但是这实际上是对正合列$A\to B\to C\to0$依次作用了左正合逆变函子$\mathrm{Hom}_{\mathbb{Z}}(-,N)$和左正合共变函子$\mathrm{Hom}_R(M,-)$,因而第二行是正合列.这就完成证明.
\end{proof}

Hom函子和张量函子与正向极限逆向极限的交换性.
\begin{enumerate}
	\item 函子$\mathrm{Hom}_R(A,-)$是共变右伴随函子,因此它和全部极限可交换,特别的逆向极限是范畴的极限的特例,于是给定一个左$R$模的逆向系统$\{M_i,\psi_i^j\}$,存在如下同构.这里我们给出一个初等证明,并证明这里的同构是自然同构.
	$$\tau_A:\mathrm{Hom} _R(A,\lim_{\leftarrow}M_i)\cong\lim_{\leftarrow}\mathrm{Hom}_R(A,M_i)$$
	\begin{proof}
		
		任意给定模$A$,那么$\psi_i^j:M_j\to M_i$诱导了一个同态$\mathrm{Hom} _R(A,M_j)\to\mathrm{Hom}_R(A,M_i)$.我们把这个诱导的映射仍然记作$\psi_i^j$.取逆向极限中的同态列$\alpha_i$,那么这诱导了$\mathrm{Hom} _R(A,\lim_{\leftarrow}M_i)\to\mathrm{Hom}_R(A,M_i)$为把$f$映射为$\alpha_i\circ f$.我们把这个映射记作$\alpha_{i,*}$.那么有$\alpha_ {j,*}\psi_i^j=\alpha_{i,*}$.于是,我们得到了交换图:
		$$\xymatrix{
			\lim_{\leftarrow}\mathrm{Hom}_R(A,M_i)\ar[ddr]_{\beta_j}\ar[dr]^{\beta_i}&&\mathrm{Hom}_R(A,\lim_{\leftarrow}M_i)\ar[dl]_{\alpha_{i,*}}\ar[ddl]^{\alpha_{j,*}}\ar[ll]_{\theta}\\
			&\mathrm{Hom}_R(A,M_i)&\\
			&\mathrm{Hom}_R(A,M_j)\ar[u]^{\psi_i^j}&
		}$$
		
		这里我们取$\lim_{\leftarrow}\mathrm{Hom}_R(A,M_i)$为上述证明$R$模范畴上逆向系统总存在逆向极限的那个例子.于是我们看到$\lim_ {\leftarrow}\mathrm{Hom}_R(A,M_i)$为$\prod_i\mathrm{Hom}_R(A,M_i)$的子模.现在我们取$\theta$为把$f:A\to\lim_{\leftarrow}M_i$映射到$(\alpha_if)$.其中$\alpha$是$\{M_i,\psi_i^j\}$逆向极限中的$\alpha_i:\lim_{\leftarrow}M_i\to M_i$.验证$\theta$是一个同态.验证它是单射,满射,并且满足自然性.
		
	\end{proof}
	\item 函子$\mathrm{Hom}_R(-,B)$是逆变右伴随函子,因此它和$\textbf{R-Mod}^{op}$上的极限交换得到$\textbf{Ab}$上的极限.特别的它和正向极限交换后成为逆向极限,即给定左$R$模的正向系统$\{M_i,\varphi_j^i\}$,那么存在同构:
	$$\mathrm{Hom}_R(\lim_{\rightarrow}M_i,B)\cong\lim_{\leftarrow}\mathrm{Hom}_R(M_i,B)$$
	\begin{proof}
		
		事实上按照$\mathrm{Hom}_R(-,B)$是一个逆变函子,导致$\{\mathrm{Hom} _R(M_i,B),\varphi_j^{i,*}\}$是一个逆向系统.接下来构造同构为$f\mapsto(f\alpha_j)$.验证即可.
	\end{proof}
    \item 函子$A\otimes_R-$是左伴随共变函子,于是它和全部余极限可交换,特别的它和正向极限可交换,即给定右$R$模$A$,设$\{B_i,\varphi_j^i\}$是一个左$R$模的正向系统,就有自然同构$A\otimes_R\lim_ {\rightarrow}B_i\simeq\lim_{\rightarrow}(A\otimes_RB_i)$.
    \begin{proof}
    	
    	按照$A\otimes_R-$是一个共变函子,于是$\{A\otimes_RB_i,1\otimes\varphi_j^i\}$是一个正向系统.我们先来构造一个映射使得它的余核就是$\lim_{\rightarrow}B_i$.对每个$i\le j$,我们定义$B_ {ij}=B_i\times\{j\}$.记它的元$(b_i,j)$为$b_{ij}$.于是$B_ {ij}$和$B_i$同构,同构可以取$b_i\mapsto b_{ij}$.我们定义$\oplus_ {ij}B_{ij}\to\oplus_iB_i$为$\sigma:b_{ij}\mapsto\lambda_j\varphi_j^ib_i-\lambda_ib_i$.这里$\lambda_i$是$B_i$到$\oplus_iB_i$的典范映射.注意到$\mathrm{im}\sigma=S$,于是我们看到$\sigma$的余核为$\lim_ {\rightarrow}B_i$.并且有正合列$\oplus_ {ij}B_{ij}\to\oplus_iB_i\to\lim_{\rightarrow}B_i\to0$.按照张量积保右正合,于是得到$A\otimes_R\left(\oplus_ {ij}B_{ij}\right)\to A\otimes_R\left(\oplus_iB_i\right)\to A\otimes\left(\lim_{\rightarrow}B_i\right)\to0$.按照$\tau:A\otimes_R\left(\oplus_ {i}B_{i}\right)\simeq\oplus_i\left(A\otimes_RB_i\right)$是自然同构,我们得到交换图:
    	$$\xymatrix{
    		A\otimes_R\left(\oplus_ {ij}B_{ij}\right)\ar[d]_{\tau}\ar[r]^{1\otimes\sigma}&A\otimes_R\left(\oplus_iB_i\right)\ar[r]\ar[d]^{\tau'}&A\otimes\left(\lim_{\rightarrow}B_i\right)\ar[r]\ar[d]&0\\
    		\oplus(A\otimes_RB_{ij})\ar[r]^{\sigma'}&\oplus(A\otimes B_i)\ar[r]&\lim_{\leftarrow}(A\otimes B_i)\ar[r]&0
    	}$$
    	
    	按照短五引理,我们得到所要的同构.最后验证自然性.
    \end{proof}
\end{enumerate}

系数的延拓和限制.这里环都是交换环,设$B$是$A$代数,设$M$是$A$模,考虑张量积$B\otimes_AM$,它自然的具备一个$B$模结构,即$b(\sum b_i\otimes x_i)=\sum bb_i\otimes x_i$.这个定义的良性可以从张量积的泛映射性质得出(下面给出证明).这里$B$模$B\otimes_AM$就称为把$A$模$M$的系数延拓至$B$.对偶的如果给定$B$模$M$,那么$M$可自然的视为$A$模,即$am=f(a)m$,这里$f$是$A$代数$B$的结构映射.
\begin{proof}
	
	这里需要验证定义良性是因为张量积中元素表示未必是唯一的.所以归结为证明如果$\sum_ib_i\otimes x_i=0$,那么有$\sum_ibb_i\otimes x_i=0$.按照张量积的泛性质,条件$\sum_ib_i\otimes x_i=0$就是指,对任意$A$模$X$和任意$A$上的双线性映射$f:B\times M\to X$,总有$\sum_if(b_i,x_i)=0$.于是特别的,任取$B\times M\to X$的双线性映射$g$,有$g(b-,-)$也是$B\times M\to X$的双线性映射,于是有$\sum_ig(bb_i,x_i)=0$,而这等价于$\sum_ibb_i\otimes x_i=0$.
\end{proof}
\newpage
\section{结合代数}
\subsection{基本概念}

给定交换环$R$,一个$R$(结合)代数,是指一个$R$模$A$上赋予了一个双线性映射$A\times A\to A$,记作$(x,y)\mapsto xy$,这个双线性映射满足结合律$x(yz)=(xy)z$,并且存在单位元$1_A$满足$1_Ax=x1_A=x,\forall x\in A$.我们提及代数只考虑结合代数.
\begin{enumerate}
	\item 粗略的讲,代数就是同时是环和模的结构.
	\item 双线性条件保证了$\forall x,y\in A$和$a\in R$,恒有$a(xy)=x(ay)=(ax)y$,即基环$R$中的元可以放在任何位置.
	\item 这里双线性映射如果不满足结合律就称为非结合代数.
	\item 代数还有一个等价定义,我们指出过代数就是同时是环和模的结构,因此可以按照我们已经给出的定义,把代数定义为模上赋予了环结构,反过来,它也可以定义为环上赋予了模结构:给定环$R$,一个$R$代数$A$是指一个环$A$以及称为结构映射的环同态$f:R\to A$,满足它把$R$中的元映射为$A$中的中心元,此时$A$自然的具备一个$R$模结构,即定义$ra=f(r)a$.这两种定义的一致性是容易看出的,因为如果$R$模$A$上赋予了适当的双线性映射,此时结构映射恰好就是$a\mapsto a1_A$.
	\item 如果$R$代数$A$的乘法满足交换律$ab=ba,\forall a,b\in A$,就称$A$是一个交换$R$代数.
	\item 如果$R=F$是域,此时$F$代数也是$F$线性空间,于是此时可以定义$F$代数的维数就是它作为线性空间的维数.
	\item 给定两个$R$代数$A,B$,称映射$f:A\to B$是$R$代数同态,如果它既是$R$模同态又是环同态(约定保乘法幺元).
	\item 子代数结构.给定$R$代数$A$,它的子代数$B$可以定义为如下两个等价定义的任一:
	\begin{enumerate}
		\item $B$是$A$的子集,包含0和1,满足在加法乘法数乘下封闭.
		\item $B$是$A$的子集,自身是一个$R$代数,满足包含映射$B\to A$是$R$代数同态.
	\end{enumerate}
	\item 理想.我们把$A$作为环的左/右/双边理想称为代数$A$的左/右/双边理想.那么代数总存在零理想和单位理想,称为平凡的双边理想.如果代数不存在非平凡的双边理想,就称它为单代数.
	\item 给定代数$A$的一族子代数,它们的交仍然是$A$的一个子代数,据此我们定义$A$的一个子集生成的子代数是全部包含该子集的子代数的交.如果一个代数可以被有限子集生成,就称它是有限生成代数.另外我们知道$A$同样可作为$R$模,对于模同样可以考虑有限生成的概念.为了区别通常称有限生成代数是有限型代数,称有限生成模为有限模.一般来讲,如果$A$是$R$代数,那么$A$是有限$R$模可推出$A$是有限型$R$代数,但是反过来未必成立,例如域$k$上的代数$k[x]$.
\end{enumerate}

代数的一些例子.
\begin{enumerate}
	\item 给定域$K$上的代数$A$,它称为可除代数,如果对每个非零元$a\in A$,存在$b\in A$满足$ab=ba=1_A$.此时记$b=a^{-1}$.
	\item 那么每个域都是可除代数,非交换可除代数的一个经典例子是四元数除环$\mathbb{H}$.它是$\mathbb{R}$上的四维代数,一组基为$\{1,i,j,k\}$,满足的关系式为:
	$$i^2=j^2=k^2=-1,ij=k,jk=i,ki=j,ji=-l,kj=-i,ik=-j$$
	
	这里非零元可逆是因为,对每个$u=a+bi+cj+dk$,取$\overline{u}=a-bi-cj-dk$,得到$u\overline{u}=\overline{u}u=a^2+b^2+c^2+d^2$.
	\item 另外$\mathbb{C}$可视为$\mathbb{H}$的子代数,于是尽管作为域的时候$\mathbb{C}$没有有限维扩张,但是它可以嵌入到一个更大的除环$\mathbb{H}$中.另外$\mathbb{H}$不能作为$\mathbb{C}$代数,否则按照代数的定义有$i(j1)=j(i1)$得到$k=-k$矛盾.
	\item 群代数.给定群$G$和域$K$,定义以$G$为基的$K$线性空间为$K[G]$,其中它的乘法约定为,$g\in G$上分量$k_g$和$h\in G$上分量$k_h$相乘得到的是$gh$分量上的$k_gk_h$.
	\item 道路代数.一个箭图是指一个有限的有向图$Q=(Q_0,Q_1)$,其中$Q_0$是有限的顶点集,$Q_1$是有限的箭头集.对每个$\alpha\in Q_1$,记它的起始点为$s(\alpha)$,终端为$t(\alpha)$.称箭头的序列$p=\alpha_r\alpha_{r-1}\cdots\alpha_1$为一条道路,如果对每个$1\le i\le r-1$都满足$t(\alpha_i)=s(\alpha_{i-1})$,这里箭头的复合约定为从右至左,其中$r$称为道路$p$的长度,记$p$的起始点为$s(p)=s(\alpha_1)$,终端为$t(p)=t(\alpha_r)$.如果$s(p)=t(p)$就称道路$p$是一个(定向)圈.另外我们对每个顶点$i\in Q_0$定义长度为0的道路$e_i$,把它的起始点和终端都记作$i$.
	
	现在给定域$K$和箭图$Q$,我们来定义道路代数$KQ$.它作为线性空间是以全体道路为基生成的$K$线性空间.为定义乘法只需定义基上的乘法,给定两个道路$p=\alpha_r\alpha_{r-1}\cdots\alpha_1$和$q=\beta_s\beta_{s-1}\cdots\beta_1$,定义$pq=\alpha_r\cdots\alpha_1\beta_s\cdots\beta_1$,如果$t(\beta_s)=s(\alpha_1)$;其它任何情况约定$pq=0$.于是可验证$\sum_{i\in Q_0}e_i$是乘法幺元,这使得$KQ$构成一个有限维$K$代数.
\end{enumerate}

全体$R$代数和$R$代数同态构成了一个范畴称为$R$代数范畴,记作$\textbf{R-Alg}$.全体$R$交换代数和$R$代数同构构成了$R$代数范畴的完全子范畴,记作$\textbf{CR-Alg}$.
\begin{enumerate}
	\item $\mathbb{Z}$代数就是一个环,于是$\mathbb{Z}$代数范畴恰好就是环范畴$\textbf{Ring}$.类似的$\mathbb{Z}/n$代数范畴就是特征$n$环范畴.
	\item 积对象.积对象即笛卡尔积$\prod_iA_i$,它作为模是有限个模的直积,作为环是有限个环的直积.
	\item 张量积,$\textbf{CR-Alg}$上的余积.给定交换环$R$,给定两个$R$代数$A,B$,它们作为$R$模的张量积$A\otimes_RB$自然具备环结构,即$(a_1\otimes b_1)(a_2\otimes b_2)=a_1a_2\otimes b_1b_2$,再线性延拓定义一般的乘法.
	\item 张量代数.
	\begin{enumerate}
		\item 给定$R$模$M$构造它的张量代数为$\mathrm{T}(M)=\oplus_{n\ge0}\mathrm{T}^n(M)$,这里$\mathrm{T}^0(M)=R$,$\mathrm{T}^1(M)=M$,$\mathrm{T}^2(M)=M\otimes_RM$,$\mathrm{T}^3(M)=M\otimes_RM\otimes_RM$.这里$\mathrm{T}(M)$自然是一个$R$模,它的乘法约定为$(x_1\otimes x_2\otimes\cdots\otimes x_n)(y_1\otimes y_2\otimes\cdots\otimes y_m)=x_1\otimes\cdots\otimes x_n\otimes x_1\otimes\cdots\otimes y_m$.
		\item 外代数.我们定义$R$模$M$的外代数$\bigwedge(M)$是张量代数$\mathrm{T}(M)$模去由全体$x\otimes x\in T^2(M)$生成的双边理想.把外代数中的乘法记作$\wedge$,那么外代数中满足$\forall x,y\in M$有$x\wedge y=-y\wedge x$.
		\item 对称代数.$R$模$M$的对称代数定义为张量代数的商$\mathrm{T}(M)/I$,这里$I$是由形如$m_1\otimes m_2-m_2\otimes m_1$生成的双边理想.对称代数记作$\mathrm{Sym}_R(M)$或者不引起歧义时记作$\mathrm{Sym}(M)$.它也是一个分次$R$代数,齐次分解记作$\mathrm{Sym}(M)=\oplus_{n\ge0}\mathrm{Sym}_n(M)$,那么有$\mathrm{Sym}_0(M)=R$和$\mathrm{Sym}_1(M)=M$.对称代数具有如下泛性质:即典范$R$模单同态$i:M=\mathrm{Sym}_1(M)\to\mathrm{Sym}(M)$,对任意$R$代数$A$,存在如下典范双射:
		$$\mathrm{Hom}_{\textbf{R-Alg}}(\mathrm{Sym}(M),A)\cong\mathrm{Hom}_{\textbf{R-Mod}}(M,A)$$
		$$\varphi\mapsto\varphi\circ i$$
		
		特别的,如果$u:M\to N$是一个$R$模同态,那么上述泛性质导致$u$诱导了唯一的$R$分次代数的分次同态$\mathrm{Sym}(M)\to\mathrm{Sym}(N)$,把它记作$\mathrm{Sym}(u)$.特别的,这说明$\mathrm{Sym}$是从$R$模范畴到交换分次$R$代数范畴的函子,并且它左伴随于遗忘函子.
		
		\qquad
		
		另外如果$R\to S$是环同态,那么有$\mathrm{Sym}_R(M)\otimes_RS\cong\mathrm{Sym}_S(M\otimes_RS)$.另外如果$M_1,M_2$是两个$R$模,那么有$\mathrm{Sym}(M_1\oplus M_2)\cong\mathrm{Sym}(M_1)\otimes_R\mathrm{Sym}(M_2)$.另外如果$M$是自由$R$模,以$\{x_1,\cdots,x_r\}$为基,那么有同构$R[T_1,\cdots,T_r]\cong\mathrm{Sym}(M)$为$T_i\mapsto x_i$.
	\end{enumerate}
	\item 自由积,$\textbf{R-Alg}$上的余积.
	\item 商对象.作为模时商掉的子集需要是子模,作为环时商掉的子集需要是双边理想,而代数的左或右理想本身就是子模,因而$R$代数$A$的商对象恰好就是$A/I$,其中$I$是$A$的一个双边理想.
	\item 自由对象.给定交换环$R$,在$\textbf{R-Alg}$上集合$X$的自由对象称为自由代数,它作为模是$X$生成的自由幺半群$S$为基生成的自由$R$模,它作为环是幺半群环$R[S]$.它和多项式环的区别在于,这里的$X_1X_2$和$X_2X_1$视为不同的单项式.如果欧美考虑的是$\textbf{CR-Alg}$范畴,此时集合$X$上的自由对象恰好就是以集合$X$为未定元集的多项式代数.
	\item 和遗忘函子的伴随性.
	\begin{enumerate}
		\item 遗忘环结构和模结构.构造自由代数是$\textbf{Sets}\to\textbf{R-Alg}$的函子,它左伴随于遗忘函子$\textbf{R-Alg}\to\textbf{Sets}$.
		\item 遗忘环结构.构造张量代数是$\textbf{R-Mod}$到$\textbf{R-Alg}$的函子,它左伴随于遗忘函子$\textbf{R-Alg}\to\textbf{R-mod}$.
		\item 遗忘模结构.函子$A\to R\otimes_{\mathbb{Z}}A$是$\textbf{Ring}\to\textbf{R-Alg}$的函子,它左伴随于遗忘函子$\textbf{R-Alg}\to\textbf{Ring}$.
	\end{enumerate}
\end{enumerate}


按照泛映射性质,我们可以得到如下关系,如果$S$是一个$R$代数,那么:
$$S\otimes_RR[x_1,\cdots,x_n]\sim S[x_1,\cdots,x_n]$$
$$R[x_1,\cdots,x_n]\otimes_RR[y_1,\cdots,y_m]\sim R[x_1,\cdots,x_n,y_1,\cdots,y_m]$$







于是,对于给定的两个左$R$模$A,B$,加性函子$T$诱导了一个交换群同态$T_{A,B}:\mathrm{Hom}_R(A,B)\to\mathrm{Hom}_{\mathbb{Z}}(TA,TB)$.特别的,加性函子总把0态射映射为0态射.事实上加性函子还把0模映射为平凡群:取左$R$模$A$,那么$A$是0模当且仅当$1_A=0$,于是按照把0态射映射为0态射,就有$T(1_{0})=T(0)=0$,导致$T(\{0\})=0$.

\newpage
\subsection{单模和半单模}

循环模.称左$R$模$M$是循环模,如果它能由单个元$m$生成.这等价于讲,存在从自由模$R$到$M$的模同态将$r$映射为$rm$是个满射,于是循环模$M$同构于某个$R/I$,其中$I$是$R$的左理想.

单模和半单模.一个左$A$模$N$称为单模,如果它不是零模,并且它的子模仅有$N$和零模.称左$A$模是半单模,如果它同构于一族单模的直和.称左$A$模$N$是不可分解模,如果它不是零模,并且如果$N=N_1\oplus N_2$,其中$N_i$都是$N$子模,能推出$N_1$或者$N_2$是零模.
\begin{enumerate}
	\item 按照对应定理,环$A$的左理想是单模当且仅当它是极小左理想,即非零左理想在包含序下的极小元.更一般的,商模$M/N$是单模当且仅当$N$是$M$的极大左子模.
	\item 单模的判定准则.给定非零左$A$模$N$,那么如下条件等价,特别的,单模总是循环模.
	\begin{enumerate}
		\item $N$是单模.
		\item $N$的每个非零元生成的循环模都是自身.
		\item $N\cong A/m$,其中$m$是某个极大左理想.
	\end{enumerate}
	\begin{proof}
		
		1推2,任取$0\not=x\in N$,那么$Ax$是$N$的非零子模,于是单性说明$Ax=N$.2推1,任取$N$的非零子模$N'$,任取$N'$的非零元$x$,那么$N=Ax\subset N'\subset N$,迫使$N'=N$,于是$N$是单模.3推1,按照对应定理,$A/m$的子模(理想)对应于$A$中包含$m$的左理想,按照$m$已经是极大左理想,于是$A/m$的子模只有0和自身,于是$A/m$是单模.1推3,任取$N$中非零元$x$,那么有左$A$模同态$A\to N$为$a\mapsto ax$.按照$Ax=N$得到这是满同态,于是同构定理得到$A/m\cong N$,这里$m$是$A$的左理想,按照$A/m$需要是单模,它不含非零非单位的左理想,这只能有$m$是极大左理想.
	\end{proof}
	\item 在证明中可以看到,如果$N$是单模,那么对每个非零元$x\in N$,总有$N\cong A/\mathrm{Ann}_A(x)$.
	\item Schur引理.设$M,N$是左$A$模,设$\varphi:M\to N$是$A$模的非零同态.
	\begin{enumerate}
		\item 若$M$是单模,那么$\ker\varphi=0$,于是$\varphi$是单射.
		\item 若$N$是单模,那么$\mathrm{im}\varphi=N$,于是$\varphi$是满射.
		\item 于是如果$M,N$都是单模,那么要么$M\cong N$,要么$\mathrm{Hom}_A(M,N)=0$.特别的,这说明对于单模$N$总有$\mathrm{End}_A(N)$是除环.
		\item 代数闭域上的Schur引理.设$k$是代数闭域,$A$是$k$代数,$M$是$A$单模,并且作为$k$模是有限维的,那么$M$上自同态$\varphi$总是数量变换$\lambda\mathrm{id}_M,\lambda\in k$.
		\begin{proof}
			
			$\varphi$可视为$k$模$M$上的自同态.按照维数有限,并且$k$是代数闭的,于是可取一个特征值$\lambda\in k$,考虑映射$\varphi-\lambda\mathrm{id}_M$,它也是$A$模$M$上的自同态并且不可逆,于是上一条说明它只能是零映射,于是$\varphi=\lambda\mathrm{id}_M$.
		\end{proof}
	\end{enumerate}
	\item 如果$N$是半单模,那么如下三个条件互相等价:
	\begin{enumerate}
		\item $N$是单模.
		\item $\mathrm{End}_A(N)$是除环.
		\item $N$是不可分解模.
	\end{enumerate}
	\begin{proof}
		
		1推2我们已经在Schur引理中给出过.2推3,首先按照$\mathrm{End}_A(N)$是除环,说明其上零元和幺元不同,于是$N$不会是零模.接下来假设$N=N_1\oplus N_2$,取$N\to N_1$的投影映射$\pi$,那么有$\pi^2=\pi$,按照除环条件只能有$\pi=0$或者$\pi=1$,分别对应于$N_1=0$或者$N_2=0$,于是$N$是不可分解模.3推1,按照定义半单模是一族单模的直和,不可分解条件说明这族单模只能恰好是1个.
	\end{proof}
\end{enumerate}

阿廷环上单模与合成链.设$A$是阿廷环(即作为自身模是有限长度模),那么$A$上单模具有形式$A/I$,其中$I$是极大左理想,那么链$I\subset A$可以延拓为$A$的一个合成链,于是每个$A$单模必然同构于某个合成因子.这说明阿廷环上的单模在同构意义下恰好是合成因子.

一些特殊代数的单模.
\begin{enumerate}
	\item 多项式代数$A=k[X]/(f)$.这里$k$是一个域,$f$是$k[X]$中的元.
	\begin{enumerate}
		\item 每个$A$单模同构于某个$k[X]/(h)$,这里$h$是$f$的一个素因子.
		\begin{proof}
			
			事实上$A$的单模恰好具有形式$A/I$,其中$I$是$A$的极大理想,按照对应定理,这等价于$I$是$k[X]$的包含$(f)$的极大理想,按照$k[X]$是PID,这等价于$I=(h)$是包含$(f)$的极大理想,等价于$h$是$f$的素因子.
		\end{proof}
		\item 如果$g,h$是$f$的两个不同的素因子,那么$A$单模$k[X]/(g)$和$k[X]/(h)$不同构.
		\begin{proof}
			
			假设有同构$\varphi:k[X]/(g)\cong k[X]/(h)$,设$1+(g)$下$\varphi$的像为$t+(h)$,那么$h+(g)$的像为$th+(h)$是零元.但是$h+(g)$不是$k[X]/(g)$中的零元:否则$h\in (g)$和$g,h$互素矛盾.
		\end{proof}
		\item 综上,记$A=k[X]/(f)$,其中$k$是域,若记$f$的唯一分解为$f=\prod_{1\le i\le n}p_i^{e_i}$,其中$p_i$两两不同的不可约多项式,那么$A$上的全部单模在同构意义下恰好就是全体$k[X]/(p_i)$.
	\end{enumerate}
	\item 有限直积上的单模.
	\begin{enumerate}
		\item 给定环$A_i,1\le i\le r$,记它们的直积为$A=\prod_{1\le i\le r}A_i$.每个$A_i$单模$M$自然的成为一个$A$单模,其中模结构定义为$(a_1,a_2,\cdots,a_r)m=a_im$.这是因为$M$作为$A$模的子模自然也是它作为$A_i$模的子模.
		\item 给定环的直积$A=\prod_{1\le i\le r}A_i$,记$\varepsilon_i\in A$是第$i$分量取$A_i$的幺元,其余分量取相应环中零元的元.给定$A$模$M$,记$M_i=\varepsilon_iM_i$,那么$M_i$是$A$模$M$的子模,并且有直和分解$M=\oplus_{1\le i\le r}M_i$.特别的,如果$M$是$A$单模,那么恰好存在一个指标$i$使得$M_i\not=0$,并且$M_i$也是$A$单模.
		\item 给定环的直积$A=\prod_{1\le i\le r}A_i$,那么每个$A$单模$M$恰好可以视为某个$A_i$单模.
	\end{enumerate}
\end{enumerate}

半单模的等价定义.
\begin{enumerate}
	\item 一个$A$模$M$是半单模当且仅当它的每个子模都存在直和补.
	\begin{proof}
		
		一方面,如果$M$半单,那么存在一个指标集$J$和一族单模$S_j,j\in J$,满足$M=\oplus_{j\in J}S_j$.任取$M$的一个子模$N$,考虑$J$的幂集$J'$,那么对任意$I\in J'$,记$S_I=\oplus_{i\in I}S_i$.按照Zorn引理,存在一个极大的子集$I\subset J$,满足$S_I\cap N=0$.我们断言$M=N\oplus S_I$.这只要证明对任意$j\in J$有$S_j\subset N+S_I$.所以不妨设$j\not\in I$,那么按照极大性有$(S_j+S_I)\cap N$非0,也就是说存在$s_j+s_I=n\not=0$,那么$s_j=n-s_I\in (N+S_I)\cap S_j$,其中$s_j\not=0$.单射按照$S_j$是单的,有$(N+S_I)\cap S_j=S_j$,于是$S_j\subset N+S_I$.
		
		反过来,如果每个$M$的子模都是$M$自身的一个直和项.我们只要说明每个非0子模都包含一个单子模.因为一旦这成立,考虑全体单子模直和构成的集合,Zorn引理说明它存在极大元$\oplus_{k\in K}S_k$,如果$D$是$M$的真子模,那么按照条件存在直和$M=D\oplus Y$,其中$Y$非0,但是$Y$包含了一个单子模$S_Y$,那么$\left(\oplus_{k\in K}S_k\right)\oplus S_Y$就和极大性矛盾.
		
		取非0子模$N$,取$x\in N$非0,按照Zorn引理,$N$存在一个极大的子模$Z$满足$x\not\in Z$.按照条件有$Z$在$M$存中在直和项$M'$,也就是$M=Z\oplus M'$,那么也就有$N=Z\oplus(M'\cap N)$.于是$Z$也是$N$中的直和项,记$Y=M'\cap N$,我们断言$Y$是一个单子模.事实上如果$Y$存在一个非0真子模$Y'$,同样的操作得到$Y=Y'\oplus Y''$,那么$N=Z\oplus Y=Z\oplus Y'\oplus Y''$,那么$Z\oplus Y'$和$Z\oplus Y''$中必有一个不含$x$,这和极大性矛盾.至此得到了每个非0子模存在单子模.
	\end{proof}
	\item 一个$A$模$M$是半单模还等价于它可以写成一族单子模的和.
	\begin{proof}
		
		如果$M$是半单模,它是一族单模的直和,于是也是一族单子模的和.反过来如果$M$可以表示为一族单子模$S=\{S_i\}$的和,任取$M$的子模$N$,我们来验证$N$存在直和补.设$E$是全体这样的$M$的子模$Y$构成的集合,满足$Y\cap N=0$.那么$E$非空,因为至少零模在其中.验证Zorn引理的条件知$E$中存在极大元$Y_0$,我们断言$Y_0\oplus N=M$.若否,那么至少存在某个$S_i\not\subset Y_0\oplus N$,于是$S_i\cap(Y_0\oplus N)=0$,这导致$S_i\oplus Y_0$也是满足和$N$的交只为零模的子模,和极大性矛盾.于是必须有$Y_0$是$N$的直和补,于是$M$是半单模.
	\end{proof}
\end{enumerate}

半单模的一些性质.
\begin{enumerate}
	\item 按照最初的定义,一族半单模的直和仍然半单.
	\item 半单模的子模总是直和项,于是半单模的商同构于它的子模.我们下面断言半单模的非零子模总是半单的.事实上任取半单模$M$的非零子模$N$,任取$N$的子模$N'$,那么按照$M$是半单模存在$M$的子模$Y$满足$M=N'\oplus Y$,验证$N=N'\oplus(Y\cap N)$即可.
\end{enumerate}

半单环.一个环称为半单的,如果它作为自身模是半单模.两个例子:
\begin{enumerate}
	\item 设$k$是域,那么环$A=M_n(k)$是半单环.这是因为,如果记$C_i$表示第$i$列以外全部取零的矩阵构成的理想,那么有模的直和分解$A=M_n(k)=\oplus_{1\le i\le n}C_i$.现在我们证明每个$C_i$作为$A$模都是单模.这是因为无论对哪个$i$,这个模都同构于$A$模$k^n$,并且容易验证每个$k^n$的非零元生成的$A$循环模都是$k^n$本身.
	\item 设$k$是域,$f\in k[X]$有唯一分解$f=\prod_{1\le i\le r}p_i^{e_i}$,其中$p_i$是两两不同的不可约多项式,并且$e_i\ge1$.那么$A=k[X]/(f)$是半单环当且仅当全部$e_i=1$.
	\begin{proof}
		
		首先有直积分解$A=\oplus_{1\le i\le r}k[X]/(p_i^{e_i})$.按照环的有限直积是半单的当且仅当做直积的每个环都半单,我们只需说明$k[X]/(p^e)$是半单环当且仅当$e=1$,这里$p\in k[X]$是一个不可约多项式.但是$k[X]/(p^e)$是半单环等价于讲$k[X]/(p^e)$作为$k[X]$模是半单模.一方面当$e=1$时按照$(p)$是$k[X]$的极大理想得到$k[X]/(p)$已经是$k[X]$单模.另一方面假设$e>1$,我们来证明$k[X]/(p^e)$不是半单环.因为它是零维诺特环,所以它是阿廷环,但是它的大根是$(p)/(p^e)$非零,于是它不是半单环.
	\end{proof}
\end{enumerate}

半单环的一些性质.
\begin{enumerate}
	\item 按照定义,$A$是半单环当且仅当它可以写作一族单子模的直和$A=\oplus_iM_i$,但是这里我们实际上可以证明这族子模实际上只有有限个.事实上如果直和是对无限个单子模的直和,考虑幺元在直和下的分解,这只涉及到有限个$M_i$,导致$A$包含于这有限个单子模的直和中,于是其余分量处的单子模只能是零,这矛盾.
	\item 半单环$A$上每个非零模都是半单模.
	\begin{proof}
		
		任取$A$模$M$,需要验证$M$是半单模,但是事实上我们可以取自由模$F$使得存在$F$到$M$的满同态,按照$A$是半单$A$模,得到$F$是半单$A$模,于是商模$M$也是半单$A$模.
	\end{proof}
	\item 半单环的商环总是半单环.特别的,这说明环的有限直积是半单环,那么做直积的每个环都是半单环.
	\begin{proof}
		
		设环$A$是半单环,设$B=A/I$,其中$I$是某个双边理想.任取$B$模$M$,那么$M$也就是满足被$I$零化的$A$模,按照条件$M$作为$A$模是半单的,于是$M$可以表示为一族$A$单模的直和$M=\oplus_{i\in I}M_i$.这里每个$M_i$都被$I$零化,于是它们也是$B$模,这就得到$M$作为$B$模也总是半单的,于是$B$是半单环.
	\end{proof}
	\item 另外上一条的证明也可以看出,设$A$是环,$B=A/I$是商环,这里$I$是$A$的某个双边理想,那么$M$是$B$半单模当且仅当它是$A$半单模并且被$I$零化.
	\item 给定环的有限直积$A=\prod_{1\le i\le r}A_i$,那么$A$是半单环当且仅当每个$A_i$都是半单环.
	\begin{proof}
		
		必要性在第四条已经得证,因为每个$A_i$都是$A$的商环.下面证明充分性,假设每个$A_i$都是半单环,需要验证$A$是半单环,$A$作为$A$模有子模的直和分解$A=\oplus_{1\le i\le r}M_i$,其中$M_i=\varepsilon_iA$是$A$的理想.那么$M_i$作为$A_i$模实际上就是$A_i$自身,按照$A_i$半单得到$M_i$作为$A_i$模是半单模,于是$M_i$作为$A$模也是半单模,按照半单的直和是半单的,就得到$A$作为$A$模是半单的.
	\end{proof}
\end{enumerate}

半单环上的单模.
\begin{enumerate}
	\item 半单环$A$上的单模总同构于$A$的某个极小左理想.事实上按照$A$的单模同构于$A/I$,其中$I$是一个极大左理想,半单条件说明$I$是$A$的直和项,那么$A/I$同构的左理想必然是极小左理想,否则和$I$的极大性矛盾.
	\item 设$A$是半单环,设$N_1,N_2$是两个极小左理想,那么如下条件等价:
	\begin{enumerate}
		\item 作为左$A$模有同构$N_1\cong N_2$.
		\item $N_1N_2\not=0$.
		\item 存在元$x\in A$使得$N_1x=N_2$.
	\end{enumerate}
	\begin{proof}
		
		1推2,设$\varphi:N_2\to N_1$是模同构.那么$\varphi(N_1N_2)=N_1\varphi(N_2)=N_1^2$.于是倘若$N_1N_2=0$,就有$N_2^2=0$.但是这导致$N_2\subset\mathrm{Rad}(A)$和半单环的大根为零矛盾:任取极大左理想$m$,那么$P=A/m$是$A$单模,那么$N_2P$是单模$P$的子模,倘若有$N_2P=P$,那么$P=N_2P=N_2^2P=0$矛盾,于是$N_2(A/m)=0$,于是$N_2\subset m$,于是$N_2\subset\mathrm{Rad}(A)$和半单环的大根为零矛盾.
		
		2推3,至少存在一个$x\in N_2$使得$N_1x\not=0$,这是单模$N_2$的非零子模,于是它只能是$N_1$.3推1,存在$N_1$到$N_1x=N_2$的满同态,按照Schur引理这个非零同态只能是同构.
	\end{proof}
	\item 我们已经解释了半单环$A$写作单子模的直和只涉及到有限个单子模,另外我们还说明了它的单模实际上就是极小左理想,于是存在分解$A=\oplus_{1\le i\le r}N_i$,其中每个$N_i$都是$A$的极小左理想,我们断言$A$的每个极小左理想同构于这里出现的极小左理想,特别的这就说明半单环上的极小左理想在同构意义下唯一.
	\begin{proof}
		
		事实上,我们可以直接构造出合成因子恰好为全体$N_i$的合成链.我们证明过$A$单模同构于它的某个合成因子.
	\end{proof}
\end{enumerate}

大根.设$M$是$A$左模,$M$的大根定义为$M$的全部极大左子模的交,记作$\mathrm{Rad}_A(M)$.一些简单的观察有:如果$\mathrm{Rad}(M/N)=0$,那么包含$N$的极大左子模的交是$N$自身,于是必然有$\mathrm{Rad}(M)\subset N$;另外总有$\mathrm{Rad}(M/\mathrm{Rad}(M))=0$;另外如果$M$是半单$A$模,那么$M$的大根必然是零:设$M=\oplus_iM_i$,其中每个$M_i$都是单模,设$N_i=\oplus_{j\not=i}M_j$,那么$M/N_i\cong M_i$是单模,导致$\mathrm{Rad}(M/N_i)=0$,于是$\mathrm{Rad}(M)\subset\cap_iN_i=0$.

半单性和大根.设$M$是$A$模,那么$M$是有限模且半单当且仅当$M$是Artinian模并且$\mathrm{Rad}(M)=0$.特别的,环$A$是半单环当且仅当它是阿廷环并且$\mathrm{Rad}(A)=0$.
\begin{proof}
	
	$M$半单说明它是一族单模的直和,但是无限个非零模的直和必然不是有限生成的,于是只能有$M$是有限个单模的直和,此时合成链是直接构造而出的,于是$M$是阿廷模.另外我们已经证明过半单模就推出大根为零.这就证明了必要性.
	
	充分性,按照大根为零,存在一族极大左子模$S=\{N_i,i\in I\}$使得它们的交为零.考虑$S$的有限个元的交构成的集族,阿廷条件说明这个集族存在极小元,记作$N_1\cap N_2\cap\cdots\cap N_m$,我们断言这个交只能是零模.否则按照全部$N_i$的交是零,说明至少存在一个$N_i$使得这个交不包含于$N_i$,那么这个交再交上$N_i$会严格变小,和极小条件矛盾.这个交为零保证了存在单射$M\cong M/\cap_{1\le i\le m}N_i\to\oplus_{1\le i\le m}M/N_i$,后者是半单模,半单模的子模半单.最后半单和阿廷条件说明$M$只能是有限个单模的直和(否则直接构造无限长度的严格降链),而单模是循环模,被单个元生成,于是此时$M$是有限生成模.
\end{proof}

一个注解.事实上半单模上如下条件是互相等价的:
\begin{enumerate}
	\item $M$是有限生成$A$模.
	\item $M$是有限个单模的直和.
	\item $M$是阿廷模.
	\item $M$是诺特模.
\end{enumerate}

单环.一个环$A$称为单环,如果它仅有的双边理想是零理想和单位理想.
\begin{enumerate}
	\item 单环未必是半单环.
	\item 设$A$是(左)单环,那么如下条件互相等价:
	\begin{enumerate}
		\item $A$是(左)半单环.
		\item $A$是(左)阿廷的.
		\item $A$有极小左理想.
	\end{enumerate}
	\begin{proof}
		
		1推2是直接的,比方说上面注解.2推3是阿廷条件.最后来说明3推1,设$N$是$A$的极小左理想,那么$NA=\sum_{a\in A}Na$是$A$的非零双边理想,于是单环条件得到$A=NA=\sum_{a\in A}Na$.我们证明过这里$Na$要么是零要么同构于$N$,于是$A$是单模的和,我们证明过此时$A$是半单环.
	\end{proof}
	\item 设$A$是半单环,如下条件互相等价:
	\begin{enumerate}
		\item $A$是单环.
		\item $A$的全部极小左理想互相同构.
		\item $A$的全部左单模互相同构.
	\end{enumerate}
	\begin{proof}
		
		我们证明过半单环上极小左理想在同构意义下恰好就是单模,于是2和3等价.下面假设$A$还是单环,任取$A$的两个极小左理想$N_1,N_2$,那么$N_1A=\sum_{a\in A}N_1a$是$A$的非零双边理想,于是单环条件说明$N_1A=A$.同理$N_2A=A$,于是$(N_1N_2)A=A$,特别的$N_1N_2\not=0$,我们证明过这个条件等价于讲$N_1\cong N_2$.下面假设$A$的全部极小左理想互相同构,任取$A$的双边非零理想$J$,按照$A$是半单环,说明存在包含于$J$的极小左理想$N$,按照全部极小左理想互相同构,得到$A=NA=\sum_{a\in A}Na\subset J$,于是$A=J$,于是$A$是单环.
	\end{proof}
	\item 一个例子.设$A=M_n(D)$,其中$D$是一个除环,记$N_i=\varepsilon_{i}A$表示第$i$列以外全部取零的矩阵构成的子集.
	\begin{enumerate}
		\item $A$是半单环,$N_i$是$A$的极小左理想,并且$_AA=\oplus_{1\le i\le n}N_i$,并且对每个$i,j$总有$N_i\cong N_j$.于是$A$是单环,并且$A\cong N^{\oplus n}$.
		\item $\mathrm{End}_A(N_i)\cong D$.事实上任取$D$中的元$z$,数乘$z$是$N_i$上的模同态,这诱导了同态$D\to\mathrm{End}_A(N_i)$.按照$D$是除环说明它是单射.下面任取$\varphi\in\mathrm{End}_A(N_i)$,设$\varepsilon$为$N_i$中第$i$列全部取$1_D$的矩阵,记$\varphi(\varepsilon)=\beta$,那么有$\beta=\varphi(\varepsilon)=\varphi(\varepsilon^2)=\varepsilon\varphi(\varepsilon)=\varepsilon\beta$,于是$\beta$的第$i$列的全部元互相相同,设为$z\in D$,那么有$\beta=z\varepsilon$,于是对任意$\alpha\in N_i$有$\varphi(\alpha)=\alpha\varphi(\varepsilon)=z\alpha$,这得到满射.
	\end{enumerate}
\end{enumerate}

为了证明本节的核心定理:Artin-Wedderburn定理,我们来引入同态矩阵.给定$R$代数$A$,给定$(M_1,M_2,\cdots,M_n)$是一组左$A$模,我们记$[\mathrm{Hom}_A(M_i,M_j)]$为全体$n$阶矩阵$(\varphi_{ij})$,其中每个$\varphi_{ij}\in\mathrm{Hom}_A(M_j,M_i)$.
\begin{enumerate}
	\item 定义$[\mathrm{Hom}_A(M_i,M_j)]$上的加法为对应项相加,定义乘法为矩阵的乘法.这使得$[\mathrm{Hom}_A(M_i,M_j)]$构成一个$R$代数.此时$R$代数同构于$\mathrm{End}_A(M)$,其中$M=\oplus_{1\le i\le n}M_i$.这个只要直接构造同构即可.
	\item 特别的,$\mathrm{End}_A(M^{\oplus n})\cong M_n(\mathrm{End}_A(M))$.更特别的,如果$M$是存在一组元素个数为$n$的自由$A$模,那么$\mathrm{End}_A(M)\cong M_n(A)$.
	\item 特别的,如果$\{M_i,1\le i\le n\}$是一组左$A$模满足$i\not=j$时总有$\mathrm{Hom}_A(M_i,M_j)=0$,那么$\mathrm{End}_A(\oplus_{1\le i\le n}M_i)\cong\oplus_{1\le i\le n}\mathrm{End}_A(M_i)$.
\end{enumerate}

Artin-Wedderburn定理.设$A$是半单$R$代数(这是指$A$是$R$代数并且$A$作为环半单),
\begin{enumerate}
	\item 存在自然数$n_1,n_2,\cdots,n_r$和$R$可除代数$D_1,D_2,\cdots,D_r$,满足$A\cong M_{n_1}(D_1)\times M_{n_2}(D_2)\times\cdots\times M_{n_r}(D_r)$.
	\item 这里$(n_1,D_1),(n_2,D_2),\cdots,(n_r,D_r)$在同构意义下被$A$唯一决定.其中$r$是同构意义下的全部不同极小左理想(单模)的个数,设全部为$S_1,S_2,\cdots,S_r$,那么$n_i$是$S_i$在合成因子中出现的个数,$D_i$就是除环$\mathrm{End}_A(S_i)$.
	\item 反过来给定一组自然数$n_1,n_2,\cdots,n_r$和一组$R$可除代数$D_1,D_2,\cdots,D_r$,那么$\bigtimes_{1\le i\le r}M_{n_i}(D_i)$是(无论作为左模还是右模)$R$半单代数.
\end{enumerate}
\begin{proof}
	
	第一条.设$A$是左半单$R$代数,此时$A$可以写作有限个单模的直和,我们说明过此时$A$单模就是极小左理想,于是有$_AA\cong M_1\otimes M_2\otimes\cdots\otimes M_r$,其中$M_i$是$A$的极小左理想$N_i$的$r_i$次直和,并且这里可以约定$N_i$两两不同.按照Schur引理,以及Hom函子和有限直和可交换,得到$\mathrm{Hom}_A(M_i,M_j)=0,\forall i\not=j$.于是有$A\cong\mathrm{End}_A(_AA)\cong\bigtimes_{1\le i\le r}\mathrm{End}_A(M_i)\cong\bigtimes_{1\le i\le r}M_{n_i}(D_i)$,其中按照Schur引理有$D_i=\mathrm{End}_A(N_i)$是除环.
	
	第二条.假设还有同构$A\cong\bigtimes_{1\le i\le r}A_i$,其中$A_i\cong M_{k_i}(C_i)$,其中$k_i$是自然数,$C_i$是可除$R$代数.我们证明过对每个$i$,存在$A$的极小左理想$P_i$使得$M_{k_i}(C_i)$作为$A$模同构于$k_i$个$P_i$的直和.按照$i\not=j$时$P_iP_j=0$,说明$P_i$两两不同构.于是按照半单环写作极小左理想直和的唯一性(Jordan-H\"older定理),说明适当排序后有$k_i=n_i$且$N_i\cong P_i$.最后我们证明过$C_i\cong\mathrm{End}_{A_i}(P_i)\cong\mathrm{End}_A(P_i)$,于是$C_i\cong\mathrm{End}_A(P_i)\cong\mathrm{End}(N_i)=D_i$,得证.
	
	第三条.我们证明过$M_n(D)$总是半单环,这里$D$是除环,于是$\bigtimes_{1\le i\le r}M_{n_i}(D_i)$总是半单环.
\end{proof}


不可分解模.给定$R$模$M$,如果存在两个模$N,N'$使得$M\cong N\oplus N'$,就称$N$和$N'$均为$M$的直和项.如果$M$不是零模,$N$和$N'$都是非平凡的,此时称$M$是可分解模;如果这样的分解仅可能在$N$和$N'$中存在零模的时候实现,就称$M$是不可分解$R$模.可分解性总是由幂等自同构元所提供的:给定$R$模$M$,那么她是不可分解模当且仅当自同态环$\mathrm{Hom}_R(M,M)$不存在非平凡的幂等元。
\begin{proof}
	
	一方面如果$M$存在非平凡的(即非0非1)幂等自同态$f$,那么$f(M)$与$(1-f)(M)$都是$M$的子模,并且有$M=f(M)+(1-f)(M)$.现在验证交为零,如果存在$a,b\in M$使得$f(a)=(f-1)(b)$,两边复合$f-1$,就得到$0=(f-1)(b)=f(a)$.另一方面如果$M$存在非平凡的直和分解$M\cong N\oplus N'$,那么$1_N\oplus 0_{N'}$就是一个非平凡的自同态.
\end{proof}



\newpage
\section{正合性}
\subsection{基本概念}

$R$模范畴中的正合性.$S_*=\{S_n\}$是一列左(右)$R$模,$\partial=\{\partial_n:S_n\to S_{n-1}\}$是一列模同态,满足$\partial_n\partial_ {n+1}=0,n\in \mathbb{Z}$,则称$(S_*,\partial)$为一个链复形.这里条件$\partial_n\partial_ {n+1}=0,n\in Z$等价于$\mathrm{im}f_{n+1}\subset\ker f_n,n\in\mathbb{Z}$.如果这个包含关系对$n\in\mathbb{Z}$总取等号,就称链复形为正合的,或者称为正合列.

$$\xymatrix{
	\cdots \ar[rr] && S_{n+1} \ar[rr]^{\partial_{n+1}} && S_n \ar[rr]^{\partial_n} && S_{n-1} \ar[rr]^{} &&\cdots }$$
\begin{enumerate}
	\item 形如下图的正合列等价于讲$f$是单射:
	$$\xymatrix{
		0\ar[r]&A\ar[r]^{f}&B}$$
	\item 形如下图的正合列等价于讲$g$是满射:
	$$\xymatrix{
		B\ar[r]^{g}&C\ar[r]&0}$$
	\item 形如下图的正合列等价于讲$A$和$B$是同构的:
	$$\xymatrix{
		0\ar[r]&A\ar[r]&B\ar[r]&0}$$
	\item 称下图形式的正合列为短正合列.此时有$A\cong\ker f$,并且$C\cong\frac{B}{\mathrm{im}A}$.
	$$\xymatrix{
		0\ar[r]&A\ar[r]^{f}&B\ar[r]^{g}&C\ar[r]&0}$$
	\item 于是按照同态定理,短正合列可以分解一个模同态.即对模同态$f:A\to B$,有短正合列:
	$$\xymatrix{
		0\ar[r]&\ker f\ar[r]&A\ar[r]^{f}&\mathrm{im}f\ar[r]&0}$$
\end{enumerate}

以下是涉及到正合列的若干引理.
\begin{enumerate}
	\item 短五引理的若干版本.
	\begin{enumerate}
		\item 给定如下交换图,其中两行均为正合列,那么存在唯一的同态$h:A''\to B''$使得图表交换,并且倘若$f,g$均是同构,则这唯一的$h$也是同构.
		$$\xymatrix{A'\ar[r]^i\ar[d]_f&A\ar[r]^p\ar[d]^g&A''\ar[r]\ar[d]^h&0\\ B'\ar[r]_j&B\ar[r]_q&B''\ar[r]&0}$$
		\begin{proof}
			
			任取$a''\in A''$,那么存在$a\in A$使得$a''=p(a)$,按照图表交换性,只能定义$h(a'')=q\circ g(a)$.先来说明这个定义良性.假如存在$a_0\in A$使得$p(a)=p(a_0)=a''$,那么$a-a_0\in\ker p=\mathrm{im}i$,于是可取$a'\in A$使得$i(a')=a-a_0$.那么有$g(a)-g(a_0)=g(a-a_0)=g\circ i(a')=j\circ f(a')$,于是$q\circ g(a)-q\circ g(a_0)=q\circ j\circ f(a')=0$,于是$q\circ g(a)=q\circ g(a_0)$.
			
			再验证是模同态,任取$a'',a_1''\in A''$,那么$h(a''+a_1'')=q\circ g(a+a_0)$,其中$p(a)=a''$,$p(a_0)=a_0''$.那么$q\circ g(a+a_0)=q\circ g(a)+q\circ g(a_0)=h(a)+h(a_0)$.另外任取$r\in R$,那么$h(ra'')=q\circ g(ra)=rq\circ g(a)=rh(a'')$.
			
			验证是单同构.如果$a''\in A''$使得$h(a'')=0$,于是$a''=p(a)$满足$q\circ g(a)=0$,那么$g(a)\in\ker q=\mathrm{im}j$,于是$g(a)=j(b')$,其中$b'\in B$.按照$f$是同构,存在$a'\in A$使得$f(a')=b'$,于是$g(a)=j\circ f(a')=g\circ i(a')$.于是按照$g$是同构得到$a=i(a')$,导致$a\in\mathrm{im}i=\ker p$,导致$a''=p(a)=0$,于是$h$是单同态.
			
			验证是满同态.任取$b''\in B''$,按照$q$是满同态,得到$b\in B$使得$q(b)=b''$.按照$g$是同构,说明存在$a\in A$使得$g(a)=b$,于是$q\circ g(a)=b''$,导致$h\circ p(a)=b''$,说明满同态.
		\end{proof}
		\item 给定如下交换图,其中两行均为正合列,那么存在唯一的同态$f:A'\to B'$使得图表交换,并且倘若$g,h$均是同构,则这唯一的$f$也是同构.
		$$\xymatrix{0\ar[r]&A'\ar[r]^i\ar[d]_f&A\ar[r]^p\ar[d]^g&A''\ar[d]^h\\0\ar[r]&B'\ar[r]_j&B\ar[r]_q&B''}$$
		\item 给定如下交换图,其中两行都是短正合列,那么$\alpha,\gamma$单射/满射/双射蕴含$\beta$单射/满射/双射.在双射情况下称两个短正合列同构.
		$$\xymatrix{0\ar[r]&A\ar[d]_{\alpha}\ar[r]^{f}&B\ar[d]_{\beta}\ar[r]^{g}&C\ar[d]_{\gamma}\ar[r]&0\\0\ar[r]&A'\ar[r]^{f'}&B'\ar[r]^{g'}&C'\ar[r]&0}$$
		\begin{proof}
			
			单射情况.设$b\in B$满足$\beta(b)=0$,那么有$\gamma\circ g(b)=g'\circ\beta(b)=0$,按照$\gamma$是单射,得到$g(b)=0$,于是$b\in\ker g=\mathrm{im}f$,于是存在$a\in A$使得$f(a)=b$.那么$0=\beta\circ f(a)=f'\circ\alpha(a)$,按照$f'$和$\alpha$都是单射,得到$a=0$.
			
			满射情况.设$b'\in B'$,按照$g$和$\gamma$都是满射,说明存在$b\in B$使得$g'(b')=\gamma\circ g(b)=g'\circ\beta(b)$,于是$b'-\beta(b)\in\ker g'=\mathrm{im}f'$,于是存在$a'\in A'$满足$b'-\beta(b)=f'(a')$,又按照$\alpha$是满射,存在$a\in A$满足$b'-\beta(b)=f'\circ\alpha(a)=\beta\circ f(a)$,于是$b'=\beta(b+f(a))$,这说明$\beta$是满同态.
			
			最后当$\alpha,\gamma$同时单满,上两段说明$\beta$是同构.
		\end{proof}
		\item 给定如下交换图,其中$\alpha,\beta,\gamma$都是同构,那么上下两行其中一个是正合列,就得到另一个是短正合列.
		$$\xymatrix{0\ar[r]&A\ar[d]_{\alpha}\ar[r]^{f}&B\ar[d]_{\beta}\ar[r]^{g}&C\ar[d]_{\gamma}\ar[r]&0\\0\ar[r]&A'\ar[r]^{f'}&B'\ar[r]^{g'}&C'\ar[r]&0}$$
		$$\xymatrix{&A\ar[d]_{\alpha}\ar[r]^{f}&B\ar[d]_{\beta}\ar[r]^{g}&C\ar[d]_{\gamma}&\\&A'\ar[r]^{f'}&B'\ar[r]^{g'}&C'&}$$
		\begin{proof}
			
			以短正合列情况为例.不妨设上一行是短正合列,另一种情况只要把同构箭头变方向即可.先来说明$f'$是单射.任取$a'\in A'$使得$f'(a')=0$,按照$\alpha$同构知存在$a\in A$使得$\alpha(a)=a'$,那么得到$0=f'\circ\alpha(a)=\beta\circ f(a)$,再按照$\beta$和$f$都是单射,得到$a=0$,于是$f'$是单射.
			
			再说明$g'$是满射.任取$c'\in C'$,按照$\gamma$和$g$均为满射说明存在$b\in B$使得$c'=\gamma\circ g(b)=g'\circ\beta(b)$,这说明$g'$是满射.
			
			再说明$\mathrm{im}f'\subset\ker g'$.任取$a'\in A$,那么$g'\circ f'(a)=g'\circ \beta\circ\beta^{-1}\circ f'\circ\alpha\circ\alpha^{-1}(a)=\gamma\circ g\circ\beta^{-1}\circ \beta\circ f\circ\alpha^{-1}(a)=0$.
			
			最后说明$\ker g'\subset\mathrm{im}f'$.任取$b'\in\ker g'$,那么存在$b\in B$使得$\beta(b)=b'$.于是$0=g'\circ\beta(b)=\gamma\circ g(b)$,从$\gamma$是同构说明$g(b)=0$,于是$b\in\ker g=\mathrm{im}f$,于是存在$a\in A$使得$b=f(a)$.于是$b'=\beta\circ f(a)=f'\circ\alpha(a)$,于是$b'\in\mathrm{im}f'$,这就得证.
		\end{proof}
		\item 上一条说明,如果有如下交换图,其中每个$h_n$都是模同构,那么$\{A_n,f_n\}$是正合列当且仅当$\{B_n,g_n\}$是正合列.
		$$\xymatrix{\cdots\ar[r]&A_{n-1}\ar[d]_{h_{n-1}}\ar[r]^{f_n}&A_n\ar[d]_{h_n}\ar[r]^{f_{n+1}}&A_{n+1}\ar[d]_{h_{n+1}}\ar[r]&\cdots\\\cdots\ar[r]&B_{n-1}\ar[r]^{g_n}&B_n\ar[r]^{g_{n+1}}&B_{n+1}\ar[r]&\cdots}$$
	\end{enumerate}
    \item 分离短正合列.短正合列$\xymatrix{0\ar[r]&A_1\ar[r]^{f}&B\ar[r]^{g}&A_2\ar[r]&0}$称为分离的如果它满足如下等价条件的任意一个:
    \begin{enumerate}
    	\item 存在模同态$h:A_2\to B$使得$g\circ h=1_{A_2}$.
    	\item 存在模同态$h:B\to A_1$使得$h\circ f=1_{A_1}$.
    	\item 它同构于短正合列$\xymatrix{0\ar[r]&A_1\ar[r]^{\iota}&A_1\oplus A_2\ar[r]^{\pi}&A_2\ar[r]&0}$.
    \end{enumerate}
    \begin{proof}
    	
    	1推3.考虑如下图表,取$\varphi$为$(a_1,a_2)\mapsto f(a_1)+h(a_2)$,这使得图表交换,按照短五引理,得到$\varphi$是同构.
    	$$\xymatrix{
    		0\ar[r]&A_1\ar[r]^{\iota_1}\ar[d]_{1_{A_1}}&A_1\oplus A_2\ar[r]^{\pi_2}\ar[d]_{\varphi}&A_2\ar[r]\ar[d]^{1_{A_2}}&0\\
    		0\ar[r]&A_1\ar[r]_{f}&B\ar[r]_{g}&A_2\ar[r]&0}$$
    	
    	2推3.考虑如下图表,取$\varphi$为$b\mapsto(h(b),g(b))$,这使得图表交换,按照短五引理,得到$\varphi$是同构.
    	$$\xymatrix{
    		0\ar[r]&A_1\ar[r]^{f}\ar[d]_{1_{A_1}}&B\ar[r]^{g}\ar[d]_{\varphi}&A_2\ar[r]\ar[d]^{1_{A_2}}&0\\
    		0\ar[r]&A_1\ar[r]_{\iota_1}&A_1\oplus A_2\ar[r]_{\pi_2}&A_2\ar[r]&0}$$
    	
    	3推1和3推2.考虑如下短正合列之间的同构.构造$h:A_2\to B$为$\varphi\iota_2\beta^{-1}$,那么有$g\circ h=g\circ\varphi\iota_2\circ\beta^{-1}=\beta\circ\pi_2\circ\iota_2\circ\beta^{-1}=1_{A_2}$.构造$h:B\to A_1$为$\alpha\circ\pi_1\circ\varphi^{-1}$,那么有$h\circ f=\alpha\circ\pi_1\circ\varphi^{-1}\circ f=\alpha^{-1}\circ\pi_1\circ\iota_1\circ\alpha^{-1}=1_{A_1}$.
    	$$\xymatrix{
    		0\ar[r]&A_1\ar[r]^{l_1}\ar[d]_{\alpha}&A_1\oplus A_2\ar[r]^{\pi_2}\ar[d]_{\varphi}&A_2\ar[r]\ar[d]^{\beta}&0\\
    		0\ar[r]&A_1\ar[r]_{f}&B\ar[r]_{g}&A_2\ar[r]&0}$$
    \end{proof}
    \item 蛇形引理的若干版本.
    \begin{enumerate}
    	\item 给定如下交换图,其中上下两行都是正合列:
    	$$\xymatrix{
    		& A_1 \ar[d]_{\alpha} \ar[r]^{f_1} & B_1 \ar[d]_{\beta} \ar[r]^{g_1} & C_1 \ar[d]_{\gamma} \ar[r] & 0 \\
    		0 \ar[r] & A_0 \ar[r]^{f_0} & B_0 \ar[r]^{g_0} & C_0 &  }$$
    	
    	那么它诱导了一个正合列:
    	$$\xymatrix{
    		\ker\alpha\ar[r]^{f_1'}&\ker\beta\ar[r]^{g_1'}&\ker\gamma\ar[r]^{\delta}&
    		\mathrm{coker}\alpha\ar[r]^{f_0'}&\mathrm{coker}\beta\ar[r]^{g_0'}&\mathrm{coker}\gamma}$$
    	\begin{proof}
    		
    		先来说明$f_1,g_1,f_0,g_0$的确诱导了相应的映射$f_1',g_1',f_0',g_0'$.从图表交换性说明如果$a_1\in\ker\alpha$,那么$\beta\circ f_1(a_1)=f_0\circ\alpha(a_1)=0$,于是$f_1(a_1)\in\ker\beta$;如果$b_1\in\ker\beta$,那么$\gamma\circ g_1(b_1)=g_0\circ\beta(b_1)=0$说明$g_1(b_1)\in\ker\gamma$;任取$\alpha(a_1)\in\mathrm{im}\alpha$,那么$f_0\circ\alpha(a_1)=\beta\circ f_1(a_1)\in\mathrm{im}\beta$,于是$f_0$诱导了$\mathrm{coker}\alpha\to\mathrm{coker}\beta$的映射为$a_0+\mathrm{im}\alpha\mapsto f_0(a_0)+\mathrm{im}\beta$;同理$g_0$诱导了$\mathrm{coker}\beta\to\mathrm{coker}\gamma$为$b_0+\mathrm{im}\beta\mapsto g_0(b_0)+\mathrm{im}\gamma$.
    		
    		$\delta$的构造.任取$c_1\in\ker\gamma$,那么存在$b_1\in B_1$使得$g_1(b_1)=c_1$,于是有$0=\gamma\circ g_1(b_1)=g_0\circ\beta(b_1)$,于是$\beta(b_1)\in\ker g_0=\mathrm{im}f_0$,于是存在$a_0\in A_0$使得$\beta(b_1)=f_0(a_0)$,这里$a_0$是唯一的,因为$f_0$是单射.我们就定义$\delta(c_1)=a_0+\mathrm{im}\alpha\in\mathrm{coker}\alpha$.现在验证它的定义良性.假设存在$b_1'\in B_1$同样满足$g_1(b_1)=c_1$,于是同样得到$\beta(b_1')\in\ker g_0=\mathrm{im}f_0$,于是存在唯一的$a_0'\in A_0$使得$f_0(a_0')=\beta(b_1')$,我们需要说明$a_0-a_0'\in\mathrm{im}\alpha$.按照$g_1(b_1-b_1')=0$,得到$b_1-b_1'\in\ker g_1=\mathrm{im}f_1$,于是存在$a_1\in A_1$使得$b_1-b_1'=f_1(a_1)$,导致$f_0(a_0-a_0')=\beta(b_1-b_1')=\beta\circ f_1(a_1)=f_0\circ\alpha(a_1)$.按照$f_0$是单射,得到$a_0-a_0'=\alpha(a_1)$,这就得到良性.
    		
    		验证$\delta$是模同态.任取$c_1,c_1'\in C_1$,那么存在$b_1,b_1'\in B_1$满足$g_1(b_1)=c_1$和$g_1(b_1')=c_1'$,那么得到$g_1(b_1+b_1')=c_1+c_1'$,再设$a_0,a_0'\in A_0$满足$f_0(a_0)=\beta(b_0)$和$f_0(a_0')=\beta(b_0')$,于是$\delta(c_1)=a_0+\mathrm{im}\alpha$和$\delta(c_1')=a_0'+\mathrm{im}\alpha$.又因为$f_0(a_0+a_0')=\beta(b_0+b_0')$,说明$h(c_1+c_1')=h(c_1)+h(c_1')$.再取$r\in R$和$c_1\in C_1$,设$b_1\in B_1$满足$g_1(b_1)=c_1$,那么$g_1(rb_1)=rc_1$.如果设$a_0\in A_0$满足$f_0(a_0)=\beta(b_1)$,那么得到$f_0(ra_0)=\beta(rb_1)$.于是$\delta(rc_1)=r\delta(c_1)$.
    		
    		接下来验证正合性.首先是$\ker\beta$处的正合性.按照$f_1',g_1'$是限制到了$\ker$上的$f_1,g_1$,说明$g_1'\circ f_1'=0$,即$\mathrm{im}f_1\subset\ker g_1$.现在任取$b_1\in\ker g_1\cap\ker\beta$,那么按照原图表的交换性,存在$a_1\in A_1$使得$f_1(a_1)=b_1$,于是$f_0\circ\alpha(a_1)=0$,导致$\alpha(a_1)=0$,于是$a_1\in\ker\alpha$.
    		
    		$\ker\gamma$处的正合性.任取$g_1(b_1)\in\mathrm{im}g_1'$,其中$b_1\in\ker\beta$.那么$\beta(b_1)=0$导致$f_0(0)=\beta(b_1)$,于是$\delta(g_1(b_1))=0$.于是$\mathrm{im}g_1'\subset\ker\delta$.反过来取$c_1\in\ker\delta$,设$b_1\in B_1$满足$g_1(b_1)=c_1$,那么有$\beta(b_1)=f_0(a_0)$,其中$a_0=\alpha(a_1)$.那么$\beta(b_1)=\beta\circ f_1(a_1)$,于是$b_1-f_1(a_1)\in\ker\beta$,并且$g_1(b_1-f_1(a_1))=g_1(b_1)=c_1$,于是$c_1\in\mathrm{im}g_1'$.
    		
    		$\mathrm{coker}\alpha$处的正合性.任取$a_0+\mathrm{im}\alpha\in\mathrm{im}\delta$,那么存在$c_1\in C_1$,存在$b_1\in B_1$使得$g_1(b_1)=c_1$,使得$\beta(b_1)=f_0(a_0')$,使得$a_0'-a_0=\alpha(a_1)$.这说明$f_0(a_0)=f_0(a_0')-f_0\circ\alpha(a_1)=\beta(b_1-f_1(a_1))\in\mathrm{im}\beta$,说明$a_0+\mathrm{im}\alpha\subset\ker f_0'$.反过来任取$a_0+\mathrm{im}\alpha\in\ker f_0'$,即$f_0(a_0)=\beta(b_1)$,于是$g_1(b_1)\in C_1$,并且$\gamma\circ g_1(b_1)=g_0\circ\beta(b_1)=g_0\circ g_0(a_0)=0$,于是$g_1(b_1)\in\ker\gamma$,并且有$\delta(g_1(b_1))=a_0$.
    		
    		$\mathrm{coker}\beta$处的正合性.任取$f_0(a_0)+\mathrm{im}\beta\in\mathrm{im}f_0$,那么$g_0\circ f_0(a_0)=0$,说明$\mathrm{im}f_0\subset\ker g_0$.反过来取$b_0+\mathrm{im}\beta\in\ker g_0'$,即存在$c_1\in C_1$使得$\gamma(c_1)=g_0(b_0)$.那么存在$b_1\in B_1$使得$g_1(b_1)=c_1$,于是$g_0(b_0)=\gamma\circ g_1(b_1)=g_0\circ\beta(b_1)$,于是$b_0-\beta(b_1)\in\ker g_0=\mathrm{im}f_0$,即$b_0-\beta(b_1)=f_0(a_0)$.于是$b_0+\mathrm{im}\beta=f_0(a_0)+\mathrm{im}\beta\in\mathrm{im}f_0'$,得证.
    	\end{proof}
    	\item 如果取被同态连接的两个短正合列:
    	$$\xymatrix{0\ar[r]
    		& A_1 \ar[d]_{\alpha} \ar[r]^{f_1} & B_1 \ar[d]_{\beta} \ar[r]^{g_1} & C_1 \ar[d]_{\gamma} \ar[r] & 0 \\
    		0 \ar[r] & A_0 \ar[r]^{f_0} & B_0 \ar[r]^{g_0} & C_0\ar[r] & 0 }$$
    	
    	那么蛇形引理的结论为:
    	$$\xymatrix{
    		0\ar[r]&\ker\alpha\ar[r]^{f_1'}&\ker\beta\ar[r]^{g_1'}&\ker\gamma\ar[r]^{\delta}&
    		\mathrm{coker}\alpha\ar[r]^{f_0'}&\mathrm{coker}\beta\ar[r]^{g_0'}&\mathrm{coker}\gamma\ar[r]&0}$$
    	\begin{proof}
    		
    		在上一条的基础上,只需在添加条件下再证明$f_1'$是单射和$g_0'$是满射.首先是$f_1'$是单射,如果$a_1\in\ker\alpha\cap\ker f_1$,结合$f_1$是单射就得到$a_1=0$.最后是$g_0'$是满射,任取$c_0+\mathrm{im}\gamma\in\mathrm{coker}\gamma$,按照$g_0$是满射,就存在$b_0\in B_0$使得$g_0(b_0)=c_0$,于是$g_0'(b_0+\mathrm{im}\beta)=c_0+\mathrm{im}\gamma$,于是$g_0'$是满射.
    	\end{proof}
    \end{enumerate}
    \item 正合列的拼接.给定如下两个正合列:
    $$\xymatrix{\ar[r]&A_1\ar[r]^{f_1}&A_0\ar[r]^{f_0}&K\ar[r]&0}$$
    $$\xymatrix{0\ar[r]&K\ar[r]^{g_0}&B_0\ar[r]^{g_1}&B_1\ar[r]&}$$
    
    那么它可拼接为正合列:
    $$\xymatrix{\ar[r]&A_1\ar[r]^{f_1}&A_0\ar[r]^{g_0\circ f_0}&B_0\ar[r]^{g_1}&B_1\ar[r]&}$$
    \item $3\times3$引理的若干版本.
    \begin{enumerate}
    	\item 给出左$R$模范畴中的如下交换图,其中三列都是短正合列,那么前两行是短正合列推出第三行是短正合列;后两行是短正合列推出第一行是短正合列.
    	$$\xymatrix{&0\ar[d]&0\ar[d]&0\ar[d]&\\0\ar[r]&A'\ar[r]\ar[d]&A\ar[r]\ar[d]&A''\ar[r]\ar[d]&0\\0\ar[r]&B'\ar[r]\ar[d]&B\ar[r]\ar[d]&B''\ar[r]\ar[d]&0\\0\ar[r]&C'\ar[r]\ar[d]&C\ar[r]\ar[d]&C''\ar[r]\ar[d]&0\\&0&0&0&}$$
    	\item 给定如下模的交换图,其中每行和每列都是正合列.那么$B'\to B$和$A''\to B''$是单同态推出$C'\to C$是单同态;$C'\to C$和$A\to B$是单同态推出$A''\to B''$是单同态.
    	$$\xymatrix{A'\ar[r]\ar[d]&A\ar[r]\ar[d]&A''\ar[r]\ar[d]&0\\B'\ar[r]\ar[d]&B\ar[r]\ar[d]&B''\ar[r]\ar[d]&0\\C'\ar[r]\ar[d]&C\ar[r]\ar[d]&C''\ar[r]\ar[d]&0\\0&0&0&}$$
    \end{enumerate}
\end{enumerate}

正合函子.给定左$R$模范畴到左$S$模范畴的共变加性函子$F$与逆变加性函子$G$,那么$F,G$都把零对象映射为零对象,零态射映射为零态射.对任意复形:
$$\xymatrix{\cdots\ar[r]&A\ar[r]^f&B\ar[r]^g&C\ar[r]&\cdots}$$

将函子$F$作用上,得到$\xymatrix{\cdots\ar[r]&FA\ar[r]^{F(f)}&FB\ar[r]^{F(g)}&FC\ar[r]&\cdots}$,那么按照加性函子把零态射映射为零态射,$F(g)\circ F(f)=F(g\circ f)=F(0)=0$,即这仍然是一个复形.一个自然的问题是加性函子能否保证复形的正合性?这便是正合函子的概念.考虑短正合列:
$$\xymatrix{0\ar[r]&A\ar[r]&B\ar[r]&C\ar[r]&0}$$
\begin{enumerate}
	\item 如果对任意短正合列总有$\xymatrix{0\ar[r]&F(A)\ar[r]&F(B)\ar[r]&F(C)}$,则称$F$是共变左正合函子;如果对任意短正合列总有$\xymatrix{0\ar[r]&G(C)\ar[r]&G(B)\ar[r]&G(A)}$,则称$G$是逆变左正合函子.
	\item 如果对任意短正合列总有$\xymatrix{F(A)\ar[r]&F(B)\ar[r]&F(C)\ar[r]&0}$,则称$F$是共变的右正合函子;如果对任意短正合列总有$\xymatrix{G(C)\ar[r]&G(B)\ar[r]&G(A)\ar[r]&0}$,则称$G$是逆变的右正合函子.
	\item 如果对任意短正合列总有$\xymatrix{0\ar[r]&F(A)\ar[r]&F(B)\ar[r]&F(C)\ar[r]&0}$,则称$F$是共变正合函子;如果对任意短正合列总有$\xymatrix{0\ar[r]&G(C)\ar[r]&G(B)\ar[r]&G(A)\ar[r]&0}$,则称$G$是逆变正合函子.
\end{enumerate}

关于各种类型的正合函子,有如下等价描述:设$F$是共变加性函子,$G$是逆变加性函子,它们都是左$R$模范畴到左$S$模范畴的函子.
\begin{enumerate}
	\item $F$左正合,当且仅当对任意正合列$0\to A\to B\to C$得到$0\to F(A)\to F(B)\to F(C)$是正合的.
	\item $F$右正合,当且仅当对任意正合列$A\to B\to C\to0$得到$F(A)\to F(B)\to F(C)\to0$是正合的.
	\item $G$左正合,当且仅当对任意正合列$A\to B\to C\to0$得到$0\to G(C)\to G(B)\to G(A)$是正合的.
	\item $G$右正合,当且仅当对任意正合列$0\to A\to B\to C$得到$G(C)\to G(B)\to G(A)\to0$是正合的.
	\item $F$是正合函子当且仅当对任意正合列$A\to B\to C$得到$F(A)\to F(B)\to F(C)$,这也等价于任意长度的正合列在作用$F$之后仍为正合列.
	\item $G$是正合函子当且仅当对任意正合列$A\to B\to C$得到$F(C)\to F(B)\to F(A)$,这也等价于对任意长度正合列在作用$G$之后仍为正合列.
\end{enumerate}
\begin{proof}
	
	前四条的证明是类似的,以第一条为例.如果共变加性函子$F$满足这个新条件,那么对任意给定的短正合列$0\to A\to B\to C\to0$,则$0\to A\to B\to C$必然也是正合列,于是得到$0\to F(A)\to F(B)\to F(C)$是正合列.如果$F$是左正合函子,现在任取正合列$0\to A\to B\to C$,则$f:B\to C$的核就是$A$,于是按照同构定理得到$\mathrm{im}f=B/A$是$C$的子模.有短正合列$0\to A\to B\to B/A\to0$,那么按照左正合性,得到$0\to F(A)\to F(B)\to F(\mathrm{im}f)$是正合的.而$F$左正合说明从$0\to\mathrm{im}f\to C$得到$0\to F(\mathrm{im}f)\to F(C)$,于是$\ker(F(B)\to F(C))=\ker(F(B)\to F(\mathrm{im}f))$,即$0\to F(A)\to F(B)\to F(C)$是正合的.
	
	五六条是类似的.以证明第五条为例,首先保任意长度的正合列说明保形如$A\to B\to C$的正合列,也保全部短正合列.而保形如$A\to B\to C$的正合列说明保全部任意长度的正合列.于是只需证明保短正合列可以说明保形如$A\to B\to C$的正合列.任取正合列$\xymatrix{A'\ar[r]^f&A\ar[r]^g&A''}$.做如下交换图:
	$$\xymatrix{
		0\ar[dr]&&&&0\ar[dr]&&0&&\\
		&\ker f\ar[dr]&&&&\mathrm{im}g\ar[ur]\ar[dr]&&&\\
		&&A'\ar[rr]^f\ar[dr]&&A\ar[rr]^g\ar[ur]&&A''\ar[dr]&&\\
		&&&\mathrm{im}f\ar[ur]\ar[dr]&&&&\mathrm{coker}g\ar[dr]&\\
		&&0\ar[ur]&&0&&&&0
	}$$
	
	现在把保短正合列的函子$F$作用其上,那么:
	\begin{align*}
	\mathrm{im}Ff&=\mathrm{im}\left(F(A')\to F(\mathrm{im}f)\to F(A)\right)  \\
	&= \mathrm{im}\left(F(\mathrm{im}f)\to F(A)\right) \\
	&= \ker\left(F(A)\to F(\mathrm{im}g)\right) \\
	&= \ker\left(F(A)\to F(\mathrm{im}g)\to F(A'')\right) \\
	&= \ker Fg
	\end{align*}
\end{proof}

加性条件下左右正合函子的等价描述.设$T$是左$R$模范畴到左$S$模上的加性函子.这里$T$可能是共变的或者是逆变的函子.那么$T$是左正合函子当且仅当它保核,即$T(\ker f)=\ker T(f)$.它右正合等价于保余核,即$T(\mathrm{coker}f)=\mathrm{coker}Tf$.【】
\begin{proof}
	
	首先如果$T$保核,需要证明对任意的左正合列$\xymatrix{0\ar[r]&A\ar[r]^{f}&B\ar[r]^g&C}$,有正合列:
	$$\xymatrix{0\ar[r]&TA\ar[r]^{Tf}&TB\ar[r]^{Tg}&TC}$$
	
	首先必然有$0=T(\ker f)=\ker Tf$.又有$\ker Tg=T(\ker g)=T(\mathrm{im}f)\cong TA=\mathrm{im}Tf$.对偶的,如果有右正合列$\xymatrix{A\ar[r]^{f}&B\ar[r]^g&C\ar[r]&0}$,需要证明$\xymatrix{TA\ar[r]^{Tf}&TB\ar[r]^Tg&TC\ar[r]&0}$是正合列.一方面$g$满射得到$\mathrm{coker}g=0$,于是$\mathrm{coker}Tg=T(\mathrm{coker}g)=0$.另一方面有$TB/\mathrm{im}Tf=\mathrm{coker}Tf=T\mathrm{coker}f=T(B/\mathrm{im}f)
	=T(B/\ker g)=T(C)=TB/\ker Tg$,再结合$\mathrm{im}Tf\subset\ker Tg$,得到二者相同.
\end{proof}

如果左$R$模范畴到左$S$模范畴的两个同时共变或者同时逆变的加性函子$F,G$是自然同构的,那么它们具有相同类型的正合性.
\begin{proof}
	
	这实际上是短五引理第四条的推论.以共变左正合为例.假设$F$是左正合函子,任取$R$模的正合列$\xymatrix{0\ar[r]&A\ar[r]^f&B\ar[r]^g&C}$,那么有正合列$0\to FA\to FB\to FC$.按照函子的自然同构,存在同构$\eta_i:Fi\to Gi,i=A,B,C$,就得到如下交换图,按照短五引理第四条,其中一行是正合的推出另一行是正合的.
	$$\xymatrix{
		0\ar[r]&FA\ar[d]_{\eta_A}\ar[r]^{F(f)}&FB\ar[d]_{\eta_B}\ar[r]^{F(g)}&FC\ar[d]_{\eta_C}\\
		0\ar[r]&GA\ar[r]_{G(f)}&GB\ar[r]_{G(g)}&GC}$$
\end{proof}

左正合函子的另一个例子,取挠子模.给定整环$R$,考虑$R$模范畴的函子$\mathrm{Tor}$为,把模$M$映射为挠子模$\mathrm{Tor}(M)$.现在设有模同态$f:M\to N$,那么$f(\mathrm{Tor}(M))\subset\mathrm{Tor}(N)$.就把$\mathrm{Tor}(f)$约定为限制映射$\mathrm{Tor}(M)\to\mathrm{Tor}(N)$.
\begin{enumerate}
	\item $\mathrm{Tor}$是$R$模范畴上的加性函子.
	\item $\mathrm{Tor}$是左正合函子.即等价于它与核可交换$\mathrm{Tor}(\ker f)=\ker\mathrm{Tor}(f)$,这个直接验证即可.
	\item $\mathrm{Tor}(M)$实际上就是$M\to F\otimes_RM$,$m\mapsto 1\otimes m$的核.证明只需注意到$F$可记作$S^{-1}R$,于是$F\otimes_RM\cong S^{-1}M$.
\end{enumerate}

正向极限保右正合列,对偶的逆向极限保左正合列.
\begin{enumerate}
	\item 给定指标集是同一个$I$的三个正向系统$\{A_i,\alpha_j^i\}$,$\{B_i,\beta_j^i\}$,$\{C_i,\gamma_j^i\}$.假设对每个$i$存在正合列$\xymatrix{A_i\ar[r]^{r_i}&B_i\ar[r]^{s_i}&C_i\ar[r]&0}$.那么正向极限满足右正合性:
	$$\xymatrix{\lim\limits_{\rightarrow}K_i\ar[r]^ {r'}&\lim\limits_{\rightarrow}N_i\ar[r]^{s'}&\lim\limits_{\rightarrow}M_i\ar[r]&0}$$
	
	一般来讲正向极限不保正合列,不过如果约定指标集$I$是有向集,则正向极限是保正合列的.
	\begin{proof}
		
		我们只来证明$r'$是单同态,其他的证明都是直接的.假设$r'(x)=0$,$x\in\lim_{\rightarrow}A_i$.按照$I$是有向集,于是$x$可以表示为$\lambda_ia_i+S$,其中$S$是正向系统$A$上定义正向极限的关系集.于是$\mu_ir_ia_i\in T$,这里$T$是正向系统$B$上定义正向极限的关系集.验证这等价于讲存在指标$k\ge i$使得$\beta_k^ir_ia_i=0$,于是得到$r_k\alpha_k^ia_i=0$,按照$r_k$是单同态,得到$x=\lambda_ia_i+S=0$.
	\end{proof}
    \item 同样的一般来讲逆向极限同样不保正合列,不过有如下特殊情况:设指标集为$\mathbb{N}$,并且每个$K_{i+1}\to K_i$都是满射,那么从逆向系统的短正合列$0\to(K_i)\to(L_i)\to(M_i)\to0$推出逆向极限的短正合列$0\to\lim\limits_{\leftarrow}K_i\to\lim\limits_{\leftarrow}L_i\to\lim\limits_{\leftarrow}M_i\to0$.
\end{enumerate}
\newpage
\subsection{投射模}

我们已经看到Hom函子和张量函子一般来讲都不是正合函子.为了完善函子的正合性,一个合理的思路是讨论能够使得这些函子正合的特殊的模.称一个左/右$R$模$P$是投射模,如果$\mathrm{Hom}_R(P,-)$是共变的正合函子,称左/右$R$模$Q$是内射模,如果$\mathrm{Hom}_R(-,Q)$是逆变的正合函子.称一个左或右$R$模$S$是平坦的,如果$S\otimes_R-$或$-\otimes_R S$是共变的正合函子.

自由模总是投射模和平坦模.但是一般来讲自由模不是内射模,例如我们看到过$\mathrm{Hom}_{\mathbb{Z}}(-,\mathbb{Z})$不是正合的.
\begin{proof}
	
	给定左自由模$R^J$,有自然同构$\mathrm{Hom}_R(R^J,M)\cong\oplus_JM$和$M\otimes_RR^J\cong\oplus_JM$.另外一个短正合列$0\to A\to B\to C\to0$总诱导出短正合列$0\to\oplus_JA\to\oplus_JB\to\oplus_JC\to0$,这就说明函子$\mathrm{Hom}_R(R^J,-)$和$-\otimes_RR^J$都是正合的.
\end{proof}

投射模的等价描述.左模$P$是投射模,如果它满足如下等价条件中任一个:
\begin{enumerate}
	\item (提升定义)对任意左$R$模$B$和$C$,以及满同态$f:B\to C$,每个同态$g:P\to C$存在提升,即存在$h:P\to B$使得如下图表交换:
	$$\xymatrix{&P\ar[dl]_{h}\ar[d]^{g}\\B\ar[r]_{f}&C\ar[r]&0}$$
	\item (正合性定义)对任意短正合列$\xymatrix{0\ar[r]&A\ar[r]^{f}&B\ar[r]^{g}&C\ar[r]&0}$诱导了短正合列:
	$$\xymatrix{0\ar[r]&\mathrm{Hom}_R(P,A)\ar[r]^{f_*}&\mathrm{Hom}_R(P,B) \ar[r]^{g_*}&\mathrm{Hom}_R(P,C)\ar[r]&0}$$
	
	由于Hom函子$\mathrm{Hom}_R(P,-)$是左正合函子,于是这一条等价于对每个满的模同态$f:M\to N$,有$\mathrm{Hom}_R(P,f)$是$\mathrm{Hom}_R(P,M)\to\mathrm{Hom}_R(P,N)$的满同态.
	\item (分离定义)短正合列$\xymatrix{0\ar[r]&A\ar[r]^{f}&B\ar[r]^{g}&P\ar[r]&0}$总是分离的.
	\item (直和定义)$P$是$R$上某个左自由模$F$的直和项,即存在左自由模$F$,左模$E$使得$F\cong P\oplus E$.另外一个有限生成模是投射模当且仅当它是一个有限秩的投射模的直和项.
	\item $^*$模$P$存在投射基.
\end{enumerate}
\begin{proof}
	
	提升定义和正合性定义的等价性是非常直接的.函子$\mathrm{Hom}_R(P,-)$已经是左正合的共变函子,于是它是正合函子当且仅当每个满同态$f:B\to C$诱导了满同态$\mathrm{Hom}(1_P,f)$,换句话讲对每个模同态$g:P\to C$,存在(未必唯一)模同态$h:P\to B$提升了$g$.
	
	提升定义推分离定义.如果$P$是投射模,按照$g:B\to P$是满同态,说明$1_P:P\to P$可提升为模同态$h:P\to B$,即$p\circ h=1_P$,这是分离短正合列的一个等价描述.
	
	分离定义推直和定义.给定满足分离定义的模$P$,取自由模$F$使得存在满同态$h:F\to P$,这个满同态可以延拓为一个短正合列$0\to\ker h\to F\to P$.于是条件说明了$P$同构于$F$的一个直和项.另外当$P$有限生成的时候,这里的$F$自然可以取有限秩的自由模.
	
	最后从直和定义推提升定义.对任意满同态$g:A\to B$,任意$f:P\to B$,按照自由模$F$总是投射的,得出同态$f\circ\pi$存在提升$h$,按照直和条件知存在$\iota:P\to F$使得$\pi\circ\iota=1_P$,取$h'=h\circ\iota$,它就是$f$的提升,即$g\circ h'=g\circ h\circ\iota=f\circ\pi\circ\iota=f$.
	$$\xymatrix{
		&F\ar[d]^{\pi}\ar[ddl]_h&\\
		&P\ar[d]^f\ar[dl]^{h'}&\\
		A\ar[r]^g&B\ar[r]&0}$$
\end{proof}

下面给出一些关于投射模的观察.
\begin{enumerate}
	\item 一族模的直和是投射模当且仅当每个分支是投射模.这只需用投射模的提升定义结合余积的图表定义,或者借助直和定义投射模等价于是自由模的直和项.
    \item 交换环上,投射模的有限张量积是投射的.只需证明$P_1,P_2$投射时$P_1\otimes P_2$是投射的.
    \begin{proof}
	
	按照直和定义,存在$R$上自由模$F_1,F_2$和模$K_1,K_2$,使得$F_i\cong K_i\oplus P_i,i=1,2$,那么有$F_1\otimes F_2\cong(K_1\oplus P_1)\otimes(K_2\oplus P_2)=(P_1\otimes P_2)\oplus X$,而$F_1\otimes F_2$是自由模,于是$P_1\otimes P_2$是投射模.
	
	我们还可以给出一个范畴证明.投射告诉我们函子$\mathrm{Hom}_R(P_1,\mathrm{Hom}_R(P_2,-))$是正合的,按照Hom和张量的伴随性,存在自然的同构$\mathrm{Hom}_R(P_1,\mathrm{Hom}_R(P_2,-))\cong \mathrm{Hom}_R(P_1\otimes P_2,-)$,那么这说明后者也是正合函子,这就得到$P_1\otimes P_2$的投射性.
    \end{proof}
    \item 我们曾指出过有限表现模是比有限生成模更强的条件,但是对于投射模这是等价的,即每个有限生成投射模都是有限表现的.事实上对于有限生成投射模$P$,可取有限秩自由模$F$,使得有满同态$\varphi:F\to P$,那么从直和定义得出$F$同构于$\ker\varphi\oplus P$,特别的$\ker\varphi$是$F$的商,于是它是有限生成的,从而$P$是有限表现模.
\end{enumerate}

下面给出一些投射模的例子和反例.
\begin{enumerate}
	\item 寻找全部投射模是一个非常依赖环本身的问题.如果环是PID,按照投射模总是自由模的直和项,而PID上自由模的子模(特别的,直和项)是自由模,就说明PID上投射模恰好就是自由模.一个复杂得多的结论是,域上多项式环$R=k[x_1,x_2,\cdots,_n]$中的投射模也总是自由模.
	\item 非自由模的投射模.考虑环$R=\mathbb{Z}/6$,它作为自身的模有直和分解$\mathbb{Z}/6\cong\mathbb{Z}/2\oplus\mathbb{Z}/3$.于是$\mathbb{Z}/2$和$\mathbb{Z}/3$都是投射$R$模,但是作为自由$R$模的元素个数要么无穷要么应该是6的次幂,这说明它们两个都不会是自由模.
	\item 另外存在整环上存在投射模不是自由模.代数整数环总是一个戴德金整环,戴德金整环的一个等价描述是全部理想都是投射模.存在非PID的代数整数环,即类数非1,此时一个非主理想总是一个投射模但总不是自由模.
	\item 投射模的无限直积未必是投射模.例如,由于$\mathbb{Z}$是PID,其上的投射模恰好就是自由模,但是可数个$\mathbb{Z}$的直积不是一个自由交换群(群论里我们证明过).
\end{enumerate}

投射模和分式化.
\begin{enumerate}
	\item 设$M$是交换环$R$上的投射模,取$R$的乘性闭子集$S$,则$S^{-1}M$是$S^{-1}R$投射模.我们证明过自由模的分式化是分式环的自由模,按照投射模的直和定义,以及分式化作为正合函子与直和可交换,就立刻得到这个结论.这个结论还可以运用提升定义给出证明:
	\begin{proof}
	
	取$S^{-1}R$模之间的满同态$p:N_1\to N_2$.取模同态$f:S^{-1}M\to N_2$.那么$N_1$和$N_2$同样是$R$模.取$g:M\to N_2$为$g(m)=f(m/1)$,这是$R$模同态,并且满足分式化泛映射性质的唯一的提升$S^{-1}M\to N_2$是$f$.按照$M$是投射$R$模,得到$g$存在提升$g':M\to N_1$,也即满足$p\circ g'=g$.现在取$g'$的满足分式化泛映射性质的唯一提升$S^{-1}M\to N_1$为$f'$.我们断言$f'$提升了$f$,这是因为$p\circ f'(m/s)=p(g'(m)/s)=p\circ g'(m)/s=g(m)/s=f(m/s)$,这就得证.
	\end{proof}
    \item 这一条说明在诺特交换环上,有限生成模是投射模是一个局部性质.即$M$是交换环$R$的投射模当且仅当对每个素理想$p$有$M_p$是$R_p$投射模,当且仅当对每个极大理想$m$有$M_m$是$R_m$投射模.其中一侧是上一条结论,对于另一侧只要注意到对于有限生成诺特环$R$,如果$A$是有限生成模,那么$\mathrm{Ext}$函子和分式化函子可交换,这说明$\mathrm{Ext}_R^1(M,B)_m=0$对每个极大理想$m$和每个$R$模$B$成立.再结合零模是一个局部性质就得到$\mathrm{Ext}_R^1(M,B)=0$对任意$R$模$B$成立,而这等价于$M$是投射$R$模.
\end{enumerate}

另一件有趣的事情是局部环上的投射模总是自由模.首先如果投射模是有限的,那么这一事实仅需借助Nakayama引理:取局部环$R$上有限投射模$M$的最小生成元集$\{m_1,m_2,\cdots,m_n\}$,取满同态$R^n\to M$为$(r_1,r_2,\cdots,r_n)\mapsto\sum r_im_i$,取这个满同态的核为$K$,从局部环上有限模极小生成元集与剩余类域上自由模的基的对应关系,得到$\sum a_im_i=0$可推出$a_i\in m$,这说明$K\subset mR^n$.现在从$M$投射得到单同态$\psi:M\to R^n$使得$R^n=\psi(M)\oplus K$,这说明$\psi(M)\cap K=\{0\}$,于是$K\subset mF=m\psi(M)\oplus mK$,而按照二者交为零只能得到$K\subset mK$,于是$K=mK$.从Nakayama引理这就得到$K=0$,于是$F\cong M$.一般投射模是自由模的证明,见Matsumura的<Commutative rings theory>第10页.

投射基.给定环$R$,取一个左$R$模$A$,称$A$的子集$\{a_i\}_{i\in I}$是一个投射基,如果存在一族$R$模同态$\{\varphi_i:A\to R\}_{i\in I}$满足如下两个性质:对每个$x\in A$有$\varphi_i(x)$不为0的$i$是有限的;对任意$x\in A$有等式$x=\sum_{i\in I}(\varphi_ix)a_i$.我们也会把这个子集和对应的同态族并称为一个投射基$(a_i,\varphi_i)$.那么一个左$R$模是投射模当且仅当它存在投射基.
\begin{proof}
	
	假设$A$投射,那么存在一个左自由$R$模$F$,和满同态$\psi:F\to A$.按照投射定义1得到一个模同态$\varphi:A\to F$满足$\psi\varphi=1_A$.取$F$的一组基为$\{e_i,i\in I\}$.记$\psi(e_i)=a_i$.那么对任意的$x\in A$,存在唯一的表示$\varphi(x)=\sum_ir_ie_i$,记$\varphi_i:A\to R$为$x\mapsto r_i$.那么$\{a_i\}$和$\varphi_i$是投射基.
	
	反过来,如果存在投射基$\{a_i\}_{i\in I}$和一族$R$模同态$\{\varphi_i:A\to R\}_{i\in I}$,取集合$I$上的左自由$R$模$F$,取$R$模同态$\psi:F\to A$为$i\mapsto a_i$.现在只要构造一个$R$模同态$\varphi:A\to F$满足$\psi\varphi=1_A$,这就使得$A$是$F$的直和项,由此得到$A$是投射模.定义$\varphi(x)=\sum_i(\varphi_ix)i$,这定义良性,并且满足$\psi\varphi=1_A$,得证.
\end{proof}

投射对象.模仿提升定义可定义一般范畴上的投射对象.称一个对象$P$是投射对象,如果对每个满态射(epic)$g:A\to B$,每个态射$f:P\to B$可提升为态射$\overline{f}:P\to A$.称一个范畴具有足够多的投射对象,如果对每个对象$A$,存在投射对象$P$和满同态$P\to A$.下面给出一些例子.
\begin{enumerate}
	\item 集合范畴上,在承认选择公理的前提下这样的提升总是存在的,于是每个集合都是投射对象.事实上这个结论是等价于选择公理的.集合范畴说具有足够多的投射对象.
	\item 本节主要给出了模范畴上的投射对象,另外按照自由模是投射的,说明模范畴上总是具有足够多的投射对象.特别的交换群范畴等价于$\mathbb{Z}$模范畴,$\mathbb{Z}$是PID,于是交换群范畴上自由对象(自由交换群)就是投射对象.
	\item 群范畴上的自由对象就是自由群.我们给出的这个证明要借助自由群的子群自由这一事实.另外这一事实说明群范畴上具有足够多的投射对象.
	\begin{proof}
		
		先给定集合$S$上的自由群$F(S)$,任取满的群同态$p:G\to H$,任意给定群同态$f:F(S)\to H$,我们定义提升$f'$如下,对每个$s\in S$,有$f(s)\in H$,按照$p$是满同态,存在$g\in G$使得$p(g)=f(s)$,就定义$f'(s)=g$,那么按照自由对象的泛映射性质,这诱导了群同态$f':F(S)\to G$.并且对每个$x\in F(S)$,记$x=s_1s_2\cdots s_n$,其中$s_i\in S$或者$s_i^{-1}\in S$,那么$p\circ f'(x)=f(s_1)f(s_2)\cdots f(s_n)=f(x)$,于是$p\circ f'=f$.
		
		反过来,设群$P$是投射对象.可取自由群$F$使得存在满同态$p:F\to P$,取恒等映射$1_P:P\to P$,按照投射定义存在提升$f:P\to F$.我们断言这是一个单射,如果$f(x)=1$,那么$x=1_P(x)=p\circ f(x)=1$.于是$P$同构于$F$的子群.按照自由群的子群自由,这就得到$P$是自由群.
	\end{proof}
    \item 有限群范畴上投射对象只有平凡群.于是该范畴不具有足够多的投射对象.
\end{enumerate}

超余子模.模$M$的一个子模$S$称为超余子模,如果只要子模$L\subset M$满足$L+S=M$,那么就有$L=M$.通常把$S$中的元称为$M$的非生成元,因为如果$M$被一组元素生成,那么我们扣去$S$中的元依旧是一组生成元.
\begin{enumerate}
	\item 如果$S$是$M$的超余子模,如果$M$是$N$的子模,那么$S$也是$N$是超余子模.
	\begin{proof}
		
		因为如果$N$的子模$L$满足$S+L=N$,那么有$S+(L\cap M)=M$,于是$L\cap M=M$,于是$M\subset L$,于是$N=S+L=L$.
	\end{proof}
    \item 如果$S_i$是$M_i$的超余子模,那么有限直和$\oplus_iS_i$是有限直和$\oplus M_i$的超余子模.
    \begin{proof}
    	
    	按照归纳法归结为证明两个模的情况.如果$M_1\oplus M_2$的子模$L$满足$L+S_1\oplus S_2=M_1\oplus M_2$,那么$L\cap M_1+S_1=M_1$,$L\cap M_2+S_2=M_2$,得到$M_1,M_2\subset L$,于是$M_1\oplus M_2=L$.
    \end{proof}
    \item 设$A$是环,设大根为$J(A)$(即全体极大左理想的交),如果$M$是有限左$A$模,那么$JM$是$M$的超余子模.特别的,如果$(A,m)$是局部环,如果$M$是有限$A$模,那么$mM$是$M$的超余子模.
    \begin{proof}
    	
    	这就是一步NAK引理:如果$M$的子模$L$满足$L+JM=M$,那么有$M/L=J(M/L)$,导致$M=L$.
    \end{proof}
\end{enumerate}

投射盖(projective cover).一个$A$模$M$的投射盖是指一个对$(P,\varphi)$,其中$\varphi$是$P\to M$的满同态,使得$\ker\varphi$是$P$的超余子模.
\begin{enumerate}
	\item 和它的对偶概念内射包不同,投射盖未必总是存在的.例如考虑$\mathbb{Z}$模$M=\mathbb{Z}/2\mathbb{Z}$,此时投射模等价于自由模,任取源端为自由模的满同态$\varphi:F\to M$,设$M=\langle a\rangle$,设$x\in F$是$a$的原像,那么$\varphi(3x)=a$,于是有$F=\ker\varphi+\langle 3x\rangle$,但是这导致$F=\langle 3x\rangle$,矛盾.
	\item 如果$\varphi:P\to M$是$M$的投射盖,任取$P$的真子模$N$,那么总有$\varphi(N)\subsetneqq M$.
	\begin{proof}
		
		假设$P$有真子模$N$使得$\varphi(N)=M$.按照它是真子模,必然有$\ker\varphi+N$是$P$的真子模,于是可取不在这个真子模的元素$x$,按照$\varphi(N)=M$,存在$y\in N$使得$\varphi(y)=\varphi(x)$,于是$x-y\in\ker\varphi$,导致$x=(x-y)+y\in\ker\varphi+N$,这矛盾.
	\end{proof}
	\item 如果$(A,m)$是局部环,那么每个有限模$M$都有投射盖:存在短正合列$0\to K\to F\to M\to0$,其中$F\to M$记作$\varphi$,$F$是有限生成自由模,并且$K\subset mF$.
	\begin{proof}
		
		事实上选取$M$的一组极小生成元集合$\{m_1,m_2,\cdots,m_r\}$,取秩$r$的自由模$F$,把一组基双射的映射到$\{m_1,m_2,\cdots,m_r\}$上,那么必然有$\ker\varphi\subset mF$.我们解释过$mF$是$F$的超余子模,于是超余子模的子模$\ker\varphi$也是$F$的超余子模.
	\end{proof}
    \item 设$\varphi:P\to M$是源端为投射模的满同态,那么它是$M$的投射盖当且仅当如果$g$是$P$上自同态使得$fg=f$,那么总有$g$是自同构.
    \begin{proof}
    	
    	必要性.按照$fg=f$,得到$P=\ker f+\mathrm{Im}g$:$\forall x\in P$,有$x=(x-g(x))+g(x)$,其中$x-g(x)\in\ker f$,$g(x)\in\mathrm{Im}g$.于是条件说明$\mathrm{Im}g=P$.再说明$\ker g=0$:按照$g$是满射,$P$是投射模,提升定义说明存在同态$h:P\to P$使得$gh=1_P$.此时有$P=\mathrm{Im}h\oplus\ker g$.倘若这里$\ker g$不是零,那么$\mathrm{Im}h$是$P$的真子模,我们解释过投射盖的真子模的像理应也是真子模.但是这里$fh=fgh=f$导致$f(\mathrm{Im}h)=M$,这矛盾,于是$\ker g=0$.
    	
    	充分性.假设从$fg=f$总能推出$g$是同构,需要证明$\ker f$是$P$的超余子模.假设$P$的子模$L$满足$\ker f+L=P$,那么$f(L)=P$,于是依旧有满同态$f:L\to M$,按照$P$是投射模,同态$f:P\to M$就提升为$g:P\to L$,使得$fg=f$,把$g$视为$P\to P$的自同态,条件说明$g$是同构,于是只能有$L=P$
    \end{proof}
    \item 投射盖可以理解为模的最小的投射覆盖,这里覆盖理解为存在满同态:如果$\pi:P\to M$是$M$的投射盖,如果$g:Q\to M$是一个源端为投射模的满同态,那么$P$总是$Q$的直和项,并且$\ker\pi$总是$\ker g$的直和项.特别的,这件事说明投射盖如果存在则在同构意义下唯一.
    \begin{proof}
    	
    	按照投射模的提升定义,可构造如下交换图中的$\alpha:P\to Q$和$\beta:Q\to P$使得图表交换,进而可构造$\alpha'$和$\beta'$使得如下图表交换.
    	$$\xymatrix{0\ar[r]&\ker\pi\ar[r]\ar[d]^{\alpha'}&P\ar[r]^{\pi}\ar[d]^{\alpha}&M\ar[r]\ar@{=}[d]&0\\0\ar[r]&\ker g\ar[r]\ar[d]^{\beta'}&Q\ar[r]^{g}\ar[d]^{\beta}&M\ar[r]\ar@{=}[d]&0\\0\ar[r]&\ker\pi\ar[r]&P\ar[r]^{\pi}&M\ar[r]&0}$$
    	
    	于是按照$\pi$是投射盖,得到$\beta\alpha$是同构,进而$\beta'\alpha'$也是同构.按照如下引理得到结论:如果模同态$f:M\to N$和$g:N\to M$满足$gf=1_M$,那么$N=\mathrm{Im}f\oplus\ker g$,并且$M\cong\mathrm{Im}f$:直接构造$N$的直和分解为$n=f(g(n))+(n-f(g(n)))$,其中$n-f(g(n))\in\ker g$,$f(g(n))\in\mathrm{Im}f$.并且有$\mathrm{Im}f\cap\ker g=\{0\}$.
    \end{proof}
\end{enumerate}
\newpage
\subsection{内射模}

先总结下$D$是左内射模的等价描述.
\begin{enumerate}
	\item (提升定义)对任意左$R$模$A,B$,$f:A\to B$单同态,$g:A\to D$,存在模同态$h:P\to B$使得$h\circ f=g$.
	$$\xymatrix{0\ar[r]&A\ar[d]^{g}\ar[r]^{f}&B\ar[dl]^{h}\\&D}$$
	\item (正合性定义)对任意短正合列$\xymatrix{0\ar[r]&A\ar[r]^{f}&B\ar[r]^{g}&C\ar[r]&0}$诱导出了短正合列:
	$$\xymatrix{0\ar[r]&\mathrm{Hom}(C,D)\ar[r]^{g_*}&\mathrm{Hom}(B,D)\ar[r]^{f_*}&\mathrm{Hom}(A,D)\ar[r]&0}$$
	
	由于Hom函子$\mathrm{Hom}_R(-,D)$是左正合函子,于是这一条等价于验证对每个单同态$f:M\to N$,有$\mathrm{Hom}(f,1_D)$是$\mathrm{Hom}_R(N,D)\to\mathrm{Hom}_R(M,D)$的满同态.
	\item (分离定义)短正合列$\xymatrix{0\ar[r]&D\ar[r]^{f}&B\ar[r]^{g}&C\ar[r]&0}$总是分离的.
	\item (直和定义)每个以$D$为子模的左模$B$,有$D$是$B$的直和项.
	\item Baer准则:左$R$模$D$是内射模当且仅当对任意的$R$的左理想$I$,模同态$I\to D$总能提升为$R\to D$的模同态.
	\item $^*$模$D$没有真本性扩张.
\end{enumerate}

等价性的部分证明.提升定义和正合性定义的等价性:函子$\mathrm{Hom}_R(-,D)$是左正合的,所以它是正合函子等价于对每个单同态$f:A\to B$诱导了满同态$\mathrm{Hom}(f,1_D)$,即对每个$g:A\to D$,存在同态$h:B\to D$使得$h\circ f=g$.正合性定义可推出分离定义,因为如果考虑单同态$f:D\to B$,那么恒等映射$1_D:D\to D$可提升为映射$h:B\to D$,满足$h\circ f=1_D$,这是分离短正合列的一个等价定义.分离定义可推出直和定义,如果$B$以$D$为子模,那么有包含映射$D\to B$,于是有短正合列$0\to D\to B\to B/D\to0$.于是$D$是$B$的直和项.最后从直和定义推提升定义还需要一些工具.

接下来我们寻找一些特殊的单同态,然后探究作为内射模需要满足的一些条件.首先把$R$视为自身左模,任取$R$的非零因子$r$,那么左乘$r$构成了$R$上的单同态$\mu_r$.每个$R\to D$的模同态$1\mapsto x$要提升为$R\to D$的同态,于是满足存在$D$中的$y$使得$ry=x$,换句话说对每个$R$中的非零因子$r\in D$,有$rD=D$.
$$\xymatrix{0\ar[r]&R\ar[d]_{1\mapsto x}\ar[r]^{\mu_r}&R\ar[dl]^{?}\\&D}$$

提炼出这个条件,称整环上的模$D$是可除模,如果对$R$的每一个非0因子$r$,总有$rD=D$.于是整环上的内射模总是可除模.对一般环也可以定义可除模,不过这种定义不是唯一的.下面给出整环上可除模的一些基本性质.
\begin{enumerate}
	\item 给定整环$R$,商域记作$K$,那么$K$是一个可除$R$模.
	\item 整环上可除模的直和与直积都是可除模.特别的$K$上线性空间作为$R$模是可除模.
	\item 整环上可除模的商是可除模.特别的可除模的直和项是可除模.
\end{enumerate}

这里再考虑下可除这个条件.对于整环$R$,考虑模$A$,左乘元$r\in R$是$A$上的同态$\mu_r$.对每个非零元$r$有$\mu_r$是$A$上单同态当且仅当$A$是无挠模,即不存在非零元$r\in R$和非零元$a\in A$使得$ra=0$.对每个非零元$r$有$\mu_r$是$A$上满同态当且仅当$A$是可除$R$模.于是对每个非零元$r\in R$有$\mu_r$是同构当且仅当$A$是无挠可除$R$模.如果记$R$的商域为$F$,我们断言$R$模$A$是无挠可除模当且仅当$A$是$F$模,这导致整环上的无挠可除模就是商域上的线性空间.一方面$F$模自然总是无挠可除$R$模.反过来如果$A$是无挠可除$R$模,相当于讲$A$上的模结构$R\mapsto\mathrm{End}(A)$的像每个元都是可逆元,于是(分式化的泛映射性质说明)这诱导了模结构$F\mapsto\mathrm{End}(A)$,即$A$是$F$模.

我们再考虑一种特殊的单同态.取$R$的左理想$I$,那么包含映射$I\hookrightarrow R$是单同态.按照内射模的提升定义,一个左模$D$是内射模需要使得每一个模同态$I\to D$总必须能提升为模同态$R\to D$.神奇的是这一性质实际上是充要的.这就是\textbf{Baer准则}.
\begin{proof}
	
	任取单同态$i:A\to B$,不妨约定这是包含映射,即$A\subset B$,给定模同态$f:A\to E$,这里$E$是一个左内射模.考虑全体对$(A',g')$构成的集合$X$,其中$A'$是满足$A\subset A'\subset B$的左模,$g'$是满足延拓了$f:A\to E$的模同态$g':A'\to E$.那么按照$(A,f)\in X$得到$X$不是空集.
	
	赋予$X$偏序$\le$,$(A',g')\le(A'',g'')$当且仅当$A'\subset A''$并且$g''$在$A'$上的限制为$g'$.验证$X$中的升链总有上界,于是按照Zorn引理,$X$具有最大元$(A_0,g_0)$.
	
	假设$A_0$不是整个$B$,那么存在某个$b\in B$不在$A_0$中.定义$I=\{r\in R\mid rb\in A_0\}$,这是$R$的一个左理想,定义$h:I\to E$为$h(r)=g_0(rb)$,于是按照条件,存在映射$h':R\to E$提升了$h$.取$A_1=A_0+(b)$,和$g_1:A_1\to E$为$a_0+rb\mapsto g_0(a_0)+rh'(1)$.验证定义良性,这和$A_0$的极大性矛盾,于是$A_0=B$,完成证明.
\end{proof}

关于内射模的直和与直积.
\begin{enumerate}
	\item 类似直和是投射模当且仅当每个分量是投射模,我们有:一族模的直积是内射模当且仅当每个分量是内射模.
	\begin{proof}
		
		充分性.设$\{E_i\}$是一族内射模,设$E=\prod_iE_i$.任取单同态$i:A\to B$,任取模同态$f:A\to E$,需要说明的是存在提升$f':B\to E$.现在取投影映射$p_i:E\to E_i$,那么$p_i\circ f$是$A\to E_i$的模同态,按照$E_i$是内射模,存在提升$g_i:B\to E_i$使得$g_i\circ i=p_i\circ f$.现在定义$\{g_i\}$诱导的模同态$g:B\to E$,那么它提升了$f$.
		$$\xymatrix{0\ar[r]&A\ar[r]^i\ar[d]^{f}&B\ar[dl]^g\\&E&}$$
		
		必要性.设$E=\prod_iE_i$是一族模的直积,并且$E$是内射模.那么任取单同态$i:A\to B$,任取模同态$f':A\to E_i$,取$l_i:E_i\to E$为把$e_i\in E_i$映射为$i$分量为$e_i$其余分量为零的$E$中元,那么这是一个模同态.设$f=l_i\circ f'$,按照$E$是内射模说明存在$f$的提升映射$g:B\to E$,设投影映射为$p_i:E\to E_i$那么$p_i\circ l_i=1_{M_i}$,取$g_i=p_i\circ g:B\to E_i$,那么它提升了$f'$:$g_i\circ i=p_i\circ g\circ i=p_i\circ f=p_i\circ l_i\circ f'=f'$.
		$$\xymatrix{0\ar[r]&A\ar[d]_f\ar[r]^i&B\ar[dl]_g\ar[ddl]^{g_i}\\&E\ar[d]_{p_i}&\\&E_i&}$$
	\end{proof}
	\item 按照模范畴上有限直和和有限直积同构,说明有限个模的直和是内射模当且仅当每个分量都是内射模.
	\item 内射模的无限直和一般不是内射模,下面两条说明非左诺特环上总存在左模构成反例.
	\item 左诺特环上,一族内射模的直和是内射模.本质上讲,诺特条件保证了一个左理想到$\oplus_iD_i$的模同态只会涉及到有限个分量不恒取零,这就归结到了有限直和的情况.
	\begin{proof}
		
		给定左诺特环$R$上的一族内射模$\{D_i,i\in K\}$,需要说明$\oplus_{i\in K}D_i$是内射模.按照Baer准则,只需说明对$R$的任意理想$I$,记包含映射$\iota:I\to R$,那么每个模同态$f:I\to\oplus_{i\in K}D_i$可延拓为模同态$f^*:R\to\oplus_{i\in K}D_i$.
		
		取$x=(d_i)$,其中每个$d_i\in D_i$,定义支集$\mathrm{Supp}(x)=\{i\in K\mid e_i\not=0\}$.诺特条件保证了$I$是有限生成的,记$I=(a_1,a_2,\cdots,a_n)$.按照直和的定义,每个$f(a_i)$的支集都是有限集.于是并$S=\cup_{1\le i\le n}\mathrm{Supp}(f(a_i))$是有限集.于是$\mathrm{im}f$只涉及到指标集$K$的有限个分量,即集合$S$.这有限个分量的直和是内射模,于是$f$存在提升$f^*:R\to\oplus_{i\in S}D_i$.
	\end{proof}
    \item Bass-Papp定理.即上述定理的逆命题,如果环$R$满足内射模的直和总是内射模,那么它是左诺特环.
\end{enumerate}

下面给出内射模的一些例子和性质.
\begin{enumerate}
	\item 我们曾证明过$\mathbb{Z}$不是内射交换群.这一事实还可以从另一角度给出证明.考虑一个整环$R$,任取非零元$r\in R$,那么左乘$r$是$R$自身的模同态$\mu_r$,取$R$上的恒等映射$1_R$,倘若$R$是自身的内射模,按照提升定义,需要满足存在$R$中的元$s$使得$rs=1$.所以只要整环上非零元不是单位就是一个反例,例如$\mathbb{Z}$.
	\item PID上模是内射模当且仅当是可除模.于是可除模的性质在PID上就是内射模满足的性质,例如内射模任意直和任意直积都是内射模,另外内射模的商总是内射模.于是比方说$\mathbb{Q}/\mathbb{Z}$是一个内射交换群.
	\begin{proof}
		
		我们说明过整环上内射模总是可除模.现在任取PID上的可除模$D$,按照Baer准则,需要说明每个理想$I$到$D$的模同态$f$可延拓为$R\to D$的模同态.注意PID上的理想总是主理想,即$I=(r)$,于是按照$D$是可除模,存在$x\in D$使得$rx=f(1)$,就定义$\overline{f}:R\to D$为$1\mapsto x$,这满足$\overline{f}\circ\iota=f$.这就说明$D$是内射模.
	\end{proof}
	\item 如果$R$是整环,设商域$Q$,那么$Q$是一个内射$R$模.
	\begin{proof}
		
		按照Baer准则,只要证明模同态$f:I\to Q$,其中$I$是$R$的任意理想,可以延拓为$R\to Q$的模同态.为此注意到如果$a,b\in I$非0,那么有$af(b)=f(ab)=bf(a)$,于是在$Q$中得到$f(a)/a=f(b)/b$,记$c\in Q$为这个公共的值.现在定义$g:R\to Q$为$g(r)=rc$.那么$g$就是所求延拓.
	\end{proof}
	\item 给定整环$R$,设商域为$Q$,那么$R$上的无挠可除模$E$恰好就是$Q$上线性空间.此时$E$是$R$上内射模.
	\begin{proof}
		
		同样考虑Baer引理.给定模同态$f:I\to E$,现在对每个非0的$a\in I$,按照可除性,可以设$f(a)=ae_a$.我们断言对任意的$a,b\in I$不为0,有$e_a=e_b$.事实上从$f(ab)=af(b)=a(be_b)=b(ae_a)$,按照无挠就得证.现在把这个公共的元记作$e$,那么定义$g:R\to E$为$g(r)=re$.容易验证这延拓了$f$,于是得证.
	\end{proof}
    \item 上一条说明域上的模总是内射模.这一事实还可以从左诺特环上内射模的直和是内射模得到.
\end{enumerate}

在投射模等价描述的证明中我们实际运用了模范畴具有足够多的投射对象,只不过运用了自由模这个特殊的投射模.为例完成内射模等价描述的证明,我们这回需要证明模范畴上具有足够多的内射对象,换句话说每个模都可以嵌入到某个内射模中.自由模不存在对偶的范畴概念,因为余自由这个概念只能是平凡模.为此我们的思路是先解决$\mathbb{Z}$模范畴上具有足够多的内射对象,然后借助Hom函子转化为任意模范畴的情况.

每个交换群都可以嵌入到某个内射交换群.任取交换群$G$,那么存在自由群$F$到$G$的满同态,于是$G\cong F/K$.而$F/K\cong\oplus_i\mathbb{Z}/K\subset\oplus_i\mathbb{Q}/K$.最后一项是可除模的直和的商,于是它是可除交换群,而PID上可除模和内射模一致,这就完成证明.

模范畴具有足够多的内射对象.以作$R$模范畴为例,我们的思路是先证明对内射交换群$D$,有$\mathrm{Hom}_{\mathbb{Z}}(R,D)$是内射左$R$模,借助这一事实把左$R$模嵌入到这样形式的左内射$R$模.
\begin{proof}
	
	先来证明$D^*=\mathrm{Hom}_{\mathbb{Z}}(R,D)$是内射左$R$模.首先我们说明过这是一个左$R$模.为证明它是内射$R$模,只需证明函子$\mathrm{Hom}_R(-,D^*)$是正合函子.按照Hom函子和张量函子的伴随性,只需验证$\mathrm{Hom}_{\mathbb{Z}}(R\otimes_R-,D)$是正合函子.而它可以分解为两个函子的复合$\mathrm{Hom}_{\mathbb{Z}}(-,D)\circ(R\otimes_R-)$.由于$D$是内射交换群,说明前者是正合函子.而后者自然同构于左$R$模范畴上的恒等函子,它们都是正合的,于是复合函子是正合的.
	
	接下来任取左$R$模$M$,我们需要把它嵌入到上述形式的内射左$R$模中.先把$M$视为交换群,得到交换群的嵌入同态$\varphi:M\to\mathrm{Hom}_{\mathbb{Z}}(R,M)$,即$m\mapsto(\varphi_m:1\mapsto m)$.按照交换群范畴上具有足够多的内射对象,存在内射交换群$D$使得有交换群嵌入$i:M\to D$.Hom函子的左正合性说明它诱导了单同态$i_*:\mathrm{Hom}_{\mathbb{Z}}(R,M)\to\mathrm{Hom}_{\mathbb{Z}}(R,D)$.于是复合交换群同态$i_*\circ\varphi:M\to\mathrm{Hom}_{\mathbb{Z}}(R,D)$是单同态.验证它实际是左$R$模同态,这就把$M$嵌入到了左$R$模$\mathrm{Hom}_{\mathbb{Z}}(R,D)$.
\end{proof}

现在可以完善内射模等价定义中缺失的直和定义推出提升定义了.如果$D$满足,对每个以它为子模的模$B$,有$D$为$B$的直和项.按照模范畴上具有足够多的内射对象,存在内射模$B$以$D$为子模,那么$D$是$B$的直和项.而模的有限直和是内射模当且仅当每个分量都是内射模,因此$D$是内射模.

事实上每个左/右模都可以嵌入到某个$\prod_{S}\mathrm{Hom}_ {\mathbb{Z}}(R,\mathbb{Q}/\mathbb{Z})$中.这个性质导致我们称$\mathrm{Hom}_{\mathbb{Z}}(R,\mathbb{Q}/\mathbb{Z})$的直积为余自由模,注意这并不是$R$模范畴上的余自由对象,余自由对象是平凡的,我们称这个名字是因为这个模对偶了自由模的性质"任意模都是自由模的商".
\begin{proof}
	
	先证明交换群$A$总可以嵌入到$\mathbb{Q}/\mathbb{Z}$的直积中,对每个$A$中非零元$a$,如果它的阶数无限,取$f(a)$是$\mathbb{Q}/\mathbb{Z}$中任意非零元,如果是有限的$n$,取$f(a)$是阶数恰好整除$n$的非零元,那么$f$定义了交换群$\langle a\rangle$到$\mathbb{Q}/\mathbb{Z}$的模同态,按照$\mathbb{Q}/\mathbb{Z}$是内射模,有$f$提升为同态$f_a:A\to \mathbb{Q}/\mathbb{Z}$.按照积的泛映射性质,得到从$A$到$\prod_{a\in A,a\not=0}(\mathbb{Q}/\mathbb{Z})_a$的模同态,并且这个映射是单的,这是因为如果$a$是非零元,那么$f_a(a)\not=0$.
	
	于是说明$R$中每个模都可以嵌入到内射模$\mathrm{Hom}_ {\mathbb{Z}}(R,\prod_{S}(\mathbb{Q}/\mathbb{Z}))$中.最后按照$\mathrm{Hom}_R(S,-)$是右伴随函子,它保直积,于是就有同构$\prod_a\mathrm{Hom}_{\mathbb{Z}}(R,\mathbb{Q}/\mathbb{Z})\cong \mathrm{Hom}_{\mathbb{Z}}(R,(\mathbb{Q}/\mathbb{Z})^{\prod S})$.
\end{proof}

\subsubsection{内射模的Matlis理论.}

本性扩张.称左$R$模$E$是左$R$模$M$的本性扩张,如果存在单的$R$模同态$\alpha:M\to E$,使得对任意非零的$E$的子模$S$,总有$S\cap\alpha(M)\not=0$.如果$\alpha(M)\subsetneqq E$,就称$E$是$M$的真本性扩张.
\begin{enumerate}
	\item 例如交换群$\mathbb{Q}$是交换群$\mathbb{Z}$的真本性扩张,这时候嵌入只能有一个.事实上每个满足$\mathbb{Z}\subset G\subset\mathbb{Q}$的交换群$G$都是$\mathbb{Z}$的本性扩张.
	\item 本性扩张具有传递性.如果$N$是$M$的本性扩张,$E$是$N$的本性扩张,取$i:M\to N$和$j:N\to E$为满足本性扩张定义的嵌入,记复合为$k$,那么对$E$的任意子模$S$,必然有$S\cap j(N)$是非平凡子模,于是$S\cap j(N)\cap i(M)=S\cap k(M)$非平凡.事实上$M\subset E$是本性扩张当且仅当$M\subset N$和$N\subset E$都是本性扩张.
	\item 一个左$R$模$M$是内射模当且仅当$M$没有真本性扩张.
	\begin{proof}
		
		必要性,设$M$是一个内射模,假设取到$M$的一个真本性扩张$E$,那么存在单同态$\alpha:M\to E$,满足$\alpha(M)\not=E$并且对$E$的每个非零子模$S$有$S\cap\alpha(M)\not=0$.按照$\alpha(M)$是内射模,有$\alpha(M)$是$E$的一个直和项.即存在$S\subset E$使得$E=\alpha(M)\oplus S$.按照$\alpha(M)\subsetneqq E$,得到$S$非0,那么$S\cap\alpha(M)=0$,这就矛盾.
		
		充分性,设$M$没有真本性扩张.那么存在一个内射模$E$使得存在单射$i:M\to E$.如果$E$是$i(M)$的本性扩张,那么$E\cong i(M)\cong M$于是$M$是内射模.现在设$E$不是$i(M)$的本性扩张,那么存在$E$的一个非0子模$S$使得$S\cap i(M)=0$.按照Zorn引理,可以取一个子模$N\subset E$极大的满足$N\cap i(M)=0$.取$\pi$为典范映射$E\to E/N$.于是$\ker\pi\cap i(M)=0$.于是$\pi$限制在$i(M)$上是单射,按照引理这导致$\pi$是单射,于是$N=0$,导致矛盾.
	\end{proof}
    \item 设$A$是诺特交换环,设$S\subseteq A$是乘性闭子集,如果$M$是$A$模,$N\subseteq M$是$A$子模,如果$N\subseteq M$是本性扩张,那么$S^{-1}N\subseteq S^{-1}M$作为$S^{-1}A$模也是本性扩张.
    \begin{proof}
    	
    	对$\xi\in M$,记$\xi_S=\xi/1\in S^{-1}M$.那么$S^{-1}M$中的每个元都可以写作$u\xi_S$,其中$u\in S^{-1}A$是单位,而$\xi\in M$.我们只需证明对每个非零的$\xi_S$,都有$S^{-1}N\cap(S^{-1}A\xi_S)\not=0$.考虑理想集合$\{\mathrm{Ann}(t\xi)\mid t\in S\}$.诺特条件允许我们取一个极大元$\mathrm{Ann}(t_0\xi)$,记$\eta=t_0\xi$,那么$\xi_S=t_0^{-1}\eta_S\not=0$,于是$\eta_S\not=0$.记$\mathfrak{b}=\{a\in A\mid a\eta\in N\}$,由于$N\subseteq M$是本性扩张,导致$\mathfrak{b}\eta=A\eta\cap N\not=0$.记$\mathfrak{b}=(b_1,\cdots,b_r)$.如果$b_i\eta_S=0$对任意$i$成立,那么存在$t\in S$使得$t\mathfrak{b}\eta=0$,但是按照$\eta$的极大性,导致$\mathrm{Ann}(\eta)=\mathrm{Ann}(t\eta)\supseteq\mathfrak{b}$,所以$\mathfrak{b}\eta=0$,这个矛盾导致存在某个指标$i$使得$b_i\eta_S=0$,但是$b_i\eta_S\in(S^{-1}A\eta_S)\cap S^{-1}N=(S^{-1}A\xi_S)\cap S^{-1}N$.得证.
    \end{proof}
\end{enumerate}

内射闭包.
\begin{enumerate}
	\item 给定左$R$模$M$和包含它的一个模$E$,那么如下条件等价.满足这个条件的左模$E$就称为$M$的内射闭包,记作$E(M)$或者$E_R(M)$.
	\begin{enumerate}
		\item $E$是$M$的极大本性扩张,这是说,$E$是$M$的本性扩张,并且$E$不存在真本性扩张.
		\item $E$是内射模并且是$M$的本性扩张.
		\item $E$是内射模并且不存在内射模$E'$满足$M\subset E'\subsetneqq E$.即,$E$是包含$M$的极小内射模.
	\end{enumerate}
	\begin{proof}
		
		不妨设$M$不是内射模,否则上述三个情况都有$E=M$.(a)推(b),按照本性扩张具有传递性,于是(a)说明$E$没有真本性扩张,于是$E$是内射模.(b)推(c),设$E$是内射的本性扩张,如果存在$M\subset E'\subsetneqq E$使得$E'$是内射模,这导致$E'$为$E$的直和项,于是$E=E'\oplus E''$,其中$E''\not=0$.这导致$M\cap E''\subset E'\cap E''=0$,矛盾.最后(c)推(a),如果$E$满足条件,需要证明$E$是极大的本性扩张.取$\mathscr{F}$为$E$的全体子模$S$使得它是$M$的本性扩张.赋予$\mathscr{F}$包含序,现在证明对每个升链$E_1\subset E_2\subset\cdots$具有上界$E_0=\cup_kE_k$.任取$E_0$的非0子模$S$,$S\cap i(M)=\cup_i(S\cap E_i)\cap i(M)$.必然存在一个$i$使得$S\cap E_i$非平凡,于是导致$S\cap i(M)$非平凡.于是按照Zorn引理得到$\mathscr{F}$具有极大元$E'$.现在证明$E'$不存在真本性扩张,若否,按照$E$是内射模得到交换图:
		$$\xymatrix{
			&E&\\
			0\ar[r]&E'\ar[u]^i\ar[r]_j&N\ar[ul]_h
		}$$
		
		按照$hj$为$h$在$E$上限制,是单射,于是$h$是单射,就有$h(N)$是$E'$的本性扩张,按照$E'$的极大性得到$E'=h(N)$.于是得到$E'$没有真本性扩张,于是$E'$是内射模.按照条件,没有内射中间模,导致$E=E'$,于是$E$是$M$的极大本性扩张.
	\end{proof}
    \item 每个模$M$都存在内射闭包,并且在同构意义下是唯一的(但是同构不是典范的).
    \begin{proof}
    	
    	内射闭包的存在性.只要让$M$嵌入到一个内射模$E$中,然后利用Zorn引理证明存在一个极小内射模包含$M$.任意两个内射闭包是同构的,这个同构可以取为限制在$M$上是恒等映射.事实上如果有两个内射闭包$i:M\to E,j:M\to E'$.按照$E'$和$E$内射,我们会得到一个提升$\varphi:E\to E'$,并且它是同构.
    \end{proof}
    \item 例子.
    \begin{enumerate}
    	\item 设$A$是整环,$K$是商域,那么$K$作为$A$模是内射的.另外$A\subseteq K$是本性扩张,就说明$K$是$A$的内射闭包.
    	\item 设$A$是DVR,取素元$x$,记商域$K$,记剩余域$k=A/xA$.那么$k\cong x^{-1}A/A\subseteq K/A$是本性扩张,并且$K/A$是内射$A$模:如果$0\not=I\subseteq A$是理想,可记$I=x^rA,r\ge0$,如果$f:I\to K/A$是$A$模同态,可把它延拓为$A\to K/A$,把1映射为$f(x^r)/x^r$在$K/A$中的像.于是$K/A$是$k$的内射闭包.
    \end{enumerate}
\end{enumerate}

设$A$是诺特交换环,设$\mathfrak{p},\mathfrak{q}\in\mathrm{Spec}A$.
\begin{enumerate}
	\item 内射闭包$E(A/\mathfrak{p})$是不可约分解的,也即它是非零,并且不能表示为两个非零的$A$子模的直和.
	\begin{proof}
		
		取$E(A/\mathfrak{p})$的两个非零子模$N_1$和$N_2$,因为$E(A/\mathfrak{p})$是$A/\mathfrak{p}$的本性扩张,有$I_i=N_i\cap(A/\mathfrak{p})\not=0$.并且$I_1,I_2$是整环$A/\mathfrak{p}$的两个非零理想,所以有$0\not=I_1I_2\subseteq I_1\cap I_2$,这说明$N_1\cap N_2\not=0$,即$E(A/\mathfrak{p})$是不可分解的.
	\end{proof}
    \item $A$的不可分解内射模总可以表示为$E(A/\mathfrak{p})$,其中$\mathfrak{p}\in\mathrm{Spec}A$.
    \begin{proof}
    	
    	设$N$是不可分解内射$A$模,特别的有$N\not=0$.取$\mathfrak{p}\in\mathrm{Ass}(N)$,那么有单射$A/\mathfrak{p}\subseteq N$.按照$N$是内射模,所以$A/\mathfrak{p}\subseteq N$就可以延拓为$\alpha:E(A/\mathfrak{p})\to N$.并且$\alpha$必须是单射,因为$E(A/\mathfrak{p})$是本性扩张保证从$\ker(\alpha)\cap(A/\mathfrak{p})=0$推出$\ker\alpha=0$.最后因为$E(A/\mathfrak{p})$也是内射的,就有$\alpha$分裂,但是按照$E(A/\mathfrak{p})$是本性扩张就有$\alpha:E(A/\mathfrak{p})\to N$必须是同构.
    \end{proof}
    \item 如果$x\in A-\mathfrak{p}$,那么数乘$x$诱导了$E(A/\mathfrak{p})$上的自同构.这是指按照内射模的定义存在提升映射$\varphi$,断言$\varphi$是同构:
    $$\xymatrix{0\ar[r]&A/\mathfrak{p}\ar[r]\ar[d]_{x}&E(A/\mathfrak{p})\ar[d]^{\varphi}\\0\ar[r]&A/\mathfrak{p}\ar[r]&E(A/\mathfrak{p})}$$
    \begin{proof}
    	
    	因为$x\not\in\mathfrak{p}$,所以$\varphi\mid_{A/\mathfrak{p}}$是单射.于是$\ker(\varphi)\cap(A/\mathfrak{p})=0$,按照本性扩张的定义就有$\ker\varphi=0$.所以$\varphi$是单射.但是由于$E(A/\mathfrak{p})$是内射模,这个$\varphi$就要分裂,迫使$\varphi$是同构.
    \end{proof}
    \item 如果$\mathfrak{p}\not=\mathfrak{q}$,那么$E(A/\mathfrak{p})\not\cong E(A/\mathfrak{q})$.
    \begin{proof}
    	
    	不妨设$\mathfrak{p}\not\subseteq\mathfrak{q}$,那么可取$x\in\mathfrak{p}-\mathfrak{q}$.那么数乘$x$诱导了$E(A/\mathfrak{q})$上的单射,但在$E(A/\mathfrak{p})$上不是单射,所以它们不同构.
    \end{proof}
    \item 对每个$\xi\in E(A/\mathfrak{p})$,存在一个依赖于$\xi$的整数$v$使得$\mathfrak{p}^v\xi=0$.
    \begin{proof}
    	
    	我们先断言有$\mathrm{Ass}(E(A/\mathfrak{p}))=\{\mathfrak{p}\}$.一方面因为$E(A/\mathfrak{p})$是$A/\mathfrak{p}$的本性扩张,至少有单射$A/\mathfrak{p}\to E(A/\mathfrak{p})$,所以$\mathfrak{p}\in\mathrm{Ass}(E(A/\mathfrak{p}))$.另外倘若有单射$\mathfrak{q}\to E(A/\mathfrak{p})$,按照第二条一样可证明$E(A/\mathfrak{q})\cong E(A/\mathfrak{p})$,迫使$\mathfrak{p}=\mathfrak{q}$.证明了断言.于是子模$A\xi\subseteq E(A/\mathfrak{p})$仍然以$\mathfrak{p}$为唯一的伴随素理想,所以$\mathrm{Ann}(\xi)$是$\mathfrak{p}$准素的.诺特条件就导致有$\mathfrak{p}^v\xi=0$.
    \end{proof}
    \item 如果$\mathfrak{q}\subseteq\mathfrak{p}$,那么$E(A/\mathfrak{q})$是一个$A_{\mathfrak{p}}$模,并且它是模$(A/\mathfrak{q})_{\mathfrak{p}}=A_{\mathfrak{p}}/\mathfrak{q}A_{\mathfrak{p}}$的内射闭包.
    \begin{proof}
    	
    	第三条就说明$E(A/\mathfrak{q})$的确是一个$A_{\mathfrak{p}}$模.我们有$A/\mathfrak{q}\subseteq(A/\mathfrak{q})_{\mathfrak{p}}\subseteq E(A/\mathfrak{q})$,因为$A/\mathfrak{q}\subseteq E(A/\mathfrak{q})$是本性扩张,有$(A/\mathfrak{q})_{\mathfrak{p}}\subseteq E(A/\mathfrak{q})$也是本性扩张.最后$E(A/\mathfrak{q})$作为$A$模是内射的,它本身还是$A_{\mathfrak{p}}$模,所以它作为$A_{\mathfrak{p}}$模也是内射的.于是$E(A/\mathfrak{q})$是$A_{\mathfrak{p}}$模$(A/\mathfrak{q})_{\mathfrak{p}}$的内射闭包.
    \end{proof}
\end{enumerate}

关于内射模分解为不可分解内射模的直和.设$A$是诺特环.
\begin{enumerate}
	\item 每个内射$A$模都是一族不可分解内射$A$模的直和.
	\begin{proof}
		
		设$\mathscr{F}=\{E_{\lambda}\}$是一族不可分解内射模,并且是内射$A$模$M$的子模.我们称$\mathscr{F}$是自由的,如果和$\sum_{\lambda}E_{\lambda}$是直和,也即对任意不同的有限项$\{E_{\lambda_1},\cdots E_{\lambda_n}\}$都有$E_{\lambda_1}\cap\sum_{i=2}^nE_{\lambda_i}=\{0\}$.设全体自由的族$\mathscr{F}$构成的集合为$\mathfrak{M}$,赋予包含序,那么它的每个链都有上界,即取并.所以按照Zorn引理$\mathscr{M}$就有极大元$\mathscr{F}_0$.记$N=\sum_{E\in\mathscr{F}_0}E\subseteq M$.那么$N$是内射模(内射模的直和是内射的),于是$N$在$M$中有直和补$M=N\oplus N'$.如果$N'\not=0$,由于$M$是内射的,导致$N'$也是内射的.任取$\mathfrak{p}\in\mathrm{Ass}_A(N')$,就有单射$A/\mathfrak{p}\subseteq N'$,于是$E(A/\mathfrak{p})\subseteq N'$.于是$N'$具有直和项$E'$同构于$E(A/\mathfrak{p})$.但是$\mathscr{F}_0\cup\{E'\}\in\mathfrak{M}$,就和$\mathscr{F}_0$的极大性矛盾.于是$N'=0$且$M=N$.
	\end{proof}
    \item 上一条中的分解是唯一的.具体地讲,如果$M$是内射模,如果有$M$的子模族$\{M_i\}$使得$M=\oplus_iM_i$,其中每个$M_i$都是不可分解内射模.那么对每个$\mathfrak{p}\in\mathrm{Spec}A$,在$\{M_i\}$中所有与$E(A/\mathfrak{p})$同构的子模的和$M(\mathfrak{p})$只与$M$和$\mathfrak{p}$有关,并且$\{M_i\}$中所有与$E(A/\mathfrak{p})$同构的子模的个数是$\dim_{\kappa(\mathfrak{p})}\mathrm{Hom}_{A_{\mathfrak{p}}}(\kappa(\mathfrak{p}),M_{\mathfrak{p}})$.
    \begin{proof}
    	
    	取$\mathfrak{p}\in\mathrm{Spec}A$,我们先证明对每个同构于$E(A/\mathfrak{p})$的$M$的子模$E$,有$E\subseteq M(\mathfrak{p})$.一旦这得证,就说明$M(\mathfrak{p})$是被所有这样的子模$E$生成的,于是$M(\mathfrak{p})$就只依赖$M$和$\mathfrak{p}$.首先按照上一条有$M=\oplus_{\mathfrak{q}\in\mathrm{Spec}A}M(\mathfrak{q})$.对每个$\xi\in E$,记$\xi=\xi_1+\cdots+\xi_n$,其中$\xi_i\in M(\mathfrak{p_i})$,约定$\mathfrak{p}=\mathfrak{p}_1$,约定$\mathfrak{p}_1,\cdots,\mathfrak{p}_n$是两两不同的素理想.记$\xi_1-\xi=\eta_1$和$\xi_i=\eta_i,2\le i\le n$.那么有$\eta_1+\cdots+\eta_n=0$.我们断言每个$\eta_i=0$.设$\mathfrak{p}_r$在$\mathfrak{p}_1,\cdots,\mathfrak{p}_n$中极小,那么对每个正整数$m$都有$(\prod_{j\not=r}\mathfrak{p}_j)^m\not\subseteq\mathfrak{p}_r$.所以如果取足够大的$m$,取$a\in(\prod_{j\not=r}\mathfrak{p}_j)^m-\mathfrak{p}_r$,就有$a\mathfrak{p}_j=0,\forall j\not=r$.另外数乘$a$是$M(\mathfrak{p}_r)$上的自同构,就导致$\eta_r=0$.于是对$n$归纳,反复取出$\mathfrak{p}_1,\cdots,\mathfrak{p}_n$中的极小元做相同的事情,就得到每个$\eta_i=0$.于是$\xi=\xi_1\in M(\mathfrak{p})$.
    	
    	\qquad
    	
    	接下来求$\{M_i\}$中和$E(A/\mathfrak{p})$同构的子模的个数.首先断言对每个$\mathfrak{q}\in\mathrm{Spec}A$有:
    	$$E(A/\mathfrak{q})_{\mathfrak{p}}=\left\{\begin{array}{cc}0&\mathfrak{q}\not\subseteq\mathfrak{p}\\E(A/\mathfrak{q})&\mathfrak{q}\subseteq\mathfrak{p}\end{array}\right.$$
    	
    	这是因为,如果$\mathfrak{q}\not\subseteq\mathfrak{p}$,取$a\in\mathfrak{q}-\mathfrak{p}$,那么对每个$x\in E(A/\mathfrak{q})$,就有足够大的正整数$v$使得$a^vx=0$.并且数乘$a$诱导在$E(A/\mathfrak{q})_{\mathfrak{p}}$上是双射.所以$x=0$,所以$E(A/\mathfrak{q})_{\mathfrak{p}}=0$.如果$\mathfrak{q}\subseteq\mathfrak{p}$,我们之前证明过$E(A/\mathfrak{q})$已经是$A_{\mathfrak{p}}$模.于是$E(A/\mathfrak{q})_{\mathfrak{p}}=E(A/\mathfrak{q})$.完成断言的证明.接下来有$M_{\mathfrak{p}}=\oplus_{\mathfrak{q}\in\mathrm{Spec}A}M(\mathfrak{q})_{\mathfrak{p}}=\oplus_{\mathfrak{q}\subseteq\mathfrak{p}}M(\mathfrak{q})_{\mathfrak{p}}=\oplus_{\mathfrak{q}\subseteq\mathfrak{p}}M(\mathfrak{q})$.所以用$A_{\mathfrak{p}}$代替$A$,可不妨设$(A,\mathfrak{p})$本身就是局部环.设$\mathfrak{q}\subsetneqq\mathfrak{p}$,取$y\in\mathfrak{p}-\mathfrak{q}$,那么数乘$y$在$M(\mathfrak{q})$上是单射,但是有$x\kappa(\mathfrak{p})=0$/于是有$\mathrm{Hom}_A(\kappa(\mathfrak{p}),M(\mathfrak{q}))=0$.于是得到$\mathrm{Hom}_A(\kappa(\mathfrak{p}),M)\cong\oplus_{\mathfrak{q}}\mathrm{Hom}_A(\kappa(\mathfrak{p}),M(\mathfrak{q}))=\mathrm{Hom}_A(\kappa(\mathfrak{p}),M(\mathfrak{p}))$.这里第一个同构依赖于$\kappa(\mathfrak{p})$作为$A_{\mathfrak{p}}$模可被单个元生成(不然按理说$\mathrm{Hom}$函子关于第二个位置是和直积可交换的).类似的,如果$M(\mathfrak{p})=\oplus_{i\in I_{\mathfrak{p}}}M_i$,那么有$\mathrm{Hom}_A(\kappa(\mathfrak{p}),M(\mathfrak{p}))\cong\oplus_{i\in I_{\mathfrak{p}}}\mathrm{Hom}_A(\kappa(\mathfrak{p}),M_i)$.最后按照$M_i\cong E(A/\mathfrak{p})$是$\kappa(\mathfrak{p})$的本性扩张,就有$\mathrm{Hom}_A(\kappa(\mathfrak{p}),M_i)\cong\mathrm{Hom}_A(\kappa(\mathfrak{p}),E(A/\mathfrak{p}))\cong\kappa(\mathfrak{p})$.综上我们得到$|I_{\mathfrak{p}}|=\dim_{\kappa(\mathfrak{p})}\mathrm{Hom}_A(\kappa(\mathfrak{p}),M)$.
    \end{proof}
\end{enumerate}

设$(A,\mathfrak{m},k)$是诺特局部环,设$E=E_A(k)$是$k$的内射闭包,对$A$模$M$,记$M'=\mathrm{Hom}_A(M,E)$.
\begin{enumerate}
	\item 如果$M$是$A$模,并且$0\not=x\in M$,那么存在$\varphi\in M'$使得$\varphi(x)\not=0$.特别的,典范映射$\theta_M:M\to M''=(M')'$,$\theta_M(x)(\varphi)=\varphi(x),x\in M,\varphi\in M'$是单射.
	\begin{proof}
		
		取$Ax\to E$是典范映射的复合$Ax\cong A/\mathrm{Ann}(x)\to A/\mathfrak{m}\to E$,按照$E$是内射的,这个映射$Ax\to E$就延拓为$\varphi:M\to E$,使得$\varphi(x)\not=0$.
	\end{proof}
    \item 如果$M$是有限长度$A$模,那么$l(M)=l(M')$,于是此时$\theta_M$是同构.
    \begin{proof}
    	
    	设$l(M)=n<\infty$,那么$M$有子模$M_1$的长度是$n-1$,并且有短正合列$0\to M_1\to M\to k\to0$.因为$E$是内射模,所以$\mathrm{Hom}_A(-,E)$是正合函子,所以有短正合列$0\to k'\to M'\to M_1'\to0$.因为$k\subseteq E$是本性扩张,有$\mathrm{Hom}_k(k,E)\cong k$.所以$M'$是有限长度模当且仅当$M_1'$是有限长度模.所以我们只要对$l(M)$归纳即得证.
    \end{proof}
    \item 设$\widehat{A}$是$A$关于$\mathfrak{m}$的完备化,那么$E$也是$\widehat{A}$模,并且它也是$k=\widehat{A}/\mathfrak{m}\widehat{A}$作为$\widehat{A}$模的内射闭包.
    \begin{proof}
    	
    	我们之前解释过$E=E(A/\mathfrak{m})$被$\mathfrak{m}$的某个次幂零化.于是$E$可以自然的视为$\widehat{A}$模,于是$k=A/\mathfrak{m}=\widehat{A}/\mathfrak{m}\widehat{A}\subseteq E$也是本性扩张.最后我们只要证明$E$作为$\widehat{A}$模仍然是内射模.设$F$是$E$作为$\widehat{A}$模的内射闭包,于是它也是$k=\widehat{A}/\mathfrak{m}\widehat{A}$的内射闭包.于是$F$被$\mathfrak{m}\widehat{A}$的某个次幂零化.按照$E$是内射$A$模,有$F$作为$A$模可以表示为$F=E\oplus C$,其中$C$是一个$A$模.但是$C\subseteq F$的每个元都被$\mathfrak{m}\widehat{A}$的某个次幂零化,所以$C$也是一个$\widehat{A}$模,所以$F=E\oplus C$是作为$\widehat{A}$模的直和分解,这导致$C=0$,于是$E=F$得证.
    \end{proof}
    \item 典范映射$\widehat{A}\to\mathrm{Hom}_{\widehat{A}}(E,E)$都是同构.这里第一个映射是数乘$a\in\widehat{A}$的$E$上的同态,第二个映射是把$\widehat{A}$模同态$E\to E$视为$A$模同态.
    \begin{proof}
    	
    	因为$E$中每个元可以被$\mathfrak{m}$的某个次幂零化,所以每个$A$模同态$E\to E$都可以视为$\widehat{A}$模同态,所以有$\mathrm{Hom}_{\widehat{A}}(E,E)=\mathrm{Hom}_A(E,E)$.接下来对每个正整数$v$,记$E_v=\{x\in E\mid\mathfrak{m}^vx=0\}$,那么有$(A/\mathfrak{m}^v)'=\mathrm{Hom}_A(A/\mathfrak{m}^v,E)\cong E_v$.并且有$\mathrm{Hom}_A(E_v,E_v)\cong\mathrm{Hom}_A(E_v,E)=E_v'\cong(A/\mathfrak{m}^v)''$.按照第二条有$(A/\mathfrak{m}^v)''\cong A/\mathfrak{m}^v$.于是我们证明了有典范同构$A/\mathfrak{m}^v\cong\mathrm{Hom}_A(E_v,E)$.接下来按照$E=\cup_vE_v$,所以可以写成正向极限$E=\lim\limits_{\rightarrow}E_v$,于是有$\mathrm{Hom}_A(E,E)=\mathrm{Hom}_A(\lim\limits_{\rightarrow}E_v,E)=\lim\limits_{\leftarrow}\mathrm{Hom}_A(E_v,E)=\lim\limits_{\leftarrow}A/\mathfrak{m}^v=\widehat{A}$.
    \end{proof}
    \item $E$作为$A$模和作为$\widehat{A}$模都是阿廷模.如果$A$本身是完备的,对诺特$A$模$M$,有$M'$是阿廷的,并且我们的$\theta_M$是同构;对阿廷$A$模$M$,有$M'$是诺特的,并且$\theta_M$是同构.(matsumura书上)
\end{enumerate}
\newpage
\subsection{平坦模}

先总结下平坦模的等价描述.给定环$A$,一个右$A$模$M$称为平坦的,如果它满足如下等价描述的任意一个:
\begin{enumerate}
	\item $M\otimes_A-$是正合函子,即对任意左$A$模的短正合列$\xymatrix{0\ar[r]&N'\ar[r]^i&N\ar[r]^{p}&N''\ar[r]&0}$,有交换群的短正合列:
	$$\xymatrix{0\ar[r]&M\otimes_AN'\ar[r]^{1\otimes i}&M\otimes_AN\ar[r]^{1\otimes p}&M\otimes_AN''\ar[r]&0}$$
	
	由于张量函子是右正合函子,于是这个条件等价于讲,对任意单同态$i:N'\to N$,有$1\otimes i:M\otimes_AN'\to M\otimes_AN$是单同态.
	\item $M$的特征模$M^*=\mathrm{Hom}_{\mathbb{Z}}(M,\mathbb{Q}/\mathbb{Z})$作为左$A$模是内射模.
	\item 理想准则:$M$是平坦右$A$模等价于$M$满足如下条件的任一:
	\begin{enumerate}
		\item 对每个左理想$I$,典范映射$M\otimes_AI\cong MI$,$m\otimes i\mapsto mi$是同构.
		\item 对每个有限生成左理想$J$,典范映射$M\otimes_AJ\cong MJ$,$m\otimes j\mapsto mj$是同构.
	\end{enumerate}
    
    理想准则还有几个简单的等价描述:
    \begin{enumerate}
    	\item 首先典范映射$M\otimes_AI\to MI$总是满射,于是上面条件中的典范映射是同构也可以改为典范映射$M\otimes_AI\to M$是单射.
    	\item 另外上述这个单射条件等价于有$\mathrm{Tor}_1^A(M,A/I)=0$.于是理想准则可以描述为,对每个有限生成左理想(或左理想)$I$,总有$\mathrm{Tor}_1^A(M,A/I)=0$.
    \end{enumerate}
	\item 给定右$R$模的短正合列$0\to K\to F\to A\to0$,如果$F$是平坦的,那么$A$平坦当且仅当对每个有限生成的左理想$I$,有$K\cap FI=KI$.
\end{enumerate}

按照正合性定义可以直接得到一些结论:
\begin{enumerate}
	\item 一族模的直和是平坦模当且仅当每个分量模是平坦模.
	\begin{proof}
		
		我们知道给定一族模同态$f_i:U_i\to V_i$诱导了直和的同态$f:\oplus_iU_i\to\oplus_iV_i$,并且$f$是单同态当且仅当每个$f_i$是单同态.现在任取单同态$i:A\to B$,任取一族模$\{M_i\}$,按照直和与张量可交换的自然性,有交换图:
		$$\xymatrix{(\oplus_iM_i)\otimes_RA\ar[d]\ar[r]^{1\otimes i}&(\otimes_iM_i)\otimes_RB\ar[d]\\ \oplus_i(M_i\otimes_RA)\ar[r]_f&\oplus_i(M_i\otimes_RB)}$$
		
		其中$f$是同态族$\{1_{M_i}\otimes i\}$诱导的直和的同态.那么$1\otimes i$是单射当且仅当$f$是单射,这又等价于每个$1_{M_i}\otimes i$是单同态,即$\otimes_iM_i$是平坦模当且仅当每个$M_i$是平坦模.
	\end{proof}
	\item 投射模是平坦模.首先左自由模总是平坦模,而左投射模总是某个左自由模的直和项,再利用一族模的直和是平坦的当且仅当每个分量平坦,这就得到左投射模是左平坦模.
	\item 分式化是平坦模.我们之前证明过分式化函子$S^{-1}(-)$和函子$S^{-1}R\otimes_R-$是自然同构的.并且证明过分式化函子是正合函子,于是$S^{-1}R$是平坦$R$模.
\end{enumerate}

平坦性和正向极限.
\begin{enumerate}
	\item 引理.给定左$R$模的单同态$i:A\to B$,设$M$是右$R$模,如果$u\in\ker(1_M\otimes i)\subset M\otimes_RA$,那么存在有限生成子模$N\subset M$和$u'\in\ker(1_N\otimes i)\subset N\otimes_RA$,满足$u=(k\otimes 1_A)(u')$,其中$k:N\to M$是包含映射.
	\begin{proof}
		
		设$u=\sum_jm_j\otimes a_j\in\ker(1_M\otimes i)$.于是在$M\otimes_RB$中有$0=\sum_jm_j\otimes ia_j$.现在设$M\times B$生成的自由左模是$F$,设$M\otimes_RB\cong F/S$,这里$S$由张量积定义中三种形式的$F$中元生成的子群.于是得到短正合列$0\to S\to F\to M\otimes_RB\to0$,其中倒数第二个同态$v$为$(m,b)\mapsto m\otimes b$.于是从正合性得到$\sum_jm_k\otimes ia_j=\sum_kv(m_k',b_k')$.现在定义$N$为由有限个$m_i$和有限个$m_k'$生成的子模,于是它是$M$的有限生成子模.现在取$u'=\sum_jm_j\otimes a_j\in N\otimes_RA$,那么有$(k\otimes1_A)(u')=u$,其中$k$是$N\to M$的包含映射.最后需要说明$(1_N\otimes i)(u')=0$,这是因为我们约定了$m_k'\in N$,保证了所有使得$(1_N\otimes i)(u')=0$的关系集都在$N$中,完成证明.	
	\end{proof}
    \item 如果右$R$模$M$的所有有限生成子模都是平坦的,那么$M$是平坦右模.
    \begin{proof}
    	
    	按照正合性定义,只需证明对每个左$R$模的单同态$i:A\to B$,诱导了交换群的单同态$1_M\otimes i:M\otimes_RA\to M\otimes_RB$.任取$u\in\ker(1_M\otimes i)$,上一条保证了存在$M$的有限生成子模$N$使得存在$u'\in\ker(1_N\otimes i)\subset N\otimes_RA$,并且$u=(k\otimes1_A)(u')$,其中$k:N\to M$是包含映射.现在按照条件有$1_N\otimes i$是单射,于是$u'=0$,这导致$u=0$,即$M$是平坦右$R$模.
    \end{proof}
    \item 正向极限保平坦性.更一般的,一个有向集为指标集的正向系统,如果每个模都是平坦模,那么正向极限是平坦模.
    \begin{proof}
    	
    	记指标集为有向集$I$,记正向系统为$\{F_i,\varphi_j^i\}$,其中每个$F_i$都是平坦模.任取左正合列$0\to A\to B$.按照每个$F_i$平坦,得到左正合列$0\to F_i\otimes_RA\to F_i\otimes_RB$.考虑如下交换图:
    	$$\xymatrix{
    		0\ar[r]&\lim_{\rightarrow}\left(F_i\otimes_RA\right)\ar[r]^{k^*}\ar[d]_{\varphi}&\lim_{\rightarrow}\left(F_i\otimes B\right)\ar[d]^{\psi}\\
    		0\ar[r]&\left(\lim_{\rightarrow}F_i\right)\otimes_RA\ar[r]^{1\otimes k}&\left(\lim_{\rightarrow}F_i\right)\otimes_RB
    	}$$
    	
    	这里$\varphi$和$\psi$是自然的同构.按照每个$F_i$平坦,于是$1_i\otimes k$都是单射.按照有向集上的正向极限保短正合列,得到上述交换图第一行是左正合的,于是$k^*$是单射.这就导致$1\otimes k=\psi k^*\varphi^{-1}$是单射的复合,于是单,于是$\lim_{\rightarrow}F_i$是平坦的.
    \end{proof}
    \item 正向极限保平坦模可以得到一些有趣的结论.
    \begin{enumerate}
    	\item 如果$R$是整环,取商域$Q$,那么$Q$作为$R$模是平坦的.
    	\begin{proof}
    		
    		考虑$Q$的全部具有形式$\langle1/r\rangle,r\not=0\in R$的循环子模.赋予包含序,按照$\langle1/s\rangle+\langle1/t\rangle\subset\langle1/st\rangle$说明全体这样的子模构成一个有向集.由于$Q$就是这个正向系统的正向极限.按照每个这样的循环子模是无挠的,于是同构于自由模$R$,于是正向极限$Q$是平坦$R$模.
    	\end{proof}
    	\item 一个模的所有有限生成子模如果是平坦的,那么这个模是平坦的.事实上全体有限生成子模在包含序下是一个有向集,它的正向极限就是原本的模.于是全部有限生成模平坦可以推出原本的模平坦.
    \end{enumerate}
    \item 上述定理存在逆命题Lazard定理:对任意环$A$,一个左$A$模$M$是平坦模当且仅当它是一个以有向集为指标集的由有限生成自由模构成的正向系统的正向极限.
\end{enumerate}

无挠模和平坦模的关系.
\begin{enumerate}
	\item 对整环上的模$M$,有如下短正合列,即每个模都是挠模和无挠模的扩张.$0\to\mathrm{tor}M\to M\to M/\mathrm{tor}M\to0$.
	\item 整环上平坦模都是无挠模.
	\begin{proof}
		
		给定一个整环$R$,如果$A$是平坦$R$模,取$R$的商域$Q$,按照$A$平坦,那么有$-\otimes_RA$是正合函子,于是从$0\to R\to Q$得到$0\to R\otimes_RA\to Q\otimes_RA$.注意到$R\otimes_RA\cong A$,现在说明$Q\otimes_RA$具有$Q$模结构,对每个非零元$r\in R$,左乘$r$是$Q\otimes_RA$上的双射,于是$R$模结构自然的延拓为$Q$模结构,于是$Q\otimes_RA$是线性空间,于是它是无挠模,于是它作为$R$模也是无挠模,按照无挠模子模是无挠模就得到$A$是无挠的.
	\end{proof}
    \item 在PID上平坦模和无挠模等价.至此我们看到PID上投射模等价于自由模,内射模等价于可除模,平坦模等价于无挠模.另外注意到PID上无限生成模未必是自由模(投射模),例如$\mathbb{Q}$作为$\mathbb{Z}$模.
    \begin{proof}
    	
    	设$R$是PID,上一条已经证明整环上的平坦模是无挠模.现在任取无挠$R$模$M$,为证明它是平坦模,只需说明它的每个有限生成子模都是平坦模.任取$M$的有限生成子模$N$,按照无挠模的子模无挠,得到$N$是无挠模.按照PID上有限生成模的结构定理,说明$N$是自由模,于是它是平坦模,这就得证.
    \end{proof}
\end{enumerate}

特征模.给定右$R$模$B$,称$B^*=\mathrm{Hom}_{\mathbb{Z}}(B,\mathbb{Q}/\mathbb{Z})$作为左$R$模为$B$的特征模.
\begin{enumerate}
	\item 模的正合性和对应的特征模的正合性是等价的.即$A\to B\to C$是正合列当且仅当$C^*\to B^*\to A^*$是正合列.
	\begin{proof}
		
		必要性,按照$\mathbb{Q}/\mathbb{Z}$是内射交换群,得到$\mathrm{Hom}_{\mathbb{Z}}(-,\mathbb{Q}/\mathbb{Z})$是右$R$模范畴到左$R$模范畴的逆变正合函子,于是必要性得证.
		
		对于充分性,记$A\to B$为$\alpha$,$B\to C$为$\beta$,那么如果$C^*\to B^*\to A^*$正合,等价于说$\ker\alpha^*=\mathrm{im}\beta^*$.
		
		先证明$\mathrm{im}\alpha\subset\ker\beta$,对$x\in A$,如果$\alpha x\not\in\ker\beta$,那么$\beta\circ\alpha(x)\not=0$,于是按照$\mathbb{Q}/\mathbb{Z}$是内射交换群,存在$f:C\to\mathbb{Q}/\mathbb{Z}$使得$f\circ\beta\circ\alpha(x)\not=0$,于是$f\in C^*$满足$f\circ\beta\circ\alpha\not=0$,这和$\alpha^*\beta^*=0$矛盾.
		
		最后证明$\ker\beta\subset\mathrm{im}\alpha$.如果$y\in\ker\beta$,但是$y\not\in\mathrm{im}\alpha$,那么$y+\mathrm{im}\alpha$是$B/\mathrm{im}\alpha$中非零元.于是按照内射交换群定义,存在$g:B/\mathrm{im}\alpha\to\mathbb{Q}/\mathbb{Z}$满足$g(y+\mathrm{im}\alpha)\not=0$.记典范同态$v:B\to B/\mathrm{im}\alpha$,取$g'=gv\in B^*$,那么$g'(y)\not=0$,但是$g'(\mathrm{im}\alpha)=0$,于是$0=g'\alpha=\alpha^*(g')$,于是$g'\in\ker\alpha^*=\mathrm{im}\beta^*$,于是存在$h'\in C^*$满足$g'(y)=h\beta(y)\not=0$,这和$y\in\ker\beta$矛盾.
	\end{proof}
    \item 由特征模可以给出平坦模的一个等价描述:一个右$R$模$B$是平坦模当且仅当它的特征模$B^*$是一个内射左$R$模.
    \begin{proof}
    	
    	有自然同构$\mathrm{Hom} _R(-,\mathrm{Hom}_{\mathbb{Z}}(B,\mathbb{Q}/\mathbb{Z}))\cong\mathrm{Hom}_{\mathbb{Z}}(B\otimes_R-,\mathbb{Q}/\mathbb{Z})$是自然同构的.如果$B$是平坦的,注意$\mathbb{Q}/\mathbb{Z}$是$\mathbb{Z}$上的内射模,于是右侧函子是两个正合函子的复合,导致右侧是正合函子,这使得左侧是正合函子,于是$B^*$必然是内射模.
    	
    	反过来,如果$B^*$是左内射模,任取左$R$模之间单的模同态$A'\to A$,按照自然同构$\mathrm{Hom}_R(-,B^*)\cong\mathrm{Hom}_R(-,\mathrm{Hom}_{\mathbb{Z}}(B,\mathbb{Q}/\mathbb{Z}))$,得到交换图:
    	$$\xymatrix{\mathrm{Hom}_R(A,B^*)\ar[rr]\ar[d]&&\mathrm{Hom}_R(A',B^*)\ar[d]\ar[r]&0\\
    		\mathrm{Hom}_{\mathbb{Z}}(B\otimes_RA,\mathbb{Q}/\mathbb{Z})\ar[rr]\ar[d]_{\cong}&&\mathrm{Hom}_{\mathbb{Z}}(B\otimes_RA',\mathbb{Q}/\mathbb{Z})\ar[r]\ar[d]&0\\
    		(B\otimes_RA)^*\ar[rr]&&(B\otimes_RA')^*\ar[r]&0}$$
    	
    	于是第一行的正合性传递给第三行,这就得到正合列正合列$0\to B\otimes_RA'\to B\otimes_RA$,于是$B$是平坦模.
    \end{proof}
    \item 这里我们给出左$R$模范畴上具有足够多内射对象的另一证明.任取左$R$模$M$,那么$M^*$是右$R$模,而$M^{**}$是左$R$模,存在典范单同态$M\to M^{**}$为$m\mapsto\varphi_m:M^*\to\mathbb{Q}/\mathbb{Z}$,$f\mapsto f(m)$.取自由右$R$模$F$使得存在满同态$F\to M^*$,那么上一条说明存在左$R$模的单同态$M^{**}\to F^*$.那么存在单同态的复合$M\to M^{**}\to F^*$,于是它也是单同态.最后按照$F$是平坦右$R$模,得到$F^*$是内射左$R$模.这就把$M$嵌入到了左内射模中.
\end{enumerate}

有限表现模和平坦模.
\begin{enumerate}
	\item 引理.给定左$R$模$X$,$(R,S)$双侧模$Y$和右$S$模$Z$,如果$X$是有限生成自由模,那么存在自然同构$\tau_{X,Y,Z}:\mathrm{Hom}_S(Y,Z)\otimes_RX\simeq\mathrm{Hom}_S(\mathrm{Hom}_R(X,Y),Z)$.这一事实即有限直和与Hom函子的可交换性.特别的如果$X$是有限表现模,那么有同构$\tau:Y^*\otimes_RX\to\mathrm{Hom}_R(X,Y)^*$.
	\begin{proof}
		
		取有限生成自由模$F',F$使得有正合列$F'\to F\to X\to0$.取$X=F$和$F'$,取$Z=\mathbb{Q}/\mathbb{Z}$.按照所提的自由同构,得到如下交换图,其中前两垂直映射是同构:
		$$\xymatrix{Y^*\otimes_RF'\ar[d]\ar[r]&Y^*\otimes_RF\ar[d]\ar[r]&Y^*\otimes_RX\ar[r]\ar[d]&0\\\mathrm{Hom}_R(F',Y)^*\ar[r]&\mathrm{Hom}_R(F,Y)^*\ar[r]&\mathrm{Hom}_R(X,Y)^*\ar[r]&0}$$
		
		按照张量积是右正合函子,说明第一行是正合列.按照$\mathrm{Hom}_R(-,Y)$是左正合函子,$\mathrm{Hom}_{\mathbb{Z}}(-,\mathbb{Q}/\mathbb{Z})$是正合函子,于是第二行是正合列.于是短五引理说明第三个垂直映射是同构.
	\end{proof}
	\item 我们曾经证明过\textbf{投射$\rightarrow$平坦}.这里我们证明\textbf{平坦 +有限表现$\rightarrow$投射},换句话讲有限表现模是平坦的等价于是投射的.
	\begin{proof}
		
		设$B$是有限表现的平坦左模,那么存在有限生成自由左模$F',F$使得存在正合列$F'\to F\to B\to0$.为了说明$B$是投射左模,只需验证对每个满同态$A\to A''$诱导的$\mathrm{Hom}(B,A)\to\mathrm{Hom}(B,A'')$是满同态.按照引理,对每个左$R$模$Y$,有自然同构$Y^*\otimes_RB\to\mathrm{Hom}_R(B,Y)^*$.于是得到交换图,其中两个垂直映射都是同构:
		$$\xymatrix{0\ar[r]&(A'')^*\otimes_RB\ar[r]\ar[d]&A^*\otimes_RB\ar[d]\\0\ar[r]&\mathrm{Hom}_R(B,A'')^*\ar[r]&\mathrm{Hom}_R(B,A)^*}$$
		
		按照$A\to A''$是满同态,得到$(A'')^*\to A^*$是单同态,于是从$B$平坦得到第一行是正合列.于是第一行的正合性传递给第二行.即$\mathrm{Hom}_R(B,A'')^*\to\mathrm{Hom}_R(B,A)^*$是单同态,于是$\mathrm{Hom}_R(B,A)\to\mathrm{Hom}_R(B,A'')$是满同态,这就得证.
	\end{proof}
	\item 我们说明过左诺特模上有限生成模和有限表现模是一致的,于是左诺特模上有限生成模是平坦的当且仅当是投射的.另外我们证明过投射模是有限生成模等价于有限表现模,于是有限生成模是投射模等价于是有限表现的平坦模.
\end{enumerate}

现在证明平坦性的理想准则.$M$是平坦右$A$模等价于$M$满足如下条件的任一:
\begin{enumerate}
	\item 对每个左理想$I$,存在同构$M\otimes_AI\cong MI$,$m\otimes a\mapsto ma$.
	\item 对每个有限生成左理想$J$,存在同构$M\otimes_AJ\cong MJ$,$m\otimes a\mapsto ma$.
\end{enumerate}
\begin{proof}
	
	先说明这两个条件互相等价.1自然能推出2,现在假设2成立,对任意的左理想$J$可以表示为它有限生成子理想的正向极限.对每个有限生成子理想$I$,有单射$M\otimes_AI\to M$.这族单射诱导的正向极限$M\otimes_AJ$到$M$的同态也是单射.此即1成立.
	
	假设$M$是平坦右$A$模,把它张量在左$A$模的短正合列$0\to J\to A\to A/J\to0$就得到$M\otimes_AJ\to M$是单射.现在反过来设条件成立,需要验证$M$是平坦右$A$模.
	
	任取左$A$模$N'$的子模$N$.需要验证诱导的同态$M\otimes_AN\to M\otimes_AN'$是单射.我们断言仅需验证$N'/N$是有限生成模的情况就能得到一般情况.这是因为对于一般情况,如果存在$M\otimes_AN$中的元$\sum_im_i\otimes n_i$,它在$M\otimes_AN'$中为零.按照张量积的具体构造,这个为零的元是若干关系集(例如$a(m\otimes n)-am\otimes n$,$(m_1+m_2)\otimes n-m_1\otimes n-n_2\otimes n$等)的$A$线性组合.涉及到的关系集只有有限个,取这些关系集中出现的$N'$中的元,这也是一个有限集,记作$S$,那么按照假设,$M\otimes_AN\to M\otimes_A(N+AS)$是单射,并且$\sum_im_i\otimes n_i$在$M\otimes(N+AS)$中为零,于是$\sum_im_i\otimes n_i$在$M\otimes_AN$中为零,于是$M\otimes_AN\to M\otimes_AN'$是单射.
	
	至此问题归结为证明$M\otimes_AN\to M\otimes_A(N+AS)$是单射,其中$S$是有限集.而这可以归结为$S$是单元集合的情况(比方说$S=\{x_1,x_2,\cdots,x_n\}$,那么这个映射可分解为$M\otimes_AN\to M\otimes_A(N+Ax_1)\to\cdots\to M\otimes_A(N+AS)$).我们有同构$(N+Ax)/N\cong A/I$,其中$I=(N:x)_A$,于是有左$A$模的短正合列$0\to N\to N+Ax\to A/I\to0$,于是得到正合列$\mathrm{Tor}_1^A(M,A/I)\to M\otimes_AN\to M\otimes_A(N+Ax)$.而从短正合列$0\to I\to A\to A/I\to0$得到$0=\mathrm{Tor}_1^A(M,A)\to\mathrm{Tor}_1^A(M,A/I)\to M\otimes_AI\to M\otimes_AA$,按照条件$M\otimes_AI\to M$是单射,得到$\mathrm{Tor}_1^A(M,A/I)=0$,于是$M\otimes_AN\to M\otimes_A(N+Ax)$是单射,完成证明.
\end{proof}

这里整理下平坦模和其他算子的关系.
\begin{enumerate}
	\item 平坦模的子模未必平坦.取$R=\mathbb{Z}/4$,取理想$I=2\mathbb{Z}/4$,那么包含映射$I\hookrightarrow R$诱导了模同态$I\otimes_RI\to I\otimes_RR\cong I$,即$r\otimes s\mapsto rs$,那么这是一个零同态.但是$I\otimes_RI\cong\mathbb{Z}/2$非零,于是$I$不是平坦的.
	\item 平坦模的商未必平坦.例如$\mathbb{Z}/n$总不是平坦的,$n\not=0$,因为将它作用到包含映射$\mathbb{Z}\to\mathbb{Q}$上,有$\mathbb{Z}/n\otimes_{\mathbb{Z}}\mathbb{Z}=\mathbb{Z}/n$和$\mathbb{Z}/n\otimes_{\mathbb{Z}}\mathbb{Q}=0$,于是诱导的模同态不是单射.
	\item 已经证明过一族模的直和平坦当且仅当每个分量平坦.另外由于无限直积和张量积未必可交换,所以不能模仿直和的情况证明一族模的直积平坦当且仅当每个分量平坦,并且这件事也是不对的.事实上一个环满足所有左平坦模的直积平坦当且仅当它是左遗传环,即每个有限生成左理想是优先表现的.
	\item 如果$M,N$分别是平坦右$R$模和平坦左$R$模,那么$M\otimes_RN$是平坦交换群;如果$R$交换,此时$M\otimes_RN$是平坦$R$模.事实上按照张量积的结合律,说明$(M\otimes_RN)\otimes_{\mathbb{Z}}-$自然同构于$M\otimes_R(N\otimes_{\mathbb{Z}}-)$,其中$N$视为$(R,\mathbb{Z})$模.于是它是两个正合函子$N\otimes_{\mathbb{Z}}-$和$M\otimes_R-$的复合,于是它正合.
	\item 给定左$R$模的短正合列$0\to A\to B\to C\to0$,那么$A,C$平坦推出$B$平坦,$B,C$平坦推出$A$平坦.这个结论用$\mathrm{Tor}$函子是直接的,对每个右$R$模$D$,取短正合列诱导的长正合列,那么从$B,C$平坦推出$\mathrm{Tor}_1^R(D,A)=0$,从$A,C$平坦推出$\mathrm{Tor}_1^R(D,B)=0$.于是这两种情况下分别有$A$平坦和$B$平坦.但是注意$A,B$平坦不能推出$C$平坦.例如$0\to\mathbb{Z}\to\mathbb{Z}\to\mathbb{Z}/2\to0$,但是$\mathbb{Z}/2$不是无挠的,于是不是$\mathbb{Z}$上的平坦模.
\end{enumerate}

忠实平坦模.一个右$R$模$B$称为忠实平坦模,如果它满足如下等价描述中的任一:
\begin{enumerate}
	\item $A'\to A\to A''$是正合列当且仅当$B\otimes_RA'\to B\otimes_RA\to B\otimes_RA''$是正合列.
	\item $0\to A'\to A\to A''\to0$是短正合列当且仅当$0\to B\otimes_RA'\to B\otimes_RA\to B\otimes_RA''\to0$是短正合列.
	\item $B$是平坦模并且只要$X$不是零模就有$B\otimes_RX$不是零模.
	\item $B$是平坦模并且对每个真左理想$I$有$B\otimes_R(R/I)$非零模.
	\item $B$是平坦模并且对每个真左极大理想$m$有$B\otimes_R(R/m)$非零模.按照$B\otimes_R(R/m)\cong B/mB$,这等价于$mB\not=B$对每个左极大理想$m$成立.
\end{enumerate}
\begin{proof}
	
	1推2是直接的,2推3只要注意到如果$B\otimes_RX$是零模,那么条件得到$0\to X\to0\to0\to0$是短正合列,于是$X$是零模.3推4只要取$X=R/I$.4推5平凡.
	
	5推3.设$X$是非零模,取非零元$x\in X$,于是$\mathrm{Ann}(x)$是$R$的左真理想.设$\alpha:R/\mathrm{Ann}(x)\to X$为$r+\mathrm{Ann}(x)\mapsto rx$.按照$B$是平坦模,得到单同态$1_B\otimes\alpha:B\otimes_RR/\mathrm{Ann}(x)\to B\otimes_RX$.现在取包含$\mathrm{Ann}(x)$的左极大理想$m$,从$mB\not=B$得到$B\otimes_RR/\mathrm{Ann}(x)\cong B/\mathrm{Ann}(x)B\not=0$.
	
	3推1.$B$是平坦模说明正合列$A'\to A\to A''$诱导出$B\otimes_RA'\to B\otimes_RA\to B\otimes_RA''$.现在假设后者是正合列,需要验证前者是正合列.先说明$\mathrm{im}\alpha\subset\ker\beta$.按照$B$是平坦模,$B\otimes_R-$是正合函子,于是它与核余核可交换,按照像是余核的核,这说明有$B\otimes_R(\mathrm{im}(\beta\circ\alpha))=\mathrm{im}((1\otimes\beta)\circ(1\otimes\alpha))=0$,于是条件说明$\mathrm{\beta\circ\alpha}=0$,这就得到$\mathrm{im}\alpha\subset\ker\beta$.接下来注意到右正合保余核,于是右正合函子和商可交换,这导致$B\otimes_R(\ker\beta/\mathrm{im}\alpha)\cong\ker(1\otimes\beta)/\mathrm{im}(1\otimes\alpha)=0$.于是条件得到$\ker\beta/\mathrm{im}\alpha=0$,即$\ker\beta=\mathrm{im}\alpha$.
\end{proof}

这里给出忠实平坦性的一些例子和性质.
\begin{enumerate}
	\item $R[x]$是一个忠实平坦$R$模.首先$R[x]$作为左$R$模可视为可数秩的左自由模,于是它是平坦模.现在任取左$R$模$X$,设$R[x]\otimes_RX=0$.按照函子$-\otimes_RX$是右正合的,于是它和商可交换,于是$M\cong R\otimes_RM\cong(R[x]/(x))\otimes_RX\cong R[x]\otimes_RX/(x)\otimes_RX$.这导致$X=0$.
	\item $\mathbb{Q}$不是忠实平坦交换群,因为$\mathbb{Q}\otimes_{\mathbb{Z}}\mathbb{Z}/n=0$.按照等价定义,以及PID上平坦模等价于无挠模,说明交换群$G$是忠实平坦的当且仅当它无挠并且对每个素数$p$有$pG\not=G$.
	\item 如果$A$是平坦$R$模,$B$是忠实平坦$R$模,那么$A\oplus B$是忠实平坦的.首先$A,B$平坦推出$A\oplus B$平坦.接下来假设右$R$模$X$满足$(A\oplus B)\otimes_RX=0$,按照直和与张量积的交换性,得到$(A\otimes_RX)\oplus(B\otimes_RX)=0$.于是特别的有$B\otimes_RX=0$,从$B$忠实平坦就得到$X=0$.
	\item 更一般的,给定短正合列$0\to A\to B\to C\to0$,设$A,C$均平坦,并且其中某个是忠实平坦的,那么$B$就是忠实平坦的.事实上按照平坦模总是投射模,说明$C$是投射模.按照投射模的分离定义,得到这个短正合列实际上是分离的,于是$B\cong A\oplus C$,于是上一条说明此时$B$是忠实平坦的.
\end{enumerate}

纯正合列.回顾本节和前两节的内容,对于函子的正合性的缺失,利用寻找特殊的能够完善正合性的模来弥补这一缺失.在这里提供一种新的思路,即寻找可以保证函子正合性的特殊正合列.称一个右$R$模的正合列$0\to B'\to B\to B''\to0$是纯正合列,如果对任意左模$A$,以函子$A\otimes_R-$作用这个正合列仍然保持正合性.称$B'$嵌入到$B$的子模为$B$的纯子模.

那么首先,分离的短正合列是纯正合列.
\begin{proof}
	
	给定分离的短正合列$0\to B'\to B\to B''\to0$,对任意的模$A$,记$B'\to B$为$f$,那么需要证明$1_A\otimes f:A\otimes B'\to A\otimes B$是单射.取$\sum a_i\otimes b_i'\in\ker 1_A\otimes f$,那么按照短正合列分离,存在模同态$h:B\to B'$满足$hf=1_{B'}$,于是,$(1_A\otimes h)\circ(1_A\otimes f)=1_A\otimes 1_{B'}$,于是$0=(1_A\otimes h)\circ(1_A\otimes f)(\sum a_i\otimes b_i')=\sum a_i\otimes b_i'$,完成证明.
\end{proof}

一个左$R$模$B''$是平坦模当且仅当,对每个左$R$模的正合列$0\to B'\to B\to B''\to0$是纯正合列.
\begin{proof}
	
	必要性.取右$R$模的正合列$A'\to A\to A''\to 0$,按照张量积是一个双侧函子,对每个分量都是右正合的,如果取$A$是右自由$R$模,那么得到如下交换图:
	$$\xymatrix{
		A'\otimes_RB'\ar[r]\ar[d]&A'\otimes_RB\ar[r]\ar[d]&A'\otimes_RB''\ar[r]\ar[d]&0\\
		A\otimes_RB'\ar[r]\ar[d]&A\otimes_RB\ar[r]\ar[d]&A\otimes_RB''\ar[r]\ar[d]&0\\
		A''\otimes_RB'\ar[r]&A''\otimes_RB\ar[r]&A''\otimes_RB''\ar[r]&0
	}$$
	
	按照$A$自由和$B''$平坦,得到第三列与第二行补上左侧$0\to$时候是正合.不难由此推出$A''\otimes_RB'\to A''\otimes_RB$是单射,于是第三行补上左侧$0\to$是正合列.
	
	现在假设固定$B''$,对每个短正合列$0\to B'\to B\to B''\to0$是纯正合列,下面证明$B''$是平坦的.事实上,可以取$B$为自由模的正合列.现在取短正合列$0\to A'\to A\to A''\to0$,考虑如上交换图.那么现在第二列补上左侧成为短正合列,并且按照条件三行都会在补上左侧情况下得到短正合列.不难由此推出第三列补上左侧是短正合列,于是得到$B''$是平坦的.
\end{proof}

这里借助Tor函子给出上一定理的第二种证明.假设$B''$是平坦模,对任意右$R$模$A$,诱导的长正合列为$\mathrm{Tor}_1^R(A,B'')\to A\otimes_RB'\to A\otimes_RB\to A\otimes_RB''\to0$.从$B''$平坦得到第一项为零,于是$0\to B'\to B\to B''\to0$是纯正合列.充分性,假设每个第三项固定为$B''$的短正合列是纯正合列,取$B$是自由模的短正合列,对每个右$R$模$A$,得到正合列$\mathrm{Tor}_1^R(A,B)\to\mathrm{Tor}_1^R(A,B'')\to A\otimes_RB'\to A\otimes_RB$.按照$B$自由得到第一项为零,按照该短正合列是纯正合列得到$1\otimes i:A\otimes_RB'\to A\otimes_RB$是单射,这得到$\mathrm{Tor}_1^R(A,B'')=0$,于是得到$B''$是平坦模.

根据这个定理,我们可以给出一个纯短正合列不是分离短正合列的例子.在交换群范畴下,平坦等价于无挠,于是对每个交换群$G$,$0\to\mathrm{tor}G\to G\to G/\mathrm{tor}G\to0$是纯短正合列,但是它不总是分离短正合列.

\newpage
\subsection{特殊环}
\subsubsection{半单环}

最简单的模是域上的线性空间,这时候所有模都是自由模,投射模,内射模和平坦模.我们首要描述的是所有满足这样条件的环.称一个环是左(右)半单环,如果它作为自身的左(右)模是半单的.
\begin{enumerate}
	\item Wedderburn-Artin定理给出了半单环的结构定理.Maschke定理给出了群环半单性的完全描述.见结合代数内容.
	\item 我们知道环作为自身的模,那么子模就是理想.于是半单环就是说环是若干极小非0理想的直和.但是在这种情况下,直和必然是只有有限个直和项的.即如果环$R$是若干左理想的直和$R=\oplus_{i\in I}L_i$,那么只会有有限个$L_i$是非0的.
	\begin{proof}
		
		对任意$R$中元按照这个直和分解可以写作有限元的和.那么特别的单位1可以写作和$1=e_1+\cdots+e_n$,其中$e_i\in L_i$,那么对任意$a\in$某个不同于$L_1,\cdots,L_n$的$L_j$,就有$a=a1=ae_1+\cdots+ae_n$,但是每个$ae_i$必然是0,这导致$L_j=0$.得证.
	\end{proof}
    \item 特别的,上一条说明左(右)半单环是左(右)诺特环.
    \item 事实上一个环是左半单环等价于是右半单环,今后我们略去左右的区别.
\end{enumerate}

一个环上的如下条件是等价的:
\begin{enumerate}
	\item $R$是半单环.
	\item 每个左(右)$R$模都是半单模.
	\item 每个左(右)$R$模是内射的.
	\item 每个左(右)$R$模的短正合列都是分离的.
	\item 每个左(右)$R$模是投射的.
\end{enumerate}
\begin{proof}
	
	1推2,按照$R$半单,它作为自身的模是半单的,这导致$R$上自由模都是半单的,而$M$必然是自由模的商,于是也是半单的.
	
	2推3,给定一个左$R$模$E$,$E$是内射模当且仅当任意的正合列$0\to E\to B\to C\to0$分裂.但是按照假设$B$是半单模,于是任意子模都是直和项,于是$E$是内射模.
	
	3推4,直接按照刚刚2到3的方法.
	
	4推5,给定模$M$,那么存在自由模$F$满足$0\to F'\to F\to M\to 0$,于是这说明$M$是自由模的直和项,于是$M$是投射的.
	
	5推1,给定$R$的一个左理想$I$,于是存在正合列$0\to I\to R\to R/I\to0$,按照条件,$R/I$是投射模,于是这短正合列分裂,于是$I$是$R$的直和项,于是$R$作为自身左模是半单的.
\end{proof}
\subsubsection{绝对平坦环}

现在讨论所有模都是平坦模的环.称一个环是冯诺依曼正则环或者绝对平坦环,如果对任意的元素$r$,存在一个元素$r'$满足$rr'r=r$.此时称$r'$是$r$的广义逆.例子:
\begin{enumerate}
	\item Boolean环,即元素都满足$r^2=r$的环,直接取$r'=r$.
	\item 我们举一个稍微复杂的例子,考虑域上一个模$V$,记自同态环$R=\rm{End}_k(V)$,我们断言$R$是一个冯诺依曼正则环.给定一个线性变换$\varphi:V\to V$,存在直和分解$V=\ker\varphi\oplus W$.记$\ker\varphi$的一组基为$X$,记$W$的一组基为$Y$,那么$X\cup Y$是$V$上一组基.由$W\cap\ker\varphi=0$知$\varphi(Y)$是一组线性无关组,那么它可以延拓为一组$V$的基$\varphi(Y)\cup Z$,取$\varphi':V\to V$为,$\varphi'(\varphi(y))=y$,而$\varphi'(z)=0,\forall z\in Z$,那么得到$\varphi\varphi'\varphi=\varphi$.
	\item 注意上一条例子说明绝对平坦环未必满足维数不变性,但是半单环是诺特环,于是满足维数不变性.
\end{enumerate}

引理.如果$R$是冯诺依曼正则环,那么每个有限生成的左(右)理想都是主理想,并且被一个幂等元生成.
\begin{proof}
	
	如果给定一个主的左理想$Ra$,取$a$的广义逆为$a'$,那么$aa'$是一个幂等元,并且$Ra=R(aa')$.
	
	现在证明有限生成的左理想$I$都是主理想,按照归纳只要证明对任意$a,b\in R$有$Ra+Rb$是主理想.存在一个幂等元$e$满足$Ra=Re$.注意到$Ra+Rb=Re+Rb(1-e)$,现在取幂等元$f$满足$Rb(1-e)=Rf$,那么存在$r\in R$满足$f=rb(1-e)$,那么$fe=0$,注意到这不说明$ef=0$,取$g=(1-e)f$,那么$g^2=g$,并且$eg=ge=0$,并且$Rg=Rf$,于是$Ra+Rb=Re+Rg$,并且一方面$R(e+g)\subset Re+Rg$,另一方面,对任意$u,v\in R$有$(ue+vg)(e+g)=ue+vg$,于是$R(e+g)=Re+Rg$,并且$e+g$是幂等元.
\end{proof}

一个环是冯诺依曼正则环当且仅当它的每个右模都是平坦模,也等价于每个左模都是平坦模.
\begin{proof}
	
	一方面给定冯诺依曼正则环$R$,并且$B$是右$R$模,取短正合列$0\to K\to F\to B\to0$,其中$F$是自由模,那么我们断言给定任意有限生成左理想$I$有$KI=K\cap FI$,于是按照平坦模的等价描述就得到$B$是平坦模.按照引理$I$是主理想,记作$Ra$,其中$a$是一个幂零元,需要证明如果$k\in K$,$k=fa,f\in F$,那么$k\in Ka$,但是有$k=fa=faa'a=ka'a\in Ka$,得证.
	
	反过来,任取$a\in R$,那么按照条件,$R/aR$是平坦的,考虑短正合列$0\to aR\to R\to R/aR\to0$,按照$R$是平坦模,于是$(aR)I=aR\cap RI$,于是如果取$I=Ra$,得到$aRa=aR\cap Ra$,于是存在$a'$满足$aa'a=a$.
\end{proof}

关于冯诺依曼正则环有如下若干等价描述:(这里把所有左改成右同样都是等价的)
\begin{enumerate}
	\item $R$是冯诺依曼正则环,即每个元都存在广义逆.
	\item 每个主左理想都是幂等的.
	\item 每个有限生成左理想都是幂等的.
	\item 每个主左理想是左$R$模$R$的直和项.
	\item 每个有限生成左理想是左$R$模$R$的直和项.
	\item 每个左投射模$P$的有限生成左子模都是$P$的直和项.
	\item 每个左$R$模都是平坦左$R$模(这个性质通常称为绝对平坦性).
	\item 每个$R$模的短正合列都是纯正合列.
\end{enumerate}

对于交换环$R$,它是冯诺依曼正则环还有如下两个等价描述:
\begin{enumerate}
	\item $R$是零维的简化环.即它的素理想都是极大理想,并且不存在非平凡的幂零元.
	\item $R$在每个极大理想处的局部化都是域.
\end{enumerate}
\subsubsection{遗传环}

前两节我们证明了:环的所有左/右模都是投射模等价于环是左/右半单环;环的所有左/右模都是内射模等价于环是左/右半单环;环的所有左(右)模都是平坦模等价于环是冯诺依曼正则环.特别的,半单环总是冯诺依曼正则环.如果我们要求环上的全体理想具有统一的性质会如何?首先,一个环$R$上所有左/右理想是内射模当且仅当环是左/右半单环.
\begin{proof}
	
	一方面按照每个左理想都是内射模,于是每个左理想是$R$的直和项,于是环是左半单的.反过来如果$R$半单,那么每个模都是内射模.
\end{proof}

一个环称为左遗传环,如果每个左理想都是投射模,右遗传环就是每个右理想都是投射模.称每个理想都是投射模的整环为戴德金整环.
\begin{enumerate}
	\item 半单环必然是左右遗传环.
	\item Small给出过一个右诺特环不是左诺特环的例子,它是右遗传环但不是左遗传环.
	\item PID的每个非零理想作为模都同构于$R$本身,于是PID是遗传环,于是按照定义PID都是戴德金整环.反过来戴德金整环不都是PID,我们在下文会看到戴德金整环是PID当且仅当是UFD,当且仅当类数为1,但是存在类数为2的戴德金整环,例如$\mathbb{Z}(\sqrt{-5})$.
	\item 取$k$为域,取两个不交换的未定元$x,y$,那么$k[x,y]$同时是左右遗传环,但是不是左或右诺特环.不过我们在下文会看到戴德金整环总是诺特的.
\end{enumerate}

左遗传环上每个自由左模$F$的子模同构于若干左理想的直和.
\begin{proof}
	
	取自由模$F$的一组基$\{x_k:k\in K\}$.赋予$K$以良序.记$K$上最小元为0,记$F_0=\{0\}$.对每个$k\in K$,归纳定义$F_k=\oplus_{i<k}Rx_i$,记$F_k'=\oplus_{i\le k}Rx_i=R_k\oplus Rx_k$.于是有$F_0'=F_0$.取$F$的子模$A$,对任意的$a\in A\cap F_k'$,存在唯一的表示$a=b+rx_k$,其中$b\in F_k$.那么得到一个同态$\varphi_k:A\cap F_k'\to R$为$a\mapsto r$.于是得到短正合列$\xymatrix{0\ar[r]&A\cap F_k\ar[r]&A\cap F_k'\ar[r]&\mathrm{im}\varphi_k\ar[r]&0}$.按照左遗传的定义,左理想$\mathrm{im}\varphi_k$是投射模,于是得到同构$A\cap F_k'\sim (A\cap F_k)\oplus C_k$.其中$C_k\sim\mathrm{im}\varphi_k$.现在我们断言$A=\oplus_ {k\in K}C_k$就完成证明.
	
	一方面,记$C=\oplus_{k\in K}C_k$,如果存在某些$a\in A$并不包含在$C$中,记$\mu(a)$表示最小的$k\in K$使得$a\in F_k'$.那么集合$J=\{\mu(a)\mid a\in A-C\}$非空,于是可以取最小元$j$,记$y\in A-C$使得$\mu(y)=j$.于是有$y\in A\cap F_j'=(A\cap F_j)\oplus C_j$.于是有$y=b+c$,$b\in A\cap F_j$,$c\in C_j$.于是$b=y-c$,那么$b\not\in C$,但是$\mu(b)<j$,这矛盾.
	
	另一方面,如果存在$c_1+\cdots+c_n=0$,其中$c_i\in C_{k_i}$,不妨设$k_1<\cdots<k_n$.并且按照良序约定取的$c_i$是使得$k_n$最小的情况.那么$c_1+\cdots+c_ {n-1}=-c_n\in(A\cap F_{k_n})\cap C_{k_n}=\{0\}$.这导致$c_n=0$,和极小性矛盾.
\end{proof}

一个环是左遗传环等价于如下任一条:
\begin{enumerate}
	\item 每个左投射模的子模都是投射的.
	\item 每个左内射模的商都是内射的.
\end{enumerate}
\begin{proof}
	
	先证明左遗传环上每个投射左模的子模都是投射左模.取左遗传环上的投射左模$P$,那么它是某个自由左模的直和项.于是$P$的任意子左模$S$是某个自由左模的子模,按照上一定理$S$是若干左理想的直和,但是左遗传环上每个左理想都是投射的,于是$S$是若干投射模的直和,于是它是投射模.
	
	反过来$R$自身是自由左模,那么必然投射,于是全部左理想都是投射的.这就证明了和1等价.
	
	现在证明2的等价性,这里只证明3推2,至于2推3是类似的.考虑如下交换图:
	$$\xymatrix{
		P\ar[d]\ar[dr]&P'\ar[l]_j\ar[d]^f\ar[dl]&0\ar[l]\\
		Q\ar[r]_r&Q''\ar[r]&0
	}$$
	
	其中$P$投射,$Q$内射,那么按照条件$Q$的商$Q''$是内射模,利用投射模的提升定义,只需证明,存在$P'\to Q$提升了$P'\to Q''$.因为存在$P\to Q''$提升了$P'\to Q''$.再按照$P$是投射的,存在$P\to Q$提升了$P\to Q''$,于是复合$P'\to P\to Q$满足要求.
\end{proof}

如果$R$是左诺特环,那么每个左理想是平坦的当且仅当$R$是左遗传环.
\begin{proof}
	
	必要性,每个左理想是有限生成的,于是$I$平坦能推出它是投射的,于是按照定义$R$是左遗传环.充分性,如果$R$是左遗传的,那么每个左理想是投射的,投射模都是平坦的.
\end{proof}


\subsubsection{半遗传环和Pr\"ufer环}



现在来讨论有限生成理想特殊的环.一个环$R$称为左/右半遗传的,如果它的每个有限生成左/右理想都是投射模.称一个整环为Prufer环,如果它的每个有限生成理想都是投射模.称一个环是Bezout环,如果它的每个有限生成理想都是主理想.
\begin{enumerate}
	\item 左遗传环总是左半遗传环,在左诺特环上两个概念一致.
	\item 存在左半遗传环不是右半遗传环,见gtm189.但是对于单侧的诺特环,它同时是左右半遗传环或者同时不是.
	\item 冯诺依曼正则环总是左和右半遗传环.因为冯诺依曼正则环上每个有限生成的左/右理想都是主理想,记作$I=(a)$,那么有$a'$使得$aa'a=a$.取同态$\varphi:R\to I$为$r\mapsto ra'a$,这在$I$上的限制是恒等,导致$I$是$R$的一个直和项,于是$I$是投射的.
	\item 赋值环是满足对任意元$a,b$有$a\mid b$或$b\mid a$的整环,这导致它是一个Bezout环.
	\item 设$X$是一个非紧的黎曼曲面,设$R$为其上全部复值解析函数,那么这是一个Bezout环;$C$上代数整数环是Bezout环.
\end{enumerate}

环$R$是左半遗传环,那么每个自由模的有限生成子模是有限个有限生成左理想的直和.
\begin{proof}
	
	取$R$上左自由模$F$,取基$\{x_k\mid k\in K\}$,取$A=\langle a_1,\cdots,a_m\rangle$为一个有限生成子模.取每个$a_i$表示为$x_k$线性组合,这只依赖于有限个$x_k$,记作$X$,那么$A\subset \langle X\rangle$.并且$\langle X\rangle$是一个有限生成的自由模,于是可以不妨设$F$是有限生成的自由模,基为$\{x_1,\cdots,x_n\}$.现在对$n$归纳,证明$A$同构于有限个有限生成左理想的直和.如果$n=1$,那么$A$自身就是一个有限生成左理想.假设$n>1$,记$B=A\cap(Rx_1+\cdots+Rx_{n-1})$,按照归纳假设,$B$是有限个有限生成左理想的直和.对$a\in A$可以唯一的写作$b+rx_n$,其中$b\in B$,$r\in R$.定义$\varphi:A\to R$为$a\mapsto r$.那么$\mathrm{im}\varphi$是一个有限生成左理想.考虑短正合列$0\to B\to A\to\mathrm{im}\varphi\to0$.按照$\mathrm{im}\varphi$投射得到这个短正合列分裂,于是$A\cong B\oplus\mathrm{im}\varphi$.完成归纳.
\end{proof}

$R$是左半遗传环当且仅当,投射左$R$模$P$的每个有限生成子模$A$都是投射的.
\begin{proof}
	
	按照$P$是自由模的直和项,于是也是子模,于是$A$是自由模的有限生成子模,于是按照上一定理得到$A$是有限个有限生成左理想的直和.其中每个分支都是投射模,于是$A$投射.反过来是直接的.
\end{proof}

一个环称为左一致环,如果每个有限生成左理想都是有限表现的.那么每个左诺特环必然都是左一致环.考虑无限个未定元的多项式环,它是左一致环,但是不是左诺特环.存在左一致但是不右一致的环.

左一致环$R$具有如下等价描述:
\begin{enumerate}
	\item 对每个集合$X$,$X$作为基生成的右自由$R$模$R^X$是平坦的.
	\item 每个左自由模的有限生成子模是有限表现的.
\end{enumerate}