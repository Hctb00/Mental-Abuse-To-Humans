\chapter{概型论}
\section{仿射概型}
\subsection{素谱作为拓扑空间}

设$A$是(含幺交换)环,它的全部素理想构成的集合记作$\mathrm{Spec}A$,称为$A$的素谱.我们来给它赋予拓扑.
\begin{enumerate}
	\item 映射$V(-)$,$I(-)$和$D(-)$.
	\begin{itemize}
		\item 映射$V(-):\textbf{P}(A)\to\textbf{P}(\mathrm{Spec}A)$:对$A$的子集$S$,定义$V(S)=\{P\in\mathrm{Spec}(A)\mid S\subseteq P\}$.其中$\textbf{P}(X)$表示$X$的幂集,$\mathrm{Spec}A$是环$A$的素谱.
		\item 如果$f\in A$,我们把$V(\{f\})$简单记作$V(f)$,并且记补集$\mathrm{Spec}A-V(f)$为$D(f)$,称为$f$的主开集.
		\item 映射$I(-):\textbf{P}(\mathrm{Spec}A)\to\textbf{P}(A)$:对素谱的子集$X$,定义$I(X)=\cap_{\mathfrak{p}\in X}\mathfrak{p}$.
	\end{itemize}
    \begin{enumerate}[(1)]
    	\item 如果$S\subseteq A$生成的理想是$\mathfrak{a}$,那么有$V(S)=V(\mathfrak{a})=V(\sqrt{\mathfrak{a}})$.并且$\sqrt{\mathfrak{a}}$是最大的满足等式$V(S)=V(\sqrt{\mathfrak{a}})$的$A$的子集.
    	\item 全体$\{V(\mathfrak{a})\}$满足闭集公理:$V(0)=\mathrm{Spec}(A)$;$V(A)=\emptyset$;$V(\mathfrak{a})\cup V(\mathfrak{b})=V(\mathfrak{a}\mathfrak{b})=V(\mathfrak{a}\cap \mathfrak{b})$;$\cap_jV(\mathfrak{a}_j)=V(\sum_j\mathfrak{a}_j)$.这个拓扑称为素谱上的Zariski拓扑.
    	\begin{proof}
    		
    		其中按照$(\mathfrak{a}\cap\mathfrak{b})^2\subseteq\mathfrak{a}\mathfrak{b}\subseteq\mathfrak{a}\cap\mathfrak{b}$直接得到$V(\mathfrak{a}\mathfrak{b})=V(\mathfrak{a}\cap\mathfrak{b})$.
    	\end{proof}
    	\item 如果$X\subseteq\mathrm{Spec}A$是子集,那么有$I(X)=I(\overline{X})$.
    	\item $V(-)$和$I(-)$都是反序包含的.并且$V(\mathfrak{a})\subseteq V(\mathfrak{b})$当且仅当$\sqrt{\mathfrak{b}}\subseteq\sqrt{\mathfrak{a}}$;$I(X)\subseteq I(Y)$当且仅当$\overline{Y}\subseteq\overline{X}$.
    	\item 二次复合都使得点集变大:对任意$S\subseteq A$,有$S\subseteq I(V(S))$;对任意$X\subseteq\mathrm{Spec}A$,有$X\subseteq V(I(X))$.
    	\item 三次复合和一次复合一致:对任意$S\subseteq A$,有$V(S)=V(I(V(S)))$;对任意$X\subseteq\mathrm{Spec}A$,有$I(X)\subseteq I(V(I(X)))$.
    	\item 对任意理想$\mathfrak{a}\subseteq A$,有$I(V(\mathfrak{a}))=\sqrt{\mathfrak{a}}$.
    	\item 对任意子集$X\subseteq\mathrm{Spec}A$,有$V(I(X))=\overline{X}$,这也是素谱上的闭包表达式.
    	\item $V(-)$和$I(-)$是素谱中的闭集和环$A$中的根理想之间的反序一一对应.
    	\item $D(a)=\emptyset$等价于$a$是幂零元;$D(a)=\mathrm{Spec}(A)$等价于$a$是单位;$D(a)=D(b)$等价于$\sqrt{(a)}=\sqrt{(b)}$.
    	\item 主开集构成Zariski拓扑的拓扑基.一方面任取开集$U$,有$U=\mathrm{Spec}(A)-V(\mathfrak{a})=\cup_{f\in\mathfrak{a}}D(f)$,另一方面有$D(fg)=D(f)\cap D(g)$.
    \end{enumerate}
    \item 紧性.
    \begin{enumerate}[(1)]
    	\item Zariski拓扑是拟紧致的(在交换代数和代数几何中通常把点集拓扑中所定义的开覆盖有有限子覆盖称为拟紧致的,而把拟紧致的Hausdorff空间称为紧致的).为此只需说明主开集的覆盖总有有限子覆盖:从$\cup_{i\in I}D(f_i)=\mathrm{Spec}(A)$得到$\sum_{i\in I}(f_i)=A$,但是单位元$1_A$据此可以写成有限和,这只涉及到有限个$f_i$,于是这些$f_i$对应的主开集是有限子覆盖.
    	\item 每个主开集是拟紧致的.给定$D(f)$的一个额开覆盖,不妨取为主开集覆盖.并且由于$D(f)\cap D(f_i)=D(ff_i)$,可以取主开覆盖为$\{g_i,i\in I\}$使得$g_i\in (f)$.开覆盖意味着$\cup_{i\in I}D(g_i)=D(f)$,也就是$V(\sum_i (g_i))=\cap_{i\in I}V(g_i)=V(f)$,于是有$\sum_{i\in I}(g_i)$和$(f)$的根理想相同,而$\sum_i (g_i)$的根理想可以被全体$g_i$的次幂生成,这就导致存在一个等式$f=\sum_{j=1}^r a_ig_{t_i}$,那么$(f)$包含在$(g_{t_1},\cdots,g_{t_r})$的根理想中,于是二者根理想相同,于是$V(\sum_{i=1}^r(g_i{t_i}))\cap_{j=1}^rV(g_{t_i})=V(f)$,这就说明了$D(f)$被$D(g_{t_i}),i=1,2,\cdots,r$覆盖.
    	\item $X$的一个开集是拟紧集当且仅当它可以写作有限个主开集的并.上一条说明了充分性,对于必要性,取开准紧集$U$,如果记补集是闭集$V(I)$,则有$U=\cup_{f\in I}D(f)$,抽有限子覆盖$U=\cup_{f_i\in I,1\le i\le n}D(f_i)$,就得到结论.
    	\item 于是上一条得到素谱上可能存在开集不是拟紧的.例如$A=k[x_1,x_2,\cdots]$,取开集$U$为$V(x_1,x_2,\cdots)$的补集.
    \end{enumerate}
    \item 诺特性.这是一种较强的紧致性.
    \begin{enumerate}[(1)]
    	\item 按照$V(I)\subseteq V(J)$当且仅当$\sqrt{J}\subseteq\sqrt{I}$,说明素谱是诺特空间当且仅当环对根理想满足升链条件.于是一个诺特环的素谱总是诺特的.
    	\item 但是反过来素谱诺特并不能说明环是诺特的,本质原因是幂零根里蕴藏的非诺特性不会被素谱发现,即环商去幂零根中的理想不改变素谱,例如可构造$R=k[x_1,x_2,\cdots]/(x_1^2,x_2^2,\cdots)$,那么$(x_1,x_2,\cdots)$包含在幂零根中,导致$R$和$k=k[x_1,x_2,\cdots]/(x_1,x_2,\cdots)$的素谱相同,于是恰好由单个元构成,这是诺特的.但是$R$的极大理想$(x_1,x_2,\cdots)/(x_1^2,x_2^2,\cdots)$不是有限生成的.
    	\item 另外素谱诺特并不等价于素理想上满足升链条件,仅仅能有前推出后.例如考虑可数个$\mathbb{F}_2$的直积.这是Boolean环,于是它是零维的,于是素理想上满足升链条件.但是Boolean环的每个理想都是根理想,导致如果取前n分量生成的理想$I_n$,从$I_n\subsetneqq I_{n+1}$得到$V(I_n)\supsetneqq V(I_{n+1})$,这导致素谱不是诺特的.
    \end{enumerate}
    \item 分离性.
    \begin{enumerate}[(1)]
    	\item 闭点.任取$P\in\mathrm{Spec}(A)$,那么$\{P\}$的闭包恰好就是$V(P)$,于是$\{P\}$是闭集当且仅当$P$是极大理想.据此把素谱中的极大理想称为闭点.交换环总存在极大理想,于是素谱上总存在闭点.
    	\item 素谱总是$T_0$的:这是因为对任两个不同的素理想$P_1,P_2$,$V(P_1)$和$V(P_2)$至少有一个不包含另外的$P_i$,否则会得到$P_1=P_2$.
    	\item 素谱的如下条件两两等价:
    	\begin{enumerate}
    		\item $A/\mathrm{nil}(A)$是绝对平坦环.
    		\item $A$是零维环,即每个素理想都是极大理想.
    		\item 素谱是$T_1$空间.
    		\item 素谱是$T_2$空间.
    	\end{enumerate}
    	\begin{proof}
    		
    		1等价于2.必要性,$A/\mathrm{nil}(A)$是绝对平坦环这个条件等价于讲对每个$x\in A$,存在$y\in A$使得$x(xy-1)$是幂零元,也即它包含于$A$的每个素理想中.任取素理想$P$,任取不在$P$中的元$x$,那么存在$y$使得$x'(x'y'-1)=0$,这里$x'\not=0$,于是从$A/P$是整环得到$x'y'-1=0$,即$x'$是单位元,这就得到$A/P$是域,于是$P$是极大理想.充分性,需要验证的是$(A')_{m'}$总是域,这里$A'=A/\mathrm{nil}(A)$,$m'=m/\mathrm{nil}(A)$其中$m$是$A$的极大理想.取$S=A'-m'$.假设可以取到$S^{-1}(m')$中的非零元$a$,$A'_{m'}$没有非平凡幂零元说明$a$不是幂零元,于是存在真包含于$S^{-1}(m')$内的素理想与$\{1,a,a^2,\cdots\}$无交,按照商环素理想对应定理,存在真包含于$m$内的素理想,这和零维条件矛盾,于是$S^{-1}(m')=0$,于是$S^{-1}(A')$是域.
    		
    		2和3的等价性是直接的,素谱是$T_1$空间当且仅当每个素理想都是闭点,当且仅当每个素理想都是极大理想.
    		
    		4和前三项的等价性.4推3是直接的,只需证明1推4.我们在下文会验证$\mathrm{Spec}(A)$和$\mathrm{Spec}(A/\mathrm{nil}(A))$是同胚的,于是问题归结为对绝对平坦环$A$,总有它的素谱是$T_2$空间.任取$A$的两个不同的素理想$P,Q$,那么$P\not\subseteq Q$和$Q\not\subseteq P$至少有一个不成立,不妨设可取元$a\in P-Q$,按照绝对平坦条件存在广义逆$b$,即有$a^2b=a$.那么$b\not\in Q$,于是$ab\in P-Q$,而$(ab)^2=ab$,于是$ab(ab-1)=0$,于是素性得到$ab-1\in Q-P$,$P\in D(ab-1)$和$Q\in D(ab)$,而且$D(ab)\cap D(ab-1)=D(0)=\emptyset$.
    	\end{proof}
    	\item 在上一条任一条件成立下,素谱是全不连通空间.事实上如果存在连通子集$Y$包含至少两个点$P,Q$,上一条最后一段的证明中,我们看到存在既开又闭的两个子集$D(ab)$和$D(ab-1)$分离这两个点,这导致它们分离子集$X$,矛盾.于是连通分支都是单点集.
    \end{enumerate}
    \item 不可约性.
    \begin{enumerate}[(1)]
    	\item 素谱$\mathrm{Spec}A$的子集$Y$是不可约的当且仅当$\mathfrak{p}=I(Y)=\cap_{\mathfrak{q}\in Y}\mathfrak{q}$是素理想.特别的,素谱的闭子集$E$是不可约的当且仅当它可以表示为$V(p)=\overline{\{p\}}$,其中$p$是一个素理想.并且这里的$p$是唯一的,它称为不可约闭子集$E$的一般点.
    	\begin{proof}
    		
    		如果$Y$是不可约的,任取$f,g\in A$满足$fg\in\mathfrak{p}$,那么从$Y\subseteq V(fg)=V(g)\cup V(f)$,得到$Y\subseteq V(f)$或者$Y\subseteq V(g)$,于是$f\in\mathfrak{p}$或者$g\in\mathfrak{p}$,于是$\mathfrak{p}$是素理想.
    		
    		\qquad
    		
    		反过来如果$\mathfrak{p}$是素理想.我们有$\overline{Y}=V(\mathfrak{p})=V(I(\{\mathfrak{p}\}))=\overline{\{p\}}$.于是$\overline{Y}$是$\{\mathfrak{p}\}$的闭包,但是单点集一定是不可约的,不可约子集的闭包还是不可约的.
    		
    		\qquad
    		
    		最后如果$Y$本身是闭子集,那么它可以表示为$V(\mathfrak{p})$,其中$\mathfrak{p}=I(Y)$是根理想.把前面的结论套上就得到$Y$是不可约的当且仅当$\mathfrak{p}$是素理想.
    	\end{proof}
    	\item 上一命题说明如下若干事实:素谱$\mathrm{Spec}(A)$是不可约的当且仅当幂零根是素理想(特别的,整环的素谱是不可约的);素谱上的不可约闭集一一对应于素谱中的点;并且在上述对应下,不可约分支(极大不可约闭子集)一一对应于$A$的极小素理想.另外注意不同于连通分支的概念,不同不可约分支的交可能非空.
    	\item 另外不可约空间必然是连通的,但是反过来存在连通的可约空间:考虑素谱$X=\mathrm{Spec}\mathbb{C}[x,y]/(xy)$,它的幂零根是零理想,但是这不是素理想,因为$(x)\cdot(y)=(xy)$.它是连通空间因为这个环没有非平凡幂等元:如果$e\in R=\mathbb{C}[x,y]/(xy)$是幂等元,那么$e(e-1)\in(xy)\subseteq(x)$,这里$(x)$是素理想,于是不妨设$e\in(x)$,否则以$1-e$取代$e$.如果$e\in(y)$那么$e\in(x)\cap(y)=(xy)$得到$R$没有非平凡幂等元.否则有$e-1\in(y)$,导致$R=(e,e-1)\subseteq(x,y)$矛盾.
    	\item 空间$X$上的组合维数定义为严格包含的闭不可约子空间的长度的上确界.于是环$A$的krull维数就是它的素谱的组合维数.
    \end{enumerate}
    \item 连通性.
    \begin{enumerate}[(1)]
    	\item 素谱不连通的等价描述.给定环$A$,设素谱$X=\mathrm{Spec}(A)$,则如下三个条件等价:
    	\begin{enumerate}[(a)]
    		\item $X$是不连通空间.
    		\item $A\cong A_1\times A_2$,其中$A_1,A_2$都不是零环.因为这个描述,我们把不能写成两个非零环直积的环称为连通环.
    		\item $A$包含一个非平凡的幂等元.
    	\end{enumerate}
    	
    	特别的,$e\mapsto D(e)$是从$A$的幂等元到$\mathrm{Spec}A$的既开又闭子集的一一对应.
    	\begin{proof}
    		
    		(c)推(b),假设$A$上有非平凡的幂等元$u$,即$u\not=0,1$且$u^2=u$.那么$u$和$1-u$互素,且$u(1-u)=0$,于是中国剩余定理得到$A\cong A/(u(1-u))\cong A/(u)\times A/(1-u)$.按照$u\not=0,1$说明右侧两个环都不是零环.
    		
    		\qquad
    		
    		(a)推(c),假设素谱$X=\mathrm{Spec}(A)$不连通,那么存在非空不交的两个闭子集$V(E),V(F)$满足$V(E)\cup V(F)=X$,其中$E,F$都是理想.这几个条件等价于讲:$E,F$不是单位理想,$E+F=A$,$E\cap F$包含于$A$的幂零根中.于是可取非单位元$e\in E,f\in F$满足$e+f=1$,按照$ef\in E\cap F$落在幂零根中,于是存在一个足够大的正整数$n$满足$e^nf^n=0$.现在考虑$1=(e+f)^{2n}$的展开式,记:
    		$$e_1=e^{2n}+\left(\begin{array}{c} 2n\\1\end{array}\right)e^{2n-1}f+\cdots+\left(\begin{array}{c} 2n\\n-1\end{array}\right)e^{n+1}f^{n-1}$$
    		$$e_2=f^{2n}+\left(\begin{array}{c} 2n\\1\end{array}\right)f^{2n-1}e+\cdots+\left(\begin{array}{c} 2n\\n-1\end{array}\right)f^{n+1}e^{n-1}$$
    		
    		其中没有包含项$a^nb^n$是因为它已经为零了,于是$e_1+e_2=(e+f)^{2n}=1$.另外由于$e_1$中每一项都包含$e^n$,$e_2$中每一项都包含$f^n$,而$e^nf^n=0$,于是$e_1e_2=0$.于是得到$0=e_1e_2=e_1(1-e_1)=e_1-e_1^2$,于是$e_1$是幂等元,同理$e_2$是幂等元,为了说明它们都是非平凡的,只需说明它们都不能是1,这从$e_1\in E$和$e_2\in F$得到.
    		
    		\qquad
    		
    		如果借助结构层,(a)推(c)可以更直接一些:如果$X=U\cup V$,其中$U,V$是不交非空开集.定义$X$上的一个整体截面$s$满足在$U$上恒取$1/1$,在$V$上恒取$0/1$,那么在$\mathscr{O}_{\mathrm{Spec}(A)}(\mathrm{Spec}(A))\cong A$下截面$s$会对应一个元$a'\in A$.按照$U,V$非空知这个元不为0和1,于是它就是非平凡的幂等元.
    		
    		\qquad
    		
    		(b)推(c):只需取$A\cong A_1\times A_2$中的元$(1,0)$,它是幂等元,但是按照$A_1,A_2$都不是零环得到这个元不是零元也不是幺元.
    		
    		\qquad
    		
    		(c)推(a),如果$A$中存在非平凡幂等元$e$,那么$e(e-1)=0$,于是$V(e)\cup V(1-e)=V(0)=X$,并且$V(e)\cap V(e-1)=V((e)+(e-1))=V(1)=\emptyset$.按照$e$是非平凡的幂等元,得到$e$和$(1-e)$都不是单位元,于是$V(e)$和$V(1-e)$都非空,这就得到$X$是不连通的.
    	\end{proof}
    	\item 设$A$是环,设$I\subseteq A$是一个幂零理想,那么典范映射$A\to A/I$把$A$的幂等元集合一一对应于$A/I$的幂等元集合.
    	\begin{proof}
    		
    		满射:任取$A+I$的幂等元$a+I$,那么$1-a+I$也是幂等元,存在正整数$n$使得$a^n(1-a)^n=0$.有$a^n+(1-a)^n\equiv a+(1-a)\equiv1(\mathrm{mod}I)$.于是$1-a^n-(1-a)^n=b\in I$.如果记$b^m=0$,那么$1-b$有逆元$1+b+\cdots+b^{m-1}=c$.于是有$1=(1-b)c=ca^n+c(1-a)^n$.在等式两边乘以$ca^n$,结合$a^n(1-a)^n=0$得到$(ca^n)^2=ca^n$.按照$b\in I$得到$c=1+b+\cdots+b^{m-1}\equiv1(\mathrm{mod}I)$,于是$ca^n\equiv a(\mathrm{mod}I)$.于是$ca^n$就是$a+I$在$A$中提升的幂等元.
    		
    		\qquad
    		
    		单射:归结为证明如果$a_1,a_2\in A$是两个幂等元,满足$a_1-a_2\in I$,那么$a_1=a_2$.设正整数$n$使得$(a_1-a_2)^n=0$.我们还可以不妨设$n$是奇数,因为否则的话可以对这个式子再乘上一个$(a_1-a_2)$.把它展开,按照$a_1^k=a_1$和$a_2^k=a_2$,得到:
    		$$0=(a_1-a_2)^n=a_1-a_2+(-C_n^1+C_n^2-\cdots+C_n^{n-1})a_1a_2=a_1-a_2$$
    	\end{proof}
    	\item 例子.
    	\begin{itemize}
    		\item 局部环的素谱总是连通的.这是因为若否则$A$存在非平凡的幂等元$e$,那么$e-1$和$e$都不是单位,于是它们落在唯一的极大理想中,但是这导致$1$也落在极大理想中,矛盾.
    		\item 不可约空间总是连通的.这是因为不可约空间里两个非空开集的交总非空.
    	\end{itemize}
    \end{enumerate}
    \item 函子性.给定环同态$\varphi:A\to B$,那么它诱导了一个谱空间的映射:$^a\varphi:\mathrm{Spec}(B)\to\mathrm{Spec}(A)$,即$P\mapsto\varphi^{-1}(P)$.
    \begin{enumerate}[(1)]
    	\item 对$\mathrm{Spec}(A)$的每个子集$S$,有$^a\varphi^{-1}(V(S))=V(\varphi(S))$.特别的,$^a\varphi$是连续映射.于是$A\mapsto\mathrm{Spec}(A)$和$\varphi\mapsto^a\varphi$是从环范畴到拓扑空间范畴的逆变函子.
    	\begin{proof}
    		
    		这是因为$q\in^a\varphi^{-1}(V(S))$当且仅当$S\subseteq\varphi^{-1}(q)$当且仅当$\varphi(S)\subseteq q$当且仅当$q\in V(\varphi(S))$.
    	\end{proof}
    	\item 对任意$f\in A$,总有$^a\varphi^{-1}(D(f))=D(\varphi(f))$.
    	\begin{proof}
    		
    		这是因为$q\in^a\varphi^{-1}(D(f))$等价于$\varphi^{-1}(q)\in D(f)$等价于$f\not\in\varphi^{-1}(q)$,等价于$\varphi(f)\not\in q$,等价于$q\in D(\varphi(f))$.
    	\end{proof}
        \item 对任意$B$的理想$J$,总有$\overline{^a\varphi(V(J))}=V(\varphi^{-1}(J))$.
        \begin{proof}
        	
        	按照闭包公式有$\overline{^a\varphi(V(J))}=V(\cap_{p\in^a\varphi(V(J))}p)=V(\cap_{q\in V(J)}\varphi^{-1}(q))=V(\varphi^{-1}(\cap_{q\in V(J)}q))=V(\varphi^{-1}(\sqrt{J}))=V(\sqrt{\varphi^{-1}(J)})=V(\varphi^{-1}(J))$.
        \end{proof}
        \item 记$\varphi:A\to B$是环同态,那么它诱导了连续映射$^a\varphi:X=\mathrm{Spec}(B)\to Y=\mathrm{Spec}(A)$.那么$^a\varphi(X)$在$Y$中稠密(这个条件有个名字,即$^a\varphi$是支配态射)当且仅当$\ker\varphi$落在$A$的幂零根中.
        \begin{proof}
        	
        	按照上一条结论,有$\overline{^a\varphi(X)}=\overline{^a\varphi(V(0))}=V(\ker\varphi)$,于是这个等式恰好是$Y$当且仅当$\ker\varphi\subseteq\mathrm{nil}(A)$.
        \end{proof}
    \end{enumerate}
    \item 关于分式化和商的素谱.
    \begin{enumerate}[(1)]
    	\item 一个引理.如果环同态$\varphi:A\to B$,满足$B$中每个元可以表示为$\varphi(a)h$的形式,这里$h$是$B$中的一个单位,那么它诱导的素谱之间的连续映射$^a\varphi:\mathrm{Spec}(B)\to\mathrm{Spec}(A)$是单射,并且它甚至是$\mathrm{Spec}(B)$到像空间$\mathrm{im}^a\varphi$的同胚.
    	\begin{proof}
    		
    		我们先来说明$^a\varphi$是单射.如果$p_1,p_2\in\mathrm{Spec}(B)$满足$\varphi^{-1}(p_1)=\varphi^{-1}(p_2)$.任取$b\in p_1\subseteq B$,按照定义它可以表示为$b=\varphi(a)h$,其中$h$是$B$中单位,那么$\varphi(a)=bh^{-1}\in p_1$,于是$a\in\varphi^{-1}(p_1)=\varphi^{-1}(p_2)$,于是$\varphi(a)=bh^{-1}\in p_2$,于是$b\in p_2$.这说明$p_1\subseteq p_2$,同理有$p_2\subseteq p_1$,于是$p_1=p_2$,于是$^a\varphi$是单射.
    		
    		于是$^a\varphi$就是$\mathrm{Spec}(B)\to\mathrm{im}^a\varphi$的连续双射,为证明它是同胚,仅需验证它是闭映射.任取$\mathrm{Spec}(B)$中的闭集$Y'=V(E')$,其中$E'\subseteq B$.如果我们把$E'$中某个元乘以一个单位,不会改变$V(E')$,于是按照条件$E'$中的元肯定表示为$\varphi(A)$中的元乘以一个单位,我们不妨约定$E'\subseteq\varphi(A)$.记$E'=\varphi(E)$,$E\subseteq A$.那么得到$Y'=V(E')=V(\varphi(E))=(^a\varphi)^{-1}(V(E))$.按照$^a\varphi$是双射得到$^a\varphi(Y')=V(E)$是闭集.这里用到了一步一般成立的等式$V(\varphi(E))=(^a\varphi)^{-1}V(E)$.
    	\end{proof}
    	\item 借助上述引理,得到如下两个结论:典范满同态$A\to A/I$诱导了素谱$\mathrm{Spec}(A/I)$到$\mathrm{Spec}(A)$的闭子集$V(I)$的同胚,于是$\mathrm{Spec}(A/I)$就可以视为$\mathrm{Spec}(A)$的闭子集$V(I)$;对任意乘性闭子集$S$,典范同态$A\to S^{-1}A$诱导了素谱$\mathrm{Spec}(S^{-1}A)$到$\mathrm{Spec}(A)$的全部和$S$无交的素理想构成的子空间之间的同胚,特别的,局部化$A_f,f\in A$的素谱可视为$\mathrm{Spec}(A)$的开子集$D(f)$.
    	\item 特别的,对环$A$总有$\mathrm{Spec}(A)$和$\mathrm{Spec}(A/\mathrm{nil}(A))$同胚,这里$\mathrm{nil}(A)$表示$A$的幂零根.
    \end{enumerate}
\end{enumerate}
\subsection{素谱上的结构层}

给定交换环$A$,记$\mathrm{Spec}(A)$是它的素谱,我们来定义$\mathrm{Spec}(A)$上的环层$\mathscr{O}$.
\begin{enumerate}
	\item 截面的定义.对开集$U\subseteq\mathrm{Spec}(A)$,定义$\mathscr{O}(U)$是全体满足如下两个条件的映射$s:U\to\coprod_{p\in\mathrm{Spec}(A)}A_p$构成的集合:对每个$p\in U$有$s(p)\in A_p$;第二个条件是$s$局部上看会是一个固定的分式形式,即对每个$p\in U$,存在$p$的开邻域$U_p\subseteq U$,以及元$a,f\in A$,满足对每个$U_p$中的素理想$q$总有$f\not\in q$,并且有$s(q)=\frac{a}{f}\in A_q,\forall q\in U_q$.
	\item 验证$\mathscr{O}(U)$的确是一个交换环.给定两个元$s,t\in\mathscr{O}(U)$,就定义加法和乘法为$(s+t)(p)=s(p)+t(p)\in A_p$和$(st)(p)=s(p)t(p)\in A_p$.需要验证的是定义中的第二条.倘若$p\in U$存在开邻域$U_1$满足在其上恒有表达式$s(q)=\frac{a_1}{f_1}$,其中$f_1$不被$U_1$中任一素理想包含,还存在$p$的开邻域$U_2$满足在其上恒有表达式$s(q)=\frac{a_2}{f_2}$,其中$f_2$不被$U_2$中任一素理想包含.那么在开集$U_1\cap U_2$上$f_1f_2$不被其中任一素理想所包含,此时有$(s+t)(q)=\frac{a_1f_2+a_2f_1}{f_1f_2}$和$(st)(q)=\frac{a_1a_2}{f_1f_2}$.另外这个环中加法零元是$s(p)=\frac{0}{1}\in A_p,\forall p\in U$;乘法幺元是$s(p)=\frac{1}{1}\in A_p,\forall p\in U$.
	\item 定义限制映射.设素谱上的开集满足$V\subseteq U$,定义限制映射$\mathscr{O}(U)\to\mathscr{O}(V)$就是把$U$上满足那两条的映射限制到$V$上.至此得到$\mathscr{O}$是$\mathrm{Spec}(A)$上的预层.容易验证它是一个层.
	\item 茎.对每个点$p\in\mathrm{Spec}(A)$,有$p$处的茎$\mathscr{O}_p\cong A_p$.
	\begin{proof}
		
		$\mathscr{O}_p$就是以$\mathscr{O}(U)$,其中$U$是包含点$p$的开集,以及它们之间限制映射作为正向系统的正向极限,这里正向系统的指标集是一个有向集.对每个包含$p$的开集$U$,定义环同态$\mathscr{O}(U)\to A_p$为$s\mapsto s(p)$,于是按照正向极限的泛映射性质,诱导了一个环同态$\mathscr{O}_p\to A_p$.现在验证它的单满性.
		
		验证满射.在指标集是有向集的前提下,正向极限中的每个元素可以表示为正向系统中某个环中元素在典范映射下的像,于是验证上述映射的满射性,只需验证对$A_p$中的每个元$\frac{a}{f},f\not\in p$,存在包含$p$的开集$U$,满足存在$s\in\mathscr{O}(U)$使得$s(p)=\frac{a}{f}$.对此可取$U=D(f)$,有$s:q\mapsto\frac{a}{f}$是$\mathscr{O}(D(f))$中的元.再考虑如下交换图得到满射性:
		$$\xymatrix{\mathscr{O}_p\ar[r]&A_p\\\mathscr{O}(U)\ar[u]\ar[ur]&}$$
		
		验证单射.只需验证如果$s\in\mathscr{O}(U)$,其中$p\in U$,满足$s(p)=0$,那么有$s$在正向极限中的像是零元.这个需要验证的内容等价于验证(在指标集是有向集的前提下),存在一个$p$的开邻域$V\subseteq U$满足$s$在$V$上的限制是零映射(即$\mathscr{O}(V)$中的零元).为此先取$V\subseteq U$是$p$的开邻域,满足$\forall q\in V$有$s(q)=\frac{a}{f}$.特别的从$s(p)=\frac{a}{f}=0\in A_p$得到存在$t\not\in p$使得$ta=0$,于是在包含$p$的开集$V\cap D(t)\subseteq U$中就有$s(q)=0,\forall q\in V\cap D(t)$.这得到单射.
	\end{proof}
    \item 主开集上的截面.对$f\in A$,主开集上的截面$\mathscr{O}(D(f))\cong A_f$.特别的这说明$\mathscr{O}(\mathrm{Spec}(A))\cong A$.基于这个结论我们会把$A$中的元视为素谱上的映射,即$a\in A$作用在素理想$p$上为$a\in A_p$.
    \begin{proof}
    	
    	定义$A_f\to\mathscr{O}(D(f))$的映射为,把每个$a/f^k$映射为把$D(f)\to\coprod_{p\in\mathrm{Spec}(A)}A_p$,$p\mapsto\frac{a}{f^k}\in A_p$.
    	
    	验证单射.假设$\frac{a}{f^k}$落在上述同态的核中,那么对每个$p\in D(f)$有$\frac{a}{f^k}=0$,也即对每个$p\in D(f)$存在$t\not\in p$使得$ta=0$.于是$p\in D(f)$得到$\mathrm{Ann}(a)\not\subseteq p$.于是$V(\mathrm{Ann}(a))\subseteq V((f))$.于是$f\in\sqrt{\mathrm{Ann}(a)}$,也即存在$f$的次幂满足$f^na=0$.于是在$A_f$中有$\frac{a}{f^k}=0$.
    	
    	验证满射要复杂一些.对$\mathscr{O}(D(f))$中的每个映射$s$,按照定义,$D(f)$可以被若干开集$\{V_i\}$覆盖,并且在每个$V_i$上$s$的取值可以表示为$\frac{a_i}{f_i}$,其中$f_i$不被$V_i$中每个素理想包含,也即$V_i\subseteq D(f_i)$.由于主开集是Zariski拓扑的一个拓扑基,于是我们可以索性把$V_i$取为主开集记作$D(g_i)$.那么从$D(g_i)\subseteq D(f_i)$得到$V(f_i)\subseteq V(g_i)$,而这等价于$\sqrt{(g_i)}\subseteq\sqrt{(f_i)}$,于是存在某个$g_i^{k_i}=h_if_i$,于是$\frac{a_i}{f_i}=\frac{h_ia_i}{g_i^{k_i}}$.按照$D(g_i)=(g_i^{k_i})$,于是不妨把$g_i^{k_i}$替换为$g_i$,把$a_ih_i$替换为$a_i$.
    	
    	我们接下来要说明$D(f)$可被有限个这样的$D(g_i)$所覆盖.从$D(f)=\cup_iD(g_i)$得到$V(f)=\cap_i V((g_i))=V(\sum(g_i))$得到$\sqrt{f}=\sqrt{\sum(g_i)}$,于是存在某个$f^n\in\sum(g_i)$.于是$f^n=a_1g_{t_1}+a_2g_{t_2}+\cdots+a_rg_{t_r}$,这得到$f\in\sqrt{\sum_i(g_{t_i})}$,于是$\sqrt{f}\subseteq\sqrt{\sum_i(g(t_i))}$,导致$D(f)\subseteq\cup_iD(g_{t_i})$.不妨就设$D(g_i)$的指标集是有限集.
    	
    	接下来对每个$q\in D(g_i)\cap D(g_j)=D(g_ig_j)$,在$A_q$中有$\frac{a_i}{g_i}=\frac{a_j}{g_j}$,也即存在$t\not\in q$使得$t(a_ig_j-a_jg_i)=0$.于是$\mathrm{Ann}(a_ig_j-a_jg_i)\subseteq q$,于是$V(\mathrm{Ann}(a_ig_j-a_jg_i))\subseteq V((g_ig_j))$.于是$g_ig_j\in\sqrt{\mathrm{Ann}(a_ig_j-a_jg_i)}$,按照只有有限个$g_i$,于是可取足够大的正整数$n$使得$(g_ig_j)^n(a_ig_j-a_jg_i)=0$对任意的$i,j$成立.再把$g_i$替换为$g_i^{k+1}$,把$a_i$替换为$a_ig_i^k$,于是对每个$q\in D(f)$依然有$s(q)=\frac{a_i}{g_i}\in A_q$,并且此时有$a_ig_j=a_jg_i,\forall i,j$.
    	
    	最后从$D(f)=\cup_iD(g_i)$得到$V((f))=V(\sum_i(g_i))$,于是存在$f$的某个次幂满足$f^k=\sum_ib_ig_i$.于是有$a_jf^k=\sum_ia_jg_ib_i=\sum_ia_ig_jb_i$.于是得到在每个$D(g_j)$中都有$\frac{a_j}{g_j}=\frac{\sum_ia_ib_i}{f^k}$,这得到满射性.
    \end{proof}
    \item 仿射概型是指同构于$(\mathrm{Spec}(A),\mathscr{O}_{\mathrm{Spec}(A)})$的环空间.概型$(X,\mathscr{O}_X)$是指一个局部环空间,存在开覆盖$\{U_i\}_{i\in X}$使得每个$(U_i,\mathscr{O}_{X}\mid_{U_i})$是仿射概型.称$X$是底空间,$\mathscr{O}_X$是结构层.定义概型之间的态射是它们作为局部环空间的态射.
    \item 定义结构层还可以直接在主开集(作为拓扑基)上定义,我们解释过此时结构层可以唯一延拓至所有开集.设$X=\mathrm{Spec}A$,对主开集$D(f)$,定义截面环为$\mathscr{O}_X(D(f))=A_f$.如果$D(f)\subseteq D(g)$,那么有$\sqrt{f}\subseteq\sqrt{g}$,换句话讲$f^n=gs$,定义典范映射$A_g\to A_f$就是把$A_f$视为$(A_g)_{f/1}$的典范映射,具体的写就是$a/g^r\mapsto as^r/f^{nr}$.我们断言这是主开集上的一个层,于是按照延拓的唯一性,这个结构层和我们之前定义的结构层是一致的.
    \begin{proof}
    	
    	这明显是一个预层,我们只需验证粘合的存在性和唯一性.任取一个主开集$D(f)$,设它被主开集$\{D(f_i),i\in I\}$覆盖,如果在每个$\mathscr{O}_X(D(f_i))$中选取一个元,它们在交集上的限制相同,我们要证明这些元可以粘合为$D(f)$上的截面,并且这样的粘合结果是唯一的.我们可以把问题先做一些简化.首先主开集$D(f)$本身可以直接取为全集,或者说$f$可以取为1,因为$D(f)$上的截面环是$A_f$,可以用它替代$A$.另外按照素谱是拟紧的,我们可以不妨设$I$是有限的.
    	
    	证明粘合的唯一性.换句话讲典范映射$A\to\prod_iA_{f_i}$是单射.如果$a\in A$满足在每个$A_{f_i}$中都有$a/1=0$.按照$I$是有限的,可取一个足够大的$n$使得$f_i^na=0,\forall i$成立.按照$\{D(f_i)=D(f_i^n),i\in I\}$覆盖了整个空间,说明$(f_i^n,i\in I)$生成了整个$A$,所以有$g_i\in A$使得$1=\sum_ig_if_i^n$.那么有$a=\sum_ig_if_i^na=0$,这得到粘合唯一性.
    	
    	证明粘合的存在性.如果选取$s_i\in A_{f_i}$使得$s_i,s_j$打到$A_{f_if_j}$中是相同的.我们要证明存在$a\in A$使得在每个$A_{f_i}$中都有$a/1=s_i$.因为$I$是有限集,可取一个足够大的$n$使得$s_i=a_i/f_i^n$.按照$s_i=s_j\in A_{f_if_j}$,以及$I$是有限集,可取足够大的$m$使得$(f_if_j)^m(f_j^na_i-f_i^na_j)=0$对任意$i,j$成立.用$f_i^ma_i$代替$a_i$,记$n+m=r$,于是有$f_j^ra_i=f_i^ta_j,\forall i,j$成立.这些$f_i^r$依旧生成了整个$A$,所以有$1=\sum_ig_if_i^r$.取$a=\sum_ig_ia_i$,那么有如下等式,这得证.
    	$$f_j^ra=\sum_ig_if_j^ra_i=\sum_ig_if_i^ra_j=(\sum_ig_if_i^r)a_j=a_j$$
    \end{proof}
\end{enumerate}
\subsection{交换环范畴和仿射概型范畴}

交换环范畴和仿射概型范畴是逆变范畴等价的,也即存在环范畴到仿射概型范畴的完全忠实本质满函子.这里本质满的条件(即给定仿射概型,必然存在交换环和它对应)仅仅是定义.需要说明的是态射集之间存在双射.
\begin{enumerate}
	\item 给定环同态$\varphi:A\to B$,那么它诱导了局部环空间之间的态射:
	$$(f,f^{\#}):(\mathrm{Spec}(B),\mathscr{O}_{\mathrm{Spec}(B)})\to(\mathrm{Spec}(A),\mathscr{O}_{\mathrm{Spec}(A)})$$
	\begin{proof}
		
		给定环同态$\varphi:A\to B$,记$f:\mathrm{Spec}(B)\to\mathrm{Spec}(A)$为$q\mapsto\varphi^{-1}(q)$.为说明它是连续映射,仅需注意到对任意$A$中理想$a$,总有$f^{-1}(V(a))=V(f(a))$:一方面如果$B$中素理想$q$满足$a\subseteq\varphi^{-1}(q)$,那么有$\varphi(a)\subseteq q$,于是$q\in V(aB)$;另一方面如果$B$中素理想$q$满足$aB\subseteq q$,于是$\varphi(a)\subseteq q$,于是$a\subseteq\varphi^{-1}(q)$,于是$q\in f^{-1}(V(a))$.
		
		定义$f^{\#}$.对每个$p\in\mathrm{Spec}(B)$,$\varphi$可诱导局部同态$\varphi_p:A_{\varphi^{-1}(p)}\to B_p$(即$a/s\mapsto f(a)/f(s)$,它是局部同态因为$\varphi_p(qA_q)\subseteq\varphi(q)B_p\subseteq pB_p$,其中$q=\varphi^{-1}(p)$).给定开集$V\subseteq\mathrm{Spec}(A)$,考虑如下图表:
		$$\xymatrix{\mathscr{O}_{\mathrm{Spec}(A)}(V)\ar[rr]^{f^{\#}(V)}\ar[d]^{s\mapsto s(f(q))}&&\mathscr{O}_{\mathrm{Spec}(B)}(f^{-1}(V))\ar[d]^{t\mapsto t(q)}\\A_{f(q)}\ar[rr]_{\varphi_p}&&B_q}$$
		
		我们定义$f^{\#}$是使得图表对每个素理想$q\in f^{-1}(V)$交换的映射:即对任意截面$s\in\mathscr{O}_{\mathrm{Spec}(A)}(V):V\to\coprod_{p\in\mathrm{Spec}(A)}A_p$,构造截面$f^{\#}(s)\in\mathscr{O}_{\mathrm{Spec}(B)}(f^{-1}(V)):f^{-1}(V)\to\coprod_{q\in\mathrm{Spec}(B)}B_q$为,把$f^{-1}(V)$中每个素理想$q$,映射为$\varphi_q(s(f(q)))\in B_q$.
		
		最后只需验证$f^{\#}_q$总是局部同态.为此只需验证如下图表是交换的,这就说明$f^{\#}_q$等同于同态$\varphi_q$是局部同态.
		$$\xymatrix{\mathscr{O}_{\mathrm{Spec}(A),\varphi^{-1}(q)}\ar[d]^{s\mapsto s(\varphi^{-1}(q))}\ar[rr]^{f_q^{\#}}&&\mathscr{O}_{\mathrm{Spec}(B),q}\ar[d]^{t\mapsto t(q)}\\A_{\varphi^{-1}(q)}\ar[u]\ar[rr]_{\varphi_q}&&B_q\ar[u]}$$
		
		按照正向极限的性质,$\mathscr{O}_{\mathrm{Spec}(A),f(q)}$中的每个元可以表示为某个截面环$\mathscr{O}_{\mathrm{Spec}(A)}(V)$中元的像,于是如下交换图结合上面第一个交换图就说明了我们这个图表交换.
		$$\xymatrix{\mathscr{O}_{\mathrm{Spec}(A)}(V)\ar[rr]^{f^{\#}}\ar[d]&&\mathscr{O}_{\mathrm{Spec}(B)}(f^{-1}(V))\ar[d]\\\mathscr{O}_{\mathrm{spec(A),f(q)}}\ar[rr]_{f^{\#}_q}&&\mathscr{O}_{\mathrm{Spec}(B),q}}$$
	\end{proof}
    \item 另外注意上述证明中,一个环同态$\varphi:A\to B$诱导的主开集之间的态射$D(h),h\in A\to D(\varphi(h))$的态射是$A_h\to B_{\varphi(h)}$,$a/h^r\mapsto\varphi(a)/\varphi(h)^r$.
	\item 任意态射$(f,f^{\#}):(\mathrm{Spec}(B),\mathscr{O}_{\mathrm{Spec}(B)})\to(\mathrm{Spec}(A),\mathscr{O}_{\mathrm{Spec}(A)})$都是由环同态这样诱导出来的.
	\begin{proof}
		
		给定态射$(f,f^{\#})$.按照$(f_*\mathscr{O}_{\mathrm{Spec}(B)})(\mathrm{Spec}(A))=\mathscr{O}_{\mathrm{Spec}(B)}(\mathrm{Spec}(B))$,$f^{\#}$在整体截面环上的限制诱导了一个环同态$\varphi:A\to B$.我们的目标就是证明$\varphi$按照第一条诱导的态射$(g,g^{\#})=(f,f^{\#})$.
		$$\xymatrix{A\ar[rr]^{\varphi}\ar[d]^{\cong}&&B\ar[d]^{\cong}\\\mathscr{O}_{\mathrm{Spec}(A)}(\mathrm{Spec}(A))\ar[rr]^{f^{\#}}\ar[u]&&\mathscr{O}_{\mathrm{Spec}(B)}(\mathrm{Spec}(B))\ar[u]}$$
		具体的讲,验证$g=f$等价于验证$f(q)=\varphi^{-1}(q),\forall q\in\mathrm{Spec}(B)$;验证$g^{\#}=f^{\#}$等价于验证$f^{\#}(s)(p)=\varphi_p(s(f(p)))$对任意的截面$s$和$p\in\mathrm{Spec}(A)$.为证第一条,我们先来考虑如下图表:
		$$\xymatrix{A\ar[rrrr]^{\varphi}\ar[ddd]_{p_A}\ar[dr]&&&&B\ar[dl]\ar[ddd]^{p_B}\\&\mathscr{O}_{\mathrm{Spec}(A)}(\mathrm{Spec}(A))\ar[ul]\ar[rr]^{f^{\#}}\ar[d]&&\mathscr{O}_{\mathrm{Spec}(B)}(\mathrm{Spec}(B))\ar[ur]\ar[d]&\\&\mathscr{O}_{\mathrm{Spec}(A),f(q)}\ar[rr]^{f^{\#}_q}\ar[dl]&&\mathscr{O}_{\mathrm{Spec}(B),q}\ar[dr]&\\A_{f(q)}\ar[rrrr]_{\varphi_q'}\ar[ur]&&&&B_q\ar[ul]}$$
		
		上方梯形的交换性就是$\varphi$的定义,中间小矩形交换性我们之前给出过,我们定义$\varphi_q'$是使得下方梯形交换的同态,注意我们目前还不知道这里$\varphi_q'$是我们之前定义的典范同态$\varphi_q:A_{f(q)}\to B_q$.我们的目标就是证明它.这等价于验证大矩形的交换性,为此只需再验证左侧和右侧的梯形交换,这是容易的.据此我们得到需要验证的第一条$f(q)=\varphi^{-1}(q)$,注意这里中间一步$\varphi_q^{-1}(qB_q)=f(q)A_{f(q)}$是因为$\varphi_q$是局部同态:
		$$\varphi^{-1}(q)=\varphi^{-1}\circ p_B^{-1}(qB_q)=p_A^{-1}\circ\varphi_q^{-1}(qB_q)=p_A^{-1}(f(q)A_{f(q)})=f(q)$$
		
		下面验证第二条,为此我们只需验证如下图表的大矩形是交换的,这样两侧的复合相同得到$f^{\#}(s)(q)=\varphi_q(s(f(q)))$.但是这两个小矩形的交换性我们已经给出过了.这就完成证明.
		$$\xymatrix{\mathscr{O}_{\mathrm{Spec}(A)}(V)\ar[rr]^{f^{\#}}\ar[d]&&\mathscr{O}_{\mathrm{Spec}(B)}(f^{-1}(V))\ar[d]\\\mathscr{O}_{\mathrm{Spec}(A),f(q)}\ar[rr]^{f^{\#}_q}\ar[d]&&\mathscr{O}_{\mathrm{Spec}(B),q}\ar[d]\\A_{f(q)}\ar[rr]^{\varphi_q}&&B_q}$$
	\end{proof}
    \item 这里我们来验证上面两条给出了态射集之间的同态$\mathrm{Hom}_{\textbf{Ring}}(A,B)\cong\mathrm{Mor}(\mathrm{Spec}(B),\mathrm{Spec}(A))$.在第一条中对环同态$\varphi:A\to B$,我们诱导了态射$(f,f^{\#})$,这就给出了一个映射$\alpha:\mathrm{Hom}_{\textbf{Ring}}(A,B)\to\mathrm{Mor}(\mathrm{Spec}(B),\mathrm{Spec}(A))$.在第二条中对态射$(f,f^{\#})$,我们诱导了环同态$A\to B$,这就给出了映射$\beta:\mathrm{Mor}(\mathrm{Spec}(B),\mathrm{Spec}(A))\to\mathrm{Hom}_{\textbf{Ring}}(A,B)$.第二条的证明中我们实际上验证了$\alpha\circ\beta=\mathrm{id}$.下面我们证明$\beta\circ\alpha=\mathrm{id}$,这就得到交换环范畴和仿射概型范畴是逆变范畴等价的.
    \begin{proof}
    	
    	事实上,任取环同态$\varphi:A\to B$,设它诱导的态射为$(f,f^{\#})$,那么这个态射诱导的环同态也就是$f^{\#}(\mathrm{Spec}(A)):\mathscr{O}_{\mathrm{Spec}(A)}(\mathrm{Spec}(A))\to\mathscr{O}_{\mathrm{Spec}(B)}(\mathrm{Spec}(B))$.它把$\mathscr{O}_{\mathrm{Spec}(A)}(\mathrm{Spec}(A))\cong A$中的元$a$(此应视为处处取$a/1\in A_p$的整体截面),映射为这样一个$B$中的整体截面$f^{\#}(a)$,它在素理想$q$处的取值是$\varphi_q(s(f(q)))=\varphi(a)/1=b/1\in B_q$,于是$(f,f^{\#})$诱导的环同态恰好还是$\varphi$自身.也即$\beta\circ\alpha=\mathrm{id}$.
    \end{proof}
    \item 概形范畴上的单满态射未必是同构,因为在交换环范畴上这已经不成立了.
\end{enumerate}
\subsection{仿射概形上的拟凝聚层}

伴随模层.设$M$是$A$模,$\mathrm{Spec}A$上的关于$M$的伴随模层$\widetilde{M}$定义为:对每个开集$U\subseteq\mathrm{Spec}A$,定义$\widetilde{M}(U)$是全体函数$s:U\to\coprod_{p\in U}M_p$,满足如下两件事.这个层的限制映射自然的定义为映射在更小开集上的限制.仿射概形$\mathrm{Spec}A$上的模层$F$如果同构于某个$\widetilde{M}$,其中$M$是$A$模,就称它是仿射概形上的拟凝聚层.
\begin{itemize}
	\item $\forall p\in U$有$s(p)\in M_p$.
	\item 对每个$p\in U$,存在它的开邻域$V\subseteq U$,使得存在$m\in M,f\in A$对$q\in V$恒有$f\not\in q,s(q)=m/f$.
\end{itemize}
\begin{enumerate}
	\item $\widetilde{M}$是一个$\mathscr{O}_A$模层.特别的$\widetilde{A}=\mathscr{O}_{\mathrm{Spec}A}$.
	\item 茎.对每个$p\in\mathrm{Spec}A$,模层的局部环$\widetilde{M}_p$同构于局部化$M_p$.
	\item 主开集.对$f\in A$,$A_f$模$\widetilde{M}(D(f))$同构于局部化$M_f$.特别的有$\Gamma(X,\widetilde{M})=M$.
	\item 关于模层同构的注释.给定仿射概形上两个模层$\mathscr{F}$和$\mathscr{G}$,我们要证明它们同构可以用如下两个方式:如果预先构造了一个态射$\mathscr{F}\to\mathscr{G}$,那么接下来仅需验证在stalk上处处是同构,就得到这个态射是同构;如果预先没有态射,我们只要构造一个拓扑基的每个开集上(在仿射概形情况下就可以取主开集)相应截面的同构,再证明这些同态和限制映射是可交换的,那么这些局部的同构就可以粘合为一个整体的同构.
	\item 函子性.给定$A$模同态$\varphi:M\to N$,它自然的诱导了模层之间的态射$\widetilde{\varphi}:\widetilde{M}\to\widetilde{N}$为,把$s:U\to\coprod_{p\in U}M_p$映射为$t:U\to\coprod_{p\in U}N_p$,其中如果$s(p)=m/f$则$t(p)=\varphi(m)/f$.于是$M\mapsto\widetilde{M}$是从$A$模范畴到$\mathscr{O}_A$模层范畴的函子.
	\begin{itemize}
		\item 这个函子和任意直和可交换:$\widetilde{\oplus_iM_i}\cong\oplus_i\widetilde{M_i}$.于是特别的它是加性函子(等价于和二元直和可交换).
		\item 这个函子是完全忠实的,换句话讲对$A$模$M,N$有典范的同构$\mathrm{Hom}_A(M,N)\cong\mathrm{Hom}_{\mathscr{O}_A}(\widetilde{M},\widetilde{N})$.
		\begin{proof}
			
			给定$A$模同态$\varphi:M\to N$,我们解释了它诱导了对应的模层同态.反过来给定模层同态$\widetilde{\varphi}:\widetilde{M}\to\widetilde{N}$,取整体截面环之间的同态得到一个$A$模同态$\varphi:M\to N$.这两个映射互为逆映射,于是这个函子是完全忠实的.
		\end{proof}
		\item 这个函子是正合函子,更强的:$L\to M\to N$是正合的当且仅当$\widetilde{L}\to\widetilde{M}\to\widetilde{N}$是正合的.于是特别的这个函子与核,余核,像都可交换.
		\begin{proof}
			
			$\xymatrix{L\ar[r]^{\alpha}&M\ar[r]^{\beta}&N}$是模的正合列当且仅当对任意$p\in X=\mathrm{Spec}(A)$有正合列$L_p\to M_p\to N_p$,此即对任意$p\in X$有正合列$(\widetilde{L})_p\to(\widetilde{M})_p\to(\widetilde{N})_p$,于是有模层的正合列$\widetilde{L}\to\widetilde{M}\to\widetilde{N}$.
			
			\qquad
			
			反过来如果$\xymatrix{\widetilde{L}\ar[r]^{\alpha}&\widetilde{M}\ar[r]^{\beta}&\widetilde{N}}$是伴随模层的正合列,对任意素理想$\mathfrak{p}\in\mathrm{Spec}A$,有如下交换图表,其中第二行是正合列,于是有$(\ker\beta(X)/\mathrm{im}\alpha(X))_{\mathfrak{p}}=0$对任意素理想$\mathfrak{p}$成立,这得到$\ker\beta(X)/\mathrm{im}\alpha(X)=0$,也即第一行是正合的.
			$$\xymatrix{L\ar[r]^{\alpha(X)}\ar[d]&M\ar[r]^{\beta(X)}\ar[d]&N\ar[d]\\L_{\mathfrak{p}}\ar[r]&M_{\mathfrak{p}}\ar[r]&N_{\mathfrak{p}}}$$
		\end{proof}
		\item 特别的,$M\mapsto\widetilde{M}$是从$A$模范畴到$\mathscr{O}_A$拟凝聚层范畴的正合范畴等价.
		\item 如果$N_1$和$N_2$是$A$模$M$的子模,那么$\widetilde{N_1+N_2}=\widetilde{N_1}+\widetilde{N_2}$和$\widetilde{N_1\cap N_2}=\widetilde{N_1}\cap\widetilde{N_2}$.因为$N_1+N_2$可视为$N_1\oplus N_2\to M$的像,$N_1\cap N_2$可视为$M/N_1\oplus M/N_2\to M$的核,它们都和这个正合函子可交换.进而模层的任意和总是模层,有限个模层的交总是模层.
		\item $\widetilde{A}$的一族伴随模层的余积就是对应的一族模的直和的伴随模层.更一般的,我们解释过模层范畴上存在全部极限和余极限.如果$A$模$M$是正向系统$(M_{\lambda},g_{\mu\lambda})$的正向极限,那么$\widetilde{M}$是$(\widetilde{M_{\lambda}},\widetilde{g_{\mu\lambda}})$作为$\widetilde{A}$的伴随模层范畴里的正向系统的正向极限.
		\begin{proof}
			
			我们直接证明第二个结论.设正向系统$(M_{\lambda},g_{\mu\lambda})$的正向极限为$(M,g_{\lambda})$,其中$g_{\lambda}:M_{\lambda}\to M$.那么$(\widetilde{M},\widetilde{g_{\lambda}})$是正向系统$(\widetilde{M_{\lambda}},\widetilde{g_{\mu\lambda}})$的余锥.于是存在唯一的模层态射$\theta$使得如下图表交换.		
			$$\xymatrix{\lim\limits_{\rightarrow}\widetilde{M_{\lambda}}\ar[rr]^{\theta}&&\widetilde{M}\\\widetilde{M_{\lambda}}\ar[u]\ar@/_1pc/[urr]_{\widetilde{g_{\lambda}}}&&}$$
			
			最后仅需验证$\theta$是模层之间的同构,为此只需验证茎上是同构.而这只要注意到:
			$$\left(\lim\limits_{\rightarrow}\widetilde{M_{\lambda}}\right)_p=\lim\limits_{\rightarrow}(\widetilde{M_{\lambda}})_p=\lim\limits_{\rightarrow}(M_{\lambda})_p=M_p$$
		\end{proof}
	\end{itemize}
	\item 另外类似仿射概形上结构层的定义,给定$A$模$M$,定义伴随模层还可以从主开集出发,定义$D(f)$上的截面是$M_f$,可以验证此时在主开集上这已经构成一个层,再延拓到任意开集上即可.
	\item 张量和Hom函子.
	\begin{enumerate}
		\item 取伴随模层和张量积可交换,即有典范同构$\widetilde{M\otimes_AN}\cong\widetilde{M}\otimes_{\mathscr{O}_A}\widetilde{N}$.
		\begin{proof}
			
			先构造一个同态为,对开集$U$,$\widetilde{M}(U)$中的元$\left(s:U\to\coprod_{p\in U}M_p\right)$和$\widetilde{N}(U)$中的元$\left(t:U\to\coprod_{p\in U}N_p\right)$,把$s\otimes t$映射为$\widetilde{M\otimes_AN}(U)$中的元,为映射$U\to\coprod_{p\in U}M_p\otimes_{A_p}N_p$,$p\mapsto s(p)\otimes t(p)$.接下来容易验证stalk上是同构.
		\end{proof}
		\item 如果$M$是有限表示$A$模,那么伴随模层和Hom模层可交换,即有典范同构$\mathrm{HOM}_{\mathscr{O}_A}(\widetilde{M},\widetilde{N})\cong\widetilde{\mathrm{Hom}_A(M,N)}$.
		\begin{proof}
			
			对任意主开集$D(f)$,我们有典范同构:
			\begin{align*}
				\mathrm{HOM}_{\mathscr{O}_A}(\widetilde{M},\widetilde{N})(U)&=\mathrm{Hom}_{\mathscr{O}_A\mid_{D(f)}}(\widetilde{M}\mid_{D(f)},\widetilde{N}\mid_{D(f)})\\&=\mathrm{Hom}_{\mathscr{O}_{A_f}}(\widetilde{M_f},\widetilde{N_f})\\&=\mathrm{Hom}_{A_f}(M_f,N_f)
			\end{align*}
			
			再因为$M$是有限表示模,就有:
			\begin{align*}
				\mathrm{Hom}_{A_f}(M_f,N_f)&=\mathrm{Hom}_A(M,N)_f\\&=\widetilde{\mathrm{Hom}_A(M,N)}(D(f))
			\end{align*}
			
			容易验证这些典范同构和主开集之间的限制可交换,所以这些同构粘合为一个整体同构.
		\end{proof}
	\end{enumerate}
	\item 给定环同态$\varphi:A\to B$,它对应于态射$f:\mathrm{Spec}B\to\mathrm{Spec}A$.对$B$模$N$,有$f_*(\widetilde{N})=\widetilde{_AN}$,这里$_AN$是$N$经$\varphi$视为的$A$模.对$A$模$M$,有$f^*(\widetilde{M})=\widetilde{M\otimes_AB}$.
	\begin{proof}
		
		取$g\in A$,有$f^{-1}(D(g))=D(\varphi(g))$.于是有:
		$$\Gamma(D(g),f_*(\widetilde{N}))=\Gamma(D(\varphi(g)),\widetilde{N})=N_{\varphi(g)}=(_AN)_g=\Gamma(D(g),\widetilde{_AN})$$
		
		这说明$f_*(\widetilde{N})=\widetilde{_AN}$.再按照$f_*$与$f^*$的伴随性(另外用到了拟凝聚层的逆像总是拟凝聚层,下文有证),对任意$B$上的模层$\mathscr{F}$有:
		\begin{align*}
			\mathrm{Hom}_B(f^*\widetilde{M},\mathscr{F})&=\mathrm{Hom}_A(\widetilde{M},f_*\mathscr{F})\\&=\mathrm{Hom}_A(\widetilde{M},\widetilde{_A\mathscr{F}(X)})\\&=\mathrm{Hom}_A(M,_A\mathscr{F}(X))\\&=\mathrm{Hom}_B(B\otimes_AM,\mathscr{F}(X))\\&=\mathrm{Hom}_B(\widetilde{B\otimes_AM},\mathscr{F})
		\end{align*}
		
		按照Yoneda引理,存在自然同构$f^*\widetilde{M}\cong\widetilde{B\otimes_AM}$.
	\end{proof}
	\item 设$X=\mathrm{Spec}A$是仿射概型,那么函子$M\mapsto\widetilde{M}$和整体截面函子$\Gamma$是互相伴随的,换句话讲总存在如下自然同构:
	$$\mathrm{Hom}_A(M,\Gamma(X,\mathscr{F}))\cong\mathrm{Hom}_{\mathscr{O}_X}(\widetilde{M},\mathscr{F})$$
	\begin{proof}
		
		一方面给定$A$模同态$\varphi:M\to\mathscr{F}(X)$,任取主开集$D(f)$,那么$\mathscr{F}(D(f))$是一个$A_f$模,于是$\varphi$诱导了$A_f$模同态$M_f\to\mathscr{F}(D(f))$,这粘合成一个模层态射$\widetilde{M}\to\mathscr{F}$.另一方面给定模层态射$\widetilde{M}\to\mathscr{F}$,取整体截面得到$A$模同态$M\to\mathscr{F}(X)$.证明这两个映射互为逆映射.
	\end{proof}
	\item 设$X$是仿射概型,设$0\to\mathscr{F}_1\to\mathscr{F}_2\to\mathscr{F}_3\to0$是$\mathscr{O}_X$模层的短正合列,如果$\mathscr{F}_1$是拟凝聚层,那么整体截面函子作用其上是正合的.这件事也可以直接从$H^1(X,\mathscr{F}_1)=0$得到.
	$$\xymatrix{0\ar[r]&\Gamma(X,\mathscr{F}_1)\ar[r]&\Gamma(X,\mathscr{F}_2)\ar[r]&\Gamma(X,\mathscr{F}_3)\ar[r]&0}$$
	\begin{proof}
		
		我们解释过整体截面函子是左正合的.于是归结为证明$\Gamma(X,\mathscr{F}_2)\to\Gamma(X,\mathscr{F}_3)$是满射.任取$s\in\Gamma(X,\mathscr{F}_3)$,按照$\mathscr{F}_2\to\mathscr{F}_3$是满态射,于是在茎的层面是满同态,于是对每个$x\in X$,存在开邻域$D(f)$,使得$s\mid D(f)$可提升为一个截面$t\in\mathscr{F}_2(D(f))$.我们断言存在$n>0$使得$f^ns$提升为$\mathscr{F}_2$的一个整体截面.
		
		事实上,取$X$的有限主开集覆盖$\{D(g_i)\}$,对每个指标$i$,$s\mid D(g_i)$可提升为一个截面$t_i\in\mathscr{F}(D(g_i))$.现在$t,t_i$都是$s$在$D(fg_i)$上的提升,于是有$t-t_i\in\mathscr{F}_1(D(fg_i))$,按照$\mathscr{F}_1$是拟凝聚层,说明存在一个足够大的$n$使得对每个$i$都有$f^n(t-t_i)$可延拓为一个截面$u_i\in\mathscr{F}_1(D(g_i))$.记$t_i'=f^nt_i+u_i$,那么$t_i'$是$f^ns$在$D(g_i)$的提升,并且$t_i'$和$f^nt_i$在$D(fg_i)$的限制相同.现在在$D(g_ig_j)$上$t_i'$和$t_j'$都是$f^ns$的提升,于是$t_i'-t_j'\in\mathscr{F}_1(D(g_ig_j))$,另外$t_i',t_j'$在$D(fg_ig_j)$上的限制相同,我们解释过这导致存在一个足够大的$m$使得$f^m(t_i'-t_j')=0$对任意$i,j$成立.于是$f^mt_i'$粘合为一个$\mathscr{F}_2$的整体截面$t''$,它提升了$f^{n+m}s$.这证明了断言.
		
		现在设$X$有有限主开集覆盖$\{D(f_i),1\le i\le r\}$,满足$s\mid D(f_i)$提升为$\mathscr{F}_2$在$D(f_i)$的截面.按照我们的断言,存在一个足够大的$n$和整体界面$t_i\in\Gamma(X,\mathscr{F}_2)$使得$t_i$是$f_i^ns$的提升对任意$i$成立.按照$\{D(f_i)\}$是$X$的开覆盖,得到$\{D(f_i^n)\}$也是$X$的开覆盖,于是存在$a_i\in A$使得$1=\sum_{1\le i\le r}a_if_i^n$.取$t=\sum_ia_it_i$,则$t$是$\mathscr{F}_2$的整体截面使得它在$\mathscr{F}_3$中的像是$\sum_ia_if_i^ns=s$,完成证明.
	\end{proof}
\end{enumerate}
\newpage
\section{射影概形}
\subsection{$\mathrm{Proj}A$作为拓扑空间}

给定分次环$A=\oplus_{n\ge0}A_n$,记$A_+=\oplus_{n\ge1}A_n$,它是$A$的一个理想,称为无关理想.记全体不包含$A_+$的$A$中的齐次素理想(称为相关素理想)构成的集合为$\mathrm{Proj}(A)$,这是素谱$\mathrm{Spec}(A)$的子集.
\begin{enumerate}
	\item 对相关素理想$\mathfrak{p}\subseteq A$,记$\mathfrak{p}_+=\mathfrak{p}\cap A_+$(这当然未必还是素理想,这个理想满足对正次齐次元$a,b$如果都不在里面,那么$ab$也不在,但是$a,b$如果有一个是零次元,有一个是正次元,并且它们都不在里面,可能会出现$ab$仍在里面的情况).如果两个相关素理想$\mathfrak{p}$和$\mathfrak{p}'$满足$\mathfrak{p}_+=\mathfrak{p}_+'$,那么有$\mathfrak{p}=\mathfrak{p}'$.另外一个齐次理想$I\subsetneqq A_+$可以表示为$\mathfrak{p}_+$,其中$\mathfrak{p}$是相关素理想,当且仅当对任意正次齐次元$a,b\not\in I$,总有$ab\not\in I$.
	\begin{proof}
		
		任取$a\in\mathfrak{p}$,记$a=\sum a_i$是齐次分解,那么$a_i\in\mathfrak{p}_+=\mathfrak{p}'_+\subseteq\mathfrak{p}',i\ge1$.所以不妨设$a\in\mathfrak{p}$是零次齐次元.按照$A_+\not\subseteq\mathfrak{p}'$,可取正次齐次元$f\in A_+-\mathfrak{p}'$.那么$fa\in\mathfrak{p}$是正次齐次元,导致$fa\in\mathfrak{p}'$,但是$f\not\in\mathfrak{p}'$,迫使$a\in\mathfrak{p}'$.这就得到$\mathfrak{p}\subseteq\mathfrak{p}'$.同理得到另一侧的包含关系.
		
		\qquad
		
		下面证明第二个命题.首先如果$I=\mathfrak{p}_+$,那么明显的,如果$a,b$是不在$\mathfrak{p}$中的正次齐次元,则有$ab$是不在$\mathfrak{p}$中的正次齐次元.反过来设齐次理想$I$满足这个条件.由于$I\subsetneqq A_+$,可取$f\in A_+-I$.我们取$\mathfrak{p}_0=\{a\in A_0\mid f^ra\in I_{r\deg f},\forall r\ge1\}$.倘若我们能证明$\mathfrak{p}=\mathfrak{p}_0\oplus I$是一个齐次素理想(它明显是一个齐次理想),那么$I=\mathfrak{p}\cap A_+=\mathfrak{p}_+$就得证.我们设$g_s$和$h_t$是两个不在$\mathfrak{p}$中的齐次元,次数分别约定为$s$和$t$.我们只需证明$g_sh_t\not\in\mathfrak{p}$.下面断言存在自然数$a$使得$f^ag_s$是正次齐次元并且不在$I$中.这是因为如果$s\ge1$则就取$a=0$,如果$s=0$,那么$g_0\not\in\mathfrak{p}_0$导致存在正整数$a$使得$f^ag_0\not\in I$.同理存在自然数$b$使得$f^bh_t$是正次齐次元并且不在$I$中.于是按照$I$的条件知$f^{a+b}g_sh_t\not\in I\subseteq\mathfrak{p}$,从而$g_sh_t\not\in\mathfrak{p}$.
	\end{proof}
	\item 设$S\subseteq A_+$是一个非空的乘性闭子集,设$S$不包含所有正次齐次元.那么全体和$S$不交的严格包含于$A_+$的齐次理想构成的集合是非空的.按照Zorn引理这个集合存在极大元,我们断言每个极大元$I$一定具有形式$\mathfrak{p}_+$,其中$\mathfrak{p}$是相关素理想.
	\begin{proof}
		
		我们来用上一条结论.设正次齐次元$a,b\not\in I$,按照$I$的极大性,$I+(a)$和$I+(b)$都要和$S$有交,于是存在$c,d\in I$和$f,g\in A$使得$h=(c+fa)(d+gb)\in S$.但是如果$ab\in I$,导致$h\in I$,就和$I\cap S$是空集矛盾.于是$ab\not\in I$,得证.
	\end{proof}
	\item 设$I\subseteq A_+$是齐次理想,那么$\sqrt{I}$就是所有包含$I$的相关素理想的交,于是它也是齐次理想.特别的有$(\sqrt{I})_+=\sqrt{I}\cap A_+$也是齐次素理想.
	\begin{proof}
		
		用$A/I$替代$A$可不妨设$I=0$.一方面$\sqrt{0}$一定包含在所有相关素理想的交$\mathfrak{a}$中.反过来如果正次齐次元$f\not\in\sqrt{0}$,取$S=\{1,f,f^2,\cdots\}$,那么和$S$不交的齐次理想是非空的,并且极大元具有形式$\mathfrak{p}_+$,其中$\mathfrak{p}$是相关素理想.于是$f\not\in\mathfrak{p}$.
	\end{proof}
	\item 对$A$的子集$E$,记$V_+(E)=\{\mathfrak{p}\in\mathrm{Proj}A\mid E\subseteq\mathfrak{p}\}=V(E)\cap\mathrm{Proj}A$.这里$V(E)$是素谱$\mathrm{Spec}A$上定义的闭子集.我们有一系列熟知的事实:
	\begin{itemize}
		\item 如果子集$E$的所有元的所有齐次分支生成的齐次理想为$\mathfrak{a}$,那么有:
		$$V_+(E)=V_+(\mathfrak{a})=V_+(\mathfrak{a}_+)=V_+(\sqrt{\mathfrak{a}})=V_+(\sqrt{\mathfrak{a}}_+)$$
		\item 如果$I_i$是一族齐次理想,那么$\cap_iV_+(I_i)=V_+(\sum_iI_i)$.
		\item 如果$I,J$是齐次理想,那么$V_+(I)\cup V_+(J)=V_+(I\cap J)$.
		\item $\mathrm{Proj}A$是空集当且仅当$A_+$中的元都是幂零元.
	\end{itemize}
	\item 按照$V_+(E)=V(E)\cap\mathrm{Proj}A$,说明全体$\{V_+(E)\}$也满足闭集公理,这使得$\mathrm{Proj}A$是$\mathrm{Spec}A$的子空间.它也称为Zariski拓扑.
	\item 算子$I_+$.设$Y\subseteq\mathrm{Proj}A$是子集,定义$I_+(Y)=\left(\cap_{\mathfrak{p}\in Y}\mathfrak{p}\right)\cap A_+$.
	\begin{itemize}
		\item 设$I\subseteq A_+$是齐次理想,那么有$I_+(V_+(I))=(\sqrt{I})_+$.如果$Y\subseteq\mathrm{Proj}A$,那么$V_+(I)(I_+(Y))=\overline{Y}$.
		\item $Y\mapsto I_+(Y)$和$I\mapsto V_+(I)$定义了全体满足$I=(\sqrt{I})_+$的齐次理想$I\subseteq A_+$和$\mathrm{Proj}A$的全体闭子集之间的双射.在这个对应下,$\mathrm{Proj}A$的不可约闭子集对应于理想$\mathfrak{p}_+$,其中$\mathfrak{p}$是相关素理想.
		\item 如果$I\subseteq A_+$是齐次理想,那么$V_+(I)=\emptyset$当且仅当$(\sqrt{I})_+=A_+$.特别的$\mathrm{Proj}A$是空集当且仅当$A_+$中的元都是幂零元.
	\end{itemize}
	\item 对齐次元$f\in S$,定义$D_+(f)=\mathrm{Proj}(S)-V_+((f))$为齐次主开集.当$f$跑遍正次数的齐次元时,$\{D_+(f)\}$构成拓扑基.另外如果固定一个正整数$d$,所有$dn$次数的齐次元的主开集也构成拓扑基.
	\begin{proof}
		
		要证明两件事,首先对任意$p\in D_+(f)\cap D_+(g)$,存在$D_+(fg)$使得$p\in D_+(fg)=D_+(f)\cap D_+(g)$.第二件事是任意开集可以表示为这种形式的开集的并.首先闭集总是具有形式$V_+(I)$,这里$I$是一个齐次理想.于是开集总具有形式$V_+(I)^c=\mathrm{Proj}(S)-V_+(I)$,任取这个开集中的素齐次理想$p$,那么从$I\not\subseteq p$和$S_+\not\subseteq p$得到$IS_+\not\subseteq p$,于是可取正次数的齐次元$f\in IS_+$但$f\not\in p$,这得到$p\in D_+(f)\subseteq V_+(I)^c$.把这个$f$记作$f_p$,那么有$V_+(I)^c=\cup_{p\in D_+(I)}D_+(f_p)$.这就验证了它是拓扑基.第二个命题只要注意到$D_+(f)=D_+(f^d)$.
	\end{proof}
	\item 设$A$是$\mathbb{Z}$分次环,设$f$是可逆正次齐次元,那么存在从$A_0$的素理想到$A$的齐次理想之间的双射.这件事说明取$\mathbb{Z}_{\ge0}$分次环$S$,取正次齐次元$f$,那么$S_f$是$\mathbb{Z}$分次环,这里$1/f$的次数约定为$-\deg f$,那么$S_{(f)}$的素理想一一对应于$S_f$的齐次素理想.
	\begin{proof}
		
		我们有一个典范的环同态$A_0\to A$,于是$A$中每个齐次素理想拉回为$A_0$中的素理想.反过来取$A_0$的素理想$p_0$,定义$p=\oplus_iq_i\subseteq A$,其中$q_i=\{a\in A_i\mid a^{\deg f}/f^i\in p_0\}$.那么$q_0=p_0$.下面证明$p$是一个理想:如果$a\in q_i$和$b\in q_j$,那么$ab\in q_{i+j}$;如果$a\in q_i$和$r\in A_j$,那么$ra\in q_i$;如果$a,b\in q_i$,那么$a^2+2ab+b^2\in q_{2i}$,导致$a+b\in q_i$.$p$是齐次理想是直接的,因为$p$本身已经表示为直和形式.最后验证$p$是齐次素理想:如果$a\in A_i$和$b\in A_j$使得$ab\in q_{i+j}$,也即$(ab)^{\deg f}/f^{i+j}\in p_0$,于是$a\in A_i$和$b\in A_j$至少一个成立.最后验证这两个映射互为逆映射即可.
	\end{proof}
	\item 给定正次数的齐次元$f\in S$,包含映射$D_+(f)\subseteq D(f)=\mathrm{Spec}(S_f)$和环同态$S_{(f)}\to S_f$诱导的连续映射$\mathrm{Spec}(S_f)\to\mathrm{Spec}(S_{(f)})$(此即把$S_f$中的素理想$Q$映射为$Q\cap S_{(f)}$,换句话讲这是把一个分次环中的一般理想$I$映射为了它所包含的全部齐次元生成的齐次理想$I^h$,另外$I$是素理想则$I^h$是素理想)的复合映射记作$\varphi$,它是$D_+(f)\cong\mathrm{Spec}(S_{(f)})$的同胚映射.换句话讲,$\varphi$把$D_+(f)$中的素理想$p$映射为$pS_f\cap S_{(f)}$.
	$$\xymatrix{D_+(f)\ar[r]&D(f)\cong\mathrm{Spec}(S_f)\ar[r]&\mathrm{Spec}(S_{(f)})\\ p&pS_f&pS_f\cap S_{(f)}}$$
	\begin{proof}
		
		验证$\varphi$是单射.如果$p_1,p_2\in D_+(f)$满足$\varphi(p_1)=\varphi(p_2)$.任取$x\in p_1$,那么有$\frac{x^{\deg f}}{f^{\deg x}}\in p_1S_f\cap S_{(f)}\subseteq p_2S_f$(这里用到了$\deg f>0$),这导致$x\in p_2$,于是$p_1\subseteq p_2$,同理$p_2\subseteq p_1$,于是$p_1=p_2$.
		
		验证$\varphi$是满射.任取$S_{(f)}$中的素理想$q$,我们先证明$qS_f\cap S_{(f)}=q$.这是因为$qS_f=q(S_{(f)}\oplus S_+)$导致$qS_f\cap S_{(f)}\subseteq qS_{(f)}=q$.下面证明$\sqrt{qS_f}$是$S_f$中的齐次素理想(这里齐次因为齐次理想的根理想是齐次的).任取$S_f$中的两个齐次元$a,b$使得$ab\in\sqrt{qS_f}$,那么有$a^rb^r\in qS_f$,于是$\left(a^{r\deg f}f^{-r\deg a}\right)\left(b^{r\deg f}f^{-r\deg b}\right)\in qS_f\cap S_{(f)}=q$,按照$q$是素理想,得到二者之一属于$q$,不妨设$a^{r\deg f}f^{-r\deg a}\in q$,于是$a^{r\deg f}\in qS_f$,于是$a\in\sqrt{qS_f}$.这里说明$\sqrt{qS_f}\cap S_{(f)}=q$,因为如果$a\in\sqrt{qS_f}\cap S_{(f)}$,那么存在某个次幂$a^n\in qS_f\cap S_{(f)}=q$,导致$a\in q$,于是$\sqrt{qS_f}\cap S_{(f)}=q$.接下来按照$D(f)\cong\mathrm{Spec}(S_f)$是经$\rho:S\to S_f$诱导的,于是素理想$p=\rho^{-1}(qS_f)$满足$p\mapsto\sqrt{qS_f}$.最后需要说明$p$是$\mathrm{Spec}(S)$中的不包含$S_+$的齐次理想,这只需注意到$\rho$下齐次元的原像还是齐次元,另外$p$不含$f$于是自然不包含$S_+$.
		$$\xymatrix{D_+(f)\ar[r]&D(f)\cong\mathrm{Spec}(S_f)\ar[r]&\mathrm{Spec}(S_{(f)})\\p=\rho^{-1}(\sqrt{qS_f})&\sqrt{qS_f}&q}$$
		
		验证$\varphi$是同胚.我们已经得到$\varphi$是连续双射,接下来仅需验证它是开映射.为此我们仅需说明$\varphi$把$D_+(f)$中主开集都映射为$\mathrm{Spec}(S_{(f)})$的开集(因主开集构成拓扑基).对任意的$D_+(g)\subseteq D_+(f)$,取$\alpha=g^rf^{-\deg g}\in S_{(f)}$,我们来证明$\varphi(D_+(g))=D(\alpha)$.为此只需注意到任取$p\in D_+(f)$,那么$g\in p$当且仅当$\alpha\in\varphi(p)$,于是$p\in D_+(g)$当且仅当$\varphi(p)\in D(\alpha)$,于是$\varphi(D_+(g))=D(\alpha)$.
	\end{proof}
\end{enumerate}
\subsection{$\mathrm{Proj}A$作为概形}

$\mathrm{Proj}A$作为概形称为射影概形.
\begin{enumerate}
	\item 定义层.对开集$U\subseteq\mathrm{Proj}(S)$,定义截面环$\mathscr{O}_{\mathrm{Proj}(S)}(U)$是全体这样的映射$s:U\to\coprod_{p\in\mathrm{Proj}(S)}S_{(p)}$,这里$S_{(p)}$表示局部化$S_p$自然的作为$\mathbb{Z}$分次环(对$S$中齐次元$a$和$S-p$中齐次元$f$,定义$a/f$的次数是$\deg a-\deg f$)后全体零次元构成的子环.$s$满足如下两个条件:对每个$p\in U$有$s(p)\in S_{(p)}$;对每个$p\in U$,存在$p$的开邻域$U_p\subseteq U$,以及同次数的齐次元$a,f\in S$,使得对每个$q\in U_p$,有$f\not\in q$,并且恒有表达式$s(q)=\frac{a}{f}\in S_{(q)}$.这里限制映射就定义为映射限制在更小的开集上.容易验证这构成了一个层,于是$(\mathrm{Proj}(S),\mathscr{O}_{\mathrm{Proj}(S)})$是环空间.
	\item 对每个$p\in\mathrm{Proj}(S)$,有典范的同构$\mathscr{O}_{\mathrm{Proj}(S),p}\cong S_{(p)}$.这个证明完全类似素谱的情况.于是$(\mathrm{Proj}(S),\mathscr{O}_{\mathrm{Proj}(S)})$是局部环空间.
	\item 我们之前构造的$D_+(f)\cong\mathrm{Spec}(S_{(f)})$实际上是如下局部环同构.这导致$\mathrm{Proj}S$是概形.
	$$(D_+(f),\mathscr{O}_{\mathrm{Proj}(S)}\mid_{D_+(f)})\cong(\mathrm{Spec}(S_{(f)}),\mathscr{O}_{\mathrm{Spec}(S_{(f)})})$$
	\begin{proof}
		
		这个映射可以视为两个局部环空间态射的复合,而这两个态射在stalk上都是同构,开嵌入的部分在stalk上肯定是同构,而另一个态射在stalk上是同构因为我们解释过$S_{(f)}$的素理想和$S_f$的齐次素理想一一对应.
	\end{proof}
	\item 如果$A$是分次环,那么$X=\mathrm{Proj}A$总是一个分离概形.
	\begin{proof}
		
		$\{D_+(f)\}$构成了仿射开覆盖,其中$f$跑遍正次齐次元.满足$D_+(f)\cap D_+(g)=D_+(fg)$仍然是仿射的,并且对应的同态$A_{(f)}\otimes_{\mathbb{Z}}A_{(g)}\to A_{(fg)}$是满射,于是$X$是分离概形.
	\end{proof}
    \item 和仿射概形一样,射影概形的结构层也可以从主开集定义,再唯一延拓到所有开集上.设$X=\mathrm{Proj}A$,其中$A$是分次环.
    \begin{enumerate}
    	\item 对主开集$D_+(f)$,其中$f$是正$d$次齐次元,定义$\mathscr{O}_X(D_+(f))=A_{(f)}$,这是$A_f$赋予诱导分次结构后的零次子环,也即$\{a/f^k\in A_f\mid a\in A_{kd}\}$.
    	\item 如果正次数齐次元$f,g$满足$D_+(f)\subseteq D_+(g)$,等价于讲$\sqrt{f}\cap A_+\subseteq\sqrt{g}\cap A_+$,也即存在齐次元$h$和正整数$n$满足$f^n=hg$,那么$A_f$是$A_g$的局部化,有典范的保次数的同态$A_g\to A_f$,具体写出来就是$a/g^k\mapsto ah^k/f^{nk}$.这限制为零次子环之间的同态$A_{(g)}\to A_{(f)}$,而这视为限制映射.
    	\item 这些信息可以构成主开集上的层,于是它唯一延拓到所有开集上.
    	\item 另外我们也可以在$f$跑遍正次元时对每个主开集$D_+(f)$赋予仿射概形结构$\mathrm{Spec}A_{(f)}$.然后粘合得到$\mathrm{Proj}A$上的概形结构.
    \end{enumerate}
\end{enumerate}
\subsection{射影空间}
\begin{enumerate}
	\item 环$R$上的$n$维射影空间定义为$\mathbb{P}_R^n=\mathrm{Proj}R[x_0,x_1,\cdots,x_n]$.
	\item 我们还可以像射影线一样粘合定义射影空间,环$R$上的$n$维射影空间$\mathbb{P}_R^n$定义为$n+1$个仿射空间$\mathbb{A}_R^n$的粘合:对每个$0\le i\le n$,记$U_i$是坐标为$\{x_{0/i},x_{1/i},\cdots,x_{i-1/i},x_{i+1/i},\cdots,x_{n/i}\}$的$n$维仿射空间.对$j\not=i$,取$U_i$的开子集$U_{ij}=D(x_{j/i})$.那么有典范的同构$\varphi_{ij}:U_{ij}\to U_{ji}$.粘合得到的概型即$\mathbb{P}_R^n$.
	\item 类似射影线的情况,有$\Gamma(\mathbb{P}_R^n,\mathscr{O})=R$,于是如果$R$是一个域,那么射影空间总不是仿射的.
	\item 一个非常平凡的情况.我们有$\mathbb{P}^0_R=\mathrm{Proj}R[X]=\mathrm{Spec}(R)$.因为$\mathrm{Proj}R[X]$上唯一的主开集是$D(X)\cong\mathrm{Spec}A$.
\end{enumerate}
\subsection{分次环同态诱导的态射}
\begin{enumerate}
	\item 分次环之间的同态.设$S$和$R$是两个$\mathbb{Z}_{\ge0}$分次环.一个分次环同态$\varphi:S\to R$是指一个环同态,使得存在正整数$d>0$,有$\varphi(S_n)\subseteq R_{dn}$.
	\item 如果$\varphi:S\to R$是分次环同态,如果$p$是$R$中的齐次素理想,那么$\varphi^{-1}(p)$的确是$S$的齐次素理想.但是问题是如果$p$是不包含无关理想$R_+$的齐次素理想,可能会出现$\varphi^{-1}(p)$反而包含无关理想$S_+$的情况.所以严格讲我们只能得到$\mathrm{Proj}R$的开子集$\mathrm{Proj}R-V_+(\varphi(S_+))\to\mathrm{Proj}S$的映射.倘若$V_+(\varphi(S_+))=\emptyset$(例如$\sqrt{\varphi(S_+)}=R_+$会推出这个条件),那么此时诱导的映射的确是两个射影概型之间的映射$\mathrm{Proj}R\to\mathrm{Proj}S$.另外这个映射恰好就是$\varphi$作为环同态诱导的素谱之间的连续映射$\mathrm{Spec}R\to\mathrm{Spec}S$,在$\mathrm{Proj}R-V_+(\varphi(S_+))$的限制.
	\item 现在我们说明上述映射$^a\widetilde{\varphi}$是概形态射.任取齐次元$f\in S_+$,记$f'=\varphi(f)$也是齐次元.那么$\varphi$诱导了局部化之间的分次环同态$\varphi_f:S_f\to R_{f'}$,限制在零次子环上得到环同态$\varphi_{(f)}:S_{(f)}\to R_{(f')}$,它诱导了素谱之间的态射$^a\varphi_{(f)}$.有$^a\widetilde{\varphi}^{-1}(D_+(f))=D_+(f')$(这还说明分次环同态诱导的态射必然是仿射的),那么有如下图表交换:
	$$\xymatrix{D_+(f')\ar[rr]^{^a\widetilde{\varphi}}\ar[d]&&D_+(f)\ar[d]\\\mathrm{Spec}R_{(f')}\ar[rr]^{^a\varphi_{(f)}}&&\mathrm{Spec}S_{(f)}}$$
	
	所以我们验证了$^a\widetilde{\varphi}$在主开集上的限制的确是概形的态射.要想这些态射粘合还要说明两个主开集上的态射在交集上是一致的,而这只是因为有如下交换图表:
	$$\xymatrix{S_f\ar[rr]\ar[d]&&R_{f'}\ar[d]\\S_{fg}\ar[rr]&&R_{f'g'}}$$
	\item 特别的如果$\varphi:S\to R$是满的分次环同态,那么$V_+(\varphi(S_+))=\emptyset$,此时在主开集上的限制都被满的环同态诱导,所以此时诱导的$^a\widetilde{\varphi}$是闭嵌入.并且它的像集恰好就是$V_+(K)$,其中$K=\ker\varphi$.
	\begin{proof}
		
		$\varphi$诱导的$^a\varphi$是$\mathrm{Spec}R\to\mathrm{Spec}S$的像集为$V(K)$的映射.我们需要验证的是$^a\varphi(\mathrm{Proj}R)=V_+(K)$.这里左侧包含于右侧是平凡的.按照$^a\varphi$是单射,证明右侧包含于左侧等价于证明$^a\varphi^{-1}(V_+(K))\subseteq\mathrm{Proj}R$.任取左侧的$q$,也即$\varphi^{-1}(q)$是包含了$K$的齐次素理想,但是按照$\varphi$是满射,得到$q=\varphi(\varphi^{-1}(q))$也是齐次理想.这得证.
	\end{proof}
	\item 分次环同态和态射之间的对应是非常糟糕的.不同的分次环同态可能会诱导相同的态射;存在射影概形之间的态射不被任何分次环同态诱导.
	\begin{itemize}
		\item 如果$\varphi:S\to T$是分次环之间的同态,满足$\varphi(S_n)\subseteq T_n$.如果对$d\ge d_0$都有$\varphi_d:S_d\to T_d$是同构.那么有$V_+(\varphi(S_+))=\emptyset$,并且此时诱导的$(f,f^{\#}):\mathrm{Proj}T\to\mathrm{Proj}S$是同构.
		\item 例如设$A$是分次环,设$I$是分次理想,任取正整数$d_0>0$,记$I'=\oplus_{d\ge d_0}I_d$,那么$I'$也是齐次理想,并且$A\to A/I$和$A\to A/I'$都定义了相同的闭嵌入.
	\end{itemize}
\end{enumerate}
\subsection{有限性条件和Veronese子环}

我们始终设$A=\oplus_{n\ge0}A_n$是分次环,设$A_+$是无关理想.
\begin{enumerate}
	\item 一个由正次齐次元构成的子集$E\subseteq A_+$生成了整个$A_+$当且仅当$E$在$A_0$上代数生成了整个$A$.
	\begin{proof}
		
		充分性是显然的.下面证明必要性:为此我们对$d$做归纳证明$A_d$是被$E$中次数$\le d$的元在$A_0$上生成的模.按照$E$生成了理想$A_+$,说明$A_d=\sum_{i=0}^{d-1}A_i(E\cap A_{d-i})$,按照归纳假设就得证.
	\end{proof}
    \item $A_+$是有限生成理想当且仅当$A$是有限型$A_0$代数.这个条件成立时我们称$A$是有限生成分次环.
    \begin{proof}
    	
    	因为一个齐次理想是有限生成的当且仅当它存在有限生成元集由齐次元构成.于是这一条由上一条得到.
    \end{proof}
    \item $A$是诺特环当且仅当$A_0$是诺特环并且$A$是有限型$A_0$代数.
    \begin{proof}
    	
    	充分性就是希尔伯特基定理.必要性是因为如果$A$是诺特的,那么$A_0=A/A_+$当然也是诺特的.而上一条说明从$A_+$是有限生成理想得到$A$是有限型$A_0$代数.
    \end{proof}
    \item 设$A$是有限生成分次环.设$M=\oplus_{n\in\mathbb{Z}}M_n$是有限生成$A$分次模.我们断言$A_0$模$M_n$总是有限生成的,并且存在整数$n_0$,使得$n\le n_0$时总有$M_n=0$.另外存在整数$n_1$和$d>0$,使得对$n\ge n_1$时总有$A_dM_n=M_{n+d}$(一般的我们只能有$A_dM_n\subseteq M_{n+d}$).
    \begin{proof}
    	
    	可设有正次齐次元$f_1,\cdots,f_r\in A$使得$A=A_0[f_1,\cdots,f_r]$.设$\deg f_i=d_i>0$.再设$x_1,\cdots,x_s\in M$是齐次元生成了$A$模$M$.记$\deg x_j=n_j$(这未必是正的).那么$M_n$中的元一定可以表示为形如$f_1^{\alpha_1}\cdots f_r^{\alpha_r}x_j$的$A_0$线性组合,这个元要满足$n_j+\sum_i\alpha_id_i=n,\alpha_i\ge0$.满足这个等式的$(\alpha_1,\cdots,\alpha_r,j)$至多只有有限个,于是$M_n$作为$A_0$模是有限生成的.另外当$n<\max\{n_j\}$时不存在满足等式的$(\alpha_1,\cdots,\alpha_r,j)$,所以此时$M_n=0$.
    	
    	\qquad
    	
    	取$d$是$d_1,\cdots,d_r$的最小公倍数,再记$g_i=f_i^{d/d_i}$,那么$\deg g_i=d,\forall i$.再设$E=\{f_1^{\beta_1}\cdots f_r^{\beta_r}x_j\mid0\le\beta_i\le d/d_i,\forall i\}$,这是一个有限集合,设$E$中元素的最大次数是$n_1$.设$n\ge n_1$,原本$M_{n+d}$中的元要表示为$f_1^{\alpha_1}\cdots f_r^{\alpha_r}x_j$的$A_0$线性组合,其中次数要满足$n_j+\sum_i\alpha_id_i=n+d$.我们用$\alpha_i$对$d/d_i$做带余除法,记作$\alpha_i=k_i(d/d_i)+\beta_i$,那么$0\le\beta_i<d/d_i$,并且有$f_1^{\alpha_1}\cdots f_r^{\alpha_r}x_j$可以表示为$f_1^{\beta_1}\cdots f_r^{\beta_r}x_j$乘以一个关于$g_1,\cdots,g_r$的单项式(而$n\ge n_1$的目的是这个单项式至少是1次的),这就导致$M_{n+d}\subseteq A_dM_n$.
    \end{proof}
    \item 设$A=\oplus_{n\ge0}A_n$是分次环.我们定义$A$的第$\delta$个Veronese子环为$A^{(\delta)}=\oplus_{n\ge0}A_{\delta n}$.我们先断言如果$A$是有限生成分次环,那么$A^{(\delta)}$也是有限生成分次环.
    \begin{proof}
    	
    	等价于证明$A^{(\delta)}_+$是有限生成$A^{(\delta)}$理想.按照条件我们知道$A_+$是有限生成$A$理想,所以可取$f_1,\cdots,f_r$是正次齐次元生成了整个$A_+$,设$\deg f_i=d_i$,设$d_1,\cdots,d_r$的最小公倍数是$d$.我们解释过每个$A_n$都是有限$A_0$模,另外类似上一条的证明可以说明存在正整数$n_1$使得$n\ge n_1$时有$A_{d\delta}A_n=A_{n+d\delta}$.那么对任意正整数$m$,记带余除法$m=dk+r$,其中$0\le r<d$,那么有$A_{m\delta}=(A_{d\delta})^kA_{r\delta}$.于是$A_{\delta},\cdots,A_{d\delta}$作为有限$A_0$模的生成元凑在一起是$A^{(\delta)}_+$的生成元,并且这是有限集合.
    \end{proof}
    \item 设$A$是有限生成分次环,设$A^{(\delta)}$是Veronese子环,我们断言包含映射$A^{(\delta)}\to A$诱导了射影概形之间的同构$\mathrm{Proj}A\to\mathrm{Proj}A^{(\delta)}$.
    \begin{proof}
    	
    	任取$f\in A$是次数被$\delta$整除的齐次元,那么$A_{(f)}=(A^{(\delta)})_{(f)}$.另一方面对每个$r$次元$f$,有$\deg f^{\delta}=r\delta$被$\delta$整除,并且$D(f)=D(f^{\delta})$,于是当$f$跑遍次数被$n$整除的齐次元时,有$D(f)$覆盖整个空间.
    \end{proof}
    \item 设分次环$A$被有限个正次齐次元$f_1,\cdots,f_r$在$A_0$上代数生成,我们断言存在一个Veronese子环$A^{(\delta)}$,满足$A^{(\delta)}$是被有限个$A^{\delta}_1$中的元在$A_0$上代数生成的.我们解释过有典范同构$\mathrm{Proj}A\cong\mathrm{Proj}A^{(\delta)}$,于是这件事告诉我们如果射影概形$X=\mathrm{Proj}A$,其中$A$是有限型$A_0$代数,那么我们可以设这有限个生成元都是1次元.
    \begin{proof}
    	
    	我们解释过存在正整数$\delta'$和正整数$n_1$使得$n\ge n_1$时有$A_{\delta'}A_n=A_{n+\delta'}$.于是存在正整数$d_1$使得$d\ge d_1$时有$A_{d\delta'}=(A_{\delta'})^d$.下面取$\delta=d_1\delta$,那么对任意$d\ge1$,我们有$A_{d\delta}=A_{dd_1\delta'}=(A_{\delta'})^{dd_1}=(A_{\delta})^d$.于是$A^{(\delta)}=\oplus_{n\ge0}A_{\delta n}=\oplus_{n\ge0}(A_{\delta})^n$,而每个$(A_{\delta})^n$都能被$A_{\delta}$生成,这说明
    \end{proof}
    \item 设$A$是有限生成分次环,那么存在闭嵌入$\mathrm{Proj}A\to\mathbb{P}_{A_0}^n$,换句话讲有限生成分次环诱导的射影概形都是射影空间的闭子概型.特别的此时$\mathrm{Proj}A$总是有限型$A_0$概形.另外如果$A_0$是诺特环,那么$\mathrm{Proj}A$是诺特概形.
    \begin{proof}
    	
    	按照上一条我们可以设$A$是被有限个1次元代数生成的,记作$f_0,f_1,\cdots,f_n\in A_1$.那么有分次满同态$A_0[X_0,\cdots,X_n]\to A$为$X_i\mapsto f_i$.于是这诱导了闭嵌入$\mathrm{Proj}A\to\mathrm{Proj}A_0[X_0,\cdots,X_n]=\mathbb{P}_{A_0}^n$.又因为$\mathbb{P}_{A_0}^n$是有限型$A_0$概形,得到$\mathrm{Proj}A$是有限型$A_0$概形.最后如果$A_0$是诺特环,那么有$\mathbb{P}_{A_0}^n$是诺特概形,进而有闭子概型$\mathrm{Proj}A$也是诺特概形.
    \end{proof}
\end{enumerate}
\subsection{射影概形的闭子概型}
\begin{enumerate}
	\item 引理.设$A$是被有限个一次元生成的分次环,记$X=\mathrm{ProjA}$,记$M$是分次$A$模,如果$\mathscr{G}\subseteq\widetilde{M}$是拟凝聚子层,那么存在齐次$A$子模$N\subseteq M$,使得$\widetilde{N}=\mathscr{G}$.如果$M$是饱和分次模,那么$N$也可以选取为饱和的.
	\begin{proof}
		
		取典范同态$\alpha:M\to\Gamma_*(\widetilde{M})$,它的像集记作$P$.按照射影概形上$M\mapsto\widetilde{M}$和滤过余极限可交换,有$\widetilde{P}$是$\widetilde{\alpha}$的像,但是$\widetilde{\alpha}$是同构,就导致$\widetilde{P}=\widetilde{M}$.下面取$N=\alpha^{-1}(\Gamma_*(\mathscr{G}))$.于是$\alpha$在$N$上的限制就诱导了单态射$\widetilde{N}\to\widetilde{\Gamma_*(\mathscr{G})}=\mathscr{G}$(这里用到$\mathscr{G}$是拟凝聚的),按照$M\mapsto\widetilde{M}$与交可交换(因为正合性),于是$\alpha\mid_N$的像是$\widetilde{(\Gamma_*(\mathscr{G})\cap P)}=\mathscr{G}\cap\widetilde{P}=\mathscr{G}$.于是取分次子模$N=\Gamma_*(\mathscr{G})\cap P$就有$\widetilde{N}=\mathscr{G}$.最后如果$M$是饱和的,那么这里$\alpha$是同构,那么限制的$N\to\Gamma_*(\mathscr{G})$就也是同构,于是$N$也是饱和的.
	\end{proof}
	\item 设$A$是被有限个一次元生成的分次环,设$X=\mathrm{Proj}A$,设$Z\subseteq X$是闭子概型,那么存在不包含$A_+$的齐次理想$I_Z\subseteq A$,使得$Z=\mathrm{Proj}A/I_Z$.并且如果$A$作为自身模是饱和的(比方说$A$是某个环上的多项式环),那么存在唯一的饱和分次理想$I_Z$满足$Z=\mathrm{Proj}A/I_Z$.
	\begin{proof}
		
		先设$Z$对应的拟凝聚理想层为$\mathscr{I}$.按照上一条引理就存在分次理想$I_Z\subseteq A$使得$\widetilde{I}=\mathscr{I}$.于是有$Z=\mathrm{Proj}A/I_Z$(严格讲是$\mathscr{O}_X/\mathscr{I}=\widetilde{A}/\widetilde{I_Z}\cong\widetilde{A/I_Z}$,最后一个同构是因为$M\mapsto\widetilde{M}$是正合的).最后如果$A$是饱和的,上一条引理告诉我们$I_Z$也可以选取饱和的.再按照饱和分次模与拟凝聚层一一对应,就得到饱和的$I_Z$是唯一的.
	\end{proof}
	\item 关于饱和齐次理想.设$A=R[T_0,\cdots,T_n]$是环$R$上的多项式环,记$X=\mathrm{Proj}A=\mathbb{P}_R^n$.任取$A$的齐次理想$J$,那么$Z=\mathrm{Proj}A/J$是$X$的闭子概型.另外这里$J$可能不是唯一的,但是存在唯一的饱和齐次理想$I=I_Z$使得$Z=\mathrm{Proj}A/I$.我们称$I$是$J$的饱和化(saturation).它可以表示为$\Gamma_*(\mathscr{I})$其中$\mathscr{I}$是$Z$对应的拟凝聚理想层,$I$也是所有定义了$Z$的齐次理想的并,$I$还有关于$J$的如下具体描述:
	$$I=J^{\mathrm{sat}}=\{f\in A\mid\exists N\ge0,\forall i,T_i^Nf\in J\}$$
	\item 设$A=k[T_0,\cdots,T_n]$是域$k$上的多项式环,设$X=\mathrm{Proj}A=\mathbb{A}_k^n$.设$Z\subseteq X$是闭子概型,对应的拟凝聚理想层为$\mathscr{I}$,对应的饱和齐次理想为$I$.那么$I_d-\{0\}$恰好由那些在$Z$上恒为零的$d$次齐次多项式构成,并且两个$d$次齐次多项式定义了相同的超曲面当且仅当它们相差一个$k^{\times}$中的元.另外按照$I$是饱和的我们有$I_d\cong\Gamma(\mathbb{P}_k^n,\mathscr{I}(d))$,综上我们有如下一一对应:
	$$\{\text{全体包含}Z\text{的}d\text{次超曲面}\}\leftrightarrow\left(\Gamma(\mathbb{P}_k^n,\mathscr{I}(d))-\{0\}\right)/k^{\times}$$
	\item 推论.设$Z$是射影空间$X=\mathbb{P}_R^n$的闭子概型.那么存在齐次理想$I\subseteq A=R[T_0,\cdots,T_n]$使得$Z$同构于$\mathrm{Proj}A/I$.
	\begin{proof}
		
		设$Z$对应的拟凝聚理想层为$\mathscr{I}$.按照整体截面函子是左正合的,得到$\Gamma_*(\mathscr{I})$是$\Gamma_*(\mathscr{O}_X)$的齐次子模.而我们证明过$\Gamma_*(\mathscr{O}_X)=A$,所以$\Gamma_*(\mathscr{I})$是$A$的齐次理想,记作$I$.按照$\beta$是同构就有$\mathscr{I}=\widetilde{I}$.最后$\mathscr{O}_X/\mathscr{I}=\widetilde{A/I}$对应的闭子概型恰好就是$V_+(I)=\mathrm{Proj}A/I$.
	\end{proof}
	\item 推论.概形$X$是$\mathbb{P}_R^n$的闭子概型当且仅当存在分次环$A$,满足$A_0=R$,并且$A$被有限个一次元生成,使得$X\cong\mathrm{Proj}A$.
	\begin{proof}
		
		必要性:如果$X$是$\mathbb{P}_R^n$的闭子概型,可取齐次理想$I\subseteq S=R[T_0,\cdots,T_n]$使得$X\cong\mathrm{Proj}S/I$.但是这里$I$可以选取落在$S_+$中(取$I'=\oplus_{d\ge1}I_d$),此时就有$(S/I)_0=R$.充分性是因为$A\cong R[T_0,\cdots,T_n]/I$,其中$I$是齐次理想,所以$\mathrm{Proj}A$是$\mathbb{P}_R^n$的闭子概型.
	\end{proof}
	\item 这里我们统一下定义.概形$Y$上的射影概形是指它是$Y$上某个射影空间$\mathbb{P}_Y^r$的闭子概型.等价的讲结构态射$X\to Y$是H射影态射.在$Y=\mathrm{Spec}R$是仿射的情况下这等价于讲它是$R$上某个有限生成分次代数的$\mathrm{Proj}$.
\end{enumerate}
\subsection{Serge嵌入}

射影概型的纤维积,Segre嵌入.
\begin{enumerate}
	\item 我们期望证明射影$A$概型的纤维积仍然是一个射影$A$概型.为此只需证明两个$A$射影空间的纤维积是射影$A$概型即可.这是因为射影$A$概型恰好是$A$射影空间的闭子概型,结合闭嵌入的纤维积仍然是闭嵌入,就说明如果$X\to\mathbb{P}_A^n$和$Y\to\mathbb{P}_A^m$是闭嵌入,那么$X\times_AY\to\mathbb{P}_A^n\times_A\mathbb{P}_A^m$是闭嵌入,于是问题归结为射影空间的纤维积是射影概型.
	\item Segre嵌入.我们来构造闭嵌入$\mathbb{P}_A^n\times_A\mathbb{P}_A^m\to\mathbb{P}_A^{(n+1)(m+1)-1}$.记$\mathbb{P}_A^n=\mathrm{Proj}A[x_0,x_1,\cdots,x_n]$和$\mathbb{P}_A^m=\mathrm{Proj}A[y_0,y_1,\cdots,y_m]$,再记$U_i=D(x_i)$和$V_i=D(y_i)$.于是纤维积可被仿射开子集$U_i\times_AV_j$覆盖.记$\mathbb{P}_A^{(n+1)(m+1)-1}=\mathrm{Proj}A[z_{00},z_{01},\cdots,z_{nm}]=\mathrm{Proj}A[x_0y_0,x_0y_1,\cdots,x_ny_m]$.记$W_{ij}=D(z_{ij})$.定义态射$U_i\times_AV_j\to W_{ij}$被环同态$z_{ab/ij}\mapsto x_{a/i}y_{b/j}$所诱导.这些局部构造的态射在定义域交集上的限制是相同的,于是所有局部闭嵌入态射粘合为一个整体闭嵌入.此称为Segre嵌入.
	\item Segre嵌入得到的$A$射影概型是如下关系式定义的零点概型(即这个矩阵的所有二阶子式为零).
	$$\mathrm{rank}\left(\begin{array}{ccc}z_{00}&\cdots&z_{0m}\\\vdots&\ddots&\vdots\\z_{n0}&\cdots&z_{nm}\end{array}\right)=1$$
	\item 特别的,取$m=n=1$,上一条说明纤维积$\mathbb{P}_k^1\times_k\mathbb{P}_k^1$恰好是$\mathrm{Proj}k[w,x,y,z]/(wz-xy)$.如果$k$是特征非2的代数闭域,那么$wz-xy$可以对角化,得到它同构于$\mathrm{Proj}k[w,x,y,z]/(x^2+y^2+z^2+w^2)$.这里代数闭域是必要的,例如$k=\mathbb{R}$就不会成立,因为后者没有$\mathbb{R}$值点.
	\item 不依赖坐标的射影概型纤维积描述.设$S$和$T$是两个有限生成$A$分次环,那么$\mathrm{Proj}S\times_A\mathrm{Proj}T\cong\mathrm{Proj}\oplus_{n\ge0}(S_n\otimes_AT_n)$.
\end{enumerate}
\subsection{射影概形上的伴随模层}

伴随模层.设$A=\oplus_{n\ge0}A_n$是分次环,设$M=\oplus_{n\in\mathbb{Z}}M_n$是分次$A$模,定义$M$在$\mathrm{Proj}A$上的伴随模层,记作$\widetilde{M}$,在每个开集$U\subseteq\mathrm{Proj}A$上的截面定义为$U\to\coprod_{\mathfrak{p}\in U}M_{(\mathfrak{p})}$的函数$s$(这里$M_{(\mathfrak{p})}$表示分式模$M_{\mathfrak{p}}$作为$\mathbb{Z}$分次模的零次子模),满足如下两个条件.限制映射直接取为函数在更小开集上的限制.
\begin{itemize}
	\item 对每个$\mathfrak{p}\in U$有$s(\mathfrak{p})\in M_{(\mathfrak{p})}$.
	\item 对每个$\mathfrak{p}\in U$存在开邻域$W_{\mathfrak{p}}\subseteq U$,存在$m\in M$和$f\in A$,使得对每个$\mathfrak{q}\in W_{\mathfrak{p}}$都有$f\not\in\mathfrak{q}$和$s(\mathfrak{q})=m/f$.
\end{itemize}
\begin{enumerate}
	\item 茎.$\mathfrak{p}\in\mathrm{Proj}A$有$(\widetilde{M})_{\mathfrak{p}}=M_{(\mathfrak{p})}$.
	\item 主开集.任取正次齐次元$f\in A_+$,有$\widetilde{M}\mid_{D_+(f)}=\widetilde{M_{(f)}}$.其中开子概型$D_+(f)$自然视为$\mathrm{Spec}A_{(f)}$.于是特别的$\widetilde{M}$是$\mathrm{Proj}A$上的拟凝聚层.另外如果$S$是诺特的,$M$是有限生成分次模,那么$\widetilde{M}$是凝聚层.
	\item 定义伴随模层也可以从主开集出发,对正次齐次元$f\in A_+$定义$D_+(f)$上的模层是$\widetilde{M_{(f)}}$,然后把它们粘合即可.
	\item 设$f\in A_+$是正次齐次元,那么函子$M\mapsto\widetilde{M_{(f)}}$是从分次$A$模范畴到拟凝聚$\mathscr{O}_{D_+(f)}$模层范畴的函子,与直和可交换,与滤过正向极限可交换.进而有$M\mapsto\widetilde{M}$是从分次$A$模范畴到拟凝聚$\mathscr{O}_{\mathrm{Proj}A}$模层的函子,它也满足与直和可交换,与滤过正像极限可交换.但是相比仿射情况,这个函子有很多缺陷,例如它不是忠实的:设$A$是分次环,设$M=\oplus_nM_n$是分次模,取整数$n_0$,记$N=\oplus_{n\ge n_0}M_n$,那么对每个齐次元$f\in B$,都有$M_{(f)}=N_{(f)}$,这说明$\widetilde{M}=\widetilde{N}$.换句话讲分次模的前有限个分量不影响它诱导的伴随模层.于是如果取$M$上的恒等同态,和取$n\le n_0$时$M_n\to M_n$是零同态,其余分量是恒等同态,那么它们诱导了相同的概形态射.
	\item 射影概形上$M\mapsto\widetilde{M}$和张量积可交换.即如果$M,N$都是分次环$S$的分次模,那么有:
	$$\widetilde{M}\otimes_{\mathrm{Proj}S}\widetilde{N}\cong\widetilde{M\otimes_SN}$$
	\item 我们会证明如果$A$是被一次齐次元生成的有限生成分次环,那么$\mathrm{Proj}A$上的拟凝聚层一定是分次模的伴随模层.
\end{enumerate}
\subsection{射影概形上的扭曲层}

设$M$是分次环$A$上的分次模.对每个整数$n$,记$M(n)=\oplus_{d\in\mathbb{Z}}M_{n+d}$,那么有$M(n)=M\otimes_AA(n)$.记$X=\mathrm{Proj}A$是一个射影概形.对$\mathscr{O}_X$模层$\mathscr{F}$,定义它的$n$次扭曲层为$\mathscr(n)=\mathscr{F}\otimes_{\mathscr{O}_X}\mathscr{O}_X(n)$,其中$\mathscr{O}_X(n)=\widetilde{A(n)}$.于是特别的$\mathscr{O}_X$本身的$n$次扭曲层就是$\mathscr{O}_X(n)$.
\begin{enumerate}
	\item 设$A$是被一次元生成的分次环,设$X=\mathrm{Proj}A$,关于扭曲层有如下结论:
	\begin{enumerate}
		\item $\mathscr{O}_X(n)$总是$X$上的可逆层.
		\begin{proof}
			
			按照$A$被一次齐次元生成,有当$f$跑遍$A$的一次齐次元时$\{D_+(f)\}$覆盖整个$A$.于是只需验证$A(n)_{(f)}$是秩1自由$A_{(f)}$模,这只需直接构造$A_{(f)}$模同构$S_{(f)}\to S(n)_{(f)}$为$s\mapsto f^ns$(这里用到了$f$必须是一次元).
		\end{proof}
		\item 我们知道线丛的同构类构成Picard群,并且对概形$X$(其实甚至是环空间),有同构$\mathrm{Pic}(X)\cong\mathrm{H}^1(X,\mathscr{O}_X^*)$.在这个对应下,$\mathscr{O}_X(n)$作为线丛对应于\v{C}ech上同调的如下1-余圈:开覆盖取为$\{D_+(f)\mid f\in A_1-\{0\}\}$,对$f,g\in A_1$取$f^{-n}g^n\in\Gamma(D_+(f)\cap D_+(g),\mathscr{O}_X)^*=A_{(fg)}^*$.那么$\{f^{-n}g^n\}$构成1-余圈.
		\item 当$n$取负整数时我们把扭曲层定义为线丛的逆$\mathscr{O}_X(-1)=\mathscr{O}_X(1)^{\vee}=\mathrm{HOM}_{\mathscr{O}_X}(\mathscr{O}_X(1),\mathscr{O}_X)$.
		\item $\widetilde{M}(n)=\widetilde{M}\otimes_{\mathscr{O}_X}\widetilde{S(n)}=\widetilde{M\otimes_SS(n)}=\widetilde{M(n)}$.
		\item $\mathscr{O}_X(n)\otimes_{\mathscr{O}_X}\mathscr{O}_X(m)=\widetilde{S(n)}\otimes_{\mathscr{O}_X}\widetilde{S(m)}=\widetilde{S(n)\otimes_SS(m)}=\mathscr{O}_X(m+n)$.进而有$\mathscr{O}_X(n)=\mathscr{O}_X(1)^{\otimes n}$.这件事也可以从线丛的\v{C}ech上同调描述得到(因为1-余圈$\{f^{-n}g^n\}$和$\{f^{-m}g^m\}$的和就是$\{f^{-(m+n)}g^{m+n}\}$).
		\item 设$B$是另一个分次环,也被一次齐次元生成.设$\varphi:A\to B$是保次数的分次环同态,记$U=\mathrm{Proj}B-V_+(\varphi(A_+))$,我们解释过$\varphi$诱导了概形态射$f:U\to\mathrm{Proj}S$.对任意$A$分次模$M$,都有$f^*\widetilde{M}=\widetilde{B\otimes_AM}\mid_U$;对任意$B$分次模$N$,都有$f_*(\widetilde{N}\mid_U)=\widetilde{_AN}$.
	\end{enumerate}
	\item 射影概形上模层伴随的分次模.设$A$是分次环,设$X=\mathrm{Proj}A$,设$\mathscr{F}$是$\mathscr{O}_X$模层,定义和$\mathscr{F}$伴随的分次$A$模$\Gamma_*(\mathscr{F})=\oplus_{n\in\mathbb{Z}}\Gamma(X,\mathscr{F}(n))$.它的分次$A$模结构是由$\mathscr{O}_X(n)\otimes_{\mathscr{O}_X}\mathscr{F}(m)\cong\mathscr{F}(m+n)$在整体截面上的同态$A_n\otimes_A\Gamma(X,\mathscr{F}(m))\to\Gamma(X,\mathscr{F}(n+m))$.
	\item 自然变换$\alpha$.对$A$分次模$M$我们构造自然的$\alpha_M:M\to\Gamma_*(\widetilde{M})$.
	\begin{enumerate}
		\item 设$A$是被一次齐次元生成的分次环,设$X=\mathrm{Proj}A$,我们构造一个保次数的典范分次同态$\alpha:A\to\Gamma_*(\mathscr{O}_X)$.构造$A_0$代数同态$\alpha_n:A_n\to\Gamma(X,\mathscr{O}_X(n))$如下:任取$a\in A_n$,任取$f\in A_1$,取$a/f^n\in\Gamma(D_+(f),\mathscr{O}_X)$,那么这些$a/f^n$在$D_+(f)\cap D_+(g)$上相差一个$f^{-n}g^n$,满足余圈条件,于是它们可以粘合为一个整体截面,这个整体截面定义为$\alpha_n(a)$.
		\item 设$A$是环,设$S=A[X_0,\cdots,X_r],r\ge1$是多项式环,那么有$\alpha$是同构(后面会把$\alpha_M$是同构的分次$A$模$M$称为饱和的,这里就是说如果$A$是某个环上的多项式环,那么$A$作为自身分次模是饱和的).但是如果$S$不是某个环上的多项式环,仅是分次环,这个结论未必成立.
		\begin{proof}
			
			主开集$\{D_+(x_i)\}$覆盖了整个$X$.选取一个$t\in\Gamma(X,\mathscr{O}_X(n))$,等价于选取一族$t_i\in\mathscr{O}_X(n)(D_+(x_i))=(S(n))_{(x_i)}$,使得$t_i,t_j$在$D_+(x_ix_j)$上的限制相同.换句话讲,等价于选取一族$t_i$为$S_{x_i}$的$n$次元,使得$t_i,t_j$在$S_{x_ix_j}$中是相同的.于是任取$t\in\Gamma_*(\mathscr{O}_X)$就等价于选取一族$(t_0,t_1,\cdots,t_r)$,其中$t_i\in S_{x_i}$,满足$t_i,t_j$在$S_{x_ix_j}$中相同.
			
			\qquad
			
			现在典范映射$S\to S_{x_i}\to S_{x_ix_j}\to S'=S_{x_0x_1\cdots x_r}$全是单射,所以我们可以把$S,S_{x_i},S_{x_ix_j}$全部视为$S'$的子环.在这个意义下就有$\Gamma_*(\mathscr{O}_X)=\cap S_{x_i}$.并且一方面$S\subseteq\cap S_{x_i}$,另一方面我们来证明$S_{x_i}\cap S_{x_j}=S$,于是有$\cap S_{x_i}\subseteq S$:$S'$中的元可以唯一的表示为$x_0^{i_0}\cdots x_r^{i_r}f(x_0,\cdots,x_r)$,其中$i_j\in\mathbb{Z}$,而$f$是不被任一$x_i$整除的多项式.那么这个元在$S_{x_i}$中等价于讲当$j\not=i$时有$i_j\ge0$,这就得到$S_{x_i}\cap S_{x_j}=S$.
		\end{proof}
		\item 更一般的,设$A$是被一次元生成的分次环,设$X=\mathrm{Proj}A$,设$M$是分次$A$模.我们来构造自然同态$\alpha=\alpha_M:M\to\Gamma_*(\widetilde{M})$:任取$x\in M_n$,任取$f\in A_d$,定义$\alpha_n^f(x)=x/1\in(M_{(f)})_n=M(n)_{(f)}=\Gamma(D_+(f),\widetilde{M}(n))$.这些$\alpha_n^f$满足余圈条件,于是可以粘合为$\Gamma(X,\widetilde{M}(n))$中的元,把它记作$\alpha_n(x)$.
	\end{enumerate}
	\item 设$R$是整环,设$A=R[T_0,\cdots,T_d]$是多项式环,设$f\in A_n$是正次齐次元,设$V_+(f)$是被$f$割掉的$X=\mathrm{Proj}A$的闭子概型,那么乘以$f$诱导了一个分次$A$模同态$A(-n)\to A$,并且它的余核是$A/(f)$,于是我们得到$A$模的短正合列:
	$$\xymatrix{0\ar[r]&A(-n)\ar[r]&A\ar[r]&A/(f)\ar[r]&0}$$
	
	它诱导的伴随模层的短正合列就是:
	$$\xymatrix{0\ar[r]&\mathscr{O}_X(-n)\ar[r]&\mathscr{O}_X\ar[r]&\mathscr{O}_{V_+(f)}\ar[r]&0}$$
	\item 自然变换$\beta$.对$X=\mathrm{Proj}A$上的模层$\mathscr{F}$我们构造典范的态射$\beta_{\mathscr{F}}:\widetilde{\Gamma_*(\mathscr{F})}\to\mathscr{F}$.
	\begin{enumerate}
		\item 设$\mathscr{F}$是$\mathscr{O}_X$模层,设$M=\Gamma_*(\mathscr{F})$,设$f\in A_d$是正次齐次元,取$A_{(f)}$模同态$\beta^f:\Gamma(D_+(f),\widetilde{M})=M_{(f)}\to\Gamma(D_+(f),\mathscr{F})$为对$x\in M_{nd}$,把$x/f^n\in M_{(f)}$映为$(x\mid_{D_+(f)})(\alpha_{dn}(f^n)\mid_{D_+(f)})^{-1}$.这些局部的态射粘合得到的整体态射记作$\beta_{\mathscr{F}}$.
		\item 设$A$被有限个$A_1$中的元生成,设$\mathscr{F}$是拟凝聚层,那么我们断言$\beta$是同构.
		\begin{proof}
			
			此时$\beta$有如下描述:任取$f\in A_1$,按照伴随性,在$D_+(f)$上定义$\widetilde{\Gamma_*(\mathscr{F})}\to\mathscr{F}$等价于构造$A_{(f)}$模同态$\Gamma_*(\mathscr{F})_{(f)}\to\Gamma(X,\mathscr{F})$.任取$m/f^d$,其中$m\in\Gamma(X,\mathscr{F}(d)),d\ge0$.按照同构$\mathscr{O}_X(d)\otimes_{\mathscr{O}_X}\mathscr{F}(-d)\cong F$,定义$m/f^n$在$\mathscr{F}(X)$中的像为$m\otimes f^{-d}$在这个同构下的像.
			
			\qquad
			
			这里$f$可视为可逆层$L=\mathscr{O}_X(1)$上的整体截面.设$A$被$f_0,f_1,\cdots,f_r\in A_1$在$A_0$上生成.这里$D_+(f_i)\cap D_+(f_j)=D_+(f_if_j)$拟紧,$L\mid D_+(f_i)$自由,所以拟凝聚层上截面延拓定理的条件全部满足,所以有$\Gamma(X,\mathscr{F})_f=\Gamma(X_f,\mathscr{F})$.于是得到如下同构,于是$\beta$是同构.
			$$\Gamma_*(\mathscr{F})_{(f)}=\left(\oplus_n\Gamma(X,\mathscr{F}(n))\right)_{(f)}=\Gamma(X,\mathscr{F})_f=\Gamma(X_f,\mathscr{F})$$
		\end{proof}
	\end{enumerate}
	\item 如果记函子$F:M\mapsto\widetilde{M}$和函子$G:\mathscr{F}\mapsto\Gamma_*(\mathscr{F})$.那么$\alpha$就是$1\to GF$的自然变换,$\beta$就是$FG\to1$的自然变换.倘若$\alpha$和$\beta$都是同构,则有$F,G$定义了范畴等价.设$A$是分次环,一个分次$A$模$M$称为饱和的(saturated),如果$\alpha_M:M\to\Gamma_*(\widetilde{M})$是同构.我们用$\textbf{GrMod}^{\textbf{sat}}_{\text{A}}$表示饱和分次$A$模构成的完全子范畴,用$\textbf{QCoh}_{\textbf{X}}$表示$X$上的拟凝聚层范畴,那么如果$A$是被有限个一次元生成的分次环,就有如下范畴等价:
	$$\xymatrix{\left(\textbf{GrMod}^{\textbf{sat}}_{\text{A}}\right)\ar@<0.5ex>[rrr]^{M\mapsto\widetilde{M}}&&&\textbf{QCoh}_{\textbf{X}}\ar@<0.5ex>[lll]^{\mathscr{F}\mapsto\Gamma_*(\mathscr{F})}}$$
	\begin{proof}
		
		从$\alpha$和$\beta$是自然同构推出$F,G$定义了范畴等价是范畴等价的标准等价描述.唯一还需要说明的是如果$\mathscr{F}$是$X$上的拟凝聚层,那么$M=\Gamma_*(\mathscr{F})$的确是一个饱和分次$A$模.而这是因为从$\beta_M$是同构得到$\Gamma_*(\beta_M):\Gamma_*(\widetilde{M})\to\Gamma_*(\mathscr{F})=M$也是同构,但是如下复合是$M$上的恒等,于是$\alpha$是同构.
		$$\xymatrix{M\ar[rr]^{\alpha}&&\Gamma_*(\widetilde{M})\ar[rr]^{\Gamma_*(\beta)}&&M}$$
	\end{proof}
	\item 推论.设$A$是被有限个一次元生成的分次环,那么$X=\mathrm{Proj}A$上的拟凝聚层一定是分次$A$模的伴随模层.
	\item
	【】
	
	
	设$Y$是概形,我们定义过其上的一般射影空间为$\mathbb{P}_Y^r=\mathbb{P}_{\mathbb{Z}}^r\times_{\mathbb{Z}}Y$.定义$\mathbb{P}_Y^r$上的扭曲层$\mathscr{O}(1)$为$g^*(\mathscr{O}(1))$,其中$g:\mathbb{P}^r_Y\to\mathbb{P}_{\mathbb{Z}}^r$的典范投影态射.那么如果取$Y$是仿射概形,这个定义吻合于我们之前在$\mathbb{P}_A^r$上定义的扭曲层.
\end{enumerate}
\newpage
\section{概型}
\subsection{仿射开子集}

环空间$(X,\mathscr{O}_X)$的开子集$U$称为仿射开子集,如果存在从$(U,\mathscr{O}_X\mid_U)$到某个仿射概型的同构(此时同构自然也是局部环空间之间的同构).局部环空间称为概型,如果它存在仿射开集构成的开覆盖.仿射开子集构成了概型的一个拓扑基.

概型上两个仿射开子集的交非仿射的例子.给定两个仿射平面$X=\mathrm{Spec}k[x,y]$和$Y=\mathrm{Spec}k[s,t]$.取主开集$D(x,y)=\mathrm{Spec}k[x,y,1/x,1/y]$和$D(s,t)=\mathrm{Spec}k[s,t,1/s,1/t]$.按照$\varphi:x\mapsto s,y\mapsto t$的方式粘合,得到了带两个原点的仿射平面.这里$X$和$Y$的像集是它的两个仿射开子集,它们的交是仿射平面扣去原点,我们已经解释了这不是仿射的.

\subsection{开子概型}

我们接下来要证明概型的开子集总是自然的继承一个概型结构,它称为开子概型.为此先需要验证主开集上成立.
\begin{enumerate}
	\item 对每个$f\in A$,有局部环空间之间的同构$(D(f),\mathscr{O}_{\mathrm{Spec}(A)}\mid_{D(f)})\cong(\mathrm{Spec}(A_f),\mathscr{O}_{\mathrm{Spec}(A_f)})$.
	\begin{proof}
		
		记典范同态$A\to A_f$诱导的态射$(\varphi,\varphi^{\#}):(\mathrm{Spec}A_f,\mathscr{O}_{\mathrm{Spec}A_f})\to(\mathrm{Spec}A,\mathscr{O}_{\mathrm{Spec}A})$,我们之前证明过分式化的典范映射诱导的连续映射总是到像集的同胚,于是这里$\varphi$是同胚.
		
		接下来要证明层的态射$\varphi^{\#}$是同构.为此我们仅需证明诱导的每个茎之间的同态都是同构.我们解释过尽管$f^{\#}_p$一般来讲不是$f^{\#}$诱导的Stalk之间的同态,但是倘若$f$已经是同胚,此时$f^{\#}_p$恰好就是.于是为了验证这里$f^{\#}$是层同构,仅需验证$f^{\#}_p$对每个$p\in D(f)$都是同构.但是我们之前给出过此时$f^{\#}_p$恰好就是典范的同态$A_p\to (A_f)_{\hat{p}}$,这里$\hat{p}$是$p$在$A_f$中对应的素理想,可以之间验证这个同态就是同构.
	\end{proof}
    \item 设$(X,\mathscr{O}_X)$是概型,取开子集$U$,那么$(U,\mathscr{O}_X\mid_U)$是概型,它称为$(X,\mathscr{O}_X)$的开子概型.
    \begin{proof}
    	
    	设$X=\cup_iU_i$,其中每个$U_i$同构于仿射概型$\mathrm{Spec}(A_i)$.于是开集$U=\cup_i(U\cap U_i)$.我们只需验证每个$U\cap U_i$可以表示为若干仿射概型的并.但是$U\cap U_i$是仿射概型$U_i$的开子集.按照主开集是仿射概型的拓扑基,说明$U\cap U_i$可以表示为若干主开集的并,上一条验证了仿射概型的主开子集继承局部环空间结构都是仿射概型.这就说明了$X$可以表示为若干仿射概型的并.
    \end{proof}
\end{enumerate}

\subsection{开集$X_f$}

设$X$是概型,取整体截面$f\in\mathscr{O}_X(X)$,记$X_f$是全体$X$中这样的点$p$,满足$f$在$\mathscr{O}_{X,p}$中是单位.换句话讲,如果把$f$在$p$的取值取在剩余类域$\kappa(p)=\mathscr{O}_{X,p}/m_p$中,那么$X_f$是$f$的不取零的点构成的集合.
\begin{enumerate}
	\item 设$U$是仿射开子集,记$f'=f\mid_U$,那么$X_f\cap U=D(f')$.因为$U=\mathrm{Spec}(A)$中点$p$使得$f$在$\mathscr{O}_{X,p}$中是单位,等价于$f'$在$A_p$中是单位,即$f'\not\in p$.
	\item $X_f$是开集.取$X$的仿射开覆盖$X=\cup_iU_i$,按照上一条,每个$U_i\cap X_f=D(f_i)$是开集,于是$X_f$是开集.
	\item 设$f$是$X$上在剩余类域中处处不取零的函数,那么$f$是可逆的函数.这只要在局部上取逆映射,再粘合即可.
	\item $X_f=\emptyset$当且仅当对任意$X$的仿射开覆盖$\{U_i=\mathrm{Spec}(A_i)\}$,使得每个$f\mid_{U_i}$是$A_i$中的幂零元.
	\item 若$f,g\in\mathscr{O}_X(X)$,那么$X_f\cap X_g=X_{fg}$.事实上$p\in X_{fg}$等价于$fg$在$\mathscr{O}_{X,p}$是单位,等价于$f$和$g$都在$\mathscr{O}_{X,p}$中是单位,等价于$p\in X_f\cap X_g$.
	\item 设$(\varphi,\varphi^{\#}):X\to Y$是概型的态射,这里$\varphi^{\#}$诱导了整体截面之间的环同态$\mathscr{O}_Y(Y)\to\mathscr{O}_X(X)$.那么有$\varphi^{-1}(Y_f)=X_{\varphi^{\#}(f)}$.
	\begin{proof}
		
		考虑如下交换图,这里下一行的映射是局部映射,于是它只把单位映射为单位.那么$p\in X_{\varphi^{\#}(p)}$等价于$\mathscr{O}_{Y}(Y)\to\mathscr{O}_X(X)\to\mathscr{O}_{X,p}$把$f$映射为单位.于是$\mathscr{O}_Y(Y)\to\mathscr{O}_{Y,\varphi(p)}\to\mathscr{O}_{X,p}$把$f$映为单位.按照局部映射只把单位映射为单位,于是$\mathscr{O}_Y(Y)\to\mathscr{O}_{Y,\varphi(p)}$把$f$映为单位.等价于$\varphi(p)\in Y_f$,等价于$p\in\varphi^{-1}(Y_f)$.
		
		$$\xymatrix{\mathscr{O}_Y(Y)\ar[rr]^{\varphi^{\#}}\ar[d]&&\mathscr{O}_X(X)\ar[d]\\\mathscr{O}_{Y,\varphi(p)}\ar[rr]&&\mathscr{O}_{X,p}}$$
	\end{proof}
\end{enumerate}
\subsection{概型是$T_0$空间}

任取两个不同点$x,y$,倘若不存在仿射开子集同时覆盖这两个点,那么取$x$的仿射开邻域就不包含点$y$;倘若存在同时覆盖这两个点的仿射开子集,按照仿射概型是$T_0$空间,得到存在$X$的开集只覆盖这两个点中的一个.
\subsection{特殊化和一般化}

拓扑空间上两个点$x,y$,称$y$是$x$的特殊化或者$x$是$y$的一般化,如果满足$y\in\overline{\{x\}}$.换句话讲闭包更大的点是更一般的.如果一个点的一般化只有它自身,就称它为一般点.
\begin{enumerate}
	\item 这不是一个偏序关系,它只满足自反性和传递性.但是$T_0$条件等价于讲它满足反对称性,于是概形上特殊化总是一个偏序关系.
	\item 如果空间是$T_1$的,特殊化和一般化的概念就没有任何用处,这时候一个点的特殊化和一般化只有它自己.概形通常都不是$T_1$的.
	\item 闭集包含它每个点的特殊化,开集包含它每个点的一般化.
	\item 如果$f:X\to Y$是连续映射,如果在$X$中有点$x$是$y$的一般化,那么在$Y$中就有$f(x)$是$f(y)$的一般化.即连续映射是保特殊化和一般化的.
	\item 仿射概形的情况.设$X=\mathrm{Spec}A$,素理想$\mathfrak{p}$是$\mathfrak{q}$的一般化当且仅当$\mathfrak{p}\subseteq\mathfrak{q}$,素理想是一般点当且仅当它是极小素理想.
\end{enumerate}
\subsection{概形的不可约性和不可约分支}
\begin{enumerate}
	\item 设$X$为概型,对任意非空的不可约闭子集$Y$,存在唯一的点$y\in X$满足$\overline{\{y\}}=Y$,它称为不可约闭子集$Y$的一般点(generic point).于是概型上的点和它的非空不可约闭子集之间存在一一对应$y\mapsto\overline{\{y\}}$.
	\begin{proof}
		
		取仿射开子集$U=\mathrm{Spec}(A)\subseteq X$使得$U\cap Y$非空,那么$U\cap Y$是不可约的(不可约集的开子集不可约),它是仿射概型$U=\mathrm{Spec}(A)$的不可约闭子集,于是按照仿射情况的结论,存在极小素理想$p\in\mathrm{Spec}(A)$使得$V(p)=U\cap Y$.按照不可约空间的非空开子集总是稠密集,得到$\overline{U\cap Y}=Y$,即$\overline{V(p)}=Y$.而$V(p)=\overline{\{p\}}\cap U$,于是$\overline{\{p\}}=\overline{V(p)}=Y$.
		
		验证唯一性,如果$p,q\in X$满足$\overline{\{p\}}=\overline{\{q\}}$,倘若$p\not=q$,按照概型总是$T_0$空间,存在开集只包含$p,q$中的一个点,不妨设开集$U$包含$p$不包含$q$,这导致$q\not\in\overline{\{p\}}$矛盾.
	\end{proof}
	\item 于是如果$A$是整环或者$S$是分次整环,那么自然有$\mathrm{Spec}(A)$和$\mathrm{Proj}(S)$都是不可约空间.特别的$\mathbb{A}^n_k$和$\mathbb{P}_k^n$都是不可约空间.
	\item 不可约的拓扑性质.
	\begin{itemize}
		\item 设$X$是拓扑空间,设$U\subseteq X$是开子集,那么$U$的全部不可约分支恰好是$\{U\cap X_i\}$,其中$X_i$取遍和$U$有交的$X$的不可约分支.
		\begin{proof}
			
			一方面取$X$的不可约分支$X_i$,使得$X_i\cap U$非空,如果$X_i\cap U$包含于$U$的某个不可约分支$Z$中,那么有$X_i=\overline{X_i\cap U}\subseteq\overline{Z}$,于是按照$X_i$的极大性得到$X_i=\overline{Z}$(因为不可约子集的闭包还是不可约的).这说明$Z=\overline{Z}\cap U=X_i\cap U$,于是$Z=X_i\cap U$.
			
			另一方面任取$U$的不可约分支$Z$,那么有$X$的不可约分支$X_i$包含了整个$Z$,于是有$\overline{Z}\subseteq X_i$,于是$Z=\overline{Z}\cap U\subseteq X_i\cap U$.但是后者是$U$中的不可约闭子集,于是$Z=X_i\cap U$.
		\end{proof}
		\item 如果$X$是有限个不可约闭子集$Z_j$的并,那么$X$的每个不可约分支等于某个$Z_j$.另外倘若这些$Z_j$之间没有包含关系,那么$\{Z_j\}$恰好是$X$的全部不可约分支.
		\begin{proof}
			
			我们只要说明第一个结论.任取$X$的不可约分支$Z$,那么有$Z=\cup_j(Z_j\cap Z)$.按照$Z$的不可约性,说明$Z$包含于某个$Z_j$中,按照$Z$是极大的不可约子集,得到$Z=Z_j$.
		\end{proof}
	\end{itemize}
	\item 设$X$是概型,任取点$x\in X$,在后文局部概形那里我们会解释$\mathrm{Spec}\mathscr{O}_{X,x}$典范的一一对应于经$x$的不可约闭子集,也即$x$的所有一般化,特别的,这个素谱的不可约分支(也即这个局部环的极小素理想)恰好一一对应于$X$的经过点$x$的不可约分支.
	\item 设$X$是概型,那么下面条件满足(1)推出(2),(2)和(3)等价,如果$X$的不可约分支是局部有限集合,此即对每个点$x\in X$,存在开邻域只和有限个不可约分支有交(特别的,局部诺特概型满足这个条件),那么这三个条件互相等价.【特别的,这件事可以说明局部诺特正规概形如果连通,则它一定是整概形】
	\begin{enumerate}[(1)]
		\item $X$的每个连通分支都是不可约分支.
		\item $X$的不可约分支两两不交.
		\item 每个点的局部环都有唯一的极小素理想(等价于讲幂零根本身是一个素理想).
	\end{enumerate}
	\begin{proof}
		
		(1)推(2)是因为连通分支总是两两不交的.(2)和(3)等价是因为每个点$x\in X$的局部环的极小素理想一一对应于经点$x$的不可约分支.最后证明如果不可约分支是局部有限集合,那么(2)推(1)成立:首先我们解释条件下概型的每个点存在一个不可约的开邻域.任取$x\in X$,可取开邻域$U_x$使得它只和有限个不可约分支有交,记作$\{X_1,X_2,\cdots,X_r\}$,于是每个$X_i\cap U_x$是$U_x$中的不可约分支,于是是闭集.按照$X_i$两两不交,得到每个$X_i\cap U_x$是开集,设$x\in X_t$,那么$X_t\cap U_x$是$x$的不可约开邻域.
		
		现在任取概型$X$的连通分支$Y$,按照不可约空间总是连通的,说明连通分支总是若干不可约分支的并.任取$Y$的不可约分支$Z$,这是$Y$的闭子集.但是任取$x\in Z$,有$x$的不可约开邻域,这个开邻域包含于$Z$,这说明$Z$是$Y$中开集.按照$Y$是连通的,只能有$Y=Z$.
	\end{proof}
	\item 设概形之间的态射$f:X\to Y$在拓扑层面是开映射,设$Y$是以$\eta$为一般点的不可约概形,那么$X$是不可约的当且仅当$f^{-1}(\eta)$是不可约集合.
	\begin{proof}
		
		我们先断言有$\overline{f^{-1}(\eta)}=f^{-1}(\overline{\{eta\}})$:按照$f$连续,闭集的原像就是闭集,于是从$f^{-1}(\eta)\subseteq f^{-1}(\overline{\{\eta\}})$得到$\overline{f^{-1}(\eta)}\subseteq f^{-1}(\overline{\{\eta\}})$.另一方面如果$x\in f^{-1}(\overline{\{\eta\}})$,任取$x$的开邻域$U$,按照$f$是开映射得到$f(U)$也是$f(x)$的开邻域,于是从$f(x)\in\overline{\{\eta\}}$,得到$\eta\in f(U)$,于是$f^{-1}(\eta)$和$U$的交非空,于是$x\in\overline{f^{-1}(\eta)}$.
		
		\qquad
		
		于是有$\overline{f^{-1}(\eta)}=f^{-1}(\overline{\{\eta\}})=f^{-1}(Y)=X$.而我们知道空间的一个子集是不可约的当且仅当它的闭包是不可约的,这说明$X$是不可约的等价于$f^{-1}(\eta)$是不可约的.
	\end{proof}
\end{enumerate}
\subsection{概形的粘合}

\begin{enumerate}
	\item 拓扑的粘合.设$\{U_i,i\in I\}$是一族拓扑空间,对任意$i\not=j$,存在$U_i$的开子集$U_{ij}$.存在同胚$\varphi_{ij}:U_{ij}\to U_{ji}$.满足:
	\begin{enumerate}
		\item $\varphi_{ij}$和$\varphi_{ji}$互为逆映射.
		\item 对两两不同的指标$i,j,k$,条件a就保证了$\varphi_{ij}(U_{ij}\cap U_{ik})=U_{ji}\cap U_{jk}$,于是$\varphi_{jk}\circ\varphi_{ij}$和$\varphi_{ik}$具有相同的定义域,这一条约定这两个映射总是相同的,即$\varphi_{jk}\circ\varphi_{ij}=\varphi_{ik}$.
	\end{enumerate}
	
	那么存在拓扑空间$X$,以及以$I$为指标的拓扑嵌入(这是指映射是连续单射,并且是到像集的同胚)$\{\varphi_i:U_i\to X\}$,满足如下三条,并且这样的$(X,\varphi_i)$在同胚意义下唯一.
	\begin{enumerate}
		\item $\{\varphi_i(U_i)\}$是$X$的一个开覆盖.
		\item $\forall i,j,\varphi_i(U_{ij})=\varphi_j(U_{ji})=\varphi_i(U_i)\cap\varphi_j(U_j)$.
		\item $\forall i,j,\varphi_j\circ\varphi_{ij}=\varphi_i$.
	\end{enumerate}
    \begin{proof}
    	
    	取$X=\coprod_{i\in I}U_i/\sim$,其中等价关系$\sim$定义为,若$x\in\coprod_iU_i$不在任一$U_{ij}$中,则$x$只和自身等价;若存在$i\not=j$使得$x\in U_{ij}$,则$x$还与$\varphi_{ij}(x)\in U_{ji}$等价(条件a是对称性,条件b是传递性).赋予$\coprod_iU_i$无尽并拓扑,再赋予$X$商拓扑.(于是$X$的子集$S$是开集当且仅当它的全部等价元构成的集合和每个$U_i$的交是$U_i$的开子集).取$\varphi_i$为典范映射的复合$U_i\to\coprod_iU_i\to X$,于是$\varphi_i$都是连续映射.
    	\begin{enumerate}
    		\item $\varphi_i(U_i)$即那些包含了$U_i$中的点的等价类构成的集合,于是它在$\coprod_iU_i\to X$下的原像$E_i$满足$E_i\cap U_j=U_{ji}$,这是$U_j$的开子集,于是$\varphi_i(U_i)$是$X$中的开集.
    		\item $X$中每个点都存在代表元$x'\in$某个$U_i$,于是$x\in\varphi_i(U_i)$,于是$\{\varphi_i(U_i)\}$是$X$是开覆盖.
    		\item $\varphi_i(U_{ij})$中的点即这样的等价类,它同时包含了$U_i$中的点以及$U_j$中的点,于是$\varphi_i(U_{ij})=\varphi_j(U_{ji})=\varphi_i(U_i)\cap\varphi_j(U_j)$.
    		\item 任取$x\in U_i$,那么$\varphi_i$把$x$映射为$x$所在的等价类$[x]$,同样有$\varphi_j\circ\varphi_{ij}(x)=[\varphi_{ij}(x)]$,这两个等价类按照定义相同,于是$\varphi_j\circ\varphi_{ij}=\varphi_i$.
    	\end{enumerate}
    	
    	最后我们来证明$X$在同胚意义下唯一.假设还有$(Y,\psi_i)$满足结论中的三个条件.构造$\theta:\coprod_iU_i\to Y$为在$U_i$上取映射$\psi_i$,这个映射自然是连续的.按照$\psi_j=\varphi_{ij}\circ\psi_i$,说明如果$x,y\in\coprod_iU_i$等价,那么$\theta(x)=\theta(y)$.于是$\theta$诱导了商拓扑出发的连续映射$\theta':X\to Y$.我们需要说明它是同胚.
    	
    	\begin{enumerate}
    		\item 验证满射.这是因为$\{\psi_i(U_i)\}$是$X$的开覆盖.
    		\item 验证单射.如果$\theta(x)=\theta(y)$,记$x\in U_i$,$y\in U_j$,不妨设$i\not=j$,那么$\psi_i(x)=\psi_j(y)=\psi_i\circ\varphi_{ji}(y)$,按照$\psi_i$是单射,得到$x=\varphi_{ji}(y)$,于是$x,y$是等价的.
    		\item 验证开映射.取$X$的开集$V$,记它在$\coprod_iU_i$中的原像是$W$,需要验证$\theta(W)=\theta'(V)$是开集.按照$\{\psi_i(U_i)\}$是开覆盖,以及$\theta(W)\cap\psi_i(U_i)=\psi_i(U_i\cap W)$是开集,证毕.
    	\end{enumerate}
    \end{proof}
    \item 层的粘合.设$X$是拓扑空间,有开覆盖$\{U_i\}$,每个$U_i$上有环层$F_i$,并且对任意指标$i,j$有层同构$\varphi_{ij}:F_i\mid_{U_i\cap U_j}\to F_j\mid_{U_i\cap U_j}$,满足:
    \begin{enumerate}
    	\item $\varphi_{ii}$是$F_i$上的恒等自然同构.
    	\item 对任意指标$i,j,k$,有$\varphi_{ik}=\varphi_{jk}\circ\varphi_{ij}$在$U_i\cap U_j\cap U_k$上成立.
    \end{enumerate}

    那么存在$X$上的层$F$,和层同构$\varphi_i:F\mid_{U_i}\to F_i$,满足$\varphi_j=\varphi_{ij}\circ\varphi_i$.并且这样的层$F$在同构意义下唯一.
    \begin{proof}
    	
    	先构造$F$,对每个开集$V\subseteq X$,定义$F(V)=\{(s_i)_{i\in I}\mid s_i\in F_i(V\cap U_i),\exists i,j\in I,\varphi_{ij}(s_i\mid_{V\cap U_i\cap U_j})=s_j\mid_{V\cap U_i\cap U_j}\}$.如果开集$U\subseteq V$,定义限制映射$F(V)\to F(U)$为$(s_i\in F_i(V\cap U_i))\mapsto(s_i\mid_{U\cap U_i})$.按照每个$F_i$是层,容易验证$F$是层.
    	
    	构造$\varphi_i:F\mid_{U_i}\to F_i$为,对任意开集$V\subseteq U_i$,有$F(V)\to F_i(V)$把$(s_j)$映射为$s_i$.它是层同构因为有逆映射$s\in F_i(V)$映射为$(\varphi_{kj}(s\mid_{V\cap U_j}))_{k\in I}$.容易验证它满足$\varphi_j=\varphi_{ij}\circ\varphi_i$.
    	
    	唯一性.倘若有$(G,\psi_i)$同样满足结论.按照$\psi_i:G\mid_{U_i}\to F_i$是同构,得到$\alpha_i:\psi_i^{-1}\circ\varphi_i:F\mid_{U_i}\to G\mid_{U_i}$的同构.只需验证在$U_i\cap U_j$上恒有$\alpha_i=\alpha_j$,就得到$\alpha_i$可粘合为$F\to G$的层同构.但是这是因为$\alpha_j=\psi_j^{-1}\circ\varphi_j=\psi_i^{-1}\circ\varphi_{ji}\circ\varphi_j=\psi_i^{-1}\circ\varphi_i=\alpha_i$.
    \end{proof}
    \item 概型的粘合.设$\{U_i\}_{i\in I}$为一族概型,对任意$i\not=j$,有$U_i$的开子概型$U_{ij}$,以及概型的同构$\varphi_{ij}:U_{ij}\to U_{ji}$,满足:
    \begin{enumerate}
    	\item $\forall i\not=j$,有$\varphi_{ij}$和$\varphi_{ji}$互为逆映射.
    	\item 对任意两两不同的$i,j,k$,第一条得到$\varphi_{ij}(U_{ij}\cap U_{ik})=U_{ji}\cap U_{jk}$.于是$\varphi_{jk}\circ\varphi_{ij}$和$\varphi_{ik}$具有相同的定义域,这一条要求这两个映射相同.
    \end{enumerate}

    那么存在概型$X$,以及以$I$为指标集的开嵌入$\{\varphi_i:U_i\to X\}$,满足如下三条,并且这样的$(X,\varphi_i)$在同构意义下唯一.
    \begin{enumerate}
    	\item $\{\varphi_i(U_i)\}$是$X$的一个开覆盖.
    	\item $\forall i,j,\varphi_i(U_{ij})=\varphi_j(U_{ji})=\varphi_i(U_i)\cap\varphi_j(U_j)$.
    	\item $\forall i,j,\varphi_j\circ\varphi_{ij}=\varphi_i$.
    \end{enumerate}
    \begin{proof}
    	
    	先构造概型$X$,由第一条构造出拓扑空间$X$,以及连续映射族$\{\varphi_i:U_i\to X\}$,使得$\{\varphi_i(U_i)\}$是$X$的开覆盖,并且$\varphi_i(U_i)\cap\varphi_j(U_j)=\varphi_i(U_{ij})=\varphi_j(U_{ji})$,以及$\varphi_j\circ\varphi_{ij}=\varphi_i$.按照$\varphi_i:U_i\to\varphi_i(U_i)$的同胚,可定义$V_i=\varphi_i(U_i)$上的层结构$F_i=(\varphi_i)_*\mathscr{O}_{U_i}$,即对$V_i$中的开集$W$有$F_i(V)=\mathscr{O}_{U_i}(\varphi_i^{-1}(W))$.下面构造$\widetilde{\varphi_{ij}}:F_i\mid_{V_i\cap V_j}\to F_j\mid_{V_i\cap V_j}$,即$\mathscr{O}_{U_i}(\varphi_i^{-1}(-))\to\mathscr{O}_{U_j}(\varphi_j^{-1}(-))$为,对$V\subseteq V_i\cap V_j$,有$\widetilde{\varphi_{ij}}$为$s\in\mathscr{O}_{U_j}(\varphi_j^{-1}(V))$映射为$\varphi_{ji}(s)$.于是$\widetilde{\varphi_{jk}}\circ\widetilde{\varphi_{ij}}=\widetilde{\varphi_{ik}}$.于是由第二条,层族$F_i$可粘合为$X$上的层$F$,使得$F\mid_{V_i}=F_i$.于是$(X,F)$是概型.
    	
    	取$(\varphi_i,\varphi_i^{\#}):U_i\to X$为,$\varphi_i^{\#}:F\to(\varphi_i)_*\mathscr{O}_{U_i}$,于是$(\varphi_i,\varphi_i^{\#})$是开嵌入.并且满足结论中的三条.
    	
    	最后验证唯一性.如果概型$(Y,G)$和开嵌入族$\{\psi_i:U_i\to Y\}$也满足结论,那么$\psi_i\circ\varphi_i^{-1}$是$V_i\to U_i\to\psi_i(U_i)$的同构.而在$V_i\cap V_j$中有$\psi_i\circ\varphi_i^{-1}=\psi_j\circ\varphi_j^{-1}$,于是这些同构可粘合为$X\to Y$的同构.
    \end{proof}
\end{enumerate}

\subsection{粘合的具体例子}

如果$X,Y$是两个概型,分别有开子空间$U\subseteq X$和$V\subseteq Y$,取同构$\varphi:U\to V$,那么$X$和$Y$就可经$\varphi$粘合.
\begin{enumerate}
	\item 带两个原点的仿射线.取$X=\mathrm{Spec}k[t]$和$Y=\mathrm{Spec}k[s]$.取$U=D(t)=X-\{(t)\}$和$V=D(s)=Y-\{(s)\}$,那么$U$和$V$经$\varphi:s\mapsto t$同构,这个粘合是带两个原点的仿射线.
	\item 射影线的粘合定义.设$k$是域,取两个仿射线$X=\mathrm{Spec}k[t]$和$Y=\mathrm{Spec}k[u]$,取开子集$U=D(t)=\mathrm{Spec}k[t,1/t]$和$V=D(u)=\mathrm{Spec}k[u,1/u]$.我们按照$\varphi:t\mapsto1/u$的方式粘合这两个概型,得到的概型称为$k$上的一维射影空间,或者$k$上的射影线,记作$\mathbb{P}_k^1$.
	\item 粘合结构可以得到整体截面环:我们设$k$总是代数闭域,考虑带两个原点的仿射线,一个整体截面分别限制在粘合的两个部分是两个多项式$f(s)$和$g(t)$,按照粘合条件$s\mapsto t$,就有$f(x)=g(x)$,于是整体截面环就是$k[x]$.考虑射影线,一个整体截面分别限制在粘合的两个部分是两个多项式$f(t)$和$g(u)$,按照粘合条件$t\mapsto 1/u$,得到$f(t)=g(1/t)$,但是这只能有$f=g$是常值多项式,于是整体截面环就是$k$.
	\item 求出整体截面环后可以说明它们都不是仿射的:取$X$在带两个原点的仿射线$Z$中的像是$X'$,那么$X'$和$Z$的整体截面环相同,包含映射$X'\to Z$会诱导整体截面环之间的同构,于是倘若$Z$是仿射的按照仿射概型范畴和环范畴逆变范畴等价,得到包含映射$X'\to Z$理应是同构,但是它都不是满射.
	\item 还有一个粘合例子是$X=\mathrm{Spec}k[x,y]$扣去原点$(x,y)$,这是$X$的开子概型,记作$Y$.它可以表示为$D(x),D(y)\subseteq X$的粘合(严格说是$D(x)$的开子集$D(xy)$与$D(y)$的开子集$D(xy)$按恒等映射粘合).按照层公理,如下交换图表是一个纤维积:
	$$\xymatrix{Y&&D(x)\ar[ll]\\D(y)\ar[u]\ar[rr]&&D(xy)\ar[u]\ar[ll]}$$
	
	取截面环,这里$D(x)$的截面环是$k[x,y,1/x]$,$D(y)$的截面环是$k[x,y,1y]$,$D(xy)$的截面环是$k[x,y,1/xy]=k[x,y,1/x,1/y]$.于是得到纤维积:
	$$\xymatrix{\mathscr{O}_X(Y)\ar[rr]\ar[d]&&k[x,y,1/x]\ar[d]\\k[x,y,1/y]\ar[rr]&&k[x,y,1/x,1/y]}$$
	
	这样$\mathscr{O}_X(Y)$可以表示为两个元$f(x,y)\in k[x,y,1/x]$和$g(x,y)\in k[x,y,1/y]$,使得它们视为$k[x,y,1/x,1/y]$中的元是相同的,这迫使$f=g$并且$x,y$没有负次数项,于是$\mathscr{O}_X(Y)=k[x,y]$.接下来一样的套路可以说明它不是仿射的:考虑$Y\to X$的包含映射,它诱导了整体截面环上的同构,如果$Y$是仿射的,那么这个包含映射理应是同构,但是它甚至不是满射.
\end{enumerate}
\subsection{局部有限型$k$概形上的闭点}
\begin{enumerate}
	\item 设$k$是域,$X$是局部有限型$k$概形,设$x\in X$,那么如下条件互相等价:
	\begin{itemize}
		\item $x$是闭点.
		\item 域扩张$k\subseteq\kappa(x)$是有限的(这里的扩张来自于$X$作为$k$概形时所有截面环局部环都具有典范的$k$模结构,并且都是由典范态射诱导的).
		\item 域扩张$k\subseteq\kappa(x)$是代数的.
	\end{itemize}
    \begin{proof}
    	
    	1推2,如果$x$是闭点,任取它的仿射开邻域$U=\mathrm{Spec}A$,那么$A$是有限型$k$代数,$x$也是$U$的闭点,于是它对应于$A$的极大理想$m$,按照零点定理得到$\kappa(x)=A/m$是$k$的有限扩张.2推3平凡,3推1是因为选取$x$附近的仿射开邻域$U=\mathrm{Spec}A$,考虑如下复合映射:
    	$$k\to A\to A/p_x\to\mathrm{Frac}(A/p_x)=\kappa(x)$$
    	
    	按照$\kappa(x)$在$k$上整,得到子环$A/p_x$在域$k$上整,这导致$A/p_x$是域,所以$x$是$U$的闭点,对$x$的任意仿射开邻域$U$成立.这说明$x$是$X$的闭点:如果$\{x\}$闭包中还有另一个点$y$,任取$y$的仿射开邻域$U$就必须包含点$x$,但是$\{x\}$在$U$中的闭包是它在$X$中闭包与$U$的交,所以$\{x\}$在$U$中的闭包至少还含有$y$,这和它在$U$中是闭点矛盾.
    \end{proof}
    \item 上一条里我们还证明了这样一件事:如果$X$是局部有限型$k$概形,如果$x\in X$是某个开集的闭点,那么$x$是$X$的闭点.
    \item 拓扑空间$X$的子集$Y$称为非常稠密,如果它满足如下等价描述中的任意一个:
    \begin{enumerate}
    	\item 映射$U\mapsto U\cap Y$是从$X$上全部开集到$Y$上全部开集的双射.
    	\item 映射$F\mapsto F\cap Y$是从$X$上全部闭集到$Y$上全部闭集的双射.
    	\item 对每个闭集$F\subseteq X$,有$F=\overline{F\cap Y}$.
    	\item 每个$X$的非空的局部闭子集$Z$都包含了$Y$中的点.
    \end{enumerate}
    \begin{proof}
    	
    	前三条等价是直接的,其中第三条仅仅是给出了具体的逆映射.下证3推4:记$Z=F-F'$,其中$F'\subsetneqq F$是两个闭集,假设有$(F\cap Y)-(F'\cap Y)=Z\cap Y=\emptyset$,那么$F\cap Y=F'\cap Y$,条件3得到$F=F'$.最后证4推2,假设有两个$X$的闭子集$F,F'$满足$F\cap Y=F'\cap Y$,此即$((F\cup F')-(F\cap F'))\cap Y=\emptyset$,条件4得到$F\cup F'=F\cap F'$,于是$F=F'$.
    \end{proof}
    \item 设$X$是局部有限型$k$概型,那么$X$的全部闭点集构成的子集$Z$是非常稠密的.
    \begin{proof}
    	
    	我们来验证每个局部闭子集$E$都包含$X$的闭点.$E$可以表示为开集$U$和闭集$C$的交,我们可以缩小这个开集$U$变成仿射开子集,那么这个局部闭子集就是素谱的闭子概型,这还是仿射的,所以必存在极大理想,但是这个极大理想也必须是$U$的极大理想,于是我们找到了$U$的闭点,但是我们解释过局部有限型$k$概形上开集的闭点就是全集的闭点,这得证.
    \end{proof}
    \item 特别的,如果$k$是代数闭域,设$X$是局部有限型$k$概型,那么$X$的闭点集恰好就是$\{x\in X\mid k=\kappa(x)\}$,另外按照局部概型为起点的态射的描述,这个集合恰好就是$\mathrm{Hom}_k(\mathrm{Spec}(k),X)$.
\end{enumerate}
\subsection{概形语言下的簇}

我们要给出如下范畴等价,设$k$是代数闭域.
\begin{itemize}
	\item $k$上仿射簇,即$k$上不可约仿射代数集$\Leftrightarrow$$k$上有限型整仿射概形.
	\item $k$上预簇,即$k$上被有限个仿射簇覆盖的不可约空间(这里不可约改成连通是等价的定义)$\Leftrightarrow$$k$上有限型整概形.
	\item $k$上射影簇$\Leftrightarrow$$k$上射影概形.
\end{itemize}
\begin{enumerate}
	\item 关于簇的定义歧义的注解.如果不要求预簇是不可约的,仅要求被有限个仿射簇覆盖,那么它对应的是既约有限型$k$概形;如果在预簇的定义中要求可以被无限个仿射簇覆盖,那么它对应的是局部有限型整$k$概形.
	\item 设$k$是代数闭域,我们解释过有如下三个等价的范畴,这得到第一个对应.
	\begin{itemize}
		\item 有限型仿射整的$k$概型范畴.
		\item 有限生成$k$整环范畴的反范畴.
		\item 仿射簇(不可约仿射代数集)范畴.
	\end{itemize}
    \item 一个拓扑引理.如果空间$X$的闭点集$Y$是稠密的,那么$X$是不可约空间当且仅当$Y$是不可约的.
    \item 设$X$是局部有限型$k$概形,闭点集就是$k$有理点集$X(k)$,我们解释过这是一个非常稠密集.取包含映射$\alpha:X(k)\subseteq X$,取回拉层$\mathscr{O}_{X(k)}=\alpha^{-1}\mathscr{O}_X$.这得到一个环空间$(X(k),\mathscr{O}_{X(k)})$.那么首先断言对开集$U\subseteq X$有:
    $$\mathscr{O}_{X(k)}(U\cap X(k))=\mathscr{O}_X(U)$$
    \begin{proof}
    	
    	按照回拉层的定义,$\mathscr{O}_{X(k)}$是预层$U\cap X(k)\mapsto\lim\limits_{\substack{\rightarrow\\U\cap X(k)\subseteq V}}\mathscr{O}_X(V)$的层化.但是这里如果开集$V\subseteq X$满足$U\cap X(k)\subseteq V$,那么必须有$U\subseteq V$,因为否则的话$U\cap V^c$是非空的,这是一个局部闭子集,按照$X(k)$是非常稠密的,说明$U\cap V^c$中有$X(k)$的点,这和$U\cap X(k)\subseteq V$矛盾.于是$\lim\limits_{\substack{\rightarrow\\U\cap X(k)\subseteq V}}\mathscr{O}_X(V)$本身就是$\mathscr{O}_X(U)$.按照$\mathscr{O}_X$是层就说明这个预层已经是层.
    \end{proof}
    \item 我们构造的$(X,\mathscr{O}_X)\mapsto(X(k),\mathscr{O}_{X(k)})$是从$k$上有限型整概形范畴到$k$上预簇范畴的函子.
    \begin{proof}
    	
    	如果$(X,\mathscr{O}_X)$是有限型整$k$仿射概形,可记$X=\mathrm{Spec}k[X_1,\cdots,X_n]/I$,其中$I$是素理想.按照零点定理,就有$X(k)$是由$I$定义的仿射代数集,这里$I$是素理想导致$X(k)$是仿射簇.另外要说明结构层是仿射簇上的正则函数层,任取$f\in\mathscr{O}_{X(k)}(U\cap X(k))=\mathscr{O}_X(U)$.它要视为$U\cap X(k)$上的正则函数,即把$x\in U\cap X(k)$映射为$f$在映射$\mathscr{O}_X(U)\to\mathscr{O}_{X,x}\to\kappa(x)=k$下的像.如果选取$U=\mathrm{Spec}A$是仿射开子集,那么$A$是一个有限型$k$代数$k[X_1,\cdots,X_n]/I$,那么$f\in\mathscr{O}_X(U)=A$自然是一个正则函数(多项式函数),它把$U\cap X(k)$中的点(此为素谱$U$的极大理想,由零点定理对应为$k^n$中的满足$I$的点)$(a_1,\cdots,a_n)$映射为$f(a_1,\cdots,a_n)$.于是$\mathscr{O}_{X(k)}(U\cap X(k))$就是代数簇$X(k)$开集$U\cap X(k)$上的正则函数.
    	
    	\qquad
    	
    	另外还要说明如果$f,g\in\mathscr{O}_X(U)$,如果它们视为正则函数相同,那么$f=g$.不妨设$U=\mathrm{Spec}A$是仿射开子集,如果$f,g\in A$在$U\cap X(k)$的每个点的剩余类域相同,此即$f-g$落在$A$的每个极大理想中,按照$A$是既约有限型$k$代数,它的Jacobson根和幂零根相同并且都是0,这说明$f-g=0$.
    	
    	\qquad
    	
    	接下来如果$(X,\mathscr{O}_X)$是一般的有限型整$k$概形.那么它是不可约的并且可以被有限个有限型整$k$仿射概形$\{U_i\}$覆盖.我们解释过$X(k)$还是不可约的,我们还解释过$U_i$的闭点还是$X$的闭点,所以$X(k)=\cup_iU_i(k)$就是有限个仿射簇的并,所以这是预簇.并且$\mathscr{O}_{X(k)}$也就是预簇上的正则函数层.
    	
    	\qquad
    	
    	最后要说明函子性,即有限型整$k$概形之间的$k$态射必然对应于诱导的预簇之间的态射.归结为仿射的情况,态射$f:X\to Y$对应的环同态就诱导了$X(k)\to Y(k)$的仿射簇态射.这里我们仅需要验证下有限型整$k$概形之间的态射$f:X\to Y$必然把闭点打到闭点.而这是因为如果$x\in X$,$y=f(x)\in Y$,那么$f$诱导了同态$\kappa(y)\to\kappa(x)=k$使得如下图表交换,这就说明$\kappa(y)=k$,于是$y=f(x)$也是闭点.
    	$$\xymatrix{&k\ar[dr]^{\mathrm{id}_k}\ar[dl]&\\\kappa(y)\ar[rr]&&k}$$
    \end{proof}
    \item 拓扑空间的sober化.设$X$是$T_1$空间,记$t(X)$表示$X$上所有不可约闭子集构成的集合.其上拓扑定义为,闭子集具有形式$t(Z)\subseteq t(X)$,其中$Z\subseteq X$是闭子集.这满足闭集公理是因为$t(\cap_iZ_i)=\cap t(Z_i)$和$t(Z_1\cup Z_2)=t(Z_1)\cup t(Z_2)$.
    \begin{itemize}
    	\item 函子性.如果$f:X\to Y$是$T_1$空间之间的连续映射,定义$t(f):t(X)\to t(Y)$为把$X$的不可约闭子集$E$映射为$Y$的不可约闭子集$\overline{f(E)}$.这是连续映射因为对$Y$上的闭子集$N$总有如下等式:
    	$$t(f)^{-1}(t_Y(N))=t_X(f^{-1}(N))$$
    	\begin{proof}
    		
    		设$X$上的不可约闭子集$E$在左边,也即$\overline{f(E)}\subseteq N$,那么$E\subseteq f^{-1}(N)$,此即左侧包含于右侧.
    		
    		设$X$上的不可约闭子集$E$在右边,也即$E\subseteq f^{-1}(N)$,那么$f(E)\subseteq N$,于是$\overline{f(E)}\subseteq N$,此即右侧包含于左侧.
    	\end{proof}
        \item $t(X)$的不可约闭子集具有形式$t(Z)$,其中$Z\subseteq X$是不可约闭子集,此时$Z$是$t(Z)$的唯一一般点.这说明$t(X)$是sober空间(即每个不可约闭子集恰好存在唯一的一般点),所以我们这里的构造称为空间的sober化(sobrification).
        \begin{proof}
        	
        	一方面如果$t(X)$的不可约闭子集记作$t(Z)$,其中$Z\subseteq X$是闭子集,那么必须有$Z$是不可约的,因为否则$Z$可以表示为两个更小的闭子集的并$Z_1\cup Z_2$,导致$t(Z)=t(Z_1)\cup t(Z_2)$.按照$t(Z)$是不可约的不妨设$t(Z)=t(Z_1)$.由于单点都是不可约闭子集,说明$t(Z)=t(Z_1)$推出$Z=Z_1$.
        	
        	\qquad
        	
        	另一方面如果$Z\subseteq X$是不可约闭子集,那么$t(Z)$必须是不可约的.否则有$t(Z)=t(Z_1)\cup t(Z_2)=t(Z_1\cup Z_2)$,那么$Z=Z_1\cup Z_2$,于是可不妨设$Z=Z_1$,导致$t(Z)=t(Z_1)$.
        	
        	\qquad
        	
        	最后$Z$是$t(Z)$的唯一一般点是因为对不可约闭子集$Z$,在$t(X)$中总有$\overline{\{Z\}}=t(Z)$.而这明显是唯一的一般点.
        \end{proof}
        \item 如果$X$是$T_1$空间,它的单点子集都是不可约闭子集,所以有典范的单连续映射$\beta:X\to t(X)$.这是从$X$到$t(X)$闭点子集的同胚,并且是$t(X)$的非常稠密子集.
    \end{itemize}
    \item 设$(X,\mathscr{O}_X)$是$k$预簇,其中$\mathscr{O}_X$是正则函数层.我们证明$(t(X),\beta_*\mathscr{O}_X)$是$k$有限型整概形.
    \begin{proof}
    	
    	如果$X$是仿射簇,设仿射坐标环为$A$,那么$X$同胚于$\mathrm{Spec}A$的闭点子集,于是$t(X)$同胚于$\mathrm{Spec}A$.对$f\in A$,有$\beta_*\mathscr{O}_X(D(f))=\mathscr{O}_X(\beta^{-1}(D(f)))=\mathscr{O}_X(D(f)\cap X)=A_f$,这说明$(t(X),\beta_*\mathscr{O}_X)$就是一个仿射概形.并且按照仿射簇是不可约的,它的坐标环$A$是既约的还是有限型$k$代数,说明$(t(X),\beta_*\mathscr{O}_X)$是既约的有限型仿射$k$概形.
    	
    	\qquad
    	
    	现在设$X$是预簇,也即它是不可约的,并且可以被有限个仿射簇覆盖,于是$(t(X),\beta_*\mathscr{O}_X)$也是不可约的,并且可以被有限个既约仿射概形覆盖,于是$t(X)$是不可约的有限型既约$k$概形,换句话讲它是有限型整$k$概形.
    	
    	\qquad
    	
    	最后说明函子性,如果$f:X\to Y$是预簇之间的态射,那么$t(f):t(X)\to t(Y)$是连续映射.取层态射为$\beta^Y_*\mathscr{O}_Y\to t(f)_*(\beta^X_*\mathscr{O}_X)$.在仿射情况下,仿射簇之间的的态射$f:X\to Y$对应于坐标环之间的$k$代数同态$\varphi:\mathbb{A}[Y]\to\mathbb{A}[X]$.于是$t(f)$为把$X$上的不可约闭子集,也即$\mathbb{A}[X]$的素理想$p$,映射为$Y$上的不可约闭子集$\varphi^{-1}(p)$,也即$\mathbb{A}[Y]$的素理想.于是$t(f)$就是环同态诱导的连续映射.层态射也是环同态诱导的层态射.
    \end{proof}
    \item 我们构造的两个函子互为拟逆函子,于是我们证明了代数闭域$k$上的预簇范畴和$k$上的有限型整概形范畴是范畴等价的.在这个对应下射影簇是和射影概形对应的(如果$V$是齐次坐标环为$S$的射影簇,那么$t(V)\cong\mathrm{Proj}S$).至此我们证明了承诺的三个范畴等价.
\end{enumerate}
\subsection{Grassmannian概形}
\begin{enumerate}
	\item 粗略的讲,Grassmannian概形是射影空间的一种推广.射影空间可以视为线性空间上一维子空间的参数化,Grassmannian概形则是全体$d$维子空间的参数化.以仿射情况为例,给定环$A$,模$A^{\oplus n}$的全体$d$维子自由模构成的集合记作$\mathbb{G}(d,n)$.任取$A^{\oplus n}$的一组基$\{v_1,v_2,\cdots,v_n\}$,那么$A^{\oplus}$中的每个点对应于$A$上的一个$n$元组.一个$d$维子空间被$d$个线性无关的$n$元组生成.两组这样的$d$元对对应了相同的$d$维子空间当且仅当它们之间的过渡矩阵是一个$A$上的$d$阶可逆矩阵.于是$d$维线性子空间可以表示为$d\times n$矩阵的等价类,两个这样的矩阵等价当且仅当它们相差一个左乘可逆矩阵.下面定义概形结构,取定一组基$v=\{v_1,v_2,\cdots,v_n\}$,考虑这样的$d\times n$的矩阵集合:当$1\le i,j\le d$时定义$a_{ij}=\delta_{ij}$,当$i\ge d+1$时$a_{ij}$任取.那么这样定义的不同矩阵对应于不同的$d$线性子空间,这定义了$\mathbb{G}(k,n)$的一个子集,记作$U_v$.在每个$U_v$上定义概型结构$\mathbb{A}_R^{k(n-k)}$,再粘合得到一个参数化了所有$d$维子空间的概形.但这只是粗略的讲,对于一般概形我们要给出$d$维子空间的具体含义.
	\item 引理.设$S$是概形,设$i:\mathscr{G}\to\mathscr{F}$是$\mathscr{O}_S$模层之间的态射,考虑如下命题,如果$\mathscr{F}$是拟凝聚的,那么有(a)$\Rightarrow$(b)$\Rightarrow$(c)$\Rightarrow$(d)成立;如果$\mathscr{G}$是有限生成模层,$\mathscr{F}$是有限局部自由模层,那么所有命题互相等价.
	\begin{enumerate}
		\item $i$是单态射,并且$\mathscr{O}_S$模层$\mathscr{F}/i(\mathscr{G})$是局部自由模层.
		\item 对任意仿射开子集$U\subseteq S$,存在模层态射$\pi:\mathscr{F}\mid_U\to\mathscr{G}\mid_U$使得$\pi\circ i\mid_U=\mathrm{id}$.这等价于讲$i(\mathscr{G})$是$\mathscr{F}$的局部直和项(此即对任意$s\in S$存在开邻域$V$使得$\mathscr{G}\mid_V$是$\mathscr{F}\mid_V$的直和项).
		\item 对任意概形态射$f:T\to S$,有$\mathscr{O}_T$模层态射$f^*(i):f^*(\mathscr{G})\to f^*(\mathscr{F})$是单态射.
		\item 对任意$s\in S$有$\kappa(s)$线性空间之间的同态$i\otimes\mathrm{id}_{\kappa(s)}:\mathscr{G}_s\to\mathscr{F}_s$是单射.
		\item $\mathscr{G}$是有限局部自由$\mathscr{O}_S$模层,并且对偶态射$i^{\vee}:\mathscr{F}^{\vee}\to\mathscr{G}^{\vee}$是满射.
	\end{enumerate}
    \begin{proof}
    	
    	问题都是局部的,不妨设$S=\mathrm{Spec}R$是仿射的,设$\mathscr{F}=\widetilde{M}$,按照$\mathscr{F}$是局部自由的,得到$M$是投射$R$模.(a)推(b):不妨设$U=S$,我们有短正合列$0\to\mathscr{G}\to\mathscr{F}\to\mathscr{F}/i(\mathscr{G})\to0$,于是从后两项是拟凝聚的得到$\mathscr{G}$是拟凝聚的,可记$\mathscr{G}=\widetilde{N}$,于是有$R$模的短正合列$0\to N\to M\to M/N\to0$,由于$\widetilde{F}/i(\mathscr{G})$是局部自由的,得到这里$M/N$是投射模,于是这个短正合列分裂,此即(b)成立.下面(b)推(c)平凡,(c)推(d)只要取典范的$f:T=\mathrm{Spec}\kappa(s)\to S$.
    	
    	\qquad
    	
    	下面设$\mathscr{G}$是有限生成模层,设$\mathscr{F}$是有限局部自由模层,我们来证明(d)推(a),要证明的是$\mathscr{F}/i(\mathscr{G})$是局部自由的,这等价于讲$\mathscr{F}/i(\mathscr{G})$是有限表示的并且$(\mathscr{F}/i(\mathscr{G}))_s$总是自由$\mathscr{O}_{S,s}$模.按照$\mathscr{F}$和$\mathscr{G}$的条件知$\mathscr{F}/i(\mathscr{G})$是有限表示的,另一方面局部环上自由模和投射模等价,等价于是某个自由模的直和项.于是问题归结为如下交换代数:记$\mathscr{O}_{S,s}=(A,\mathfrak{m},k)$,记$N=\mathscr{G}_s$和$M=\mathscr{F}_s$,记$i':i_s$.条件是$k$线性空间的态射$i_0=i'\otimes\mathrm{id}_{A/\mathfrak{m}}:N/\mathfrak{m}N\to M/\mathfrak{m}M$是单射,我们只需证明$i'$是单射,并且$i'(N)$是$M$的直和项.
    	
    	\qquad
    	
    	设$r_0$是$i_0$的左逆同态,那么它是满射,于是复合上满射$M\to M/\mathfrak{m}M$也是满射.可取同态$r':M\to N$使得$\left(M\to M/\mathfrak{m}M\to N/\mathfrak{m}N\right)=\left(M\to N\to N/\mathfrak{m}N\right)$.现在$r'\circ i'$是$N$的自同态,并且模$\mathfrak{m}$下是恒等,于是NAK引理导致$r'\circ i'$是满射,但是有限模上的满射自同态一定是同构,于是$r'\circ i'$是同构,于是$r=(r'\circ i')^{-1}\circ r':M\to N$满足$r\circ i'=\mathrm{id}_N$,这说明$i'$是单射并且$i'(N)$是$M$的直和项.这完成了(d)推(a)的证明.
    	
    	\qquad
    	
    	(e)和前四条的等价性:首先如果前四条成立,那么$i(\mathscr{G})$是$\mathscr{F}$的局部直和项,但是按照$\mathscr{F}/i(\mathscr{G})$是局部自由模层,导致$0\to\mathscr{G}\to\mathscr{F}\to\mathscr{F}/i(\mathscr{G})\to0$分裂,又因为$\mathscr{F}$是有限局部自由的,就得到$\mathscr{G}\cong i(\mathscr{G})$是有限局部自由的.接下来按照$\mathscr{G}$是有限局部自由的,我们有$(i^{\vee})^{\vee}=i$和$(i\otimes\mathrm{id}_{\kappa(s)})^{\vee}=i^{\vee}\otimes\mathrm{id}_{\kappa(s)}$.于是(d)等价于讲对任意$s\in S$有$i^{\vee}\otimes\mathrm{id}_{\kappa(s)}$是满射,从NAK引理得到这等价于讲$i^{\vee}$是满射.
    \end{proof}
    \item 设$n$是正整数,设$1\le d\le n$是一个整数,对概形$S$,我们记$\mathrm{Grass}_{d,n}(S)$为$\mathscr{O}_S^n$的所有子模层$\mathscr{G}$,满足$\mathscr{O}_S^n/\mathscr{G}$是一个秩为$n-d$的局部自由$\mathscr{O}_S$模层.下面考虑函子性,如果$f:T\to S$是概形的态射,设$\mathscr{G}\in\mathrm{Grass}_{d,n}(S)$,我们定义$\mathrm{Grass}_{d,n}(f)(\mathscr{G})=f^*\mathscr{G}$,那么仍然有$f^*\mathscr{G}\in\mathrm{Grass}_{d,n}(T)$:上面引理告诉我们$f^*\mathscr{G}\to f^*\mathscr{O}_S^n=\mathscr{O}_T^n$也是单射,$f^*$是右正合的并且保局部自由模层说明$f^*(\mathscr{O}_S^n/\mathscr{G})=\mathscr{O}_T^n/f^*\mathscr{G}$也是局部秩$n-d$自由模层.于是我们定义了逆变函子$\mathrm{Grass}_{d,n}:\textbf{S-Sch}^{\mathrm{op}}\to\textbf{Sets}$,它称为Grassmannian函子.
    \item Grassmannian函子被一个概形表示,它称为Grassmannian概形,也记作$\mathrm{Grass}_{d,n}$.
    \begin{proof}
    	
    	我们当然要用可表准则,对任意由$n-d$个不同元构成的子集$I\subseteq\{1,\cdots,n\}$,我们定义$\mathrm{Grass}_{d,n}$的子函子$\mathrm{Grass}_{d,n}^I$为,在概形$S$上取为$\mathrm{Grass}_{d,n}(S)$的子集,由那些满足$\mathscr{O}_S^I\to\mathscr{O}_S^n\to\mathscr{O}_S^n/\mathscr{G}$的有限局部自由模层$\mathscr{G}$构成.其中$\mathscr{O}_S^I\to\mathscr{O}_S^n$取为$I\to\{1,\cdots,n\}$诱导的包含映射.换句话讲$\mathrm{Grass}_{d,n}^I(S)$中的元是满足$\mathscr{G}\oplus\mathscr{O}_S^I=\mathscr{O}_S^n$的有限局部自由模层$\mathscr{G}$构成的集合.包含映射$\mathrm{Grass}_{d,n}^I(S)\to\mathrm{Grass}_{d,n}(S)$定义了自然变换$i^I:\mathrm{Grass}_{d,n}^I\to\mathrm{Grass}_{d,n}$.
    	
    	我们先证明$\mathrm{Grass}_{d,n}^I$是$\mathrm{Grass}_{d,n}$的开子函子.任取概形$X$,记函子$h=\mathrm{Grass}_{d,n}^I\times_{\mathrm{Grass}_{d,n}}h_X$,我们要证明的是函子$h$被$X$的一个开子概形$U$表示.按照定义对任意概形$S$有$h(S)=\{(f,\mathscr{G})\mid f\in h_X(S)=\mathrm{Hom}(S,X),\mathscr{G}\in\mathrm{Grass}_{d,n}^I,f^* \}$【】
    	
    	再证明$\mathrm{Grass}_{d,n}^I$被$\mathbb{A}^{d(n-d)}$表示.设$S$是任意概形,设$\mathscr{G}\in\mathrm{Grass}_{d,n}^I(S)$,按照定义有同构$\mathscr{O}_S^I\cong\mathscr{O}_S^n/\mathscr{G}$.考虑典范商态射和这个同构的复合,记作$u_{\mathscr{G}}:\mathscr{O}_S^n\to\mathscr{O}_S^n/\mathscr{G}\to\mathscr{O}_S^I$.如果记典范包含态射$\mathscr{O}_S^I\to\mathscr{O}_S^n$为$u^I$,那么有$u_{\mathscr{G}}\circ u^I=\mathrm{id}_{\mathscr{O}_S^I}$,并且$u_{\mathscr{G}}$的核就是$\mathscr{G}$.另一方面如果态射$u:\mathscr{O}_S^n\to\mathscr{O}_S^I$满足$u\circ u^I=\mathrm{id}_{\mathscr{O}_S^I}$,那么短正合列$0\to\ker u\to\mathscr{O}_S^n\to\mathscr{O}_S^I$局部分裂,于是$\ker u\in\mathrm{Grass}_{d,n}^I(S)$.综上我们证明了有$F(S)=\{u\in\mathrm{Hom}_{\mathscr{O}_S}(\mathscr{O}_S^n,\mathscr{O}_S^I)\mid u\circ u^I=\mathrm{id}_{\mathscr{O}_S^I}\}$到$\mathrm{Grass}_{d,n}^I(S)$的双射为$u\mapsto\ker u$.如果记$J=\{1,\cdots,n\}-I$,那么$F(S)\cong\mathrm{Hom}_{\mathscr{O}_S}(\mathscr{O}_S^J,\mathscr{O}_S^I)$,这明显被$\mathbb{A}^{d(n-d)}$表示.
    	
    	最后证明当$I$跑遍$\{1,\cdots,n\}$的$n-d$元子集时$\{i^I:\mathrm{Grass}_{d,n}^I\to\mathrm{Grass}_{d,n}\}$是$\mathrm{Grass}_{d,n}$的Zariski开覆盖.设$S$是任意概形,设$\mathrm{Grass}_{d,n}^I\times_{\mathrm{Grass}_{d,n}}h_X$被$X$的开子概型$U_I$表示.我们要证明的就是$\{U_I\}$覆盖了整个$X$,或者等价的讲$f:\coprod_IU_I\to X$是一个满射.而这只需证明对任意域$K$有$f$在$K$值点上诱导了满射(我们解释过这个条件,一般的态射是满射未必推出这个条件,但是这个条件总能推出态射是满射).任取$K$值点$x:\mathrm{Spec}K\to X$,它在自然变换$g$下的像是$\mathrm{Grass}_{d,n}(K)$的元,也即$K^n$的一个$d$维子空间记作$U$.于是$x$落在$U_I(K)\to X(K)$的像中当且仅当$K^I$是$U$的直和补,并且我们总能取$I$实现这件事,这就说明了$f$在$K$值点上是满射.
    \end{proof}
    \item 推论.按照上面证明,概形$\mathrm{Grass}_{d,n}$被有限个$\mathbb{A}^{d(n-d)}$覆盖,特别的它是一个光滑概形,并且相对光滑维数是$d(n-d)$.
    \item 射影空间是$d=1$的情况.设$S$是概形,设$\mathscr{L}$是$\mathscr{O}_S^n$的秩1局部自由的子模层,那么存在$S$的仿射开覆盖$\{U_{\alpha}\}$,使得在$U_{\alpha}$上$\mathscr{L}\mid_{U_{\alpha}}$就要被$\Gamma(U_{\alpha},\mathscr{O}_S^n)=A_{\alpha}^n$的单个元$x=(x_0,\cdots,x_{n-1})$生成.另外$\mathscr{L}\mid_{U_{\alpha}}$是$\mathscr{O}_{U_{\alpha}}^n$的秩1自由的直和项当且仅当$x_0,\cdots,x_{n-1}$生成了$A_{\alpha}$的单位理想.下面任取$i\in\{0,\cdots,n-1\}$,记$I_i=\{1,\cdots,n\}-\{i+1\}$,那么$\mathrm{Grass}_{1,n}^{I_i}(S)$是由这样的线丛$\mathscr{L}$构成:它满足局部上被单个元$(x_0,\cdots,x_{n-1})$生成,并且这里$x_i$要落在截面环的单位中(因为有$\mathscr{L}\oplus\mathscr{O}_S^{I_i}=\mathscr{O}_S^n$),按照$\mathrm{Grass}_{1,n}\cong h_{\mathbb{A}^{n-1}}$.于是$(x_0,\cdots,x_{n-1})\in U_i\mapsto(x_0/x_i,\cdots,\widehat{x_i/x_i},\cdots,x_{n-1}/x_i)$就粘合为一个同构$\mathrm{Grass}_{1,n}\cong\mathbb{P}_{\mathbb{Z}}^{n-1}$.
    \item 拟凝聚模层的Grassmannian概形.更一般的,设$S$是概形,设$\mathscr{E}$是拟凝聚$\mathscr{O}_S$模层,设$e\ge0$是整数,我们来构造$\textbf{S-Sch}^{\mathrm{op}}\to\textbf{Sets}$的函子$\mathrm{Grass}^e(\mathscr{E})$:
    \begin{itemize}
    	\item 对$S$概形$h:T\to S$,集合$\mathrm{Grass}^e(\mathscr{E})(T)$由全体$\mathscr{O}_T$子模层$\mathscr{G}\subseteq h^*(\mathscr{E})$构成,使得$h^*(\mathscr{E})/\mathscr{G}$是秩$e$的局部自由$\mathscr{O}_T$模层.
    	\item 对$S$态射$f:T'\to T$,集合之间的映射$\mathrm{Grass}^e(\mathscr{E})(T)\to\mathrm{Grass}^e(\mathscr{E})(T')$取为$\mathscr{G}\mapsto f^*(\mathscr{G})$.
    	\item 特别的,如果取$\mathscr{E}=\mathscr{O}_S^n$,取$e=n-d$,那么$\mathrm{Grass}^e(\mathscr{E})=\mathrm{Grass}_{d,n}\times_{\mathbb{Z}}S$,也会记作$\mathrm{Grass}_{d,n,S}$.这是前面的情况.
    \end{itemize}
    \item 性质.
    \begin{enumerate}
    	\item 可表性.设$\mathscr{E}$是拟凝聚$\mathscr{O}_S$模层,那么$\mathrm{Grass}^e(\mathscr{E})$被一个$S$概形表示.【】
    	\item 对应闭嵌入.如果$v:\mathscr{E}_1\to\mathscr{E}_2$是$\mathscr{O}_S$模层的满射态射,那么它诱导了自然变换$i_v:\mathrm{Grass}^e(\mathscr{E}_2)\to\mathrm{Grass}^e(\mathscr{E}_1)$.并且如果$v$是拟凝聚模层之间的满射态射,那么$i_v$是一个闭嵌入.【】
    	\item 对偶.设$\mathscr{E}$是秩$n$的局部自由$\mathscr{O}_S$模层,我们有典范同构:
    	$$\mathrm{Grass}^e(\mathscr{E})\cong\mathrm{Grass}^{n-e}(\mathscr{E}^{\vee})$$
    	$$\left(i:\mathscr{G}\to h^*\mathscr{E}\right)\mapsto\left(\mathrm{coker}(i)^{\vee}\to h^*\mathscr{E}^{\vee}\right)$$
    	【】
    	\item 基变换.设$S$是概形,设$\mathrm{E}$是拟凝聚$\mathscr{O}_S$模层,设$e\ge0$是整数,设$u:S'\to S$是态射.设$T$是$S'$概形,那么复合$u$导致它总能视为$S$概形.我们有如下典范同构:
    	$$\mathrm{Hom}_{S'}(T,\mathrm{Grass}^e(\mathscr{E})\times_SS')=\mathrm{Hom}_S(T,\mathrm{Grass}^e(\mathscr{E}))=\mathrm{Hom}_{S'}(T,\mathrm{Grass}^e(u^*\mathscr{E}))$$
    	
    	这里第二个同构是因为如果记$h:T\to S'$,记$h'=u\circ h$,那么$h^*\mathscr{E}=h'^*(u^*\mathscr{E})$.另外按照米田引理我们有如下典范的$S'$同构:
    	$$\mathrm{Grass}^e(\mathscr{E})\times_SS'=u^*\mathrm{Grass}^e(\mathscr{E})\cong\mathrm{Grass}^e(u^*\mathscr{E})$$
    \end{enumerate}
    \item 射影丛.设$S$是概形,设$\mathscr{E}$是拟凝聚$\mathscr{O}_S$模层,定义$\mathscr{E}$的射影丛是$S$概形$\mathrm{Grass}^1(\mathscr{E})$,记作$\mathbb{P}(\mathscr{E})$.于是按照定义,对$S$概形$h:T\to S$,有$\mathbb{P}(\mathscr{E})(T)$典范的一一对应于满射$h^*\mathscr{E}\to\mathscr{L}$的等价类,其中$\mathscr{L}$是秩1局部自由$\mathscr{O}_T$模层.
    \begin{enumerate}
    	\item 特别的,取$S=\mathrm{Spec}\mathbb{Z}$和$\mathscr{E}=(\mathscr{O}_{\mathrm{Spec}\mathbb{Z}}^{n+1})^{\vee}$,那么对偶性告诉我们:
    	$$\mathbb{P}((\mathscr{O}_{\mathrm{Spec}\mathbb{Z}}^{n+1})^{\vee})=\mathrm{Grass}^n(\mathscr{O}_{\mathrm{Spec}\mathbb{Z}}^{n+1})=\mathbb{P}^n$$
    	\item 基变换.设$u:S'\to S$是概形态射,那么对拟凝聚$\mathscr{O}_S$模层$\mathscr{E}$就有:
    	$$u^*\mathbb{P}(\mathscr{E})=\mathbb{P}(u^*\mathscr{E})$$
    \end{enumerate}
\end{enumerate}
\subsection{拟紧概形}

一个概型称为拟紧的,如果它作为拓扑空间是拟紧的.
\begin{enumerate}
	\item 一个概型$X$是拟紧的当且仅当它可以被有限个仿射开子集覆盖.必要性是因为有限个拟紧集的并是拟紧的;充分性是因为仿射开覆盖可取有限子覆盖.
	\item 于是按照射影概型的粘合构造,说明对任意环$A$,有$\mathbb{P}_A^n$总是拟紧的.
	\item 一个拓扑性质:非空拟紧$T_0$空间上总存在闭点.我们解释了概型总是$T_0$空间,于是特别的非空拟紧概型上总存在闭点.注意存在这样的非空概型上不存在闭点.
	\begin{proof}
		
		记$S$表示这个空间上全部单点集闭包构成的集合,赋予反向包含序,我们来验证Zorn引理:取$S$的全序子集$\{Z_i,i\in I\}$,这里$I$是一个全序集,这个链的有限子集的交总是非空的,按照拟紧条件,说明这个全序子集的交是非空的,可取$x\in\cap_iZ_i$,于是$\overline{\{x\}}$是这个全序子集的上界.于是按照Zorn引理,可取$S$的极大元$Z=\overline{\{x_0\}}$.如果$Z$中有异于$x_0$的点$x_1$,那么有$\overline{\{x_1\}}\subseteq\overline{\{x_0\}}$,极大性说明这个包含式是等式,于是$x_1$的每个开邻域包含了$x_0$,而$x_0$的每个开邻域包含了$x_1$,这和$T_0$条件矛盾.于是$Z$是单点集$\{x_0\}$,于是$x_0$是闭点.
	\end{proof}
    \item 如果$P$是一个茎局部性质(即称开集$U$上有性质$P$等价于$U$中每个点的局部环具有性质$P$),那么只要在仿射情况下性质$P$只需验证闭点(即从每个极大理想处满足性质$P$可推出每个素理想处满足性质$P$),那么在拟紧概型上性质$P$只需验证闭点.
    \begin{proof}
    	
    	对于拟紧概型$X$,可取有限的仿射开覆盖$\{X_i=\mathrm{Spec}A_i\}$.假设$X$的闭点处的局部环都满足性质$P$,我们来证明每个$X_i$的闭点处满足性质$P$.这样按照条件得到每个$X_i$的每个点都满足性质$P$.
    	
    	任取$X_1$的闭点$x_1$,如果它不是$X$的闭点,那么存在$X_i\not=X$,$x_1\in X_i$,使得$x_1$不是$X_i$的闭点.不妨记$i=2$,否则可以重排这些仿射开子集的指标.取$A_2$中包含素理想$p_{x_1}$的极大理想$x_2$,那么$\overline{\{x_2\}}\subsetneqq\overline{\{x_1\}}$,如果$x_2$不是$X$的闭点,类似的可取$x_3,x_4,\cdots$.注意有$\overline{\{x_i\}}\cap X_i=\{x_i\}$,说明$i\not=j$时$X_i\not=X_j$.按照$i$是有限的,这个操作会在有限步后终止,最后得到的$x_n$是$X$的闭点.
    	
    	设$x_n$对应$A_n$的极大理想$m_n$,那么$x_{n-1}$处的局部环是$m_n$处局部环的局部化,于是从$x_n$的局部环满足性质$P$,得到$x_{n-1}$处局部环满足性质$P$,归纳得到$x_1$处局部环满足性质$P$.
    \end{proof}
    \item 拟紧概型上的整体函数$f$处处取值为零,当且仅当存在某个正整数$n$使得$f^n=0$.这是因为对于仿射情况,一个整体函数处处取零等价于它在幂零根中,于是有$n$使得$f^n=0$.而拟紧概型可取有限仿射开覆盖,取这些使得它次幂为零的最大次数,得证.这个结论对非拟紧概型是不成立的.例如取$X_n=\mathrm{Spec}k[X]/(X^n),n\ge1$的无交并,在每个$X_n$上取函数$X+(X^n)$,这粘合得到整体函数$f$,但是每个$f^n$都是非零的.
\end{enumerate}
\subsection{茎局部性质和仿射局部性质}
\begin{enumerate}
	\item 设$\mathrm{Spec}A$和$\mathrm{Spec}B$是概型$X$的两个仿射开子集,那么$\mathrm{Spec}A\cap\mathrm{Spec}B$可以被一族同时是$\mathrm{Spec}A$和$\mathrm{Spec}B$的主开集的开子集覆盖.
	\begin{proof}
		
		任取点$p\in\mathrm{Spec}A\cap\mathrm{Spec}B$,我们需要找$p$在$\mathrm{Spec}A\cap\mathrm{Spec}B$中的开邻域,使得它同时是$\mathrm{Spec}A$和$\mathrm{Spec}B$的主开集.先取$p$在开集$\mathrm{Spec}A\cap\mathrm{Spec}B$中的主开集邻域$\mathrm{Spec}A_f$.再取$p$再开集$\mathrm{Spec}A_f$中的主开集邻域$\mathrm{Spec}B_g$.这里$g\in B=\Gamma(\mathrm{Spec}B,\mathscr{O}_X)$,记它限制到$\mathrm{Spec}A_f$为$g'\in A_f$.那么$g$在$\mathrm{Spec}A_f$中的零点集恰好是$g'$在其上的零点集.于是$\mathrm{Spec}B_g=\mathrm{Spec}(A_f)_{g'}$.记$g'=g''/f^n$,得到$\mathrm{Spec}(A_f)_{g'}=\mathrm{Spec}A_{fg''}$.
	\end{proof}
	\item 仿射交换引理.设$X$是概型,设$P$是仿射开子集上可定义的性质,满足这样两个条件:如果仿射开子集$\mathrm{Spec}A$满足$P$,那么它的每个主开集$\mathrm{Spec}A_f$都满足$P$;如果$(f_1,f_2,\cdots,f_n)=A$,并且每个$\mathrm{Spec}A_i$具有性质$P$可推出$\mathrm{Spec}A$具有性质$P$.那么如果存在一个仿射开覆盖上每个开子集满足这个性质$P$,可推出对每个仿射开子集$\mathrm{Spec}A$都满足这个性质$P$.
	\item 仿射局部性.设$P$是一个关于交换环的性质,我们称概型是仿射$P$的,如果概型的每个仿射开子集对应的环满足性质$P$.那么如果性质$P$满足仿射交换引理中的两个条件:即满足性质$P$的环$A$的局部化$A_f,\forall f\in A$都有性质$P$;如果$A=(f_1,f_2,\cdots,f_n)$,那么从$A_{f_i},\forall i$有性质$P$可推出$A$具有性质$P$.那么验证概型是仿射$P$的,只需验证概型存在一个仿射开覆盖上都满足性质$P$.
	\item 茎局部性.设$P$是一个关于交换环的性质,称概型是茎$P$的,如果概型的每个点处的局部环满足性质$P$.我们之前证明过如果这样的性质$P$在仿射情况是只需验证闭点的,换句话讲如果对每个极大理想$m$有$A_m$满足这个性质,那么对每个素理想$p$有$A_p$满足这个性质,那么对拟紧概型,验证它是茎$P$的只需验证闭点处的局部环都满足性质$P$.
	\item 设$P$是交换环上的局部性质,也即一个交换环满足性质$P$等价于对每个素理想$p$有$A_p$满足这个性质,等价于对每个极大理想$m$有$A_m$满足这个性质.那么概型上仿射$P$是等价于茎$P$的,等价于存在一组仿射开覆盖使得每个对应的环都满足性质$P$.并且对于仿射概型,满足这个性质$P$当且仅当它对应的环满足这个性质.但是注意这未必说明对每个开子集$U$有$\mathscr{O}_X(U)$满足性质$P$.
\end{enumerate}
\subsection{拟分离概形}

一个拓扑空间称为拟分离的,如果任意两个拟紧开子集的交是拟紧的.
\begin{enumerate}
	\item 一个概型是拟分离的当且仅当它的任意两个仿射开子集的交是有限个仿射开子集的并.
	\item 仿射概型是拟分离的.任取两个拟紧开子集,可表示为主开集的并$U=\cup_iD(f_i)$和$V=\cup_jD(g_j)$,于是$U\cap V=\cup_{i,j}D(f_ig_j)$是有限个主开集的并,于是拟紧.
	\item 概形$X$是拟分离的当且仅当对任意拟紧开集$U$,开嵌入$U\to X$是拟紧的.
	\item 概型$X$是拟紧和拟分离的当且仅当它可以被有限个仿射开子集覆盖,并且其中任意两个的交也可以表示为有限个仿射开子集覆盖.
	\begin{proof}
		
		必要性是直接的.对于充分性,设$X$存在有限仿射开覆盖$X=\cup_iU_i$,并且每个$U_i\cap U_j$也存在有限仿射开覆盖$\{V_1^{ij},V_2^{ij},\cdots,V_{r_{ij}}^{ij}\}$.于是$X$是拟紧的,下面需要说明它是拟分离的,也即任取拟紧开子集$U$和$U'$,有$U\cap U'$是拟紧的.为此我们先证明每个$U\cap U_i$是拟紧的.一旦这成立,那么$U\cap U_i$和$U'\cap U_i$是拟紧的,按照我们解释过的仿射开子集$U_i$是拟分离的,就得到$U\cap U'\cap U_i=(U\cap U_i)\cap(U'\cap U_i)$是拟紧的,对$i$取有限并就得到$U\cap U'$是拟紧的.
		
		取每个$U\cap U_j$的拟紧开子集覆盖$W_{jk},k\in K_j$.得到$U=\cup_j\left(U\cap U_j\right)=\cup_j\cup_{k\in K_j}W_{jk}$.按照$U$是拟紧的,可取$K_j$的有限子集$K'_j$,使得$U=\cup_j\left(U\cap U_j\right)=\cup_j\cup_{k\in K'_j}W_{jk}$.于是有$U\cap U_1=\cup_j\cup_{k\in K'_j}\left(W_{jk}\cap U_1\right)$.现在$V_r^{ij}$和$W_{jk}$都是$U_j$的拟紧开子集,按照$U_j$上的拟分离性,得到$W_{jk}\cap V_r^{ij}$是$U_j$中的拟紧开子集.于是有如下等式,于是$U\cap U_1$是拟紧的.
		\begin{align*}
		U\cap U_1&=\cup_j\cup_{k\in K'_j}\left(W_{jk}\cap U_1\right)=\cup_j\cup_{k\in K'_j}\left(W_{jk}\cap U_1\cap U_j\right)\\&=\cup_j\cup_{k\in K'_j}\cup_{1\le r\le r_{1j}}\left(W_{jk}\cap V_r^{1j}\right)
		\end{align*}
	\end{proof}
    \item 设$S$是$A$上的分次环,此即$S_0=A$.设$S$被有限个一次元生成$\{x_1,x_2,\cdots,x_n\}$,那么$\{D(x_i)\}$是$S$对应的射影$A$概型$X$的有限仿射开覆盖,并且$D(x_i)\cap D(x_j)=D(x_ix_j)$是仿射的.这说明有限生成的射影$A$概型总是拟紧和拟分离的.
    \item 拟紧非拟分离的例子.设$X=\mathrm{Spec}k[x_1,x_2,\cdots]$,记$U=X-\{m\}$,其中$m$是闭点$(x_1,x_2,\cdots)$.取两个相同的$X$,把它们经$U$粘合,这得到了带两个$m$的概型.按照它的两个仿射开子集$X$的并,说明这个空间是拟紧的.但是做粘合的两个空间的像集都是拟紧的,它们的交即$U$,这不是拟紧的,因为它的开覆盖$\{D(x_i)\}$没有有限子覆盖.
    \item 设$X$是拟紧拟分离概形,设$\textbf{P}$是一个关于$X$的开子集的性质,满足如下两个条件:
    \begin{enumerate}
    	\item $\textbf{P}(U)$对任意仿射开子集$U$成立.
    	\item 如果$U\subseteq X$是拟紧开子集,使得$U=\cup_{1\le i\le n}U_i$是有限的仿射开覆盖,满足$\textbf{P}(U_i\cap U_j)$对任意$1\le i,j\le n$都成立,则有$\textbf{P}(U)$成立.
    \end{enumerate}

    那么$\textbf{P}(U)$对任意拟紧开子集$U$都成立.
\end{enumerate}
\subsection{既约概形}

一个交换环称为既约的,如果它没有非平凡幂零元,换句话讲幂零根是零理想.一个概形$(X,\mathscr{O}_X)$称为既约概形,如果对每个开集$U\subseteq X$都有$\mathscr{O}_X(U)$是既约环.
\begin{enumerate}
	\item 概形$(X,\mathscr{O}_X)$是既约概形等价于如下任一条:
	\begin{itemize}
		\item $X$上存在仿射开覆盖,使得其中每个开集的截面都是既约环.
		\item $X$的每个点的局部环都是既约环.
	\end{itemize}

    另外这里仿射开覆盖改为开覆盖是不成立的,如果概形不是仿射的,即便整体截面环是既约环,也不能说明它是既约概形.
	\item 仿射概形是既约环当且仅当它对应的环是既约环.
	\item 按照既约性是一个局部性质,对于拟紧概型,如果每个闭点处的局部环都是既约环,那么它是既约概型.
	\item 一个概型上的既约点未必构成开集.例如考虑$X=\mathrm{Spec}\mathbb{C}[x,y_1,y_2,\cdots]/(y_1^2,y_2^2,\cdots,(x-1)y_1,(x-2)y_2,\cdots)$.它和$\mathrm{Spec}\mathbb{C}[x]$是一一对应的,即$(x-a)$对应于$(x-a,y_1,y_2,\cdots)$.这个概型的非既约点恰好一一对应于$(x-n),n\in\mathbb{Z}$.它的补集不是开集.不过我们会证明对于局部诺特概型,它的既约点构成开子集.
	\item 关于纤维积.
	\begin{itemize}
		\item 设$k$是代数闭域,那么两个$k$既约概型的纤维积是既约的.按照既约性是仿射局部的,我们只需验证仿射情况是成立的,换句话讲要证明这样一个代数结论:如果$k$是代数闭域,如果$A,B$是既约$k$代数,那么$A\otimes_kB$是既约$k$代数.
		\begin{proof}
			
			首先我们解释可设$A,B$是有限生成$k$代数.假设有幂零元$\alpha\in A\otimes_kB$,可取$A$和$B$分别的有限生成$k$子代数$A'$和$B'$,使得$\alpha\in A'\otimes_kB'$.按照域上的模都是平坦模,得到$A'\otimes_kB'$可视为$A\otimes_kB$的子环,于是问题归结为$A$和$B$是有限生成$k$代数的情况.
			
			任取$\alpha=\sum_{1\le i\le n}a_i\otimes b_i$,其中$a_i\in A$和$b_i\in B$.我们可以设$\{b_1,b_2,\cdots,b_n\}$是$k$线性无关的.否则比方说$b_n=\sum_{1\le i\le n-1}c_ib_i$,其中$c_i\in k$,那么$\alpha=\sum_{1\le i\le n-1}(a_i+c_ia_n)\otimes b_i$.每次操作会把$n$变小,于是有限步后操作终止,此时$\{b_1,b_2,\cdots,b_n\}$是$k$线性无关的.
			
			现在设$A,B$是既约$k$代数,设$\alpha$是幂零元.取$A$的极大理想$m$,零点定理得到$A/m=k$.考虑环同态$A\to A/m=k$,它诱导了单同态$A\otimes_kB\to k\otimes_kB\cong B$的映射,也即$a\otimes b\mapsto\overline{a}\otimes b\mapsto\overline{a}b$.那么按照$\alpha$是幂零元,得到它的像$\sum_i\overline{a_i}b_i$是$B$中幂零元.但是按照$B$既约,得到$\sum_i\overline{a_i}b_i=0$.再按照$\{b_1,b_2,\cdots,b_n\}$是$k$线性无关的,得到$\overline{a_i}=0$.也即对每个$A$的极大理想$m$,都有每个$a_i\in m$.但是按照有限生成$k$代数上有零点定理,即它每个理想的根理想和大根相同,于是极大理想的交是幂零根,按照$A$既约得到$a_i=0,\forall i$,这说明$\alpha=0$.
		\end{proof}
	    \item 当$k$非代数闭域时这个结论不成立.例如考虑$k$是特征$p>0$的非完全域,此即存在一个元$a\in k$的$p$次根$\alpha$不落在$k$中.考虑域$k'=k[\alpha]$,那么在$k'\otimes_kk'$中有$\alpha\otimes1-1\otimes\alpha\not=0$,因为$\alpha^i\otimes\alpha^j,0\le i,j\le p-1$是张量积的一组基.但是它的$p$次幂$(\alpha\otimes1-1\otimes\alpha)^p=a\otimes1-1\otimes a=0$.这说明$k'\otimes_kk'$不是既约环.
	\end{itemize}
\end{enumerate}
\subsection{整概形}

概型$X$称为整概型,如果对每个非空开集$U$,有截面环$\mathscr{O}_X(U)$是整环.概型$X$称为局部整概型,如果每个$\mathscr{O}_{X,x}$都是整环.注意交换环的没有非平凡零因子的性质不是一个局部性质.
\begin{enumerate}
	\item 等价描述.概型$X$是整概型当且仅当它是既约的和不可约的.
	\begin{proof}
		
		必要性.整环是既约环,于是整概型是既约概型.假设$X$不是不可约的,那么可找两个非空不交的开集$U_1,U_2$,那么有$\mathscr{O}_X(U_1\cup U_2)=\mathscr{O}_X(U_1)\times\mathscr{O}_X(U_2)$.此时后者必然不是整环,这矛盾.
		
		注意.这里我们需要证明下对概型$X$,开集$U$,有$\mathscr{O}_X(U)$是零环当且仅当$U$是空集.这是因为,如果$U$不是空集,可取非空仿射开子集$V$,倘若$\mathscr{O}_X(U)$是零环,导致$1_{\mathscr{O}_X(U)}=0_{\mathscr{O}_X(U)}$,于是$\mathscr{O}_X(V)$上恒等元和零元相同,这矛盾.
		
		充分性.设$X$是既约和不可约的,需要验证对每个开集$U$有$\mathscr{O}_X(U)$是整环.但是既约和不可约这两个条件均可以传递给开子集,于是仅需验证$\mathscr{O}_X(X)$是整环.如果$f,g\in\mathscr{O}_X(X)$满足$fg=0$,那么$X_f\cap X_g=X_{fg}=\emptyset$.于是按照不可约性,不妨设$X_f=\emptyset$,我们来证明$f=0$.
		
		取$X$的仿射开覆盖$\{U_i=\mathrm{Spec}(A_i)\}$,记$f\mid_{U_i}=f_i$,那么我们解释过$X_f\cap U_i=D(f_i)$.这里$X_f$是空集得到$D(f_i)=\emptyset$,于是$f_i$落在$A_i$的幂零根中,但是$A_i$是既约环,导致$f_i=0$.于是在每个仿射开子集$U_i$上$f$的限制都是零,于是$f=0$.
	\end{proof}
	\item 整环的局部化总是整环,于是整概型总是局部整概型.这个逆命题不成立,例如我们取两个整环对应仿射整概型的无交并,这是局部整概型,但不是整概型.
	\item 仿射概型$\mathrm{Spec}(A)$是整概型当且仅当$A$是整环.因为仿射概型是既约概型当且仅当$A$是既约环,即全部极小素理想的交是零理想,在这基础上是不可约的当且仅当极小素理想唯一,于是这唯一的极小素理想只能是零理想.
	\item 概型$X$是局部整概型当且仅当$X$既约,并且不可约分支两两不交.这是因为不可约分支两两不交等价于每个点的局部环只有唯一极小素理想,而一个环是整环当且仅当它既约并且只有唯一极小素理想.
    \item 关于纤维积.
    \begin{itemize}
    	\item 设$k$是代数闭域,两个整$k$概型的纤维积是整概型,两个局部整$k$概型的纤维积是局部整概型.这个性质是仿射的,于是我们只需证明这样一个代数结论:如果$k$是代数闭域,如果$A$和$B$是$k$整环,那么$A\otimes_kB$是$k$整环.
    	\begin{proof}
    		
    		按照既约情况的证明,我们不妨设$A$和$B$都是有限生成$k$整环.假设$\alpha,\alpha'\in A\otimes_kB$使得$\alpha\alpha'=0$,按照既约情况的证明,可设$\alpha=\sum_ia_i\otimes b_i$和$\alpha'=\sum_ja_j'\otimes b_j'$,并且约定其中$\{b_1,b_2,\cdots,b_s\}$和$\{b_1',b_2',\cdots,b_t'\}$是$B$中的两组$k$线性无关组.同样的按照单射$A\to A/m=k$诱导了单同态$A\otimes_kB\to k\otimes_kB\cong B$,得到$\sum_i\overline{a_i}b_i\sum_j\overline{a_j'}b_j'=0$,于是要么$\{a_1,a_2,\cdots,a_s\}\subseteq m$,要么$\{a_1',a_2',\cdots,a_t'\}\subseteq m$.这个结论对任意极大理想$m$成立,于是$A$的极大谱满足$\mathrm{Spm}(A)=V(a_1,a_2,\cdots,a_s)\cup V(a_1',a_2',\cdots,a_t')$.按照$A$是代数闭域上的有限生成整环,它的极大谱是不可约的,可不妨设$\mathrm{Spm}(A)=V(a_1,a_2,\cdots,a_s)$,于是每个$a_i$落在大根中,但是有限生成$k$代数上大根和幂零根相同,导致$a_i=0,\forall i$,这说明$\alpha=0$.
    	\end{proof}
        \item 如果$k$不是代数闭域,这个结论未必成立:例如有$\mathbb{C}\otimes_{\mathbb{R}}\mathbb{C}=\mathbb{R}[x]/(x^2+1)\otimes_{\mathbb{R}}\mathbb{C}\cong\mathbb{C}[x]/(x^2+1)\cong\mathbb{C}[x]/(x+i)\times\mathbb{C}[x]/(x-i)\cong\mathbb{C}\times\mathbb{C}$不是整环.
    \end{itemize}
\end{enumerate}
\subsection{函数域}

如果仿射概型$\mathrm{Spec}(A)$是整概型,这等价于$A$是整环.此时概型的一般点对应于$A$的零理想,并且此时局部化$A_{(0)}$就是$A$的分式域(函数域).更一般的,如果$X$是整概型,设$\eta\in X$是它的一般点,那么局部环$\mathscr{O}_{X,\eta}$是域,它称为$X$的函数域,记作$K(X)$.设$X$是整概型,$\eta$是一般点,$K(X)$是它的函数域.
\begin{enumerate}
	\item 我们先解释下$\mathscr{O}_{X,\eta}$的确是一个域.任取仿射开子集$U=\mathrm{Spec}A$,那么$U$包含了一般点$\eta$,所以$\eta$在$U$里是$A$的极小素理想,但是$A$是整环,导致$\mathscr{O}_{X,\eta}=A_{(0)}$就是$A$的商域,所以是域.
	\item 如果$U=\mathrm{Spec}(A)$是$X$的非空仿射开集,那么$K(X)=\mathrm{Frac}(A)$.如果$x\in X$,那么$\mathrm{Frac}(\mathscr{O}_{X,x})=K(X)$.如果$U$不是仿射开集,那么$\mathscr{O}_X(U)$的商域未必是函数域,比方说域$k$上射影空间$\mathbb{P}_k^n$上的整体截面环是$k$本身,但是函数域是$k(T_1,\cdots,T_n)$.
	\begin{proof}
		
		任取$x$的仿射开邻域$U$,那么$\eta\in U$,并且对应于整环$A$的零理想,并且$\mathscr{O}_{X,x}=A_{p_x}$.于是$K(X)=\mathscr{O}_{U,\eta}=\mathrm{Frac}(A)=\mathrm{Frac}(A_{p_x})$.
	\end{proof}
	\item 设$U\subseteq V$是整概型$X$的非空开集,那么典范映射$\mathscr{O}_X(V)\to\mathscr{O}_X(U)\to K(X)$都是单射.
	\begin{proof}
		
		我们只需说明后者对任意非空开集$U$总是单射,然后按照这个映射的复合就是$\mathscr{O}_X(U)\to K(X)$的典范映射,也是单的,就说明限制映射总是单射.
		
		任取非空开集$U$和$f\in\mathscr{O}_X(U)$,假设$f_{\eta}=0$,需要验证$f=0$.任取仿射开子集$V=\mathrm{Spec}(A)$,那么$A\to K(X)=\mathrm{Frac}(A)$自然是单射,于是从$(f\mid_V)_{\eta}=0$得到$f\mid_V=0$.于是层公理得到$f=0$.
	\end{proof}
    \item 于是整概型上截面的粘合非常简单:取$f_i$是开覆盖$U_i$上的函数族,那么这个函数族可以粘合到整个$U=\cup_iU_i$当且仅当$f_i$视为$K(X)$上的元是相同的.换句话讲对$X$的每个非空开子集$U$,任取开覆盖$U=\cup_iU_i$,那么总有$\mathscr{O}_X(U)=\cap_i\mathscr{O}_X(U_i)=\cap_{x\in U}\mathscr{O}_{X,x}$.
    \begin{proof}
    	
    	我们来证明最后一个等式$\mathscr{O}_X(U)=\cap_{p\in U}\mathscr{O}_{X,p}$.在整概形上有典范单射$\mathscr{O}_X(V)\to\mathscr{O}_X(U)\to\mathscr{O}_{X,p}\to K(X)$.所以截面局部环中的元素都可以视为函数域$K(X)$中的元.于是有$\mathscr{O}_X(U)\subseteq\cap_{p\in U}\mathscr{O}_{X,p}$.反过来任取$s\in\cap_{p\in U}\mathscr{O}_{X,p}$,对每个点$p\in U$,课取开邻域$p\in U_p\subseteq U$,使得$s$可视为$\mathscr{O}_X(U_p)$中的元,这样$\{s\in\mathscr{O}_X(U_p),p\in U\}$可粘合为元素$s\in\mathscr{O}_X(U)$.
    \end{proof}
\end{enumerate}
\subsection{局部诺特概形和诺特概形}

概型$X$称为局部诺特概型,如果它可被一族诺特环对应的仿射开集覆盖.概型$X$称为诺特概型,如果它可被有限个诺特环对应的仿射开集覆盖.
\begin{enumerate}
	\item 诺特条件满足仿射交换引理中的两个条件:如果$A$是诺特环,那么每个局部化$A_f,f\in A$是诺特环;如果$(f_1,f_2,\cdots,f_n)=A$,并且每个$A_{f_i}$是诺特环,那么$A$是诺特环.于是按照仿射交换引理,局部诺特条件是仿射局部的,于是有:概型是局部诺特概型当且仅当它的每个仿射开子集对应的环都是诺特环;仿射概型是局部诺特概型当且仅当它对应的环是诺特环;局部诺特概型的开子概型是局部诺特的.
	\begin{proof}
		
		我们只需验证第二个条件.注意条件$(f_1,f_2,\cdots,f_n)=A$等价于讲$\cup_{1\le i\le n}D(f_i)=\mathrm{Spec}A$.问题归结为这个条件下如果每个$A_{f_i}$都是诺特环,证明$A$是诺特环.
		
		任取$A$的理想$I$,记$IA_{f_i}$作为$A_{f_i}$的理想的生成元集可取$\{\frac{a_{ij}}{f_i^{r_{ij}}},j\in J_i\}$.我们断言$I$可被有限个元$\{a_{ij},j\in J_i,1\le i\le n\}$生成:任取$a\in I$,对每个$1\le i\le n$都有$\frac{a}{1}=\sum_{j\in J_i}\frac{b_{ij}}{f_i^{e_{ij}}}\cdot\frac{a_{ij}}{f_i^{r_{ij}}}$.于是存在足够大的正整数$k_i$使得$f_i^{k_i}a=\sum_{j\in J_i}c_{ij}a_{ij}$.另外按照$D(f_i)=D(f_i^{k_i})$,于是全部$D(f_i^{k_i})$也是$X$的开覆盖.于是$V(\sum_i(f_i^{k_i}))=\emptyset$,于是存在系数使得$1=\sum_ib_if_i^{k_i}$.于是$a=\sum_{i,j}b_ic_{ij}a_{ij}$.于是$A$是诺特的.
	\end{proof}
	\item 诺特概型等价于拟紧的局部诺特概型.于是特别的仿射概型是诺特的等价于是局部诺特的,也等价于它对应的环是诺特环.另外诺特概型是诺特空间,因为诺特环对应的素谱是诺特空间,而有限个诺特空间的并是诺特的.最后按照诺特空间的所有子集都是拟紧的,这说明诺特概型的开子概型总是诺特的.
	\item 但是注意诺特概型只是说仿射开子集上的截面环是诺特环,存在这样的诺特概型,它的整体截面环不是诺特环.
	\item 诺特概型上每个点的stalk都是诺特环.这个逆命题即便对于仿射概型都不成立:即环的诺特性不是局部性质,存在这样的非诺特环,它每个素理想处的局部化都是诺特环,例如设$k$是代数闭域,设$(a_i)_{i\ge0}$是两两不同的$k$中的元,取:$$A=k[U,T_1,T_2,\cdots]/((U-a_i)T_{i+1}-T_i,T_i^2)$$
	\item 局部诺特概形的纤维积未必还是局部诺特的,事实上域扩张就有反例了,如果$F/K$是不能有限生成的域扩张,那么$F\otimes_KF$不是诺特的.
	\item 局部诺特概型是拟分离的.这是因为局部诺特概型的仿射开子集是诺特仿射概型,它的任意开子集都是拟紧的.于是诺特概形总是拟紧拟分离的.
	\item 按照诺特概型是诺特空间,其上的不可约分支只有有限个,按照不可约空间都是连通的,说明连通分支是若干不可约分支的并,于是诺特概型上的连通分支个数也是有限的,并且每个连通分支是有限个不可约分支的并.
	\item 诺特归纳(Noetherian Induction).设$X$是一个诺特拓扑空间,设$P$是关于$X$闭子集的一个性质.对任意闭子集$Y$,如果$Y$的所有真闭子集满足$P$能推出$Y$满足$P$,那么$X$满足$P$.
\end{enumerate}
\subsection{有限型概形}

设$A$是环,一个$A$概型$X$称为局部有限型$A$概型,如果存在$X$的仿射开覆盖$\{\mathrm{Spec}B_i\}$,使得每个$B_i$经典范映射诱导的结构映射都是有限生成$A$代数.如果这个仿射开覆盖可取为有限开覆盖,就称$X$是有限型$A$概型.
\begin{enumerate}
	\item 有限型条件满足仿射交换引理中的两个条件:如果$B$是有限型$A$代数,那么每个局部化$B_f,f\in B$是有限型$A$代数;如果$(f_1,f_2,\cdots,f_n)=B$,并且每个$B_{f_i}$是有限型$A$代数,那么$B$是有限型$A$代数.于是按照仿射交换引理,局部有限型条件是仿射局部的,于是有:$A$概型是局部有限型的当且仅当它的每个仿射开子集对应的环都是有限型$A$代数;仿射$A$概型是局部有限型的当且仅当它对应的环是有限型$A$代数;局部有限型$A$概型的开子概型是局部有限型$A$概型.
	\begin{proof}
		
		只需验证第二个条件.这里条件$(f_1,f_2,\cdots,f_n)=B$等价于讲$\cup_{1\le i\le n}D(f_i)=\mathrm{Spec}B$.记$B_{f_i}$作为$A$代数可由$\{\frac{b_{ij}}{f_i^{r_j}}\mid 1\le j\le K_i\}$生成.我们断言$B$作为$A$代数可由$\{a_{ij}\mid 1\le i\le n,1\le j\le K_i\}$生成.任取$b\in B$,对每个指标$i$,在$B_{f_i}$中有$b/1=\sum_{t_1,\cdots,t_{k_i}}a_{t_1,\cdots,t_{k_i}}^i\left(\frac{b_{i1}}{f_i^{r_1}}\right)^{t_1}\left(\frac{b_{i2}}{f_i^{r_2}}\right)^{t_2}\cdots\left(\frac{b_{i,k_i}}{f_i^{r_{k_i}}}\right)^{t_n}$.于是存在足够大的$n_i$和$A$系数多元多项式$P_i$使得$f_i^{n_i}b=P_i(b_{i1},b_{i2},\cdots,b_{i,k_i})$.再按照$U=\cup_{1\le i\le n}D(f_i)=\cup_{1\le i\le n}D(f_i^{n_i})$.得到$1=\sum_{1\le i\le n}a_if_i^{n_i}$.于是得到$b=\sum_{1\le i\le n}a_if_i^{n_i}b=\sum_{1\le i\le n}a_iP_i(b_{i1},b_{i2},\cdots,b_{i,k_i})$.这说明$b$由$\{b_{ij}\}$生成.
	\end{proof}
    \item 有限型$A$概型等价于拟紧的局部有限型$A$概型.例如如果$X$是拟射影$A$概型,它是有限个有限型$A$代数的仿射概型的并,于是$X$是有限型$A$概型.如果额外的有$A$是诺特环,那么此时$X$是诺特概型.
\end{enumerate}
\subsection{正规概形}

一个概型称为正规概型,如果每个点处的局部环都是正规整环,即它是整环,并且在商域中整闭.
\begin{enumerate}
	\item 正规概形是局部整概形,于是特别的它的不可约分支两两不交.但是它未必是整概形(例如考虑$k$是域,取两个$\mathrm{Spec}k$的无交并,它的整体截面环是$k\times k$不是整环,但是这是一个正规概型),所以它开子集的截面未必是整环,一个正规概形是整概形当且仅当它是不可约的.
	\item 正规整环对应的仿射概型是正规概型,但是反过来如果一个仿射概形是正规整环,它未必是被正规整环诱导的.不过一个仿射概形是正规整概形当且仅当它被正规整环诱导.这件事用交换代数的语言说就是:如果$A$是正规整环,那么它的每个局部化$A_p$是正规整环,反过来如果$A$的每个局部化都是正规整环,那么$A$未必是整环.但是如果$A$是整环,并且每个局部环$A_p$是正规整环,那么$A$是正规整环.
	\item 如果概形$X$存在一个仿射开覆盖$X=\cup_iU_i$,使得每个$\mathscr{O}_X(U_i)$都是正规整环,那么$X$是正规概形.如果$X$是整概形,那么$X$是正规概形当且仅当对每个开集$U$有$\mathscr{O}_X(U)$是正规整环.
	\item 拟紧概型上如果闭点处的局部环都是正规整环,那么它是正规概型.
	\item 设$X$是局部诺特概形,如果$U\subseteq X$是连通开集,那么$\mathscr{O}_X(U)$是正规整环.这件事是因为局部诺特概形上连通分支分解和不可约分解是一致的,所以连通开集$U$也是不可约的,所以开子概型$U$是正规整概形,于是整体截面是正规整环.
	\item Serre准则.一个局部诺特概形是正规的当且仅当对$\dim\mathscr{O}_{X,x}=1$的点$x$都是正则点,对$\dim\mathscr{O}_{X,x}=2$的点有$\mathscr{O}_{X,x}$是CM局部环.
	\item 例子.设$X=\mathrm{Spec}A$是正则的,考虑它的整闭子概型$Z=V(f)=\mathrm{Spec}A/(f)$(整就要求了$A/(f)$是整环,也即$(f)$是素理想),那么$Z$是正规的当且仅当$Z$是余维数1正则的,此即对任意$x\in X$满足$\dim\mathscr{O}_{X,x}=1$,都有$x$是正则点,它还有个名字是Serre条件$(R_1)$.
	\begin{proof}
		
		我们有Serre的正规性准则,即正规性等价于$(R_1)+(S_2)$.所以我们只需验证$A/(f)$满足$(S_2)$,而这是因为$f$是正则元,导致$A/(f)$是CM环,于是它满足所有$(S_n)$.
	\end{proof}
	\item 设$X$是域上的有限型概形,那么全体正规点构成的正规中心是一个开集.【】
	\item 称一个维数$\le1$的局部诺特正规概形为戴德金概形(戴德金整环是维数$\le1$的诺特正规整环).那么对于诺特整概形$X$,它是戴德金概形当且仅当对每个开集$U$有$\mathscr{O}_X(U)$是戴德金整环.特别的,戴德金整环对应的仿射概形是戴德金概形.
	\item Hartogs定理(这个定理在复几何中有对应).设$X$是局部诺特正规整概形,设$U\subseteq X$是开子集,满足$\mathrm{codim}_X(X-U)\ge2$,那么限制映射$\mathscr{O}_X(X)\to\mathscr{O}_X(U)$是同构,换句话讲,每个$f\in\mathscr{O}_X(U)$都可以唯一延拓为$\mathscr{O}_X(X)$中的元.
	\begin{proof}
		
		归结为设$X=\mathrm{Spec}A$是仿射的,这里$A$是诺特正规整环,因为$X-U$的余维数$\ge2$,所以高度1的素理想必须都在$U$中.于是对每个高度1的素理想$\mathfrak{p}$,有$\mathscr{O}_X(U)\subseteq A_{\mathfrak{p}}$,由于$A$是诺特正规整环,让$\mathfrak{p}$跑遍高度1的素理想时有$A=\cap_{\mathfrak{p}}A_{\mathfrak{p}}$,这得到$\mathscr{O}_X(U)\subseteq A=\mathscr{O}_X(X)$,但是另一侧的包含关系是因为$X$是整概形,这得到它们相同.
	\end{proof}
    \item 推论.如果$X$是局部诺特正规整概形,如果$\mathscr{F}$是$X$上的局部自由模层,那么$\mathscr{F}$仍然满足Hartogs定理,也即如果开子集$U\subseteq X$满足$X-U$的余维数$\ge2$,那么限制映射$\mathscr{F}(X)\to\mathscr{F}(U)$是同构.
    \begin{proof}
    	
    	上一条已经证明了$\mathscr{F}$是自由模层的时候命题成立.现在设$\mathscr{F}$是局部自由模层,那么存在$X$的开覆盖$\{V_i\}$,使得每个$\mathscr{F}\mid_{V_i}$是$\mathscr{O}_{V_i}$的自由模层.由于$V_i\cap U$在$V_i$中的补集仍然是余维数$\ge2$的,并且$V_i$仍然是局部诺特正规整概形,于是任取$a\in\mathscr{F}(U)$,记$a_i=a\mid_{U\cap V_i}$,那么$a_i$可以唯一延拓为一个元$b_i\in\mathscr{F}(V_i)$.任取指标$i,j$,由于$\mathscr{F}$限制在$V_i\cap V_j$上是自由模层,并且$X$是整概形,于是从$b_i\mid_{V_i\cap V_j\cap U}=b_j\mid_{V_i\cap V_j\cap U}$就得到$b_i\mid_{V_i\cap V_j}=b_j\mid_{V_i\cap V_j}$.于是这些$\{b_i\in\mathscr{F}(V_i)\}$可以唯一的粘合为一个元素$b\in\mathscr{F}(X)$,它是$a$在整个$X$上的唯一延拓.
    \end{proof}
    \item 例如取$X=\mathrm{Spec}k[x,y]$,其中$k$是域,取$F=\{(0,0)\}$(这里$(0,0)$指的是极大理想$(x-0,y-0)$),取补集为$U$,那么$F$的余维数是2,于是有$\mathscr{O}_X(U)=\mathscr{O}_X(X)=k[x,y]$.
\end{enumerate}
\subsection{正则概形}

设$X$是局部诺特概形,称$x\in X$是正则点,如果$\mathscr{O}_{X,x}$是正则局部环.这也等价于讲$\mathscr{O}_{X,x}$的嵌入维数,此为$\mathfrak{m}_x/\mathfrak{m}_x^2$作为$\kappa(x)=\mathscr{O}_{X,x}/\mathfrak{m}_x$线性空间的维数,也即点$x$的切空间维数,和$\mathscr{O}_{X,x}$的维数相同.不是正则点的点称为奇点.称$X$是正则概形,如果它的所有点都是正则点.
\begin{enumerate}
	\item 例如一维正规局部诺特概形是正则概形,因为DVR总是正则局部环.再例如如果$A$是诺特环,那么$\mathrm{Spec}A$是正则概形当且仅当$A$是正则环(它的定义就是每个素理想处局部化都是正则局部环的诺特环).
	\item 一个正则局部环总是唯一分解整环,于是特别的正则概形总是局部分解概形(即局部环都是唯一分解整环),也总是正规概形.
	\item 例子.考虑$\mathbb{A}_{\mathbb{Z}}^d=\mathrm{Spec}A=\mathrm{Spec}\mathbb{Z}[T_1,\cdots,T_d]$,它的闭点$\mathfrak{m}=(p,T_1,\cdots,T_d)$是正则点,因为$A_{\mathfrak{m}}$的维数是$d+1$,它的极大理想可以被$d+1$个元生成.取非零多项式$F(T_1,\cdots,T_d)\in\mathfrak{m}$,那么$\mathrm{Spec}\mathbb{Z}[T_1,\cdots,T_d]/(F)$还在$\mathfrak{m}$处正则当且仅当$F\not\in(p,T_1,\cdots,T_d)^2$.一般的,如果$(A,\mathfrak{m})$是正则局部环,如果$f\in\mathfrak{m}-\{0\}$,那么$A/fA$是正则局部环当且仅当$f\not\in\mathfrak{m}^2$.更一般的,如果$(A,\mathfrak{m})$是正则局部环,设$I\subseteq A$是真理想,那么$A/I$是正则局部环当且仅当$I$被某组参数系统的一部分生成,如果这组$I$的生成元的个数是$r$,那么有$r=\dim A-\dim A/I$.
	\item 设$X$是诺特概形(于是拟紧),那么$X$是正则概形当且仅当它的所有闭点是正则点.
	\item Jacobian准则.设$X$是域$k$上的有限型仿射概形,即$X=\mathrm{Spec}k[T_1,\cdots,T_n]/(F_1,\cdots,F_r)$,对$k$有理点$x\in X(k)$,我们记它的Jacobian矩阵为$J_x=\left(\frac{\partial F_i}{\partial T_j}(x)\right)_{1\le i\le r,1\le j\le n}$.那么$X$在点$x$正则当且仅当$\mathrm{rank}J_x=n-\dim\mathscr{O}_{X,x}$.
	\begin{proof}
		
		记$Y=\mathbb{A}_k^n$,记$X=V(I)$,其中$I=(F_1,\cdots,F_r)$.我们解释过$x$在$Y$中的切空间等同于$k^n$,而$\mathrm{T}_xX$作为$k^n$的子空间是雅各比矩阵的核.于是$\dim\mathrm{T}_xX=n-\mathrm{rank}J_x$.于是$x$是正则点当且仅当$\dim\mathscr{O}_{X,x}=n-\mathrm{rank}J_x$.
	\end{proof}
    \item 设$X$是域$k$上的几何既约局部有限型概形,那么$X$有正则闭点.特别的,代数闭域上的既约局部有限型概形有正则闭点.
    \begin{proof}
    	
    	我们知道局部有限型$k$概形上开集的闭点一定是全空间的闭点.所以问题归结为设$X=\mathrm{Spec}A$是仿射和整的,并且$A$是有限型$k$代数.几何既约性保证函数域$K(X)$可以分解为$k\subseteq k(T_1,\cdots,T_d)\subseteq K(X)$,前者是$\dim X$维纯超越扩张,后者是有限可分扩张.于是可记$K(X)=k(T_1,\cdots,T_n)[f]$,其中$f\in A$.设它在$k(T_1,\cdots,T_d)$上的极小多项式是$P(S)$,可约定$P(S)$的系数都落在$k[t_1,\cdots,T_d]$中.适当把$A$缩小到主开集上可以要求$T_1,\cdots,T_d\in A$(比方说$T_1=a_1/a_2$,其中$a_1,a_2\in A$,那么$T_1\in\Gamma(D(a_2),\mathscr{O})$).因为$A$是有限生成$k$代数,可以找到一个$R(T)\in k[T_1,\cdots,T_d]$使得$A\subseteq k[T_1,\cdots,T_d,R^{-1}][f]$(比方说设$A=k[x_1,\cdots,x_n]$,每个$x_i$可以表示为$k(T_1,\cdots,T_d)$系数的$f$多项式,把系数的分母都乘起来作为$R$,那么每个$x_i\in k[T_1,\cdots,T_d,R^{-1}][f]$).于是再对$A$做一次局部化,可以约定$A=k[T_1,\cdots,T_d,R^{-1}][f]$.于是$\mathrm{Spec}A$是$\mathrm{Spec}B$的主开集$D(R)$,这里$B=k[T_1,\cdots,T_d,S]/(P(S))$.记$\delta(S)$是$P(S)$和$P'(S)$的结式,这是$k[T_1,\cdots,T_d]$中的元,按照$P$可分知$\delta(T)$不为零.
    	
    	\qquad
    	
    	设$\mathfrak{m}_x$是$k[T_1,\cdots,T_d]$的不包含$R(T)\delta(T)$的极大理想.设$\mathfrak{m}_y$是$k[T_1,\cdots,T_d,S]$的包含$P(S)+\mathfrak{m}_x$的极大理想.那么$\mathfrak{m}_y$对应于一个点$y\in\mathrm{Spec}A\subseteq\mathrm{Spec}B=V(P)\subseteq\mathbb{A}_k^{d+1}$.我们断言$B$在$y$处正则:存在$Q(S)\in k[T_1,\cdots,T_d][S]$使得$\mathfrak{m}_y=\mathfrak{m}_x+(Q(S))$,如果设$S$在$\kappa(y)$中的像是$s$,那么就有$\kappa(y)=\kappa(x)[s]$.【】
    \end{proof}
    \item 设$X$是域上的局部有限型概形,那么正则点构成了$X$的开子集.另外如果$X$还是正规的,把奇点集记作$F$,那么$\mathrm{codim}_XF\ge2$,于是特别的Hartogs定理告诉我们正则点集上的正则函数都可以唯一延拓到所有奇点上.
    \begin{proof}
    	
    	条件下正则点构成开集见交换代数中的Nagata准则.下面额外要求$X$是正规的,如果$\xi\in X$是余维数1的点,也即$\mathrm{codim}_X\overline{\{\xi\}}=1$,那么$\mathscr{O}_{X,\xi}$是1维正规诺特局部环,它是DVR,所以是正则的.这说明余维数是1的点都是正则点.但是如果$F$的余维数是1,因为它是闭集,导致它的不可约分支的一般点是余维数1的,但是这理应是正则点,矛盾.
    \end{proof}
    \item 设$X$是$k$概形,设$x\in X$是正则点,那么$\mathscr{O}_{X,x}$关于$\mathfrak{m}_x$的完备化满足有$k$代数同构$\widetilde{\mathscr{O}_{X,x}}\cong k[[T_1,\cdots,T_d]]$,其中$d=\dim\mathscr{O}_{X,x}$.这件事是因为$X$是$k$概形保证了$A=\mathscr{O}_{X,x}$是等特征局部环,它的完备化就是等特征的完备正则局部环,必然是形式幂级数环.
\end{enumerate}
\subsection{Cohen-Macaulay概形}

设$X$是局部诺特概形,称一个点$x\in X$是Cohen-Macaulay点,简称CM点,如果局部环$\mathscr{O}_{X,x}$是CM局部环.如果局部诺特概形$X$的每个都是CM点,则称它是CM概形.
\begin{enumerate}
	\item 正则环一定是CM环,所以正则概形一定是CM概形.关于这些性质更精细的描述见Serre条件.
	\item CM概形一定是universally catenary概形.
	\item CM概形的嵌入点只有一般点.这是因为CM环的伴随素理想只有极小素理想.
	\item 如果CM概形$X$是连通的,那么它是余维数2连通的,换句话讲如果$Z\subseteq X$是余维数$\ge2$的闭子概型,那么$X-Z$是连通的.
	\begin{proof}
		
		设$X-Z$可以表示为两个非空的开闭子集$x_1,X_2$的无交并.【】
	\end{proof}
	\item 如果$X$是域$k$上的局部有限型概形,并且是CM概形,那么$X$是等维数的.【】
\end{enumerate}
\subsection{Serre条件}

设$X$是局部诺特概形,设$k\ge0$是一个自然数.
\begin{itemize}
	\item 称$X$满足条件$(R_k)$,如果只要$\dim\mathscr{O}_{X,x}\le k$,就有$x$是正则点.
	\item 称$X$满足条件$(S_k)$,如果只要$\mathrm{depth}\mathscr{O}_{X,x}<k$,那么$x$是CM点.条件$(S_0)$是平凡成立的.
\end{itemize}
\begin{enumerate}
	\item 局部诺特概形$X$是CM概形当且仅当它满足所有$(S_k),k\ge0$.
	\item 局部诺特概形$X$是正则概形当且仅当它满足所有$(R_k),k\ge0$.
	\item 局部诺特概形$X$满足$(S_1)$当且仅当它的伴随点只有一般点.
	\item 局部诺特概形$X$是既约概形当且仅当它满足$(R_0)$和$(S_1)$.
	\item 局部诺特概形$X$是正规概形当且仅当它满足$(R_1)$和$(S_2)$.
\end{enumerate}
\newpage
\section{态射}
\subsection{仿射概形之间的态射}

设$\varphi:A\to B$是环同态,设诱导的仿射概形之间的态射为$f:Y=\mathrm{Spec}B\to X=\mathrm{Spec}A$.
\begin{enumerate}
	\item 设$\mathfrak{p}\in\mathrm{Spec}A$,那么$f^{\#}_{\mathfrak{p}}$就是$A_{\mathfrak{p}}\to B_{\mathfrak{p}}=B\otimes_AA_{\mathfrak{p}}$.
	\begin{proof}
		\begin{align*}
			(f_*\mathscr{O}_Y)_{\mathfrak{p}}&=\lim\limits_{\substack{\rightarrow\\p\in V}}\mathscr{O}_Y(f^{-1}(V))\\&=\lim\limits_{\substack{\rightarrow\\p\in D(a)}}\mathscr{O}_Y(f^{-1}(D(a)))\\&=\lim\limits_{\substack{\rightarrow\\p\in D(a)}}B_{f(a)}\\&=B_{\mathfrak{p}}=B\otimes_AA_{\mathfrak{p}}
		\end{align*}
	\end{proof}
	\item $\varphi$是单射当且仅当层态射$f^{\#}:\mathscr{O}_X\to f_*\mathscr{O}_Y$是单态射.在这个条件成立时$f$是支配的,即$f(Y)$在$X$中稠密.
	\begin{proof}
		
		一方面如果$f^{\#}$是单态射,取整体截面得到$\varphi$是单射.另一方面如果$\varphi$是单射,取$\mathfrak{p}\in\mathrm{Spec}A$,那么$f^{\#}_{\mathfrak{p}}:A_{\mathfrak{p}}\to(f_*\mathscr{O}_Y)_{\mathfrak{p}}=B_{\mathfrak{p}}=B\otimes_AA_{\mathfrak{p}}$.于是从$A_{\mathfrak{p}}\to B_{\mathfrak{p}}$是单射就得到$f^{\#}_{\mathfrak{p}}$是单射,最后$\mathfrak{p}$是任意的,就得到$f^{\#}$是单态射.另外我们解释过$f(Y)$在$X$中稠密当且仅当$\ker\varphi$在$A$的幂零根中.
	\end{proof}
    \item $\varphi$是满射当且仅当$f$是$Y$到$X$的一个闭子集的同胚,并且$f^{\#}:\mathscr{O}_X\to f_*\mathscr{O}_Y$是满态射.
    \begin{proof}
    	
    	必要性,此时存在理想$I$使得$B\cong A/I$.我们解释过此时$f$作为连续映射是到闭子集的同胚.而$f^{\#}$的茎$f^{\#}_{\mathfrak{p}}:A_{\mathfrak{p}}\to A/I\otimes_AA_{\mathfrak{p}}$明显是满射.充分性,如果$f^{\#}$是满态射,那么每个$f^{\#}_{\mathfrak{p}}$都是满同态.于是对每个$b\in B$,对每个$A$的素理想$\mathfrak{p}_i$,可以找到主开集$D(f_i)$,使得$b$在$\mathfrak{p}_i$的stalk是某个$a_i/f_i^{n_i}$.也即有$f_i^{m_i}(a_i-f_i^{n_i}b)=0$.可以取$\{D(f_i)\}$的有限子覆盖$D(f_1),\cdots,D(f_r)$,那么这里$n_i$和$m_i$分别可以替换为一个足够大的正整数$n,m$,并且$f_i^{m+n}$生成了整个$A$,于是有$1=\sum g_if_i^{n+m}$,进而有$b=\sum g_if_i^{n+1}b=\sum g_if_i^ma_i$,这在$\varphi$的像中,于是$\varphi$是满射.
    \end{proof}
    \item 概型之间的态射不会被它在底空间之间的映射所决定,而是被底空间之间的映射和层态射同时决定.例如$\mathbb{C}$上的恒等映射和共轭映射诱导了不同的仿射概型之间的态射$\mathrm{Spec}\mathbb{C}\to\mathrm{Spec}\mathbb{C}$,但是这两个态射的底空间上的映射是相同的,因为$\mathrm{Spec}\mathbb{C}$是单点集合.
\end{enumerate}
\subsection{态射的粘合}
\begin{enumerate}
	\item 开子概型的自然包含态射.给定概型$(X,\mathscr{O}_X)$的开子概型$U\subseteq X$.我们约定包含态射$(f,f^{\#}):(U,\mathscr{O}_X\mid_U)\to(X,\mathscr{O}_X)$为,$f$就是$U\to X$的包含映射,$f^{\#}$在开集$V\subseteq X$上为层$\mathscr{O}_X$上的限制映射$\mathscr{O}_X(V)\to \mathscr{O}_X(V\cap U)$.
	\item 开嵌入.开嵌入就是到开子概型的同构,具体的讲,态射$(f,f^{\#}):(X,\mathscr{O}_X)\to(Y,\mathscr{O}_Y)$称为开嵌入,如果$f$是$X$到$\mathrm{im}f$的同胚,并且$\mathrm{im}f$是$Y$的开子集,并且$f^{\#}$在$\mathrm{im}f$上的限制是层同构.
	\item 态射的粘合.给定概型$(X,\mathscr{O}_X)$,设一族开子概型$(U_i,\mathscr{O}_X\mid_{U_i}),i\in I$满足$X=\cup_iU_i$.如果对每个$i\in I$存在态射$(f_i,f_i^{\#}):(U_i,\mathscr{O}_X\mid_{U_i})$满足在$U_i\cap U_j$上恒有$(f_i,f_i^{\#})=(f_j,f_j^{\#}),\forall i,j\in I$,那么存在唯一的态射$(f,f^{\#}):(X,\mathscr{O}_X)\to(Y,\mathscr{O}_Y)$满足在每个$U_i$上恒有$(f,f^{\#})=(f_i,f_i^{\#})$.
	\begin{proof}
		
		首先连续映射$f_i:U_i\to Y$粘合为连续映射$f:X\to Y$是平凡的.下面定义粘合的$f^{\#}$.对$Y$中每个开集$V$,$f^{\#}$要把$\mathscr{O}_Y(V)$中的$s$映射为$\mathscr{O}_X(f^{-1}(V))$中的元.我们记$s_i=f_i^{\#}(s)\in\mathscr{O}_X(f^{-1}(V)\cap U_i)$.按照条件$U_i\cap U_j$上恒有$f_i^{\#}=f_j^{\#}$,说明$s_i$和$s_j$限制在$f^{-1}(V)\cap U_i\cap U_j$上是相同的.再结合$\cup_i(f^{-1}(V)\cap U_i)=f^{-1}(V)$,按照层公理就存在唯一的$f^{\#}(s)\in\mathscr{O}_X(f^{-1}(V))$满足$f^{\#}(s)$在$U_i$上的限制是$s_i$.容易验证$f^{\#}(V)$是环同态,下面我们来说明它是自然变换,从而是层态射.任取开集$U\subseteq V$:
		$$\xymatrix{\mathscr{O}_{Y}(V)\ar[rrrr]^{f^{\#}(V)}\ar[drr]\ar[ddd]_{\mathrm{res}_{V,U}}&&&&\mathscr{O}_X(f^{-1}(V))\ar[ddd]^{\mathrm{res}_{f^{-1}V,f^{-1}U}}\\&&\left((f_i)_*\mathscr{O}_X\right)_Y(V)=\mathscr{O}_X(f^{-1}(V)\cap U_i)\ar[d]\ar[urr]&&\\&&\left((f_i)_*\mathscr{O}_X\right)_Y(U)=\mathscr{O}_X(f^{-1}(U)\cap U_i\ar[drr]&&\\\mathscr{O}_Y(U)\ar[urr]\ar[rrrr]_{f^{\#}(U)}&&&&\mathscr{O}_X(f^{-1}(U))}$$
		
		其中上下小三角形的交换性是因为$f_i^{\#}$粘合为$f^{\#}$的定义,左侧小梯形交换是因为每个$f_i^{\#}$是层态射(自然变换),右侧小梯形交换是因为限制映射的复合只取决于初始和终端的开集.最后要验证粘合的唯一性,这是因为对每个开集$V\subseteq Y$,$f^{\#}(V)(s)$必然要满足在$U_i$上的限制是$s_i$,这从层公理中粘合的唯一性就说明$f^{\#}(V)$是唯一的.
	\end{proof}
    \item 我们可以换一种方式描述态射的粘合.如果$X,Y$是概形(或者局部环空间),对$X$的开子集$U$,取$U\mapsto\mathrm{Hom}(U,Y)$,那么这是$X$上的一个取值在$\textbf{Sets}$的层.层的两条公理就是这里态射粘合的存在性和唯一性.
\end{enumerate}
\subsection{终端为仿射概形的态射}
\begin{enumerate}
	\item 设$X$是概型,$Y$是仿射概型,那么存在如下态射集之间的双射:
	$$\mathrm{Hom}_{\textbf{Ring}}(\mathscr{O}_Y(Y),\mathscr{O}_X(X))\cong\mathrm{Hom}_{\textbf{Sch}}(X,Y)$$
	\begin{proof}
		
		设$X=\cup_iU_i$,其中每个$U_i$是仿射开子集.按照函子性,得到如下交换图,这里$\rho$是把$X\to Y$的态射$(f,f^{\#})$映射为交换环同态$f^{\#}(Y):\mathscr{O}_Y(Y)\to\mathscr{O}_X(X)$.
		$$\xymatrix{\mathrm{Hom}(X,Y)\ar[rr]^{\rho}\ar[d]_{\alpha}&&\mathrm{Hom}_{\textbf{Ring}}(\mathscr{O}_Y(Y),\mathscr{O}_X(X))\ar[d]^{\beta}\\\prod_i\mathrm{Hom}(U_i,Y)\ar[rr]^{\gamma}&&\prod_i\mathrm{Hom}_{\textbf{Ring}}(\mathscr{O}_Y(Y),\mathscr{O}_X(U_i))}$$
		
		按照我们之前给出的态射粘合的唯一性,说明$\alpha$是单射.按照之前给出的$X$是仿射概型时$\mathrm{Hom}$集合是双射,得到$\gamma$是双射.于是图表交换性得到$\rho$是单射.下面验证$\rho$的满射性,任取同态$\varphi\in\mathrm{Hom}_{\textbf{Ring}}(\mathscr{O}_Y(Y),\mathscr{O}_X(X))$,记它和限制映射$\mathscr{O}_X(X)\to\mathscr{O}_X(U_i)$的复合是$\varphi_i$.于是$\beta(\varphi)=(\varphi_i)$.按照之前证明的$\gamma$是双射,于是对每个$i$存在态射$f_i:U_i\to Y$使得$\gamma((f_i))=(\varphi_i)$.现在我们仅需说明$f_i\mid_{U_i\cap U_j}=f_j\mid_{U_i\cap U_j}$对任意$i,j$成立,就可以按照之前态射的粘合引理得到存在$f\in\mathrm{Hom}(X,Y)$使得$\alpha(f)=(f_i)$,于是$\beta$是单射就说明$\rho(f)=\varphi$完成满射的证明.
		
		为此任取仿射开集$V\subseteq U_i\cap U_j$,按照限制映射的复合只取决于初始和终端的开集,说明$f_i$和$f_j$在$\mathrm{Hom}(\mathscr{O}_Y(Y),\mathscr{O}_X(V))$中的像相同,结合仿射情况态射集是双射,就得到$f_i\mid_V=f_j\mid_V$,再结合仿射开子集是$U_i\cap U_j$的拓扑基,就说明了$f_i\mid_{U_i\cap U_j}=f_j\mid_{U_i\cap U_j}$.
	\end{proof}
    \item 这个定理说明$\mathrm{Spec}\mathbb{Z}$是概型范畴上的终对象.另外关于初对象,我们考虑的环同态要把乘法幺元映射为乘法幺元,按照定义一个零环只有一个元素,它同时是零元素和乘法幺元.于是任意一个环都存在唯一的到零环的环同态.零环的素谱是一个空集,按照定义空集为源端的映射是唯一的,这说明零环的素谱是概形上的初对象.
    \item 这个同构实际上说明了$\Gamma(-)$和$\mathrm{Spec}(-)$作为局部环空间范畴和环空间范畴之间的两个逆变函子是互相伴随性.于是$\mathrm{Spec}$把$\textbf{CRing}$上的正向极限映射为局部环空间上的逆向极限.
    $$\mathrm{Hom}_{\textbf{Rings}^{op}}(\Gamma(X,\mathscr{O}_X),B)\cong\mathrm{Hom}_{\textbf{LRS}}((X,\mathscr{O}_X),\mathrm{Spec}(B))$$
\end{enumerate}
\subsection{概形上的概形}
\begin{enumerate}
	\item 给定概型$S$,当我们称概型$X$是一个$S$概型的时候,是指约定了一个态射$\varphi:X\to S$,它称为$S$概型$X$的结构态射.当$S$是某个环$A$对应的仿射概型的时候,我们也称$S$概型$X$是一个$A$概型.如果$S$概型$X$中的点$x$在结构态射下的像是点$s\in S$,我们称点$x$在点$s$上.给定两个$S$概型$X,Y$,设结构态射分别为$\varphi:X\to S$和$\psi:Y\to S$,它们之间的$S$态射是指满足如下图表交换的概型的态射$f:X\to Y$,$S$概型$X,Y$之间的全体态射记作$\mathrm{Hom}_S(X,Y)$,如果$S$是仿射概型,我们也会把这个记号中的$S$改为它对应的环.对于每个概型$S$,全体$S$概型和$S$态射构成一个范畴,称为$S$概型范畴.
	$$\xymatrix{X\ar[rr]^f\ar[dr]_{\varphi}&&Y\ar[dl]^{\psi}\\&S&}$$
	\item 我们证明过概型到仿射概型的态射集是典范的逆变的一一对应于整体截面环之间的环同态,特别的按照$\mathbb{Z}$是含幺环范畴上的初对象,就得到$\mathrm{Spec}(\mathbb{Z})$是概型范畴上的终对象.于是每个概型都可以视为$\mathbb{Z}$概型,甚至有概型范畴和$\mathbb{Z}$概型范畴是同构的.
	\item 如果$S=\mathrm{Spec}(A)$是仿射概型,设$X$是$S$概型,$Y=\mathrm{Spec}(B)$是仿射$S$概型.容易验证此时诱导了态射$X\to Y$的环同态实际上是$A$代数同态,此时有如下双射:
	$$\mathrm{Hom}_{\textbf{R-Alg}}(B,\mathscr{O}_X(X))\cong\mathrm{Hom}_{\textbf{S-Sch}}(X,Y)$$
	\item 给定态射$S\to S'$,它诱导了$S$概型范畴到$S'$概型范畴的函子为,对象上把$S$概型$X$对应为由$X\to S\to S'$作为结构态射的$S'$概型$X$,态射上不变.
	\item 设$A$是一个环,直接称$\mathrm{Spec}A$上的概型为$A$概型.对于$A$概型$X$,其上的所有截面环,茎,剩余类域都自然的具备$A$代数结构.它们的结构映射都是典范映射诱导的$A\to\mathscr{O}_X(X)\to\mathscr{O}_X(U)\to X_p\to\kappa(p)$.
	\item 如果$k_1$和$k_2$是两个不同特征的域,$X_1$和$X_2$分别是$k_1$和$k_2$上的概型,那么不存在$X_1$和$X_2$之间的态射.这是因为倘若存在,那么可限制为两个仿射开子概型之间的态射,导致它由不同特征的环之间的映射所诱导,但是不同特征之间的环是不存在同态的.
\end{enumerate}
\subsection{源端为局部概型的态射}

一个概型称为局部概型,如果它同构于某个局部环对应的仿射概型.对概型$Y$上一个点$y$,称$\mathrm{Spec}(\mathscr{O}_{Y,y})$为$Y$在点$y$处的局部概型.
\begin{enumerate}
	\item 局部概型存在一个点,它落在任意一个非空闭集中.
	\item 对概型$Y$上一个点$y$,任取一个包含点$Y$的仿射开子集$V=\mathrm{Spec}(B)$,那么$\mathscr{O}_{Y,y}$典范同构于局部化$B_y$.典范同态$B\to B_y$则诱导了一个概型的态射$\mathrm{Spec}(\mathscr{O}_{Y,y})\to V$.我们断言复合包含映射$V\to Y$得到的概型之间的态射$\mathrm{Spec}(\mathscr{O}_{Y,y})\to Y$不依赖于$y$的仿射开邻域$V$的选取.这个态射我们称为$Y$上局部概型到$Y$的典范态射.
	\begin{proof}
		
		假设还有$y$的仿射开邻域$V'=\mathrm{Spec}(B')$,那么可取$y$的仿射开邻域$W\subseteq V\cap V'$.于是问题归结为$V\subseteq V'$的情况:一旦包含的情况得证,那么$\mathrm{Spec}(\mathscr{O}_{Y,y})\to W\to Y$和$\mathrm{Spec}(\mathscr{O}_{Y,y})\to V\to Y$一致,也有$\mathrm{Spec}(\mathscr{O}_{Y,y})\to W\to Y$和$\mathrm{Spec}(\mathscr{O}_{Y,y})\to V'\to Y$一致.
		
		但是对于$V\subseteq V'$的情况,同态的复合$B'\to B\to\mathscr{O}_{Y,y}$恰好就是典范的$B'\to\mathscr{O}_{Y,y}$,于是它们诱导了相同的仿射概型之间的态射,也即$\mathrm{Spec}(\mathscr{O}_{Y,y})\to V'$和$\mathrm{Spec}(\mathscr{O}_{Y,y})\to V\to V'$是一致的,进而它们复合包含映射是一致的.
	\end{proof}
    \item 上一条中的典范态射$(\varphi,\varphi^{\#}):(\mathrm{Spec}(\mathscr{O}_{Y,y}),\mathscr{O}')\to(Y,\mathscr{O}_Y)$中$\varphi$的像集$S_y$就是点$y$在$Y$中所有一般化构成的集合,也即$Y$中全部满足$y\in\overline{\{x\}}$的$x$构成的集合.换句话讲这里的典范态射把$y$的局部概型中的点,一一对应于$X$的包含点$y$的不可约闭集.并且这个对应是保序的,如果$\mathrm{Spec}\mathscr{O}_{Y,y}$中的两个点$s,t$分别对应于$Y$的包含点$y$的不可约闭子集$S,T$,那么有$S\subseteq T$当且仅当$V(s)\subseteq V(t)$.
    \begin{proof}
    	
    	一方面,任取$x\in\mathrm{im}\varphi$,那么$\varphi^{-1}(\overline{\{x\}})$是局部概型的非空闭集,于是它包含了唯一的极大理想,于是$y\in\overline{\{x\}}$.另一方面,按照闭包的性质,任取$y$的仿射开子集$V$,有$S_y$落在$V$中,于是不妨约定$Y$本身是仿射的$\mathrm{Spec}(B)$,此时$y\in\overline{\{x\}}$等价于$p_x\subseteq p_y$,并且$\varphi$恰好是典范映射$B\to B_y$诱导的,于是像集恰好是和$p_y$补集不交的素理想,而$p_x\subseteq p_y$得到$p_x$和$p_y$的补集不交,于是$x\in\mathrm{im}\varphi$.
    \end{proof}
    \item 点$y\in Y$是$Y$的某个不可约分支的一般点当且仅当$\mathscr{O}_{Y,y}$的素谱是单点集合,等价于这个局部环零维;点$y$处局部环的极小素理想一一对应于$X$中经$y$点的不可约分支.
\end{enumerate}    
    
源端为局部概形的态射.
\begin{enumerate}
    \item 设$(X,\mathscr{O}_X)$是局部环$A$对应的局部概型,设$a$是它的唯一闭点,设$(Y,\mathscr{O}_Y)$是任意概型,那么每个态射$(\varphi,\varphi^{\#}):(X,\mathscr{O}_X)\to(Y,\mathscr{O}_Y)$可以唯一分解为$X\to\mathrm{Spec}(\mathscr{O}_{Y,\varphi(a)})\to Y$,其中后一个态射是我们之前给出的典范态射,前一个态射对应于唯一的局部映射$\mathscr{O}_{Y,\varphi(a)}\to A$.于是存在典范的从$(X,\mathscr{O}_X)\to(Y,\mathscr{O}_Y)$的态射到局部环同态$\mathscr{O}_{Y,y}\to A,y\in Y$的双射.
    \begin{proof}
    	
    	我们说明过$\varphi(X)$包含在每个覆盖了$\varphi(a)$的仿射开子集中.于是问题归结为$Y$是仿射的情况.记$Y=\mathrm{Spec}(B)$,此时$(\varphi,\varphi^{\#})$就由某个环同态$f:B\to A$诱导.按照$f^{-1}(p_a)=p_{\varphi(a)}$,说明$f(B-p_{\varphi(a)})$在$A$中可逆,于是按照分式化的泛性质,这诱导了环同态$B\to B_a\to A$,并且这个复合就是$f$.由此得到结论中的分解.
    	
    	反过来对每个局部环同态$\mathscr{O}_{Y,y}\to A$,对应了唯一的态射$(\varphi,\varphi^{\#}):X\to\mathrm{Spec}(\mathscr{O}_{Y,y})$,并且此时必然满足$\varphi(a)=y$.这证明了唯一性.
    \end{proof}
    \item 设$k$是域,从$\mathrm{Spec}(k)$到某个概型$X$的态射集一一对应于二元对$(x,l)$,其中$x\in X$,$l$是$\kappa(x)\to k$的同态.
    \begin{proof}
    	
    	首先给定态射$(f,f^{\#}):\mathrm{Spec}(k)\to X$.这里$\mathrm{Spec}(k)$是单点集,它的像集只由一个点构成,记作$x\in X$.这个态射诱导的茎之间的同态为$f^{\#}_x:\mathscr{O}_{X,x}\to\mathscr{O}_{\mathrm{Spec}(k),x}\cong k$.按照商域的泛性质,它诱导了单同态$l:\kappa(x)=\mathscr{O}_{X,x}/m_x\to k$.我们就记$(f,f^{\#})$对应于$(x,l)$.
    	
    	反过来如果$x\in X$,$l:\kappa(x)\to k$是单同态.定义连续映射$f:\mathrm{Spec}(k)\to X$把单点映射为$x$.接下来要构造$f^{\#}:\mathscr{O}_X\to f_*\mathscr{O}_{\mathrm{Spec}(k)}$.如果开集$U\subseteq X$包含了点$x$,构造$f^{\#}:\mathscr{O}_X(U)\to\mathscr{O}_{\mathrm{Spec}(k)}(f^{-1}(U))=k$为如下复合映射:
    	$$\xymatrix{\mathscr{O}_X(U)\ar[r]&\mathscr{O}_{X,x}\ar[r]&\frac{\mathscr{O}_{X,x}}{m_x}=\kappa(x)\ar[r]^{\quad l}&k}$$
    	
    	如果开集$U\subseteq X$不包含点$x$,那么取$\mathscr{O}_X(U)\to\mathscr{O}_{\mathrm{Spec}(k)}(f^{-1}(U))=\{0\}$是平凡映射.只要验证$(f,f^{\#})$是态射,再验证上述两种对应互逆即可.
    \end{proof}
\end{enumerate}
\subsection{$S$值点和有理点}

设$X,T$是概形.
\begin{itemize}
	\item 称$X$上的$T$值点是指一个态射$T\to X$.所以$X$上的全体$T$值点构成的集合就是$\mathrm{Hom}_{\textbf{Sch}}(T,X)$.
	\item 如果$X,T$都是$S$概形,称$S$概形$X$上的一个$T$值点是指作为$S$概形的态射$T\to X$,于是$X$上全体$T$值点构成的集合为$\mathrm{Hom}_{\textbf{S-Sch}}(T,X)$.在$S$不引起歧义的情况下记作把$X$上全部$T$值点集合记作$X(T)$.
	\item 如果$X$是$S$概形,设$S$以恒等态射作为自身上的概形,我们把$S$概形$X$上的$S$值点称为$X$上的$S$有理点,按照定义它是结构态射$\pi:X\to S$的截面,也即一个态射$\sigma:S\to X$满足$\pi\circ\sigma=1_S$.
\end{itemize}
\begin{enumerate}
	\item 设$\varphi:X\to Y$是$S$概形之间的态射,那么它诱导了$T$值点之间的映射为$\varphi^*_T:\mathrm{Hom}_{\textbf{S-Sch}}(T,X)\to\mathrm{Hom}_{\textbf{S-Sch}}(T,Y)$为$\sigma\mapsto\varphi\circ\sigma$.米田引理告诉我们态射$\varphi$被全体$\varphi^*_T$,其中$T$取遍$S$概形,唯一确定.
	\item 设$k$是域,设$X$是$k$概形,那么存在从$X(k)$到$X$的剩余域为$k$的点之间的双射:一方面如果$\sigma\in X(k)$,因为$\mathrm{Spec}k$是单点集,设这个单点在$\sigma$下的像是$x$,那么$\sigma$诱导了stalk上的扩张$k\subseteq\kappa(x)\subseteq k$,迫使$\kappa(x)=k$.另一方面如果$x\in X$是剩余域为$k$的点,按照源端为局部概形的态射的描述,$x$和$k$上的恒等映射诱导了$k$概形之间的态射$\mathrm{Spec}k\to X$.
	\item 设$k$是域,设$X$是$k$概形,设$Y\subseteq X$是开或者闭子概型,那么对每个点$y\in Y$,它在$Y$中的剩余域和在$X$中的剩余域一致,于是有$Y(k)=X(k)\cap Y$.
	\item 仿射空间的$S$值点.设$X=\mathbb{A}_{\mathbb{Z}}^n=\mathrm{Spec}\mathbb{Z}[T_1,\cdots,T_n]$,那么有:
	$$X(S)=\mathrm{Hom}_{\textbf{Sch}}(S,X)=\mathrm{Hom}_{\textbf{Rings}}(\mathbb{Z}[T_1,\cdots,T_n],\Gamma(S,\mathscr{O}_S))=\Gamma(S,\mathscr{O}_S)^n$$
	$$\varphi\mapsto(\varphi(T_1),\cdots,\varphi(T_n))$$
	
	类似的,对域$k$上的仿射空间$X=\mathbb{A}_k^n=\mathrm{Spec}k[T_1,\cdots,T_n]$,对$k$概形$S$我们有$k$代数同构:
	$$\mathrm{Hom}_k(S,X)\cong\Gamma(S,\mathscr{O}_S)^n$$
	$$\varphi\mapsto(\varphi(T_1),\cdots,\varphi(T_n))$$
	
	特别的,域$k$上仿射空间$\mathbb{A}_k^n$的$k$有理点集同构于$k^n$.
	\item 域上有限型仿射概形的有理点.设$k$是域,记$X=\mathrm{Spec}k[x_1,x_2,\cdots,x_n]/(f_1,f_2,\cdots,f_r)$,那么$X(k)$恰好一一对应域$k$上多项式方程组$f_i(x_1,x_2,\cdots,x_n)=0,\forall 1\le i\le r$在$k$上的解.
	\begin{proof}
		
		设$\alpha=(\alpha_1,\cdots,\alpha_n)\in k^n$是$\{f_i\}$的一个公共零点,记$\mathfrak{m}_{\alpha}=(X_1-\alpha_1,\cdots,X_n-\alpha_n)$是$k[X_1,X_2,\cdots,X_n]/(f_1,f_2,\cdots,f_r)$的一个极大理想,它对应了$X$的一个点,并且剩余域是$k$,于是零点对应了一个$k$值点.反过来任取$X(k)=\mathbb{A}_k^n(k)\cap X$中的点,它作为$\mathbb{A}_k^n$的有理点可表示为$(X_1-\alpha_1,\cdots,X_n-\alpha_n)$,它在$Y$中就导致$(\alpha_1,\cdots,\alpha_n)$是$\{f_1,\cdots,f_r\}$的公共零点.
	\end{proof}
    \item 域上射影空间的有理点.设$k$是域,用$\mathbb{P}(k^{n+1})$表示古典射影空间$\left(k^{n+1}-\{0\}\right)/k^*$.任取$\alpha\in\mathbb{P}(k^{n+1})$,设它的一个齐次坐标为$[\alpha_0,\cdots,\alpha_n]$,定义$\rho(\alpha)$是$k[T_0,\cdots,T_n]$的被$\alpha_jT_i-\alpha_iT_j,0\le i,j\le n$生成的齐次理想,这是一个剩余域为$k$的极大理想.我们断言$\rho$诱导了从$\mathbb{P}(k^{n+1})$到有理点集$\mathbb{P}_k^n(k)$的双射.
    \begin{proof}
    	
    	$\rho(\alpha)$不包含无关理想$(T_0,\cdots,T_n)$(可对$n$归纳).设$\alpha_t\not=0$,那么$k[T_0,\cdots,T_n]/\rho(\alpha)\cong k[T_0]$,于是$\rho(\alpha)$是素理想.因为$T_i-\alpha_t^{-1}\alpha_iT_t\in\rho(\alpha),\forall i$,所以不能有$T_t\in\rho(\alpha)$,否则$\rho(\alpha)$就要包含无关理想.于是$\rho(\alpha)\in D_+(T_t)$.在仿射概形$D_+(T_t)$里有$\rho(\alpha)$是由$T_i/T_t-\alpha_i/\alpha_t$生成的极大理想,并且它的剩余域是$k$本身,局部有限型概形上开集的闭点是原空间的闭点,于是$\rho(\alpha)$是射影空间$\mathbb{P}_k^n$的闭点.综上我们证明了$\rho(\alpha)$是$\mathbb{P}_k^n$上的$k$有理点.
    	
    	\qquad
    	
    	假设$\beta=[\beta_0,\cdots,\beta_n]\in\mathbb{P}(k^{n+1})$满足$\rho(\alpha)=\rho(\beta)$.如果$\beta_t=0$,那么$T_t\in\rho(\beta)=\rho(\alpha)$,但是$\alpha_t\not=0$,导致所有$T_j\in\rho(\alpha)$矛盾.于是$\beta_t\not=0$.但是在$D_+(T_0)$中$\rho(\alpha)$被$T_i-\alpha_t^{-1}\alpha_iT_t$生成,$\rho(\beta)$被$T_i-\alpha_t^{-1}\alpha_iT_t$生成.它们相同必须有$\alpha_t^{-1}\alpha_i=\beta_t^{-1}\beta_i$,也即$[\alpha_0,\cdots,\alpha_n]=[\beta_0,\cdots,\beta_n]$.这说明$\rho$是单射.
    	
    	\qquad
    	
    	最后设$x$是$\mathbb{P}_k^n$的$k$有理点,不妨设$x\in D_+(T_0)$,那么$x$是仿射概形$D_+(T_0)$的$k$有理点,它就可以表示为$(T_0^{-1}T_1-a_1,\cdots,T_0^{-1}T_n-a_n)$,其中$a_i\in k$.我们记$\alpha=(1,a_1,\cdots,a_n)$就满足$\rho(\alpha)=x$.
    \end{proof}
    \item 设$k$是域,设齐次理想$I=(P_1(T),\cdots,p_m(T))\subseteq k[T_0,\cdots,T_n]$,其中每个$p_i(T)$是齐次多项式,那么存在从$\mathrm{Proj}k[T_0,\cdots,T_n]/I$上全部$k$有理点到$\{P_1,\cdots,P_m\}$的公共零点集之间的双射.
    \begin{proof}
    	
    	记$B=k[T_0,\cdots,T_n]$,记$Z_+$表示$\{P_1,\cdots,P_m\}$在仿射空间$\mathbb{P}(k^{n+1})$的公共零点集.由于$\mathrm{Proj}B/I$是$\mathrm{Proj}B=\mathbb{P}_k^n$的闭子概型,于是$\mathrm{Proj}(B/I)$的$k$有理点集就是$V_+(I)\cap\mathbb{P}_k^n(k)$.记典范双射$\rho:\mathbb{P}(k^{n+1})\to\mathbb{P}_k^n(k)$,我们要证明的就是$\rho(Z_+)=V_+(I)\cap\mathbb{P}_k^n(k)$.
    	
    	\qquad
    	
    	对每个$0\le i\le n$,记$U_i=\rho^{-1}(D_+(T_i))\subseteq\mathbb{P}(k^{n+1})$,此即齐次坐标点$[\alpha_0,\cdots,\alpha_n]$满足$\alpha_i\not=0$.归结为证明$\rho(Z_+\cap U_i)=V_+(I)\cap D_+(T_i)(k)$.此为仿射有限型$k$概形的情况.
    \end{proof}
	\item 设$X$是域$k$上的概形,点$x\in X$上的Zariski切空间$\mathrm{T}_xX$通常定义为$\kappa(x)$线性空间$m_x/m_x^2$的对偶空间.我们考虑环$k[x]/(x^2)$称为域$k$上的对偶数环,那么$k$概形态射$\mathrm{Spec}k[x]/(x^2)\to X$一一对应于$(x,v)$,其中$x\in X$是$k$有理点,而$v\in\mathrm{T}_xX$.
	\item 如果$X,Y,W$都是$S$概型,按照纤维积的定义,有$\mathrm{Hom}_S(W,X\times_SY)\cong\mathrm{Hom}_S(W,X)\times\mathrm{Hom}_S(W,Y)$.换句话讲,概型值点在纤维积下表现良好:纤维积$X\times_SY$上的$W$值点恰好一一对应于一对$X$上的$W$值点和$Y$上的$W$值点.
\end{enumerate}
\subsection{关于可表性}

考虑从$\textbf{S-Sch}^{\mathrm{op}}\to\textbf{Sets}$的函子,如果它自然同构于某个$h_X:Y\mapsto\mathrm{Hom}_S(Y,X)$,就称它是一个可表函子.下面主要给出一些简单例子.
\begin{enumerate}
	\item 考虑概型范畴到集合范畴的逆变函子$X\mapsto\{(f_1,f_2,\cdots,f_n)\mid f_i\in\Gamma(X,\mathscr{O}_X)\}$.换句话讲它把概形对应为全体由截面构成的$n$元对.它把概形的态射$f:X\to Y$映射为集合间的映射$(g_1,\cdots,g_n)\in\Gamma(Y,\mathscr{O}_Y)^n\mapsto(f^{\#}(g_1),\cdots,f^{\#}(g_n))\in\Gamma(X,\mathscr{O}_X)^n$.这个函子被$\mathbb{A}_{\mathbb{Z}}^n$表示.
	\item 乘法群概型.考虑概型范畴到集合范畴的逆变函子把概型$X$映射为它整体截面环的乘法群.这个函子被$\mathrm{Spec}\mathbb{Z}[t,t^{-1}]$表示,称它为乘法群概型.
\end{enumerate}
\subsection{支配态射}

概形之间的态射$f:X\to Y$称为支配态射,如果$f(X)$在$Y$中稠密.
\begin{enumerate}
	\item 对于仿射概形之间的态射$f:\mathrm{Spec}B\to\mathrm{Spec}A$,我们解释过它是支配的当且仅当对应的环同态$\varphi:A\to B$满足$\ker\varphi$落在$A$的幂零根中.特别的如果$\varphi$是单射那么$f$是支配态射.
	\item 设$f:X\to Y$是拟紧态射,那么如下命题互相等价.
	\begin{enumerate}
		\item $f$是支配态射.
		\item 对$Y$的每个一般点$y$,都有$f^{-1}(y)$是非空的.
		\item 对$Y$的每个一般点$y$,都有$f^{-1}(y)$包含了$X$的某个一般点.
	\end{enumerate}
	\begin{proof}
		
		明显有(c)推(b)推(a)成立,并且这不需要拟紧条件,下面只需证明拟紧条件下有(a)推(c)成立:这里问题都是拓扑上的,所以可以不妨设$X,Y$都是既约概形,设$f$是拟紧态射,设$y\in Y$是一个一般点.任取$y$的仿射开邻域$V$,那么$f^{-1}(V)$是$X$的拟紧开子集,于是它是$X$的有限个仿射开子集$U_i$的并.任取$y$的开邻域$W'$,那么可取$y$的仿射开邻域$W$满足$W\subseteq V$,则按照$f$是支配的有$W\cap f(X)$非空,任取$f(x)\in W\cap f(X)$,那么这里$x\in f^{-1}(V)=U$,于是我们证明了$y$的任意开邻域都和$f(U)$有交,于是得到$y\in\overline{f(U)}=\cup\overline{f(U_i)}$,换句话讲存在某个指标$i$使得$y$落在$f(U_i)$在$U$中的闭包中.我们知道$U_i$的一般点也是$X$的一般点,所以不妨用$U_i$替代$X$,用$U$的既约闭子概型$\overline{f(U_i)\cap V}$替代$Y$,这使得$f$仍然是支配的,并且这里$X=\mathrm{Spec}A$和$Y=\mathrm{Spec}B$都是仿射的.那么$f$是支配态射等价于讲它对应的环同态$B\to A$是单射,也即$B$可视为$A$的子环.于是我们的结论就来自于子环$B$的极小素理想一定是$A$的极小素理想和$B$的交.
	\end{proof}
	\item 下降性质.设$f:X\to Y$是态射,设$X'$是它关于$g:Y'\to Y$的基变换,如果$g$是满射,那么从$f'$是支配态射得到$f$也是支配态射.
	\begin{proof}
		
		我们有$f\circ g'=g\circ f'$.设$g$是满射,设$f'$是支配态射,按照$Y=g(\overline{f'}(X'))\subseteq\overline{g(f'(E))}$,得到$g\circ f'$也是支配态射,进而有$f\circ g'$也是支配态射.但是这里$g'$是满射,于是$f$是支配的.
	\end{proof}
	
	
	
	
	\item 设$f:X\to Y$是不可约概形之间的态射,它们的唯一一般点分别记作$\eta_X$和$\eta_Y$.我们断言如下命题互相等价:
	\begin{enumerate}
		\item $f$是支配态射.
		\item $f(\eta_X)=\eta_Y$.
		\item $\eta_Y\in f(X)$.
		\item 对$Y$的任意开子集$V$和$X$的任意满足$U\subseteq f^{-1}(V)$的开子集$U$,有$\mathscr{O}_Y(V)\to\mathscr{O}_X(U)$是单射.
		\item $f^{\#}:\mathscr{O}_Y\to f_*\mathscr{O}_X$是单态射.
		\item 存在$Y$的非空开子集$V$和一个$X$的满足$U\subseteq f^{-1}(V)$的开子集$U$,使得$\mathscr{O}_Y(V)\to\mathscr{O}_X(U)$是单射.
		\item 对任意$x\in X$,记$y=f(x)$,有$f^{\#}_x:\mathscr{O}_{Y,y}\to\mathscr{O}_{X,x}$总是单射.
		\item 存在$x\in X$,记$y=f(x)$,使得$f^{\#}_x:\mathscr{O}_{Y,y}\to\mathscr{O}_{X,x}$是单射.
	\end{enumerate}
    \begin{proof}
    	
    	(b)推(c)推(a)是平凡的,(a)推(b)因为$Y=\overline{f(X)}=\overline{f(\overline{\{\eta_X\}})}=\overline{\{f(\eta_X)\}}$,迫使$\eta_Y=f(\eta_X)$.
    	
    	【】
    \end{proof}
\end{enumerate}
\newpage
\section{有理映射和亚纯映射}
\subsection{有理映射}

设$X,Y$是概形.
\begin{itemize}
	\item 考虑全体从$X$的稠密开子集到$Y$的态射构成的集合$S(X,Y)$,其上定义一个等价关系为,态射$f:U\to Y$和$g:V\to Y$等价当且仅当存在$U\cap V$的稠密开子集$W$使得$f\mid_W=g\mid_W$(这是等价关系是因为有限个稠密开子集的交仍然是稠密开子集).把等价类称为一个有理映射$X\to Y$.全体有理映射构成的环记作$\mathrm{Rat}(X,Y)$.
	\item 如果$X,Y$都是$S$概形,在全体从$X$的稠密开子集到$Y$的$S$态射构成的集合上定义类似的等价关系,其中等价类称为一个$S$有理映射$X\to Y$.全体$S$有理映射构成的环记作$\mathrm{Rat}_S(X,Y)$.
	\item 概形$X$上的有理函数指的是$X\to\mathbb{A}_X^1=X\times_{\mathrm{Spec}\mathbb{Z}}\mathrm{Spec}\mathbb{Z}[T]$的$X$有理映射.这等价于讲考虑$X$上全体稠密开集上截面构成的集合,两个截面$(U,s),(V,t)$定义为等价的如果存在稠密开子集$W\subseteq U\cap V$使得$s\mid_W=t\mid_W$,那么有理函数就是等价类.全体有理函数构成一个环,记作$\mathrm{Rat}(X)$,称为$X$的有理函数环.
\end{itemize}
\begin{enumerate}
	\item 有理映射在开子集上的限制.设$f:X\to Y$是有理映射,设$U\subseteq X$是开子集.任取$f$的两个表示$f_1:U_1\to Y$和$f_2:U_2\to Y$,那么$f_1\mid_{U_1}$和$f_2\mid_{U_2}$也是等价的,就把这个等价类记作$f\mid_U$,称为$f$在$U$上的限制.
	\item 对于不可约概形$X$,一般点记作$\eta$,它的非空开子集都是稠密开子集,此时两个态射$f:U\to Y$和$g:V\to Y$是等价的当且仅当$f_{\eta}=g_{\eta}$.特别的,此时$\mathrm{Rat}(X)\cong\mathscr{O}_{X,\eta}$,这总是零维局部环.如果$X$是诺特不可约的,那么$\mathrm{Rat}(X)$是阿廷局部环;如果$X$是整概形,那么$\mathrm{Rat}(X)$是域,也即$X$的函数域.
	\item 设$X,Y$是$S$概形,设$X$只有有限个不可约$\{X_i\}$,记$X_i$的一般点为$\eta_i$,记$R_i$是从$X$的包含$\eta_i$的开子集映到$Y$的态射在$\eta_i$处芽构成的环.那么$\mathrm{Rat}_S(X,Y)=\prod_iR_i$.
	\begin{proof}
		
		条件下可以取$\eta_i$的开邻域$U_i\subseteq X_i$,满足$\{U_i\}$两两不交,那么$U=\cup_iU_i$是$X$的稠密开集.任取$X$的稠密开子集$V$,那么$V_i=V\cap U_i$在$U_i$中稠密,并且这些$V_i$两两不交,于是$V\to Y$的$S$态射在等价意义下恰好一一对应于一族$V_i\to Y$的$S$态射.
	\end{proof}
    \item 推论.设$X$是诺特概形,那么$\mathrm{Rat}(X)=\prod_i\mathscr{O}_{X,\eta_i}$,其中$\{\eta_i\}$是$X$的全部有限个一般点.
    \item 仿射情况:设$X=\mathrm{Spec}A$是诺特环的素谱,那么$X$上的稠密主开集具有形式$D(f)$,其中$f$落在$A$的全部极小素理想的并的补集$Q$中.并且$X$的任意稠密开子集都包含了一个形如$D(f)$的稠密主开集.此时$\mathrm{Rat}(X)=Q^{-1}A$.
    \begin{proof}
    	
    	记$X$的全部有限个不可约分支是$\{X_i=V(\mathfrak{p}_i)\}$.主开集$D(f)$是稠密开集当且仅当$D(f)\cap X_i$总非空,也即$f\not\in\cup_i\mathfrak{p}_i$.
    	
    	\qquad
    	
    	任取$X$的稠密开子集$U$,那么它的补集可以表示为$V(\mathfrak{a})$,这里理想$\mathfrak{a}$不能包含在每个$\mathfrak{p}_i$中,进而也不能包含在它们的并中.于是可取$f\in\mathfrak{a}$满足$f\in Q$,此时$D(f)\subseteq U$.
    	
    	\qquad
    	
    	按照$\{D(f)\mid f\in Q\}$和$X$上全体稠密开集构成的偏序集是共尾的,于是$\mathrm{Rat}(X)$就是$\{\Gamma(D(f),\mathscr{O}_X)=A_f,f\in Q\}$的正向极限,也即$Q^{-1}A$.
    \end{proof}
    \item 设$X$是不可约概形,一般点记作$\eta$,任取一个有理映射$X\to Y$,取它的两个表示$f:U\to Y$和$g:V\to Y$.那么它们和典范态射$\mathrm{Spec}\mathscr{O}_{X,\eta}\to U$和$\mathrm{Spec}\mathscr{O}_{X,\eta}\to V$的复合是相同的态射,于是我们证明了存在典范映射$\mathrm{Rat}(X,Y)\to\mathrm{Hom}_{\textbf{Sch}(S)}(\mathrm{Spec}\mathscr{O}_{X,\eta},Y)$.
    \item 反过来我们知道有限型态射的stalk可以决定局部,于是有:设$X,Y$是$S$概形,$X$是不可约的,一般点记作$\eta$,$Y$在$S$上有限型,那么如果两个$S$有理映射$X\to Y$对应的$\mathrm{Spec}\mathscr{O}_{X,\eta}\to Y$是相同的,则它们是相同的有理映射.进一步如果$S$还是局部诺特的,那么上一段的对应是双射.特别的,此时$X\to Y$的$S$有理映射只依赖于$S$概形$\mathrm{Spec}\mathscr{O}_{X,\eta}$,
    \item 推论.
    \begin{enumerate}[(1)]
    	\item 设$X,Y$是$S$概形,$S$是局部诺特的,$X$是不可约的,一般点记作$\eta$,$Y$是有限型$S$概形.按照源端是局部概形的态射的描述,$\mathrm{Rat}_S(X,Y)$恰好一一对应于$(y,\varphi)$,其中$y\in Y$在$S$中的像是$\eta$在$S$中的像$s$,$\varphi:\mathscr{O}_{Y,y}\to\mathscr{O}_{X,x}=\mathrm{Rat}(X)$是局部$\mathscr{O}_{S,s}$同态.
    	\item 如果$X$还是整概形,那么$\mathscr{O}_{X,\eta}$是域,按照源端是域的态射的描述,$\mathrm{Rat}_S(X,Y)$恰好一一对应于$(y,\varphi)$,其中$y\in Y$在$S$中的像是$\eta$在$S$中的像$s$,$\varphi:\kappa(y)\to\kappa(x)=\mathrm{Rat}(X)$是$\kappa(s)$嵌入.
    	\item 如果额外的还有$S=\mathrm{Spec}k$是域,$X$也是有限型的,那么此时$\mathrm{Rat}_S(X,Y)$恰好一一对应于$Y$的$\mathrm{Rat}(X)$值点.
    \end{enumerate}
    \item 设$f:X\to Y$是有理映射,称它在点$x\in X$有定义,如果它存在一个表示$U\to Y$满足$x\in U$.全体有定义的点构成的集合称为$f$的定义域,它就是$\cup_{(U\to Y)\in f}U$,这是一个稠密开集.设$f:X\to Y$是一个$S$有理映射,定义域记作$U_0$.如果$X$是既约的并且$Y\to S$是分离的,那么等价类$f$中存在唯一一个$S$态射$U_0\to Y$.这件事有如下推论:
    \begin{enumerate}[(1)]
    	\item 此时$\mathrm{Rat}_S(X,Y)$和不能再延拓的源端稠密的$S$态射$U\to Y$是一一对应的.
    	\item 对稠密开集$U\subseteq X$,有$\mathrm{Hom}_S(U,Y)$恰好一一对应于在$U$上处处有定义的$S$有理映射.
    	\item 设$X$是既约概形,对稠密开集$U\subseteq X$,有$\Gamma(U,\mathscr{O}_X)$恰好一一对应于在$U$上处处有定义的有理函数.
    \end{enumerate}
    \begin{proof}
    	
    	设$U_1,U_2\subseteq X$是两个稠密开集,$f_1:U_1\to Y$和$f_2:U_2\to Y$是两个态射,满足存在稠密开集$V\subseteq U_1\cap U_2$使得$f_1\mid_V=f_2\mid_V$,问题归结为证明$f_1\mid_{U_1\cap U_2}=f_2\mid_{U_1\cap U_2}$.于是我们可以把$f$所在等价类中的全部态射粘合得到一个最大定义域的态射.这件事在分离态射中证明过了.
    \end{proof}
\end{enumerate}
\subsection{有理函数层}

设$X$是概形,对开子集$U$,记$\mathrm{Rat}(U)$表示$U$上的有理函数环.那么$U\mapsto\mathrm{Rat}(U)$是$X$上的预层,它的层化称为$X$上的有理函数层,记作$\mathscr{R}_X$.
\begin{enumerate}
	\item 有理函数层明显和开子概型可交换,也即$\mathscr{R}_X\mid_U=\mathscr{R}_U$.
	\item 设$X$是概形,设它的不可约分支$\{X_i\}$是一个局部有限子集族.那么$\mathscr{R}_X$是拟凝聚代数层,并且对任意只和有限个$X_i$有交的开子集$U$,有$\mathrm{Rat}(U)=\mathscr{R}_X(U)$,并且它就是$\prod_i\mathscr{O}_{X,\eta_i}$,其中$\eta_i$跑遍那些有限个和$U$有交的不可约分支的一般点.
	\begin{proof}
		
		问题可以归结为设$X$只有有限个不可约分支$\{X_1,\cdots,X_n\}$,记$X_i$的一般点是$\eta_i$.我们已经解释过此时$\mathrm{Rat}(U)=\prod_{X_i\cap U\not=\emptyset}\mathscr{O}_{X,\eta_i}$.此时$\mathrm{Rat}$已经是$X$上的层.最后为证明$\mathscr{R}_X$是拟凝聚的只需设$X=\mathrm{Spec}A$是仿射的,那么$\mathscr{R}_X=\widetilde{M}$,其中$M$是那些$A$模$A_{\eta_i}$的直和.
	\end{proof}
    \item 设$X$是只有有限个不可约分支的既约概形,设全部不可约分支赋予既约闭子概型为$\{X_1,\cdots,X_n\}$.记典范闭嵌入$h_i:X_i\to X$,那么$\mathscr{R}_X=\oplus_{1\le i\le n}(h_i)_*(\mathscr{R}_{X_i})$.
    \item 如果$X$是不可约概形,那么$\mathscr{R}_X$是常值层,甚至任何拟凝聚$\mathscr{R}_X$模层$\mathscr{F}$都是常值层.
    \begin{proof}
    	
    	因为不可约空间上常值层和局部常值层等价,问题归结为设$X=\mathrm{Spec}A$本身是仿射的.那么$\mathscr{F}$是某个同态$(\mathscr{R}_X)^{(I)}\to(\mathscr{R}_X)^{(J)}$的余核,于是问题归结为证明$\mathscr{R}_X$是常值层,而这是因为对任意非空开集$U\subseteq X$,都包含了一般点$\eta$,于是$\Gamma(U,\mathscr{R}_X)=A_{\eta}$.
    \end{proof}
    \item 如果$X$是不可约分支是局部有限的既约概形,那么典范态射$\mathscr{O}_X\to\mathscr{R}_X$是单的.
    \begin{proof}
    	
    	因为我们解释过既约条件下截面和开集上处处有定义的有理函数一一对应.
    \end{proof}
\end{enumerate}




\subsection{有理映射}

【这里的有理映射是EGAIV的pseudo-morphism】

概形稠密开集.设$X$是概形,一个开子集$U$称为概形稠密开集(schematically dense open subset),如果对任意开子概型$V\subseteq X$,都有$V$的支配了$U\cap V$的闭子概型只有$V$本身,换句话讲如果$Z\subseteq V$是$V$的闭子概型使得$U\cap V\subseteq V$要经$Z\subseteq V$分解,那么有$Z=V$.如果开嵌入$j:Y\to X$满足$j(Y)$是$X$的概形稠密开集,则称$j$是概形支配的(schematically dominant).
\begin{enumerate}
	\item 设$X$是$S$概形,设$j:U\to X$是开嵌入,那么如下命题互相等价:
	\begin{enumerate}
		\item $U$是$X$的概形稠密开子集.
		\item 典范层态射$\mathscr{O}_X\to j_*\mathscr{O}_U$是单态射.
		\item 对任意开子概型$V\subseteq X$和任意分离$S$概形$Y$,对任意$S$态射$f,g:V\to Y$如果$f\mid_{U\cap V}=g\mid_{U\cap V}$,则在$V$上有$f=g$.
	\end{enumerate}
	\begin{proof}
		
		(b)推(a):设$V$是$X$的开子概型,设$i:Z\to V$是闭子概型,对应的拟凝聚理想层记作$\mathscr{I}\subseteq\mathscr{O}_V$.设$Z$支配了$U\cap V$,考虑态射的复合$\mathscr{O}_V\to j_*\mathscr{O}_{U\cap V}\to j_*\mathscr{O}_Z$,前者是单态射的限制所以是单态射,后者是单态射因为$U\cap V\subseteq Z$是开子概型.这导致$\mathscr{O}_V\to\mathscr{O}_V/\mathscr{I}$是单态射,但是它也是层范畴上的满态射,在层范畴上单满态射的确是同构,于是$\mathscr{I}=0$,也即$Z=V$.
		
		\qquad
		
		(a)推(c):设$V\subseteq X$是开子概型,设$Y$是分离$S$概形,设$f,g:V\to Y$使得$f\mid_{U\cap V}=g\mid_{U\cap V}$.那么按照纤维积的泛性质,存在$U\cap V\to\ker(f,g)$使得如下图表交换.由于$Y\to S$是分离态射,我们解释过它等价于讲对任意$X$和任意态射$f,g:X\to Y$有$\ker(f,g)\subseteq X$是闭嵌入,于是这里$\ker(f,g)$是$V$的闭子概型,并且$U\cap V$经$\ker(f,g)$分解.那么按照$U$是概形稠密开集,说明$\ker(f,g)=V$,并且这里$p=q=1_V$.这就得到$f=g$.
		$$\xymatrix{U\cap V\ar@/^1pc/[drr]\ar@/_1pc/[ddr]\ar[dr]&&\\&\ker(f,g)\ar[r]^p\ar[d]_q&V\ar[d]^g\\&V\ar[r]_f&Y}$$
		
		(c)推(b):任取开子集$V\subseteq X$,我们要证明限制映射$\rho:\mathscr{O}_X(V)\to\mathscr{O}_X(V\cap U)$是单射.而这是因为取$Y$是分离概形$\mathbb{A}_S^1$,那么$\mathrm{Hom}_S(W,\mathbb{A}_S^1)=\Gamma(W,\mathscr{O}_X)$.于是从$\mathrm{Hom}_S(V,\mathbb{A}_S^1)\to\mathrm{Hom}_S(U\cap V,\mathbb{A}_S^1)$是单射就得到限制映射$\rho$是单射.
	\end{proof}
	\item 条件(b)说明概形稠密开集总是稠密开集.另外我们解释过如果$X$是既约$S$概形,$Y$是分离$S$概形,那么如果$S$态射$f,g:X\to Y$在稠密开集$U$上相同,则它们在整个$X$上相同.这结合条件(c)说明既约概形上的稠密开子集一定是概形稠密开集.
	\item 如果$U,U'\subseteq X$是两个概形稠密开集,那么$U\cap U'$也是概形稠密开集.
	\begin{proof}
		
		任取开集$W$,我们要证明$\mathscr{O}_X(W)\to\mathscr{O}_X(U'\cap U\cap W)$是单射.但是它可以分解为$\mathscr{O}_X(W)\to\mathscr{O}_X(U\cap W)\to\mathscr{O}_X(U'\cap U\cap W)$,前者是单射因为$U$是概形稠密开集,后者是单射因为$U'$是概形稠密开集.
	\end{proof}
	\item 如果$U$是$X$的概形稠密开集,$W$是$U$的概形稠密开集,那么$W$也是$X$的概形稠密开集.
	\begin{proof}
		
		任取分离$S$概形$Y$,任取开子概型$V\subseteq X$,任取态射$f,g:V\to Y$使得$f\mid_{V\cap W}=g\mid_{V\cap W}$,那么按照$W$在$U$中是概形稠密开集,说明$f\mid_{V\cap U}=g\mid_{V\cap U}$,又按照$U$在$X$中是概形稠密开集,得到在整个$V$上有$f=g$.这就得到$W$是$X$的概形稠密开集.
	\end{proof}
	\item 设$X=\cup_jV_j$是开覆盖,那么一个开集$U\subseteq X$是概形稠密开集当且仅当对每个$j$有$U\cap V_j$是$V_j$的概形稠密开集.
	\begin{proof}
		
		必要性是因为任取开集$W\subseteq V_j$,那么$\mathscr{O}_{V_j}(W)=\mathscr{O}_X(W)\to\mathscr{O}_X(W\cap V_j\cap U)=\mathscr{O}_{V_j}(W\cap U)$是单射.充分性是因为任取开集$W\subseteq X$,我们有如下交换图表,按照粘合的唯一性得到$\alpha$和$\beta$都是单射.按照$V_j\cap U$都是$V_j$的概形稠密开集得到$\gamma$是单射,这就说明$\mathscr{O}_X(W)\to\mathscr{O}_X(W\cap U)$是单射.
		$$\xymatrix{\mathscr{O}_X(W)\ar[rr]\ar[d]_{\alpha}&&\mathscr{O}_X(W\cap U)\ar[d]^{\beta}\\\prod_i\mathscr{O}_X(W\cap V_i)\ar[rr]^{\gamma}&&\prod_i\mathscr{O}_X(W\cap U\cap V_i)}$$
	\end{proof}
\end{enumerate}

对于局部诺特概形我们可以用伴随点描述概形稠密开集.回顾一个环$A$上的素理想$\mathfrak{p}$称为伴随素理想,如果存在$a\in A$使得$\mathfrak{p}=\mathrm{Ann}(a)$.如果$X$是局部诺特概形,一个点$x\in X$称为伴随点(associated point),如果$\mathfrak{m}_x$是$\mathscr{O}_{X,x}$的伴随素理想.$X$的所有伴随点构成的集合记作$\mathrm{Ass}(X)$.如果$x\in X$是伴随点,我们称不可约闭子集$\overline{\{x\}}$是$x$的伴随分支(associated component).
\begin{enumerate}
	\item 仿射情况.伴随素理想在局部化下不变,所以如果$X=\mathrm{Spec}A$是仿射概形,那么它的伴随点就由伴随素理想构成.
	\item 我们知道诺特环上的极小素理想总是伴随素理想,于是对于局部诺特概形$X$,它的一般点总是伴随点.称局部诺特概形$X$的不是一般点的伴随点为嵌入点.例如局部诺特既约概形$X$上没有嵌入点,因为诺特既约环没有嵌入素理想(因为零理想可以表示为全部极小素理想的交,而伴随素理想肯定在这个表达式中).再比如$X=\mathrm{Spec}k[u,v]/(u^2,uv)$,那么$(u,v)=\mathrm{Ann}(u)$和$(u)=\mathrm{Ann}(v)$都是伴随点,其中$(u,v)$是嵌入点.
	\item 对开子概形$U\subseteq X$,明显有$\mathrm{Ass}(X)\cap U=\mathrm{Ass}(U)$.
	\item 因为诺特环上有限模的伴随素理想有限,导致$\mathrm{Ass}(X)$总是局部有限集.
	\item 仿射情况下伴随点和概形稠密开集.设$A$是环,设$U\subseteq X=\mathrm{Spec}A$是开子集,设理想$\mathfrak{a}$满足$X-U=V(\mathfrak{a})$.那么如下命题有(b)$\Rightarrow$(a)$\Rightarrow$(c)$\Rightarrow$(d).如果$A$是诺特环那么这些命题都是等价的.
	\begin{enumerate}
		\item $U$是概形稠密开集.
		\item $U$包含了一个主开集$D(t)$,其中$t\in A$不是零因子.等价的讲,$\mathfrak{a}$包含了一个元$t\in A$不是零因子.
		\item $\mathrm{Ann}(\mathfrak{a})=\{s\in A\mid s\mathfrak{a}=0\}$是零.
		\item $U$包含了$\mathrm{Ass}(A)$.
	\end{enumerate}
	\begin{proof}
		
		(a)等价于(d):我们要证明$A\to\mathscr{O}_X(U)$是单射当且仅当$\mathrm{Ass}(A)\subseteq U$.先证充分性,设$a\in A$使得$a\mid_U=0$,倘若$a\not=0$,那么非零模$aA$理应有伴随素理想,记作$\mathfrak{p}=\mathrm{Ann}(ab),b\in A$.但是在$A_{\mathfrak{p}}$里有$a=0$,所以存在$s\in A-\mathfrak{p}$使得$sa=0$,导致$s\in\mathrm{Ann}(ab)$矛盾,于是$a=0$,这证明充分性.对于必要性,假设存在一个伴随素理想$\mathfrak{p}=\mathrm{Ann}(a)\not\in U$.对$U$中的每个素理想$\mathfrak{q}$,有$\mathfrak{p}$包含了一个元$b$不在$\mathfrak{q}$中,这导致在$A_{\mathfrak{q}}$中有$a/1=ba/b=0$,于是$a\mid_U=0$.
		
		\qquad
		
		(d)和(c)等价是因为,如果$\mathfrak{a}$包含在某个伴随素理想$\mathfrak{p}=\mathrm{Ann}(a)$中,那么$a\mathfrak{a}=0$,反过来如果有某个$a\in A$满足$a\mathfrak{a}=0$,而包含$\mathrm{Ann}(a)$的形如$\mathrm{Ann}(x)$的真理想的极大元是伴随素理想,导致$V(\mathfrak{a})$包含了伴随素理想.
		
		\qquad
		
		(b)和(d)等价是因为,如果$\mathfrak{a}$包含了某个非零因子$t$,那么$\mathrm{Ann}(\mathfrak{a})=0$.反过来如果$\mathfrak{a}$由零因子构成,我们知道所有伴随素理想的并就是环上的所有零因子,于是$\mathfrak{a}$包含在一族伴随素理想的并中,但是环是诺特的,所以$\mathfrak{a}$包含在有限个伴随素理想的并中,按照prime avoidance引理,就有$\mathfrak{a}$包含在某个伴随素理想中,于是$V(\mathfrak{a})$包含了伴随素理想.
	\end{proof}
	\item 一般情况下伴随点和概形稠密开集.设$X$是局部诺特概形,设$U\subseteq X$是开子集,那么$U$是概形稠密开集当且仅当$\mathrm{Ass}(X)\subseteq U$.这件事是因为无论概形稠密开集还是伴随点集都是局部的,所以归结为上一条结论.
	\item 对于一般概形$X$,一个开集如果包含全部一般点,那么这个开集在$X$中稠密.逆命题一般不成立,但是对于局部诺特概形成立:如果$X$是局部诺特概形,那么稠密开集总包含全部一般点.特别的,这件事说明如果一个局部诺特概形$X$没有嵌入点,那么$\mathrm{Ass}(\mathscr{O}_X)\subseteq U$当且仅当$U$是稠密的.
	\begin{proof}
		
		设$X$是局部诺特概形,设$U$是稠密开集,假设有一般点$\eta\not\in U$.设$X$的全部不可约分支为$\{Z_i\mid i\in I\}$.设指标$i_0$满足$Z_{i_0}=\overline{\{\eta\}}$.我们断言$C=\cup_{i\not=i_0}Z_i$是$X$的闭子集,一旦这成立,那么从$\eta\in U^c$得到$Z_{i_0}\subseteq U^c$,得到$U\subseteq C$,这和$\overline{U}=X$矛盾.下面任取仿射开子集$V$,它是诺特环的素谱,我们有$V=\cup_i(V\cap Z_i)$,其中$V\cap Z_i$是不可约空间$Z_i$的开子集,所以不可约.按照诺特空间下不可约分解是有限并,所以存在$I$的有限子集$I_0$使得$V=\cup_{i\in I_0}(V\cap Z_i)$,于是$V\cap C=\cup_{i\in I_0-\{i_0\}}(V\cap Z_i)$是$V$的闭子集,这证明了$C$是闭集.
	\end{proof}
\end{enumerate}

有理映射就是具有极大定义域的态射.
\begin{itemize}
	\item 设$X$和$Y$是概型,考虑全体对$(U,f)$,其中$U$是$X$的概形稠密开子集,$f:U\to Y$是概型之间的态射,定义$(U,f)$和$(V,g)$等价如果存在概形稠密开子集$W\subseteq U\cap V$,使得$f\mid_W=g\mid_W$.这个等价关系的等价类称为$X\to Y$的一个有理映射(rational map).一个有理映射$\widetilde{f}$作为等价类的元素$(U,f)$称为这个有理映射的一个表示.
	\item 如果$X,Y$都是$S$概形,一个有理映射$f:X\to Y$称为$S$有理映射,如果存在一个表示$(U,f)$使得$f:U\to Y$是$S$态射.
	设$X$是$S$概形,$X$上的有理$S$函数是指一个$S$有理映射$X\to\mathbb{A}_S^1$,把$X$上的所有有理函数构成的集合记作$R(X)$.
\end{itemize}
\begin{enumerate}
	\item 如果$S$是分离概形,如果$\widetilde{f}:X\to Y$是一个$S$有理映射,那么它的每一个表示$(U,f)$都满足$f:U\to Y$是一个$S$态射.
	\begin{proof}
		
		按照$\widetilde{f}$是$S$有理映射,所以存在一个表示$(V,g)$满足$g:V\to Y$是$S$态射.下面任取另外一个表示$(U,f)$,要证明$\alpha:U\to Y\to S$和$\beta:U\to S$是相同的.但是按照$(U,f)\sim(V,g)$,说明存在概形稠密开集$W\subseteq U\cap V$使得$f\mid_W=g\mid_W$.那么按照$W$是概形稠密的,从$\alpha\mid_W=\beta\mid_W$就得到$\alpha=\beta$.
	\end{proof}
	\item 记$\mathrm{Rat}_S(X,Y)$表示所有$S$有理映射$\widetilde{f}:X\to Y$构成的集合,那么有典范映射$\mathrm{Hom}_S(X,Y)\to\mathrm{Rat}_S(X,Y)$.如果$S$是分离概形,那么这个典范映射是单射.
	\begin{proof}
		
		这个典范映射就是把一个$X\to Y$的$S$态射对应到等价类中.它是单射是因为如果$f,g:X\to Y$是两个$S$态射使得存在概形稠密开集$W\subseteq X$满足$f\mid_W=g\mid_W$,那么$S$是分离概形保证$f=g$,于是这个典范映射是单射.
	\end{proof}
	\item 开子概型上的限制.设$\widetilde{f}:X\to Y$是$S$有理映射,设$W\subseteq X$是开子概型,取$\widetilde{f}$的表示$(U,f)$,其中$f:U\to Y$是$S$态射,那么$U\cap W$是$W$的概形稠密开集,我们断言$(U\cap W,f\mid_{U\cap W})$所在的等价类不依赖于$(U,f)$的选取,这个等价类称为$\widetilde{f}$在$W$上的限制.
	\begin{proof}
		
		设$(V,g)$是$\widetilde{f}$的另一个表示,使得$g:V\to Y$是$S$态射,我们要证明$(U\cap W,f\mid_{U\cap W})$和$(V\cap W,g\mid_{V\cap W})$等价,而这是因为$(U,f)\sim(V,g)$导致存在概形稠密开集$W'\subseteq U\cap V$使得$f\mid_{W'}=g\mid_{W'}$,于是$f\mid_{W'\cap W}=g\mid_{W'\cap W}$,于是它们等价.
	\end{proof}
	\item 设$\widetilde{f}:X\to Y$是$S$有理映射,设$Y$是分离$S$概形,那么$\widetilde{f}$作为等价类存在唯一一个表示$(U,f)$使得$U$是所有表示中极大的概形稠密开集,它就是所有表示中的概形稠密开集的并,而$f$就是所有表示的粘合.我们称这个极大的$U$是$\widetilde{f}$的定义域.
	\begin{proof}
		
		我们只要证明不同的表示总可以粘合,换句话讲如果$(U,f)$和$(V,g)$是$\widetilde{f}$不同的表示,那么$f$和$g$在$U\cap V$上是一致的,而这是因为$Y$是分离$S$概形以及$U,U'$是概形稠密开集.
	\end{proof}
	\item 有理函数.设$X$是$S$概形,$X$上的有理$S$函数是指一个$S$有理映射$X\to\mathbb{A}_S^1$,把$X$上的所有有理函数构成的集合记作$R(X)$.我们知道$\mathrm{Hom}_S(U,\mathbb{A}_S^1)=\Gamma(U,\mathscr{O}_X)$,所以我们有如下公式,其中$U$跑遍所有概形稠密开集:
	$$R(X)=\lim\limits_{\rightarrow}\Gamma(U,\mathscr{O}_X)$$
	\item 我们解释过既约概形上概形稠密开集和稠密开集是一致的,而在整概形上非空开集和稠密开集是一致的,所以对于整概形,一个开集是概形稠密开集当且仅当它是非空开集.另外整概形上非空开集总包含唯一的一般点$\eta$,于是我们有$R(X)=\lim\limits_{\rightarrow}\Gamma(U,\mathscr{O}_X)=\mathscr{O}_{X,\eta}=K(X)$,此即整概形$X$上的函数域.
\end{enumerate}

双有理等价.一个有理$S$映射$\widetilde{f}:X\to Y$称为双有理的,如果它存在一个表示$(U,f)$,使得$f$是$U$到$Y$的某个概形稠密开集$V$的$S$同构.如果两个$S$概形之间存在双有理$S$映射,就称它们是双有理等价的.如果$S$态射$f:X\to Y$对应的$S$有理映射(此即它所在的等价类)是双有理的,就称$f$是双有理的,换句话讲态射$f$是双有理的当且仅当存在$X$的概形稠密开集$U$,使得$f\mid_U$是到$Y$的某个概形稠密开集$V$的同构.
\begin{enumerate}
	\item 如果$X$是整概形,那么它的每个非空开集都是概形稠密开集,并且包含了一般点$\eta$.如果$\widetilde{f}:X\to Y$是$S$有理映射,设$(U,f)$是它的一个表示,那么$f(\eta)$不依赖于表示的选取,这个点记作$\widetilde{f}(\eta)$.那么每个$f\in\mathrm{Rat}_S(X,Y)$,记$f(\eta)=y$,就对应了一个$S$态射$\mathrm{Spec}K(X)\to\mathrm{Spec}\kappa(y)$.
	\item 设$X,Y$是整$S$概形,它们的一般点分别记作$\eta,\theta$,设$\widetilde{f}:X\to Y$是$S$有理映射,那么如下命题互相等价,满足这个条件的$S$有理映射$\widetilde{f}$称为支配的(dominant).
	\begin{enumerate}
		\item 存在$\widetilde{f}$一个表示$(U,f)$,使得$f(U)$在$Y$中稠密.
		\item $f(\eta)=\theta$.
	\end{enumerate}
	\begin{proof}
		
		明显(b)推出(a).反过来如果存在$\widetilde{f}$的表示$(U,f)$使得$f(U)$是$Y$的稠密子集.那么任取$Y$的非空开集$V$,有$V\cap f(U)$是非空的,于是$f^{-1}(V\cap U)$是$X$的非空开集,所以包含了一般点$\eta$.我们断言$V$的全部非空开子集的交就是$\theta$:一方面$\theta$一定包含在$V$的所有非空开子集中因为它是一般点,另一方面任取$\theta\not=x\in V$,按照概形都是$T_0$空间,所以要么$\theta$存在开邻域不包含$x$,要么$x$存在开邻域不包含$\theta$,但是后者是不能成立的因为非空开集必须包含唯一一般点,所以必须是前者,所以$x$不能落在$V$的所有非空开子集的交,于是$V$的所有非空开子集的交就是$\{\theta\}$.于是我们有$\eta\in\cap_Uf^{-1}(V\cap U)=f^{-1}(\theta)$,这说明$f(\eta)=\theta$.
	\end{proof}
	\item 有理映射的复合.设$\widetilde{f}:X\to Y$和$\widetilde{g}:Y\to Z$是$S$有理映射,如果$X,Y,Z$都是不可约的,我们定义$\widetilde{g}\circ\widetilde{f}$是$X\to Z$的$S$有理映射如下:设$\widetilde{f}$的一个表示为$(U,f)$,其中$U\subseteq X$是概形稠密开集,$f:U\to Y$是$S$态射;设$\widetilde{g}$的一个表示为$(V,g)$,其中$V\subseteq Y$是概形稠密开集,$g:V\to Z$是$S$态射.我们定义$\widetilde{g}\circ\widetilde{f}$是$(U\cap f^{-1}(V),g\circ f)$所在的等价类(不可约条件用在$U\cap f^{-1}(V)$是非空的),而这不依赖于$\widetilde{f}$和$\widetilde{g}$的表示的选取.另外如果$\widetilde{f}$和$\widetilde{g}$是支配有理映射,那么它们的复合也是支配有理映射.
	\item 设$k$是域,考虑$k$上的整有限型概形和支配有理映射构成的范畴,我们断言这个范畴和$k$上的有限生成域扩张范畴(态射就是域扩张)是逆变范畴等价的.于是特别的,域$k$上两个有限型整概形是双有理等价的当且仅当它们的函数域是同构的.
	\begin{itemize}
		\item 我们把$k$上的一个整有限型概形$X$对应于它的函数域$K(X)$,那么这是域$k$的有限生成域扩张.我们解释过$k$上的整有限型概形之间的支配有理映射$f:X\to Y$把一般点打到一般点,所以诱导的局部环同态就是对应的$k$的有限生成扩张之间的扩张.这是一个逆变函子.
		\item 这个函子是完全忠实的.换句话讲对$k$上有限型整概形$X,Y$,我们有如下典范双射:
		\begin{align*}
			\{f\in\mathrm{Rat}_k(X,Y)\mid f\text{是支配的}\}&=\mathrm{Hom}_k(\mathrm{Spec}K(X),\mathrm{Spec}K(Y))\\&=\mathrm{Hom}_k(K(Y),K(X))
		\end{align*}
		
		对此我们来证明更一般的事:设$X$是整$k$概型,设$Y$是整的有限型$k$概型,给定函数域之间的$k$同态$\varphi^*:k(Y)\to k(X)$,那么存在$k$概型之间的支配有理映射$\varphi:X\to Y$诱导了这个同态$\varphi^*$.
		\begin{proof}
			
			用$Y$的仿射开子集$\mathrm{Spec}B$代替$Y$,这不改变$Y$的函数域,也不改变$X\to Y$的有理映射(作为等价类).按照定义可设$B$作为$k$代数由$y_1,y_2,\cdots,y_n$生成.为了构造诱导$\varphi^*$的支配有理映射,只需在每个仿射开子集$\mathrm{Spec}A$定义支配有理映射.换句话讲,我们不妨设$X,Y$都是仿射的.
			
			现在域扩张$\varphi^*:k(Y)=k(B)\to k(X)=k(A)$诱导了单同态$B\to k(A)$.记$y_i$在这个映射下的像是$f_i/g_i$,其中$f_i,g_i\in A$,记$g=\prod_ig_i$,那么$\varphi^*$诱导了单同态$B\to A_g$.它诱导了仿射概型之间的态射$\varphi:\mathrm{Spec}A_g\to\mathrm{Spec}B$,它诱导了$\varphi^*$.最后解释$\varphi$诱导了支配有理映射,换句话讲$B\to A_g$把零理想(一般点)映射为零理想(一般点),但是按照这是单同态直接得证.
		\end{proof}
		\item 这个函子是本质满的.设$k$是域,设$k\subseteq K$是有限生成域扩张.证明存在不可约仿射$k$代数簇使得它的函数域恰好是$K$.事实上,按照有限生成条件,设$K=k(x_1,x_2,\cdots,x_n)$,取同态$k[t_1,t_2,\cdots,t_n]\to K$为$t_i\mapsto x_i$,这个同态的核必然是一个素理想$p$,因为像集是整环.此时仿射$k$代数簇$k[t_1,t_2,\cdots,t_n]/p$函数域为$K$.
	\end{itemize}
	\item 整概型之间的有理映射逆变的诱导了函数域之间的域同态.但是反过来,给定整概型的函数域之间的逆变的同态未必对应于某个有理映射.例如整概型$\mathrm{Spec}k[x]$和$\mathrm{Spec}k(x)$具有相同的函数域$k(x)$,但是如果存在$\mathrm{Spec}k[x]\to\mathrm{Spec}k(x)$的有理映射,它具有某个代表元可以定义在主开集$D(f(x))$上,于是这个代表元对应于态射$\mathrm{Spec}k[x,1/f(x)]\to\mathrm{Spec}k(x)$,也即对应于环同态$k(x)\to k[x,1/f(x)]$,但是对任意$f(x)$都不存在这样的环同态.
\end{enumerate}

有理映射的例子.
\begin{enumerate}
	\item 毕达哥拉斯三元组是指$x^2+y^2=z^2$的整数解,或者等价于讲$x^2+y^2=1$上的有理点,或者$\mathbb{Q}$概型$X=\mathbb{Q}[x,y]/(x^2+y^2-1)$的$\mathbb{Q}$值点.一个平凡的$\mathbb{Q}$值点是$p=(1,0)$,假设$q$是另一个$\mathbb{Q}$值点,它映射到$pq$线的斜率$y/(x-1)$是$X\to\mathbb{A}_{\mathbb{Q}}^1$的有理映射.反过来给定一个有理数斜率$m$,直线$y=m(x-1)$和圆$x^2+y^2=1$的另外一个交点$(x,y)=(\frac{m^2-1}{m^2+1},\frac{-2m}{m^2+1})$是$\mathbb{Q}$值点.于是我们构造了从$\mathbb{A}_{\mathbb{Q}}^2$上单位圆$C$到$\mathbb{A}_{\mathbb{Q}}^1$的双有理等价.
	\item 现在我们把上述有理映射嵌入到$\mathbb{P}_{\mathbb{Q}}^1$中,也即复合映射$C\to\mathbb{A}_{\mathbb{Q}}^1\to\mathbb{P}^1_{\mathbb{Q}}$为$(x,y)\mapsto y/(x-1)\mapsto[y,x-1]$.这个有理映射定义在除去$p=(0,1)$以外的所有点.但是实际上可以延拓到点$p$:按照$[y,x-1]=[x+1.-y]$就可以延拓到点$p$.
	\item 换句话讲当域$k$的特征非2的时候,$\mathbb{P}_k^2$中的圆锥$x^2+y^2=z^2$和$\mathbb{P}_k^1$是同构的.另外我们会证明$\mathbb{P}_k^2$中的每个存在有理点的秩3二次曲线都同构于$\mathbb{P}_k^1$.
\end{enumerate}
\subsection{亚纯函数层}
\begin{enumerate}
	\item 对交换环$A$,记$R(A)$表示$A$的所有正则元(即非零因子)构成的乘性闭子集.记$\mathrm{Frac}(A)=R(A)^{-1}A$表示全商环(total ring of fractions).如果$A$是整环,全商环就是熟知的商域.
	\item 设$X$是环空间,对每个开集$U$,记$\mathscr{R}_X(U)=\{a\in\mathscr{O}_X(U)\mid a_x\in R(\mathscr{O}_{X,x}),\forall x\in U\}$.换句话讲$\mathscr{R}_X(U)$由$U$上那些在$U$的每个stalk上都是正则元的截面构成,特别的$\mathscr{R}_X(U)\subseteq R(\mathscr{O}_X(U))$.限制映射取为$\mathscr{O}_X$上的限制映射,我们断言这使得$\mathscr{R}_X$构成$\mathscr{O}_X$的子层.并且如果$X$是概形,$U$是仿射开子集,那么有$\mathscr{R}_X(U)=R(\mathscr{O}_X(U))$.
	\begin{proof}
		
		我们先解释限制映射是定义良性的,如果$a\in\mathscr{R}_X(U)$,我们要证明的是对开集$V\subseteq U$,有$a\mid_V\in\mathscr{R}_X(V)$.但是这是因为对$x\in V\subseteq U$有$(a\mid_V)_x=a_x$.进而有$\mathscr{R}_X$是$\mathscr{O}_X$的子预层,于是它已经满足粘合的唯一性.我们来证明$\mathscr{R}_X$满足粘合的存在性,换句话讲,如果存在开集的开覆盖$U=\cup_iU_i$,如果$a\in\mathscr{O}_X(U)$满足$a\mid_{U_i}\in\mathscr{R}_X(U_i)$,那么明显的对$x\in U$,设$x\in U_i$,那么有$a_x=(a\mid_{U_i})_x$是正则元,这说明$a\in\mathscr{R}_X(U)$.
		
		\qquad
		
		最后设$X$是概形,设$U=\mathrm{Spec}A$是仿射开子集,我们来证明$\mathscr{R}_X(U)=R(A)$.首先有$\mathscr{O}_X(U)=A$,如果$x\in R(A)$,那么对$A$的任意素理想$\mathfrak{p}$,有$x/1\in R(A_{\mathfrak{p}})$.这说明$R(A)\subseteq\mathscr{R}_X(U)$.反过来设$a\in A$,设$a$在每个$A_{\mathfrak{p}}$中是正则元,那么明显的$a$在$A$中也是正则元,这就得证.
	\end{proof}
	\item 设$X$是环空间,对每个开集$U$,取预层$\mathscr{K}'_X(U)=\mathscr{R}_X(U)^{-1}\mathscr{O}_X(U)$.限制映射取为$\mathscr{O}_X(U)\to\mathscr{O}_X(V)$诱导的$\mathscr{R}_X(U)^{-1}\mathscr{O}_X(U)\to\mathscr{R}_X(V)^{-1}\mathscr{O}_X(V)$.于是$\mathscr{K}'_X$是$X$上的包含$\mathscr{O}_X$的代数预层.它满足:
	\begin{enumerate}
		\item 如果$X$是概形,$U$是仿射的,那么$\mathscr{K}'_X(U)=\mathrm{Frac}(\mathscr{O}_X(U))$.
		\item 对每个开集$U\subseteq X$,典范映射$\mathscr{K}'_X(U)\to\prod_{x\in U}\mathscr{K}'_{X,x}$是单射.
		\begin{proof}
			
			设$s=a/b$,其中$a\in\mathscr{O}_X(U)$和$b\in\mathscr{R}_X(U)$,如果每个$x\in U$有$s_x=0$,那么$a_x=0,\forall x\in U$,于是$a=0$,于是$s=0$.
		\end{proof}
		\item 如果$X$是局部诺特的,那么$\forall x\in X$,有$\mathscr{K}'_{X,x}\cong\mathrm{Frac}(\mathscr{O}_{X,x})$.
		\begin{proof}
			
			我们有$\lim\limits_{\rightarrow}\mathscr{K}'_X(U)=\lim\limits_{\rightarrow}R_X(U)^{-1}\mathscr{O}_X(U)=\mathscr{R}_{X,x}^{-1}\mathscr{O}_{X,x}$.另外有$\mathscr{R}_{X,x}\subseteq R(\mathscr{O}_{X,x})$.问题归结为证$R(\mathscr{O}_{X,x})\subseteq\mathscr{R}_{X,x}$.任取$b_x\in R(\mathscr{O}_{X,x})$,取一个提升截面为$b\in\mathscr{O}_X(W)$,其中$W=\mathrm{Spec}A$是仿射的.记$I=\mathrm{Ann}(b)$,那么$I\mathscr{O}_{X,x}=0$,因为如果$a\in I$和$r/s\in\mathscr{O}_{X,x}$,那么$abr/s=0$,但是$b_x=b/1$是正则元导致$ar/s=0$.取$V=\{y\in X\mid x\in\overline{\{y\}}\}\cong\mathrm{Spec}\mathscr{O}_{X,x}$,那么$V\subseteq W$是仿射开子集,有$I\mathscr{O}_X(V)=0$.由于这里$A$是诺特的,所以$I$是有限生成的,于是此时有$R(\mathscr{O}_X(V))=\mathscr{R}_X(V)$.这就导致$b\mid_V\in\mathscr{R}_X(V)$,于是$b_x\in\mathscr{R}_{X,x}$.
		\end{proof}
	\end{enumerate}
	\item 设$X$是概形,定义预层$\mathscr{K}''_X$为对开子集$U\mapsto\mathrm{Frac}(\mathscr{O}_X(U))$.这个预层可能并不存在(可能存在非零因子$t\in\mathscr{O}_X(X)$,使得限制在某个$\mathscr{O}_X(U)$上是零因子,这导致$1/t\in\mathrm{Frac}(\mathscr{O}_X(X))$没法映入$\mathrm{Frac}(\mathscr{O}_X(U))$).但是在仿射开子集上是和$\mathscr{K}'_X$一致的.另外如果$X$是概形,$\mathscr{K}'_X$和$\mathscr{K}''_X$在仿射开子集上未必构成一个层(存在这样的仿射概形$X=\mathrm{Spec}A$,使得$\mathrm{Frac}(A)$是$\mathscr{K}_X(X)$的真子集).
	\item 设$X$是环空间,称$\mathscr{K}_X'$的层化(如果$X$是概形,这等价于讲是$\mathscr{K}''_X$在仿射开子集上限制的层化)为亚纯函数层,记作$\mathscr{K}_X$.称$\mathscr{K}_X(U)$中的元是$U$上的亚纯函数.
	\begin{enumerate}
		\item 如果$S\subseteq A$是由正则元构成的乘性闭子集$A\to S^{-1}A$是单射,所以典范态射$\mathscr{O}_X\to\mathscr{K}_X'$是单态射,所以它在stalk上处处是单射,于是诱导的$\mathscr{O}_X\to\mathscr{K}_X$是单态射.于是$\mathscr{O}_X$可视为$\mathscr{K}_X$的子层.
		\item 如果$X$是整概形,我们解释过每个非空开集的商域,每个stalk的商域都是相同的,于是此时亚纯函数层是一个常值层(不可约空间上常值预层本身就是常值层),于是整概形上的亚纯函数和有理函数等价.
		\item 即便$X$是诺特概形,也未必有$\mathscr{K}_X$是拟凝聚层.不过如果$X$是诺特既约概形,或者$X$是整概形,就有$\mathscr{K}_X$是拟凝聚$\mathscr{O}_X$模层.【】
		\item 如果$X$是局部诺特的,那么有$\mathscr{K}_{X,x}=\mathscr{K}_{X,x}'=\mathrm{Frac}(\mathscr{O}_{X,x})$.因为层化不改变stalk.
	\end{enumerate}
	\item 设$f\in\Gamma(X,\mathscr{K}_X)$是亚纯函数,记$\mathrm{dom}(f)=\{x\in X\mid f_x\in\mathscr{O}_{X,x}\subseteq\mathscr{K}_{X,x}\}$称为亚纯函数$f$的定义域.我们断言亚纯函数的定义域总是概形稠密开集.
	\begin{proof}
		
		记$U=\mathrm{dom}(f)$,任取$x\in U$,那么$f_x\in\mathscr{O}_{X,x}$,于是存在$x$的开邻域$W$和$g\in\mathscr{O}_X(W)$,使得$g\mid_W=f\mid_W$,于是$W\subseteq U$,于是$U\subseteq X$是开子集.
		
		\qquad
		
		任取$x\in X$,那么存在$x$的仿射开邻域$V=\mathrm{Spec}A$和一个正则元$g\in A$使得$gf\mid_U\in A$.那么$D(g)\subseteq U\cap V$,于是有$U\cap V$是概形稠密开集,取遍$V$为$X$的开覆盖得到$U$是概形稠密开集.
	\end{proof}
	\item 这一条开始我们用$\mathscr{R}_X$表示概形$X$上的有理函数层,换句话讲$\mathscr{R}_X(U)$表示$U$上的全体有理函数.那么上一条说明对任意开集$U$我们有典范同态$\alpha_U:\mathscr{K}_X(U)\to\mathscr{R}_X(U)$,这定义了一个层态射$\alpha:\mathscr{K}_X\to\mathscr{R}_X$.我们断言如果$X$是整概形或者$X$是局部诺特概形,那么$\alpha$是同构.
	\begin{proof}
		
		整概形的情况我们解释过了.下面设$X$是局部诺特概形,我们只需证明对任意仿射开子集$U=\mathrm{Spec}A$上有$\alpha_U$是双射.我们有$\mathscr{K}_X(U)'=\mathrm{Frac}(A)=\lim\limits_{\rightarrow}\Gamma(D(t),\mathscr{O}_X)$,这里$t$跑遍$A$的正则元;我们还有$\mathscr{R}_X(U)=\lim\limits_{\rightarrow}\Gamma(V,\mathscr{O}_X)$,其中$V$跑遍$\mathrm{Spec}A$的概形稠密开集.但是我们解释过诺特环的素谱上概形稠密开集恰好是包含$D(t)$,其中$t$是正则元的开集,这两个正向极限是相同的,于是对仿射开子集$U$总有$\mathscr{K}_X(U)'=\mathscr{K}_X(U)=\mathscr{R}_X(U)$.这得到$\alpha$是同构.
	\end{proof}
	\item 特别的,上一条的证明中解释了如果$X$是局部诺特概形,如果$U$是仿射开子集,那么有$\mathscr{K}_X(U)=\mathscr{K}_X'(U)=\mathrm{Frac}(\mathscr{O}_X(U))$.我们解释过这件事对一般概形是不成立的.
	\item 设$X$是局部诺特概形,设$U$是概形稠密开集(等价于讲$\mathrm{Ass}(\mathscr{O}_X)\subseteq U$),设$i:U\to X$是包含映射,那么典范映射$\mathscr{K}_X\to i_*\mathscr{K}_U$是同构.
	\begin{proof}
		
		因为$\mathrm{Ass}(\mathscr{O}_X)\subseteq U$,我们解释过$\mathscr{O}_X\to i_*\mathscr{O}_U$是单态射,于是它诱导的$\mathscr{K}'_X\to i_*\mathscr{K}_U'$也是单态射,于是层化$\mathscr{K}_X\to i_*\mathscr{K}_U$也是单态射.
		
		\qquad
		
		下面证明$\mathscr{K}_X\to i_*\mathscr{K}_U$是满态射.问题是局部的,不妨设$X=\mathrm{Spec}A$是仿射的.归结为证明$\mathscr{K}_X(X)\to\mathscr{K}_X(U)$是满射.我们先证明$\mathscr{K}'_X(X)\to\mathscr{K}'_X(U)$是满射.记$X-U=V(I)$,其中$I$是$A$的理想,那么$V(I)$不包含$A$的伴随素理想,但是$\mathrm{Ass}(A)$是有限集($A$是有限$A$模),于是$I\not\subseteq\cup_{\mathfrak{p}\in\mathrm{Ass}(A)}\mathfrak{p}$.设$a\in I-\cup_{\mathfrak{p}\in\mathrm{Ass}(A)}\mathfrak{p}$,那么$a$必然是正则元(因为零因子肯定包含在伴随素理想中),并且从$V(I)\subseteq V(a)$得到$D(a)\subseteq U$,并且有$\mathrm{Ass}(A)\subseteq D(a)$,这说明单态射的复合映射$\mathscr{K}'_X(X)\to\mathscr{K}_X'(U)\to\mathscr{K}_X'(D(a))$就是典范映射$\mathscr{K}_X'(X)\to\mathscr{K}_X'(D(a))$,但是由于$a$是正则元,这个映射是同构,这就说明了$\mathscr{K}'_X(X)\to\mathscr{K}_X'(U)$是满态射.
		
		\qquad
		
		下面证明$\mathscr{K}_X(X)\to\mathscr{K}_X(U)$是满射,只需证明$\mathscr{K}_X'(X)\to\mathscr{K}_X(U)$是满射.任取$s\in\mathscr{K}_X(U)$,存在$U$的仿射开覆盖$\{U_i\}$,使得$s\mid_{U_i}\in\mathscr{K}_X'(U_i)$,记$s\mid_{U_i}=a_i/b_i$,其中$b_i\in R(\mathscr{O}_X(U_i))$.记$V_i=D(b_i)\subseteq U_i$,记$V=\cup_iV_i$.我们有$\mathrm{Ass}(\mathscr{O}_X(U_i))\subseteq V_i$,于是$\mathrm{Ass}(\mathscr{O}_U)\subseteq V$.那么按照上一段,存在$t\in\mathscr{K}_X'(X)$使得$t\mid_V=s\mid_V$.因为$\mathscr{K}_X(U)\to\mathscr{K}_X(V)$是单射,就有$s=t\mid_U$,这就证明了$\mathscr{K}_X'(X)\to\mathscr{K}_X(U)$是满射.
	\end{proof}
\end{enumerate}

\newpage
\section{态射的基本类型}

一般来讲,相比直接处理对象本身,考虑态射能更本质的反应对象的性质.如果$P$是一个关于概型的性质,我们称一个概型之间的态射$\varphi:X\to Y$满足性质$P$,如果对$Y$的每个仿射开子集$V$,有$\varphi^{-1}(V)$满足性质$P$.我们主要关心的性质(例如拟紧,仿射,拟分离等等)都会满足仿射交换引理的条件,于是态射满足性质$P$等价于存在$Y$的一个仿射开覆盖$\{V_i\}$,使得每个$\varphi^{-1}(V_i)$满足性质$P$.当我们定义一种关于态射的性质时,我们期望这种性质满足如下三个要求:它在终端局部;它是保复合的;它是保基变换的,下一节会解释这个条件.和终端局部性质不同,我们会遇到很多不满足源端局部的性质.

终端局部性质,源端局部性质,仿射局部性质.一个关于态射的性质称为终端局部性质,如果对态射$\varphi:X\to Y$,它满足这个性质当且仅当对任意一个$Y$的开覆盖$\{V_i\}$,有$\varphi$限制在每个$\varphi^{-1}(V_i)$上满足这个性质.一个性质称为源端局部性质,如果对态射$\varphi:X\to Y$,它满足这个性质当且仅当对任意的$X$的开覆盖$\{U_i\}$,有$\varphi$限制在每个$U_i$上满足这个性质.如果一个关于态射的性质只需验证终端或者源端的一个仿射开覆盖,就称它是终端或者源端的仿射局部性质.例如单射,满射,双射,同胚,开映射,闭映射都是终端局部性质,但是通常只有开映射是源端局部性质.

另外如果关于态射的性质P是终端局部的,那么对任意概形$S$和任意$S$概形之间的态射$f:X\to Y$,有$f$满足P当且仅当对$S$的任意开覆盖$\{S_i\}$,都有$f$限制为$g^{-1}(S_i)\to h^{-1}(S_i)$满足性质P,这里$g,h$分别是$X,Y$作为$S$概形的结构态射.

\subsection{开嵌入}

我们已经给出了开嵌入的定义,现在来讨论它的性质.
\begin{enumerate}
	\item 开嵌入在终端局部.即如果$\varphi:X\to Y$是开嵌入,那么对每个开子集$V\subseteq Y$,有$\varphi$限制在$\varphi^{-1}(V)$上仍然是开嵌入;如果$\{V_i\}$是$Y$的开覆盖,那么$\varphi$是开嵌入当且仅当$\varphi$限制在每个$\varphi^{-1}(V_i)$上是开嵌入.
	\item 开嵌入是保复合的.
	\item 开嵌入在基变换下不变.换句话讲,如果$i:U\to Z$是开嵌入,对任意态射$p:Y\to Z$,纤维积$U\times_ZY$存在,它实际上就是$p^{-1}(U)$,此时基变换$U\times_ZY\to Y$是开嵌入.
	$$\xymatrix{T\ar@/^1pc/[drrrr]\ar[drr]\ar@/_1pc/[ddrr]&&&&\\&&p^{-1}(U)\ar[rr]\ar[d]&&Y\ar[d]^p\\&&U\ar[rr]^i&&Z}$$
	\item 开嵌入不是一个源端局部性质.例如我们考虑两个$\mathbb{A}^1$的无交并$X$,设两个$\mathbb{A}^1$上都是到$\mathbb{A}^1$的恒等态射,这得到了$X\to\mathbb{A}^1$的态射,但是它不是开嵌入,因为它甚至不是单射.
\end{enumerate}
\subsection{拟紧态射}

一个概型之间的态射$\varphi:X\to Y$称为拟紧的,如果$Y$的每个拟紧开子集的原像是拟紧的.概形上的拟紧开子集等价于讲可以表示为有限个仿射开子集的并.
\begin{enumerate}
	\item 关于局部性质.拟紧是一个终端局部性质,更强的,我们来证明一个态射$\varphi:X\to Y$的如下条件互相等价:
	\begin{enumerate}
		\item 拟紧开集的原像拟紧.
		\item 仿射开集的原像拟紧.
		\item 对任意$Y$的仿射开覆盖$\{V_i\}$,使得每个$\varphi^{-1}(V_i)$都是$X$的拟紧子集.
		\item 存在$Y$的仿射开覆盖$\{V_i\}$,使得每个$\varphi^{-1}(V_i)$都是$X$的拟紧子集.
		\item 对任意$Y$的开覆盖$\{V_i\}$,使得$\varphi$在每个$\varphi^{-1}(V_i)$上的限制都是拟紧态射.
		\item 存在$Y$的开覆盖$\{V_i\}$,使得$\varphi$在每个$\varphi^{-1}(V_i)$上的限制都是拟紧态射.
	\end{enumerate}
	\begin{proof}
		
		1推2推3推4是平凡的.下面2推1因为拟紧开集可取有限的仿射开覆盖,每个仿射开集的原像拟紧,有限个拟紧集合的并拟紧.验证4推2,任取$Y$的仿射开子集$V$,可取$V$的有限开覆盖$\{W_j\}$,使得每个$W_j$同时是$V$的和某个$W_j$的主开集.下面只需验证每个$f^{-1}(W_j)$是拟紧的.记$f^{-1}(V_i)$表示为有限个仿射开子集的覆盖$\{T_{ik},k\in K_i\}$.那么$f^{-1}(W_j)\cap T_{ik}$是$T_{ik}$的主开集,于是拟紧,于是这些有限个拟紧集的并是拟紧的.
		
		\qquad
		
		1推5推6是平凡的.下面6推2,任取$Y$的仿射开子集$V$,我们可取$V$上的主开集覆盖,使得每个主开集落在某个$V_i$中,按照$V$是拟紧的,这个主开集覆盖可取有限子覆盖,并且每个主开集的原像是拟紧的,这说明$V$的原像是有限个拟紧子集的并,于是是拟紧的.
	\end{proof}
    \item 关于复合.
    \begin{enumerate}
    	\item 拟紧态射是保复合的.
    	\item 如果$X\to Y\to Z$是拟紧的,$Y\to Z$是拟分离的,那么$X\to Y$是拟紧的.
    	\item 一个态射$f$是拟紧的当且仅当$f_{\mathrm{red}}$是拟紧的,这件事是因为拟紧态射实际上是一个拓扑性质.
    	\item 如果$X\to Y\to Z$是拟紧的,$X\to Y$是满射,那么$Y\to Z$是拟紧的.
    	\begin{proof}
    		
    		记$f:X\to Y$和$g:Y\to Z$,任取$Z$的拟紧开子集$W$,那么$f^{-1}\circ g^{-1}(W)$是$X$的拟紧开子集,按照$f$是满射有$g^{-1}(W)=f\circ f^{-1}\circ g^{-1}(W)$,但是拟紧子集的像还是拟紧的,这就得证.
    	\end{proof}
    \end{enumerate}
    \item 关于基变换.
    \begin{enumerate}
    	\item 拟紧态射是保基变换的.如果$f:X\to Y$是拟紧态射,对每个态射$g:S\to Y$,需要证明$p:X\times_YS\to S$是拟紧态射.取$Y$的仿射开覆盖$\{Y_i\}$,记$X_i=f^{-1}(Y_i)$,按照定义这是拟紧集,再记$g^{-1}(Y_i)=S_i$,那么$\{S_i\}$是$S$的开覆盖,于是按照拟紧是一个终端局部的性质,只需验证每个$X_i\times_{Y_i}S_i\to S_i$是拟紧的.于是我们不妨设$Y$本身是仿射的,再取$S$的仿射开覆盖,按照拟紧是一个终端局部性质,又不妨设$S$是仿射的,现在只要证明$X\times_YS$本身是拟紧的.但是现在$X$本身是拟紧的,它可以被有限个仿射开子集$\{X_i\}$覆盖,于是$X\times_YS$被有限个仿射开子集$\{X_i\times_YS\}$覆盖,这说明它是拟紧的.
    	\item 下降性质.设$f:X\to Y$是态射,设$X'$是它关于$g:Y'\to Y$的基变换,如果$g$是拟紧满射,那么$f':X'\to Y'$是拟紧态射可以推出$f$是拟紧态射.
    	\begin{proof}
    		
    		我们有$f\circ g'=g\circ f'$.如果$g$是拟紧满射,如果$f'$是拟紧的,那么$f\circ g'=g\circ f'$是拟紧的,但是按照$f\circ g'$拟紧和$g'$是满射,就得到$f$是拟紧的.
    	\end{proof}
    \end{enumerate}
    \item 以诺特概型为源端的态射总是拟紧的,这是因为诺特概型是诺特空间,其上所有子集都是拟紧的.
    \item 我们会证明从拟紧概型到拟分离概型的态射总是拟紧的.但是这个结论对拟紧概型之间的态射不成立:取$X=\mathrm{Spec}k[x_1,x_2,\cdots]$,取开子集$U=X-\{(x_1,x_2,\cdots)\}$,把两个$X$按照在$U$上的恒等态射粘合,得到了带两个点$(x_1,x_2,\cdots)$的概型$Z$,设两个$X$在$Z$中的像集是$X_1,X_2$,考虑$X\to Z$为$X$和$X_1$的恒等态射,这是一个开嵌入.按照有限个拟紧空间的并是拟紧的,得到$Z$是拟紧的.另外$Z$不是拟分离的,因为$X_1\cap X_2$不是拟紧的.最后这个开嵌入不是拟紧的:$X_2$的原像是$X$中的$U$,这不是拟紧子集.
    \item 拟紧态射和拟紧概形的联系.按照仿射概型上全空间是一个仿射开覆盖,说明概型$X$是拟紧的等价于$X$到终对象$\mathrm{Spec}\mathbb{Z}$的唯一结构态射是拟紧的.另外如果$X$是环$A$上的概形,那么$X$在$\mathrm{Spec}A$上拟紧等价于$X$在$\mathrm{Spec}\mathbb{Z}$上拟紧.这件事是因为,仿射概形之间的态射一定是拟紧的.
\end{enumerate}
\subsection{拟分离态射}

一个态射$\varphi:X\to Y$称为拟分离的,如果对$Y$的每个仿射开子集$U$,有$\varphi^{-1}(U)$是拟分离概型.
\begin{enumerate}
	\item 等价定义.一个态射$\varphi:X\to Y$是拟分离的当且仅当对角态射$\Delta_{Y/X}:X\to X\times_YX$是拟紧的.
	\begin{proof}
		
		充分性,任取$Y$的仿射开子集$W$,需要证明$\varphi^{-1}(W)$中的任意两个仿射开子集$U,V$,都有$U\cap V$是拟紧的.为此考虑如下图表,这是一个笛卡尔图(这是指它是纤维积图表),按照$\Delta$是拟紧态射,它的基变换$\Delta'$也是拟紧的态射.但是按照$U\times_WV$是仿射的,它的原像$U\times_XV=U\cap V$是拟紧的.
		$$\xymatrix{U\times_XV\ar[rr]^{\Delta'}\ar[d]&&U\times_WV\ar[d]\\X\ar[rr]^{\Delta}&&X\times_YX}$$
		
		必要性,任取$Y$的仿射开子集$\mathrm{Spec}A$,记$W=\varphi^{-1}(\mathrm{Spec}A)$,任取$W$的仿射开子集$U,V$,那么$U\times_YV=U\times_AV$覆盖了整个$X\times_YX$,现在$U\times_AV=p^{-1}(U)\cap q^{-1}(V)$在对角态射下的原像为$\Delta^{-1}\left(p^{-1}(U)\cap q^{-1}(V)\right)=U\cap V$.但是按照拟分离的定义,这里$U\cap V$是拟紧的.
	\end{proof}
	\item 拟分离是一个仿射终端局部性质.换句话讲,一个态射$\varphi:X\to Y$的如下条件互相等价:
	\begin{enumerate}
		\item $\varphi$是拟分离的.
		\item 对任意$Y$的仿射开覆盖$\{V_i\}$,有$\varphi$限制在每个$\varphi^{-1}(V_i)$上是拟分离的.
		\item 存在$Y$的仿射开覆盖$\{V_i\}$,有$\varphi$限制在每个$\varphi^{-1}(V_i)$上是拟分离的.
	\end{enumerate}
	\begin{proof}
		
		1推2推3是平凡的.下证3推1,需验证明的是对角态射$\Delta_{Y/X}:X\to X\times_YX$是拟紧的.记$\varphi^{-1}(V_i)=U_i$,那么$U_i\times_{V_i}U_i$构成了$X\times_YX$的开覆盖,这里$\Delta^{-1}(U_i\times_{V_i}U_i)=U_i$.按照拟紧是终端局部性质,说明只需验证$\varphi$限制在每个$U_i$上是拟紧的.但是这就是指$\varphi$限制在$U_i$上是拟分离的.
	\end{proof}
	\item 关于复合.
	\begin{enumerate}
		\item 拟分离态射的复合是拟分离的.
		\begin{proof}
			
			设$\varphi:X\to Y$和$\psi:Y\to Z$都是拟分离态射.考虑如下交换图,其中右侧方块图表是一个magic交换图,于是拟紧态射$\Delta$的基变换$\Delta'$是拟紧的,再按照拟紧态射复合是拟紧的,这说明$X\to X\times_ZX$是拟紧的,于是复合态射$X\to Z$是拟分离的.
			$$\xymatrix{X\ar[rr]&&X\times_YX\ar[rr]\ar[d]&&X\times_ZX\ar[d]\\&&Y\ar[rr]&&Y\times_ZY}$$
		\end{proof}
		\item 如果$X\to Y\to Z$是拟分离态射,那么$X\to Y$是拟分离态射.这件事是因为对角态射总是嵌入,而嵌入总是拟分离的.
		\item 如果$f:X\to Y$是拟分离的,那么$f_{\mathrm{red}}:X_{\mathrm{red}}\to Y_{\mathrm{red}}$是拟分离的.
	\end{enumerate}
	\item 关于基变换.
	\begin{enumerate}
		\item 拟分离态射是保基变换的.
		\begin{proof}
			
			如果如下图表是一个笛卡尔图表:
			$$\xymatrix{W\ar[rr]\ar[d]&&X\ar[d]\\Y\ar[rr]&&Z}$$
			
			那么对角态射构成了如下笛卡尔图表:
			$$\xymatrix{W\ar[rr]^{\Delta_{X/W}}\ar[d]&&W\times_XW\ar[d]\\Y\ar[rr]^{\Delta}_{Z/Y}&&Y\times_ZY}$$
			
			于是如果$Y\to Z$是拟分离的,那么$\Delta_{Z/Y}$是拟紧的,它的基变换$\Delta_{X/W}$是拟紧的,于是$W\to X$是拟分离的.
		\end{proof}
		\item 下降性质.设$f:X\to Y$关于$g:Y'\to Y$的基变换为$f':X'=X\times_YY'\to Y'$,如果$g$是拟紧满射,则从$f'$是拟分离态射可推出$f$是拟分离态射.
		\begin{proof}
			
			这件事是因为$\Delta_{f'}=\Delta_f\times_YY'$,并且我们解释过在$Y'\to Y$拟紧满射时有$\Delta_{f'}$拟紧可以推出$\Delta_f$拟紧.
		\end{proof}
	\end{enumerate}
    \item qcqs条件.如果一个概形或者一个态射是拟紧和拟分离的,就简记它是qcqs(quasi-compact,quasi-separated)的.于是一个概型$X$是qcqs概型等价于讲$X$可被有限个仿射开子集$\{U_i\}$覆盖,并且每个$U_i\cap U_j$可被有限个仿射开子集覆盖.
	\item 关于源端和终端添加条件.
	\begin{enumerate}
		\item 以拟分离概型为源端的态射都是拟分离的.
		\item 如果态射$f:X\to Y$满足$Y$是拟分离的,那么$f$是拟分离的当且仅当$X$是拟分离的.
		\item 以诺特概型为源端的态射总是qcqs的.
	\end{enumerate}
    \item 拟分离概形和拟分离态射.
    \begin{enumerate}
    	\item 一个概型$X$是拟分离的当且仅当它到终对象$\mathrm{Spec}\mathbb{Z}$的唯一结构态射是拟分离的.
    	\item 另外如果$X$是环$A$上的概形,那么$X$在$\mathrm{Spec}A$上拟分离当且仅当$X$在$\mathrm{Spec}\mathbb{Z}$上拟分离.这件事是因为仿射概形之间的态射一定是拟分离的(因为仿射概形之间的态射的对角态射是闭嵌入,而闭嵌入是拟紧的).
    	\item 我们解释过对拟紧概形也有上述两个性质,于是一个环$A$上的概形$X$在$\mathrm{Spec}A$上是qcqs的当且仅当$X$在$\mathrm{Spec}\mathbb{Z}$上是qcqs的.
    \end{enumerate}
    \item qcqs引理:如果$X$是qcqs概型,记$A=\mathscr{O}_X(X)$,那么对每个$f\in A$,有$\mathscr{O}_X(X_f)\cong A_f$.
    \begin{proof}
    	
    	先证限制映射$\mathscr{O}_X(X)\to\mathscr{O}_X(X_f)$把$f$映射为单位.归结为验证如果开集$U$和$f\in\mathscr{O}_X(U)$满足$\forall p\in U$有$f$在$\mathscr{O}_{X,p}$中是单位,则$f$是$\mathscr{O}_X(U)$中单位.为此取$U$的仿射开覆盖$\{U_i\}$,只需验证每个$f\mid_{U_i}$是$\mathscr{O}_X(U_i)$中单位,于是对应的逆元可以粘合为$\mathscr{O}_X(U)$中的元,于是$f$存在逆元.
    	
    	\qquad
    	
    	按照分式化的泛性质,限制映射诱导了一个环同态$A_f\to\mathscr{O}_X(X_f)$.它把$a/f^n$映射为$a'(f')^{-n}$.下面验证这个映射是同构.记$U_i=\mathrm{Spec}(A_i)$,记$f_i=f\mid_{U_i}$,那么$X_f\cap U_i=D(f_i)$.
    	
    	\qquad
    	
    	验证单射.如果$a'(f')^{-n}=0$,需要验证$a/f^n=0$,换句话讲有$f^ra=0$.这就归结为倘若$a\in\mathscr{O}_X(X)$满足$a\mid_{x_f}=0$,那么存在某个正整数$r$使得$f^ra=0$.
    	
    	\qquad
    	
    	$a\mid_{U_i}$在限制映射$\mathscr{O}_X(U_i)\to\mathscr{O}_X(U_i\cap X_f)$的核中,即$A_i\to (A_i)_{f_i}$的核中,于是有$f_i^{n_i}(s\mid_{U_i})=0$.按照条件$U_i$只有有限个,于是存在足够大的$n$使得$f_i^n(s\mid_{U_i})=0,\forall i$.于是层公理得到$f^ns=0$.
    	
    	\qquad
    	
    	验证满射.如果$t\in\mathscr{O}_X(X_f)$,需要验证存在整体截面$a$使得$t=a'(f')^{-n}$.即归结为证对任意$t\in\mathscr{O}_X(X_f)$,存在正整数$n$使得$f^nt$可延拓为一个整体截面.
    	
    	\qquad
    	
    	首先仿射情况可延拓说明,$t\mid_{U_i\cap X_f}$存在某个$n$使得$f^nt\mid_{U_i\cap X_f}$可延拓为$\mathscr{O}_X(U_i)$中的元$t_i$.考虑如下交换图,说明$t_i\mid_{U_i\cap U_j\cap X_f}=t_j\mid_{U_i\cap U_j\cap X_f}$.于是$t_i\mid_{U_i\cap U_j}-t_j\mid_{U_i\cap U_j}$限制在$X_f\cap U_i\cap U_j$上为零.按照条件$U_i\cap U_j$可被有限个仿射开子集覆盖,于是按照单射情况我们所证的结论,说明存在正整数$m_{ij}$使得$f^{m_{ij}}(t_i\mid_{U_i\cap U_j}-t_j\mid_{U_i\cap U_j})=0$.按照$i,j$只有有限个,所以可取足够大的正整数$m$使得$f^mt_i\mid_{U_i\cap U_j}=f^mt_j\mid_{U_i\cap U_j}$对任意$i,j$成立.最后将$f^mt_i$粘合为$s$,那么$s$延拓了$f^{m+n}t$.
    	$$\xymatrix{&\mathscr{O}_X(X_f)\ar[dl]\ar[dr]&\\\mathscr{O}_X(X_f\cap U_i)\ar[dr]&&\mathscr{O}_X(X_f\cap U_j)\ar[dl]\\&\mathscr{O}_X(X_f\cap U_i\cap U_j)&}$$
    \end{proof}
\end{enumerate}
\subsection{仿射态射}

一个态射$\varphi:X\to Y$称为仿射态射,如果$Y$的每个仿射开子集的原像都是仿射的.
\begin{enumerate}
	\item 仿射准则.概型$X$是仿射的当且仅当可找到有限个整体截面$f_1,f_2,\cdots,f_n\in\mathscr{O}_X(X)$生成了整个$\mathscr{O}_X(X)$,并且每个$(X_{f_i},\mathscr{O}_X\mid_{X_{f_i}})$都是仿射的.
	\begin{proof}
		
		必要性是直接的,记$X=\mathrm{Spec}(A)$,取$\{f_1,f_2,\cdots,f_n\}=\{1_A\}$,那么$X_{1_A}=D(1_A)=X$.
		
		充分性.记$A=\mathscr{O}_X(X)$,那么恒等映射$A\to\mathscr{O}_X(X)$诱导了概型的态射$\varphi:X\to\mathrm{Spec}(A)$.需要验证$\varphi$是同构.从$\{f_i\}$生成$A$得到$\cup_iD(f_i)=\mathrm{Spec}(A)$.我们解释过总有$\varphi^{-1}(D(f_i))\cong X_{f_i}$.一旦我们验证qcqs引理的条件,就得到$\mathscr{O}_X(X_{f_i})\cong A_{f_i}$.按照条件$X_{f_i}$是仿射的,于是$\varphi$诱导了同构$X_{f_i}\cong D(f_i)$.于是$\varphi$是同构.(相当于把$\varphi$的终端分解为若干子空间,每个子空间的原像到它是同构,这样$\varphi$本身必然是同构).
		
		最后验证qcqs引理的条件.首先$X$被有限个仿射开子集$\{X_{f_i},1\le i\le n\}$覆盖.记$X_{f_i}=\mathrm{Spec}(B_i)$,记$f_j'=f_j\mid_{X_{f_i}}$,那么$X_{f_i}\cap X_{f_j}=D(f_j')$是仿射的.
	\end{proof}
    \item 一个态射$\varphi:X\to Y$是仿射的当且仅当存在$Y$的仿射开覆盖$\{V_i\}$使得每个$f^{-1}(V_i)$是$X$的仿射开子集.特别的,这说明仿射态射是一个终端局部性质.
    \begin{proof}
    	
    	必要性平凡,下证充分性.任取$Y$的仿射开子集$U=\mathrm{Spec}(A)$,那么$U$存在主开集构成的有限仿射开覆盖$\{U_j\}$,并且每个$U_j$是某个$V_i$的主开集.记$U_j=D(f_j)$,其中$f_j\in A$,那么$f_1,f_2,\cdots,f_n$生成整个$A$,记$f^{-1}(U)=V$,记$f^{\#}(f_i)=g_i\in\mathscr{O}_X(V)$,那么$f^{-1}(U_j)=f^{-1}(D(f_j))=V_{g_i}$是仿射的,并且从$f_j$生成$\mathscr{O}_Y(U)$得到$g_j$生成$\mathscr{O}_X(V)$,于是按照仿射准则知$V$是仿射的.
    \end{proof}
    \item 仿射态射的复合自然是仿射的.
    \item 仿射态射的基变换是仿射态射.设$\varphi:X\to Y$是仿射态射,任取态射$\psi:S\to Y$,那么基变换$X\times_YS\to S$是仿射态射.为此取$Y$的仿射开覆盖$\{Y_i,i\in I\}$,那么每个$Y_i$在$\varphi$下的原像$X_i$是仿射的,取$Y_i$在$\psi$下的原像的仿射开覆盖$\{S_{ij},j\in J_i\}$,那么$\{S_{ij},j\in J_i,i\in I\}$是$S$的仿射开覆盖.这里每个$S_{ij}$在基变换态射下的原像$X_i\times_{Y_i}S_{ij}$是仿射的,于是基变换态射是仿射的.
    \item 例如闭嵌入是仿射态射,开嵌入未必是仿射态射.另外如果$Y\to X$是$X$概形,其中$X$是仿射概形,那么$Y\to X$是仿射态射当且仅当$Y$是仿射概形.
    \item 设$f:X\to Y$是qcqs态射,我们解释过$\mathscr{A}(Y)=f_*\mathscr{O}_X$是一个拟凝聚$\mathscr{O}_Y$代数层,我们解释过有如下典范双射,设$f_*\mathscr{O}_X$上的恒等态射对应的$Y$态射为$c_X:X\to\mathrm{Spec}f_*\mathscr{O}_X$.
    $$\mathrm{Hom}_Y(X,\mathrm{Spec}f_*\mathscr{O}_X)\cong\mathrm{Hom}_{\textbf{Alg}(\mathscr{O}_X)}(f_*\mathscr{O}_X,f_*\mathscr{O}_X)$$
    
    我们断言一个态射$f:X\to Y$是仿射的当且仅当$f$是拟紧和分离的态射,并且$c_X$是同构.
    \begin{proof}
    	
    	充分性,此时$f_*\mathscr{O}_X$是$Y$上的拟凝聚代数层,而$f$是$\mathrm{Spec}f_*\mathscr{O}_X\to Y$,而我们解释过对$Y$的任意仿射开子集$V$,有$f^{-1}(V)\cong\mathrm{Spec}\Gamma(V,f_*\mathscr{O}_X)$,于是$f$是仿射态射.必要性,首先仿射态射一定是拟紧和分离的,拟紧因为它仿射开子集的像是拟紧的,分离是因为仿射概形之间的态射的对角态射一定是闭嵌入.下面证明$c_X$是同构,同构是一个终端局部性质,所以归结为设$Y=\mathrm{Spec}A$是仿射的情况,但是此时有$X=f^{-1}(Y)$也是仿射的,记作$X=\mathrm{Spec}B$,此时明显有$c_X$是同构.
    \end{proof}
    \item 关于仿射态射还有一种观点,如果$Y$是$X$概形,使得结构态射$Y\to X$是仿射态射,那么$Y$一定可以实现为某个$\mathscr{O}_X$拟凝聚代数层$\mathscr{B}$的素谱,并且具有相应函子性.见拟凝聚代数层的素谱.
\end{enumerate}
\subsection{(局部)有限型态射}

设$f:X\to Y$是概形之间的态射,称$f$在点$x\in X$处是有限型的,如果存在$y=f(x)$的仿射开邻域$V=\mathrm{Spec}A$和$x$的仿射开邻域$U=\mathrm{Spec}B$,满足$f(U)\subseteq V$,并且$B$是一个有限型$A$代数.如果$f$在$X$的每个点都说有限型的,就称$f$是局部有限型态射(locally of finite type).称拟紧的局部有限型态射为有限型态射.
\begin{enumerate}
	\item 关于局部性质.
	\begin{enumerate}
		\item 我们解释过环上局部有限型概型满足仿射交换引理的条件,于是局部有限型态射定义中的:原像$\varphi^{-1}(\mathrm{Spec}B)$中的每个仿射开子集$\mathrm{Spec}A$,诱导的结构同态$B\to A$都使得$A$是有限型$B$代数,等价于讲原像$\varphi^{-1}(\mathrm{Spec}B)$存在一个仿射开覆盖$\{U_j=\mathrm{Spec}A_j\}$,使得诱导的结构同态$B\to A_j$都使得$A_j$是有限型$B$代数.
		\item 局部有限型是仿射终端局部性质,也是终端局部性质,也是源端局部性质.而有限型态射只是仿射终端局部的和终端局部的.以局部有限型态射是仿射终端局部性质为例,此即$Y$存在仿射开覆盖$\{V_i=\mathrm{Spec}(B_i)\}$,满足$\varphi$在每个$\varphi^{-1}(V_i)$的限制都是局部有限型态射.
		\begin{proof}
			
			任取$Y$的仿射开子集$V=\mathrm{Spec}B$,取$V$的仿射开覆盖$S$,使得$S$中每个开集同时是$V$的和某个$V_i$的主开集.任取$W=D(g)=D(g_i)$,其中$g\in B$和$g_i\in B_i$.现在$\varphi^{-1}(W)$被$\{U_{ij},j\in J_i\}$覆盖.$\varphi^{-1}(W)\cap U_{ij}$是$U_{ij}$的主开集$D(f_{ij})$,这里$f_{ij}\in A_{ij}$实际上就是$\varphi^{\#}(g_i)$.$\varphi^{-1}(V)$可被$\{\varphi^{-1}(W)\cap U_{ij}\mid W\in S,j\in J_i\}$覆盖.每个$\varphi^{-1}(W)\cap U_{ij}$的整体截面环是$(A_{ij})_{f_{ij}}$.从$B_i\to A_{ij}$有限型得到$(B_i)_{g_i}\to(A_{ij})_{f_{ij}}$是有限型,从而每个$B_g\to (A_{ij})_{f_{ij}}$有限型,于是$B\to(A_{ij})_{f_{ij}}$是有限型的.
		\end{proof}
		\item 整理一下,态射$\varphi:X\to Y$是局部有限态射,如果它满足如下等价条件中的任一个:
		\begin{itemize}
			\item $Y$存在仿射开覆盖$\{V_i=\mathrm{Spec}(B_i)\}$,并且每个$\varphi^{-1}(V_i)$存在仿射开覆盖$\{U_{ij}=\mathrm{Spec}(A_{ij})\}$.使得$A_{ij}$都是有限型$B_i$代数.
			\item 对$Y$的每个仿射开子集$V=\mathrm{Spec}(B)$,存在$\varphi^{-1}(V)$的仿射开覆盖$\{U_i=\mathrm{Spec}(A_i)\}$,满足$A_i$都是有限型$B$代数.
			\item 对$Y$的每个仿射开子集$V=\mathrm{Spec}(B)$和每个满足$U\subseteq\varphi^{-1}(V)$的$X$的仿射开子集$X=\mathrm{Spec}(A)$,总有$A$是有限型$B$代数.
		\end{itemize}
		\item 类似的,$\varphi:X\to Y$是有限型态射有如下互相等价的描述:
		\begin{itemize}
			\item $\varphi$是拟紧和局部有限型态射.
			\item 对每个仿射开子集$V=\mathrm{Spec}(B)\subseteq Y$,存在$\varphi^{-1}(V)$的有限仿射开覆盖$\{U_i=\mathrm{Spec}(A_i)\}$,使得每个$A_i$都是有限型的$B$代数.
			\item 存在$Y$的仿射开覆盖$\{V_j=\mathrm{Spec}B_j\}$,每个$\varphi^{-1}(V_j)$存在有限的仿射开覆盖$\{U_{ij}=\mathrm{Spec}(A_{ij}),i\in I_j\}$,使得每个$A_{ij}$都是$B_j$有限型代数.
		\end{itemize}
	\end{enumerate}
    \item 仿射情况.设环同态$\varphi:A\to B$对应的态射为$f:\mathrm{Spec}B\to\mathrm{Spec}A$,那么$f$是局部有限表示态射当且仅当它是有限表示态射,当且仅当$\varphi$使得$B$是有限型$A$代数.
	\item 关于复合.
	\begin{enumerate}
		\item 局部有限型态射的复合是局部有限型态射,有限型态射的复合是有限型态射.
		\item 如果$X\to Y\to Z$是局部有限型态射,那么$X\to Y$是局部有限型态射.这件事是因为任意态射的对角态射都是嵌入,而嵌入总是局部有限型态射.
		\item 如果$X\to Y\to Z$是有限型态射,并且$Y\to Z$是拟分离的,那么$X\to Y$是有限型态射.
		\item 如果$f$是局部有限型态射或者有限型态射,那么$f_{\mathrm{red}}$也是.
	\end{enumerate}
	\item 关于基变换.局部有限型是保基变换的,有限型是保基变换的.
	\item 和诺特概形的联系.
	\begin{enumerate}
		\item 如果态射$f:X\to Y$是局部有限型态射,$Y$是局部诺特概型,那么$X$是局部诺特概型.如果$f$是有限型态射,$Y$是诺特概型,那么$X$是诺特概型.这件事是因为希尔伯特基定理,即诺特环上的有限生成代数是诺特环.
		\item 推论.设$f:X\to Y$是局部有限型态射,设$Y'\to Y$是一个源端为局部诺特概形的态射,那么$X\times_YY'$是局部诺特概形.因为基变换$X\times_YY'\to Y'$是局部有限型态射.
		\item 推论.设$f:X\to Y$是有限型态射,设$Y'\to Y$是源端为诺特概形的态射,那么$X\times_YY'$是诺特概形.
		\item 设$S$是局部诺特概形,设$X\to S$是有限型态射,那么任意$S$态射$f:X\to Y$都是有限型态射.
		\begin{proof}
			
			首先从$(X\to S)=(X\to Y\to S)$是局部有限型的可推出$X\to Y$是局部有限型的,问题只剩证明$X\to Y$是拟紧的,这是终端局部的性质,于是不妨设$S$本身是诺特概形,那么按照$X\to Y$是有限型的,就得到$X$是诺特概形,于是$X\to Y$一定是拟紧态射.
		\end{proof}
	\end{enumerate}
	\item 关于嵌入.嵌入总是局部有限型态射,开嵌入可能不是拟紧的,但是闭嵌入一定是拟紧的从而是有限型态射.
    \item 设$X,Y$是$S$概形,设$y\in Y$在$S$中的像是$s$,设$Y$在点$s$是局部有限型的(此即存在$s$的仿射开邻域$V=\mathrm{Spec}A$和$y$的仿射开邻域$U=\mathrm{Spec}B$,使得$U$在结构态射下打到$V$内,并且$B$是有限型$A$代数).如果$f,g:X\to Y$是$S$态射,满足$f(x)=g(x)=y$,并且它们诱导的$\mathscr{O}_{Y,y}\to\mathscr{O}_{X,x}$相同,那么存在$x$的仿射开邻域上使得$f=f'$.
    \begin{proof}
    	
    	设$W=\mathrm{Spec}A$是点$x$的仿射开邻域,使得$f(W),g(W)\subseteq U$.设$\mathfrak{p},\mathfrak{q},\mathfrak{r}$分别对应于$x,y,z$在$C,B,A$中的素理想.态射$f,g$限制在$W$上对应了两个$A$代数同态$\varphi,\psi:B\to C$,并且满足$\varphi^{-1}(\mathfrak{p})=\psi^{-1}(\mathfrak{p})=\mathfrak{q}$.条件要求$\varphi$和$\psi$诱导了相同的同态$B_{\mathfrak{q}}\to C_{\mathfrak{p}}$.设$A$代数$B=A[b_1,\cdots,b_n]$.那么在$C_{\mathfrak{p}}$中有$\varphi(b_i)/1=\psi(b_i)/1$,于是存在$s_i\in C-\mathfrak{p}$使得$s_i\varphi(b_i)=s_i\psi(b_i)$.记$s=s_1\cdots s_n$,那么在$C_s$中对每个$i$都有$\varphi(b_i)/1=\psi(b_i)/1$.于是$\varphi$和$\psi$诱导了相同的$B\to C_s$,于是$f,g$限制在$x$的开邻域$D(s)\subseteq W$上是相同的.
    \end{proof}
    \item 设$X,Y$是$S$概形,设$f:X\to Y$是$S$概形,设$x\in X$,$y=f(x)$,并且$x,y$在$S$中的像是$s$,如果$X\to S$在点$x$是局部有限型的,那么$X\to Y$在点$x$也是局部有限型的.
    \begin{proof}
    	
    	因为如果$A\to B\to C$是环同态的复合,如果$C$是$A$上有限型代数,那么$C$也是$B$上有限型态射.
    \end{proof}
    \item 关于本质有限型.设$A$是环,一个$A$代数$B$称为本质有限型的(essentially of finite type),如果存在一个有限型$A$代数$C$,和$C$的一个乘性闭子集$S$,使得$B$是$A$代数同构于$S^{-1}C$的.
    \begin{enumerate}
    	\item 如果$B$是本质有限型$A$代数,$C$是本质有限型$B$代数,那么$C$也是本质有限型$A$代数.
    	\begin{proof}
    		
    		按照定义,存在有限型$A$代数$B'$以及它的乘性闭子集$S'$满足$B={S'}^{-1}B'$,也存在有限型$B$代数$C'$以及它的乘性闭子集$T'$满足$C={T'}^{-1}C'$.那么存在有限型$B'$代数$C''$使得$C'={S'}^{-1}C''$(因为比方说设$C'=B[T_1,\cdots,T_n]/I={S'}^{-1}B'\otimes_{B'}B'[T_1,\cdots,T_n]/I'$,那么取$C''=B'[T_1,\cdots,T_n]/I'$就是有限型$B'$代数,并且满足$C'={S'}^{-1}C''$).则$C$也是$C''$的分式化,又因为$C''$也是有限型$A$代数,这就得证.
    	\end{proof}
        \item 基变换.设$B$是本质有限型$A$代数,设$A'$是任意$A$代数,那么$B'=B\otimes_AA'$是本质有限型$A'$代数.
        \begin{proof}
        	
        	按照定义,存在有限型$A$代数$C$和乘性闭子集$S\subseteq C$使得有$A$代数同构$S^{-1}C\cong B$,进而有$A'$代数同构$S^{-1}(C\otimes_AA')\cong S^{-1}C\otimes_AA'\cong B'$.
        \end{proof}
        \item 如果$B$是本质有限型$A$代数,并且$B$是局部环,那么存在有限型$A$代数$C$,以及一个素理想$\mathfrak{q}\subseteq C$,使得有$A$代数同构$B\cong C_{\mathfrak{q}}$【】.如果记$\mathfrak{q}\subseteq C$在$A$中的原像为$\mathfrak{p}$,再记$S=A-\mathfrak{p}$,那么$C_{\mathfrak{q}}$也是$S^{-1}C$的分式化,又因为$S^{-1}C$也是有限型$A_{\mathfrak{p}}$代数,于是$B=C_{\mathfrak{q}}$也是$A_{\mathfrak{p}}$本质有限型代数,并且此时$A_{\mathfrak{p}}\to B$是局部同态.
        \item 设$B$是局部环,也是本质有限型$A$代数,那么有$A$代数同构$B\cong C_{\mathfrak{q}}/I$,其中$C=A[T_1,\cdots,T_n]$,$\mathfrak{q}\subseteq C$是素理想.
        \begin{proof}
        	
        	我们解释了$B$同构于$C'_{\mathfrak{q}'}$,其中$C'$是一个有限型$A$代数,$\mathfrak{q}'\subseteq C'$是一个素理想.那么可记$C'=A[T_1,\cdots,T_n]/J$,按照分式化和商可交换,就有$C'_{\mathfrak{q}'}\cong C_{\mathfrak{q}}/I$.
        \end{proof}
    \end{enumerate}
\end{enumerate}
\subsection{整态射}

概型之间的态射$f:X\to Y$称为整态射,如果它是仿射态射,并且对$Y$的每个仿射开子集$\mathrm{Spec}A$,设它的原像为$\mathrm{Spec}B$,那么诱导的结构同态$A\to B$是一个整同态.
\begin{enumerate}
	\item 整态射是一个终端局部性质,并且满足仿射交换引理,于是对态射$f:X\to Y$如下条件互相等价,并且在命题成立时称$f$是整态射.
	\begin{enumerate}
		\item 对$Y$的每个仿射开子集$\mathrm{Spec}A$,它在$f$下的原像是仿射的,如果记作$\mathrm{Spec}B$,那么$A\to B$是整同态.
		\item 对任意$Y$的仿射开覆盖$\{V_i=\mathrm{Spec}A_i\}$,每个$\mathrm{Spec}A_i$在$f$下的原像是仿射的,如果记作$\mathrm{Spec}B_i$,那么诱导的结构同态$A_i\to B_i$都是整同态.
		\item 存在$Y$的仿射开覆盖$\{V_i=\mathrm{Spec}A_i\}$,每个$\mathrm{Spec}A_i$在$f$下的原像为$\mathrm{Spec}B_i$,那么诱导的结构同态$A_i\to B_i$是整同态.
	\end{enumerate}
	\begin{proof}
		
		我们已经解释过仿射态射是一个仿射终端局部性质,于是1推2推3是平凡的.3推1归结为证明仿射交换引理,也即如下两个交换代数事实:设$\varphi:A\to B$是环同态,
		\begin{itemize}
			\item 如果$A\to B$是整同态,那么诱导的$A_a\to B_{\varphi(a)}$是整同态.
			\item 设$A=a_1A+\cdots+a_nA$,如果每个诱导的$A_{a_i}\to B_{\varphi(a_i)}$是整同态,那么$A\to B$是整同态.
		\end{itemize}
	\end{proof}
    \item 整同态的复合是整同态,于是整态射的复合是整态射.
    \item 整态射在基变换下不变.问题归结为仿射情况,也即如果$A\to B$是整同态,对任意同态$A\to C$,证明$C\to C\otimes_AB$是整同态.此时按照$B$在$A$上整,得到每个$b\otimes1$在$C$中整,于是$c(b\otimes1)=b\otimes c$在$C$中整.
    \item 闭嵌入一定是整态射,于是如果$X\to Y\to Z$是整态射,如果$Y\to Z$是分离的,那么$X\to Y$是整态射.
    \item 概型之间的整态射是一个闭映射.事实上假设$f:X\to Y$是整态射,任取$Y$的仿射开覆盖$\{V_i=\mathrm{Spec}A_i\}$,设$V_i$在$f$下的原像是$U_i=\mathrm{Spec}B_i$.按照闭性是一个局部性质,只需验证$f$在每个$U_i$上的限制是闭映射.换句话讲我们只需验证仿射概型之间的整态射是闭映射.设$f:\mathrm{Spec}B\to\mathrm{Spec}A$,设对应的环同态为$\varphi:A\to B$.任取$\mathrm{Spec}B$的闭子集$V(J)$,这里$J$是$B$中的理想,记$I=\varphi^{-1}(J)$,那么$\varphi$诱导了环同态$\widetilde{\varphi}:A/I\to B/J$,这是单射并且是整同态,于是它满足提升条件,也即它诱导了$V(J)\to V(I)$的满射,于是$f(V(J))=V(I)$是闭集.
    \item 我们能证明更强的事情:整态射等价于泛闭的仿射态射.
    \begin{proof}
    	
    	按照定义整态射是仿射的,另外上一条解释了整态射是闭映射,但是整态射的基变换还是整态射,于是整态射是泛闭的,这解决了必要性.对于充分性,归结为仿射的情况,设$f:\mathrm{Spec}A\to\mathrm{Spec}R$是泛闭的态射,设$a\in A$,我们要证明的是$a$在$R$上整,换句话讲证明$R[X]\to A$,$X\mapsto a$的核$I$包含了首一多项式.记$B=A[X]/(aX-1)$,设$R[X]\to A[X]\to B$的核为$J$,如果$f\in J$,那么存在$q\in A[X]$使得$f=(aX-1)q$.如果记$f=\sum_if_iX^i$和$q=\sum_iq_iX^i$,那么有$f_i=aq_{i-1}-q_i$.设$n\ge\deg q+1$,那么有$\sum_if_iX^{n-i}=\sum_i(aq_{i-1}-q_i)X^{n-i}=(a-x)\sum_iq_iX^{n-i-1}$落在$I$中.并且如果$f_0=1$则这个多项式是首一的,所以归结为证明$J$包含了一个常数项为1的多项式.等价于讲有$R[X]=J+(X)$,等价于证明$\mathrm{Spec}R[X]/(J+(X))$是空集.
    	
    	\qquad
    	
    	因为$f$是泛闭的,所以基变换$\mathrm{Spec}A[X]\to\mathrm{Spec}R[X]$是闭映射.于是$\mathrm{Spec}B\subseteq\mathrm{Spec}A[X]$的像集就是闭子集$\mathrm{Spec}R[X]/J\subseteq\mathrm{Spec}R[X]$,特别的有$\mathrm{Spec}B\to\mathrm{Spec}R[X]/J$是满射.我们有如下纤维积图表,这里左下角是空集是因为$R\otimes_{R[X]}B$是零环,而这是因为$R[X]\to B$把$X$映为可逆元,但是$R[X]\to R$把$X$映为零.于是从$g$是满射得到提升$g'$也是满射,这导致$\mathrm{Spec}R[X]/(J+(X))$是空集.
    	$$\xymatrix{\mathrm{Spec}B\ar[rr]^g&&\mathrm{Spec}R[X]/J\ar[rr]&&\mathrm{Spec}R[X]\\\emptyset\ar[rr]_{g'}\ar[u]&&\mathrm{Spec}R[X]/(J+(X))\ar[rr]\ar[u]&&\mathrm{Spec}R\ar[u]_0}$$
    \end{proof}
\end{enumerate}
\subsection{有限态射}

概型之间的态射$f:X\to Y$称为有限态射,如果它是仿射态射,并且对每个$Y$的仿射开子集$\mathrm{Spec}A$,设它在$f$下的原像是$\mathrm{Spec}B$,那么$A\to B$使得$B$是有限$A$模.
\begin{enumerate}
	\item 有限态射是一个终端局部性质,也满足仿射交换引理.于是对态射$f:X\to Y$有如下条件互相等价,并且在条件成立时称$f$是有限态射.
	\begin{enumerate}
		\item 对$Y$的每个仿射开子集$V=\mathrm{Spec}A$,有$f^{-1}(V)=\mathrm{Spec}B$是仿射的,并且$B$是有限$A$模.
		\item 对任意$Y$的仿射开覆盖$\{V_i=\mathrm{Spec}A_i\}$,有每个$f^{-1}(V_i)$都是$X$的仿射开子集,记作$\mathrm{Spec}B_i$,并且有$B_i$都是有限$A_i$模.
		\item 存在$Y$的仿射开覆盖$\{V_i=\mathrm{Spec}A_i\}$,使得每个$f^{-1}(V_i)$都是$X$的仿射开子集,记作$\mathrm{Spec}B_i$,并且$B_i$都是有限$A_i$模.
	\end{enumerate}
    \begin{proof}
    	
    	1推2推3平凡,下证3推1,我们之前证明过此时$f^{-1}(V)$是仿射的.取$V$的开覆盖$S$,使得$S$中每个开集同时是$V$和某个$V_i$的主开集.取$W\in S$,记$W=D(g)=D(g_i)$,其中$g\in A$,$g_i\in A_i$.那么$f^{-1}(W)$是$f^{-1}(V)$的主开集$D(f^{\#}(g_i))$.条件是$B_i$是有限$A_i$模,于是$(B_i)_{g_i}$是有限$(A_i)_{g_i}$模.于是$B_g$是有限$A_g$模,这里$g$跑遍一个生成单位理想的$A$的有限子集,这可推出$B$是有限$A$模.
    \end{proof}
    \item 有限态射的复合是有限态射.
    \item 有限态射的基变换是有限态射.这归结为仿射情况,如果$A\to B$使得$B$是有限$A$模,那么对任意环同态$A\to C$,都有$C\to B\otimes_AC$使得$B\otimes_AC$是有限$C$模.
    \item $f:X\to Y$是有限态射当且仅当它是仿射态射,并且$f_*\mathscr{O}_X$作为拟凝聚$\mathscr{O}_Y$模层是有限生成的.这件事是因为归结到仿射情况,有$B$是有限$A$模当且仅当$\widetilde{B}$是$\mathrm{Spec}A$上的有限生成模层.
    \item 闭嵌入一定是有限态射,于是如果$X\to Y\to Z$是有限态射,$Y\to Z$是分离态射,那么$X\to Y$是有限态射.
    \item 设$k$是域,态射$X\to\mathrm{Spec}k$是有限态射,那么$X$是有限的离散空间,每个点的剩余类域都是$k$的有限扩张.
    \begin{proof}
    	
    	首先有限态射是仿射的,于是$X$是仿射概型,记对应的环是$A$.从$A$是有限$k$模得到$A$是零维环,于是它是有限个点构成的离散空间.最后因为$A/\mathfrak{p}$总是$k$上有限模,得到$k\subseteq\kappa(\mathfrak{p})$是有限扩张.
    \end{proof}
    \item 有限态射的每个纤维集合都是有限集合,这个条件今后我们称为具有有限纤维.按照有限态射是仿射的,问题归结为证明仿射概型之间的有限态射的每个纤维都是有限集合.但是我们可以做商和局部化使问题归结为终端是一个域对应的仿射概型的情况.这就归结为上一条结论.
    \item 一些反例.开嵌入$\mathbb{A}_k^2-\{(0,0)\}\to\mathbb{A}_k^2$具有有限纤维,但它不是仿射的;开嵌入$\mathbb{A}_{\mathbb{C}}^1-\{(x)\}\to\mathbb{A}_{\mathbb{C}}^1$是仿射的,并且具有有限纤维,但它不是有限态射,因为$\mathbb{C}[x]_x$不是有限$\mathbb{C}[x]$模;我们接下来会证明有限等价于有限型和整,但是整态射未必总具有有限纤维.
    \item 设$B$是$A$代数,那么$B$是有限$A$模等价于$B$是有限型和整$A$代数.这件事说明概型之间的态射是有限态射等价于它是有限型态射和整态射.
\end{enumerate}
\subsection{拟有限态射}

概型之间的一个态射称为拟有限态射(quasi-finite morphism),如果它是有限型的,并且具有有限纤维.称$f:X\to Y$在点$x\in X$是局部拟有限的,如果存在$Y$的仿射开邻域$V$和$X$的仿射开邻域$U$满足$f(U)\subseteq V$,那么$f\mid_U:U\to V$是拟有限态射.
\begin{enumerate}
	\item 设$f:X\to Y$是有限型态射,设$x\in X$和$y=f(x)$,那么如下命题互相等价.在条件成立时我们称$x$是关于$f$或者关于$Y$的纤维孤立点.
	\begin{enumerate}
		\item $x$是纤维$X_y$的孤立点(这是指这个点同时是单点开集和单点闭集).
		\item $\mathscr{O}_{X,x}$在$\mathscr{O}_{Y,y}$上拟有限.如果$(A,\mathfrak{m})\to(B,\mathfrak{n})$是局部环之间的局部态射,称它是拟有限的,如果$B/\mathfrak{m}B$是$A/\mathfrak{m}$上的有限维线性空间.
	\end{enumerate}
	\begin{proof}
		
		如果把$X\to Y$替换为纤维态射$X_y\to\mathrm{Spec}\kappa(y)$,不改变$\mathscr{O}_{X,x}/\mathfrak{m}_y\mathscr{O}_{X,x}$和$\mathscr{O}_{Y,y}/\mathfrak{m}_y\mathscr{O}_{Y,y}$.于是问题归结为设$Y=\mathrm{Spec}k$的情况.
		
		\qquad
		
		(a)推(b):取$x$的开邻域,我们不妨设$X$由单点构成.因为$f$是有限型态射,所以有$X=\mathrm{Spec}B$,其中$B$是有限型$k$代数,并且只存在唯一的素理想$\mathfrak{n}$.那么$B/\mathfrak{n}$是有限维$k$线性空间(弱零点定理),并且有$\mathfrak{n}$是幂零理想,设幂零指数是$n$.因为$B$是诺特环,所以每个$1\le i\le n$有$\mathfrak{n}^{i-1}/\mathfrak{n}^i$是有限维$B/\mathfrak{n}$线性空间,进而它们在$k$上也是有限维的.这得到$B$在$k$上有限维.
		
		\qquad
		
		(b)推(a):取$x$的仿射开邻域$U=\mathrm{Spec}B$,其中$B$是有限型$k$代数.设$x$对应于$B$的素理想$\mathfrak{q}$.按照条件$B_{\mathfrak{q}}$是$k$上有限维线性空间,于是$B_{\mathfrak{q}}$是一个阿廷环,于是$\mathfrak{q}$是$B$的一个极小素理想,于是$x$是$U$的一个一般点,进而它是$X$的一般点(因为开集保一般化).另一方面有$B_{\mathfrak{q}}/\mathfrak{q}B_{\mathfrak{q}}$是$k$上有限维空间,于是如果$B/\mathfrak{q}$在$k$上无限维,理应有它的局部化$B_{\mathfrak{q}}/\mathfrak{q}B_{\mathfrak{q}}=(B/\mathfrak{q})_{\mathfrak{q}}$也在$k$上无限维.说明$B/\mathfrak{q}$是$k$的有限维空间,所以$B/\mathfrak{q}$是一个零维整环,所以它是域,所以$\mathfrak{q}$还是$B$的极大理想.于是$x$是$U$的闭点,但是有限型$k$概形上开集的闭点也是全空间的闭点,于是$x$是$X$的闭点,于是$\{x\}$自己构成一个不可约分支.但是$X$是诺特的,它只有有限个不可约分支,其余不可约分支的并还是闭集,就导致$\{x\}$是开集.
	\end{proof}
	\item 设$f:X\to Y$是有限型态射,那么如下命题互相等价.满足这些等价条件的有限型态射称为拟有限态射.
	\begin{enumerate}
		\item 对任意$y\in Y$,纤维$X_y$是离散空间.
		\item 对任意$x\in X$,记$y=f(x)$,有$\mathscr{O}_{X,x}$在$\mathscr{O}_{Y,y}$上拟有限.
		\item 对任意$y\in Y$,纤维$X_y$是有限点集.
	\end{enumerate}
	\begin{proof}
		
		上一条给出了(a)和(b)的等价性,至于(a)和(c)的等价性,问题归结为设$Y=\mathrm{Spec}k$,那么$X$就是一个有限型$k$概形.我们解释过此时$X$是零维的当且仅当它是有限点集当且仅当它是离散空间.
	\end{proof}
	\item 拟有限态射保复合,是终端局部的,保基变换.
	\begin{proof}
		
		前两个性质平凡.设有如下纤维积图表的复合,那么大长方形图表也是纤维积的,我们要证明的是$\alpha$具有有限纤维,就归结为不妨设$Y'=\mathrm{Spec}L$的情况.
		$$\xymatrix{X'\times_{Y'}\kappa(y')\ar[rr]\ar[d]^{\alpha}&&\mathrm{Spec}\kappa(y')\ar[d]\\X'=X\times_YY'\ar[rr]\ar[d]&&Y'\ar[d]\\X\ar[rr]&&Y}$$
		
		如果设$\mathrm{Spec}L$在$Y$中的像是$y$,那么$\mathrm{Spec}L\to Y$就要经$\mathrm{Spec}K$分解,其中$K=\kappa(y)$.按照$X_y$作为纤维积的泛性质,我们有如下交换图表.大长方形图表和下方小长方形图表是笛卡尔的可以推出上面小长方形图表是笛卡尔的.于是问题又归结为设$Y=\mathrm{Spec}K$的情况.
		$$\xymatrix{X_L\ar[rr]\ar[d]&&\mathrm{Spec}L\ar[d]\\X_y\ar[rr]\ar[d]&&\mathrm{Spec}K\ar[d]\\X\ar[rr]&&Y}$$
		
		换句话讲,$X$是域$K$上的有限型概形,具有有限个点,这等价于$X$是零维的.那么对任意域扩张$L/K$,就有$X_L$也是零维的(局部有限型概形做域扩张的基变换不改变维数),于是它只有有限个点.
	\end{proof}
    \item 设$k$是域,设$f:X\to Y$是$k$有限型概形之间的态射.固定一个包含$k$的代数闭域$K$,那么$f$是拟有限的当且仅当它诱导的$K$值点之间的态射$f(K):X(K)\to Y(K)$作为集合之间的映射具有有限纤维.
    \begin{proof}
    	
    	首先我们解释过域$k$上有限型概形之间的$k$态射是有限型的.记满足$X_y$是有限点集的$y$构成的$Y$的子集为$Y_0$,这也就是满足$\dim X_y=0$的点$y$构成的子集.维数是一个可构造性质,于是$Y_0$是可构造集.于是$Y_0=Y$当且仅当$Y_0$包含所有$Y$的闭点.设$y\in Y$是闭点,那么$\kappa(y)$就是$k$的有限扩张,于是存在嵌入$\kappa(y)\to K$.我们解释过域上的局部有限型概形做任意域扩张的基变换不改变维数,于是$\dim X_y=\dim X_y\otimes_{\kappa(y)}K$.但是这里$\dim X_y\times_{\kappa(y)}K=0$当且仅当$X_y\times_{\kappa(y)}K$只有有限个点,而这等价于$(X_y\times_{\kappa(y)}K)(K)$是有限点集(一方面如果$X_y\times_{\kappa(y)}K$只有有限个点,那么它的$K$值点自然是有限的,反过来如果$X_y\times_{\kappa(y)}K$有无限个点,因为投影映射$X_y\times_{\kappa(y)}K\to X_y$是满射,说明$X_y$也是无限点集,因为$X_y$是$\kappa(y)$上有限型概形,每个点的剩余域都是$\kappa(y)$的有限扩张,所以每个点都可以作为$K$值点的像集,按照纤维积的泛性质$X_y\times_{\kappa(y)}K$上就有无穷个$K$值点).
    \end{proof}
    \item 嵌入总是局部有限型态射,拟紧嵌入总是有限型态射,而嵌入总是具有有限纤维的,于是拟紧嵌入总是拟有限态射.按照拟分离态射的对角态射一定是拟紧嵌入,从而是拟有限态射,说明如果$X\to Y\to Z$是拟有限的,并且$Y\to Z$是拟分离态射,那么$X\to Y$是拟有限态射.
	\item 我们解释过有限态射具有有限纤维,于是有限态射都是拟有限态射.但是反过来拟有限态射未必是有限态射,例如开嵌入$\mathbb{A}_k^2-\{(0,0)\}\to\mathbb{A}_k^2$是拟有限的,但是它不是仿射态射.
	\item 如果$\varphi:X\to Y$是有限态射,它在每个开集上的限制都是拟有限态射.并且每个拟有限态射都可以按照这种方式得到.这个结论是Zariski主定理的一种形式.
	\item 关于惯性次数和分歧指数.设$k$是域,设$A$是有限$k$代数,那么$X=\mathrm{Spec}A$是有限离散空间,并且满足$A=\prod_{x\in X}\mathscr{O}_{X,x}$.对点$x\in X$,我们记:
	\begin{itemize}
		\item $e_x(X)=\mathrm{l}_{\mathscr{O}_{X,x}}\mathscr{O}_{X,x}$是$\mathscr{O}_{X,x}$的长度.
		\item $f'_x(X)=[\kappa(x):k]_{\mathrm{insep}}$是$\kappa(x)/k$的纯不可分次数.
		\item $f''_x(X)=[\kappa(x):k]_{\mathrm{sep}}$是可分次数.
		\item $f_x(X)=f_x'f_x''=[\kappa(x):k]$是扩张次数.
	\end{itemize}
	
	我们断言:
	\begin{enumerate}
		\item $\dim_k\mathscr{O}_{X,x}=e_xf_x$.
		\item $\dim_kA=\sum_{x\in X}e_xf_x$.
		\item $e_x=1$当且仅当$\mathscr{O}_{X,x}=\kappa(x)$,也即$\mathfrak{m}_x$是零理想.
	\end{enumerate}
    \begin{proof}
    	
    	因为$\mathscr{O}_{X,x}$是阿廷局部环,所以它是完备的,并且有作为$\mathscr{O}_{X,x}$模的有限直和分解$\mathscr{O}_{X,x}\cong\oplus_i\mathfrak{m}_x^i/\mathfrak{m}_x^{i+1}$.这里$\mathfrak{m}_x^i/\mathfrak{m}_x^{i+1}$作为$\kappa(x)$线性空间的维数和它作为$\mathscr{O}_{X,x}$模的长度是一致的.于是:
    	\begin{align*}
    		e_xf_x&=f_x\mathrm{l}_{\mathscr{O}_{X,x}}\mathscr{O}_{X,x}\\&=[\kappa(x):k]\sum_{i\ge0}\dim_{\kappa(x)}\mathfrak{m}_x^i/\mathfrak{m}_x^{i+1}\\&=\sum_{i\ge0}\dim_k\mathfrak{m}_x^i/\mathfrak{m}_x^{i+1}\\&=\dim_k\oplus_i\mathfrak{m}_x^i/\mathfrak{m}_x^{i+1}\\&=\dim_k\mathscr{O}_{X,x}
    	\end{align*}
    	
        进而有:
        $$\dim_kA=\sum_{x\in X}\dim_k\mathscr{O}_{X,x}=\sum_{x\in X}e_xf_x$$
        
        最后$e_x=1$当且仅当$\mathscr{O}_{X,x}\cong\oplus_i\mathfrak{m}_x^i/\mathfrak{m}_x^{i+1}$分解中的$i$取一个指标,于是此时$\mathscr{O}_{X,x}=\mathscr{O}_{X,x}/\mathfrak{m}_x=\kappa(x)$.
    \end{proof}
    \item 设$f:X\to Y$是拟紧态射,设$y\in Y$,那么纤维$X_y$是有限离散空间,我们记:
    \begin{itemize}
    	\item $e_{x/y}=e_x(X_y)$称为$x$在$y$上的分歧指数.
    	\item $f_{x/y}'=f_x'(X_y)$称为$x$在$y$上的纯不可分惯性次数.
    	\item $f_{x/y}''=f_x''(X_y)$称为$x$在$y$上的可分惯性次数.
    	\item $f_{x/y}=f_x(X_y)$称为$x$在$y$上的惯性次数.
    \end{itemize}

    我们断言$e_{x/y}f'_{x/y}$和$f''_{x/y}$是纤维积下不变的.具体的讲,考虑如下纤维积图表,其中$f$是拟有限态射.
    $$\xymatrix{X'\ar[rr]^{\alpha'}\ar[d]_{f'}&&X\ar[d]^f\\Y'\ar[rr]^{\alpha}&&Y}$$
    
    设$y'\in Y'$,设$\alpha(y')=y$,设$x\in f^{-1}(y)$,那么有:
    $$e_{x/y}f'_{x/y}=\sum\limits_{x'\in f'^{-1}(y')\cap\alpha'^{-1}(x)}e_{x'/y'}f'_{x'/y'}$$
    $$f''_{x/y}=\sum\limits_{x'\in f'^{-1}(y')\cap\alpha'^{-1}(x)}f''_{x'/y'}$$
    \begin{proof}
    	
    	我们的问题对于$Y$来讲只涉及到$y$,对于$Y'$来讲只涉及到$y'$,于是不妨设$Y=\mathrm{Spec}k$和$Y'=\mathrm{Spec}k'$都是域的素谱.那么$X=\mathrm{Spec}A$就是一个有限$k$代数$A$的素谱.因为对$X$来讲我们的问题也只涉及到$x$,所以不妨设$A=\mathscr{O}_{X,x}$是局部的,并且剩余域就是$\kappa(x)$.另外按照$X,Y,Y'$都约化到单点空间,所以$f'^{-1}(y')\cap\alpha'^{-1}(x)$就是$X'$本身.我们先来证明第二个等式,选取一个$k'$的代数闭包$\Omega$,我们有:
    	\begin{align*}
    		f''_{x/y}&=[\kappa(x):k]_{\mathrm{sep}}=|\mathrm{Hom}_k(\kappa(x),\Omega)|\\&=|\mathrm{Hom}_k(A,\Omega)|=|\mathrm{Hom}_{k'}(A\otimes_kk',\Omega)|\\&=\sum_{x'\in X'}[\kappa(x'):k']_{\mathrm{sep}}=\sum_{x'\in X'}f''_{x'/y'}
    	\end{align*}
    	
    	从第二个等式可以推出第一个等式,【】
    \end{proof}
    \item 【EGAIV8.11节】
\end{enumerate}
\subsection{有限局部自由态射}
\begin{enumerate}
	\item 设$f:X\to Y$是概形之间的态射,那么如下命题互相等价,在命题成立时称$f$是有限局部自由态射.
	\begin{enumerate}
		\item $f$是仿射态射,并且$f_*\mathscr{O}_X$是有限局部自由的$\mathscr{O}_Y$模层.
		\item $f$是有限,平坦和有限表示的态射.
		\item $f$是仿射态射,并且对$Y$的任意仿射开子集$\mathrm{Spec}A$,记它在$f$下的原像是$\mathrm{Spec}B$,那么诱导的$A\to B$使得$B$是有限投射$A$模.
	\end{enumerate}
    \begin{proof}
    	
    	这件事是因为我们解释过如果$M$是$A$模,那么如下三件事互相等价:
    	\begin{enumerate}
    		\item $\widetilde{M}$是$\mathrm{Spec}A$上有限生成的局部自由模层.
    		\item $M$是有限表示的平坦$A$模.
    		\item $M$是有限投射$A$模.
    	\end{enumerate}
    \end{proof}
    \item 【】
\end{enumerate}











\subsection{(局部)有限表示态射}

设$f:X\to Y$是概形之间的态射,称它在点$x\in X$处有限表示,如果存在$y=f(x)$的仿射开邻域$V=\mathrm{Spec}A$,存在$x$的仿射开邻域$U=\mathrm{Spec}B$,使得$f(U)\subseteq V$,并且满足$B$是有限表示$A$代数,此即有$A$代数同构$B\cong A[X_1,\cdots,X_n]/(f_1,\cdots,f_r)$.如果$f$在$X$的每个点都是有限表示的,就称$f$是局部有限表示态射.称一个态射是有限表示态射,如果它是局部有限表示态射,并且是拟紧和拟分离的(这三个条件任意两个都推不出第三个).
\begin{enumerate}
	\item 回顾有限表示代数.
	\begin{enumerate}
		\item 设$A$是环,它的有限表示代数是指形如$B=A[T_1,\cdots,T_n]/\mathfrak{a}$的代数,其中$\mathfrak{a}$是有限生成理想.对于有限表示$A$代数$B$,如果$B'$是一个有限生成$A$代数,如果有$\varphi:B'\to B$是满代数同态,那么$\ker\varphi$总是$B'$的有限生成理想.换句话讲,尽管有限表示代数定义中我们只要求了存在一个表示是$A[T_1,\cdots,T_n]/\mathfrak{a}$,但是倘若它还能表示成$A[S_1,\cdots,S_m]/\mathfrak{b}$,那么$\mathfrak{b}$总是有限生成的.
		\item 基变换.如果$B$是有限表示$A$代数,如果$A'$是任意$A$代数,那么$B'=B\otimes_AA'$是有限表示$A'$代数.这是因为如果$\mathfrak{a}$是$A[T_1,\cdots,T_n]$的有限生成理想,那么$\mathfrak{a}\otimes_AA'$是$A'[T_1,\cdots,T_n]$的有限生成理想.
		\item 复合.如果$B$是有限表示$A$代数,$C$是有限表示$B$代数,那么$C$也是有限表示$A$代数.
		\begin{proof}
			
			可记$B=A[S_1,\cdots,S_m]/\mathfrak{a}$和$C=B[T_1,\cdots,T_n]/\mathfrak{b}$,其中$\mathfrak{a}$和$\mathfrak{b}$都是有限生成理想.那么$B[T_1,\cdots,T_n]$同构于$A[S_1,\cdots,S_m,T_1,\cdots,T_n]/\mathfrak{a}\otimes_AA[T_1,\cdots,T_n]$,而这里$\mathfrak{a}\otimes_AA[T_1,\cdots,T_n]$的确是$A[S_1,\cdots,S_m,T_1,\cdots,T_n]$的有限生成理想.那么$B[T_1,\cdots,T_n]$的理想$\mathfrak{b}$就对应于$\mathfrak{b}'/\mathfrak{a}'$,其中$\mathfrak{b}'$是$A[S_1,\cdots,S_m,T_1,\cdots,T_n]$的理想.因为$\mathfrak{a}'$和$\mathfrak{b}'/\mathfrak{a}'$都是有限生成理想,这就得到$\mathfrak{b}'$也是有限生成理想.于是$C\cong A[S_1,\cdots,S_m,T_1,\cdots,T_n]/\mathfrak{b}'$是有限表示$A$代数.
		\end{proof}
	    \item 设$B$是$A$代数,并且作为$A$模是有限的.那么$B$是有限表示$A$代数当且仅当$B$是有限表示$A$模.特别的,这说明商代数$A/I$是有限表示$A$代数当且仅当$I$是有限生成理想.
	    \begin{proof}
	    	
	    	我们先来证明一个引理:设$B$是$A$代数,并且作为$A$模是有限的,那么存在有限表示$A$代数$B'$,使得它作为$A$模是有限自由模,并且存在满的$A$代数同态$u:B'\to B$.
	    	
	    	\qquad
	    	
	    	因为$A$代数$B$如果作为模是有限模,则它作为代数也一定是有限型的,并且$B$在$A$上整.我们记$B$作为$A$代数被有限个元$\{a_1,\cdots,a_m\}$生成.按照$B$是有限$A$模,对每个$i$,就有$a_i$要满足某个$A$系数的首一非零多项式$F_i\in A[X]$,有$\deg F_i>0$.下面记$B_i'=A[X]/(F_i)$,这是一个有限自由$A$模,记$X$在$B_i'$中的像为$c_i$.构造$A$代数同态$u_i:B_i'\to B$为$c_i\mapsto a_i$.于是我们只需取$B'=B_1'\otimes_A\cdots\otimes_AB_m'$,这作为$A$模就是一个有限自由模,取同态$u_1\otimes\cdots\otimes u_m:B'\to B$,按照$B_i'$的构造这是一个满射.接下来只剩下验证$B'$的确是有限表示$A$代数.为此设$B''=A[T_1,\cdots,T_m]$,记$c_i\in B_i'$在$B'$中的像为$b_i'$,构造$A$代数同态$v:B''\to B'$为$T_i\mapsto b_i'$,那么这是一个满射,它的核$\mathfrak{b}'$是$B''$的理想,并且被$F_i(T_i)$生成,这就得到$B'$是有限表示$A$代数.
	    	
	    	\qquad
	    	
	    	回到原命题,但是我们保持上面证明中相同的记号.记$w=v\circ u:B''\to B$.我们记$\mathfrak{a}=\ker u\subseteq B'$,记$\mathfrak{b}=\ker w\subseteq B''$,那么按照$v$是满射,就有$\mathfrak{a}=v(\mathfrak{b})$.
	    	
	    	\qquad
	    	
	    	必要性.下面设$B$是有限表示$A$代数,那么这里$\mathfrak{b}$就是$B''$的有限生成理想,进而从$\mathfrak{a}=v(\mathfrak{b})$得到$\mathfrak{a}$是有限生成$B'$理想,按照$B'$是有限自由$A$模,就得到$\mathfrak{a}$是有限$A$模,从$A$模同构$B'/\mathfrak{a}\cong B$就得到$B$是有限表示$A$模.
	    	
	    	\qquad
	    	
	    	充分性.如果$B$是有限表示$A$模,那么这里$\mathfrak{a}$就是有限$A$模,从而它当然是$B'$的有限生成理想.于是$B$是有限表示$B'$代数,而$B'$是有限表示$A$代数,于是复合得到$B$是有限表示$A$代数.
	    \end{proof}
	\end{enumerate}
	\item 关于局部性质.
	\begin{enumerate}
		\item 局部有限表示态射是终端局部性质和源端局部性质.另外它满足仿射交换引理,于是$f:X\to Y$是局部有限表示态射还等价于:
		\begin{enumerate}
			\item 对任意仿射开子集$V=\mathrm{Spec}A\subseteq Y$,和任意仿射开子集$U=\mathrm{Spec}B\subseteq f^{-1}(V)$,总有$B$是有限表示$A$代数.
			\item 存在$Y$的仿射开覆盖$Y=\cup_iV_i$,其中$V_i=\mathrm{Spec}A_i$,对每个$i$又存在$f^{-1}(V_i)$的仿射开覆盖$\cup_jU_{ij}$,其中$U_{ij}=\mathrm{Spec}B_{ij}$,那么有每个$B_{ij}$都是有限表示$A_i$代数.
		\end{enumerate}
	    \item 有限表示态射是仿射终端局部性质.
	\end{enumerate}
    \item 仿射情况.态射$f:\mathrm{Spec}B\to\mathrm{Spec}A$是局部有限表示态射当且仅当它是有限表示态射,当且仅当$B$是有限表示$A$代数.
	\item 关于有限表示和有限型.一般的,局部有限表示态射一定是局部有限型态射,有限表示态射一定是有限型态射.如果一个态射的终端是局部诺特概形,那么这个态射是局部有限型态射等价于是局部有限表示态射;它是有限型态射等价于是有限表示态射.
	\begin{proof}
		
		设$f:X\to Y$是终端为局部诺特概形的态射.如果$f$是局部有限型态射,按照希尔伯特基定理,诺特环上有限型代数自动是有限表示的,于是$f$是局部有限表示态射.下面设$f$是有限型态射,我们只需证明$f$是拟分离的,但是拟分离是仿射终端局部性质,所以不妨设$Y$本身是仿射诺特概形,则从$f$是有限型态射得到$X$是诺特概形,但是以诺特概形为源端的态射一定是拟分离的,这就得证.
	\end{proof}
    \item 关于对角态射.局部有限型态射的对角态射是局部有限表示态射.
    \begin{proof}
    	
    	设$f:X\to Y$是局部有限型态射,我们来证明$\Delta_f:X\to X\times_YX$是局部有限表示态射,为此不妨设$Y=\mathrm{Spec}A$和$X=\mathrm{Spec}B$都是仿射的,那么$B$就是一个有限型$A$代数,并且对角态射对应于环同态$B\otimes_AB\to B$,$b_1\otimes b_2\mapsto b_1b_2$.但是我们知道这个环同态的核$I$被全体$\{1\otimes s-s\otimes1\}$生成,其中$s$跑遍$B$作为$A$代数的生成元集,于是$I$是有限生成理想.
    \end{proof}
    \item 关于复合.
    \begin{enumerate}
    	\item 局部有限表示和有限表示都是保复合的.
    	\item 设$X\to Y\to Z$是局部有限表示态射,设$Y\to Z$是局部有限型态射,那么$X\to Y$是局部有限表示态射.这是因为局部有限型态射的对角态射是局部有限表示的.
    	\item 设$X\to Y\to Z$是有限表示态射,设$Y\to Z$是局部有限型的拟分离态射,那么$X\to Y$是有限表示态射.
    \end{enumerate}
	\item 关于嵌入.
	\begin{enumerate}
		\item 开嵌入是局部有限表示态射,由于开嵌入自动是拟分离的(因为单态射都是拟分离的),一个开嵌入是有限表示态射当且仅当它是拟紧的.
		\item 闭嵌入自动是拟紧拟分离的,一个闭嵌入$j:Z\to X$是有限型态射当且仅当它对应的拟凝聚理想层$\mathscr{I}_Z\subseteq\mathscr{O}_X$是有限型模层,这是因为我们解释过$A/I$是有限表示$A$代数当且仅当$I$是有限生成理想.
	\end{enumerate}
	\item 局部有限表示和有限表示都在基变换下不变.
    \item 设$X,Y$是$S$概形,设$y\in Y$,设$y$在$S$中的像是$s$,设$Y\to S$在点$y$是局部有限表示的(此即存在$y$的仿射开邻域$U=\mathrm{Spec}B$和$s$的仿射开邻域$V=\mathrm{Spec}A$使得$Y\to S$把$U$打到$V$内,并且$B$是一个有限型$A$代数,并且生成元集可以取有限集),设$\varphi_x:\mathscr{O}_{Y,y}\to\mathscr{O}_{X,x}$是一个$\mathscr{O}_{S,s}$代数同态,也是一个局部同态,那么存在$x$的开邻域$U$以及一个$S$态射$f:U\to Y$满足$f(x)=y$,并且$f$在点$x$诱导的$\mathscr{O}_{Y,y}\to\mathscr{O}_{X,x}$就是$\varphi_x$.
    \begin{proof}
    	
    	取$x$的仿射开邻域$W=\mathrm{Spec}C$使得$W$在$X\to S$下打到$V$内.我们可设$B=A[T_1,\cdots,T_n]/(f_1,\cdots,f_m)$,设$x,y$在$C$和$B$中对应的素理想分别是$\mathfrak{q}$和$\mathfrak{p}$.于是$\varphi_x$是$B_{\mathfrak{q}}\to C_{\mathscr{p}}$的局部同态.记典范同态的复合$A[T_1,\cdots,T_n]\to B\to B_{\mathfrak{q}}\to C_{\mathfrak{p}}$为$\theta$.设$T_i$在$B$中的像是$b_i$,记$\varphi_x(b_i/1)=a_i/s_i$,其中$s_i\in C-\mathfrak{p}$.于是有$\theta(f_i(T_1,\cdots,T_n))=0$,于是在$C_{\mathfrak{p}}$中有$\varphi_x(f_i(b_1,\cdots,b_n)/1)=0$,但是我们有:
    	\begin{align*}
    		\varphi_x\left(\frac{f_i(b_1,\cdots,b_n)}{1}\right)&=\sum_{0\le i_1,\cdots,i_n\le d}\left(\frac{a_1}{s_1}\right)^{i_1}\cdots\left(\frac{a_n}{s_n}\right)^{i_n}\\&=\frac{\sum_{0\le i_1,\cdots,i_n\le d}c_{i,i_1,\cdots,i_n}s_1^{d-i_1}a_1^{i_1}\cdots s_n^{d-i_n}a_n^{i_n}}{s_1^d\cdots s_n^d}
    	\end{align*}
    
        于是存在$t_i\in C-\mathfrak{p}$使得$t_i$乘以上述分子为零.我们取$s=s_1\cdots s_nt_1\cdots t_m$,那么$s\in C-\mathfrak{p}$.有典范同态$A[T_1,\cdots,T_n]\to C_s$为$T_i\mapsto a_i/s_i=a_is_1\cdots\widehat{s_i}\cdots s_nt_1\cdots t_m/s$,并且它把$f_i(T_1,\cdots,T_n)$都映射为零,于是它诱导了$B\to C_s$,这对应了态射$f:D(s)\to Y$满足要求.
    \end{proof}
    \item 设$X,Y$是$S$概形,设$x\in X$和$y\in Y$在$S$中的像都是$s$,设$X\to S$在点$x$是局部有限表示的,设$Y\to S$在点$y$是局部有限表示的,如果存在$\mathscr{O}_{S,s}$代数同构$\varphi:\mathscr{O}_{Y,y}\to\mathscr{O}_{X,x}$,那么存在$x$和$y$的开邻域$U$和$V$,以及一个概形同构$f:U\to V$,满足$f(x)=y$,并且它在点$x$诱导的局部环同态就是$\varphi$.
    \begin{proof}
    	
    	按照上一条,存在$x$的开邻域$U$和态射$f:U\to Y$使得$f(x)=y$并且$f^{\#}_x=\varphi$;同理存在$y$的开邻域$V$和态射$g:V\to X$使得$g(y)=x$并且$g^{\#}_y=\varphi^{-1}$.另外我们解释过如果$X,Y$在$s\in S$是局部有限型的,两个$S$态射$X\to Y$如果满足$f^{\#}_x=g^{\#}_x$,那么$f,g$在$x$足够小的开邻域上相同.于是从这里$f\circ g$在$y$的stalk是恒等态射,说明存在$y$足够小的开邻域上有$f\circ g$就是恒等态射,同理对$g\circ f$做相同的事,适当缩小开邻域就得到一个同构$U\cong V$.
    \end{proof}
	\item 一个态射$\varphi:X\to Y$是局部有限表示态射当且仅当对每个$Y$仿射概型的投射系统$\{S_i,i\in I\}$,典范映射$\lim\limits_{\substack{\rightarrow\\i}}\mathrm{Hom}_Y(S_i,X)\to\mathrm{Hom}_Y(\lim\limits_{\substack{\leftarrow\\i}}S_i,X)$是双射.
\end{enumerate}
\subsection{单态射和单射}

概形之间的态射$f$称为单态射(monomorphism),如果它是概形范畴中的单态射,也即对任意态射$g,h$,如果有$f\circ g=f\circ h$,那么有$g=h$.态射$f$称为单射(injection),如果它在集合层面是单射.
\begin{enumerate}
	\item 一个态射$f:X\to Y$是单态射当且仅当对角态射$\Delta_{X/Y}:X\to X\times_YX$是同构(此即$(X,1_X,1_X)$就是$f:X\to Y$和自身的纤维积).
	\item 单态射的复合是单态射,单态射的纤维积是单态射.这在任意范畴上都成立.
	\item 单态射一定是泛单射(比方说,泛单射等价于对角态射是满射,而单态射的对角态射是同构).
	\item 单态射总是分离态射.
	\item 单射的复合也是单射,但是单射的基变换未必是单射:设$k\subseteq K$是$n$次的有限可分扩张,设$k\subseteq\overline{k}$是到代数闭包的域扩张,那么扩张诱导的态射$f:\mathrm{Spec}K\to\mathrm{Spec}k$拓扑上是同胚,因为它们都是单点集.但是它关于$\mathrm{Spec}\overline{k}\to\mathrm{Spec}k$的基变换$\mathrm{Spec}K\times_k\mathrm{Spec}\overline{k}\to\mathrm{Spec}\overline{k}$就不是单射,因为终端是1个点,源端是$n$个点.这个例子也说明双射不是基变换不变的性质.
	\item 单射也总是分离态射.
	\begin{proof}
		
		回顾一下我们证明一个态射$f:X\to Y$的对角态射$\Delta_f$总是嵌入的证明中,我们取$Y$的仿射开子集$V$,取$X$的仿射开子集$U$使得$f(U)\subseteq V$,对所有满足这个条件的$(U,V)$,我们记$U\times_VU$的并为$W$,这一般未必是整个$X\times_YX$,但是是它的开子集,然后我们证明$\Delta(X)$是$W$的闭子集,于是$\Delta$总是嵌入.这里我们来证明如果$f$是单射,那么$W$就是整个$X\times_YX$,这也就说明了$\Delta$是闭嵌入,也即$f$是分离的.事实上,任取$X\times_YX$中的点$z$,设$X\times_YX$的泛性质中的到$X$的两个投影为$\pi_1$和$\pi_2$,由于$f(\pi_1(z))=f(\pi_2(z))$,从$f$是单射就得到$\pi_1(z)=\pi_2(z)$记作$x$,再记$f(x)=y$.取$Y$的包含$y$的仿射开子集$V$,那么$f(x)\in V$,于是可取$x$的仿射开子集$U$使得$f(U)\subseteq V$,那么$\Delta(x)=z$,于是$z$落在$U\times_VU$中,于是所有形如$U\times_VU$的开子集覆盖了整个$X\times_YX$.
	\end{proof}
    \item 如果态射$f:X\to Y$是单射,并且对任意$x\in X$,有$f^{\#}_x:\mathscr{O}_{Y,f(x)}\to\mathscr{O}_{X,x}$都是满射,那么$f$是单态射.
    \begin{proof}
    	
    	假设有两个概型的态射$a,b:Z\to X$满足$f\circ a=f\circ b$.按照$f$在拓扑角度是单射,得到$a,b$作为连续映射是相同的.对任意$z\in Z$,有$a_z^{\#}\circ f_{a(z)}^{\#}=b_z^{\#}\circ f_{b(z)}^{\#}$,按照$f_x^{\#}$是满射,得到$a_z^{\#}=b_z^{\#}$,于是作为概型的态射有$a=b$.
    \end{proof}
    \item 例如如果$X$是概形,$x\in X$,那么典范态射$\mathrm{Spec}\kappa(x)\to X$是一个单态射.
\end{enumerate}
\subsection{满射}
\begin{enumerate}
	\item 满射是基变换下不变性质.
	\begin{proof}
		
		设$f:X\to S$是态射,并且集合层面是满射,对任意态射$g:Y\to S$,我们要证明提升$X\times_SY\to Y$也是满射.任取$y\in Y$,此即证明纤维$X\times_SY\times_Y\mathrm{Spec}\kappa(y)=X\times_S\mathrm{Spec}\kappa(y)$非空.记$g(y)=s$,那么$\mathrm{Spec}\kappa(s)\to S$是单态射,所以$X\times_S\mathrm{Spec}\kappa(y)=X\times_{\mathrm{Spec}\kappa(s)}\mathrm{Spec}\kappa(y)$.但是按照$f$是满射,可取$s$在$X$的一个原像$x$,取$x$的仿射开邻域$U=\mathrm{Spec}A$,那么$U\times_{\kappa(s)}\kappa(y)=\mathrm{Spec}(A\otimes_{\kappa(s)}\kappa(y))$,由于$\kappa(s)$是域,从$A$和$\kappa(y)$不是零环就得到$A\otimes_{\kappa(s)}\kappa(y)$不是零环,于是$U\times_{\kappa(s)}\kappa(y)$非空,于是$X\times_{\kappa(s)}\kappa(y)$非空,这说明$X\times_SY\to Y$是满射.
	\end{proof}
    \item 满射准则.设$f:X\to Y$是概形之间的态射,那么它是满射当且仅当对任意域$K$和任意$K$值点$y\in Y(K)$,都存在$K$的域扩张$L$和一个$L$值点$x\in X(L)$,使得如下图表交换:
    $$\xymatrix{\mathrm{Spec}L\ar@{-->}[rr]\ar@{-->}[d]_x&&\mathrm{Spec}K\ar[d]^y\\X\ar[rr]_f&&Y}$$
    \begin{proof}
    	
    	充分性.设$y_0\in Y$,我们有典范的态射$y:\mathrm{Spec}\kappa(y_0)\to Y$.那么可以找到域扩张$\kappa(y_0)\subseteq L$和一个$L$值点$x:\mathrm{Spec}L\to X$使得上述图表交换.设$x$的像集中的唯一点是$x_0$,那么有$f(x_0)=y_0$.
    	
    	\qquad
    	
    	必要性.设$f$是满射,设$y\in Y(K)$,设$y$像集中的唯一点是$y_0$.那么存在$x_0\in X$使得$f(x_0)=y_0$.我们有剩余域扩张$\kappa(y_0)\subseteq\kappa(x_0)$.取域$L$使得存在$\kappa(y_0)$嵌入$\kappa(x_0)\to L$和$\kappa(y_0)$嵌入$K\to L$,比方说可以取$L=(\kappa(x_0)\otimes_{\kappa(y_0)}K)/\mathfrak{m}$.那么态射$x:\mathrm{Spec}L\to\mathrm{Spec}\kappa(x_0)\to X$满足结论,因为有如下交换图表:
    	$$\xymatrix{\mathrm{Spec}L\ar[rr]\ar[d]&&\mathrm{Spec}K\ar[d]\\\mathrm{Spec}\kappa(x_0)\ar[rr]\ar[d]&&\mathrm{Spec}\kappa(y_0)\ar[d]\\X\ar[rr]_f&&Y}$$
    \end{proof}
    \item 特别的如果态射$f:X\to Y$满足对任意域$K$都有诱导的$X(K)\to Y(K)$都是满射,那么$f$是满射.这个逆命题不对.不过我们有下一条结论.
    \item 设$f:X\to Y$是局部有限型态射,那么$f$是满射当且仅当对任意代数闭域$\Omega$,有$f$诱导的映射$X(\Omega)\to Y(\Omega)$是满射.
    \begin{proof}
    	
    	充分性和前面定理的充分性证明是一样的,任取$y\in Y$,我们取$\kappa(y)$的一个代数闭包$\Omega$,取$\mathrm{Spec}\Omega\to Y$为把唯一点映射为$\{y\}$.那么按照条件就有$X(\Omega)$中的点$x$使得如下图表交换,也即$f(x)=y$,于是$f$是满射.
    	$$\xymatrix{X\ar[rr]^f&&Y\\&\mathrm{Spec}\Omega\ar[ul]\ar[ur]&}$$
    	
    	必要性,设$f$是满射,取态射$g:\{\xi\}=\mathrm{Spec}\Omega\to Y$,其中$\Omega$是某个代数闭域.考虑如下纤维积图表:
    	$$\xymatrix{\mathrm{Spec}\Omega\ar@{-->}@/_1pc/[ddr]\ar@{=}@/^1pc/[drr]&&\\&X_{(\Omega)}\ar[r]^{f_{(\Omega)}}\ar[d]&\mathrm{Spec}\Omega\ar[d]\\&X\ar[r]_f&Y}$$
    	
    	我们要证明的是有虚线态射使得上面图表交换,按照纤维积的泛性质,这等价于讲存在一个态射$\mathrm{Spec}\Omega\to X_{(\Omega)}$.因为$f$是满射,得到$f_{(\Omega)}$是满射,于是$X_{(\Omega)}$非空.又因为$f$是局部有限型态射,得到$f_{(\Omega)}$也是局部有限型态射,于是存在$X_{(\Omega)}$的非空仿射开集$Z=\mathrm{Spec}R$,使得$R$是$\Omega$上的非零有限型代数.记作$R=\Omega[T_1,\cdots,T_n]/I$,如果取$I$在$\Omega[T_1,\cdots,T_n]$中的极大理想$\mathfrak{m}$,那么有典范$\Omega$同态$R\to\Omega[T_1,\cdots,T_n]/\mathfrak{m}\cong\Omega$(最后这个同构是希尔伯特零点定理),这个同态对应于一个$\Omega$态射$\mathrm{Spec}\Omega\to X_{(\Omega)}$,这就得证.
    \end{proof}
\end{enumerate}

\subsection{平坦和忠实平坦态射}

设$f:X\to Y$是概形的态射.
\begin{itemize}
	\item 称它在点$x\in X$是平坦的(flat),如果$f_x^{\#}:\mathscr{O}_{Y,f(x)}\to\mathscr{O}_{X,x}$使得$\mathscr{O}_{X,x}$是平坦$\mathscr{O}_{Y,f(x)}$代数.如果$f$在$X$上处处是平坦的,就称$f$是平坦态射.
	\item 如果$f$是平坦态射也是满射,就称它是忠实平坦态射(faithfully flat).
\end{itemize}
\begin{enumerate}
	\item 如果$f:X\to Y$是态射,设$x\in X$和$y=f(x)$,记$y$的纤维诱导的态射是$g:X\times_Y\mathrm{Spec}\mathscr{O}_{Y,y}\to\mathrm{Spec}\mathscr{O}_{Y,y}$.设$x'$是$x$在纤维中的对应,那么我们知道$f^{\#}_x=g^{\#}_{x'}$.于是$f$在一个点$x$是平坦的当且仅当它对应的纤维态射(这里不是$X_y\to\mathrm{Spec}\kappa(y)$的哪个纤维态射)在对应的点$x'$是平坦的.
	\item 局部性质,复合与基变换.
	\begin{enumerate}
		\item 平坦态射是一个终端局部性质,也是一个源端局部性质,也是一个茎局部性质.忠实平坦态射是终端局部性质.
		\item 平坦态射的基变换是平坦态射.因为如果$B$是平坦$A$代数,那么$C\otimes_AB$总是平坦$B$代数.进而忠实平坦态射也在基变换下不变,因为满射在基变换下不变.
		\item 平坦态射的复合是平坦态射.因为如果$A\to B$和$B\to C$都是平坦环同态,那么$A\to B\to C$也是平坦环同态.进而忠实平坦态射也在复合下不变,因为满射的复合还是满射.
	\end{enumerate}
	\item 例子.
	\begin{enumerate}
		\item 开嵌入总是平坦的,因为开嵌入诱导的stalk上的同态是恒等.
		\item 设$k$是域,终端为$\mathrm{Spec}k$的态射总是平坦态射,因为域上的模总是平坦的.
		\item 仿射空间的结构态射$\mathbb{A}_S^n\to S$总是平坦态射,因为环上的多项式环总是平坦代数.进而有射影空间的结构态射$\mathbb{P}_S^n\to S$是平坦态射(因为$\mathbb{P}^n_S$被一族$\mathbb{A}_S^n$覆盖).更一般的如果$\mathscr{E}$是有限局部自由$\mathscr{O}_S$模层,那么射影丛$\mathbb{P}(\mathscr{E})$就要被一族$\mathbb{P}^n_S$覆盖,导致$\mathbb{P}(\mathscr{E})\to S$也是平坦态射.
	\end{enumerate}
	\item 平坦态射满足下降条件.
	\begin{enumerate}
		\item 设$f:X\to Y$是平坦态射,设$x\in X$,记$y=f(x)$,设$y'\in Y$是$y$的一般化,那么存在$x$的一般化$x'$使得$f(x')=y'$.这个结论也可以等价的写为总有$f(\mathrm{Spec}\mathscr{O}_{X,x})=\mathrm{Spec}\mathscr{O}_{Y,f(x)}$.特别的,平坦态射把一般点映为一般点.
		\begin{proof}
			
			记$A=\mathscr{O}_{Y,y}$和$B=\mathscr{O}_{X,x}$,记平坦环同态$\varphi=f_x^{\#}:A\to B$,记诱导的态射是$g:\mathrm{Spec}B\to\mathrm{Spec}A$.我们要证明的是$g$是满射,而这是因为局部环之间的平坦扩张一定是忠实平坦扩张,于是对任意$y'\in\mathrm{Spec}A$,从$\kappa(y')\not=0$得到纤维环$B\otimes_A\kappa(y')\not=0$,于是纤维总非空.
		\end{proof}
	    \item 推论.我们解释过局部有限表示态射是开映射当且仅当它满足下降条件,于是局部有限表示的平坦态射是开映射.
	\end{enumerate}
	\item 仿射情况.设$f:\mathrm{Spec}B\to\mathrm{Spec}A$是环同态$\varphi:A\to B$诱导的,那么$f$是平坦态射当且仅当$\varphi$使得$B$在$A$上平坦;$f$是忠实平坦态射当且仅当$\varphi$使得$B$在$A$上忠实平坦.
	\begin{proof}
		
		交换代数里解释过平坦是一个局部性质.我们来证明忠实平坦的部分:设$f$是平坦态射,我们来证明$f$是满射当且仅当$B$是忠实平坦$A$代数.按照等价定义,$B$在$A$上忠实平坦当且仅当对$A$的每个极大理想$\mathfrak{m}$都有$\mathfrak{m}B\not=B$,等价于$\mathrm{Spec}A$的每个闭点的纤维都是非空的.但是我们解释了平坦态射满足下降条件,于是$f$是满射当且仅当$\mathrm{Spec}A$的每个闭点都有原像,这就得证.
	\end{proof}
    \item 平坦和正合性.设$f:X\to Y$是平坦态射,那么$f^*$是从拟凝聚$\mathscr{O}_Y$模层范畴到$\mathscr{O}_X$模层范畴的正合函子.反过来如果$Y$是qcqs概形,如果$f^*$是正合函子,那么$f$是平坦态射.
    \begin{proof}
    	
    	【】
    \end{proof}
    \item 如果$X\to Y$是忠实平坦态射,那么$X\to Y\to Z$是平坦态射当且仅当$Y\to Z$是平坦态射;$X\to Y\to Z$是忠实平坦态射当且仅当$Y\to Z$是忠实平坦态射.
    \begin{proof}
    	
    	第一个命题归结为局部环之间的环同态$A\to B\to C$,其中$B\to C$是平坦的(等价于忠实平坦),$A\to B\to C$是平坦的,那么$A\to B$是平坦的.这是因为$B\to C$是忠实平坦的导致$-\otimes_AB$是正合函子当且仅当$-\otimes_AB\otimes_BC$是正合函子,当且仅当$-\otimes_AC$是正合函子.第二个命题是第一个命题结合上$X\to Y\to Z$是满射推出$Y\to Z$是满射.
    \end{proof}
    \item 设$S'\to S$是平坦态射,设$f:X\to S$是态射,满足把一般点映为一般点.那么基变换$X'=X\times_SS'\to S'$也把一般点映为一般点.
    \begin{proof}
    	
    	任取$x'\in X'$,设它在$X$中的像是$x$,按照$X'\to X$是$S'\to S$的基变换,于是它也是平坦态射,于是它把一般点映为一般点,于是$x$是$X$的一般点,进而有$x$在$S$中的像$s$也是一般点.再设$x'$在$S'$中的像是$s'$,我们要证明$s'$是$S'$的一般点.由于$s'$的一般化在$S$中的像仍然是$s$,导致$s'$是$S$的一般点等价于它是纤维$S'_s$的一般点.于是我们不妨用$\mathrm{Spec}\kappa(s)=\{s\}$替代$S$,用$X_s$替代$X$,用$S'_s$替代$S'$.那么$x'$也在$X'_s=X_s\times_{\{s\}}S'_s$中.按照$\{s\}$是域的素谱,有$X_s\to\{s\}$是平坦态射,于是基变换$X'_s\to S'_s$也是平坦态射,于是按照平坦态射把一般点映为一般点,得到$s'$是纤维$S'_s$的一般点,从而$s$也是$S'$的一般点.
    \end{proof}
    \item 平坦和闭嵌入.粗略的讲,平坦的闭嵌入是开嵌入.
    \begin{enumerate}
    	\item 设$f:X\to Y$是闭嵌入,那么$f$是平坦和有限表示态射当且仅当$f$还是开嵌入.必要性我们会证明这样一件事:设$f:X\to Y$是有限表示的闭嵌入,设$f$在$x\in X$处平坦,我们断言存在$y=f(x)$的开邻域$V$,使得$f$可以限制为同构$f^{-1}(V)\to V$.
    	\begin{proof}
    		
    		充分性是因为开嵌入一定是平坦和局部有限表示的,而$f$是闭嵌入导致它是拟紧和分离的态射,于是它是有限表示的.必要性归结为证明这样一件事:设$f:X\to Y$是有限表示的闭嵌入,设$f$在$x\in X$处平坦,我们断言存在$y=f(x)$的开邻域$V$,使得$f$可以限制为同构$f^{-1}(V)\to V$.
    		
    		\qquad
    		
    		设$\mathscr{I}$是$X$作为闭子概型对应的拟凝聚理想层,设$\mathscr{O}_{Y,y}=A$,设$\mathscr{I}_y=I$,那么条件说明非零$A$模$A/I$是平坦的,它还是有限表示的,于是$A/I$是投射模,但是局部环上投射模等价于自由,于是$A/I$是自由$A$模,这迫使$I=0$.又因为$\mathscr{I}$是有限型模层,它的支集是闭的,于是存在$y$的开邻域使得$\mathscr{I}\mid_V=0$.
    	\end{proof}
        \item 特别的,设$X$是诺特概形,它的闭子概型$Y$是平坦的,当且仅当$Y$还是开子概型.交换代数的语言就是说,设$A$是诺特环,那么$A/I$是平坦$A$模当且仅当$I=I^2$,当且仅当$I$被一个幂等元$e$生成,也即$e=e^2$.此时有$V(e)$也是$\mathrm{Spec}A$的开子集.
    \end{enumerate}
    \item 平坦和正规化.粗略的讲,平坦的正规化是同构.
    \begin{enumerate}
    	\item 引理.设$X$是具有有限个不可约分支的概形,设$f:X\to Y$是分离态射,设有稠密开集的开覆盖$X=\cup_iU_i$,满足$f$在每个$U_i$上的限制都是嵌入/开嵌入,那么$f$是嵌入/开嵌入.
    	\begin{proof}
    		
    		从$f\mid_{U_i}$都是嵌入/开嵌入得到$f$在层上是满态射/同构.所以问题归结为证明$f$是单射,一旦这成立,从$f\mid_{U_i}$是到局部闭子集/开子集的拓扑嵌入,就得到$f$是到局部闭子集/开子集的拓扑嵌入.因为$f$是单射只依赖于底空间,所以我们不妨设$X,Y$都是既约的,否则可以用$X_{\mathrm{red}},Y_{\mathrm{red}},f_{\mathrm{red}}$替代$X,Y,f$.我们还可以用$Y$的既约闭子概型$\overline{f(X)}$替代$Y$(一般的如果$f:X\to Y$是源端既约的态射,如果$f(X)$作为集合落在$Y$的闭子概型$Z$中,那么$f$会经闭嵌入$Z\to Y$分解),于是不妨设$f$是支配态射.设$f$是浸入,那么对任意$x\in X$有$f_x^{\#}:\mathscr{O}_{Y,f(x)}\to\mathscr{O}_{X,x}$是满射.但是又因为$f$是支配的并且$Y$是既约的,我们有这些茎同态还是单射,于是这些茎同态都是同构,于是$f$是平坦态射.进而两个投影态射$p_1,p_2:X\times_YX\to X$作为$f$的基变换也是平坦的.我们解释过平坦满足下降条件,于是$p_i(X\times_YX)$在一般化下不变.
    		
    		\qquad
    		
    		下面设$\eta_1,\cdots,\eta_n\in X$是全部一般点.那么每个$U_i$要包含全部这些点,所以按照$f\mid_{U_i}$是单射得到$f$限制在$\{\eta_1,\cdots,\eta_n\}$上是单射.又因为$f$是平坦的,所以每个$\theta_i=f(\eta_i)$都是$Y$的一般点,并且按照$f$是支配的说明这些是$Y$的全部一般点.
    		
    		\qquad
    		
    		下面设$\xi$是$X\times_YX$的一般点,那么$\xi$在$Y$中的像是一般点,记作$\theta_j$.另外$p_1(\xi)$和$p_2(\xi)$都是$X$的一般点,并且被$f$打到相同的点$\theta_j$,所以有$p_1(\xi)=p_2(\xi)=\eta_j$.于是$\xi$落在纤维$Z=\mathrm{Spec}(\kappa(\eta_j)\otimes_{\kappa(\theta_j)}\kappa(\eta_j))=p_1^{-1}(\eta_j)\cap p_2^{-1}(\eta_j)$中.但是因为$\kappa(\theta_j)\to\kappa(\eta_j)$是同构,于是$Z$是域的素谱,于是纤维积泛性质得到的$Z\to X\times_YX$要经对角态射$\Delta:X\to X\times_YX$分解,这导致$Z$的唯一点$\{\xi\}$落在$\Delta(X)$中,于是我们证明了$\Delta(X)$包含了$X\times_YX$的所有一般点.又因为$f$是分离的,所以$\Delta(X)$是闭集,这就导致作为集合$\Delta(X)=X\times_YX$,也即$\Delta$是满射,于是$f$是泛单的(我们解释过泛单等价于对角态射是满射,这里也可以直接证明$f$是单射:设$x+1,x_2\in X$使得$f(x_1)=f(x_2)$,那么存在$z\in X\times_YX$使得$p_1(z)=x_1$和$p_2(z)=x_2$,按照$\Delta$是满射得到$x_0\in X$使得$\Delta(x_0)=z$,于是$x_0=x_1=x_2$).
    	\end{proof}
        \item 设$f:X'\to X$是整概形之间的分离态射也是满射,满足它诱导的函数域扩张$K(X)\to K(X')$是同构.再设$f$满足如下两个条件的任意一个,那么$f$是平坦态射当且仅当它是同构.
        \begin{enumerate}[(a)]
        	\item $f$是仿射态射(比方说$f$是正规化).
        	\item $f$是局部有限表示态射(比方说$f$是诺特整概形$X$沿某个闭子概型的爆破).
        \end{enumerate}
        \begin{proof}
        	
        	所有条件以及结论都是终端局部的,所以不妨设$X=\mathrm{Spec}A$是仿射的,于是$A$是一个整环.先设(a)成立,那么有$X'=\mathrm{Spec}A'$也是仿射的.按照$f$平坦以及满射,说明它是忠实平坦态射,于是$A'$在$A$上忠实平坦,于是结构同态$A\to A'$是单射,于是整环$A'$落在$A$的商域中.我们断言有$A'=A$:设$a'=a/s\in A'$,其中$a,s\in A$,$s\not=0$,那么$a\in sA'\cap A=sA$(这个等式是因为从$A\to B$是忠实平坦的得到对任意理想$\mathfrak{a}\subseteq A$有$\mathfrak{a}B\cap A=\mathfrak{a}$).这就导致$a'\in A$.于是$f$是同构.
        	
        	\qquad
        	
        	再设(b)成立:任取$x'\in X'$,因为局部环上平坦和忠实平坦等价,于是$\mathscr{O}_{X',x'}$在$\mathscr{O}_{X,f(x')}$上忠实平坦.又因为$X,X'$都是整概形,这两个局部环都是整环,于是$\mathscr{O}_{X',x'}$落在$\mathscr{O}_{X,f(x')}$的商域中.所以按照上述讨论依旧得到$\mathscr{O}_{X,f(x')}\to\mathscr{O}_{X',x'}$是同构.又因为$f$是局部有限表示态射,我们解释过这个条件下stalk同构的点是局部同构的,也即存在$x'$的开邻域$U'$使得$f\mid_{U'}$是开嵌入.让$x'$跑遍$X'$的所有点,按照上述引理得到$f$是开嵌入,又因为$f$是满射,这就得到$f$是同构.
        \end{proof}
    \end{enumerate}
    \item 终端是戴德金概形的平坦态射.这一条中我们提及戴德金概形总是指一个维数$\le1$的诺特正规整概形$X$,这也等价于对每个仿射开子集$U$有$\Gamma(U,\mathscr{O}_X)$是戴德金整环.
    \begin{enumerate}
    	\item 设$f:X\to Y$是平坦态射,$Y$是不可约的,那么对$X$的每个非空开集$U$,都有$U$支配了$Y$(此为$f(U)$在$Y$中稠密).如果$X$只有有限个不可约分支,那么每个不可约分支都支配了$Y$(此即每个不可约分支在$f$下的像都在$Y$中稠密).
    	\begin{proof}
    		
    		平坦条件是终端和源端局部的,$Y$不可约所以非空开子集都是不可约的,所以归结为仿射情况.设$Y=\mathrm{Spec}A$,设$U=\mathrm{Spec}B$,那么$f:U\to Y$是平坦态射.记$Y$的一般点为$\eta$,记$A$的幂零根为$N$,由于$B$是平坦$A$代数,得到$B/NB=B\otimes_A(A/N)\subseteq B\otimes_A\mathrm{Frac}(A/N)=B\otimes_A\kappa(\eta)$.假设$\eta$不在$f$的像集中,那么纤维环$B\otimes_A\kappa(\eta)$没有素理想,所以是零环,所以$B=NB$是幂零的,导致$B$是零环,导致$U=\emptyset$矛盾.这说明$\eta$在$f$的像集中,于是$f(U)$在$Y$中稠密.
    		
    		\qquad
    		
    		下面设$X$只有有限个不可约分支,那么每个不可约分支都包含一个非空开集,所以按照前半段命题得到不可约分支的像集在$Y$中稠密.
    	\end{proof}
        \item 设$Y$是诺特整戴德金概形,设$\eta$是它的一般点,设$f:X\to Y$是态射,考虑如下三个条件,我们有(1)$\Leftrightarrow$(2)$\Rightarrow$(3),如果额外的$X$还是既约的,那么三个条件互相等价.这个结论告诉我们的是,要想平坦化一个戴德金概形上的概形$X$,我们只要把$X$替换为它一般纤维的概形闭包即可.另外如果$X$是局部诺特既约概形,那么条件(2)等价于讲$f^{-1}(\eta)$(这是个开集,因为它是诺特概形并且保一般化)包含了$X$的所有伴随点.
        \begin{enumerate}[(1)]
        	\item $f$是平坦态射.
        	\item $X$(作为概形)是一般纤维$f^{-1}(\eta)$在$X$中的概形闭包.
        	\item $X$的每个不可约分支都支配了$Y$.
        \end{enumerate}
        \begin{proof}
        	
        	(1)和(2)的等价性:所有条件都是关于$X$和$Y$局部的,所以不妨设$Y=\mathrm{Spec}R$,其中$R$是戴德金整环,$X=\mathrm{Spec}A$.记$f:X\to Y$对应的环同态为$\varphi:R\to A$.它的一般纤维(即在$Y$的一般点的纤维)是$\mathrm{Spec}S^{-1}A$,其中$S=R-\{0\}$,我们有$S^{-1}A=A\otimes_RK$,其中$K=\mathrm{Frac}(R)$.另外有同态$A\to A\otimes_RK$为$a\mapsto a\otimes1$,它的核是$A_{\mathrm{tors}}$,即$A$的所有挠元构成的理想($a\in A$是关于$R$的挠元指的是存在$0\not=r\in R$使得$\varphi(r)a=0$,那么$a\in\ker(A\to A\otimes_RK)$当且仅当$a/1=0\in S^{-1}A$,当且仅当存在$0\not=r\in R$使得$\varphi(r)a=0$,也即$a\in A_{\mathrm{tors}}$).于是我们有环的单射$A/A_{\mathrm{tors}}\to A\otimes_RK$,这诱导了支配态射$\mathrm{Spec}A\otimes_RK\to\mathrm{Spec}A/A_{\mathrm{tors}}$,后者是$\mathrm{Spec}A$的闭子集,我们还解释过对拟紧嵌入$i:S_1\to S_2$,有$\overline{X}$就是$i$的概形闭包,于是这里$f$的一般纤维的概形闭包就是$\mathrm{Spec}A/A_{\mathrm{tors}}$.这作为概形是$X$当且仅当$A$关于$R$是无挠的,而戴德金整环上的模是平坦模等价于无挠模,这就得到前两个命题等价.
        	
        	\qquad
        	
        	(1)推(3)因为平坦态射把一般点映为一般点.最后如果$X$是既约的,从(3)得到$f^{-1}(\eta)$包含了$X$的所有一般点,于是$f^{-1}(\eta)$的闭包作为集合是整个$X$,但是因为$X$既约,具有相同底空间的闭子概型只有自身一个,于是$f^{-1}(\eta)$的概形闭包作为概形是整个$X$.
        \end{proof}
        \item 推论.设$f:X\to Y$是整概形到戴德金概形的非常值态射,那么$f$是平坦的.
        \begin{proof}
        	
        	因为$f$不是常值的,导致$\overline{f(X)}$是维数至少为1的闭子集,但是$Y$是不可约的至多1维的空间,于是迫使$\overline{f(X)}=Y$,也即$X$的不可约分支支配了$Y$,于是$f$是平坦的.
        \end{proof}
        \item DVR的情况.设$Y=\mathrm{Spec}R$,其中$R$是DVR,我们知道$Y$包含两个点,一个是零理想也是一般点,一个是唯一的闭点$(\pi)$,其中$\pi$是$R$的uniformizer.设$f:X\to Y$是概形之间的态射,则这两个点的纤维分别称为一般纤维和闭纤维.此时对$R$模$M$,一个元$m\in M$是关于$R$的挠元当且仅当$m$被$\pi$零化,于是一个$R$模$M$是平坦模当且仅当它的非零元都不被$\pi$零化,于是我们有如下互相等价的命题.其中(3)和(4)等价是因为如果$x$在一般纤维中,环同态$R_{0}\to\mathscr{O}_{X,x}$把$\pi$映射为单位,所以此时$\pi$总不零化$\mathscr{O}_{X,x}$的非零元.
        \begin{enumerate}[(1)]
        	\item $f$是平坦态射.
        	\item 对任意仿射开子集$U\subseteq X$,有$\Gamma(U,\mathscr{O}_X)$中的非零元不被$\pi$零化.
        	\item 对任意$x\in X$,有$\mathscr{O}_{X,x}$中的非零元不被$\pi$零化.
        	\item 对闭纤维中的任意点$x$,有$\mathscr{O}_{X,x}$中的非零元不被$\pi$零化.
        \end{enumerate}
        \item 设$R$是DVR,设它的一般点是$\eta$,唯一闭点是$s$.设$f:X\to\mathrm{Spec}R$是源端为局部诺特概形的态射,设一般纤维的拓扑闭包是整个$X$,并且闭纤维$X_s$是既约概形,那么$f$是平坦态射.
        \begin{proof}
        	
        	记$\pi$是$R$的uniformizer,按照上一条,我们要证明的是对$x\in X_s$,有$\pi$是$\mathscr{O}_{X,x}$的正则元.这是stalk上的问题,不妨设$X=\mathrm{Spec}A$是仿射的,其中$A$是诺特局部环.记$f$对应的环同态是$\varphi:R\to A$.不妨设$\varphi(\pi)$落在$A$的极大理想中,否则$\varphi(\pi)$是$A$的单位元已经是正则元.设有$a\in A$满足$\varphi(\pi)a=0$,我们只需证明$a=0$.一般纤维是$f^{-1}(\eta)=\mathrm{Spec}A\otimes_RK$,其中$K=\mathrm{Frac}R$.由于$A$的素理想$\mathfrak{p}\in f^{-1}(\eta)$要满足$\varphi^{-1}(\mathfrak{p})=\{0\}$,于是$\varphi(\pi)\not\in\mathfrak{p}$,于是从$\varphi(\pi)a=0\in\mathfrak{p}$就得到$a\in\mathfrak{p}$,于是$f^{-1}(\eta)=\mathrm{Spec}A\otimes_RK$落在$V(a)$中.但是条件要求了一般纤维的拓扑闭包是整个$X$,于是这里$V(a)$作为集合是整个$\mathrm{Spec}A$,于是$a$是$A$的幂零元.于是$a$是闭纤维$R/\pi R\otimes_RA=A/\varphi(\pi)A$的幂零元,但是闭纤维按照条件是既约的,于是$a=\varphi(\pi)a'$.那么依旧有$f^{-1}(\eta)=\mathrm{Spec}A\otimes_RK$包含在$V(a')$中,进而得到$a'\in\varphi(\pi)A$,操作下去得到$a\in\cap_i\varphi(\pi)^iA$,按照$\varphi(\pi)$落在$A$的极大理想中,NAK引理导致$\cap_i\varphi(\pi)^iA=\{0\}$,于是$a=0$,得证.
        \end{proof}
        \item 上面几条告诉我们DVR上的平坦态射很简单,后文会给出平坦性的赋值准则,告诉我们在一些条件下DVR的基变换还能决定一般态射的平坦性.
    \end{enumerate}
    \item 平坦性的纤维准则.
    \begin{enumerate}
    	\item 设$S$是概形,设$g:X\to S$和$h:Y\to S$是两个$S$概形,设$f:X\to Y$是$S$态射.设$\mathscr{F}$是拟凝聚$\mathscr{O}_X$模层,设$x\in X$,设$y=f(x)$,设有$s=h(y)=g(x)$.设$\mathscr{F}_x\not=0$.设如下两个条件之一成立:
    	\begin{enumerate}[(1)]
    		\item $S,X,Y$都是局部诺特概形,$\mathscr{F}$是凝聚层.
    		\item $g,h$是局部有限表示态射,$\mathscr{F}$是有限表示模层.
    	\end{enumerate}
    
        那么如下两个条件互相等价:
        \begin{enumerate}[(i)]
        	\item $\mathscr{F}$在点$x$是$g$平坦的,并且$\mathscr{F}_s=\mathscr{F}\times_XX_s$在点$x$是$f_s=f\times_S\mathrm{Spec}\kappa(s)$平坦的.
        	\item $h$在点$y$平坦,并且$\mathscr{F}$在点$x$是$f$平坦的.
        \end{enumerate}
        \begin{proof}
        	
        	首先(ii)推(i)是容易的并且不需要任何额外条件:(ii)要求$\mathscr{F}_x$在$\mathscr{O}_{Y,y}$上平坦,和$\mathscr{O}_{Y,y}$在$\mathscr{O}_{S,s}$上平坦.于是就有$\mathscr{F}_x$在$\mathscr{O}_{S,s}$上平坦,此为(i)的前半部分.后半部分因为平坦性在基变换下不变.
        	
        	\qquad
        	
        	下面证明(i)推(ii).条件(b)的情况可以约化到(a)的情况,见EGA,我们来证明条件(a)的情况.由于平坦性是茎局部的,所以问题归结为所有概形都是局部环的仿射概形的情况.改写一下条件:设$\varphi:A\to B$和$\psi:B\to C$是诺特局部环之间的局部同态,设$M$是有限$C$模,我们要证明如下条件有(i)推(ii):
        	\begin{enumerate}[(i)]
        		\item $M$是平坦$A$模,并且$M\otimes_Ak$是平坦$B\otimes_Ak$模,这里$k$是$A$的剩余域.
        		\item $B$是平坦$A$模,并且$M$是平坦$B$模.
        	\end{enumerate}
        
            我们先来证明$M$是平坦$B$模.首先回顾下局部平坦准则,设$I\subseteq R$是理想,设$N$是$R$模,记$\mathrm{gr}^I(N)=\oplus_{n\ge0}N_n$,其中$N_n=I^nN/I^{n+1}N$.于是有典范的满的分次模同态$\gamma_N^I:\mathrm{gr}^I(R)\otimes_{R_0}N_0\to\mathrm{gr}^I(N)$.局部平坦准则指的是$N$是平坦$R$模当且仅当$N_0$是平坦$R_0$模(这个条件在我们取$I$是极大理想时是一定成立的,因为域上的模总是平坦的),并且$\gamma_N^I$是同构.回到原命题的证明,设$\mathfrak{m}$是$A$的唯一极大理想,考虑如下交换图表,其中$v=\gamma_M^{\mathfrak{m}}$,$u=\gamma_B^{\mathfrak{m}}$,$w=\gamma_M^{\mathfrak{m}B}$,我们有$\mathrm{gr}_A^{\mathfrak{m}}(B)=\mathrm{gr}^{\mathfrak{m}B}_B(B)$,并且$B_0=B/\mathfrak{m}B$:
            $$\xymatrix{\mathrm{gr}^{\mathfrak{m}}(A)\otimes_{A_0}M_0\ar[rrr]^v\ar@{=}[d]&&&\mathrm{gr}^{\mathfrak{m}}(M)\\(\mathrm{gr}^{\mathfrak{m}}(A)\otimes_{A_0}B_0)\otimes_{B_0}M_0\ar[rrr]_{u\otimes 1}&&&\mathrm{gr}^{\mathfrak{m}}(B)\otimes_{B_0}M_0\ar[u]_w}$$
        
        	因为$M$是平坦$A$模得到$v$是同构,又因为$u\otimes1$和$w$都是满射,于是得到$u\otimes1$和$w$都是同构.又因为(i)还要求了$M\otimes_Ak=M/\mathfrak{m}M$是平坦$B\otimes_Ak=B/\mathfrak{m}B$模,结合$w$是同构,把局部平坦准则用在$B$的理想$\mathfrak{m}B$上得到$M$是平坦$B$模.这证明了(ii)的后半部分.
        	
        	\qquad
        	
        	因为$B$是局部环,所以$M$甚至是忠实平坦$B$模,进而有基变换$M_0=M\otimes_Ak$是忠实平坦$B_0=B\otimes_Ak$模.于是从$u\otimes1$是同构得到$u$是同构,把局部平坦准则再用到$A$模$B$上,就得到$B$在$A$上平坦.
        \end{proof}
        \item 推论.取$\mathscr{F}=\mathscr{O}_X$,我们有如下结论:设$S$是概形,设$g:X\to S$和$h:Y\to S$是局部有限表示的$S$概形,设$f:X\to Y$是$S$态射,那么如下两个条件互相等价:
        \begin{enumerate}[(i)]
        	\item $g$是平坦态射,并且对任意$s\in S$有$f_s:X_s\to Y_s$是平坦的.
        	\item $h$在$f(X)$中的点平坦,并且$f$是平坦的.
        \end{enumerate}
        \item 设$S$是概形,设$g:X\to S$和$h:Y\to S$是两个$S$概形,设$g$和$h$都是有限表示的紧合态射,并且$g$还是平坦的.设$f:X\to Y$是$S$态射,设$s\in S$使得$f_s=f\times\mathrm{id}_{\kappa(s)}:X\times_S\mathrm{Spec}\kappa(s)\to Y\times_S\kappa(s)$是同构,那么存在$s$的开邻域$U$使得$f\mid_{g^{-1}(U)}:g^{-1}(U)\to h^{-1}(U)$是同构.
        \begin{proof}
        	
        	我们解释过$X\to Y\to S$和$Y\to S$是有限表示态射得到$f:X\to Y$是有限表示态射.我们之前在Zariski主定理推论中解释过这里$f_s$是闭嵌入可以推出存在$s$的开邻域$U$使得$f\mid_{g^{-1}(U)}:g^{-1}(U)\to h^{-1}(U)$是闭嵌入.所以适当把$X,Y,S$替换为开子概型可以不妨设$f$本身是闭嵌入.按照平坦的纤维准则,有$f$在所有$g^{-1}(s)$中的点是平坦的.另外这里$f$是局部有限表示的闭嵌入,我们解释过对平坦点$x\in g^{-1}(s)$,记$y=f(x)$,那么存在$y$的开邻域$V_y$使得$f$限制为$f^{-1}(V_y)\to V_y$的同构.
        	
        	这里$f$是有限表示的闭嵌入,而且在$x\in g^{-1}(s)$上平坦.按照$f$是闭嵌入,有$f(g^{-1}(s))=h^{-1}()$.整理下现在$f:X\to Y$是有限表示的闭嵌入,并且在【】
        \end{proof}
    \end{enumerate}
    \item 平坦性的赋值准则.
    \begin{enumerate}
    	\item 平坦性的下降.设$(A,\mathfrak{m})$【】
    \end{enumerate}
    \item 平坦性和泛开态射.
    \begin{enumerate}
    	\item 我们解释过局部有限表示的平坦态射是泛开态射.特别的光滑态射总是泛开态射;特别的局部有限表示的忠实平坦态射作为集合映射是商映射(一个连续映射$f:X\to Y$称为商映射,如果子集$V\subseteq Y$是开集当且仅当$f^{-1}(V)\subseteq X$是开集,满射开映射一定是商映射).另外我们会证明拟紧的完全忠实平坦态射是开映射.这里满射是必须的,存在诺特概形之间的拟紧平坦态射不是开映射:取$Y=\mathrm{Spec}\mathbb{Z}$,设一般点是$\eta$,取$X=Y\coprod\mathrm{Spec}\kappa(\eta)$,考虑典范态射$X\to Y$,它满足所有条件,但是$\{\eta\}$不是$Y$的开子集导致它不是开映射.
    	\item 但是在适当去掉平凡反例的时候这个逆命题是成立的:设$Y$是既约局部诺特概形,设$f:X\to Y$是局部有限型的泛开态射,设每个纤维$X_y=f^{-1}(y),y\in Y$都是几何既约的$\kappa(y)$概形,那么$f$是平坦态射.
    	\begin{proof}
    		
    		这里$Y$是既约局部诺特概形,验证$f$的平坦性可以用赋值准则.所以设$R$是DVR,设有态射$\mathrm{Spec}R\to Y$,我们要证明基变换$f':X'=X\times_Y\mathrm{Spec}R\to\mathrm{Spec}R$是平坦的.设$s,\eta$是$R$的唯一闭点和一般点,设$X'_s$和$X'_{\eta}$是对应的闭纤维和一般纤维,按照DVR终端的态射的平坦性的描述,只需证明$X'_s$是既约的,并且$X'_s$包含在$X'_{\eta}$的闭包中.前一件事没什么需要证的,设$y$是$s$在$Y$中的像,那么$X_y$是几何既约的得到关于域的基变换$X'_s$是既约的($X'_s=X\times_Y\mathrm{Spec}R\times_{\mathrm{Spec}R}\mathrm{Spec}\kappa(s)=X\times_Y\mathrm{Spec}\kappa(s)=X_y\times_{\mathrm{Spec}\kappa(y)}\mathrm{Spec}\kappa(s)$).对于后一件事,任取$X'_s$的一般点$\xi$,按照闭集保特殊化,我们只需验证$X'_{\eta}$里存在$\xi$的一般化.假设这不成立,那么$\xi$在整个$X'$里都不存在非平凡的一般化,于是$\xi$是整个$X'$的一般点,它的闭包理应是$X'$的一个不可约分支,$\xi$是闭纤维中的点,所以这个不可约分支$\overline{\{\xi\}}$由闭纤维中的点构成.下面取$\xi$的仿射开邻域$U=\mathrm{Spec}A$,使得$A$是一个有限型$R$代数,那么$A$也是诺特环(DVR自动是诺特的),并且$\overline{\{\xi\}}\cap U$是$U$的不可约分支,但是按照$U$的不可约分支只有有限个,存在$\overline{\{\xi\}}\cap U$的开子集$U_0$使得它也是$U$的开子集,那么这个$U_0$也是$X'$的开子集,由于它是$X'_s$的子集导致它的像集是$\{s\}$,但是这不是$\mathrm{Spec}R$的开子集,这就和$X'\to\mathrm{Spec}R$是开映射矛盾.
    	\end{proof}
    \end{enumerate}
    \item 拟紧平坦态射的拓扑性质.我们主要证明的是拟紧忠实平坦态射是泛浸没.
    \begin{enumerate}
    	\item 引理.设$Y$是qcqs概形,设$Z\subseteq Y$是预可构造集,设$f:X\to Y$是概形的态射满足下降条件(此为如果$x\in X$,记$y=f(x)$,如果$y'$是$y$的一般化,那么存在$x$的一般化$x'$使得$f(x')=y'$.这也等价于讲对任意$x\in X$,记$y=f(x)$,那么有$f(\mathrm{Spec}\mathscr{O}_{X,x})=\mathrm{Spec}\mathscr{O}_{Y,y}$).那么有$\overline{f^{-1}(Z)}=f^{-1}(\overline{Z})$.
    	\begin{proof}
    		
    		取$Y$的有限仿射开覆盖$\{V_i\}$,那么每个$V_i$都是$Y$的预可构造集,预可构造集的交仍然是预可构造集,于是每个$V_i\cap Z$都是$Y$的预可构造集.所以一旦我们对$V_i$和$V_i\cap Z$证明了结论中的等式,按照有限并和闭包可交换,就得到结论成立.换句话讲不妨设$Y=\mathrm{Spec}A$是仿射的.
    		
    		\qquad
    		
    		一方面有$\overline{f^{-1}(Z)}\subseteq f^{-1}(\overline{Z})$.下面设$x\in U=X-\overline{f^{-1}(Z)}$,我们要证明的是$y=f(x)\not\in\overline{Z}$.按照预可构造集的定义,我们有仿射概形为源端的态射$g:Y'=\mathrm{Spec}A'\to Y$,满足$g(Y')=Z$.因为$x\in U$,这里$U$是开集,所以$x$的一般化都在$U$中,也即$\mathrm{Spec}\mathscr{O}_{X,x}\subseteq U\subseteq f^{-1}(Y-Z)$.按照$f$满足下降条件,我们有$\mathrm{Spec}\mathscr{O}_{Y,y}=f(\mathrm{Spec}\mathscr{O}_{X,x})\subseteq f(f^{-1}(Y-Z))\subseteq Y-Z$.设$y$对应的$A$的素理想是$\mathfrak{p}$.那么有$g^{-1}(\mathrm{Spec}\mathscr{O}_{Y,y})$是空集,也即$A_{\mathfrak{p}}\otimes_AA'=0$,按照$A_{\mathfrak{p}}\otimes_AA'=\lim\limits_{\rightarrow}(A_s\otimes_AA')$,其中$s$跑遍$A-\mathfrak{p}$中的元,于是按照正向极限$A_{\mathfrak{p}}\otimes_AA'$中$1=0$,就存在$s\in A-\mathfrak{p}$使得$A_s\otimes_AA'$中也有$1=0$.换句话讲$D(s)\cap Z$是空集,于是$y\not\in\overline{Z}$.
    	\end{proof}
        \item 引理.设$f:X\to Y$是拟紧态射,满足下降条件,那么$f$作为$X\to f(X)$的集合映射是一个商映射,换句话讲$f(X)$的子集$V$是开集当且仅当$f^{-1}(V)$是$X$的开子集.
        \begin{proof}
        	
        	问题是局部的,不妨设$Y$是仿射的.我们只需证明如果子集$Z\subseteq f(X)$满足$F=f^{-1}(Z)$是$X$的闭子集,那么$Z$是$f(X)$的闭子集,归结为证明$Z\supseteq\overline{Z}\cap f(X)$.赋予$F$既约闭子概型结构,那么$F\to X\to Y$仍是拟紧态射,于是它的像集$Z$是预可构造集.按照上一条引理我们有$\overline{Z}\cap f(X)=f(f^{-1}(\overline{Z}))=f(\overline{f^{-1}(Z)})=f(f^{-1}(Z))\subseteq Z$.
        \end{proof}
        \item 推论.设$f:X\to Y$是拟紧忠实平坦态射,那么$f$作为集合映射是一个商映射.
        \item 推论.我们称一个满态射是浸没(submersive),如果它作为集合映射是商映射.于是我们证明了拟紧忠实平坦态射是泛浸没.
    \end{enumerate}
    \item 平坦点的开集性质.这里我们给出满足某些条件的点构成开集的性质.
    \begin{enumerate}
    	\item 【】
    \end{enumerate}
    
    【】
    
    
    
    
\end{enumerate}
\subsection{Zariski主定理}

局部版本.
\begin{enumerate}
	\item 引理1.设$X$是域上的局部有限型概形,设$x\in X$,有如下命题互相等价.
	\begin{enumerate}
		\item $x$是$X$的孤立点,此即它是单点开集也是单点闭集(闭点).
		\item $x$是$X$的单点开集.
		\item $\dim_xX=0$,此为$x\in X$的局部维数,在局部有限型概形上局部维数就是过该点的不可约分支的维数的上确界,于是这一条等价于讲过该点的$X$的不可约分支都是零维的.
	\end{enumerate}
	\begin{proof}
		
		我们解释过域上局部有限型概形$X$的非空开子集$U$上总包含$X$的闭点,于是这里(a)和(b)是等价的.(b)推(c)是因为$\dim_xX=\dim_x\{x\}=0$.(c)推(b)是因为,选取$x$的仿射开邻域$U$,那么$\dim_xU=\dim_xX=0$,并且$\{x\}$在$U$中是开集等价于$\{x\}$在$X$中是开集.于是归结为设$X$是仿射的.那么$\dim_xX=0$也就是说$x$对应的素理想同时是极大的也是极小的.但是由于$X$对应的环是诺特的,它只有有限个极小素理想,于是$\{x\}$作为其它极小素理想的闭包的并的补集,就是开集.
	\end{proof}
	\item 纤维孤立点.如果$f:X\to Y$是概形之间的态射,任取点$x\in X$,记$y=f(x)$,如果$x$在纤维$X_y$中是孤立点(此即它是单点开集也是闭点),就称$x\in X$是纤维孤立点,态射$f$的全部纤维孤立点构成的集合记作$\mathrm{Isol}(f)$或者$\mathrm{Isol}(X/Y)$,或者对于仿射情况记作$\mathrm{Isol}(B/A)$.
	\begin{itemize}
		\item 明显的,如果$f:X\to Y$和$g:Y\to Z$是两个概形之间的态射,那么有:
		$$\mathrm{Isol}(g\circ f)\subseteq\mathrm{Isol}(f)$$
		\item 如果$f:X=\mathrm{Spec}B\to Y=\mathrm{Spec}A$是仿射概形之间的有限型态射,设对应的环同态为$\varphi:A\to B$,那么按照引理1,素理想$\mathfrak{q}\in\mathrm{Spec}B$是$f$的纤维孤立点当且仅当$\mathfrak{q}$在纤维里是极大元也是极小元,换句话讲如果记$\mathfrak{p}=\mathfrak{q}\cap A$,如果$\mathfrak{q}'$是$\mathfrak{p}$的另一个提升素理想,并且和$\mathfrak{q}$有包含关系,那么$\mathfrak{q}=\mathfrak{q}'$(也称为提升素理想的不可比条件).
	\end{itemize}
	\item 引理2.设$A\subseteq B$是子环,记$Y=\mathrm{Spec}A$和$X=\mathrm{Spec}B$,设$f:X\to Y$是包含同态$A\subseteq B$对应的态射.下面设$A$在$B$中整闭,设$B=A[b]$,其中$b\in B$(换句话讲环扩张$A\subseteq B$是单扩张),设$y\in Y$对应的素理想是$\mathfrak{p}\subseteq A$.那么纤维$X_y=f^{-1}(y)=\mathrm{Spec}B_{\mathfrak{p}}/\mathfrak{p}B_{\mathfrak{p}}$只会发生如下三种情况:
	\begin{enumerate}
		\item $X_y$是空集.
		\item $X_y$恰由单点构成,并且此时存在$y$的开邻域$V$使得$f$诱导了同构$f^{-1}(V)\cong V$.
		\item 我们有$(A/\mathfrak{p})[T]\cong B/\mathfrak{p}B$.此时$X_y\cong\mathbb{A}_{\kappa(y)}^1$,并且$X_y$的一般点$x$满足$\kappa(x)/\kappa(y)$是超越扩张.
	\end{enumerate}
    \begin{proof}
    	
    	设$\varepsilon:A[T]\to B$为$T\mapsto b$诱导的满环同态.我们先断言$\ker\varepsilon$作为$A[T]$的理想是被形如$cT+d$其中$c,d\in A$的一次多项式生成的理想.设$g(T)=a_nT^n+a_{n-1}T^{n-1}+\cdots+a_0\in\ker\varepsilon$,其中$a_n\not=0$.那么$T=b$带入为零,对这个式子两边乘以$a_n^{n-1}$,得到:
    	$$(a_nb)^n+a_{n-1}(a_nb)^{n-1}+a_{n-2}a_n(a_nb)^{n-2}+\cdots+a_0a_n^{n-1}=0$$
    	
    	按照$A$在$B$中整闭,于是$a=a_nb\in A$,于是$a_nT-a\in\ker\varepsilon$,并且有$g(T)=T^{n-1}(a_nT-a)+h(T)$,其中$\deg h<n$,于是对$n$归纳得到$g(T)$一定可以被若干一次多项式生成.完成断言的证明.
    	
    	\qquad
    	
    	设$\varepsilon:A[T]\to B$复合上$B\to B/\mathfrak{p}B$是$\overline{\varepsilon}$.那么$\mathfrak{p}A[T]\subseteq\ker\overline{\varepsilon}$.假设$\ker\overline{\varepsilon}\subseteq\mathfrak{p}A[T]$,这等价于讲$\ker\varepsilon\subseteq\mathfrak{p}A[T]$.那么此时有同构$(A/\mathfrak{p})[T]\cong B/\mathfrak{p}B$,于是此时纤维$X_y=\mathrm{Spec}B_{\mathfrak{p}}/\mathfrak{p}B_{\mathfrak{p}}=\mathrm{Spec}(B/\mathfrak{p}B)_{\mathfrak{p}}=\mathrm{Spec}\kappa(\mathfrak{p})[T]=\mathbb{A}_{\kappa(y)}^1$,于是$\kappa(x)=\kappa(y)(T)$在$\kappa(y)$上超越.此为情况(c).
    	
    	\qquad
    	
    	下面设$\ker\varepsilon\not\subseteq\mathfrak{p}A[T]$.也即存在$c,d\in A$满足$cb=d$,但是$cT+d\not\in\mathfrak{p}A[T]$,换句话讲$c$和$d$中至少有一个不在$\mathfrak{p}$中.先设$c\not\in\mathfrak{p}$,那么有$A_c=B_c$(一方面$A_c\subseteq B_c$,另一方面$b/1=d/c\in A_c$得到$B_c\subseteq A_c$),于是$V=D(c)$就是$y$的开邻域使得$f$限制在$f^{-1}(V)$上是同构$f^{-1}(V)\cong V$.此为(b)的情况.
    	
    	\qquad
    	
    	最后设$c\in\mathfrak{p}$,但是$d\not\in\mathfrak{p}$.那么从$cb=d$得到$b$是$B_d$中的单位,并且$b^{-1}\in\mathfrak{p}B_d$,于是这迫使$\mathfrak{p}B_d=B_d$,按照$d\in B-\mathfrak{p}$再做局部化得到$\mathfrak{p}B_{\mathfrak{p}}=B_{\mathfrak{p}}$,导致$X_y=\mathrm{Spec}B_{\mathfrak{p}}/\mathfrak{p}B_{\mathfrak{p}}=\emptyset$.此为情况(a).
    \end{proof}
    \item 局部同构点.设$f:X\to Y$是概形之间的态射,设$x\in X$,记$y=f(x)$,称$x\in X$是$f$的局部同构点,如果存在$y$的开邻域$V$使得$f$可以限制为同构$f^{-1}(V)\cong V$.特别的,如果$A\subseteq B$是子环,设$f:X=\mathrm{Spec}B\to Y=\mathrm{Spec}A$是包含同态$A\subseteq B$对应的态射,由于主开集构成了$Y$上的拓扑基,所以一个素理想$\mathfrak{q}\subseteq B$是局部同构点当且仅当存在$s\in A$且$s\not\in\mathfrak{q}$,使得$A_s=B_s$.我们把$f:X\to Y$的全部局部同构点构成的集合记作$\mathrm{LocIsom}(X/Y)$或者$\mathrm{LocIsom}(f)$,或者对于环情况记作$\mathrm{LocIsom}(B/A)$.
    \item 纤维孤立点和局部同构点的一些性质.设$A\subseteq B$是子环,设环扩张对应的态射为$f:X=\mathrm{Spec}B\to Y=\mathrm{Spec}A$.
    \begin{enumerate}
    	\item 我们总有$\mathrm{LocIsom}(B/A)\subseteq\mathrm{Isol}(B/A)$.因为对于局部同构点$x\in X$,记$y=f(x)$,那么$y$的纤维已经只有单点$x$,所以$x$自然是纤维孤立点.
    	\item 对$A\subseteq B$的任意中间环$A'$(等价于讲$A$代数$B$的子代数$A'$),我们总有:
    	$$\mathrm{Isol}(B/A)\subseteq\mathrm{Isol}(B/A')$$
    	$$\mathrm{LocIsom}(B/A)\subseteq\mathrm{LocIsom}(B/A')$$
    	\begin{proof}
    		
    		第一个包含是因为我们解释过$\mathrm{Isol}(g\circ f)\subseteq\mathrm{Isol}(f)$.第二个包含是因为设素理想$\mathfrak{q}\in\mathrm{Spec}B$,设有$s\in A$使得$s\not\in\mathfrak{q}$,并且有$B_s=A_s$.那么$\mathfrak{q}'=\mathfrak{q}\cap A'$也满足$s\not\in\mathfrak{q}'$,并且从$A_s\subseteq A'_s\subseteq B_s$得到$A'_s=B_s$.
    	\end{proof}
        \item $\mathrm{LocIsom}(B/A)$是$\mathrm{Spec}B$的开子集.这按照定义就保证.
        \item 如果$S\subseteq A$是乘性闭子集,那么有如下关系式,如果$B$是子环$A$上的有限型代数,那么第二行的包含号可以改为等号.
        $$\mathrm{Isol}(S^{-1}B/S^{-1}A)=\mathrm{Isol}(B/A)\cap\mathrm{Spec}S^{-1}B$$
        $$\mathrm{LocIsom}(S^{-1}B/S^{-1}A)\supseteq\mathrm{LocIsom}(B/A)\cap\mathrm{Spec}S^{-1}B$$
        \begin{proof}
        	
        	如果$\mathfrak{p}\subseteq A$是和$S$不交的素理想,那么它纤维中的素理想,也即它在$B$中的提升素理想$\mathfrak{q}$也总和$S$不交(因为$\mathfrak{q}\cap A=\mathfrak{p}$,而$S\subseteq A$).所以对于$A$的和$S$不交的素理想,它在态射$\mathrm{Spec}B\to\mathrm{Spec}A$下的纤维和在态射$\mathrm{Spec}S^{-1}B\to\mathrm{Spec}S^{-1}A$下的纤维是同胚的.所以纤维中的素理想是否为孤立点是一致的,这得到第一个等式.
        	
        	\qquad
        	
        	如果$\mathfrak{q}\subseteq B$是素理想,满足和$S$不交,并且存在$s\in A$满足$s\not\in\mathfrak{q}$,使得$A_s=B_s$.那么$(S^{-1}A)_s=(S^{-1}B)_s$,并且从$\mathfrak{q}$和$S$不交得到$s/1$仍然不在$\mathfrak{q}$在$S^{-1}B$中对应的素理想,这得到第二个包含关系式.
        	
        	\qquad
        	
        	下面设$B$是子环$A$上的有限型代数,我们来证明第二行式子在改变包含号方向后成立.设$\mathfrak{q'}\subseteq B$是素理想,满足和$S$不交,并且存在$f'\in S^{-1}A$满足$f'\not\in S^{-1}\mathfrak{q}'$,使得$(S^{-1}A)_{f'}=(S^{-1}B)_{f'}$.设$B=A[b_1,\cdots,b_n]$,那么有$b_i/1=a_i/s_if'^{r_i}$,于是如果记$s=s_1s_2\cdots s_n$,那么在$B_{f's}$中就有$b_i/1=a_i'/sf'^{r_i}\in A_{f's}$,于是有$A_{f's}=B_{f's}$,并且从$s\in S\subseteq A-\mathfrak{q}$得到$s\in A$和$s\not\in\mathfrak{q}$,进而有$f's\in A$和$f's\not\in\mathfrak{q}$,这就得到$\mathfrak{q}\in\mathrm{LocIsom}(B/A)\cap\mathrm{Spec}S^{-1}B$.
        \end{proof}
        \item 设$A\subseteq B$是有限型代数.设存在$b\in B$使得$A[b]\subseteq B$是整扩张,且$A$在$A[b]$中整闭.那么$\mathrm{Isol}(B/A)$在一般化下不变.
        \begin{proof}
        	
        	设$\mathfrak{q}\in\mathrm{Isol}(B/A)$,设有$B$的素理想$\mathfrak{q}'\subseteq\mathfrak{q}$,但$\mathfrak{q}'\not\in\mathrm{Isol}(B/A)$.记$\mathfrak{q}_1'=\mathfrak{q}'\cap B_1$,其中$B_1=A[b]$,由于$B_1\subseteq B$是整扩张和有限型代数,从$\mathfrak{q}'\not\in\mathrm{Isol}(B/A)$可推出$\mathfrak{q}_1'\not\in\mathrm{Isol}(B_1/A)$.再记$\mathfrak{p}'=\mathfrak{q}'\cap A$.
        	
        	\qquad
        	
        	我们把引理2用在$A\subseteq B_1$和$A$的素理想$\mathfrak{p}'$上,因为$\mathfrak{q}'_1$不是纤维孤立点,所以唯一可以发生的情况是(c).于是$\mathfrak{q}'_1$是$\mathbb{A}_{\kappa(\mathfrak{p}')}^1$中的元,把$\mathfrak{q}'_1$替换为这个仿射线的一般点(如果$\mathfrak{q}'_1$本身就是一般点,则不需要任何操作),因为域上仿射线就没有孤立点,所以这样替换后$\mathfrak{q}'_1$仍然$\not\in\mathrm{Isol}(B_1/A)$.
        	
        	\qquad
        	
        	按照情况(c)的$(A/\mathfrak{p}')[T]\cong B_1/\mathfrak{p}'B_1$,说明$\mathfrak{q}'_1=\mathfrak{p}'B_1$.那么我们依旧有环扩张$A/\mathfrak{p}'\subseteq B_1/\mathfrak{q}'_1$,并且右侧的零理想仍然不是纤维孤立点(因为环扩张$A\subseteq B_1$中$\mathfrak{q}_1'$的纤维,和这个新的环扩张中右侧零理想的纤维是同胚的),所以用这个环扩张替换我们的环扩张$A\subseteq B_1$,我们找到了一个整环扩张$A\subseteq B_1=A[b]$,使得右侧的零理想不是纤维孤立点.另外按照$B_1/\mathfrak{q}_1'\subseteq B/\mathfrak{q}'$仍然是单射,把$B$替换为$B/\mathfrak{q}'$仍然保证$\mathfrak{q}$是$A\subseteq B$的纤维孤立点.
        	
        	\qquad
        	
        	再设$B_2$是$B_1$在$\mathrm{Frac}(B_1)$中的整闭包,按照$B_1\subseteq B_2$是整扩张,满足上升条件,说明$B_2$的零理想也不是关于$A$零理想的纤维孤立点,于是我们不妨设$B_1$本身是正规整环.类似的可以把$A$替换为$B_1$中的整闭包,于是此时$A$也是正规整环,进而有$A[T]$是正规整环.
        	
        	\qquad
        	
        	按照$A[T]\to B_1\to B$是整环的整扩张,并且$A[T]$是正规整环,得到$A[T]\to B$同时满足上升条件和下降条件,这导致从$\mathfrak{q}$是$A\subseteq B$的纤维孤立点得到$\mathfrak{a}=\mathfrak{q}\cap A[T]$是$A\subseteq A[T]$的纤维孤立点.但是$\mathrm{Spec}A[T]\to\mathrm{Spec}A$的纤维都是某个域上的仿射线,这总没有孤立点,这个矛盾导致我们最开始设的$\mathfrak{q}'$一定在$\mathrm{Isol}(B/A)$,得证.
        \end{proof}
        \item 设$A\subseteq B$是有限型代数.设中间环$A'$在$A$上整,那么有如下等式,其中$A''$跑遍全部在$A$上是有限模的子代数$A\subseteq A''\subseteq A'$:
        $$\mathrm{LocIsom}(B/A')=\cup_{A''}\mathrm{LocIsom}(B/A'')$$
        \begin{proof}
        	
        	因为$A\subseteq A''\subseteq A'$,所以$\mathrm{LocIsom}(B/A'')\subseteq\mathrm{LocIsom}(B/A')$,于是$\cup_{A''}\mathrm{LocIsom}(B/A'')\subseteq\mathrm{LocIsom}(B/A')$.
        	
        	\qquad
        	
        	反过来设$\mathfrak{q}\in\mathrm{LocIsom}(B/A')$,此即$\mathfrak{q}$是$B$的素理想,并且存在$f\in\overline{A}-\mathfrak{q}$使得$A'_f=B_f$.由于$B_f$是有限型$A$代数,说明存在$A'$的有限型子代数$A''$使得$f\in A''$并且$A''_f=A'_f=B_f$.这里$A\subseteq A''$同时是有限型代数和整扩张,导致$A''$是有限$A$模.依旧有$f\in A''-\mathfrak{q}$,于是$\mathfrak{q}\in\mathrm{LocIsom}(B/A'')$.这得到另一侧的包含关系.
        \end{proof}
        \item 设$A\subseteq B$是有限型代数,设中间环$A\subseteq A'\subseteq B$满足$A'$在$A$上整,那么有:
        $$\mathrm{LocIsom}(B/A)\subseteq\mathrm{LocIsom}(B/A')\subseteq\mathrm{Isol}(B/A)\subseteq\mathrm{Isol}(B/A')$$
        \begin{proof}
        	
        	这里第一个包含号和第三个包含号对任意中间环$A'$成立.下面设$A'$在$A$上整,我们证明第二个包含号.按照上一条归结为设$A'$在$A$上有限的情况.但是我们解释过此时有$\mathrm{Isol}(B/A)=\mathrm{Isol}(B/A')$,于是得证.
        \end{proof}
    \end{enumerate}
    \item 局部版本的Zariski主定理.设环扩张$A\subseteq B$是有限型代数,设$A$在$B$中的整闭包是$\overline{A}$,那么有:
    $$\mathrm{Isol}(B/A)=\mathrm{LocIsom}(B/\overline{A})=\mathrm{Isol}(B/\overline{A})$$
    \begin{proof}
    	
    	我们证明过条件下有$\mathrm{LocIsom}(B/\overline{A})\subseteq\mathrm{Isol}(B/A)\subseteq\mathrm{Isol}(B/\overline{A})$.于是只需证明有$\mathrm{Isol}(B/\overline{A})\subseteq\mathrm{LocIsom}(B/\overline{A})$.由于$B$依旧是有限型$\overline{A}$代数,我们不妨设$A$本身已经在$B$中整闭.我们来对$B$作为有限型$A$代数的生成元个数$n$做归纳.如果$n=1$,那么$B=A[b]$,这是引理2的情况,我们解释过此时纤维会出现三种可能,纤维中出现孤立点的情况只在第二种情况发生,并且此时这个纤维孤立点的确是局部同构点,这就证明了$n=1$的情况.
    	
    	\qquad
    	
    	下面设$B=A[b,b_2,\cdots,b_n]$,设命题对$n-1$个生成元总成立,由于如果$B$作为$A$代数被$n-1$个元生成,那么$B$作为$\overline{A}$也可以被$n-1$个元生成,所以按照归纳假设我们得到的命题并不一定要求$A$一定在$B$中整闭,这得到下式的第一个等号.下式中$\overline{A[b]}$表示$A[b]$在$B$中的整闭包.另外下式中$A'$跑遍的是$A[b]\subseteq\overline{A[b]}$的所有中间环,使得$A'$在$A[b]$上有限.
    	$$\mathfrak{q}\in\mathrm{Isol}(B/A)\subseteq\mathrm{Isol}(B/A[b])=\mathrm{LocIsom}(B/\overline{A[b]})=\cup_{A'}\mathrm{LocIsom}(B/A')$$
    	
    	于是存在$A[b]\subseteq B$的中间环$A'$使得$A'$在$A[b]$上有限,并且存在$f'\in A'-\mathfrak{q}$使得$B_{f'}=A'_{f'}$.换句话讲,如果记$\mathfrak{q}'=\mathfrak{q}\cap A'$,那么存在$\mathfrak{q}'$在$\mathrm{Spec}A'$中的开邻域$U'$使得$U'$在$\mathrm{Spec}B$中的原像是同构到$U'$的.于是从$\mathfrak{q}\in\mathrm{Isol}(B/A)$就得到$\mathfrak{q}'\in\mathrm{Isol}(A'/A)$.我们下面断言问题归结为证明$\mathfrak{q}'\in\mathrm{LocIsom}(A'/A)$.因为一旦这成立,那么存在$f_1\in A-\mathfrak{q'}$使得$A'_{f_1}=A_{f_1}$.那么$f'$在$A'_{f_1}=A_{f_1}$中的像就可以表示为$g/f_1^m$,其中$g\in A$.如果把$f'$替换为$f'f_1^m$,自然仍然有$B_{f'}=A'_{f'}$,但是此时我们可以保证$f'\in A$.于是此时取$f=f'f_1$,那么$f\in A-\mathfrak{q}$,并且$A_f=(A_{f_1})_{f'}=(A'_{f_1})_{f'}=(A'_{f'})_{f_1}=B_f$,这就说明$\mathfrak{q}\in\mathrm{LocIsom}(B/A)$.
    	
    	\qquad
    	
    	于是问题归结为证明$\mathfrak{q}'\in\mathrm{LocIsom}(A'/A)$.用$A'$替换$B$,归结为设$B$是有限$A[b]$模的情况.重述一下,设$A\subseteq B$是有限型代数,使得$A$在$B$中整闭,有中间环$A[b]$使得$B$是有限$A[b]$模(从而$A[b]\subseteq B$是整扩张),我们要证明的是$\mathrm{Isol}(B/A)\subseteq\mathrm{LocIsom}(B/A)$.假设可以找到$\mathfrak{q}\in\mathrm{Isol}(B/A)-\mathrm{LocIsom}(B/A)$,这个条件下我们证明过$\mathrm{Isol}(B/A)$在一般化下不变.我们断言可以约定$\mathfrak{q}$满足对任意素理想$\mathfrak{q}'\subsetneqq\mathfrak{q}$,都有$\mathfrak{q}'\in\mathrm{LocIsom}(B/A)$:只需证明$\mathrm{Isol}(B/A)-\mathrm{LocIsom}(B/A)$中有极小元,但是这里没有诺特条件,我们只能用Zorn引理,设$I$是一个全序集,设每个$i\in I$有素理想$\mathfrak{p}_i\in\mathrm{Isol}(B/A)-\mathrm{LocIsom}(B/A)$,并且如果$i\ge j$就有$\mathfrak{p}_i\subseteq\mathrm{p}_j$,我们要证明的是$\mathfrak{p}=\cap_i\mathfrak{p}_i$是这个链的上确界,而这是因为,首先$\mathfrak{p}$作为素理想链的交一定是素理想,按照$\mathrm{Isol}(B/A)$在一般化下不变说明$\mathfrak{p}\in\mathrm{Isol}(B/A)$,另外由于$\mathfrak{p}$是子集$\{\mathfrak{p}_i\}$的聚点,于是它自然落在闭集$\mathrm{LocIsom}(B/A)^c$中,完成断言的证明.
    	
    	\qquad
    	
    	另外我们解释过分式化和$\mathrm{Isol}(B/A)$和$\mathrm{LocIsom}(B/A)$都可交换,所以可设$A$是以$\mathfrak{m}=\mathfrak{q}\cap A$为唯一极大理想的局部环.我们断言此时有$B/\mathfrak{q}$包含了一个在$k=\kappa(\mathfrak{m})$上的超越元.一旦这成立,那么$\kappa(\mathfrak{q})$是$k$的超越扩张,但是$\mathfrak{q}$理应是孤立点,它是闭点结合$B/A$是有限型代数说明闭点的剩余域扩张应该是代数扩张(甚至是有限扩张),这个矛盾就说明$\mathrm{Isol}(B/A)\subseteq\mathrm{LocIsom}(B/A)$从而完成证明.
    	
    	\qquad
    	
    	倘若我们的断言不成立,那么$B/\mathfrak{q}$的元都在$A/\mathfrak{m}$上代数,由于$\mathfrak{m}$在$B/\mathfrak{q}$上零化,所以这等价于讲$B/\mathfrak{q}$中的元都在$A$上整.那么存在首一$A$系数多项式$h$使得$y=h(b)\in\mathfrak{q}$.因为$B$在$A[b]$上有限,而$A[b]$在$A[y]$上有限,导致$B$在$A[y]$上有限.按照我们对$\mathfrak{q}$的选取,对每个素理想$\mathfrak{r}\subsetneqq\mathfrak{q}$,都存在一个元$f_{\mathfrak{r}}\in A-\mathfrak{r}$使得$B_{f_{\mathfrak{r}}}=A[y]_{f_{\mathfrak{r}}}=A_{f_{\mathfrak{r}}}$.按照$B$在$A[y]$上有限,设生成元为有限集合$\{b_i\}$,那么存在足够大的正整数$N$使得$f_{\mathfrak{r}}^Nb_i\in A[y],\forall i$.用这个$f_{\mathfrak{r}}^N$替换$f_{\mathfrak{r}}$,我们可以额外约定总有$f_{\mathfrak{r}}B\subseteq A[y]$.下面取$\mathfrak{b}=\{g\in A[y]\mid gB\subseteq A[y]\}$,于是每个$f_{\mathfrak{r}}\in\mathfrak{b}$,特别的这说明每个严格包含于$\mathfrak{q}$的素理想都不包含$\mathfrak{b}$,于是$\mathfrak{q}$是$B/\mathfrak{b}$的极小素理想.
    	
    	\qquad
    	
    	如果$\mathfrak{b}\subseteq\mathfrak{q}$.我们断言存在$z\in B-A[y]$满足$zy\in A[y]$:按照$\mathfrak{q}$是$B/\mathfrak{b}$的极小素理想,有$y\in\mathfrak{q}$在$B/\mathfrak{b}$中是零因子,于是存在$z'\in B-\mathfrak{b}$使得$z'y\in\mathfrak{b}\subseteq A[y]$.倘若$z'\not\in A[y]$我们就取$z=z'$;倘若$z'\in A[y]$,按照$z'\not\in\mathfrak{b}$说明存在$g\in B$使得$gz'\not\in A[y]$,于是取$z=gz'$成立.这完成了本段断言的证明.接下来设$zy=a_0+a_1y+\cdots+a_my^m\in A[y]$,用$z-(a_1+a_2y+\cdots+a_my^{m-1})$替代$z$,那么我们甚至可以要求$z\in B$满足$z\not\in A[y]$且$zy\in A$.按照$A$在$B$中整闭知$z$不会是幂零元且$z\not\in A$.于是$B_z\not=0$.用$\overline{A}$,$\overline{y}$,$\overline{z}$分别表示$A,y,z$在$B_z$中的像.按照$1/\overline{z}\in\overline{A}$得到$\overline{z}$在$\overline{A}$上整,进而有$z$在$A$上整,从$A$在$B$中整闭得到$z\in A$,这矛盾,说明不会出现$\mathfrak{b}\subseteq\mathfrak{q}$的情况.
    	
    	\qquad
    	
    	如果$\mathfrak{b}\not\subseteq\mathfrak{q}$,可取$f\in\mathfrak{b}-\mathfrak{q}$,那么$fB\subseteq A[y]$导致$B_f=A[y]_f$.我们把$\mathrm{Spec}B_f$典范的视为$\mathrm{Spec}B$的开子集,把$\mathrm{Spec}A[y]_f$典范的视为$\mathrm{Spec}A[y]$的开子集.按照$\mathfrak{q}\not\in\mathrm{LocIsom}(B/A)$得到$\mathfrak{q}\not\in\mathrm{LocIsom}(B_f/A)$.再从$\mathrm{LocIsom}(B_f/A)=\mathrm{LocIsom}(A[y]_f/A)$得到$\mathfrak{q}_0=\mathfrak{q}\cap A[y]$步骤$\mathrm{LocIsom}(A[y]_f/A)$中.因为我们选取的$\mathfrak{q}$是$\mathrm{Spec}B-\mathrm{LocIsom}(B/A)$的极小元,导致$\mathfrak{q}_0$是$\mathrm{Spec}A[y]-\mathrm{LocIsom}(A[y]/A)$的极小元.但是此时$A\subseteq A[y]$满足$A$在$A[y]$中整闭,这又是引理2的情况,在这种情况下只能是情况(c),此时$A[y]/\mathfrak{q}_0\subseteq B/\mathfrak{q}$包含了一个$k$上的超越元,这也矛盾.至此说明我们(中间部位)的断言成立,完成证明.
    \end{proof}
    \item 推论.设$A\subseteq B$是有限型代数,设$\mathfrak{q}\in\mathrm{Spec}B$是纤维孤立点,那么存在$A$代数$B$的子代数$A'$,使得$A'$在$A$上有限,以及一个元$f\in A'-\mathfrak{q}$,使得$B_f=A_f'$.
    \item 推论.设$A\subseteq B$是有限型代数,使得$A$在$B$中整闭,设$\mathfrak{q}\in\mathrm{Spec}B$是纤维孤立点,那么存在$f\in A-\mathfrak{q}$使得$B_f=A_f$.
    \item 推论.设$f:X\to Y$是局部有限型态射,设$U=\mathrm{Isol}(f)=\{x\in X\mid\dim_xf^{-1}(f(x))=0\}$,那么$U$是$X$的开子集.
    \begin{proof}
    	
    	开集是一个局部性质,所以归结为仿射情况,记$X=\mathrm{Spec}B$和$Y=\mathrm{Spec}A$,设$f$对应的环同态为$\varphi:A\to B$,这使得$B$是有限型$A$代数.因为我们要证明的开集是$X$上的,并且把$A$替换为$A/\ker\varphi$不改变纤维,于是不妨设$\varphi$是单射,即$A\subseteq B$是子环.任取$x\in U$,我们解释过此时$x$是$A\subseteq B$的纤维孤立点,于是按照我们的推论存在$A$代数$B$的子代数$A'$,使得$A'$在$A$上有限,如果记$A'\subseteq B$对应的态射是$f'$,那么存在$f'(x)$的开邻域$U'$使得$f'$诱导了同构$f'^{-1}(U')\cong U'$.这导致开集$f'^{-1}(U)$中的点都在$U$中,于是$U$是开集.
    \end{proof}
\end{enumerate}

整体版本.
\begin{enumerate}
	\item 引理1.一个局部环$A$称为hensel环,如果每个有限$A$代数都同构于若干局部环的直积,例如完备局部环总是hensel环.另外如果$(A,\mathfrak{m})$是局部环,那么总存在一个局部hensel环$A'$和一个忠实平坦局部同态$A\to A'$,使得$\mathfrak{m}A'$恰好就是$A'$的极大理想.例如如果$A$是诺特局部环,那么这里$A'$可以取为它的完备化.
	\item 引理2.设$A\to A'$是局部环之间的局部平坦同态,设$f:X\to Y$是$A$概形态射,那么$f$是同构当且仅当$f_{A'}:X\times_AA'\to Y\times_AA'$是同构.
	\item 设$(A,\mathfrak{m})$是局部hensel环,设$B$是有限型$A$代数,设$f:\mathrm{Spec}B\to\mathrm{Spec}A$是对应的态射.设$y\in\mathrm{Spec}A$是唯一闭点,设$x\in f^{-1}(y)$是纤维中的孤立点.那么$B_{\mathfrak{p}_x}$是有限$A$代数,并且$\mathrm{Spec}B_{\mathfrak{p}_x}$是$\mathrm{Spec}B$的开子集.
	\begin{proof}
		
		首先我们可以把$A$替换为$A/I$,其中$I=\ker(A\to B)$,因为局部hensel环的商还是局部hensel环,并且这样替换不改变$y$的纤维,并且按照$A/I$是有限$A$代数,说明从$B_{\mathfrak{p}_x}$是有限$A/I$代数得到$B_{\mathfrak{p}_x}$是有限$A$代数.换句话讲我们不妨设$A\subseteq B$是子环.
		
		\qquad
		
		那么按照$x$是纤维孤立点,从局部版本的Zariski主定理得到存在$A$代数$B$的子代数$A'$,使得$A'$在$A$上有限,并且存在$s\in A'-\mathfrak{p}_x$使得$B_s=A'_s$.如果记$\mathfrak{p}_x\cap A'=\mathfrak{p}_z$,那么进而有$B_{\mathfrak{p}_x}=A'_{\mathfrak{p}_z}$.并且这里$z$一定是$\mathrm{Spec}A'$的闭点,因为如果$z'$是$z$的特殊化,那么$z$和$z'$同时在$y$的纤维中,但是由于$A'$是有限$A$模,理应有$y$在$A'$中的纤维是离散空间,这导致$z'=z$.
		
		\qquad
		
		现在按照$A$是局部hensel环,以及$A'$在$A$上有限,说明$A'=\prod_{y'}A'_{\mathfrak{p}_{y'}}$,其中$y'$跑遍$\mathrm{Spec}A'$的全体有限个闭点.这个直积对应于$\mathrm{Spec}A'$的连通分支分解,每个$\mathrm{Spec}A'_{\mathfrak{p}_{y'}}$是以$y'$为唯一闭点的连通分支.如果记$A''=A'_{\mathfrak{p}_z}=B_{\mathfrak{p}_x}$,按照连通分解等价于幂等元分解,就有非平凡幂等元$e\in A'$使得$A''=A'_e$,那么有$e\not\in\mathfrak{p}_x$(否则的话$1-e\not\in\mathfrak{p}_x$,导致$1-e$是$A'_e$中的可逆元,但是$(1-e)e=0$,导致$e=0$和非平凡矛盾),于是有$B_e\subseteq B_{\mathfrak{p}_x}$.反过来$B_{\mathfrak{p}_x}=A'_e\subseteq B_e$,综上有$B_{\mathfrak{p}_x}=B_e$,于是$\mathrm{Spec}B_{\mathfrak{p}_x}=\mathrm{Spec}B_e$是$\mathrm{Spec}B$的开子集.
	\end{proof}
    \item 引理3.设$f:X\to Y$是有限型分离态射,设$f^{\#}:\mathscr{O}_Y\to f_*\mathscr{O}_X$是同构.记$V\subseteq X$是开集$\{x\in X\mid\dim_xf^{-1}(f(x))=0\}$(我们解释过局部有限型态射就保证这是开集).那么$f$限制在$V$上是一个开嵌入,并且有$f^{-1}(f(V))=V$.
    \begin{proof}
    	
    	第一步,我们证明如果$y\in f(V)$,那么投影态射$X^y=X\times_Y\mathrm{Spec}\mathscr{O}_{Y,y}\to\mathrm{Spec}\mathscr{O}_{Y,y}$是同构.由于$\mathrm{Spec}\mathscr{O}_{Y,y}\to Y$是拓扑嵌入,并且像集上点的stalk处处是同构,我们解释过这导致诱导的基变换$X^y=X\times_Y\mathrm{Spec}\mathscr{O}_{Y,y}\to X=X\times_YY$仍然是拓扑嵌入,于是我们可以把$X^y$典范的视为$X$的子空间.考虑如下纤维积图表,从$f$是分离有限型态射得到$f'$也是.
    	$$\xymatrix{X'=X^y\ar[rr]^{f'}\ar[d]_{g'}&&Y'=\mathrm{Spec}\mathscr{O}_{Y,y}\ar[d]^g\\X\ar[rr]_f&&Y}$$
    	
    	另外这里$f$是拟紧拟分离的,并且$g$是平坦的,我们解释过这个条件下有$g^*f_*\mathscr{F}\cong f'_*g'^*\mathscr{F}$,其中$\mathscr{F}$是拟凝聚$\mathscr{O}_X$模层.如果取$\mathscr{F}=\mathscr{O}_X$就得到:
    	$$\mathscr{O}_{Y'}=g^*\mathscr{O}_Y=g^*f_*\mathscr{O}_X=f'^*g'^*\mathscr{O}_X=f'_*\mathscr{O}_{X'}$$
    	
    	换句话讲$f':X'\to Y'$仍然满足$f'^{\#}:\mathscr{O}_{Y'}\to f'_*\mathscr{O}_{X'}$是同构.综上为证明第一步的断言,把$f',X',Y'$替换$f,X,Y$,归结为设$Y=\mathrm{Spec}A$是一个局部环的素谱,$y$是$Y$的唯一闭点,要证明的是$f:X\to Y$本身是一个同构.
    	
    	\qquad
    	
    	设$A\to A'$是引理1中的到hensel环的同态.记$Y'=\mathrm{Spec}A'$,记$X'=X\times_YY'$.按照引理2有$f:X\to Y$是同构当且仅当投影态射$f':X'\to Y'$是同构.另外由于$Y'\to Y$也是平坦的,所以和上一段相同的做法可得依旧有$f'^{\#}:\mathscr{O}_{Y'}\to f'_*\mathscr{O}_{X'}$是同构.另外如果$x\in V\cap f^{-1}(y)$,取$x$在$X'$中的提升元(这总存在,我们解释过对于纤维积$(X\times_SY,p,q)$,如果$x\in X$和$y\in Y$在$S$中的像相同,那么$p^{-1}(x)\cap q^{-1}(y)$是一个非零环的素谱,它总是非空的),那么$x'$是$f'^{-1}(f'(x))$的孤立点【】.综上我们不妨用$X',Y',f'$替换$X,Y,f$,换句话讲不妨设$A$本身是局部hensel环.
    	
    	\qquad
    	
    	任取$x\in V\cap f^{-1}(y)$,按照它是纤维孤立点,说明有$x\in X$的仿射开邻域$U=\mathrm{Spec}B$,使得$U\cap f^{-1}(y)=\{x\}$.按照上一条,存在$x\in U$的仿射开邻域$V'=\mathrm{Spec}B_{\mathfrak{p}_x}$使得$B_{\mathfrak{p}_x}$是有限$A$代数.按照$V'\to X\to Y$是有限的,而$X\to Y$是分离的,得到开嵌入$V'\to X$是有限态射,从而它也是闭映射,于是$V'$作为$X$的子集是既开又闭的,它的补集记作$W$.于是有$\mathscr{O}_X(X)=\mathscr{O}_X(V')\times\mathscr{O}_X(W)$.我们还有$\mathscr{O}_X(X)=f_*\mathscr{O}_X(Y)=\mathscr{O}_Y(Y)=A$是局部环.但是我们知道局部环的素谱一定是连通的(因为局部环不能有非平凡幂等元,否则导致极大理想包含非平凡幂等元$e,1-e$,导致1在极大理想里矛盾),这迫使$W=\emptyset$,于是$V'=X$导致$X$是仿射的,于是从$f^{\#}$是同构得到$f:X\to Y$是同构.这完成了第一步.
    	
    	\qquad
    	
    	第二步,证明原命题.设$x\in V$和$y=f(x)$,把$\mathrm{Spec}\mathscr{O}_{Y,y}$视为$Y$的子空间,由于$\mathrm{Spec}\mathscr{O}_{Y,y}\to Y$的像集中的点的stalk都是同构,我们解释过此时投影态射$p:X^y\to X$作为集合映射是从$X^y$到$f^{-1}(\mathrm{Spec}\mathscr{O}_{Y,y})$的拓扑嵌入.并且我们有如下集合映射的图表:
    	$$\xymatrix{X^y\ar[rr]^{f'}\ar[dr]&&\mathrm{Spec}\mathscr{O}_{Y,y}\\&f^{-1}(\mathrm{Spec}\mathscr{O}_{Y,y})\ar[ur]_f&}$$
    	
    	第一步告诉我们这里$f'$是概形同构.于是这个交换图表告诉我们$f$限制为$f^{-1}(\mathrm{Spec}\mathscr{O}_{Y,y})\to\mathrm{Spec}\mathscr{O}_{Y,y}$是同胚.特别的,如果有$x\in V$和$x'\in X$使得$y=f(x)=f(x')$,那么$x,x'\in f^{-1}(\mathrm{Spec}\mathscr{O}_{Y,y})$,但是按照$f$在这个集合的限制是同胚,导致$x=x'$.这件事说明$f^{-1}(f(V))=V$(因为$V\subseteq f^{-1}(f(V))$总成立).另外如果再要求$x,x'\in V$,说明$f$限制在$V$上是单射.至此我们还差证明$g=f\mid_V$是开嵌入,倘若我们能证明$g$是开映射,那么$g$是$V\to f(V)$的同胚,而层上面按照条件有$\mathscr{O}_Y\to f_*\mathscr{O}_X\to f_*\mathscr{O}_U$是同构,这就说明$g$是开嵌入.另外因为开映射是一个局部性质,我们不妨设$Y=\mathrm{Spec}A$是仿射的.
    	
    	\qquad
    	
    	第一步还告诉我们$X^y$实际上是一个局部环的素谱,按照$f$是$f^{-1}(\mathrm{Spec}\mathscr{O}_{Y,y})\to\mathrm{Spec}\mathscr{O}_{Y,y}$的同胚,得到$f^{-1}(\mathrm{Spec}\mathscr{O}_{Y,y})=\mathrm{Spec}\mathscr{O}_{X,x}$.于是对$x\in V$有典范同构:$\mathrm{Spec}\mathscr{O}_{X,x}\cong X^{g(x)}\cong\mathrm{Spec}\mathscr{O}_{Y,g(x)}$.于是这里stalk同态$\mathscr{O}_{Y,g(x)}\to\mathscr{O}_{X,x}$都是同构【】.
    	
    	\qquad
    	
    	任取$x\in V$,记$y=g(x)$,取$x$在$V$中的仿射开邻域$U=\mathrm{Spec}B$,记$S=A-\mathfrak{p}_y$和$T=B-\mathfrak{p}_x$,那么$S\subseteq T$,并且$S$同时是$A$和$B$的乘性闭子集.我们解释了$\mathscr{O}_{X,x}\cong\mathscr{O}_{Y,g(x)}$,于是$S^{-1}A=T^{-1}B$,但是有$S^{-1}A\subseteq S^{-1}B\subseteq T^{-1}B$,导致$S^{-1}B=S^{-1}A=B_{\mathfrak{p}_x}$.按照$f$是有限型态射,有$B=A[b_1,\cdots,b_n]$.于是我们可以选取$s\in S$使得$b_i/1\in A_s$(把$A_s$视为$B_s$的子集).用$D(s)$替换$Y$,用$f^{-1}(D(s))$替换$X$,用$\mathrm{Spec}B_s$替换$U$,则我们可以不妨设$A\to B$是满射.这导致$U$同构于$Y$的某个闭子概型$Z=\mathrm{Spec}A/\mathfrak{a}$.再设$U'=f^{-1}(Y-Z)$.因为我们解释的$f\mid_V$是单射以及$f^{-1}(f(V))=U$,得到$f^{-1}(f(U))=U$.于是$U$和$U'$不交并且都是开集并且并是整个$X$.于是有$A=\mathscr{O}_Y(Y)=f_*\mathscr{O}_X(Y)=\mathscr{O}_X(X)=\mathscr{O}_X(U)\times\mathscr{O}_X(U')\cong A/\mathfrak{a}\times\mathscr{O}_X(U')$.而这导致$Z$作为$Y$的闭子概型也是开子概型(因为环的二元直积使得空间是两个开子概型的无交并),特别的$g(U)$是开集,完成证明.
    \end{proof}
    \item 引理4.设$S$是拟紧概形,设$\mathscr{A}$是一个拟凝聚$\mathscr{O}_S$代数层,设$I$是一个有向偏序集,设$\{\mathscr{A}_i\}$是由$\mathscr{A}$的某些拟凝聚子代数层构成的集合,满足$\cup_i\mathscr{A}_i=\mathscr{A}$.记$X=\mathrm{Spec}\mathscr{A}$和$X_i=\mathrm{Spec}\mathscr{A}_i$.设$V$是一个有限型$S$概形,设$g:V\to X$是开嵌入,那么存在指标$t\in I$使得$\xymatrix{V\ar[r]^g&X\ar[r]&X_t}$是开嵌入.
    \begin{proof}
    	
    	【】
    \end{proof}
    \item 整体版本的Zariski主定理.设$Y$是qcqs概形,设$f:X\to Y$是分离和有限型态射.那么$\mathrm{Isol}(f)$是$X$的开子集,并且对任意拟紧开子集$V'\subseteq\mathrm{Isol}(f)$,可以把$f$分解为$\xymatrix{X\ar[r]^h&Y'\ar[r]^g&Y}$,其中$g$是有限态射,而$h\mid_{V'}:V'\to Y'$是拟紧开嵌入,并且满足$h^{-1}(h(V'))=V'$.
    \begin{proof}
    	
    	因为$f$是拟紧拟分离态射,导致$\mathscr{B}=f_*\mathscr{O}_X$是拟凝聚$\mathscr{O}_Y$代数层.记$X'=\mathrm{Spec}\mathscr{B}$,记$Z=\mathrm{Spec}\mathscr{C}$是$Y$在$\mathscr{B}$中的整闭包.设$\{\mathscr{C}_i\mid i\in I\}$是由$\mathscr{C}$的有限型$\mathscr{O}_Y$子代数层构成的正向系统.那么$\mathscr{C}$就是$\mathscr{C}_i$的并(也即极限,这个结论依赖于$Y$是qcqs概形).再记$Z_i=\mathrm{Spec}\mathscr{C}_i$.因为$\mathscr{C}$在$\mathscr{O}_Y$上整,得到每个$\mathscr{C}_i$都是有限生成的$\mathscr{O}_Y$模层.于是每个$Z_i$都在$Y$上有限,于是$Z_i$是qcqs概形(因为有限态射一定是qcqs态射,结合$Y$是qcqs概形就得到$Z_i$是qcqs概形).另外按照拟凝聚代数层素谱的泛性质,有包含态射$\mathscr{C}_i\to\mathscr{B}=f_*\mathscr{O}_X$和$\mathscr{C}\to\mathscr{B}$对应于概形态射$X\to Z_i$和$X\to Z$.再设$\mathscr{C}\to\mathscr{B}$对应的仿射态射为$f':X'\to Z'$(见下面公式).那么$f'$也是qcqs态射,导致$f'_*\mathscr{O}_{X'}$也是拟凝聚$\mathscr{O}_Z$代数层.并且有$\mathrm{Spec}f'_*\mathscr{O}_{X'}=X'$因为$f'$是仿射的.这里$f'_*\mathscr{O}_{X'}$也是它全体有限型$\mathscr{O}_Z$子代数层$\{\mathscr{D}_j\mid j\in J\}$的极限.记$X'_j=\mathrm{Spec}\mathscr{D}_j$,记$f_j'$是结构态射$X_j'\to Z$.
    	$$\mathrm{Hom}_Y(X,\mathrm{Spec}\mathscr{C})\cong\mathrm{Hom}_{\mathscr{O}_Y-\textbf{Alg}}(\mathscr{C},f_*\mathscr{O}_X)\cong\mathrm{Hom}_Y(\mathrm{Spec}\mathscr{B},\mathrm{Spec}\mathscr{C})$$
    	$$(X\to Z)\mapsto(\mathscr{C}\to\mathscr{B})\mapsto f'$$
    	
    	下面设$V'\subseteq\mathrm{Isol}(f)$是拟紧开子集,我们定义$h,h_j,g,g_i$是使得如下图表交换的虚线态射(就是依次定义复合):
    	$$\xymatrix{&X\ar[r]^c&X'\ar[d]\ar@/^1pc/[dd]^{f'}&\\V'\ar[ur]\ar@{-->}[urr]^h\ar@{-->}[rr]^{h_j}\ar@{-->}[drr]_g\ar@{-->}[ddrr]_{g_i}&&X'_j\ar[d]_{f'_j}&\\&&Z\ar[d]&\\&&Z_i\ar[r]^{\mathrm{finite}}&Y}$$
    	
    	这里$c$满足如下交换图表,由于$i$是仿射态射,从$f$是分离和有限型态射就得到$c$是分离和有限型态射.另外有$c$诱导了同构$\mathscr{O}_{X'}\cong c_*\mathscr{O}_X$【】.于是引理3说明$c$限制在$\mathrm{Isol}(c)$上是开嵌入.由于$\mathrm{Isol}(f)\subseteq\mathrm{Isol}(c)$,说明$h$是开嵌入.那么引理4说明存在指标$t\in J$使得$h_t$是开嵌入.因为$\mathscr{O}_Z$在$\mathscr{B}$中整闭,于是$\mathscr{O}_Z$在子代数层$f'_{t,*}\mathscr{O}_{X'_t}$上整闭.因又为$f_t'$是仿射和有限型态射,于是局部Zariski主定理的推论告诉我们存在$Z$的开子概型$W$,使得$f_t'^{-1}(W)=\mathrm{Isol}(f'_t)$,并且$f_t'$限制在这上面是开嵌入.又因为$\mathrm{Isol}(f)\subseteq\mathrm{Isol}(f'_t)$【】,于是特别的$g$是开嵌入.再按照引理4,有指标$s\in I$使得$g_s$是开嵌入.最后$\mathscr{O}_{Z_s}$在$\mathscr{O}_Y$上整,于是$Z_s\to Y$是整态射,而$Z_s\to Y$又是有限型态射,导致$Z_s\to Y$是有限态射.于是选取$Y'=Z_s$得到我们想要的分解.
    \end{proof}
\end{enumerate}

Zariski主定理的推论.
\begin{enumerate}
	\item 推论.设$Y$是qcqs概形,设$f:X\to Y$是拟有限和分离的态射,那么$f$可以分解为$\xymatrix{X\ar[r]^j&Y'\ar[r]^g&Y}$,其中$j$是拟紧开嵌入,$g$是有限态射.
	\begin{proof}
		
		因为$f$拟有限的时候$X$的所有点都是纤维孤立点,而且拟有限定义里要求了拟紧(也要求了有限型),导致$X$本身是纤维孤立点构成的拟紧开集.
	\end{proof}
	\item 推论.另外在上一条的基础上如果把$j$替换为它的概形像,那么$j$依旧是拟紧开嵌入(因为这里$\mathrm{im}j\to Y'$是闭嵌入,它的对角态射是同构,导致$X\to Y'$的拟紧和开嵌入都传递给$X\to\mathrm{im}j$).而$g$要多复合一个闭嵌入$\mathrm{im}j\to Y'$,替换后的$g$仍然是有限态射.换句话讲我们可以额外要求结论中的分解还满足$j$是概形支配的(换句话讲$j(X)$是$Y'$的概形稠密开集).
	\item 推论.设$S$是诺特概形,设$X,Y$是分离和有限型的$S$概形.那么每个具有有限纤维的态射$f:X\to Y$可以分解为$\xymatrix{X\ar[r]^j&Y'\ar[r]^g&Y}$,其中$j$是开嵌入,$g$是有限态射.
	\begin{proof}
		
		因为$X,Y$都是$S$分离概形,于是所有$S$态射$X\to Y$都是分离态射.又因为$X\to S$是有限型态射,得到$X\to Y$是有限型态射.而诺特概形一定是qcqs概形.并且$f$是拟有限的.所以满足Zariski主定理的条件.
	\end{proof}
	\item 设$f:X\to Y$是分离拟有限的双有理态射,并且$X$是整概形,$Y$正规整概形.那么$f$是开嵌入.
	\begin{proof}
		
		开嵌入是终端局部的,不妨设$Y=\mathrm{Spec}A$是仿射的.因为$f$是双有理的,于是诱导了整概形函数域之间的同构$K(Y)\cong K(X)$.这里$f$满足Zariski主定理推论的条件,所以$f=g\circ j$,其中$j:X\to Y'$是一个开嵌入,$g:Y'\to Y$是有限态射.把$Y'$替换为$j$的概形像.回顾在既约概形条件下$j$的概形像就是集合像的闭包上赋予既约闭子概型结构.于是我们不妨约定$Y'$是整概形,并且$j$是支配态射.但是$j$是开嵌入,导致它诱导了函数域的同构$K(Y')\cong K(X)$.这导致$g$也是双有理的.因为$g$是有限态射,有$Y'=\mathrm{Spec}A'$也是仿射的,并且$A'$是有限$A$代数,导致$A'$在$A$上整,但是函数域同构说明$A$和$A'$的商域一致,从$A$是整闭的就得到$A=A'$,于是$g$是同构,于是$f$是开嵌入.
	\end{proof}
	\item 推论.设$f:X\to Y$是概形之间的态射,那么如下命题互相等价.我们解释过(a)和(c)等价,这里借助Zariski主定理给出证明.
	\begin{enumerate}
		\item $f$是有限态射.
		\item $f$是拟有限和紧合态射.
		\item $f$是仿射和紧合态射.
	\end{enumerate}
    \begin{proof}
    	
    	(b)推(a):设$f$是拟有限和紧合态射,问题是终端局部的,我们设$Y$是仿射的,于是$Y$是qcqs概形,于是按照Zariski主定理的推论有$f$可以分解为$X\to Y'\to Y$,其中$X\to Y'$是开嵌入,$Y'\to Y$是有限态射,那么从$X\to Y'\to Y$是紧合态射和$Y'\to Y$是分离态射得到$X\to Y'$是紧合态射.于是$X\to Y'$是开嵌入还是闭映射,这导致它是闭嵌入,于是$X\to Y'$是有限态射,于是作为有限态射的复合$f$也是有限态射.
    	
    	\qquad
    	
    	(a)推(c)是平凡的.最后证明(c)推(b):设$f$是仿射和紧合态射,我们只需证明$f$具有有限纤维,于是归结为设$Y=\mathrm{Spec}k$是域的素谱,那么按照$f$是仿射和有限型态射,可设$X=\mathrm{Spec}A$,其中$A$是有限型$k$代数.按照诺特正规化引理,存在满射有限态射$g:X\to\mathbb{A}_k^n$.把它复合上开嵌入$\mathbb{A}_k^n\to\mathbb{P}_k^n$得到态射$X\to\mathbb{P}_k^n$.由于$X\to\mathrm{Spec}k$是紧合态射,而$\mathbb{P}_k^n\to\mathrm{Spec}k$是分离态射,于是$X\to\mathbb{P}_k^n$是紧合态射,所以它是闭映射.但是$X\to\mathbb{A}_k^n$是满射,于是$\mathbb{A}_k^n\to\mathbb{P}_k^n$的像集是闭集,这迫使$n=0$,也即$X\to\mathrm{Spec}k$已经是有限态射,此时$A$是$k$上有限维代数,于是$A$是零维的,于是$X=\mathrm{Spec}A$是有限点集.
    \end{proof}
    \item 设$f:X\to Y$是紧合态射,设$V=\{y\in Y\mid f^{-1}(y)\text{有限点集}\}$.那么$V\subseteq Y$是开集,并且$f$限制在$f^{-1}(V)\to V$是有限态射.
    \begin{proof}
    	
    	设$U=\mathrm{Isol}(f)$,因为$f$是有限型态射,我们解释过此时$U\subseteq X$是开子集.接下来我们断言有$V=Y-f(X-U)$,一旦这成立,按照$f$是闭映射就得到$V$是开集.进而$f^{-1}(V)\to V$是拟有限和紧合态射,于是上一条得到它是有限态射.最后证明我们的断言.一方面如果$y\in V$,那么$f^{-1}(y)$是有限点集,所以它是离散空间(域上有限型概形,零维等价于有限点集等价于离散空间),所以$f^{-1}(y)\subseteq U$,于是$y\not\in f(X-U)$,也即$V\subseteq Y-f(X-U)$;另一方面我们证明$V\supseteq Y-f(X-U)$,也即$Y-V\subseteq f(X-U)$.任取$y\in Y-V$,那么$f^{-1}(y)$是无限点集,因为$f$是拟紧的归结为设$\mathrm{Spec}A\to\mathrm{Spec}k$是有限型的,并且$\mathrm{Spec}A$是无限点集,要证明的是$\mathrm{Spec}A$中存在非孤立点.但是倘若$\mathrm{Spec}A$是离散空间,同样按照域上有限型概形离散空间等价于有限点集,这就矛盾,于是得证.
    \end{proof}
    \item 闭嵌入准则.概形之间的态射是闭嵌入当且仅当它是紧合态射和单态射.
    \begin{proof}
    	
    	一方面闭嵌入一定是单态射和紧合态射.反过来设$f:X\to Y$是单态射和紧合态射,因为闭嵌入是终端局部的,不妨设$Y=\mathrm{Spec}A$是仿射的.任取$y\in Y$,那么基变换$f^{-1}(y)\to\mathrm{Spec}\kappa(y)$也是单态射,单态射一定是单射,所以$f^{-1}(y)$至多包含单个点.于是$f$是拟有限态射和紧合态射,于是$f$是有限态射.记$X=\mathrm{Spec}B$,其中$B$是有限$A$代数.记$f^{-1}(y)=\mathrm{Spec}B_y$.其中$B_y=B_{\mathfrak{p}}/\mathfrak{p}B_{\mathfrak{p}}$是有限$\kappa(\mathfrak{p})$代数,其中$\mathfrak{p}$是$y$对应的素理想.按照$f^{-1}(y)\to\mathrm{Spec}\kappa(y)$是单态射,得到对角态射$B_y\otimes_{\kappa(y)}B_y\to B_y$是同构.这迫使$B_y$作为$\kappa(y)$线性空间是$\le1$维的,于是$B_y=\kappa(y)$或者$B_y=0$.于是$A_{\mathfrak{p}}/\mathfrak{p}A_{\mathfrak{p}}\to B_{\mathfrak{p}}/\mathfrak{p}B_{\mathfrak{p}}$总是满射,于是NAK引理导致$A_{\mathfrak{p}}\to B_{\mathfrak{p}}$是满射,进而有$A\to B$是满射,于是$f$是闭嵌入.
    \end{proof}
\end{enumerate}
\newpage
\section{嵌入和子概型}
\subsection{闭嵌入(闭浸入)和闭子概型}
\begin{itemize}
	\item 一个态射$\varphi:X\to Y$称为闭嵌入,如果它是拓扑嵌入,并且像集$\varphi(Y)$是$X$的闭子集,并且$\varphi^{\#}$是$\mathrm{im}\varphi$上的满的层态射$\mathscr{O}_X\mid_{\varphi(Y)}\to\varphi_*\mathscr{O}_Y$.
	\item 概形$X$的闭子概形是指它的一个闭子集$Y$,赋予了一个结构层使得它成为概形,并且存在闭嵌入$(j,j^{\#}):Y\to X$,其中$j:Y\subseteq X$是包含映射.
\end{itemize}
\begin{enumerate}
	\item 仿射情况的闭嵌入.
	\begin{itemize}
		\item 由满的环同态$A\to A/I$诱导(等价的讲,环的满同态诱导的)的仿射概型之间的态射是闭嵌入.
		\begin{proof}
			
			我们说明过态射$(\varphi,\varphi^{\#})\mathrm{Spec}(A/I)\to\mathrm{Spec}(A)$是$\mathrm{Spec}(A/I)$到像集$V(I)$的同胚,并且$V(I)$是$\mathrm{Spec}(A)$的闭子集.另外对每个$p\in V(I)$,有$\varphi^{\#}_p:A_p\to A_p/IA_p$是满射,于是$\varphi^{\#}$是满态射.
		\end{proof}
		\item 仿射概型$\mathrm{Spec}(A)$的闭子概型$Z$总是仿射的,并且具有形式$\mathrm{Spec}(A/I)$.换句话讲仿射概型的闭嵌入总是具有第一条的形式.
		\begin{proof}
			
			第一步证明$Z$是仿射的.对任意$p\in Z$,取$p$在$Z$中的仿射开邻域,它具有形式$U_p\cap Z=\mathrm{Spec}(A^{(p)})$,其中$U_p$是$p$在$\mathrm{Spec}(A)$中的开邻域.再取$p$再$\mathrm{Spec}(A)$中的主开集开邻域$D(f_p)\subseteq U_p$.那么有$D(f_p)\cap Z$是$U_p\cap Z$的主开集,记作$D(g_p)$,$g_p\in A^{(p)}$.拟紧空间的闭子集拟紧,于是可取$Z$的有限开覆盖$\{D(g_i),1\le i\le n\}$,记$D(g_i)=D(f_i)\cap Z$,$f_i\in A$,记$A\to\mathscr{O}_Z(Z)$下把$f_i$映射为$f_i'$,那么$Z_{f_i'}=D(f_i)\cap Z=D(g_i)$,于是$Z_{f_i'}$覆盖了整个$Z$并且每个$Z_{f_i'}=\mathrm{Spec}(A^{(i)}_{g_i})$仿射,于是按照仿射准则,得到$Z$是仿射的.
			
			第二步,设$\mathrm{Spec}(B)\to\mathrm{Spec}(A)$是闭嵌入,我们来证明$\mathrm(B)$同构于某个$\mathrm{Spec}(A/I)$.设这个态射由$\varphi:A\to B$诱导,记$\ker\varphi=I$,则有环同态的复合$A\to A/I\to B$,它诱导了仿射概型的态射复合:
			$$\xymatrix{\mathrm{Spec}(B)\ar[rr]^{\alpha}\ar[d]_{\theta}&&\mathrm{Spec}(A)\\\mathrm{Spec}(A/I)\ar[urr]_{\beta}&&}$$
			
			按照闭嵌入的定义,拓扑上看$\theta$是一个同胚.接下来仅需验证验证$\theta^{\#}$是一个层同构.一方面如果复合态射$\alpha$和$\beta$都是闭嵌入,可验证$\theta$也是闭嵌入,于是$\theta^{\#}$是满态射.最后说明$\theta^{\#}$是单态射.为此我们来证明如果$f:A\to B$是单的环同态,那么诱导的层态射$\varphi^{\#}:\mathscr{O}_{\mathrm{Spec}(A)}\to\varphi_*\mathscr{O}_{\mathrm{Spec}(B)}$是单态射.
			
			事实上这里$(\varphi_*\mathscr{O}_{\mathrm{Spec}B})_p$就是$\mathscr{O}_{\mathrm{Spec}(B)}(\varphi^{-1}(D(h)))$的逆向极限,其中$D(h)$取遍包含$p$的主开集.这也就是$B_{f(h)}$的逆向极限,其中$h$取遍满足$p\in D(h)$的元,这个逆向极限实际上就是$S^{-1}B$,其中$S=f(A-p)$,由此可推出对每个$\varphi^{\#}_p$都是单射:它是$A_p\to S^{-1}B$的映射,把$a/s$映射为$f(a)/f(s)$.于是倘若$f(a)/f(s)=0$,那么存在$t\in A-p$使得$f(a)f(t)=0$,单射导致$at=0$,于是$a/s=at/st=0$.
		\end{proof}
		\item 最后我们强调下同一个闭子集上可能存在不同的闭子概型结构.例如对于任意环$A$,它素谱这个集合上还可以定义闭子概型$\mathrm{Spec}(A/\mathrm{nil}(A))$.
	\end{itemize}
	\item 一个态射是闭嵌入当且仅当它是仿射态射,并且如果终端的仿射开子集$\mathrm{Spec}B$的原像记作$\mathrm{Spec}A$,那么诱导的结构同态$B\to A$是满同态.于是闭嵌入有如下性质:
	\begin{itemize}
		\item 闭嵌入是有限态射.
		\item 闭嵌入的复合是闭嵌入.
		\item 闭嵌入是仿射终端局部性质.
	\end{itemize}
    \item 闭嵌入在基变换下不变.
    \begin{proof}
    	
    	设$\varphi:X\to Y$是闭嵌入,我们需要验证提升态射$X\times_YS\to S$也是闭嵌入.按照闭嵌入是仿射终端局部性质,可不妨设$Y=\mathrm{Spec}A$是仿射的,$S=\mathrm{Spec}B$是仿射的.按照仿射概形的闭子概型的结构,有$X=\mathrm{Spec}A/I$.那么这里$\varphi$的提升映射就是被$B\to A/I\otimes_AB\cong B/IB$诱导的态射,所以它是闭嵌入.
    	$$\xymatrix{X\times_YS\ar[rr]\ar[d]&&S\ar[d]\\X\ar[rr]^{\varphi}&&Y}$$
    \end{proof}
\end{enumerate}
\subsection{嵌入和子概型}
\begin{itemize}
	\item 态射$\varphi:X\to Y$称为嵌入(有时也称为局部闭嵌入或者浸入),如果它可以分解为$X\to Z\to Y$,其中$X\to Z$是闭嵌入,$Z\to Y$是开嵌入.
	\item 概形$X$的子概型是指一个局部闭子集(此为一个开子集和一个闭子集的交),和其上的一个结构层使得它成为一个概形$Z$,并且存在嵌入$(j,j^{\#}):Z\to X$,使得其中$j:Z\subseteq X$是包含映射.
\end{itemize}
\begin{enumerate}
	\item 概型的态射$(j,j^{\#}):X\to Y$是嵌入当且仅当$j$是拓扑嵌入,并且像集$j(X)$是$Y$的局部闭子集(即可表示为某个开集和闭集的交,也等价于它在自身闭包中开),并且$j^{\#}$是$\mathrm{im}j$上的满的层态射$\mathscr{O}_Y\to j_*\mathscr{O}_X$.
	\item 闭嵌入是拟紧态射,但是开嵌入未必是拟紧的.要想开嵌入是拟紧的需要加一些条件,比方说终端是局部诺特概形.
	\item 开嵌入是局部有限表示态射,闭嵌入是有限型态射,嵌入是局部有限型态射.因为一方面$A/I$是有限$A$模说明嵌入是有限型态射,另一方面从$A_f\cong A[T]/(fT-1)$是有限型$A$代数说明开嵌入总是局部有限表示态射.
	\item 嵌入的复合是嵌入.如果一个态射$X\to Y$可以分解为$X\to Z\to Y$,其中$X\to Z$是开嵌入,$Y\to Z$是闭嵌入,那么称它为H-嵌入(Hartshorne上定义的嵌入).那么这一条就是说H-嵌入一定是嵌入.反过来一般不成立,但是按照概形像的分解,拟紧嵌入一定是H-嵌入.
	\begin{proof}
		
		为此只要说明概型$X$的闭子概型$V$的开子概型$U$可视为$X$的开子概型的闭子概型.而这是因为$U$在拓扑层面上可视为$X$的某个开集$U'$与$V$的交,于是$V\to U'$是闭嵌入,而$U'\to X$是开嵌入.
	\end{proof}
	\item 嵌入是终端局部性质.
	\begin{proof}
		
		验证层态射部分局部是直接的.$j^{\#}$是满态射或者同构等价于讲$\mathscr{O}_{X,f(y)}\to\mathscr{O}_{Y,y}$都是满同态或者同构.这等价于对每个$X$的开覆盖的每个开集上有这个典范映射都是满同态或者同构.验证拓扑部分也是容易的,即空间上的开集闭集和局部闭都是局部性质.
	\end{proof}
    \item 嵌入是概型上的单态射.我们解释过如果概型的态射$j:X\to Y$是单射,并且对任意$x\in X$有$j^{\#}_x:\mathscr{O}_{Y,j(x)}\to\mathscr{O}_{X,x}$满射,那么$j$就是概型的单态射.
    \item 嵌入在基变换下不变.这是因为我们分别证明过开嵌入和闭嵌入都在基变换下不变.另外如果$f:X\to Y$是态射,如果$i:Z\to Y$是嵌入,那么$Z\times_YX$的底空间是$f^{-1}(Z)$,它是$X$的子概型,称为$Z$在$f$下的原像.如果$i:Z\to X$是闭嵌入或者开嵌入,那么$f^{-1}(Z)$就是$X$的闭子概型或者开子概型.例如对于仿射情况,考虑环同态$\varphi:A\to B$,它对应于态射$f:X=\mathrm{Spec}B\to Y=\mathrm{Spec}A$,取$\mathrm{Spec}A$的闭子概型$Z=V(\mathfrak{a})=\mathrm{Spec}A/\mathfrak{a}$.它的原像是$f^{-1}(V(\mathfrak{a}))=V(\varphi(\mathfrak{a})B)$.
    \item 子概型的交.设$i:Y\to X$和$j:Z\to X$是两个子概型.我们定义$Y\cap Z=Y\times_XZ=i^{-1}(Z)=j^{-1}(Y)$称为它们的概形交.它的泛性质是说一个态射$T\to X$经$Y\cap Z$分解当且仅当它要经$Y$和$Z$分解.例如对仿射情况,两个闭子概型$V(\mathfrak{a})$和$V(\mathfrak{b})$的交是$V(\mathfrak{a}+\mathfrak{b})=\mathrm{Spec}A/(\mathfrak{a}+\mathfrak{b})$.
\end{enumerate}
\subsection{概形像}

设$f:X\to Y$是概形之间的态射,它的概形像(schematic image)是指$Y$的一个闭子概型$\mathrm{im}f$,满足如下两个条件.概形像作为$Y$闭子概型的拟凝聚理想层记作$\mathscr{I}_f$.
\begin{itemize}
	\item $f$经闭嵌入$\mathrm{im}f\to Y$分解.
	\item 如果$Z\to Y$是另一个闭嵌入,使得$f$经它分解,那么$Z$要控制$\mathrm{im}f$,此即$\mathrm{im}f\to Y$要经$Z\to Y$分解.
\end{itemize}
\begin{enumerate}
	\item $\mathrm{im}f$的底空间一定包含闭包$\overline{f(X)}$,但是它们一般不相等.
	\item 如果$f:X\to Y$是拟紧态射,记$f^{\#}:\mathscr{O}_Y\to f_*\mathscr{O}_X$的核是$\mathscr{K}_f$,那么它是$Y$上的拟凝聚理想层,并且概形像$\mathrm{im}f$就是被这个理想层定义的闭子概型,并且此时有$\mathrm{im}f$的底空间恰好就是$\overline{f(X)}$.
	\begin{proof}
		
		设$i:Z\to Y$是任意的闭子概型,设它对应的拟凝聚理想层是$\mathscr{I}_Z$.那么$f$经$Z\to Y$分解当且仅当拓扑上$f(X)\subseteq Z$;结构层上$f^{\#}:\mathscr{O}_Y\to f_*\mathscr{O}_X$经$i^{\#}:\mathscr{O}_Y\to i_*\mathscr{O}_Z=\mathscr{O}_Y/\mathscr{I}_Z$分解.而后者等价于讲$\mathscr{I}_Z\subseteq\mathscr{K}_f$.所以一旦我们证明$\mathscr{K}_f$本身是拟凝聚理想层,那么它定义的$Y$的闭子概型就必然是概形像.另外一旦这成立,按照$\mathrm{Supp}(\mathscr{O}_Y/\mathscr{K}_f)=\overline{f(X)}$(这是因为,首先有$f(X)\subseteq\mathrm{Supp}\mathscr{O}_Y/\mathscr{K}_f$,因为如果$x\in X$,$y=f(x)$,如果$\mathscr{O}_{Y,y}/\mathscr{K}_{f,y}=0$,那么$\mathscr{O}_{Y,y}\to(f_*\mathscr{O}_X)_y$就也是零态射,但是stalk同态$\mathscr{O}_{Y,y}\to\mathscr{O}_{X,x}$是要经这个同态分解的,这导致这个stalk同态是零同态,这不可能,于是$y=f(x)\in\mathrm{Supp}(\mathscr{O}_Y/\mathscr{K}_f)$;反过来如果$y\in Y-\overline{f(X)}$,那么存在$Y$的开邻域$V$使得$f^{-1}(V)$是空集,这迫使$(f_*\mathscr{O}_X)_y=\lim\limits_{\substack{\rightarrow\\y\in V}}\mathscr{O}_X(f^{-1}(V))=0$,迫使整个$V$都不在$\mathrm{Supp}(\mathscr{O}_Y/\mathscr{K}_f)$中,综上有$\mathrm{Supp}(\mathscr{O}_Y/\mathscr{K}_f)=\overline{f(X)}$),就得到$\mathrm{im}f$的底空间就是$\overline{f(X)}$.
		
		\qquad
		
		下面设$f$是拟紧态射,我们来证明$\mathscr{K}_f$是拟凝聚层.不妨设$Y$是仿射的,那么拟紧条件保证$X=\cup_iU_i$是有限个仿射开覆盖,设$j_i:U_i\to X$是包含映射,按照层公理我们有典范单射$\mathscr{O}_X\to\oplus_i(j_i)_*\mathscr{O}_{U_i}$.正像函子是左正合的,所以作用$f_*$得到单射$f_*\mathscr{O}_X\to\mathscr{G}=\oplus_i(f\circ j_i)_*\mathscr{O}_{U_i}$.由于$f\circ j_i$是仿射概形之间的态射,所以$(f\circ j_i)_*\mathscr{O}_{U_i}$是$Y$上的拟凝聚层,于是$\mathscr{G}$是拟凝聚层,而$\mathscr{K}_f$是$\mathscr{O}_Y\to f_*\mathscr{O}_X$的核,所以它也是复合上单态射$f_*\mathscr{O}_X\to\mathscr{G}$的核,即$\mathscr{K}_f$是$\mathscr{O}_Y\to\mathscr{G}$的核,但是这是$Y$上两个拟凝聚层之间态射的核,所以核还是拟凝聚层.
	\end{proof}
    \item 设$f:X\to Y$是态射,设$X$是既约概形,设$Y'=\overline{f(X)}$并且赋予既约闭子概型结构,那么$Y'$就是$f$的概形像.进而如果$X$是整概形,那么概形像也是整概形.
    \begin{proof}
    	
    	因为从$X$既约得到$f$要经$f_{\mathrm{red}}$分解,而$f_{\mathrm{red}}$要经$Y'\to Y$分解(在仿射情况下,如果$\varphi:A\to B$是既约环之间的同态,那么$\mathrm{Spec}B$在$\mathrm{Spec}A$中像集的闭包是$V(\varphi^{-1}(0))$,那么$\varphi$必然要经$A\to A/\sqrt{\varphi^{-1}(0)}$分解).$\overline{f(X)}$的既约闭子概型结构保证了它是底空间不变的闭子概型在控制偏序下的极小元,于是对任意闭子概型$Z\to Y$使得$f$经它分解,都有$Y'\to Y$要经$Z\to Y$分解.
    \end{proof}
    \item 概形像的传递性.设$f:X\to Y$和$g:Y\to Z$是概形态射,设$f$的概形像$Y'$存在,设$g$限制在$Y'$上得到的态射$g'$的概形像$Z'$也存在.那么$g\circ f$的概形像存在,并且就是$Z'$.
    \begin{proof}
    	
    	【EGAI9.5.5】
    \end{proof}
    \item 设$f:X\to Y$是$S$态射,使得$Y$本身就是$f$的概形像.设$Z$是分离$S$概形,如果两个$S$态射$g_1,g_2:Y\to Z$满足$g_1\circ f=g_2\circ f$,则$g_1=g_2$.特别的,考虑分离$S$概形范畴,其中的$S$态射$f:X\to Y$的概形像如果是$Y$本身,那么$f$是该范畴中的满态射.这件事的逆命题也是对的:如果$f:X\to Y$是分离$S$概形范畴中的满态射,如果$f$的概形像存在,那么它的概形像就是$Y$本身.
    \begin{proof}
    	
    	设$g_1,g_2$的纤维积为$h:Y\to Z\times_SZ$.对角线$T=\Delta_Z(Z)$是$Z\times_SZ$的闭子集,于是$Y'=h^{-1}(T)$是$Y$的闭子概型.记$u=g_1\circ f=g_2\circ f$.那么$h'=h\circ f$就是被两个$u$诱导的纤维积态射,于是$h\circ f=\Delta_Z\circ u$.按照$\Delta_Z^{-1}(T)=Z$,于是${h'}^{-1}(T)=u^{-1}(Z)=X$.于是$f^{-1}(Y')=X$.于是闭嵌入$Y'\to Y$分解了$f$,按照$f$的概形像是$Y$本身,就有$Y'=Y$.于是$h=\Delta_Z\circ v$,于是$v=g_1=g_2$.
    \end{proof}
    \item 概形像和开子概型可交换.设$f:X\to Y$是拟紧态射,概形像记作$Y'$,设$V\subseteq Y$是开子概型,那么$f\mid_V:f^{-1}(V)\to V$的概形像存在,并且就是$V\times_YY'$.
    \begin{proof}
    	
    	如果记$\mathscr{O}_Y\to f_*\mathscr{O}_X$的核是$\mathscr{I}$,那么$\mathscr{O}_V\to (f\mid_V)_*\mathscr{O}_{f^{-1}(V)}$的核就是$\mathscr{I}\mid_V$.
    \end{proof}
    \item 平坦基变换.设$f:X\to Y$是拟紧$S$态射,设$Z\subseteq Y$是$f$的概形像,如果$S'\to S$是平坦态射,那么$Z'=Z\times_SS'$是基变换$f':X\times_SS'\to Y\times_SS'$的概形像.
    \begin{proof}
    	
    	因为$S'\to S$是平坦的,它的基变换$p:Y'=Y\times_SS'\to Y$也是平坦的.我们有$X'=X\times_YY'$,所以不妨设$Y=S$和$Y'=S'$.我们解释过拟紧态射的概形像对应的拟凝聚理想层就是$\mathscr{K}_f=\ker(\mathscr{O}_Y\to f_*\mathscr{O}_X)$.但是按照$Y'\to Y$是平坦的,就有$p^*\mathscr{K}_f=\ker(\mathscr{O}_{Y'}\to f'_*\mathscr{O}_{X'})$,拟凝聚层的逆像还是拟凝聚层,这说明$p^*\mathscr{K}_f$对应的闭子概型$p^*Z=Z\times_YY'$是$f'$的概形像(我们解释过$\ker(\mathscr{O}_Y\to f_*\mathscr{O}_X)$已经满足概形像的泛性质,问题只在于它是不是拟凝聚层,一旦是则它对应的闭子概型就是概形像).
    \end{proof}
    \item 设$i:X\to Y$是拟紧嵌入,设概形像为$Z$,那么$j:X\to Z$是概形支配开嵌入(一个开嵌入$j:X\to Z$称为概形支配的,如果$j(X)$是$Z$的概形稠密开集):我们解释过拟紧态射的概形像作为集合就是$\overline{i(X)}$,而我们有$i(X)$是$\overline{i(X)}$的开子集(因为$i(X)$是局部闭子集,比方说记作$U\cap F$,其中$U$是$Y$的开集,$F$是$Y$的闭集,那么我们有$U\cap F=U\cap\overline{U\cap F}$).另一方面按照拟紧态射的概形像就是$\mathscr{O}_Y\to i_*\mathscr{O}_X$的核对应的闭子概型.这说明:
    \begin{itemize}
    	\item 按照$i$是拓扑嵌入,有$(i_*\mathscr{O}_X)_x=\mathscr{O}_{X,x}$,于是$(\mathscr{O}_Y/\mathscr{K})_z\cong\mathscr{O}_{X,x}$,这说明$j$是开嵌入.
    	\item 按照$\mathscr{K}$的定义有$\mathscr{O}_Z=(\mathscr{O}_Y/\mathscr{K})\mid_Z\to j_*\mathscr{O}_X$是单态射(在stalk上验证单态射就是验证$\lim\limits_{\substack{\rightarrow\\z\in U}}\mathscr{O}_X(U\cap X)\to\lim\limits_{\substack{\rightarrow\\z\in V}}\mathscr{O}_X(V\cap X)$是单射,这里$U$跑遍$Y$的开子集,$V$跑遍$Z$的开子集,但是$U\cap X$一定可以写成$V\cap X$的性质,即把$V$取为$U\cap Z$,于是前一个正向系统有更多的分量,所以限制映射是单射),这说明$j(X)$是$Z$的概形稠密开集.
    \end{itemize}
    
    综上$i$可以分解为$X\to Z\to Y$,其中$X\to Z$是概形支配开嵌入,$Z\to Y$是闭嵌入,我们称$Z$是$Y$的子概型$X$的概形闭包.另外一般的一个嵌入可以分解为$X\to Z\to Y$,其中$X\to Z$是闭嵌入,$Z\to Y$是开嵌入.嵌入一般不能分解为前面是开嵌入后面是闭嵌入,不过我们解释过能这样分解的态射一定是嵌入.这里我们说明了拟紧嵌入是可以分解为前面是开嵌入后面是闭嵌入的.
    
\end{enumerate}
\subsection{既约子概型}

概型$X$上给定局部闭子集$Y$,可存在多种结构层使得$Y$是一个闭子概型,但是这些闭子概型中存在层意义上最小的一个结构,它可以被既约性质描述.
\begin{enumerate}
	\item 幂零层.给定概型$X$,预层$U\mapsto\mathrm{nil}(\mathscr{O}_X(U))$的层化称为幂零层,记作$\mathscr{N}_X$.它可以具体的表示为理想层$\mathscr{N}(U)=\{s\in\mathscr{O}_X(U)\mid s_x\in\mathrm{nil}(\mathscr{O}_{X,x}),\forall x\in U\}$.
	\item 仿射情况.设$X=\mathrm{Spec}(A)$,记$X_{\mathrm{red}}=(X,\mathscr{O}_X/\mathscr{N})$,那么$X_{\mathrm{red}}=\mathrm{Spec}(A/\mathrm{nil}(A))$是一个既约闭子概型.
	\begin{proof}
		
		首先我们说明$\mathscr{N}_X$实际上是主开集上的一个层.它是层$\mathscr{O}_X$的子预层,于是它满足唯一性公理.验证粘合性公理就是在讲,如果环上一个元在每个主开集上是幂零元,验证这个元是幂零的,注意仿射概型$\mathrm{Spec}(A)$是拟紧的,于是主开集可取有限子覆盖$\{D(f_1),D(f_2),\cdots,D(f_n)\}$.假设$f\in A$在每个$A_{f_i}$上幂零,也就是存在正整数$k_i$和足够大的正整数$N$,使得$f_i^{k_i}f^N=0$,按照$\mathrm{Spec}(A)=\cup_{1\le i\le n}D(f_i)=\cup_{1\le i\le n}D(f_i^{k_i})$,得到$1=\sum_{1\le i\le n}b_if_i^{k_i}$,于是$f^N=\sum_{1\le i\le n}b_if_i^{k_i}f^N=0$.
		
		接下来预层$U\mapsto\mathscr{O}_X(U)/\mathscr{N}_X(U)$在主开集上和$\mathrm{Spec}(A/\mathrm{nl}(A))$一致,于是它们是相同的结构层(或者说,一个预层和一个层如果在一个拓扑基上一致,那么这个层是这个预层的层化),即$X_{\mathrm{red}}=\mathrm{Spec}(A/\mathrm{nil}(A))$.
	\end{proof}
	\item 一般情况.设$X$是概形,那么$X_{\mathrm{red}}=(X,\mathscr{O}_X/\mathscr{N})$是一个概型,它是$X$的一个既约闭子概型.
	\begin{proof}
		
		在上一条中我们证明了如果$U\subseteq X$是仿射开子集,那么$\mathscr{N}_U=\widetilde{\mathrm{nil}(\mathscr{O}_X(U))}$,于是$\mathscr{N}_X$是$X$上的拟凝聚理想层,于是$X_{\mathrm{red}}$就是$X$的闭子概型.它是既约的因为在每个仿射开子集上是既约的.
	\end{proof}
	\item 既约闭子概型的泛性质.如果$X'$是$X$的具有相同底拓扑空间的闭子概型,那么包含映射$X_{\mathrm{red}}\to X$可经一个闭嵌入$X_{\mathrm{red}}\to X'$分解.按照$X'\to X$是单态射,这个分解自然是唯一的.特别的,既约闭子概型结构是唯一的.
	\begin{proof}
		
		我们需要验证层态射$\mathscr{O}_X\to\mathscr{O}_X/\mathscr{N}$经$\mathscr{O}_{X'}$分解(此时$\mathscr{O}_{X'}\to\mathscr{O}_X/\mathscr{N}$是满态射是直接的,于是$X_{\mathrm{red}}\to X'$是闭嵌入).设$\ker(\mathscr{O}_X\to\mathscr{O}_{X'})=K$,倘若我们可以证明$K\subseteq\mathscr{N}_X$,那么按照核的泛性质,得到如下图表,也就证明了可经$X'$分解.
		$$\xymatrix{K\ar[r]&\mathscr{O}_X\ar[r]\ar[d]&\mathscr{O}_X/\mathscr{N}_X\\&\mathscr{O}_{X'}\ar[ur]&}$$
		
		归结为验证仿射开子集$U$上恒有这个包含关系,不妨设$X=\mathrm{Spec}(A)$,既然$X'$是一个闭子概型,可记$X'=\mathrm{Spec}(B)$,条件即有满同态$A\to B$诱导了素谱之间的同胚映射,于是这个环同态的核会落在么个素理想里,于是落在幂零根中.
	\end{proof}
	\item 函子性.设$f:X\to Y$是概型的态射,那么存在唯一的概型的态射$f_{\mathrm{red}}:X_{\mathrm{red}}\to Y_{\mathrm{red}}$使得如下图表交换:
	$$\xymatrix{X_{\mathrm{red}}\ar[d]_{f_{\mathrm{red}}}\ar[rr]^{i_x}&&X\ar[d]^f\\Y_{\mathrm{red}}\ar[rr]_{i_Y}&&Y}$$
	
	另外如果$g:Y\to Z$是第二个概型的态射,那么有$(g\circ f)_{\mathrm{red}}=g_{\mathrm{red}}\circ f_{\mathrm{red}}$.
	\begin{proof}
		
		因为$i_Y$是单态射,说明满足图表交换的$f_{\mathrm{red}}$必然是唯一的.下面验证存在性,我们只要验证仿射情况,一般情况将仿射情况粘合即可(能保证可以粘合是因为在两个不同的仿射开子集的交$U\cap V$上,再取仿射开子集$W$,那么$U$和$V$分别限制在$W$上仍然使得$W$具备既约闭子概型结构,但是对于仿射情况这样的既约闭子概型是唯一的,于是$U\cap V$无论作为$U$还是$V$的开子概型是一致的,这就说明可以粘合).记$X=\mathrm{Spec}(B)$和$Y=\mathrm{Spec}(A)$,记$f$被$\varphi:A\to B$诱导,那么按照$\varphi(\mathrm{nil}(A))\subseteq\mathrm{nil}(B)$,说明$\varphi$诱导了$\varphi_{\mathrm{red}}:A/\mathrm{nil}(A)\to B/\mathrm{nil}(B)$,那么它诱导的态射使得图表交换.
		
		\qquad
		
		最后$(g\circ f)_{\mathrm{red}}=g_{\mathrm{red}}\circ f_{\mathrm{red}}$只要运用$(g\circ f)_{\mathrm{red}}$的唯一性.
	\end{proof}
	\item 局部闭子集上的既约子概型.设$X$是概型,设$Z\subseteq X$是局部闭子集,那么存在$Z$上唯一的既约的子概型结构,它记作$Z_{\mathrm{red}}$.
	\begin{proof}
		
		唯一性按照泛性质直接得到.下面证明存在性.首先$X$可替换为$Z$的某个开邻域,此时$Z$在$X$中闭,于是归结为假设$Z$是$X$的闭子集.对于仿射情况,$Z$具有形式$V(I)$,此时闭子概型结构即$\mathrm{Spec}(A/\sqrt{I})$.对于一般的概型$X$,先取仿射开覆盖$X=\cup_iU_i$,对每个指标$i$,存在$U_i$的唯一的既约闭子概型$Z_i$,它的拓扑空间就是$Z\cap U_i$.记$U_{ij}=U_i\cap U_j$,记概型$Z_{ij}=(Z\cap U_{ij},\mathscr{O}_{Z_i}\mid_{Z\cap U_{ij}})$.那么$Z_{ij}=Z_{ji}$(因为它们的底空间相同,而我们说明了相同底空间上的既约闭子概型结构是唯一的),这得到一族概型的粘合信息.于是按照粘合引理,这些$Z_i$粘合为一个底空间是$Z$的既约概型(因为它是既约概型的粘合).
	\end{proof}
	\item 既约子概型是控制偏序下的极小元.给定概型$X$,给定两个子概型$Z,Z'$,称$Z'$控制了$Z$,如果包含映射$Z\to X$经$Z'$分解,于是局部闭子集$Z$上的既约子概型恰好就是底空间为$Z$的全部子概型在控制偏序下的极小元.
\end{enumerate}
\subsection{谱空间}

满足如下四条性质的拓扑空间称为谱空间.
\begin{itemize}
	\item 拟紧的.
	\item 拟分离的,即两个拟紧开子集的交还是拟紧的.
	\item 拟紧开子集构成拓扑基.
	\item Sober,即每个不可约闭子集恰好存在唯一的一般点.等价于讲每个不可约闭子集存在一般点,并且空间是$T_0$的.
\end{itemize}
\begin{enumerate}
	\item Hochster's PHD Thesis证明了对拓扑空间$X$如下命题互相等价:
	\begin{itemize}
		\item 存在交换环$A$使得$X\cong\mathrm{Spec}A$.
		\item $X$是谱空间.
		\item $X$可以表示为有限$T_0$空间的逆向极限.
	\end{itemize}
    \item 有限$T_0$空间都是谱空间.
    \begin{proof}
    	
    	先说明非空的有限不可约空间都存在一般点:设$X$是有限不可约空间,那么$X$可以表示为有限并$X=\cup_{x\in X}\overline{\{x\}}$,不可约性导致$X$是某个$\overline{\{x_0\}}$,于是存在一般点.
    	
    	\qquad
    	
    	按照有限点集都是拟紧的,有限空间总是拟紧拟分离,拟紧开子集构成拓扑基(此时所有开子集都是拟紧的).最后要说明Sober,每个不可约闭子集存在一般点因为上一段,一般点唯一因为$T_0$条件.
    \end{proof}
    \item 例子.把$\mathrm{Spec}\mathbb{Z}$写成有限$T_0$空间的逆向极限.
    \begin{proof}
    	
    	记$X=\mathrm{Spec}\mathbb{Z}=\{0,p_1,p_2,\cdots\}$,其中$0$是零理想对应的一般点,$p_i$是全体素数.它的非空开集的一般形式为$\{0,p_{i_1}.p_{i_2},\cdots\}$,其中$\{p_{i_1},p_{i_2},\cdots\}$在全体素数集中余有限(即补集是有限集).
    	
    	\qquad
    	
    	取$X_0=\{0\}$,$X_n=\{0,p_1,\cdots,p_n\}$是$\mathrm{Spec}\mathbb{Z}$的子空间,那么这些$X_n$是有限$T_0$空间($T_0$传递给子空间).$X_n$的非空开集的一般形式为$\{0,p_{i_1}.p_{i_2},\cdots\}$,其中$\{p_{i_1},p_{i_2},\cdots\}$是$\{p_1,p_2,\cdots,p_n\}$的任意子集.
    	
    	\qquad
    	
    	构造$\varphi_n:X_{n+1}\to X_n$为:
    	$$0,p_{n+1}\mapsto 0$$
    	$$p_i\mapsto p_i,\forall 1\le i\le n$$
    	
    	这是连续的因为$X_n$上一般非空开集$\{0,p_{i_1},p_{i_2},\cdots\}$的原像是$\{0,p_{i_1},p_{i_2},\cdots,p_{n+1}\}$,这是$X_{n+1}$上的开集.
    	
    	\qquad
    	
    	这样$(X_n,\varphi_n)$构成一个逆像系统,它的逆向极限中的点和$\mathrm{Spec}\mathbb{Z}$是一一对应的,并且可验证这是一个同胚.
    	$$(\cdots,0,0)\mapsto 0$$
    	$$(\cdots,p_1,\cdots,p_1,0)\mapsto p_1$$
    	$$(\cdots,p_2,\cdots,p_2,0,0)\mapsto p_2$$
    	$$\cdots$$
    \end{proof}
\end{enumerate}
\subsection{Zariski空间}

我们称诺特的谱空间为Zariski空间.换句话讲拓扑空间$X$称为Zariski空间,如果它是诺特的,并且每个非空不可约闭子集都存在唯一一般点.因为诺特条件等价于所有子集都是拟紧的,这覆盖了谱空间全部前三个条件.
\begin{enumerate}
	\item 例如诺特概形的底空间是Zariski空间.
	\item Zariski空间的非空闭子集肯定包含闭点,特别的,这件事说明Zariski空间上每个闭点处任取一个开邻域,那么这些开邻域一定覆盖整个空间,因为否则这些开邻域并的补集中是非空的闭集,就理应还有闭点.
	\begin{proof}
		
		任取诺特空间$X$的非空闭子集$Y$,按照诺特条件,包含于$Y$的所有闭子集存在极小元,任取一个极小元$E$,极小性说明这是不可约的闭子集,所以存在一般点$\eta\in E$.假设还能取到$x\in E-\{\eta\}$,那么有$\overline{\{x\}}\subseteq E$,极小性保证$E=\overline{\{x\}}$,但是一般点的唯一性导致$x=\eta$,这就说明$E=\{\eta\}$只能是单点集,也即$\eta$是闭点.
	\end{proof}
    \item 如果$X$是不可约的Zariski空间,那么它的一般点包含在每个非空开子集中.
    \begin{proof}
    	
    	因为如果一般点$\eta$不在开集$U$中,那么它在闭集$U^c$中,导致$X=\overline{\{\eta\}}\subseteq U^c$,这导致$U$是空集.
    \end{proof}
    \item 设$X$是Zariski空间,如果$y\in\overline{\{x\}}$,就称$y$是$x$的特殊化,记作$x\ge y$.那么这是$X$上的一个偏序关系,我们断言这个偏序的极小元是闭点,极大元是$X$不可约分支的一般点.
\end{enumerate}
\newpage
\section{光滑态射}
\subsection{微分层}

设$f:X\to Y$是概形之间的态射,我们要构造一个和$f$相关的$X$上的拟凝聚层$\Omega_{X/Y}$,称为微分层.考虑对角态射$\Delta_{X/Y}:X\to X\times_YX$,我们知道对角态射的像集是一个局部闭子集,记$\Delta(X)$是$X\times_YX$的开子集$W$的闭子集.记$\Delta(X)$作为$W$的闭子概型的拟凝聚理想层为$\mathscr{I}$,定义$\Delta^*(\mathscr{I}/\mathscr{I}^2)$为$f$的微分层,记作$\Omega_{X/Y}$.这是拟凝聚层的回拉,所以是$X$上的拟凝聚层.
\begin{enumerate}
	\item 仿射情况.设$U=\mathrm{Spec}A$是$Y$的仿射开子集,取$f^{-1}(V)$中的一个仿射开子集$U=\mathrm{Spec}B$,那么$\Omega_{X/Y}\mid_U=\widetilde{\Omega_{B/A}}$.特别的,这说明微分层就是仿射局部上的微分模的伴随模层粘合得到的.
	\begin{proof}
		
		我们知道$V\times_UV$是$X\times_YX$的仿射开子集,同构于$\mathrm{Spec}B\otimes_AB$.于是如果记$\ker(B\otimes_AB\to B)=I$,那么$\Delta(X)\cap(V\times_UV)$就是被$I$定义的仿射的闭子概型.这里$I/I^2$就是微分模$\Omega_{B/A}$,于是得到$\Omega_{V/U}=\widetilde{\Omega_{B/A}}$.
	\end{proof}
	\item 对$x\in X$有$\left(\Omega_{X/Y}\right)_x=\Omega_{\mathscr{O}_{X,x}/\mathscr{O}_{Y,f(x)}}$;对仿射开子集$U\subseteq X$,如果$V\subseteq Y$是任意一个满足$f(U)\subseteq V$的仿射开子集,那么有$\Omega_{X/Y}\mid_U=\widetilde{\Omega_{\mathscr{O}_X(U)/\mathscr{O}_Y(V)}}$;对任意开子集$U\subseteq X$,有$\Omega_{X/Y}\mid_U=\Omega_{U/Y}$.
	\begin{proof}
		
		按照微分模的基变换,对$B$的素理想$\mathfrak{q}$,记$f(\mathfrak{q})=\mathfrak{p}$,那么有$\left(\Omega_{B/A}\right)_{\mathfrak{q}}=\Omega_{B_{\mathfrak{q}}/A}$.另外对$A\to A_{\mathfrak{p}}\to B_{\mathfrak{q}}$用微分模的第一基本正合列,结合分式化的微分模是平凡的(甚至是0-平展的),就得到$\Omega_{B_{\mathfrak{q}}/A}=\Omega_{B_{\mathfrak{q}}/A_{\mathfrak{p}}}$.
	\end{proof}
	\item 如果$Y$是诺特的,$f$是有限型态射,那么$X\times_YX$也是诺特的,此时$\mathscr{I}/\mathscr{I}^2$就是凝聚层,此时逆像$\Omega_{X/Y}$就也是凝聚层.
	\item 基变换.设$f:X\to Y$是态射,设$Y'$是$Y$概形,设$X'=X\times_YY'$,设$p:X'\to X$是投影态射,那么有$\Omega_{X'/Y'}\cong p^*\Omega_{X/Y}$.
	\item 第一基本正合列.设$f:X\to Y$,$g:Y\to Z$是概形之间的态射,那么有如下正合列:
	$$\xymatrix{f^*\Omega_{Y/Z}\ar[r]&\Omega_{X/Z}\ar[r]&\Omega_{X/Y}\ar[r]&0}$$
	\item 第二基本正合列.设$f:X\to Y$是概形之间的态射,设$Z$是$X$的闭子概型,对应的拟凝聚理想层是$\mathscr{I}$,那么有如下正合列:
	$$\xymatrix{\mathscr{I}/\mathscr{I}^2\ar[r]&\Omega_{X/Y}\otimes_{\mathscr{O}_X}\mathscr{O}_Z\ar[r]&\Omega_{Z/Y}\ar[r]&0}$$
\end{enumerate}
\subsection{光滑概形}

设$X$是域$k$上的概形.
\begin{itemize}
	\item 称点$x\in X$是相对光滑维数$d$的点,如果它有仿射开邻域$\mathrm{Spec}k[T_1,\cdots,T_n]/(f_1,\cdots,f_{n-d})$,满足它的雅各比矩阵在$x$处取值的秩是$n-d$.这里我们把$\frac{\partial f_i}{\partial T_j}(x)$理解为$\frac{\partial f_i}{\partial T_j}(T_1,\cdots,T_n)$在剩余域$\kappa(x)$中的像.如果$x\in k^n$是$k$有理点,此即把$x$带入多项式的取值;如果$x\in L^n$是$k$的某个扩域$L$值点,那么这也是把$x$带入多项式的取值.
	\item 称$X$是相对光滑维数为$d$的$k$概形,如果它每个点都是相对光滑维数$d$的点,换句话讲,$X$有仿射开覆盖,使得其中每个仿射开集可以表示为$\mathrm{Spec}k[T_1,\cdots,T_n]/(f_1,\cdots,f_{n-d})$,满足它的雅各比矩阵在这个仿射开集的每个点的秩都是$n-d$.
	\item 如果存在自然数$d$使得$k$概形$X$是相对光滑维数$d$的,我们就称$X$是光滑概形.
\end{itemize}
\begin{enumerate}
	\item 按照定义,$k$光滑概形一定是局部有限型$k$概形.
	\item 首先我们断言这个定义即结构态射$X\to\mathrm{Spec}k$是相对光滑维数$d$的态射.
	\begin{proof}
		
		一方面这个定义必然保证结构态射$X\to\mathrm{Spec}k$是相对光滑维数$d$的态射.反过来如果$X\to\mathrm{Spec}k$是相对光滑维数$d$的态射,对$X$的任意仿射开子集$U$,它就是某个$k$概形$Y=\mathrm{Spec}k[T_1,\cdots,T_n]/(f_1,\cdots,f_{n-d})$的开子概型,满足雅各比矩阵的秩为$n-d$.对每个$x\in U\subseteq Y$,我们可以选取$Y$的主开集$D(f)$,满足$x\in D(f)\subseteq U$.此时$D(f)=\mathrm{Spec}k[T_1,\cdots,T_n,Y]/(f_1,\cdots,f_{n-d},Yf-1)$.这个仿射有限型概形的雅各比矩阵的秩是$n+1-d$,并且$(n+1)-(n+1-d)=d$,于是此时$X$是我们这里定义的相对光滑维数$d$的$k$概形.
	\end{proof}
	\item 如果概型$X=\mathrm{Spec}k[x_1,x_2,\cdots,x_n]/(f_1,f_2,\cdots,f_r)$在每个闭点处的雅各比矩阵的秩为$n-d$,那么它在每个点处雅各比矩阵的秩都是$n-d$.这件事说明域上局部有限型概型验证光滑性只需验证全部闭点处.
	\begin{proof}
		
		考虑雅各比矩阵的所有$n-d$阶子式,这些式子全为零的点构成一个闭子集,它就是雅各比矩阵的秩$\le n-d$的点构成的闭子集,仿射概型的闭子集总包含闭点,这说明这个闭子集必须是空集,于是所有点的雅各比矩阵的秩$\ge n-d$.再考虑全部$n-d+1$阶子式为零的点构成的闭子集,它的补集是开子集,此即雅各比矩阵的秩$\ge n-d+1$的点构成的开子集,按照域上有限生成代数的素谱的闭点集构成稠密子集,这个开子集中理应含闭点,于是说明它也是空集.
	\end{proof}
	\item 光滑点一定是正则点.设$X$是域$k$上的局部有限型概形,设$x\in X$是相对维数$d$的光滑点.我们断言$\mathscr{O}_{X,x}$是维数$\le d$的正则局部环.如果$x$是闭点,那么$\dim\mathscr{O}_{X,x}=d$.
	\begin{proof}
		
		取$x$的仿射开邻域$U$使得其上处处光滑.我们解释过局部有限型$k$概形的闭点集是非常稠密的(此即非空局部闭子集上总存在全空间的闭点),于是存在$X$的闭点$x'\in U\cap\overline{\{x\}}$.一旦我们证明$\mathscr{O}_{X,x'}$是维数为$d$的正则局部环,那么$\mathscr{O}_{X,x}$作为它的局部化,就是维数$\le d$的正则局部环.于是问题归结为设$x$是闭点.
		
		\qquad
		
		现在设$x$存在仿射开邻域$U=\mathrm{Spec}A$,其中$A=k[T_1,\cdots,T_n]/(f_1,\cdots,f_{n-d})$,满足雅各比矩阵的秩是$n-d$.把$U$视为$\mathbb{A}_k^n$的开子集,那么这个雅各比矩阵的秩是$n-d$就是在说$f_1,\cdots,f_{n-d}$是余切空间$\mathrm{T}^*_x\mathbb{A}_k^n=\mathfrak{m}_x/\mathfrak{m}_x^2$的线性无关组,于是$f_1,\cdots,f_{n-d}$是正则局部环$k[T_1,\cdots,T_n]_{\mathfrak{m}_x}$的参数系统的一部分,于是$\mathscr{O}_{\mathbb{A}_k^n,x}/(f_1,\cdots,f_{n-d})\cong\mathscr{O}_{X,x}$是维数为$n-(n-d)=d$的正则局部环(闭点用在了$\dim\mathscr{O}_{\mathbb{A}_k^n,x}=n$).
	\end{proof}
	\item 如果$X$是域$k$上的相对光滑维数$d$的概形,那么$X$的等维数$d$的.
	\begin{proof}
		
		我们解释过此时$\dim_xX=d,\forall x\in X$.设$X$的全部不可约分支是$\{Z_i\mid i\in I\}$.对不可约分支$Z$,任取点$x\in Z$的仿射开子集$U=\mathrm{Spec}A$使得$A$是有限型$k$代数.按照不可约空间的非空开子集维数不变,说明只要$U\cap Z_i$非空,那么$\dim U=\dim U\cap Z_i$.由于$U$只有至多有限个不可约分支,并且它们就是非空的$U\cap Z_i$,于是可取$U$的开子集$V\subseteq Z$,并且不和其它的$Z_i$有交,按照$\dim_xX=\sup_{x\in C}\dim C$,其中$C$跑遍包含$x$的不可约分支,就得到$\dim Z=d$.于是$X$是等维数的.
	\end{proof}
	\item 设$k$是域,设$X=V(g_1,\cdots,g_s)\subseteq\mathbb{A}_k^n$,设$x\in X$是闭点,满足$\mathrm{rank}_{\kappa(x)}J_{g_1,\cdots,g_s}(x)=n-\dim\mathscr{O}_{X,x}$.那么$x$是相对光滑维数$d=\dim\mathscr{O}_{X,x}$的光滑点.
	\begin{proof}
		
		雅各比矩阵在点$x$处的秩是$n-d$.重排$g_i$的顺序,我们不妨设雅各比矩阵的前$n-d$行是线性无关的.记$Y=V(g_1,\cdots,g_{n-d})$,那么有$x\in X\subseteq Y\subseteq\mathbb{A}_k^n$.明显的$Y$在点$x$是相对光滑维数$d$的点,于是上一条说明$\mathscr{O}_{Y,x}$是维数是$d$的正则局部环.因为$X\subseteq Y$是闭嵌入,所以它在局部环上诱导的同态$\mathscr{O}_{Y,x}\to\mathscr{O}_{X,x}$是满同态,这导致$\mathrm{Spec}\mathscr{O}_{X,x}\to\mathrm{Spec}\mathscr{O}_{Y,x}$是闭嵌入,并且它们的维数都是$d$,并且后一个仿射概形是不可约的(因为它是正则局部环的素谱),这些条件下迫使这个闭嵌入是同构,于是$\mathscr{O}_{Y,x}\to\mathscr{O}_{X,x}$实际上是同构.但是我们解释过域$k$上两个局部有限型概形如果存在各自的点使得局部环是$k$代数同构的,那么可以找这两个点分别的开邻域$U,V$,使得存在$k$概形同构$U\cong V$,满足它在这两个点诱导的局部环同构恰好是初始的那个同构.这说明$x\in X$存在开邻域恰好是$Y$里的开邻域,于是$x$也是$X$的光滑点.
	\end{proof}
	\item 特别的,上一条告诉我们,设$X$是域$k$上的局部有限型概形,一个点$x\in X$是相对光滑维数$d$的点当且仅当它存在一个仿射开邻域$U=V(g_1,\cdots,g_s)\subseteq\mathbb{A}_k^n$,使得雅各比矩阵的秩是$n-d$(也即不要求这里雅各比矩阵必须是行满秩的).理由是在证明里我们解释了此时存在$x$在$X$中的开邻域$U$使得有开嵌入$U\subseteq V(g_1,\cdots,g_{n-d})$,这里我们重排了$\{g_i\}$使得雅各比矩阵前$n-d$行在$x$处的取值是行满秩的.
	\item 设$X$是域$k$上的局部有限型概形,设$x\in X$是闭点,设$d\ge0$是整数,设$K$是某个包含$k$的代数闭域(未要求$K$一定是$k$的代数闭包).记$X_K=X\otimes_kK$,那么如下命题互相等价:
	\begin{enumerate}
		\item $k$概形$X$在点$x$是相对光滑维数$d$的.
		\item 对任意的(或者存在一个)$\overline{x}\in X_K$是$x$的提升,都有$\overline{x}\in X_K$是相对光滑维数$d$的.
		\item 对任意的(或者存在一个)$\overline{x}\in X_K$是$x$的提升,都有$\mathscr{O}_{X_K,\overline{x}}$是维数$d$的正则局部环.
		\item 我们有$\dim_{\kappa(x)}\mathrm{T}_x(X/k)=\dim\mathscr{O}_{X,x}=d$.
		\item 对任意的(或者存在一个)$\overline{x}\in X_K$是$x$的提升,都有$\widetilde{\mathscr{O}_{X_K,\overline{x}}}=K[[T_1,\cdots,T_d]]$.
	\end{enumerate}
	\begin{proof}
		
		(a)推(b)是光滑性在基变换下不变.(b)推(c)是我们前面证明的相对光滑维数$d$的闭点是局部环的维数$d$的正则点.(c)和(d)的等价性是因为我们解释过$\dim_{\kappa(x)}\mathrm{T}_x(X/k)=\dim_K\mathrm{T}_{\overline{x}}(X_K)$.还解释过$\dim\mathscr{O}_{X,x}=\dim\mathscr{O}_{X_K,\overline{x}}$.
		
		\qquad
		
		(c)推(a):我们只要证明(c)中"对任意的"改为"存在一个"时能推出(a)即可.因为代数闭域$K$上局部有限型概形的闭点就是$K$有理点,此时正则点和光滑点一致,于是$\overline{x}$是$X_K$的光滑点.不妨用光滑性定义中的$x$的开邻域$V(g_1,\cdots,g_{n-d})\subseteq\mathbb{A}_k^n$替代$X$,使得雅各比矩阵在点$x$的秩是$n-d$.为了证明$x$是$X$的相对光滑维数$d$的点,只需验证$\mathrm{rank}_{\kappa(x)}J_{g_1,\cdots,g_{n-d}}(x)=n-d$.但是矩阵的秩是在域扩张下不变的,于是$\mathrm{rank}_{\kappa(x)}=\mathrm{rank}_KJ_{g_1,\cdots,g_{n-d}}(\overline{x})$.
		
		\qquad
		
		最后(c)和(e)等价是因为域上概形的局部环一定是等特征的,一个局部环是正则的当且仅当它的完备化是正则局部环,而等特征的完备正则局部环一定具有形式$K[[T_1,\cdots,T_d]]$.
	\end{proof}
	\item 推论.如果$x$是域$k$上概形$X$的$k$有理点,那么$x$是光滑点当且仅当它是正则点.并且此时它的相对光滑维数就是它局部环的维数.更一般的,如果$x$是域$k$上局部诺特概形$X$的$K$值点,满足$K/k$是有限可分扩张,那么$x$是光滑点等价于是正则点.特别的,如果$X$是完全域$k$上的局部有限型概形,那么$X$是光滑的当且仅当它是正则的.
	\begin{proof}
		
		首先我们断言投影态射$X_K\to X$是有限平展态射.它是$\mathrm{Spec}K\to\mathrm{Spec}k$的基变换,而这个态射是有限态射因为$K/k$有限维,平坦态射因为域上模都是平坦的,$K/k$是有限可分扩张导致它是非分歧态射,而这些性质都是基变换不变的.于是$x\in X$是正则点当且仅当它的原像$x'\in X_K$是正则点.但是$x$的原像$x'\in X_K$都是$K$有理点,于是$x'$是$X_K$的光滑点,于是上一条说明$x\in X$是光滑点.
	\end{proof}
	\item 推论.下降性质.设$X$是域$k$上的概形.
	\begin{enumerate}
		\item 设$K$是包含$k$的一个代数闭域(未要求$K$一定是$k$的代数闭包),那么$X$是相对光滑维数$d$的$k$概形当且仅当$X_K$是相对光滑维数$d$的$k$概形.
		\item 设$L/k$是任意域扩张,那么$X_L$是相对光滑维数$d$的$L$概形当且仅当$X$是相对光滑维数$d$的$k$概形.
	\end{enumerate}
	\begin{proof}
		
		如果$X$是相对光滑维数$d$的$k$概形,基变换$X_K$自然是相对光滑维数$d$的$K$概形.反过来如果$X_K$是相对光滑维数$d$的$K$概形.按照满射是基变换不变性质,有投影态射$X_K\to X$是满射,于是上一条说明$X$的每个点都是相对光滑维数$d$的点.
		
		\qquad
		
		现在设$L/k$是任意域扩张,设$X_L$是相对光滑维数$d$的$L$概形,取$L$的代数闭包为$K$,那么$K$是同时包含$L$和$k$的一个代数闭域.于是(a)告诉我们$X_K$是相对光滑维数$d$的$K$概形,再用一次(a)告诉我们$X$是相对光滑维数$d$的$k$概形.
	\end{proof}
	\item 推论.设$X$是域$k$上的局部有限型概形,那么如下命题互相等价.
	\begin{enumerate}
		\item $X$是光滑$k$概形.
		\item $X$是几何正则的,也即对任意域扩张$L/k$,都有$X_L=X\times_kL$是正则概形.
		\item 存在某个包含$k$的代数闭域$K$(未要求$K$一定是$k$的代数闭包),使得$X_K$是正则概形.
	\end{enumerate}
	\item 正则但不光滑的例子.考虑态射$f:\mathrm{Spec}\mathbb{F}_p(T)\to\mathrm{Spec}\mathbb{F}_p(T^p)$,其中$p$是一个素数,那么这里$\mathrm{Spec}\mathbb{F}_p(T)$明显是一个正则概形.但是我们考虑$f$和自己做纤维积,得到$\mathrm{Spec}\mathbb{F}_p(T)\otimes_{\mathbb{F}_p(T^p)}\mathbb{F}_p(T)$.我们断言这个张量积就是$\mathbb{F}_p(T)[X]/(X^p)$.一旦这成立,这个张量积甚至不是既约的,就不可能是正则的,于是$f$就不是光滑态射.验证这个断言只要验证如下图表具有泛映射性质即可:
	$$\xymatrix{\mathbb{F}_p(T^p)\ar[rr]\ar[d]&&\mathbb{F}_p(T)\ar[d]^{T\mapsto T+X}\\\mathbb{F}_p(T)\ar[rr]^{T\mapsto T}&&\mathbb{F}_p(T)[X]/(X^p)}$$
	
	事实上域扩张是光滑态射当且仅当它是有限可分扩张:因为如果$\mathrm{Spec}L\to\mathrm{Spec}K$是光滑态射,由于纤维的局部维数是零,导致它是相对光滑维数零的,于是它是平展态射,于是$L/K$是有限可分扩张.
	\item 设$X$是域$k$上的光滑概形,那些剩余域在$k$上是可分扩张的闭点构成的集合在$X$里稠密.
	\begin{proof}
		
		对每个点$x_0\in X$,存在开邻域$U$和分解$\xymatrix{U\ar[r]^g&\mathbb{A}_k^n\ar[r]^p&\mathrm{Spec}k}$,使得$g$是平展态射.于是按照定义,对$x\in U$,有$\kappa(x)/\kappa(g(x))$是有限可分扩张.所以我们只要证明$g(U)$中包含一个闭点$y$,使得$\kappa(y)$是$k$的可分扩张.但是由于$g$是平坦的,它是开映射,所以$g(U)$是开集.问题归结为$\mathbb{A}_k^n$上剩余域是$k$的可分扩张的点是稠密的.如果$k$是无限域,对每个非空主开集我们总可以取一个$k$有理点在非空主开集里,于是此时可分扩张的点是稠密的;如果$k$是有限域,那它是完全域,它的代数扩张总是可分的,所以它的闭点都是可分扩张点,但是域上局部有限型概形的闭点是稠密的,这得证.
	\end{proof}
\end{enumerate}
\subsection{非分歧态射}

设$f:X\to Y$是概形之间的局部有限表示态射,设$x\in X$和$y=f(x)$,称$f$在点$x$非分歧(unramified),如果$\mathscr{O}_{Y,y}\to\mathscr{O}_{X,x}$满足$\mathfrak{m}_y\mathscr{O}_{X,x}=\mathfrak{m}_x$,并且对应的剩余域扩张$\kappa(y)\to\kappa(x)$是有限可分扩张.如果局部有限表示态射$f$在$X$上处处非分歧,就称它是非分歧态射.
\begin{enumerate}
	\item 例如设$L/K$是数域的扩张,设$\mathscr{O}_L$和$\mathscr{O}_K$都是代数整数环,我们解释过$f:\mathrm{Spec}\mathscr{O}_L\to\mathrm{Spec}\mathscr{O}_K$是平坦态射.对$\mathscr{O}_L$的素理想$\mathfrak{q}$,记$\mathfrak{p}=\mathfrak{q}\cap\mathscr{O}_K$,那么域扩张$\kappa(\mathfrak{q})/\kappa(\mathfrak{p})$是可分的,于是$f$在$\mathfrak{q}$上非分歧(平展),当且仅当$\mathfrak{p}$在$\mathfrak{q}$上的分歧指数是1,也即和数域上的非分歧概念吻合.
	\item 等价描述.设$f:X\to Y$是局部有限表示态射,设$x\in X$和$y=f(x)$,如下命题互相等价:
	\begin{enumerate}
		\item $f$在点$x$非分歧.
		\item $(\Omega_{X/Y})_x=0$.
		\item 对角态射$\Delta:X\to X\times_YX$在$x$的某个开邻域是开嵌入.特别的,这一条说明一点非分歧能推出在某个开邻域上非分歧.
		\item 纤维态射$X_y=X\times_Y\mathrm{Spec}\kappa(y)\to\mathrm{Spec}\kappa(y)$在点$x$是非分歧的.
	\end{enumerate}
	\begin{proof}
		
		(a)等价于(d):问题是局部的,都设为仿射的$X=\mathrm{Spec}B$和$Y=\mathrm{Spec}A$,按照微分模和基变换的公式,得到$\Omega_{X_y/\kappa(y),x}=\Omega_{X/Y,x}\otimes_{\mathscr{O}_{Y,y}}\kappa(y)=\Omega_{X/Y,x}/\mathfrak{m}_x\Omega_{X/Y,x}$.于是$\Omega_{X/Y,x}=0$推出$\Omega_{X_y/\kappa(y),x}=0$.反过来如果后者为零,因为$f$是局部有限型态射,得到$\mathscr{O}_{Y,y}$是有限型$\mathscr{O}_{X,x}$代数,按照微分模的第二基本正合列,$(\Omega_{X/Y})_x$是有限$\mathscr{O}_{X,x}$模,于是从NAK引理得到$(\Omega_{X/Y})_x=0$.
		
		\qquad
		
		(a)推(b):首先我们有如下张量积图表,这是因为$\mathscr{O}_{X,x}\otimes_{\mathscr{O}_{Y,y}}\kappa(y)\cong\mathscr{O}_{X,x}/\mathfrak{m}_y\mathscr{O}_{X,x}=\mathscr{O}_{X,x}/\mathfrak{m}_x=\kappa(x)$.
		$$\xymatrix{\mathscr{O}_{Y,y}\ar[rr]\ar[d]&&\mathscr{O}_{X,x}\ar[d]\\\kappa(y)\ar[rr]&&\kappa(x)}$$
		
		于是按照微分模的基变换(此即如果$B'=B\otimes_AA'$,那么$\Omega_{B'/A'}=\Omega_{B/A}\otimes_BB'=\Omega_{B/A}\otimes_AA'$),有$(\Omega_{X/Y})_x\otimes_{\mathscr{O}_{X,x}}\kappa(x)\cong\Omega_{\kappa(x)/\kappa(y)}$.但是我们解释过有限可分扩张的微分模平凡,于是这里$(\Omega_{X/Y})_x\otimes_{\mathscr{O}_{X,x}}\kappa(x)=0$.因为$f$是局部有限型态射,得到$\mathscr{O}_{Y,y}$是有限型$\mathscr{O}_{X,x}$代数,按照微分模的第二基本正合列,$(\Omega_{X/Y})_x$是有限$\mathscr{O}_{X,x}$模,于是从NAK引理得到$(\Omega_{X/Y})_x=0$.
		
		\qquad
		
		(b)推(c):设$W$是$X\times_YX$的开子集,使得$\Delta(X)\subseteq W$,于是$\Delta$诱导了闭嵌入$X\to W$.并且如果记这个闭嵌入的拟凝聚理想层是$\mathscr{I}$,那么我们知道微分模层就是$\Omega_{X/Y}=\Delta^*(\mathscr{I}/\mathscr{I}^2)$.设$A=\mathscr{O}_{Z,\Delta(x)}$,其中$Z=X\times_YX$,那么$\mathscr{I}_zz=I$是$A$的有限生成理想,其中$z=\delta(x)$.于是有$0=(\Omega_{X/Y})_x=(\Delta^*\mathscr{I}/\mathscr{I}^2)_x=A/I\otimes_AI/I^2=I/I^2$.于是$I=I^2$.从$I=I^2\subseteq\mathfrak{m}_zI\subseteq I$得到$I=\mathfrak{m}_zI$,于是NAK引理得到$I=0$,也即$\mathscr{I}_z=0$,那么存在$z$的开邻域$U$上有$\mathscr{I}$处处为零,也即$\Delta$在$x$的某个开邻域上是同构.
		
		\qquad
		
		(c)推(a):做基变换$\mathrm{Spec}\kappa(y)\to Y$,因为有如下magic图表,开嵌入的基变换还是开嵌入,所以问题归结为设$Y=\mathrm{Spec}k$是域的素谱的情况.
		$$\xymatrix{X_y\ar[rr]\ar[d]&&X_y\times_{\kappa(y)}X_y\ar[d]\\X\ar[rr]&&X\times_YX}$$
		
		把$X$替换为$x$的仿射开邻域,我们可以要求$\Delta$是开嵌入.先设$k$是代数闭域,任取闭点$t\in X$,也即有$k$态射$t:\mathrm{Spec}k\to X$,它的像集是$\{t\}$.把$t$的图像态射记作$\Gamma_t$,这和$t$是相同的态射,我们有如下纤维积图表:
		$$\xymatrix{\mathrm{Spec}k\ar[rr]^{\Gamma_t}\ar[d]&&\mathrm{Spec}k\times_kX\ar[d]\\X\ar[rr]^{\Delta}&&X\times_kX}$$
		
		因为$\Delta$是开嵌入,得到$\Gamma_t$也是开嵌入,于是$t$也是开嵌入.换句话讲我们证明了$X$的每个闭点都是孤立点,并且$\Gamma(\{t\},\mathscr{O}_X)=k$.任取一个非闭点$y$,它理应是某个闭点$x$的特殊化,但是由于$\{x\}$是开集,它理应保特殊化,于是$y\in\{x\}$,换句话讲$X$上只有闭点.另外$X=\mathrm{Spec}A$满足$A$是诺特的,$X$是有限个$\mathrm{Spec}k$的无交并.这满足(a)的条件.
		
		\qquad
		
		再设$k$是一般的域,设$\overline{k}$是它的一个代数闭包.上述证明告诉我们$X_{\overline{k}}=X\times_k\overline{k}$是有限个$\mathrm{Spec}\overline{k}$的无交并.于是$X$也必须是有限集($X_{\overline{k}}\to X$是满射).另外这里$X_{\overline{k}}\to X$是开映射(我们可以用,比方说,如果$S$的底空间是离散空间,那么$f:X\to S$总是泛开的),于是$X$也是离散空间,用一个单点替换$X$,那么$X=\mathrm{Spec}A$,其中$A$是有限维$k$代数.按照$A\otimes_k\overline{k}$是既约的,说明$A$也是既约的,但是只有一个素理想的既约环只能是域,所以$A$是域,并且从从$A$是有限维$k$代数得到$A/k$是有限扩张,从$A\otimes_k\overline{k}$既约得到$A/k$是可分代数扩张.
	\end{proof}
	\item 推论.设$X$是域$k$上的局部有限型概形,如下命题互相等价:
	\begin{enumerate}
		\item $X$是非分歧的.
		\item $X$是平展的.
		\item $X$是若干$\mathrm{Spec}L_i$的无交并(于是此为离散空间),这里$L_i/k$是有限可分扩张.
	\end{enumerate}
	\begin{proof}
		
		(a)和(b)等价因为域上模都是平坦的.(c)推(a)就是非分歧的定义.(a)推(c):可设$X=\mathrm{Spec}A$是有限生成$k$代数,非分歧条件要求了$\mathscr{O}_{X,x}$是域,并且是$k$的有限可分扩张.于是特别的对$A$的每个极大理想$\mathfrak{m}$有$A_{\mathfrak{m}}$是域,这导致$\dim A=0$,于是$A$是阿廷环,它是有限个阿廷局部环的积,这些阿廷局部环在这里就是域,并且是$k$的有限可分扩张.
	\end{proof}
	\item 推论.设$f:X\to Y$是局部有限表示态射,设它在点$x$非分歧,那么它在点$x$是拟有限的,换句话讲记$y=f(x)$,那么存在$x$和$y$的仿射开邻域$U$和$V$,满足$f(U)\subseteq V$,并且$f\mid_U:U\to V$是拟有限的.
	\begin{proof}
		
		首先我们解释了在一点非分歧可以推出在某个开邻域上非分歧,于是我们可以归结为设$Y=\mathrm{Spec}A$和$X=\mathrm{Spec}B$,其中$B$是有限型$A$代数,设$f$是非分歧的.我们要证明的是$f$处处具有有限纤维.任取$y\in Y$,我们解释了$X_y\to\mathrm{Spec}\kappa(y)$也是非分歧态射.上一条说明$X_y$是离散空间,又因为它是诺特环的素谱,所以拟紧,所以它是有限点集.
	\end{proof}
	\item 设$f:X\to Y$和$g:Y\to Z$是局部有限表示态射,设$f$在点$x$非分歧,$g$在点$y=f(x)$非分歧,那么$g\circ f$在点$x$非分歧.
	\begin{proof}
		
		问题是局部的,不妨都设为仿射概形.我们有如下第一基本正合列.
		$$\xymatrix{\Omega_{B/A}\otimes_BC\ar[r]&\Omega_{C/A}\ar[r]&\Omega_{C/B}\ar[r]&0}$$
		
		设$\mathfrak{q}\in\mathrm{Spec}C$满足$(\Omega_{C/B})_{\mathfrak{q}}=0$,设$\mathfrak{q}\cap B=\mathfrak{p}$(这是指$\mathfrak{q}$在结构同态$B\to C$下的原像,这个记号方便在于不用设出这个具体的映射),我们有:
		\begin{align*}
			(\Omega_{B/A}\otimes_BC)_{\mathfrak{q}}&=\Omega_{B/A}\otimes_BC\otimes_CC_{\mathfrak{q}}\\&=\Omega_{B/A}\otimes_BC_{\mathfrak{q}}\\&=\Omega_{B/A}\otimes_B(C_{\mathfrak{q}}\otimes_BB_{\mathfrak{p}})\\&=(\Omega_{B/A})_{\mathfrak{p}}\otimes_{B_{\mathfrak{p}}}C_{\mathfrak{q}}
		\end{align*}
		
		对上面正合列做$\mathfrak{q}$的局部化,就得到$(\Omega_{C/A})_{\mathfrak{q}}=0$,此即$g\circ f$在$x_{\mathfrak{q}}\in X=\mathrm{Spec}C$上非分歧.当然也可以用初始的定义,有$\mathfrak{m}_z\mathscr{O}_{X,x}=\mathfrak{m}_y\mathscr{O}_{X,x}=\mathfrak{m}_x$以及有限可分扩张的复合还是有限可分扩张.
	\end{proof}
	\item 非分歧态射是终端和源端局部性质.
	\item 非分歧态射在基变换下还是非分歧态射.这是因为微分层在基变换下是投影态射的逆像.
	\item 嵌入总是非分歧的.事实上只要$f$是单态射,那么$\Delta$就是同构,于是如果$f$还是局部有限型态射,那么它就是非分歧的.
	\item 如果局部有限型态射的复合$\xymatrix{X\ar[r]^f&Y\ar[r]^g&Z}$是非分歧的,那么$f$就是非分歧的.这从微分层的第一基本正合列直接得到.
	\item 如果局部有限表示态射的复合$\xymatrix{X\ar[r]^f&Y\ar[r]^g&Z}$满足$g$是非分歧的,那么$\Gamma_f:X\to X\times_ZY$是开嵌入.这件事是因为有如下纤维积图表:
	$$\xymatrix{X\ar[rr]^{\Gamma_f}\ar[d]&&X\times_ZY\ar[d]\\Y\ar[rr]^{\Delta}&&Y\times_ZY}$$
\end{enumerate}
\subsection{平展态射}

设$f:X\to Y$是局部有限型态射,称$f$在点$x\in X$是平展的(\'etale),如果它在点$x$是非分歧和平坦的.如果$f$在$X$上处处平展,就称$f$是平展态射.
\begin{enumerate}
	\item 我们解释过平坦和非分歧都保复合与基变换,于是平展态射保复合与基变换.
	\item 开嵌入是平坦和非分歧的,所以总是平展态射.另外我们解释过闭嵌入是平坦的当且仅当还是开嵌入,所以一个嵌入是平展的当且仅当它是开嵌入.
	\item 设态射的复合$\xymatrix{X\ar[r]^f&Y\ar[r]^g&Z}$是平展的,$g$是非分歧的,那么$f$是平展的.因为有如下交换图表:
	$$\xymatrix{X\ar[rr]^{\text{open imm}}&&X\times_ZY\ar[rr]^{\text{\'etale}}\ar[d]&&Y\ar[d]\\&&X\ar[rr]^{\text{\'etale}}&&Z}$$
	\item 设$A$是环,设$p(T)\in A[T]$是首一多项式,设$B=A[T]/(p(T))$,典范态射$f:\mathrm{Spec}B\to\mathrm{Spec}A$在素理想$\mathfrak{q}\in\mathrm{Spec}B$是平展的当且仅当$\mathfrak{q}$不包含$p'(T)$在$B$中的像.于是$f$是平展态射当且仅当$p'(T)$是$B$中的单位元,换句话讲$p(T)$和$p'(T)$在$A[T]$中生成了单位理想.
	\begin{proof}
		
		因为$p$是首一的,于是$B$是有限秩自由$A$模,于是$B$在$A$上平坦.设$C=A[T]$,我们有微分模的第二基本正合列:
		$$\xymatrix{I/I^2\ar[r]^{\delta}&\Omega_{C/A}\otimes_CB\ar[r]&\Omega_{B/A}\ar[r]&0}$$
		
		其中$I=(p(T))$,$\delta$是把$p(T)$映射为$\mathrm{d}p(T)\otimes1$.我们有$\Omega_{A[T]/A}$是以$\mathrm{d}T$为基的自由$A[T]$模.于是$\Omega_{C/A}\otimes_CB$是以$\mathrm{d}T$为基的自由$B$模.而$\delta$的像集是$p'(T)\mathrm{d}T$和$p(T)\mathrm{d}T$生成的$B$子模,于是上述正合列得到$\Omega_{B/A}\cong A[T]/(p(T),p'(T))$.于是$(\Omega_{B/A})_{\mathfrak{q}}=0$当且仅当$\mathfrak{q}$不包含$p'(T)$.
	\end{proof}
	\item 设$f:X\to Y$是局部诺特概形之间的平展态射,设$x\in X$和$y=f(x)$.
	\begin{enumerate}
		\item $\dim\mathscr{O}_{X,x}=\dim\mathscr{O}_{Y,y}$,且$f$是拟有限态射.
		\item $f$诱导的切映射$\mathrm{T}_xX\to\mathrm{T}_yY\otimes_{\kappa(y)}\kappa(x)$是同构.
		\item 于是特别的,$X$在点$x$正则当且仅当$Y$在点$y$正则.不过在交换代数里我们证明了更强的结论:只要$f$是平坦的,从$x\in X$正则能推出$y=f(x)\in Y$正则.
	\end{enumerate}
	\begin{proof}
		
		我们解释过非分歧态射就已经是拟有限态射,并且此时纤维$X_y$是零维空间.我们还解释过局部诺特概形之间的平坦态射满足维数等式,结合纤维是零维的就得到两个局部环的维数相同.下面证明(b),设$A=\mathscr{O}_{Y,y}$和$B=\mathscr{O}_{X,x}$,那么有:
		$$\mathfrak{m}_y/\mathfrak{m}_y^2\otimes_{\kappa(y)}\kappa(x)=(\mathfrak{m}_y\otimes_A\kappa(y))=\mathfrak{m}_y\otimes_A(B/\mathfrak{m}_yB)$$
		
		按照$B$在$A$上平坦,我们有$\mathfrak{m}_y\otimes_AB\cong\mathfrak{m}_yB=\mathfrak{m}_x$.于是上式同构于$\mathfrak{m}_x/\mathfrak{m}_x^2$.
	\end{proof}
	\item 引理.设$X\to X'\to Y$都是有限型态射,设$Y$是局部诺特概形,设$x\in X$,它在$X'$和$Y$中的像分别记住$x'$和$y$.如果这三个态射中任意两个在标记的点上平展,那么第三个态射在标记的点上也平展.【SGA1】
\end{enumerate}
\subsection{光滑态射}

设$f:X\to Y$是概形之间的态射,设$d\ge0$是整数.
\begin{itemize}
	\item 称$f$在点$x\in X$的相对光滑维数为$d$,如果存在$x$和$f(x)$的仿射开邻域$U$和$V=\mathrm{Spec}(R)$,使得$f(U)\subset V$,并且存在$R$概型的开嵌入(也即$U\to V$要经这个开嵌入分解)$j:U\to\mathrm{Spec}R[T_1,T_2,\cdots,T_n]/(f_1,f_2,\cdots,f_{n-d})$,使得Jacobian矩阵$J_{f_1,f_2,\cdots,f_{n-d}}(x)=\left(\frac{\partial f_i}{\partial T_j}(x)\right)_{i,j}\in M_{(n-d)\times n}(\kappa(x))$的秩为$n-d$.(这里$\frac{\partial f_i}{\partial T_j}(x)$理解为$\frac{\partial f_i}{\partial T_j}(T_1,\cdots,T_n)$在剩余域$\kappa(x)$中的像,如果$R$是域,如果$x\in R^n$是$R$有理点,那么此即把$x$带入这个多项式的值).
	\item 如果$f$在$x\in X$处存在相对光滑维数,就称$f$在$x\in X$处是光滑的.如果$f$在$X$上处处光滑,就称$f$是光滑态射.如果$f$处处是相对光滑维数$d$的,就称$f$是相对光滑维数$d$的态射.
\end{itemize}
\begin{enumerate}
	\item 我们断言定义中的$U$是某个$\mathrm{Spec}R[T_1,\cdots,T_n]/(f_1,\cdots,f_{n-d})$的开子概型可以直接改为$U$具有形式$\mathrm{Spec}R[T_1,\cdots,T_n]/(f_1,\cdots,f_{n-d})$.
	\begin{proof}
		
		一方面如果$U$干脆可以取为$\mathrm{Spec}R[T_1,\cdots,T_n]/(f_1,\cdots,f_{n-d})$的话自然满足光滑定义.反过来如果$U$可以开嵌入到$\mathrm{Spec}R[T_1,\cdots,T_n]/(f_1,\cdots,f_{n-d})$,那么存在后者的主开集$D(f)$包含$x$并且落在$U$中.此时$D(f)=\mathrm{Spec}R[T_1,\cdots,T_n,Y]/(f_1,\cdots,f_{n-d},Yf-1)$.它的雅各比矩阵的秩是$n+1-d$.
	\end{proof}
	\item 按照定义,光滑态射一定是局部有限表示态射,于是特别的它一定是局部有限型态射.并且对环$R$总有结构态射$\mathbb{A}_R^n\to\mathrm{Spec}R$是相对光滑维数$n$的,进而仿射空间粘合的$\mathbb{P}_R^n\to\mathrm{Spec}R$也是相对光滑维数$n$的.再比如开嵌入一定是平展态射.还有比较简单的是投影态射$\mathbb{A}_k^n\times_kY\to Y$是相对光滑维数$n$的.
	\item 设$f:X\to Y$是概型的态射,设$f$在$x\in X$的相对光滑维数是$d$,那么存在$x$的开邻域$U$上处处相对光滑维数是$d$.特别的,这说明$X$连通条件下,如果$f$光滑,那么处处相对光滑维数相同.特别的,这还说明光滑点构成一个开子集,这称为$X$关于$f$的光滑中心(smooth locus),记作$X_{\mathrm{sm}}(f)$.
	\begin{proof}
		
		存在$x$的开邻域$U$,可以开嵌入到某个$Z=\mathrm{Spec}R[T_1,\cdots,T_n]/(f_1,\cdots,f_{n-d})$,并且它的雅各比矩阵在$x$点的秩是$n-d$,换句话讲存在一个$n-d$阶子式,在剩余域$\kappa(x)$中不为零.换句话讲这个$n-d$阶子式$a$不落在$\mathfrak{p}_x$中,那么对开集$D(a)$中的每个点$y$都有$a$在$\kappa(y)$中不为零.换句话讲存在$x$的开邻域使得其上处处是相对维数$d$的.
	\end{proof}
	\item 光滑性在终端和源端是局部的.换句话讲,取开集$U\subset X$,取$x\in U$,那么$f$在$x$处光滑当且仅当$f\mid U$在$x$处光滑;取开集$V\subset X$,取$x\in f^{-1}(V)$,那么$f$在$x$处光滑当且仅当$f$限制为$f^{-1}(V)\to V$在$x$处光滑.
	\item 基变换.设$f:X\to Y$是相对光滑维数$d$的态射,任取态射$Y'\to Y$,那么对$X\times_YY'$的每个投影到$x$的点,基变换$X\times_YY'\to Y'$在这个点的相对光滑维数都是$d$.
	\begin{proof}
		
		因为光滑性是终端局部的,我们不妨设$Y'=\mathrm{Spec}C$是仿射的.任取$Y$的仿射开子集$V=\mathrm{Spec}B$,取$V$在$X$中原像里的任意仿射开子集$U=\mathrm{Spec}A$,我们知道有分解$B\to B[X_1,\cdots,X_n]/(f_1,\cdots,f_{n-d})\to A$,满足两件事:后一个同态诱导了开嵌入;中间有限型$B$代数的Jacobian矩阵的秩是$n-d$.我们有$U\times_VY'=\mathrm{Spec}C\otimes_BA$覆盖了$Y'$在$X'$中的原像,并且$C\to A\otimes_BC$可分解为$C\to C\otimes_BB[X_1,\cdots,X_n]/(f_1,\cdots,f_{n-d})\to A\otimes_BC$.按照开嵌入在基变换下不变,说明这里$C\otimes_BB[X_1,\cdots,X_n]/(f_1,\cdots,f_{n-d})\to A\otimes_BC$诱导的态射也是开嵌入.另外$C\otimes_B[X_1,\cdots,X_n]/(f_1,\cdots,f_{n-d})=C[X_1,\cdots,X_n]/(f_1,\cdots,f_{n-d})$的Jacobian矩阵的秩也是$n-d$,因为在扩环下一个元素是否为零是不变的.这就说明$X'\to Y'$是光滑的.
	\end{proof}
	\item 复合.设$f:X\to Y$和$g:Y\to Z$是两个态射,$f$在$x\in X$处相对光滑维数$d$,$g$在$y=f(x)\in Y$处相对光滑维数$e$,那么$g\circ f$在$x$处相对光滑维数$d+e$.特别的,平展态射的复合仍然是平展态射.
	\begin{proof}
		
		任取$z=g(y)\in Z$的仿射开邻域$W=\mathrm{Spec}C$,那么存在$y$的仿射开邻域$V=\mathrm{Spec}B$,其中$B=C[T_1,\cdots,T_n]/(f_1,\cdots,f_{n-e})$,使得雅各比矩阵在$y$的秩是$n-e$,并且$g(V)\subseteq W$.也存在$x\in X$的仿射开邻域$U=\mathrm{Spec}A$,其中$A=B[S_1,\cdots,S_m]/(g_1,\cdots,g_{m-d})$,使得雅各比矩阵在$x$的秩是$m-d$,并且$f(U)\subseteq V$.那么$g\circ f(U)\subseteq W$,并且$A=C[T_1,\cdots,T_n,S_1,\cdots,S_m]/(f_1,\cdots,f_{n-e},g_1,\cdots,g_{m-d})$的雅各比矩阵在点$x$是一个分块下对角矩阵,它的秩是$(n-e)+(m-d)$,于是$g\circ f$在点$x$处相对光滑维数是$d+e$.
	\end{proof}
	\item 设态射$f:X\to Y$在点$x$的相对光滑维数是$d$,记$y=f(x)$,那么$d=\dim_xX_y$.
	\begin{proof}
		
		因为基变换$X_y\to\mathrm{Spec}\kappa(y)$也在点$x$的相对光滑维数是$d$,所以不妨设$Y=\mathrm{Spec}k$本身就是域的素谱,此时$X$是$k$上的局部有限型概形.因为对$x$的开邻域$U$总有$\dim_xX=\dim_xU$,我们可以用$x$的仿射开子集代替$X$,于是不妨设$X$是$k$上有限型代数的素谱,还可以设$X$是$k$上的光滑概形.设此时$X$的全部不可约分支是$C_1,\cdots,C_r$,那么$\dim_xX=\max_{x\in C_i}\dim C_i$.设$i_0$满足$\dim_xX=\dim C_{i_0}$.我们可以取一个包含在$C_{i_0}$中的开子集$V$不和其它的$C_i$有交.我们知道域上不可约的局部有限型概形上非空开子集的维数全相同,于是$\dim V=\dim C_{i_0}$.用$V$替代$X$,我们归结为设$X$是不可约的光滑$k$概形,要证明的是$\dim X=d$.但是任取$X$的闭点$x$,我们有$\dim X=\dim_xX=\dim\mathscr{O}_{X,x}$,而我们解释过对光滑$k$概形上的闭点$x$有$\dim\mathscr{O}_{X,x}$就是相对光滑维数$d$.
	\end{proof}
	\item 设$f:X\to Y$在点$x\in X$是相对光滑维数$d$的,那么有$(\Omega_{X/Y})_x$是秩为$d$的自由$\mathscr{O}_{X,x}$模.于是如果$f$是相对光滑维数$d$的,那么$\Omega_{X/Y}$是$X$上的局部秩$d$自由模层.
	\begin{proof}
		
		因为$B=A[T_1,\cdots,T_n]/(f_1,\cdots,f_{n-d})$如果满足雅各比矩阵在$\kappa(x)$上的秩是$n-d$,那么存在一个$n-d$阶子式在$\kappa(x)$上不为零,于是这个子式在$\mathscr{O}_{X,x}$里就是可逆元,于是我们可以把雅各比矩阵扩充为$n$阶可逆矩阵,对应的多项式记作$f_1,\cdots,f_n$,这是$\Omega_{A[T_1,\cdots,T_n]/A}$的一组基,于是$\mathrm{d}f_{n-d+1},\cdots,\mathrm{d}f_n$是$\Omega_{B/A}$的一组基.
	\end{proof}
	\item 等价定义.设$f:X\to Y$是概形之间的局部有限表示态射,那么它在点$x\in X$的相对光滑维数是$d$当且仅当满足如下条件:
	\begin{enumerate}
		\item $f$在点$x$平坦态射.
		\item 对$y=f(x)\in Y$,有$\kappa(y)$概形$X_y$是$d$维的几何正则概形.或者等价的讲,$X_y\to\mathrm{Spec}\kappa(y)$在点$x$相对光滑维数$d$.
	\end{enumerate}
	\begin{proof}
		
		先设$f$在点$x$相对光滑维数是$d$,按照光滑性在基变换下不变,得到(b).只需证明$f$在点$x$是平坦的.不妨设$Y=\mathrm{Spec}A$是仿射的,那么存在$x$的仿射开邻域$U$,使得$U=\mathrm{Spec}A[T_1,\cdots,T_n]/(f_1,\cdots,f_{n-d})$满足雅各比矩阵在点$x$是行满秩的,并且$f(U)\subseteq V$.把$f\mid_U:U\to V$分解为如下复合:
		$$\xymatrix{\mathrm{Spec}\frac{A[T_1,\cdots,T_n]}{(f_1,\cdots,f_{n-d})}\ar[r]&\cdots\ar[r]&\mathrm{Spec}\frac{A[T_1,\cdots,T_n]}{(f_1)}\ar[r]&\mathrm{Spec}A[T_1,\cdots,T_n]\ar[r]&\mathrm{Spec}A}$$
		
		我们来归纳的证明这些态射都在相应的点平坦.首先最右端是平坦的因为$A[T_1,\cdots,T_n]$是自由$A$模.下面假设$Z=\mathrm{Spec}\frac{A[T_1,\cdots,T_n]}{(f_1,\cdots,f_i)}$,假设$Z'$是$Z$的被$V(f_{i+1})$定义的闭子概型.设$z'\in Z'$映到$z\in Z$,使得$z$在$Z\to\mathrm{Spec}A$下是平坦的,再设$\mathrm{d}f_{i+1}$在$\kappa(y)$里不为零.我们要证明的是$Z'\to Z\to\mathrm{Spec}A$在点$z'$平坦.还可以不妨设$A$是诺特环,因为$A$总可以写成它的有限生成$\mathbb{Z}$代数子环$A_0$的正向极限,一旦我们证明了$\mathrm{Spec}\frac{A_0[T_1,\cdots,T_n]}{(f_1,\cdots,f_i)}\to\mathrm{Spec}\frac{A_0[T_1,\cdots,T_n]}{(f_1,\cdots,f_i,f_{i+1})}$在点$z'\cap(A_0[T_1,\cdots,T_n]/(f_1,\cdots,f_i))$是平坦的(这里我们在选取$A_0$的时候当然要包含$f_1,\cdots,f_{i+1}$的全部有限个系数),那么它基变换到$Z'\to Z$就也是平坦的.
		
		\qquad
		
		按照局部平坦性准则,如果$(R,\mathfrak{m},k)$是诺特局部环,那么$M$是平坦的等价于$\mathrm{Tor}_1^R(M,k)=0$.于是归结为证明$\mathrm{Tor}_1^{\mathscr{O}_{Y,y}}(\mathscr{O}_{Z',z'},\kappa(y))=0$.有如下$\mathscr{O}_{Z,z}$模的正合列:
		$$\xymatrix{\mathscr{O}_{Z,z}\ar[r]^{f_{i+1}}&\mathscr{O}_{Z,z}\ar[r]&\mathscr{O}_{Z',z'}\ar[r]&0}$$
		
		如果把这个正合列张量$\kappa(z)$,那么$\mathscr{O}_{Z,z}\otimes_{\mathscr{O}_{Y,y}}\kappa(y)$是一个整环,按照条件有$f_{i+1}\times1\not=0$,于是得到如下短正合列,这迫使$\mathrm{Tor}_1^{\mathscr{O}_{Y,y}}(\mathscr{O}_{X,x},\kappa(y))=0$.
		$$\xymatrix{0\ar[r]&\mathscr{O}_{Z,z}\otimes_{\mathscr{O}_{Y,y}}\kappa(y)\ar[r]^{f_{i+1}\otimes1}&\mathscr{O}_{Z,z}\otimes_{\mathscr{O}_{Y,y}}\kappa(y)\ar[r]&\mathscr{O}_{Z',z'}\otimes_{\mathscr{O}_{Y,y}}\kappa(y)\ar[r]&0}$$
		
		\qquad
		
		再设局部有限表示态射满足(a)和(b),问题是局部的,不妨设$Y=\mathrm{Spec}A$和$X=\mathrm{Spec}A[T_1,\cdots,T_n]/(f_1,\cdots,f_r)$.那么$f$的纤维态射是$X_y=\mathrm{Spec}K[T_1,\cdots,T_n]/(f_1,\cdots,f_r)\to\mathrm{Spec}K$,其中$K=\kappa(y)$.但是这个态射在点$x$相对光滑维数$d$导致它的雅各比矩阵的秩是$n-d$,于是$r\ge n-d$,不妨设$f_1,\cdots,f_{n-d}$的微分是线性无关的.记$B=A[T_1,\cdots,T_n]/(f_1,\cdots,f_{n-d})$,记$I=(f_1,\cdots,f_r)/(f_1,\cdots,f_{n-d})$,那么$A[T_1,\cdots,T_n]/(f_1,\cdots,f_r)=B/I$.考虑短正合列$0\to I\to B\to B/I\to0$.因为$(B/I)_{\mathfrak{p}_y}$是平坦$A$模,于是对这个短正合列张量$\otimes_A\kappa(y)$仍然是短正合列.但是$B\otimes_A\kappa(y)=(B/I)\otimes_A\kappa(y)$,就导致$J\otimes_A\kappa(y)=0$.但是$J$是有限$B$模,按照NAK引理得到$J_{\mathfrak{m}_x}=0$,此即闭嵌入$\mathrm{Spec}B/I\to\mathrm{Spec}B$在点$x$的某个开邻域上是恒等,于是从$\mathrm{Spec}B$在点$x$光滑得到$\mathrm{Spec}B/I$在点$x$光滑.
	\end{proof}
	\item 态射$f:X\to Y$在点$x$平展当且仅当在点$x$是相对光滑维数零的.于是平展态射吻合于相对光滑维数零的态射.
	\begin{proof}
		
		设$f:X\to Y$是局部有限表示态射,设$x\in X$和$y=f(x)$,我们解释过$f$在点$x$非分歧当且仅当纤维态射$X_y\to\mathrm{Spec}\kappa(y)$是非分歧的.还解释过这等价于讲$X_y=\mathrm{Spec}\prod_{1\le i\le n}L_i$,其中每个$L_i$都是$k=\kappa(y)$的有限可分扩张.可记$L_i=k[X]/(p(X))$,其中$p$是$k$上的一个不可约可分多项式,于是$0\not=p'(X)\in L_i$,这说明如果$X_y\to\mathrm{Spec}k$是相对光滑维数零的,于是上一条说明平展态射是相对光滑维数零的态射.反过来如果$f$是相对光滑维数零的,上一条已经证明了$f$是平坦的,另外从$(\Omega_{X/Y})_x=0$得到$f$是非分歧的,这说明$f$是平展态射.
	\end{proof}
	\item 一个光滑态射$f:X\to Y$是平展态射当且仅当它是局部拟有限的.这是因为局部拟有限态射的纤维是离散空间,所以局部维数是零,就导致$f$的相对光滑维数是零.
	\item 光滑,平展,非分歧的无穷小提升性质(infinitesimal lifting property).设$f:X\to S$是局部有限表示态射,那么如下命题互相等价:
	\begin{enumerate}
		\item $f$是非分歧的/光滑的/平展的.
		\item 对任意$S$概形$\mathrm{Spec}A$,设理想$I$满足$I^2=0$,那么典范映射:
		$$\mathrm{Hom}_S(\mathrm{Spec}A,X)\to\mathrm{Hom}_S(\mathrm{Spec}A/I,X)$$
		
		是单射/满射/双射.
	\end{enumerate}
	\begin{proof}
		
		先证明非分歧的等价命题.首先(b)可以归结为$X$和$S$都是仿射的情况,因为(b)中的典范同态是单射并不涉及到粘合问题(满射才涉及到).并且$\mathrm{Spec}A/I$在主开集上是$\mathrm{Spec}A_f/IA_f$,依旧满足$(IA_f)^2=0$.于是我们不妨设$S=\mathrm{Spec}R$,设$X=\mathrm{Spec}B$.那么条件(b)等价于讲如下实线交换图表,至多存在一个提升同态$B\to A/I$,此即$R$代数$B$是0-非分歧的,这等价于讲$\Omega_{B/R}=0$,此即$f$是非分歧态射.
		$$\xymatrix{B\ar[rr]\ar@{-->}[drr]&&A/I\\R\ar[u]\ar[rr]&&A\ar[u]}$$
		
		再证明光滑的等价命题.我们先处理仿射情况,先设$f$是光滑态射,那么可设$S=\mathrm{Spec}R$和$X=\mathrm{Spec}B$,其中$B=C/J$,$B=R[T_1,\cdots,T_n]$,$J=(f_1,\cdots,f_{n-d})$满足雅各比矩阵是行满秩的.那么有如下分裂短正合列:
		$$\xymatrix{0\ar[r]&J/J^2\ar[r]&\Omega_{C/R}\otimes_CB\ar[r]&\Omega_{B/R}\ar[r]&0}$$
		
		现在考虑如下实线交换图表,按照多项式环的泛性质存在$\psi$使得图表交换.那么$\psi(J)\subseteq I$和$\psi(J^2)=0$,于是$\psi$诱导了$\psi':J/J^2\to I$,按照上述短正合列分裂,可以把$\psi'$提升为$\psi'':\Omega_{C/R}\otimes_CB\to I$,这样的$B$模同态对应了$A\to I$的导数,记作$\delta$,那么$\delta$和$\psi$在$J$上是相同的映射,据此可验证$\psi-\delta:C\to A$就是一个提升,这得到了条件(b).
		$$\xymatrix{B\ar[rr]^u\ar@{-->}[ddrr]&&A/I\\C\ar[u]\ar@{-->}[drr]^{\psi}&&\\R\ar[u]\ar[rr]&&A\ar[uu]}$$
		
		反过来设$f$是有限表示态射,满足条件(b),设$S=\mathrm{Spec}R$,设$B=C/J$,其中$C=R[T_1,\cdots,T_n]$,$J=(f_1,\cdots,f_r)$.条件(b)即$B$在$R$上0-光滑,这说明有如下短正合列.这对$B$的每个剩余域做张量积仍然是短正合列,于是$f$是光滑的.
		$$\xymatrix{0\ar[r]&J/J^2\ar[r]&\Omega_{C/R}\otimes_CB\ar[r]&\Omega_{B/R}\ar[r]&0}$$
		
		下面处理一般情况.因为(b)推(a)是局部的问题,我们只要证明(a)推(b)的情况.我们已经看到仿射局部上提升总是存在的,问题在于这些局部上的提升能不能粘合.一般的,如果有阿贝尔层的短正合列$0\to\mathscr{F}'\to\mathscr{F}\to\mathscr{F}''$,取$t\in\mathscr{F}''(X)$,假设存在开覆盖$\mathscr{U}=\{U_i,i\in I\}$,使得存在$s_i\in\mathscr{F}(U_i)$,打到$\mathscr{F}''$中是$t\mid_{U_i}$.那么使得这些$\{s_i\}$可以粘合为$t$的一个整体提升的条件是$\check{\mathrm{H}}^1(\mathscr{U},X)=0$.在我们这里任取态射$\varphi:\mathrm{Spec}A/I\to X$,就有$\mathrm{Spec}A/I$上模层的左正合列:
		$$\xymatrix{0\ar[r]&\mathrm{HOM}_{A/I}(\varphi^*\Omega_{X/S},\widetilde{I})\ar[r]&\mathrm{HOM}_{A/I}(\varphi^*\Omega_{X/S},\mathrm{Spec}A)\ar[r]&\mathrm{HOM}_{A/I}(\varphi^*\Omega_{X/S},\mathrm{Spec}A/I)}$$
		
		由于$\mathscr{F}'=\mathrm{HOM}_{A/I}(\varphi^*\Omega_{X/S},\widetilde{I})$是$\mathrm{Spec}A/I$上的伴随模层,就有$\check{\mathrm{H}}^1(\mathrm{Spec}A/I,\mathscr{F}')=0$,于是这些局部提升是可以粘合的.
	\end{proof}
	\item 推论.设$X,Y$是$S$概形.
	\begin{enumerate}
		\item 如果$X\to S$是非分歧的,那么$X\to Y$是非分歧的.
		\item 如果$Y\to S$是非分歧的,如果$X\to S$是光滑的/平展的,那么$X\to Y$是光滑的/平展的.
	\end{enumerate}
	$$\xymatrix{X\ar[rr]\ar[dr]&&Y\ar[dl]\\&S&}$$
	\begin{proof}
		
		(a)是因为有如下交换图表的上一个方格.(b)只需验证光滑的情况,任取$c\in\mathrm{Hom}_Y(\mathrm{Spec}A/I,X)$,那么它也可以视为$\mathrm{Hom}_S(\mathrm{Spec}A/I,X)$中的元,按照$X\to S$是光滑的,就存在$b\in\mathrm{Hom}_S(\mathrm{Spec}A,X)$使得$b$复合上$\mathrm{Spec}A/I\to\mathrm{Spec}A$就是$c$.为了说明$b$在$\mathrm{Hom}_Y(\mathrm{Spec}A,X)$中,我们只需验证$b$是一个$Y$态射,也即$b$和$X\to Y$的复合是$\mathrm{Spec}A\to Y$.但是我们要证明相同的这两个态射都满足复合上$\mathrm{Spec}A/I\to\mathrm{Spec}A$是$c$在$\mathrm{Hom}_S(\mathrm{Spec}A/I,Y)$中的像,结合$Y\to S$是非分歧的,迫使这两个态射相同,这就得证.
		$$\xymatrix{\mathrm{Hom}_Y(\mathrm{Spec}A,X)\ar[rr]\ar@{^{(}->}[d]&&\mathrm{Hom}_Y(\mathrm{Spec}A/I,X)\ar@{^{(}->}[d]\\\mathrm{Hom}_S(\mathrm{Spec}A,X)\ar[rr]\ar[d]&&\mathrm{Hom}_S(\mathrm{Spec}A/I,X)\ar[d]\\\mathrm{Hom}_S(\mathrm{Spec}A,Y)\ar[rr]^{\mathrm{inj}}&&\mathrm{Hom}_S(\mathrm{Spec}A/I,Y)}$$
	\end{proof}
	\item 设$f:X\to Y$是$S$概形之间的态射,设$x\in X$和$y=f(x)\in Y$在结构态射下都是光滑点,设$f$在点$x$也光滑,那么有如下自由模$\mathscr{O}_{X,x}$的分裂短正合列:
	$$\xymatrix{0\ar[r]&(f^*\Omega_{Y/S})_y\ar[r]&\Omega_{X/S,x}\ar[r]&\Omega_{X/Y,x}\ar[r]&0}$$
	
	如果$X,Y$作为$S$概形的结构态射都是光滑的,并且$S$概形态射$f:X\to Y$也是光滑的,那么有如下$\mathscr{O}_X$模层的短正合列,并且它在stalk上是分裂的短正合列(事实上,如果只约定$f$是光滑的,也有这个结论).
	$$\xymatrix{0\ar[r]&f^*\Omega_{Y/S}\ar[r]&\Omega_{X/S}\ar[r]&\Omega_{X/Y}\ar[r]&0}$$
	\begin{proof}
		
		我们总有$\mathscr{O}_X$模层的正合列:
		$$\xymatrix{f^*\Omega_{Y/S}\ar[r]^{\alpha}&\Omega_{X/S}\ar[r]&\Omega_{X/Y}\ar[r]&0}$$
		
		在光滑点$x$处的stalk都是$\mathscr{O}_{X,x}$自由模,秩就是相对光滑维数,所以按照光滑性是保复合的,说明$\ker\alpha$是零维的,于是这个正合列可以补为短正合列.
	\end{proof}
	\item 雅各比准则.设$Z$是$S$概形,$j:X\to Z$是闭嵌入并且是局部有限表示的,设对应的理想层为$\mathscr{I}$,设$x\in X$,$z=j(x)$,设结构态射$Z\to S$在点$z\in Z$是相对光滑维数$n$的,那么如下命题互相等价:
	\begin{enumerate}
		\item 结构态射$X\to S$在点$x$的相对光滑维数是$r$.
		\item 有如下$\mathscr{O}_{X,x}$模的分裂短正合列,并且有$r=\mathrm{rank}(\Omega_{X/S}\otimes\kappa(x))$
		$$\xymatrix{0\ar[r]&(\mathscr{I}/\mathscr{I}^2)_x\ar[r]&(j^*\Omega_{Z/S})_x\ar[r]&\Omega_{X/S,x}\ar[r]&0}$$
		\item 设$\mathrm{d}z_1,\cdots,\mathrm{d}z_n$是$\Omega_{Z/S,z}$的一组基,其中$z_i\in\mathscr{O}_{Z,z}$.设$g_1,\cdots,g_N\in\mathscr{O}_{Z,z}$生成了理想$\mathscr{I}_z$(这用到了$X$是有限表示的闭子概型),那么存在$z_1,\cdots,z_n$和$g_1,\cdots,g_N$的重排,使得$g_{r+1},\cdots,g_n$生成了$\mathscr{I}_z$,并且$\mathrm{d}z_1,\cdots,\mathrm{d}z_r,\mathrm{d}g_{r+1},\cdots,\mathrm{d}g_n$生成了$\Omega_{Z/S,z}$.
		\item 存在$g_1,\cdots,g_{n-r}\in\mathscr{O}_{Z,z}$生成了$\mathscr{I}_z$,满足$\mathrm{d}g_{r+1},\cdots,\mathrm{d}g_n(z)$在$\Omega_{Z/S,z}\otimes\kappa(z)$里线性无关.
	\end{enumerate}
	\item 设$f:X\to Y$是$S$概形态射,设$x\in X$和$y=f(x)\in Y$,设$x$是结构态射$X\to S$的光滑点,$y$是结构态射$Y\to S$的光滑点.那么如下命题互相等价:
	\begin{enumerate}
		\item $f$在点$x$光滑.
		\item 典范同态$(f^*\Omega_{Y/S})_x\to\Omega_{X/S,x}$存在左同态逆,换句话讲如下短正合列分裂:
		$$\xymatrix{0\ar[r]&(f^*\Omega_{Y/S})_y\ar[r]&\Omega_{X/S,x}\ar[r]&\Omega_{X/Y,x}\ar[r]&0}$$
		\item 典范同态$(f^*\Omega_{Y/S})_x\otimes\kappa(x)\to\Omega_{X/S,x}\otimes\kappa(x)$是单射.
	\end{enumerate}
	\begin{proof}
		
		(a)推(b)推(c)是容易的.下面证明(c)推(a):先证明$Y=\mathbb{A}_S^s$的情况.此时$f$就是$\mathscr{O}_X$的$s$个整体截面$\overline{f_1},\cdots,\overline{f_s}$(积的泛性质),那么条件(c)就是在讲$\mathrm{d}\overline{f_1}(x),\cdots,\mathrm{d}\overline{f_s}(x)$在$\kappa(x)$上线性无关.另一方面因为$X\to S$在点$x$光滑,可设$X$是某个$\mathbb{A}_S^m$的子概型,存在$\mathbb{A}_S^m$的截面$h_{r+1},\cdots,h_m$使得$\mathrm{d}h_{r+1}(x),\cdots,\mathrm{d}h_m(x)$在$\kappa(x)$上线性无关.
		
		\qquad
		
		下面考虑图像态射的嵌入$X\to X\times_SY\to\mathbb{A}_S^m\times_S\mathbb{A}_S^s$.于是在点$(x,f(x))$附近可以把$X$视为$\mathbb{A}_S^{m+s}=\mathbb{A}_Y^m$的子概型.【】
	\end{proof}
	\item 推论.在上一条中如果$f$在点$x$是平展的,那么$\Omega_{X/Y,x}=0$,于是(b)变成典范同态$(f^*\Omega_{Y/S})_x\to\Omega_{X/S,x}$是同构.
	\item 设$f:X\to S$是态射,设$x\in X$和$s=f(x)$,那么$f$在点$x$是相对光滑维数$n$的当且仅当存在$x$的开邻域$U$,使得$f\mid_U$可以分解为$\xymatrix{U\ar[r]^g&\mathbb{A}_s^n\ar[r]^p&S}$,其中$g$是平展态射,$p$是仿射空间的结构态射.
	$$\xymatrix{U\ar[rr]^g\ar[drr]_{f\mid_U}&&\mathbb{A}_S^n\ar[d]^p\\&&S}$$
	\begin{proof}
		
		充分性就是光滑态射的复合.下面证明必要性,选取$g_1,\cdots,g_n\in\mathscr{O}_{X,x}$使得$\mathrm{d}g_1,\cdots,\mathrm{d}g_n$构成$\Omega_{X/S,x}$的一组基.于是由$g_1,\cdots,g_n$定义的态射$g:U\to Y=\mathbb{A}_S^n$,其中$U$是$x$的一个合适的开邻域,就使得$(f^*\Omega_{Y/S})_x\to\Omega_{X/S,x}$是双射,于是$g$在点$x$平展,并且$f\mid_U$要分解为$p\circ g$.
	\end{proof}
	\item 整理一下,设$x\in X$是$f:X\to S$的相对光滑维数$n$的点.那么$g_1,\cdots,g_n\in\mathscr{O}_{X,x}$满足$\mathrm{d}g_1,\cdots,\mathrm{d}g_n$构成$\Omega_{X/S,x}$的一组基当且仅当$\mathrm{d}g_1(x),\cdots,\mathrm{d}g_n(x)$构成$\kappa(x)$线性空间$\Omega_{X/S,x}\otimes\kappa(x)$的一组基,当且仅当$g_1,\cdots,g_n$定义了从$x$的某个开邻域$U$到$\mathbb{A}_S^n$的平展态射.满足这个条件的$g_1,\cdots,g_n$称为$x$处的局部坐标系统.
\end{enumerate}

\newpage
\section{形式概形}
\subsection{仿射形式概形}

\begin{enumerate}
	\item 形式谱.设$A$是可容拓扑环,取定义理想$J$,那么$\mathrm{Spec}A/J$恰好是由$A$的开素理想构成的闭子集,于是这个集合不依赖于定义理想$J$的选取,它定义为$A$的形式谱的底空间,记作$\mathfrak{X}$;选取$0\in A$的由定义理想构成的开邻域基$\{J_i\}$,那么$\mathrm{Spec}A/J_i$都具有相同的底空间$\mathfrak{X}$,由于这个空间存在拟紧开集构成的拓扑基,于是每个$\mathscr{O}_i=\mathscr{O}_{\mathrm{Spec}A/J_i}$可以改进为伪离散拓扑环层(这是指它的每个截面都不变,但是拓扑改进,不过拟紧开集上的拓扑总是离散拓扑),改进后的拓扑环层仍然记作$\mathscr{O}_i$.如果$J_i\subseteq J_j$,就有典范同态$A/J_i\to A/J_j$,它对应了层态射$\mathscr{O}_i\to\mathscr{O}_j$,于是$\{\mathscr{O}_i\}$是$\mathfrak{X}$上的拓扑环层的逆向系统.它的极限记作$\mathscr{O}_{\mathfrak{X}}$,这是一个拓扑环层.我们称拓扑环空间$(\mathfrak{X},\mathscr{O}_{\mathfrak{X}})$为$A$对应的形式谱,记作$\mathrm{Spf}A$.一个仿射形式概形就是指同构于某个形式谱的拓扑环空间.
	\item 按照定义,对$\mathfrak{X}$的任意开子集$U$有$\mathscr{O}_{\mathfrak{X}}(U)=\varprojlim\Gamma(U,\mathscr{O}_{\mathrm{Spec}A/J_i})$;如果$U\subseteq\mathfrak{X}$是拟紧开集,那么上述同构还是拓扑环同构(伪离散层具有这样的性质),其中右侧逆向极限取的是离散拓扑环的逆向极限.特别的,这说明$\Gamma(\mathfrak{X},\mathscr{O}_{\mathfrak{X}})$拓扑同构于$A$.
	\item 主开集.设$A$是可容拓扑环,设$\mathfrak{X}=\mathrm{Spf}A$.对$f\in A$,记$\mathfrak{D}(f)=D(f)\cap\mathfrak{X}$.那么拓扑环空间$(\mathfrak{D}(f),\mathscr{O}_{\mathfrak{X}}\mid_{\mathfrak{D}(f)})$同构于仿射形式概形$\mathrm{Spf}A_{\{f\}}$.
	\begin{proof}
		
		首先$A_{\{f\}}$的开素理想恰好一一对应于$A$的和$f$不交的开素理想,于是作为拓扑空间$\mathrm{Spf}A_{\{f\}}$典范等同于$\mathfrak{D}(f)$.接下来任取包含在$\mathfrak{D}(f)$中的拟紧开集$U$,那么有典范拓扑环同构:
		$$\Gamma(U,\mathscr{O}_{\mathfrak{X}})=\varprojlim\Gamma(U,\mathscr{O}_{\mathrm{Spec}A/J_i})\cong\varprojlim\Gamma(U,\mathscr{O}_{\mathrm{Spec}S_f^{-1}A/S_f^{-1}J_i})=\Gamma(U,\mathscr{O}_{\mathrm{Spec}A_{\{f\}}})$$
	\end{proof}
    \item 茎.形式谱$\mathfrak{X}=\mathrm{Spf}A$在忽略环层上的拓扑结构时依旧可以定义茎,即对任意$x\in\mathfrak{X}$,有$\mathscr{O}_{\mathfrak{X},x}=\varinjlim A_{\{f\}}=A_{\{S_x\}}$,这里$f$跑遍$A$的乘性子集$S_x=A-\mathfrak{p}_x$.那么$\mathscr{O}_{\mathfrak{X},x}$总是局部环(这一整条的结论可见可容环和进制环的性质),并且剩余域同构于$\kappa(x)=A_{\mathfrak{p}_x}/\mathfrak{p}_xA_{\mathfrak{p}_x}$.如果$A$是诺特进制环,那么$\mathscr{O}_{\mathfrak{X},x}$总是诺特环.
    \item 同态诱导的形式谱态射.设$\varphi:B\to A$是可容环之间的连续同态,那么$f:\mathrm{Spec}A\to\mathrm{Spec}B$把开素理想映为开素理想,对任意$g\in B$,有$\varphi$诱导了连续同态$B_{\{g\}}\to A_{\{\varphi(g)\}}$,也即$\Gamma(\mathfrak{D}(g),\mathscr{O}_{\mathrm{Spf}B})\to\Gamma(\mathfrak{D}(\varphi(g)),\mathscr{O}_{\mathrm{Spf}A})$,这些同态和限制映射可交换,于是它诱导了一个拓扑环层的$f$态射$f^{\#}:\mathscr{O}_{\mathrm{Spf}B}\to\mathscr{O}_{\mathrm{Spf}A}$.进而$(f,f^{\#}):\mathrm{Spf}A\to\mathrm{Spf}B$就是拓扑环层空间之间的态射.我们定义形式谱之间的态射是被可容环之间连续同态按如上方式诱导的态射.于是按照这个定义,形式谱范畴和可容环范畴是逆变范畴等价的.
    \item 形式谱态射诱导的茎同态.设$\varphi:B\to A$是可容环之间的连续同态,它对应的态射记作$f:\mathfrak{X}=\mathrm{Spf}A\to\mathrm{Spf}B=\mathfrak{Y}$,对任意$x\in\mathrm{Spf}A$,有$f^{\#}$在忽略拓扑结构后定义了一个同态$f^{\#}_x:\mathscr{O}_{\mathfrak{Y},f(x)}\to\mathscr{O}_{\mathfrak{X},x}$.这总是一个局部环同态,事实上我们还能证明下一条更强的结论.
    \item 设$A,B$是可容环,设$\mathfrak{X}=\mathrm{Spf}A$和$\mathfrak{Y}=\mathrm{Spf}B$,设$(\psi,\theta):\mathfrak{X}\to\mathfrak{Y}$是拓扑环层空间之间的态射,那么要想它是被某个连续同态$B\to A$所诱导的,等价于要求对任意$x\in\mathfrak{X}$有$\theta_x:\mathscr{O}_{\mathfrak{Y},\psi(x)}\to\mathscr{O}_{\mathfrak{X},x}$是局部环同态.
    \begin{proof}
    	
    	必要性.设$\varphi:B\to A$诱导了该态射,设$\mathfrak{p}=\mathfrak{p}_x\in\mathrm{Spf}A$,设$\psi(x)=y$,设$\mathfrak{q}=\mathfrak{q}_y=\varphi^{-1}(\mathfrak{p})$,那么$\theta_x$是如下同态的复合,这些同态都是局部环之间的局部同态:
    	$$\xymatrix{\mathscr{O}_{\mathfrak{Y},y}=\varinjlim_{g\not\in\mathfrak{q}}B_{\{g\}}\ar[r]&A_{\{S\}}=\varinjlim_{g\not\in\mathfrak{q}}A_{\{\varphi(g)\}}\ar[r]&\mathscr{O}_{\mathfrak{X},x}=\varinjlim_{f\not\in\mathfrak{p}}A_{\{f\}}}$$
    	
    	充分性.设$(\psi,\theta)$诱导的茎同态都是局部环同态.取$\theta$的整体截面,得到一个拓扑环之间的连续同态$\varphi=\Gamma(\theta):B\to A$.按照茎同态是局部的,整体截面$\varphi(g)$在$x$处可逆当且仅当$g$在$\psi(x)$处可逆,换句话讲如果$x$对应的素理想是$\mathfrak{p}$,记$\mathfrak{q}=\varphi^{-1}(\mathfrak{p})$,那么$\psi(x)$对应的素理想就是$\mathfrak{q}$.这就说明$\varphi$诱导的态射在拓扑上和$\psi$一致.最后由于$\varphi$在主开集$\mathfrak{D}(g)$上和$\theta$是一致的,于是$\varphi$诱导的态射在层上和$\theta$一致.
    \end{proof}
\end{enumerate}
\subsection{形式概形}

设$\mathfrak{X}$是拓扑环空间.
\begin{itemize}
	\item 一个开子集$U$称为仿射形式开子集/进制仿射形式开子集/诺特仿射形式开子集,如果它的开子拓扑环空间结构是一个可容环/进制环/诺特进制环对应的形式谱.
	\item $\mathfrak{X}$称为形式概形/进制形式概形/局部诺特形式概形,如果它每个点都存在仿射形式开子集/进制仿射形式开子集/诺特仿射形式开子集.
	\item $\mathfrak{X}$称为诺特形式概形,如果它是拟紧的局部诺特形式概形.
	\item 形式概形之间的态射定义为它们作为拓扑环空间之间的态射,并且满足茎同态总是局部环同态.形式概形构成的范畴记作$\textbf{FSch}$.
\end{itemize}
\begin{enumerate}
	\item 如果$\mathfrak{X}$是形式概形/局部诺特形式概形,那么它的仿射形式开子集/诺特仿射形式开子集构成拓扑基(一个诺特进制环$A$,任取$f\in A$,那么$A_{\{f\}}$也是诺特进制环).
	\item 形式概形/局部诺特形式概形的开子拓扑环空间总是形式概形/局部诺特形式概形,并且此时典范嵌入是形式概形态射.
	\item 设$\mathfrak{X}$是形式概形,设$\mathfrak{Y}=\mathrm{Spf}A$是仿射形式概形,那么存在如下双射:
	$$\mathrm{Hom}_{\textbf{FSch}}(\mathfrak{X},\mathfrak{Y})\cong\mathrm{Hom}_{\textbf{TopRings}}(A,\Gamma(\mathfrak{X},\mathscr{O}_{\mathfrak{X}}))$$
	\item 对任意环$A$,赋予离散拓扑时它自动是一个可容环,并且此时它的素谱和形式谱是一致的,记作$\mathfrak{X}$,对$\mathfrak{X}$的开集$U$,它作为素谱和形式谱的截面作为集合是相同的,如果$U$是拟紧开集那么作为形式谱的截面环是离散拓扑的,对非拟紧开集截面环上的拓扑可能未必是离散的.于是一个仿射概形我们总可以添加截面环上拓扑实现为一个形式谱.于是概形总是形式概形,并且两个概形之间的作为形式概形的态射和它们作为概形的态射是一致的.
\end{enumerate}
\subsection{仿射情况的定义理想层}

设$A$是可容环,记$\mathfrak{X}=\mathrm{Spf}(A)$.
\begin{enumerate}
	\item 取$A$的一个开理想$J$,记$\mathfrak{X}=\mathrm{Spf}(A)$,记$\{J_i\}$是全部包含在$J$中的定义理想构成的逆向系统.那么$\widetilde{J/J_i}$是$\widetilde{A/J_i}$上的理想层,并且当$i$变动时$\{\widetilde{J/J_i}\}$构成逆向系统,它的极限是$\mathscr{O}_{\mathfrak{X}}$的理想层,记作$J^{\Delta}$.
	\item 按照定义,任取$f\in A$,那么$\Gamma(\mathfrak{D}(f),J^{\Delta})$就是$\{S_f^{-1}J/S_f^{-1}J_i\}$的逆向极限,也即$A_{\{f\}}$的开理想$J_{\{f\}}$.特别的有$\Gamma(\mathfrak{X},J^{\Delta})=J$.进而有$J^{\Delta}\mid_{\mathfrak{D}(f)}=(J_{\{f\}})^{\Delta}$.特别的,这说明$\mathfrak{X}$上的定义理想层(见下文)总能限制为$\mathfrak{D}(f)$上的定义理想层
	\item 典范同构$\mathscr{O}_{\mathfrak{X}}/J^{\Delta}\cong(\widetilde{A/J})\mid_{\mathfrak{X}}$.任取$f\in A$,那么典范同态$A_{\{f\}}=\varprojlim S_f^{-1}A/S_f^{-1}J_i\to S_f^{-1}A/S_f^{-1}J$是连续满射,这也是$\Gamma(\mathfrak{D}(f),\mathscr{O}_{\mathfrak{X}})\to\Gamma(\mathfrak{D}(f),(\widetilde{A/J})\mid_{\mathfrak{X}})$的连续同态,它的核是$\Gamma(\mathfrak{D}(f),J^{\Delta})=J_{\{f\}}$.这个典范同态定义了一个层态射$\mathscr{O}_{\mathfrak{X}}\to\widetilde{A/J}\mid_{\mathfrak{X}}$,后者是离散环层,这个层态射也就是典范连续满射$A\to A/J$诱导的仿射形式概形之间的态射$\mathrm{Spec}(A/J)\to\mathfrak{X}$的层态射.另外这个层态射的核就是$J^{\Delta}$,进而有典范同构$\mathscr{O}_{\mathfrak{X}}/J^{\Delta}\cong(\widetilde{A/J})\mid_{\mathfrak{X}}$.另外这件事还说明$J\mapsto J^{\Delta}$是保序的单射,因为如果$J\subseteq J'$,那么${J'}^{\Delta}/J^{\Delta}$典范同构于$\widetilde{J'/J}$,这非平凡.
	\item 定义理想层.$\mathscr{O}_{\mathfrak{X}}$的一个理想层$\mathscr{I}$称为定义理想层,如果它存在形如$\mathfrak{D}(f)$的开覆盖,使得$\mathscr{I}\mid_{\mathfrak{D}(f)}$具有形式$J^{\Delta}$,其中$J$是$A_{\{f\}}$的一个定义理想.事实上仿射情况下定义理想层一定可以唯一的表示为$J^{\Delta}$,其中$J$是$A$的定义理想.
	\begin{proof}
		
		任取$\mathfrak{X}$的定义理想层$\mathscr{I}$,因为仿射形式概形的拓扑是拟紧的,按照定义就可以选取有限个元$f_i\in A$使得这些$\mathfrak{D}(f_i)$覆盖了整个$\mathfrak{X}$,并且有$\mathscr{I}\mid_{\mathfrak{D}(f_i)}=h_i^{\Delta}$,这里$h_i$是$A_{\{f_i\}}$的一个定义理想.于是可以选取$A$的开理想$R_i$使得$(R_i)_{\{f_i\}}=h_i$.再选取$A$的定义理想$R$包含在每个$R_i$中.那么$\mathscr{I}/R^{\Delta}$在典范同构$\mathscr{O}_{\mathfrak{X}}/R^{\Delta}\cong(\widetilde{A/R})\mid_{\mathfrak{X}}$下的像就满足限制在$\mathfrak{D}(f_i)$上是$\widetilde{R_i/R}$,于是这个像是$\mathrm{Spec}A/R$上的拟凝聚层,进而它就具有形式$\widetilde{J/R}$,于是$\mathscr{I}=J^{\Delta}$.最后只要证明$J$是$A$的定义理想:我们知道固定一个定义理想$\mathfrak{a}$,那么一个理想$\mathfrak{b}$包含在某个定义理想中当且仅当$\mathfrak{b}$的某个次幂落在$\mathfrak{a}$中.这里每个$R_i$自身都是定义理想,于是存在足够大的正整数$n$使得$R_i^n\subseteq R$,进而有$(\mathscr{I}/R^{\Delta})^n=0$,于是$(J/R)^n=0$,于是$J^n\subseteq R$,再结合$J$本身是开理想就得到$J$是定义理想.
	\end{proof}
    \item 设$A$是进制环,设$J$是定义理想,设$J/J^2$是有限$A/J$模,那么对任意正整数$n$都有$(J^{\Delta})^n=(J^n)^{\Delta}$.
    \begin{proof}
    	
    	因为条件下有$(J\{S^{-1}\})^n=J^n\{S^{-1}\}$.
    \end{proof}
    \item 称$\mathfrak{X}=\mathrm{Spf}A$的一族定义理想层$\{\mathscr{I}_i\}$构成基本系统,如果对应的定义理想$\{J_i\}$构成0元的开邻域基,这也等价于讲对$\mathfrak{X}$的任意定义理想$\mathscr{I}$,都包含了某个$\mathscr{I}_i$.我们断言如果定义理想层构成的一族滤过系统$\{\mathscr{I}_i\}$满足,存在一族$f_j\in A$使得$\mathfrak{D}(f_j)$覆盖整个$\mathfrak{X}$,并且$\{\mathscr{I}_i\mid\mathscr{D}(f_j)\}$是$\mathfrak{D}(f_j)$的基本系统,那么$\{\mathscr{I}_i\}$就是$\mathfrak{X}$的基本系统.这件事可以说明形式概形上定义的基本系统和这里仿射概形上的基本系统是一致的.
    \begin{proof}
    	
    	任取定义理想层$\mathscr{I}$,那么$\mathscr{I}\mid_{\mathscr{D}(f_j)}$也是$\mathscr{D}(f_j)$上的定义理想,于是按照基本系统的定义,存在和$j$有关的指标$k_j$,使得$\mathscr{I}_{k_j}\mid_{\mathfrak{D}(f_j)}$包含于$\mathscr{I}\mid_{\mathfrak{D}(f_j)}$.按照$\mathfrak{X}$是拟紧的,可以让$j$跑遍有限个指标使得$\mathfrak{D}(f_j)$覆盖了整个$\mathfrak{X}$.按照滤过条件可以选取足够大的指标$t$使得$\mathscr{I}_t\subseteq\mathscr{I}_{j}$对$j$的有限个选取成立,此时$\mathscr{I}_t\subseteq\mathscr{I}$.
    \end{proof}
\end{enumerate}
\subsection{一般情况的定义理想层}

设$\mathfrak{X}$是形式概形,它的一个$\mathscr{O}_{\mathfrak{X}}$理想层$\mathscr{I}$称为定义理想层,如果存在形式仿射开覆盖$\mathscr{U}$,使得对任意$U\in\mathscr{U}$都有$\mathscr{I}\mid_U$是仿射形式概形$U$上的定义理想层.
\begin{enumerate}
	\item 任取开子集$V\subseteq\mathfrak{X}$,任取$\mathfrak{X}$上的定义理想层$\mathscr{I}$,有$\mathscr{I}\mid_U$是形式概形$U$上的定义理想层.
	\item 基本系统.称$\mathfrak{X}$的一族定义理想层$\{\mathscr{I}_i\}$构成基本系统,如果存在$\mathfrak{X}$的仿射形式开覆盖$\{U_{\alpha}\}$,使得对任意指标$\alpha$,都有$\{\mathscr{I}\mid_{U_{\alpha}}\}$是仿射形式概形$U_{\alpha}$上的基本系统.
	\begin{enumerate}[(1)]
		\item 之前已经验证了形式概形上的基本系统和仿射形式概形上的基本系统的定义是一致的.
		\item 任取开子集$V\subseteq\mathfrak{X}$,那么$\{\mathscr{I}_i\mid_V\}$也是$V$上的基本系统.
		\item 如果$\mathfrak{X}$是局部诺特形式概形(这是指它局部上是诺特进制环的形式谱),设$\mathscr{I}$是一个定义理想层,那么$\{\mathscr{I}^n\}$就构成基本系统.
	\end{enumerate}
	\item 设$\mathfrak{X}$是形式概形,设$\mathscr{I}$是一个定义理想,那么$(\mathfrak{X},\mathscr{O}_{\mathfrak{X}}/\mathscr{I})$是概形.并且如果$\mathfrak{X}$是仿射形式概形/局部诺特形式概形/诺特形式概形时,后者相应的是仿射概形/局部诺特概形/诺特概形.另外如果取典范商态射$\theta:\mathscr{O}_{\mathfrak{X}}\to\mathscr{O}_{\mathfrak{X}}/\mathscr{I}$,那么$(1_{\mathfrak{X}},\theta):(\mathfrak{X},\mathscr{O}_{\mathfrak{X}}/\mathscr{I})\to(\mathfrak{X},\mathscr{O}_{\mathfrak{X}})$是形式概形之间的态射,这约定为典范的.
	\item 设$\mathfrak{X}$是形式概形,设$\{\mathscr{I}_i\}$是基本系统,那么拓扑环层$\mathscr{O}_{\mathfrak{X}}$就是伪离散环层$\mathscr{O}_{\mathfrak{X}}/\mathscr{I}_i$的正向极限.
	\begin{proof}
		
		$\mathfrak{X}$上的拓扑有拟紧仿射形式开子集构成的拓扑基,我们知道此时伪离散环层拓扑上在拟紧开集上的截面是离散的,所以只要在仿射形式开子集上不考虑拓扑的验证这件事就可以了.仿射情况这件事是平凡的.
	\end{proof}
	\item 形式概形上的定义理想层是未必存在的,但是局部诺特形式概形上总存在:设$\mathfrak{X}$是局部诺特形式概形,那么存在$\mathfrak{X}$的最大的定义理想层$\mathscr{J}$,记使得概形$(\mathfrak{X},\mathscr{O}_{\mathfrak{X}}/\mathfrak{I})$既约的唯一的定义理想层为$\mathscr{I}$,那么$\mathscr{J}$也就是$\mathscr{O}_{\mathfrak{X}}/\mathfrak{I}$的幂零根在典范态射$\mathscr{O}_{\mathfrak{X}}\to\mathscr{O}_{\mathfrak{X}}/\mathfrak{I}$下的原像层.把既约概形$(\mathfrak{X},\mathscr{O}_{\mathfrak{X}}/\mathscr{J})$记作$\mathfrak{X}_{\mathrm{red}}$.
	\begin{proof}
		
		先考虑仿射情况,记$\mathfrak{X}=\mathrm{Spf}(A)$,其中$A$是诺特进制环,此时它的定义理想层恰好具有形式$J^{\Delta}$,其中$J$是$A$的定义理想,此时$J$是最大定义理想当且仅当$A/J$是既约的.
		
		\qquad
		
		对于一般的形式概形,归结为证明如果$V\subseteq U$是$X$的两个诺特形式仿射开子集,那么$U$上的最大定义理想层$\mathscr{J}_U$限制在$V$上是$V$上的最大定义理想层$\mathscr{J}_V$,而这件事是因为$V$上的概形结构层$\mathscr{O}_{\mathfrak{X}}\mid_V/\mathscr{J}_U\mid_V$仍然是既约的.
	\end{proof}
    \item 设$\mathfrak{X}$是局部诺特形式概形,设$\mathscr{J}$是最大定义理想层,那么对任意开集$V\subseteq\mathfrak{X}$,有$\mathscr{J}\mid_V$也是局部诺特形式概形$V$的最大定义理想层.
    \item 设$\mathfrak{X}$和$\mathfrak{Y}$是形式概形,设$\mathscr{I}$和$\mathscr{K}$分别是它们的定义理想层,设$f:\mathfrak{X}\to\mathfrak{Y}$是形式概形态射.
    \begin{enumerate}[(1)]
    	\item 如果$f^*(\mathscr{K})\mathscr{O}_{\mathfrak{X}}\subseteq\mathscr{I}$,那么存在唯一的态射$f':(\mathfrak{X},\mathscr{O}_{\mathfrak{X}}/\mathscr{I})\to(\mathfrak{Y},\mathscr{O}_{\mathfrak{Y}}/\mathscr{K})$使得如下图表交换,其中垂直态射是之前定义的典范态射:
    	$$\xymatrix{(\mathfrak{X},\mathscr{O}_{\mathfrak{X}})\ar[rr]^f&&(\mathfrak{Y},\mathscr{O}_{\mathfrak{Y}})\\(\mathfrak{X},\mathscr{O}_{\mathfrak{X}}/\mathscr{I})\ar[rr]^{f'}\ar[u]&&(\mathfrak{Y},\mathscr{O}_{\mathfrak{Y}}/\mathscr{K})\ar[u]}$$
    	\item 设$\mathfrak{X}=\mathrm{Spf}(A)$和$\mathfrak{Y}=\mathrm{Spf}(B)$是仿射的,记$\mathscr{I}=I^{\Delta}$和$\mathfrak{K}=K^{\Delta}$,其中$I$和$K$分别是$A$和$B$的定义理想,再记$f$被连续同态$\varphi:B\to A$所诱导.那么条件$f^*(\mathscr{K})\mathscr{O}_{\mathfrak{X}}\subseteq\mathscr{I}$等价于$\varphi(K)\subseteq I$.并且在这个条件成立时,这里的$f'$恰好是$\varphi$诱导的$\varphi':B/K\to A/I$所对应的态射.
    \end{enumerate}
\end{enumerate}
\subsection{形式概形作为概形的正向极限}
\begin{enumerate}
	\item 设$\mathfrak{X}$是形式概形,设$\{\mathscr{I}_i\}$是基本系统,记典范的形式概形态射$f_i:X_i=(\mathfrak{X},\mathscr{O}_{\mathfrak{X}}/\mathscr{I}_i)\to(\mathfrak{X},\mathscr{O}_{\mathfrak{X}})$.如果$\mathscr{I}_i\subseteq\mathscr{I}_j$,那么典范态射$\mathscr{O}_{\mathfrak{X}}/\mathscr{I}_i\to\mathscr{O}_{\mathfrak{X}}/\mathscr{I}_j$就诱导了典范概形态射$f_{ij}:(\mathfrak{X},\mathscr{O}_{\mathfrak{X}}/\mathscr{I}_j)\to(\mathfrak{X},\mathscr{O}_{\mathfrak{X}}/\mathscr{I}_i)$.并且明显满足$f_j=f_i\circ f_{ij}$.我们断言$\{\mathfrak{X},f_i\}$恰好就是$\textbf{FSch}$中的正向系统$\{X_i,f_{ij}\}$的极限.
	\begin{proof}
		
		验证泛性质.设$\mathfrak{Y}$是形式概形,对任意指标$i$存在形式概形态射$g_i=(\psi_i,\theta_i):X_i\to\mathfrak{Y}$,使得只要$\mathscr{I}_i\subseteq\mathscr{I}_j$就有$g_j=g_i\circ f_{ij}$.按照$X_i$的底空间都是一致的,于是$\psi_i$都是相同的连续映射$\psi:\mathfrak{X}\to\mathfrak{Y}$.至于层态射的部分,$\theta_i^{\#}:\psi^*\mathscr{O}_{\mathfrak{Y}}\to\mathscr{O}_{X_i}=\mathscr{O}_{\mathfrak{X}}/\mathscr{I}_i$构成逆向系统,进而诱导了层态射$\omega:\psi^*\mathscr{O}_Y\to\mathscr{O}_{\mathfrak{X}}$.明显的$g=(\psi,\omega)$是满足对任意$i$有$g_i=g\circ f_i$的唯一环空间态射,最后只要验证$g$是形式概形态射.
		
		\qquad
		
		问题是局部的,于是不妨设$\mathfrak{X}=\mathrm{Spf}(A)$和$\mathfrak{Y}=\mathrm{Spf}(B)$都是仿射的,其中$A$和$B$都是可容环.那么可记定义理想层$\mathscr{I}_i=I_i^{\Delta}$.此时$g$就是被$B\to A/I_i$诱导的唯一的$B\to\varprojlim A/I_i=A$所对应的形式概形态射.
	\end{proof}
	\item 设$\mathfrak{X}$是拓扑空间,设$(\mathscr{O}_i,u_{ji})$是其上环层的逆向系统,指标集是自然数集$\mathbb{N}$,设$u_{0i}:\mathscr{O}_i\to\mathscr{O}_0$的核为$\mathscr{I}_i$,满足:
	\begin{enumerate}[(a)]
		\item 环空间$X_i=(\mathfrak{X},\mathscr{O}_i)$是概形.
		\item 对任意$x\in\mathfrak{X}$和任意指标$i$,存在$x$的开邻域$U_i$使得$\mathscr{I}_i\mid_{U_i}$是幂零的.
		\item 态射$u_{ji}$总是满态射.
	\end{enumerate}
	
	赋予$\mathscr{O}_i$伪离散拓扑环层结构,它在拓扑环层中的逆向极限记作$\mathscr{O}_{\mathfrak{X}}$和$u_i:\mathscr{O}_{\mathfrak{X}}\to\mathscr{O}_i$.那么$(\mathfrak{X},\mathscr{O}_{\mathfrak{X}})$是形式概形,态射$u_i$都是满态射,并且它们的核$\mathscr{I}^{(i)}$构成$\mathfrak{X}$的基本系统,并且$\mathscr{I}^{(0)}$就是$\mathscr{I}_i$的逆向极限.
	\begin{proof}
		
		首先由于$u_{ji}$都是满态射,于是它诱导的茎同态是局部环之间的满态射,从而自动是局部同态,于是$v_{ij}=(1_{\mathfrak{X}},u_{ji}):X_j\to X_i$都是概形态射,它们构成了概形的正向系统.
		
		\qquad
		
		先设$X_i=\mathrm{Spec}A_i$都是仿射概形,记$u_{ji}$被$\varphi_{ji}:A_i\to A_j$诱导.由于这里是仿射的,有$\mathscr{O}_j$是拟凝聚$\mathscr{O}_i$模层.任取$f\in A_i$,记$\varphi_{ji}(f)=f'$,那么$D(f)$和$D(f')$是$\mathfrak{X}$上一致的开子集,并且$(\varphi_{ji})_f:(A_i)_f\to(A_j)_{f'}$恰好对应于$u_{ji}\mid_{D(f)}$.于是$u_{ji}=\widetilde{\varphi_{ji}}$,这里的$\varphi_{ji}$视为$A_i$模同态,于是从$u_{ji}$是满态射得到$\varphi_{ji}$是满同态,它的核记作$J_{ji}$,那么$u_{ji}$的核就是$\widetilde{J_{ji}}$.特别的,记$\varphi_{0i}:A_i\to A_0$的核为$J_i$,那么$\mathscr{I}_i=\widetilde{J_i}$.
		
		\qquad
		
		我们断言此时(b)导致这里$\mathscr{I}_i$都是幂零的,进而有$A=\varprojlim A_i$是可容环,记典范同态$\varphi_i:A\to A_i$,这是满射,它的核$J^{(i)}$恰好就是$J_{ik},k\ge i$的逆向极限,那么$J^{(i)}$构成$A$的基本系统,此时$\mathscr{O}_{\mathfrak{X}}$就是$\mathrm{Spf}(A)$的结构层,满足命题中的结论.下面证明断言:固定$i$,按照$\mathfrak{X}$是拟紧的,可以取有限开覆盖$\{U_k\}$,使得存在自然书$n_k$使得$(\mathscr{I}_i\mid_{U_k})^{n_k}=0$,记$n=\max\{n_k\}$,那么有$\mathscr{I}_i^n=0$.
		
		\qquad
		
		另外此时如果$f=(f_i)$是$A=\varprojlim A_i$中的元,那么这些$D(f_i)$都是相同的,并且就是$\mathfrak{D}(f)$.
		
		\qquad
		
		对于$X_i$是一般概形的情况,如果$U\subseteq\mathfrak{X}$是拟紧开集,那么依旧可以证明$\mathscr{I}_i\mid_U$是幂零的.问题归结为如下观察:对任意$x\in\mathfrak{X}$,存在它的开邻域$U$使得$X_i\mid_U$都是仿射的:取$U$使得$X_0\mid_U$是仿射的.从$\mathscr{O}_{X_0}=\mathscr{O}_{X_i}/\mathscr{I}_i$就得到$U\in X_i$是仿射的(Chevalley仿射准则).	
	\end{proof}
    \item 推论.条件同上一条,如果对$i\ge j$有$\ker u_{ji}=\mathscr{I}_i^{j+1}$,并且$\mathscr{I}_1/\mathscr{I}_1^2$是$\mathscr{O}_0=\mathscr{O}_1/\mathscr{I}_1$上有限模层,那么$\mathfrak{X}$局部上都是进制环的形式谱.如果额外的还有$X_0$是局部诺特/诺特的,那么$\mathfrak{X}$是局部诺特形式概形/诺特形式概形.特别的,这件事说明局部诺特形式概形一定是由局部诺特概形构成的正向系统的极限,并且可以要求满足本条和上一条的内容,为此只要取局部诺特形式概形$\mathfrak{X}$的定义理想层$\mathscr{I}$,再取$X_n=(\mathfrak{X},\mathscr{O}_{\mathfrak{X}}/\mathscr{I}^{n+1})$.
    \item 推论.设$A$是可容环,那么$\mathfrak{X}=\mathrm{Spf}(A)$是诺特形式概形,当且仅当$A$是诺特进制环.
    \begin{proof}
    	
    	充分性平凡.必要性设$\mathfrak{X}$是诺特形式概形,取$A$的定义理想$J$,记$\mathscr{I}=J^{\Delta}$,取概形$X_n=(\mathfrak{X},\mathscr{O}_{\mathfrak{X}}/\mathscr{I}^{n+1})$,它是环$A_n=A/J^{n+1}$的素谱.此时它们满足使得$A$是诺特进制环的条件(见进制环的内容)
    \end{proof}
    \item 设$\mathfrak{X}$和$\mathfrak{Y}$是两个局部诺特形式概形,设$\mathscr{I}$和$\mathscr{K}$分别是它们的定义理想层,给定形式概形态射$f:\mathfrak{X}\to\mathfrak{Y}$满足$f^*(\mathscr{K})\mathscr{O}_{\mathfrak{X}}\subseteq\mathscr{I}$,进而对任意正整数$n$有$f^*(\mathscr{K}^n)\mathscr{O}_{\mathfrak{X}}\subseteq\mathscr{I}^n$,我们解释过这个条件诱导了唯一的概形态射$f_n:X_n=(\mathfrak{X},\mathscr{O}_{\mathfrak{X}}/\mathscr{I}^{n+1})\to Y_n=(\mathfrak{Y},\mathscr{O}_{\mathfrak{Y}}/\mathscr{K}^{n+1})$,并且对任意$m\le n$都有如下交换图表:
    $$\xymatrix{X_m\ar[rr]^{f_m}\ar[d]&&Y_m\ar[d]\\X_n\ar[rr]^{f_n}&&Y_n}$$
    
    于是$\{f_n\}$是正向系统之间的态射$\{X_n\}\to\{Y_n\}$.反过来如果给定这样的正向系统之间的态射$\{f_n\}$,那么它的极限是形式概形之间的态射$f:\mathfrak{X}\to\mathfrak{Y}$.我们断言上述对应是从全体满足$f^*(\mathscr{K})\mathscr{O}_{\mathfrak{X}}\subseteq\mathscr{I}$的形式态射$f$到全体满足上述交换图表对任意$m\le n$成立的正向系统之间的态射$\{f_n\}$之间的一一对应.
    \begin{proof}
    	
    	只需证明如果$\{f_n\}$是正向系统之间的态射,它的极限记作$f$,那么$f$满足$f^*(\mathscr{K})\mathscr{O}_{\mathfrak{X}}\subseteq\mathscr{I}$.问题关于源端和终端都是局部的,不妨设$\mathfrak{X}=\mathrm{Spf}A$和$\mathfrak{Y}=\mathrm{Spf}B$,这里$A$和$B$都是诺特进制环,记$\mathscr{I}=I^{\Delta}$和$\mathscr{K}=J^{\Delta}$,这里$I$和$J$分别是$A$和$B$的定义理想.那么$X_n=\mathrm{Spec}A_n$和$Y_n=\mathrm{Spec}B_n$,其中$A_n=A/I^{n+1}$和$B_n=B/J^{n+1}$.设$f_n$被环同态$\varphi_n:B_n\to A_n$诱导,那么有$\varphi_n(J/J^{n+1})\subseteq I/I^{n+1}$,取极限得到$\varphi(J)\subseteq I$,也即$f^*(\mathscr{K})\mathscr{O}_{\mathfrak{X}}\subseteq\mathscr{I}$.
    \end{proof}
    \item 推理.设$\mathfrak{X}$和$\mathfrak{Y}$是两个局部诺特形式概形,设$\mathscr{I}$是$\mathfrak{X}$的最大定义理想层.
    \begin{enumerate}[(1)]
    	\item 对$\mathfrak{Y}$的任意定义理想层$\mathscr{K}$,和任意形式态射$f:\mathfrak{X}\to\mathfrak{Y}$,都有$f^*(\mathscr{K})\mathscr{O}_{\mathfrak{X}}\subseteq\mathscr{I}$.
    	\item 于是$\mathrm{Hom}_{\textbf{FSch}}(\mathfrak{X},\mathfrak{Y})$和$\mathrm{Hom}(\{X_n\},\{Y_n\})$是一一对应的,这里$X_n=(\mathfrak{X},\mathscr{O}_{\mathfrak{X}}/\mathscr{I}^{n+1})$和$Y_n=(\mathfrak{Y},\mathscr{O}_{\mathfrak{Y}}/\mathscr{K}^{n+1})$.
    \end{enumerate}
    \begin{proof}
    	
    	只需证明(1),问题依旧是关于源端和终端是局部的,不妨设$\mathfrak{X}=\mathrm{Spf}A$和$\mathfrak{Y}=\mathrm{Spf}B$,这里$A$和$B$都是诺特进制环,记$\mathscr{I}=I^{\Delta}$和$\mathscr{K}=J^{\Delta}$,这里$I$是$A$的最大定义理想,$J$是$B$的定义理想.设$f$被连续同态$\varphi:B\to A$诱导.我们解释过$J$中的元素都是拓扑幂零元(即$x^n$的极限为零),进而$\varphi(J)$中的元也是拓扑幂零元,于是自然有$\varphi(J)\subseteq I$,此即$f^*(\mathscr{K})\mathscr{O}_{\mathfrak{X}}\subseteq\mathscr{I}$.
    \end{proof}
    \item 上一条的$S$对象版本.设$\mathfrak{Z}$是局部诺特形式概形,设$\mathfrak{X}$和$\mathfrak{Y}$是两个局部诺特形式$\mathfrak{Z}$概形,结构态射分别记作$f:\mathfrak{X}\to\mathfrak{Z}$和$g:\mathfrak{Y}\to\mathfrak{Z}$.设$\mathscr{I}$,$\mathscr{J}$,$\mathscr{L}$分别是$\mathfrak{X}$,$\mathfrak{Y}$,$\mathfrak{Z}$的定义理想层,满足$f^*(\mathscr{L})\mathscr{O}_{\mathfrak{X}}\subseteq\mathscr{I}$和$f^*(\mathscr{L})\mathscr{O}_{\mathfrak{Y}}\subseteq\mathscr{K}$.记$Z_n=(\mathfrak{Z},\mathscr{O}_{\mathfrak{Z}}/\mathscr{L}^{n+1})$,$X_n=(\mathfrak{X},\mathscr{O}_{\mathfrak{X}}/\mathscr{I}^{n+1})$和$Y_n=(\mathfrak{Y},\mathscr{O}_{\mathfrak{Y}}/\mathscr{K}^{n+1})$.那么存在从$\mathrm{Hom}_{\mathfrak{Z}}(\mathfrak{X},\mathfrak{Y})$到$\mathrm{Hom}_{\{Z_n\}}(\{X_n\},\{Y_n\})$之间的一一对应.
\end{enumerate}
\subsection{形式概形的纤维积}

设$\mathfrak{Z}$是形式概形,其上全体形式概形构成的范畴记作$\textbf{FSch}(\mathfrak{Z})$,这个范畴的积对象定义为形式概形的纤维积.
\begin{enumerate}
	\item 仿射情况.设$\mathfrak{X}=\mathrm{Spf}B$和$\mathfrak{Y}=\mathrm{Spf}C$是仿射形式概形$\mathfrak{Z}=\mathrm{Spf}A$上的两个仿射形式概形.它们的纤维积就是完备张量积的形式谱$\mathfrak{W}=\mathrm{Spf}\widehat{B\otimes_AC}$.回顾一下$\widehat{B\otimes_AC}$的定义是环$B\otimes_AC$以$\{\mathrm{Im}(I\otimes_AC)+\mathrm{Im}(B\otimes_AJ)\}$为零元开邻域基(此即张量积拓扑)的完备化.
	\begin{proof}
		
		此归结为完备张量积的泛性质.
	\end{proof}
    \item 形式概形的纤维积总是存在的.
    \item 局部诺特形式概形上的局部诺特形式概形的纤维积未必是局部诺特的.
    \item 纤维积和正向极限.
    \begin{enumerate}[(1)]
    	\item 设$f:\mathfrak{X}\to\mathfrak{Z}$和$g:\mathfrak{Y}\to\mathfrak{Z}$是两个$\mathfrak{Z}$形式概形.设$\{\mathscr{I}_i\}$,$\{\mathscr{J}_i\}$和$\{\mathscr{K}_i\}$分别是$\mathfrak{Z}$,$\mathfrak{X}$和$\mathfrak{Y}$的具有相同指标集的定义理想层.设对每个指标$i$都有$f^*\mathscr{I}_i\mathscr{O}_{\mathfrak{X}}\subseteq\mathscr{J}_i$和$g^*\mathscr{I}_i\mathscr{O}_{\mathfrak{Y}}\subseteq\mathscr{K}_i$.再记$Z_i=(\mathfrak{Z},\mathscr{O}_{\mathfrak{Z}}/\mathscr{I}_i)$,$X_i=(\mathfrak{X},\mathscr{O}_{\mathfrak{X}}/\mathscr{J}_i)$和$Y_i=(\mathfrak{Y},\mathscr{O}_{\mathfrak{Y}}/\mathscr{K}_i)$.如果指标$(i,j)$满足$\mathscr{I}_j\subseteq\mathscr{I}_i$,$\mathscr{J}_j\subseteq\mathscr{J}_i$和$\mathscr{K}_j\subseteq\mathscr{K}_i$,那么$X_j$是$X_i$的具有相同底空间的闭子概型,类似的对$Y$和$Z$都成立,进而有$X_i\times_{Z_i}Y_i$是$X_j\times_{Z_j}Y_j$的具有相同底空间的闭子概形.于是有$\mathfrak{X}\times_{\mathfrak{Z}}\mathfrak{Y}$是$\{X_i\times_{Z_i}Y_i\}$的正向极限.
    	\item 进一步,设$\mathfrak{W}$是$\mathfrak{Z}$形式概形,设$\{\mathscr{L}_i\}$是定义理想层,指标集和其它形式概形的定义理想层一致.设$u:\mathfrak{W}\to\mathfrak{X}$和$v:\mathfrak{W}\to\mathfrak{Y}$是两个$\mathfrak{Z}$形式态射,并且对任意指标$i$都满足$(u^*\mathscr{J}_i)\mathscr{O}_{\mathfrak{W}}\subseteq\mathscr{L}_i$和$(v^*\mathscr{K}_i)\mathscr{O}_{\mathfrak{W}}\subseteq\mathscr{L}_i$.记$W_i=(\mathfrak{W},\mathscr{O}_{\mathfrak{W}}/\mathscr{L}_i)$,设$u,v$诱导的唯一的态射为$u_i:W_i\to X_i$和$v_i:W_i\to Y_i$.那么$(u,v)$诱导的形式态射$\mathfrak{W}\to\mathfrak{X}\times_{\mathfrak{Z}}\mathfrak{Y}$就是$(u_i,v_i)$诱导的概形态射$\mathfrak{W}_i\to\mathfrak{X}_i\times_{\mathfrak{Z}_i}\mathfrak{Y}_i$的正向极限.
    	\item 特别的,如果考虑局部诺特形式概形,那么上两条中的前提条件总是成立的.
    \end{enumerate}
\end{enumerate}
\subsection{概形沿闭子概型的形式完备化}
\begin{enumerate}
	\item 设$X$是局部诺特概形,设$X'\subseteq X$是非空闭子集,记全体使得$\mathscr{O}_X/\mathscr{I}$的支集为$X'$的凝聚理想层$\mathscr{I}$构成的集合为$S$,赋予包含序.我们断言这是一个有向集,并且如果$X$是诺特概形,那么对任意$\mathscr{I}_0\in S$,都有$\{\mathscr{I}_0^n\}$是$S$的共尾子集.
	\begin{proof}
		
		设$\mathscr{I}_1,\mathscr{I}_2\in S$,那么$\mathscr{I}=\mathscr{I}_1\cap\mathscr{I}_2$也是凝聚理想层,因为它可以写成两个凝聚层之间态射的核.另外对$x\in X$有$\mathscr{I}_x=(\mathscr{I}_1)_x\cap(\mathscr{I}_2)_x$,于是$\mathscr{I}$的支集也是$X'$,于是$\mathscr{I}\in S$.
		
		\qquad
		
		如果$X$是诺特概形,设$\mathscr{I}_0,\mathscr{I}\in S$,那么【EGAI.9.3.4】说明存在正整数$n$使得$\mathscr{I}_0^n(\mathscr{O}_X/\mathscr{I})=0$,也即$\mathscr{I}_0^n\subseteq\mathscr{I}$.
	\end{proof}
	\item 凝聚模层沿闭子集的完备化.设$X$是局部诺特概形,设$X'$是闭子集,设$\mathscr{F}$是凝聚$\mathscr{O}_X$模层.称层$\varprojlim_{\mathscr{I}\in S}(\mathscr{F}\otimes_{\mathscr{O}_X}\mathscr{O}_X/\mathscr{I})$在$X'$上的限制(这里$\mathscr{F}\otimes_{\mathscr{O}_X}\mathscr{O}_X/\mathscr{I}$的支集总包含在$X'$中)为$\mathscr{F}$沿$X'$的完备化,记作$\mathscr{F}_{/X'}$.这个层在$X'$上的截面称为$\mathscr{F}$沿$X'$的形式截面.
	\item 限制到开子集和沿闭子集的完备化是可交换的:对任意开子集$U\subseteq X$,有$(\mathscr{F}\mid_U)_{/(U\cap X')}=(\mathscr{F}_{/X'})\mid_{U\cap X'}$.
	\item 拓扑结构.$(\mathscr{O}_X)_{/X'}$是$X$上的环层,并且$\mathscr{F}_{/X'}$是$(\mathscr{O}_X)_{/X'}$模层.由于$X$具有拟紧开集构成的拓扑基,于是可以把$(\mathscr{O}_X)_{/X'}$改进为伪离散环层(此即赋予一个唯一的拓扑环层,使得拟紧开集上的截面环的拓扑取为离散拓扑),类似的$\mathscr{F}_{/X'}$改进为伪离散群层,此时$\mathscr{F}_{/X'}$仍然是$(\mathscr{O}_X)_{/X'}$是$X$上的拓扑环层.
	\item 函子性.设$u:\mathscr{F}\to\mathscr{G}$是$\mathscr{O}_X$模层态射,取$\mathscr{I}\in S$,那么它诱导了模层态射$u_{\mathscr{I}}:\mathscr{F}\otimes_{\mathscr{O}_X}(\mathscr{O}_X/\mathscr{I})\to\mathscr{G}\otimes_{\mathscr{O}_X}(\mathscr{O}_X/\mathscr{I})$,进而$\{u_{\mathscr{I}}\mid\mathscr{I}\in S\}$构成逆向系统,进而取极限并限制在$X'$上得到一个连续$(\mathscr{O}_X)_{/X'}$模层态射$\mathscr{F}_{/X'}\to\mathscr{G}_{/X'}$,这个态射称为$u$沿$X'$的完备化,记作$u_{/X'}$.于是我们构造了一个$\textbf{Coh}(\mathscr{O}_X)\to\textbf{TopMod}((\mathscr{O}_X)_{/X'})$的函子.
	\item 概形沿闭子集的完备化.设$X$是局部诺特概形,设$X'$是闭子集,那么$(\mathscr{O}_X)_{/X'}$的支集就是$X'$,拓扑环空间$(X',(\mathscr{O}_X)_{/X'})$是一个局部诺特形式概形,并且如果$\mathscr{I}\in S$,那么$\mathscr{I}_{/X'}$就是该形式概形的一个定义理想.这个形式概形称为$X$沿$X'$的完备化,记作$X_{/X'}$.平凡的如果$X'=X$,那么$\mathscr{I}=0$已经是$S$的共尾子集,于是$X_{/X'}=X$.
	\begin{proof}
		
		这归结为仿射情况:设$X=\mathrm{Spec}A$是诺特仿射概形,设$\mathscr{I}=\widetilde{I}$,其中$I$是$A$的理想,满足$X'=V(I)$,那么$(X',(\mathscr{O}_X)_{/X'})$典范同构于$\mathrm{Spf}\widehat{A}$,其中$\widehat{A}$是$A$在$I$-adic拓扑下的完备化.
	\end{proof}
	\item 关于开子概型.如果$U\subseteq X$是局部诺特概形的开子概型,那么$U_{/(U\cap X')}$典范等同于$X_{/X'}$在$X'$的开子集$U\cap X'$上诱导的形式开子概形.
	\item 推论.$(X_{/X'})_{\mathrm{red}}$就是$X$的以$X'$为底空间的唯一的既约闭子概型.并且此时$X_{/X'}$是诺特的当且仅当$(X_{/X'})_{\mathrm{red}}$是诺特的.特别的$X$是诺特的可以推出$X_{/X'}$是诺特的.
	\begin{proof}
		
		归结为仿射情况,设$X=\mathrm{Spec}A$是诺特仿射概形,设理想$I$使得$X'=V(I)$,记$A$的$I$进制完备化为$\widehat{A}$,那么$X_{/X'}=\mathrm{Spf}\widehat{A}$.此时$\widehat{A}$的最大定义理想$\mathfrak{a}$就是$A/I$的幂零根在$\widehat{A}\to\widehat{A}/\widehat{I}=A/I$下的原像,进而有$\widehat{A}/\mathfrak{a}\cong(A/I)_{\mathrm{red}}$,这证明了第一句话.
		
		\qquad
		
		如果$(X_{/X'})_{\mathrm{red}}$是诺特概形,也即$(X')_{\mathrm{red}}$是诺特概形,那么$X_n'=\mathrm{Spec}(\mathscr{O}_X/\mathscr{I}^n)$都是诺特概形,这里$\mathscr{I}\in S$.那么$\{\mathscr{O}_X/\mathscr{I}^n\}$和它们之间的典范态射满足形式概形作为概形正向极限的条件,并且保证极限$X_{/X'}$是诺特的(【ref】见形式概形作为概形正向极限那节的推论).
	\end{proof}
	\item 典范态射$\{\mathscr{O}_X\to\mathscr{O}_X/\mathscr{I}\mid\mathscr{I}\in S\}$构成逆向系统,它的极限是环层态射$\theta:\mathscr{O}_X\to\psi_*((\mathscr{O}_X)_{/X'})=\varprojlim_{\mathscr{I}\in S}(\mathscr{O}_X/\mathscr{I})$,其中$\psi:X'\to X$是典范嵌入.记环空间态射$i=(\psi,\theta):X_{/X'}\to X$.
	\item 类似的,设$\mathscr{F}$是$\mathscr{O}_X$凝聚层,典范态射$\mathscr{O}_X\to\mathscr{O}_X/\mathscr{I}$诱导了$\mathscr{O}_X$模层态射$\mathscr{F}\to\mathscr{F}\otimes_{\mathscr{O}_X}(\mathscr{O}_X/\mathscr{I})$,进而取极限得到函子性的态射$\gamma:\mathscr{F}\to\psi_*(\mathscr{F}_{/X'})$.我们断言:
	\begin{enumerate}[(1)]
		\item 函子$\mathscr{F}\mapsto\mathscr{F}_{/X'}$是正合的.
		\item $(\mathscr{O}_X)_{/X'}$模层态射$\gamma^{\#}:i^*\mathscr{F}\to\mathscr{F}_{/X'}$是同构.
	\end{enumerate}
	\begin{proof}
		
		给定凝聚$\mathscr{O}_X$模层的正合列:
		$$\xymatrix{0\ar[r]&\mathscr{F}'\ar[r]&\mathscr{F}\ar[r]&\mathscr{F}''\ar[r]&0}$$
		
		任取仿射开子集$U=\mathrm{Spec}A\subseteq X$,那么$A$是诺特的,记$\mathscr{F}\mid_U=\widetilde{M}$,$\mathscr{F}'\mid_U=\widetilde{M'}$和$\mathscr{F}''\mid_U=\widetilde{M''}$,其中$M,M',M''$是有限$A$模,并且有短正合列:
		$$\xymatrix{0\ar[r]&M'\ar[r]&M\ar[r]&M''\ar[r]&0}$$
		
		任取$\mathscr{I}\in S$,记$\mathscr{I}\mid_U=\widetilde{I}$,其中$I$是$A$的理想,那么有:
		$$\Gamma(U\cap X',\mathscr{F}\otimes_{\mathscr{O}_X}\mathscr{O}_X/\mathscr{I}^n)=M\otimes_A(A/I^n)$$
		
		进而取逆向极限得到:
		$$\Gamma(U\cap X',\mathscr{F}_{/X'})=\varprojlim(M\otimes_A(A/I^n))=\widehat{M}$$
		
		于是按照诺特环上有限模关于固定理想的完备化是正合的,就得到如下短正合列,进而得到要证的正合性:
		$$\xymatrix{0\ar[r]&\Gamma(U\cap X',\mathscr{F}'_{/X'})\ar[r]&\Gamma(U\cap X',\mathscr{F}_{/X'})\ar[r]&\Gamma(U\cap X',\mathscr{F}''_{/X'})\ar[r]&0}$$
		
		下面证明(2).问题是局部的,结合凝聚层定义,不妨设有正合列:
		$$\xymatrix{\mathscr{O}_X^m\ar[r]&\mathscr{O}_X^n\ar[r]&M\ar[r]&0}$$
		
		按照$i^*$是右正合的,$(-)_{/X'}$是正合的,得到交换图表:
		$$\xymatrix{i^*(\mathscr{O}_X^m)\ar[r]\ar[d]_{\gamma^{\#}}&i^*(\mathscr{O}_X^n)\ar[r]\ar[d]_{\gamma^{\#}}&i^*\mathscr{F}\ar[r]\ar[d]_{\gamma^{\#}}&0\\(\mathscr{O}_X^m)_{/X'}\ar[r]&(\mathscr{O}_X^n)_{/X'}\ar[r]&(\mathscr{F})_{/X'}\ar[r]&0}$$
		
		结合这两个函子是加性的(和有限直和可交换),以及短五引理,归结为证明$\mathscr{F}=\mathscr{O}_X$的情况.但是此时有$i^*\mathscr{O}_X=(\mathscr{O}_X)_{/X'}$,由于$\gamma^{\#}$是$(\mathscr{O}_X)_{/X'}$模层态射,于是只需验证$\gamma^{\#}$在任意开集上都把截面的单位元映为单位元,这是平凡的.
	\end{proof}
	\item 典范态射$i:X_{/X'}\to X$是平坦的.
	\begin{proof}
		
		一般的如果$f:X\to Y$是环空间之间的态射,满足$\mathscr{O}_Y$是自身的凝聚层(这对局部诺特概形一定成立),并且$(f^*-\otimes_{\mathscr{O}_X}\mathscr{F})$局部上总是正合的,则$f$是平坦态射.这里按照上一条有$i^*\mathscr{F}\cong\mathscr{F}_{/X'}$是正合的.
	\end{proof}
	\item 张量积和HOM.设$\mathscr{F}$和$\mathscr{G}$是凝聚$\mathscr{O}_X$模层,那么有如下自然同构:
	$$(\mathscr{F}_{/X'})\otimes_{(\mathscr{O}_X)_{/X'}}(\mathscr{G}_{/X'})\cong(F\otimes_{\mathscr{O}_X}\mathscr{G})_{/X'}$$
	$$(\mathrm{HOM}_{\mathscr{O}_X}(\mathscr{F},\mathscr{G}))_{/X'}\cong\mathrm{HOM}_{(\mathscr{O}_X)_{/X'}}(\mathscr{F}_{/X'},\mathscr{G}_{/X'})$$
	\begin{proof}
		
		这两个态射是可以构造的,第一个不需要条件,第二个依赖于$i$是平坦的.
	\end{proof}
	\item $\mathscr{F}\to\mathscr{F}_{/X'}$诱导的$\Gamma(X,\mathscr{F})\to\Gamma(X',\mathscr{F}_{/X'})$的核由那些限制在$X'$的某个非空开邻域上是零截面的整体截面构成.
	\begin{proof}
		
		一方面这样的整体截面映为$\mathscr{F}\otimes_{\mathscr{O}_X}(\mathscr{O}_X/\mathscr{I})$的零整体截面,进而映为$\mathscr{F}_{/X'}$的零整体截面.反过来,设$s\in\Gamma(X,\mathscr{F})$映为$\mathscr{F}_{/X'}$的零整体截面.归结为证明对任意$x\in X'$都存在$x$在$X$中的开邻域使得$s$在其上的限制为零.于是问题归结为设$X=\mathrm{Spec}A$是仿射的,于是这里$A$是诺特环.记$X'=V(\mathfrak{a})$,其中$\mathfrak{a}\subseteq A$是理想.记$\mathscr{F}=\widetilde{M}$,其中$M$是有限$A$模.那么$\Gamma(X',\mathscr{F}_{/X'})$就是$M$的$\mathfrak{a}$-adic完备化$\widehat{M}$.整体截面映射$\Gamma(X,\mathscr{F})\to\Gamma(X',\mathscr{F}_{/X'})$就是典范的完备化映射$M\to\widehat{M}$.它的核由被$1+\mathfrak{a}$某个元零化的$z\in M$构成.设$f\in\mathfrak{a}$使得$(1+f)s=0$.于是对$x\in X'$有$(1_x+f_x)s_x=0$.但是$1_x+f_x$在$A_x$中可逆,于是$s_x=0$.
	\end{proof}
	\item 推论.
	\begin{enumerate}[(1)]
		\item $\mathrm{Supp}\left(\mathscr{F}_{/X'}\right)=\mathrm{Supp}(\mathscr{F})\cap X'$.
		\item 设$u:\mathscr{F}\to\mathscr{G}$是凝聚$\mathscr{O}_X$模层,那么$u_{/X'}$是零态射当且仅当$u$限制在$X'$的某个开邻域上是零态射.
		\item 设$u:\mathscr{F}\to\mathscr{G}$是凝聚$\mathscr{O}_X$模层,那么$u_{/X'}$是单态射/满态射当且仅当$u$限制在$X'$的某个开邻域上是单态射/满态射.
	\end{enumerate}
	\begin{proof}
		
		(1):回拉把有限型模层映为有限型模层,于是$\mathscr{F}_{/X'}$是有限型$(\mathscr{O}_X)_{/X'}$模层.于是它的支集是闭集.它包含在$\mathrm{Supp}(\mathscr{F})\cap X'$中.为了证明它们相同【】
		
		\qquad
		
		(2):我们解释过$u_{/X'}=i^*u$,于是如果把$u$视为模层$\mathscr{H}=\mathrm{HOM}_{\mathscr{O}_X}(\mathscr{F},\mathscr{G})$的截面,就有$u_{/X'}$是如下模层的截面,于是按照上一条得证.
		$$i^*\mathscr{H}=\mathscr{H}_{/X'}\cong\mathrm{HOM}_{(\mathscr{O}_X)_{/X'}}(\mathscr{F}_{/X'},\mathscr{G}_{/X'})$$
		
		\qquad
		
		(3):设$u$的核与余核分别为$\mathscr{N}$和$\mathscr{P}$,于是我们有如下$\mathscr{O}_X$模层的正合列:
		$$\xymatrix{0\ar[r]&\mathscr{N}\ar[r]^v&\mathscr{F}\ar[r]^u&\mathscr{G}\ar[r]^w&\mathscr{P}\ar[r]&0}$$
		
		进而按照$i^*$是正合的,有如下$(\mathscr{O}_X)_{/X'}$模层的正合列:
		$$\xymatrix{0\ar[r]&\mathscr{N}_{/X'}\ar[r]^{v_{/X'}}&\mathscr{F}_{/X'}\ar[r]^{u_{/X'}}&\mathscr{G}_{/X'}\ar[r]^{w_{/X'}}&\mathscr{P}_{/X'}\ar[r]&0}$$
		
		于是$u_{/X'}$是单态射/满态射当且仅当$v_{/X'}=0$或者$w_{/X'}=0$,按照(2)这当且仅当$v$或者$w$在$X'$的某个开邻域上为零.
	\end{proof}
\end{enumerate}
\subsection{态射的完备化}

设$f:X\to Y$是局部诺特概形之间的态射,设$X'\subseteq X$和$Y'\subseteq Y$是闭子集,满足$f(X')\subseteq Y'$.再设$\mathscr{I}$和$\mathscr{K}$是$\mathscr{O}_X$和$\mathscr{O}_Y$的对应于上述闭子集的凝聚理想层,并且满足$f^*(\mathscr{K})\mathscr{O}_X\subseteq\mathscr{I}$.我们之前解释过此时对任意正整数$n$都有$f^*(\mathscr{K}^n)\subseteq\mathscr{I}^n$.如果记$X_n'=(X',\mathscr{O}_X/\mathscr{I}^{n+1})$和$Y_n'=(Y',\mathscr{O}_Y/\mathscr{K}^{n+1})$,那么$f$诱导了$f_n:X_n'\to Y_n'$,并且$\{f_n\}$构成逆向系统,它的极限记作$\widehat{f}:X_{/X'}\to Y_{/Y'}$,称为$f$关于$(X',Y')$的完备化态射.这个态射不依赖于对应于$X',Y'$的满足$f^*(\mathscr{K})\mathscr{O}_X\subseteq\mathscr{I}$的凝聚理想层$\mathscr{I},\mathscr{K}$的选取.
\begin{enumerate}
	\item 仿射情况.设$X=\mathrm{Spec}A$和$Y=\mathrm{Spec}B$,分别取$A,B$的理想$I,J$.记$A,B$分别关于这两个理想的完备化为$\widehat{A},\widehat{B}$.记$f$对应的环同态$\varphi:B\to A$.那么$\widehat{f}$对应的环同态就是$\widehat{\varphi}:\widehat{B}\to\widehat{A}$.而且这的确不依赖于定义了相应闭子集的理想$I,J$的选取.
	\item $\widehat{f}:X_{/X'}\to Y_{/Y'}$作为集合映射恰好就是$f$在$X'\to Y'$上的限制.
	\item 按照定义有如下交换图表:
	$$\xymatrix{\widehat{X}\ar[rr]^{\widehat{f}}\ar[d]_{i_X}&&\widehat{Y}\ar[d]^{i_Y}\\X\ar[rr]^f&&Y}$$
	\item 如果$Z$是第三个概形,闭子集$Z'\subseteq Z$和态射$g:Y\to Z$满足和$f$相同的条件,那么有$\widehat{g\circ f}=\widehat{g}\circ\widehat{f}$.
	\item 设$X,Y$是局部诺特$S$概形,其中$Y$是有限型$S$概形.设$f,g:X\to Y$是$S$态射,满足$f(X')\subseteq Y'$和$g(X')\subseteq Y'$.那么$\widehat{f}=\widehat{g}$当且仅当$f,g$在$X'$的某个开邻域的限制相同.
	\begin{proof}
		
		充分性是直接的.必要性归结为$X=\mathrm{Spec}A,Y=\mathrm{Spec}B,S=\mathrm{Spec}R$仿射的情况.按照$\widehat{f}=\widehat{g}$,已经有$y=f(x)=g(x),\forall x\in X'$.另外$B$是有限型$R$代数.记$f,g$对应的$R$代数同态为$\rho,\sigma:B\to A$,它们满足$\widehat{\rho}=\widehat{\sigma}$.之前解释过按照$\rho(s)$和$\sigma(s)$映射到相同的整体截面$s$,所以$\rho(s)$和$\sigma(s)$在$X'$的某个开邻域上的限制一致.这个开邻域是依赖于$s$的,但是按照$B$是有限型$R$代数,我们可以选取一个统一的$X'$的开邻域$V$,使得对每个$s\in B$都有$\rho(s)$和$\sigma(s)$限制在$V$上一致.取$h\in A$使得$x\in D(h)\subseteq V$,那么$f,g$在$D(h)$上一致.
	\end{proof}
	\item 如果$\mathscr{G}$是凝聚$\mathscr{O}_Y$模层,那么有如下典范$(\mathscr{O}_X)_{/X'}$模层同构:
	$$(f^*\mathscr{G})_{/X'}\cong\widehat{f}^*(\mathscr{G}_{/Y'})$$
	\begin{proof}
		
		此即因为如下交换图表:
		$$\xymatrix{\widehat{X}\ar[rr]^{\widehat{f}}\ar[d]_{i_X}&&\widehat{Y}\ar[d]^{i_Y}\\X\ar[rr]^f&&Y}$$
	\end{proof}
	\item 设$\mathscr{F}$是凝聚$\mathscr{O}_X$模层,$\mathscr{G}$是凝聚$\mathscr{O}_Y$模层.设$u:\mathscr{G}\to\mathscr{F}$是$f$态射,也即一个$\mathscr{O}_X$模层态射$f^*\mathscr{G}\to\mathscr{F}$.它的完备化是一个$(\mathscr{O}_X)_{/X'}$模层态射$(f^*\mathscr{G})_{/X'}\to\mathscr{F}_{/X'}$,按照上一条的同构,此即一个$\widehat{f}$态射$v:\mathscr{G}_{Y'}\to\mathscr{F}_{/X'}$.考虑全体三元组$(X,X',\mathscr{F})$构成的范畴,其中$X'\subseteq X$是闭子集,$\mathscr{F}$是凝聚$\mathscr{O}_X$模层,两个三元组之间的态射定义为二元组$(f,u):(X,X',\mathscr{F})\to(Y,Y',\mathscr{G})$,其中$f:X\to Y$是概形态射,满足$f(X')\subseteq Y'$,而$u:\mathscr{G}\to\mathscr{F}$是一个$f$态射.这个范畴记作$\mathscr{C_1}$.再记带一个模层的局部诺特形式概形范畴为$\mathscr{C}_2$.那么$(X,X',\mathscr{F})\mapsto(X_{/X'},\mathscr{F}_{/X'})$是$\mathscr{C}_1\to\mathscr{C}_2$的han'z
	\item 设$S,X,Y$是局部诺特概形,$g:X\to S$和$h:Y\to S$是态射.取$S',X',Y'$分别是$S,X,Y$的闭子集,满足$g(X')\subseteq S'$和$h(Y')\subseteq S'$.取$Z=X\times_SY$,设这是局部诺特的(局部诺特概形的纤维积一般未必是局部诺特的).取$Z'=p^{-1}(X')\cap q^{-1}(Y')$,其中$p$和$q$是纤维积定义中的投影态射.那么作为$S_{/S'}$概形,有如下纤维积的典范同构,其中右侧是形式概形的纤维积,投影态射就是$\widehat{p}$,$\widehat{q}$.
	$$Z_{/Z'}\cong(X_{/X'})\times_{S_{/S'}}Y_{/Y'}$$
	
	进而如果额外的还有局部诺特$S$概形$T$和$S$态射$u:T\to X$,$v:T\to Y$,有$T$的闭子集$T'$使得$u(T')\subseteq X'$和$v(T')\subseteq Y'$.那么完备化态射$\widehat{(u,v)_S}$典范等同于$(\widehat{u},\widehat{v})_{S_{/S'}}$.
	\begin{proof}
		
		问题是局部的,设$S=\mathrm{Spec}A$,$X=\mathrm{Spec}B$,$Y=\mathrm{Spec}C$.设$g,h$对应的环同态为$\varphi$和$\psi$.设$S',X',Y'$分别对应于理想$\mathfrak{a},\mathfrak{b}$和$\mathfrak{c}$,使得$\varphi(\mathfrak{a})\subseteq\mathfrak{b}$和$\psi(\mathfrak{a})\subseteq\mathfrak{c}$.那么有$Z=\mathrm{Spec}B\otimes_AC$,并且$Z'$对应的理想是$\mathrm{im}(\mathfrak{b}\otimes_AC)+\mathrm{im}(B\otimes_A\mathfrak{c})$.那么$B\otimes_AC$关于这个理想的完备化恰好就是$\widehat{B}\otimes_{\widehat{A}}\widehat{C}$.
	\end{proof}
    \item 推论.设$X,Y$是局部诺特$S$概形,使得$X\times_SY$也是局部诺特的,设$S',X',Y'$分别是$S,X,Y$的闭子集,使得$X',Y'$在$S$中的像都包含在$S'$中.对任意满足$f(X')\subseteq Y'$的$S$态射$f:X\to Y$,都有图像态射$\Gamma_{\widehat{f}}$等同于$\widehat{\Gamma_f}$.
    \item 推论.设$f:X\to Y$是局部诺特概形之间的态射,$Y'\subseteq Y$是闭子集,取$X'=f^{-1}(Y')$,那么$X_{/X'}$典范同构于形式概形的纤维积$X\times_Y(Y_{/Y'})$,也即如下交换图表是纤维积图表:
    $$\xymatrix{X_{/X'}\ar[rr]\ar[d]_{\widehat{f}}&&X\ar[d]^f\\Y_{/Y'}\ar[rr]&&Y}$$
\end{enumerate}
\subsection{仿射形式概形上的凝聚层}

设$A$是诺特进制环,设$\mathfrak{a}$是定义理想,设$X=\mathrm{Spec}A$和$\mathfrak{X}=\mathrm{Spf}A$.那么按照定义,$\mathfrak{X}$等同于概形$X$沿$\mathfrak{X}$底空间的完备化$X_{/\mathfrak{X}}$.凝聚$\mathscr{O}_X$模层具有形式$\widetilde{M}$,其中$M$是有限$A$模.记$M^{\Delta}=(\widetilde{M})_{/\mathfrak{X}}$,这是一个$\mathfrak{X}$凝聚层.如果$u:M\to N$是有限$A$模之间的同态,它诱导的$\widetilde{u}_{/\mathfrak{X}}:M^{\Delta}\to N^{\Delta}$记作$u^{\Delta}$.于是我们定义了函子$\textbf{CohMod}(\mathscr{O}_X)\to\textbf{CohMod}(\mathscr{O}_{\mathfrak{X}})$.
\begin{enumerate}
	\item 
	\begin{enumerate}[(1)]
		\item $M\mapsto M^{\Delta}$是正合函子,并且有$A$模的典范同构$\Gamma(\mathfrak{X},M^{\Delta})\cong M$.
		\item 设$M,N$是有限$A$模,那么有:
		$$(M\otimes_AN)^{\Delta}\cong M^{\Delta}\otimes_{\mathscr{O}_{\mathfrak{X}}}N^{\Delta}$$
		$$(\mathrm{Hom}_A(M,N))^{\Delta}\cong\mathrm{HOM}_{\mathscr{O}_X}(M^{\Delta},N^{\Delta})$$
		\item 映射$u\mapsto u^{\Delta}$是如下自然同构:
		$$\mathrm{Hom}_A(M,N)\cong\mathrm{Hom}_{\mathscr{O}_X}(M^{\Delta},N^{\Delta})$$
	\end{enumerate}
    \begin{proof}
    	
    	(1):正合性是因为$M\mapsto\widetilde{M}$和$\mathscr{F}\mapsto\mathscr{F}_{/X'}$都是正合函子.按照定义$\Gamma(\mathfrak{X},M^{\Delta})$是$\Gamma(X,\widetilde{M})$的$\mathfrak{a}$进制完备化,但是$A$已经是完备的,于是这就是$M$本身.(2):这些典范同构对于函子$M\mapsto\widetilde{M}$和$\mathscr{F}\mapsto\mathscr{F}_{/X'}$都成立.(3):结合(1)和(2)的第二个同构得证.
    \end{proof}
    \item 按照$\Delta$的正合性有短正合列$0\to\mathfrak{a}^{\Delta}\to\mathscr{O}_{\mathfrak{X}}\to\mathscr{O}_{\mathfrak{X}}/\mathfrak{a}^{\Delta}\to0$.于是$\mathfrak{a}^{\Delta}$和我们之前的定义理想层是一致的.
    \item 和仿射概形的情况一样,$\mathscr{O}_{\mathfrak{X}}$作为自身模层是凝聚的.
    \item 记$A_n=A/\mathfrak{a}^{n+1}$,记$X_n=\mathrm{Spec}A_n$,对$m\le n$用$u_{mn}:X_m\to X_n$表示$A_n\to A_m$诱导的态射.我们知道$\{X_n,u_{mn}\}$的正向极限就是仿射形式概形$\mathfrak{X}$.设$\mathscr{F}$是$\mathscr{O}_{\mathfrak{X}}$模层,那么如下命题互相等价.特别的,满足(2)中的逆向系统$\{\mathscr{F}_n\}$同构于$\{\mathscr{F}\otimes_{\mathscr{O}_{\mathfrak{X}}}\mathscr{O}_{X_n}\}$.
    \begin{enumerate}[(1)]
    	\item $\mathscr{F}$是凝聚$\mathscr{O}_{\mathfrak{X}}$模层.
    	\item $\mathscr{F}$是一族$\{\mathscr{F}_n\}$的逆向极限,其中$\mathscr{F}_n$是凝聚$\mathscr{O}_{X_n}$模层,满足$u_{mn}^*\mathscr{F}_n=\mathscr{F}_m$.
    	\item 存在有限$A$模使得$\mathscr{F}\cong M^{\Delta}$.
    \end{enumerate}
    \begin{proof}
    	
    	(2)和(3)的等价性是诺特进制环上模的逆向极限描述.(3)推(1)因为$\mathscr{O}_{\mathfrak{X}}$作为自身模层是凝聚的.最后证明(1)推(2):作为$\mathscr{O}_{\mathfrak{X}}$模层,有$\mathscr{O}_{X_n}=\mathscr{O}_{\mathfrak{X}}/\mathscr{I}^{n+1}=A_n^{\Delta}$是凝聚层,进而凝聚层的张量积$\mathscr{F}_n=\mathscr{F}\otimes_{\mathscr{O}_{\mathfrak{X}}}\mathscr{O}_{X_n}$是$\mathscr{O}_{\mathfrak{X}}$凝聚层,也是$\mathscr{O}_{X_n}$凝聚层.并且有$u_{mn}^*\mathscr{F}_n=\mathscr{F}_m$.验证$\mathscr{F}_n$的逆向极限是$\mathscr{F}$即可.
    \end{proof}
    \item 设$A,B$是诺特进制环,设$\varphi:B\to A$是连续同态,记$\mathfrak{a}$和$\mathfrak{b}$分别是$A$和$B$的定义理想,并且满足$\varphi(\mathfrak{b})\subseteq\mathfrak{a}$.记$X=\mathrm{Spec}A$,$Y=\mathrm{Spec}B$,$\mathfrak{X}=\mathrm{Spf}A$和$\mathfrak{Y}=\mathrm{Spf}B$,记$\varphi$对应的概形态射是$f:X\to Y$,对应的完备化态射是$\widehat{f}:\mathfrak{X}\to\mathfrak{Y}$.我们断言对任意有限$B$模$N$,都有典范的$\mathscr{O}_{\mathfrak{X}}$模层同构:
    $$\widehat{f}^*(N^{\Delta})\cong(N\otimes_BA)^{\Delta}$$
    
    特别的,对$B$的任意理想$\mathfrak{b}$都有$\widehat{f}^*(\mathfrak{b}^{\Delta})\mathscr{O}_{\mathfrak{X}}=(\mathfrak{b}A)^{\Delta}$.
    \begin{proof}
    	\begin{align*}
    		(N\otimes_BA)^{\Delta}&\cong i_X^*(\widetilde{N\otimes_BA})\\&\cong i_X^*f^*(\widetilde{N})\\&\cong\widehat{f}^*i_Y^*\widetilde{N}\\&\cong\widehat{f}^*(N^{\Delta})
    	\end{align*}
    \end{proof}
\end{enumerate}
\subsection{形式概形上的凝聚层}

设$\mathfrak{X}$是局部诺特形式概形.
\begin{enumerate}
	\item $\mathscr{O}_{\mathfrak{X}}$是自身凝聚层,并且任意定义理想层都是凝聚层.
	\item 设$\mathscr{I}$是定义理想层,记局部诺特概形$X_n=(\mathfrak{X},\mathscr{O}_{\mathfrak{X}}/\mathscr{I}^{n+1})$,于是$\mathfrak{X}$是$\{(X_n),(u_{mn}:X_m\to X_n)\}$的正向极限.一个$\mathscr{O}_{\mathfrak{X}}$模层$\mathscr{F}$是凝聚层当且仅当它同构于一族模层$\{\mathscr{F}_n\}$的逆向极限,其中$\mathscr{F}_n$是一个凝聚$\mathscr{O}_{X_n}$模层,并且满足$u_{mn}^*\mathscr{F}_n=\mathscr{F}_m,\forall m\le n$.这个逆向系统同构于$\{u_n^*\mathscr{F}=\mathscr{F}\otimes_{\mathscr{O}_{\mathfrak{X}}}\mathscr{O}_{X_n}\}$.
	\item 特别的,上一条说明,$\mathscr{O}_{\mathfrak{X}}$凝聚层$\mathscr{F}$自然的具备一个拓扑层结构,即伪离散群层$\mathscr{F}_n$的极限.我们解释过此时一个$\mathscr{O}_{\mathfrak{X}}$凝聚层态射是自动连续的.另外如果$\mathscr{H}\subseteq\mathscr{F}$是凝聚层的凝聚子模层,那么对任意开子集$U\subseteq\mathfrak{X}$,都有$\Gamma(U,\mathscr{H})$是$\Gamma(U,\mathscr{F})$的闭子群:按照截面函子$\Gamma$是左正合的,有$\Gamma(U,\mathscr{H})$是$\Gamma(U,\mathscr{F})\to\Gamma(U,\mathscr{F}/\mathscr{H})$的核,按照$\mathscr{F}/\mathscr{H}$也是凝聚层,$\mathscr{F}\to\mathscr{F}/\mathscr{H}$是连续的,以及$\Gamma(U,\mathscr{F}/\mathscr{H})$是可分拓扑群,得到核是闭子集.
	\item 设$\mathscr{F}$和$\mathscr{G}$是凝聚$\mathscr{O}_{\mathfrak{X}}$模层,那么$\theta\mapsto(u_n^*\theta)_n$是自然同构:
	$$\mathrm{Hom}_{\mathscr{O}_{\mathfrak{X}}}(\mathscr{F},\mathscr{G})\cong\varprojlim_n\mathrm{Hom}_{\mathscr{O}_{X_n}}(\mathscr{F}_n,\mathscr{G}_n)$$
	\item 一个$\mathscr{O}_{\mathfrak{X}}$模层态射$\theta:\mathscr{F}\to\mathscr{G}$是满射当且仅当$\theta_0=u_0^*\theta:\mathscr{F}_0\to\mathscr{G}_0$是满射.
	\begin{proof}
		
		问题归结为有限$A$模之间的同态$u:M\to N$是满射当且仅当$M/\mathfrak{a}M\to N/\mathfrak{a}N$是满射.这件事依赖于$A$是可容环.
	\end{proof}
    \item 张量和Hom.设$\mathscr{F}$和$\mathscr{G}$是凝聚$\mathscr{O}_{\mathfrak{X}}$模层,那么有如下拓扑层的典范同构:
    $$\mathscr{F}\otimes_{\mathscr{O}_{\mathfrak{X}}}\mathscr{G}\cong\varprojlim_n\left(\mathscr{F}_n\otimes_{\mathscr{O}_{X_n}}\mathscr{G}_n\right)$$
    $$\mathrm{HOM}_{\mathscr{O}_{\mathfrak{X}}}(\mathscr{F},\mathscr{G})\cong\varprojlim_n\mathrm{HOM}_{\mathscr{O}_{X_n}}(\mathscr{F}_n,\mathscr{G}_n)$$
    \begin{proof}
    	
    	第一个拓扑层同构是因为:
    	$$\mathscr{F}_n\otimes_{\mathscr{O}_{X_n}}\mathscr{G}_n=\left(\mathscr{F}\otimes_{\mathscr{O}_{\mathfrak{X}}}\mathscr{G}\right)\otimes_{\mathscr{O}_{\mathfrak{X}}}\mathscr{O}_{X_n}$$
    	
    	至于第二个同构,我们已经给出了去掉拓扑层结构的时候这在截面环上典范同构.于是只需仿射局部上验证它还是同胚即可.而这件事我们已经证明过了.
    \end{proof}
    \item 按照$\mathrm{HOM}_{\mathscr{O}_{\mathfrak{X}}}(\mathscr{F},\mathscr{G})$是拓扑模层,有$\mathrm{Hom}_{\mathscr{O}_{\mathfrak{X}}}(\mathscr{F},\mathscr{G})$自然的具备一个拓扑结构.如果$\mathfrak{X}$是诺特的,那么$\mathrm{Hom}_{\mathscr{O}_{\mathfrak{X}}}(\mathscr{F},\mathscr{I}^n\mathscr{G})$构成了这个群零元的基本系统;并且全体满射/单射/双射构成了开子集.
    \begin{proof}
    	
    	我们解释过全体满射是典范连续映射$\mathrm{Hom}_{\mathscr{O}_{\mathfrak{X}}}(\mathscr{F},\mathscr{G})\to\mathrm{Hom}_{\mathscr{O}_{X_0}}(\mathscr{F}_0,\mathscr{G}_0)$的原像,但是后者是离散空间(因为全空间是拟紧的),所以满射构成开子集.接下来取$\mathfrak{X}$的有限诺特仿射开覆盖$\{\mathscr{U}_i\}$.那么$\theta\in\mathrm{Hom}_{\mathscr{O}_{\mathfrak{X}}}(\mathscr{F},\mathscr{G})$是单射,当且仅当对任意指标$i$有$\theta$在$\mathrm{Hom}_{\mathscr{O}_{\mathfrak{X}}}(\mathscr{F},\mathscr{G})\to\mathrm{Hom}_{\mathscr{O}_{\mathfrak{X}}\mid_{\mathscr{U}_i}}(\mathscr{F}\mid_{\mathscr{U}_i},\mathscr{G}\mid_{\mathscr{U}_i})$的像都是单射.于是这归结为仿射情况,前文已经证明过了.
    \end{proof}
\end{enumerate}
\subsection{进制态射}

局部诺特形式概形之间的态射$f:\mathfrak{X}\to\mathfrak{Y}$称为进制态射(adic morphism),如果存在$\mathfrak{Y}$的定义理想层$\mathscr{I}$,使得$\mathscr{K}=f^*(\mathscr{I})\mathscr{O}_{\mathfrak{X}}$是$\mathfrak{X}$的定义理想层.此时也称$\mathfrak{X}$是进制形式$\mathfrak{Y}$概形.
\begin{enumerate}
	\item 如果$f:\mathfrak{X}\to\mathfrak{Y}$是进制态射,那么对$\mathfrak{Y}$的任意定义理想层$\mathscr{I}_1$,都有$\mathscr{K}_1=f^*(\mathscr{I}_1)\mathscr{O}_{\mathfrak{X}}$是$\mathfrak{X}$的定义理想层.
	\begin{proof}
		
		问题是局部的,不妨设$\mathfrak{X}$和$\mathfrak{Y}$是诺特仿射的.按照$\mathscr{I}_1$和$\mathscr{I}$都是定义理想层,存在正整数$a,b$满足$\mathscr{I}^a\subseteq\mathscr{I}_1$和$\mathscr{I}_1^b\subseteq\mathscr{I}$.进而有$\mathscr{K}^a\subseteq\mathscr{K}_1$和$\mathscr{K}_1^b\subseteq\mathscr{K}$.其中第一个包含关系说明$\mathscr{K}_1$具有形式$\mathfrak{a}^{\Delta}$,第二个包含关系说明它是定义理想.
	\end{proof}
    \item 如果$f:\mathfrak{X}\to\mathfrak{Z},g:\mathfrak{Y}\to\mathfrak{Z}$都是进制$\mathfrak{Z}$概形,那么任意$\mathfrak{Z}$态射$h:\mathfrak{X}\to\mathfrak{Y}$都是进制的.
    \begin{proof}
    	
    	任取$\mathfrak{Z}$的定义理想层$\mathscr{I}$,有$f^*(\mathscr{I})\mathscr{O}_{\mathfrak{X}}=h^*\left(g^*(\mathscr{I})\mathscr{O}_{\mathfrak{Y}}\right)\mathscr{O}_{\mathfrak{X}}$是$\mathfrak{X}$的定义理想层,并且其中$g^*(\mathscr{I})\mathscr{O}_{\mathfrak{Y}}$是$\mathfrak{Y}$的定义理想层.
    \end{proof}
    \item 设$\mathfrak{Z}$是局部诺特形式概形,设$\mathscr{I}$是一个定义理想,记概形$S_n=(\mathfrak{Z},\mathscr{O}_{\mathfrak{Z}}/\mathscr{I}^{n+1})$.一个进制正向$\{S_n\}$系统指的是一族概形态射$\{f_n:X_n\to S_n\}$,其中$X_n$是局部诺特概形,对任意$m\le n$有如下纤维积图表:
    $$\xymatrix{X_m\ar[rr]\ar[d]_{f_m}&&X_n\ar[d]^{f_n}\\S_m\ar[rr]&&S_n}$$
    
    全体进制正向$\{S_n\}$系统构成范畴,态射约定为一族态射$\{u_n:X_n\to Y_n\}$,其中$u_n$是$S_n$态射,满足对任意$m\le n$有如下纤维积图表:
    $$\xymatrix{X_m\ar[rr]\ar[d]_{u_m}&&X_n\ar[d]^{u_n}\\Y_m\ar[rr]&&Y_n}$$
    
    我们断言把$\mathfrak{Z}$进制形式概形$f:\mathfrak{X}\to\mathfrak{Z}$映为$\{X_n=(\mathfrak{X},\mathscr{O}_{\mathfrak{X}}/\mathscr{K}^{n+1})\}$,其中$\mathscr{K}=f^*(\mathscr{I})\mathscr{O}_{\mathfrak{X}}$,$\mathscr{I}$是一个固定的$\mathfrak{Z}$的定义理想,是从进制$\mathfrak{Z}$形式概形到进制正向$\{S_n\}$系统范畴的范畴等价.
    \begin{proof}
    	
    	范畴等价归结为下一条.这里我们先证明$\{X_n\}$的确是进制正向$\{S_n\}$系统.这件事归结为这样一件事:设$f:X\to Y$是概形态射,设$Y'$是$Y$的闭子概型,被一个拟凝聚理想层$\mathscr{I}$定义.那么$X$的闭子概型$X\times_YY'$被$f^*(\mathscr{I})\mathscr{O}_X$所定义.
    \end{proof}
    \item 设$\{X_n\}$是$\{S_n\}$正向系统,设结构态射$f_n:X_n\to S_n$,满足对任意$m\le n$有如下纤维积图表:
    $$\xymatrix{X_m\ar[rr]\ar[d]_{f_m}&&X_n\ar[d]^{f_n}\\S_m\ar[rr]&&S_n}$$

    满足形式概形描述为概形正向极限【ref】的条件(b)和(c).把$\{X_n\}$的正向极限记作$\mathfrak{X}$,设$f_n$的正向极限是$f:\mathfrak{X}\to\mathfrak{Y}$.那么如果$X_0$是局部诺特的,就有$\mathfrak{X}$是局部诺特的,并且$f$是进制态射.
    \begin{proof}
    	
    	【EGAI.10.12.3.1】
    \end{proof}
    \item 推论.对任意$\mathfrak{Z}$进制形式概形$\mathfrak{X}$和$\mathfrak{Y}$,有如下典范双射:
    $$\mathrm{Hom}_{\mathfrak{Z}}(\mathfrak{X},\mathfrak{Y})\cong\varprojlim_n\mathrm{Hom}_{S_n}(X_n,Y_n)$$
\end{enumerate}
\subsection{有限型形式态射}
\begin{enumerate}
	\item 设$\mathfrak{Y}$是局部诺特形式概形,设$\mathscr{K}$是定义理想,设$f:\mathfrak{X}\to\mathfrak{Y}$是形式态射,那么如下命题互相等价.此时称$f$是有限型态射,或者称$\mathfrak{X}$是有限型$\mathfrak{Y}$形式概形.
	\begin{enumerate}[(1)]
		\item $\mathfrak{X}$是局部诺特的,$f$是进制态射,并且$f$诱导的$f_0:(\mathfrak{X},\mathscr{O}_{\mathfrak{X}}/\mathscr{I})\to(\mathfrak{Y},\mathscr{O}_{\mathfrak{Y}}/\mathscr{K})$是有限型态射,其中$\mathscr{K}=f^*(\mathscr{K})\mathscr{O}_{\mathfrak{X}}$.
		\item $\mathfrak{X}$是局部诺特的,并且它是一个进制正向$\{Y_n\}$系统$\{X_n\}$的极限,满足$X_0\to Y_0$是有限型的.
		\item $\mathfrak{Y}$的每个点都存在一个诺特仿射形式开邻域$V$,满足如下条件:$f^{-1}(V)$是有限个诺特仿射形式开子集$\{\mathscr{U}_i\}$的并,满足诺特进制环$\Gamma(\mathscr{U}_i,\mathscr{O}_{\mathfrak{X}})$拓扑同构于$\Gamma(V,\mathscr{O}_{\mathfrak{Y}})$上限制幂级数环的商.
		\item 上一条中的每个点存在诺特仿射形式开子集满足这个条件改为任意诺特仿射形式开子集满足这个条件.
	\end{enumerate}
    \begin{proof}
    	
    	(1)推(2)是因为进制$\mathfrak{Y}$形式概形和进制正向$\{Y_n\}$系统范畴是范畴等价的.(2)推(3):问题是终端局部的,可设$\mathfrak{Y}=\mathrm{Spf}(B)$,其中$B$是诺特进制环.记$\mathscr{K}=\mathfrak{b}^{\Delta}$,其中$\mathfrak{b}$是$B$的定义理想.按照$X_0\to Y_0$是有限型态射,$X_0$可以被有限个仿射开子集$\{U_i=\mathrm{Spec}A_{i0}\}$覆盖,其中$A_{i0}$是有限型$B/\mathfrak{b}$代数.设$\Gamma(U_i,\mathscr{O}_{X_n})=A_{in}$,那么对$m\le n$就有$A_{im}$同构于$A_{in}/\mathfrak{b}^{m+1}A_{in}$.于是$\mathfrak{X}\mid_{U_i}=\mathrm{Spf}(A_i)$,其中$A_i=\varprojlim_nA_{in}$.这里$A_i$是一个$\mathfrak{b}A_i$进制环,并且$A_i/\mathfrak{b}A_i$同构于$A_{i0}$是有限型$B/\mathfrak{b}$代数.这两个条件保证了$A_i$是限制$B$幂级数环的商.
    	
    	\qquad
    	
    	(3)推(1):此时还可以不妨设$\mathfrak{X}=\mathrm{Spf}(A)$也是仿射的,其中$A$是诺特进制环,并且同构于$B$上限制幂级数环的商.那么$A/\mathfrak{b}A$是有限型$B/\mathfrak{b}$代数,并且$\mathfrak{b}A$是$A$的定义理想(见限制幂级数环的商的等价描述).另外我们之前解释过$(\mathfrak{b}A)^{\Delta}=\widehat{f}^*(\mathfrak{b}^{\Delta})\mathscr{O}_{\mathfrak{X}}$.
    \end{proof}
    \item 推论.如果$f:\mathfrak{X}\to\mathfrak{Y}$是有限型形式态射,如果$\mathfrak{Y}$是诺特的,那么$\mathfrak{X}$是诺特的.
    \item 仿射情况.设$\mathfrak{X}=\mathrm{Spf}(A)$和$\mathfrak{Y}=\mathrm{Spf}(B)$是诺特仿射形式概形,那么$\mathfrak{X}\to\mathfrak{Y}$是有限型态射当且仅当$A$同构于$B$上限制幂级数环的商.
    \item 
    \begin{enumerate}[(1)]
    	\item 复合.有限型形式态射的复合仍然是有限型的.
    	\item 基变换.设$f:\mathfrak{X}\to\mathfrak{Y}$是局部诺特形式概形之间的有限型态射,它关于任意局部诺特形式概形之间的态射$\mathfrak{Y}'\to\mathfrak{Y}$的基变换$\mathfrak{X}\times_{\mathfrak{Y}}\mathfrak{Y}'\to\mathfrak{Y}'$都是有限型态射.
    \end{enumerate}
    \item 推论.设$f:X\to Y$是局部诺特概形之间的态射,设$Y'\subseteq Y$是闭子集,设$X'=f^{-1}(Y')$,如果$f$是有限型态射,那么$\widehat{f}:X_{/X'}\to Y_{/Y'}$是有限型形式态射.
\end{enumerate}
\subsection{形式概形的闭子概型}
\begin{enumerate}
	\item 设$\mathfrak{X}$是局部诺特形式概形,设$\mathscr{A}$是一个凝聚理想层,设$\mathscr{O}_{\mathfrak{X}}/\mathscr{A}$的支集为$\mathfrak{Y}$,这是一个闭子集,那么拓扑环空间$(\mathfrak{Y},(\mathscr{O}_{\mathfrak{X}}/\mathscr{A})\mid_{\mathfrak{Y}})$是局部诺特形式概形.并且如果$\mathfrak{X}$是诺特的,则它也是诺特的.定义这个形式概形为$\mathfrak{X}$的被凝聚理想层$\mathscr{A}$所定义的形式闭子概型.按照定义,形式闭子概型和$\mathscr{O}_{\mathfrak{X}}$的凝聚理想层是一一对应的.
	\begin{proof}
		
		因为$\mathscr{O}_{\mathfrak{X}}/\mathscr{A}$是凝聚层,所以它的支集是闭子集.设$\mathscr{I}$是$\mathfrak{X}$的定义理想层,设概形$X_n=(\mathfrak{X},\mathscr{O}_{\mathfrak{X}}/\mathscr{I}^{n+1})$.那么环层$\mathscr{O}_{\mathfrak{X}}/\mathscr{A}$就是环层族$\{\mathscr{O}_{\mathfrak{X}}/(\mathscr{A}+\mathscr{I}^{n+1})=(\mathscr{O}_{\mathfrak{X}}/\mathscr{A})\otimes_{\mathscr{O}_{\mathfrak{X}}}(\mathscr{O}_{\mathfrak{X}}/\mathscr{I}^{n+1})\}$的逆向极限.这些层的支集都是$\mathfrak{Y}$.层$(\mathscr{A}+\mathscr{I}^{n+1})/\mathscr{I}^{n+1}$是凝聚$\mathscr{O}_{\mathfrak{X}}/\mathscr{I}^{n+1}$层,设$Y_n$是$X_n$的被这个凝聚理想层定义的闭子概型,那么$(\mathfrak{Y},(\mathscr{O}_{\mathfrak{X}}/\mathscr{A})\mid_{\mathfrak{Y}})$就是$\{Y_n\}$的正向极限.按照局部诺特形式概形的正向极限描述,得到它是局部诺特的,并且从$\mathfrak{X}$诺特得到$Y_0$诺特,进而它是诺特的.
	\end{proof}
    \item 闭嵌入.设$\mathfrak{Y}$是$\mathfrak{X}$的形式闭子概型,典范连续单射$\mathfrak{Y}\to\mathfrak{X}$和层态射$\mathscr{O}_{\mathfrak{X}}\to\mathscr{O}_{\mathfrak{X}}/\mathscr{I}$构成了形式概形态射,它称为形式闭子概型的典范闭嵌入.仿射上典范闭嵌入就是被$A\to A/\mathfrak{a}$所诱导的形式态射.局部诺特形式概形之间的态射$\mathfrak{X}\to\mathfrak{Y}$称为闭嵌入,如果它可以分解为$\xymatrix{\mathfrak{X}\ar[r]^g&\mathfrak{Y}'\ar[r]^j&\mathfrak{Y}}$,其中$g$是同构,$j$是典范闭嵌入.
    \item 闭嵌入是有限型态射.因为仿射局部上闭嵌入被$A\to A/\mathfrak{a}$诱导.
    \item 设$f:\mathfrak{Y}\to\mathfrak{X}$是局部诺特形式概形之间的态射,设$\{\mathscr{U}_{\alpha}\}$是$f(\mathfrak{Y})$在$\mathfrak{X}$中的一族诺特仿射形式开覆盖,满足$f^{-1}(\mathscr{U}_{\alpha})$是$\mathfrak{Y}$的诺特仿射形式开子集.那么$f$是闭嵌入当且仅当$f(\mathfrak{Y})$是$\mathfrak{X}$的闭子集,并且对任意$\alpha$,有$f$限制在$f^{-1}(\mathscr{U}_{\alpha})$上对应于一个满同态$\Gamma(\mathscr{U}_{\alpha},\mathscr{O}_{\mathfrak{X}})\to\Gamma(f^{-1}(\mathscr{U}_{\alpha}),\mathscr{O}_{\mathfrak{Y}})$.
    \begin{proof}
    	
    	只需证明充分性.设满同态$\Gamma(\mathscr{U}_{\alpha},\mathscr{O}_{\mathfrak{X}})\to\Gamma(f^{-1}(\mathscr{U}_{\alpha}),\mathscr{O}_{\mathfrak{Y}})$的核为$\mathfrak{a}_{\alpha}$.考虑$\mathscr{U}_{\alpha}$上的层$\mathfrak{a}_{\alpha}$,以及$f(\mathfrak{Y})$的补集这个开子集上的零层,它们可以粘合为一个凝聚层$\mathscr{A}$,这定义了闭嵌入$f$.
    \end{proof}
    \item 
    \begin{enumerate}[(1)]
    	\item 复合.局部诺特形式概形之间的闭嵌入的复合仍然是闭嵌入.
    	\item 基变换.局部诺特形式概形范畴上闭嵌入的基变换仍然是闭嵌入.
    \end{enumerate}
    \item 推论.考虑如下局部诺特形式概形的纤维积图表:
    $$\xymatrix{\mathfrak{X}\times_{\mathfrak{Z}}\mathfrak{Y}\ar[rr]^p\ar[d]_q&&\mathfrak{X}\ar[d]^f\\\mathfrak{Y}\ar[rr]^g&&\mathfrak{Z}}$$
    
    对任意凝聚$\mathscr{O}_{\mathfrak{X}}$模层$\mathscr{F}$,都有典范的$\mathscr{O}_{\mathfrak{Y}}$模层同构:
    $$u:g^*f_*(\mathscr{F})\cong q_*p^*(\mathscr{F})$$
    \item 推论.设$X$是局部诺特概形,$Y$是$X$的闭子概型,典范闭嵌入为$j:Y\to X$,设$X'$是$X$的闭子集,设$Y'=Y\cap X'$.按照$Y'=j^{-1}(X')$,就有$\widehat{j}:Y_{/Y'}\to X_{/X'}$是闭嵌入,并且对任意凝聚$\mathscr{O}_Y$模层$\mathscr{F}$都有:
    $$\widehat{j}_*(\mathscr{F}_{/Y'})\cong(j_*(\mathscr{F}))_{/X'}$$
\end{enumerate}
\subsection{分离形式概形}

设$\mathfrak{Z}$是形式概形,一个形式态射$f:\mathfrak{X}\to\mathfrak{Z}$称为分离态射,或者称形式概形$\mathfrak{X}$是分离$\mathfrak{Z}$形式概形,如果对角态射$\Delta_{\mathfrak{X}\mid\mathfrak{Z}}:\mathfrak{X}\to\mathfrak{X}\times_{\mathfrak{Z}}\mathfrak{X}$的像集是终端的闭子集.
\begin{enumerate}
	\item 设形式态射$f:\mathfrak{X}\to\mathfrak{Z}$是一族概形态射$\{f_n:X_n\to S_n\}$的正向极限.那么$f$是分离形式态射当且仅当$f_0:X_0\to S_0$是分离态射.
	\begin{proof}
		
		因为$\Delta=\Delta_{\mathfrak{X}\mid\mathfrak{Z}}$是$\Delta_n=\Delta_{X_n\mid S_n}$的正向极限,并且$\Delta$的像集是$\Delta_0$的像集.
	\end{proof}
    \item 设这里所有形式概形和形式态射都是通常概形和态射的正向极限.
    \begin{enumerate}[(1)]
    	\item 复合.两个分离形式态射的复合都是分离形式态射.
    	\item 基变换.分离形式态射的基变换是分离形式态射.
    	\item 如果形式态射的复合$g\circ f$是分离的,那么$f$是分离的.
    \end{enumerate}
    \begin{proof}
    	
    	这件事是因为$(g\circ f)_0=g_0\circ f_0$和$(f\times_{\mathfrak{Z}}g)_0=f_0\times_{S_0}g_0$.
    \end{proof}
    \item 设$\mathfrak{Z}$是局部诺特形式概形,设$\mathfrak{X}$和$\mathfrak{Y}$是$\mathfrak{Z}$形式概形,其中有一个是有限型的,并且$\mathfrak{Y}$是分离$\mathfrak{Z}$形式概形.设$f:\mathfrak{X}\to\mathfrak{Y}$是$\mathfrak{Z}$形式态射,那么被$1_{\mathfrak{X}}$和$f$诱导的图像态射$\Gamma_f:\mathfrak{X}\to\mathfrak{X}\times_{\mathfrak{Z}}\mathfrak{Y}$是闭嵌入.
    \begin{proof}
    	
    	【】
    \end{proof}
    \item 推论.设$\mathfrak{Z}$是局部诺特形式概形,设$\mathfrak{X}$是有限型$\mathfrak{Z}$形式态射,那么$\mathfrak{X}\to\mathfrak{Z}$分离当且仅当对角态射$\Delta_f:\mathfrak{X}\to\mathfrak{X}\times_{\mathfrak{Z}}\mathfrak{X}$是闭嵌入.
    \item 局部诺特形式概形之间的闭嵌入一定是分离态射.
    \item 设$X$是局部诺特概形,设$X'\subseteq X$是闭子集,记$\widehat{X}=X_{/X'}$,那么$\widehat{X}$是分离的当且仅当$\widehat{X}_{\mathrm{red}}$是分离的,当且仅当$X$是分离的.
    \begin{proof}
    	
    	这件事是因为$\widehat{X}_{\mathrm{red}}=(X'_0)_{\mathrm{red}}$.
    \end{proof}
\end{enumerate}
\newpage
\section{刚性几何}
\subsection{Tate代数}

设$(K,|\bullet|)$是完备非阿基米德赋值域.这个赋值可以唯一延拓到$\overline{K}$上,尽管延拓后的赋值未必是完备的,但是对$K$在$\overline{K}$中的每个有限扩张上这个赋值都是完备的.定义$K$上的$n$元限制Tate代数为:
$$T_n=K\langle X_1,\cdots,X_n\rangle=\left\{\sum_Ia_I\underline{X}^T\in K[[X_1,\cdots,X_n]]\middle|\lim\limits_{|I|\to\infty}|a_I|_K=0\right\}$$
$K$的赋值环为$R=\{a\in K\mid|a|\le1\}$,定义$R\langle X_1,\cdots,X_n\rangle$是系数落在$R$中的$T_n$的子代数.
\begin{enumerate}
	\item 记$\mathbb{B}^n(\overline{K})=\{(x_1,\cdots,x_n)\in\overline{K}^n\mid|x_i|\le1\}$称为$\overline{K}^n$上的单位球.记$f=\sum_Ia_I\underline{X}^I\in K[[X_1,\cdots,X_n]]$,那么$\lim\limits_{|I|\to\infty}|a_I|=0$当且仅当$f$在$\mathbb{B}^n(\overline{K})$中收敛.
	\begin{proof}
		
		充分性:如果$f$在$(1,1,\cdots,1)$处收敛,也即$\sum_Ia_I$收敛,也即$\lim\limits_{|I|\to\infty}|a_I|=0$.必要性:选取$x\in\mathbb{B}^n(\overline{K})$,那么存在$K$的有限扩张$L$包含了$x$的全部分量.那么$\lim\limits_{|I|\to\infty}|a_I|=0$推出$\lim\limits_{|I|\to\infty}|a_I||x^I|=0$,于是$f(x)$在$L$上收敛.
	\end{proof}
    \item 赋予$T_n$上范数$|f|=\max_I|a_I|$,其中$f=\sum_Ia_I\underline{X}^I$.这里范数是指对任意$f,g\in T_n$有:
    \begin{enumerate}[(1)]
    	\item $|f|=0$当且仅当$f=0$.
    	\item $|cf|=|c||f|,\forall c\in K$.
    	\item $|fg|=|f||g|$.
    	\item $|f+g|\le\max\{|f|,|g|\}$.
    \end{enumerate}
    $K$的赋值环为$R=\{a\in K\mid|a|\le1\}$,这是局部环,它的唯一极大理想是$\mathfrak{m}=\{a\in K\mid|a|<1\}$,它的剩余域记作$k=A/\mathfrak{m}$.那么典范商同态$R\to k$可以延拓为$\pi:R\langle X_1,\cdots,X_n\rangle\to k[X_1,\cdots,X_n]$.在这个映射下$f$的像$\pi(f)$称为$f$的约化.那么$\pi(f)=0$当且仅当$|f|<1$.
    \item $(T_n,|\bullet|)$是Banach $K$-代数.
    \begin{proof}
    	
    	取$f_n=\sum_Ia_{n,I}\underline{X}^I,n\ge0$,设$\lim\limits_{n\to\infty}f_n=0$.那么$|a_{n,I}|\le|f_n|$,进而$\sum_{n\ge0}a_{n,I}$收敛,记作$a_I$.我们断言$f=\sum_Ia_I$收敛,并且$f=\sum_{n\ge0}f_n$.
    	
    	\qquad
    	
    	取$\varepsilon>0$,那么存在$N$使得$n>N$时$|f_n|<\varepsilon$,也即$|a_{n,I}|<\varepsilon$对任意$n>N$和任意$I$成立.又因为$f_0,\cdots,f_{N-1}$是收敛的级数,所以当指标足够大时那些系数的赋值也$<\varepsilon$,综上对几乎所有$a_{n,I}$都有$|a_{n,I}|<\varepsilon$.于是$f$收敛且$f=\sum_{n\ge0}f_n$.
    \end{proof}
    \item $T_n$中的单位.设$f\in T_n$满足$|f|=1$,那么$f$是单位当且仅当它在$k[X_1,\cdots,X_n]$中的约化$\pi(f)$是单位.更一般的,一个$f\in T_n$是单位当且仅当$|f-f(0)|<|f(0)|$,此即$f$的常数项的绝对值严格大于其它系数的绝对值.
    \begin{proof}
    	
    	问题归结为设$f$满足$|f|=1$,设$\pi(f)$在$k^*$中,证明$f$是单位.按照$\pi(f)\in k^*$,那么$|f(0)|=1$,我们不妨设$f(0)=1$,因为总归相差一个$K^*$中的元.此时$f=1-g$,其中$|g|<1$,那么此时$\sum_{n\ge0}g^n$就是$f$的逆.
    \end{proof}
    \item 极大模原理.设$f\in T_n$,那么对任意$x\in\mathbb{B}^n(\overline{K})$都有$|f(x)|\le|f|$,并且存在$x_0\in\mathbb{B}^n(\overline{K})$使得$|f(x_0)|=|f|$.
\end{enumerate}







\subsection{刚性空间}

设$R$是完备离散赋值环,uniformizer记作$\pi$,剩余域记作$k$,商域记作$K$.
\begin{enumerate}
	\item 一个形式$R$概形$\mathcal{X}$是指如下方式构造的环空间$(\mathcal{X},\mathscr{O}_{\mathcal{X}})$:对每个自然数$r\ge0$,选取一个$R/\pi^{r+1}$概形$X_r$,使得$X_r$在模掉$\pi^r$后就是$X_{r-1}$.于是全部$X_r,r\ge0$具有相同的底空间,把它取为$\mathcal{X}$,它通常记作$X_k$,称为特殊纤维.结构层取为$\mathscr{O}_{\mathcal{X}}=\varprojlim_r\mathscr{O}_{X_r}$.
	\item 形式概形的爆破.设$\mathcal{X}$是形式概形,设$\mathscr{I}$是它的一个定义理想层,那么$\mathcal{X}$沿$\mathscr{I}$的爆破定义为如下形式概形$\widetilde{\mathcal{X}}$:记$\mathcal{X}=\cup_{j\in J}\mathcal{U}_j$是形式概形$\mathcal{X}$的仿射开覆盖,记$\mathcal{U}_j=\mathrm{Spf}A_j$,记$U_j=\mathrm{Spec}A_j$,记$V_j$是$U_j$关于$\Gamma(U_j,\mathscr{I})$的爆破.取$\widetilde{\mathcal{U_j}}$是$V_j$沿自身特殊纤维的完备化,再粘合得到的形式概形记作$\widetilde{\mathcal{X}}$.
	\item 刚性空间.定义$\textbf{Form}(R)$为有限型分离,关于$\pi$-adic拓扑完备的$R$形式概形构成的范畴.这个范畴关于全体容许爆破的完备化范畴称为刚性$K$空间范畴,记作$\textbf{Rig}(K)$.换句话讲这两个范畴的对象类是一致的,儿$\textbf{Rig}(K)$中的一个态射$\mathcal{D}\to\mathcal{X}$理解为$\textbf{Form}(R)$中态射构成的如下图表,其中$\sigma$是一个容许爆破,而$f'$是一个形式态射.两个刚性空间$X,X'$是同构的当且仅当存在一个形式概形$X''$同时是它们的容许爆破.
	$$\xymatrix{\mathcal{D}'\ar[d]_{\sigma}\ar[rr]^{f'}&&\mathcal{X}\\\mathcal{D}&&}$$
	\item 对任意形式$R$概形,它的一般纤维视为和它伴随的刚性$K$空间.反过来在刚性空间的同构意义下,它总是某些形式概形的一般纤维,称这样的形式概形为该刚性空间的整模型(integer models).如果$X_1,X_2$是同一个刚性空间的两个整模型,那么它们一定被第三个整模型$X_3$控制,并且$X_3$是$X_1$和$X_2$的容许爆破.
	\item Tate代数.
	\begin{enumerate}[(1)]
		\item 定义$R$上的限制幂级数代数或者Tate代数为如下代数的商,记作$R\langle T_1,\cdots,T_N\rangle$.
		$$R\{T_1,\cdots,T_N\}=\left\{\sum_Ia_I\underline{T}^I\mid\lim_{|I|\to\infty}|a_I|=0\right\}$$
		\item 定义Tate $K$代数为$K\{T_1,\cdots,T_N\}=R\{T_1,\cdots,T_N\}\otimes_RK$的商.
		\item 设$A_K$是一个Tate $K$代数,定义它的标准Tate $K$代数为具有如下形式的代数,其中$f_0,\cdots,f_m\in A_K$生成了单位理想:
		$$B_K=A_K\{Z_1,\cdots,Z_m\}/(f_1-f_0Z_1,\cdots,f_m-f_0Z_m)$$
		\item Tate $K$代数$A_K$上的系数绝对值上确界是一个范数,这使得$A_K$是一个赋范空间,并且在这个拓扑下,代数同态总是连续的.
		\item 设$A_K$是Tate代数.设$x\in\mathrm{Spm}A_K$是极大谱中的点,那么$\kappa(x)/K$是一个有限扩张,于是$K$上的完备离散赋值可以延拓为$\kappa(x)$上的完备离散赋值.于是对$f\in A_K$,就有$|f(x)|$是有意义的.
		\item 设$\varphi:A_K\to C_K$是两个Tate $K$代数之间的连续同态,任取闭点$x\in\mathrm{Spm}A_K$,那么有单射$C_K/\varphi^{-1}(x)\subseteq A_k/x=\kappa(x)$,后者是完备赋值域,它的非零素理想一定是极大理想,于是$\varphi^{-1}(x)$是极大理想.
		\item 于是如果$B_K$是$A_K$上的标准Tate代数,那么典范同态$A_K\to B_K$诱导了$\mathrm{Spm}B_K$到如下$\mathrm{Spm}A_K$的子集之间的双射,这个子集我们称为标准开集.
		$$U=\{x\in\mathrm{Spm}A_K\mid|f_i(x)|\le|f_0(x)|,i=1,\cdots,m\}$$
	\end{enumerate}
	\item 设$A_K$是Tate $K$代数,设$X$是它的极大谱.我们定义$X$上的Grothendieck拓扑和一个换层$\mathscr{O}_X$如下.同构于$(X,\mathscr{O}_X)$的环空间称为域$K$上的affinoid空间.
	\begin{enumerate}[(1)]
		\item $X$的开集定义为这样的子集$V$:它可以表示为一族标准开集$\{U_i\mid i\in I\}$的并,满足对任意Tate代数的极大谱$Y$和任意一个满足$f(Y)\subseteq V$的由连续$K$代数同态诱导的态射$f:Y\to X$,都有$f(Y)$被有限个$U_i$覆盖.
		\item $X$的开集$V$的一个开覆盖$\{V_j\mid j\in J\}$称为容许的(admissible),如果它满足如下条件:对任意Tate代数的极大谱$Y$和任意一个满足$f(Y)\subseteq V$的由连续$k$代数同态诱导的态射$f:Y\to X$,都有$\{f^{-1}(V_j)\mid j\in J\}$可以加细为$Y$的有限个标准开集.
		\item $X$上的环层$\mathscr{O}_X$定义为对标准开集$U=\mathrm{Spm}B_K$有$\Gamma{U,\mathscr{O}_X}=B_K$.
		\item Tate定理.容许开覆盖构成的\v{C}ech复形是环层$\mathscr{O}_X$的预解.
		\item 如果$C_K$是$A_K$上的平坦Tate代数,并且$Y=\mathrm{Spm}C_K\to X$是单态射,那么【Bosch】$Y$是$X$上有限个标准开集的并,并且这个开覆盖是容许的.特别的$Y$是$X$上的拟紧开集.
	\end{enumerate}
	\item 刚性空间.一个刚性$K$空间指的是能被有限个affinoid空间覆盖的环空间.取一般纤维的函子$\textbf{Rig}(K)\to\{\text{刚性空间}\}$是范畴等价,于是这个定义吻合于我们之前定义的刚性空间.
	\item 凝聚层.设$(X,\mathscr{O}_X)$是affinoid空间,一个$\mathscr{O}_X$模层$\mathscr{F}$称为凝聚层,如果它存在由affinoid空间构成的容许开覆盖$\{U_i=\mathrm{Spm}A_i\}$,以及有限$A_i$模$M_i$,使得$\mathscr{F}\mid_{U_i}\cong M_i\otimes\mathscr{O}_{U_i}$.进而可定义刚性空间上的凝聚层.
	\item 如果$X$是$K$-affinoid空间,$\mathcal{X}$是它的一个$R$整模型,那么任意$\mathscr{O}_X$凝聚模层都来自于某个$\mathscr{O}_{\mathcal{X}}$凝聚模层.
	\item 紧合态射.
	\begin{enumerate}[(1)]
		\item 设$f:\mathcal{Y}\to\mathcal{X}$是形式$R$概形之间的态射,它称为紧合的如果特殊纤维$f_0:Y_0\to X_0$作为概形态射是紧合的.
		\item 设$f:Y\to X$是刚性空间之间的态射,那么它对应于如下图表,它称为紧合的如果其中$f'$作为形式概形之间的态射是紧合的($\sigma$作为blowup自动是紧合的,甚至是射影的).
		$$\xymatrix{&Y'\ar[dl]_{\sigma}\ar[dr]^{f'}&\\Y&&X}$$
	\end{enumerate}
	\item 有限性定理和存在性定理.
	\begin{enumerate}[(1)]
		\item 形式版本的有限性定理.设$f:\mathcal{Y}\to\mathcal{X}$是形式$R$概形之间的紧合态射,对任意$\mathscr{O}_{\mathcal{Y}}$凝聚层$\mathscr{F}$和任意整数$q\ge0$,总有$\mathrm{R}^qf_*\mathscr{F}$是凝聚$\mathscr{O}_{\mathcal{X}}$模层.
		\item 刚性版本的有限性定理.设$f:Y\to X$是$K$刚性空间之间的紧合态射,对任意$\mathscr{O}_Y$凝聚层$\mathscr{F}$和任意整数$q\ge0$,总有$\mathrm{R}^qf_*\mathscr{F}$是凝聚$\mathscr{O}_X$模层.
		\item 形式版本的存在性定理.设$X$是紧合$R$概形,设$\mathscr{F}$是凝聚$\mathscr{O}_X$模层,设它们的$\pi$-adic完备化分别为$\widetilde{X}$和$\widetilde{\mathscr{F}}$.
		\begin{enumerate}[(a)]
			\item 对任意整数$i\ge0$,都有典范同构:
			$$\mathrm{H}^i(X,\mathscr{F})\cong\mathrm{H}^i(\widetilde{X},\widetilde{\mathscr{F}})$$
			\item 函子$\mathscr{F}\mapsto\widetilde{\mathscr{F}}$是$\textbf{Coh}(\mathscr{O}_X)\to\textbf{Coh}(\mathscr{O}_{\widetilde{X}})$的范畴等价.
		\end{enumerate}
	    \item 刚性版本的存在性定理.设$X_K$是紧合$K$概形,在固定一个紧合$R$模型的前提下,存在同构意义下唯一的紧合$K$刚性空间$X$和它对应,类似的对任意凝聚$\mathscr{O}_{X_K}$模层$\mathscr{F}_K$,存在同构意义下唯一的凝聚$\mathscr{O}_X$模层$\mathscr{F}$.
	    \begin{enumerate}[(a)]
	    	\item 对任意整数$i\ge0$,都有典范同构:
	    	$$\mathrm{H}^i(X_K,\mathscr{F}_K)\cong\mathrm{H}^i(X,\mathscr{F})$$
	    	\item 函子$\mathscr{F}_K\mapsto\mathscr{F}$是$\textbf{Coh}(\mathscr{O}_{X_K})\to\textbf{Coh}(\mathscr{O}_X)$的范畴等价.
	    \end{enumerate}
	\end{enumerate}
\end{enumerate}
\subsection{形式曲线}

设$K$是非阿基米德域,整数环记作$R$.一个平坦$R$形式曲线定义为有限型平坦可分$R$形式概形,满足特殊纤维是等维数1的.
\begin{enumerate}
	\item 按照GAGA原理,函子$X\mapsto\widehat{X}$是从紧合光滑$R$曲线到紧合光滑形式$R$曲线的范畴等价.
	\item 按照定义,平坦$R$形式曲线的维数是2.
	\item 
\end{enumerate}

\subsection{解析函数的牛顿多边形}

设$(K,|\bullet|_K)$是完备离散赋值域,设$R$是它的赋值环,剩余域记作$k$,单值化参数记作$\pi$.因为$K$是完备的,其上的离散赋值$|\bullet|_K$可以唯一的延拓到它的固定的代数闭包$\overline{K}$上,这个延拓赋值我们仍然用相同的记号.本节介绍刚性几何观点下的代数曲线.按照半稳定约化定理,域$K$上的一条几何连通光滑紧合代数曲线总存在在$R$上的半稳定约化.这个模型的形式纤维是圆盘或者圆环.于是这样的圆盘和圆环就是曲线在解析观点下的组成单位.
\begin{enumerate}
	\item 解析函数.设$I$是$\mathbb{R}_{\ge0}$中的区间,定义其上的解析函数代数为:
	$$\mathscr{A}(I)=\{f=\sum_{v\in\mathbb{Z}}c_vT^v\in K[[T,T^{-1}]]\mid\forall x\in\overline{K},|x|_K\in I\Rightarrow f(x)\text{收敛}\}$$
	\item 解析函数的牛顿多边形.设$f=\sum_{v\in\mathbb{Z}}c_vT^v\in\mathscr{A}(I)$,记$S=\{(v,-\log|c_v|_K)\in\mathbb{R}^2\mid v\in\mathbb{Z}\}$.如果$\rho\in I$,那么对$|z|_K=\rho$的点$z$有$f(z)$收敛,于是$\lim\limits_{|v|\to+\infty}|c_v|_K\rho^v=0$.这说明$\mathbb{R}^2$中任意一条斜率为$\log\rho$的直线的下面只能包含$S$中有限个点.记:
	$$|f|_{\rho}=\sup_{|x|_K=\rho}|f(x)|_K=\sup_{v\in\mathbb{Z}}|c_v|_K\rho^v$$
	
	那么$-\log|f|_{\rho}$就是最小的斜率,使得存在该斜率的直线和$S$有交.再定义:
	$$n(f,\rho)=\inf\{v\in\mathbb{Z}\mid|c_v|_K\rho^v=|f|_{\rho}\}$$
	$$N(f,\rho)=\sup\{v\in\mathbb{Z}\mid|c_v|_K\rho^v=|f|_{\rho}\}$$
	
	如果$n(f,\rho)\not=N(f,\rho)$则称$\log\rho$为$f$的异常斜率(exceptional slope).异常斜率构成了$I$的一个离散子集;在两个异常斜率之间,总有$n=n(f,\rho)=N(f,\rho)$是常值,并且$|f|_{\rho}=|c_n|_K\rho^n$.换句话讲异常斜率就是关于$\log\rho$的函数$\log|f|_{\rho}$的不连续点.我们定义解析函数$f$的牛顿多边形为如下函数的图像:
	$$t\mapsto P(f,t)=\sup_{\rho\in I}(-\log|f|_{\rho}+t\log\rho)$$
	\item 借助牛顿多边形可以得到解析函数零点的赋值信息.
	\begin{enumerate}[(1)]
		\item 解析函数$f$的牛顿多边形恰好是$S=\{(v,-\log|c_v|_K)\mid v\in\mathbb{Z}\}$的下凸闭包的边界.
		\item 如果$x$是$f$的零点,记$\rho=|x|_K$,那么$\log\rho$是$f$的异常斜率.
		\item 设$\log\rho$是$f$的异常斜率,如果$n(f,\rho)\le t\le N(f,\rho)$,那么$P(f,t)=-\log|f|_{\rho}+t\log\rho$,并且在$\overline{K}$中恰好存在$N(f,\rho)-n(f,\rho)$个$f$的零点满足赋值为$\rho$.
	\end{enumerate}
    \item 用$L_{\rho}$表示$K[[T,T^{-1}]]$中的满足$|x|_K=\rho$的$x$总收敛的洛朗级数构成的代数.设$\rho$是正实数并且$\log\rho$是$f$的异常斜率,那么存在唯一的对$(P,g)$满足如下条件:
    \begin{enumerate}[(1)]
    	\item $P$是一个$K$系数的多项式,次数为$N(f,\rho)-n(f,\rho)$.
    	\item $N(P,\rho)=\deg P=N(f,\rho)-n(f,\rho)$,并且$n(P,\rho)=0$.
    	\item $g\in L_{\rho}$,并且有$f=Pg$.
    \end{enumerate}
    \item 推论.$L_{\rho}$的可逆元具有形式$f=\alpha T^r(1+u)$,其中$\alpha\in K$非零,$r\in\mathbb{Z}$,而$u\in L_{\rho}$满足$|u|_{\rho}<1$.
    \item 准备定理.设$I=[\rho_1,\rho_2]$是$\mathbb{R}_{\ge0}$的闭区间,设$f\in\mathscr{A}(I)$.
    \begin{enumerate}[(1)]
    	\item $f$是$\mathscr{A}(I)$中的可逆元当且仅当$N(f,\rho_2)=n(f,\rho_1)$.于是此时存在$0\not=\alpha\in K$,$r\in\mathbb{Z}$和$u\in\mathscr{A}(I)$使得$f=\alpha T^r(1+u)$,并且有$\inf_{\rho_1\le|z|_K\le\rho_2}|u(z)|_K<1$.
    	\item 如果$f\not=0$,那么存在唯一的次数为$N(f,\rho_2)-n(f,\rho_1)$的多项式$P\in K[T]$,使得$P(0)=1$和$f=Pg$,其中$g$是$\mathscr{A}(I)$中的可逆元.
    \end{enumerate}
    \item 解析函数在闭圆盘或者闭圆环上只有有限个零点,单射对于开圆盘或者开圆环结论不成立.不过一个有界解析函数在开圆盘或者开圆环上只有有限个零点.
\end{enumerate}
\subsection{超度量圆盘和圆环}

\begin{enumerate}
	\item 设$A$是一个Tate $K$代数,对任意$f\in A$,定义$\vert f\vert_{\mathrm{sp}}=\sup\{|f(x)|_K\mid x\in\mathrm{Spm}A\}$,这里$|f(x)|_K$有意义是因为$\kappa(x)/K$是有限扩张,于是$K$上的完备离散赋值可以唯一的延拓到$\kappa(x)$中.记$A^0=\{a\in A\mid\vert a\vert_{\mathrm{sp}}\le1\}$和$A^{00}=\{a\in A\mid\vert a\vert_{\mathrm{sp}}<1\}$,再记$\overline{A}=A^0/A^{00}$.那么$\overline{A}$是一个有限型$k$代数,它的极大谱称为affinoid空间$\mathrm{Spm}A$的典范约化.定义特殊化态射$\mathrm{sp}:\mathrm{Spm}A\to\mathrm{Spm}\overline{A}$如下:设$x\in\mathrm{Spm}A$,记$L=A/x$,记典范同态$\varphi:A\to L$.记$L$的赋值环为$L^0$,记剩余域为$\overline{L}$.定义$\mathrm{sp}(x)=\ker\overline{\varphi}$,这里$\overline{\varphi}$是由如下图表定义的:
	$$\xymatrix{A\ar[rr]^{\varphi}&&L\\A^0\ar[u]\ar[rr]^{\varphi^0}\ar[d]&&L^0\ar[u]\ar[d]\\\overline{A}\ar[rr]^{\overline{\varphi}}&&\overline{L}}$$
	\item 闭单位圆盘.$R$上的标准形式圆盘是指$R$上仿射线的$\pi$-adic完备化,记作$\mathscr{D}$.那么$\mathscr{D}$就是$R{T}$对应的仿射形式概形.$\mathscr{D}$的一般纤维就是affinoid空间$D=\mathrm{Spm}(R{T}\otimes_RK)=\mathrm{Spm}K{T}$.
	【】
	\item 开单位圆盘.
	
\end{enumerate}




\newpage
\section{可构造集}
\subsection{一般拓扑空间上的可构造集}

设$X$是拓扑空间.
\begin{itemize}
	\item 一个子集$C$称为反紧的(retro-compact),如果对任意拟紧开子集$U\subseteq X$有$C\cap U$是拟紧的.换句话讲包含映射$C\to X$是一个拟紧映射(这个定义就是拟紧开子集的原像拟紧).按照定义空集也是反紧的.
	\item 考虑由全体开集在有限交和取补集下封闭生成的最小的子集族,其中的集合称为$X$的拟整体可构造集(global quasi-constructible set).
	\item 考虑由反紧开集在有限交和取补集下封闭生成的最小的子集族,其中的集合称为$X$的整体可构造集(global constructible set).
	\item 一个子集$C$称为拟可构造的(quasi-constructible set),如果每个点$x\in X$都存在开邻域$U$使得$C\cap U$是$U$的拟整体可构造子集;它称为可构造的(constructible set),如果每个点$x\in X$都存在开邻域$U$使得$C\cap U$是$U$的拟整体可构造子集.
\end{itemize}
\begin{enumerate}
	\item 关于反紧性的一些简单事实.闭子集总是反紧的;拟紧空间的反紧子集总是拟紧的(取拟紧开集是$X$本身,那么按照定义$X\cap C=C$是拟紧的);反紧子集的有限并仍然是反紧的;反紧开子集的有限交仍然是反紧的;由于在局部诺特空间(这是指每个点都存在开邻域是诺特空间)中拟紧子集总是诺特的,就导致局部诺特空间中的所有子集都是反紧的.
	\item 
	\begin{enumerate}
		\item 空间$X$的子集是整体可构造集当且仅当它可以表示为有限个形如$U\cap(X-V)$的子集的并,其中$U,V$是$X$的反紧开集.类似的空间$X$的子集是拟整体可构造集当且仅当它可以表示为有限个形如$U\cap(X-V)$的子集的并,其中$U,V$是$X$的开子集,此时形如$U\cap(X-V)$的子集换句话讲就是可以表示为一个开集和一个闭集的交的子集,这样的子集称为局部闭子集,于是我们断言的是拟整体可构造集恰好就是可以表示为有限个局部闭子集的并的子集.
		\begin{proof}
			
			我们只证明前半部分,后半部分是一样的.充分性是因为这样的集合的确是由反紧开集在取有限交有限并和补集操作下得到的.必要性归结为证明可以表示为有限个形如$U\cap(X-V)$的子集的并,其中$U,V$是$X$的反紧开集,的子集构成的集族在有限并和取补集下封闭.在有限并下封闭是平凡的,取补集封闭是因为如果$Y=\cup_i(U_i\cap(X-V_i))$,其中$U_i,V_i$都是反紧开集,那么$Y^c=\cap_i((X-U_i)\cup V_i)$是形如$(X-U_i)\cap V_j$的有限并,所以$Y^c$是整体可构造的.
		\end{proof}
		\item 可构造集/拟可构造集的有限并有限交和补集都是可构造集/拟可构造集.这件事是因为整体可构造集在更小开集上的限制仍然是这个更小开集的整体可构造集.
	\end{enumerate}
	\item 整体可构造集总是反紧的.特别的,如果$U$是开集,那么它是整体可构造集当且仅当它是反紧的;如果$F$是闭集,那么它是整体可构造集当且仅当$X-F$是反紧的.
	\begin{proof}
		
		我们解释过反紧子集的有限并还是反紧的,于是只需证明$U\cap(X-V)$是反紧的,其中$U$和$V$都是反紧开集.而这是因为任取拟紧开子集$V'\subseteq X$,那么按照$U$是反紧的,得到$U\cap V'$是拟紧的,而$U\cap V'\cap(X-V)$是拟紧集$U\cap V'$的闭子集,所以它仍然是拟紧的.剩下两个命题是因为反紧开集和反紧开集的补集都是整体可构造集.
	\end{proof}
	\item 按照定义,空间$X$是拟分离的(虽然这是概形上的性质,但是它其实是拓扑的,即两个拟紧开子集的交总是拟紧的)当且仅当它的拟紧开子集都是反紧的;空间$X$是拟紧的可以推出它的反紧开子集一定是拟紧的.于是如果空间$X$是qcqs的,那么它的反紧开子集和拟紧开子集是一致的(上面还说过反紧开集总等价于整体可构造开子集).此时$X$的整体可构造集就是有限个$U\cap(X-V)$的并,其中$U,V$都是$X$的拟紧开子集.
	\item 拟整体可构造集在连续映射下的原像一定是拟整体可构造集,拟可构造集在连续映射下的原像一定是拟可构造集.这个结论对一般的整体可构造集和可构造集当然不对.
	\item 关于各种可构造集在开子集上的限制.设$U\subseteq X$是开子集.
	\begin{enumerate}
		\item 如果$T$是$X$的整体可构造集/拟整体可构造集/可构造集/拟可构造集,那么$T\cap U$是$U$的整体可构造集/拟整体可构造集/可构造集/拟可构造集.
		\begin{proof}
			
			归结为设$T$是$X$的反紧开子集的情况,此时对任意拟紧开子集$W\subseteq U$,有$T\cap U\cap W=T\cap W$是拟紧的,于是$T\cap U$是反紧开子集.
		\end{proof}
		\item 如果$U\subseteq X$是反紧开集,那么$U$的子集$Z$在$X$中是整体可构造集当且仅当它在$U$中是整体可构造集.对任意开集$U\subseteq X$,有$U$的子集$Z$在$X$中是拟整体可构造集当且仅当它在$U$中是拟整体可构造集.
		\begin{proof}
			
			上一条说明了必要性.至于充分性,我们先设$Z$是$U$的反紧开集,设$W$是$X$的任意拟紧开子集,那么$Z\cap W=Z\cap(W\cap U)$,按照$U$是$X$的反紧开集,得到$W\cap U$是$U$的拟紧开集,于是$Z\cap(W\cap U)$是拟紧的,这就说明$Z$是$X$的反紧开子集.于是我们证明了$U$中的反紧开子集$Z$在$X$中还是反紧的,进而$U-Z=U\cap Z^c$是$X$的整体可构造集,进而对$U$的任意反紧开子集$Z,Z'$都有$Z\cap(U-Z')$是$X$的整体可构造集,这说明$U$的整体可构造子集都是$X$的整体可构造子集.
		\end{proof}
		\item 推论.设$\{U_i\}$是$X$的由反紧开集构成的有限开覆盖,那么一个子集$Z\subseteq X$是整体可构造的当且仅当对每个指标$i$,有$U_i\cap Z$是$U_i$的整体可构造子集;设$\{U_i\}$是$X$的有限开覆盖,那么一个子集$Z\subseteq X$是拟整体可构造的当且仅当对每个指标$i$,有$U_i\cap Z$是$U_i$的拟整体可构造子集.
	\end{enumerate}
	\item (拟)整体可构造集和(拟)可构造集之间的关系.
	\begin{enumerate}
		\item 整体可构造集一定是可构造集.反过来如果$X$是拟紧的,并且$X$有反紧开集构成的拓扑基(等价于每个点存在反紧开邻域构成的开邻域基),那么可构造集一定是整体可构造集.当$X$是诺特空间时这两个条件都是满足的.
		\begin{proof}
			
			设$Z$是$X$的可构造集.对每个$x\in X$,选取开邻域$V$使得$V\cap Z$是$V$的整体可构造集,我们可以取$x$的反紧开子集$U$满足$U\subseteq V$,那么$U\cap Z$是$U$的整体可构造子集,又因为$U$是反紧开集,我们解释了此时有$U\cap Z$是$X$的整体可构造子集.满足这个条件的反紧开集$U$可以开覆盖整个$X$,按照$X$是拟紧的可以取有限开覆盖,于是$Z$是有限个$X$的整体可构造子集$U\cap Z$的并,于是它是整体可构造子集.
		\end{proof}
		\item 拟整体可构造集一定是拟可构造集.反过来如果$X$是拟紧的,那么拟可构造集一定是拟整体可构造集.
	\end{enumerate}
	\item 推论.设$X$有反紧开集构成的拓扑基,那么$X$的可构造子集总是反紧的.
	\begin{proof}
		
		设$T$是$X$的可构造子集,任取$X$的拟紧开子集$U$,我们要证明的是$T\cap U$是拟紧的.我们可以取有限个反紧开子集$U_i\subseteq U$,满足$\cup_iU_i=U$,并且$T\cap U_i$在$U_i$中是整体可构造的,那么我们解释过$T\cap U_i$就在$U$中是整体可构造的,于是$T\cap U$作为$T\cap U_i$的有限并就也在$U$中整体可构造.于是$T\cap U$在$U$中反紧,但是$U$本身是拟紧的,就得到$T\cap U$是拟紧的.
	\end{proof}
\end{enumerate}
\subsection{诺特空间上的可构造集}

\begin{enumerate}
	\item 诺特空间上的整体可构造集,可构造集,拟整体可构造集,拟可构造集全部都是一致的概念,就统一称为可构造集.它具有如下等价描述:
	\begin{enumerate}
		\item 它是诺特空间的子集族$\mathscr{A}$中的元素,这里$\mathscr{A}$是由全体开集在有限交和保补集下生成的最小的子集族.
		\item 它是诺特空间的有限个局部闭子集的并,这里局部闭子集指的是可以写作某个开子集和某个闭子集的交的子集.
		\item 它是诺特空间的有限个局部闭子集的无交并.
	\end{enumerate}
	\begin{proof}
		
		我们之前没解释过(a)和(c)的等价性.把所有能表示为有限个局部闭子集的无交并的子集构成的子集族为$\mathscr{B}$.按照局部闭子集是可构造子集,得到$\mathscr{B}\subseteq\mathscr{A}$.按照$\mathscr{A}$的极小性,问题归结为证明$\mathscr{B}$满足可构造子集定义中的三个条件.
		
		\qquad
		
		首先开集自然是局部闭子集,所以全体开集包含在$\mathscr{B}$中.如果任取两个$\mathscr{B}$中的元$\coprod_{i=1}^nE_i\cap U_i$和$\coprod_{j=1}^mE_j'\cap U_j'$.那么有:
		$$\left(\coprod_{i=1}^nE_i\cap U_i\right)\cap\left(\coprod_jE_j'\cap U_j'\right)=\coprod_{i,j}(E_i\cap E_j')\cap(U_i\cap U_j')\in\mathscr{B}$$
		
		\qquad
		
		验证保补集.任取局部闭子集$E\cap U$,其中$E$是闭集,$U$是开集,那么有$(E\cap U)^c=U^c\cup E^c=U^c\coprod(E^c\cap U)$在$\mathscr{B}$中.现在任取有限个局部闭子集的无交并$\coprod_iS_i$,它的补集就是$\cap_iS_i^c$,每个$S_i^c$在$\mathscr{B}$中,我们验证了$\mathscr{B}$是保有限交的,于是这个补集落在$\mathscr{B}$中.
	\end{proof}
	\item 推论.两个诺特空间之间的连续映射,一定把可构造集回拉为可构造集;如果$E\subseteq X$是可构造集,那么对任意子空间$Y\subseteq X$,有$Y\cap E$是$Y$的可构造子集;如果$Y$是$X$的可构造子集,那么$Y$的子集$Z$是$Y$的可构造子集当且仅当$Z$是$X$的可构造子集.
	\item 诺特空间$X$上的子集$E$是可构造集当且仅当对每个不可约闭子集$Y\subseteq X$,有$E\cap Y$要么包含了$Y$的一个非空开子集,要么$E\cap Y$在$Y$中无处稠密.回顾一个子集称为无处稠密的,如果它的闭包不包含内点,等价的讲它闭包的补集是一个稠密开集,另外如果全空间是不可约的,那么一个子集是无处稠密的当且仅当它的闭包不是全空间(也即此时无处稠密和非稠密等价),当且仅当这个子集的补集包含一个非空开集.
	\begin{proof}
		
		先设$E$是可构造集,那么$E\cap Y$是$Y$的可构造子集,所以它是有限个$Y$的局部闭子集的并.如果其中某个局部闭子集在$Y$中稠密,那么这个局部闭子集一定是$Y$的开集,于是此时$E\cap Y$包含了$Y$的非空开集.如果$E\cap Y$的每个局部闭子集都不是稠密的,也即每个局部闭子集在$Y$中的补集包含了一个非空开集,因为$Y$是不可约的,所以这些非空开集的交非空,于是这些非稠密的局部闭子集的并仍然是非稠密的,于是此时$E\cap Y$在$Y$中无处稠密.
		
		\qquad
		
		充分性,我们来对$X$的闭子集用诺特归纳.换句话讲我们假设对任意真闭子集$Y\subseteq X$,有$E\cap Y$是$Y$的可构造子集(但是因为$Y\subseteq X$是闭子集,所以这也等价于讲$E\cap Y$是$X$的可构造子集),我们来证明$E$是$X$的可构造子集.先设$X$不是不可约的,设它的不可约分支是有限个$\{X_i\mid i\in I\}$.那么$X_i$都是$X$的闭子集,所以归纳假设保证了$E\cap X_i$都是$X$的可构造子集,所以它们的(有限个)并就也是$X$的可构造子集.于是问题归结为设$X$是不可约空间.那么条件要求$E$要么在$X$中无处稠密,此时$E$包含在真闭子集$\overline{E}$中,于是归纳假设要求$E$是无处稠密的,要么$E$包含了一个非空开子集$U$,但是$E-U=E\cap(X-U)$按照归纳假设是可构造的,于是$E=U\cup(E-U)$也是可构造的.
	\end{proof}
	\item 推论.不可约诺特空间$X$上的子集$E$是可构造子集具有如下刻画:要么$E$是非稠密的(在不可约条件下等价于无处稠密的),要么$E$是稠密的,并且包含了一个非空开集.
	\item 设$X$是诺特空间,设$\{E_i\}$是$X$的可构造集的在包含偏序下的有向集,满足:
	\begin{enumerate}
		\item $X$是$\{E_i\}$的并.
		\item $X$的任何不可约闭子集都包含在某个$E_i$的闭包中.
	\end{enumerate}
	
	那么可以找到一个指标$i$使得$X=E_i$.另外如果$X$的每个不可约闭子集都有一般点(这甚至不需要同一个不可约闭子集的一般点是唯一的),则此时条件(b)是自动满足的.
	\begin{proof}
		
		按照诺特归纳法,我们可以假设$X$的任意真闭子集都包含在某个$E_i$中.倘若$X$不是不可约的,那么它只有有限个不可约分支$\{X_1,\cdots,X_s\}$,于是按照归纳假设,每个$X_i$都包含在某个$E_{j_i}$中,按照有向集的定义,就可以找到一个$E_{j_0}$包含了每个$X_i$,从而$E_{j_0}=X$.现在设$X$是不可约的,按照条件可以找到一个指标$j$使得$X=\overline{E_j}$.我们解释过不可约诺特空间的一个可构造集是稠密的当且仅当它包含了一个非空开集记作$U$,那么$X-U$是$X$的真闭子集,于是存在指标$i$使得$X-U\subseteq E_i$,于是选取$E_k$使得$E_i,E_j\subseteq E_k$,就有$X=E_k$.
	\end{proof}
	\item 设$X$是诺特空间,设$x\in X$,设$E$是$X$的可构造集,那么$E$是$x$的邻域(此为存在开集$V$满足$x\in V\subseteq E$)当且仅当对$X$的任一个包含$x$的不可约闭子集$Y$,有$E\cap Y$总在$Y$中稠密(如果$Y$具有一般点$y$,那么$E\cap Y$在$Y$中稠密等价于讲$y\in E$).
	\begin{proof}
		
		必要性是因为不可约空间的非空开子集一定是稠密的.对于充分性,我们记$\mathfrak{M}$由$X$的那些包含$x$并且$E\cap Y$是$x$在$Y$中邻域的闭子集$Y$构成,我们要证明的是$X$本身落在$\mathfrak{M}$中,为此我们可以证明更强的所有闭子集都落在$\mathfrak{M}$中,而这按照诺特归纳就不妨设$X$的所有真闭子集都落在$\mathfrak{M}$中.先设$X$不是不可约的,那么它有有限个不可约分支$\{X_1,\cdots,X_s\}$,设$X_i$是包含$x$的不可约分支,那么$X_i\in\mathfrak{M}$,于是$E\cap X_i$是$x$在$X_i$中的邻域.我们设$1\le i\le r$的时候$X_i$都包含$x$,其余的不可约分支不包含$x$,那么对$1\le i\le r$就存在开集$U_i$使得$x\in U_i\cap X_i\subseteq E\cap X_i$,那么$\cup_{1\le i\le r}(E_i\cap U_i)$是$\cup_{1\le i\le r}X_i$的开子集,并且包含在$E$中,于是$E$是$x$在$\cup_{1\le i\le r}X_i$中的邻域,又因为$\cup_{1\le i\le r}X_i$是$x$在$X$中的邻域,这就导致$E$是$x$在$X$中的邻域.
		
		\qquad
		
		下面设$X$是不可约的,条件要求$E$在$X$中稠密,但是我们解释了这导致$E$要包含$X$的一个非空开集$U$.倘若$x\in U$,则$E$就是$x$的邻域.倘若$x\in Y=X-U$,按照诺特归纳假设,就有$E\cap Y$在$Y$中稠密,而$E\cap Y$是$Y$的可构造集,于是$E\cap Y$包含了$x$在$Y$中的某个开邻域,也即存在$X$的开集$V$使得$V\cap Y\subseteq E\cap Y$.那么这里$(V\cap Y)\cup U=V\cup U$是$X$的开集,并且包含$x$,并且这个开集落在$E$中,换句话讲我们找到了$x$在$X$中的开邻域满足整个落在$E$中,这就导致$x\not\in\overline{X-E}$,从而$\overline{X-E}$的补集是$x$在$X$中的开邻域,并且这个开邻域在$E$中,于是此时也有$E$是$x$在$X$中的开邻域.
	\end{proof}
	\item 推论.设$X$是诺特空间,子集$E$是开集当且仅当对任意与$E$有交的不可约闭子集$Y$,都有$E\cap Y$包含了$Y$的非空开集.
	\begin{proof}
		
		必要性是直接的,对于充分性,条件保证了$E$是可构造集,而上一条又保证了$E$是自身任一点的邻域,这就得到$E$是开集.
	\end{proof}
	\item 推论.设$X$是sober的诺特空间(这里sober指的是$X$的每个不可约闭子集都存在唯一的一般点),设$E\subseteq X$是子集,那么$E$是$X$的开子集当且仅当$E$是保一般化的,并且对$X$的任意开子集$V$和$V$的任意不可约闭子集$Y$,如果$V-Y\subseteq E$并且$Y$的一般点落在$E$中,那么总有$E\cap Y$包含了$Y$的非空开集.
	\begin{proof}
		
		首先尽管$X$是sober的,但是它的子空间(即便是开子空间)的不可约闭子集我们即不知道有没有一般点也不知道一般点是不是唯一的.首先我们断言$Y$作为$V$的不可约闭子集恰好存在唯一的一般点:考虑$Y$在$X$中的闭包$\overline{Y}$,因为不可约子空间的闭包还是不可约的,所以它是$X$的不可约闭子集,于是按照sober条件它存在唯一的一般点$\eta$,这个一般点是$Y$中点的一般化,按照$E$是保一般化的导致$\eta\in V$中,于是$\eta$也是$Y$作为$V$的不可约闭子集的一般点.倘若$Y$作为$V$的不可约闭子集还有另一个一般点$\eta'$,那么在$X$中$\eta'$也是$\eta$的一般化,于是按照$X$的不可约闭子集恰有唯一一般点,就导致$\eta'=\eta$.完成断言的证明.
		
		\qquad
		
		如果$E$是$X$的开集,那么当然它保一般化,并且此时$E\cap Y$是非空的(因为含$Y$的一般点),那么$E\cap Y$本身就是$Y$的非空开集,于是必要性明显成立.下面证明充分性,设$E$在空间$X$中的内点集为$U$,那么$X-U$作为$X$的闭子空间也是诺特的,所以$X-U$的不可约分支个数是有限的,并且每个不可约分支都也是$X$的不可约闭子集,所以它们都存在唯一的一般点.下面假设$U\not=E$,那么$X-U$和$E$有交,按照$E$是保一般化的,于是至少存在$X-U$的某个不可约分支的某个一般点$z$落在$E$中.并且$z$只能是$X-U$的一个不可约分支的一般点,设$X-U$的其余(有限个)不可约分支的并为$T$,那么$V=X-T$是$X$的开子集,并且是$U$和$Y=\overline{z}\cap V$的并.并且这里$Y$是$V$的不可约闭子集.那么按照条件,$E\cap Y$就要包含$Y$的一个非空开子集$W$.按照$V-U\cup W=Y-W$是$V$的闭子集,于是$U\cup W$是$V$的开子集.又因为$V$是$X$的开子集,就导致$U\cup W$是$X$的开子集.但是这个$W\subseteq E$,按照$U$是$E$在$X$中的内点集,就导致$W\subseteq U$.但是$W$作为$Y$的非空开集一定包含一般点$z$,这导致$z\in U$,矛盾.于是$U=E$,也即$E$是$X$的开集.
	\end{proof}
\end{enumerate}
\subsection{拓扑空间上的可构造映射}

设$h$是从拓扑空间$X$到某个集合$T$的映射,称$h$是整体可构造的,如果对任意$t\in T$,都有$h^{-1}(t)$是$X$的整体可构造子集,并且除了有限个$t\in T$以外,$h^{-1}(t)$总是空集.后面这个条件保证了对$T$的任意子集$S$,都有$h^{-1}(S)$是整体可构造集.称$h$是可构造的,如果对任意$x\in X$,都存在开邻域$V$,使得$h\mid_V$是整体可构造的.
\begin{enumerate}
	\item 整体可构造映射当然是可构造映射,反过来如果$X$是拟紧的,并且存在由反紧开集构成的拓扑基(比方说,$X$是诺特空间),则逆命题成立.
	\item 设$h:X\to T$是映射,其中$X$是诺特空间,那么$h$是可构造映射映射当且仅当对$X$的任意不可约闭子集$Y$,都可以找到$Y$的一个非空开子集$U$,使得$h$在$U$上是常值的.
	\begin{proof}
		
		必要性.按照定义$h$至多取有限个值,记作$\{t_i\}$,并且每个$h^{-1}(t_i)\cap Y$都是$Y$的可构造集.但是不可约空间的无处稠密集等价于补集包含非空开集的集合,这可以说明不可约空间的有限个无处稠密子集的并还说无处稠密的,这说明$h^{-1}(t_i)\cap Y$不能都是$Y$的无处稠密集,否则这有限个无处稠密集的并集是$Y$本身,却又在$Y$中无处稠密.但是可构造子集如果和不可约子空间的交不是无处稠密的,则交中必须包含非空开集,换句话讲某个$h^{-1}(t_i)\cap Y$要包含非空开集,于是$h$在其上的限制是常值的.
		
		\qquad
		
		充分性.我们考虑$X$全体满足$h\mid_Y$是可构造映射的闭子集$Y$构成的集合$\mathfrak{M}$,在条件下我们要证明的就是$X\in\mathfrak{M}$,我们来证明更强的,所有闭子集都在$\mathfrak{M}$,于是按照诺特归纳,可设$X$的所有真闭子集都落在$\mathfrak{M}$中.如果$X$不是不可约的,那么它只有有限个不可约分支,并且$h$在其中任意一个上的限制都是可构造映射,按照可构造映射的定义,这能推出$h$在$X$上是可构造映射.下面设$X$是不可约的,那么条件要求存在$X$的非空开子集$U$使得$h\mid_U$是常值的,按照诺特归纳的假设,$h$限制在$X-Y$上也是可构造映射,这就得到$h$在整个$X$上是可构造的(可构造集的有限并是可构造的).
	\end{proof}
	\item 设诺特空间$X$的每个不可约闭子集都有一般点,设$h:X\to T$是映射,满足对任意$t\in T$有$h^{-1}(t)$是可构造集,那么$h$是可构造映射.
	\begin{proof}
		
		我们用上一条结论,任取不可约闭子集$Y$,记一般点为$y$,那么$Y\cap h^{-1}(h(y))$是$Y$的包含一般点$y$的可构造集,于是它在$Y$中稠密,但是我们解释过不可约诺特空间的可构造集是稠密的当且仅当包含了非空开集,于是$Y\cap h^{-1}(h(y))$包含了$Y$的非空开集,于是$h$在这个开集上的限制是常值的,于是上一条说明$h$是可构造映射.
	\end{proof}
	\item 设诺特空间$X$的每个不可约闭子集都有一般点,设$h:X\to T$是可构造映射,这里$T$是一个全序集(从而自动赋予序拓扑),则$h$是上半连续的当且仅当对任意$x\in X$和$x$的任意一般化$x'$,都有$h(x')\le h(x)$.这里$h$是上半连续指的是,对任意$b>f(x_0)$,存在$x_0$的开邻域$U$,使得$f(U)<b$.因为$h$的取值只有有限个,这等价于讲对任意$x\in X$,满足$f(y)\le f(x)$的点$y$总构成$x$的邻域.
	\begin{proof}
		
		设满足$f(y)\le f(x)$的点$y$构成的集合是$E$,那么$E$是可构造集.我们解释过可构造集$E$是$x$的邻域当且仅当对$X$的任意一个包含$x$的不可约闭子集$Y$,都有$E\cap Y$在$Y$中稠密.但是如果记$Y$的一般点为$y$,那么$E\cap Y$在$Y$中稠密当且仅当$y\in E$,而$x\in Y$当且仅当$y$是$x$的一般化.于是对$X$的任意一个包含$x$的不可约闭子集$Y$,都有$E\cap Y$在$Y$中稠密,等价于讲对$x$的任意一般化$x'$,都有$h(x')\le h(x)$.
	\end{proof}
\end{enumerate}
\subsection{概形上的可构造集}
\begin{enumerate}
	\item 一些基本性质.
	\begin{enumerate}
		\item 首先按照定义,概形$X$的开子集$U$是反紧的当且仅当对任意拟紧开子集$V$有$U\cap V$是拟紧的.
		\item 特别的如果$X$是拟分离概形,那么它的拟紧开子集总是反紧的.于是此时$X$存在由反紧开子集构成的拓扑基,我们解释过这个条件下$X$的可构造集都是反紧的.
		\item 如果$X$是qcqs概形,那么反紧开子集等价于拟紧开子集等价于可构造开子集,此时$X$的整体可构造集和可构造集是一致的.
	\end{enumerate}
	\item 设$f:X\to Y$是概形之间的态射,设$E\subseteq Y$是一个可构造集或者整体可构造集,那么$f^{-1}(E)$是$X$的可构造集或者整体可构造集.
	\begin{proof}
		
		如果$V\subseteq Y$是反紧开集,此即包含态射$V\to Y$是拟紧开嵌入,那么它的基变换$f^{-1}(V)=V\times_YX\to X$也是拟紧开嵌入,于是$f^{-1}(V)$也是反紧开集.这就说明整体可构造集的原像仍然是整体可构造集.再设$E$是可构造集,任取$x\in X$,设$V\subseteq Y$是$f(x)$的开邻域,使得$E\cap V$是$V$的整体可构造集,那么$f^{-1}(E)\cap f^{-1}(V)$是$f^{-1}(V)$的整体可构造集,而$f^{-1}(E)\cap f^{-1}(V)$是$x$的开邻域,这就说明$f^{-1}(E)$是可构造集.
	\end{proof}
	\item 设$X$是qcqs概形,设$Z$是可构造集,那么存在以仿射概形为源端的有限表示态射$f:X'\to X$,使得$f(X')=Z$.
	\begin{proof}
		
		我们先解释下问题归结为设$X$本身是仿射的.首先因为$X$是拟紧的,它是有限个仿射开子集$\{X_i\mid i\in I\}$的并,又因为$X$是拟分离的,于是开嵌入$g_i:X_i\to X$总是拟紧开嵌入,我们解释过拟紧开嵌入一定是有限表示态射,于是这些$g_i$都是有限表示态射.另外我们知道$Z\cap X_i$是$X_i$的可构造集.倘若对$X$仿射情况已经得证,那么存在仿射概形$X_i'$,以及有限表示态射$f_i:X_i'\to X_i$,使得$f_i(X_i')=Z\cap X_i$.那么$h_i=g_i\circ f_i:X_i'\to X$也是有限表示态射,并且如果记概形的和$X'=\coprod_iX_i'$,那么泛性质诱导的态射$h:X'\to X$仍然是有限表示态射,并且$h(X')=\cup_ih(X_i')=\cup_iZ\cap X_i=Z$.于是问题归结为设$X=\mathrm{Spec}A$是仿射的.
		
		\qquad
		
		按照定义,有$Z=\cup_{1\le i\le n}Z_i$,其中每个$Z_i$都是形如$U\cap(X-V)$的子集,其中$U,V$是$X$的拟紧开子集.倘若命题对$Z=U\cap(X-V)$,其中$U,V$是$X$的拟紧开子集成立.那么对每个指标$i$,都存在仿射开子集$X_i'$以及有限表示态射$g_i:X_i\to X$满足$g_i(X_i)=Z_i$,于是只要取$X'=\coprod_iX_i'$,取$g_i$按照概形的和的泛性质诱导的态射$g:X'\to X$就仍然是一个有限表示态射,并且满足$f(X')=\cup_if(X_i')=\cup_iZ_i=Z$.于是问题又归结为设$Z=U\cap(X-V)$,其中$U,V$都是$X$的拟紧开子集.
		
		\qquad
		
		接下来按照$U$是拟紧开子集,于是它是有限个仿射开子集的并,做上两段中相同的约化,可以不妨设$U$本身是仿射开子集.接下来设$V$作为$X=\mathrm{Spec}A$的拟紧开子集,被有限个主开集$D(f_1),\cdots,D(f_r)$覆盖,那么$X-V$就是$I=(f_1,\cdots,f_r)$对应的闭子概型$Z'$,并且这里$f_i$有限导致$Z'\to X$是有限表示态射(闭嵌入自动是qcqs的).下面取$X'=Z'\times_XU$,取$f:X'\to X$是纤维积的结构态射,按照拟紧开嵌入是有限表示的,并且有限表示在基变换下不变,就得到这个结构态射$f:X'\to X$是有限表示的.最后我们知道对于纤维积如下图表,对$U$的任意子集$M$总有$j^{-1}\circ i(M)=q\circ p^{-1}(M)$,取$M=U$,那么得到$Z'\cap U=q(X')$,但是按照$j$是闭嵌入,于是$q(X')\cong j\circ q(X')=f(X')$.
	\end{proof}
	\item 推论.设$X$是诺特概形,设$C$是可构造子集,那么存在一个仿射概形$X'$和一个有限型态射$f:X'\to X$,使得$f(X')=C$.
	\begin{proof}
		
		把$C$写成有限个局部闭子集的无交并,所以问题归结为设$C$本身是一个局部闭子集,记作$C=U\cap Z$,其中$U$是$X$的开集,$Z$是$X$的闭集.再把$U$写为有限个仿射开子集的并$U=\cup_iU_i$,倘若我们证明了有态射$\mathrm{Spec}A_i\to X$使得集合意义的像集是$U_i\cap Z$,那么取余积泛性质中的唯一态射$\coprod_i\mathrm{Spec}A_i\to X$,就有它的集合意义的像集是$\cup_i(U_i\cap Z)=U\cap Z$.有限个仿射概形的余积仍然是仿射概形.于是问题归结为设$U$本身是仿射的,此时就赋予$U\cap Z$仿射开子集的闭子概型结构,诺特条件下嵌入是有限型态射(嵌入是局部有限型态射,诺特条件提供了拟紧性).
	\end{proof}
	\item 设$X$是概形,设$C\subseteq X$是可构造集,设$Y\subseteq X$是闭不可约子集,那么$C\cap Y$和$Y-C$中至少有一个包含了一个$Y$的稠密开子集.
	\begin{proof}
		
		设$Y$的一般点是$\eta$,设$C$包含了$\eta$,否则取$C$的补集仍然是可构造集.我们断言$C$包含了$Y$的一个稠密开集.取$\eta$的仿射开邻域$W$,那么$W\cap C$是$W$的整体可构造集,它可以表示为若干$U_i\cap(X-V_i)$的并,其中$U,V$是反紧开集,我们设其中$U\cap(X-V)$覆盖了点$\eta$,但是$U\cap(X-V)$的闭包肯定在$X-V$中,这迫使$Y\subseteq X-V$,于是$U\cap Y$就是$C$所包含的$Y$的稠密开子集.
	\end{proof}
	\item 推论.设$Y$是不可约概形,设一般点为$\eta$,设$f:X\to Y$是局部有限型态射,如果$f^{-1}(\eta)$非空,那么存在$\eta$的开邻域$U$,使得$U\subseteq f(X)$.如果$\mathscr{F}$是有限型拟凝聚$\mathscr{O}_X$模层,并且设$\mathscr{F}_{\eta}=\mathscr{F}\otimes_{\mathscr{O}_Y}\kappa(\eta)$非空,则存在$\eta$的开邻域$U'$,使得对任意$y\in U'$都有$\mathscr{F}_y=\mathscr{F}\otimes_{\mathscr{O}_Y}\kappa(y)$非零(也即它的支集非空).
	\begin{proof}
		
		我们记典范投影态射$p_y:f^{-1}(y)=X\times_Y\mathrm{Spec}\kappa(y)\to X$,则有$\mathscr{F}_y=p_y^*\mathscr{F}$.进而有$\mathrm{Supp}(\mathscr{F}_y)=p_y^{-1}(\mathrm{Supp}(\mathscr{F}))=\mathrm{Supp}(\mathscr{F})\cap f^{-1}(y)$(这个等式依赖于$\mathscr{F}$是有限型的).另外按照$\mathscr{F}$是有限型的,就有支集$\mathrm{Supp}(\mathscr{F})$是$X$的闭子集,设它对应的唯一的既约闭子概型是$Z$,记闭嵌入$j:Z\to X$,这也是局部有限型态射,于是复合$f\circ j:Z\to Y$也是局部有限型态射.所以一旦我们证明了命题的前半部分,把它用在局部有限型态射$f\circ j$上,就存在$\eta$的开邻域$U'$满足对任意的$y\in U'$都有$j^{-1}\circ f^{-1}(y)=f^{-1}(y)\cap Z=\mathrm{Supp}(\mathscr{F}_y)$非空,这也就说明了$\mathscr{F}_y$非空.于是命题的后半部分得证.
		
		\qquad
		
		于是问题归结为证明命题的前半部分.由于问题是拓扑上的,所以不妨设$X,Y$都是既约概形,那么特别的$Y$是整概形.任取$f^{-1}(\eta)$中的点$\xi$,因为我们的问题是局部的,我们可以用$\eta$的仿射开邻域$V=\mathrm{Spec}A$,以及$\xi$的满足$f(U)\subseteq V$的仿射开邻域$U=\mathrm{Spec}B$分别替换$Y$和$X$.那么$B$是既约环,$A$是整环,并且$B$是有限型$A$代数.再设$\overline{\{\xi\}}$对应的$X$的既约闭子概型为$Z$,那么$X\to Y$复合上$Z\to X$仍然是有限型态射,并且用$Z$替换$X$可以让我们不妨设$X=\mathrm{Spec}B$就是以$\xi$为一般点的整仿射概形,换句话讲$B$是整环.那么由于$f$把一般点映射为一般点,它就是支配态射,所以它对应的环同态$\varphi:A\to B$是单射.于是我们只需证明存在$g\in A$使得$D(g)\subseteq f(X)$(这件事还在Chevalley定理一节有证明),为此只需证明存在这样的$g\in A$,使得对任意$y\in D(g)$(换句话讲$g\not\in I(y)$),都有$I(y)$是$A$和$B$的某个素理想的交.而这归结为如下交换代数结论:设$A,B$是整环,单同态$A\to B$使得$B$是有限型$A$代数,那么存在$g\in A$,使得对任意在$g$处不取零的环同态$A\to\Omega$,都可以延拓为同态$B\to\Omega$.因为任取$A$的素理想$\mathfrak{p}$,使得$g\not\in\mathfrak{p}$,我们可以取$\Omega$是$\kappa(\mathfrak{p})$的代数闭包,则$A\to\Omega$的核是$\mathfrak{p}$,的确满足在$g$不取零,那么它可以延拓为同态$B\to\Omega$,那么它的核就是$B$的素理想$\mathfrak{q}$,按照$A\to B$是单同态就得到$\mathfrak{p}=\mathfrak{q}\cap A$,这就得证.
	\end{proof}
	\item 设$S$是概形,设$g:X\to S$和$h:Y\to S$是两个有限表示态射,设$f:X\to Y$是$S$态射.对任意$s\in S$,我们记$X_s=g^{-1}(s)=X\times_S\mathrm{Spec}\kappa(s)$和$Y_s=h^{-1}(s)=Y\times_S\mathrm{Spec}\kappa(s)$和$f_s=f\times\mathrm{Spec}\kappa(s):X_s\to Y_s$.那么使得$f_s$是满射(或者改成泛单射)的点$s$构成了$S$的可构造集.
	\begin{proof}
		
		$s\in S$使得$f_s$不是满射,当且仅当存在$Y-f(X)$中的点在$h$下的像是$s$,换句话讲如果记使得$f_s$是满射的点$s\in S$构成的集合为$E$,那么$S-E=h(Y-f(X))$.接下来按照$g,h$都是有限表示的,导致$f$也是有限表示的(我们解释过这里$h$只要拟分离局部有限表示,就有$f$是有限表示的).于是按照Chevalley定理(实际上只要$f$是局部有限表示就够了),$f(X)$是$Y$的可构造集,进而$Y-f(X)$是$Y$的可构造集,再用一次Chevalley定理,就有$S-E=h(Y-f(X))$是$S$的可构造集,从而$E$是$S$的可构造集.
		
		\qquad
		
		按照$f$是有限表示的,得到结构态射$X\times_YX\to S$也是有限表示的,另外$f_s$是泛单射当且仅当$\Delta_{f_s}$是满射,而这里$\Delta_{f_s}$是$\Delta_f:X\to X\times_YX$关于$\mathrm{Spec}\kappa(s)\to S$的基变换.所以把前半部分的结论应用在$S$态射$\Delta_f:X\to X\times_YX$上,按照$X$和$X\times_YX$作为$S$概形的结构态射都是有限表示的,于是使得$\Delta_{f_s}$是满射的$s\in S$构成可构造集,进而使得$f_s$是泛单射的$s\in S$构成可构造集.
	\end{proof}
	\item Zariski空间$X$的非空子集是闭集当且仅当它是可构造集并且保点的特殊化,对偶的非空子集是开集当且仅当它是可构造集并且保点的一般化.更一般的,设$X$是概形,设$C$是可构造集,那么$C$是开集当且仅当它保特殊化,它是闭集当且仅当它保一般化.我们把证明放在后面.
	\begin{proof}
		
		以闭集为例.一方面闭集肯定是可构造集,并且保点的特殊化.反过来设$S=\coprod_{i=1}^nE_i\cap U_i$是可构造子集,不妨设这里每个$E_i\cap U_i$是非空的,否则可以删去无交并中的这一项.选取$E_i$的和$U_i$的交非空的那个不可约分支的一般点$x$,那么$x$必然在$U_i$中,所以按照$S$保特殊化,说明$\overline{\{x\}}\subseteq S$.所以每个$E_i$的和$S$有交的不可约分支都落在$S$中,按照诺特性这是有限个不可约闭子集的并落在$S$中,反过来$S$的每个点肯定落在某个$E_i$的某个不可约分支中,这就说明$S$恰好是这有限个不可约闭子集的并,所以$S$是闭集.
	\end{proof}
\end{enumerate}
\subsection{Chevalley定理}

Chevalley定理断言诺特概形之间的有限型态射一定把可构造集映射为可构造集.更一般的如果$f:X\to Y$是拟紧的局部有限表示态射,那么$f(X)$是$Y$的可构造集.
\begin{enumerate}
	\item 一般来讲概形之间态射的像集可以非常随意,例如任取概形$Y$的子集$Z$,那么$\coprod_{z\in Z}\mathrm{Spec}\kappa(z)\to Y$的像集就是$Z$.
	\item 引理.设$R$是整环,$R\subseteq A$是一个单的环同态,使得$A$是有限型$R$代数,那么存在$0\not=s\in R$,使得存在单同态$R_s[T_1,\cdots,T_n]\to A_s$,使得$A_s$是$R_s[T_1,\cdots,T_n]$上的有限模.
	\begin{proof}
		
		记$S=R-\{0\}$,记$K=\mathrm{Frac}(R)$,那么$S^{-1}A$就是$S^{-1}R=K$上的有限型代数,按照诺特正规化引理,可取$y_1,\cdots,y_n\in S^{-1}A$在$K$上代数无关,并且$S^{-1}A$在$K[y_1,\cdots,y_n]$上有限.我们可以取$s\in S$使得每个$y_i=y_i'/s$,其中$y_i'\in A$.那么有$S^{-1}A$在$K[y_1',\cdots,y_n']$上有限.按照$A$作为$R$代数是有限生成的,可以选取一组生成元,那么它们是以$K[y_1',\cdots,y_n']$为系数的首一多项式的根.我们可以适当把$s$再乘以一个$S$中的元使得这里作为系数的所有出现过的$K[y_1',\cdots,y_n']$中的多项式都落在$R_s[y_1',\cdots,y_n']$中.那么这使得$R_s[y_1',\cdots,y_n']\subseteq A_s$是单环同态并且$A_s$是有限$R_s[y_1',\cdots,y_n']$模.
	\end{proof}
	\item 验证Chevalley定理可归结为证$f(X)$是可构造子集,并且可设$X,Y$都是仿射整诺特概形,并且可设$f$本身是支配态射.
	\begin{proof}
		
		首先$X$上的可构造子集$S$可以表示为有限个局部闭子集的无交并$S=\coprod_iS_i$.每个$S_i$上存在子流形结构,使得包含映射$S_i\subseteq X$是嵌入,而嵌入是有限型态射.这些包含映射的粘合$S\subseteq X$仍然是有限型态射,所以它和$f:X\to Y$的复合也是有限型态射,并且像集是$C$.这说明问题归结为证明$f(X)$是$Y$的可构造子集.
		
		\qquad
		
		取$Y$的有限仿射开覆盖$\{V_i\}$,取每个$f^{-1}(V_i)$的有限仿射开覆盖$\{U_{ij}\}$.一旦证明$f(U_{ij})$是可构造集,就有$f(X)=\cup_{ij}f(U_{ij})$是可构造集,所以归结为设$X,Y$是仿射的.类似的可设$X,Y$是不可约的:如果$Y$是可以的,取不可约分支分解$Y=\cup_iY_i$,取$f^{-1}(Y_i)$的不可约分支$X_{ij}$,那么诱导的$X_{ij}\to Y_i$仍然是有限型态射,所以一旦$f(X_{ij})$是可构造集,那么有限并$f(X)=\cup_{ij}f(X_{ij})$也是可构造集.另外把仿射概形的结构层改为既约的不影响$f$是有限型态射,也不影响$X,Y$的拓扑,所以可设$X,Y$是既约概形,换句话讲我们可设$X,Y$是整的仿射诺特概形.
		
		\qquad
		
		最后说明可设$f$是支配态射.假设对支配态射$f$总有$f(X)$是可构造子集.现在任取态射$f:X\to Y$,那么$f$可视为态射$f':X\to\overline{f(X)}=Y'$的态射.于是$f'$是支配态射,所以$f'(X)$在$Y'$中是可构造子集.所以$f'(X)=\coprod_iU_i\cap E_i$,这里$E_i$是$Y'$的闭子集,所以也是$Y$的闭子集.而$U_i$是$Y'$的闭子集,所以可表示为$U_i=V_i\cap Y'$,其中$V_i$是$Y$的开子集.于是有$f(X)=\coprod_iU_i\cap E_i=U_i\cap(Y\cap E_i)$是$Y$的可构造子集.
	\end{proof}
	\item 在上述条件下,$f(X)$包含了$Y$的稠密开子集.
	\begin{proof}
		
		设$f:\mathrm{Spec}B\to\mathrm{Spec}A$被环同态$\varphi:A\to B$所诱导,按照$f$是支配的,得到$\varphi$是单射.于是我们的引理说明存在$s\in A-\{0\}$使得$\varphi_s:A_s\to B_s$可以分解为$A_s\to A_s[T_1,\cdots,T_n]\to B_s$,使得第二个同态是有限和单的.于是$\varphi_s$诱导了满射$\mathrm{Spec}B_s\to\mathrm{Spec}A_s$,这说明$D(s)\subseteq f(X)$.
	\end{proof}
	\item 证明诺特情况的Chevalley定理:
	\begin{proof}
		
		上一条解释了存在$a\in A$使得$D(a)\subseteq f(X)$.但是开集是可构造集,所以问题归结为证明$V(a)\cap f(X)$是可构造集.而它就是$\varphi:A\to B$诱导的$A/a\to B/aB$的素谱态射$f':\mathrm{Spec}B/aB\to\mathrm{Spec}A/a$的像集.这里$\varphi:A\to B$是单的导致$A/a\to B/aB$也是单的,所以这里$f'$是支配态射.按照$A$是诺特环,可取准素分解$(a)=\cap_iq_i$,其中$q_i$是$p_i$准素理想.那么有$\sqrt{(a)}=\cap_ip_i$.于是在拓扑层面就有$V(a)=V(\sqrt{a})=\cup_iV(p_i)$.按照$B$是诺特的,每个$p_iB$的根理想也可以表示为$B$的有限个素理想的交$\cap_j\pi_{ij}$.于是我们得到态射族$\{\mathrm{Spec}B/\pi_{ij}\to\mathrm{Spec}A/p_i\}$,这些态射的像集的并是整个$f(X)\cap V(a)$.按照诺特归纳,这里每个态射$\mathrm{Spec}B/\pi_{ij}\to\mathrm{Spec}A/p_i$的像集都是$V(p_i)$的可构造集.但是$V(p_i)$是$\mathrm{Spec}A$的闭子集,所以这些像集都是$\mathrm{Spec}A$的可构造子集,所以它们的并$f(X)\cap V(a)$是$\mathrm{Spec}A=Y$的可构造子集.
	\end{proof}
	\item 一般情况的Chevalley定理.
	\begin{enumerate}
		\item 如果$f:X\to Y$是拟紧的局部有限表示态射,那么$f(X)$是$Y$的可构造集.
		\item 如果$f:X\to Y$是有限表示态射,设$C\subseteq X$是可构造子集,那么$f(C)$是$Y$的可构造子集.
		\item 如果$f:X\to Y$是局部有限表示态射,那么$f(X)$是$Y$的一族可构造子集的并.
	\end{enumerate}
	\begin{proof}
		
		(a):取$Y$的仿射开覆盖$\{V_i\}$,如果我们能证明$f(X)\cap V_i$是可构造集,因为$V_i$是拟紧的,它的可构造子集和整体可构造子集一致,于是$f(X)\cap V_i$是整体可构造子集,于是$f(X)=\cup_i(f(X)\cap V_i)$就是可构造子集.于是问题归结为设$Y=\mathrm{Spec}A$是仿射的情况.此时$X$是拟紧的(环上概形的结构态射的拟紧的当且仅当这个概形自身是拟紧的),所以$X$是有限个仿射开子集的并,但是我们知道可构造子集的有限并仍然是可构造子集,于是归结为设$X=\mathrm{Spec}B$也是仿射的.按照条件$B$是有限表示$A$代数.把$A$写作它的有限型$\mathbb{Z}$子代数的正向极限,按照诺特消除原理,存在$A$的有限型$\mathbb{Z}$子代数$A_0$,存在$A_0$上的有限表示代数$B_0$,记结构态射$f_0:\mathrm{Spec}B_0\to\mathrm{Spec}A_0$,那么$A,B,f$就是$A_0,B_0,f_0$的基变换.有如下基变换图表,那么有$f(\mathrm{Spec}B)=p^{-1}(f_0(\mathrm{Spec}B_0))$(这里左包含于右是直接的,右包含于左是因为如果有$x\in\mathrm{Spec}B_0$和$y\in\mathrm{Spec}A$使得它们在$\mathrm{Spec}A_0$中的像相同,那么$q^{-1}(x)\cap f^{-1}(y)$必然是非空的,于是$y\in f(\mathrm{Spec}B)$).最后按照诺特版本的Chevalley定理,有$f_0(\mathrm{Spec}B_0)$是$\mathrm{Spec}A_0$的可构造子集,我们还解释过可构造子集的原像总是可构造子集,于是$f(X)$是$Y$的可构造子集.
		$$\xymatrix{\mathrm{Spec}B\ar[rr]^f\ar[d]_q&&\mathrm{Spec}A\ar[d]^p\\\mathrm{Spec}B_0\ar[rr]_{f_0}&&\mathrm{Spec}A_0}$$
		
		(b):我们依旧可以设$Y$是仿射的,那么此时$X$是qcqs概形.我们解释过qcqs概形上的可构造子集$C$总是某个有限表示态射$g:X'\to X$的像集.那么$f(C)=f\circ g(X')$,而这里$f\circ g$是有限表示的,于是(a)说明$f(C)=f\circ g(X')$是$Y$的可构造子集.
		
		\qquad
		
		(c):同样可以设$Y=\mathrm{Spec}A$是仿射的,可取$X$的仿射开覆盖$\{U_i=\mathrm{Spec}B_i\}$,使得每个$B_i$都是有限表示$A$代数,那么每个$f(U_i)$都是可构造集,于是$f(X)$是一族可构造集的并.
	\end{proof}
\end{enumerate}
\subsection{pro-可构造集和ind-可构造集}

设$X$是拓扑空间,一个子集$E$称为pro-可构造集(pro-constructible set),如果每个点$x\in X$存在开邻域$U$,使得$E\cap U$是$U$的一族可构造集的交;一个子集$E$称为ind-可构造集(ind-constructible set),如果每个点$x\in X$存在开邻域$U$,使得$E\cap U$是$U$的一族可构造集的并.
\begin{enumerate}
	\item\label{pro-可构造集和ind-可构造集1} 设$X$是qcqs概形,设$E\subseteq X$是子集.那么如下命题互相等价:
	\begin{enumerate}
		\item 存在拟紧态射$f:X'\to X$,使得$E=f(X')$.
		\item 存在仿射概形为源端的态射$f:X'\to X$,使得$E=f(X')$(此时$f$一定是拟紧的).
		\item 对$X$的任意仿射开覆盖$\{X_i\mid i\in I\}$,记$E_i=E\cap X_i$,对任意指标$i$,都存在拟紧态射$f_i:X_i'\to X_i$使得$E_i=f_i(X_i')$.
		\item 对$X$的任意仿射开覆盖$\{X_i\mid i\in I\}$,记$E_i=E\cap X_i$,对任意指标$i$,存在仿射概形为源端的态射$f_i:X_i'\to X_i$使得$f_i(X_i')=E_i$.
		\item $E$是$X$的一族可构造集的交(而在qcqs条件下这等价于pro-可构造集).
		\item $F=X-E$是$X$的一族可构造集的并(在qcqs条件下这等价于ind-可构造集).
	\end{enumerate}
	\begin{proof}
		
		(a)推(c)是因为直接取$X_i'=f^{-1}(X_i)=X'\times_XX_i$,取$f_i:X_i'\to X_i$是$f$的基变换,那么按照$E=f(X')$就有$f_i(X_i')=f(X')\cap X_i=E_i$.下面证明(c)推(d):按照$X_i$是仿射的,$f_i$是拟紧的,于是每个$X_i'$都是有限个仿射开子集的并,记作$\{X_{ij}''\}$,固定$i$的时候取$X_{ij}''$的概形和,记作$X_i''$,那么这也是仿射概形.再记概形和的泛性质诱导的态射为$g_i:X_i''\to X_i'$,这个态射在每个$X_{ij}''$上的限制都是恒等态射,特别的它是拟紧的.于是拟紧态射的复合$f_i'=f_i\circ g_i:X_i''\to X_i$也是拟紧的,并且按照$g_i$是满射得到$f_i'(X_i'')=E_i$.再证明(d)推(a):首先按照$X$是拟紧的,我们取有限子集$J\subseteq I$,使得$\{X_i\mid i\in J\}$已经是$X$的开覆盖.我们记$\{X_i\mid i\in J\}$的概形和为$X'$,那么这是一个仿射概形.再记概形和的泛性质诱导的态射$f:X'\to X$.那么对$i\in J$就有$f$和$j_i\circ f_i$是一致的,这里$j_i$典范包含态射$X_i\to X$.另外按照$\{X_i\mid i\in J\}$已经是$X$的开覆盖,于是$\cup_{i\in J}E_i=E$,从而$E=f(X')$.还剩下证明$f$是拟紧的.按照$X$是拟分离的,就有典范包含态射$j_i:X_i\to X$总是拟紧的,于是复合态射$j_i\circ f_i$是拟紧的,也即$f$限制在每个$X_i',i\in J$上都是拟紧的,从而$f$是拟紧的.
		
		\qquad
		
		(d)推(b)只要把开覆盖取为$\{X\}$即可,(b)推(a)是因为满足(b)的态射$f$一定是拟紧的,因为$X'\to X\to\mathrm{Spec}\mathbb{Z}$是拟紧的,而$X$是拟分离的导致$X\to\mathrm{Spec}\mathbb{Z}$的对角态射是拟紧的,这两个条件可以推出$X'\to X$一定是拟紧的.至此我们证明了(a),(b),(c),(d)的等价性.另外(e)和(f)的等价性是平凡的.
		
		\qquad
		
		(d)推(e):设有限子集$J\subseteq I$使得$\{X_i\mid i\in J\}$覆盖了整个$X$.我们只需证明对任意$i\in J$(当然实际上证明了对任意$i\in I$,没什么区别,只不过我们在把$E$写成一族可构造集的交的时候需要用到$J$是有限集),都有$E_i$是$X_i$的一族可构造集$F_{ij}$的并,因为一旦这成立,我们解释过qcqs条件下$X_i$作为拟紧开集,它的可构造集一定是$X$的可构造集,进而让$i$跑遍$J$,就有$E$是有限个子集的并,这里每个子集都是$X$的可构造集的交,但是我们知道可构造集的有限并(这里用到$J$是有限集)还是可构造集,这就把$E$写作一族可构造集的交,于是得到(e).综上我们归结为设$X=\mathrm{Spec}A$本身是仿射的,$X'=\mathrm{Spec}A'$是仿射的,态射$f:X'\to X$满足$f(X')=E$,我们要证明的是$E$可以表示为$X$的一族可构造集的交.$f$对应了一个环同态$A\to A'$,我们先把$A'$写作有限型$A$代数的正向极限$\varinjlim A_i'$(这个正向系统的偏序就是包含序).再记$X_i'=\mathrm{Spec}A_i'$,那么$f$就可以分解为$X'\to X_i'\to X$,右侧的态射是结构环同态$A\to A_i'$诱导的态射,记作$f_i$.在环的正向极限里我们解释过此时有$f(X')=\cap_if_i(X_i')$.于是归结为证明$f_i(X_i')$是$X$的一族可构造集的交.用$X_i'$和$f_i$分别替代$X'$和$f$,我们不妨设$A'$本身已经是有限型$A$代数.我们在环的正向极限中还解释过一个有限型$A$代数可以写成有限表示$A$代数的正向极限,于是$A'=\varinjlim A_i'$,其中每个$A_i'$都是有限表示$A$代数.仍然记$X_i'=\mathrm{Spec}A_i'$,记$A\to A_i'$对应的态射是$f_i':X_i'\to X$,于是我们仍然有$f(X')=\cap_if_i(X_i')$.但是这回$f_i$是有限表示态射,按照Chevalley定理有$f_i(X_i')$是$X$的可构造集,从而我们把$f(X')$表示为$X$的一族可构造集的交.
		
		\qquad
		
		(e)推(d):设$\{X_i\mid i\in I\}$是$X$的仿射开覆盖,记$E_i=E\cap X_i$,再记$E$是$X$的一族可构造集$\{G_t\mid t\in T\}$的交.那么$E_i$就是$\{G_t\cap X_i\mid t\in T\}$的交,我们解释过可构造集在开子集上的限制是该开子集的可构造集,于是这里$G_t\cap X_i$就总是$X_i$的可构造集.我们要证明的是存在仿射概形为源端的态射$f_i:X_i'\to X_i$满足$f_i(X_i')=E_i$.所以我们不妨用$X_i',X_i,E_i,f_i$分别替代$X',X,E,f$,于是不妨设$X=\mathrm{Spec}A$本身是仿射的.我们之前已经证明过对于qcqs概形$X$,如果$Z$是$X$的可构造集,那么存在仿射概形为源端的态射使得它的集合像集就是$Z$.所以这里对每个指标$t\in T$,都存在$A$代数$A_t'$,使得结构同态$A\to A_t'$对应的态射$f_t:X_t'\to X$满足$f_t(X_t')=G_t$.我们要证明的是找到一个$A$代数$A'$,使得结构同态$A\to A'$对应的态射$f:X'=\mathrm{Spec}A'\to X$满足$f(X')=\cap G_t=E$.
		
		\qquad
		
		对$T$的有限子集$J$,我们记$A$代数$A_J'=\otimes_{t\in J}A_t'$和$X_J'=\mathrm{Spec}A_J'$(换句话讲$X_J'$就是$X_t',t\in J$的纤维积)以及结构态射$f_J:X_J'\to X$.我们先断言有$f_J(X_J')=\cap_{t\in J}f_t(X_t')$:这件事是因为一般的如果$X$是$S$概形$X_1,\cdots,X_n$的纤维积,那么存在$x\in X$满足它在$X_i$中的投影恰好是$x_i$当且仅当这些$x_i$在$S$中的像是相同的.接下来取$A'=\varinjlim A_J'$,其中$J$跑遍$T$的有限子集并且赋予包含序.再记$X'=\mathrm{Spec}A'$,记结构态射$f:X'\to X$,那么在环的正向极限中我们证明过有$f(X')=\cap_Jf_J(X_J')$,结合前一个断言就得到$f(X')=\cap_{t\in T}f_t(X_t')=\cap_tG_t=E$.综上得证.
	\end{proof}
	\item\label{pro-可构造集和ind-可构造集2} 基本性质.这些性质我们在后文会给出拓扑解释.设$X$是概形.
	\begin{enumerate}
		\item 一个子集$E$是$X$的pro-可构造集当且仅当$X-E$是$X$的ind-可构造集.
		\item pro-可构造集的有限并和任意交都是pro-可构造集;ind-可构造集的有限交和任意并都是ind-可构造集.
		\item $X$的任意一族可构造集的交都是pro-可构造集,$X$的任意一族可构造集的并都是ind-可构造集;反过来如果$X$是qcqs概形,那么$X$的pro-可构造集恰好是可以表示为一族可构造集的交的集合,$X$的ind-可构造集恰好是可以表示为一族可构造集的并的集合.
		\item 设$\{U_i\}$是$X$的开覆盖,那么一个子集$E$是$X$的pro-可构造集/ind-可构造集当且仅当对每个$i$都有$U_i\cap E$是$U_i$的pro-可构造集/ind-可构造集.
		\item $X$的pro-可构造集总是$X$的反紧子集,于是对于$X$的一个局部闭子集$E$,它在$X$中是反紧的当且仅当它是$X$的pro-可构造集.特别的,这说明闭子集总是pro-可构造集(闭子集总是反紧的),进而开子集总是ind-可构造集(而这实际上是说可构造拓扑包含了概形拓扑).
		\begin{proof}
			
			因为pro-可构造集和反紧都是局部性质,所以我们可以不妨设$X$是仿射的,特别的$X$是拟分离的,所以它的拟紧子集【】
			
			\qquad
			
			下面设$E$是$X$的反紧的局部闭子集,赋予$E$子概型结构,典范嵌入记作$j:E\to Y$,按照$E$是反紧的就有$j$是拟紧的,于是(g)告诉我们$j(E)=E$是$X$的pro-可构造集.
		\end{proof}
		\item 设$f:X'\to X$是概形之间的态射,对$X$的任意pro-可构造集/ind-可构造集$E$,都有$f^{-1}(E)$是$X'$的pro-可构造集/ind-可构造集.
		\item 设$f:X'\to X$是拟紧态射,则对$X'$的任意pro-可构造集$E'$,都有$f(E')$是$X$的pro-可构造集.特别的$f(X')$总是$X$的pro-可构造集.
		\begin{proof}
			
			我们要证明$f(E')$是$X$的pro-可构造集,因为问题是局部的,所以不妨设$X$是仿射的.则按照$f$拟紧就得到$X'$可以表示为有限个仿射开子集$\{X_i'\}$的并.于是$f(E')$可以表示为$\cup_if(E'\cap X_i')$,于是我们还可以设$X'$也是仿射的.但是按照qcqs概形上pro-可构造集的等价描述,存在仿射概形为源端的拟紧态射$g:X''\to X'$满足$g(X'')=E'$,进而有$f(E')=f(g(X''))$,这里$f\circ g$是拟紧态射,于是$f(E')$是$X$的可构造集.
		\end{proof}
		\item 设$f:X'\to X$是局部有限表示态射,设$E'$是$X'$的ind-可构造集,那么$f(E')$是$X$是ind-可构造集.特别的$f(X')$总是$X$的ind-可构造集.
		\begin{proof}
			
			按照ind-可构造集是局部性质,我们不妨设$X$是仿射的.按照$E'$是ind-可构造集,可取$X'$的仿射开覆盖$\{X_i'\}$,使得$E'\cap X_i'$是$X_i'$的一族可构造集的并.那么$f(E')=\cup_if(E'\cap X_i')$.这里$f$限制为$X_i'\to X$是有限表示的,于是按照Chevalley定理就有$f(E'\cap X_i')$是$X$的一族可构造集的并,进而$f(E')$是$X$的一族可构造集的并,于是它是ind-可构造集.
		\end{proof}
	\end{enumerate}
	\item\label{pro-可构造集和ind-可构造集3} 概形$X$的有限子集一定是pro-可构造集.但是单点子集一般也未必是ind-可构造集:取$X=\mathrm{Spec}\mathbb{Z}$,那么这是一个诺特整概形,取$x$是一般点,假设单点集$\{x\}$是ind-可构造集,那么它是一族可构造集的并,所以它本身是可构造集.但是我们解释过不可约诺特空间的稠密子集一定包含非空开集,这迫使$\{x\}$本身是$X$的开集,但这矛盾.
	\begin{proof}
		
		我们知道pro-可构造集的有限并还是pro-可构造集,所以归结为证明$X$的单点子集$\{x\}$总是pro-可构造集.考虑典范态射$\mathrm{Spec}\kappa(x)\to X$,这明显是一个拟紧态射,于是它的像集$\{x\}$是pro-可构造集.
	\end{proof}
	\item\label{pro-可构造集和ind-可构造集4} 如果$X$是诺特概形,则可构造集恰好是有限个局部闭子集的并,那么$X$的ind-可构造集恰好是一族局部闭子集的并.
	\item\label{pro-可构造集和ind-可构造集5} 
	\begin{enumerate}
		\item 设$X$是拟紧概形,设$\{F_i\mid i\in I\}$是$X$的一族pro-可构造集,并且它们的交是空集.那么存在$I$的非空有限子集$J$使得$\{F_i\mid i\in J\}$的交是空集.
		\begin{proof}
			
			按照$X$是拟紧的,可以取有限仿射开覆盖$\{X_t\mid t\in T\}$,一旦我们证明了$X$仿射的情况下结论成立,按照$\{X_t\cap F_i\mid i\in I\}$也是$X_t$的一族pro-可构造集,并且交也是空集,就存在$I$的有限子集$J_t$,使得$\{X_t\cap F_i\mid i\in J_t\}$的交是空集.则取$J=\cup_{t\in T}J_t$也是$I$的有限子集,并且$\{F_i\mid i\in I\}$的交集和任意$X_t$的交都是空集,从而$\{F_i\mid i\in I\}$的交集是空集.综上问题归结为设$X=\mathrm{Spec}A$本身是仿射的.
			
			\qquad
			
			按照仿射概形是qcqs概形,以及qcqs概形上pro-可构造集的等价描述,就存在$X_i=\mathrm{Spec}A_i'$和态射$f_i:X_i'\to X$满足$f_i(X_i')=F_i$.接下来对$I$的有限子集$J$,记$A$代数$\{A_i'\mid i\in J\}$的张量积为$A_J'$,再让$J$跑遍$I$的有限子集,赋予包含序,则$\{A_J'\}$构成一个正向系统,它的正向极限记作$A'$,再记$X'=\mathrm{Spec}A'$,记$X'_J=\mathrm{Spec}A_J'$,记结构同态$A\to A_J'$对应的态射为$f_J$,记结构同态$A\to A'$对应的态射为$f:X'\to X$.那么我们解释过(在证明qcqs概形上pro-可构造集等价描述那里)有$f(X')=\cap_Jf_J(X_J')$以及$f_J(X_J')=\cap_{i\in J}f_i(X_i')$.现在按照条件有$f(X')=\cap_{i\in I}f_i(X_i')=\cap_{i\in I}F_i$是空集,于是概形$X'$是空集.这导致它正向系统中某个分量$X_J'$是空集,于是这个$J$满足$f_J(X_J')=\cap_{i\in J}F_i$是空集.
		\end{proof}
		\item 设$X$是拟紧概形,设$E$是$X$的ind-可构造集,设$\{F_i\mid i\in I\}$是$X$的一族pro-可构造集,满足$\cap_{i\in I}F_i\subseteq E$,那么存在$I$的有限子集$J$满足$\cap_{i\in J}F_i\subseteq E$.
		\begin{proof}
			
			$F=X-E$是$X$的pro-可构造集.则每个$F\cap F_i$也是pro-可构造集,并且满足$\{F_i\cap F\mid i\in I\}$的交是空集,按照上一条就存在$I$的非空有限子集$J$使得$\{F_i\cap F\mid i\in J\}$的交是空集,换句话讲$\cap_{i\in J}F_i\subseteq E$.
		\end{proof}
		\item 对偶的,设$X$是拟紧概形,设$F$是$X$的pro-可构造集,设$\{E_i\mid i\in I\}$是$X$的一族ind-可构造集,满足$F\subseteq\cup_{i\in I}E_i$,那么存在$I$的有限子集$J$满足$F\subseteq\cup_{i\in J}E_i$.
	\end{enumerate}
	\item\label{pro-可构造集和ind-可构造集6} 设$X$是概形,如果$E$是$X$的ind-可构造集,则对任意$x\in X$,都有$E\cap\overline{\{x\}}$包含了$x$在$\overline{\{x\}}$中的开子集.如果$X$是局部诺特概形,那么这个命题甚至是充要的.
	\begin{proof}
		
		先设$E$是ind-可构造集,任取$x\in X$的仿射开子集,设它和$\overline{\{x\}}$的交为$Y$,则我们可以赋予$Y$上一个$X$的子概型结构(并且$Y$是仿射的,因为它是仿射概形的闭子概型),记对应的嵌入为$j:Y\to X$.那么$E\cap Y=j^{-1}(E)$是$Y$的ind-可构造集.按照$Y$是拟紧和分离的,就有$E\cap Y$是$Y$的一族可构造集的并.进而存在$Y$的两个开子集$U,V$使得$x\in U\cap(Y-V)\subseteq E\cap Y$.但是这里$x$是不可约空间$Y$的一般点,所以$Y$的任意非空开子集都要包含$x$,这迫使$V$是空集,进而有$U\subseteq E\cap Y$.
		
		\qquad
		
		下面设$X$是局部诺特概形,设$E$满足命题中的条件.因为我们的条件是局部的,ind-可构造集的条件也是局部的,所以归结为设$X$是诺特概形.按照条件对任意$x\in E$,都存在$X$的一个非空开集$U$满足$U\cap\overline{\{x\}}\subseteq E\cap\overline{\{x\}}$.这说明$E$是一族局部闭子集的并.但是诺特条件下局部闭子集是可构造集,于是$E$是ind-可构造集.
	\end{proof}
	\item\label{pro-可构造集和ind-可构造集7} 设$X$是概形,设$E\subseteq X$,那么如下命题互相等价:
	\begin{enumerate}
		\item $E$是可构造集.
		\item $E$同时是ind-可构造集和pro-可构造集.
		\item $E$和$X-E$都是pro-可构造集.
		\item $E$和$X-E$都是ind-可构造集.
	\end{enumerate}
	\begin{proof}
		
		这里(b),(c),(d)的等价性以及(a)推出它们是平凡的.最后只需证明(d)推(a):问题是局部的,我们可设$X$是仿射概形,那么$E$就是一族$X$的可构造集$\{E_i\mid i\in I\}$的并,$X-E$也是一族$X$的可构造集$\{E_j'\mid j\in J\}$的并.那么$\{E_i,E_j'\}$就覆盖了整个$X$.按照前面定理,存在$I$和$J$的有限子集$I_0$和$J_0$,使得$\{E_i,E_j'\mid i\in I_0,j\in J_0\}$构成$X$的覆盖.特别的$E=\cup_{i\in I_0}E_i$,从而$E$是可构造集.
	\end{proof}
	\item\label{pro-可构造集和ind-可构造集8} 我们解释过只要$f:X\to Y$是概形之间的态射,那么可构造集/pro-可构造集/ind-可构造集的原像就还是可构造集/pro-可构造集/ind-可构造集.这里我们断言如果$f:X\to Y$是一个满射态射,并且要么是拟紧的要么是局部有限表示的,那么上述逆命题也成立.
	\begin{proof}
		
		我们要证明的是如果$Y$的子集$E$满足$f^{-1}(E)$是$X$的可构造集/pro-可构造集/ind-可构造集,那么$E$是$Y$的可构造集/pro-可构造集/ind-可构造集.又因为我们解释过同时是ind-可构造集和pro-可构造集的子集是可构造集,所以归结为证明pro-可构造集和ind-可构造集的情况,又因为$Y-f^{-1}(E)=f^{-1}(X-E)$.所以问题归结为证明$f$是pro-可构造集和ind-可构造集的任意一个.
		
		\qquad
		
		下面先设$f$是满射和拟紧的态射,那么我们解释过从$f^{-1}(E)$是pro-可构造集得到$E=f\circ f^{-1}(E)$是pro-可构造集;再设$f$是满射和局部有限表示态射,那么我们解释过从$f^{-1}(E)$是ind-可构造集得到$E=f\circ f^{-1}(E)$是ind-可构造集.这就得证.
	\end{proof}
\end{enumerate}
\subsection{可构造拓扑}

设$X$是概形,我们解释过它的ind-可构造集在有限交和任意并下不变,于是ind-可构造集满足开集公理,这个拓扑称为概形$X$的可构造拓扑(constructible topology).对概形$X$,我们用$\mathscr{T}$表示它的概形拓扑,用$\mathscr{T}^{\mathrm{cons}}$表示它的可构造拓扑,用$X^{\mathrm{cons}}$表示$X$的底集合上赋予可构造拓扑得到的拓扑空间.
\begin{enumerate}
	\item $\mathscr{T}\subseteq\mathscr{T}^{\mathrm{cons}}$,换句话讲可构造拓扑细于概形拓扑.这件事是因为我们解释了闭子集都是pro-可构造集.
	\item 可构造集就是可构造拓扑中的既开又闭子集.因为我们解释过同时为pro-可构造集和ind-可构造集的子集恰好是可构造集.
	\item 设$f:X\to Y$是概形之间的态射,那么它作为集合映射也是$X^{\mathrm{cons}}\to Y^{\mathrm{cons}}$的连续映射,此时我们把它记作$f^{\mathrm{cons}}$.
	\item 如果$f:X\to Y$是拟紧态射,那么$f^{\mathrm{cons}}$是闭映射.特别的如果$f$是拟紧双射,那么$f^{\mathrm{cons}}$是同胚.
	\item 如果$f:X\to Y$是局部有限表示态射,那么$f^{\mathrm{cons}}$是开映射.特别的如果$f$局部有限表示双射,那么$f^{\mathrm{cons}}$是同胚.
	\item 设$X$是概形,设$U\subseteq X$是概形拓扑的开子集,那么我们解释过$U$也是$X$的可构造拓扑的开子集,把$U$视为$X^{\mathrm{cons}}$的开子空间的拓扑和把$U$视为开子概型所对应的$U^{\mathrm{cons}}$的拓扑是一致的.
	\begin{proof}
		
		设$j:U\to X$是开嵌入,那么这是局部有限表示态射,于是$j^{\mathrm{cons}}:U^{\mathrm{cons}}\to X^{\mathrm{cons}}$是连续开映射而且是单射.从而$V\subseteq U$是$U^{\mathrm{cons}}$是开子集当且仅当$j(V)=V$是$X^{\mathrm{cons}}$的开子集.
	\end{proof} 
	\item $X$是拟紧概形当且仅当$X^{\mathrm{cons}}$是拟紧空间.这件事的必要性我们已经证明过了,而充分性只是因为可构造拓扑细于概形拓扑.
	\item $X$是拟分离概形当且仅当$X^{\mathrm{cons}}$是Hausdorff的,并且在条件成立时有$X^{\mathrm{cons}}$是局部紧和全不连通的.局部紧空间指的是Hausdorff空间并且存在拟紧开集构成的拓扑基,我们今后提到紧总是指拟紧和Hausdorff的.
	\begin{proof}
		
		先设$X$是拟分离的,我们先证明对$X$中的任意两个不同点$x,y$,都存在子集$U$,使得$U$和$X-U$都不是空集,分别包含点$x,y$,并且它们都是可构造集.这件事同时说明$X^{\mathrm{cons}}$是Hausdorff的和全不连通空间(后者是因为这件事说明$X^{\mathrm{cons}}$的连通分支不会包含多于一个点).按照概形是$T_0$空间,不妨设存在$X$的仿射开子集$U$,满足包含了$x$但不包含$y$.由于$X$是拟分离的,于是$U$是$X$的反紧开集,进而$U$和$X-U$都是$X$的可构造集.
		
		\qquad
		
		再证明$X^{\mathrm{cons}}$是局部紧的:任取$X$的仿射开子集$U$,那么$U$视为$X^{\mathrm{cons}}$的开子空间和$U$作为开子概型诱导的$U^{\mathrm{cons}}$的拓扑是一致的.但是$U$是拟紧的导致$U^{\mathrm{cons}}$是拟紧的,进而$U$视为$X^{\mathrm{cons}}$的开子空间也是拟紧的,这说明$X^{\mathrm{cons}}$是局部紧的.
		
		\qquad
		
		最后只需证明如果$X^{\mathrm{cons}}$是Hausdorff的,那么$X$是拟分离概形:取$X$的仿射开子集$U$,我们已经解释过$U$视为$X^{\mathrm{cons}}$的开子空间是拟紧的,结合Hausdorff就得到它是紧的.进而有$U^{\mathrm{cons}}$也是紧的.下面取第二个仿射开子集$V\subseteq X$.因为Hausdorff空间的两个拟紧子集的交总是拟紧的(Hausdorff空间的拟紧子集是闭的,而拟紧子集的闭子集还是拟紧的),于是$U\cap V$视为$X^{\mathrm{cons}}$的开子集是拟紧的,于是$(U\cap V)^{\mathrm{cons}}$也是拟紧的.接下来考虑集合上的恒等映射$(U\cap V)^{\mathrm{cons}}\to U\cap V$(后者是概形拓扑的子空间),按照可构造拓扑细于概形拓扑,这个恒等映射是连续的,但是拟紧空间的连续像是拟紧的,这就说明$U\cap V$作为概形拓扑的开集是拟紧的.
	\end{proof}
	\item $X$是qcqs概形当且仅当$X^{\mathrm{cons}}$是紧空间(此为点集拓扑中的紧致Hausdorff空间).
	\item 对任意概形$X$,有$X^{\mathrm{cons}}$的每个点都存在开邻域是紧的.这件事是因为仿射开邻域总是可构造拓扑的开集,而仿射开集本身是qcqs概形,对应的可构造拓扑就是紧的.
	\item 概形之间的态射$f:X\to Y$是拟紧的,当且仅当$f^{\mathrm{cons}}:X^{\mathrm{cons}}\to Y^{\mathrm{cons}}$是proper映射,此为拟紧子集的原像总是拟紧的连续映射.
	\begin{proof}
		
		先设$f$是拟紧的,那么我们解释过$f^{\mathrm{cons}}$是闭映射.任取$y\in Y$,按照$f$是拟紧的,并且拟紧保基变换,我们得到$Z=f^{-1}(y)=X\times_Y\mathrm{Spec}\kappa(y)$是拟紧的.另外按照$\mathrm{Spec}\kappa(y)\to Y$是泛单射,就有纤维积的结构态射$p:Z\to X$也是单射.由于$f^{-1}(y)$视为$X$的子空间是拟紧的,就得到$f^{-1}(y)$视为$X^{\mathrm{cons}}$的子空间也是拟紧的.一个点集拓扑事实是$f$是proper映射当且仅当每个点的纤维都是拟紧的.
		
		\qquad
		
		下面设$f^{\mathrm{cons}}$是proper映射.设$V\subseteq Y$是拟紧开子集,记典范包含态射$h:V\to Y$,那么$h^{\mathrm{cons}}:V^{\mathrm{cons}}\to Y^{\mathrm{cons}}$也是连续单射,并且这里$V^{\mathrm{cons}}$也是拟紧的.于是$V$视为$Y^{\mathrm{cons}}$的开子空间也是拟紧的.于是按照$f^{\mathrm{cons}}$是proper映射,就有$f^{-1}(V)$视为$X^{\mathrm{cons}}$的开子空间是拟紧的.于是$f^{-1}(V)$视为$X$的开子空间也是拟紧的,这就说明$f$是拟紧态射.
	\end{proof}
\end{enumerate}
\subsection{应用在开映射上}

\begin{enumerate}
	\item 设$X$是概形,设$E$是$X$的ind-可构造集,设$x\in E$,那么$x$是$E\subseteq X$的内点当且仅当$x$的任意一般化都在$E$中,也即$\mathrm{Spec}\mathscr{O}_{X,x}\subseteq E$.
	\begin{proof}
		
		必要性是平凡的,因为开集保一般化.下面证明充分性,因为条件和问题都是局部的,于是可不妨设$X=\mathrm{Spec}A$是仿射的,那么$x$就对应于$A$的一个素理想$\mathfrak{p}$.按照qcqs概形上pro-可构造集的等价描述,存在仿射概形$X'=\mathrm{Spec}A'$和态射$f:X'\to X$,满足$f(X')=X-E$.再记$Y=\mathrm{Spec}\mathscr{O}_{X,x}$和$Y'=X'\times_XY=f^{-1}(x)$.但是按照$x$不在$f(X')=X-E$中,就导致$Y'$是空集.于是$A'\otimes_AA_{\mathfrak{p}}$是零环.我们有$A_{\mathfrak{p}}=\varinjlim A_s$,其中$s$跑遍$A-\mathfrak{p}$中的元.按照正向极限和张量积可交换,于是$\varinjlim A_s'=0$,于是存在某个$t\in A-\mathfrak{p}$使得$A_t'=0$,也即$D(t)$中的素理想都不在$f$的集合像集中,也即$D(t)\subseteq E$,完成证明.
	\end{proof}
	\item 推论.设$f:X\to Y$是局部有限表示态射,设$y\in Y$,那么$y$是$f(X)$的内点当且仅当$y$的所有一般化都在$f(X)$中.这件事是因为我们解释过如果$f$是局部有限表示态射,那么对$X$的任意ind-可构造集$E$,都有$f(E)$是$Y$的ind-可构造集,再结合上一条.
	\item 设$f:X\to Y$是概形之间的态射,设$x\in X$,记$y=f(x)$.考虑如下条件:
	\begin{enumerate}
		\item $f$在点$x$处开,这是指$f$把$x\in X$的任意邻域都映为$y$的邻域.这里邻域不止指的是开邻域,$X$的子集$E$称为点$x$的邻域,如果$x$是$E\subseteq X$的内点,也即$E$包含了$x$在$X$中的一个开邻域.
		\item 对$y$的任意一般化$y'$,都存在$x$的一般化$x'$满足$f(x')=y'$.
		\item $f(\mathrm{Spec}\mathscr{O}_{X,x})=\mathrm{Spec}\mathscr{O}_{Y,y}$.
		\item 对$Y$的任意一个包含$y$的不可约闭子集$Y'$,都存在$X'=f^{-1}(Y')$的一个不可约分支包含了$x$并且支配了$Y'$.
	\end{enumerate}
	
	我们有(a)$\Rightarrow$(b)$\Leftrightarrow$(c)$\Leftrightarrow$(d).并且如果$f$是局部有限表示态射,那么这些命题都是等价的.
	\begin{proof}
		
		首先(b)和(c)的等价性就是$\mathrm{Spec}\mathscr{O}_{X,x}$典范的视为$X$的子集就是由$x$的一般化构成的子集.证明(d)推(b):任取$y$的一般化$y'$,记$Y=\overline{\{y'\}}$,则这是包含$y$的不可约闭子集,于是按照条件存在$X'=f^{-1}(Y')$的一个不可约分支满足包含$x$并且支配$Y'$,设这个不可约分支的一般点为$x'$,那么$x'$就是$x$的一般化,而这个不可约分支支配$Y'$导致$f(x')=y'$,这就得到(b).反过来证明(b)推(d):设$Y'$是$Y$的一个包含$y$的不可约闭子集,设一般点为$y'$,于是$y'$就是$y$的一般化,按照条件就可以找到$x$的一般化$x'$满足$f(x')=y'$.设$Z'$是$f^{-1}(Y')$的包含点$x'$的不可约分支,那么$Z'$页包含了点$x$,另外按照$f(x')$是$Y'$的一般点,就有$f(Z')$支配了$Y'$.综上我们证明了(b),(c),(d)的等价性.
		
		\qquad
		
		(a)推(c):因为开集保一般化,所以我们不妨取$y$的仿射开邻域$V=\mathrm{Spec}A$,和$x$的仿射开邻域$U=\mathrm{Spec}B$,满足$f(U)\subseteq V$,用$U,V$分别替代$X,Y$.那么$x$就对应$B$的一个素理想$\mathfrak{q}$.我们有$\mathscr{O}_{X,x}=B_{\mathfrak{q}}=\varinjlim B_s$,其中$s$跑遍$B-\mathfrak{q}$中的元.那么在环的正向极限中我们解释过有$f(\mathrm{Spec}\mathscr{O}_{X,x})=\cap_sf(\mathrm{Spec}B_s)=\cap_sf(D(s))$.按照条件,$y$总是$f(D(s))$的内点,于是按照开集保一般化就有$\mathrm{Spec}\mathscr{O}_{Y,y}\subseteq f(D(s))$,进而有$\mathrm{Spec}\mathscr{O}_{Y,y}\subseteq f(\mathrm{Spec}\mathscr{O}_{X,x})$.而另一侧的包含关系是平凡的,因为连续映射就保一般化和特殊化.
		
		\qquad
		
		最后如果$f$是局部有限表示态射,我们来证明(b)推(a):设$U$是$x$的开邻域,那么$f$限制为$U\to Y$也是局部有限表示的(因为它是局部有限表示态射复合上开嵌入$U\to X$,而开嵌入总是局部有限表示的),于是如果(b)成立,那么$y$的一般化都在$f(U)$中,于是按照上一条得到$y$是$f(U)\subseteq Y$的内点,也即$f(U)$是$y$的邻域,于是(a)成立.
	\end{proof}
	\item 推论.设$f:X\to Y$是概形之间的态射,考虑如下命题:
	\begin{enumerate}
		\item $f$是开映射.
		\item 对任意$x\in X$,记$y=f(x)$,对$y$的任意一般化$y'$,都存在$x$的一般化$x'$满足$f(x')=y'$.
		\item 对任意$x\in X$都有$f(\mathrm{Spec}\mathscr{O}_{X,x})=\mathrm{Spec}\mathscr{O}_{Y,y}$.
		\item 对$Y$的任意不可约闭子集$Y'$,有$f^{-1}(Y')$的任意不可约分支都支配了$Y'$.
	\end{enumerate}
	
	我们有(a)$\Rightarrow$(b)$\Leftrightarrow$(c)$\Leftrightarrow$(d).并且如果$f$是局部有限表示态射,那么这些命题都是等价的.
	\begin{proof}
		
		几乎所有内容都是因为$f$是开映射当且仅当$f$在$X$的每个点都是开的.我们只来证明下(b)推(d):任取$f^{-1}(Y')$的一般点$x'$,记$y'=f(x')$,我们要证明的是$y'$恰好是$Y'$的一般点.假设$y'$还有一般化$y''\in Y'$,按照条件就存在$x'$的一般化$x''$满足$f(x'')=y''$.但是这个$x''\in f^{-1}(Y')$,就导致$x''=x'$,于是$y''=y'$,这就得到$y'$是$Y'$的一般点.
	\end{proof}
\end{enumerate}
\newpage
\section{概形的逆向极限}
\subsection{环的正向极限}

设$B$是环,设$\{(A_i)_{i\in I},(\varphi_{ji}:A_i\to A_j)_{i\le j}\}$是$B$代数范畴上的正向系统(这涵盖了环范畴上正向系统的情况,只要取$B=\mathbb{Z}$).记$B$代数范畴中的正向极限$A=\varinjlim A_i$,记正向极限的典范同态$\varphi_i:A_i\to A$(这个正向极限总是存在的,因为$B$代数范畴是余完备的).记$X_i=\mathrm{Spec}A_i$和$X=\mathrm{Spec}A$,再记典范环同态$\varphi_i:A_i\to A$对应的概形态射是$f_i:X\to X_i$,记$\varphi_{ji}$对应的态射是$f_{ji}:X_j\to X_i$.再记作为$B$代数的结构同态$B\to A_i$和$B\to A$对应的态射为$u_i:X_i\to Y$和$u:X\to Y$,其中$Y=\mathrm{Spec}B$.
\begin{enumerate}
	\item\label{环的正向极限1} $A=0$当且仅当存在指标$i$使得$A_i=0$(此时对任意$j\ge i$都有$A_j=0$),换句话讲$X=\emptyset$当且仅当存在指标$i$使得$X_i=\emptyset$(此时对任意$j\ge i$都有$X_j=\emptyset$).
	\begin{proof}
		
		充分性是直接的.对于必要性,我们记典范同态$\varphi_i:A_i\to A$,那么从$A=0$得到$\varphi_i(1)=0=\varphi_i(0)$,这等价于讲存在$j\ge i$使得$\varphi_{ji}:A_i\to A_j$满足$\varphi_{ji}(1)=\varphi_{ji}(0)$,这迫使$A_j$中的零元和幺元相同,也即$A_j$是零环.
	\end{proof}
	\item\label{环的正向极限2} $(X,f_i)$就是$\mathrm{Spec}B$概形范畴上的逆向系统$\{(X_i)_{i\in I},(f_{ji}:X_j\to X_i)_{i\le j}\}$的极限.
	\begin{proof}
		
		我们要证明的是对任意概形$T$有如下典范双射:
		$$\mathrm{Hom}_{\textbf{Sch}(Y)}(T,X)\cong\varprojlim\mathrm{Hom}_{\textbf{Sch}(Y)}(T,X_i)$$
		
		但是按照终端为仿射概形的态射的描述,这等价于有如下典范双射,而这是环的正向极限的定义:
		$$\varinjlim\mathrm{Hom}_{\textbf{Alg}(B)}(A_i,\Gamma(T,\mathscr{O}_T))\cong\mathrm{Hom}_{\textbf{Alg}(B)}(A,\Gamma(T,\mathscr{O}_T))$$
	\end{proof}
	\item\label{环的正向极限3} 我们断言有$u(X)=\cap_{i\in I}u_i(X_i)$.
	\begin{proof}
		
		首先按照$u=f_i\circ u_i$,说明$u(X)\subseteq u_i(X_i)$对任意指标$i$成立,于是有$u(X)\subseteq\cap_{i\in I}u_i(X_i)$.接下来设$y\in Y-u(X)$,我们只要证明$y\not\in\cap_{i\in I}u_i(X_i)$即可.按照$u^{-1}(y)$是空集,得到$\mathrm{Spec}(A\otimes_B\kappa(y))$是空集.按照模范畴上张量积和正向极限可交换(正向极限是正合的),就有$A\otimes_B$是$A_i\otimes_B\kappa(y)$的正向极限.但是这个正向极限的素谱是空集,就说明存在指标$i$使得$A_i\otimes_B\kappa(y)$是零环,换句话讲$u_i^{-1}(y)$是空集,于是$y$不在$u_i(X_i)$中,这就得证.
	\end{proof}
	\item\label{环的正向极限4} 设$A$是环,设$A'$是有限型$A$代数,那么存在有限表示$A$代数构成的正向系统$\{A_i''\mid i\in I\}$,使得有$A$代数同构$A'\cong\varinjlim A_i''$.
	\begin{proof}
		
		我们可以记$A'=B/J$,其中$B=A[T_1,\cdots,T_n]$,而$J\subseteq B$是理想.那么$J$可以表示为它的有限生成子理想构成的正向系统(以包含序为偏序)的正向极限$J=\varinjlim J_i$.按照模范畴上正向极限是正合的,于是有$A'=B/J=\varinjlim B/J_i$,这里$B/J_i$都是有限表示$A$代数,这就得证.
	\end{proof}
    \item\label{环的正向极限5} 关于有限表示模和有限表示代数.设$\{(A_i)_{i\in I},(\varphi_{ji}:A_i\to A_j)_{1\le j}\}$是环范畴上的正向系统,约定指标集是一个有向集,记正向极限为$\{A,(\varphi_i:A_i\to A)\}$.
    \begin{enumerate}[(1)]
    	\item 设$M$是有限表示$A$模,那么存在指标$i$和一个有限表示$A_i$模$M_i$,使得有$A$模同构$M\cong M_i\otimes_{A_i}A$,进而对任意$j\ge i$,取$M_j=M_i\otimes_{A_i}A_j$也是有限表示$A_j$模,那么总有$A$模同构$M\cong M_j\otimes_{A_j}A$.
    	\begin{proof}
    		
    		记$M=A^r/P$,其中$P$是$A^r$的有限子模.我们有$A^r=\varinjlim(A_i^r)$.按照$P$是有限生成的,我们可以取一个指标$i$,使得存在$A_i^r$的有限子模$P_i$,使得有满同态$P_i\otimes_{A_i}A\to P$.于是按照张量积的有正合性,我们有如下正合列之间的交换图表,从这里$\alpha,\beta$是满射得到$\beta$还是同构,于是$M\cong M_i\otimes_{A_i}A$.
    		$$\xymatrix{P_i\otimes_{A_i}A\ar[r]\ar[d]_{\alpha}&A^r\ar[r]\ar@{=}[d]&(A_i^r/P_i)\otimes_{A_i}A\ar[r]\ar[d]^{\beta}&0\\P\ar[r]&A^r\ar[r]&A^r/P\ar[r]&0}$$
    	\end{proof}
        \item 设$B$是有限表示$A$代数,那么存在指标$i$和一个有限表示$A_i$代数$B_i$,使得有$A$代数同构$B\cong B_i\otimes_{A_i}A$,进而对任意$j\ge i$,去$B_j=B_i\otimes_{A_i}A_j$也是有限表示$A_j$代数,那么总有$A$代数同构$B\cong B_j\otimes_{A_j}A$.
        \begin{proof}
        	
        	记$B=A[T_1,\cdots,T_n]/(F_1,\cdots,F_m)$.我们可以选取一个足够大的指标$i$使得每个$F_s$的系数都落在$\varphi_i(A_i)$中.换句话讲,存在$F_{si}\in A_i[T_1,\cdots,T_n]$使得$F_s=\varphi_i(F_{si}),\forall1\le s\le m$.记$A_i$的由$\{F_{si},1\le s\le m\}$生成的理想为$J_i$,记$A$的由$\{F_s,1\le s\le m\}$生成的理想为$J$.于是$J_i\otimes_{A_i}A=J$,于是取$B_i=A_i[T_1,\cdots,T_n]/J_i$满足要求.
        \end{proof}
    \end{enumerate}
    \item\label{环的正向极限6} 环的正向极限和理想.设$\{(A_i)_{i\in I},(\varphi_{ji}:A_i\to A_j)_{1\le j}\}$是环范畴上的正向系统,约定指标集是一个有向集,记正向极限为$\{A,(\varphi_i:A_i\to A)\}$.
    \begin{enumerate}
    	\item 对每个指标$i$,取$A_i$的理想$\mathfrak{a}_i$,满足对任意$i\le j$都有$\varphi_{ji}(\mathfrak{a}_i)\subseteq\mathfrak{a}_j$.那么$\{(\mathfrak{a}_i),(\varphi_{ji}\mid_{\mathfrak{a}_i})\}$构成集合范畴上的正向系统,它的正向极限$\mathfrak{a}=\varinjlim\mathfrak{a}_i$视为$A$的子集时是一个理想,并且有$A/\mathfrak{a}=\varinjlim(A_i/\mathfrak{a}_i)$.如果对任意$i\le j$总有$\mathfrak{a}_i=\varphi_{ji}^{-1}(\mathfrak{a}_j)$,那么对任意指标$i$总有$\mathfrak{a}_i=\varphi_i^{-1}(\mathfrak{a})$.
    	\item 反过来,对$A$的任意理想$\mathfrak{a}$,记$\mathfrak{a}_i=\varphi_i^{-1}(\mathfrak{a})$,那么$\{\mathfrak{a}_i\}$构成正向系统,并且有$\mathfrak{a}=\varinjlim\mathfrak{a}_i$.
    \end{enumerate}
    \begin{proof}
    	
    	明显的(b)是(a)的特殊情况.所以我们只需证明(a).我们知道环范畴到阿贝尔群范畴的遗忘函子以及阿贝尔群范畴到集合范畴的遗忘函子和正向极限可交换(这依赖于指标集是有向集),据此容易验证$\mathfrak{a}$的确是$A$的理想,并且的确有$A/\mathfrak{a}=\varinjlim(A_i/\mathfrak{a}_i)$.下面证明最后一个命题,如果$x_i\in A_i$满足$\varphi_i(x_i)\in\mathfrak{a}$,按照$\mathfrak{a}=\cup_i\varphi_i(\mathfrak{a}_i)$就能找到一个$j\ge i$和一个$y_j\in\mathfrak{a}_j$,满足$\varphi_i(x_i)=\varphi_j(y_j)$.于是$\varphi_j(\varphi_{ji}(x_i))=\varphi_j(y_j)$,但是按照正向极限的性质,我们可以找到指标$k\ge j$,使得$\varphi_{kj}(\varphi_{ji}(x_i))=\varphi_{kj}(y_j)$,于是$\varphi_{ki}(x_i)\in\mathfrak{a}_k$,于是按照条件$\mathfrak{a}_i=\varphi_{ki}^{-1}(\mathfrak{a}_k)$就有$x_i\in\mathfrak{a}_i$.
    \end{proof}
    \item\label{环的正向极限7} 环的正向极限和素理想.依旧设$\{(A_i)_{i\in I},(\varphi_{ji}:A_i\to A_j)_{1\le j}\}$是环范畴上的正向系统,约定指标集是一个有向集,记正向极限为$\{A,(\varphi_i:A_i\to A)\}$.
    \begin{enumerate}
    	\item 对任意指标$i$,取$A_i$的素理想$\mathfrak{p}_i$,满足对任意$i\le j$,都有$\varphi_{ji}(\mathfrak{p}_i)\subseteq\mathfrak{p}_j$.那么$\mathfrak{p}=\varinjlim\mathfrak{p}_i$是$A$的素理想,进而有$A/\mathfrak{p}=\varinjlim(A_i/\mathfrak{p}_i)$.特别的,这说明$A_i$是整环可以推出正向极限$A$也是整环.
    	\item 接上一条,如果对任意$i\le j$总有$\mathfrak{p}_i=\varphi_{ji}^{-1}(\mathfrak{p}_j)$,则有$A_{\mathfrak{p}}=\varinjlim(A_i)_{\mathfrak{p}_i}$.
    	\item 如果$(A,\mathfrak{p})$是局部环,如果每个$\varphi_i:A_i\to A$都是单射,记$\mathfrak{p}_i=\varphi_i^{-1}(\mathfrak{p})$,那么有$A=\varinjlim(A_i)_{\mathfrak{p}_i}$,并且同态$(A_i)_{\mathfrak{p}_i}\to A$总是单射.
    \end{enumerate}
    \begin{proof}
    	
    	(a)只需证明如果$A_i$都是整环,那么正向极限$A$也是整环.设$x,y\in A$满足$xy=0$,那么按照有向集的条件,我们可以找到同一个指标$i$,以及$x_i,y_i\in A_i$,使得$x=\varphi_i(x_i)$和$y=\varphi_i(y_i)$.那么有$\varphi_i(x_iy_i)=0$,于是存在指标$j\ge i$,使得$\varphi_{ji}(x_iy_i)=0$.如果记$x_j=\varphi_{ji}(x_i)$和$y_j=\varphi_{ji}(x_j)$,那么有$x_jy_j=0$.但是按照$A_j$是整环,我们可以不妨设$x_j=0$,于是$x=\varphi_j(x_j)=0$,这说明$A$是整环.
    	
    	\qquad
    	
    	下面证明(b):首先对任意$i\le j$有$\mathfrak{p}_i=\varphi_{ji}^{-1}(\mathfrak{p}_j)$保证了$\{(A_i-\mathfrak{p}_i)_{i\in I},(\varphi_{ji})\}$构成了一个正向系统,并且它的正向极限就是$A-\mathfrak{p}$.进而有$\{(A_i)_{\mathfrak{p}_i}\}$构成一个正向系统,并且该正向系统中的同态都是局部环之间的局部同态.典范同态$(A_i)_{\mathfrak{p}_i}\to A_{\mathfrak{p}}$就构成了这个正向系统的一个余锥,进而有同态$\psi:\varinjlim(A_i)_{\mathfrak{p}_i}\to A_{\mathfrak{p}}$,并且这是一个满射.
    	
    	\qquad
    	
    	我们只需再证明$\psi$是单射:设有$x_i\in A_i$和$s_i\in A_i-\mathfrak{p}_i$使得$\varphi_i(x_i)/\varphi_i(s_i)$在$A_{\mathfrak{p}}$中为零,于是存在$s'\in A-\mathfrak{p}$使得$s'\varphi_i(x_i)=0$.但是我们知道$A-\mathfrak{p}=\varinjlim(A_i-\mathfrak{p}_i)$,所以把$i$替换为一个足够大的指标,可以找到$s_i'\in A_i-\mathfrak{p}_i$,满足$\varphi_i(s_i'x_i)=0$,于是存在指标$j\ge i$使得$\varphi_{ji}(s_i'x_i)=0$.进而在$(A_j)_{\mathfrak{p}_j}$中就有$\varphi_{ji}(x_i)/\varphi_{ji}(s_i)=0$,这说明$x_i/s_i$在$\varinjlim(A_i)_{\mathfrak{p}_i}$中的像为零,也即$\psi$是单射.
    	
    	\qquad
    	
    	证明(c):第一个断言我们只要把(b)用在$A=A_{\mathfrak{p}}$上.这里条件还要求$A_i\to A$是单射,于是$A_i-\mathfrak{p}_i\subseteq A-\mathfrak{p}$是$A$的可逆元,于是总有$A_{\mathfrak{p}_i}=A$.于是从$A_i\to A$是单射,以及分式化是正合的,得到$(A_i)_{\mathfrak{p}_i}\to A_{\mathfrak{p}_i}=A$是单射.
    \end{proof}
    \item\label{环的正向极限8} $\{A_i\}$的性质传递给正向极限$A$.我们始终约定指标集是有向集.
    \begin{enumerate}[(1)]
    	\item 关于既约环.设$\mathfrak{R}_i$是$A_i$的幂零根,那么$\mathfrak{R}=\varinjlim\mathfrak{R}_i$是$A=\varinjlim A_i$的幂零根.进而有$A_{\mathrm{red}}=\varinjlim(A_i)_{\mathrm{red}}$.特别的,如果$A_i$都是既约环,那么它的正向极限$A$也是既约环.
    	\begin{proof}
    		
    		按照我们给出的环的正向极限和理想的关系.一切结论归结于证明$\mathfrak{R}=\varinjlim\mathfrak{R}_i$是$A$的幂零根.我们知道环范畴上的正向极限是这些环的无交并上赋予一个等价关系的等价类集合,据此得到$\mathfrak{R}$一定落在$A$的幂零根中.反过来任取$A$的幂零元$x$,设$x^n=0$,可记有指标$i$和$x_i\in A_i$使得$x=\varphi_i(x_i)$,于是有$\varphi_i(x_i^n)=0$,进而存在指标$j\ge i$满足$\varphi_{ji}(x_i^n)=0$,换句话讲$x_j=\varphi_{ji}(x_i)$满足$x_j^n=0$,于是$x_j\in\mathscr{R}_j$,于是$x=\varphi_j(x_j)$落在$A$的幂零根中(环同态$f:A\to B$当然把$A$的幂零元映为$B$的幂零元).
    	\end{proof}
        \item 关于整环.如果每个$A_i$都是整环,那么正向极限$A$也是整环.
        \item 关于域.如果每个$A_i$都是域,那么正向极限$A$也是域.这件事是在环的正向极限和素理想(b)中取$\mathfrak{p}_i$是$A_i$的零理想.
        \item 关于局部环.如果每个$(A_i,\mathfrak{m}_i,\kappa_i)$都是局部环,并且$\varphi_{ji}:A_i\to A_j$是局部环同态,那么正向极限$A$也是局部环,极大理想就是$\mathfrak{m}=\varinjlim\mathfrak{m}_i$,剩余域就是$\kappa=\varinjlim\kappa_i$.
        \begin{proof}
        	
        	我们只需证明第一个命题,剩余域的部分已经在环的正向极限和理想那里证明过了.按照$\varphi_{ji}$是局部环同态,得到$\{\mathfrak{m}_i,\varphi_{ji}\}$构成一个正向系统,记它的正向极限是$\mathfrak{m}$,那么我们已经解释过这是$A$的理想,现在任取$s\in A-\mathfrak{m}$,那么存在指标$i$和$s_i\in A_i$使得$\varphi_i(s_i)=s$.如果$s_i\in\mathfrak{m}_i$,导致$s=\varphi_i(s_i)\in\mathfrak{m}$矛盾,于是$s_i\in A_i-\mathfrak{m}_i$,但是$A_i$是局部环,于是$s_i$是$A_i$的单位,于是$s=\varphi_i(s_i)$是$A$的单位,综上我们证明了$A-\mathfrak{m}$中的元都是$A$的单位.反过来任取$A$的单位$s$,那么存在$t\in A$使得$st=1$,那么存在指标$i$和$s_i,t_i\in A_i$满足$s_it_i=1$.但是如果$s$落在$\mathfrak{m}$中,把$i$替换为更大的指标,会导致$s_i$落在$\mathfrak{m}_i$中,但是这就和$s_i$是$A_i$的单位矛盾.于是$A$的单位一定不在$\mathfrak{m}$中,换句话讲$A-\mathfrak{m}$恰好由$A$的单位构成,于是$A$是以$\mathfrak{m}$为极大理想的局部环.	
        \end{proof}
        \item  关于整闭包.设$\{A_i,\varphi_{ji}\}$是由整环构成的环的正向系统,并且$\varphi_{ji}$总是单射.如果记$A_i$的整闭包是$A_i'$,那么$\{A_i',\varphi_{ji}'\}$构成了环的正向极限,并且$A'=\varinjlim A_i'$恰好就是$A$的整闭包.特别的,如果额外的$A_i$都是整闭整环,那么$A$也是整闭整环.
        \begin{proof}
        	
        	设$A_i$的商域为$K_i=(A_i)_{(0)}$,我们解释了$K=\varinjlim K_i$是整环$A$的商域.那么$A'=\varinjlim A_i'$就是$K$的子环,并且按照环上正向极限描述为等价类,$A'$明显在$A$上整.反过来如果$z\in K$是$A$上的整元,也即存在$c_i\in A$满足$z^n+\sum_{t=1}^nc_tz^{n-t}=0$.则存在指标$i$,存在$z_i\in K_i$和$c_{ti}\in A_i$,使得它们在$K$和$A$中的像分别是$z$和$c_t$,进而得到$z_i\in A_i'$,于是$z\in A'$.这证明了前半部分断言,后半部分是因为从$A_i'=A_i$取正向极限得到$A'=A$.
        \end{proof}
        \item 关于单分支局部环.如果每个$A_i$都是单分支局部环或者几何单分支局部环,并且$\varphi_{ji}$都是局部单同态,那么$A$也是单分支局部环或者几何单分支局部环.这里一个环$A$称为单分支局部环(unibranch local ring),如果$B=A_{\mathrm{red}}$是整环,并且$B$的整闭包还是局部环;一个单分支局部环$A$称为几何单分支局部环(geometrically unibranch local ring),如果$B$的剩余域是$A_{\mathrm{red}}$剩余域的纯不可分扩张【单分支局部环更多内容见EGA4的0.23.2】.
        \begin{proof}
        	
        	如果$A_i$都是局部环,并且$\varphi_{ji}$都是局部同态,那么我们解释过正向极限$A$也是局部环,并且如果记$A_i$的剩余域为$\kappa_i$,那么$\kappa=\varinjlim\kappa_i$是$A$的剩余域.进而如果$A_i'$也是局部环,那么$A$的整闭包$A'$就还是局部环,于是从$A_i$都是单分支局部环得到正向极限$A$也是.最后再设$A'$的剩余域$\kappa_i'$在$\kappa_i$上是纯不可分扩张,那么$\kappa'=\varinjlim\kappa'_i$是$A'$的剩余域,并且也是$\kappa$的纯不可分扩张,换句话讲从$A_i$都是几何单分支局部环得到正向极限$A$是几何单分支局部环.
        \end{proof}
        \item 关于正规环.设$A_i$都是正规环(此为在每个素理想处的局部化都是正规整环),对任意$i\le j$,有$\mathrm{Spec}A_j$的不可约分支支配了$\mathrm{Spec}A_i$的某个不可约分支.那么$A=\varinjlim A_i$是正规环.
        \begin{proof}
        	
        	取$A$的素理想$\mathfrak{p}$,对任意指标$i$记$\mathfrak{p}_i=\varphi_i^{-1}(\mathfrak{p})$.按照$A_i$都是正规环,每个$\mathfrak{p}_i$只包含了$A_i$的一个极小素理想$\mathfrak{q}_i$.那么$\varphi_{ji}^{-1}(\mathfrak{q}_j)$是包含在$\mathfrak{p}_i$中的极小素理想(这个原像是极小素理想是因为条件中的$\mathrm{Spec}A_j$的每个不可约分支都支配了$\mathrm{Spec}A_i$的某个不可约分支),这迫使$\varphi_{ji}^{-1}(\mathfrak{q}_j)=\mathfrak{q}_i$.于是$\{\mathfrak{q}_i,\varphi_{ji}\}$构成一个正向系统.下面记$B_i=A_i/\mathfrak{q}_i$,记$\varphi_{ji}:A_i\to A_j$诱导的$B_i\to B_j$为$\varphi_{ji}'$,并且这是一个单射,那么$\{(B_i),(\varphi_{ji}')\}$是正向系统.另外按照$A_i$是正规环得到$B_i$是正规整环,于是我们解释了$B=\varinjlim B_i$是正规整环.记$\mathfrak{p}_i'=\mathfrak{p}_i/\mathfrak{q}_i$,那么$(B_i)_{\mathfrak{p}_i'}=(A_i)_{\mathfrak{p}_i}$(因为$(A_i)_{\mathfrak{p}_i}$按照条件是整环,于是极小素理想$(\mathfrak{q}_i)_{\mathfrak{p}_i}$是零理想).进而有$\varinjlim(B_i)_{\mathfrak{p}_i'}=\varinjlim(A_i)_{\mathfrak{p}_i}=A_{\mathfrak{p}}$,并且这是$B$在一个素理想处的局部化,于是按照$B$是正规整环,就得到$A_{\mathfrak{p}}$是正规整环.
        \end{proof}
        \item 关于正则环.设$A_i$都是正则环(这是指在每个素理想处的局部化都是局部正则环),对任意$i\le j$,设每个$\varphi_{ji}:A_i\to A_j$使得$A_j$是平坦$A_i$代数,再设$A=\varinjlim A_i$是诺特环,那么我们断言$A$也是正则环.
        \begin{proof}
        	
        	任取$A$的素理想$\mathfrak{p}$,我们要证明的是诺特局部环$A_{\mathfrak{p}}$是正则局部环.对每个指标$i$记$\mathfrak{p}_i=\varphi_i^{-1}(\mathfrak{p})$,那么我们解释过$\mathfrak{p}=\varinjlim\mathfrak{p}_i$和$A_{\mathfrak{p}}=\varinjlim(A_i)_{\mathfrak{p}_i}$.对$i\le j$,从$A_j$是平坦$A_i$代数,得到$(A_j)_{\mathfrak{p}_j}$是平坦$(A_i)_{\mathfrak{p}_i}$代数(这件事是因为一般的,如果$B$是$A$代数,$T\subseteq B$是乘性闭子集,$P$是$B$模,如果$P$作为$A$模是平坦的,那么$T^{-1}P$作为$A$模也是平坦的).于是问题归结为设$A_i$和$A$都是局部环.此时为了证明$A$是正则的,只需证明对任意有限$A$模$M,N$,都存在正整数$n_0$,使得当$n\ge n_0$时总有$\mathrm{Tor}_n^A(M,N)=0$.
        	
        	\qquad
        	
        	这里条件要求了$A$是诺特的,于是有限模$M,N$总是有限表示模,我们解释过此时存在指标$i$和有限$A_i$模$M_i,N_i$满足$M=M_i\otimes_{A_i}A$和$N=N_i\otimes_{A_i}A$.我们知道$\{A_j,j\ge i\}$也构成一个正向系统,并且它的极限仍然是$A$,而平坦模的正向极限还是平坦模,于是$A$是平坦$A_i$代数.我们有平坦基变换公式$\mathrm{Tor}_n^A(M,N)=\mathrm{Tor}_n^{A_i}(M_i,N_i)\otimes_{A_i}A$【EGAIII,6.3.9】,于是按照$A_i$是正则的,那么存在正整数$n_0$使得$n\ge n_0$时$\mathrm{Tor}_n^{A_i}(M_i,N_i)=0$,进而$n\ge n_0$时$\mathrm{Tor}_n^A(M,N)=0$,于是$A$是正则局部环.
        \end{proof}
        \item 关于Serre条件$(R_k)$.设存在正整数$k$,使得所有$A_i$都是诺特环并且满足$(R_k)$,此为在任意高度不超过$k$的素理想处的局部化总是正则局部环.设对任意$i\le j$,有$\varphi_{ji}:A_i\to A_j$使得$A_j$是平坦$A_i$代数,并且$A=\varinjlim A_i$是诺特环,并且对任意$i$有$\varphi_i$诱导的$\mathrm{Spec}A\to\mathrm{Spec}A_i$是同胚.则有$A$也满足$(R_k)$.
        \begin{proof}
        	
        	取$A$的素理想$\mathfrak{p}$,对任意指标$i$记$\mathfrak{p}_i=\varphi_i^{-1}(\mathfrak{p})$,那么我们解释过有$\mathfrak{p}=\varinjlim\mathfrak{p}_i$和$A_{\mathfrak{p}}=\varinjlim(A_i)_{\mathfrak{p}_i}$.进而按照同胚条件,对任意指标$i$就有$\mathrm{Spec}A_{\mathfrak{p}}\to\mathrm{Spec}(A_i)_{\mathfrak{p}_i}$也是同胚,于是特别的有$\dim A_{\mathfrak{p}}=\dim(A_i)_{\mathfrak{p}_i}$对任意指标$i$成立.下面设$\dim(A_{\mathfrak{p}})\le k$,则按照条件就有所有$(A_i)_{\mathfrak{p}_i}$都是正则局部环.那么对任意$i\le j$,我们有$(A_j)_{\mathfrak{p}_j}$是平坦$(A_i)_{\mathfrak{p}_i}$代数.于是上一条告诉我们$A_{\mathfrak{p}}$是正则局部环,从而$A$满足条件$(R_k)$.
        \end{proof}
    \end{enumerate}
\end{enumerate}
\subsection{概形的逆向极限}
\begin{enumerate}
	\item\label{概形的逆向极限1} 设$S_0$是环空间,设$\textbf{S}=\{(\mathscr{A}_i)_{i\in I},(\varphi_{ji}:\mathscr{A}_i\to\mathscr{A}_j)_{i\le j}\}$是以有向集$I$为指标集的$\mathscr{O}_{S_0}$代数层(未必交换)范畴上的正向系统.我们知道环空间上的模层范畴总是余完备的,于是把$\textbf{S}$视为$\mathscr{O}_{S_0}$模层范畴的正向系统时它总有正向极限.这里我们断言$\textbf{S}$作为$\mathscr{O}_{S_0}$代数层范畴上的正向系统也是总存在正向极限的,并且遗忘函子$\textbf{Alg}(\mathscr{O}_{S_0})\to\textbf{Mod}(\mathscr{O}_{S_0})$总和正向极限可交换.更具体地讲,如果$\{\mathscr{A},\varphi_i:\mathscr{A}_i\to\mathscr{A}\}$是$\textbf{S}$在$\mathscr{O}_{S_0}$代数层范畴的正向极限,那么把$\mathscr{A}$和$\varphi_i$都视为模层范畴中的对象和态射后它就是$\textbf{S}$在$\mathscr{O}_{S_0}$模层范畴中的正向极限.另外如果每个$\mathscr{A}_i$都是交换的$\mathscr{O}_{S_0}$代数层,则正向极限$\mathscr{A}$也是交换的代数层.
	\begin{proof}
		
		记$m_i:\mathscr{A}_i\otimes\mathscr{A}_i\to\mathscr{A}_i$是定义了$\mathscr{O}_{S_0}$代数层$\mathscr{A}_i$上乘法的$\mathscr{O}_{S_0}$模层态射.那么$\varphi_{ji}$满足的余圈条件保证了$\{m_i\}$构成了模层态射的正向系统,并且按照取正向极限和张量积可交换,我们有$m=\varinjlim m_i$(这个存在性还是因为模层范畴是余完备的)是$\mathscr{A}\otimes\mathscr{A}\to\mathscr{A}$的$\mathscr{O}_{S_0}$模层态射,这作为乘法使得$\mathscr{A}$是一个$\mathscr{O}_{S_0}$代数层,并且每个$\varphi_i$都是$\mathscr{O}_{S_0}$代数层之间的态射.我们来证明$\{\mathscr{A},\varphi_i:\mathscr{A}_i\to\mathscr{A}\}$实际上就是正向系统$\textbf{S}$在$\mathscr{O}_{S_0}$代数层范畴上的正向极限.换句话讲,我们要证明对任意$\mathscr{O}_{S_0}$代数层$\mathscr{B}$,如下典范态射是双射:
		$$\mathrm{Hom}_{\textbf{Alg}(\mathscr{O}_{S_0})}(\mathscr{A},\mathscr{B})\to\varprojlim\mathrm{Hom}_{\textbf{Alg}(\mathscr{O}_{S_0})}(\mathscr{A}_i,\mathscr{B})$$
		$$\left(f:\mathscr{A}\to\mathscr{B}\right)\mapsto\left(f\circ\varphi_i:\mathscr{A}_i\to\mathscr{B}\right)_{i\in I}$$
		
		首先按照$\{\mathscr{A},\varphi_i\}$是$\textbf{S}$在模层范畴中的正向极限,所以这个映射是单射.现在任取右侧正向极限中的元$(f_i:\mathscr{A}_i\to\mathscr{B})$,它在模层范畴中有极限$f:\mathscr{A}\to\mathscr{B}$,我们只要证明这是代数层之间的态射即可.而这只需要对如下交换图表取正向极限,再按照正向极限和张量积可交换就得到$f$是代数层之间的态射.
		$$\xymatrix{\mathscr{A}_i\otimes\mathscr{A}_i\ar[rr]^{f_i\otimes f_i}\ar[d]_{m_i}&&\mathscr{B}\otimes\mathscr{B}\ar[d]\\\mathscr{A}_i\ar[rr]&&\mathscr{B}}$$
		
		最后一个断言也是平凡的,如果每个$\mathscr{A}_i$都是交换代数层,记$t_i$是代数层态射$a\otimes b\mapsto b\otimes a$,那么有如下交换图表,取正向极限就得到$\mathscr{A}$是交换代数层.
		$$\xymatrix{\mathscr{A}_i\otimes\mathscr{A}_i\ar[rr]^{t_i}\ar[dr]_{m_i}&&\mathscr{A}_i\otimes\mathscr{A}_i\ar[dl]^{m_i}\\&\mathscr{A}_i&}$$
	\end{proof}
	\item\label{概形的逆向极限2} 特别的,设$S_0$是概形,设$\{\mathscr{A}_i,\varphi_{ji}\}$是拟凝聚$\mathscr{O}_{S_0}$代数层范畴上的正向系统,那么$\mathscr{A}=\varinjlim\mathscr{A}_i$也是拟凝聚$\mathscr{O}_{S_0}$代数层.接下来记$S_i=\mathrm{Spec}\mathscr{A}_i$和$S=\mathrm{Spec}\mathscr{A}$,记$\varphi_{ji}:\mathscr{A}_i\to\mathscr{A}_j$和$\varphi_i:\mathscr{A}_i\to\mathscr{A}$分别对应的$S_0$概形之间的态射是$u_{ij}:S_j\to S_i$和$u_i:S\to S_i$(并且由于结构态射$S_i\to S_0$和$S\to S_0$都是仿射态射,于是这里$u_{ij}$和$u_i$总是仿射的),则$\{(S_i)_{i\in I},(u_{ij})_{i\le j}\}$是概形范畴上的逆向系统.这里我们断言$\{S,(u_i)_{i\in I}\}$就是它(在概形范畴中)的逆向极限.另外,如果取定一个结构态射$h:S_0\to T$,据此让每个$S_0$概形都具备$T$概形结构,那么$\{S,(u_i)_{i\in I}\}$也是这个逆向系统在$T$概形范畴中的极限.
	\begin{proof}
		
		我们先来证明$\{S,(u_i)_{i\in I}\}$是逆向系统$\{(S_i)_{i\in I},(u_{ij})_{i\le j}\}$在$S_0$概形范畴中的逆向极限.换句话讲我们要证明对任意$S_0$概形$X$,如下典范映射是同构:
		$$\mathrm{Hom}_{S_0}(X,S)\to\varprojlim\mathrm{Hom}_{S_0}(X,S_i)$$
		$$\left(v:X\to S\right)\mapsto\left(u_i\circ v\right)_{i\in I}$$
		
		记结构态射$g:X\to S_0$,再记$\mathscr{B}=g_*(\mathscr{O}_X)$,按照拟凝聚代数层素谱的泛性质,以及上一条的,我们有:
		\begin{align*}
			\mathrm{Hom}_{S_0}(X,S)&\cong\mathrm{Hom}_{\textbf{Alg}(\mathscr{O}_{S_0})}(\mathscr{A},\mathscr{B})\\&=\varprojlim\mathrm{Hom}_{\mathscr{O}_{S_0}}(\mathscr{A}_i,\mathscr{B})\\&\cong\varprojlim\mathrm{Hom}_{S_0}(X,S_i)
		\end{align*}
	
	    我们剩下需要证明的部分是形式上的(或者说范畴上的)结论:设$\mathscr{C}$是范畴,设$T$是一个对象,设$\mathscr{C}_T$是以$T$对象和$T$态射构成的$\mathscr{C}$的子范畴.设$\textbf{S}=\{(S_i)_{i\in I},(\varphi_{ij}:S_j\to S_i)_{i\le j}\}$是$\mathscr{C}_T$上的逆向系统,则它当然也是$\mathscr{C}$上的逆向系统.我们断言它视为这两个范畴上的逆向系统的逆向极限如果有一个存在,那么另一个也存在,并且它们是同构的.
	    
	    \qquad
	    
	    证明上述结论:记结构态射$f_i:S_i\to T$.先设$\{S,(u_i:S\to S_i)\}$是这个逆向系统在$\mathscr{C}$中的逆向极限.假设我们有该逆向系统在$\mathscr{C}_T$范畴中的锥$\{Y,(w_i:Y\to S_i)\}$,则它也是$\mathscr{C}$范畴中该逆向系统的锥,于是存在$\mathscr{C}$中的态射$w:Y\to S$满足$w_i=u_i\circ w,\forall i$,但是这个等式导致$u_i$也是$T$态射,进而$S$也是$T$对象,于是这个逆向极限实际落在子范畴$\mathscr{C}_T$中,于是它也是该逆向系统在这个子范畴中的极限.
	    
	    \qquad
	    
	    再设$\{S,(u_i:S\to S_i)\}$是这个逆向系统在$\mathscr{C}_T$中的逆向极限,为了证明它也是$\mathscr{C}$中的逆向极限,只需证明该逆向系统在$\mathscr{C}$中的任意一个锥$\{Y,(w_i:Y\to S_i)\}$都也是该逆向系统在$\mathscr{C}_T$中的锥.我们先断言对任意指标$i$,有态射$f_i\circ w_i:Y\to T$是同一个态射:任取指标$i$和$j$,可以取指标$k$满足$i\le k$和$j\le k$,则按照锥的定义就有$f_k=f_i\circ u_{ik}=f_j\circ u_{jk}$,进而有$f_i\circ w_i=f_i\circ u_{ik}\circ w_k=f_j\circ u_{jk}\circ w_k=f_j\circ w_j$,完成断言的证明.我们把这个统一的态射记作$g:Y\to T$,它使得$Y$是$T$对象,并且$w_i$都是$T$态射,于是这个锥也是$\mathscr{C}_T$范畴中的锥.
	\end{proof}
    \item\label{概形的逆向极限3} 更特别的,如果$S_0=\mathrm{Spec}B$本身是仿射的,那么这里$\mathscr{O}_{S_0}$拟凝聚代数层范畴上的正向系统$\{\mathscr{A}_i,\varphi_{ji}\}$就等同于$B$代数范畴上的正向系统$\{A_i,\varphi_{ji}\}$,把正向极限记作$\{A,(\varphi_i:A_i\to A)\}$.那么有$S_i=\mathrm{Spec}A_i$和$S=\mathrm{Spec}A$,并且$\varphi_{ji}$和$\varphi_i$分别对应于概形态射$u_{ij}:S_j\to S_i$和$u_i:S\to S_i$.那么$\{(S_i),(u_{ij})_{i\le j}\}$构成了概形范畴上的逆向系统,并且我们证明了(无论这一小节还是上一小节)$\{S,(u_i)\}$是它在概形范畴中的逆向极限.
    \item\label{概形的逆向极限4} 逆向系统的基变换.继续延续第二条的记号,固定一个指标$i_0$,取一个$S_{i_0}$概形$X_{i_0}$,对任意$i\ge i_0$取$X_i=X_{i_0}\times_{S_{i_0}}S_i$,对$i_0\le i\le j$,取$v_{ij}=1_{X_{i_0}}\times u_{ij}:X_j\to X_i$,那么$I'=\{i\in I\mid i\ge i_0\}$是$I$的共尾子集,并且也是有向集,进而$\{(X_i)_{i\in I'},(v_{ij})_{i\le j}\}$是$X_{i_0}$概形范畴上的逆向系统.再记$X=X_{i_0}\times_{S_{i_0}}S$和$v_i=1_{X_{i_0}}\times u_i$,那么$\{X,(v_i)_{i\in I'}\}$是该逆向系统在$X_{i_0}$概形范畴中的逆向极限,也是在概形范畴中的逆向极限.
    \begin{proof}
    	
    	这件事实际上是形式上结论(或者说范畴上的结论).我们只需证明如下结论:设$\mathscr{C}$是总具有纤维积的范畴,设$g:T'\to T$是一个态射,我们用$\mathscr{C}_T$表示$T$对象和$T$态射构成的$\mathscr{C}$的子范畴,同理记子范畴$\mathscr{C}_{T'}$.记$\textbf{S}=\{(S_i),(u_{ij})\}$是$\mathscr{C}_T$中的逆向系统,并且不要求它的指标集一定是有向集(只要是偏序集即可),记$S_i'=S_i\times_TT'$和$u_{ij}'=u_{ij}\times1_{T'}$,那么$\textbf{S}'=\{(S_i'),(u_{ij}')\}$是$\mathscr{C}_{T'}$范畴中的逆向系统.如果$\{S,(u_i)\}$是$\textbf{S}$在$\mathscr{C}_T$范畴中的逆向极限,那么$\textbf{S}'$在$\mathscr{C}_{T'}$中的逆向极限就是$\{S\times_TT',(u_i\times1_{T'})\}$.
    	
    	\qquad
    	
    	下面给出证明:我们记$S'=S\times_TT'$,$u_i'=u_i\times1_{T'}$,记$S_i$作为$T$概形的结构态射为$h_i$,再记$h_i'=h_i\times1_{T'}$.那么我们有如下实线纤维积图表(小方格和大矩形都是纤维积图表):
    	$$\xymatrix{Y\ar@{-->}@/^1pc/[drrr]^{g'}\ar@{-->}@/^1pc/[drr]_{w_i'}\ar@{-->}@/_1pc/[ddr]_w\ar@{-->}[dr]_{w'}&&&\\&S'\ar[d]_p\ar[r]^{u_i'}&S_i'\ar[r]^{h_i'}\ar[d]_{p_i}&T'\ar[d]^q\\&S\ar[r]_{u_i}&S_i\ar[r]_{h_i}&T}$$
    	
    	下面取$\textbf{S}'$在$\mathscr{C}_{T'}$范畴中的锥$\{Y,(w_i':Y\to S_i')\}$,其中$Y$作为$T'$概形的结构态射记作$g'$.那么$Y$以$q\circ g'$作为$T$概形的结构态射,并且$w_i=p_i\circ w_i'$也是$T$态射,因为按照图表交换性有$h_i\circ w_i=q\circ g'$是$Y$作为$T$对象的结构态射.于是$\{Y,(w_i)\}$是逆向系统$\textbf{S}$的锥,于是存在态射$w:Y\to S$满足$u_i\circ w=w_i$.进而按照(左侧小方格)纤维积的泛性质,存在态射$w':Y\to S'$满足$p\circ w'=w$和$u_i'\circ w'=w_i'$.这说明$\{S',(w_i')\}$是$\textbf{S}'$的逆向极限.
    \end{proof}
    \item\label{概形的逆向极限5} 特别的,我们知道取开子集是基变换的特殊情况,于是如果取$S_{i_0}$的开子集$U_{i_0}$,对$i_0\le i$取$U_i=u_{i_0i}^{-1}(U_{i_0})$,这构成了逆向系统,它的逆向极限$U$是$S$的开子集,并且满足$U=u_i^{-1}(U_i),\forall i\ge i_0$.
    \item\label{概形的逆向极限6} 另外如果存在某个指标$i$使得$S_i$是拟紧或者拟分离的,那么按照$u_{ij}$和$u_i$是仿射态射,对任意$i\le j$就有$S_j$和$S$都是拟紧和拟分离的.
    \item\label{概形的逆向极限7} 另外实际上上述所有结论在额外要求代数层交换的情况下,都可以不要求指标集是有向集.这件事是因为如果$S$是环空间,在$\mathscr{O}_S$交换代数层范畴上,两个态射$\mathscr{A}\to\mathscr{B}$和$\mathscr{A}\to\mathscr{C}$的amalgamted sum总是存在的,它就是张量积$\mathscr{B}\otimes_{\mathscr{A}}\mathscr{C}$,这个条件结合有向集作为指标集的逆向系统总存在极限,就能推出指标集未必是有向集的逆向系统也总存在极限.如果$S$是概形,它的拟凝聚代数层的张量积还是拟凝聚代数层,于是在拟凝聚(要求交换)代数层范畴上amalgamted sum总是存在的,进而关于概形和拟凝聚代数层(交换)的结论也可以推广到指标集未必是有向集的逆向系统.
    \item\label{概形的逆向极限8} 设$S_0$是概形,记号$\mathscr{A}_i$,$S_i$,$S$,$\varphi_{ji}$,$\varphi_i$,$u_{ij}$,$u_i$的含义同上,并且要求指标集是有向集(的确用到了这个条件).现在我们设$u_{ij}=(\psi_{ij},\theta_{ij})$,这里$\psi_{ij}$是连续映射,$\theta_{ij}$是相应的层态射.同理记$u_i=(\psi_i,\theta_i)$.那么$\{(S_i)_{i\in I},(\psi_{ij}:S_j\to S_i)_{i\le j}\}$构成了拓扑空间范畴中的逆向系统(严格写的话这里$S_i$应该改为概形$S_i$的底空间,但是这实际不引起歧义,所以我们省略这个解释),我们断言$\{S,(\psi_i)\}$就是它在拓扑空间范畴中的逆向极限.
    \begin{proof}
    	
    	设$\{S_i,(\psi_i)\}$在拓扑空间范畴中的逆向极限是$\{T,\pi_i:T\to S_i\}$.那么$\{S,(\psi_i)\}$构成了这个逆向系统在拓扑空间范畴中的一个锥,进而有连续映射$\psi=\varprojlim\psi_i:S\to T$.我们可以把$S_0$添加到逆向系统中,结构态射当然也满足逆向系统中的余圈条件,此时指标集$I$就被添加了一个最小元$0$,换句话讲对任意$i\in I$都有$i\ge0$,并且这不改变无论是概形范畴上还是拓扑空间范畴上的逆向极限(因为新添加的余圈条件原来就满足).我们的目标就是证明$\psi$是一个同胚.为此我们先证明$S$中的开集一定具有形式$\psi^{-1}(V)$,其中$V\subseteq T$是开集,然后证明$\psi$是双射,这就导致$\psi$是连续双射并且是开映射,于是$\psi$是同胚.
    	
    	\qquad
    	
    	我们断言$S$中的开集一定可以写成$\psi^{-1}(V)$,其中$V\subseteq T$是开集.而我们知道按照拓扑空间范畴上逆向极限的描述,这里$V$一定是形如$\pi_i^{-1}(U_i)$的并,其中$U_i$是$S_i$的开集.取$S_0$的仿射开子集$U_0=\mathrm{Spec}A_0$,那么$\pi_0^{-1}(U_0)$是$S$的仿射开邻域,记作$U=\mathrm{Spec}A$,对任意指标$i$,有$\pi_i^{-1}(U_0)$也是$S_i$的仿射开邻域,记作$U_i=\mathrm{Spec}A_i$.任取$s\in A$,按照$A=\varinjlim A_i$,就存在一个指标$j$使得$s$可以提升到$s_j\in A_j$.再按照局部化和张量积可交换,得到$D_A(s)=D_{A_j}(s_j)\times_{S_j}S$(此即$A_j\times_AA_s=(A_j)_{s_j}$),换句话讲$D(s)=\psi_j^{-1}(D(s_j))$.最后当$U_0$跑遍$S_0$的仿射开子集,$s$跑遍$A=\Gamma(\pi_0^{-1}(U_0),\mathscr{O}_S)$中的元时,$D_A(s)$构成了$S$的拓扑基,所以$S$中的开集可以被这种形式的开集覆盖,而我们解释了这样的开集可以表示为$\psi_i^{-1}(U_i)$的形式,这里$U_i\subseteq S_i$是开集.进而$S$中的开集一定可以表示为形如$\psi_i^{-1}(U_i)$子集的并.
    	
    	\qquad
    	
    	再证明$\psi$是双射.我们知道概形总是$T_0$空间,所以任取$S$的两个不同点$x,y$,一定存在一个开集只包含这两个点中的一个.但是倘若$\psi$不是单射,就存在两个点$x,y$的像相同,由于$S$上的开集必然具有形式$\psi^{-1}(V)$,其中$V\subseteq T$是开集,就导致$S$中的开集要么同时包含$x,y$,要么同时不包含这两个点.这就和$T_0$矛盾,于是$\psi$是单射.因为满射是终端局部性质,于是我们可以不妨设$S_0$是仿射的,进而有$S_i$和$S$都是仿射的,它们对应的环分别记作$A_0,A_i,A$.那么$\{A_i,(\varphi_{ji}:A_i\to A_j)\}$构成一个正向系统,并且$A=\varinjlim A_i$.
    	
    	\qquad
    	
    	我们知道拓扑空间范畴上有向集上的正向极限和遗忘到集合范畴上的正向极限作为集合是一致的.所以$T$中的一个元$\alpha$可以描述为$\{\mathfrak{p}_i\in\mathrm{Spec}A_i\mid\varphi_{ji}^{-1}(\mathfrak{p}_j)=\mathfrak{p}_i\}$.而这构成一个正向系统,它的极限是$A$的素理想$\mathfrak{p}$(这件事在环的正向极限中证明过了,并且这需要有向集条件),并且满足$\mathfrak{p}_i=\varphi_i^{-1}(\mathfrak{p})$,于是$\psi(\mathfrak{p})=\alpha$,这就得到$\psi$是满射.
    \end{proof}
    \item\label{概形的逆向极限9} 推论.在上一条的条件下.
    \begin{enumerate}[(1)]
    	\item 设$\{(A_i),(\varphi_{ji})\}$是指标集是有向集的正向系统,记正向极限为$\{A,(\varphi_i)\}$.那么$\mathfrak{p}\mapsto\left(\varphi_i^{-1}(\mathfrak{p})\right)$就是从$\mathrm{Spec}A\to\varprojlim\mathrm{Spec}A_i$的同胚.
    	\item 记号$S,S_i,\varphi_i$的含义同上,要求指标集是有向集.如果$U$是$S$的拟紧开集,那么存在指标$i$和$S_i$的拟紧开集$U_i$满足$U=\psi_i^{-1}(U_i)$.这件事是因为当$U_i$取遍$S_i$的拟紧开集时,$\psi_i^{-1}(U_i)$构成了$S$的拓扑基,再结合$U$是拟紧的,以及指标集是有向集就得证.
    	\item 记号$u_{ij}=(\psi_{ij},\theta_{ij})$和$u_i=(\psi_i,\theta_i)$的含义同上,那么$\{\psi_i^*(\mathscr{O}_{S_i})\}$构成了$S$上环层的正向系统,我们断言$\theta_i^*:\psi_i^*(\mathscr{O}_{S_i})\to\mathscr{O}_S$诱导了如下同构,其中正向极限取为在$S$上环层范畴的正向极限.
    	$$\varinjlim\psi_i^*(\mathscr{O}_{S_i})\cong\mathscr{O}_S$$
    	\begin{proof}
    		
    		问题是局部的,不妨设$S_i=\mathrm{Spec}A_i$都是仿射的,那么这个同构就是说如果$\mathfrak{p}\in\mathrm{Spec}A$对应于$\varprojlim\mathrm{Spec}A_i$中的$\{\mathfrak{p}_i\in\mathrm{Spec}A_i\}$,则典范同态$\varinjlim(A_i)_{\mathfrak{p}_i}\to A_{\mathfrak{p}}$是同构.而这在环的正向极限中证过了.
    	\end{proof}
        \item $\{S,(u_i)\}$甚至是逆向系统$\{(S_i)_{i\in I},(u_{ij})_{i\le j}\}$在环空间范畴(从而也是在局部环空间范畴中)的逆向极限.
        \begin{proof}
        	
        	设$\{Y,(w_i:Y\to S_i)\}$是该逆向系统在环空间范畴中的一个锥.可记$w_i=(\rho_i,\omega_i)$.依旧记$u_{ij}=(\psi_{ij},\theta_{ij})$和$u_i=(\psi_i,\theta_i)$.那么$(Y,(\rho_i))$是该逆向系统在拓扑空间范畴中的锥,于是按照我们证明的$\{S,(\psi_i)\}$是该逆向系统在拓扑空间范畴中的极限,就存在连续映射$\rho:Y\to S$满足$\rho_i=\psi_i\circ\rho,\forall i$.另一方面,按照$\omega_i^{\#}:\rho_i^*(\mathscr{O}_{S_i})\to\mathscr{O}_Y$构成环层之间态射的逆向系统,其中$\rho_i^*(\mathscr{O}_{S_i})=\rho^*(\psi^*_i(\mathscr{O}_{S_i}))$.于是上一条结合$\rho^*$是右正合的得到$\varinjlim\rho_i^*(\mathscr{O}_{S_i})=\rho^*(\mathscr{O}_S)$.于是$\omega_i^{\#}:\rho_i^*(\mathscr{O}_{S_i})\to\mathscr{O}_Y$就诱导了层态射$\omega^{\#}:\rho^*(\mathscr{O}_S)\to\mathscr{O}_Y$,满足$\omega_i^{\#}=\omega^{\#}\circ\rho^*(\theta_i^{\#})$.综上取$w=(\rho,\omega):Y\to S$是环空间之间的态射,就得到$w_i=u_i\circ w$,于是$\{S,(u_i)\}$是该逆向系统在环空间范畴中的极限.        	
        \end{proof}
        \item 设$u_{ij}:S_j\to S_i$总是开嵌入,那么对任意指标$i$,都有$S$典范等同于$S_i$的开子概型$\{S_j\mid i\le j\}$的交.并且结构层$\mathscr{O}_S$就是$\mathscr{O}_{S_i}$在交集$S$上诱导的(这是指如果记$u_i:S\to S_i$的拓扑映射部分是$\psi_i$,那么$\mathscr{O}_S=\psi_i^*(\mathscr{O}_{S_i})$).更一般的,此时对任意$\mathscr{O}_{S_i}$模层$\mathscr{F}_i$,都有$u_i^*(\mathscr{F}_i)$等同于$\mathscr{F}_i$在交集$S$上的限制(同样的,这是指$u$).
        \begin{proof}
        	
        	第一个断言是因为我们解释了$S$恰好就是$\{S_i,\psi_{ji}\}$在拓扑空间范畴上的正向极限.【EGAIV的8.2.13】.
        \end{proof}
    \end{enumerate}
\end{enumerate}
\subsection{逆向极限和可构造集}

本节我们始终做如下约定:设$S_0$是概形,设$\{(\mathscr{A}_i),(\varphi_{ji}:\mathscr{A}_i\to\mathscr{A}_j)\}$是$\mathscr{O}_{S_0}$拟凝聚代数层范畴上的正向系统,它的极限记作$\{\mathscr{A},(\varphi_i:\mathscr{A}_i\to\mathscr{A})\}$.再记$S_i=\mathrm{Spec}\mathscr{A}_i$和$S=\mathrm{Spec}\mathscr{A}$.把$\varphi_{ji}$和$\varphi_i$对应的概形之间的态射记作$u_{ij}:S_j\to S_i$和$u_i:S\to S_i$.
\begin{enumerate}
	\item\label{逆向极限和可构造集1} 对每个指标$i$,取$S_i$的两个子集$E_i$和$F_i$,记$E=\cap_iu_i^{-1}(E_i)$和$F=\cup_iu_i^{-1}(F_i)$.设如下条件成立:
	\begin{enumerate}[(i)]
		\item 对任意$i$有$E_i$是pro-可构造集,$F_i$是ind-可构造集.
		\item 对任意$i\le j$有$E_j\subseteq u_{ij}^{-1}(E_i)$和$F_j\supseteq u_{ij}^{-1}(F_i)$.
		\item 存在一个指标$i_0$使得$S_{i_0}$是拟紧的(进而对任意$i\ge i_0$,有$S_i=u_{i_0i}^{-1}(S_{i_0})$是拟紧的).
	\end{enumerate}

    那么如下命题互相等价:
    \begin{enumerate}
    	\item $E\subseteq F$.
    	\item 存在指标$i\ge i_0$,使得$u_i^{-1}(E_i)\subseteq u_i^{-1}(F_i)$(进而对任意$j\ge i$都有$u_j^{-1}(E_j)\subseteq u_j^{-1}(F_j)$).
    	\item 存在指标$i\ge i_0$使得$E_i\subseteq F_i$(进而对任意$j\ge i$有$E_i\subseteq F_i$).
    \end{enumerate}
    \begin{proof}
    	
    	首先(b)和(c)中括号里的等价性是因为(ii):对任意$j\ge i$,我们有:
    	$$u_j^{-1}(E_j)\subseteq u_j^{-1}(u_{ij}^{-1}(E_i))=u_i^{-1}(E_i)\subseteq u_i^{-1}(F_i)=u_j^{-1}(u_{ij}^{-1}(F_i))\subseteq u_j^{-1}(F_j)$$
    	$$E_j\subseteq u_{ij}^{-1}(E_i)\subseteq u_{ij}^{-1}(F_i)\subseteq F_j$$
    	
    	我们记$G_i=E_i\cap(S_i-F_i)$和$G=E\cap(S-F)$,那么$G_i$也是$S_i$的pro-可构造集.并且对任意$j\ge i$有$G_j\subseteq u_{ij}^{-1}(G_i)$和$G=\cap_iu_i^{-1}(G_i)$.于是用$G_i,G$替代$E_i,E$,问题归结为设$F_i=\emptyset$的情况.此为下面推论的第一条.
    \end{proof}
    \item\label{逆向极限和可构造集2} 推论.
    \begin{enumerate}[(1)]
    	\item 对任意指标$i$,取$S_i$的一个pro-可构造集$E_i$,满足对任意$i\le j$都有$E_j\subseteq u_{ij}^{-1}(E_i)$,再设存在指标$i_0$使得$S_{i_0}$是拟紧的,那么如下命题互相等价:
    	\begin{enumerate}[(a)]
    		\item $E=\cap_iu_i^{-1}(E_i)=\emptyset$.
    		\item 存在一个指标$i$使得$u_i^{-1}(E_i)=\emptyset$(从而对任意$i\le j$都有$u_j^{-1}(E_j)=\emptyset$).
    		\item 存在一个指标$i$使得$E_i=\emptyset$(从而对任意$i\le j$都有$E_j=\emptyset$).
    	\end{enumerate}
    	\begin{proof}
    		
    		明显的有(c)推(a).下面证明(a)推(b):按照$S_{i_0}$是拟紧的,以及$u_{i_0}:S\to S_{i_0}$是仿射的,于是$S$也是拟紧的.另外从$E_i$是pro-可构造集得到$u_i^{-1}(E_i)$总是$S$的可构造集.我们解释过拟紧概形的一族pro-可构造集的交如果是空集,那么存在$I$的有限子集$J$使得$\{u_i^{-1}(E_i),i\in J\}$的交是空集,取一个指标$k$满足$k\ge i,\forall J$,那么$u_k^{-1}(E_k)\subseteq\cap_{i\in J}u_i^{-1}(E_i)$(因为从$i\le k$得到$u_k^{-1}(E_k)\subseteq u_k^{-1}(u_{ik}^{-1}(E_i))=u_i^{-1}(E_i)$).
    		
    		\qquad
    		
    		最后证明(b)推(c):问题是局部的,我们不妨用$S_{i_0}$的仿射开邻域替代$S_{i_0}$,进而按照$u_{ij}$和$u_i$都是仿射态射,得到对任意$i\le i_0$有$S_i$是仿射的,并且$S$也是仿射的.我们解释过此时对任意$i\ge i_0$,有$u_i(S)=\cap_{j\ge i}u_{ij}(S_j)$.进而有$E_i\cap u_i(S)=\cap_{j\ge i}(E_i\cap u_{ij}(S_j))$.这里$u_i$和$u_{ij}$是拟紧态射,我们解释过拟紧态射把pro-可构造集映为pro-可构造集,于是这里$u_i(S)$和$u_{ij}(S_j)$都是$S_i$的pro-可构造集.按照$u_i^{-1}(E_i)$是空集,也即$u_i(S)\cap E_i$是空集,于是从$S_i$是拟紧的得到存在$I$的有限子集$J$使得$\{E_i\cap u_{ij}(S_j),j\in J\}$的交是空集,进而取指标$k\ge j,j\in J$,有$E_i\cap u_{ik}(S_k)$是空集,于是$u_{ik}^{-1}(E_i)$是空集,于是$E_i\subseteq u_{ik}^{-1}(E_i)$是空集.
    	\end{proof}
    	\item 对任意指标$i$,取$S_i$的一个ind-可构造集$F_i$,满足对任意$i\le j$有$u_{ij}^{-1}(F_i)\subseteq F_j$.设存在指标$i_0$使得$S_{i_0}$是拟紧的,那么如下命题互相等价:
    	\begin{enumerate}[(a)]
    		\item $F=\cup_iu_i^{-1}(F_i)=S$.
    		\item 存在指标$i$使得$u_i^{-1}(F_i)=S$(进而对任意$i\le j$有$u_j^{-1}(F_j)=S$).
    		\item 存在指标$i$使得$F_i=S_i$(进而对任意$i\le j$有$F_j=S_j$).
    	\end{enumerate}
        \item 如果存在某个指标$i_0$使得$S_{i_0}$是拟紧的,那么$S$是空集当且仅当存在某个指标$i$使得$S_i$是空集(进而对任意$i\le j$有$S_j$是空集).
        \item 对任意指标$i_0$,我们有$u_{i_0}(S)=\cap_{i\ge i_0}u_{i_0i}(S_i)$(我们解释过环的素谱的情况有这个等式,这里是一般的拟凝聚代数层的素谱的情况).
        \begin{proof}
        	
        	明显的按照$u_{i_0}$经$u_{i_0i}$分解,得到$u_{i_0}(S)\subseteq\cap_{i\ge i_0}u_{i_0i}(S_i)$.接下来设$s\in S_{i_0}$,记$X_{i_0}=\mathrm{Spec}\kappa(s)$,对任意$j\ge i\ge i_0$,记$X_i=X_{i_0}\times_{S_{i_0}}S_i$和$v_{ij}=1\times u_{ij}$.再记$X=X_{i_0}\times_{S_{i_0}}S$和$v_i=1\times u_i$.如果$s\in\cap_{i\ge i_0}u_{i_0i}(S_i)$,则对任意$i\ge i_0$有$X_i$非空,于是上一条说明$X$非空(可以用上一条是因为$X_i$和$X$仍然是拟凝聚代数层的素谱构成的逆向系统),也即$s\in u_{i_0}(S)$.
        \end{proof}
        \item 设有一个指标$i_0$使得$S_{i_0}$是拟紧的,取指标$i$,取两个可构造集$E_i,F_i\subseteq S_i$,对$i\le j$取$E_j=u_{ij}^{-1}(E_i)$和$F_j=u_{ij}^{-1}(F_i)$.那么对任意$i\le j$有$u_i^{-1}(E_i)=u_j^{-1}(E_j)=E$和$u_i^{-1}(F_i)=u_j^{-1}(F_j)=F$.那么$E\subseteq F$当且仅当存在指标$j$使得$E_j\subseteq F_j$(进而对任意指标$k\ge j$都有$E_k\subseteq F_k$);并且$E=F$当且仅当存在指标$j$使得$E_j=F_j$(进而对任意指标$k\ge j$都有$E_k=F_k$).这件事相当于后面定义的$u_{\textbf{P}}$是单射.
    \end{enumerate}
    \item\label{逆向极限和可构造集3} 逆向系统中一些态射性质的传递.
    \begin{enumerate}
    	\item 取一个指标$i$,那么$u_i:S\to S_i$是支配态射/满射态射,当且仅当对任意$j\ge i$有$u_{ij}:S_j\to S_i$是支配态射/满射态射.
    	\begin{proof}
    		
    		我们先解释满射态射的命题.首先对$i\le j$总有$u_i(S)\subseteq u_{ij}(S_j)$,于是从$u_i$是满射当然得到$u_{ij},\forall j\ge i$是满射.反过来的情况是因为我们证明了$u_i(S)=\cap_{j\ge i}u_{ij}(S_j)$.
    		
    		\qquad
    		
    		下面处理支配态射的命题.因为支配态射是终端局部性质,我们不妨设$S_i$是仿射态射,进而按照$u_i$和$u_{ij}$都是仿射态射,得到$S_j,j\ge i$和$S$都是仿射态射.记$u_{ij}$和$u_i$对应的环同态是$\varphi_{ji}$和$\varphi_i$,我们知道仿射概形之间的态射是支配态射当且仅当对应的环同态是单射.所以我们要证明的是$\varphi_i$是单射当且仅当$\forall i\le j$有$\varphi_{ji}$是单射.必要性是因为$\varphi_i$经$\varphi_{ji}$分解.充分性是因为如果$x,y\in A_i$满足$\varphi_i(x)=\varphi_i(y)$,按照环的正向系统的极限的描述,存在$j\ge i$使得$\varphi_{ji}(x)=\varphi_{ji}(y)$,但是$\varphi_{ji}$是单射,于是$x=y$.
    	\end{proof}
    	\item 取一个指标$i$,如果对任意$i\le j$有$u_{ij}:S_j\to S_i$是平坦态射/忠实平坦态射,那么$u_i:S\to S_i$是平坦态射/忠实平坦态射.
    	\begin{proof}
    		
    		这件事是因为平坦模构成的正向极限仍然是平坦模,另外忠实平坦是平坦+满射,而满射情况我们已经在上一条解释过了.
    	\end{proof}
    	\item 取一个指标$i$,设对任意$i\le j$有$u_j:S\to S_j$是满射态射.那么$u_i$是开映射/泛开的态射,当且仅当对任意$i\le j$有$u_{ij}$是开映射/泛开的态射.另外这里$u_j,j\ge i$总是满射是必须的:取$A=\mathbb{Z}$,把商域记作$K$,让$f$跑遍$A-\{0\}$,则$\{A_f\}$构成一个正向系统,记$S_0=\mathrm{Spec}A$,记$S_f=\mathrm{Spec}A_f$,那么$S=\varprojlim S_f=\mathrm{Spec}K$.这里每个$S_f\to S_0$都是开映射,但是$S\to S_0$不是开映射.
    	\begin{proof}
    		
    		首先我们解释过逆向系统和基变换是兼容的,结合满射在基变换下不变,所以泛开的映射的情况归结为开映射的情况.先设$u_i$是开映射.按照$i\le j$时$u_i=u_{ij}\circ u_j$,以及$u_j$总是满射,对$S_j$的开集$U_j$就有$u_{ij}(U_j)=u_i\circ u_j^{-1}(U_j)$,于是从$u_i$是开映射得到$u_{ij}(U_j)$是开集,于是$u_{ij}$是开映射.
    		
    		\qquad
    		
    		下面设对任意$i\le j$有$u_{ij}$是开映射.为了证明$u_i$是开映射,只需证明对$S$的任意拟紧开子集$V$有$u_i(V)$是$S_i$的开集.但是我们解释过对拟紧开子集$V$,此时一定存在指标$j$和$S_j$的拟紧开子集$U_j$,满足$U=u_j^{-1}(U_j)$.按照有向集条件我们可以不妨设$j\ge i$.按照$u_j$是满射,就有$V_j=u_j(V)$,从而$u_i(V)=u_{ij}\circ u_j(V)=u_{ij}(V_j)$,于是从$u_{ij}$是开映射得到$u_i(V)$是开集,从而$u_i$是开映射.
    	\end{proof}
    \end{enumerate}
    \item\label{逆向极限和可构造集4} 设$X$是概形,我们定义如下记号:
    \begin{align*}
    	\textbf{P}(X)&=\{X\text{的所有子集}\}\\
    	\textbf{C}(X)&=\{X\text{的所有可构造集}\}\\
    	\textbf{OC}(X)&=\{X\text{的所有可构造开集}\}\\
    	\textbf{FC}(X)&=\{X\text{的所有可构造闭集}\}\\
    	\textbf{LFC}(X)&=\{X\text{的所有可构造局部闭子集}\}
    \end{align*}

    按照可构造集的原像还是可构造集,我们有如下正向系统:
    \begin{align*}
    	\{(\textbf{P}(S_i))_{i\in I},&(u_{ij}^{-1})_{i\le j}\}\\
    	\{(\textbf{C}(S_i))_{i\in I},&(u_{ij}^{-1})_{i\le j}\}\\
    	\{(\textbf{OC}(S_i))_{i\in I},&(u_{ij}^{-1})_{i\le j}\}\\
    	\{(\textbf{FC}(S_i))_{i\in I},&(u_{ij}^{-1})_{i\le j}\}\\
    	\{(\textbf{LFC}(S_i))_{i\in I},&(u_{ij}^{-1})_{i\le j}\}
    \end{align*}

    并且$\{\textbf{P}(S),(u_i^{-1})_{i\in I}\}$等构成了上述每个正向系统的余锥,于是诱导了如下集合映射:
    \begin{align*}
    	u_{\textbf{P}}:\varinjlim\textbf{P}(S_i)&\to\textbf{P}(S)\\
    	u_{\textbf{C}}:\varinjlim\textbf{C}(S_i)&\to\textbf{C}(S)\\
    	u_{\textbf{OC}}:\varinjlim\textbf{OC}(S_i)&\to\textbf{OC}(S)\\
    	u_{\textbf{FC}}:\varinjlim\textbf{FC}(S_i)&\to\textbf{FC}(S)\\
    	u_{\textbf{LFC}}:\varinjlim\textbf{LFC}(S_i)&\to\textbf{LFC}(S)
    \end{align*}
    
    再设$g_0:X_0\to S_0$是态射,对任意指标$i\le j$取$X_i=X_0\times_{S_0}S_i$和$v_{ij}=1_{X_0}\times u_{ij}$.再取$X=X_0\times_{S_0}S$和$v_i=1_{X_0}\times u_i$,那么我们解释过$\{X,(v_i)\}$是$\{(X_i),(v_{ij})\}$(在概形范畴或者$S_0$概形范畴)的逆向极限.记投影态射$g_i:X_i\to S_i$和$g:X\to S$,那么$g_i^{-1}:\textbf{P}(S_i)\to\textbf{P}(X_i)$和$g^{-1}:\textbf{P}(S)\to\textbf{P}(X)$,进而我们有如下交换图表:
    $$\xymatrix{\textbf{P}(S_i)\ar[rr]^{u_i^{-1}}\ar[d]_{g_i^{-1}}&&\textbf{P}(S)\ar[d]^{g^{-1}}\\\textbf{P}(X_i)\ar[rr]_{v_i^{-1}}&&\textbf{P}(X)}$$
    
    进而取正向极限得到如下交换图表:
    $$\xymatrix{\varinjlim\textbf{P}(S_i)\ar[rr]^{u_{\textbf{P}}}\ar[d]_{\varinjlim g_i^{-1}}&&\textbf{P}(S)\ar[d]^{g^{-1}}\\\varinjlim\textbf{P}(X_i)\ar[rr]_{v_{\textbf{P}}}&&\textbf{P}(X)}$$
    
    类似的对$\textbf{C},\textbf{OC},\textbf{FC},\textbf{LFC}$有相同的交换图表.
    \item\label{逆向极限和可构造集5}
    \begin{enumerate}
    	\item 如果存在指标$i_0$使得$S_{i_0}$是拟紧的,那么上面定义的$u_{\textbf{C}}$,$u_{\textbf{OC}}$,$u_{\textbf{FC}}$,$u_{\textbf{LFC}}$都是单射.
    	\item 如果存在指标$i_0$使得$S_{i_0}$是qcqs概形,那么$u_{\textbf{C}}$,$u_{\textbf{OC}}$,$u_{\textbf{FC}}$,$u_{\textbf{LFC}}$都是双射.
    \end{enumerate}
    \begin{proof}
    	
    	我们解释过如果存在指标$i_0$使得$S_{i_0}$是拟紧概形,那么$u_{\textbf{P}}$是单射,进而限制定义域得到的映射$u_{\textbf{C}}$,$u_{\textbf{OC}}$,$u_{\textbf{FC}}$,$u_{\textbf{LFC}}$都是单射.于是我们还剩下证明这些映射都是满射.
    	
    	\qquad
    	
    	先证明$u_{\textbf{OC}}$是满射:按照$S_{i_0}$是qcqs概形,则对任意$i\ge i_0$有$S_i$和$S$是qcqs概形.而我们知道qcqs概形上的可构造开集就是拟紧开集,任取$S$的拟紧开集$U$,我们解释过存在指标$i$和拟紧开集$U_i\subseteq S_i$使得$u_i^{-1}(U_i)=U$.那么对$j\ge i$取拟紧开集$U_j=u_{ij}^{-1}(U_i)$就保证$\{U_j,j\ge i\}$是$\varinjlim\textbf{OC}(S_i)$中的元,并且在$u_{\textbf{OC}}$下的像是$U$,这就得到$u_{\textbf{OC}}$是满射.
    	
    	\qquad
    	
    	把上一条结论取补集就得到$u_{\textbf{FC}}$是满射.另外我们知道qcqs概形$X$上可构造集恰好可以表示为有限个形如$U\cap(X-V)$的并集,其中$U,V$是可构造开集,于是从$u_{\textbf{OC}}$是满射得到$u_{\textbf{C}}$是满射.
    	
    	\qquad
    	
    	最后我们证明$u_{\textbf{LFC}}$是满射:取$S$的可构造局部闭子集$Z$,那么它是拟紧的(因为整体可构造集都是反紧的,而拟紧空间的反紧子集总是拟紧的).按照局部闭子集的定义,对任意$x\in Z$,可取$x$在$S$中的拟紧开邻域$V_x$,使得$Z\cap V_x$是$V_x$的闭子集.按照$Z$是拟紧的,就可以取有限个这样的$V_x$覆盖$Z$.取这有限个$V_x$的并为$U$,那么$U$就是一个包含$Z$的拟紧开集,使得$Z$在$U$中是闭集.因为$Z$是$S$的可构造集,于是$Z$也是$U$的可构造集.另外由于$U$是拟紧的,我们解释过可以找到指标$i$以及一个拟紧开集$U_i\subseteq S_i$,使得$U=u_i^{-1}(U_i)$.对$j\ge i$取$U_j=u_{ij}^{-1}(U_i)$,那么$U$就是$\{U_j,j\ge i\}$的正向极限.进而按照我们证明的$u_{\textbf{FC}}$是满射,对每个$j\ge i$就可以找到$U_j$的可构造闭集$Z_j$,满足$Z=u_j^{-1}(Z_j)$.又因为$S_j$是拟分离的,得到开嵌入$U_j\to S_j$是拟紧的,从而它是有限表示的,于是按照Chevalley定理,$U_j$的可构造集$Z_j$在$S_j$中的像集$Z_j$是$S_j$的可构造集.于是$Z_j$是$S_j$的可构造局部闭子集,而$\{Z_j\}$的极限是$Z$,于是$u_{\textbf{LFC}}$是满射.
    \end{proof}
    \item\label{逆向极限和可构造集6} 推论.设有指标$i_0$使得$S_{i_0}$是拟紧的,并且对任意指标$i$设$Z_i$是$S_i$的可构造集,并且对$i\le j$总有$Z_j=u_{ij}^{-1}(Z_i)$.再记$Z=u_i^{-1}(Z_i)$(这不依赖$i$的选取).则$Z$是$S$的开子集/闭子集/局部闭子集,当且仅当存在指标$j\ge i_0$,使得$Z_j$是$S_j$的开子集/闭子集/局部闭子集(进而对任意$k\ge j$有$Z_k$是$Z_j$的开子集/闭子集/局部闭子集).
    \begin{proof}
    	
    	充分性是因为$Z=u_j^{-1}(Z_j)$.只需证明必要性.我们设$S_{i_0}$是有限个仿射开子集$U_{i_0}^{(t)}$的并,再记$U^{(t)}=u_{i_0}^{-1}(U_{i_0}^{(t)})$,则这些$\{U^{(t)}\}$构成$S$的有限仿射开覆盖.那么从$Z$是$S$的开子集/闭子集/局部闭子集,得到$Z\cap U^{(t)}$是$U^{(t)}$的开子集/闭子集/局部闭子集.于是我们不妨设$S_{i_0}$是仿射的,特别的它是qcqs概形,于是这归结为上一条结论.
    \end{proof}
    \item\label{逆向极限和可构造集7} 设对任意$i\le j$有$u_{ij}:S_j\to S_i$是平坦态射,设存在指标$i_0$使得$S_{i_0}$是拟紧的.对任意指标$i$,取$Z_i',Z_i''$为$S_i$的两个pro-可构造集.对$i\le j$取$Z_j'=u_{ij}^{-1}(Z_i')$和$Z_j''=u_{ij}^{-1}(Z_i'')$.再设$\overline{Z_{i_0}'}$是$S_{i_0}$的可构造集.记$Z'=u_i^{-1}(Z_i')$和$Z''=u_i^{-1}(Z_i'')$.我们断言$Z''\subseteq\overline{Z'}$当且仅当存在$i\ge i_0$满足$Z_i''\subseteq\overline{Z_i'}$(进而对任意$i\le j$有$Z_j''\subseteq\overline{Z_j'}$).
    \begin{proof}
    	
    	我们解释过从$u_{ij},\forall j\ge i$平坦得到$u_i$是平坦的.那么对任意$i\le j$就有$\overline{Z_j'}=u_{ij}^{-1}(\overline{Z_i'})$和$\overline{Z'}=u_i^{-1}(\overline{Z_i'})$【这件事是EGAIV的2.3.10】.但是我们知道可构造集的原像还是可构造集,于是这里$\overline{Z_j'}$和$\overline{Z'}$都是可构造集.于是结论归结为\ref{逆向极限和可构造集1}.
    \end{proof}
\end{enumerate}
\subsection{逆向极限和不可约性与连通性}

本节我们始终做如下约定:设$S_0$是概形,设$\{(\mathscr{A}_i),(\varphi_{ji}:\mathscr{A}_i\to\mathscr{A}_j)\}$是$\mathscr{O}_{S_0}$拟凝聚代数层范畴上的正向系统,它的极限记作$\{\mathscr{A},(\varphi_i:\mathscr{A}_i\to\mathscr{A})\}$.再记$S_i=\mathrm{Spec}\mathscr{A}_i$和$S=\mathrm{Spec}\mathscr{A}$.把$\varphi_{ji}$和$\varphi_i$对应的概形之间的态射记作$u_{ij}:S_j\to S_i$和$u_i:S\to S_i$.
\begin{enumerate}
	\item\label{逆向极限和不可约性与连通性1} 设存在指标$i_0$使得$S_{i_0}$是拟紧的.
	\begin{enumerate}
		\item 设$S$不是不可约的,并且$S$的底空间是诺特的,并且$S_{i_0}$额外是拟分离的.则存在指标$i\ge i_0$,使得对任意$j\ge i$有$S_j$都不是不可约的.
		\begin{proof}
			
			设$S$是两个不是全集也不是空集的闭子集$S',S''$的并.因为$S$是诺特的,所以闭子集$S',S''$都是可构造集.于是按照我们证明的$u_{\textbf{FC}}$是满射(这里用到了拟分离条件),就存在指标$i$和$S_i$的两个可构造闭集$S_i',S_i''$满足$S'=u_i^{-1}(S_i')$和$S''=u_i^{-1}(S_i'')$.那么从$S=S'\cup S''$得到$S_i=S_i'\cup S_i''$,那么$S_i'$和$S_i''$都不是全集也不是空集,这就得到$S_i$不是不可约的,同理对任意$i\le j$有$S_j$不是不可约的.
		\end{proof}
		\item 设$S$不是连通的,那么存在$i\ge i_0$,使得对任意$j\ge i$都有$S_j$不是连通的.
		\begin{proof}
			
			设$S$是两个不是全集也不是空集的闭子集$S',S''$的无交并.因为$S$是拟紧的,于是$S',S''$也是拟紧开集.我们解释过存在指标$i$和$S_i$的拟紧开集$S_i',S_i''$,满足$S'=u_i^{-1}(S_i')$,$S''=u_i^{-1}(S_i'')$,并且这两个等式说明$S_i'$和$S_i''$是非全集和非空集的.但是按照$S',S''$都是$S$的既开又闭子集,所以它们同时是$S$的pro-可构造集和ind-可构造集(我们解释过闭集是pro-可构造集),从而它们都是$S$的可构造集.再对$i\le j$取$S_j'=u_{ij}^{-1}(S_i')$和$S_j''=u_{ij}^{-1}(S_i'')$.那么从$S'=S-S''$,并且$S-S''$是$\{S_j-S''_j\}$的正向极限,以及它们都是可构造集,于是我们可以适当把指标$i$变大,使得它满足$S_i'=S_i-S_i''$,也即$S_i'\cap S_i''$恰好是$S_i$的非全集和非空集的开集的无交并,这就说明$S_i$不是连通的.同理对任意$i\le j$有$S_j$不是连通的.
		\end{proof}
	\end{enumerate}
    \item\label{逆向极限和不可约性与连通性2} 设$S$的底空间是诺特的,并且如下两个条件之一成立:
    \begin{enumerate}
    	\item 对任意$i\le j$,有$u_{ij}:S_j\to S_i$是支配态射,并且存在一个指标$i_0$使得$S_{i_0}$是拟紧的.
    	\item 对任意$i\le j$,有$u_{ij}:S_j\to S_i$在底空间上是一个拓扑嵌入(也即它是到终端的某个子集的同胚,例如它们总是嵌入),并且存在一个指标$i_0$使得$S_{i_0}$的底空间是诺特的.
    \end{enumerate}

    那么存在一个指标$i$,使得对任意$j\ge i$:
    \begin{enumerate}[(i)]
    	\item 对$S$的任意不可约分支$Y$,有$\overline{u_j(Y)}$是$S_j$的不可约分支,并且如果记$S$的全部不可约分支为$\{Y_t\mid1\le t\le m\}$,那么$Y_t\mapsto\overline{u_j(Y_t)}$恰好是从$S$的全体不可约分支到$S_j$的全体不可约分支的双射.
    	\item 对$S$的任意连通分支$C$,有$\overline{u_j(C)}$是$S_j$的连通分支,并且如果记$S$的全部连通分支为$\{C_t\mid1\le t\le n\}$,那么$C_t\mapsto\overline{u_j(C_t)}$恰好是从$S$的全体连通分支到$S_j$的全体连通分支的双射.
    \end{enumerate}
    \begin{proof}
    	
    	我们断言在条件(a)或者(b)下,总存在指标$i$使得对任意$i\le j$有$u_j:S\to S_j$总是支配态射:对于(a),我们已经证明过了即便不要求$S$是诺特空间,从$u_{ij},i\le j$是支配态射就得到$u_i$是支配态射.对于(b),我们记$Z_{i_0}=\overline{u_{i_0}(S)}$,并且因为$S_{i_0}$是诺特空间,所以这个闭子集是可构造集.于是从$u_{i_0}^{-1}(Z_{i_0})=S$,就可以找到$i\ge i_0$使得对任意$i\le j$都有$Z_j=u_{i_0j}^{-1}(Z_{i_0})=S_j$.又因为$u_{i_0j}$在底空间上是拓扑嵌入,于是从复合态射$S\to Z_j\to Z_{i_0}$是支配的就得到$S\to Z_j=S_j$也必须是支配的.完成断言的证明.于是我们不妨设对任意指标$i$有$u_i$是支配态射.
    	
    	\qquad
    	
    	证明(i):固定$i$时,每个$S_i$都是有限个不可约闭子集$\{\overline{u_i(Y_t)}\}$的并(这个闭包是$S_i$的不可约闭子集是因为不可约空间的连续像是不可约的,不可约子空间的闭包是不可约的).我们解释过如果一个空间写成了有限个不可约闭子集的并,那么其中的全体极大元恰好就是这个空间的全部不可约分支.于是问题归结为证明$\{\overline{u_i(Y_t)}\}$两两之间没有包含关系.
    	
    	\qquad
    	
    	取定一个$1\le t\le m$,取$Y_t$以外的$S$的不可约分支的并,再取它在$S$中的补集记作$U_t$,那么$\{U_t\mid1\le t\le m\}$是两两不交的,并且有$Y_t=\overline{U_t}$(不可约空间的非空开子集总是稠密的).按照$S$的底空间是诺特的,于是这里$U_t$都是拟紧的.那么我们解释过存在指标$i$和$S_i$的拟紧开集$U_{ti}$,使得$U_t=u_i^{-1}(U_{ti}),\forall1\le t\le m$成立.对任意$i\le j$,我们记$U_{tj}=u_{ij}^{-1}(U_{ti})$.那么按照$u_j$是支配的,以及$U_t=u_j^{-1}(U_{tj})$,就有$\{U_{tj}\mid1\le t\le m\}$是两两不交的.进而有$\{\overline{U_{tj}}\mid1\le t\le m\}$两两之间没有包含关系.并且按照$u_j$是支配的,就有$u_j(U_t)$是$U_{tj}$的稠密子集.进而有$\overline{U_{tj}}=\overline{u_j(Y_t)}$.这就说明了$\{\overline{u_i(Y_t)}\}$两两之间没有包含关系.
    	
    	\qquad
    	
    	证明(ii):我们知道诺特空间的连通分支$C_t$总是既开又闭并且拟紧的(因为诺特空间的不可约分支都是闭子集,而且有限个,而每个连通分支都是若干不可约分支的并).于是我们依旧可以找到一个指标$i$,以及$S_i$的拟紧开集$V_{ti},1\le t\le n$,满足$C_t=u_{ij}^{-1}(V_{ti})$.对任意$i\le j$,记$V_{tj}=u_{ij}^{-1}(V_{ti})$.按照$u_j$是支配的,我们依旧有$\{V_{tj}\mid1\le t\le n\}$是两两不交的,并且$u_j(C_t)$在$V_{tj}$中稠密,进而$V_{tj}$是连通的(因为连通空间的连续像连通,并且连通子集的闭包连通).另外按照每个$C_t=u_j^{-1}(V_{tj})$都是$S$的ind-可构造集,每个$V_{tj}$都是$S_j$的ind-可构造集,并且$\cup_tC_t=S$,我们解释过此时适当把$j$变大可以有$\cup_tV_{tj}=S_j$,进而把$j$替换为任意更大的指标都有这个等式成立.于是此时$S_j$是一族两两不交的连通子集的并,于是它们恰好是$S_j$的所有连通分支.
    \end{proof}
    \item\label{逆向极限和不可约性与连通性3} 另外在上一条的(a)和(b)实际上只用在当$i$足够大时$u_i$是支配态射,并且存在$i_0$使得$S_{i_0}$是拟紧的.所以如果只要求这两个条件以及$S$的底空间是诺特的,那么结论依旧成立.而且这里要求$i$足够大时$u_i$是支配态射是必须的,甚至在$S_i,S$的底空间都是诺特空间的情况下也是必须的:我们把指标集取为$\mathbb{N}$,总取$S_n=\mathrm{Spec}(A\times K)=\mathrm{Spec}A\coprod\mathrm{Spec}K$,其中$K$是域,$A$是一个$K$代数,把结构同态记作$j:K\to A$.定义$u_{n,n+1}$为同态$(x,y)\mapsto(j(y),y)$诱导的态射.这个环的正向系统的极限是$K$本身,同态$u_n:A\times K\to K$为第二分量的投影映射.此时$S=\mathrm{Spec}K$是不可约的,但是所有$S_n$都是非连通的.
    \item\label{逆向极限和不可约性与连通性4} 推论.设$S$是诺特空间,并且满足上一条中的(a)或者(b),那么$S$是不可约的当且仅当存在指标$i$,使得对任意$i\le j$都有$S_j$是不可约的;$S$是连通的当且仅当存在指标$i$,使得对任意$i\le j$都有$S_j$是连通的
    \item\label{逆向极限和不可约性与连通性5} 推论.设有指标$i_0$使得$S_{i_0}$是拟紧的,设对任意$i\le j$有$u_{ij}:S_j\to S_i$是支配态射.那么$S$是连通的当且仅当存在指标$i$,使得对任意$i\le j$都有$S_j$是连通的(这和上一条不同在这里$S$没要求是诺特的).
    \begin{proof}
    	
    	充分性是本节的第一个命题.至于必要性,首先从$u_{ij}$总是支配的得到当$i$足够大时$u_i$总是支配的.又因为$S$是连通的,按照$u_i(S)$是$S_i$的稠密子集,当然有在$i$足够大时$S_i$是连通的.
    \end{proof}
\end{enumerate}
\subsection{逆向系统和有限表示模层}

本节我们始终做上一节开篇相同的约定:设$S_0$是概形,设$\{(\mathscr{A}_i),(\varphi_{ji}:\mathscr{A}_i\to\mathscr{A}_j)\}$是$\mathscr{O}_{S_0}$拟凝聚代数层范畴上的正向系统,它的极限记作$\{\mathscr{A},(\varphi_i:\mathscr{A}_i\to\mathscr{A})\}$.再记$S_i=\mathrm{Spec}\mathscr{A}_i$和$S=\mathrm{Spec}\mathscr{A}$.把$\varphi_{ji}$和$\varphi_i$对应的概形之间的态射记作$u_{ij}:S_j\to S_i$和$u_i:S\to S_i$.并且再做如下额外约定:
\begin{itemize}
	\item 我们知道这里逆向系统$\{S_i\mid i\in I\}$都是$S_0$概形,于是我们可以把$0$作为最小元添加到$I$中,这仍然保证$I$是有向集,并且对$i_0\le i$,定义$u_{i_0i}:S_i\to S_0$就是$S_i$作为$S_0$的结构态射,那么添加$S_0$和这些$u_{i_0i}$后仍然构成逆向系统,并且它的极限是不变的.
	\item 对每个指标$i$取一个$\mathscr{O}_{S_i}$模层$\mathscr{F}_i$,约定对任意$i\le j$有$\mathscr{F}_j=u_{ij}^*(\mathscr{F}_i)$,则$\mathscr{O}_S$模层$\mathscr{F}=u_i^*(\mathscr{F}_i)$的定义不依赖指标$i$的选取.类似的定模层$\mathscr{G}_i$和$\mathscr{G}$.
	\item 另外我们有时候做如下约定,选取一个指标$i_0$,选取一个$\mathscr{O}_{S_{i_0}}$模层$\mathscr{F}_{i_0}$,对任意$i_0\le i$定义$\mathscr{F}_i=u_{i_0i}^*(\mathscr{F}_{i_0})$,则这些$\{\mathscr{F}_i\mid i\ge i_0\}$仍然满足上面的等式$\mathscr{F}_j=u_{ij}^*(\mathscr{F}_i)$.再记$\mathscr{O}_S$模层$\mathscr{F}=u_i^*(\mathscr{F}_{i})$,这个定义不依赖于$i\ge i_0$的选取.尽管这个约定下并不是对所有指标$i$都定义了$\mathscr{F}_i$,但是我们涉及到的所有问题都只是对足够大的指标来讲的(比方说,如果$\mathscr{F}_i$是有限型拟凝聚模层或者有限表示拟凝聚模层,那么$\mathscr{F}_j=u_{ij}^*(\mathscr{F}_i)$也是有限型拟凝聚模层或者有限表示拟凝聚模层),所以这不会导致本质区别.
\end{itemize}
\begin{enumerate}
	\item\label{逆向系统和有限表示模层1} 如果$i\le j$,那么定义如下典范映射:
	\begin{align*}
		u_{ij}^*:\mathrm{Hom}_{S_i}(\mathscr{F}_i,\mathscr{G}_i)&\to\mathrm{Hom}_{S_j}(\mathscr{F}_j,\mathscr{G}_j)\\f_i&\mapsto u_{ij}^*(f_i)
	\end{align*}

    于是$\{(\mathrm{Hom}_{S_i}(\mathscr{F}_i,\mathscr{G}_i))_{i\in I},(u_{ij}^*)\}$构成了阿贝尔群范畴上的正向系统.对任意指标$i$,我们有典范同态:
    $$u_i^*:\mathrm{Hom}_{S_i}(\mathscr{F}_i,\mathscr{G}_i)\to\mathrm{Hom}_S(\mathscr{F},\mathscr{G})$$
    
    进而$\{\mathrm{Hom}_S(\mathscr{F},\mathscr{G}),(u_i^*)\}$构成了上述正向系统的一个余锥,于是泛性质诱导了阿贝尔群的典范同态:
    $$u_{\mathscr{F},\mathscr{G}}:\varinjlim\mathrm{Hom}_{S_i}(\mathscr{F}_i,\mathscr{G}_i)\to\mathrm{Hom}_S(\mathscr{F},\mathscr{G})$$
 
    特别的,如果取$\mathscr{F}_i=\mathscr{O}_{S_i}$,那么当然对$i\le j$有$\mathscr{O}_{S_j}=u_{ij}^*(\mathscr{O}_{S_i})$,并且有$\mathscr{F}=\mathscr{O}_S$,我们知道$\mathrm{Hom}_S(\mathscr{O}_S,\mathscr{G})=\Gamma(S,\mathscr{G})$,于是此时$u_{\mathscr{F},\mathscr{G}}$就是如下阿贝尔群同态:
    $$u_{\mathscr{G}}:\varinjlim\Gamma(S_i,\mathscr{G}_i)\to\Gamma(S,\mathscr{G})$$
    \item\label{逆向系统和有限表示模层2} 定理.粗略的讲,如果$S_0$是qcqs概形,对每个指标$i$,记$\mathscr{C}_i$表示有限表示$\mathscr{O}_{S_i}$拟凝聚模层范畴,记$\mathscr{C}$表示有限表示$\mathscr{O}_S$拟凝聚模层范畴.对$i\le j$,就有$u_{ij}^*:\mathscr{C}_i\to\mathscr{C}_j$是函子,并且满足余圈条件(此即$u_{ij}^*\circ u_{ki}^*=u_{kj}^*$),于是$\{(\mathscr{C}_i),(u_{ij}^*)\}$构成一个正向系统.本条定理就是说范畴$\mathscr{C}$几乎就是正向极限$\varinjlim\mathscr{C}_i$.换句话讲,给定一个有限表示$\mathscr{O}_S$拟凝聚模层$\mathscr{F}$,等同于在$i$足够大时给定有限表示$\mathscr{O}_{S_i}$拟凝聚模层$\mathscr{F}_i$,使得它以$\mathscr{F}$为原像,并且如果还有一个有限表示拟凝聚$\mathscr{O}_{S_i}$模层$\mathscr{F}_j$满足相同的事情,那么$\mathscr{F}_i$和$\mathscr{F}_j$就在某个指标更大的$S_k$里具有相同的原像.
    \begin{enumerate}
    	\item 设$S_0$是拟紧的,设对任意指标$i$有$\mathscr{F}_i$都是有限型拟凝聚模层,并且$\mathscr{G}_i$都是拟凝聚模层,那么$u_{\mathscr{F},\mathscr{G}}$是单射.
    	\item 设$S_0$是qcqs概形,设对任意指标$i$有$\mathscr{F}_i$都是有限表示拟凝聚模层,并且$\mathscr{G}_i$都是拟凝聚模层,那么$u_{\mathscr{F},\mathscr{G}}$是双射.
    	\item 设$S_0$是qcqs概形,如果我们预先任意给定一个有限表示拟凝聚$\mathscr{O}_S$模层$\mathscr{F}$,那么可以找到一个指标$i$和一个有限表示拟凝聚$\mathscr{O}_{S_i}$模层$\mathscr{F}_i$,使得$\mathscr{F}\cong u_i^*(\mathscr{F}_i)$.
    \end{enumerate}
    \begin{proof}
    	
    	我们先设$S_0=\mathrm{Spec}A_0$是仿射的来证明(a)和(b).此时所有$S_i,S$都是仿射的,它们的拟凝聚模层都是伴随模层,所以问题归结为交换代数:设$(R_i)$是环的逆向系统,逆向极限记作$R$,设$M_0,N_0$是$R_0$模,设$M_i=M_0\otimes_{R_0}R_i$和$M=M_0\otimes_{R_0}R$,类似定义$N_i$和$N$.如果$M_0$是有限$R_0$模,那么如下典范同态是单射,如果$M_0$是有限表示模,那么它是双射.
    	$$\lim\limits_{\rightarrow}\mathrm{Hom}_{R_i}(M_i,N_i)\to\mathrm{Hom}_R(M,N)$$
    	
    	这件事是因为借助如下典范同构转化后的命题是熟知的.
    	$$\mathrm{Hom}_{R_i}(M_i,N_i)=\mathrm{Hom}_{R_0}(M_0,N_i)$$
    	$$\mathrm{Hom}_R(M,N)=\mathrm{Hom}_{R_0}(M_0,N)$$
    	
    	接下来设$S_0$是拟紧的,我们来证明(a):那么它存在有限仿射开覆盖$\{U_t\mid 1\le t\le m\}$,对每个指标$i$我们记$U_{ti}=u_{0i}^{-1}(U_t)$,那么固定$i$时$\{U_{ti}\}$就构成了$S_i$的有限仿射开覆盖.记$V_t=u_0^{-1}(U_t)$,则$\{V_t\}$构成了$S$的有限仿射开覆盖.为了证明$u_{\mathscr{F},\mathscr{G}}$是单射,需要证明如果$f_i:\mathscr{F}_i\to\mathscr{G}_i$满足$f=u_i^*(f_i)=0$,那么存在指标$i\le j$,使得$f_j=u_{ij}^*(f_i)=0$.那么按照我们解决的仿射情况,对每个$t$,都存在$i\le j_t$使得$f_{j_t}$限制在$U_{t,j_t}$上为零,进而把$j_t$替换为更大的指标仍然满足这件事.于是按照有向集条件,可以取指标$i\le j$(这里用到$S_0$是拟紧的,即$t$只有有限个),满足$f_j$在每个$U_{tj}$上为零,从而在整个$S_j$上为零.
    	
    	\qquad
    	
    	再设$S_0$还是拟分离的,设$\mathscr{F}_i$都是有限表示模层,我们来证明(b):任取$\mathscr{O}_S$模层态射$f:\mathscr{F}\to\mathscr{G}$,我们已经处理了仿射情况,所以对每个$t$,都存在指标$i_t$和态射$f_{i_t}^{(t)}:\mathscr{F}_{i_t}\mid U_{t,i_t}\to\mathscr{G}_{i_t}\mid U_{t,i_t}$,并且满足$u_{i_t}^*(f_{i_t}^{(t)})=f\mid V_t$.并且对任意$i_t\le i$,都有$f_i^{(t)}=u_{i_t,i}^*(f_{i_t}^{(t)}):\mathscr{F}_i\mid U_{ti}\to\mathscr{G}_i\mid U_{ti}$仍然满足$u_i^*(f_i^{(t)})=f\mid V_t$.于是按照有向集的条件,就可以取一个统一的指标$i$,使得上述等式对任意$t$都成立.我们接下来就要把$\{f_i^{(t)}\mid1\le t\le m\}$粘合成一个态射,并说明它是$f$的原像.
    	
    	\qquad
    	
    	按照$S_0$是拟分离的得到$S_i$是拟分离的,于是对任意$1\le s,t\le m$,就有$U_{sti}=U_{si}\cap U_{ti}$是拟紧的.并且有$u_i^*(f_i^{(t)}\mid U_{sti})=u_i^*(f_i^{(s)}\mid U_{sti})=f\mid(V_s\cap V_t)$.于是按照我们已经证明的$u_{\mathscr{F},\mathscr{G}}$是单射,就存在指标$i\le i_{st}$,使得对任意$i_{st}\le j$,都有$u_{ij}^*(f_i^{(s)}\mid U_{sti})=u_{ij}^*(f_i^{(t)}\mid U_{sti})$.那么我们可以选取一个指标$k\ge$每个$i_{st},1\le s,t\le m$,那么对任意$j\ge k$就有$\{u_{ij}^*(f_i^{(t)})\mid 1\le t\le m\}$在定义域相交的地方是相同的,所以它们可以粘合为一个态射$f_j:\mathscr{F}_j\to\mathscr{G}_j$,并且满足$f=u_j^*(f_j)$.
    	
    	\qquad
    	
    	接下来我们证明(c):首先$S_0=\mathrm{Spec}A_0$是仿射的情况我们已经在环的正向极限那节已经证明过了.下面对一般的qcqs概形$S_0$,任取$S$的有限表示拟凝聚模层$\mathscr{F}$.取$S_0$的有限仿射开覆盖$\{U_t\mid1\le t\le m\}$.对任意指标$i$记$U_{ti}=u_{0i}^{-1}(U_t)$,那么$\{U_{ti}\mid1\le t\le m\}$就是$S_i$的有限仿射开覆盖.再记$V_t=u_0^{-1}(U_t)$,那么$\{V_t\mid1\le t\le m\}$是$S$的有限仿射开覆盖.按照已经得证的仿射情况,存在指标$i_t$和有限表示拟凝聚$\mathscr{O}_{U_{ti_t}}$模层$\mathscr{F}_t$满足$u_{i_t}^*(\mathscr{F}_t)=\mathscr{F}\mid V_t$.于是我们可以选取一个指标$i\ge i_t,\forall1\le t\le m$,以及有限表示拟凝聚$\mathscr{O}_{U_{ti}}$模层$\mathscr{F}_t^{(i)},1\le t\le m$,满足$u_i^*(\mathscr{F}_t^{(i)})=\mathscr{F}\mid V_t$.按照$S_i$是qcqs概形,对任意$1\le s,t\le m$,记$U_{sti}=U_{si}\cap U_{ti}$,那么这也是qcqs概形.按照$u_i^*(\mathscr{F}_t^{(i)}\mid U_{sti})=u_i^*(\mathscr{F}_s^{(i)}\mid U_{sti})=\mathscr{F}\mid(V_s\cap V_t)$,从下一条推论中的(b)得到(因为下一条的推论(b)只用到了本条的(a)和(b),所以没有循环论证)存在指标$i_{st}\ge i$,使得有模层同构$\theta_{st}:u_{i,i_{st}}^*(\mathscr{F}_t^{(i)}\mid U_{sti})\cong u_{i,i_{st}}^*(\mathscr{F}_s^{(i)}\mid U_{sti})$,并且有$u_{i_{st}}^*(\theta_{st})$是模层$\mathscr{F}\mid(V_s\cap V_t)$的自同构.接下来对$1\le r,s,t\le m$,记$U_{rst,i}=U_{ri}\cap U_{si}\cap U_{ti}$,那么这也是拟紧的,再把$\theta_{st},\theta_{sr},\theta_{tr}$限制在$U_{rst,i}$上为$\theta_{st}',\theta_{sr}',\theta_{tr}'$,那么有$u_i^*(\theta'_{st}\circ\theta_{tr}')=u_i^*(\theta_{sr}')$.于是按照(a),存在$i\le j$使得$u_{ij}^*(\theta'_{st}\circ\theta_{tr}')=u_{ij}^*(\theta_{sr}')$,此为余圈条件,于是这些$S_j$的开子集$U_{tj}$上的有限表示拟凝聚层就可以粘合为整个$S_j$上的有限表示拟凝聚层$\mathscr{F}_j$,并且满足$\mathscr{F}\cong u_j^*(\mathscr{F}_j)$.从而得证.
    \end{proof}
    \item\label{逆向系统和有限表示模层3} 推论.
    \begin{enumerate}[(1)]
    	\item 设$S_0$是拟紧的,$\mathscr{F}_i$总是有限型拟凝聚模层,$\mathscr{G}_i$总是有限表示拟凝聚模层.取一个指标$i_0$,取一个同态$f_{i_0}:\mathscr{F}_{i_0}\to\mathscr{G}_{i_0}$,对任意$i_0\le i$记$f_i=u_{i_0i}^*(f_{i_0})$,再记$f=u_{i_0}^*(f_{i_0})$,那么$f:\mathscr{F}\to\mathscr{G}$是同构当且仅当存在$i_0\le i$使得$f_i=u_{i_0i}^*(f_{i_0})$是同构(从而对任意$i\le j$都有$f_j=u_{i_0j}^*(f_{i_0})$是同构).
    	\begin{proof}
    		
    		我们可以不妨设这里的$i_0$就是指标$0$.倘若我们对仿射情况解决了这个问题,取$S_0$的有限仿射开覆盖$\{U_t\mid1\le t\le m\}$,则对每个$U_t$都可以找到指标$i_t$使得$f_{i_t}$可以限制为$\mathscr{F}_{i_t}\mid U_{ti}\to\mathscr{G}_{i_t}\mid U_{ti}$的同构.所以当我们选取$i\ge i_t,\forall1\le t\le m$,则$f_i$就是一个同构(这里不涉及粘合问题,因为$f_i$已经存在了).综上问题归结为设$S_0$是仿射的情况.
    		
    		\qquad
    		
    		特别的此时$S_0$是qcqs概形.充分性显然是成立的,对于必要性,设$f$是同构,那么存在$\mathscr{O}_S$模层态射$g:\mathscr{G}\to\mathscr{F}$满足$g\circ f=1_{\mathscr{F}}$和$f\circ g=1_{\mathscr{G}}$.那么按照$\mathscr{G}$是有限表示拟凝聚模层,就有$u_{\mathscr{G},\mathscr{F}}$是满射,于是可以找到指标$i$和态射$g_i:\mathscr{G}_i\to\mathscr{F}_i$满足$g=u_i^*(g_i)$.进而有$u_i^*(g_i\circ f_i)=1_{\mathscr{F}}=u_i^*(1_{\mathscr{F}_i})$和$u_i^*(f_i\circ g_i)=1_{\mathscr{G}}=u_i^*(1_{\mathscr{G}_i})$.又因为$\mathscr{F}_i$和$\mathscr{G}_i$都是有限型拟凝聚层,于是$u_{\mathscr{F},\mathscr{F}}$和$u_{\mathscr{G},\mathscr{G}}$都是单射,于是存在$i\le j$,记$f_j=u_{ij}^*(f_i)$和$g_j=u_{ij}^*(g_i)$,那么就有$g_j\circ f_j=1_{\mathscr{F}_j}$和$f_j\circ g_j=1_{\mathscr{G}_j}$,这就得到$f_j$是同构.
    	\end{proof}
        \item 设$S_0$是qcqs概形,设$\mathscr{F}_i$和$\mathscr{G}_i$都是有限表示拟凝聚$\mathscr{O}_{S_i}$模层,那么$\mathscr{F}\cong\mathscr{G}$当且仅当存在指标$i\le j$使得$\mathscr{F}_j\cong\mathscr{G}_j$.更具体的,对任意同构$f:\mathscr{F}\cong\mathscr{G}$,都存在指标$i\le j$和一个同构$f_j:\mathscr{F}_j\cong\mathscr{G}_j$满足$f=u_j^*(f_j)$.
        \item 设$S_0$是qcqs概形,对任意拟凝聚$\mathscr{O}_S$模层$\mathscr{G}$,都有如下典范同态是同构:
        $$u_{\mathscr{G}}:\varinjlim\Gamma(S_i,\mathscr{G}_i)\cong\Gamma(S,\mathscr{G})$$
    \end{enumerate}
    \item\label{逆向系统和有限表示模层4} 设$S_0$是拟紧的,取一个指标$i_0$,设$\mathscr{F}_{i_0}$是有限表示拟凝聚$\mathscr{O}_{S_{i_0}}$模层.对$i\ge i_0$定义$\mathscr{F}_i=u_{i_0i}^*(\mathscr{F}_{i_0})$,再定义$\mathscr{F}=u_i^*(\mathscr{F}_i)$,并且这个定义不依赖于$i\ge i_0$的选取.那么$\mathscr{F}$是局部自由模层/局部秩$n$自由模层,当且仅当存在指标$i\ge i_0$使得$\mathscr{F}_i$也是如此(进而对任意$j\ge i$有$\mathscr{F}_j$也是如此).
    \begin{proof}
    	
    	充分性是平凡的,因为按照$u_i$是仿射的,局部自由/局部秩$n$自由的模层$\mathscr{F}_i$的回拉$\mathscr{F}=u_i^*(\mathscr{F}_i)$就也是局部自由/局部秩$n$自由的.下面证明必要性,如果$\mathscr{F}$是局部自由的,那么存在$S$的有限仿射开覆盖$\{V_t\mid1\le t\le m\}$,使得对任意$t$有$\mathscr{F}\mid V_t$总同构于某个$\mathscr{O}_S^{n_t}\mid V_t$.我们解释过对于拟紧开集$V_t$,可以找到一个统一的指标$i\ge i_0$,以及$S_i$的拟紧开子集$U_{ti}$,满足对任意$t$有$V_t=u_i^{-1}(U_{ti})$.而这里$U_{ti}$是有限个仿射开子集的并$\{U_{ti}^{(s)}\}$,所以一旦我们证明了$S_0$是仿射的并且$\mathscr{F}$是自由模层的情况成立,按照有向集条件就可以选取一个足够大的指标$i$使得总有$\mathscr{F}_i\mid U_{ti}^{(s)}\cong\mathscr{O}_{S_i}^{n_t}\mid U_{ti}^{(s)}$.于是$\mathscr{F}_i$是局部自由模层.接下来设$S_0$是仿射的,设$\mathscr{F}=\mathscr{O}_S^n$,特别的此时$S_0$是qcqs概形,那么从$u_i^*(\mathscr{F}_i)=u_i^*(\mathscr{O}_{S_i}^n)$和上面的定理就得到有$i\le j$使得$\mathscr{F}_j=\mathscr{O}_{S_j}^n$得证.最后局部秩$n$自由的情况也即所有$n_t\equiv n$的情况,这依旧归结为仿射情况,于是得证.
    \end{proof}
    \item\label{逆向系统和有限表示模层5} 设$S_0$是拟紧的,设有指标$i_0$和$\mathscr{O}_{S_0}$模层态射$\mathscr{F}_{i_0}\to\mathscr{G}_{i_0}$和$\mathscr{G}_{i_0}\to\mathscr{H}_{i_0}$,满足它们的复合是零态射.并且$\mathscr{F}_{i_0}$和$\mathscr{G}_{i_0}$是有限型拟凝聚层,$\mathscr{H}_{i_0}$是有限表示拟凝聚层.定义$\mathscr{F}_i=u_{i_0i}^*(\mathscr{F}_{i_0})$,再定义$\mathscr{F}=u_i^*(\mathscr{F}_i)$,并且这个定义不依赖于$i\ge i_0$的选取.类似的定义$\mathscr{G}_i,\mathscr{H}_i,\mathscr{G},\mathscr{H},i\ge i_0$.那么诱导的$\mathscr{O}_S$模层态射$\mathscr{F}\to\mathscr{G}\to\mathscr{H}\to0$是正合列,当且仅当存在指标$i\ge i_0$有诱导的$\mathscr{O}_{S_i}$模层态射$\mathscr{F}_i\to\mathscr{G}_i\to\mathscr{H}_i\to0$是正合列(进而对任意$i\le j$有$\mathscr{O}_{S_j}$模层态射$\mathscr{F}_j\to\mathscr{G}_j\to\mathscr{H}_j\to0$是正合列).
    \begin{proof}
    	
    	充分性是因为逆像函子$u_i^*$和$u_{ij}^*$都是右正合的.下面证明必要性,首先按照$S_0$是拟紧的,以及$\mathscr{F}_i$是有限型的,就有$u_{\mathscr{F},\mathscr{H}}$是单射,于是当指标足够大时就有复合$\mathscr{F}_i\to\mathscr{G}_i\to\mathscr{H}_i$是零态射.再设$\mathscr{H}_i'=\mathrm{coker}(\mathscr{F}_i\to\mathscr{G}_i)$,那么由于上述复合是零态射,就诱导了一个模层态射$f_i:\mathscr{H}_i'\to\mathscr{H}_i$.则按照条件,有$u_i^*(f_i)=f$是同构,那么按照前面结论,就可以取一个足够大的指标$j$使得$u_{ij}^*(f_i)=f_j$是同构.
    \end{proof}
    \item\label{逆向系统和有限表示模层6} 推论.设$S_0$是拟紧的,取指标$i_0$,设$\mathscr{F}_{i_0}$是拟凝聚$\mathscr{O}_{S_{i_0}}$模层,$\mathscr{G}_{i_0}$是有限型拟凝聚$\mathscr{O}_{S_{i_0}}$模层.那么模层态射$f=u_{i_0}^*(f_{i_0}):\mathscr{F}\to\mathscr{G}$是满态射当且仅当存在指标$i\ge i_0$使得$f_i=u_{i_0i}^*(f_{i_0}):\mathscr{F}_i\to\mathscr{G}_i$是满态射(进而对任意$i\le j$有$f_j=u_{ij}^*(f_i)$是满态射).
    \begin{proof}
    	
    	此为在上一条中取$0\to\mathscr{H}_i\to0\to0$,其中$\mathscr{H}_i=\mathrm{coker}(f_i)$,这也是有限型拟凝聚模层.
    \end{proof}
    \item\label{逆向系统和有限表示模层7} 推论.设$S_0$是拟紧的,设每个$u_{ij}:S_j\to S_i,i\le j$都是平坦的.
    \begin{enumerate}
    	\item 设$\xymatrix{\mathscr{F}_{i_0}\ar[r]^{f_{i_0}}&\mathscr{G}_{i_0}\ar[r]^{g_{i_0}}&\mathscr{H}_{i_0}}$是拟凝聚$\mathscr{O}_{S_{i_0}}$模层之间的态射.满足$\mathrm{im}f_{i_0}$和$\ker g_{i_0}$是有限型的.那么它诱导的$\mathscr{O}_S$模层之间的态射$\xymatrix{\mathscr{F}\ar[r]^f&\mathscr{G}\ar[r]^g&\mathscr{H}}$是正合的当且仅当存在$i\ge i_0$使得$\xymatrix{\mathscr{F}_i\ar[r]^{f_i}&\mathscr{G}_i\ar[r]^{g_i}&\mathscr{H}_i}$是正合的(进而对任意$i\le j$都有$\xymatrix{\mathscr{F}_j\ar[r]^{f_j}&\mathscr{G}_j\ar[r]^{g_j}&\mathscr{H}_j}$是正合的).
    	\begin{proof}
    		
    		首先我们解释过这里$u_{ij}$都是平坦态射可以推出$u_i$都是平坦态射.另外我们知道一般的如果$f:X\to Y$是平坦态射,那么$f^*:\textbf{QCoh}(Y)\to\textbf{QCoh}(X)$的正合函子.特别的,正合函子与$\ker$和$\mathrm{im}$都可交换.于是特别的我们有$u_i^*(\ker f_i)=\ker f$和$u_i^*(\mathrm{im}g_i)=\mathrm{im}g$.于是充分性只要作用函子$u^*_i$即可.对于必要性,设$\xymatrix{\mathscr{F}\ar[r]^f&\mathscr{G}\ar[r]^g&\mathscr{H}}$是正合的.那么$\mathrm{im}f\to\mathscr{H}$是零态射,于是按照$\mathrm{im}f$是有限型以及$S_0$是拟紧的,得到$u_{\mathrm{im}f,\mathscr{H}}$是单射,于是从$\mathrm{im}f\to\mathscr{H}$是零态射得到当$i$足够大时有$\mathrm{im}f_i\to\mathrm{H}_i$也是零态射,也即$g_i\circ f_i=0$.接下来按照$\mathrm{im}f=\ker g$,得到态射$\mathscr{F}\to\ker g$是满射,那么按照$\ker g$是有限型的,上一条推论就告诉我们当$i$足够大时有$\mathrm{F}_i\to\ker g_i$是满射,也即$\mathrm{im}f_i=\ker g_i$,于是综上当$i$足够大时$\xymatrix{\mathscr{F}_i\ar[r]^{f_i}&\mathscr{G}_i\ar[r]^{g_i}&\mathscr{H}_i}$是正合的.
    	\end{proof}
    	\item 设拟凝聚$\mathscr{O}_{S_{i_0}}$模层态射$f_{i_0}:\mathscr{F}_{i_0}\to\mathscr{G}_{i_0}$满足$\ker f_{i_0}$是有限型的.那么$f=u_i^*(f_i):\mathscr{F}\to\mathscr{G}$是单态射当且仅当存在$i\ge i_0$使得$f_i=u_{i_0i}^*(f_{i_0})$是单态射(进而对任意$i\le j$有$f_j=u_{ij}^*(f_i)$是单态射).
    	\begin{proof}
    		
    		这只要把(a)的结论用在$0\to\mathscr{F}_i\to\mathscr{G}_{i_0}$上即可.
    	\end{proof}
    \end{enumerate}
    \item\label{逆向系统和有限表示模层8} 设$S_0$是拟紧的,设$\mathscr{F}_{i_0}$是有限型拟凝聚模层,设$\mathscr{G}_{i_0}'$和$\mathscr{G}_{i_0}''$是$\mathscr{F}_{i_0}$的两个拟凝聚商模层,其中$\mathscr{G}_{i_0}'$是有限表示模层.那么$\mathscr{G}''$是$\mathscr{G}'$的商模层,当且仅当存在$i\ge i_0$使得$\mathscr{G}''_i$是$\mathscr{G}_i'$的商模层(从而对任意$i\le j$有$\mathscr{G}''_j$是$\mathscr{G}_j'$的商模层).
    \begin{proof}
    	
    	按照条件,存在两个满态射$p'_{i_0}:\mathscr{F}_{i_0}\to\mathscr{G}_{i_0}'$和$p''_{i_0}:\mathscr{F}_{i_0}\to\mathscr{G}_{i_0}''$.那么按照$u_i^*$和$u_{i_0i}^*$的右正合性,就有$p'_i=u_{i_0i}^*(p_{i_0}')$,$p''_i=u_{i_0i}^*(p_{i_0}'')$,$p'=u_i^*(p_i')$,$p''=u_i^*(p_i'')$都是满态射.另外$\mathscr{G}''$是$\mathscr{G}'$的商模层等价于讲存在态射$f:\mathscr{G}'\to\mathscr{G}''$满足$p''=f\circ p'$(满足这个等式的$f$自动是满态射,因为$p''$是满态射),由于$p'$是满态射,导致满足这个等式的$f$是唯一的.这个唯一性说明,如果我们在仿射局部上找到了满足这个等式的态射$f$,那么这些局部上的$f$在相交的地方是自动一致的,从而它们可以粘合为一个唯一满足这个等式的定义在整个$S$上的态射.综上我们解释了验证$\mathscr{G}''$是$\mathscr{G}'$的商模层只需在$S_0$仿射的情况下验证即可.同理验证$\mathscr{G}_i''$是$\mathscr{G}_i'$的商模层也只要在$S_0$仿射的情况下验证.综上我们不妨设$S_0$是仿射的.
    	
    	\qquad
    	
    	充分性是平凡的,只要在$p_i''=f_i\circ p_i'$上作用$u_i^*$,按照$u_i^*$把满态射映射为满态射,就得到$f=u_i^*(f_i)$是满态射.对于必要性,此时已经存在满足$p''=f\circ p'$的满态射$f$,那么按照$S_0$是qcqs概形,以及$\mathscr{G}''$是有限表示模层,就有$u_{\mathscr{G}',\mathscr{G}''}$是满射,于是存在足够大的指标$i$以及态射$f_i$满足$p''_i=f_i\circ p'_i$.完成证明.
    \end{proof}
    \item\label{逆向系统和有限表示模层9} 对概形上的拟凝聚层$\mathscr{F}$,我们用$\textbf{Q}(\mathscr{F})$表示$\mathscr{F}$的所有有限表示拟凝聚商模构成的集合.设$\mathscr{F}_i$是拟凝聚$\mathscr{O}_{S_i}$模层,任取$\mathscr{G}_i\in\textbf{Q}(\mathscr{F}_i)$,按照$u_i^*$和$u_{ij}^*$都是右正合的,就有$\mathscr{G}_j=u_{ij}^*(\mathscr{G}_i)\in\textbf{Q}(\mathscr{F}_j)$和$\mathscr{G}=u_{ij}^*(\mathscr{G})\in\textbf{Q}(\mathscr{F})$.这导致$\{(\textbf{Q}(\mathscr{F}_i))_{i\in I},(u_{ij}^*)_{i\le j}\}$构成集合范畴上的正向系统,并且$\{\textbf{Q}(\mathscr{F}),(u_i^*)\}$构成它的一个余锥,于是泛性质诱导了典范映射:
    $$u_{\textbf{Q}}:\varinjlim\textbf{Q}(\mathscr{F}_i)\to\textbf{Q}(\mathscr{F})$$
    
    如果$(\mathscr{F}_i')$是另一组拟凝聚$\mathscr{O}_{S_i}$模层,满足对任意$i$有$\mathscr{F}_i'$是$\mathscr{F}_i$的商模层,那么$\mathscr{F}'$就是$\mathscr{F}$的商模层,并且有如下交换图表:
    $$\xymatrix{\varinjlim\textbf{Q}(\mathscr{F}_i')\ar[rr]^{u_{\textbf{Q}}}\ar[d]&&\textbf{Q}(\mathscr{F}')\ar[d]\\\varinjlim\textbf{Q}(\mathscr{F}_i)\ar[rr]_{u_{\textbf{Q}}}&&\textbf{Q}(\mathscr{F})}$$
    \begin{enumerate}
    	\item 如果$S_0$是拟紧的,并且每个$\mathscr{F}_i$都是有限型拟凝聚模层,那么$u_{\textbf{Q}}$是单射.
    	\item 如果$S_0$是qcqs的,并且每个$\mathscr{F}_i$是有限表示拟凝聚模层,那么$u_{\textbf{Q}}$是双射.
    \end{enumerate}
    \begin{proof}
    	
    	上一条命题已经推出(a)成立.接下来设$\mathscr{G}$是$\mathscr{F}$的商模,并且是有限表示拟凝聚模层.按照$S_0$是qcqs概形,我们解释过这些条件下存在足够大的指标$i$使得存在有限表示拟凝聚$\mathscr{O}_{S_i}$模层$\mathscr{G}_i$,满足$\mathscr{G}=u_i^*(\mathscr{G}_i)$.记典范满态射$p:\mathscr{F}\to\mathscr{G}$,那么按照$u_{\mathscr{F},\mathscr{G}}$是满射,可以把$i$替换为更大的指标使得$p_i:\mathscr{F}_i\to\mathscr{G}_i$满足$p=u_i^*(p_i)$.我们解释过此时从$p$是满态射,可以把$i$再替换为更大的指标使得$p_i$是满态射,这就得证.
    \end{proof}
\end{enumerate}
\subsection{逆向系统和有限表示子概型}

我们做上节相同的约定:设$S_0$是概形,设$\{(\mathscr{A}_i),(\varphi_{ji}:\mathscr{A}_i\to\mathscr{A}_j)\}$是$\mathscr{O}_{S_0}$拟凝聚代数层范畴上的正向系统,它的极限记作$\{\mathscr{A},(\varphi_i:\mathscr{A}_i\to\mathscr{A})\}$.再记$S_i=\mathrm{Spec}\mathscr{A}_i$和$S=\mathrm{Spec}\mathscr{A}$.把$\varphi_{ji}$和$\varphi_i$对应的概形之间的态射记作$u_{ij}:S_j\to S_i$和$u_i:S\to S_i$.并且约定指标$0$作为最小元也在指标集$I$中(见上一节开头).
\begin{enumerate}
	\item\label{逆向系统和有限表示子概型1} 设$Y$是概形,用$\textbf{Sub}(Y)$表示$Y$的所有有限表示子概型构成的集合,那么其上自然具备一个偏序,即$X_1\le X_2$表示$X_1$是$X_2$的子概型.用$\textbf{OSub}(Y)$和$\textbf{FSub}(Y)$分别表示$Y$的所有有限表示开子概型和有限表示闭子概型构成的偏序集.设$Z$是$Y$的子概型
	\begin{enumerate}
		\item $Z\in\textbf{OSub}(Y)$当且仅当$Z$是$Y$的反紧开子集.
		\item $Z\in\textbf{FSub}(Y)$当且仅当$Z$是$Y$的闭子概型,并且对应的拟凝聚理想层是有限型的.
	\end{enumerate}
	\item\label{逆向系统和有限表示子概型2} 取一个指标$i_0$,取$S_{i_0}$的子概形$Y_{i_0}$,对任意$i\ge i_0$记$Y_i=u_{i_0i}^{-1}(Y_{i_0})$和$Y=u_i^{-1}(Y_i)$(不依赖$i\ge i_0$的选取).我们知道开嵌入和闭嵌入和有限表示态射都在基变换下不变,于是我们有如下三个集合范畴上的正向系统:
	\begin{align*}
		\{(\textbf{Sub}(S_i))_{i\in I},&(u_{ij}^{-1})_{i\le j}\}\\
		\{(\textbf{OSub}(S_i))_{i\in I},&(u_{ij}^{-1})_{i\le j}\}\\
		\{(\textbf{FSub}(S_i))_{i\in I},&(u_{ij}^{-1})_{i\le j}\}
	\end{align*}

    进而如下集合依次构成这些正向系统的余锥:
    \begin{align*}
    	\{\textbf{Sub}(Sub),&(u_i^{-1})_{i\in I}\}\\
    	\{\textbf{Sub}(OSub),&(u_i^{-1})_{i\in I}\}\\
    	\{\textbf{Sub}(FSub),&(u_i^{-1})_{i\in I}\}
    \end{align*}
	
	进而泛性质诱导了如下典范映射:
	\begin{align*}
		u_{\textbf{Sub}}:\varinjlim\textbf{Sub}(S_i)&\to\textbf{Sub}(S)\\
		u_{\textbf{OSub}}:\varinjlim\textbf{OSub}(S_i)&\to\textbf{OSub}(S)\\
		u_{\textbf{FSub}}:\varinjlim\textbf{FSub}(S_i)&\to\textbf{FSub}(S)
	\end{align*}

    另外上述集合和映射都具有序结构:我们知道$\textbf{Sub}(S_i)$上以子概型偏序构成偏序集.这使得正向极限$\varinjlim\textbf{Sub}(S_i)$也构成偏序集,给定两个指标$i,j$和$Y_i\in\textbf{Sub}(S_i)$和$Z_j\in\textbf{Sub}(S_j)$,那么可以找到指标$k\ge i,j$,记$Y_k=u_{ik}^{-1}(Y_i)$和$Z_k=u_{jk}^{-1}(Z_j)$,如果$Y_k\le Z_k$,就定义$u_i^{-1}(Y_i)\le u_j^{-1}(Z_j)$.容易验证这个偏序不依赖于代表元的选取,从而我们定义了$\varinjlim\textbf{Sub}(S_i)$上的偏序,并且明显的$u_{\textbf{Sub}}$是保序映射.
    \item\label{逆向系统和有限表示子概型3} 定理.
    \begin{enumerate}
    	\item 如果$S_0$是拟紧的,那么$u_{\textbf{Sub}},u_{\textbf{OSub}},u_{\textbf{FSub}}$都是单射.
    	\item 如果$S_0$是qcqs的,那么$u_{\textbf{Sub}},u_{\textbf{OSub}},u_{\textbf{FSub}}$都是双射.
    \end{enumerate}
    \begin{proof}
    	
    	在\ref{逆向系统和有限表示模层9}中取$\mathscr{F}_i=\mathscr{O}_{S_i}$就得到关于$u_{\textbf{FSub}}$的结论.另外我们这里反紧开集是可构造开集,于是从\ref{逆向极限和可构造集5}得到关于$u_{\textbf{OSub}}$的结论.最后我们来处理关于$u_{\textbf{Sub}}$的结论.先设$S_0$是qcqs概形,我们来证明$u_{\textbf{Sub}}$是满射:设$Z$是$S$的有限表示子概型,那么从$S$是拟紧的就得到$Z$也是拟紧的.那么存在$S$的拟紧开集$U$包含了$Z$,满足$Z$是$U$的闭子概型.那么按照\ref{概形的逆向极限5}知存在指标$i$和$S_i$的拟紧开子集$U_i$满足$U=u_i^{-1}(U_i)$.再按照$S_i$是拟分离的,就有它的拟紧开子集$U_i$也是拟分离的.综上我们把问题归结为$U=S$的情况,但是此为$u_{\textbf{FSub}}$的结论.最后要证明$S_0$拟紧的情况下$u_{\textbf{Sub}}$是单射,归结为证明下一条推论.
    \end{proof}
    \item\label{逆向系统和有限表示子概型4} 推论.设$S_0$是拟紧的,设有指标$i_0$,和$S_{i_0}$的两个有限表示子概型$Z_{i_0}'$和$Z_{i_0}''$.记$Z'=u_{i_0}^{-1}(Z_i')$和$Z_i'=u_{i_0i}^{-1}(Z_{i_0}'),i\ge i_0$,同理记$Z_i''$和$Z''$.那么$Z'$是$Z''$的子概型当且仅当存在指标$i\ge i_0$使得$Z_i'$是$Z_i''$的子概型(从而对任意$i\le j$有$Z_j'$是$Z_j''$的子概型).
    \begin{proof}
    	
    	充分性是平凡的,因为嵌入在基变换下不变.下面证明必要性,首先$Z_{i_0}'$和$Z_{i_0}''$都是$S_{i_0}$的可构造集(这是因为Chevalley定理,有限表示态射的像集总是终端的可构造集).那么从$Z'\subseteq Z''$就存在指标$i\ge i_0$使得$Z_i'\subseteq Z_i''$.按照定义,有$Z_i'$和$Z_i''$都是$S_i$的拟紧子集.下面对任意$x\in Z_i'$,可以找到$x$在$S_i$中的拟紧开邻域$V_x$,使得$V_x\cap Z_i'$和$V_x\cap Z_i''$都是$V_x$的闭子集.这些$\{V_x,x\in S_i\}$构成了$S_i$的开覆盖,于是从$Z_i'$是拟紧的,就可以取有限个$\{V_{x_i},1\le i\le n\}$覆盖了$Z_i'$,把它们的并记作$U_i$,于是我们找到了$S_i$的拟紧开集$U_i$,使得$Z_i'\subseteq U_i$,并且$Z_i''\cap U_i$是$U_i$的闭子集.记$Z_i''$在开子集$U_i\cap Z_i''$上诱导的开子概型为$Y_i$,记$Y_j=u_{ij}^{-1}(Y_i)$和$Y=u_i^{-1}(Y_i)$,则$Y_j$就是$Z_j''$的开子集$U_j\cap Z_j''$上的开子概型,$Y$就是$Z''$在开子集$U\cap Z''$上的开子概型.按照$Z'$是$Z''$的子概型,并且$Z'\subseteq U'$,于是$Z'$也是$Y$的子概型.进而我们归结为证明在指标$i$足够大时有$Z_i'$是$Y_i$的子概型.我们还可以用$U_i$替换$S_i$,于是问题归结为设$Z_i'$和$Z_i''$都是$S_i$的闭子概型,这又归结为上一条中我们证明过的$u_{\textbf{FSub}}$在$S_0$拟紧时是单射.
    \end{proof}
    \item\label{逆向系统和有限表示子概型5} 推论.设$S_0$是拟紧的,取一个指标$i_0$,设$Z_{i_0}$是$S_{i_0}$的有限表示子概型,那么$Z=u_{i_0}^{-1}(Z_{i_0})$是$S$的开子概型/闭子概型,当且仅当存在指标$i\ge i_0$使得$Z_i=u_{i_0i}^{-1}(Z_{i_0})$是$S_i$的开子概型/闭子概型(进而对任意$i\le j$都有$Z_j=u_{ij}^{-1}(Z_i)$是$S_j$的开子概型/闭子概型).
    \begin{proof}
    	
    	我们取$S_{i_0}$的有限仿射开覆盖$\{U_{i_0}^(t),1\le t\le m\}$,对任意$i\ge i_0$记$U_i^{(t)}=u_{i_0i}^{-1}(U_{i_0}^{(t)})$和$U^{(t)}=u_i^{-1}(U_i^{(t)})$,这个定义不依赖$i\ge i_0$的选取.如果$Z$是$S$的开集/闭集,那么$Z\cap U^{(t)}$就总是$U^{(t)}$的开集/闭集.结合指标集是有向集,我们不妨设$S_{i_0}$是仿射的,于是特别的它是qcqs概形,于是按照前面证明的$u_{\textbf{OSub}}$和$u_{\textbf{FSub}}$都是双射就得到结论.
    \end{proof}
\end{enumerate}
\subsection{逆向极限和既约概形与整概形}

我们做上节相同的约定:设$S_0$是概形,设$\{(\mathscr{A}_i),(\varphi_{ji}:\mathscr{A}_i\to\mathscr{A}_j)\}$是$\mathscr{O}_{S_0}$拟凝聚代数层范畴上的正向系统,它的极限记作$\{\mathscr{A},(\varphi_i:\mathscr{A}_i\to\mathscr{A})\}$.再记$S_i=\mathrm{Spec}\mathscr{A}_i$和$S=\mathrm{Spec}\mathscr{A}$.把$\varphi_{ji}$和$\varphi_i$对应的概形之间的态射记作$u_{ij}:S_j\to S_i$和$u_i:S\to S_i$.再约定指标$0$作为最小元也在指标集$I$中(见上节开头).
\begin{enumerate}
	\item\label{逆向极限和既约概形与整概形1} 设$S$不是既约概形,那么存在指标$i_0$使得对任意$i\ge i_0$都有$S_i$不是既约概形.
	\begin{proof}
		
		我们知道既约是局部的性质,所以归结为设$S_0=\mathrm{Spec}A_0$是仿射的,此时$S_i=\mathrm{Spec}A_i$和$S=\mathrm{Spec}A$都是仿射的,其中$A=\varinjlim A_i$是$A_0$代数范畴中的正向极限.我们在环的正向极限中解释了$A$的幂零根恰好就是$A_i$的幂零根的正向极限.于是倘若$A$存在非平凡幂零元$a\not=0$,则有足够大的指标$i$和$A_i$的非平凡幂零元$a_i$满足$\varphi_i(a_i)=a$,进而对任意$i\le j$都有$a_j=\varphi_{ji}(a_i)$也是非平凡的幂零元.于是对任意$i\le j$有$A_j$不是既约环.
	\end{proof}
    \item\label{逆向极限和既约概形与整概形2} 设如下两个条件之一成立:
    \begin{enumerate}
    	\item $S_0$是拟紧的,$\mathscr{O}_{S_0}$的幂零理想层$\mathscr{N}_0$是有限型的(比方说$S_0$是诺特的),并且态射$u_{ij}:S_j\to S_i$总是开嵌入.
    	\item 态射$u_{ij}:S_j\to S_i$总是忠实平坦态射.
    \end{enumerate}

    此时$S$是既约概形当且仅当存在指标$i_0$使得对任意$i\ge i_0$都有$S_i$是既约概形(但是事实上按照上一条,充分性不需要这些额外条件).
    \begin{proof}
    	
    	充分性就是上一条,下面证明必要性.我们解释过从$u_{ij},i\le j$是忠实平坦态射可以推出$u_i$是忠实平坦态射.另外如果$f:X\to Y$是平坦态射,取$x\in X$,记$y=f(x)$,则从$\mathscr{O}_{X,x}$是既约环可以推出$\mathscr{O}_{Y,y}$是既约环,于是对于忠实平坦态射,从$S$是既约的就得到$S_i$是既约的(因为既约概形等价于每个局部环都是既约环,所以这里的确用到了忠实平坦态射是满射).这解决了(b)的情况.
    	
    	\qquad
    	
    	下面证明(a)条件下的必要性.按照\ref{概形的逆向极限9}的(e),我们知道如果对$i\le j$把$S_j$典范的视为$S_i$的开子概型,把$S$典范的视为$S_i$的开子概型,那么$S$在拓扑上就是$\{S_i\}$的交.并且$\mathscr{O}_S$就是$\mathscr{O}_{S_i}$在拓扑上的在$S$上的回拉(此即如果记$u_i:S\to S_i$的连续映射部分为$\psi_i:S\to S_i$,那么$\mathscr{O}_S=\psi_i^*(\mathscr{O}_{S_i})$).于是特别的对任意$s\in S$,局部环$\mathscr{O}_{S,s}$等同于$\mathscr{O}_{S_i,s}$.【需要EGAIV8.2.13】
    \end{proof}
    \item\label{逆向极限和既约概形与整概形3} 推论.设如下两个条件之一成立:
    \begin{enumerate}
    	\item $S_0$是诺特的,并且态射$u_{ij}:S_j\to S_i$总是开嵌入.
    	\item 态射$u_{ij}:S_j\to S_i$总是忠实平坦态射.
    \end{enumerate}
    
    此时$S$是整概形当且仅当存在指标$i_0$使得对任意$i\ge i_0$都有$S_i$是整概形.
    \begin{proof}
    	
    	因为整概形等价于不可约的既约概形,我们分别处理了既约概形和不可约概形的情况.
    \end{proof}
\end{enumerate}
\subsection{逆向极限和有限表示态射}

我们做上一节相同的约定:设$S_0$是概形,设$\{(\mathscr{A}_i),(\varphi_{ji}:\mathscr{A}_i\to\mathscr{A}_j)\}$是$\mathscr{O}_{S_0}$拟凝聚代数层范畴上的正向系统,它的极限记作$\{\mathscr{A},(\varphi_i:\mathscr{A}_i\to\mathscr{A})\}$.再记$S_i=\mathrm{Spec}\mathscr{A}_i$和$S=\mathrm{Spec}\mathscr{A}$.把$\varphi_{ji}$和$\varphi_i$对应的概形之间的态射记作$u_{ij}:S_j\to S_i$和$u_i:S\to S_i$.再做如下约定:
\begin{itemize}
	\item 任取指标$i_0$,设$X_{i_0}$和$Y_{i_0}$是两个$S_{i_0}$概形.对$i_0\le i\le j$,取$X_i=X_{i_0}\times_{S_{i_0}}S_i$,$Y_i=Y_{i_0}\times_{S_{i_0}}S_i$和$v_{ij}=1_{X_{i_0}}\times u_{ij}$,$w_{ij}=1_{Y_{i_0}}\times u_{ij}$.那么如果记$I'=\{i\in I\mid i_0\le i\}$,这仍然是一个有向集,并且我们有两组概形的逆向系统$\{(X_i)_{i\in I'},(v_{ij}:X_j\to X_i)_{i\le j}\}$和$\{(Y_i)_{i\in I'},(w_{ij}:Y_j\to Y_i)_{i\le j}\}$.并且我们证明过这两组逆向系统的极限分别是$\{X=X_{i_0}\times_{S_{i_0}}S,(v_i=1_{X_{i_0}}\times u_i:X\to X_i)\}$和$\{Y=Y_{i_0}\times_{S_{i_0}}S,(w_i=1_{Y_{i_0}}\times u_i:Y\to Y_i)\}$.
	\item 对任意$i_0\le i\le j$,如果$f_i:X_i\to Y_i$是$S_i$态射,那么$f_j=f_i\times1_{S_j}:X_j\to X_j$是$S_j$态射.于是我们构造了如下集合映射:
	\begin{align*}
		e_{ji}:\mathrm{Hom}_{S_i}(X_i,Y_i)&\to\mathrm{Hom}_{S_j}(X_j,Y_j)\\
		f_i&\mapsto f_j
	\end{align*}
	
	于是$\{(\mathrm{Hom}_{S_i}(X_i,Y_i))_{i\in I'},(e_{ji})_{i\le j}\}$构成了集合范畴上的正向系统.下面对$S_i$态射$f_i:X_i\to Y_i$取$f=f_i\times1_S:X\to Y$,于是有如下集合映射:
	\begin{align*}
		e_i:\mathrm{Hom}_{S_i}(X_i,Y_i)&\to\mathrm{Hom}_S(X,Y)\\
		f_i&\mapsto f
	\end{align*}

    于是$\{\mathrm{Hom}_S(X,Y),(e_i)_{i\in I'}\}$构成了上述正向系统的余锥,于是泛性质诱导了如下集合映射:
    $$e:\varinjlim\mathrm{Hom}_{S_i}(X_i,Y_i)\to\mathrm{Hom}_S(X,Y)$$
\end{itemize}
\begin{enumerate}
	\item\label{逆向极限和有限表示态射1} 定理.粗略的讲,如果$S_0$是qcqs概形,对每个指标$i$,记$\mathscr{C}_i$表示有限表示$S_i$概形范畴,记$\mathscr{C}$表示有限表示$S$概形,对$i\le j$就有函子$\alpha_{ji}:\mathscr{C}_i\to\mathscr{C}_j$为把$X_i$映为$X_j=X_i\times_{S_i}S_j$,把$f_i:X_i\to Y_i$映射为$f_j=f_i\times1_{S_j}:X_j\to Y_j$.于是$\{(\mathscr{C}_i),(\alpha_{ji})\}$构成一个正向系统.本条定理就是说范畴$\mathscr{C}$几乎就是正向极限$\varinjlim\mathscr{C}_i$.
	\begin{enumerate}[(1)]
		\item 如果$X_{i_0}$是拟紧概形,$Y_{i_0}$是局部有限型$S_{i_0}$概形,那么$e$是单射.
		\item 如果$X_{i_0}$是qcqs概形,$Y_{i_0}$是局部有限表示$S_{i_0}$概形,那么$e$是双射.
		\item 如果$S_0$是qcqs概形,对任意有限表示$S$概形$X$,都存在一个指标$i$和一个有限表示$S_i$概形$X_i$,使得有$S$同构$X\cong X_i\times_{S_i}S$.
	\end{enumerate}
    \begin{proof}
    	
    	我们先来处理仿射情况(另外我们实际上证明了只要$S_{i_0}$和$Y_{i_0}$是仿射的则结论成立),为此设$S_0=\mathrm{Spec}A_0$,$X_{i_0}=\mathrm{Spec}B_{i_0}$,$Y_{i_0}=\mathrm{Spec}C_{i_0}$,于是有$S_i=\mathrm{Spec}A_i$和$S=\mathrm{Spec}A$也都是仿射的,并且有$A=\varinjlim A_i$.那么(1)和(2)转化为交换代数:设$A_0$是环,设$\{(A_i)_{i\in I},(\varphi_{ji}:A_i\to A_j)_{i\le j}\}$是$A_0$代数范畴上的正向系统,设$A=\varinjlim A_i$,再取$A_{i_0}$代数$B_{i_0}$和$C_{i_0}$,如果$C_{i_0}$是有限型$A_{i_0}$代数,那么如下典范同态是单射;如果$C_{i_0}$是有限表示$A_{i_0}$代数,那么如下典范同态是双射.
    	$$\varinjlim\mathrm{Hom}_{\textbf{Alg}(A_i)}(C_{i_0}\otimes_{A_{i_0}}A_i,B_{i_0}\otimes_{A_{i_0}}A_i)\to\mathrm{Hom}_{\textbf{Alg}(A)}(C_{i_0}\otimes_{A_{i_0}}A,B_{i_0}\otimes_{A_{i_0}}A)$$
    	
    	按照张量积的泛性质,我们有如下典范双射:
    	$$\mathrm{Hom}_{\textbf{Alg}(A_i)}(C_{i_0}\otimes_{A_{i_0}}A_i,B_{i_0}\otimes_{A_{i_0}}A_i)\cong\mathrm{Hom}_{\textbf{Alg}(A_{i_0})}(C_{i_0},B_{i_0}\otimes_{A_{i_0}}A_i)$$
    	$$\mathrm{Hom}_{\textbf{Alg}(A)}(C_{i_0}\otimes_{A_{i_0}}A,B_{i_0}\otimes_{A_{i_0}}A)\cong\mathrm{Hom}_{\textbf{Alg}(A_{i_0})}(C_{i_0},B_{i_0}\otimes_{A_{i_0}}A)$$
    	
    	于是问题归结为如下一般的代数事实:设$E$是环,设$G$是$E$代数,设$\{F_i\}$是$E$代数范畴上的一个正向系统,我们有如下典范同态,如果$G$是有限型$E$代数,那么它是单射;如果$G$是有限表示$E$代数,那么它是双射.
    	$$\varinjlim\mathrm{Hom}_{\textbf{Alg}(E)}(G,F_i)\to\mathrm{Hom}_{\textbf{Alg}(E)}(G,\varinjlim F_i)$$
    	
    	我们来证明下这个事实:首先这个典范同态是把$E$代数同态$\theta_i:G\to F_i$映为按照$\varinjlim F_i$的泛性质诱导的$E$代数同态$\varinjlim\theta_i:G\to\varinjlim F_i$.设$G$是有限型$E$代数,可记一组生成元集是$\{t_s,1\le s\le n\}$.假设有两组$\{\theta_i'\},\{\theta_i''\}\in\varinjlim\mathrm{Hom}_{\textbf{Alg}(E)}(G,F_i)$,使得它们诱导了相同的同态$\varinjlim\theta_i'=\varinjlim\theta_i''$.那么我们可以选取一个足够大的指标$i$,使得$\varphi_i(\theta'_i(t_s))=\varphi_i(\theta''_i(t_s))$对任意$1\le s\le n$成立.于是按照正向极限的性质,可以取$i\le j$使得$\varphi_{ji}(\theta'_i(t_s))=\varphi_{ji}(\theta''_i(t_s))$对任意$1\le s\le n$成立.于是$\theta'_j(t_s)=\theta''_j(t_s)$对任意$1\le s\le n$成立,于是$\theta_j'=\theta_j''$.
    	
    	\qquad
    	
    	再设$G$是有限表示$E$代数,也即有$G=E[T_1,\cdots,T_n]/J$,其中$J$是$E[T_1,\cdots,T_n]$的有限生成理想,于是可记一组生成元集为$\{P_s,1\le s\le m\}$.再记$T_s$在$G$中的像是$t_s$.下面任取$E$代数同态$\theta:G\to F=\varinjlim F_i$.记$b_s=\theta(t_s)$.我们有$P_s(b_1,\cdots,b_n)=\theta(P_s(t_1,\cdots,t_n))=0,\forall1\le s\le m$.于是按照正向极限中元素的描述,可以选取一个足够大的指标$i$,以及$x_1,\cdots,x_n\in F_i$,使得$b_s=\varphi_i(x_s),\forall1\le s\le n$.我们有$\varphi_i(P_s(x_1,\cdots,x_n))=P_s(b_1,\cdots,b_n)=0,\forall1\le s\le m$,这里$s$只取有限个数,所以可以找到$i\le j$使得$P_s(\varphi_{ji}(x_1),\cdots,\varphi_{ji}(x_n))=\varphi_{ji}(P_s(x_1,\cdots,x_n))=0$对任意$1\le s\le m$成立.于是我们可以构造$E$代数同态$\theta_j:G\to F_j$为$t_s\mapsto\varphi_{ji}(x_s),\forall1\le s\le n$.它满足复合上$F_j\to F$就是$\theta$,于是$\theta_j$在$\varinjlim\mathrm{Hom}_{\textbf{Alg}(E)}(G,F_i)$中的像,满足打到$\mathrm{Hom}_{\textbf{Alg}(E)}(G,\varinjlim F_i)$中就是$\theta$,这证明了满射.
    	
    	\qquad
    	
    	综上我们解决了仿射情况下的(1)和(2).对于一般情况,我们先解释下不妨设$X_{i_0}=S_{i_0}$.记$S_{i_0}$概形$Z_{i_0}=X_{i_0}\times_{S_{i_0}}Y_{i_0}$,再对$i\ge i_0$取$Z_i=Z_{i_0}\times_{S_{i_0}}S_i=X_i$,那么这也构成概形的逆向系统,它的极限就是$Z=Z_{i_0}\times_{S_{i_0}}S=X\times_SY$.于是按照纤维积的泛性质,我们有如下交换图表:
    	$$\xymatrix{\mathrm{Hom}_{S_i}(X_i,Y_i)\ar[r]^{e_{ji}}\ar@{=}[d]&\mathrm{Hom}_{S_j}(X_j,Y_j)\ar[r]^{e_j}\ar@{=}[d]&\mathrm{Hom}_S(X,Y)\ar@{=}[d]\\\mathrm{Hom}_{X_i}(X_i,Z_i)\ar[r]_{e_{ji}}&\mathrm{Hom}_{X_j}(X_j,Z_j)\ar[r]_{e_j}&\mathrm{Hom}_X(X,Z)}$$
    	
    	于是问题归结为设$X_{i_0}=S_{i_0}$.下面证明一般情况下的(1):设$X_{i_0}$是拟紧概形,设$Y_{i_0}$是有限型$X_{i_0}$概形.取两个指标$i,j$,设有$X_i$态射$f_i':X_i\to Y_i$和$X_j$态射$f_j'':X_j\to Y_j$,使得它们分别对应的$X$态射$f',f'':X\to Y$是相同的.我们可以选取一个指标$\ge i,j$,那么$f_i'$和$f_j''$分别基变换到这个指标上的态射在相同的等价类中,于是我们不妨设起初的$f_{i_0}',f_{i_0}''$都是$X_{i_0}\to Y_{i_0}$的态射,满足它们诱导的态射$f',f'':X\to Y$是相同的.我们的目标就是证明存在$i\ge i_0$使得$f'_i=f_i''$.
    	
    	\qquad
    	
    	按照条件$X_{i_0}$是拟紧的,拟紧空间的连续像是拟紧的,拟紧集的有限并是拟紧的,于是$f_{i_0}'(X_{i_0})\cup f''_{i_0}(X_{i_0})\subseteq Y_{i_0}$也是拟紧的.另外按照$Y_{i_0}$在$X_{i_0}$上是有限型概形,于是存在$f_{i_0}'(X_{i_0})\cup f''_{i_0}(X_{i_0})$在$Y_{i_0}$中的有限仿射开覆盖$\{V_{ti_0}\}$,满足它们都是$X_{i_0}$上的有限型概形.下面设$U_{ti_0}'={f'}_{i_0}^{-1}(V_{ti_0})$和$U_{ti_0}''={f''}_{i_0}^{-1}(V_{ti_0})$,再设$U_{ti_0}=U_{ti_0}'\cap U_{ti_0}''$和$U_{i_0}=\cup_tU_{ti_0}$.那么按照$f'=f''$,就有$v_{i_0}^{-1}(U_{ti_0}')={f'}^{-1}(w_{i_0}^{-1}(V_{ti_0}))={f''}^{-1}(w_{i_0}^{-1}(V_{ti_0}))=v_{i_0}^{-1}(U_{ti_0}'')$.进而有$v_{i_0}^{-1}(U_{i_0})={f'}^{-1}(Y)=X$.如果记$U_{ti}=v_{i_0i}^{-1}(U_{ti_0})$和$U_i=v_{i_0i}^{-1}(U_{i_0})$.那么它们都是$X_i$的开子集,从而都是$X_i$的ind-可构造集.于是从$\cup_iv_i^{-1}(U_i)=v_{i_0}^{-1}(U_{i_0})=X$,按照\ref{逆向极限和可构造集2}的(2)就可以找到足够大的指标$i$,使得$v_i^{-1}(U_i)=X_i$,换句话讲$\{U_{ti}\}$是$X_i$的有限开覆盖.于是不妨用$i$替代$i_0$,我们可以设$\{U_{ti_0}\}$构成了$X_{i_0}$的有限开覆盖.最后任取$x\in X_{i_0}$,可取它的仿射开邻域$W(x)$落在某个$U_{ti_0}$中.进而$f_{i_0}'$和$f_{i_0}''$都可以限制为态射$W(x)\to V_{ti_0}$,所以按照仿射情况,可以选取足够大的指标$i$使得$f_i'$和$f_i''$限制在$W(x)$上是一致的.最后$\{W(x),x\in X_{i_0}\}$构成了$X_{i_0}$的仿射开覆盖,按照$X_{i_0}$是拟紧的,可以选取有限子覆盖,这就保证把$i$替换为一个足够大的指标时在整个$X_i$上都有$f_i'=f_i''$,完成(1)的证明.
    	
    	\qquad
    	
    	下面证明一般情况的(2).我们已经解释过不妨设$X_{i_0}=S_{i_0}$,进而有$X=S$.下面设$X_{i_0}$是qcqs概形,设$Y_{i_0}$是有限表示$S_{i_0}$概形.我们只需证明典范的$e$是满射.任取$X$态射$f:X\to Y$.由于$X$是拟紧的,于是$f(X)$也是拟紧的,于是存在$Y$的拟紧开子集$Y'$包含了整个$f(X)$.于是按照\ref{概形的逆向极限9}的(2),存在指标$i\ge i_0$和$Y_i$的拟紧开集$Y_i'$,使得$Y'=w_i^{-1}(Y_i')$.我们不妨用$i$替换$i_0$,再用$Y_{i_0}'$替换$Y_{i_0}$,于是不妨设$Y_{i_0}$本身就是拟紧的,进而$Y_i$和$Y$都是拟紧的.于是可取$Y_{i_0}$的有限仿射开覆盖$\{V_{ti_0}\}$,那么$\{V_{ti}=w_{i_0i}^{-1}(V_{ti_0})\}$是$Y_i$的有限仿射开覆盖,$\{V_t=w_{i_0}^{-1}(V_{ti_0})\}$是$Y$的有限仿射开覆盖.于是$\{f^{-1}(V_i)\}$是$X$的开覆盖,于是对每个点$x\in X$,选取它的仿射开邻域$W(x)$包含在某个$f^{-1}(V_i)$中,按照$X$是拟紧的$\{W(x)\mid x\in X\}$就存在有限子覆盖$\{U_s\}$.我们可以扩充开覆盖$\{V_t\}$,使得$\{U_s\}$和$\{V_t\}$具有相同的指标集,并且满足$f(U_t)\subseteq V_t$.于是按照\ref{概形的逆向极限9}的(2),我们可以适当把$i_0$替换为更大的指标,使得存在$X_{i_0}$的拟紧开子集$U_{ti_0}$,满足$U_t=w_{i_0i}^{-1}(U_{ti_0})$.接下来似于上一段的操作,按照\ref{逆向极限和可构造集2}的(2)我们可以不妨设这些$\{U_{ti_0}\}$构成了$X_{i_0}$的开覆盖.
    	
    	\qquad
    	
    	现在问题归结为找到一个指标$i$,以及态射$f_{ti}:U_{ti}=v_{i_0i}^{-1}(U_{ti_0})\to V_{ti}=w_{i_0i}^{-1}(V_{ti_0})$,使得态射$f_t=e_i(f_{ti})$恰好是$f$在$U_t$上的限制.因为一旦找到这样的$f_{ti}$,按照$X_i$是拟分离的,有$U_{si}\cap U_{ti}$是拟紧的,于是我们证明的(1)就保证了把$i$替换为更大的指标时对任意$s,t$都有$f_{ti}$和$f_{si}$限制在$U_{ti}\cap U_{si}$上是相同的,于是这些$\{f_{ti}\}$就可以粘合为一个$X_i$态射$f_i:X_i\to Y_i$,并且它满足$e_i(f_i)=f$.于是我们归结为在(2)中设$Y_{i_0}$是仿射的.我们还可以设$S_{i_0}$是仿射的【】.我们在前面仿射情况的证明中解释了只要$S_{i_0}$和$Y_{i_0}$是仿射的则结论成立.至此(2)得证.
    	
    	\qquad
    	
    	接下来证明(3).首先仿射情况已经在环的正向极限中证明过了.对于一般情况,按照$S_0$是拟紧的,可以取有限仿射开覆盖$\{U_s\}$,按照$S\to S_0$是仿射态射,所以$U_s$的原像$W_s$也是仿射的,于是$\{W_s\}$是$S$的有限仿射开覆盖.取$W_s$在$X\to S$下原像的仿射开覆盖,让$s$跑遍所有指标,这些仿射开子集构成了$X$的仿射开覆盖,由于$X\to S$是拟紧的,$S$是拟紧概形,于是$X$也是拟紧的,于是可取有限子覆盖.换句话讲我们找到了$X$的有限仿射开覆盖$\{X_r\}$,使得每个$X_r$在$X\to S$下打到某个$W_s$中,我们把这个和$r$相关的指标$s$记作$s(r)$.我们暂时把仿射概形$X$对应的环记作$A(X)$.那么按照$X\to S$是有限表示态射,得到$A(X_r)$是有限表示$A(W_{s(r)})$代数.记$U_{s(r)}$在$S_i\to S_0$下的原像是$W_{s(r),i}$.按照已经得证的仿射情况,以及这里$r$只取有限个,我们可以选取一个足够大的指标$i$,使得对每个$r$都存在有限表示$W_{s(r),i}$仿射概形$Z_{ri}$,满足有$S$同构$g_r:Z_{ri}\times_{S_i}S\cong X_r$.我们接下来就要证明在$i$足够大时这些$Z_{ri}$可以粘合为一个$S_i$概形$X_i$,那么按照构造这是有限表示$S_i$概形,并且满足有$S$同构$X_i\times_{S_i}S\cong X$.
    	
    	\qquad
    	
    	设$X_{r_1}\cap X_{r_2}$在$g_{r_1}$下的原像是$Z_{r_1r_2}$,按照$X$是拟分离的得到$Z_{r_1r_2}\cong X_{r_1}\cap X_{r_2}$是拟紧的,把这个同构记作$g'_{r_1r_2}$.对$i\le j$记$Z_{rj}=v_{ij}^{-1}(Z_{ri})$.我们可以把$i$替换为更大的指标,使得对任意$r_1,r_2$,都存在$Z_{r_1i}$的拟紧开子集$Z_{r_1r_2i}$,使得$Z_{r_1r_2}$就是$Z_{r_1r_2i}$在$Z_{r_1i}\times_{S_i}S\to Z_{r_1i}$下的原像,换句话讲$Z_{r_1,r_2}=Z_{r_1r_2i}\times_{S_i}S$.另外按照$S_i$是拟分离的,$W_{s(r),i}\to S_i$是拟紧的,于是$Z_{r_1r_2i}\to S_i$是有限表示态射.记$h_{r_2r_1}={g'}_{r_2r_1}^{-1}\circ g'_{r_1r_2}:Z_{r_1r_2}\to Z_{r_2r_1}$.于是下面推论告诉我们可以让$i$变得更大,使得对任意$r_1,r_2$,都有同构$h_{r_2r_1i}:Z_{r_1r_2i}\to Z_{r_2r_1i}$使得$h_{r_2r_1}=h_{r_2r_1i}\times1_S$.接下来固定指标$r_1,r_2,r_3$,把$h_{r_2r_1}$限制在$Z_{r_1r_2}\cap Z_{r_1r_3}$上记作$h'_{r_2r_1}$,那么它是$Z_{r_1r_2}\cap Z_{r_1r_3}$到$Z_{r_2r_1}\cap Z_{r_2r_3}$的同构.于是有$h'_{r_3r_2}\circ h'_{r_2r_1}=h'_{r_3r_1}$.于是再用一次下面的推论,把$i$替换为足够大的指标可以保证对任意$r_1,r_2,r_3$有$h'_{pq}$满足余圈条件$h'_{r_3r_2i}\circ h'_{r_2r_1i}=h'_{r_3r_1i}$.于是这些$Z_{r_1r_2}$就可以粘合为一个$S_i$概形$X_i$,并且按照构造满足有$S$同构$X\cong X_i\times_{S_i}S$.按照构造还有$X_i\to S_i$是局部有限表示态射,也是拟紧的,最后按照$Z_{r_1i}\cap Z_{r_2i}$是拟紧的得到$X_i$是拟分离概形,再按照$S_i$是拟分离概形就得到$X_i\to S_i$是拟分离态射,综上得到$X_i\to S_i$是有限表示态射.完成证明.
    \end{proof}
    \item\label{逆向极限和有限表示态射2} 推论.
    \begin{enumerate}[(1)]
    	\item 设$S_0$是拟紧的,设$X_{i_0}\to S_{i_0}$是有限表示态射,设$Y_{i_0}\to S_{i_0}$是有限型的拟分离态射(例如有限表示态射满足这个要求).设$f_{i_0}:X_{i_0}\to Y_{i_0}$是$S_{i_0}$态射,对$i\ge i_0$记$f_i=f_{i_0}\times1_{S_i}$和$f=f_{i_0}\times1_S$.那么$f$是同构当且仅当存在$i\ge i_0$使得$f_i:X_i\to Y_i$是同构(从而对任意$i\le j$都有$f_j:X_j\to Y_j$是同构).
    	\begin{proof}
    		
    		充分性就是因为同构在基变换下不变.下面证明必要性.问题关于$S_0$是局部的,所以不妨设$S_0$是仿射的,特别的它是拟分离的.按照条件有$S$态射$g:Y\to X$使得$g\circ f=1_X$和$f\circ g=1_Y$.因为$Y_{i_0}\to S_{i_0}$是qcqs态射,并且$S_{i_0}$作为仿射概形本身是qcqs概形,于是$Y_{i_0}$是qcqs概形.再按照$X_{i_0}$是有限表示$S_{i_0}$概形,于是上一条的(2)告诉我们存在足够大的指标$i$和$S_i$态射$g_i:Y_i\to X_i$,满足$g=g_i\times1_S$.又因为$X_{i_0}$和$Y_{i_0}$都是有限型$S_{i_0}$概形,于是按照$e$是单射,从$g\circ f=1_X$和$f\circ g=1_Y$可以把$i$替换为更大的指标使得$g_i\circ f_i=1_{X_i}$和$f_i\circ g_i=1_{Y_i}$.于是$f_i:X_i\to Y_i$是同构.
    	\end{proof}
        \item 设$S_0$是qcqs概形,设$X_{i_0}$和$Y_{i_0}$都是有限表示$S_{i_0}$概形,记$X_i=X_{i_0}\times_{S_{i_0}}X_i$和$Y_i=Y_{i_0}\times_{S_{i_0}}S_i$,再记$X=X_{i_0}\times_{S_{i_0}}S$和$Y=Y_{i_0}\times_{S_{i_0}}S$.那么$X$和$Y$是$S$同构的当且仅当存在指标$i\ge i_0$使得$X_i$和$Y_i$是$S_i$同构的(从而对任意$i\le j$都有$X_j$和$Y_j$是$S_j$同构的).更具体的,对任意$S$同构$f:X\cong Y$,都存在指标$i\ge i_0$和$S_i$同构$X_i\cong Y_i$,使得$f=f_i\times1_S$.
    \end{enumerate}
\end{enumerate}
\subsection{消除诺特条件}
\begin{enumerate}
	\item\label{消除诺特条件1} 设$A$是环,设$X$是$A$概形.
	\begin{enumerate}[(1)]
		\item 如下条件互相等价:
		\begin{enumerate}[(a)]
			\item $X$是有限表示$A$概形.
			\item 存在一个诺特环$A_0$,和一个有限型$A_0$概形$X_0$,和一个环同态$A_0\to A$,使得有$A$概形同构$X_0\times_{A_0}A\cong X$.
			\item 存在$A$的子环$A_0$,使得它作为$\mathbb{Z}$代数是有限型的,存在一个有限型$A_0$概形$X_0$,使得有$A$概形同构$X_0\times_{A_0}A\cong X$.
		\end{enumerate}
		\begin{proof}
			
			因为环$A$总可以写成它的有限型$\mathbb{Z}$代数子环的正向极限,于是\ref{逆向极限和有限表示态射1}就说明(a)推(c).而(c)推(b)是因为有限型$\mathbb{Z}$代数一定是诺特环.最后(b)推(a)是因为按照$A_0$是诺特环,它上面的有限型概形一定是有限表示概形,再结合有限表示态射在基变换下不变得证.
		\end{proof}
		\item 设$X$是有限表示$A$概形,设$\mathscr{F}$是有限表示拟凝聚$\mathscr{O}_X$模层,那么我们可以找到$A$的诺特子环$A_0$,以及一个$A_0$概形$X_0$使得有$A$概形同构$X\cong A\times_{A_0}X_0$,并且可以找到凝聚$\mathscr{O}_{X_0}$模层$\mathscr{F}_0$使得有$\mathscr{O}_X$模层同构$\mathscr{F}\cong\mathscr{F}_0\otimes_{A_0}A$.另外这里$\mathrm{Supp}(\mathscr{F})$是$X$的可构造闭集,并且其上可以赋予闭子概型结构,使得典范闭嵌入$Z\to X$是有限表示态射.
		\begin{proof}
			
			这里满足等式的拟凝聚层$\mathscr{F}_0$的存在性是因为\ref{逆向系统和有限表示模层2},这里它是凝聚层是因为$A_0$是诺特环.【后半部分】
		\end{proof}
	    \item 设$X,Y$是两个有限表示$A$概形,设$f:X\to Y$是$A$态射,那么有$A$的诺特子环$A_0$,以及有限型$A_0$概形$X_0$和$Y_0$,使得有$A$概形同构$X\cong X_0\times_{A_0}A$和$Y\cong Y_0\times_{A_0}A$.还存在一个$A_0$态射$f_0:X_0\to Y_0$(此时由于$X_0\to\mathrm{Spec}A$和$Y_0\to\mathrm{Spec}A$都是有限型态射,得到$f_0$必然也是有限型态射),满足$f=f_0\times1_A$.
	    \begin{proof}
	    	
	    	这件事就是\ref{逆向极限和有限表示态射1}的(2).
	    \end{proof}
	\end{enumerate}
	\item\label{消除诺特条件2} 设$Y$是整概形,设$u:X\to Y$是有限型态射和局部有限表示态射,设$\mathscr{F}$是有限表示拟凝聚$\mathscr{O}_X$模层,那么存在$Y$的非空开集$U$,使得$\mathscr{F}\mid_{u^{-1}(U)}$是$U$上平坦(这是指对任意$x\in u^{-1}(U)$,记$y=u(x)$,那么$\mathscr{F}_x$经$\mathscr{O}_{Y,y}\to\mathscr{O}_{X,x}$视为$\mathscr{O}_{Y,y}$模是平坦的).
	\begin{proof}
		
		首先问题明显是关于$Y$局部的,于是我们不妨设$Y=\mathrm{Spec}A$是仿射的.按照$u:X\to Y$是拟紧的,于是此时$X$可以表示为有限个仿射开子集$\{U_i\}$的并.那么$X_i\to\mathrm{Spec}A$同样是有限型和有限表示态射,并且$\mathscr{F}$限制在$X_i$上同样是有限表示拟凝聚模层.于是一旦我们解决了$X,Y$都是仿射的情况,那么对每个$i$就存在$Y$的非空开子集$U_i$使得$\mathscr{F}$限制在$X_i\cap u^{-1}(U_i)$上是平坦的.于是只要取$U=\cap_iU_i$,那么任取$x\in u^{-1}(U)$,就有$y=u(x)$落在每个$U_i$中,记$x$落在$X_t$中,那么按照$\mathscr{F}\mid_{X_t\cap u^{-1}(U_t)}$在$U_t$上平坦,就得到$\mathscr{F}_x$经$\mathscr{O}_{Y,y}\to\mathscr{O}_{X,x}$是平坦$\mathscr{O}_{Y,y}$模.综上我们解释了归结为$X,Y$都是仿射的情况.
		
		\qquad
		
		这归结为如下交换代数事实:设$A$是整环,$B$是有限表示$A$代数,$M$是有限表示$B$模,那么存在$0\not=f\in A$,使得$M_f$是自由$A_f$模.
		
		\qquad
		
		按照\ref{消除诺特条件1},存在$A$的有限型$\mathbb{Z}$代数的子环$A_0$,和一个有限型$A_0$代数$B_0$,和一个有限型$B_0$模$M_0$,使得有$A$代数同构$A\cong B_0\otimes_{A_0}A$和$B$模同构$M\cong M_0\otimes_{B_0}B$.于是按照【EGAIV6.9.2】,存在$0\not=f_0\in A_0$,使得$(M_0)_{f_0}$作为$(A_0)_{f_0}$模是自由的,于是$M_{f_0}=(M_0)_{f_0}\otimes_{A_0}A$就是$A_{f_0}=(A_0)_{f_0}\otimes_{A_0}A$自由模.
	\end{proof}
	\item\label{消除诺特条件3} 推论.设$S$是qcqs概形,设$u:X\to S$是有限表示态射,设$\mathscr{F}$是有限表示拟凝聚$\mathscr{O}_X$模层.那么存在$S$的由局部闭子集构成的有限划分$\{S_i\mid 1\le i\le n\}$,使得每个$S_i$上可以赋予子概型结构,满足嵌入$S_i\to S$是有限表示态射,并且如果记$X_i=X\times_SS_i$,那么$\mathscr{O}_{X_i}$模层$\mathscr{F}_i=\mathscr{F}\otimes_{\mathscr{O}_S}\mathscr{O}_{S_i}$在$S_i$上平坦.
	\begin{proof}
		
		取$S$的有限仿射开覆盖$\{U_j\mid1\le j\le m\}$,再记$T_1=U_1$,$T_j=U_j-\cup_{h<j}(U_j\cap U_h),j\ge2$,于是每个$T_j$都是$U_j$的闭子集,并且$T_j$两两不交.因为$U_j$是拟紧的,$S$是拟分离的,于是开嵌入$U_j\to S$是有限表示态射(开嵌入是有限表示态射当且仅当拟紧),所以只要赋予$U_j$的闭子集$T_j$是有限表示的,就有$T_j\to S$是有限表示的嵌入.另外按照$S$是qcqs概形,这里$T_j$都是可构造集.于是问题归结为设$S=U=\mathrm{Spec}A$是仿射的,设$T\subseteq S$是可构造闭集,归结为证明总可以赋予$T$有限表示的闭子概型结构,并且对任意的有限表示闭子概型结构,如果把$u,\mathscr{F}$替换为它们关于$T\to S$的基变换,那么$T$可以表示为自身的有限个局部闭子集的无交并使命题中的结论成立.
		
		\qquad
		
		我们把$A$写作它的有限型$\mathbb{Z}$代数子环的正向极限,于是$S=\varprojlim S_i$是仿射诺特概形构成的逆向系统的极限.于是按照\ref{逆向极限和可构造集5},就可以找到一个指标$i$和$S_i$的可构造闭子集$T_i$,使得$T=u_i^{-1}(T_i)$,其中$u_i:S\to S_i$是逆向极限里的典范态射.于是问题归结为赋予$T_i$一个有限表示的关于$S_i$的闭子概型结构,寻找$T_i$的有限个局部闭子集的无交并使得结论成立.但是这里$T_i$作为仿射诺特概形$S_i$的闭子概型总是仿射诺特的,进而$T_i\to S_i$总是有限表示的,于是它关于$S_i\to S$的基变换$T=T_i\times_{S_i}S$也是$S$上的有限表示态射.于是我们解释了只要$T$是$S$的可构造闭集,那么它总可以赋予有限表示的闭子概型结构.另外我们这里$S$是仿射的,它的闭子概型总是仿射的,于是我们不妨设$T$在底空间上就是整个$S$(我们当然也可以在上一段结尾就设底空间上有$T=S$,但是仍要证明$T$上总可以赋予有限表示的闭子概型结构).
		
		\qquad
		
		按照\ref{消除诺特条件1},适当把$i$替换为更大的指标,我们可以选取一个有限型态射$u_i:X_i\to S_i$,以及一个凝聚$\mathscr{O}_{X_i}$模层$\mathscr{F}_i$,使得$X\cong X_i\times_{S_i}S$和$\mathscr{F}\cong\mathscr{F}_i\otimes_{\mathscr{O}_{X_i}}\mathscr{O}_X$.于是按照【EGAIV6.9.3】,就可以找到$S_i$的有限个子概型$\{S_{it}\}$,使得它们在底空间上是$S_i$的无交并,并且如果记$X_{it}=X_i\times_{S_i}S_{it}$,那么$\mathscr{F}_{it}=\mathscr{F}_i\otimes_{\mathscr{O}_{S_i}}\mathscr{O}_{S_{it}}$就在$S_{it}$上平坦.于是$S_t=S_{it}\times_{S_i}S$在$t$跑遍有限个指标时就满足结论.
	\end{proof}
\end{enumerate}
\subsection{和逆向极限兼容的态射性质}

设$S_0$是概形,设$\{(\mathscr{A}_i),(\varphi_{ji}:\mathscr{A}_i\to\mathscr{A}_j)\}$是$\mathscr{O}_{S_0}$拟凝聚代数层范畴上的正向系统,它的极限记作$\{\mathscr{A},(\varphi_i:\mathscr{A}_i\to\mathscr{A})\}$.再记$S_i=\mathrm{Spec}\mathscr{A}_i$和$S=\mathrm{Spec}\mathscr{A}$.把$\varphi_{ji}$和$\varphi_i$对应的概形之间的态射记作$u_{ij}:S_j\to S_i$和$u_i:S\to S_i$.我们再做如下约定:
\begin{itemize}
	\item 任取指标$i_0$,设$X_{i_0}$和$Y_{i_0}$是两个$S_{i_0}$概形.对$i_0\le i\le j$,取$X_i=X_{i_0}\times_{S_{i_0}}S_i$,$Y_i=Y_{i_0}\times_{S_{i_0}}S_i$和$v_{ij}=1_{X_{i_0}}\times u_{ij}$,$w_{ij}=1_{Y_{i_0}}\times u_{ij}$.那么如果记$I'=\{i\in I\mid i_0\le i\}$,这仍然是一个有向集,并且我们有两组概形的逆向系统$\{(X_i)_{i\in I'},(v_{ij}:X_j\to X_i)_{i\le j}\}$和$\{(Y_i)_{i\in I'},(w_{ij}:Y_j\to Y_i)_{i\le j}\}$.并且我们证明过这两组逆向系统的极限分别是$\{X=X_{i_0}\times_{S_{i_0}}S,(v_i=1_{X_{i_0}}\times u_i:X\to X_i)\}$和$\{Y=Y_{i_0}\times_{S_{i_0}}S,(w_i=1_{Y_{i_0}}\times u_i:Y\to Y_i)\}$.
	\item 任取$S_{i_0}$态射$f_{i_0}:X_{i_0}\to Y_{i_0}$,对$i\in I'$记$S_i$态射$f_i=f_{i_0}\times1_{S_i}$和$S$态射$f=f_{i_0}\times1_S$.
	\item 本节我们探究是这样一个问题:设$X,Y$是$S$概形,对于什么样的态射性质$\textbf{P}$,有$S$态射$f:X\to Y$满足$\textbf{P}$当且仅当存在指标$i$,使得对任意$i\le j$都有$f_j$满足性质$\textbf{P}$.我们称满足这个条件的态射性质$\textbf{P}$是和逆向系统兼容的.另外明显的如果$\textbf{P}$是基变换不变性质,那么从$f_i$满足$\textbf{P}$推出对任意$i\le j$都有$f_j$满足$\textbf{P}$,也有$f$满足$\textbf{P}$.
\end{itemize}
\begin{enumerate}
	\item\label{和逆向极限兼容的态射性质1} 如果存在指标$i$使得对任意$i\le j$都有$S_j$态射$f_j:X_j\to Y_j$是开映射,那么$f$也是开映射(因为开映射不是基变换下不变性质,所以这件事不是形式上的结论).
	\begin{proof}
		
		因为拟紧开集构成概形的拓扑基,归结为证明如果$U\subseteq X$是拟紧开集,那么$f(U)$是$Y$的开集.但是按照\ref{概形的逆向极限9},我们可以把指标$i$变得更大,使得存在$X_i$的拟紧开集$U_i$使得$U=v_i^{-1}(U_i)$.于是按照【refI.3.4.8】有$f(U)=f(v_i^{-1}(U_i))=w_i^{-1}(f_i(U_i))$是开集.
	\end{proof}
    \item\label{和逆向极限兼容的态射性质2} 推论.设$f:X\to Y$是态射(这里不要求$X,Y$都是有限表示$S$概形,这一条推论不在我们开篇的框架下),如果对任意正整数$n$,记$Y_n=Y\times_{\mathbb{Z}}\mathbb{Z}[T_1,\cdots,T_n]=\mathbb{A}_Y^n$和$X_n=X\times_YY_n$,都有投影态射$f_n=f\times1_{Y_n}:X_n\to Y_n$是开映射,那么$f$是泛开的态射(换句话讲,虽然泛开的态射按照定义是对任意的基变换都是开映射,但是我们实际上只需验证对全部仿射空间$\mathbb{A}_{\mathbb{Z}}^n$做基变换是开映射即可).
    \begin{proof}
    	
    	泛开的态射是一个终端局部性质,所以归结为设$Y=\mathrm{Spec}A$是仿射的.我们要证明$f$关于任意态射$Y'\to Y$的基变换$f_{(Y')}$是开映射,但是由按照开映射是终端局部性质,我们又可以不妨设$Y'=\mathrm{Spec}A'$是仿射的.接下来先设$A'$是有限型$A$代数,于是$A'$可以表示为$A[T_1,\cdots,T_n]$的商代数.那么$Y'$就设$Y_n$的闭子概型,而$f_{(Y')}$就是$f_n$在$f_n^{-1}(Y')$上的限制.接下来对$X_n$的任意开子集$V$,都有$f_n(V\cap f_n^{-1}(Y'))=f_n(V)\cap Y'$,按照条件有$f_n(V)$是$Y_n$的开集,于是这个等式就说明限制映射$f_{(Y')}$是开映射.最后再设$A'$是任意的$A$代数,那么$A'$是它所有有限型$A$子代数$\{A_i'\}$的正向极限,于是$f_{(A_i')}$都是开映射,于是\ref{和逆向极限兼容的态射性质1}就说明$f_{(A')}$是开映射.
    \end{proof}
    \item\label{和逆向极限兼容的态射性质3} 设有指标$i_0$使得$S_{i_0}$是拟紧概形,并且结构态射$X_{i_0}\to S_{i_0}$拟紧态射,结构态射$Y_{i_0}\to S_{i_0}$是qcqs态射,对任意$i_0\le i\le j$有$u_{ij}:S_j\to S_i$是平坦的,并且$\overline{f_{i_0}(X_{i_0})}$是$Y_{i_0}$的可构造集.那么$f$是支配态射当且仅当存在指标$i\ge i_0$使得对任意$i\le j$都有$f_j$是支配态射.
    \begin{proof}
    	
    	条件说明$Y_{i_0}$是拟紧的,并且$f_{i_0}$也是拟紧的.我们在\ref{pro-可构造集和ind-可构造集2}解释过拟紧态射把pro-可构造集映射为pro-可构造集,于是这里$f_{i_0}(X_{i_0})=Z_{i_0}$是$Y_{i_0}$的pro-可构造集.对指标$i\ge i_0$,记$Z_i=w_{i_0i}^{-1}(Z_{i_0})$和$Z=w_{i_0}^{-1}(Z_{i_0})$.那么按照纤维积图表满足的【refI.3.4.8】,就有$Z_i=w_{i_0i}^{-1}(f_{i_0}(X_{i_0}))=f_i(v_{i_0i}^{-1}(X_{i_0}))=f_i(X_i)$.同理有$Z=f(X)$.\ref{pro-可构造集和ind-可构造集2}还说明$Z_i$是$Y_i$的pro-可构造集,于是我们的命题归结为\ref{逆向极限和可构造集7}.
    \end{proof}
    \item\label{和逆向极限兼容的态射性质4} 在我们的框架下取$i_0=0$,取$Y_0=S_0$.设$S_0$是拟紧概形,设$\mathscr{L}_0$是$\mathscr{O}_{X_0}$可逆模层,那么$\mathscr{L}_i=u_{0i}^*(\mathscr{L}_0)$就是$\mathscr{O}_{X_i}$可逆模层,我们断言$\mathscr{L}=u_0^*(\mathscr{L}_0)$是关于$f$的丰沛层或者极丰沛层当且仅当存在指标$i$使得$\mathscr{L}_i$是关于$f_i$的丰沛层或者极丰沛层.
    \begin{proof}
    	
    	【EGAIV8.10.5.2】
    \end{proof}
    \item\label{和逆向极限兼容的态射性质5} 设有指标$i_0$使得$Y_{i_0}$是拟紧的,并且$f_{i_0}$是有限型和拟分离的态射.那么$f$是分离态射当且仅当存在$i\ge i_0$使得$f_i$是分离的(进而对任意$i\le j$都有$f_j$是分离的).
    \begin{proof}
    	
    	按照$Y_{i_0}$是拟紧的,取它的有限仿射开覆盖,一旦我们对$Y_{i_0}$仿射情况证明了这件事,那么其中每个仿射开子集可以取到满足命题的指标$i$,按照这只有有限个仿射开子集以及$I$是有向集,再按照分离是终端局部性质,于是问题归结为设$Y_{i_0}$是仿射的.特别的它是拟分离的,于是$X_{i_0}$也是qcqs概形.
    	
    	\qquad
    	
    	下面对$i\ge i_0$设$X_i'=X_i\times_{Y_i}X_i$和$X'=X\times_YX$.那么我们有$X_i'=X_{i_0}'\times_{Y_{i_0}}Y_i$和$X'=X_{i_0}'\times_{Y_{i_0}}Y$(这件事就是对$X_{i_0}\times_{S_{i_0}}X_{i_0}$的整个纤维积图表再做称积$\times_{S_0}S$得到的仍然是纤维积图表).按照qcqs态射在基变换下不变,有典范投影态射$X_{i_0}'\to X_{i_0}$是qcqs态射.我们用$\Delta_i$和$\Delta$分别表示$X_i\to Y_i$和$X\to Y$的对角态射.那么我们知道对$i\le j$,有$\Delta_j$就是$\Delta_i$关于$v'_{ij}:X_j'\to X_i'$的基变换;类似的$\Delta$就是$\Delta_i$关于$v_i':X'\to X_i'$的基变换.于是按照【refI.3.4.8】得到$\Delta_j(X_j)=\Delta_j(v_{ij}^{-1}(X_i))={v_{ij}'}^{-1}(\Delta_i(X_i))$,同理有$\Delta(X)={v_i'}^{-1}(\Delta_i(X_i))$.另外按照$f_i$是拟分离的和有限型态射,得到$\Delta_i$是拟紧的局部有限表示态射,于是【Chevalley定理】说明$\Delta_i(X_i)$总是可构造集.我们要证明的就是$\Delta(X)$是可构造闭集当且仅当存在$i$使得对任意$i\le j$有$\Delta_j(X_j)$是可构造闭集,而这就是\ref{逆向极限和可构造集6}.
    \end{proof}
    \item 设$S_0$是拟紧概形,设$X_{i_0}$和$Y_{i_0}$是有限表示$S_{i_0}$概形,设$f_{i_0}:X_{i_0}\to Y_{i_0}$是$S_{i_0}$态射.设$\textbf{P}$是如下态射性质中的任意一个:
    \begin{enumerate}[(1)]
    	\item 同构.
    	\item 单态射.
    	\item 嵌入.
    	\item 开嵌入.
    	\item 闭嵌入.
    	\item 分离态射.
    	\item 满射.
    	\item 泛单的态射.
    	\item 仿射态射.
    	\item 拟仿射态射.
    	\item 有限态射.
    	\item 拟有限态射.
    	\item 紧合态射.
    \end{enumerate}

    那么性质$\textbf{P}$和逆向系统是兼容的,具体地讲,态射$f$满足性质$P$当且仅当存在指标$i\ge i_0$使得$f_i$具有性质$\textbf{P}$(进而对任意$i\le j$都有$f_j$具有性质$\textbf{P}$).另外如果$S_0$是qcqs概形,那么$\textbf{P}$还可以补充如下两个性质:
    \begin{enumerate}[(1),resume]
    	\item 射影态射.
    	\item 拟射影态射.
    \end{enumerate}
    \begin{proof}
    	
    	证明(1),(2),(6):$\textbf{P}$是同构已经在\ref{逆向极限和有限表示态射2}中得证;$\textbf{P}$是分离态射已经在\ref{和逆向极限兼容的态射性质4}中得证.另外我们知道分离态射等价于对角态射是闭嵌入,并且如果记$f_i$的对角态射是$\Delta_i$,那么对$i\le j$有$\Delta_j$是$\Delta_i$关于$v'_{ij}:X_j'\to X_i'$的基变换(可以见\ref{和逆向极限兼容的态射性质4}的证明),于是$\textbf{P}$是分离态射的命题也可以从$\textbf{P}$是闭嵌入的命题得出.类似的,由于单态射等价于对角态射是同构,所以$\textbf{P}$是同构的命题可以推出$\textbf{P}$是单态射的命题.
    	
    	\qquad
    	
    	(7),(8),(12)都是和纤维积有关的性质,我们会放在【后文】给出更一般的结论,不过下面我们也会证明(7)和(8).另外因为这里出现的性质都是基变换下不变的性质,所以所有命题的充分性都是成立的,于是我们只需这些命题的证明必要性.另外这里用$i_0$替换指标$0$没有任何影响.又因为前13个命题都是关于$S_0$上的局部性质,并且$S_0$是拟紧的说明它存在有限仿射开覆盖,再结合指标集是有向集,于是我们归结为设$S_0=\mathrm{Spec}A_0$是仿射的,进而这里$S_i=\mathrm{Spec}A_i$和$S=\mathrm{Spec}A$都是仿射的.类似的这前13个命题也都是关于$Y_0$是局部的,所以还可以不妨设$Y_0$是仿射的,那么此时$Y_i$和$Y$都是仿射的,于是我们归结为设$Y_0=S_0$的情况.
    	
    	\qquad
    	
    	证明(3),(4),(5):设$f$是嵌入,于是$X$可视为$S$的有限表示子概型.我们解释了$S_0$可以设为仿射的,于是特别的它是qcqs概形,于是\ref{逆向系统和有限表示子概型3}就说明存在指标$i$和$S_i$的有限表示子概型$X_i'$,使得$X=X_i'\times_{S_i}S$.进而对任意$i\le j$取$X_j'=X_i'\times_{S_i}S_j$就是$S_j$的有限表示子概型.并且我们知道$\{X_i'\}$的逆向极限也是$X$,所以按照\ref{逆向极限和有限表示态射2}的(1),把$i$替换为足够大的指标可以有同构$g_i:X_i\to X_i'$使得$g=g_i\times1_S$是$X$自身的同构.于是此时$X_i\cong X_i'\to S_i$就是有限表示嵌入.对于有限表示开嵌入和有限表示闭嵌入证明是完全相同的.
    	
    	\qquad
    	
    	证明(7)和(8):按照Chevalley定理,因为$f_0$是有限表示态射,于是$Z_0=f_0(X_0)$就是$S_0$的可构造集.对指标$i$取$Z_i=u_{0i}^{-1}(Z_0)$,再取$Z=u_0^{-1}(Z_0)$,那么按照【refI.3.4.8】就有$Z_i=f_i(X_i)$和$Z=f(X)$.按照\ref{逆向极限和可构造集5},我们有$u_{\textbf{C}}$是双射.于是倘若$f(X)=S$,那么$\{Z_i\}$和$\{S_i\}$都在$\varinjlim\textbf{C}(S_i)\to\textbf{C}(S)$下映为相同的元$S$,于是$i$足够大的时候就有$f_i(X_i)=Z_i=S_i$,也即$f_i$是满射.于是我们证明了(7).接下来(8)就是因为泛单的态射等价于对角态射是满射.具体地讲,我们有结构态射$X_0'=X_0\times_{S_0}X_0\to S_0$是有限表示态射,并且对角态射满足$\Delta_{f_i}=\Delta_{f_0}\times1_{S_i}$和$\Delta_f=\Delta_{f_0}\times1_S$.于是把(7)用在$X\to X'=X\times_SX$上即可.
    	
    	\qquad
    	
    	证明(9)和(10):记$S=\mathrm{Spec}A$,按照【EGAII5.1.9】,从$f:X\to S$是仿射态射得到存在一个正整数$r$和一个$S$态射闭嵌入$j:X\to\mathbb{A}_S^r=\mathrm{Spec}A[T_1,\cdots,T_r]$.如果$f$是拟仿射态射,则存在正整数$r$和一个$S$态射嵌入$j:X\to\mathbb{A}_S^r=\mathrm{Spec}A[T_1,\cdots,T_r]$.按照$\mathbb{A}^r_S=\mathbb{A}_{S_0}^r\times_{S_0}S$,并且$\mathbb{A}_{S_0}^r$是有限表示$S_0$概形,于是按照\ref{逆向极限和有限表示态射1},就存在指标$i$和一个有限表示$S_i$态射$j_i$,使得$j=j_i\times1_S$.于是按照$\textbf{P}$是闭嵌入和嵌入的情况,就有当把$i$替换为足够大的指标时有$j_i$是嵌入或者闭嵌入,对应于$X_i$是拟仿射的和仿射的.
    	
    	\qquad
    	
    	证明(11):记$S=\mathrm{Spec}A$,设$f:X\to S$是有限态射,那么$X=\mathrm{Spec}B$就也是仿射的,并且$B$作为$A$模是有限的.另外按照条件$B$还是有限表示$A$代数,我们解释过【ref】此时$B$作为$A$模也是有限表示的.于是按照\ref{逆向系统和有限表示模层2}的(3),存在指标$i$和一个有限表示$A_i$模$B_i$,使得有$A$模同构$B\cong B_i\otimes_{A_i}A$.另外$B$作为$A$代数的乘法相当于一个$A$模同态$m:B\otimes_AB\to B$.我们有$B\otimes_AB=(B_i\otimes_{A_i}B_i)\otimes_{A_i}A$,于是按照\ref{逆向系统和有限表示模层2}的(2),把$i$替换为更大的指标可以找到态射$m_i:B_i\otimes_{A_i}B_i\to B_i$满足$m=m_i\otimes1$.接下来考虑$m$满足的交换律,结合律以及存在幺元的图表,我们可以把$i$替换为足够大的指标使得$m_i$也满足相同的图表,于是此时$m_i$使得$B_i$构成一个$A_i$代数.取$X_i'=\mathrm{Spec}B_i$,那么有$S$概形同构$X\cong X_i'\times_{S_i}S$.最后按照$\textbf{P}$是同构的命题,我们可以把$i$替换为更大的指标使得$X_i$和$X_i'$是$S_i$同构的.于是$X_i\cong X_i'\to S_i$是有限态射.
    	$$\xymatrix{X_0'\ar[rr]^{j_0}\ar[d]_{g_0}&&P_0\ar[d]^{p_0}\\X_0\ar[rr]_{f_0}&&Y_0}$$
    	
    	证明(13):对有限表示态射$f_0:X_0\to S_0$用有限表示态射版本的Chow引理【EGA4的8.10.5.1】,那么存在有限表示$S_0$概形$X_0'$和$P_0$,以及射影态射$p_0:P_0\to S_0$,开嵌入$j_0:X'_0\to P_0$和射影态射且满射$g_0:X'_0\to X_0$,使得有$f_0\circ g_0=p_0\circ j_0$(上述图表).那么对每个指标$i$做相应的关于$S_i\to S_0$的基变换,我们得到如下交换图表:
    	$$\xymatrix{X_i'\ar[rr]^{j_i}\ar[d]_{g_i}&&P_i\ar[d]^{p_i}\\X_i\ar[rr]_{f_i}&&Y_i}\qquad\xymatrix{X'\ar[rr]^{j}\ar[d]_g&&P\ar[d]^p\\X\ar[rr]_f&&Y}$$
    	
    	我们知道射影态射一定是紧合的,于是$g$和$f$都是紧合态射,进而$p\circ j=f\circ g$也是紧合态射.另外从$p\circ j$紧合与$p$是分离态射,得到$j$是紧合的.但是紧合的嵌入就一定是闭嵌入.于是运用$\textbf{P}$是闭嵌入的命题,就可以找到指标$i$使得$j_i$也是闭嵌入,于是特别的它也是紧合态射.于是$p_i\circ j_i=f_i\circ g_i$也是紧合态射.因为满射也是基变换不变性质,于是$g_i$仍是满射.按照$\textbf{P}$是分离态射的命题,我们可以把$i$替换为更大的指标使得$f_i$是分离态射.最后按照消除诺特条件的操作,也即把$Y_i=\mathrm{Spec}A_i$写成$A_i$的有限型$\mathbb{Z}$代数的子环对应的仿射概形的逆向极限,就可以不妨约定$A_i$是诺特环并且$f_i$是有限型态射,于是按照【EGAII5.4.3的(ii)】就有$f_i$是紧合的.
    	
    	\qquad
    	
    	设$S_0$是qcqs概形,证明(14)和(15):在【EGAII5.5.3,条件放宽到qcqs也是成立的】中证明了终端是qcqs概形的态射,它是射影态射当且仅当它是拟射影和紧合的态射.于是按照我们证明的$\textbf{P}$是紧合态射的命题.问题就归结为证明$\textbf{P}$是拟射影的情况.设$f$是拟射影态射,按照定义就存在一个$\mathscr{O}_X$可逆模层$\mathscr{L}$是$f$丰沛的.按照\ref{逆向系统和有限表示模层2}的(3),因为$S_0$是qcqs概形,就存在$X_i$的有限表示拟凝聚模层$\mathscr{L}_i$,满足$\mathscr{L}=v_i^*(\mathscr{L}_i)$.再按照\ref{逆向系统和有限表示模层4},当$i$足够大时有$\mathscr{L}_i$是可逆模层.于是问题归结为\ref{和逆向极限兼容的态射性质4}.
    \end{proof}
\end{enumerate}
\newpage
\section{纤维积}
\subsection{概形的和}

概型的和(余积).
\begin{enumerate}
	\item 设$\{(X_i,\mathscr{O}_i),i\in I\}$是一族概型,记$X$作为拓扑空间是$\{X_i\}$的和(即集合意义上的无交并,子集是开集约定为和每个$X_i$的交是$X_i$中的开集).它的层$\mathscr{O}$取为全体$\mathscr{O}_i$的粘合.那么每个典范的态射$\varphi_i:X_i\to X$是开嵌入,并且$f\mapsto(f\circ\varphi_i)$是$\mathrm{Hom}(X,Y)\to\prod_i\mathrm{Hom}(X_i,Y)$的双射.于是$X$是$\{X_i\}$的范畴意义上的余积,也称为一族概型的和,记作$X=\coprod_iX_i$.
	\item 两个仿射概型的和.我们来验证两个仿射概型$X=\mathrm{Spec}(A)$和$Y=\mathrm{Spec}(B)$的和就是仿射概型$\mathrm{Spec}(A\times B)$.
	\begin{proof}
		
		容易说明的是$Z=X\coprod Y$的整体截面环就是$A\times B$,接下来仅需验证$Z$是仿射概型,就得到它是$\mathrm{Spec}(A\times B)$.为此我们用如下定理:一个概型$(X,\mathscr{O}_X)$是仿射概型,当且仅当存在有限个整体截面$f_1,f_2,\cdots,f_n\in\mathscr{O}(X)$生成了单位理想,并且每个$(X_{f_i},\mathscr{O}_X\mid_{f_i})$是仿射的.
		
		可取$Z$的整体截面$f=(1_A,0)$和$g=(0,1_B)$,它们生成了$A\times B$的单位理想.接下来按照定义$Z_f$表示$Z$中这样的元$p$构成的集合,使得$f$在$p$处stalk的像是单位,于是必然有$Z_f=A$和$Z_g=B$,按照定义它们都是仿射的,这就完成证明.
	\end{proof}
    \item 仿射概型总是拟紧的,说明无限个非零环的仿射概型的无交并总不是仿射概型.此时我们只有典范的单射$\coprod_i\mathrm{Spec}A_i\to\mathrm{Spec}\prod_iA_i$.例如取全部$A_i$都是一个域$k$,那么无交并中每个对应指标$i$的极大理想对应于$\prod_iA_i$的第$i$分量只取零,其余分量任取元素的极大理想.但是存在异于上述描述的极大理想.考虑积中的理想$\{(x)_i\mid x\in k\}$,这个真理想不包含在上述任意极大理想中,但是它包含在某个极大理想中,这就说明上述单射不是同构.
    \item 中国剩余定理.这个定理告诉我们知道一个数模去60等价于知道这个数模去3,4和5.我们来用概型语言做一个解释.考虑概型$\mathrm{Spec}\mathbb{Z}/60\mathbb{Z}$,它由三个素理想$(2),(3),(5)$构成.它们都是闭点,于是这个概型是离散空间.每个闭点也是开集,于是每个点的茎就是它作为开集的截面环.可依次求茎为$(\mathbb{Z}/60\mathbb{Z})_{(2)}=\mathbb{Z}/4\mathbb{Z}$和$(\mathbb{Z}/60\mathbb{Z})_{(3)}=\mathbb{Z}/3\mathbb{Z}$和$(\mathbb{Z}/60\mathbb{Z})_{(5)}=\mathbb{Z}/5\mathbb{Z}$.按照概型可表示为这三个单点开集的无交并,得到整体截面环的如下同构:
    $$\mathbb{Z}/60\mathbb{Z}\cong\mathbb{Z}/4\mathbb{Z}\times\mathbb{Z}/3\mathbb{Z}\times\mathbb{Z}/5\mathbb{Z}$$
\end{enumerate}
\subsection{一般纤维积的定义和性质}

纤维积的定义.
\begin{enumerate}
	\item 给定两个$S$概型$X,Y$,此即指定了两个态射$X\to S$和$Y\to S$,它们的一个锥是指$(Z,p,q)$,其中$Z$是概型,$p,q$是态射,满足如下图表交换:
	$$\xymatrix{Z\ar[rr]^p\ar[d]_q&&X\ar[d]\\Y\ar[rr]&&S}$$
	\item 给定$S$概型$X,Y$上的两个锥$(Z,p,q)$和$(T,p',q')$,从前者到后者的锥之间的态射是指一个$S$态射$\alpha:Z\to T$,满足如下图表交换:
	$$\xymatrix{Z\ar@/^1pc/[drr]^p\ar[dr]^{\alpha}\ar@/_1pc/[ddr]_{q}&&\\&T\ar[r]^{p'}\ar[d]_{q'}&X\ar[d]\\&Y\ar[r]&S}$$
	\item 给定两个$S$概型$X,Y$,它们的纤维积定义为一个锥$(Z,p,q)$,满足对任意一个锥$(T,p',q')$,使得存在唯一的态射$(T,p',q')\to(Z,p,q)$.这里我们把$p,q$称为投影态射,纤维积记作$X\times_SY$,一个如第一条中的方形图表称为笛卡尔的,如果$(Z,p,q)$是$X\to S$和$Y\to S$的纤维积.
	\item 纤维积即$S$概型范畴上的积对象,它的定义可以等价的用态射集的双射来描述:$(Z,p,q)$是一个锥,那么它是纤维积当且仅当$w\mapsto(p\circ w,q\circ w)$是$\mathrm{Hom}_S(T,Z)\to\mathrm{Hom}_S(T,X)\times\mathrm{Hom}_S(T,Y)$的双射.
\end{enumerate}

在一般范畴上,纤维积具有如下性质:
\begin{enumerate}
	\item 如果存在纤维积$X\times_SY$和$X'\times_SY'$,任取态射$u:X\to X'$和$v:Y\to Y'$,按照泛性质,存在唯一的态射$X\times_SY\to X'\times_SY'$使得如下图表交换,这个唯一的映射记作$u\times v$.
	$$\xymatrix{X\times_SY\ar[r]\ar[d]\ar[dr]&X\ar[dr]^u&\\Y\ar[dr]_v&X'\times_SY'\ar[d]\ar[r]&X'\ar[d]\\&Y'\ar[r]&S}$$
	\item 纤维积满足如下性质:
	\begin{enumerate}
		\item 恒等率:$X\times_SS\cong X$,这里$S\to S$取为恒等态射.
		\item 交换律:$X\times_SY\cong Y\times_SX$,前提是左右至少一个纤维积存在.
		\item 结合律:$(X\times_SY)\times_SZ\cong X\times_S(Y\times_SZ)$,前提是出现的纤维积都存在.
	\end{enumerate}
	\item 如果范畴上总存在纤维积,考虑如下交换图表,其中右侧小方块是笛卡尔的(也即是纤维积),那么左侧小方块是笛卡尔的当且仅当大矩形是笛卡尔的.
	$$\xymatrix{X''\ar[r]\ar[d]&X\ar[r]\ar[d]&X'\ar[d]\\S''\ar[r]&S\ar[r]&S'}$$
    \item 如果范畴上总存在纤维积,反复利用上一条中的结论,我们可以得到如下交换图,并且每个方格和矩形全部都是笛卡尔的.
	$$\xymatrix{X'\times_SX\ar[rr]\ar[d]&&Y'\times_SX\ar[rr]\ar[d]&&X\ar[d]^f\\X'\times_SY\ar[rr]\ar[d]&&Y'\times_SY\ar[d]\ar[rr]&&Y\ar[d]\\X'\ar[rr]_{f'}&&Y'\ar[rr]&&S}$$
	\item 单态射在纤维积中的提升态射仍然是单态射,这个命题对于满态射一般来讲不成立.
	\item "magic diagram".给定态射族:
	$$\xymatrix{X\ar[drr]&&&&Y\ar[dll]\\&&T\ar[d]&&\\&&S&&}$$
	
	那么有如下笛卡尔图表.
	$$\xymatrix{X\times_TY\ar[rr]\ar[d]&&X\times_SY\ar[d]\\T\ar[rr]^{\Delta}&&T\times_ST}$$
	
	这里态射的定义都是直接的,上方水平的态射是按照$(X\times_TY,p_T,q_T)$是$X\to T\to S$和$Y\to T\to S$上的锥,于是诱导了唯一的态射$X\times_TY\to X\times_SY$.右侧垂直态射是由$X\to T$和$Y\to T$诱导的.左侧垂直的态射是纤维积的结构态射.下方水平态射是对角态射.
	\begin{proof}
		
		首先验证图表交换,一个对象$W$到$T\times_ST$的态射被两个态射$\alpha:W\to T$和$\beta:W\to T$唯一决定,它们需要满足复合$T\to S$后得到相同的$W\to T\to S$的态射.按照这个刻画能给出图表交换性.
		
		现在任取$T\to T\times_ST$和$X\times_SY\to T\times_ST$上的一个锥$(W,\alpha,\beta)$.按照上一段,$\beta$被两个态射$\beta_1:W\to X$和$\beta_2:W\to Y$决定,它们满足复合上$\to T\to S$得到相同的态射.另外按照它是锥得到$\beta_1$和$\beta_2$复合上$\to T$恰好就是$\alpha$.于是我们得到如下交换图表,这导致存在态射$W\to X\times_TY$.容易验证这是唯一的锥之间的态射.
		$$\xymatrix{&W\ar[dr]^{\beta_2}\ar[dl]_{\beta_1}\ar[dd]_{\alpha}&\\X\ar[dr]&&Y\ar[dl]\\&T&}$$
	\end{proof}
\end{enumerate}
\subsection{概形上纤维积的存在性}

纤维积的泛性质描述说明了唯一性,下面证明$S$概型范畴上的纤维积总是存在的.思路是简单的:我们先证明$X,Y,S$都是仿射的情况下$X\times_SY$存在并且是仿射的,接下来用仿射开子集分别覆盖$X,Y,S$,然后把仿射情况的纤维积粘合就得到$X\times_SY$.我们即将看到和簇的纤维积不同的,概型的纤维积作为集合并不是集合上的笛卡尔积.
\begin{enumerate}
	\item 仿射情况.设$X=\mathrm{Spec}(A)$,$Y=\mathrm{Spec}(B)$,$S=\mathrm{Spec}(R)$.记$C=A\otimes_RB$,我们断言$(Z=\mathrm{Spec}(C),p,q)$是$S$概型$X$和$Y$的纤维积,其中$p$和$q$分别是由环同态$\alpha:A\to A\otimes_RB$,$a\mapsto a\otimes1$和$\beta:B\to A\otimes_RB$,$b\mapsto1\otimes b$诱导的态射.
	\begin{proof}
		\begin{align*}
		\mathrm{Hom}_{\textbf{S-Sch}}(T,Z)&\cong\mathrm{Hom}_{\textbf{R-Alg}}(A\otimes_RB,\mathscr{O}_T(T))\\&\cong\mathrm{Hom}_{\textbf{R-Alg}}(A,\mathscr{O}_T(T))\times\mathrm{Hom}_{\textbf{R-Alg}}(B,\mathscr{O}_T(T))\\&\cong\mathrm{Hom}_{\textbf{S-Sch}}(T,X)\times\mathrm{Hom}_{\textbf{S-Sch}}(T,Y)
		\end{align*}
		
		这个双射把$S$概型态射$\psi:T\to Z$映射为$\psi\to\varphi\to(\varphi\circ\alpha,\varphi\circ\beta)\to(p\circ\psi,q\circ\psi)$.其中第二步双射是因为张量积是$R$代数范畴上的余积.
	\end{proof}
	\item 若$(Z,p,q)$是$X$和$Y$的纤维积,设$U,V$分别是$X,Y$的开子集,记$W=p^{-1}(U)\cap q^{-1}(V)$,则$(W,p\mid_W,q\mid_W)$是$U$和$V$的纤维积.
	\begin{proof}
		
		任取$U,V$上的一个锥$(T,g,h)$,得到如下第一个交换图,记$g'$是$g$与包含映射$U\to X$的复合,记$h'$是$h$与包含映射$V\to Y$的复合.于是第一个交换图诱导了第二个交换图,按照假定了$X\times_SY$的存在性,于是存在唯一的态射$\alpha':T\to X\times_SY$使得图表交换,那么$\alpha'(T)\subseteq W$,于是诱导了唯一的使得图表交换的$\alpha:T\to W$,于是$W$是$U$和$V$的纤维积.
		$$\xymatrix{T\ar@/^1pc/[drr]^g\ar@/_1pc/[ddr]_h&&\\&W\ar[r]^{p\mid_W}\ar[d]_{q\mid_W}&U\ar[d]\\&V\ar[r]&S}\xymatrix{T\ar@/^1pc/[drr]^{g'}\ar@/_1pc/[ddr]_{h'}\ar[dr]^{\alpha'}&&\\&X\times_SY\ar[r]^{p}\ar[d]_{q}&X\ar[d]\\&Y\ar[r]&S}$$
	\end{proof}
	\item 设$(Z,p,q)$是$X,Y$的一个锥,设$\{U_i\}$和$\{V_j\}$分别是$X$和$Y$的开覆盖,对任意指标对$(i,j)$,记$W_{ij}=p^{-1}(U_i)\cap q^{-1}(V_j)$,有$(W_{ij},p\mid_{W_{ij}},q\mid_{W_{ij}})$是$U_i$和$V_j$的纤维积,那么$(Z,p,q)$是$X$和$Y$的纤维积.
	\begin{proof}
		
		设$(T,g,h)$是$X,Y$上的一个锥,考虑如下交换图,我们需要验证存在唯一的态射$T\to Z$使得图表交换.对每个指标对$(i,j)$,记$T_{ij}=g^{-1}(U_i)\cap h^{-1}(V_j)$,那么$\{T_{ij}\}$是$T$的一个开覆盖,按照$W_{ij}$是纤维积,说明存在唯一的态射$f_{ij}:T_{ij}\to W_{ij}$使得相应图表交换.这也说明了一旦使得下图交换的态射$T\to Z$存在,那么它是唯一的.为证明存在性,仅需验证这族态射$\{f_{ij}\}$可粘合为一个态射$f:T\to W$,为此仅需验证$f_{ij}$和$f_{st}$在$T_{ij}\cap T_{st}$上一致.这里$W_{ij}\cap W_{st}=p^{-1}(U_i\cap U_s)\cap q^{-1}(V_j\cap V_t)$,于是上一条说明$W_{ij}\cap W_{st}$和$p,q$在其上限制是$U_i\cap U_s$和$V_j\cap V_t$的纤维积.于是特别的$p\circ f_{ij}$和$p\circ f_{st}$在$T_{ij}\cap T_{st}$上一致,$q\circ f_{ij}$和$q\circ f_{st}$在$T_{ij}\cap T_{st}$上一致,这就说明在$T_{ij}\cap T_{st}$上$f_{ij}=f_{st}$.
		$$\xymatrix{T\ar@/^1pc/[drr]^{g}\ar@/_1pc/[ddr]_{h}&&\\&Z\ar[r]^{p}\ar[d]_{q}&X\ar[d]\\&Y\ar[r]&S}$$
	\end{proof}
	\item 设$\{U_{\alpha}\}$和$\{V_{\beta}\}$分别是$X$和$Y$的开覆盖,对每个指标对$(\alpha,\beta)$,有$U_{\alpha}$和$V_{\beta}$存在纤维积,那么$X$和$Y$存在纤维积.
	\begin{proof}
		
		记全体指标对$(\alpha,\beta)$构成的集合为$I$,对$=(\alpha,\gamma)\in I$,记$W_i=U_{\alpha}\times_SV_{\gamma}$,记$j=(\beta,\mu)\in I$.由第二条,有$W_i$的开子集$W_{ij}$同构于$(U_{\alpha}\cap U_{\beta})\times_S(V_{\gamma}\cap V_{\mu})$,记同构为$h_{ij}$.记$f_{ij}=h_{ji}^{-1}\circ h_{ij}:W_{ij}\to W_{ji}$.那么$f_{ij}=f^{-1}_{ji}$.可验证它们满足概型粘合引理的条件,于是$\{W_i\}$粘合为一个概型$Z$,并且存在一族开嵌入$\varphi_i:W_i\to Z$,使得$\{\varphi_i(W_i)\}$是$Z$的开覆盖,并且$\varphi_i(W_{ij})=\varphi_j(W_{ji})=\varphi_i(W_i)\cap\varphi_j(W_j)$,并且$\varphi_j\circ f_{ij}=\varphi_i$.
		
		记$W_i$作为纤维积的投射态射分别是$p_i$和$q_i$,记$h_i:W_i\to S$是结构态射.按照$p_i\circ\varphi_i^{-1}=p_j\circ\varphi_j^{-1}$在$\varphi_i(W_i)\cap\varphi_j(W_j)$上一致,于是$p_i$粘合为一个态射$p:Z\to  X$.同理$q_i$粘合为态射$q:Z\to Y$,另外$h_i$粘合为结构映射$Z\to S$.最后验证上一条的条件得到$(Z,p,q)$恰好是$X$和$Y$的纤维积.为此仅需验证$\varphi_i(W_i)=p^{-1}(U_{\alpha})\cap q^{-1}(V_{\lambda})$,把后者记作$W_i'$,那么有$\varphi_j(W_j)\cap W_i'=\varphi_j(W_j)\cap p^{-1}(U_{\alpha})\cap q^{-1}(V_{\lambda})=\varphi_j(W_j)\cap\varphi_jp_j^{-1}(U_{\alpha})\cap\varphi_jq_j^{-1}(V_{\lambda})=\varphi_j(W_j)\cap\varphi_j(w_{ji})=\varphi_j(W_j)\cap\varphi_i(W_i)$.再对$j$取并得到结果.
	\end{proof}
	\item 若$X\to S$和$Y\to S$有纤维积,设$S\to S'$是单态射,那么由此诱导的$X$和$Y$的$S'$概型存在纤维积,并且有同构$X\times_SY\cong X\times_{S'}Y$.这是因为按照$S\to S'$是单态射,得到$X\to S'$和$Y\to S'$上的锥也是$X\to S$和$Y\to S$上的锥.
	\item 设$\varphi:X\to S$和$\psi:Y\to S$是结构态射,设$\{S_i\}$是$S$的开覆盖.记$X_i=\varphi^{-1}(S_i)$和$Y_i=\psi^{-1}(S_i)$.如果每个$X_i\times_{S_i}Y_i$存在,那么它也是$X_i\times_SY_i$.并且此时$X\times_SY$也存在.
	\begin{proof}
		
		这里$X_i\times_SY_i$和$X_i\times_{S_i}Y_i$一致是因为$S_i\to S$是单态射.记$X_{ij}=X_i\cap X_j$和$Y_{ij}=Y_i\cap T_j$,记$Z_i=X_i\times_SY_i$,由第二条知$Z_{ij}=X_{ij}\times_SY_{ij}$存在,并且按照$\varphi\circ g(T)=\psi\circ h(T)\subseteq S_i\cap S_j$,说明$Z_{ij}$是$X_i$和$Y_j$的纤维积.于是第四条推出了$X\times_SY$存在.
		$$\xymatrix{T\ar@/^1pc/[drr]^{g}\ar@/_1pc/[ddr]_{h}&&\\&Z_{ij}\ar[r]^{p}\ar[d]_{q}&X_i\ar[d]\\&Y_j\ar[r]&S}$$
	\end{proof}
	\item 现在说明概型的纤维积的存在性.任取概型$S$,任取$S$概型$X$和$Y$,取$S$的仿射开覆盖$\{S_i\}$,按照第六条,仅需说明每个$X_i\times_{S_i}Y_i$存在.也即归结为$S$仿射的情况.现在再取$X$和$Y$分别的仿射开覆盖$\{U_i\}$和$\{V_j\}$,按照第一条,每个$U_i\times_SV_j$是存在的,再按照第四条,说明$X\times_SY$存在.
\end{enumerate}
\subsection{纤维积的例子}
\begin{enumerate}
	\item 设$A$是$B$代数,设$B[t]$是$B$上一元多项式环,那么有$A\otimes_BB[t]\cong A[t]$.这件事说明环$A$上仿射空间满足$\mathbb{A}_A^m\times_A\mathbb{A}_A^n\cong\mathbb{A}_A^{m+n}$.据此我们定义一般概型$X$上的$n$维仿射空间维$\mathbb{A}_X^n=X\times_{\mathbb{Z}}\mathrm{Spec}\mathbb{Z}[x_1,x_2,\cdots,x_n]$.类似定义$S$上的射影空间为$\mathbb{P}^n_S=\mathbb{P}^n_{\mathbb{Z}}\times_{\mathbb{Z}}S$.但是对于射影空间一般$\mathbb{P}^n_k\times_k\mathbb{P}^m_k\not\cong\mathbb{P}_k^{n+m}$.
	\item 设$A$是环,$B$是分次$A$代数(这是指$B$是分次环也是$A$代数,并且结构映射把$A$都打成零次元),$C$是$A$代数,那么$E=B\otimes_AC$上自然的存在一个分次$C$代数结构:$E=\oplus_{d\ge0}(B_d\otimes_AC)$,这也是一个分次$A$代数.我们断言有典范的同构:
	$$\mathrm{Proj}(B\otimes_AC)\cong\mathrm{Proj}B\times_{\mathrm{Spec}A}\mathrm{Spec}C$$
	\begin{proof}
		
		设分次环同态$\varphi:B\to E=B\otimes_AC$为$b\mapsto b\otimes1$.那么$\varphi(B_+)E=E_+$,所以这个分次环同态诱导了射影概形之间的态射$g:\mathrm{Proj}E\to\mathrm{Proj}B$.另外按照$E$是$C$代数,所以有一个结构态射$\mathrm{Proj}E\to\mathrm{Spec}C$.这些态射都是$A$概形之间的态射,所以按照纤维积的泛性质诱导了态射:
		$$h:\mathrm{Proj}E\to\mathrm{Proj}B\times_{\mathrm{Spec}A}\mathrm{Spec}C$$
		
		只需证明$h$是同构.任取齐次元$f\in B_+$,那么$h^{-1}(D_+(f)\times_A\mathrm{Spec}C)=g^{-1}(D_+(f))=D_+(\varphi(f))$.所以归结为证明$h$限制在$D_+(\varphi(f))\to D_+(f)\times_A\mathrm{Spec}C$上是同构.但是它对应的环同态是$\psi:B_{(f)}\otimes_AC\to E_{(\varphi(f))}$为$(b/f^n)\otimes c\mapsto (b\otimes c)/\varphi(f)^n$.这明显是满射,单射是因为它是同构$B_f\otimes_AC\cong E_{\varphi(f)}$在零次子环上的限制.
	\end{proof}
    \item 代数闭域上的预簇的纤维积仍然是预簇.无论预簇这个词用哪种定义都是成立的.因为代数闭域上概形的如下性质都是保纤维积的:整概形,既约概形,有限型,局部有限型.另外簇是指分离的预簇,而分离也在纤维积下不变,所以代数闭域上的簇的纤维积仍然是簇.但是注意这里代数闭域是必须的,因为既约或者整概形保纤维积需要这个条件.
\end{enumerate}
\subsection{可表函子语言下概型纤维积的存在性}

可表函子.考虑从$\textbf{S-Sch}^{\mathrm{op}}\to\textbf{Sets}$的函子,这样的函子称为可表的,如果它自然同构于某个$h_X$,即$h_X(Y)=\mathrm{Hom}_S(Y,X)$.
\begin{enumerate}
	\item 问题转化为一个纤维积函子是可表的.取定三个逆变函子$h_1,h_2,h_3$,取定两个自然变换$a:h_1\to h_2$和$b:h_3\to h_2$.对任意对象$W$,集合范畴上的纤维积$h_1(W)\times_{h_2(W)}h_3(W)$总是存在的.于是概形的纤维积$X\times_SY$的存在性等价于讲可表函子的纤维积$h_X\times_{h_S}h_Y$是可表的,其中两个自然变换$a:h_X\to h_S$和$b:h_Y\to h_S$按照米田引理对应于$X\times_SY$的结构态射$X\to S$和$Y\to S$.
	\item Zariski层.给定逆变函子$F:\textbf{S-Sch}\to\textbf{Sets}$,对每个$S$概形$X$,它总可以作为一个预层,即把开子集$U$映射为$F(U)$.如果对每个$S$概形$X$这个预层总是层,就称$F$是Zariski拓扑的层,简称为Zariski层.换句话讲它需要满足对每个$S$概型$X$和$X$的每个开覆盖$\{U_i\}$,如果给定一族$\{\eta_i\in F(U_i)\mid i\in I\}$满足$\eta_i\mid U_i\cap U_j=\eta_j\mid U_i\cap U_j$对任意$i,j$成立.那么必存在唯一的$\eta\in F(X)$,使得$\eta\mid U_i=\eta_i,\forall i$成立.
	\begin{itemize}
		\item 可表函子总是Zariski层.这件事是态射粘合的等价描述.
		\item 可表函子的纤维积总是Zariski层.
		\begin{proof}
			
			考虑函子$F=h_X\times_{h_S}h_Y$,设$S$概型$X,Y$的结构映射分别为$p$和$q$.任取$S$概型$T$,按照定义$h_X\times_{h_S}h_Y(T)$表示全体态射对$(f,g)$,其中$f:T\to X$和$g:T\to Y$使得$p\circ f=q\circ g$.
			
			\qquad
			
			任取$T$的开覆盖$\{U_i\}$.取$\eta_i\in F(U_i)$,使得$\eta_i\mid U_i\cap U_j=\eta_j\mid U_i\cap U_j$对任意$i,j$成立.按照定义每个$\eta_i$对应于一个对$(a_i,b_i)$,其中$a_i:U_i\to X$和$b_i:U\to Y$都是态射,并且满足$p\circ a_i=q\circ b_i$.另外有$a_i\mid U_i\cap U_j=a_j\mid U_i\cap U_j$,所以$\{a_i\}$可以唯一的粘合为一个态射$a:T\to X$.同理$\{b_i\}$可以唯一的粘合为一个态射$b:T\to Y$.那么$p\circ a=q\circ b$.于是$(a,b)$对应了$F(T)$中的元$\eta$,满足$\eta\mid U_i=\eta_i$.并且这是唯一的.
		\end{proof}
	\end{itemize}
	\item 开子函子.给定逆变函子$h,h':\textbf{S-Sch}\to\textbf{Sets}$,称$h'$是$h$的开子函子,如果存在自然变换$h'\to h$,使得对任意可表函子$h_X$和任意自然变换$h_X\to h$,纤维积$h_X\times_hh'$总是可表的,记作$h_U$,那么投影映射$h_U\to h_X$对应于某个开嵌入$U\to X$.
	$$\xymatrix{h_U\ar[rr]\ar[d]&&h_X\ar[d]\\h'\ar[rr]&&h}$$
	
	给定函子$h$的一族开子函子$\{h_i\}$,称这族开子函子覆盖了$h$,如果对每个可表函子$h_X$,定义中由$h_i$确定的开子概型记作$U_i\subseteq X$,那么$\{U_i\}$总是覆盖了整个$X$.
	\begin{itemize}
		\item 可表函子之间的自然变换$h_X\to h_Y$定义了开子函子当且仅当对应的态射$X\to Y$是开嵌入.这说明函子的开子函子的确是空间上开嵌入的一种推广.
		\begin{proof}
			
			必要性就取如下图表,图表交换说明对应的态射$X\to Y$是开嵌入:
			$$\xymatrix{h_X\ar[rr]\ar@{=}[d]&&h_Y\ar@{=}[d]\\h_X\ar[rr]&&h_Y}$$
			
			充分性是因为任取$S$概形之间的态射$f:T\to Y$,那么纤维积$X\times_YT$总是存在的,它就是$Y$的开子概型$f^{-1}(X)$.
			$$\xymatrix{h_{f^{-1}(X)}\ar[rr]\ar[d]&&h_T\ar[d]^{h_f}\\h_X\ar[rr]&&h_Y}$$
		\end{proof}
		\item 开子函子保复合.如果$F_3\to F_2$和$F_2\to F_1$都是开子函子,那么复合的$F_3\to F_1$也是开子函子.证明只要用如下交换图表:
		$$\xymatrix{F_3\times_{F_2}h_X\ar[rr]\ar[d]&&F_2\times_{F_1}h_X\ar[rr]\ar[d]&&h_X\ar[d]\\F_3\ar[rr]&&F_2\ar[rr]&&F_1}$$
		\item 设$\rho:F'\to F$是开子函子,那么对每个$S$概形$T$,都有$\rho(T):F'(T)\to F(T)$是单射.于是如果给定开子函子$F'\to F$,对任意$S$概形$T$,我们可以典范的把$F'(T)$视为$F(T)$的子集.
		\begin{proof}
			
			按照米田引理,有如下交换图表:
			$$\xymatrix{\mathrm{Nat}(h_T,F')\ar[rr]^{\sim}\ar[d]^{\rho(T)}&&F'(T)\ar[d]\\\mathrm{Nat}(h_T,F)\ar[rr]^{\sim}&&F(T)}$$
			
			设$x,y\in F'(T)$,把它们等同于对应的$h_T\to F'$的自然变换.如果有$p(T)(x)=p(T)(y)$,换句话讲有$p\circ x=p\circ y$,需要验证$x=y$.按照定义有如下纤维积图表,其中$U$是$T$的开子概型:
			$$\xymatrix{h_U\ar[rr]\ar[d]&&h_T\ar[d]^{\rho\circ x}\\F'\ar[rr]_{\rho}&&F}$$
			
			$(h_T,y,\mathrm{id}_{h_T})$是这个纤维积图表的锥,所以存在态射$h_T\to h_U$使得图表交换:
			$$\xymatrix{h_T\ar[drr]\ar@/^1pc/[drrrr]^{\mathrm{id}}\ar@/_1pc/[ddrr]_y&&&&\\&&h_U\ar[rr]\ar[d]&&h_T\ar[d]^{\rho\circ x}\\&&F\ar[rr]_{\rho}&&G}$$
			
			按照米田引理,从$h_T\to h_U\to h_T$是$h_T$上的恒等态射,说明存在态射的复合$T\to U\to T$是恒等态射.但是$U\to T$是开嵌入,这只能在$U=T$的情况下实现.于是锥$(h_T,y,\mathrm{id}_{h_T})$是纤维积,考虑如下图表,那么存在$a:h_T\to h_T$使得图表交换,那么$a=\mathrm{id}$,就导致$x=y$,证毕.
			$$\xymatrix{h_T\ar[drr]_a\ar@/^1pc/[drrrr]^{\mathrm{id}}\ar@/_1pc/[ddrr]_x&&&&\\&&h_T\ar[rr]^{\mathrm{id}}\ar[d]_y&&h_T\ar[d]^{\rho\circ x}\\&&F\ar[rr]_{\rho}&&G}$$
		\end{proof}
	    \item 如果$F_1\to F$和$F_2\to F$是两个开子函子,那么对任意$S$概形$T$有纤维积$F_1\times_FF_2(T)=F_1(T)\cap F_2(T)$(这有意义是因为上一条我们解释了$F_1(T)\to F(T)$是单射),并且这个纤维积的投影态射都是开子函子,于是特别的$F_1\times_FF_2\to F$也是开子函子.
	    \begin{proof}
	    	
	    	按照定义有$F_1\times_FF_2(T)$是集合上的纤维积$F_1(T)\times_{F(T)}F_2(T)$.但是这里结构映射都是单射,我们解释了把$F_1(T)$和$F_2(T)$就视为这典范单射的像集,那么这个纤维积实际上就是$F_1(T)\cap F_2(T)$.
	    	
	    	\qquad
	    	
	    	取$F_1\times_FF_2\to F_1\to F$是典范的结构态射,取可表函子$h_X$,按照$F_1\to F$是开子函子,说明纤维积$h_X\times_FF_1$是一个可表函子$h_Y$.再考虑纤维积$(F_2\times_FF_1)\times_{F_1}h_Y=F_2\times_Fh_Y$,按照$F_2\to F$是开子函子说明这也是一个可表函子$h_Z$,于是得到如下纤维积图表,这说明$F_1\times_FF_2\to F$是开子函子.
	    	$$\xymatrix{h_Z\ar[rr]\ar[d]&&h_Y\ar[rr]\ar[d]&&h_X\ar[d]\\F_1\times_FF_2\ar[rr]&&F_1\ar[rr]&&F}$$
	    \end{proof}
        \item 按照米田引理,如果$U\to X$是开子概型,那么对应的自然变换$h_U\to h_X$就是开子函子.如果$V\to X$也是开子概型,那么纤维积$h_V\times_{h_X}h_U$就是$h_{U\cap V}$.
        \item 设$X,Y,Z$是$S$概形,设有结构态射$X\to Z$和$Y\to Z$定义了函子的纤维积$h_X\times_{h_Z}h_Y$.如果$U\subseteq X$,$V\subseteq Y$,$W\subseteq Z$都是开子概型,使得$U,V$在上述结构态射下都映入$W$,那么$h_U\times_{h_W}h_V$是$h_X\times_{h_Z}h_Y$的开子函子.
        \begin{proof}
        	
        	任取可表函子$h_T$,任取自然变换$h_T\to h_X\times_{h_Z}h_Y$,它对应于两个自然变换$h_T\to h_X$和$h_T\to h_Y$,按照米田引理又对应于两个态射$f:T\to X$和$g:T\to Y$.考虑如下交换图表,下方小方格和整个图表都是笛卡尔的,于是上方小方格是笛卡尔的:
        	$$\xymatrix{h_{f^{-1}(U)}\ar[rr]\ar[d]&&h_T\ar[d]\\h_U\times_{h_Z}h_Y\ar[rr]\ar[d]&&h_X\times_{h_Z}h_Y\ar[d]\\h_U\ar[rr]&&h_X}$$
        	
        	类似的如下交换图表下面两个方格都是笛卡尔的.上面第一个方格是笛卡尔的是我们证明过的,于是这个图表的所有矩形都是笛卡尔的:
        	$$\xymatrix{h_{f^{-1}(U)\cap g^{-1}(V)}\ar[rr]\ar[d]&&h_{f^{-1}(U)}\ar[d]\\h_{g^{-1}(V)}\ar[rr]\ar[d]&&h_T\ar[d]\\h_U\times_{h_Z}h_V\ar[rr]\ar[d]&&h_U\times_{h_Z}h_Y\ar[d]\\h_V\ar[rr]&&h_Y}$$
        	
        	整理下就有:
        	\begin{align*}
        	h_T\times_{(h_X\times_{h_Z}h_Y)}(h_U\times_{h_W}h_V)&=h_T\times_{(h_X\times_{h_Z}h_Y)}(h_U\times_{h_Z}h_Y)\times_{(h_U\times_{h_Z}h_Y)}(h_U\times_{h_W}h_Y)\\&=h_{f^{-1}(U)}\times_{(h_U\times_{h_Z}h_Y)}(h_U\times_{h_W}h_Y)\\&=h_{f^{-1}(U)\cap g^{-1}(V)}
        	\end{align*}
        \end{proof}
		\item 可表准则.设逆变函子$F:\textbf{S-Sch}\to\textbf{Sets}$是Zariski层,设$F$被一族可表的开子函子$\{f_i:F_i=h_{X_i}\to F\mid i\in I\}$覆盖,那么$F$是可表的.
		\begin{proof}
			
			我们的目标是把$\{X_i\mid i\in I\}$粘合为一个概形$X$,使得它表示了函子$F$.对$S$概形$T$,按照$F_i\to F$是开子函子,并且$F_j$是可表的,说明$F_i\times_FF_j$是可表函子,按照$(F_i\times_FF_j)(T)=F_i(T)\cap F_j(T)$,得到函子$F_i\times_FF_j$和$F_j\times_FF_i$是相同的,设表示这个函子的概形为$X_{\{i,j\}}$.
			
			\qquad
			
			我们有被典范投影$F_i\times_FF_j\to F_i$诱导的开嵌入$X_{\{i,j\}}\to X_i$.设$U_{ij}\subseteq X_i$是这个开嵌入对应的开子概型.记同构$\psi_{ij}:X_{\{i,j\}}\cong U_{ij}$,再记同构$\varphi_{ij}=\psi_{ji}\circ\psi_{ij}^{-1}:U_{ij}\cong U_{ji}$.我们来证明$\left((X_i)_{i\in I},(U_{ij}),(\varphi)_{ij}\right)$是一组粘合信息.换句话讲它要满足所谓的cocycle条件:在$U{ij}\cap U_{jk}$上有$\varphi_{jk}\circ\varphi_{ij}=\varphi_{ik}$,这只要考虑如下交换图表,按照$F_{\{i,j,k\}}=F_i\times_FF_j\times_FF_k$上恒等映射的复合是恒等的,就得到cocycle条件:
			$$\xymatrix{h_{U_{ij}\cap U_{ik}}\ar[rr]^{\varphi_{ij}}\ar[d]_{\cong}&&h_{U_{ji}\cap U_{jk}}\ar[d]_{\cong}\\F_{\{i,j,k\}}\ar[rr]^{\mathrm{id}}&&F_{\{i,j,k\}}}$$
			
			按照米田引理,设开子函子$F_i\to F$对应于$F(X_i)$中的元$a_i$,按照$F$是Zariski层,$\{a_i\mid i\in I\}$可粘合为一个元$a\in F(X)$,设它对应于自然变换$\alpha:h_X\to F$.我们断言这个$\alpha$是自然同构.为此任取$S$范畴$T$,任取$\xi\in F(T)$.按照$\{F_i\mid i\in I\}$覆盖了$F$,说明每个$F_i\to F$对应的开嵌入$T_i\to T$覆盖了$T$.设$\xi\in F(T)$在$T_i$上的限制为$\xi_i$,则$\xi_i$对应于态射$g_i:T_i\to X_i$,态射族$\{g_i\mid i\in I\}$是可粘合的,因为$\{xi_i\}$在交集的限制上是相同的.设粘合为态射$g:T\to X$,那么$g$唯一满足$\alpha(T)(g)=\xi$.
		\end{proof}
	\end{itemize}
	\item 纤维积的存在性,即$h_X\times_{h_Z}h_Y$的可表性.
	\begin{proof}
		
		取$X,Y,Z$分别的仿射开子集$U,V,W$,使得$U,V$在结构态射下打到$W$中,那么$h_U\times_{h_W}h_V$是可表的开子函子,并且当$\{U,V,W\}$跑遍满足这个条件的仿射开子集时$\{h_U\times_{h_W}h_V\}$构成了$h_X\times_{h_Z}h_Y$的开覆盖,因为对任意$S$概形$T$,对任意态射$f:T\to X$和$g:T\to Y$,有$\{f^{-1}(U)\cap g^{-1}(V)\}$在$U,V$分别跑遍$X,Y$的仿射开子集时覆盖了整个$T$.另外我们证明过可表函子的纤维积$h_X\times_{h_Z}h_Y$是Zariski层.按照可表准则得证.
	\end{proof}
\end{enumerate}
\subsection{基变换}

基变换.给定范畴$\mathscr{C}$上的$S$对象,也即一个态射$X\to S$,给定态射$Y\to S$,这得到一个纤维积$X\times_SY$.投影态射$X\times_SY\to Y$称为$X\to S$的关于$Y\to S$的提升,反过来称$X\to S$是$X\times_SY\to Y$的下降.如果P是一个关于态射的性质,并且在复合下不变,称P是基变换不变性质,或者P在基变换下不变,如果在如下三种意义下的基变换(在P满足复合的前推下这三种定义出来的基变换不变是一致的)总是都满足性质P.设$u:S'\to S$是态射:
\begin{itemize}
	\item 如果$X\to S$是一个$S$对象,那么$X\times_SS'$是一个$S'$对象,它的结构态射是纤维积中的投影态射$X\times_SS'\to S'$.这个$S'$对象一般记作$X_{(S')}$,这个态射记作$f_{(S')}$,它称为$X\to S$的基变换.
	$$\xymatrix{X\times_SS'\ar[rr]\ar[d]&&S'\ar[d]\\X\ar[rr]&&S}$$
	\item 如果$f:X\to Y$是一个$S$态射,它诱导的$S'$态射$X\times_SS'\to Y\times_SS'$称为$f$的基变换.这是由$u$诱导的$S$概型范畴到$S'$概型范畴的函子.
	$$\xymatrix{X\times_SS'\ar[rr]^{f\times_S\mathrm{id}_{S'}}\ar[d]&&Y\times_SS'\ar[rr]\ar[d]&&S'\ar[d]\\X\ar[rr]_f&&Y\ar[rr]&&S}$$
	\item 给定两个$S$态射$f:X\to Y$和$f':X'\to Y'$,称$f\times_Sf':X\times_SY\to X'\times_SY'$是这两个态射的基变换.
	$$\xymatrix{X'\times_SX\ar[rr]\ar[d]&&Y'\times_SX\ar[rr]\ar[d]&&X\ar[d]^f\\X'\times_SY\ar[rr]\ar[d]&&Y'\times_SY\ar[d]\ar[rr]&&Y\ar[d]\\X'\ar[rr]_{f'}&&Y'\ar[rr]&&S}$$
\end{itemize}
\begin{enumerate}
	\item 我们之前给出过的态射的绝大多数性质都是基变换下不变的(闭嵌入,开嵌入,拟紧,仿射,有限,分离,有限型,紧合,射影,平坦,满射).不是基变换下不变的性质比方说具有有限纤维,单射,双射.
	\item 还有一种常见的模型是域$k$上两个概型$X$和$Y$,通常称它们满足一个关于态射的性质$P$,如果结构态射$X\to\mathrm{Spec}k$和$\mathrm{Spec}k$都满足性质$P$.如果这个性质$P$保复合并且保基变换,那么$X\times_kY$也是满足性质$P$的.
\end{enumerate}

态射性质的传递.
\begin{enumerate}
	\item 如果P是一个关于态射的性质,它在基变换和态射的复合下不变,如果$X\to Y\to Z$和$\Delta_{Y/Z}$都满足P,那么$X\to Y$满足P.
	\begin{proof}
		
		事实上,按照$\Delta_{Y/Z}$满足性质P,它的图像态射$\Gamma_{Y/Z}:X\to X\times_ZY$,作为它的提升,也满足性质P.而$X\to Y\to Z$满足性质P得到提升$X\times_ZY\to Y$满足性质P.它们的复合就是$\pi:X\to Y$,于是满足性质P.
	\end{proof}
    \begin{itemize}
    	\item 如果$X\to Y\to Z$是嵌入,那么$X\to Y$是嵌入.特别的嵌入的截面(左逆态射)一定是嵌入.
    	\item 如果$X\to Y\to Z$是拟紧的,$Y$是诺特的(诺特概型为源端的态射都是拟紧的),那么$X\to Y$是拟紧的.
    	\item 如果$X\to Y\to Z$是拟分离的,$Y$是拟分离的(拟分离概型为源端的态射是拟分离的),那么$X\to Y$是拟分离的.
    	\item 如果$X\to Y\to Z$是分离的,那么$X\to Y$是分离的(因为$Y\to Z$的对角态射是嵌入,嵌入是单态射,单态射总是分离的).
    \end{itemize}
    \item 设P是一个关于态射的性质,闭嵌入总满足P,并且性质P在基变换和态射的复合下不变,如果$Y\to Z$是分离的,$X\to Y\to Z$满足P,那么$X\to Y$满足P.另外如果$f:X\to Y$满足P,那么$f_{\mathrm{red}}:X_{\mathrm{red}}\to Y_{\mathrm{red}}$也满足P.
    \begin{proof}
    	
    	命题的前半部分是上一条的特例.对于后半部分,如果$f:X\to Y$满足性质P,考虑如下交换图表,按照闭嵌入$X_{\mathrm{red}}\to X$满足性质P,复合$X_{\mathrm{red}}\to X\to Y$就满足性质P,于是$X_{\mathrm{red}}\to Y_{\mathrm{red}}\to Y$满足性质P,但是这里$Y_{\mathrm{red}}\to Y$是闭嵌入,它的对角态射还是闭嵌入也满足性质P,所以前半部分就说明$f_{\mathrm{red}}:X_{\mathrm{red}}\to Y_{\mathrm{red}}$满足P.
    	$$\xymatrix{X_{\mathrm{red}}\ar[rr]^{f_{\mathrm{red}}}\ar[d]&&Y_{\mathrm{red}}\ar[d]\\X\ar[rr]_f&&Y}$$
    \end{proof}
    \begin{itemize}
    	\item 设$S$是分离概形,那么一个$S$概形$X$的结构态射是分离的当且仅当$X$是分离概形.特别的,由于仿射概形之间的态射总是分离的,说明如果$A$是环,那么一个$A$概形$X$的结构态射分离当且仅当$X$是分离概形.
    	\item 如果$X\to Y\to Z$是闭嵌入,$Y\to Z$是分离的,那么$X\to Y$是闭嵌入,特别的,分离态射的截面(左逆态射)总是闭嵌入.
    	\item 如果$X\to Y\to Z$是拟紧的,$Y\to Z$是分离的,那么$X\to Y$是拟紧的.
    	\item 如果$X\to Y\to Z$是局部有限型态射,那么$X\to Y$是局部有限型态射,因为嵌入总是局部有限型态射.
    	\item 如果$X\to Y\to Z$是有限型态射,$Y\to Z$是分离的,那么$X\to Y$是有限型态射.
    	\item 如果$X\to Y\to Z$是紧合态射,$Y\to Z$是分离的,那么$X\to Y$是紧合态射.
    	\item 如果$X\to Y\to Z$是射影态射,$Y\to Z$是分离的,那么$X\to Y$是射影态射.
    \end{itemize}
\end{enumerate}
\subsection{泛P性质}

泛P性质.给定态射$f:X\to S$,设P是一个关于态射的性质,称$f$是泛P的,如果对每个态射$u:S'\to S$,都有$f_{(S')}:X\times_SS'\to S'$满足性质P.
\begin{enumerate}
	\item 泛P性质是为了修复不在基变换下不变的性质用的,如果性质P本身已经在基变换下不变了,那么态射满足P和满足泛P是一回事.特别的,泛P态射必然满足性质P,泛P性质是基变换不变的.比方说单射满射双射中我们解释过只有满射是基变换不变的,所以泛满射和满射是一回事.
    $$\xymatrix{X\times_YS'\ar[rr]\ar[d]&&S'\ar[d]\\X\times_YS\ar[rr]\ar[d]&&S\ar[d]\\X\ar[rr]&&Y}$$
    \item 如果性质$P$是终端局部性质或者仿射终端局部性质,那么泛$P$也是终端局部性质或者仿射终端局部性质.例如泛单射,泛双射,泛同胚,泛开,泛闭都是终端局部性质.
\end{enumerate}

泛单态射.
\begin{enumerate}
    \item 引理.射$X,Y$是$S$概形,典范态射$X_{\mathrm{red}}\to X$和$Y_{\mathrm{red}}\to Y$诱导了如下同构:
    $$(X_{\mathrm{red}}\times_{S_{\mathrm{red}}}Y_{\mathrm{red}})_{\mathrm{red}}=(X_{\mathrm{red}}\times_SY_{\mathrm{red}})_{\mathrm{red}}\cong(X\times_SY)_{\mathrm{red}}$$
    \begin{proof}
    	
    	因为$S_{\mathrm{red}}\to S$是单态射(闭嵌入是单态射),就有典范同构$X_{\mathrm{red}}\times_{S_{\mathrm{red}}}Y_{\mathrm{red}}=X_{\mathrm{red}}\times_SY_{\mathrm{red}}$.由于$X_{\mathrm{red}}\to X$和$Y_{\mathrm{red}}\to Y$都是满射和闭嵌入,这两个性质都是基变换下不变的,并且保复合,于是诱导的$X_{\mathrm{red}}\times_SY_{\mathrm{red}}\to X\times_SY$也是满射和闭嵌入,于是它诱导了既约闭子概型上的同构.
    \end{proof}
	\item 设$f:X\to Y$是概形之间的态射,那么$f$是泛单射的当且仅当$f$是单射,并且对任意$x\in X$有$f^{\#}_x:\mathrm{O}_{Y,f(x)}\to\mathscr{O}_{X,x}$诱导的$\kappa(f(x))\to\kappa(x)$总是纯不可分扩张.
	\begin{proof}
		
		必要性.假设$\kappa(x)$不是$\kappa(f(x))$的纯不可分扩张,记$\kappa(f(x))$的代数闭包为$K$,那么存在两个不同的$\kappa(x)\to K$的$\kappa(f(x))$嵌入.这些嵌入诱导了两个不同的$\mathrm{Spec}K\to\mathrm{Spec}\kappa(x)\to X$的$Y$态射,进而诱导了两个不同的$\mathrm{Spec}K\to X\times_YK$的$K$态射,我们解释过这样的态射被$X\times_YK$中的剩余域为$K$的点唯一确定,导致了$X\times_YK$中至少有两个点,这导致基变换$X\times_YK\to\mathrm{Spec}K$不可能是单射,于是$f$不是泛单射的.
		
		\qquad
		
		充分性.我们要证明$f$是泛单射的,也即对任意态射$Y'\to Y$,对任意点$y'\in Y'$,有纤维$(X\times_YY')_{y'}=X_y\times_{\kappa(y)}\kappa(y')$至多包含一个点.这里$X_y$表示$y$在$f$下的纤维.不妨射$X_y$不是空集,那么它是单点集$\{x\}$(因为$f$是单射),并且$(X_y)_{\mathrm{red}}=\mathrm{Spec}K$,这里$K=\kappa(x)$,按照条件就有$\kappa(y)\subseteq K$是纯不可以扩张.但是要证明$X_y\times_{\kappa(y)}\kappa(y')$至多包含一个点,等价于证它的既约闭子概型至多包含一个点,按照引理就等价于证$\mathrm{Spec}K\otimes_{\kappa(y)}\kappa(y')$的既约闭子概型至多包含一个点,也就等价于证$\mathrm{Spec}K\otimes_{\kappa(y)}\kappa(y')$本身至多包含一个点.这归结为如下事实:如果$k\subseteq K$是纯不可分扩张,那么对任意域扩张$k\subseteq k'$,有$K\otimes_kk'$至多只有一个素理想.
	\end{proof}
    \item 设$f:X\to Y$是概形之间的态射,如下命题互相等价:
    \begin{enumerate}
    	\item $f$是泛单态射.
    	\item 对任意域$K$,都有$f(K):X(K)\to Y(K)$是单射.
    	\item 对任意域$K$,存在一个包含$K$的代数闭域$K'$,使得$f(K'):X(K')\to Y(K')$是单射.
    	\item 对角态射$\Delta_f$是满射.
    \end{enumerate}
    \begin{proof}
    	
    	(a)推(b):设$y\in Y(K)$,那么$x\in X(K)$使得$f(K)(x)=y$也即有如下交换图表.于是满足$f\circ x=y$的$x$等价于一个态射$\mathrm{Spec}K\to X_K$使得它复合上$p$是恒等态射.而这样的态射等价于$X_K$上的一个$K$值点.但是由于$f$是泛单的,导致$X_K\to\mathrm{Spec}K$是单射,于是$X_K$里的$K$值点只有一个,这就证明了$f(K)$是单射.
    	$$\xymatrix{\mathrm{Spec}K\ar@/^1pc/@{=}[drr]\ar@/_1pc/[ddr]_x\ar[dr]&&\\&X_K\ar[r]\ar[d]_p&\mathrm{Spec}K\ar[d]^y\\&X\ar[r]_f&Y}$$
    	
    	(b)推(c)是平凡的,下面证明(c)推(a):先考虑如下任意的基变换图表,我们要证明的是$f'$是单射.
    	$$\xymatrix{X'=X\times_YY'\ar[rr]^{f'}\ar[d]_p&&Y'\ar[d]\\X\ar[rr]&&Y}$$
    	
    	设$x_1,x_2\in X'$使得$f'(x_1)=f'(x_2)=y'$,我们要证明的就是$x_1=x_2$.我们可以选取一个域$K$使得有如下交换图表,并且把$K$替换为包含$K$的某个代数闭域$K'$不影响图表交换,我们不妨就设$K$本身就是一个满足$f(K)$是单射的代数闭域.
    	$$\xymatrix{K&&\kappa(x_1)\ar[ll]_{a_1}\\\kappa(x_2)\ar[u]^{a_2}&&\kappa(y')\ar[u]\ar[ll]}$$
    	
    	那么我们知道$x_1,x_2$诱导了两个态射$\mathrm{Spec}K\to X'$,仍然记作$x_1$和$x_2$,那么作为态射有$x_1=x_2$就推出作为点它们相同.如果把这个图表的复合态射记作$y'$,它诱导的$\mathrm{Spec}K\to Y'$仍然记作$y'$,那么$f'\circ x_1=f'\circ x_2=y'$.但是由于$p:X'\to X$满足$p\circ x_1$和$p\circ x_2$在$f(K)$下的像都是$y'$复合上$Y'\to Y$,所以$f(K)$是单射导致$p\circ x_1=p\circ x_2$记作$x$,于是我们有如下交换图表,于是按照纤维积的泛性质得到$x_1=x_2$.
    	$$\xymatrix{\mathrm{Spec}K\ar@/^1pc/[drr]^{y'}\ar@/_1pc/[ddr]_x\ar@<0.5ex>[dr]^{a_1}\ar@<-0.5ex>[dr]_{a_2}&&\\&X'\ar[r]\ar[d]^p&Y'\ar[d]\\&X\ar[r]_f&Y}$$
    	
    	(a)推(d):考虑如下交换图表,按照$f$是泛单的,得到$q:X\times_YX\to X$是单射,由于$q\circ\Delta_f=\mathrm{id}_X$,就得到$\Delta_f$是满射.
    	$$\xymatrix{X\ar@/^1pc/@{=}[drr]\ar@/_1pc/@{=}[ddr]\ar[dr]^{\Delta_f}&&\\&X\times_YX\ar[r]^q\ar[d]_p&X\ar[d]^f\\&X\ar[r]_f&Y}$$
    	
    	最后证明(d)推(a):我们知道$X\times_YX$中的元素可以表示为$(x_1,x_2,y,\mathfrak{p})$,其中$x_1,x_2\in X$,满足$f(x_1)=f(x_2)=y$,并且$\mathfrak{p}$是$\kappa(x_1)\otimes_{\kappa(y)}\kappa(x_2)$的素理想.由于$\Delta_f$是满射,所以这里总有$x_1=x_2$,并且$\kappa(x_1)\otimes_{\kappa(y)}\kappa(x_2)$只存在唯一的素理想,这就说明$f$是单射,并且$\kappa(x_1)$是$\kappa(y)$的纯不可分扩张【】,于是上一条说明$f$是泛单的.
    \end{proof}
    \item 推论.泛单态射总是分离的(事实上单射就总是分离的了),泛单态射的复合还是泛单的.
    \item 例如对$x\in X$,典范态射$\mathrm{Spec}\kappa(x)\to X$总是泛单射.
\end{enumerate}

泛开态射.
\begin{enumerate}
	\item 设$f:X\to Y$是概形之间的态射,如下命题互相等价:
	\begin{enumerate}
		\item $f$是泛开态射.
		\item 对任意局部有限表示态射$Y'\to Y$,有基变换$X_{Y'}\to Y'$是开映射.
		\item 对任意正整数$n$,有$\mathbb{A}^n\times X\to\mathbb{A}^n\times Y$是开映射.
	\end{enumerate}
    \begin{proof}
    	
    	(a)$\Rightarrow$(b)$\Rightarrow$(c)是平凡的,下面证明(c)$\Rightarrow$(a):反证,设有态射$g:T\to Y$使得基变换$X_T\to T$不是开映射,那么可取仿射开子集$V\subseteq Y$,$U\subseteq X$和$W\subseteq T$,满足$f(U)\subseteq V$和$g(W)\subseteq V$,并且投影态射$U\times_VW\to W$不是开映射.我们断言这可以推出$\mathbb{A}^n\times U\to\mathbb{A}^n\times V$不是开映射,于是与条件(c)矛盾.
    	
    	\qquad
    	
    	换句话讲设有环同态$A\to B$和$A\to A'$,使得$A'\to B'=A'\otimes_AB$诱导的态射不是开映射,要证明的是存在正整数$n$,使得$A[T_1,\cdots,T_n]\to B[T_1,\cdots,T_n]$诱导的态射也不是开映射.按照$A'\to B'$诱导的态射不是开映射,按照主开集构成拓扑基,于是存在$g\in B'$,使得$D(g)$在这个态射下的像不是$\mathrm{Spec}A'$的开集.记$g=\sum_{1\le i\le n}a_i'\otimes b_i$,其中$a_i'\in A'$和$b_i\in B$.记$h=\sum_{1\le i\le n}T_ib_i\in B[T_1,\cdots,T_n]$,我们断言$D(h)$在$A[T_1,\cdots,T_n]\to B[T_1,\cdots,T_n]$对应的态射下的像$U$不是开集.假设它是一个开集,那么$U$是拟紧开集(拟紧集的连续像是拟紧的).设$A[T_1,\cdots,T_n]\to A'$为把$T_i$映为$a_i'$,它诱导了$B$代数同态$B[T_1,\cdots,T_n]\to B'$,并且这把$h$映为$g$.考虑如下纤维积图表.我们解释过对于纤维积$X\times_SY$,如果$x\in X$满足存在$y\in Y$使得$x,y$在$S$中的像相同,当且仅当$x\in X$满足存在$z\in X\times_SY$,使得$z$在$X$下的像是$x$,于是这里有$\alpha^{-1}(\beta(D(h)))=\sigma(\tau^{-1}(D(h)))$,也即$\alpha^{-1}(U)=\sigma(D(g))$,但是从$U$是开集得到$\sigma(D(g))$是开集,这矛盾.
    	$$\xymatrix{\mathrm{Spec}B[T_1,\cdots,T_n]\ar[d]_{\beta}&&\mathrm{Spec}B'\ar[ll]_{\tau}\ar[d]^{\sigma}\\\mathrm{Spec}A[T_1,\cdots,T_n]&&\mathrm{Spec}A'\ar[ll]^{\alpha}}$$
    \end{proof}
	\item 我们解释过局部有限表示的平坦态射是泛开的,特别的光滑态射是泛开的.
    \item 设概形$Y$的底空间是离散空间,那么任意态射$f:X\to Y$都是泛开的.
    \begin{proof}
    	
    	我们要证明的是对任意态射$Y'\to Y$,基变换$f':X'=X\times_YY'\to Y'$都是开映射.问题是局部的,不妨设$Y$是单点空间.另外取既约闭子概型不改变拓扑,并且我们解释过$(X\times_SY)_{\mathrm{red}}=(X_{\mathrm{red}}\times_{S_{\mathrm{red}}}Y_{\mathrm{red}})_{\mathrm{red}}$,所以我们不妨设$X,Y$都是既约概形.特别的$Y$是一个域的素谱,记作$\mathrm{Spec}k$.于是$f$是平坦态射.取$X$的仿射开覆盖$\{U_i\}$,那么$X\times_YY'$被全体$U_i\times_YY'$覆盖,开映射是源端局部性质,所以不妨设$X=\mathrm{Spec}B$是仿射的.同理可设$Y'=\mathrm{Spec}A'$是仿射的.于是此时$X'=X\times_YY'=\mathrm{Spec}B\otimes_kA'$.我们要证明的是对任意$t\in B'$,有$V=f'(D(t))$是$\mathrm{Spec}A'$的开集.
    	
    	\qquad
    	
    	$B$是它的所有有限型$k$子代数$\{B_{\lambda}\}$的正向极限,域上代数都是平坦的,所以和正向极限可交换,于是$B'=\lim\limits_{\rightarrow}(B_{\lambda}\otimes_kA')$.可$t$落在某个$B_{\lambda}\otimes_kA'$中,记作$t_{\lambda}$.按照$\mathrm{Spec}B_{\lambda}\otimes_kA'\to\mathrm{Spec}A'$是局部有限表示的平坦态射,它是开映射,所以$D(t_{\lambda})\subseteq\mathrm{Spec}B_{\lambda}\otimes_kA'$在上述态射下的像$V_{\lambda}$是开集.最后只需证明$V=V_{\lambda}$.
    	
    	\qquad
    	
    	按照$D(t)$是$D(t_{\lambda})$在典范态射$\mathrm{Spec}B'\to\mathrm{Spec}B_{\lambda}\otimes_kA'$的原像,有$V\subseteq V_{\lambda}$.反过来任取$y\in V_{\lambda}$,要证明$y\in V$,等价于证明${f'}^{-1}(y)\cap D(t)\not=\emptyset$.如果记态射$g:{f'}^{-1}(y)\to {f'}_{\lambda}^{-1}(y)$和$W={f'}_{\lambda}^{-1}(y)\cap D(t_{\lambda})$.于是${f'}^{-1}(y)\cap D(t)=g^{-1}(W)$.下面断言$g$是支配的,导致非空开集在$g$下的原像一定非空:我们有${f'}^{-1}(y)=\mathrm{Spec}B'\otimes_{A'}\kappa(y)=\mathrm{Spec}B\otimes_k\kappa(y)$和${f'}_{\lambda}^{-1}(y)=\mathrm{Spec}B_{\lambda}\otimes_k\kappa(y)$,因为$\kappa(y)$在$k$上平坦,从$B_{\lambda}\to B$是单射得到$g$对应的环同态$B_{\lambda}\otimes_k\kappa(y)\to B\otimes_k\kappa(y)$是单射,我们解释过环之间的单同态诱导的态射是支配的.
        \end{proof}   
\end{enumerate}

泛同胚.
\begin{enumerate}
	\item 引理.设$f:X\to Y$是概形之间的态射,设$f$作为集合映射是到$Y$的闭子集的拓扑嵌入,那么$f$是仿射态射.
	\begin{proof}
		
		如果$y\not\in f(X)$,按照$f(X)$是闭集,存在$y$的仿射开邻域和$f(X)$不交,则它的原像是空集,是零环的素谱.如果$y\in f(X)$,存在唯一的$x\in X$满足$f(x)=y$,取$y$的仿射开邻域$V$,取$x$的仿射开邻域$U$使得$f(U)\subseteq V$.这里$f(U)$是$f(X)$的开子集.按照$f$是拓扑嵌入,取$h\in\Gamma(V,\mathscr{O}_Y)$使得$y\in D(h)$,并且满足$D(h)\cap f(X)\subseteq f(U)$.记$f^{\#}(h)$在$U$上的限制为$h'$,那么$D(h')\subseteq U$,并且有$D(h')=f^{-1}(D(h))$是仿射的,这说明$f$是仿射态射.
	\end{proof}
	\item 概形之间的态射是泛同胚当且仅当它是整态射,泛单射和满射.
	\begin{proof}
		
		先设$f$是泛同胚,按照引理有$f$是仿射态射,我们解释过整态射等价于泛闭的仿射态射,于是这里$f$是整态射.泛同胚自然是满射,这就解决了必要性.再设$f$是整态射,泛单射和满射.那么从$f$是整态射得到它是泛闭的,另外它也是泛双射(满射自动是泛满射的),我们知道双射的闭映射是同胚,所以$f$是泛同胚的.
	\end{proof}
    \item 例子.典范闭嵌入$X_{\mathrm{red}}\to X$是泛同胚,因为它是整态射(因为闭嵌入是仿射的和泛闭的),它是泛单射(闭嵌入是单射,闭嵌入的基变换还是闭嵌入),它是满射.
	\item 例子.设$X$是域$k$上的概形,设$k\subseteq K$是纯不可分扩张,那么投影态射$p:X_K=X\times_kK\to X$是泛同胚的.
	\begin{proof}
		
		这归结为证明它的下降$\mathrm{Spec}K\to\mathrm{Spec}k$是泛同胚的.但是态射$\mathrm{Spec}K\to\mathrm{Spec}k$是泛单射(因为它本身是单射,并且所有剩余域扩张都是纯不可分的),满射(满射自动是基变换下不变的),泛开的(终端是离散空间),于是它是泛同胚的.
	\end{proof}
\end{enumerate}
\subsection{态射的纤维}

态射的纤维.设$f:X\to Y$是概型的态射,任取点$y\in Y$,设它局部环的剩余类域是$\kappa(y)$,我们有典范的态射$\mathrm{Spec}\kappa(y)\to Y$把唯一点映射到$y$.这个典范映射和$f$的纤维积$X\times_Y\mathrm{Spec}(k(y))$称为$f$在点$y$处的(概型)纤维.如果$Y$是不可约的,它一般点的纤维称为一般纤维.
\begin{enumerate}
	\item 投影态射$X\times_Y\mathrm{Spec}(\kappa(y))\to X$是像集为$f^{-1}(y)$的拓扑嵌入.
	\begin{proof}
		
		这个问题是局部的,只需证明仿射情况.设$X=\mathrm{Spec}(B)$和$Y=\mathrm{Spec}(A)$,那么态射$f:X\to Y$由环同态$\varphi:A\to B$诱导.设$p$是点$y$对应的素理想,那么$X\times_Y\mathrm{Spec}(\kappa(y))=\mathrm{Spec}(B\otimes_AA_p/pA_p)=\mathrm{Spec}(B_p/pB_p)$.在这个同构下有对应:$B_p/pB_p$的素理想一一对应于$B_p$的包含$pB_p$的素理想,一一对应于$B$的素理想$q$,满足$q$和$\varphi(A-p)$无交,并且$\varphi(p)\subseteq q$,而这一一对应于$B$中的素理想$q$使得$\varphi^{-1}(q)=p$.这说明$X\times_Y\mathrm{Spec}(\kappa(y))\to X$是到$f^{-1}(p)$的双射.
		
		最后证明这是一个拓扑嵌入.这个概型态射可以分解为$\mathrm{Spec}(B_p/pB_p)\to\mathrm{Spec}(B_p)\to\mathrm{Spec}(B)$.这里第一个映射是闭嵌入,于是它是拓扑嵌入;而第二个映射是拓扑嵌入,于是它们的复合是拓扑嵌入.
	\end{proof}	
    \item 设$X,Y$是$S$概形,考虑纤维积$(X\times_SY,p,q)$,任取$x\in X$和$y\in Y$使得它们在$S$中的像相同,记作$s$,取两次态射的纤维得到同胚$\mathrm{Spec}\kappa(x)\otimes_{\kappa(s)}\kappa(y)\cong p^{-1}(x)\cap q^{-1}(y)$.这件事说明:
    \begin{enumerate}
    	\item 由于$\kappa(x)\otimes_{\kappa(s)}\kappa(y)$总不是零环,说明只要$x\in X$和$y\in Y$在$S$中的像相同,那么$p^{-1}(x)\cap q^{-1}(y)$总是非空的.
    	\item 设$x\in X$和$y\in Y$,那么存在$z\in X\times_SY$使得$p(z)=x$和$q(z)=y$,当且仅当$x,y$在$S$中的像相同.
    	\item $X\times_SY$中的点可以描述为四元对$(x,y,s,\mathfrak{p})$,其中$x\in X$和$y\in Y$,满足在$S$中的像都是$s$,而$\mathfrak{p}$是$\kappa(x)\otimes_{\kappa(s)}\kappa(y)$的素理想.
    \end{enumerate}
	\item 例子.考虑$\mathbb{A}_{\mathbb{Q}}^2$中的抛物线$\mathrm{Spec}\mathbb{Q}[x,y]/(y^2-x)$到$x$轴$\mathrm{Spec}\mathbb{Q}[x]$的投影映射,它由环同态$\mathbb{Q}[x]\to\mathbb{Q}[x,y]/(y^2-x)$,$x\mapsto y^2+(y^2-x)$诱导.
	\begin{itemize}
		\item $(x-1)$处的原像由两个点构成:
		\begin{align*}
		\mathrm{Spec}\mathbb{Q}[x,y]/(y^2-x)\otimes_{\mathbb{Q}[x]}\mathbb{Q}[x]/(x-1)&\cong\mathrm{Spec}\mathbb{Q}[x,y]/(y^2-x,x-1)\\&\cong\mathrm{Spec}\mathbb{Q}[y]/(y^2-1)\\&=\mathrm{Spec}\mathbb{Q}[y]/(y-1)\coprod\mathrm{Spec}\mathbb{Q}[y]/(y+1)
		\end{align*}
		\item $(x)$处的原像是一个非既约点.
		$$\mathrm{Spec}\mathbb{Q}[x,y]/(y^2-x,x)\cong\mathrm{Spec}\mathbb{Q}[y]/(y^2)$$
		\item $(x+1)$处的原像是一个既约点.
		$$\mathrm{Spec}\mathbb{Q}[x,y]/(y^2-x,x+1)\cong\mathrm{Spec}\mathbb{Q}[y]/(y^2+1)\cong\mathrm{Spec}\mathbb{Q}(i)$$
	\end{itemize}
    \item 例子.考虑$\mathbb{Z}\to\mathbb{Z}[T]$诱导的态射$f:\mathbb{A}_{\mathbb{Z}}^1\to\mathrm{Spec}\mathbb{Z}$.
    \begin{itemize}
    	\item 对$\mathbb{Z}$的零理想$(0)$,$\mathbb{Z}[T]$中的素理想$\mathfrak{p}$落在$(0)$的纤维里当且仅当$\mathfrak{p}\cap\mathbb{Z}=(0)$.于是$f^{-1}(0)$典范同构于$\mathrm{Spec}S^{-1}\mathbb{Z}[T]=\mathrm{Spec}\mathbb{Q}[T]$,其中$S=\mathbb{Z}-\{0\}$.
    	\item 对$\mathbb{Z}$的素理想$(p)$,其中$p$是素数,$\mathbb{Z}[T]$中的素理想$\mathfrak{p}$落在$(p)$的纤维里当且仅当$\mathfrak{p}\cap\mathbb{Z}=(p)$,当且仅当$p\in\mathfrak{p}$.于是$f^{-1}(p)$典范同构于$\mathrm{Spec}\mathbb{Z}[T]/(p)=\mathrm{Spec}\mathbb{F}_p[T]$.
    \end{itemize}
\end{enumerate}
\subsection{对角态射和图像态射}

一般范畴的情况.设范畴$\mathscr{C}$上任意两个终端相同的态射总存在纤维积.设$S$是一个对象,设$X,Y$是两个$S$对象,此即约定了态射$X\to S$和$Y\to S$作为结构态射.
\begin{itemize}
	\item 设$u:X\to S$是态射,它的对角态射定义为纤维积泛性质唯一确定的如下态射$\Delta_{X/S}$,它也会记作$\Delta_u$.
	$$\xymatrix{X\ar@/^1pc/[drr]^{\mathrm{id}_X}\ar@/_1pc/[ddr]_{\mathrm{id}_X}\ar[dr]^{\Delta_u}&&\\&X\times_SX\ar[r]\ar[d]&X\ar[d]\\&X\ar[r]&S}$$
	\item 设$f:X\to Y$是$S$态射,它的图像态射定义为纤维积泛性质唯一确定的如下态射$\Gamma_f$.
	$$\xymatrix{X\ar@/^1pc/[drr]^f\ar@/_1pc/[ddr]_{\mathrm{id}_X}\ar[dr]^{\Gamma_f}&&\\&X\times_SY\ar[r]\ar[d]&Y\ar[d]\\&X\ar[r]&S}$$
	\item 设$f,g:X\to Y$是两个态射,它们的核对是指一个对象$K$和一个态射$i:K\to X$,满足$f\circ i=g\circ i$,并且对任意对象$T$和任意态射$j:T\to X$使得$f\circ j=g\circ j$,都有$j$要经$i$分解.这也是纤维积的一种特殊情况.另外如果取$f=g=\left(X\to S\right)$,那么$\ker(f,g)\to X$就是对角态射.
	$$\xymatrix{K\ar[r]&X\ar@<0.5ex>[r]^f\ar@<-0.5ex>[r]_g&Y\\T\ar[u]\ar[ur]_j&&}$$
\end{itemize}
\begin{enumerate}
	\item 设$X$是$S$对象,我们有$\Delta_{X/S}=\Gamma_{\mathrm{id}_X}$.另外如果$Y$也是$S$对象,设$p:X\times_SY\to X$和$q:X\times_SY$是两个典范投影态射,那么$\Gamma_f$就是$\left(\ker(q,f\circ p)\to X\times_SY\right)$.
	\item 记号同上,如下图表的每个小方格的大方格都是笛卡尔的.
	$$\xymatrix{\ker(f,g)\ar[rr]\ar[d]&&X\ar[rr]^f\ar[d]^{\Gamma_f}&&Y\ar[d]^{\Delta_{Y/S}}\\X\ar[rr]_{\Gamma_g}&&X\times_SY\ar[rr]_{f\times\mathrm{id}_Y}&&Y\times_SY}$$
	\item 如果$s:S\to X$是$f$的截面,也即$f\circ s=\mathrm{id}_S$.那么有如下笛卡尔图表:
	$$\xymatrix{S\ar[rr]^s\ar[d]_s&&X\ar[d]^{\Gamma_{s\circ f}}\\X\ar[rr]_{\Delta_{X/S}}&&X\times_SX}$$
\end{enumerate}

我们接下来考虑概形范畴的情况.
\begin{enumerate}
	\item 设$f:X\to Y$是概形,设$\pi_1,\pi_2$是两个典范投影态射$X\times_YX\to X$.如果$s\in\Delta_f(X)$,那么有$\pi_1(s)=\pi_2(s)$.但是反过来如果$s\in X\times_YX$满足$\pi_1(s)=\pi_2(s)$,未必有$s\in\Delta_f(X)$.例如取典范嵌入$\mathbb{R}\to\mathbb{C}$诱导的$f:Y=\mathrm{Spec}\mathbb{C}\to X=\mathrm{Spec}\mathbb{R}$,由于$X$是单点集,$\Delta_f(X)$是单点集.但是此时$X\times_YX$中的两个点$s_1,s_2$都满足$\pi_1(s_1)=\pi_2(s_1)$和$\pi_1(s_2)=\pi_2(s_2)$.
	\item 仿射概形之间的态射$f:\mathrm{Spec}B\to\mathrm{Spec}A$对应了一个环同态$\varphi:A\to B$,那么$f$的对角态射$X\to X\times_YX$是被$B\otimes_AB\to B$,$b_1\otimes b_2\mapsto b_1b_2$诱导的,这是满同态;如果$S=\mathrm{Spec}R$,那么$f$的图像态射$X\to X\times_SY$是被$A\otimes_RB\to B$,$a\otimes b\mapsto\varphi(a)b$诱导的,这也是满同态.于是仿射情况下态射的对角态射和图像态射都是闭嵌入.
	\item 任意态射诱导的对角态射都是嵌入.
	$$\xymatrix{X\ar@/^1pc/[drr]^{\mathrm{id}_X}\ar@/_1pc/[ddr]_{\mathrm{id}_X}\ar[dr]^{\exists!\Delta_{Y/X}}&&\\&X\times_YX\ar[r]^p\ar[d]_q&X\ar[d]^{\varphi}\\&X\ar[r]_{\varphi}&Y}$$
	\begin{proof}
		
		设$Y$有仿射开覆盖$\{V_i\}$,$X$有仿射开覆盖$\{U_{ij}\}$,使得$\varphi$把$U_{ij}$映入$V_i$.记全体$p^{-1}(U_{ij})\cap q^{-1}(U_{ij})=U_{ij}\times_{V_i}U_{ij}$的并为开集$U$,那么$\Delta(X)\subseteq U$:因为任取$p\in X$,设它在某个$U_{ij}$中,那么$\Delta(p)\in p^{-1}(U_{ij})\cap q^{-1}(U_{ij})$.于是对角态射分解为$X\to U\to X\times_YX$,后者是开嵌入,于是我们只需验证$X\to U$是闭嵌入.有$\Delta^{-1}(U_{ij}\times_{V_i}U_{ij})=\Delta^{-1}(p^{-1}(U_{ij})\cap q^{-1}(U_{ij}))=U_{ij}$.于是归结为证明$\varphi$限制在仿射情况$U_{ij}\to U_{ij}\times_{V_i}U_{ij}$是闭嵌入.而这是上一条.
	\end{proof}
    \item 给定笛卡尔图表:
    $$\xymatrix{W\ar[rr]\ar[d]&&X\ar[d]\\Y\ar[rr]&&Z}$$
    
    那么有如下笛卡尔图表:
    $$\xymatrix{W\ar[rr]^{\Delta_{X/W}}\ar[d]&&W\times_XW\ar[d]\\Y\ar[rr]^{\Delta_{Z/Y}}&&Y\times_ZY}$$
    
    换句话讲,如果态射$\alpha$在某个纤维积中提升为态射$\beta$,那么对角态射$\Delta_{\beta}$是对角态射$\Delta_{\alpha}$在某个纤维积中的提升.
    \begin{proof}
    	
    	考虑如下交换图表:
    	$$\xymatrix{W\times_XW\ar[rr]\ar[d]&&W\ar[rr]\ar[d]&&Y\ar[d]\\W\ar[rr]&&X\ar[rr]&&Z}$$
    	
    	这两个小方块都是笛卡尔图表,我们解释过此时拼凑而成的大矩形图表是笛卡尔的,于是有$W\times_XW\cong Y\times_ZW$.再考虑如下态射族:
    	$$\xymatrix{W\ar[drr]&&&&Y\ar[dll]^{\mathrm{id}_Y}\\&&Y\ar[d]&&\\&&Z&&}$$
    	
    	按照magic图表的结论,得到如下笛卡尔图表,再用下$W\times_XW\cong Y\times_ZW$即可.
    	$$\xymatrix{W\ar[rr]\ar[d]&&Y\times_ZW\ar[d]\\T\ar[rr]&&T\times_ST}$$
    \end{proof}
    \item 给定$S$概型态射$\pi:X\to Y$.我们有如下magic图表,于是图像态射总是对角态射的提升,于是图像态射总是嵌入.另外我们解释过如果$f,g:X\to Y$是态射,那么典范态射$\ker(f,g)\to X$是图像态射$\Gamma_f$或者图像态射$\Gamma_g$的提升,于是这个典范态射总是嵌入.
    $$\xymatrix{X\ar[rr]^{\Gamma_{\pi}}\ar[d]_{\pi}&&X\times_SY\ar[d]\\Y\ar[rr]^{\Delta_{\pi}}&&Y\times_SY}$$
    \item 设$\Gamma\subseteq X\times_SY$是子概型,那么它是某个态射$f:X\to Y$的图像当且仅当典范投影态射$X\times_SY\to X$限制在$\Gamma$上是同构.并且此时$f$就是$q\circ(p\mid_{\Gamma})^{-1}$,其中$q:X\times_SY\to Y$是典范投影态射.
\end{enumerate}
\subsection{分离态射}

一个态射$\varphi:X\to Y$称为分离的,如果它的对角态射$\Delta_{Y/X}$是闭嵌入.一个$A$概型$X$称为分离的,如果它的结构态射$X\to\mathrm{Spec}A$是分离态射.
\begin{enumerate}
	\item 我们解释了一般态射诱导的对角态射总是嵌入,一个嵌入是闭嵌入当且仅当它的像集是终端的闭子集,于是一个态射是分离态射当且仅当它的对角态射的像集是闭子集.在拓扑上一个空间$X$是Hausdorff的当且仅当它的积空间$X\times X$的对角线是闭子集,于是分离态射是类似于Hausdorff条件的东西,这是分离态射的动机.
	\item 设$Y\to S$是态射,如下命题互相等价:
	\begin{enumerate}
		\item 它是分离态射,也即$\Delta_{Y/S}$是闭嵌入.
		\item 对任意$S$概形$X$和$S$态射$f:X\to Y$,有图像态射$\Gamma_f$是闭嵌入.
		\item 对任意$S$概形$X$和任意两个态射$f,g:X\to Y$,有$\ker(f,g)\subseteq X$是闭子概型.
	\end{enumerate}
	\begin{proof}
		
		我们解释过图像态射是对角态射的提升,而$\ker(f,g)$是$\Gamma_f$(也是$\Gamma_g$)的提升,于是(a)推(b)推(c)成立.最后选取$f=g=\left(Y\to S\right)$,那么典范态射$\ker(f,g)\to X$就是对角态射,于是(c)也能推出(a).
	\end{proof}
	\item 分离态射的形式性质.
	\begin{itemize}
		\item 分离态射的基变换是分离态射.我们解释过态射的基变换对应的对角态射也是基变换关系,再按照闭嵌入在基变换下不变得到这个结论.
		\item 分离态射的复合是分离态射.
		\begin{proof}
			
			设$X\to Y$和$Y\to Z$都是分离态射.考虑如下图表,其中右侧小方块是magic图表.并且上方两个态射的复合恰好是复合态射$X\to Y\to Z$的对角态射.按照条件这里$\Delta_{Y/X}$和$\Delta_{Z/Y}$是闭嵌入,按照基变换保闭嵌入,得到这里$\mu$也是闭嵌入,闭嵌入的复合是闭嵌入,得证.
			$$\xymatrix{X\ar[rr]&&X\times_YX\ar[rr]\ar[d]&&X\times_ZX\ar[d]\\&&Y\ar[rr]&&Y\times_ZY}$$
		\end{proof}
	    \item 分离是终端局部性质.换句话讲,一个态射$\pi:X\to Y$是分离的当且仅当对$Y$的某个开覆盖$\{V_i\}$,有$\pi^{-1}(U_i)\to U_i$是分离的.
	\end{itemize}
	\item 分离态射的例子.
	\begin{itemize}
		\item 单态射都是分离态射.于是特别的嵌入,闭嵌入,开嵌入都是分离态射.
		\begin{proof}
			
			设$\varphi:X\to Y$是单态射,我们来证明$(X,\mathrm{id}_X,\mathrm{id}_X)$就是纤维积$X\times_YX$,这件事就说明此时对角态射$\Delta_{Y/X}$就是同构.考虑如下图表,其中$f,g:T\to X$满足$\varphi\circ f=\varphi\circ g$,按照$\varphi$是单态射,得到$f=g$,于是存在唯一的态射$h=f=g$使得图表交换.
			$$\xymatrix{T\ar@/^1pc/[drr]^{f}\ar@/_1pc/[ddr]_{g}\ar[dr]^{\exists!h}&&\\&X\ar[r]^{\mathrm{id}_X}\ar[d]_{\mathrm{id}_X}&X\ar[d]^{\varphi}\\&X\ar[r]_{\varphi}&Y}$$
		\end{proof}
		 \item 仿射概型之间的态射总是分离的.这是因为它的对角态射由$A\otimes_BA\to A$,$a_1\otimes a_2\mapsto a_1a_2$诱导,而这是满同态.按照分离是终端仿射局部性质,得到仿射态射都是分离的.
		 \item 射影空间的结构态射$\mathbb{P}_A^n\to\mathrm{Spec}A$是分离的.
		 \begin{proof}
		 	
		 	用$U_i\times_AU_j$覆盖$\mathbb{P}_A^n\times_A\mathbb{P}_A^n$,其中$U_i=D(x_i),0\le i\le n$.我们需要说明的是对角态射限制在每个$\Delta^{-1}(U_i\times_AU_j)=U_i\cap U_j$上的像集是$U_i\times_AU_j$中的闭子集.那么对于$i=j$的情况,我们在证明一般对角态射是局部闭嵌入的时候已经证明了.于是这里设$i\not=j$.此时$U_i\times_AU_j\cong\mathrm{Spec}A[x_{0/i},\cdots,x_{n/i},y_{0/j},\cdots,y_{n/j}]/(x_{i/i}-1,y_{j/j}-1)$.于是$U_i\cap U_j\to U_i\times_AU_j$被环同态$A[x_{0/i},\cdots,x_{n/i},y_{0/j},\cdots,y_{n/j}]/(x_{i/i}-1,y_{j/j}-1)\to A[x_{0/i},\cdots,x_{n/i},x_{j/i}^{-1}]/(x_{i/i}-1)$,$x_{k/i}\mapsto x_{k/i},y_{k/j}\mapsto x_{k/i}/x_{j/i}$所诱导,而这是满同态(其中$x_{j/i}^{-1}$是$y_{i/j}$的像).
		 \end{proof}
	     \item 拟射影$A$概型都是分离的.我们定义过拟射影$A$概型是射影$A$概型的拟紧开子概型.另外我们解释过射影$A$概型就是$A$射影空间的闭子概型.于是拟射影$A$概型的结构态射是一个嵌入复合射影空间的结构态射,我们已经证明了它们都是分离的,于是复合是分离的.
	\end{itemize}
    \item 分离态射是拟分离态射.我们之前定义过拟分离态射是指对角态射拟紧的态射.按照闭嵌入是仿射的,仿射是拟紧的,说明分离态射都是拟分离态射.
    \item 设$A$是环,$X$是分离$A$概型,$X$的任意两个仿射开子集的交是仿射的.
    \begin{proof}
    	
    	任取$X$的两个仿射开子集$U,V$,那么$U\times_AV=p^{-1}(U)\cap q^{-1}(V)$是仿射的.按照对角态射$\Delta$是闭嵌入,就得到$\Delta^{-1}(p^{-1}(U)\cap q^{-1}(V))=U\cap V$是仿射的.
    \end{proof}
    \item 一个概形$X$称为分离的,如果唯一的态射$f:X\to\mathrm{Spec}\mathbb{Z}$是分离态射.等价于讲$X\to X\times_{\mathbb{Z}}X$是闭嵌入.它有如下等价描述:
    \begin{enumerate}
    	\item 对$X$的任意两个仿射开子集$U,V$,有$U\cap V$总是仿射的,并且典范同态$\mathscr{O}_X(U)\otimes_{\mathbb{Z}}\mathscr{O}_X(V)\to\mathscr{O}_X(U\cap V)$是满射.
    	\item 存在$X$的仿射开覆盖$\{U_i\mid i\in I\}$,使得任意$U_i\cap U_j$是仿射的,并且典范同态$\mathscr{O}_X(U_i)\otimes_{\mathbb{Z}}\mathscr{O}_X(U_j)\to\mathscr{O}_X(U_i\cap U_j)$总是满射.
    \end{enumerate}
    \begin{proof}
    	
    	如果$U,V$是$X$的仿射开子集,那么$U\times_{\mathbb{Z}}V$是仿射概形.对角态射满足$\Delta^{-1}(U\times_{\mathbb{Z}}V)=U\cap V$,限制态射$\Delta\mid_{U\cap V}:U\cap V=\Delta^{-1}(U\times_{\mathbb{Z}}V)\to U\times_{\mathbb{Z}}V$作为终端为仿射概形的态射,对应的环同态就是$\mathscr{O}_X(U)\otimes_{\mathbb{Z}}\mathscr{O}_X(V)\to\mathscr{O}_X(U\cap V)$.
    	
    	\qquad
    	
    	如果$f$是分离态射,那么对角态射$\Delta$是闭嵌入,由于闭嵌入是仿射终端局部性质,就有$\Delta\mid_{U\cap V}$也是闭嵌入,于是对应的环同态是满射,这得到(a).从(a)推(b)是平凡的,下面假设(b)成立,因为闭嵌入是仿射终端局部性质,得到$\Delta$是闭嵌入.
    \end{proof}
    \item 例子.取$X_1=X_2=\mathrm{Spec}\mathbb{Z}$,取素数$p$,取$X_{12}=X_1-\{p\}$和$X_{21}=X_2-\{p\}$,取恒等态射$X_{12}\cong X_{21}$,这使得$X_1$和$X_2$粘合为带两个点$(p)$的$\mathbb{Z}$,这个空间记作$X$,那么$\mathscr{O}_X(X_1\cap X_2)=\mathbb{Z}[1/p]$,而典范映射$\mathscr{O}_X(X_1)\otimes_{\mathbb{Z}}\mathscr{O}_X(X_2)\to\mathscr{O}_X(X_1\cap X_2)$的像集是$\mathbb{Z}$,这不是满射,于是上一条说明我们粘合的$X$不是分离概形.
    \item 设$X$是既约$S$概形,设$Y$是分离$S$概形,设$f,g:X\to Y$是两个$S$态射,如果它们在$X$的某个稠密开集$U$上相同,那么在整个$X$上相同.
    \begin{proof}
    	
    	先证明$f,g$在集合层面是相同的.把结构态射$Y\to S$的对角态射记作$\Delta$.记$f,g:X\to Y$诱导的$X\to Y\times_SY$为$h=(f,g)$.那么$\Delta\circ f=(f,f)$.于是在开集$U$上态射$\Delta\circ f$和$h$是一致的.于是对$x\in U$有$\pi_1(h(x))=\pi_2(h(x))$,于是$U\subseteq h^{-1}(\Delta(Y))$.但是按照条件这里$\Delta(Y)$是闭集,于是得到$X=h^{-1}(\Delta(Y))$,这说明$f(x)=g(x),\forall x\in X$.
    	
    	\qquad
    	
    	再证明$f,g$作为态射是相同的.为此我们任取$Y$的仿射开子集$V=\mathrm{Spec}B$和$f^{-1}(V)=g^{-1}(V)$中的仿射开子集$W=\mathrm{Spec}A$,然后证明$f,g$限制为$W\to V$是相同的即可,换句话讲归结为仿射情况.设$f,g$对应的环同态分别为$\varphi,\psi:B\to A$.设$b\in B$,记$a=\varphi(b)-\psi(b)$,那么$a\mid_U=0$,于是$U\subseteq V(a)$,但是后者是闭集,$U$是稠密的,导致$V(a)=\mathrm{Spec}A$,于是$a$是幂零元,再从$A$是既约的得到$a=0$.这说明$\varphi=\psi$,于是诱导的态射$f=g$.
    \end{proof}
    \item 上一条中$Y$是分离的$S$概形和$X$是既约的,去掉哪个都会导致结论不成立.
    \begin{itemize}
    	\item 取$X=\mathbb{A}_k^1$,取$Y$是$k$上带两个原点的仿射线,取$f,g:X\to Y$是把$X$中原点映射为$Y$中两个不同原点,其它位置恒等的态射.那么$X$是既约的,$k$概形$Y$不是分离的.$f,g$在稠密开子集$X-\{(T)\}$上一致,但是在原点$(T)$上取值不一致.
    	\item 取$X=Y=\mathrm{Spec}k[X,Y]/(X^2,XY)$,它由$y$轴上的点和唯一的一般点$\eta=(x)$构成.取态射$f:X\to Y$是恒等态射,取$g:X\to Y$是$y\mapsto 0$的同态诱导的态射,那么$g$把$X$上所有点映射到一般点$\eta$,于是$f,g$在稠密开集【】
    \end{itemize}
    \item 另外这个结论告诉我们既约分离概形范畴上支配态射总是满态射.这个结论对应于拓扑上Hausdorff空间范畴上的支配映射就是满态射.
    \item 一个态射$f:X\to Y$是分离态射当且仅当$f_{\mathrm{red}}:X_{\mathrm{red}}\to Y_{\mathrm{red}}$是分离的.
    \begin{proof}
    	
    	设$i:X_{\mathrm{red}}\to X$是典范闭嵌入,这是一个满射嵌入,于是它是泛同胚.我们知道由于$Y_{\mathrm{red}}\to Y$是单态射,有典范态射$X_{\mathrm{red}}\times_{Y_{\mathrm{red}}}X_{\mathrm{red}}\to X_{\mathrm{red}}\times_YX_{\mathrm{red}}$是同构.于是我们有如下交换图表,其中$i$和$i\times_Yi$都是同胚,这就说明$\Delta_f(X)$是闭集当且仅当$\Delta_{f_{\mathrm{red}}}$是闭集(因为对角态射一定是嵌入,它是闭嵌入当且仅当像集是闭集),也即$f$是分离的当且仅当$f_{\mathrm{red}}$是分离的.
    	$$\xymatrix{X_{\mathrm{red}}\ar[rr]^{\Delta_{f_{\mathrm{red}}}}\ar[d]_i&&X_{\mathrm{red}}\times_YX_{\mathrm{red}}\ar[d]^{i\times_Yi}\\X\ar[rr]^{\Delta_f}&&X\times_YX}$$
    \end{proof}
\end{enumerate}
\subsection{紧合态射}

紧合态射.一个态射称为紧合(proper)态射,如果它是分离,有限型,并且泛闭的(即任意基变换都是闭映射).
\begin{enumerate}
	\item 这个概念的动机是这样一个拓扑性质:称连续映射是proper映射,如果任意紧集的原像是紧集.那么局部紧Hausdorff空间之间的连续映射是proper映射当且仅当它是泛闭的.
	\item 紧合态射的一些形式性质.
	\begin{itemize}
		\item 紧合态射在基变换下不变.我们已经证明过分离和有限型态射都是基变换下不变的,而泛闭性自动是基变换下不变的.
		\item 紧合态射的复合是紧合态射.我们解释过分离,有限型和泛闭都是复合下不变的性质.
		\item 紧合态射是终端局部性质.
	\end{itemize}
	\item 紧合态射的例子.
	\begin{itemize}
		\item 闭嵌入都是紧合态射.它是分离的(按照仿射态射都是分离的,或者单态射都是分离的),它是有限型态射,它是泛闭的(闭嵌入的基变换都是闭嵌入,闭嵌入总是闭映射).
		\item 更一般的,有限态射都是紧合态射.事实上,按照定义有限态射都是仿射的,而仿射态射都是分离的.有限态射必然都是有限型态射.最后我们解释过整态射都是闭映射,于是有限态射都是闭映射,并且有限态射的基变换都是有限态射,于是有限态射是泛闭的.
		\item 后文会证明射影态射都是紧合态射.
	    \item 一个反例.设$k$是域,结构态射$\mathbb{A}_k^n\to\mathrm{Spec}k$是分离,有限型,和闭的,但它不是泛闭的(于是它不是紧合态射):考虑投影态射$q:\mathbb{A}_k^n\times_k\mathbb{A}_k^1\to\mathbb{A}_k^1$,记坐标$\mathbb{A}_k^n=\mathrm{Spec}k[T_1,\cdots,T_n]$和$\mathbb{A}_k^1=\mathrm{Spec}k[U]$.考虑闭子集$Z=V(UT_1-1)\subseteq\mathbb{A}_k^n\times_k\mathbb{A}_k^1$,那么$q(Z)=\mathbb{A}_k^1-\{0\}$不是闭集.另外我们会证明事实上如果$f:X\to Y$已经是有限型和分离的态射,那么它是紧合态射当且仅当$f_{(\mathbb{A}_Y^n)}:X\times_Y\mathbb{A}_Y^n\to\mathbb{A}_Y^n$是闭映射对任意$n$成立.
	\end{itemize}
    \item 设$f$是有限型态射,那么$f$是紧合态射当且仅当$f_{\mathrm{red}}$是紧合态射.
    \begin{proof}
    	
    	因为闭嵌入总是有限型态射,说明$f$是有限型态射推出$f_{\mathrm{red}}$是有限型态射.我们解释过$f$是分离态射当且仅当$f_{\mathrm{red}}$是分离的.最后泛闭只依赖于底空间上的拓扑,所以$f$是泛闭的当且仅当$f_{\mathrm{red}}$是泛闭的.
    \end{proof}
    \item 一个态射是有限态射当且仅当它是紧合态射和仿射态射.
    \begin{proof}
    	
    	我们解释过有限态射是紧合态射也是仿射的.反过来假设态射是紧合态射和仿射态射,按照紧合态射是仿射终端局部性质,归结为证明仿射情况,换句话讲如果$Y=\mathrm{Spec}B$是环$A$上的真仿射概形,需要证明$B$在$A$上有限.
    	
    	按照定义$B$是有限型$A$代数.所以问题归结为证明$B$在$A$上整.先设$B$在$A$上经一个元代数生成.此即存在满的代数同态$\varphi:A[T]\to B$.只需验证$\varphi(T)$在$A$上整,就得到$B$在$A$上整.把$\mathrm{Spec}A[T]$视为$\mathrm{Proj}A[T_1,T_2]$的开子概型$D_+(T_2)$,那么有$T=T_1/T_2$.考虑如下映射的复合,记作$f$,这里$f$和分离态射$\mathrm{Proj}A[T_1,T_2]\to\mathrm{Spec}A$的复合是真态射,我们解释过这个条件可以推出$f$是真态射.
    	$$\xymatrix{Y=\mathrm{Spec}B\ar[r]^i&\mathrm{Spec}A[T]\ar[r]&\mathrm{Proj}A[T_1,T_2]}$$
    	
    	于是$f$是闭映射,特别的有$f(Y)$是$\mathrm{Proj}A[T_1,T_2]$的闭子集,于是存在齐次理想$J\subseteq A[T_1,T_2]$使得$f(Y)=V_+(J)$.按照$f(Y)$的像实际落在$D_+(T_2)$中,于是有$V_+(T_2)\cap V_+(J)=\emptyset$,导致$(T_1,T_2)\subseteq\sqrt{J+(T_2)}$.于是存在正整数$n$以及齐次多项式$P(T_1,T_2)\in J$和齐次多项式$Q(T_1,T_2)\in A[T_1,T_2]$使得$P(T_1,T_2)=T_1^n+T_2Q(T_1,T_2)$.那么这里$P$的次数也必须是$n$次的,$Q$的次数是$n-1$次.按$T=T_1/T_2$代换这个等式,得到$P(T,1)=T^n+Q(T,1)$.其中$Q(T,1)$关于$T$的次数不超过$n-1$.
    	
    	我们有$i(Y)=V_+(J)\cap D_+(T_2)=V(J_{(T_2)})$,其中$J_{(T_2)}$是$A[T]$的理想.任取$q\in Y=\mathrm{Spec}B$,有$i(q)\in V(J_{(T_2)})$,也即$J_{(T_2)}\subseteq i(q)=\varphi^{-1}(q)$.于是$\varphi(J_{(T_2)})\subseteq q$对任意素理想$q\in\mathrm{Spec}B$成立,于是$\varphi(J_{(T_2)})\subseteq\mathrm{nil}(B)$.于是特别的$\varphi(P(T,1))$是幂零的,因为$P(T_1,T_2)\in J$导致$P(T,1)\in J_{(T_2)}$.于是在$P(T,1)=T^n+Q(T,1)$两侧取足够大的次幂,再把$\varphi$作用其上,就得到$\varphi(T)$是$A$上的整元.这解决了$B$在$A$上经一个元代数生成的情况.
    	
    	对于一般情况,设$B=A[b_1,\cdots,b_m]$,记$B_1=A[b_1,\cdots,b_{m-1}]\subseteq B$,条件要求$\mathrm{Spec}B\to\mathrm{Spec}B_1\to\mathrm{Spec}A$是紧合态射,而仿射概形之间的态射$\mathrm{Spec}B_1\to\mathrm{Spec}A$总是分离的,所以有$\mathrm{Spec}B\to\mathrm{Spec}B_1$是紧合态射.按照这里$B$在$B_1$上是单个元代数生成的,所以我们之前证明的情况就说明$B$在$B_1$上是整的,但是对$m$做归纳,可假设$B_1$在$A$上整,于是$B$在$A$上整,这就完成归纳.
    \end{proof}
    \item 设$X$是环$A$上的紧合概形,那么$\mathscr{O}_X(X)$在$A$上整.
    \begin{proof}
    	
    	我们知道一个环同态$\varphi:A\to B$是整的当且仅当$\varphi_{\mathrm{red}}:A/\mathrm{nil}(A)\to B/\mathrm{nil}(B)$是整的.所以不妨设$X$是既约概形.任取$h\in\mathscr{O}_X(X)$,定义$\varphi:A[T]\to\mathscr{O}_X(X)$为$T\mapsto h$,这诱导了紧合态射$f:X\to\mathrm{Spec}A[T]$.于是$f(X)=V(I)$,其中$I$可以取为一个根理想.进而$f$分解为满射$X\to\mathrm{Spec}A[T]/I$复合闭嵌入.我们知道一般的如果$X\to Y\to Z$是紧合的,$X\to Y$是满射,$Y\to Z$是可分有限型的,那么$Y\to Z$是紧合的.于是这里$\mathrm{Spec}A[T]/I\to\mathrm{Spec}A$是紧合的.特别的$A[T]/I$在$A$上整,迫使$I$包含一个首一非常值多项式,于是$h$在$A$上整.
    \end{proof}
    \item 设$X/k$是域上的紧合概形.
    \begin{enumerate}[(1)]
    	\item 设$X/k$是既约连通概形,那么$\mathscr{O}_X(X)$是域,并且是$k$的有限扩张.
    	\item 设$X/k$是既约几何连通概形,那么$\mathscr{O}_X(X)$是$k$的有限纯不可分扩张.
    	\item 设$X/k$是几何既约几何连通概形,那么$\mathscr{O}_X(X)=k$.
    \end{enumerate}
    \begin{proof}
    	
    	(1):我们知道这里$\mathscr{O}_X(X)$是有限维既约$k$代数,所以它可以表示为若干域的积,但是连通性导致只能出现一个域,所以$\mathscr{O}_X(X)$是$k$的有限域扩张.
    	
    	\qquad
    	
    	(2):取$k$的一个可分闭包$k^s$.我们有$\mathscr{O}(X_{k^s})=K\otimes_kk^s$.设$X$是几何连通的,那么$X_{k^s}$是连通和既约的(既约因为既约概形做可分域扩张仍然是既约的),于是$K\otimes_kk^s$是域.但是这个域包含了子环$L\otimes_kL$,其中$L=K\cap k^s$,这里$L/k$是有限可分扩张,那么$L\otimes_kL$是整环只可能有$L=k$.于是$K$在$k=K\cap k^s$上纯不可分.
    	
    	\qquad
    	
    	(3):如果$X$是几何连通和几何既约的,那么对$k$的代数闭包$\overline{k}$就有$\mathscr{O}(X_{\overline{k}})=K\otimes_k\overline{k}$是既约和连通的,从而是域.它是$\overline{k}$的有限扩张,迫使它恰好为$\overline{k}$,进而有$K=k$.
    \end{proof}
\end{enumerate}
\subsection{分离和泛闭的赋值准则}
\begin{enumerate}
	\item 我们先用一个例子解释赋值准则在做什么.设$X$是域$k$上带两个原点的仿射线,考虑如下典范包含态射$\mathbb{A}_k^1-\{0\}\to X$和$\mathbb{A}_k^1-\{0\}\to\mathbb{A}_k^1$,那么存在两种提升映射$\mathbb{A}_k^1\to X$,即把$\mathbb{A}_k^1$的原点映射为$X$上两个原点的任一个.
	$$\xymatrix{\mathbb{A}_k^1-\{0\}\ar[rr]\ar[d]&&X\\\mathbb{A}_k^1\ar[urr]&&}$$
	
	我们知道分离条件类似于拓扑上的Hausdorff条件,所以这里提升不唯一理应说明$X$不是分离的.于是一般的,如果$C$是一条光滑曲线,设$x\in C$是闭点,如果$X$是分离概形,那么对任意态射$f:C-\{x\}\to X$,理应至多存在唯一的提升$C\to X$:
	$$\xymatrix{C-\{x\}\ar[rr]^f\ar[d]&&X\\C\ar[urr]&&}$$
	
	如果我们把$C$替换为$x\in C$的所有开邻域,取极限就是$\mathrm{Spec}\mathscr{O}_{C,x}$,这是一个DVR.如果把$C-\{x\}$替换为$x\in C$的所有去心开邻域,取极限就是$\mathrm{Spec}\mathrm{Frac}(\mathscr{O}_{C,x})$,那么$X$的分离的就要满足对任意离散赋值环$A$和它的商域$K$,态射$f:\mathrm{Spec}K\to X$至多存在唯一的提升$\mathrm{Spec}A\to X$:
	$$\xymatrix{\mathrm{Spec}K\ar[rr]^f\ar[d]&&X\\\mathrm{Spec}A\ar[urr]&&}$$
	\item 引理1.设$X$是既约概形,设$f:X\to Y$是态射,$Z\to Y$是闭嵌入(也即把$Z$视为$Y$的闭子概型),如果$f(T)$落在$Z$中,那么存在唯一的虚线态射使得如下图表交换:
	$$\xymatrix{&&Z\ar[d]\\X\ar[rr]_f\ar@{-->}[urr]&&Y}$$
	\begin{proof}
		
		取$Y$的仿射开子集$V$,取$f^{-1}(V)$的仿射开子集$U$,只要我们证明此时虚线态射是唯一的,那么在不同仿射开子集的交上的虚线态射也必然是唯一的,这就自动验证了这些局部上的虚线态射可以粘合为一个整体态射.所以问题归结为$X=\mathrm{Spec}B$和$Y=\mathrm{Spec}A$都是仿射的情况.此时可记$Z=\mathrm{Spec}A/I$.那么条件$f(X)\subseteq Z$也即$f(\mathrm{Spec}B)\subseteq V(I)$,于是有$\overline{f(\mathrm{Spec}B)}\subseteq V(I)$,如果记$f$被环同态$\varphi:A\to B$诱导,那么有$V(\ker\varphi)=\overline{f(\mathrm{Spec}B)}\subseteq V(I)$,于是有$I\subseteq\sqrt{\ker\varphi}$.但是由于$B$是既约环,有单射$A/\ker\varphi\subseteq B$,导致$A/\ker\varphi$也是既约环,于是$\ker\varphi$是根理想,于是$I\subseteq\sqrt{\ker\varphi}=\ker\varphi$.那么使得如下图表交换的环同态就只有$\varphi$诱导的$\widetilde{\varphi}:A/I\to B$.
		$$\xymatrix{&&A/I\ar@{-->}[dll]\\B&&A\ar[ll]^{\varphi}\ar[u]}$$
	\end{proof}
	\item 引理2.设$f:X\to Y$是概形之间的拟紧态射,那么$f$拓扑上是闭映射当且仅当对任意$x\in X$都有$f(\overline{\{x\}})=\overline{\{f(x)\}}$.
	\begin{proof}
		
		必要性是直接的,任取$x\in X$有$\{f(x)\}\subseteq f(\overline{\{x\}})\subseteq\overline{f(x)}$,中间项是闭集,于是有$\overline{\{f(x)\}}\subseteq f(\overline{\{x\}})\subseteq\overline{\{f(x)\}}$,也即$f(\overline{\{x\}})=\overline{\{f(x)\}}$.
		
		\qquad
		
		充分性,任取$X$的闭子集$X'$,记$Y'=\overline{f(X')}$,赋予$X'$和$Y'$既约闭子概型结构,那么$f:X\to Y$诱导的态射$f':X'\to Y'$是拟紧和支配的态射.这里所有的条件和要证的结论都是拓扑性质,所以不妨设$f:X\to Y$本身是既约概型之间的拟紧和支配的态射,要证明的是$f(X)=Y$.
		
		\qquad
		
		任取$y\in Y$,设一般点$y'\in Y$满足$y\in\overline{\{y'\}}$,倘若我们证明了$y'\in f(X)$,记$y'=f(x')$,那么条件导致$y\in\overline{\{y'\}}=\overline{f(\{x'\})}=f(\overline{\{x'\}})$得到$y\in f(X)$,所以问题归结为对一般点$y'\in Y$有$f^{-1}(y')\not=\emptyset$.下面取$y$的仿射开邻域$U$,那么开集保一般化得到$y'\in U$,另外$y,y'\in U\cap\overline{\{y'\}}$,所以可以用$U\cap\overline{\{y'\}}$替换$Y$,于是我们不妨设$Y=\mathrm{Spec}A$是一个整环诱导的(既约+不可约等价于整)仿射概形.
		
		\qquad
		
		接下来按照$f$是拟紧态射,有$X$是有限个仿射开子集的并,记$X=\cup_{i=1}^nU_i$,其中$U_i$是仿射的.于是有$Y=\cup_{i=1}^n\overline{f(U_i)}$.但是按照$Y$是不可约的,所以存在某个指标$i$使得$Y=\overline{f(U_i)}$.所以归结为设$X=\mathrm{Spec}B$是仿射的.整理下,我们归结为设$f:\mathrm{Spec}B\to\mathrm{Spec}A$是支配态射,并且$A$是整环,于是诱导这个态射的同态$\varphi:A\to B$是单射.设$A$的商域为$K$,要证$A$的零理想在$f$下的纤维非空,等价于证纤维环$B\otimes_AK$不是零环,由于整环的商域是平坦的,所以$K=A\otimes_AK\to B\otimes_AK$是单射,于是$B\otimes_AK$不是零环.
	\end{proof}
    \item 引理2'.设$f:X\to Y$是概形之间的拟紧态射,那么$f(X)$是$Y$的闭子集当且仅当$f(X)$保特殊化,换句话讲如果$y\in f(X)$并且$z\in\overline{\{y\}}$,那么$z\in f(X)$.
    \begin{proof}
    	
    	必要性是平凡的,充分性的证明和上一条是一样的,首先依旧可以归结为设$f$是既约概形之间的拟紧和支配的态射,目标就是证明对每个$y\in Y$的纤维非空.在上一条中借助$f(\overline{\{x\}})=\overline{\{f(x)\}}$把问题归结为证明一般点的纤维非空.这里借助$f(X)$保特殊化,任取$y\in Y$,设$y$落在不可约分支$\overline{\{y'\}}$中,其中$y'$是一般点,倘若我们证明$y'$的纤维非空,那么$y$是$y'\in f(X)$的特殊化,也会得到$y\in f(X)$.归结为证明一般点的纤维非空后依旧约化到$Y$是整环诱导的仿射概形,用拟紧条件和$Y$不可约再约化为$X$是仿射概形,借助$\mathrm{Frac}(A)=K$是平坦$A$代数,得到纤维环$K\otimes_AB$不是零环,所以$f^{-1}(y')$非空.
    \end{proof}
	\item 引理3.设$X$是概形,设$x,x'\in X$,满足$x'$是$x$的特殊化,也即$x'\in\overline{\{x\}}$,那么存在一个赋值环$B$,和一个态射$f:\mathrm{Spec}B\to X$,它把$\mathrm{Spec}B$的唯一闭点映射到$x'$,把$\mathrm{Spec}B$的一般点$\xi$映射到$x$.如果额外的还有$X$是局部诺特概形,那么这里$B$额外的可以取为离散赋值环.
	\begin{proof}
		
		如果取$x'$的仿射开邻域$U$,那么$x'$的一般化$x$一定也落在$U$中,于是不妨设$X=\mathrm{Spec}A$是仿射的.设$\mathfrak{p}\subseteq\mathfrak{p}'$分别对应于点$x,x'$.把$A$替换为环$(A/\mathfrak{p})_{\mathfrak{p}'/\mathfrak{p}}$,几何上看它是$X$的闭子集的开子集,所以归结为设$\mathfrak{p}=(0)$和$\mathfrak{p}'$是极大理想,也即$(A,\mathfrak{p}')$是局部整环(换句话讲设这个闭子集的开子集为$Y$,那么$Y$包含了点$x,x'$,只要我们证明存在赋值环$B$对$Y$成立,再复合上包含态射$Y\subseteq X$就得到结论).设$A$的商域为$K$,那么包含$K$的某个局部子环的全部局部子环在控制偏序下的极大元按照Zorn引理总是存在的,这里称$K$的局部子环$(A,\mathfrak{m})$被$(B,\mathfrak{n})$控制是指$A\subseteq B$,并且$\mathfrak{n}\cap A=\mathfrak{m}$,并且这样的极大元总是$K$的赋值环.特别的这里局部整环$(A,\mathfrak{p}')$总可以嵌入到$K$的某个赋值环$(B,\mathfrak{n})$中,并且这里$\mathfrak{n}\cap A=\mathfrak{p}'$,换句话讲这个包含映射诱导的态射$\mathrm{Spec}B\to\mathrm{Spec}A$把$B$的唯一闭点映射为$\mathrm{Spec}A$的$\mathfrak{p}'$,并且按照这是整环之间的嵌入得到$\mathrm{Spec}B$的一般点被映射到$\mathrm{Spec}A$的一般点.
		
		\qquad
		
		如果$X$是局部诺特的,一样的理由可以不妨设$X=\mathrm{Spec}A$是诺特局部整环的仿射概形.记$A$的极大理想$\mathfrak{m}$被$n$个非零元$\{x_1,\cdots,x_n\}$生成,取$C=A[x_2/x_1,\cdots,x_n/x_1]\subseteq K=\mathrm{Frac}(A)$,那么有$\mathfrak{m}C=x_1C$是$C$的主理想.设$\mathfrak{n}$是$C$的包含$x_1C$的极小素理想,按照Krull主理想定理(结合$C$是整环),有$C_{\mathfrak{n}}$是1维诺特局部整环,于是同态$A\subseteq C_{\mathfrak{n}}$满足$\mathfrak{n}C_{\mathfrak{n}}\cap A=\mathfrak{n}\cap A\supseteq\mathfrak{m}C\cap A\supseteq\mathfrak{m}$,但是$\mathfrak{m}$已经是$A$的极大理想,于是只能有$\mathfrak{n}C_{\mathfrak{n}}\cap A=\mathfrak{m}$,换句话讲$A\subseteq C_{\mathfrak{n}}$诱导的$\mathrm{Spec}C_{\mathfrak{n}}\to\mathrm{Spec}A$把$C_{\mathfrak{n}}$的唯一闭点映射为$\mathrm{Spec}A$的唯一闭点,并且把一般点映射为一般点.但是这里$C_{\mathfrak{n}}$未必是DVR,取它在$K$中的整闭包是$B$,那么$B$是一维正规诺特局部整环,也即DVR,并且诱导的$\mathrm{Spec}B\to\mathrm{Spec}C_{\mathfrak{n}}\to\mathrm{Spec}A$依旧把唯一闭点映射为唯一闭点,把一般点映射为一般点.
	\end{proof}
    \item 分离性的赋值准则.设$f:X\to Y$是概形之间的态射,那么如下命题互相等价:
    \begin{enumerate}
    	\item $f$是分离态射.
    	\item $f$是拟分离态射(此即它的对角态射是拟紧的);对任意赋值环$A$,记商域为$K$,记典范包含映射$A\to K$诱导的仿射概形之间的态射为$i$,对任意态射构成的如下实线交换图表,那么至多存在唯一(这包含了不存在虚线态射使得图表交换的情况)一条虚线态射使得图表交换.
    	$$\xymatrix{\mathrm{Spec}K\ar[d]_i\ar[rr]&&X\ar[d]^f\\\mathrm{Spec}A\ar@{-->}[urr]\ar[rr]&&Y}$$
    	\item $f$是拟分离态射;对任意$Y$概形$\mathrm{Spec}A$,其中$A$是赋值环,商域记作$K$,如果记$A\subseteq K$诱导的态射为$i:\mathrm{Spec}K\to\mathrm{Spec}A$,那么如下映射映射是单射:
    	$$\mathrm{Hom}_Y(\mathrm{Spec}A,X)\to\mathrm{Hom}_Y(\mathrm{Spec}K,X)$$
    	$$\varphi\mapsto\varphi\circ i$$
    \end{enumerate}
    
    如果这里$f$是局部有限型态射,$Y$是局部诺特概形,那么这里(b)和(c)中的"赋值环$A$"可以改为"离散赋值环$A$".
    \begin{proof}
    	
    	(a)$\Rightarrow$(b):设$f:X\to Y$是分离态射,那么$\Delta_{X/Y}$是闭嵌入,假设有两个态射$g_1,g_2$使得如下图表交换:
    	$$\xymatrix{U=\mathrm{Spec}K\ar[d]_i\ar[rr]&&X\ar[d]^f\\T=\mathrm{Spec}A\ar[rr]\ar@<0.5ex>[urr]^{g_1}\ar@<-0.5ex>[urr]_{g_2}&&Y}$$
    	
    	右下的三角形图表交换即$f\circ g_1=f\circ g_2$,于是存在唯一的态射$g:\mathrm{Spec}A\to X\times_YX$使得如下图表交换:
    	$$\xymatrix{\mathrm{Spec}A\ar[dr]^g\ar@/^1pc/[drr]^{g_2}\ar@/_1pc/[ddr]_{g_2}&&\\&X\times_YX\ar[r]^{\pi_1}\ar[d]_{\pi_2}&X\ar[d]^f\\&X\ar[r]_f&Y}$$
    	
    	左上的三角形图表交换即$g_1\mid_U=g_2\mid_U$,这里$U=\{\xi\}$是$T$的开子概型.考虑如下实线交换图表,按照$g$是连续的,并且$\Delta(X)$是$X\times_YX$的闭子集,就有$g(T)=g(\overline{U})\subseteq\overline{g(U)}\subseteq\overline{\Delta(X)}=\Delta(X)$.于是引理1说明存在唯一的虚线态射$T\to X$使得如下图表交换:
    	$$\xymatrix{U\ar[d]\ar[rr]&&X\ar[d]^{\Delta}\\T\ar@{-->}[urr]\ar[rr]_g&&X\times_YX}$$
    	
    	于是我们有:
    	\begin{align*}
    		g_1&=g\circ\pi_1\\&=\xymatrix{T\ar[r]&X\ar[r]^{\Delta}&X\times_YX\ar[r]^{\pi_1}&X}\\&=\xymatrix{T\ar[r]&X}\\&=\xymatrix{T\ar[r]&X\ar[r]^{\Delta}&X\times_YX\ar[r]^{\pi_2}&X}\\&=g\circ\pi_2=g_2
    	\end{align*}
    
        (b)$\Rightarrow$(a):我们解释过对角态射$\Delta=\Delta_f$总是嵌入,它是闭嵌入当且仅当$\Delta(X)$是闭集.由于(b)要求了$\Delta$是拟紧的,引理2'说明为证明$\Delta(X)$是闭集,只需证明$\Delta(X)$保特殊化,也即如果$y=f(x)\in\Delta(X)$和$z\in\overline{\{y\}}$,那么$z\in\Delta(X)$.按照引理3,存在赋值环$A$,记$\mathrm{Spec}A$的唯一闭点是$s$,一般点是$\xi$,那么存在态射$g:\mathrm{Spec}A\to X\times_YX$使得$\xi$和$s$分别映射到$y$和$z$.记$A$的商域为$K$.但是此时可能未必有如下交换图表,因为$\mathrm{Spec}K\to X$把$\xi\mapsto x$未必是某个概形态射的底映射,因为这样的态射要对应于域扩张$\kappa(x)\subseteq K$,为了解决这件事我们适当扩大$K$,取$K'$为$K$和$\kappa(x)$的合成域,取$A'$为$K'$的一个控制了$A$的赋值环,再用记号$K,A$分别替换$K',A'$,那么此时的确存在概形的态射$\mathrm{Spec}K\to X$把$\xi$映射到点$x$.于是我们有如下交换图表:
        $$\xymatrix{U=\mathrm{Spec}K=\{\xi\}\ar[rr]\ar[d]&&X\ar[d]^{\Delta}\\T=\mathrm{Spec}A\ar[rr]_g&&X\times_YX}$$
        
        考虑如下交换图表,这里$g_1=\pi_1\circ g$和$g_2=\pi_2\circ g$,按照纤维积的泛性质使得图表成立的$g$是唯一的:
        $$\xymatrix{U\ar[dr]^g\ar@/^1pc/[drr]^{g_2}\ar@/_1pc/[ddr]_{g_2}&&\\&X\times_YX\ar[r]^{\pi_1}\ar[d]_{\pi_2}&X\ar[d]^f\\&X\ar[r]_f&Y}$$
        
        那么有如下交换图表,其中左上角的三角交换是因为$g_1\mid_U=g_2\mid_U$,右下角的三角交换直接验证.
        $$\xymatrix{U\ar[rr]\ar[d]&&X\ar[d]\\T\ar@<0.5ex>[urr]^{g_1}\ar@<-0.5ex>[urr]_{g_2}\ar[rr]&&Y}$$
        
        于是(b)保证了必须有$g_1=g_2$.此时有如下交换图表:
        $$\xymatrix{U\ar@/^2pc/[ddrrr]^{g_1}\ar@/_2pc/[dddrr]_{g_1}\ar[dr]^{g_1}&&&\\&X\ar@/_1pc/[ddr]^{1_X}\ar@/^1pc/[drr]^{1_X}\ar[dr]^{\Delta}&&\\&&X\times_YX\ar[r]_{\pi_1}\ar[d]^{\pi_2}&X\ar[d]^f\\&&X\ar[r]_f&Y}$$
        
        但是按照纤维积的泛性质这里$g$是唯一的,所以有$g=\Delta\circ g_1$,于是$z=g(s)=\Delta(g_1(s))\subseteq\Delta(X)$.另外如果用引理2的话,就要设$y=\Delta(x)$,证明的是$z\in\Delta(\overline{\{x\}})$,但是我们构造的$z=\Delta(g_1(s))$,由于$\xi$是$s$的一般化,连续映射是保一般化的,于是$g_1(\xi)=x$是$g_1(s)$的一般化,所以$g_1(s)\in\overline{\{x\}}$.
        
        \qquad
        
        最后如果$f$是局部有限型态射,$Y$是局部诺特概形,那么$X\times_YX$是局部诺特概形,于是引理3说明我们选取的$A$可以取为DVR,记商域是$K$,如果记$L$是$K$与$\kappa(x)$的合成域,那么$A$在$L$中的整闭包也是一个DVR,后面的证明是一样的,所以此时赋值环可以改为离散赋值环.
    \end{proof}
    \item 泛闭性的赋值准则.设$f:X\to Y$是概形之间的拟紧态射,那么如下命题互相等价:
    \begin{enumerate}
    	\item $f$是泛闭的态射.
    	\item 对任意赋值环$A$,记商域为$K$,记典范包含映射$A\to K$诱导的仿射概形之间的态射为$i$,对任意态射构成的如下实线交换图表,那么至少存在一个虚线态射使得图表交换.
    	$$\xymatrix{\mathrm{Spec}K\ar[d]_i\ar[rr]&&X\ar[d]^f\\\mathrm{Spec}A\ar@{-->}[urr]\ar[rr]&&Y}$$
    	\item 对任意$Y$概形$\mathrm{Spec}A$,其中$A$是赋值环,商域记作$K$,如果记$A\subseteq K$诱导的态射为$i:\mathrm{Spec}K\to\mathrm{Spec}A$,那么如下映射映射是满射:
    	$$\mathrm{Hom}_Y(\mathrm{Spec}A,X)\to\mathrm{Hom}_Y(\mathrm{Spec}K,X)$$
    	$$\varphi\mapsto\varphi\circ i$$
    \end{enumerate}

    如果这里$Y$是局部诺特概形,那么这里(b)和(c)中的"赋值环$A$"可以改为"离散赋值环$A$".
    \begin{proof}
    	
    	(a)$\Rightarrow$(b):设$f$是泛闭态射,任取$\varphi\in\mathrm{Hom}_Y(\mathrm{Spec}K,X)$,我们要找一个态射$\psi\in\mathrm{Hom}_Y(\mathrm{Spec}A,X)$使得$\psi\circ i=\varphi$.考虑如下图表,这里$\Gamma_{\varphi}$是$\varphi$的图像态射,$f_K$是$f$的基变换,按照条件它拓扑上是闭映射.
    	$$\xymatrix{\mathrm{Spec}K\ar@/^1pc/[drr]^{\varphi}\ar@/_1pc/@{=}[ddr]\ar[dr]^{\Gamma_{\varphi}}&&\\&\mathrm{Spec}K\times_YX\ar[r]\ar[d]^{f_K}&X\ar[d]\\&\mathrm{Spec}K\ar[r]&Y}$$
    	
    	倘若我们能找到态射如下虚线态射,使得第二行的态射复合为$\mathrm{Spec}A$上的恒等态射,则把这个虚线态射复合$\mathrm{Spec}A\times_YX\to X$就得到$\psi$.
    	$$\xymatrix{\mathrm{Spec}K\ar[rr]^{\Gamma_{\varphi}}\ar[d]&&\mathrm{Spec}K\times_YX\ar[rr]^{f_K}\ar[d]&&\mathrm{Spec}K\ar[d]\\\mathrm{Spec}A\ar@{-->}[rr]&&\mathrm{Spec}A\times_YX\ar[rr]^{f_A}&&\mathrm{Spec}A}$$
    	
    	这个图表的所有垂直态射都是开嵌入,所以上面的概形都可以视为下面概形的开子概型.$\mathrm{Spec}K=\{\xi\}$是单点集,设它在$\Gamma_{\varphi}$下的像是$z$,按照图表交换性有$f_A(z)=\xi$.按照$f_A$是闭映射,得到$f_A(\overline{\{z\}})=\overline{\{f(z)\}}=\overline{\xi}$,所以如果记$\mathrm{Spec}A$的唯一闭点是$s$,那么存在$z$的特殊化$y$使得$f_A(y)=s$.考虑如下图表,引理1说明存在提升态射$\mathrm{Spec}K\to\overline{\{z\}}$.
    	$$\xymatrix{&&\mathrm{Spec}\mathscr{O}_{\overline{\{z\}},y}\ar[d]\\&&\overline{\{z\}}\ar[d]\\\mathrm{Spec}K\ar[rr]\ar@{-->}[urr]\ar@{-->}[uurr]^{\alpha}&&\mathrm{Spec}A\times_YX}$$
    	
    	这里存在虚线态射$\mathrm{Spec}K\to\mathrm{Spec}\mathscr{O}_{\overline{\{z\}},y}$使得上方三角交换是因为有如下交换图表:
    	$$\xymatrix{&&B_{\mathfrak{p}}\ar[dll]\\K&&B\ar[u]\ar[ll]}$$
    	
    	于是我们有如下交换图表:
    	$$\xymatrix{\mathrm{Spec}K\ar[rr]^{\alpha}\ar[d]&&\mathrm{Spec}\mathscr{O}_{\overline{\{z\}},y}\ar[d]^{\gamma}\ar[dll]_{\beta}\\\mathrm{Spec}A&&\mathrm{Spec}A\times_YX\ar[ll]_{f_A}}$$
    	
    	这里$\alpha$和$\beta$都是支配态射,诱导它们的环同态都是单射并且是局部同态(唯一极大理想拉回来还是唯一极大理想),所以有环同态的交换图表:
    	$$\xymatrix{A\ar[rr]\ar[d]&&K\\\mathscr{O}_{\overline{\{z\}},y}\ar[urr]&&}$$
    	
    	按照赋值环是它商域的局部子环的控制偏序下的极大元,迫使$A=\mathscr{O}_{\overline{\{z\}},y}$.于是这里$\beta$是同构.取$\mathrm{Spec}A\to\mathrm{Spec}A\times_YX$为$\gamma\circ\beta^{-1}$满足和$f_A$的复合是$\mathrm{Spec}A$上的恒等态射.
    	
    	\qquad
    	
    	(b)$\Rightarrow$(a):对任意$Y$概形$Y'$,要证明$X'=X\times_YY'\to Y'$拓扑上是闭映射.由于对任意$Y'$概形$Z$有典范同构$\mathrm{Hom}_{Y'}(Z,X\times_YY')\cong\mathrm{Hom}_Y(Z,X)$.所以倘若我们证明了条件下有$f:X\to Y$拓扑上是闭映射,那么由于$\mathrm{Hom}_{Y'}(\mathrm{Spec}A,X\times_YY')\to\mathrm{Hom}_{Y'}(\mathrm{Spec}K,X\times_YY')=\mathrm{Hom}_Y(\mathrm{Spec}K,X)$仍然是满射,就得到$X'=X\times_YY'\to Y'$拓扑上是闭映射.
    	
    	\qquad
    	
    	按照引理2,归结为证明如果$x\in X$,记$f(x)=y$,取$y'\in\overline{\{y\}}$,那么有$y'\in f(\overline{\{x\}})$.按照引理3,存在赋值环$A$,记它的唯一闭点是$s$,它的一般点是$\xi$,存在态射$h:\mathrm{Spec}A\to Y$满足$h(\xi)=y$和$h(s)=y'$.设$\kappa(x)$和$K$的合成域是$L$,那么存在$L$的赋值环$B$控制了$A$,考虑复合态射$\mathrm{Spec}B\to\mathrm{Spec}A\to Y$(这里取$L$的目的是如果只考虑$K$,那么未必存在态射$\mathrm{Spec}K\to X$把$\xi$映射到$x$,因为这样的态射要对应于域扩张$\kappa(x)\subseteq K$,所以我们需要把$K$扩大到$L$).那么有如下实线交换图表,按照条件(b),就存在虚线态射$g$使得如下图表交换,于是$y'=f(g(s))$,又因为连续映射保一般化,于是从$\xi$是$s$的一般化得到$x=g(\xi)$是$g(s)$的一般化,于是$y'\in f(\overline{\{x\}})$.
    	$$\xymatrix{\mathrm{Spec}L\ar[d]\ar[rr]&&X\ar[d]^f\\\mathrm{Spec}B\ar[rr]\ar@{-->}[urr]^g&&Y}$$
    \end{proof}
    \item 紧合态射的赋值准则.设$f:X\to Y$是概形之间的有限型态射,那么如下命题互相等价:
    \begin{enumerate}
    	\item $f$是紧合态射.
    	\item $f$是拟分离态射;对任意赋值环$A$,记商域为$K$,记典范包含映射$A\to K$诱导的仿射概形之间的态射为$i$,对任意态射构成的如下实线交换图表,那么恰好存在唯一的虚线态射使得图表交换.
    	$$\xymatrix{\mathrm{Spec}K\ar[d]_i\ar[rr]&&X\ar[d]^f\\\mathrm{Spec}A\ar@{-->}[urr]\ar[rr]&&Y}$$
    	\item $f$是拟分离态射;对任意$Y$概形$\mathrm{Spec}A$,其中$A$是赋值环,商域记作$K$,如果记$A\subseteq K$诱导的态射为$i:\mathrm{Spec}K\to\mathrm{Spec}A$,那么如下映射映射是双射:
    	$$\mathrm{Hom}_Y(\mathrm{Spec}A,X)\to\mathrm{Hom}_Y(\mathrm{Spec}K,X)$$
    	$$\varphi\mapsto\varphi\circ i$$
    \end{enumerate}

    如果$Y$是局部诺特概形,那么这里的(b)和(c)中的赋值环可以改为离散赋值环.
\end{enumerate}
\subsection{几何P性质}

一般来讲一个域$k$上概形$X$的某些性质可能不会传递给$X_K=X\times_kK$,其中$K$是$k$的域扩张.设P是域上概形的如下四个性质之一:不可约,连通,既约或者整.我们称$k$概形$X$是几何P的,如果对任意域扩张$K/k$,都有$X_K$满足性质$P$.
\begin{enumerate}
	\item 设$X,Y$是域$k$上的概形,记典范投影态射$p:X\times_kY\to X$,于是它是$Y\to\mathrm{Spec}k$的基变换.
	\begin{enumerate}
		\item $p$是满射,并且是泛开的.
		\begin{proof}
			
			$p$是$Y\to\mathrm{Spec}k$的基变换,$Y\to\mathrm{Spec}k$是满射,它是泛开的因为有如下结论:如果$S$的底空间是离散空间,那么任意态射$f:X\to S$都是泛开的.因为满射和泛开都是基变换下不变性质,于是$p$是满射和泛开的.
		\end{proof}
	    \item $Z\mapsto\overline{p(Z)}$是从$\{X\times_kY\text{的不可约分支}\}\to\{X\text{的不可约分支}\}$的满射.换句话讲$p$把$X\times_kY$的一般点映射为$X$的一般点,并且如果$x\in X$是一般点,那么$p^{-1}(x)$中包含$X\times_kY$的一般点.
	    \begin{proof}
	    	
	    	设$z$是$X\times_kY$的一般点,由于$p$是开映射,我们解释过如果$x'$是$x=p(z)$的一般化,那么存在$z$的一般化$z'$使得$x'=p(z')$,但是$z$已经是一般点,所以$x'=p(z')=p(z)=x$,于是$x=p(z)$是$X$的一般点.另一方面如果$x$是$X$的一般点,由于$p$是满射可取$z\in X\times_kY$使得$p(z)=x$.取$z$所在的不可约分支的一般点为$z'$,因为连续映射保一般化,所以$p(z')$是$p(z)=x$的一般化,导致$p(z')=p(z)=x$.
	    	
	    	\qquad
	    	
	    	我们也可以回避$p$是开映射直接证明这件事:任取$X$的一般点$x$,设$p(x)=y$,我们要证明$y$是$Y$的一般点,取$y$的仿射开邻域$V$,开集是保一般化的,所以归结为证明$y$是$V$的一般点.再取包含在$p^{-1}(V)$内的$x$的仿射开邻域$U$,那么$x$是$U$的一般点.换句话讲问题是仿射的,归结为如果$X=\mathrm{Spec}A$,$Y=\mathrm{Spec}B$,记$C=A\otimes_kB$,如果$\mathfrak{q}\subseteq C$是极小素理想,记$\mathfrak{p}=\mathfrak{q}\cap A$是$A$的素理想,我们要证明$\mathfrak{p}$是$A$的极小素理想.也即$A_{\mathfrak{p}}$只有唯一的素理想.也即证明$A_{\mathfrak{p}}\to C_{\mathfrak{q}}$诱导的$f:\mathrm{Spec}C_{\mathfrak{q}}\to\mathrm{Spec}A_{\mathfrak{p}}$是满射.任取$A_{\mathfrak{p}}$的素理想$\mathfrak{p}'$,我们只需证明纤维环$C_{\mathfrak{q}}\otimes_{A_{\mathfrak{p}}}\kappa(\mathfrak{p}')$不是零环.但是这个环也就是$C_{\mathfrak{q}}\otimes_CC\otimes_A\kappa(\mathfrak{p}')=C_{\mathfrak{q}}\otimes_CB\otimes_k\kappa(\mathfrak{p}')$.这里$C_{\mathfrak{q}}\otimes_CB$和$\kappa(\mathfrak{p}')$都不是零环,所以它们在域上做张量积也不是零环.
	    \end{proof}
        \item $C\mapsto p(C)$是从$\{X\times_kY\text{的连通分支}\}\to\{X\text{的连通分支}\}$的满射.这件事对连续满射已经总成立了.
        \item 如果$X\times_kY$满足如下四个性质之一:不可约,连通,既约,整,那么$X$满足相同的性质.我们称这为这四个条件的下降性质.
        \begin{proof}
        	
        	不可约和连通的情况从上面的结论可以推出.整概形等价于既约和不可约,所以问题归结为既约的情况.因为既约是仿射局部性质,所以不妨设$X=\mathrm{Spec}A$和$Y=\mathrm{Spec}B$都是仿射的,那么$X\times_kY=\mathrm{Spec}A\otimes_kB$.但是由于$k$是域,典范映射$A\to A\otimes_kB$是单射,于是从$A\otimes_kB$既约得到$A$是既约.
        \end{proof}
	\end{enumerate}
    \item 设$k$是域,设$X$是$k$概形,那么如下条件互相等价,当条件成立时称$X$是几何既约的.
    \begin{enumerate}
    	\item 对任意既约$k$概形$Y$有$X_Y$是既约概形.
    	\item 对任意域扩张$k\subseteq K$有$X_K$是既约概形.
    	\item 存在一个域扩张$k\subseteq\Omega$,其中$\Omega$是完全域(比方说,包含$k$的最小的完全域是$k^{1/p^{\infty}}$),使得$X_{\Omega}$是既约概形.
    	\item 对任意有限维纯不可分扩张$k\subseteq K$有$X_K$是既约概形.
    	\item $X$是既约的,并且对任意一般点$\eta$,有$k\subseteq\kappa(\eta)$是可分扩张(此为一般意义下的可分扩张,即对任意域扩张$k\subseteq L$有$K\otimes_kL$是既约环,在代数扩张的情况下吻合于可分代数扩张的定义).
    \end{enumerate}
    \begin{proof}
    	
    	这里所有条件都是仿射局部的,所以不妨设$X=\mathrm{Spec}A$,其中$A$是一个$k$代数.另外结合上一条,这里所有条件都可以推出$X$本身是既约概形,所以不妨设这里$A$是既约环.设$X$的全部一般点为$\{\eta_i=\mathfrak{p}_i\mid i\in I\}$,它们也就是全部极小素理想.我们先断言对任意域扩张$k\subseteq L$,有$A\otimes_kL$是既约的当且仅当所有$\kappa(\eta_i)\otimes_kL$是既约的:一方面如果$A\otimes_kL$是既约的,由于$\mathfrak{p}_i$是极小素理想,那么$A_{\mathfrak{p}_i}$是具有唯一素理想的既约环(因为$A$是既约的),所以$A_{\mathfrak{p}_i}$是域,于是$A_{\mathfrak{p}_i}=\kappa(\eta_i)$.那么$A_{\mathfrak{p}_i}\otimes_kL$作为既约环$A\otimes_kL$的分式化,就也是既约的.另一方面按照$A$是既约的,有典范同态$A\to\prod_{i\in I}\kappa(\eta_i)$是单射.我们有单射的的复合$A\otimes_kL\to\left(\prod_{i\in I}\kappa(\eta_i)\right)\otimes_kL\to\prod_{i\in I}\left(\kappa(\eta_i)\otimes_kL\right)$(后一个单射是这样来的:一般的如果$M,N$是环$R$的两个模,并且$N$是自由$R$模,记一组基为$\{e_j\mid j\in J\}$,那么$M\otimes_RN$中的元可以唯一的表示为$\sum_j m_j\otimes e_j$.记$L$在$k$上的一组基为$\{e_j,j\in J\}$,那么$\left(\prod_{i\in I}\kappa(\eta_i)\right)\otimes_kL$中的元可以唯一的表示为$\sum_j(a_i^{(j)})\otimes e_j$,它映射过去的像是$(\sum_ja_i^{(j)}\otimes e_j)_{i\in I}$,如果这个像是0,也即$\forall i\in I$有$\sum_ja_i^{(j)}\otimes e_j=0$,但是$\kappa(\eta_i)\otimes_kL$中的元表示为这个形式是唯一的,迫使$a_i^{(j)}=0$.).于是从所有$\kappa(\eta_i)\otimes_kL$是既约环得到$A\otimes_kL$是既约环.
    	
    	\qquad
    	
    	扩张$k\subseteq k(\eta_i)$是可分的有如下等价描述:
    	\begin{itemize}
    		\item 对任意域扩张$k\subseteq L$有$\kappa(\eta_i)\otimes_kL$是既约的.
    		\item 对任意有限纯不可分扩张$k\subseteq L$有$\kappa(\eta_i)\otimes_kL$是既约的.
    		\item 存在正整数$n$使得$\kappa(\eta_i)\otimes_kk^{1/p^n}$是既约的.这里$k^{1/p^n}/k$总是有限纯不可分扩张,并且每个有限纯不可分扩张都是某个$k^{1/p^n}/k$的子扩张.
    		\item $\kappa(\eta_i)\otimes_kk^{1/p^{\infty}}$是既约的,这里$k^{1/p^{\infty}}$是包含$k$的最小的完全域.
    	\end{itemize}
    
        这些等价描述说明(b),(c),(d),(e),(f)是等价的.(a)推(b)是平凡的,最后证明(f)推(a):依旧因为既约是仿射局部性质,可设$Y=\mathrm{Spec}B$是仿射的.取$B$的全部一般点为$\{\xi_j\mid j\in J\}$,那么$A\otimes_kB$就可以嵌入到$\prod_{i,j}(\kappa(\eta_i)\otimes_k\kappa(\xi_j))$,但是按照(f)有后者是既约环的积,也是既约环,于是$A\otimes_kB$作为既约环的子环也是既约的.
    \end{proof}
    \item 设$k$是域,设$X$是$k$概形,那么如下条件互相等价,当条件成立时称$X$是几何不可约的.
    \begin{enumerate}
    	\item 对任意不可约$k$概形$Y$有$X\times_kY$是不可约概形.
    	\item 对任意域扩张$k\subseteq K$有$X\times_kK$是不可约概形.
    	\item 存在一个域扩张$k\subseteq\Omega$,其中$\Omega$是可分闭的域(这是指$\Omega$的有限可分代数扩张只有自己,比方说取$k$的代数可分闭包),使得$X_{\Omega}=X\times_k\Omega$是不可约概形.
    	\item 对任意有限可分扩张$k\subseteq K$有$X_K=X\times_kK$是不可约概形.
    	\item $X$是不可约的并且如果$\eta$是$X$的一般点,那么$k\subseteq\kappa(\eta)$是可分闭扩张(此即$k$在$\kappa(\eta)$中的可分代数闭包是$k$自己).
    \end{enumerate}
    \begin{proof}
    	
    	按照不可约条件的下降性质,这里的每一个命题都能推出$X$是不可约的,于是我们可以设$X$是以$\eta$为唯一一般点的不可约概形.我们解释过一个态射$f:X\to Y$如果是开映射,如果$Y$是以$\eta$为一般点的不可约概形,那么$X$是不可约的当且仅当$f^{-1}(\eta)$是不可约的.在这里对$k$概形$Y$,总有$p:X\times_kY\to X$是开映射,所以$X\times_kY$是不可约的当且仅当$p^{-1}(\eta)=\kappa(\eta)\otimes_kY$是不可约的.
    	
    	\qquad
    	
    	扩张$k\subseteq\kappa(\eta)$是可分闭的等价于如下命题中的任一条:
    	\begin{itemize}
    		\item 对任意域扩张$k\subseteq L$有$\kappa(\eta)\otimes_kL$只有唯一的极小素理想(即不可约的).
    		\item 对任意有限可分扩张$k\subseteq L$有$\kappa(\eta)\otimes_kL$只有唯一的极小素理想.
    		\item 对任意域扩张$k\subseteq L$,其中$L$是可分闭的,有$\kappa(\eta)\otimes_kL$只有唯一的极小素理想.
    	\end{itemize}
    
        这些等价描述说明(b),(c),(d)是等价的.(a)推(b)是平凡的,最后证明(d)推(a):如果$Y$是不可约概形,我们要证明$\kappa(\eta)\times_kY$是不可约的,由于$\kappa(\eta)\times_kY\to Y$也是开映射,按照第一段的讨论,如果设$Y$的一般点为$\xi$,那么$\kappa(\eta)\times_kY$是不可约的等价于$\kappa(\eta)\otimes_k\kappa(\xi)$是不可约的(即只有唯一极小素理想),但是这也归结到上述的可分闭扩张的等价描述中.
    \end{proof}
    \item 设$k$是域,设$X$是$k$概形,那么如下条件互相等价,当条件成立时称$X$是几何整的.
    \begin{enumerate}
    	\item 对任意整$k$概形$Y$有$X\times_kY$是整概形.
    	\item 对任意域扩张$k\subseteq K$有$X\times_kK$是整概形.
    	\item 存在一个域扩张$k\subseteq\Omega$,其中$\Omega$是代数闭的(比方说取$k$的代数闭包),使得$X_{\Omega}=X\times_k\Omega$是整概形.
    	\item 对任意有限扩张$k\subseteq K$有$X_K=X\times_kK$是整概形.
    	\item $X$是整的并且如果$\eta$是$X$的一般点,那么$k\subseteq\kappa(\eta)$是可分扩张,并且$k$在$\kappa(\eta)$中代数闭.
    \end{enumerate}
    \item 设$k$是域,设$X$是$k$概形,那么如下条件的前三条互相等价,当条件成立时称$X$是几何连通的.如果$X$还是拟紧的,那么它们和第四条等价.
    \begin{enumerate}
    	\item 对任意连通$k$概形$Y$有$X\times_kY$是连通概形.
    	\item 对任意域扩张$k\subseteq K$有$X\times_kK$是连通概形.
    	\item 存在一个域扩张$k\subseteq\Omega$,其中$\Omega$是可分闭的,使得$X_{\Omega}=X\times_k\Omega$是连通概形.
    	\item 对任意有限可分扩张$k\subseteq K$有$X_K=X\times_kK$是连通概形.
    \end{enumerate}
    \begin{proof}
    	
    	(a)推(b)推(c)是平凡的.先证明(b)推(a).我们有如下纯拓扑的引理:如果$f:X\to Y$是连续开映射,并且是满射,如果$Y$是连通的,并且对任意$y\in Y$有$f^{-1}(y)$是连通的,那么$X$是连通的.任取连通$k$概形$Y$,那么这里$q:X\times_kY\to Y$是开映射也是满射,按照引理为证明$X\times_kY$是连通的,就归结为证明每个纤维$q^{-1}(y)=X\times_k\kappa(y)$是连通的,但是这是条件(b).
    	
    	\qquad
    	
    	再证明(c)推(b):设$k\subseteq K$是任意的域扩张,设$L$是一个同时包含$K$和$\Omega$的域.那么基变换$X_L\to X_K$是连续满射,如果证明了$X_L$是连通的,满射像$X_K$就也是连通的.典范态射$p:X_L\to X_{\Omega}$是开映射也是满射,所以按照引理归结为证明每个$x\in X_{\Omega}$的纤维$p^{-1}(x)=\mathrm{Spec}(\kappa(x)\otimes_{\Omega}L)$是连通的.但是按照$\Omega$在$L$中是可分闭的,导致所有纤维甚至是不可约的,特别的它们都是连通的.
    	
    	\qquad
    	
    	(a)推(d)是平凡的,最后证明$X$是拟紧的条件下有(d)推(c)成立(证明来自EGAIV的8.4.5):取$k$的一个代数闭包$\Omega$,我们要证明$X\times_k\Omega$是连通的.考虑$\Omega/k$的全体在$k$上有限的中间域构成的集合为$S$,对每个$k'\in S$赋予一个概形$X\times_kk'$,如果有$k',k''\in S$满足$k\subseteq k'\subseteq k''\subseteq\Omega$,取典范态射$X\times_kk''\to X\times_kk'$,并且按照满射在基变换下不变得到它也是一个满射.按照$\Omega$是$S$作为正向系统的极限,于是$X\times_k\Omega$是$\{X\times_kk'\mid k'\in S\}$的逆向极限.我们解释过在拟紧条件下,一个逆向系统$\{(S_i),(u_{ij}:S_j\to S_i)\}$中如果$u_{ij}$总是支配的,那么$S$是连通的当且仅当$S_i$在$i$足够大时是连通的.这里$X\times_kk''\to X\times_kk'$是满射它当然是支配态射.于是我们只需证明$X\times_kk',k'\in S$总是连通的.但是如果取$k\subseteq k'$的极大可分子扩张$k\subseteq K$,那么$X\times_kk'\to X\times_kK$就明显是纯不可分态射(此即泛单射),另外它明显是有限态射(从而是闭映射)以及满射,这些条件保证了它是同胚,于是从$X\times_kK$是连通的就得到$X\times_kk'$是连通的,这就得证.
    \end{proof}
    \item 一些推论.
    \begin{enumerate}[(1)]
    	\item 设$X$是域$k$上的概形,设$k\subseteq K$是一个域扩张,其中$K$是一个代数闭域,那么$X$是几何不可约或几何整或几何连通或几何既约的当且仅当$X_K=X\times_kK$是不可约或整或连通或既约的.
    	\item 特别的,对于域$k$上的概形$X$,设$K$是包含$k$的一个代数闭域,那么$X_K$的不可约分支个数和连通分支个数是不依赖于代数闭域$K$的选取的.这个个数称为$X$的不可约分支的几何个数以及连通分支的几何个数.
    	\item 设$X$是域$k$上的概形,如果$X$是既约的,并且$k\subseteq K$是可分的,那么$X_K$是既约的.
    	\begin{proof}
    		
    		因为此时$\mathrm{Spec}K$作为$k$概形是几何既约的.
    	\end{proof}
        \item 设$X$是域$k$上的概形,如果$X$是不可约的或者连通的,设有域扩张$k\subseteq K$使得$k$在$K$中可分闭,那么$X_K$是不可约的或者连通的.
        \begin{proof}
        	
        	这里投影态射$p:X_K\to X$是开映射并且是满射,因为$X$是不可约或者连通的,所以为证明$X_K$是不可约或者连通的,只需证明每个纤维$p^{-1}(x)=\mathrm{Spec}\kappa(x)\otimes_kK$是不可约或者连通的.但是由于$k$在$K$中可分闭,得到$\mathrm{Spec}K\to\mathrm{Spec}k$是几何不可约的,于是每个$p^{-1}(x)$都是不可约的.
        \end{proof}
        \item 设$X$是域$k$上的概形,如果$X$是整的,并且$k\subseteq K$是可分扩张,并且$k$在$K$中代数闭,那么$X_K$是整的.
        \item 如果$k$是完全域(此时它的所有域扩张都是广义可分的),那么每个既约$k$概形都是几何既约的.
        \item 设$X/k$是有限型连通概形,$X$上存在有理点$x$,那么$X$是几何连通的.
        \begin{proof}
        	
        	问题归结为对有限扩张$k\subseteq L$,有$X_L$是连通的.由于$\mathrm{Spec}L\to\mathrm{Spec}k$是有限平坦的,所以它是泛开和泛闭的,于是$X_L\to X$把$X_L$的所有(有限个)连通分支都映满$X$.考虑如下纤维积图表,这说明有理点$x:\mathrm{Spec}k\to X$在$X_L$中有唯一的原像$x'$,这迫使$X_L$的所有连通分支都包含$x'$,进而$X_L$只有一个连通分支.
        	$$\xymatrix{\mathrm{Spec}L\ar[r]^{x'}\ar[d]&X_L\ar[r]\ar[d]&\mathrm{Spec}L\ar[d]\\\mathrm{Spec}k\ar[r]^x&X\ar[r]&\mathrm{Spec}k}$$
        \end{proof}
    \end{enumerate}
\end{enumerate}
\newpage
\section{下降理论}
\subsection{纤维范畴}

【临时】设$S$是概形,记$\mathscr{C}_S$表示$S$概形范畴的一个子范畴,满足在纤维积下封闭.
\begin{enumerate}
	\item $\mathscr{C}_S$上的一个纤维范畴(fibred category)是指一个伪函子(pseudofunctor)$\mathfrak{X}:\mathscr{C}_S\to\textbf{Cat}$.这是指如下信息:
	\begin{itemize}
		\item 对任意$U\in\mathscr{C}_S$,赋予一个范畴$\mathfrak{X}_U$,它称为$\mathfrak{X}$在$U\to S$上的纤维.
		\item 对任意$\mathscr{C}_S$中的态射$\varphi:V\to U$,取一个函子$\varphi^*:\mathfrak{X}_U\to\mathfrak{X}_V$称为基变换函子.
		\item 如果$\xymatrix{W\ar[r]^{\psi}&V\ar[r]^{\varphi}&U}$是$\mathscr{C}_S$中的态射,那么存在自然同构$\alpha_{\psi,\varphi}:\psi^*\varphi^*\cong(\varphi\circ\psi)^*$.
		\item 上述自然同构满足余圈条件,也即如果$\xymatrix{X\ar[r]^{\sigma}&W\ar[r]^{\psi}&V\ar[r]^{\varphi}&U}$是$\mathscr{C}_S$中的态射,那么有如下交换图表:
		$$\xymatrix{\sigma^*\psi^*\varphi^*\ar[rr]^{\sigma^*(\alpha_{\psi,\varphi})}\ar[d]_{\alpha_{\sigma,\psi}(\varphi^*)}&&\sigma^*(\varphi\circ\psi)^*\ar[d]^{\alpha_{\sigma,\varphi\circ\psi}}\\(\psi\circ\sigma)^*\varphi^*\ar[rr]_{\alpha_{\psi\circ\sigma,\varphi}}&&(\varphi\circ\psi\circ\sigma)^*}$$
	\end{itemize}
    \item 下降问题主要指的是这两个问题:
    \begin{enumerate}[(1)]
    	\item 设$U,V,S'\in\mathscr{C}_S$,记$S'\to S$为$\varphi$,设$f'$是$U'=\varphi^*(U)\to V'=\varphi^*(V)$的态射,那么什么时候存在$\mathscr{C}_S$中的态射$f:U\to V$满足$\varphi^*(f)=f'$?
    	\item 设$\varphi:S'\to S$是态射,设$U'\in\mathscr{C}_{S'}$,那么什么时候存在$U\in\mathscr{C}_S$满足$U'\cong\varphi^*(U)$?
    \end{enumerate}
    \item 对于问题1:如果存在这样的$f:U\to V$,记$S''=S'\times_SS'$,考虑如下态射:
    $$\xymatrix{S'\ar@<.5ex>[r]^{p_1}\ar@<-.5ex>[r]_{p_2}&S'\ar[r]^{\varphi}&S}$$
\end{enumerate}







\subsection{仿射概形的性质的下降}
\begin{enumerate}
	\item 模的情况.设$A\to A'$是忠实平坦环同态,设$M$是$A$模,记$M'=M\otimes_AA'$.
	\begin{enumerate}
		\item 如果$M'$是平坦$A'$模,那么$M$是平坦$A$模.
		\item 如果$M'$是有限$A'$模,那么$M$是有限$A$模.
		\item 如果$M'$是有限表示$A'$模,那么$M$是有限表示$A$模.
		\item 如果$M'$是秩$n$局部自由$A'$模,那么$M$是秩$n$局部自由$A$模.
	\end{enumerate}
	\begin{proof}
		
		(a)就是忠实平坦定义.(b)先设$M'$作为$A'$模的生成元集为$\{e_i=\sum_jm_{ij}\otimes a_{ij}\}$,其中$m_{ij}\in M$,$a_{ij}\in A$.设这里$\{m_{ij}\}$的个数是$N$,定义$A$模同态$\varphi:A^N\to M$为把标准基映射到$\{m_{ij}\}$,那么$\varphi\otimes1_{A'}$是满射.按照$A\to A'$是忠实平坦的,就有$\varphi$也是满射,于是$M$是有限$A$模.
		
		\qquad
		
		证明(c):从(b)已经知道$M$是有限$A$模,取一个满同态$A^n\to M$,核记作$N$.那么按照$A\to A'$平坦,有${A'}^n\to M'$的核就是$N'=N\otimes_AA'$,按照$M'$是有限表示$A'$模,得到$N'$是有限$A'$模,再用一次(b)得到$N$是有限$A$模,从而$M$是有限表示$A$模.
		
		\qquad
		
		最后有限秩的局部自由模等价于平坦的有限表示模,按照(a)和(c)就有$M$是有限秩局部自由$A$模.另外基变换不改变秩,从而得证.
	\end{proof}
	\item 代数的情况.设$A\to A'$是忠实平坦环同态,设$B$是$A$代数,记$B'=B\otimes_AA'$.
	\begin{enumerate}
		\item 如果$B'$是有限型$A'$代数,那么$B$是有限型$A$代数.
		\item 如果$B'$是有限表示$A'$代数,那么$B$是有限表示$A$代数.
	\end{enumerate}
\end{enumerate}
\subsection{一般态射的性质的下降}

设$S$是概形,设$f:X\to Y$是$S$概形态射,设$g:S'\to S$是态射,记$f':X'=X\times_SS'\to Y'=Y\times_SS'$.设$g$满足性质Q,如果$f'$满足性质P总能推出$f$满足性质P,我们就称性质P是Q下降的.
\begin{enumerate}
	\item 集合层面的性质下降只需要满射条件.设$g:S'\to S$是满射.
	\begin{enumerate}[(1)]
		\item 如果$f'$是满射,那么$f$也是.
		\item 如果$f'$是单射,那么$f$也是.
		\item 如果$f'$是双射,那么$f$也是.
		\item 如果$f'$具有有限纤维,那么$f$也是.
	\end{enumerate}
	$$\xymatrix{X'\ar[rr]^{f'}\ar[d]_u&&Y'\ar[d]^v\\X\ar[rr]_f&&Y}$$
	\begin{proof}
		
		(1)是容易的:按照$g$是满射,它的基变换$u,v$都是满射,于是从$f'$是满射得到$f$是满射.下面设$y'\in Y'$,设$v(y')=y$,那么$X'_{y'}=X\times_YY'\times_{Y'}\mathrm{Spec}\kappa(y')\cong X_y\times_{\mathrm{Spec}\kappa(y)}\mathrm{Spec}\kappa(y')$.按照$Y'\to Y$是满射,它的基变换$X'_{y'}\to X_y$是满射.这件事同时证明了(2)和(4).最后(1)和(2)推出(3).
	\end{proof}
	\item 拓扑层面的性质下降需要拟紧泛浸没条件(一个态射如果是满射和商映射,则称为浸没.拟紧泛浸没的例子有拟紧忠实平坦态射,拟紧满射泛闭态射,满射紧合态射,后面这两个例子是因为满射闭映射是商映射).设$g:S'\to S$是拟紧泛浸没态射.
	\begin{enumerate}[(1)]
		\item 如果$f'$是开映射,那么$f$也是.
		\item 如果$f'$是闭映射,那么$f$也是.
		\item 如果$f'$是同胚,那么$f$也是.
		\item 如果$f'$是拟紧的,那么$f$也是(但是这个证明中只用到了$g$是拟紧满射).
		\item 如果$f'$是拟分离的,那么$f$也是(这个证明中只用到$g$是拟紧满射).
		\item 如果$f'$是分离的,那么$f$也是.
		\item 如果$f'$是支配态射,那么$f$也是(这个证明只用到了$g$是满射).
	\end{enumerate}
	$$\xymatrix{X'\ar[rr]^{f'}\ar[d]_u&&Y'\ar[d]^v\\X\ar[rr]_f&&Y}$$
	\begin{proof}
		
		按照$g$是拟紧泛浸没,得到这里$u,v$都是拟紧泛浸没.证明(1):设$W\subseteq X$是开集,要证明$f(W)$是开集,按照$v$是商映射,等价于证明$v^{-1}(f(W))$是开集.但是由于这是纤维积图表,我们有$v^{-1}(f(W))=f'(u^{-1}(W))$,于是按照$f'$是开映射得到这是开集.(2)的证明是同理的.(3)是因为态射是同胚等价于它是开映射和双射,而我们证明过了开映射和双射都可以下降.(4)是因为这里$u,v$是满射和拟紧的,于是如果$V\subseteq Y$是拟紧开子集,那么$f^{-1}(V)=u\circ u^{-1}\circ f^{-1}(V)=u\circ{f'}^{-1}\circ v^{-1}(V)$是拟紧的(拟紧子集的连续像是拟紧的).最后(5)和(6)是因为$\Delta_{f'}=\Delta_f\times_YY'$,于是从(2)和(4)得到(5)和(6)成立.
	\end{proof}
	\item 关于泛P性质的下降.设P和Q是关于概形态射的两个性质,其中Q是基变换下不变的,如果P是Q下降的,那么泛P也是Q下降的.例如这说明$g$是满射时$f'$是泛单的当且仅当$f$是泛单的.
	\item 概形层面的性质下降需要拟紧忠实平坦条件.设$g$是拟紧忠实平坦态射.
	\begin{enumerate}[(1)]
		\item 如果$f'$是有限型/局部有限型态射,那么$f$也是.
		\item 如果$f'$是有限表示/局部有限表示态射,那么$f$也是.
		\item 如果$f'$是同构,那么$f$也是.
		\item 如果$f'$是单态射,那么$f$也是.
		\item 如果$f'$是开嵌入/闭嵌入/拟紧嵌入,那么$f$也是.
		\item 如果$f'$是紧合态射,那么$f$也是.
		\item 如果$f'$是仿射态射,那么$f$也是.
		\item 如果$f'$是有限态射,那么$f$也是.
		\item 如果$f'$是拟有限态射,那么$f$也是.
	\end{enumerate}
	\begin{proof}
		
		因为$Y'\to Y$也是拟紧忠实平坦的,所以不妨设$Y=S$.我们解释过仿射情况下有限型和有限表示态射都可以下降,我们还解释过条件下拟紧和拟分离都可以下降,这就证明了(1)和(2).
		
		\qquad
		
		证明(3):设$f'$是同构,于是它是泛同胚,所以$f$是泛同胚.同理按照$f'$是拟紧和分离的得到$f$是拟紧和分离的.我们只需证明层态射$\theta:\mathscr{O}_Y\to f_*\mathscr{O}_X$是同构.我们有$\theta':\mathscr{O}_{Y'}\to f'_*\mathscr{O}_{X'}$是同构.按照$f$是拟紧和分离的,并且$g$是平坦的,我们解释过有典范同构$g^*(f_*\mathscr{O}_X)\cong f'_*({g'}^*\mathscr{O}_X)=f'_*\mathscr{O}_{X'}$,而$\theta'$是这个典范同构与$g^*\theta:g^*\mathscr{O}_Y\to g^*(f_*\mathscr{O}_X)$的复合.这导致$g^*\theta$是同构,但是$g$是忠实平坦的,于是$\theta$也是同构.
		
		\qquad
		
		证明(4):我们知道一个态射是单态射当且仅当对角态射是同构,并且$\Delta_{f'}$是$\Delta_f$的基变换,所以(3)就得到(4)成立.
		
		\qquad
		
		证明(5):如果$f'$是开嵌入,那么$g^{-1}(f(X))=f'(X')$是开集,按照$g$是拟紧忠实平坦态射,有它是浸没,所以$f(X)$是$Y$的开子集.用$f(X)$和$g^{-1}(f(X))$分别替代$Y$和$Y'$,图表依旧是纤维积图表,按照(3)有$f$视为$X\to f(X)$的态射是同构,这说明$f:X\to Y$是开嵌入.
		
		\qquad
		
		再设$f'$是闭嵌入,我们要证明$f$是闭嵌入,问题是局部的,可设$Y=\mathrm{Spec}A$是仿射的.记$Z\subseteq Y$是$f$的概形像,我们解释过拟紧态射下概形像和平坦基变换可交换,所以$Z'=Z\times_YY'$就是$f'$的概形像.但是按照$f'$本身是闭嵌入,所以概形像的泛性质诱导的$X'\to Z'$就是同构,并且有$X'=X\times_ZZ'$,所以(3)说明$X\to Z$也是同构,也即$f:X\to Y$是闭嵌入.
		
		\qquad
		
		再设$f'$是拟紧嵌入,我们解释过此时$f$是拟紧的.设$Z$是$f$的概形像,按照拟紧条件下概形像和平坦基变换可交换,有$Z'=Z\times_YY'$是$f'$的概形像.我们解释过拟紧嵌入可以经概形像分解成$X'\to Z'\to Y'$,其中$X'\to Z'$是开嵌入,$Z'\to Y'$是闭嵌入.于是按照开嵌入和闭嵌入的下降,我们有$X\to Z$是开嵌入,$Z\to Y$是闭嵌入,于是$f$的嵌入(一般的,嵌入只能分解为前面是闭嵌入后面是开嵌入,不能反过来,但是如果一个态射可以分解为前面是开嵌入,后面是闭嵌入,那它一定是嵌入).
		
		\qquad
		
		(6)没什么好说的,我们已经解释过在条件下,分离,闭映射(从而泛闭,因为$g$满足的性质是基变换不变的),有限型都可以下降.(8)是因为有限态射等价于仿射和紧合态射(仿射情况放在下面),(9)是因为拟有限态射按照定义是有限型并且具有有限纤维,而这两个条件的下降我们已经解释过了.
		
		\qquad
		
		最后只剩证明(7):设$f'$是仿射态射,那么$f'$是qcqs态射,于是$f$也是qcqs态射.记$\mathscr{A}=f_*\mathscr{O}_X$和$\mathscr{A}'=f'_*\mathscr{O}_{X'}$,对于qcqs态射,拟凝聚层的前推仍然是拟凝聚层,于是这两个代数层都是拟凝聚的.另外由于$f$是qcqs态射,$g$是平坦的,我们解释过此时有$g^*f_*\mathscr{O}_X\cong f'_*{g'}^*\mathscr{O}_X$,也即$g^*\mathscr{A}=\mathscr{A}'$.于是我们有如下基变换图表,但是$f'$仿射等价于讲$\alpha'$是同构,所以按照(3)得到$\alpha$也是同构,于是$f$是仿射的.
		$$\xymatrix{X'\ar[rr]^{\alpha'}\ar[d]&&\mathrm{Spec}\mathscr{A}'\ar[rr]\ar[d]&&Y'\ar[d]\\X\ar[rr]_{\alpha}&&\mathrm{Spec}\mathscr{A}\ar[rr]&&Y}$$
	\end{proof}
\end{enumerate}

一些补充.
\begin{enumerate}
	\item 域扩张对应的态射是拟紧忠实平坦的,所以域扩张的基变换下降我们上面给出的所有性质.
	\item 设P和Q是两个关于概形态射的性质,满足如下条件:
	\begin{enumerate}
		\item 都蕴含有限表示.
		\item 【】
	\end{enumerate}
	\item 设$k$是域,$X$是$k$概形,设$k\subseteq k'$是域扩张【】
\end{enumerate}

\subsection{概形性质的下降}

设$f:X\to Y$是态射,概形性质的下降指的是,在约定$f$某些条件下,$Y$的某些性质可以传递给$X$.
\begin{enumerate}
	\item 提升的例子:设$f:X\to Y$是忠实平坦态射,那么它对应的环同态是单射,既约环的子环是既约的,所以如果$X$是既约的,就有$Y$也是既约的.再比如如果态射$f:X\to Y$是满射,如果$X$是不可约的,那么$Y$也是不可约的.
	\item 设$Y$是概形,只有有限个不可约分支,设$f:X\to Y$是平坦态射,如果$Y$是既约的/不可约的/整的,并且$f$的每个一般纤维都是既约的/不可约的/整的,那么有$X$是既约的/不可约的/整的.
	\begin{proof}
		
		先设$Y$是不可约的,它的一般点记作$\eta$,一般纤维记作$X_{\eta}$,约定也是不可约的,那么$Z=\overline{X_{\eta}}$是$X$的不可约闭子集.它的补集$X-Z$是$X$的开集,并且在$Y$中的像不是稠密的(因为要想稠密,必须包含唯一的一般点),这迫使$X-Z=\emptyset$,于是$X=Z$是不可约的.
		
		\qquad
		
		再设$Y$是既约的,设它每个一般点的纤维都是既约的.问题都是局部的,设$Y=\mathrm{Spec}A$和$X=\mathrm{Spec}B$,设$A$的全部极小素理想为$\{\mathfrak{p}_1,\cdots,\mathfrak{p}_n\}$,因为$A$是既约的,所以这些极小素理想的交是零,于是典范映射$A\to\oplus_{i=1}^nA/\mathfrak{p}_i$是单射.记$Y$的对应域$\mathfrak{p}_i$的一般点为$\eta_i$,按照$B$的平坦性,得到单射$B\to\oplus_i(B\otimes_A(A/\mathfrak{p}_i))\subseteq\oplus_i(B\otimes_A\kappa(\eta_i))$.按照条件这里一般纤维$B\otimes_A\kappa(\eta_i)$都是既约的,于是这个有限个环的积也是既约的,于是$B$是既约的.
	\end{proof}
	\item 引理.设$A\to B$是诺特局部环之间的平坦态射,设$k\ge0$是自然数,记号$(R_k)$和$(S_k)$表示Serre条件,我们有:
	\begin{enumerate}
		\item 如果$B$满足$(R_k)$或者$(S_k)$,那么$A$满足相同的性质.
		\item 如果$A$和所有纤维环$B\otimes_A\kappa(\mathfrak{p})$满足$(R_k)$或者$(S_k)$,那么$B$也满足相同的性质.
	\end{enumerate}
	\item 设$f:X\to Y$是局部诺特概形之间的忠实平坦态射,那么P$=(R_k)$或者$(S_k)$满足如下结论:
	\begin{enumerate}
		\item 如果$X$满足P,那么$Y$具有相同的性质.
		\item 如果$Y$和所有纤维$X_y$都满足P,那么$X$也满足相同的性质.
	\end{enumerate}
	
	特别的,P可以取为"正则概形","CM概形","既约概形","正规概形","嵌入点都是一般点",因为它们都是Serre条件的组合.
	\item 推论.设$f:X\to Y$是局部诺特概形之间的光滑态射和满射,那么$X$是既约/正规/正则当且仅当$Y$满足相同的性质.这件事是因为光滑态射的纤维态射是终端为域的光滑态射,从而纤维总是正则概形.
\end{enumerate}
\subsection{拟凝聚层和模层态射的下降}

\begin{enumerate}
	\item 回顾模层和态射的粘合.粘合可以视为一种特殊的下降.
	\begin{enumerate}
		\item 设$S$是概形,设$S=\cup_iU_i$是一个开覆盖,记$S'=\coprod_iU_i$是开覆盖的无交并,那么我们有典范的概形态射$S'\to S$.记$S''=S'\times_SS'=\coprod_{i,j}U_i\cap U_j$和$S'''=S'\times_SS'\times_SS'=\coprod_{i,j,k}U_i\cap U_j\cap U_k$.我们有典范投影态射$p_1,p_2:S''\to S'$,即$p_1$把$U_i\cap U_j$开嵌入到$U_i$,$p_2$把$U_i\cap U_j$开嵌入到$U_j$.我们还有典范的投影态射$p_{12},p_{23},p_{13}:S'''\to S'$.
		\item 我们知道无交并$S'$上的层$\mathscr{F}'$就是一个层的族$\{\mathscr{F}_i'\}$,其中$\mathscr{F}_i'$是$U_i$上的层;$S''$上的层$\mathscr{F}''$就是一个层的族$\{\mathscr{F}_{ij}''\}$,其中$\mathscr{F}_{ij}$是$U_{ij}$上的层.
		\item $S$上关于开覆盖$\{U_i\}$的层的一个粘合信息(gluing datum)就是$S'$上的一个层$\mathscr{F}'$以及一个同构$\varphi:p_1^*\mathscr{F}'\cong p_2^*\mathscr{F}'$,使得它满足余圈条件$p_{23}^*\varphi\circ p_{12}^*\varphi=p_{13}^*\varphi$.这个等式左侧的复合是约定了有如下典范同构:
		$$p_{12}^*(p_2^*\mathscr{F}')\cong(p_2\circ p_{12})^*\mathscr{F}'=(p_1\circ p_{23})^*\mathscr{F}'\cong p_{23}^*(p_1^*\mathscr{F}')$$
		\item 我们用$\textbf{QCoh}(S'/S)$表示$S$上的关于开覆盖$\{U_i\}$的粘合信息构成的范畴,等价于讲$S'$上的满足上述条件的层构成的范畴.那么我们有典范的函子$\textbf{QCoh}(S)\to\textbf{QCoh}(S'/S)$,即把$S$上的拟凝聚层$\mathscr{F}$映为粘合信息$\{j_i^*\mathscr{F}\}$,其中$j_i:U_i\to S$是开嵌入.那么模层可以粘合等价于讲这个函子是本质满的,模层之间的态射可以粘合等价于讲这个函子是完全忠实的.
	\end{enumerate}
	\item 下降信息与粘合函子.
	\begin{enumerate}
		\item 设$p:S'\to S$是概形之间的态射.记$S''=S'\times_SS'$和$S'''=S'\times_SS'\times_SS'$.记投影态射$p_1,p_2:S''\to S'$和$p_{12},p_{13},p_{23}:S'''\to S'$.
		\item 设$\mathscr{F}'$是$\mathscr{O}_{S'}$拟凝聚模层,$\mathscr{F}'$上的一个下降信息(descent datum)指的是一个$\mathscr{O}_{S''}$模层同构$\varphi:p_1^*\mathscr{F}'\cong p_2^*\mathscr{F}'$,满足余圈条件$p_{23}^*\varphi\circ p_{12}^*\varphi=p_{13}^*\varphi$,换句话讲有如下交换图表:
		$$\xymatrix{p_{12}^*p_1^*\mathscr{F}'\ar[rr]^{p_{12}^*\varphi}\ar@{=}[d]&&p_{12}^*p_2^*\mathscr{F}'\cong p_{23}^*p_1^*\mathscr{F}'\ar[rr]^{p_{23}^*\varphi}&&p_{23}^*p_2^*\mathscr{F}'\ar@{=}[d]\\p_{13}^*p_1^*\mathscr{F}'\ar[rrrr]^{p_{13}^*\varphi}&&&&p_{13}^*p_2^*\mathscr{F}'}$$
		\item 两个$\mathscr{O}_{S'}$模层的下降信息之间的态射$(\mathscr{F}',\varphi)\to(\mathscr{G}',\psi)$是指一个$\mathscr{O}_{S'}$模层态射$\alpha:\mathscr{F}'\to\mathscr{G}'$,满足和下降信息兼容,也即满足$p_2^*\alpha\circ\varphi=\psi\circ p_1^*\alpha$.我们用$\textbf{QCoh}(S'/S)$表示$\mathscr{O}_{S'}$模层的下降信息和态射构成的范畴.
		\item 我们有典范的函子$\Phi_{S'/S}:\textbf{QCoh}(S)\to\textbf{QCoh}(S'/S)$,它把$\mathscr{O}_S$的拟凝聚层$\mathscr{F}$映射为回拉$p^*\mathscr{F}$,赋予典范下降信息为$\varphi_{\mathrm{can}}:p_1^*(p^*\mathscr{F})\cong(p\circ p_1)^*\mathscr{F}=(p\circ p_2)^*\mathscr{F}\cong p_2^*(p^*\mathscr{F})$.
		\item 仿射情况.记$S=\mathrm{Spec}R$,$S'=\mathrm{Spec}R'$,$S''=\mathrm{Spec}R''$,其中$R''=R'\otimes_RR'$.一个拟凝聚$\mathscr{O}_{S'}$模层$\mathscr{F}'$对应于$R'$模$M'$.如果记$\pi_1,\pi_2:R'\to R''$分别是环同态$r'\mapsto r'\otimes1$和$r'\mapsto 1\otimes r'$,那么$p_1^*\mathscr{F}'$和$p_2^*\mathscr{F}'$对应的$R'$模分别是$M'\otimes_{R',\pi_1}(R'\otimes_RR')$和$M'\otimes_{R',\pi_2}(R'\otimes_RR')$.如果把前者典范同构于$R'$模$M'\otimes_RR'$,那么后者作为交换群也是$M'\otimes_RR'$,但是模结构是$(a\otimes b)(m'\otimes r')=bm'\otimes ar'$.此时作为阿贝尔群的$M'\otimes_RR'$上的恒等映射未必是$R''$线性的,所以余圈条件不是总成立的.不过如果$\mathscr{F}'$是某个$\mathscr{O}_S$拟凝聚模层的回拉,也即有$R$模$M$使得$M'=M\otimes_RR'$,此时有$R''$模同构$M\otimes_RR'\otimes_RR'\cong M\otimes_RR'\otimes_RR'$为$m\otimes a\otimes b\mapsto m\otimes b\otimes a$,这满足余圈条件,所以是下降信息.
	\end{enumerate}
	\item 我们这一条要证明的是如果$p:S'\to S$是拟紧忠实平坦态射,那么函子$\Phi_{S'/S}:\textbf{QCoh}(S)\to\textbf{QCoh}(S'/S)$是完全忠实的.这涵盖了常规的模层之间态射粘合的情况.
	\begin{enumerate}
		\item 引理.设$R\to R'$是忠实平坦环同态,设$M$是$R$模,对自然数$n\ge0$,我们用$R^{(n)}$表示$n$个$R'$在$R$上的张量积(约定$R^{(0)}=R$),对$0\le i\le n$,取$R$模同态$\varphi_i:M\otimes_RR^{(n)}\to M\otimes_RR^{(n+1)}$为$m\otimes r_1\otimes\cdots\otimes r_n\mapsto m\otimes\cdots\otimes r_i\otimes1\otimes r_{i+1}\otimes\cdots\otimes r_n$.再取$R$模同态$\delta^n:M\otimes_RR^{(n)}\to M\otimes_RR^{(n+1)}$为$x\mapsto\sum_{i=0}^n(-1)^i\varphi_i(x)$.我们断言有如下正合列:
		$$\xymatrix{0\ar[r]&M\ar[r]^{\delta^0}&M\otimes_RR'\ar[r]^{\delta^1}&M\otimes_RR^{(2)}\ar[r]&\cdots}$$
		\begin{proof}
			
			首先我们有$\delta^n\circ\delta^{n-1}=0$,所以这是一个复形.按照$R\to R'$是忠实平坦的,我们只需证明这个复形在张量$R'$后是正合的.第一个同态是$M\otimes_RR'\to M\otimes_RR'\otimes_RR'$,$m\otimes r_1\mapsto m\otimes 1\otimes r_1$,它有左逆同态为$m\otimes r_1\otimes r_2\mapsto m\otimes r_1r_2$,所以它一定是单同态,这说明第一个位置是正合的.我们有:
			$$\delta^n_{R'}=\delta^n\otimes1_{R'}:M\otimes_RR^{(n+1)}\to M\otimes_RR^{(n+2)}$$
			$$m\otimes x_1\otimes\cdots\otimes x_{n+1}\mapsto\sum_{i=0}^n(-1)^i\varphi_i(m\otimes x_1\otimes\cdots\otimes x_n)\otimes x_{n+1}$$
			
			考虑如下$R'$模同态:
			$$\gamma^n:M\otimes_RR^{(n+2)}\to M\otimes_RR^{(n+1)}$$
			$$m\otimes x_0\otimes\cdots\otimes x_{n+1}\mapsto\sum_{i=0}^n(-1)^im\otimes x_1\otimes\cdots\otimes x_ix_{i+1}\otimes\cdots\otimes x_{n+1}$$
			
			可验证$\delta_{R'}^{n-1}\circ\gamma^{n-1}+\gamma^n\circ\delta_{R'}^n$是$M\otimes_RR^{(n)}$上的恒等映射.换句话讲$\gamma$是该复形上恒等同态与零同态之间的同伦映射,这导致该复形是正合的.
		\end{proof}
		\item 下降信息之间的态射$(p^*\mathscr{F},\varphi_{\mathrm{can}}^{\mathscr{F}})\to(p^*\mathscr{F},\varphi_{\mathrm{can}}^{\mathscr{G}})$就是满足余圈条件的模层态射$\alpha:p^*\mathscr{F}\to p^*\mathscr{G}$,余圈条件就是如下图表交换:
		$$\xymatrix{p_1^*(p^*\mathscr{F})\ar[rr]^{\cong}\ar[d]_{p_1^*\alpha}&&p_2^*(p^*\mathscr{F})\ar[d]^{p_2^*\alpha}\\p_1^*(p^*\mathscr{G})\ar[rr]^{\cong}&&p_2^*(p^*\mathscr{G})}$$
		
		如果记$q=p\circ p_1=p\circ p_2$,那么函子$\Phi_{S'/S}$是完全忠实的等价于讲对任意$\mathscr{O}_S$拟凝聚模层$\mathscr{F}$和$\mathscr{G}$都有如下阿贝尔群的正合列,其中中间的态射是$\sigma\mapsto p^*\sigma$,最右侧的态射是$\alpha\mapsto p_1^*\alpha-p_2^*\alpha$(相减的两个态射的源端和终端分别是典范同构的,也即上一个图表).
		$$\xymatrix{0\ar[r]&\mathrm{Hom}_S(\mathscr{F},\mathscr{G})\ar[r]&\mathrm{Hom}_{S'}(p^*\mathscr{F},p^*\mathscr{G})\ar[r]&\mathrm{Hom}_{S''}(q^*\mathscr{F},q^*\mathscr{G})}$$
		\item 设$p:S'\to S$是拟紧忠实平坦态射,我们来证明上一条中的正合列.
		\begin{proof}
			
			首先$\mathrm{Hom}_S(\mathscr{F},\mathscr{G})\to\mathrm{Hom}_{S'}(p^*\mathscr{F},p^*\mathscr{G})$是单射等价于讲如果模层态射$\sigma:\mathscr{F}\to\mathscr{G}$满足$p^*\sigma=0$,那么$\sigma=0$,而这是因为$p$是忠实平坦态射.另外$\mathrm{Hom}_S(\mathscr{F},\mathscr{G})\to\mathrm{Hom}_{S'}(p^*\mathscr{F},p^*\mathscr{G})\to\mathrm{Hom}_{S''}(q^*\mathscr{F},q^*\mathscr{G})$是零映射是因为$p\circ p_1=p\circ p_2$.
			
			\qquad
			
			按照粘合态射是关于Zariski覆盖成立的(此即常规的态射粘合总成立),所以我们不妨设$S$是仿射的,则$S'$也是拟紧的,于是我们有有限仿射开覆盖$S'=\cup_iU_i$.记$T'=\coprod_iU_i$和$T''=T'\times_ST'$.取典范态射$r:T'\to S$,这也是一个拟紧忠实平坦态射.我们有如下交换图表,其中$s$是典范态射$T''\to S$:
			$$\xymatrix{0\ar[r]&\mathrm{Hom}_S(\mathscr{F},\mathscr{G})\ar@{=}[d]\ar[r]&\mathrm{Hom}_{S'}(p^*\mathscr{F},p^*\mathscr{G})\ar[r]\ar[d]&\mathrm{Hom}_{S''}(q^*\mathscr{F},q^*\mathscr{G})\ar[d]\\0\ar[r]&\mathrm{Hom}_S(\mathscr{F},\mathscr{G})\ar[r]&\mathrm{Hom}_{T'}(r^*\mathscr{F},r^*\mathscr{G})\ar[r]&\mathrm{Hom}_{T''}(s^*\mathscr{F},s^*\mathscr{G})}$$
			
			因为$T'\to S'$是忠实平坦的,于是第二个垂直态射是单射,按照图表追踪,从下行的正合性可推出上行的正合性,于是用$T'$替代$S'$,我们不妨设$S'$本身也是仿射的.记$S=\mathrm{Spec}R$和$S'=\mathrm{Spec}R'$,再记$\mathscr{O}_S$拟凝聚模层$\mathscr{F}$和$\mathscr{G}$分别对应于$R$模$M$和$N$.验证正合性的最后一步,也即取$R'$模同态$\varphi':M\otimes_RR'\to N\otimes_RR'$,使得它满足$p_1^*\varphi=p_2^*\varphi$,如果把这个等式的态射记作$\varphi''$,再记$\delta_M:M\otimes_RR'\to M\otimes_RR'\otimes_RR'$为$m\otimes a\mapsto m\otimes1\otimes a-m\otimes a\otimes1$,同理定义$\delta_N$,那么我们有$\varphi''\circ\delta_M=\delta_N\circ\varphi'$,于是$\varphi'(\ker\delta_M)\subseteq\ker\delta_N$.但是按照上述引理,我们有$\ker\delta_M=M$和$\ker\delta_N=N$.于是有$\varphi'(M)\subseteq N$,于是$\varphi$可限制为一个同态$\varphi:M\to N$,它就满足$\varphi\otimes1_{R'}=\varphi'$,这完成了正合性的证明.
		\end{proof}
	\end{enumerate}
	\item 拟凝聚层的下降.我们要证明的是如果$p:S'\to S$是拟紧忠实平坦态射,那么函子$\Phi_{S'/S}:\textbf{QCoh}(S)\to\textbf{QCoh}(S'/S)$是本质满的,结合上一条得到条件下有$\Phi_{S'/S}$是范畴等价.
	\begin{enumerate}
		\item 设$f:T\to S$是概形之间的态射,记基变换为$f':T'=T\times_SS'\to S'$和$f'':T''=(T\times_SS')\times_T(T\times_SS')=T\times_SS''\to S''$.我们有如下交换图表,其中右侧垂直态射是把下降信息$(\mathscr{F}',\varphi)$映射为回拉$({f'}^*\mathscr{F}',{f''}^*\varphi)$:
		$$\xymatrix{\textbf{QCoh}(S)\ar[rr]^{\Phi_{S'/S}}\ar[d]_{f^*}&&\textbf{QCoh}(S'/S)\ar[d]\\\textbf{QCoh}(T)\ar[rr]^{\Phi_{T'/T}}&&\textbf{QCoh}(T'/T)}$$
		\item 设$f':T'\to S'$是概形之间的态射,使得$p\circ f':T'\to S$也是拟紧忠实平坦态射,那么我们有如下交换图表,并且这三个函子都是完全忠实的(于是$p\circ f'$可以理解为"开覆盖"$p:S'\to S$的加细).
		$$\xymatrix{\textbf{QCoh}(S)\ar[rr]\ar[d]&&\textbf{QCoh}(S'/S)\ar[dll]\\\textbf{QCoh}(T'/S)&&}$$
		\begin{proof}
			
			考虑如下交换图表,由于$S'\times_ST'$的投影态射都是拟紧忠实平坦的,于是所有垂直和水平的函子都是完全忠实的,这导致对角线位置的函子是完全忠实的.
			$$\xymatrix{\textbf{QCoh}(S)\ar[rr]\ar[d]&&\textbf{QCoh}(S'/S)\ar[dll]\ar[d]\\\textbf{QCoh}(T/S)\ar[rr]&&\textbf{QCoh}(S'\times_ST'/S)}$$
		\end{proof}
		\item 设$p:S'\to S$是拟紧忠实平坦态射,那么典范函子$\Phi_{S'/S}:\textbf{QCoh}(S)\to\textbf{QCoh}(S'/S)$是范畴等价.我们前面证明了它是完全忠实函子,这里只需证明它是本质满的.
		\begin{proof}
			
			我们先把问题约化到$S$和$S'$均为仿射的情况.任取$S'\to S$的下降信息$(\mathscr{F}',\varphi)$,如果我们取$S$的仿射开覆盖$\{S_i\}$,记$S'_i=S_i\times_SS'$,一旦我们证明了对仿射的$S$是本质满的,有$S_i$上的拟凝聚层$\mathscr{F}_i$对应于$S'_i\to S_i$的下降信息,按照$\Phi$和基变换可交换(也即(a)),有$\{\mathscr{F}_i\}$满足粘合信息,粘合得到的$\mathscr{F}$就是对应于$(\mathscr{F}',\varphi)$的$\mathscr{O}_S$拟凝聚模层.于是问题归结为设$S$是仿射的.那么此时$S'$是拟紧的,选取它的有限仿射开覆盖,把无交并记作$T'$,则$T'\to S'$是拟紧忠实平坦的,按照(b),一旦我们证明了$\Phi_{T'/S}$是本质满的,就有$\Phi(S'/S)$是本质满的(把$\textbf{QCoh}(S'/S)\to\textbf{QCoh}(T'/S)$记作$\alpha$,如果对象$B\in\textbf{QCoh}(S'/S)$,设$\alpha(B)=C$,按照$\Phi_{T'/S}$是本质满的,就有对象$A\in\textbf{QCoh}(S)$使得$\Phi_{T'/S}(A)=C$,于是$\Phi_{S'/S}(A)$和$B$满足在$\alpha$下是同构,但是$\alpha$是完全忠实函子,它是反映同构的,所以从$\alpha(\Phi_{S'/S}(A))$和$\alpha(B)=C$同构得到$\Phi_{S'/S}(A)$和$B$同构,这得到本质满).于是问题归结为设$S'$也是仿射的.
			
			\qquad
			
			设$S=\mathrm{Spec}R$,$S'=\mathrm{Spec}R'$,设一个下降信息为$(\widetilde{M'},\varphi)$,其中$M'$是一个$R'$模,而$\varphi$是一个$R$模同构$M'\otimes_{R'}R^{(2)}\cong M'\otimes_{R'}R^{(2)}$,并且满足余圈条件.这里的两个$R^{(2)}=R'\otimes_RR'$作为$R'$模的结构是不同的(一个是到第一个分量的嵌入,一个是到第二个分量的嵌入).我们要证明的本质满就是指存在$R$模$M$使得$M'\cong M\otimes_RR'$.倘若存在这样的$M$,我们解释过就有$M=\ker(M\otimes_RR'\to M\otimes_RR^{(2)})=\ker(M'\to M\otimes_RR^{(2)})$,这里映射$M'=M\otimes_RR'\to M'\otimes_RR'$是$m\otimes a\mapsto m\otimes1\otimes a-m\otimes a\otimes1=\varphi(m'\otimes1)-m'\otimes1$.所以我们理应取$M'\to M'\otimes_RR'$,$m'\mapsto\varphi(m'\otimes1)-m'\otimes1$的核为$M$.我们把$M\otimes_RR'$上的典范下降信息记作$(M\otimes_RR',\varphi_{\mathrm{can}})$,于是问题归结为证明有下降信息的同构$(M\otimes_RR',\varphi_{\mathrm{can}})\cong(M',\varphi)$.
			
			\qquad
			
			我们要验证的是两个下降信息是同构的,但是(b)中解释了做拟紧忠实平坦基变换后,相应的下降信息范畴之间的函子是完全忠实的,所以验证这两个下降信息同构,等价于验证在某个拟紧忠实平坦基变换后仍然是同构的.由于态射$p$的拟紧忠实平坦基变换$p':S'\times_SS'\to S'$具有截面态射(此即左逆态射)$S'\to S'\times_SS'$,所以我们不妨设$p:S'\to S$本身具有截面态射$s:S\to S'$.再按照(b),要证明$\Phi_{S'/S}$是本质满的只需证明它的加细是本质满的,而这里$S\to S'\to S=1_S$是$p:S'\to S$的加细,但是$\textbf{QCoh}(S)\to\textbf{QCoh}(S/S)$本质满是平凡成立的,这就证明原命题.
		\end{proof}
	\end{enumerate}
\end{enumerate}
\subsection{概形的下降}

模仿拟凝聚层的下降我们来定义概形的下降.
\begin{itemize}
	\item 设$p:S'\to S$是概形之间的态射,设$S''=S'\times_SS'$,典范投影态射记作$p_1$和$p_2$.再设$S'''=S'\times_SS'\times_SS'$,典范投影态射记作$p_{12},p_{23},p_{13}:S'''\to S'$.关于$p$的一个下降信息指的是$(X',\varphi)$,其中$X'$是一个$S'$概形,而$\varphi$是一个$S''$概形同构$X'\times_{S,p_1}S''\cong X'\times_{S',p_2}S''$,满足余圈条件$p_{23}^*\varphi\circ p_{12}^*\varphi=p_{13}^*\varphi$.
	\item 两个关于$p:S'\to S$的下降信息之间的态射$(X',\varphi)\to(Y',\psi)$指的是一个$S'$态射$\alpha:X'\to Y'$,满足和下降信息交换,也即满足$p_2^*\alpha\circ\varphi=\psi\circ p_1^*\alpha$.我们用$\textbf{Sch}(S'/S)$表示$p:S'\to S$的下降信息构成的范畴.我们有典范的函子$p^*:\textbf{Sch}(S)\to\textbf{Sch}(S'/S)$,把$S$概形$X$对应为下降信息$(X',\varphi_{\mathrm{can}})$,其中$X'=X\times_SS'$,而$\varphi_{\mathrm{can}}$是:
	$$(X\times_SS')\times_{S',p_1}S''\cong(X\times_SS')\times_{S',p_2}S''$$
	\item 关于$p:S'\to S$的下降信息$(X',\varphi)$称为有效的(effective),如果它落在函子$p^*$的本质像中(此即那些对象$A$,使得存在$\textbf{Sch}(S)$的对象$X$,满足$p^*(X)\cong A$).
	\item 设$p:S'\to S$是拟紧忠实平坦态射,设$(X',\varphi)$是$p$的一个下降信息,我们称一个开子概形$U'\subseteq X'$在$\varphi$下稳定,如果$\varphi$可以限制为同构$U'\times_{S',p_1}S''\cong U'\times_{S',p_2}S''$(这个同构自动的是$U'$上的下降信息).
\end{itemize}
\begin{enumerate}
	\item 设$p:S'\to S$是拟紧忠实平坦态射,那么函子$p^*:\textbf{Sch}(S)\to\textbf{Sch}(S'/S)$是完全忠实的.
	\begin{proof}
		
		我们要证明的是$S$概形$X,Y$之间的态射要一一对应于相应下降信息之间的态射.但是这些态射都可以在Zariski拓扑下粘合,所以不妨设$S,X,Y$都是仿射的.那么此时$S$态射$X\to Y$就对应于拟凝聚$\mathscr{O}_S$代数层之间的态射,则按照拟凝聚层情况这个函子是完全忠实的,就得到我们这里的$p^*$是完全忠实的.
	\end{proof}
	\item 设$p:S'\to S$是拟紧忠实平坦态射,设$(X',\varphi)$是关于$p$的一个下降信息,如果$X'$存在开覆盖,使得其中每个开子概型都在$S'$上仿射,并且都在$\varphi$下稳定,则$(X',\varphi)$是有效的.
	\begin{proof}
		
		设有开覆盖$X'=\cup U_i'$,其中每个$U_i'$都在$S'$上仿射,并且每个$U_i'$都在$\varphi$下稳定.按照我们证明过拟凝聚模层的情况下函子$\Phi_{S'/S}$是本质满的,就说明这里每个$U_i'$都下降为一个$S$概形$U_i$,换句话讲有$U_i'=U_i\times_SS'$.按照上一条,即概形态射的下降,说明$\{U_i\}$也构成一个粘合信息,所以按照常规概形的粘合,这粘合为一个$S$概形$X$,它就是$X'$的下降.
	\end{proof}
	\item 一般的,即便$p:S'\to S$是拟紧忠实平坦的,也未必有函子$p^*$是本质满的,换句话讲此时$S'$概形未必能下降.不过添加一些条件可以成立:设$p:S'\to S$是拟紧忠实平坦态射.
	\begin{enumerate}
		\item 如果$p$是泛单的,那么$p$的每个下降信息都是有效的.
		\item 如果$p$是局部有限自由态射,并且$X'$在$S'$上拟射影,那么$p$和$X'$的每个下降信息都是有效的.
	\end{enumerate}
\end{enumerate}
\newpage
\section{概形的维数}
\subsection{拓扑维数}

我们要给出的概形上的维数定义只依赖于拓扑.设$X$是拓扑空间.
\begin{itemize}
	\item 定义$X$的组合维数,或者简称维数,为$X$的不可约闭子集严格包含链长度的上确界(这里一个链的长度约定为严格包含号的个数).形式上我们把空集的维数约定为$-\infty$(因为$[-\infty,+\infty]$上空集的上确界为$-\infty$).
	\item 定义点$x\in X$的组合维数,或者称为局部维数或者维数,为$\dim_xX=\inf_U\dim U$,其中$U$跑遍$x$的开邻域.
	\item 明显的空间$X$的维数是它不可约分支的维数的上确界,如果$X$的所有不可约分支的维数都相同,我们就称$X$是等维数的.
\end{itemize}
\begin{enumerate}
	\item 对空间$X$的子空间$Y$,总有$\dim Y\le\dim X$.
	\begin{proof}
		
		取定$Y$中的严格包含的不可约闭集链$Z_0\subseteq Z_1\subseteq\cdots\subseteq Z_n$,记$Z_i$在$X$中的闭包为$\overline{Z_i}$,这得到了$X$中的不可约闭集链$\overline{Z_0}\subseteq\overline{Z_1}\subseteq\cdots\subseteq\overline{Z_n}$(不可约子空间的闭包仍然是不可约的).现在我们断言这是一个严格包含链,因为按照$Z_i$是$Y$中闭集得到$Z_i=Y\cap\overline{Z_i}$,于是倘若某个$\overline{Z_i}=\overline{Z_{i+1}}$,就得到$Z_i=Z_{i+1}$矛盾.于是得到$n\le\dim X$,取上确界得到$\dim Y\le\dim X$.
	\end{proof}
	\item 给定空间$X$的开覆盖$\{U_i,i\in I\}$,那么有$\dim X=\sup_i\dim U_i$;如果$\{F_i\}$是$X$的局部有限的(此为对每个点$x\in X$都能找到开邻域只和其中有限个子集有交)闭子集满足$X=\cup_iF_i$,那么有$\dim X=\sup_i\dim F_i$.
	\begin{proof}
		
		先证明第一个命题,按照上一条得到$\sup_i\dim U_i\le\dim X$.现在取$X$中的不可约闭集链$X_0\subseteq X_1\subseteq\cdots\subseteq X_n$,约定$X_0$是一个单点集.选取开覆盖中覆盖了这个单点的开集$U$,于是$X_0\cap U\subseteq X_1\cap U\subseteq\cdots\subseteq X_n\cap U$是$U$中的不可约闭集链(不可约空间的开子集是不可约的).按照不可约空间的非空开子集总是稠密集,的$\overline{X_i\cap U}=X_i$,于是这个$U$中的链必然是严格包含的链,这就得到$n\le\dim U\le\sup_i\dim U_i$.
		
		\qquad
		
		最后倘若$\dim X=\infty$,那么$X$中存在任意长度的以单点集为最小集合的不可约闭集链,于是$\sup_i\dim U_i=\infty$.倘若$\dim X=n$,此时长度为$n$的不可约闭集链的最小集合必然是单点集,否则可以延拓链的长度,于是得到$\dim X\le\sup_i\dim U_i$.综上得到$\dim X=\sup_i\dim U_i$.
		
		\qquad
		
		证明第二个命题:设$x\in X$的开邻域$U$只和有限个$F_i$有交,记作$\{F_1,\cdots,F_n\}$,那么有$\dim U=\max_{1\le i\le n}\dim(U\cap F_i)\le\max\dim F_i$.让$U$跑遍这样的开覆盖即可.
	\end{proof}
    \item 设$x\in X$,设$U$是$x$的开邻域,设$Y_1,\cdots,Y_n$是$U$的闭子集,满足$x\in$每个$Y_i$,并且$U=\cup_iY_i$.那么有$\dim_xX=\sup_i(\dim_x(Y_i))$.
    \begin{proof}
    	
    	按照上一条,我们有$\dim_xX=\inf_V(\sup_i(\dim Y_i\cap V))$,其中$V$跑遍包含在$U$中的$x$的开邻域.类似的有$\dim_xY_i=\inf_V(\dim(Y_i\cap V))$对任意$i$成立,其中$V$仍然跑遍包含在$U$中的$x$的开邻域.
    	
    	\qquad
    	
    	倘若$\sup_i(\dim_x(Y_i))=+\infty$,这里$i$取值有限,所以存在一个指标$t$使得$\dim_x(Y_t)=+\infty$,于是对任意包含在$U$中的开邻域$V$都有$\dim(Y_t\cap V)=+\infty$,进而$\dim V=+\infty$,于是$\dim_xY_t=+\infty$,于是此时等式成立.
    	
    	\qquad
    	
    	下面设$\sup_i(\dim_x(Y_i))<+\infty$,于是对任意$i$有$\dim_x(Y_i)=\inf_V(\dim(Y_i\cap V))$是有限数,于是对任意$i$可以取包含在$U$中的$x$的开邻域$V_i$使得对任意$V\subseteq V_i$都有$\dim(Y_i\cap V)=\dim_x(Y_i)$.进而取$V_0=\cap V_i$,就有对任意$V\subseteq V_0$和任意$i$都有$\dim(Y_i\cap V)=\dim_x(Y_i)$.由此得到等式.
    \end{proof}
    \item 对$x$的每个开邻域$U$,都有$\dim_xX=\dim_xU$.
    \begin{proof}
    	
    	一方面,每个$U$中的$x$的开邻域自然是$X$中$x$的开邻域,于是$\dim_xX\le\dim_xU$.另一方面任取$x$的开邻域$V$,有$V\cap U\subseteq V$,于是$\dim_xU\le\dim V$对任意$x$的开邻域$V$成立,于是$\dim_xU\le\dim_xX$.
    \end{proof}
    \item 总有$\dim X=\sup_{x\in X}(\dim_xX)$.
    \begin{proof}
    	
    	一方面从$\dim_xX\le\dim X$得到$\sup_{x\in X}\dim X\le\dim X$.另一方面任取$X$中的不可约闭子集的链$X_0\supsetneqq X_1\supsetneqq\cdots\supsetneqq X_l$,任取点$x\in X_l$,任取$x$的开邻域$U$,那么$U\cap X_i,0\le i\le l$是$U$中的不可约闭子集链,它们是严格包含的因为每个$U\cap X_i$在$X$中的闭包是$X_i$.这说明$\dim U\ge l$对任意$x$的开邻域$U$成立,于是$\dim_xU\ge l$,于是$\sup_{x\in X}\dim_xX\ge l$,于是$\sup_{x\in X}\dim_xX\ge\dim X$.
    \end{proof}
    \item 设$X$是拟紧$T_0$空间,设$F$是$X$的闭点集,那么$\dim X=\sup_{x\in F}\dim_xX$.
    \begin{proof}
    	
    	结合上一条,仅需验证$\sup_{x\in X}\dim_xX=\sup_{x\in F}\dim_xX$.一方面自然有$\sup_{x\in F}\dim_xX\le\sup_{x\in X}\dim_xX$.另一方面任取$X$的不可约闭子集链$X_0\supsetneqq X_1\supsetneqq\cdots\supsetneqq X_r$.拟紧空间的闭子集拟紧,于是这些$X_i$都是拟紧的,它们也是$T_0$的,我们知道拟紧$T_0$空间必然有闭点,于是这里$X_r$有闭点$\{x\}$,于是$X_0\supsetneqq X_1\supsetneqq\cdots\supsetneqq X_r\supset\{x\}$,这说明$\sup_{x\in X}\dim_xX\le\sup_{x\in F}\dim_xX$.
    \end{proof}
    \item 设$X$是非空的诺特$T_0$空间,那么$\dim X=0$当且仅当$X$是有限离散空间.
    \begin{proof}
    	
    	充分性是直接的.必要性:任取不可约闭子集$F$,那么$F$是拟紧$T_0$的,所以它有闭点$x$,那么$\{x\}\subseteq F$是不可约闭子集链,按照$\dim X=0$迫使二者相等.最后对于诺特空间不可约分支是有限的,这就导致它是有限离散空间.
    \end{proof}
    \item 推论.设$X$是诺特$T_0$空间,点$x\in X$是孤立点当且仅当$\dim_xX=0$.
    \item 函数$X\to\mathbb{N}\cup\{+\infty\}$,$x\mapsto\dim_xX$是上半连续函数.换句话讲对任意自然数$n$有$\{x\in X\mid\dim_xX\le n\}$是$X$的开子集.
    \begin{proof}
    	
    	任取$x\in X$使得$\dim_xX=n$,那么存在$x$的开邻域$U$使得$\dim U=n$,于是对任意$y\in U$有$\dim_yX=\dim_yU\le n=\dim U=\sup_{y\in U}\dim_yU$.
    \end{proof}  
	\item 如果$X$是有限维的不可约空间,$Y$是它的闭子集,满足$\dim X=\dim Y$,那么必然有$X=Y$.事实上取$Y$中极大长度的不可约闭集链.这个链同样是$X$中的严格包含的不可约闭集链.倘若$Y\not=X$,那么可以对这个链添加不可约闭集$X$延拓长度,这和维数条件矛盾.
	\item 存在诺特空间具有无穷的组合维数:考虑正整数集合,约定全体闭集为$\{1,2,\cdots,n\},\forall n\ge1$.那么每个闭集都是不可约的,并且此时存在任意长度的严格包含的闭集链.
	\item 存在这样的例子,连续像的维数大于源端空间的维数.
\end{enumerate}
\subsection{拓扑余维数}

设$X$是拓扑空间.
\begin{itemize}
	\item 设$Y$是不可约闭子集,$Y$在$X$中的余维数定义为形如$Y_0\supsetneqq Y_1\supsetneqq\cdots\supsetneqq Y_r=Y$的不可约闭子集链长度的上确界.
	\item 如果$Y$是一般的闭子集,它在$X$中的余维数定义为它所有不可约分支在$X$中余维数的下确界.余维数记作$\mathrm{codim}(Y,X)$.按照这个定义有$\mathrm{codim}(\emptyset,X)=+\infty$,因为空集在$[-\infty,+\infty]$中的下确界是$+\infty$.
	\item 称空间$X$是等余维数的,如果它的任意两个极小不可约闭子集在$X$中具有相同的余维数.
	\item 设$Y$是闭子集,设$x\in X$,定义$Y$在$X$中的关于点$x$的(局部)余维数$\mathrm{codim}_x(Y,X)=\sup_U(\mathrm{codim}(Y\cap U,U))$,其中$U$跑遍$x\in X$的开邻域.明显的如果$x\in X-Y$,则$\mathrm{codim}_x(Y,X)=+\infty$.
\end{itemize}
\begin{enumerate}
	\item 设$Y$是空间$X$的闭子集,设它们的不可约分支分别为$\{Y_j\}$和$\{X_i\}$,那么每个$Y_j$都包含在某个$X_i$中,并且以$Y_j$为最小端的不可约闭子集链一定包含在这个$X_i$中,于是我们有如下等式,其中对固定的指标$j$,要求$i$跑遍所有满足$Y_j\subseteq X_i$的指标:
	$$\mathrm{codim}(Y,X)=\inf_j(\mathrm{codim}(Y_j,X))=\inf_j(\sup_i(\mathrm{codim}(Y_j,X_i)))$$
	\item 设$X$是拓扑空间.
	\begin{enumerate}[(1)]
		\item 有$\dim X=\sup_Y(\mathrm{codim}(Y,X))$,其中$Y$跑遍$X$的不可约闭子集.
		\item 对任意非空闭子集$Y\subseteq X$,有$\dim Y+\mathrm{codim}(Y,X)\le\dim X$.
		\item 设$Y_1\subseteq Y_2\subseteq Y_3$是$X$的三个闭子集,那么有:
		$$\mathrm{codim}(Y_1,Y_2)+\mathrm{codim}(Y_2,Y_3)\le\mathrm{codim}(Y_1,Y_3)$$
		\item 设$Y\subseteq X$是闭子集,那么$\mathrm{codim}(Y,X)=0$当且仅当$Y$包含了$X$的某个不可约分支.
	\end{enumerate}
	\begin{proof}
		
		(1)和(4)直接从定义得到.证明(2):如果$\dim Y=+\infty$则$\dim X=+\infty$,此时不等式成立.下设$\dim Y<+\infty$,那么存在$Y$的不可约分支$Y_0$使得$\dim Y=\dim Y_0$,另外按照定义有$\mathrm{codim}(Y,X)\le\mathrm{codim}(Y_0,X)$,并且$\dim Y_0+\mathrm{codim}(Y_0,X)$是$X$的一个不可约闭子集链的长度,于是这个长度$\le\dim X$.最后证明(3):问题可以归结为设$Y_1$是不可约的,也可以约化到$Y_2$是不可约的,此时不等式按照定义成立.
	\end{proof}
	\item 设$Y$是$X$的闭子集,对$X$的任意开集$U$,有$\mathrm{codim}(Y\cap U,U)\ge\mathrm{codim}(Y,X)$,并且这个等式成立当且仅当$\mathrm{codim}(Y,X)=\inf_j(\mathrm{codim}(Y_j,X))$,其中$Y_j$跑遍$Y$的那些和$U$有交的不可约分支.
	\begin{proof}
		
		我们知道$Z\mapsto\overline{Z}$是从$U$的所有不可约闭子集到$X$的所有和$U$有交的不可约闭子集的一一对应,特别的它可以限制为从$Y\cap U$的所有不可约分支到$Y$的所有和$U$有交的不可约分支的一一对应.于是如果取$Y$的和$U$有交的不可约分支$Y_j$,就有$\mathrm{codim}(Y_j,X)=\mathrm{codim}(Y_j\cap U,U)$,于是按照余维数定义得到该不等式.
	\end{proof}
    \item 设$Y_1,\cdots,Y_n$是$X$的有限个闭子集,设$Y=\cup_iY_i$,那么有:
    $$\mathrm{codim}(Y,X)=\inf_i(\mathrm{codim}(Y_i,X))$$
    \begin{proof}
    	
    	我们知道$Y_i$的不可约分支一定包含在$Y$的某个不可约分支中,并且反过来$Y$的不可约分支按照定义不能写成两个更小闭子集的并,导致它一定是某个$Y_i$的不可约分支.于是按照定义得到这个等式.
    \end{proof}
    \item 设$Y\subseteq X$是一个局部诺特的闭子集.
    \begin{enumerate}[(1)]
    	\item 对任意$x\in X$,只存在$Y$的有限个不可约分支$\{Y_1,\cdots,Y_n\}$包含了$x$,并且有:
    	$$\mathrm{codim}_x(Y,X)=\inf_i(\mathrm{codim}(Y_i,X))$$
    	\item 函数$x\mapsto\mathrm{codim}_x(Y,X)$是$X$上的下半连续函数.
    \end{enumerate}
    \begin{proof}
    	
    	存在$x\in X$的开邻域$U_0$使得$Y\cap U_0$是诺特空间,它只有有限个不可约分支,于是$Y$只有有限个不可约分支和$U_0$有交,于是$Y$只有有限个不可约分支$\{Y_1,\cdots,Y_n\}$包含了$x$.我们可以取$x$的开邻域$U\subseteq U_0$,使得$U$不和那些不包含$x$的$Y$的不可约分支有交.于是此时$Y_i\cap U$是$Y\cap U$的全部不可约分支.进而对$x\in X$的任意包含在$U$中的开邻域$V$,有$Y_i\cap V$是$Y\cap V$的不可约分支.这导致$\mathrm{codim}(Y_i,X)=\mathrm{codim}(Y_i\cap V,V)$,这证明了(1).最后对任意$x'\in U$,有$Y$的包含了$x'$的不可约分支一定是某个$Y_i$,由(1)这导致$\mathrm{codim}_{x'}(Y,X)\ge\mathrm{codim}_x(Y,X)$,这得到下半连续性.
    \end{proof}
\end{enumerate}
\subsection{链条件}

拓扑空间$X$上的一个不可约闭子集链$Z_0\subsetneqq Z_1\subsetneqq\cdots\subsetneqq Z_n$称为饱和的(saturated),如果对任意$0\le k\le n-1$都不存在不可约闭子集$Z$满足$Z_k\subsetneqq Z\subsetneqq Z_{k+1}$.
\begin{enumerate}
	\item 设空间$X$满足对任意两个不可约闭子集$Y\subseteq Z$,都有$\mathrm{codim}(Y,Z)<+\infty$.那么如下两个条件等价,在条件成立时我们称空间是Catenary空间.
	\begin{enumerate}[(1)]
		\item 任意两个具有相同最大端和最小端的饱和不可约闭子集链具有相同长度.
		\item 对任意不可约闭子集链$Y_1\subseteq Y_2\subseteq Y_3$,都有:
		$$\mathrm{codim}(Y_1,Y_3)=\mathrm{codim}(Y_1,Y_2)+\mathrm{codim}(Y_2,Y_3)$$
	\end{enumerate}
    \begin{proof}
    	
    	(1)推(2)是直接的.假设(2)成立,我们要证明的是两个具有相同最大端和最小端的饱和不可约闭子集链的长度如果记作$n\le m$,那么$n=m$.设$n$对应的不可约闭子集链是$Z_0\subsetneqq\cdots\subsetneqq Z_n$,那么也存在以$Z_0,Z_n$分别为最小端和最大端的饱和不可约闭子集链,于是$\mathrm{codim}(Z_0,Z_n)\ge m$.但是按照饱和条件有$\mathrm{codim}(Z_i,Z_{i+1})=1$,于是条件(2)得到$\mathrm{codim}(Z_0,Z_n)=n$,于是$m=n$.
    \end{proof}
    \item 定义.
    \begin{itemize}
    	\item 称空间$X$是弱双等维数空间,如果它是等维数,等余维数,catenary空间.
    	\item 称空间$X$是双等维数空间,如果它的所有由不可约闭子集构成的极大链都具有相同长度.
    \end{itemize}
    \item 双等维数空间一定是弱双等维数空间,反过来未必成立.
    \item 如下命题互相等价:
    \begin{enumerate}[(1)]
    	\item $X$是双等维数空间.
    	\item $X$是等维数,catenary空间,并且任意不可约分支都是等余维数空间.
    	\item $X$是等维数的,并且任意两个不可约闭子集$Y_1\subseteq Y_2$都满足:
    	$$\dim Y_2=\dim Y_1+\mathrm{codim}(Y_1,Y_2)$$
    	\item $X$是等余维数的,并且任意两个不可约闭子集$Y_1\subseteq Y_2$都满足:
    	$$\mathrm{codim}(Y_1,X)=\mathrm{codim}(Y_1,Y_2)+\mathrm{codim}(Y_2,X)$$
    \end{enumerate}
    \item 设$X$是双等维数空间,对任意闭点$x\in X$和$X$的任意不可约分支$Z$,都有:
    $$\dim X=\dim Z=\mathrm{codim}(\{x\},X)=\dim_xX$$
    \begin{proof}
    	
    	前两个等号是直接的.取极大长度的不可约闭子集链$Y_0=\{x\}\subseteq Y_1\subseteq\cdots\subseteq Y_m$.取$x$的开邻域$U$,那么$U\cap Y_i$是两两不同的,因为$\overline{U\cap Y_i}=Y_i$.于是$\dim U=\dim X$.
    \end{proof}
    \item 设$X$是双等维数空间,那么$X$的某些不可约分支的并,$X$的任意不可约闭子空间都是双等维数空间,并且对任意闭子集$Y\subseteq X$都有$\dim Y+\mathrm{codim}(Y,X)=\dim X$.
    \begin{proof}
    	
    	$X$的不可约闭子集是双等维数空间是因为满足双等维数空间等价描述中的(3).最后任取非空闭子集$Y\subseteq X$,它的不可约分支记作$\{Y_1,\cdots,Y_m\}$,如果$Y_t$满足$\dim Y=\dim Y_t$,或者说$Y$存在一个极大长度的不可约闭子集链以$Y_t$为最大端,那么按照$X$是双等维数空间,就有$Y_t$也是$\mathrm{codim}(Y_i,X)$取到最小值的$i=t$,这就得到$\dim Y+\mathrm{codim}(Y,X)=\dim X$.
    \end{proof}
\end{enumerate}
\subsection{概形的维数}

概形的维数定义为它作为拓扑空间的组合维数.
\begin{enumerate}
	\item 仿射概型的维数.
	\begin{enumerate}[(1)]
		\item 仿射概形$\mathrm{Spec}A$的维数就是$A$的Krull维数.特别的,环$A$和$A/\mathrm{nil}(A)$的维数相同.
		\item 对$\mathrm{Spec}A$的闭子集$V(I)$,其中$I\subseteq A$是理想,有:
		$$\dim(V(I))=\dim(A/I)$$
		$$\mathrm{codim}(V(I),\mathrm{Spec}A)=\mathrm{ht}(I)$$
		
		其中一个理想$I$的高度$\mathrm{ht}(I)$定义为它全部极小素理想的高度的下确界,素理想$\mathfrak{p}$的高度定义为$\dim A_{\mathfrak{p}}$.
		\item $\mathrm{Spec}A$是catenary空间当且仅当$A$是catenary环.
		\item $\mathrm{Spec}A$是等维数空间当且仅当$A$是等维数环,也即对$A$的任意极小素理想$\mathfrak{p}$,都有$\dim A/\mathfrak{p}$相同.
		\item $\mathrm{Spec}A$是等余维数空间当且仅当$A$是等余维数环,也即它的所有极大理想具有相同高度.
		\item 称$A$是双等维数环,如果$\mathrm{Spec}A$是双等维数空间.
	\end{enumerate}
	\item 设$X$是概形,设$Y\subseteq X$是不可约闭子集,它的一般点记作$y$,那么有$\mathrm{codim}(Y,X)=\dim\mathscr{O}_{X,y}$.并且:
	\begin{enumerate}[(1)]
		\item 满足$\dim\mathscr{O}_{X,x}=0$的点$x$恰好是$X$不可约分支的一般点.
		\item 对概形$X$的闭子集$Y$,有:
		$$\mathrm{codim}(Y,X)=\inf_{y\in Y}\dim\mathscr{O}_{X,y}$$
		\item 如果$X$是局部诺特概形,那么对任意$x\in Y$有:
		$$\mathrm{codim}_x(Y,X)=\inf_{y\in Y,x\in\overline{\{y\}}}\dim\mathscr{O}_{X,y}$$
	\end{enumerate}
	\begin{proof}
		
		这件事是因为$X$的过点$y$的不可约闭子集是保序一一对应于$\mathrm{Spec}\mathscr{O}_{X,y}$的.
	\end{proof}
	\item 设$X$是概型,那么:
	$$\dim X=\sup_{x\in X}\dim\mathscr{O}_{X,x}$$
	
	如果概形$X$满足任意不可约闭子集都包含闭点(例如拟紧概形),记$F$是闭点集,那么:
	$$\dim X=\sup_{x\in F}\dim\mathscr{O}_{X,x}$$
	\begin{proof}
		
		对$X$取仿射开覆盖,可把问题归结为$X$是仿射概型$\mathrm{Spec}(A)$的情况.但是此时$\mathscr{O}_{X,x}$的维数是$A$中以$\mathfrak{p}_x$(这个记号是$x$在素谱中对应的素理想)为最大端的素理想升链长度的上确界,它对$x$取上确界即环的维数.
	\end{proof}
    \item 概形$X$是catenary空间当且仅当对任意$x\in X$有$\mathscr{O}_{X,x}$是catenary环.如果额外的$X$还满足任意不可约闭子集都包含闭点,那么只要对全部闭点$x\in X$有$\mathscr{O}_{X,x}$是catenary环,就有$X$是catenary空间.另外catenary和双等维数差的很远,存在这样的正则整环$A$(从而它的素谱是catenary的),它存在不同的极大理想高度不同.
    \begin{proof}
    	
    	依旧是因为$X$的过点$y$的不可约闭子集是保序一一对应于$\mathrm{Spec}\mathscr{O}_{X,y}$的.
    \end{proof}
    \item 如果概形$X$是双等维数空间,那么对任意闭点$x$有$\mathscr{O}_{X,x}$是catenary的并且等维数$\dim X$的.如果$X$的任意非空闭子集都包含闭点(例如拟紧$T_0$空间或者局部诺特概形),那么逆命题成立.
    \begin{proof}
    	
    	设$X$是双等维数空间,我们解释过它是catenary空间,于是$\mathscr{O}_{X,x}$都是catenary环.设$x\in X$是闭点,那么$X$的包含点$x$的不可约闭子集逆变保序的一一对应于$\mathrm{Spec}\mathscr{O}_{X,x}$的素理想,于是有$\dim\mathscr{O}_{X,x}=\mathrm{codim}(\{x\},X)$.任取包含$x$的$X$的不可约分支$Z$,那么$\{x\}$和$Z$为两端的饱和链的长度是$\dim X$.
    	
    	\qquad
    	
    	反过来,任取$X$的极大长度的不可约闭子集链$X_0\subsetneqq\cdots\subsetneqq X_k$,那么$X_0=\{x\}$,于是这个链对应于$\mathrm{Spec}\mathscr{O}_{X,x}$的素理想链.因为$\mathscr{O}_{X,x}$是catenary和等维数的,就有$k=\dim\mathscr{O}_{X,x}$,于是$X$的任意极大长度的链都具有长度$\dim X$.
    \end{proof}
    \item 我们知道局部诺特环的维数有限,并且就是定义理想的最小生成元个数,于是有:设$X$是局部诺特概形,设$Y$是非空闭子集,那么$\mathrm{codim}(Y,X)$总是有限数.如果$X$是诺特仿射概形,$Y$是一个不可约闭子集,那么$\mathrm{codim}(Y,X)$是最小的整体截面个数$\{s_i\}$,使得$Y$是由$\{s_i(x)=0,\forall i\}$定义的子集的一个不可约分支.
    \item 推论.设$X$是局部诺特概形,设$\mathscr{L}$是$\mathscr{O}_X$可逆层,设$f$是$\mathscr{L}$的一个整体截面,设$Z$是由$f(x)=0$定义的子集.那么$Z$的任意不可约分支具有余维数$\le1$,并且如果$Z$不包含$X$的不可约分支则它们的余维数恰好为1.
    \begin{proof}
    	
    	归结为设$\mathscr{L}=\mathscr{O}_X$.设$y$是$Z$的某个不可约分支$Y$的一般点,那么$\mathscr{O}_{X,y}$的理想$(f_y)$就要满足$\mathscr{O}_{X,y}/(f_y)$只有单个素理想,于是$f_y$生成了$\mathscr{O}_{X,y}$的定义理想,按照上一条就有$\mathrm{codim}(Y,X)\le1$.如果$Z$不包含$X$的不可约分支,$Z$的余维数不能为0,所以此时$\mathrm{codim}(Y,X)=1$
    \end{proof}
    \item 设$X$是局部诺特概形,设$Y\subseteq X$是闭子集,设$x\in Y$使得$\mathscr{O}_{X,x}$是catenary环,那么:
    \begin{align*}
    	\mathrm{codim}_x(Y,X)&=\dim(\mathscr{O}_{X,x})-\mathrm{codim}(\overline{\{x\}},Y)\\&=\dim(\mathscr{O}_{X,x})-\dim(\mathscr{O}_{Y,x})
    \end{align*}
    \begin{proof}
    	
    	设$\{Y_1,\cdots,Y_n\}$是$Y$的那些包含$x$的不可约分支,有限性源自于$\mathscr{O}_{Y,y}$是诺特的.设$Y_i$的一般点为$y_i$.设$A=\mathscr{O}_{X,x}$,设$Y_i\cap\mathrm{Spec}A$对应的$A$的素理想为$\mathfrak{p}_i$.那么$Y$的包含$x$的不可约闭子集保序的一一对应于$A$的包含某个$\mathfrak{p}_i$的素理想.于是有$\dim\mathscr{O}_{Y,x}=\sup_i\dim(A/\mathfrak{p}_i)$.这里$\mathscr{O}_{X,y_i}=A_{\mathfrak{p}_i}$.并且按照$A$是catenary环有$\dim A=\dim(A/\mathfrak{p}_i)+\dim A_{\mathfrak{p}_i}$.结合$\mathrm{codim}_x(Y,X)=\inf_x\dim(\mathscr{O}_{X,y_i})$得到结论.
    \end{proof}
    \item 
    \begin{enumerate}[(1)]
    	\item 设拓扑空间$X$满足,对任意非空的局部闭子集$Z$,都存在局部闭子集$Z'\subseteq Z$,以及一个$Z'$的闭点$x\in Z'$.那么$X$的任意非空局部闭子集$Z$都包含了点$x$使得$x$是$\overline{\{x\}}$的单点开集.
    	\item 设$X$是概形,设$E$是非空的局部可构造子集,那么$E$总包含了一个点$x$,使得$\{x\}$是$X$的局部闭子集(这也等价于讲$x$是$\overline{\{x\}}$的单点开集).
    	\item 设$X$是局部诺特概形,设$x\in X$使得$\{x\}$是$X$的局部闭子集.那么$\dim\overline{\{x\}}\le1$,进而$\overline{\{x\}}$的任意一个异于$x$的点都是$X$的闭点.
    \end{enumerate}
    \begin{proof}
    	
    	(1):设$Z'\subseteq Z$是局部闭子集,并且$x\in Z'$是$Z'$的闭点.设$x$在$X$中的开邻域$U$使得$Z'\cap U$在$U$中闭,那么$x$也是$U$的闭子集.于是$U\cap\overline{\{x\}}=\{x\}$,也即$x$是$\overline{\{x\}}$的单点开集.
    	
    	\qquad
    	
    	(2):这是因为任意概形满足(1)中的条件:因为$Z$具备子概型结构,所以可以取$Z'$是$Z$的仿射开子集,那么$Z'$是拟紧$T_0$空间,所以包含闭点.
    	
    	\qquad
    	
    	(3):设以$\overline{\{x\}}$为底空间的$X$的唯一既约闭子概型为$Z$.条件要求$\{x\}$在$Z$中开.于是对任意$z\in Z$,都有$\mathrm{Spec}\mathscr{O}_{Z,z}$的一般点$x$在$\mathrm{Spec}\mathscr{O}_{Z,z}$中是单点开集.环$A=\mathscr{O}_{Z,z}$是局部诺特整环,迫使存在$f\in A$使得$A_f$是域.于是$\dim A\le1$,换句话讲对任意$z\in Z$有$\dim\mathscr{O}_{Z,z}\le1$,于是$\dim Z\le1$.
    \end{proof}
    \item 推论.设$X$是局部诺特概形,那么任意非空闭子集都包含闭点.
    \begin{proof}
    	
    	任取非空闭子集$E$,按照上一条的(2),存在$x\in E$使得$\{x\}$是$X$的局部闭子集.再按照(3)有$\overline{\{x\}}$中异于$x$的点都是$X$的闭点.这里$\overline{\{x\}}\subseteq E$,如果$\overline{\{x\}}=\{x\}$那么$x$本身就是闭点.
    \end{proof}
    \item 设$X$是概形,设$\mathscr{F}$是有限生成拟凝聚$\mathscr{O}_X$模层,那么我们解释过它的支集$S=\mathrm{Supp}(\mathscr{F})$是闭子集.对任意$x\in X$,把$\mathrm{Supp}\mathscr{F}_x$视为$\mathrm{Spec}\mathscr{O}_{X,x}$的子空间,那么有$\dim\mathscr{F}_x=\dim\mathrm{Supp}\mathscr{F}_x$.另一方面,赋予$S$任意$X$的闭子概型结构,总有$\mathrm{Supp}(\mathscr{F}_x)=S\cap\mathrm{Spec}\mathscr{O}_{X,x}=\mathrm{Spec}\mathscr{O}_{S,x}$.另外$\mathrm{Spec}\mathscr{O}_{S,x}$的不可约分支保序的一一对应于$S$的包含$x$的不可约分支(此即$S$的包含$x$的不可约分支和$\mathrm{Spec}\mathscr{O}_{X,x}$的交).于是我们有:
    $$\dim(\mathscr{F}_x)=\dim\mathscr{O}_{S,x}=\mathrm{codim}(\overline{\{x\}},S)$$
    \item 设$X$是概形,设$\mathscr{F}$是有限生成拟凝聚层.
    \begin{itemize}
    	\item 称$\mathscr{F}$在$x\in X$等维数,如果$\mathscr{F}_x$作为$\mathscr{O}_{X,x}$模是等维数的,此即$\mathrm{Spec}\mathscr{O}_{X,x}$的闭子空间$\mathrm{Supp}(\mathscr{F}_x)$是等维数的,也等价于讲$\mathscr{O}_{S,x}$是等维数的.
    	\item 定义$\mathscr{F}$的维数是它支集$\mathrm{Supp}(\mathscr{F})$的维数,记作$\dim\mathscr{F}$.
    \end{itemize}
    \item 按照定义有$\dim\mathscr{F}=\sup_{x\in X}\dim\mathscr{F}_x$.另外如果$X=\mathrm{Spec}A$是仿射的,$\mathscr{F}=\widetilde{M}$,其中$M$是有限$A$模.那么$\dim\mathscr{F}=\dim M$和模的维数一致.并且如果$x'$是$x$的一般化,就有$\dim\mathscr{F}_{x'}\le\mathscr{F}_x$,因为:
    $$\dim\mathscr{F}_{x'}=\dim\mathscr{O}_{S,x'}\le\dim\mathscr{O}_{S,x}=\dim\mathscr{F}_x$$
\end{enumerate}
\subsection{域上有限型概形的维数}

设$k$是域.
\begin{enumerate}
	\item 设$X/k$是不可约局部有限型概形,设$\xi$是一般点.那么$X$是双等维数空间,并且有$\dim(X)=\mathrm{tr.deg}_k\kappa(\xi)$.
	\begin{proof}
		
		对任意$x\in X$,有$\mathscr{O}_{X,x}$是有限型$k$代数,那么它是某个正则局部环的商,于是它是catenary环,于是$X$是catenary空间.为证明$\dim(X)=\mathrm{tr.deg}_k\kappa(\xi)$,归结为设$X$是整概形.任取闭点$x\in X$,只需验证$\dim\mathscr{O}_{X,x}=\mathrm{tr.deg}_k\kappa(\xi)$.这也归结为证明$\dim A=\mathrm{tr.deg}_kK$,其中$A$是有限型整$k$代数,商域为$K$,此为诺特正规化引理.再按照$X$是不可约的,它是等维数的,$\dim\mathscr{O}_{X,x}=\mathrm{tr.deg}_k\kappa(\xi)$说明它是等余维数的,于是$X$是双等维数空间.
	\end{proof}
    \item 设$X/k$是等维数的局部有限型概形,那么$X$是双等维数空间.
    \begin{proof}
    	
    	我们解释过空间$X$是双等维数的等价于它是等维数,catenary,任意不可约分支是等余维数的.它是catenary的因为它的局部环都是域$k$上有限型代数的商或者局部化.问题归结为证明它的不可约分支一定是等余维数的,但是这归结为上一条.
    \end{proof}
	\item 设$X/k$是局部有限型概形,那么$\dim(X)=\sup_x\mathrm{tr.deg}_k\kappa(x)$.
	\begin{proof}
		
		设$\{X_i\}$是$X$的全部不可约分支,赋予既约闭子概型结构,那么$X_i$是不可约局部有限型概形,于是$\dim X_i=\sup_{x\in X_i}\mathrm{tr.deg}_k\kappa(x)$.进而有$\dim(X)=\sup_i\dim(X_i)=\sup_{x\in X}\mathrm{tr.deg}_k\kappa(x)$.
	\end{proof}
    \item 推论.设$U\subseteq X$是局部有限型$k$概形的稠密开子集,那么$\dim U=\dim X$.这里局部有限型是必须的,存在这样的正则诺特双等维数概形$X$,使得它存在稠密开子集$U$维数和$X$不同,例如取DVR的素谱$X=\{s,\eta\}$,其中$s$是唯一闭点,$\eta$是唯一一般点,那么$\dim X=1$,但是$\dim\{s\}=0$.
    \begin{proof}
    	
    	这件事是因为局部诺特概形上一个开子集是稠密的当且仅当包含全部一般点,并且$U$作为$X$的开子集还是局部诺特的.
    \end{proof}
	\item 设$f:X\to Y$是域$k$上局部有限型概形之间的$k$态射.
	\begin{enumerate}[(1)]
		\item 如果$f$是拟紧和支配的,那么$\dim Y\le\dim X$.
		\item 如果$f$是拟有限的,那么$\dim X\le\dim Y$.
		\item 如果$X/k$是有限型概形,那么$\dim (X)\ge n$/$\dim(X)\le n$/$\dim X=n$当且仅当存在稠密开子集$U\subseteq X$以及一个满射/有限/满射有限$k$态射$g:U\to\mathbb{A}_k^n$.另外如果$X$是仿射的,这里的$U$可以直接取为$X$本身.
	\end{enumerate}
	\begin{proof}
		
		(1):设$y\in Y$是一般点,拟紧支配条件保证了$f^{-1}(y)$一定包含$X$的一般点$x$.于是有域扩张$\kappa(x)/\kappa(y)$,于是有$\mathrm{tr.deg}_k\kappa(y)\le\mathrm{tr.deg}_k\kappa(x)\le\dim(X)$.让$y$取遍一般点得到$\dim Y\le\dim X$.
		
		\qquad
		
		(2):设$x$是$X$的一般点,按照$f$是拟有限的,有$\kappa(x)/\kappa(f(x))$是有限扩张,所以它们在$k$上具有相同的超越维数.赋予$\overline{\{f(x)\}}$既约闭子概型结构,那么有$\mathrm{tr.deg}_k\kappa(x)=\mathrm{tr.deg}_k\kappa(f(x))\le\dim Y$.
		
		\qquad
		
		(3):充分性源自(1)和(2).下面证明必要性.设$\{U_i\}$是$X$的全部有限个一般点的仿射开邻域,那么它们必然两两不交,把并记作$U$,那么这是$X$的稠密开子集.并且如果$X$是仿射的可以直接取$U=X$.设$\dim U_i=n_i$,那么存在有限满射$k$态射$g_i:U_i\to\mathbb{A}_k^{n_i}$.那么$n'=\dim X=\max_in_i$.记$k[T_1,\cdots,T_{n'}]\to k[T_1,\cdots,T_{n_i}]$对应的$k$态射为$h_i:\mathbb{A}_k^{n_i}\to\mathbb{A}_k^{n'}$.满足$n_i=n'$的那些指标$i$就有$h_i$是恒等.取$g_i$的和为$g:U\to\mathbb{A}_k^{n'}$,在分量上是复合$U_i\to\mathbb{A}_k^{n_i}\to\mathbb{A}_k^{n'}$.那么$g$是有限满射.这就解决了$n'=n$的情况.如果$n\ge n'$,再对$g$复合上闭嵌入$\mathbb{A}_k^{n'}\to\mathbb{A}_k^n$仍然是有限态射;如果$n\le n'$,再对$g$复合上$k[T_1,\cdots,T_n]\to k[T_1,\cdots,T_{n'}]$诱导的忠实平坦态射$\mathbb{A}_k^{n'}\to\mathbb{A}_k^n$仍然是满射.
	\end{proof}
	\item 关于基变换.
	\begin{enumerate}[(1)]
		\item 设$X$是局部有限型$k$概型,设$K$是$k$的域扩张,那么基变换$X_K=X\otimes_kK$是$K$上的概型,我们有$\dim X=\dim X_K$.
		\begin{proof}
			
			取$X$的仿射开覆盖$\{U_i\}$,那么$U_i\times_kK$是$X\times_kK$的开覆盖,于是同样的仅需验证仿射情况.设$\dim X=n$,设$X=\mathrm{Spec}(A)$,那么存在有限满射态射$X\to\mathbb{A}_k^n$,基变换到$K$上得到有限满射态射$X_K\to\mathbb{A}_K^n$,于是$\dim X_K=n$.
		\end{proof}
		\item 设$X,Y$是非空的局部有限型$k$概型,那么$\dim X\times_kY=\dim X+\dim Y$.
		\begin{proof}
			
			取$X$和$Y$的仿射开覆盖$\{U_i\}_{i\in I}$和$\{V_j\}_{j\in J}$.那么$\{U_i\times_kV_j\}_{i\in I,j\in J}$构成了$X\times_kY$的开覆盖.按照维数是它开覆盖每个开集维数的上确界,说明我们仅需验证仿射情况.
			
			设$\dim X=m$和$\dim Y=n$.按照诺特正规化引理,存在两个单的整同态$k[T_1,T_2,\cdots,T_m]\to\mathscr{O}_X(X)$和$k[T_{m+1},T_{m+2},\cdots,T_{m+n}]\to\mathscr{O}_Y(Y)$,于是它们诱导了张量积之间的如下单的整同态,这说明$\dim X\times_kY=\dim X+\dim Y$.
			$$k[T_1,T_2\cdots,T_{m+n}]=k[T_1,T_2,\cdots,T_m]\otimes_kk[T_{m+1},T_{m+2},\cdots,T_{m+n}]$$
			$$\to\mathscr{O}_X(X)\otimes_k\mathscr{O}_Y(Y)=\Gamma(X\times_kY,\mathscr{O}_{X\times Y})$$
		\end{proof}
		\item 设$X$是域$k$上的概形,设$K/k$是代数扩张,那么$K$概形$X_K=X\otimes_kK$满足$\dim X=\dim X_K$.
		\begin{proof}
			
			同样取$X$的仿射开覆盖可以归结为$X=\mathrm{Spec}A$是仿射的情况.那么归结为证明$A\to A\otimes_kK$是整的同态,因为环的整扩张是保维数的.但是由于$K/k$是代数扩张,每个形如$1_A\otimes x$的元都在$A$上整,进而$A\otimes_kK$在$A$上整.
		\end{proof}
	\end{enumerate}
    \item 设$X/k$是局部有限型概形,设$x\in X$,那么有:
    $$\dim_x(X)=\dim\mathscr{O}_{X,x}+\mathrm{tr.deg}_k\kappa(x)$$
    
    特别的,如果$x\in X$是闭点,就有$\dim_x(X)=\dim\mathscr{O}_{X,x}$.
    \begin{proof}
    	
    	取$x$的开邻域$U$使得$\dim_x(X)=\dim U$.不妨设$U$的不可约分支恰好是那些$X_i\cap U_i$,其中$X_i$是$X$的那些包含$x$的不可约分支(比方说,先取$x$的仿射开邻域$V$,那么$\dim_xX=\dim_xV$,再取$U\subseteq V$包含$x$使得$\dim_xV=\dim U$,这里$V$只有有限个不可约分支,那些不包含$x$的不可约分支的补集是开集,所以可以不妨把$U$替换为这个开子集).按照$\dim X_i=\dim(U\cap X_i)$,于是$\dim_xX=\sup_i\dim(X_i)$.按照$\mathscr{O}_{X,x}$的极小素理想对应于$X_i$的一般点,我们有$\dim\mathscr{O}_{X,x}=\sup\dim\mathscr{O}_{X_i,x}$.于是问题归结为设$X$是不可约的,那么$X$是双等维数空间,于是$\dim X=\dim\overline\{x\}+\mathrm{codim}(\overline{\{x\}},X)$.我们证明过$\dim\overline{\{x\}}=\mathrm{tr.deg}_k\kappa(x)$和$\mathrm{codim}(\overline{\{x\}},X)=\dim\mathscr{O}_{X,x}$,得证.
    \end{proof}
    \item 设$X/k$是局部有限型概形,设$\mathscr{L}$是可逆层,设$f$是$\mathscr{L}$的一个整体截面,设$Y$是$f(x)=0$定义的闭子集,使得$Y$在$X$中无处稠密.那么有$\dim Y\le\dim X-1$,并且如果$Y$和$X$的每个维数恰好为$\dim X$的不可约分支有交,那么$\dim Y=\dim X-1$.
    \begin{proof}
    	
    	设$\{X_i\}$是$X$的全部不可约分支,我们有$\dim Y=\sup_i\dim(Y\cap X_i)$(因为$\{Y\cap X_i\}$是$Y$的局部有限的闭集族).并且按照$X_i$总包含$X$的开子集,就导致$Y\cap X_i$在$X_i$中无处稠密.于是不妨用$X_i$和$Y\cap X_i$替代$X$和$Y$,归结为设$X$本身是不可约的,并且可设$Y$非空.任取$Y$的一般点$x$,按照$X$是双等维数空间,就有$\dim\overline{\{x\}}=\dim X-\mathrm{codim}(\overline{\{x\}},X)$.我们之前已经证明过$\mathrm{codim}(\overline{\{x\}},X)=1$,得证.
    \end{proof}
\end{enumerate}
\subsection{态射像的维数}
\begin{enumerate}
	\item 设$f:X\to Y$是局部诺特概形之间的态射.
	\begin{enumerate}[(1)]
		\item 如果$f$是拟有限的,那么$\dim X\le\dim\overline{f(X)}\le\dim Y$.
		\item 如果$f$是开满射态射或者闭满射态射,那么$\dim X\ge\dim Y$.
	\end{enumerate}
    \begin{proof}
    	
    	(1):把$f$替换为$f_{\mathrm{red}}$,可不妨设$X,Y$都是既约的.设以$\overline{f(X)}$为底空间的$Y$的既约闭子概型为$Z$,闭嵌入记作$j:Z\to Y$,那么有分解$f=j\circ g$,其中$g:X\to Z$是一个拟有限态射.于是问题归结为设$f$是支配的.我们知道$\mathscr{O}_{X,x}$是拟有限$\mathscr{O}_{Y,f(x)}$模(此为$\mathscr{O}_{X,x}/\mathfrak{m}_{f(x)}\mathscr{O}_{X,x}$是有限秩$\kappa(x)$模,对于有限型态射此为拟有限的等价描述),并且此时$\mathfrak{m}_y\mathscr{O}_{X,x}$是$\mathscr{O}_{X,x}$的定义理想.按照$\dim X=\sup_{x\in X}\dim\mathscr{O}_{X,x}$,问题归结为如下交换代数事实:设$(A,\mathfrak{m})\to B$是诺特局部环之间的局部同态,使得$\mathfrak{m}B$是$B$的定义理想,那么有$\dim B\le\dim A$.
    	
    	\qquad
    	
    	(2):设$\dim Y=n$,那么存在不同的点$\{y_i\mid0\le i\le n\}\subseteq Y$满足$y_i\in\overline{\{y_{i+1}\}},\forall0\le i\le n-1$.问题归结为构造$X$上长度为$n$的满足这个条件的点列$\{x_i\}$满足$f(x_i)=y_i$.
    	
    	\qquad
    	
    	先设$f$是开满射.对$i$归纳.任取$x_0\in f^{-1}(y_0)$.假设已经构造了$\{x_1,\cdots,x_m\}$满足$x_i\in\overline{\{x_{i+1}\}},0\le i\le m-1$.我们知道按照$f$是开映射,因为$y_{m+1}$是$y_m$的一般化,就存在$x_m$的一般化$x_{m+1}$满足$f(x_{m+1})=y_{m+1}$.完成归纳.
    	
    	\qquad
    	
    	再设$f$是闭满射.对$i$递降的归纳.任取$x_n\in f^{-1}(y_n)$.假设已经构造了$\{x_n,x_{n-1},\cdots,x_{m+1}\}$满足$x_i\in\overline{\{x_{i+1}\}},m+1\le i\le n$.按照$f$是闭映射,有$f(\overline{\{x_{m+1}\}})=\overline{\{f(x_{m+1})\}}=\overline{\{y_{m+1}\}}$,于是可以找到$x_m\in\overline{\{x_{m+1}\}}$满足$f(x_m)=y_m$.完成归纳.
    \end{proof}
    \item 推论.如果$f:X\to Y$是局部诺特概形之间的有限态射,那么$\dim X=\dim f(X)$.如果额外的$f$还是满射,就有$\dim X=\dim Y$.
    \item 存在这样的例子,$f:X\to Y$是局部诺特概形之间的有限型双射局部嵌入,满足$\dim X<\dim Y$:设$A$是DVR,闭点记作$b$,一般点记作$a$,那么$\kappa(a)=K$是$A$的商域,$\kappa(b)=k$是$A$的剩余域.取$X=\mathrm{Spec}K\coprod\mathrm{Spec}k$和$Y=\mathrm{Spec}A$,取$f:X\to Y$是典范态射.那么$f$是双射,局部上是嵌入.并且按照$K=A[\pi^{-1}]$得到$f$是有限型态射,但是有$\dim X=0$和$\dim Y=1$.
\end{enumerate}
\subsection{泛catenary概形}

称一个局部诺特概形$X$是泛catenary概形,如果对任意$x\in X$有$\mathscr{O}_{X,x}$是泛catenary环.
\begin{enumerate}
	\item 仿射情况.我们知道一个诺特环是泛catenary环当且仅当它的全部局部环是泛catenary的,于是一个仿射概形$\mathrm{Spec}A$是泛catenary概形当且仅当$A$是泛catenary环.
	\item 设$f:X\to Y$是不可约局部诺特概形之间的支配局部有限型态射.设$X,Y$的一般点分别为$\xi,\eta$.记一般纤维的维数$e=\dim(f^{-1}(\eta))=\mathrm{tr.deg}_{\kappa(\eta)}\kappa(\xi)$.那么对任意$x\in X$,记$y=f(x)$,记$\delta(x)=\dim_x(f^{-1}(y))-e$,总有如下不等式:
	$$e+\dim\mathscr{O}_{Y,y}\ge\mathrm{tr.deg}_{\kappa(y)}\kappa(x)+\dim\mathscr{O}_{X,x}$$
	$$\dim\mathscr{O}_{X,x}\le\dim\mathscr{O}_{Y,y}+\dim(\mathscr{O}_{X,x}\otimes_{\mathscr{O}_{Y,y}}\kappa(y))-\delta(x)$$
	
	如果$Y$还是泛catenary概形,那么这两个不等式取等;如果额外的$x$是$f^{-1}(y)$的闭点,那么有$\dim\mathscr{O}_{X,x}=\dim\mathscr{O}_{Y,y}+e$.
	\begin{proof}
		
		问题归结为仿射情况,并且不妨设$X,Y$是既约的,那么这个不等式归结为维数不等式.这两个不等式是等价的是因为$f^{-1}(y)$是有限型$\kappa(y)$概形,并且$\mathscr{O}_{X,x}\otimes_{\mathscr{O}_{Y,y}}\kappa(y)$就是概形$f^{-1}(y)$在点$x$处的局部环,于是我们之前证明过:
		$$\dim\mathscr{O}_{f^{-1}(y),x}=\dim(\mathscr{O}_{X,x}\otimes_{\mathscr{O}_{Y,y}}\kappa(y))=\dim_x(f^{-1}(y))-\mathrm{tr.deg}_{\kappa(y)}\kappa(x)$$
		
		如果$Y$是泛catenary概形,那么$A=\mathscr{O}_{Y,y}$是泛catenary环,取$B'=\mathscr{O}_{X,x}$,对它们用泛catenary环满足的维数等式即得到这两个不等式取等.
	\end{proof}
    \item 推论.设$f:X\to Y$是局部诺特不可约概形之间的支配局部有限型态射.设$X,Y$的一般点分别为$\xi,\eta$.记一般纤维的维数$e=\dim(f^{-1}(\eta))=\mathrm{tr.deg}_{\kappa(\eta)}\kappa(\xi)$.那么总有$\dim X\le\dim Y+e$.如果$Y$是泛catenary概形,那么这个不等式取等当且仅当$\dim Y=\sup_{y\in f(X)}\dim\mathscr{O}_{Y,y}$.特别的,如果$Y$是域上局部有限型概形,那么这个不等式取等.
    \begin{proof}
    	
    	在上一条中取$x$是闭点得到$\dim\mathscr{O}_{X,x}\le\dim\mathscr{O}_{Y,y}+e$.由于局部诺特概形上非空闭集总包含闭点,导致对闭点的局部环的维数取上确界得到空间的维数,这个不等式得证.
    	
    	\qquad
    	
    	按照$f^{-1}(y)$如果非空则总是局部诺特的,所以纤维非空时总包含了(该纤维的)一个闭点$x$,于是如果$Y$是泛catenary概形,我们已经证明了$\dim\mathscr{O}_{X,x}=\dim\mathscr{O}_{Y,y}+e$.让$x$跑遍$f$所有纤维的所有闭点,这包含了$x$是$X$闭点的情况,所以取上确界得到$\sup\dim\mathscr{O}_{X,x}=\dim X$.另一方面因为每个非空纤维都包含闭点,所以此时$f(x)$会跑遍全部$f(X)$.于是上述等式取上确界得到$\dim X=\sup_{y\in f(X)}\dim\mathscr{O}_{Y,y}+e$.于是这个等式成立当且仅当$\dim Y=\sup_{y\in f(X)}\dim\mathscr{O}_{Y,y}$.
    	
    	\qquad
    	
    	证明最后一个断言.取$\xi$的仿射开邻域$U$使得$f\mid_U$是有限型的,那么有$f(U)$是$Y$的稠密可构造集,进而$f(U)$包含了$Y$的一个非空开子集,进而有$\dim Y=\sup_{y\in f(X)}\dim\mathscr{O}_{Y,y}$.另外按照$Y$是域上局部有限型概形导致它是泛catenary概形,于是得证.
    \end{proof}
    \item 推论.设$f:X\to Y$是局部诺特概形之间的局部有限型态射,如果$\{\dim f^{-1}(y)\mid y\in Y\}$以整数$n$为一个上界,那么有$\dim X\le\dim Y+n$.
    \begin{proof}
    	
    	因为仿射开覆盖中开子集维数的上确界是空间的维数,以及局部有限型态射在基变换下不变,所以问题归结为设$X,Y$都是仿射的.又因为问题只涉及维数,不妨设它们都是既约的.设$\{X_i\}$是$X$的全部有限个不可约分支,赋予既约(整)闭子概型结构.那么$\dim X=\sup_i\dim X_i$.再记$\overline{f(X_i)}$上赋予$Y$的既约(整)闭子概型结构为$Z_i$.那么$f\mid_{X_i}:X_i\to Y$就要经$Z_i$分解,分解为支配态射$g_i:X_i\to Z_i$和闭嵌入$j_i:Z_i\to Y$.这里$g_i$还是有限型态射.记$Z_i$的一般点为$z_i$,那么有$\dim g_i^{-1}(z_i)\le n,\forall i$.于是问题归结为证明$\dim X_i\le\dim Z_i+n$,但是前面推论已经证明了$\dim X_i\le\dim Z_i+\dim g_i^{-1}(z_i)$,得证.
    \end{proof}
    \item 推论.设$f:X\to Y$是从不可约局部诺特概形到局部诺特概形的局部有限型态射,设$Y$是泛catenary概形,设$n\ge0$是$\{\dim f^{-1}(y)\mid y\in Y\}$的一个下界,那么有$\dim X\ge\dim Y+n$;如果对任意$y\in Y$总有$\dim f^{-1}(y)=n$,那么有$\dim X=\dim Y+n$.
    \begin{proof}
    	
    	条件保证了$f$是满射,于是前面已经证明了$\dim X=\dim Y+e$,这里$e$是一般纤维的维数.于是有$\dim X=\dim Y+e\ge\dim Y+n$.
    \end{proof}
    \item 反例.设$Y=\mathrm{Spec}A$是DVR的素谱,设$X=\mathrm{Spec}K$是$A$商域的素谱,考虑典范态射$f:X\to Y$,那么$\dim X=0$,$\dim Y=1$,$e=0$,不满足$\dim X=\dim Y+e$.
    \item 设$A\subseteq B$是两个诺特环,满足$B$是有限维$A$代数,这是一个整扩张,于是有$\dim A=\dim B$.如果此时$A$是诺特局部环,那么$B$是一个诺特半局部环,它的全部(有限个)极大理想记作$\mathfrak{n}_i$,于是我们有$\dim A=\max_i\dim B_{\mathfrak{n}_i}$.但是这些$\dim B_{\mathfrak{n}_i}$未必维数都相同(反例见EGAIV.5.6.11).不过如果$A$还是泛catenary环,那么这种情况不会发生.换句话讲:设$A$是诺特整泛catenary局部环,设$B$是包含$A$的一个整有限代数,那么对$B$的任意极大理想$\mathfrak{n}$,都有$\dim B_{\mathfrak{n}}=\dim A$.
    \begin{proof}
    	
    	按照$A$是泛catenary环,就有$\dim A+\mathrm{tr.deg}_A(B_{\mathfrak{n}})=\dim B_{\mathfrak{n}}+\mathrm{tr.deg}_{\kappa}\kappa'$.但是无论商域扩张还是剩余域扩张都是代数扩张,于是得到$\dim A=\dim B_{\mathfrak{n}}$.
    \end{proof}
\end{enumerate}
\subsection{优等概形【】 }

一个局部诺特概形$X$称为优等概形(excellent scheme),如果存在(任意)仿射开覆盖$\{U_i=\mathrm{Spec}A_i\}$有$A_i$是优等环.
\begin{enumerate}
	\item 设$X$是优等概形,$f:X'\to X$是局部有限型态射,那么$X'$是优等概形.
	\item 设$X$是既约优等概形,那么完备化$X'$在$X$上有限.
	\item 优等概形$X$的全部正则点$\mathrm{Reg}(X)$,全部正规点$\mathrm{Nor}(X)$,固定$n$时全部满足Serre条件$(R_n)$的点$\mathrm{U}_{(\mathrm{R}_n)}(X)$都构成开集.
\end{enumerate}








\subsection{等维数}

设$X$是概形,设$Y$是闭子概型.
\begin{itemize}
	\item 称$X$是等维数$d$的,如果它的所有不可约分支的维数都是$d$.
	\item 称$Y$是等余维数$r$的,如果$Y$的每个不可约分支的余维数都是$r$.
\end{itemize}
\begin{enumerate}
	\item 仿射情况.一个闭子集$V(I)$的不可约分支就是$V(\mathfrak{p})$,其中$\mathfrak{p}$是包含$I$的极小素理想.于是$V(I)$是等余维数$r$的等价于讲$I$的每个极小素理想的高度都是$r$.例如对于诺特UFD,记作$A$,交换代数告诉我们它的高度1素理想恰好是素元生成的理想,于是对$0\not=f\in A$,记$f=p_1^{r_1}\cdots p_n^{r_n}$是唯一分解,其中$p_i$是两两不同的素元,那么包含$(f)$的极小素理想恰好是$(p_1),\cdots,(p_n)$,它们的高度恰好都是1,于是$V(f)$总是等余维数1的.
    \item 设$X$是域$k$上的有限型整概型,设$f\in\mathscr{O}_X(X)$满足$\emptyset\subsetneqq V(f)\subsetneqq X$,那么$V(f)$是等余维数1的.
    \begin{proof}
    	
    	按照$X$是诺特概型,闭子集$V(f)$上只有有限个不可约分支$\{Z_1,Z_2,\cdots,Z_r\}$.设$Z_i$的一般点为$\xi_i$,可取$\xi_i$的开邻域$U_i$,使得$j\not=i$时总有$U_i\cap Z_j=\emptyset$.我们解释过这个条件下有每个$\dim U_i=\dim X$.所以我们不妨用$U_i$代替$X$,用$f\mid_{U_i}$代替$f$,可设$X=\mathrm{Spec}A$是整环的素谱,并且$V(f)$是不可约闭子集.
    	
    	\qquad
    	
    	按照诺特正规化引理,如果记$\dim A=d$,存在有限单$k$同态$\varphi:k[T_1,\cdots,T_d]\to A$.记$K=k(T_1,\cdots,T_d)$,记$L=\mathrm{Frac}(A)$,记$g=\mathrm{N}_{L/K}(f)$,我们解释过$g\in k[T_1,\cdots,T_d]$,并且$\varphi$诱导的态射$X\to\mathbb{A}_k^d$可限制为一个满态射$V(f)\to V(g)$,于是$\dim V(f)=\dim V(g)$,但是这里$k[T_1,\cdots,T_d]$是诺特UFD,导致$V(g)$的维数是$d-1$,于是$\dim V(f)=d-1$.
    \end{proof}
\end{enumerate}

\subsection{射影概形的维数}
\begin{enumerate}
	\item 设$B$是域$k$上的分次代数,那么有:
	$$\dim\mathrm{Spec}B=\dim\mathrm{Proj}B+1$$
	\begin{proof}
		
		我们不妨设$B$是整环,因为$B$的极小素理想$\{\mathfrak{p}_1,\cdots,\mathfrak{p}_n\}$都是齐次的,而空间的维数是它不可约分支维数的最大值,于是一旦证明$\dim V(\mathfrak{p}_i)=\dim V_+(\mathfrak{p}_i)$,就得到$\dim\mathrm{Spec}B=\max\{\dim V(\mathfrak{p}_i)\mid i\}=\max\{\dim V_+(\mathfrak{p}_i)\mid i\}=\dim\mathrm{Proj}B$.下面设$B$是整环,任取1次齐次元$f\in B_+$,有$B$代数同构$B_{(f)}[T,1/T]\to B_f$为$T\mapsto f$,于是$\dim B_f=\dim B_{(f)}[T,1/T]=\dim B_{(f)}+1$,于是$\dim D(f)=\dim D_+(f)+1$,取上确界得到$\dim B=\dim\mathrm{Proj}B+1$.
	\end{proof}
    \item 设$X$是域$k$上的射影概形并且是整概形,那么它是$\mathbb{P}_k^n$的闭子概型,换句话讲存在$k[X_0,\cdots,X_n]$的齐次理想$I$使得$X=\mathrm{Proj}k[X_0,\cdots,X_n]/I$.我们定义$X$的仿射锥$C(X)=\mathrm{Spec}k[X_0,\cdots,X_n]/I$.我们断言有$\dim C(X)=\dim X+1$.
    \begin{proof}
    	
    	考虑态射$\pi:\mathbb{A}_k^{n+1}-\{0\}\to\mathbb{P}_k^n$,它局部上是被$\pi_i:D(X_i)=\mathrm{Spec}k[T_0,\cdots,T_n,T_i^{-1}]\to D_+(X_i)=\mathrm{Spec}k[X_{0/i},\cdots,\widehat{X_{i/i}},\cdots,X_{n/i}]$,$X_{j/i}\mapsto X_j/X_i$,$j\not=i$粘合而成的态射.那么对$\mathbb{P}_k^n$的闭子概型$X$,有$C(X)=\pi^{-1}(X)$.
    	
    	\qquad
    	
    	记$Y=\mathbb{A}_k^{n+1}-\{0\}$,那么$C(D_+(X_i)\cap X)=\pi^{-1}(D_+(X_i)\cap X)\cong(D_+(X_i)\cap X)\times_k(\mathbb{A}_k^1-\{0\})$.按照局部有限型$k$概形上纤维积的维数公式,得到$\dim C(D_+(X_i)\cap X)=\dim(D_+(X_i)\cap X)+1$.再对$i$取最大值,得到$\dim C(X)=\dim X+1$.
    \end{proof}
    \item 设$X\subseteq\mathbb{P}_k^n$是维数非零的整闭子概型,任取齐次元$f\in k[X_0,X_1,\cdots,X_n]$使得$V_+(f)\not=\emptyset$和$X\not\subseteq V_+(f)$.那么$X\cap V_+(f)$非空,并且是等余维数1的.
    \begin{proof}
    	
    	先证明$X\cap V_+(f)$非空.首先按照$\dim X>0$,得到$\dim C(X)=\dim X+1\ge2$.考虑$V(f)\subseteq\mathbb{A}_k^{n+1}$,那么原点落在$C(X)\cap V(f)$中,这个交是非空的.我们解释过有限型$k$整概型$C(X)$上的零点集$V(f\mid C(X))$是等余维数1的,这说明$V(f\mid C(X))=C(X)\cap V(f)$至少还包含一个非原点的点,这说明$X\cap V_+(f)$是非空的.【】
    \end{proof}
    \item 设$X\subseteq\mathbb{P}_k^n$是整闭子概型,设$f_1,f_2,\cdots,f_r\in k[X_0,X_1,\cdots,X_n]$是非常数的齐次多项式,那么$X\cap V_+(f_1,f_2,\cdots,f_r)$的每个不可约分支在$X$中的余维数$\le r$.于是如果$\dim X\ge r$,就有这个交是非空的.【】
    \item 设$Z\subseteq\mathbb{P}_k^n$是整闭子概型,它的余维数1当且仅当存在不可约的齐次多项式$f$使得$Z=V_+(f)$.【】
    \item 设$n\ge2$,设$X=V_+(f)\subseteq\mathbb{P}_k^n$是超曲面(也即$f$是非常数的齐次多项式),那么$X$是连通的.【】
\end{enumerate}


\subsection{纤维维数的半连续性}
\begin{enumerate}
	\item 引理.设$Y$是不可约局部诺特概形,设$X$是不可约概形,设$f:X\to Y$是有限型的支配态射,记$Y$的一般点为$\eta$,记$e=\dim f^{-1}(\eta)$,那么对任意$x\in X$,纤维$f^{-1}(f(x))$的所有不可约分支的维数都$\ge e$.这里我们只证明$Y$是universally catenary概形的情况,一般情况可以把$Y$约化到$\mathrm{Spec}A$的情况,其中$A$是$\mathbb{Z}$的有限型代数,此时它是universally catenary的【一般情况见EGA4.13.1.1】.
	\begin{proof}
		
		取$y=\in f(X)$,按照$\mathscr{O}_{Y,y}$是universally catenary环,任取$f^{-1}(y)$的不可约分支$Z$,记它的一般点为$x$,则按照universally catenary条件的维数等式描述,我们就有等式:
		$$e+\dim\mathscr{O}_{Y,y}=\mathrm{tr.deg}_{\kappa(y)}\kappa(x)+\dim\mathscr{O}_{X,x}$$
		
		另外我们知道局部诺特概形之间的态射总会满足维数不等式$\dim\mathscr{O}_{X,x}\le\dim\mathscr{O}_{Y,y}+\dim\mathscr{O}_{X_y,x}$.但是这里$x$是$X_y$的一个一般点,所以$\dim\mathscr{O}_{X_y,x}=0$,于是$\dim\mathscr{O}_{X,x}\le\dim\mathscr{O}_{Y,y}$,另一方面我们有$\dim Z=\mathrm{tr.deg}_{\kappa(y)}\kappa(x)$,综上得到$\dim Z\ge e$.
	\end{proof}
	\item 设$f:X\to Y$是局部有限型态射,考虑函数$X\to\mathbb{Z}_{\ge0}$,$x\mapsto\dim_xf^{-1}(f(x))$,那么这是上半连续函数,换句话讲对任意自然数$n$,都有$F_n=\{x\in X\mid\dim_xf^{-1}(f(x))\ge n\}$是$X$的闭子集.我们只证明$f$是局部有限表示的情况,【一般情况见EGA4.13.1.3】.
	\begin{proof}
		
		问题是局部的,归结为设$Y=\mathrm{Spec}A$和$X=\mathrm{Spec}B$是仿射的.按照$f:X\to Y$是有限表示态射,我们可以找到一个子环$A_0\subseteq A$,使得$A_0$是有限型$\mathbb{Z}$代数,并且有有限型$A_0$代数$B_0$,使得$B=A\otimes_{A_0}B_0$.记典范同态诱导的态射分别是$g:\mathrm{Spec}A\to\mathrm{Spec}A_0$和$g':\mathrm{Spec}B\to\mathrm{Spec}B_0$和$f_0:\mathrm{Spec}B_0\to\mathrm{Spec}A_0$.
		
		\qquad
		
		取$y\in Y=\mathrm{Spec}A$,记$y_0=g(y)$,我们有$f^{-1}(y)=f^{-1}(y_0)\times_{\mathrm{Spec}\kappa(y_0)}\mathrm{Spec}\kappa(y)$.再取$x\in f^{-1}(y)$,记$x_0=g'(x)$,则按照有限型条件,$\dim_xf^{-1}(y)$是$f^{-1}(y)$的所有包含$x$的不可约分支的维数的上确界.因为$\mathrm{Spec}\kappa(y)\to\mathrm{Spec}\kappa(y_0)$是泛开的(终端底空间是离散空间的态射总是泛开的),于是投影态射$f^{-1}(y)\to f_0^{-1}(y_0)$是开映射,于是$f^{-1}(y)$的每个包含$x$的不可约分支$Z$【】
	\end{proof}
    \item 推论.我们在Zariski主定理那里证明过如果$f:X\to Y$是局部有限型态射,那么$U=\{x\in X\mid\dim_xf^{-1}(f(x))=0\}$是开集.上一条也能推出这成立.
    \item 推论.如果$f:X\to Y$是局部有限型态射,并且是闭映射,那么对任意自然数整数$n$都有$Y_n=\{y\in Y\mid\dim f^{-1}(y)\ge n\}$是$Y$的闭子集.这件事是因为$Y_n=f(F_n)$【】.
    \item 设$f:X\to Y$是局部有限型态射,也是开映射.设$Y$是不可约诺特的universally catenary概形,并且$\dim Y<\infty$,再设$X$是等维数的.并且设$f$满足如下条件:对$X$的任意不可约分支$X'$,都有$\dim Y=\sup_{y\in f(X')}\dim\mathscr{O}_{Y,y}$.这个条件我们记作(D).则对任意$y\in f(X)$,有$f^{-1}(y)$是等维数的,并且总满足$\dim X=\dim Y+\dim f^{-1}(y)$.
    \begin{proof}
    	
    	记$Y$的一般点为$\eta$,取$X$的不可约分支$X'$,记它的一般点为$\theta$.按照$f$是开映射也是局部有限表示态射,说明它把一般点映射为一般点,于是特别的$f(\theta)=\eta$.于是$\{\theta\}$是$f^{-1}(\eta)\cap X'$的稠密子集,这说明$f^{-1}(\eta)\cap X'$本身是不可约的.又因为$f^{-1}(\eta)\cap X'$是$\kappa()$【】
    	
    \end{proof}
    \item 上一条中的条件(D)是不能去掉的.另外在上一条的条件下,如下额外条件可以保证(D)是满足的:
    \begin{enumerate}[(1)]
    	\item 如果$f$在$X$的每个不可约分支上的限制都是到$Y$的满射,则(D)成立.这是因为对任意概形$Y$我们都有$\dim Y=\sup_{y\in Y}\dim\mathscr{O}_{Y,y}$.特别的,如果$f$是满射并且$X$是不可约的则(D)成立.
    	\item $f$拓扑上还是闭映射,则(D)成立:因为$f$已经是开映射,它还是局部有限表示的,于是它把一般点映射为一般点,换句话讲$f$限制在$X$的每个不可约分支上都是支配的,再按照$f$是闭映射就导致$f$在每个不可约分支上的限制都是满射,这归结到(1)的情况.
    	\item $Y$是域上的有限型概形,此时$Y$自动是有限维诺特和universally catenary的,此时(D)成立:任取$X$的不可约分支$X'$,按照Chevalley定理有$f(X')$是$Y$的可构造子集.另外按照$f$是开映射也是局部有限表示态射,于是它把一般点映射为一般点,按照$Y$是不可约的,于是$f(X')$就是$Y$的稠密子集因为它包含了$Y$的一般点.但是我们解释过不可约空间的可构造子集要么是无处稠密的,要么包含了一个开集,于是$f(X')$包含了$Y$的一个非空开集$V$.进而有$\dim V\le\sup_{y\in f(X')}\dim\mathscr{O}_{Y,y}\le\dim Y$,结合$\dim V=\dim Y$得到(D)成立.
    	\item 设$Y$是概形$S$上的有限型概形,其中$S$满足如下两个条件:
    	\begin{enumerate}
    		\item $S$是universally catenary的局部诺特Jacobson概形.
    		\item 对$S$的任意不可约分支$S'$,以及任意的闭点$s\in S'$,都有$\dim\mathscr{O}_{S',s}=\dim S'$.
    	\end{enumerate}
    
        那么$Y$同样满足这两个条件,并且此时(D)成立【EGA4.10.6.2】.另外如果$S$是维数$\le1$的诺特概形,使得每个不可约分支都有无限个点,那么$S$满足这两个条件【EGA4.10.7.1】.
    \end{enumerate}
    \item 推论.设$f:X\to Y$是有限型支配态射.设$X$是不可约的,$Y$是诺特不可约有限维数的universally catenary概形.那么存在稠密开子概形$V\subseteq Y$,满足$V\subseteq f(X)$,并且如下两个命题成立:
    \begin{enumerate}
    	\item 对任意$y\in V$,有$f^{-1}(y)$是等维数的,并且$\dim f^{-1}(V)=\dim f^{-1}(y)+\dim V$.
    	\item 对任意$y\in Y$和$f^{-1}(y)$的任意不可约分支$Z$,有$\dim Z+\dim V\ge\dim f^{-1}(V)$.
    \end{enumerate}
    【】
\end{enumerate}
\newpage
\section{切空间}
\subsection{Zariski切空间}

给定局部环$(A,\mathfrak{m},k)$,它的Zariski余切空间定义为$k=A/\mathfrak{m}$线性空间$\mathfrak{m}/\mathfrak{m}^2$,这个余切空间的对偶空间定义为Zariski切空间.概型$X$上的点$p$的Zariski切空间和Zariski余切空间定义为它局部环$\mathscr{O}_{X,p}$的Zariski切空间和余切空间,记作$T_pX$和$T^{\vee}_pX$.切空间中的元素称为切向量,余切空间中的元素称为余切向量或者微分.对概型$X$,我们称$\dim_{\kappa(x)}\mathrm{T}_xX$是点$x$的嵌入维数.
\begin{enumerate}
	\item 按照NAK引理,$\dim_{\kappa(x)}\mathrm{T}_xX$就是$\mathfrak{m}_x$的极小生成元个数.于是只要$X$是局部诺特概形,它的每个点的嵌入维数都是有限的.
	\item 切空间明显是一个局部概念,如果$U\subseteq X$是开子概型,对$x\in U$,就有$\mathrm{T}_xU=\mathrm{T}_xX$.
	\item 余切映射.设$f:X\to Y$是局部诺特概型之间的态射(这个要求只是为了嵌入维数都有限),满足$f(x)=y$,那么态射诱导了局部环的局部同态$(\mathscr{O}_{Y,y},\mathfrak{m}_y)\to(\mathscr{O}_{X,x},\mathfrak{m}_x)$.于是这个局部同态把$\mathfrak{m}_y^2$映入$\mathfrak{m}_x^2$,导致它诱导了典范同态$\mathfrak{m}_y/\mathfrak{m}_y^2\to\mathfrak{m}_x/\mathfrak{m}_x^2$.它诱导了$\kappa(x)$线性映射$\mathfrak{m}_y/\mathfrak{m}_y^2\otimes_{\kappa(y)}\kappa(x)\to\mathfrak{m}_x/\mathfrak{m}_x^2$,这称为余切映射.
	\item 切映射.一般的设$V,W$是域$K$上的两个线性空间,如果其中至少一个在$K$上有限维,那么我们典范同构:
	$$V^*\otimes_KW\cong\mathrm{Hom}_K(V,W)$$
	
	下面取$V=\mathfrak{m}_y/\mathfrak{m}_y^2$和$W=\mathfrak{m}_x/\mathfrak{m}_x^2$,其中$y=f(x)$,则$f:X\to Y$诱导的阿贝尔群同态为$\varphi:V\to W$.这里$V$是域$\kappa(y)$上的线性空间,$W$是域$\kappa(x)$上的线性空间.如果$\kappa(x)$和$V$中至少有一个在$\kappa(y)$上有限维,那么我们有典范同构:
	$$\mathrm{T}_yY\otimes_{\kappa(y)}\kappa(x)\cong\mathrm{Hom}_{\kappa(y)}(V,\kappa(x))$$
	
	定义$f$在$x$处的切映射为:
	$$\mathrm{d}f_x:\mathrm{T}_xX=\mathrm{Hom}_{\kappa(x)}(W,\kappa(x))\to\mathrm{Hom}_{\kappa(y)}(V,\kappa(x))=\mathrm{T}_yY\otimes_{\kappa(y)}\kappa(x)$$
	$$\alpha\mapsto\alpha\circ\varphi$$
	
	我们来验证这是良性的,即如果$\alpha:W\to\kappa(x)$是$\kappa(x)$线性映射,那么$\alpha\circ\varphi:V\to\kappa(x)$是$\kappa(y)$线性的,首先它肯定是阿贝尔群同态,所以只需验证它和数乘可交换:设$a\in\mathscr{O}_{Y,y}^*$,设$x\in\mathfrak{m}_y$,我们把$f$诱导的$\mathscr{O}_{Y,y}\to\mathscr{O}_{X,x}$仍记作$\varphi$:
	\begin{align*}
		\alpha\circ\varphi((a+\mathfrak{m}_y)(x+\mathfrak{m}_y^2))&=\alpha\left(\varphi(a)\varphi(x)+\mathfrak{m}_x^2\right)\\&=(a+\mathfrak{m}_y)\alpha(\varphi(x)+\mathfrak{m}_x^2)\\&=(a+\mathfrak{m}_y)\alpha\circ\varphi(x+\mathfrak{m}_y^2)
	\end{align*}

    特别的,局部诺特概形上局部环都是诺特环,此时嵌入维数总是有限的,于是局部诺特概形之间的态射总能定义诱导的切映射.
	\item 态射诱导切映射具有函子性.如果$f:X\to Y$和$g:Y\to Z$是局部诺特概形之间的态射,对$x\in X$,记$y=f(x)$,那么$\mathrm{d}(g\circ f)_x=(\mathrm{d}_y\otimes\mathrm{id}_{\kappa(x)})\circ\mathrm{d}f_x$.
\end{enumerate}
\subsection{域上概形的切空间}

设$k$是域.
\begin{enumerate}
	\item 设$x$是$\mathbb{A}_k^n$的$k$有理点,可记$x=(x_1,\cdots,x_n)\in k^n$.它对应的极大理想是$(T_1-x_1,T_2-x_2,\cdots,T_n-x_n)$.并且这些$T_i-x_i$构成了$\kappa(x)=k$线性空间$\mathfrak{m}_x/\mathfrak{m}_x^2$的一组基.于是我们有同构$k^n\cong\mathrm{T}_xX$为:
	$$(v_1,\cdots,v_n)\mapsto\left(\begin{array}{c}\mathfrak{m}_x/\mathfrak{m}_x^2\to k\\\overline{g}\mapsto\sum_iv_i\frac{\partial g}{\partial T_i}(x)\end{array}\right)$$
	
	于是我们可以像微分几何一样把切向量理解为(正则)函数的方向导数,对$k$有理点$x$,对多项式$P\in k[T_1,\cdots,T_n]$,记$(\mathrm{D}_xP)(t_1,\cdots,t_n)=\sum_{1\le i\le n}\frac{\partial P}{\partial T_i}(x)t_i$,它是$P$在点$x$的方向为$(t_1,\cdots,t_n)$的方向导数.
	\item 设$X=V(f_1,\cdots,f_r)\subseteq\mathbb{A}_k^n$是闭子概型,其中$f_i\in k[T_1,\cdots,T_n]$.设$x$是$X$的$k$有理点.那么$x$也是$\mathbb{A}_k^n$的$k$有理点,并且有$\mathfrak{n}=\mathfrak{m}/I$,其中$I=(f_1,\cdots,f_r)$,而$\mathfrak{m}$和$\mathfrak{n}$分别是$x$在$\mathbb{A}_k^n$和$X$中局部环的极大理想.那么我们有短正合列$0\to I/(I\cap\mathfrak{m}^2)\to\mathfrak{m}/\mathfrak{m}^2\to\mathfrak{n}/\mathfrak{n}^2\to0$.这里$I/(I\cap\mathfrak{m}^2)=(I+\mathfrak{m}^2)/\mathfrak{m}^2=D_xI$.取对偶得到短正合列$0\to\mathrm{T}_xX\to\mathrm{T}_x\mathbb{A}_k^n\to(D_xI)^{\vee}\to0$.但是一般的如果$F$是$k$线性空间$E$的子空间,如果记$F^{\perp}=\{\varphi\in E^{\vee}\mid\varphi(v)=0,\forall v\in F\}$,那么有短正合列$0\to F^{\perp}\to E^{\vee}\to F^{\vee}\to0$.于是这里有$0\to(D_xI)^{\perp}\to\mathrm{T}_x\mathbb{A}_k^n\to(D_xI)^{\vee}\to0$.于是有$\mathrm{T}_xX$可典范的视为$(D_xI)^{\perp}\subseteq\mathrm{T}_x\mathbb{A}_k^n=k^n$.也即:
	\begin{align*}
		\mathrm{T}_xX&=\left\{(t_1,\cdots,t_n)\in k^n\mid\sum_{1\le i\le n}\frac{\partial P}{\partial T_i}(x)t_i=0,\forall P\in I\right\}\\&=\ker\left(\begin{array}{ccc}\frac{\partial f_1}{\partial T_1}(x)&\cdots&\frac{\partial f_1}{\partial T_n}(x)\\\vdots&\ddots&\vdots\\\frac{\partial f_r}{\partial T_1}(x)&\cdots&\frac{\partial f_r}{\partial T_n}(x)\end{array}\right)
	\end{align*}
	
	这个矩阵即$V(I)$的雅各比矩阵.特别的,我们有$\mathrm{rank}J_x=n-\dim_k\mathrm{T}_xX$.
	\item 例子.考虑$X=\mathrm{Spec}k[S,T]/(T^2-S^3)$.取$x_0$是$k$有理点中的原点(此即极大理想$(S,T)$).对任意$k$有理点$x$,它对应的雅各比矩阵是$(2T,-3S^2)$.于是如果$x$是不为$x_0$的有理点,那么嵌入维数$\dim_k\mathrm{T}_xX=1$,而$x_0$的嵌入维数是2.
	\item 给定一个有限生成$k$代数,它的雅各比矩阵的余秩(此为核的维数)不依赖于生成元集和关系集的选取.为此取一组生成元集合$\{T_1,T_2,\cdots,T_n\}$,取一组关系集$\{f_1,f_2,\cdots,f_r\}$,设它生成的理想为$I$,只要说明如下两个操作不改变雅各比矩阵的余秩:
	\begin{itemize}
		\item 任取$g\in I$,考虑生成元集不变,关系集为$\{f_1,f_2,\cdots,f_r,g\}$的雅各比矩阵,证明它的余秩不变.
		\item 任取$q(T_1,T_2,\cdots,T_n)\in k[T_1,T_2,\cdots,T_n]$,记$h=S-q(T_1,T_2,\cdots,T_n)\in k[T_1,\cdots,T_n,S]$.证明生成元集$\{T_1,T_2,\cdots,T_n,S\}$和关系集$\{f_1,f_2,\cdots,f_r,h\}$定义的雅各比矩阵的余秩不变.
	\end{itemize}
	\item 切空间的等价描述.对域$k$,把环$k[T]/(T^2)$记作$k[\varepsilon]$,其中$\varepsilon=T+(T^2)$满足$\varepsilon^2=0$.这个环称为$k$上的对偶数环.它的素谱可视为$\mathbb{A}_k^1$在零理想处的无穷小曲线.对$k$概形$X$,我们解释过记号$X(k[\varepsilon])$表示$\mathrm{Hom}_k(\mathrm{Spec}(k[\varepsilon]),X)$.我们断言$k[\varepsilon]$值点一一对应于一个$k$有理点$x$以及$T_xX$中的一个元.更精确的讲:
	\begin{enumerate}
		\item 如果$f:\mathrm{Spec}k[\varepsilon]\to X$是$k$态射,那么它把$\mathrm{Spec}k[\varepsilon]$的唯一点映射为$X$上的某个$k$有理点$x$,于是我们有$X(k[\varepsilon])\to X(k)$.
		\item 反过来如果$x$是$k$有理点,我们用$X(k[\varepsilon])_x$表示$x$在上述$X(k[\varepsilon])\to X(k)$下的原像,任取$f\in X(k[\varepsilon])_x$,它要诱导局部环之间的局部同态$\mathscr{O}_{X,x}\to k[\varepsilon]$,于是它诱导了$\mathfrak{m}_x\to\varepsilon k\cong k$,进而诱导了线性映射$\mathfrak{m}_x/\mathfrak{m}_x^2\to k$,此为$\mathrm{T}_xX$中的元,我们断言这个对应得到了自然同构$X(k[\varepsilon])_x\cong\mathrm{T}_xX$.
	\end{enumerate}
	\begin{proof}
		
		任取$k$态射$f:\mathrm{Spec}k[\varepsilon]\to X$,记$\mathrm{Spec}k[\varepsilon]$中的唯一点是$a$,因为$f$是源端为局部概形的态射,它就要经$\mathrm{Spec}k[\varepsilon]\to\mathrm{Spec}\mathscr{O}_{X,f(a)}$分解,于是$f$诱导了如下$k$代数同态的交换图表:
		$$\xymatrix{&k\ar[dl]\ar[dr]&\\\mathscr{O}_{X,f(a)}\ar[rr]&&k[\varepsilon]}$$
		
		取剩余域得到如下交换图表,于是$x=f(a)$是$k$有理点.这证明了(a).(事实上,这证明了$k$概形之间的态射,肯定把$k$有理点映射为$k$有理点).
		$$\xymatrix{&k\ar[dl]\ar@{=}[dr]^{\mathrm{id}}&\\\kappa(f(a))\ar[rr]&&k}$$
		
		任取$k$线性映射$t:\mathfrak{m}_x/\mathfrak{m}_x^2\to k$,这诱导了映射$\mathfrak{m}_x\to\varepsilon k$为$m\mapsto\varepsilon t(m\mathrm{mod}\mathfrak{m}_x^2)$.按照有理点的定义有$\mathscr{O}_{X,x}/\mathfrak{m}_x=k$,于是这个映射可以唯一的延拓为$k$代数同态$\mathscr{O}_{X,x}\to k[\varepsilon]$.这个对应和上面的对应是互逆的,这证明了(b).
	\end{proof}
	\item 关于$X(k[\varepsilon])_x$上加法和数乘.记$k[\varepsilon_1,\varepsilon_2]=k[T_1,T_2]/(T_1^1,T_2^2,T_1T_2)$,其中$\varepsilon_i$是$T_i$的像.我们有典范$k$代数同态$p:k[\varepsilon_1,\varepsilon_2]\to k[\varepsilon]$为$\varepsilon_i\mapsto\varepsilon$.
	\begin{enumerate}
		\item 如果$v_1,v_2\in\mathrm{T}_xX=X(k[\varepsilon])_x$,按照定义$v_i$是$\mathrm{Spec}k[\varepsilon]\to X$的态射,并且把唯一点映射为$\{x\}$.按照源端为局部概形的等价描述,$v_i$唯一的对应了一个局部同态$\mathscr{O}_{X,x}\to k[\varepsilon]$,$s\mapsto s(x)+\varepsilon\dot{v}_i$.定义$\alpha:\mathscr{O}_{X,x}\to k[\varepsilon_1,\varepsilon_2]$为$s\mapsto s(x)+\varepsilon_1\dot{v}_1(s)+\varepsilon_2\dot{v}_2(s)$.那么$v_1+v_2$就对应于$X$上的$k[\varepsilon]$值点$\mathrm{Spec}k[\varepsilon]\to\mathrm{Spec}k[\varepsilon_1,\varepsilon_2]\to X$,其中前面态射被$p$诱导,后面态射被$\alpha$诱导.
		\item 如果$a\in k$,定义$f_a:k[\varepsilon]\to k[\varepsilon]$是$k$代数同态$\varepsilon\mapsto a\varepsilon$.设$v\in\mathrm{T}_xX$对应于$v:\mathrm{Spec}k[\varepsilon]\to X$,那么$av\in\mathrm{T}_xX$就对应于$\mathrm{Spec}k[\varepsilon]\to\mathrm{Spec}k[\varepsilon]\to X$,其中前面态射被$f_a$诱导,后面态射是$v$.
	\end{enumerate}
    \item 推论.设$X,Y$是域$k$上的概形,设$x\in X(k)$和$y\in Y(k)$都是$k$有理点,那么存在$X\times_kY$上的唯一$k$有理点$z$,在两个投影态射下的像分别是$x,y$.我们断言有:
    $$\mathrm{T}_z(X\times_kY)\cong\mathrm{T}_xX\oplus\mathrm{T}_yY$$
    \begin{proof}
    	\begin{align*}
    		\mathrm{T}_z(X\times_kY)&\cong\left(\mathrm{Hom}_k(\mathrm{Spec}k[\varepsilon],X\times_kY)\right)_z\\&\cong\left(\mathrm{Hom}_k(\mathrm{Spec}k[\varepsilon],X)\right)_x\times\left(\mathrm{Hom}_k(\mathrm{Spec}k[\varepsilon],Y)\right)_y\\&\cong\mathrm{T}_xX\oplus\mathrm{T}_yY
    	\end{align*}
    \end{proof}
\end{enumerate}
\subsection{射影概形的切空间}

设$k$是域,考虑射影空间$X=\mathbb{P}_k^n=\mathrm{Proj}k[T_0,\cdots,T_n]$,取一个$k$有理点$x$,不妨设$x\in D_+(T_r)$,那么$x$也是仿射空间$D_+(T_r)$的$k$有理点,于是$\mathrm{T}_xX$是$n$维线性空间.但是这样描述切空间是依赖于基选取的,我们期望找到射影概形的一个不依赖于基选取的切空间描述.
\begin{enumerate}
	\item 引理.设$k$是域,设$f_1,\cdots,f_r,g_1,\cdots,g_r\in k[T_1,\cdots,T_n]$,记$X=D(g_1\cdots g_r)\subseteq\mathbb{A}_k^n$,考虑态射$h:X\to\mathbb{A}_k^r$为由$k[S_1,\cdots,S_r]\to k[T_1,\cdots,T_n]_{g_1\cdots g_s}$,$p(S_1,\cdots,S_n)\mapsto p(\frac{f_1(T)}{g_1(T)},\cdots,\frac{f_r(T)}{g_r(T)})$诱导的态射.我们断言$\mathrm{d}h_x$由矩阵$\left(\frac{\partial(f_i/g_i)}{\partial T_j}(x)\right)_{i,j}$
	\begin{proof}
		
		把$D(g_1\cdots g_r)$同构的视为闭子概型$V(g_1T_{n+1}-1,\cdots,g_rT_{n+r}-1)\subseteq\mathbb{A}_k^{n+1}$.那么$h$可以分解为$X\subseteq\mathbb{A}_k^{n+1}\to\mathbb{A}_k^r$,其中第二个态射是$k[S_1,\cdots,S_r]\to k[T_1,\cdots,T_{n+r}]$,$p(S_1,\cdots,S_r)\mapsto p(f_1(T_1,\cdots,T_n)T_{n+1},\cdots,f_r(T_1,\cdots,T_n)T_{n+r})$诱导的态射.按照函子性,这两个态射分别诱导的切映射的复合是这两个态射的复合诱导的切映射.
		
		\qquad
		
		设$x\in X$,我们先看第一个态射$X\subseteq\mathbb{A}_k^{n+r}$.首先把$x$视为$\mathbb{A}_k^{n+r}$的点,它的切空间是$k^{n+r}$,那么$x$在闭子概型$V(g_1T_{n+1}-1,\cdots,g_rT_{n+r}-1)$中的切空间就是如下$r\times(n+r)$矩阵的核($x$作为$\mathbb{A}_k^{n+r}$的点,它的坐标是$(x_1,\cdots,x_n,1/g_1(x_1,\cdots,x_n),\cdots,1/g_r(x_1,\cdots,x_n))$):
		$$\left(\begin{array}{ccccccc}\frac{\partial g_1}{\partial T_1}(x)\frac{1}{g_1(x)}&\cdots&\frac{\partial g_1}{\partial T_n}(x)\frac{1}{g_1(x)}&g_1(x)&0&\cdots&0\\\frac{\partial g_2}{\partial T_1}(x)\frac{1}{g_2(x)}&\cdots&\frac{\partial g_2}{\partial T_n}(x)\frac{1}{g_2(x)}&0&g_2(x)&\cdots&0\\\vdots&\ddots&\vdots&\vdots&\vdots&\ddots&\vdots\\\frac{\partial g_r}{\partial T_1}(x)\frac{1}{g_r(x)}&\cdots&\frac{\partial g_r}{\partial T_n}(x)\frac{1}{g_r(x)}&0&0&\cdots&g_r(x)\end{array}\right)$$
		
		于是$x\in X$分别作为$\mathbb{A}_k^n$的开子集和作为$\mathbb{A}_k^{n+r}$的闭子集的切空间是经如下$(n+r)\times n$的矩阵对应的:
		$$A=\left(\begin{array}{cccc}1&0&\cdots&0\\0&1&\cdots&0\\\vdots&\vdots&\ddots&\vdots\\0&0&\cdots&1\\\-\frac{\partial g_1}{\partial T_1}(x)\frac{1}{g_1^2(x)}&-\frac{\partial g_1}{\partial T_2}(x)\frac{1}{g_1^2(x)}&\cdots&-\frac{\partial g_1}{\partial T_n}(x)\frac{1}{g_1^2(x)}\\\-\frac{\partial g_2}{\partial T_1}(x)\frac{1}{g_2^2(x)}&-\frac{\partial g_2}{\partial T_2}(x)\frac{1}{g_2^2(x)}&\cdots&-\frac{\partial g_2}{\partial T_n}(x)\frac{1}{g_2^2(x)}\\\vdots&\vdots&\ddots&\vdots\\\-\frac{\partial g_r}{\partial T_1}(x)\frac{1}{g_r^2(x)}&-\frac{\partial g_r}{\partial T_2}(x)\frac{1}{g_r^2(x)}&\cdots&-\frac{\partial g_r}{\partial T_n}(x)\frac{1}{g_r^2(x)}\end{array}\right)$$
		
		再考虑第二个态射,它的切映射是如下$r\times(n+r)$矩阵:
	    $$B=\left(\begin{array}{ccccccc}\frac{\partial f_1}{\partial T_1}(x)\frac{1}{g_1(x)}&\cdots&\frac{\partial f_1}{\partial T_n}(x)\frac{1}{g_1(x)}&f_1(x)&0&\cdots&0\\\frac{\partial f_2}{\partial T_1}(x)\frac{1}{g_2(x)}&\cdots&\frac{\partial f_2}{\partial T_n}(x)\frac{1}{g_2(x)}&0&f_2(x)&\cdots&0\\\vdots&\ddots&\vdots&\vdots&\vdots&\ddots&\vdots\\\frac{\partial f_r}{\partial T_1}(x)\frac{1}{g_r(x)}&\cdots&\frac{\partial f_r}{\partial T_n}(x)\frac{1}{g_r(x)}&0&0&\cdots&f_r(x)\end{array}\right)$$
	    
	    于是$\mathrm{d}h_x$是如下$r\times n$的矩阵:
	    $$BA=\left(\frac{\partial(f_i/g_i)}{\partial T_j}(x)\right)_{i,j}$$
	\end{proof}
	\item 设$k$是域,设$x=[x_0,x_1,\cdots,x_n]$是$\mathbb{P}_k^n$的$k$有理点,那么$\mathrm{T}_x\mathbb{P}_k^n=k^{n+1}/k\cdot(x_0,\cdots,x_n)$,这里$k\cdot(x_0,\cdots,x_n)$表示的是点$(x_0,\cdots,x_n)\in k^{n+1}$所在的一维子空间.
	\begin{proof}
		
		考虑典范$k$概形态射$\mathbb{A}_k^{n+1}-\{0\}\to\mathbb{P}_k^n$(这是局部上的$D(T_i)\to D_+(T_i)$粘合得到的态射).取$\mathbb{A}_k^{n+1}$的一个$k$有理点$0\not=\dot{x}=(x_0,\cdots,x_n)$,记它在上述映射下的像是点$x$,那么$x$也是$k$有理点.于是我们有同态$k^{n+1}=\mathrm{T}_x\mathbb{A}_k^{n+1}\to\mathrm{T}_x\mathbb{P}_k^n$.我们断言这个同态的核是$(x_0,\cdots,x_n)$所在的一维子空间.一旦这成立,因为$\mathrm{T}_x\mathbb{P}_k^n$是$n$维的,就得到这个同态是满射,进而得到结论.
		
		\qquad
		
		不妨设$x_0\not=0$,于是$\dot{x}\in D(T_0)$和$x\in D_+(T_0)$,于是计算这个切空间之间切映射的核只需计算态射$D(T_0)\to D_+(T_0)$在点$x$的切映射的核,而这个态射是$(x_0,\cdots,x_n)\mapsto(x_0^{-1}x_1,\cdots,x_0^{-1}x_n)$,按照引理它的切映射是如下$n\times(n+1)$矩阵,它的核是$(x_0,\cdots,x_n)$所在的一维子空间.
		$$\left(\begin{array}{cccc}-\frac{x_1}{x_0^2}&1&&\\\vdots&&\ddots&\\-\frac{x_n}{x_0^2}&&&1\end{array}\right)$$
	\end{proof}
    \item 设$k$是域,设$X=V_+(f_1,\cdots,f_r)\subseteq\mathbb{P}_k^n$是闭子概型,其中$f_i\in k[T_0,T_1,\cdots,T_n]$是齐次多项式.设$x\in X(k)$是$k$有理点,我们断言有如下同构,其中$kx$表示$x\subseteq k^{n+1}$所在的一维子空间.
    $$\mathrm{T}_xX=\left(\ker\left(\frac{\partial f_i}{\partial T_j}(x)\right)_{i,j}\right)/kx$$
    \begin{proof}
    	
    	记$X$的仿射锥为$V(f_1,\cdots,f_r)\subseteq\mathbb{A}_k^{n+1}$.考虑典范态射$C(X)-\{0\}\to X$,取$\dot{x}\in C(X)-\{0\}$,记它在上述态射下的像是$x\in X$,那么这个态射诱导了$\mathrm{T}_{\dot{x}}C(X)\to\mathrm{T}_xX$.并且我们之前证明过$\mathrm{T}_{\dot{x}}C(X)=\ker\left(\frac{\partial f_i}{\partial T_j}(x)\right)$.于是问题归结为证明这个切映射的核是$kx$,并且是满射.
    	
    	\qquad
    	
    	不妨设$\dot{x}=x=(a_0,a_1,\cdots,a_n)$满足$a_0\not=0$.于是我们的态射$C(X)-\{0\}\to X$可以限制为$C(X)\cap D(T_0)\to X\cap D_+(T_0)$.也即如下态射,其中$g_i(S_1,\cdots,S_n)=f_i(1,S_1,\cdots,S_n)$:
    	$$\mathrm{Spec}\frac{k[T_0,\cdots,T_n,Y]}{(f_1,\cdots,f_r,T_0Y-1)}\to\mathrm{Spec}\frac{k[S_1,\cdots,S_n]}{g_1(S_1,\cdots,S_n),\cdots,g_r(S_1,\cdots,S_n)}$$
    	$$(T_0,\cdots,T_n,Y)\mapsto(T_1Y,T_2Y,\cdots,T_nY)$$
    	
    	$\mathrm{Spec}k[T_0,\cdots,T_n,Y]/(f_1,\cdots,f_r,T_0Y-1)$在点$x'=(a_0,a_1,\cdots,a_n,1/a_0)$处的切空间是如下$(r+1)\times(n+2)$矩阵的核:
    	$$A=\left(\begin{array}{ccccc}\frac{\partial f_1}{\partial T_0}(x)&\frac{\partial f_1}{\partial T_1}(x)&\cdots&\frac{\partial f_1}{\partial T_n}(x)&0\\\vdots&\vdots&\ddots&\vdots&\vdots\\\frac{\partial f_r}{\partial T_0}(x)&\frac{\partial f_r}{\partial T_1}(x)&\cdots&\frac{\partial f_r}{\partial T_n}(x)&0\\\frac{1}{a_0}&0&\cdots&0&a_0\end{array}\right)$$
    	
    	类似的$\mathrm{Spec}k[S_1,\cdots,S_n]/g_1(S_1,\cdots,S_n),\cdots,g_r(S_1,\cdots,S_n)$在点$(a_1/a_0,\cdots,a_n/a_0)$处的切空间是如下$r\times n$矩阵的核(我们有$\frac{\partial g_i}{\partial S_j}(a_1/a_0,\cdots,a_n/a_0)=a_0^{-r_i}\frac{\partial f_i}{\partial T_j}(a_0,\cdots,a_n)$,其中$r_i=\deg g_i$):
    	$$B=\left(\begin{array}{ccc}\frac{\partial f_1}{\partial T_1}(x)&\cdots&\frac{\partial f_1}{\partial T_n}(x)\\\vdots&\ddots&\vdots\\\frac{\partial f_r}{\partial T_1}(x)&\cdots&\frac{\partial f_r}{\partial T_n}(x)\end{array}\right)$$
    	
    	这个切空间之间的态射是如下矩阵$C$,也即如果$v=(v_0,\cdots,v_n,v_{n+1})\in\ker A$,那么$Cv\in\ker B$(验证这件事要用到这样一个事实:如果$f(T_0,\cdots,T_n)$是$m$次齐次多项式,那么$\sum_{0\le i\le n}\frac{\partial f}{\partial T_i}(T)T_i=mf(T)$).
    	$$C=\left(\begin{array}{cccccc}0&\frac{1}{a_0}&0&\cdots&0&a_1\\0&0&\frac{1}{a_0}&\cdots&0&a_2\\\vdots&\vdots&\vdots&\ddots&\vdots&\vdots\\0&0&0&\cdots&\frac{1}{a_0}&a_n\end{array}\right)$$
    	
    	切映射是满射是因为如果$v=(v_1,\cdots,v_n)\in\ker B$,那么$w=(0,a_0v_1,a_0v_2,\cdots,a_0v_n,0)\in\ker A$满足$Cw=v$.下面求切映射的核,如果$w=(w_0,w_1,\cdots,w_n,w_{n+1})\in\ker A$,那么$Cw=0$当且仅当$w_i=a_iw_0/a_0,\forall 0\le i\le n$,此即$(w_0,w_1,\cdots,w_n)$落在$(a_0,\cdots,a_n)$所在的一维子空间中.
    \end{proof}
\end{enumerate}
\subsection{相对切空间}

考虑如下交换图表,我们用$\mathrm{T}_{\xi}(X/S)$表示全体$S$态射$t:\mathrm{Spec}K[\varepsilon]\to X$,使得它和典范态射$\mathrm{Spec}K\to\mathrm{Spec}k[\varepsilon]$的复合是$\xi$.类似把Zariski切空间视为$k[\varepsilon]$值点中我们解释切向量的和与数乘,我们可以验证$\mathrm{T}_{\xi}(X/S)$是一个$K$线性空间,它称为$X$的关于$\xi$的在$S$上的相对切空间.如果$X$是$S$概形,$\xi$取为典范态射$\mathrm{Spec}\kappa(x)\to X$,其中$x\in X$,则把相对切空间记作$\mathrm{T}_x(X/S)$.
\begin{enumerate}
	\item 如果取$S=\mathrm{Spec}k=\mathrm{Spec}K$,并且$x$是$k$有理点,那么$\mathrm{T}_x(X/k)=\mathrm{T}_xX$.但是一般的即便$X$是域$k$上的有限型概形,如果$x$不是$k$有理点,那么$\kappa(x)$线性空间$\mathrm{T}_xX$和$\mathrm{T}_x(X/k)$可能未必同构.
	\item 按照纤维积的泛性质,态射$\xi$对应于$k$概形$X\times_S\mathrm{Spec}K$的$K$值点$\overline{x}$.同理$S$态射$\mathrm{Spec}k[\varepsilon]\to X$对应于$K$态射$\mathrm{Spec}k[\varepsilon]\to X\times_S\mathrm{Spec}K$.我们解释过对一般的$K$概形$Y$有$Y(K[\varepsilon])_x\cong\mathrm{T}_xY$,于是这里$(X\times_S\mathrm{Spec}K)(K[\varepsilon])_{\overline{x}}\cong\mathrm{T}_{\overline{x}}(X\times_S\mathrm{Spec}K)$.前者恰好是复合上$\mathrm{Spec}K\to\mathrm{Spec}K[\varepsilon]$为$\overline{x}$的态射.综上我们证明了$\mathrm{T}_{\xi}(X/S)=\mathrm{T}_{\overline{x}}(X\times_S\mathrm{Spec}K)$.
	\item 相对切空间关于$\xi$具有函子性.如果$K\to L$是域扩张,对应的态射是$p:\mathrm{Spec}L\to\mathrm{Spec}K$,那么有$L$线性空间同构:$\mathrm{T}_{\xi}(X/S)\otimes_KL\cong\mathrm{T}_{\xi\circ p}(X/S)$.【】
	\item 特别的,如果$X$是域$k$上的概形,设$x\in X$,设$k\subseteq k'$是域扩张,记$x'\in X'=X\times_kk'$投影在点$x$(满射在基变换下不变),记$\xi$是复合态射$\mathrm{Spec}\kappa(x')\to\mathrm{Spec}\kappa(x)\to X$,那么有同构:
	$$\mathrm{T}_{x'}(X'/k')=\mathrm{T}_{\xi}(X/k)\cong\mathrm{T}_x(X/k)\otimes_{\kappa(x)}\kappa(x')$$
	\item 相对切空间的维数是上半连续的:设$k$是域,设$X$是局部有限型$k$概形,对每个整数$d$,有$\{x\in X\mid\dim_{\kappa(x)}\mathrm{T}_x(X/k)\ge d\}$是$X$中的闭集.
	\begin{proof}
		
		闭集是局部性质,不妨设$X=\mathrm{Spec}A$,其中$A=k[T_1,\cdots,T_n]/(f_1,\cdots,f_r)$.对$x\in X$,我们有$\mathrm{T}_x(X/k)=\mathrm{T}_x(X\times_k\kappa(x))=\ker(J_{f_1,\cdots,f_r}(x))\subseteq\kappa(x)^n$.于是相对切空间维数$\ge d$等价于讲雅各比矩阵的行列式的秩$\le n-d$,也即雅各比矩阵的所有$n-d+1$阶子式为零,这些条件对应了闭集.
	\end{proof}
\end{enumerate}
\newpage
\section{除子}
\subsection{Cartier除子}
\begin{enumerate}
	\item 如果$\mathscr{F}$是$X$上的环层,用$\mathscr{F}^*$表示$\mathscr{F}^*(U)=(\mathscr{F}(U))^*$即取乘法群,环同态肯定把单位映成单位,单位的粘合肯定是单位,所以$\mathscr{F}^*$总是一个阿贝尔层(二元运算取为乘法).它可以作为环预层范畴到阿贝尔预层范畴的函子,并且它和层化函子是可交换的,换句话讲如果$\overline{\mathscr{F}}$是环预层$\mathscr{F}$的层化,那么$\overline{\mathscr{F}}^*$是$\mathscr{F}^*$的层化.
	\item 设$X$是概形,记$\mathscr{K}_X$是$X$上的亚纯函数层,记$\mathrm{CDiv}(X)=\mathrm{H}^0(X,\mathscr{K}^*_X/\mathscr{O}_X^*)=\Gamma(X,\mathscr{K}^*_X/\mathscr{O}_X^*)$为$X$的Cartier除子群,它的元素称为$X$的Cartier除子.按照$\mathscr{K}_X^*/\mathscr{O}_X^*$是$U\mapsto(\mathscr{K}_X')^*(U)/\mathscr{O}_X^*(U)$的层化,并且预层的层化上的截面可以表示为一族局部上的预层截面,说明一个Cartier除子可以表示为$(U_i,f_i)_{i\in I}$,其中$\{U_i,i\in I\}$是$X$的开覆盖,而$f_i\in\Gamma(U_i,(\mathscr{K}_X')^*)$是$\mathscr{O}_X(U_i)$的两个正则元(甚至是stalk处处是正则元的正则元)的商,并且满足$f_i\mid_{U_i\cap U_j}\in f_j\mid_{U_i\cap U_j}\mathscr{O}_X(U_i\cap U_j)^*$对任意$i,j\in I$成立.
	\item 如果Cartier除子$D_1,D_2$分别用$(U_i,f_i)_{i\in I}$和$(V_j,g_j)_{j\in J}$表示,用$f_ig_j$表示$f_i\mid_{U_i\cap V_i}g_j\mid_{U_i\cap V_i}$,那么$D_1=D_2$当且仅当$f_ig_j^{-1}\in\mathscr{O}_X(U_i\cap U_j)^*$对任意$i,j\in I$成立.
	\item 虽然$\mathscr{K}^*_X/\mathscr{O}_X^*$是乘法群,但是我们把除子群$\mathrm{Div}(X)$的二元运算记作加法.如果Cartier除子$D_1,D_2$分别用$(U_i,f_i)_{i\in I}$和$(V_j,g_j)_{j\in J}$表示,用$f_ij_j$表示$f_i\mid_{U_i\cap V_i}g_j\mid_{U_i\cap V_i}$,那么$D_1+D_2$可用$(U_i\cap V_j,f_ig_j)_{i\in I,j\in J}$表示.另外如果$D=(U_i,f_i)_{i\in I}$,那么$-D=(U_i,f_i^{-1})_{i\in I}$.$\mathrm{Div}(X)$的幺元就是$(X,1_{\mathscr{O}_X(X)})$.
	\item 设$f\in\mathrm{H}^0(X,\mathscr{K}_X^*)$,它在$\mathrm{H}^0(X,\mathscr{K}^*_X/\mathscr{O}_X^*)$中的像记作$\mathrm{CDiv}(f)$,称为主Cartier除子(principal Cartier divisor),换句话讲它的表示是$(X,f)$.全体主除子构成了$\mathrm{CDiv}(X)$的子群,记作$\mathrm{PCDiv}(X)$.称$\mathrm{CDiv}(X)/\mathrm{PCDiv}(X)$为Cartier除子类群,记作$\mathrm{CaCl}(X)$.称两个Cartier除子$D_1,D_2$是线性等价的,如果$D_1-D_2$是主除子.另外我们有如下正合列(前两个截面群记作乘法,后两个除子群记作加法):
	$$\xymatrix{1\ar[r]&\Gamma(X,\mathscr{O}_X)^*\ar[r]&\Gamma(X,\mathscr{K}_X)^*\ar[r]&\mathrm{Div}(X)\ar[r]&\mathrm{CaCl}(X)\ar[r]&0}$$
	\item 称$\mathrm{H}^0(X,\mathscr{O}_X\cap\mathscr{K}_X^*/\mathscr{O}_X^*)$在$\mathrm{CDiv}(X)$中的像是有效Cartier除子(effective Cartier divisor).换句话讲,有效除子$D$是指可以表示为$(U_i,f_i)_{i\in I}$,其中$f_i\in\mathscr{O}_X(U_i)$的除子,记作$D\ge0$.如果$D,D'\in\mathrm{CDiv}(X)$满足$D-D'$是有效除子,我们也记作$D\ge D'$.全体有效除子构成的集合记作$\mathrm{CDiv}_+(X)$.
	\item 有效Cartier除子一一对应于余维数1的正则闭子概型$Y$:一方面如果$D$是有效Cartier除子,那么$\mathscr{L}(-D)\subseteq\mathscr{O}_X$是可逆层,并且局部上被一个正则元定义,所以它定义了$X$的一个余维数1的正则闭子概型$Y$.反过来如果$Y$是余维数1的正则闭子概型,那么可取$X$仿射开覆盖$\{U_i\}$,可取正则元$f_i\in\mathscr{O}_X(U_i)$使得$Y\cap U_i=V(f_i),\forall i$.于是$\{(U_i,f_i)\}$对应了一个有效Cartier除子.
\end{enumerate}
\subsection{Cartier除子的支集}

设$D$是Cartier除子,它的支集定义为$\mathrm{Supp}(D)=\{x\in X\mid D_x\not=1\}$(因为阿贝尔层$\mathscr{K}_X^*/\mathscr{O}_X^*$用的是乘法记号,所以幺元记为1).如果$D=\{(U_i,f_i)\}$,那么$x\in\mathrm{Supp}(D)$当且仅当对某个指标(也等价于对每个指标)$i$有$(f_i)_x\not\in\mathscr{O}_{X,x}^*$,这也等价于讲$\mathscr{L}(D)_x\not=\mathscr{O}_{X,x}$.
\begin{enumerate}
	\item 我们解释过截面的支集总是一个闭集.
	\item 如果$D$是有效Cartier除子,那么$\mathrm{Supp}(D)$恰好是$D$对应的$X$的余维数1的正则闭子概型的底空间.这件事是因为如果记有效Cartier除子$D=\{(U_i,f_i)\}$,那么$x\in\mathrm{Supp}(D)$当且仅当使得$x\in U_i$的指标$i$要满足$(f_i)_x\not\in\mathscr{O}_{X,x}^*$,这等价于讲$\mathscr{L}(-D)\mid_{U_i}$在$x$处的stalk是$\mathscr{O}_{X,x}$的真理想,此即$x\in\mathrm{Supp}\mathscr{L}(-D)$,也即$x\in Y$.
	\item 设$D=\{(U_i,f_i)\}$是Cartier除子,对开集$V$,有$\Gamma(V,\mathscr{L}(D))$由那些$sf_i\in\Gamma(U_i\cap V,\mathscr{O}_X),\forall i$的$s\in\Gamma(V,\mathscr{K}_X)$构成.于是如果记$U_{\mathrm{eff}}$为$D$的全部有效点构成的集合(此即那些满足$(f_i)_x\in\mathscr{O}_{X,x},\forall i$的点$x\in X$构成的集合,这必然是一个开集,因为只要有一个满足$x\in U_i$的$i$使得$(f_i)_x\in\mathscr{O}_{X,x}$,那么其它的满足$x\in U_j$的指标$j$就都有$(f_j)_x\in\mathscr{O}_{X,x}$,于是我们只要找$x$仿射开邻域$U=\mathrm{Spec}A$上的主开集$D(f)$使得处处有$(f_i)_y\in\mathscr{O}_{X,y}$即可,而这是因为如果$x$对应的素理想$\mathfrak{p}$,如果$f_i\mid_U=m$,那么存在$n\in\mathscr{O}_{X,x}$和$s\in A-\mathfrak{p}$使得$m/1=n/s$,于是$y\in D(s)$满足$(f_i)_y\in\mathscr{O}_{X,y}$).那么$1\in\Gamma(U_{\mathrm{eff}},\mathscr{K}_X)$的确落在$\Gamma(U_{\mathrm{eff}},\mathscr{L}(D))$中,它称为$D$或者$\mathscr{L}(D)$的典范截面,记作$s_D$.
	\item 设$D$是Cartier除子,如果$U$是支集$\mathrm{Supp}(D)$的补集,那么有$U\subseteq U_{\mathrm{eff}}$,并且$s_D\mid_U$就定义了一个同构$\mathscr{L}(D)\mid_U\cong\mathscr{O}_X\mid_U$.
	\item 设$X$是概形,设$\mathscr{L}$是$X$上的线丛,设$\mathrm{R}_{\mathscr{L}}$由$\Gamma(X,\mathscr{L})$中的$\Gamma(X,\mathscr{O}_X)$正则元构成(此即那些$s\in\mathscr{L}(X)$,满足对任意$0\not=t\in\mathscr{O}_X(X)$有$ts\not=0$).我们有如下典范双射,这里等价关系定义为$s\sim s'$当且仅当存在$a\in\mathscr{O}_X(X)^*$使得$s'=as$:
	$$\{\text{满足}\mathscr{L}(D)\cong\mathscr{L}\text{的有效Cartier除子}D\}\cong\mathrm{R}_{\mathscr{L}}/\sim$$
	
	特别的,如果$X$是代数闭域上的连通整射影曲线,那么所有满足$\mathscr{L}(D)$线性等价于一个固定线丛$\mathscr{L}$的有限Cartier除子$D$一一对应于集合$(\mathrm{H}^0(X,\mathscr{L})-\{0\})/k^*$.
	\begin{proof}
			
		先取有效除子$D$和同构$\alpha:\mathscr{L}(D)\cong\mathscr{L}$,此时$U_{\mathrm{eff}}=X$,设典范截面$s_D$在$\alpha$下的像为$s\in\mathrm{R}_{\mathscr{L}}$,这得到左侧到右侧的映射.再取$s\in\mathrm{R}_{\mathscr{L}}$,选取$X$的开覆盖$\{U_i\}$,使得$\mathscr{L}\mid_{U_i}\cong\mathscr{O}_{U_i}$,那么$s\mid_{U_i}$在这个同构下对应了一个截面$f_i\in\mathscr{O}_X(U_i)$,并且$\{(U_i,f_i)\}$的确构成一个有效Cartier除子,并且如果把$s$替换为$as$,其中$a\in\mathscr{O}_X(X)^*$,得到的有效除子是相同的,于是按照$\mathscr{L}(D)$的定义就得到一个同构$\mathscr{L}(D)\cong\mathscr{L},$这就得到右侧到左侧的映射.最后验证它们互为逆映射即可.
	\end{proof}
	\item 如果$X$是局部诺特概形,并且$D$是Cartier除子,那么$\mathrm{Supp}(D)$的余维数$\ge1$,换句话讲,对每个$z\in\mathrm{Supp}(D)$有$\dim(\mathscr{O}_{X,z})\ge1$.
	\begin{proof}
	    	
	    如果某个点$\eta\in X$使得$\dim(\mathscr{O}_{X,\eta})=0$,那么$\mathscr{O}_{X,\eta}$是零维阿廷环,于是$\mathscr{O}_{X,\eta}$的每个正则元都是可逆元,于是$\mathscr{O}_{X,\eta}=\mathrm{Frac}(\mathscr{O}_{X,\eta})=\mathscr{K}_{X,\eta}$(这里$\mathrm{Frac}$是全商环),于是$D_{\eta}\in\mathscr{K}_{X,\eta}^*/\mathscr{O}_{X,\eta}^*=1$,这迫使$\eta$不在$\mathrm{Supp}(D)$中.
	\end{proof}
\end{enumerate}
\subsection{除子类群和Picard群}
\begin{enumerate}
	\item 设$D$是概形$X$上的Cartier除子,设它的一个表示为$\{(U_i,f_i)\}$,定义$\mathscr{L}(D)$是模层$\mathscr{K}_X$的子模层,满足$\mathscr{L}(D)\mid_{U_i}=f_i^{-1}\mathscr{O}_X\mid_{U_i}$是秩1自由模.这个定义良性是因为$f_if_j^{-1}\in\mathscr{O}_X(U_i\cap U_j)$,导致$f_i^{-1}\mathscr{O}_X(U_i\cap U_j)=f_j^{-1}\mathscr{O}_X(U_i\cap U_j)$,并且选取不同的表示$\{(V_j,g_j)\}$得到相同的子模层.于是$\mathscr{L}(D)$是可逆模层.另外在整概形上我们有如下表示:
	$$\mathscr{L}(D)=\{f\in\mathscr{K}_X^*(U)\mid\mathrm{div}(f)+D\mid_U\ge0\}\cup\{0\}$$
	\item 我们断言且$\rho:D\mapsto\mathscr{L}(D)$是$\mathrm{CDiv}(X)\to\mathrm{Pic}(X)$的群同态,$\rho$的像集是$\mathscr{K}_X$的全部可逆子模层,并且$\ker\rho=\mathrm{PCDiv}(X)$,换句话讲$\rho$诱导了单同态$\mathrm{CaCl}(X)\to\mathrm{Pic}(X)$.特别的,这说明两个Cartier除子$D_1,D_2$是线性等价的当且仅当线丛$\mathscr{L}(D_1)$和$\mathscr{L}(D_2)$是同构的.
	\begin{proof}
		
		如果$D_1$被$\{f_i\in\mathscr{O}_X(U_i)\}$表示,$D_2$被$\{g_j\in\mathscr{O}_X(V_j)\}$表示.那么$\mathscr{L}(D_1+D_2)$就是$f_ig_j$局部生成的可逆模层,也即$\mathscr{L}(D_1+D_2)=\mathscr{L}(D_1)\mathscr{L}(D_2)$,但是它们是可逆模层,就有$\mathscr{L}(D_1)\mathscr{L}(D_2)\cong\mathscr{L}(D_1)\otimes_{\mathscr{O}_X}\mathscr{L}(D_2)$.另外明显有$\mathscr{L}(0)=\mathscr{O}_X$.
		
		\qquad
		
		主除子可取表示$(X,f)$,所以它对应的是秩1自由模层,于是同构于$\mathscr{O}_X$本身.这说明主除子都落在$\ker\rho$中.于是$\rho$诱导了同态$\mathrm{CaCl}(X)\to\mathrm{Pic}(X)$.反过来如果Cartier除子$D\in\ker\rho$,于是$\mathscr{L}(D)$同构于模层$\mathscr{O}_X$,此即存在$f\in\Gamma(X,\mathscr{K}_X)^*$使得$\mathscr{L}(D)\cong f\mathscr{O}_X$.所以$D$明显是$\mathrm{Div}(f)$.
	\end{proof}
	\item 按照定义,$D$是有效除子当且仅当它的表示$\{(U_i,f_i)\}$满足$f_i\in\mathscr{O}_X(U_i)$,这等价于讲$\mathscr{L}(-D)\subseteq\mathscr{O}_X$.进而有$D_1\le D_2$当且仅当$\mathscr{L}(D_1)\subseteq\mathscr{L}(D_2)$.
	\item 存在这样的概形$X$使得$\mathrm{CaCl}(X)\to\mathrm{Div}(X)$不是满射,因为存在这样的可逆模层的同构类不在$\mathscr{K}_X$中.
	\item 整概形满足典范映射$\mathrm{CaCl}(X)\to\mathrm{Div}(X)$是同构.
	\begin{proof}
		
		只需证明$X$的每个可逆模层都同构于$\mathscr{K}_X$的子模层.但是整概形的情况下$\mathscr{K}_X$是函数域$K$的常值层.设$\mathscr{L}$是可逆模层,考虑模层$\mathscr{L}\otimes_{\mathscr{O}_X}\mathscr{K}$,如果开集$U$使得$\mathscr{L}\mid_U\cong\mathscr{O}_U$,有$\mathscr{L}\otimes_{\mathscr{O}_X}\mathscr{K}_X\mid_U\cong\mathscr{K}_X\mid_U$是常值层.按照$X$是不可约空间,开覆盖上常值层的粘合还是一个常值层,于是$\mathscr{L}\otimes_{\mathscr{O}_X}\mathscr{K}_X\cong\mathscr{K}_X$,按照$\mathscr{O}_X\to\mathscr{K}_X$是单态射,有$\mathscr{L}\cong\mathscr{L}\otimes_{\mathscr{O}_X}\mathscr{O}_X\to\mathscr{L}\otimes_{\mathscr{O}_X}\mathscr{K}_X\cong\mathscr{K}_X$是单态射,于是$\mathscr{L}$是$\mathscr{K}_X$的子层.
	\end{proof}
	\item 没有嵌入点的诺特概形(例如既约诺特概形)满足典范映射$\mathrm{CaCl}(X)\to\mathrm{Div}(X)$是同构.
	\begin{proof}
		
		同样的只需证明$X$的每个可逆模层都同构于$\mathscr{K}_X$的子模层.按照诺特条件,$X$的一般点个数有限,全部一般点记作$\{\xi_1,\cdots,\xi_n\}$,可取$\xi_i$在$X-\cup_{j\not=i}\overline\{\xi_j\}$内的开邻域$U_j$,那么$U_j\subseteq\overline{\{\xi_i\}}$,再适当缩小$U_j$,可约定$\mathscr{L}\mid_{U_j}$都是自由的.记$U=\cup_{1\le j\le n}U_j$,设$i:U\to X$是包含映射,由于$U$包含了$X$的全部伴随点,所以$\mathscr{L}\to i_*\mathscr{L}\mid_U$是单态射.于是有$\mathscr{L}\to i_*\mathscr{L}\mid_U\cong i_*\mathscr{O}_U\to i_*\mathscr{K}_U\cong\mathscr{K}_X$.
	\end{proof}
    \item 更一般的,如果$X$是局部诺特概形,并且存在一个仿射开子集是概形稠密开集,那么典范映射$\mathrm{CaCl}(X)\to\mathrm{Div}(X)$是同构.
    \begin{proof}
    	
    	我们依旧要证明的是$X$的每个可逆层都同构于$\mathscr{K}_X$的某个子模层.设$U$是概形稠密仿射开集,设$\mathscr{L}$是$X$上的可逆层,我们先证明嵌入$s:\mathscr{L}\mid_U\to\mathscr{K}_X\mid_U$可以延拓为嵌入$\widetilde{s}:\mathscr{L}\to\mathscr{K}_X$.一旦这成立,我们就可以设$X$本身是仿射的.使得$\mathscr{L}\mid_V\cong\mathscr{O}_V$的开子集$V\subseteq X$构成了$X$的拓扑基.那么对这样的$V$有$U\cap V$是$V$的概形稠密开集.所以我们只要证明$s\mid_{U\cap V}$总能唯一的延拓到$V$上.但是$s\mid_{U\cap V}$可对应于一个截面$t\in\Gamma(U,\mathscr{K}_X)$,而$s$是单射等价于讲$t\in\Gamma(U,\mathscr{K}_X)^*$.由于$X$是局部诺特的,它的有理函数层和亚纯函数层是一致的,于是有$\Gamma(X,\mathscr{K}_X)\cong\Gamma(U,\mathscr{K}_X)$,这就说明$t$可以唯一延拓为$\Gamma(X,\mathscr{K}_X)^*$中的元,于是这个唯一延拓是单射.
    	
    	\qquad
    	
    	下面归结到仿射情况.设$X=\mathrm{Spec}A$,其中$A$是诺特环.设$\mathscr{L}$是可逆$\mathscr{O}_X$模层,于是它可以表示为$\widetilde{M}$,其中$M$是局部秩1的投射$A$模.取$S\subseteq A$是正则元构成的乘性闭子集,那么我们解释过诺特条件下有$\Gamma(X,\mathscr{K}_X)=S^{-1}A$.于是问题归结为构造一个模的单同态$M\to S^{-1}A$.我们知道正则元集合$S$的补集是有限个伴随素理想的并,这导致和$S$不交的极大理想个数是有限的,也即$S^{-1}A$总是半局部环.但是一般的如果$N$是半局部环$R$上的有限表示平坦模,使得对每个极大理想$\mathfrak{m}$有$\dim_{\kappa(\mathfrak{m})}N/\mathfrak{m}N$是相等的,那么有$N$是这个秩的自由模.这里$S^{-1}M$是有限秩1投射$A$模,于是$S^{-1}M\cong S^{-1}A$,于是我们有模的单同态$M\to S^{-1}M$.
    \end{proof}
    \item 再例如如果$X$是诺特环$R$上的拟射影概形,那么典范映射$\mathrm{CaCl}(X)\to\mathrm{Div}(X)$是同构.这件事是因为首先$X$是诺特概形,按照上一条归结为证明存在概形稠密的仿射开子集.但是诺特条件保证$X$只有有限个伴随点,并且一个开集是概形稠密开集当且仅当它包含全部伴随点.但是qcqs概形如果存在丰沛可逆层,对任意有限点集$Z$,一定能找到仿射开子集包含了$Z$.
\end{enumerate}
\subsection{Weil除子}
\begin{enumerate}
	\item 设$X$是诺特概形,素除子(prime divisor)指的是它的余维数1的整闭子概型(对一个不可约闭子集$Y$,它的一般点记作$\xi$,那么$Y$的余维数定义为$\dim\mathscr{O}_{X,\xi}$),全部素除子生成的自由阿贝尔群记作$\mathrm{WDiv}(X)$,它的元素称为Weil除子.所以一个Weil除子可以表示为有限和$D=\sum_in_iY_i$,其中$Y_i$是素除子.如果这里所有$n_i\ge0$,就称$D$是有效Weil除子(effective Weil divisor).全体Weil有效除子构成的集合记作$\mathrm{WDiv}_+(X)$.
	\item 亚纯函数沿素除子的阶数.设$f\in\Gamma(U,\mathscr{K}_X)$是亚纯函数,设$C=\overline{\{\xi\}}$是一个素除子,设$\xi\in U$,记$A=\mathscr{O}_{X,\xi}$,于是$\dim A=1$,并且$f_{\xi}\in\mathrm{Frac}(A)$(我们解释过对局部诺特概形$X$,对$x\in X$总有$\mathscr{K}_{X,x}=\mathrm{Frac}(\mathscr{O}_{X,x})$),于是它可表示为$a/b$,其中$a,b\in\mathscr{O}_{X,\xi}$.定义$f$沿素除子$C$的阶数$\mathrm{ord}_C(f)=l_A(A/(a))-l_A(A/(b))$,其中$l_A(M)$表示$A$模$M$的长度.我们断言定义的$a/b\mapsto l_A(A/(a))-l_A(A/(b))$是加法群同态,并且有$A^*\subseteq\ker(\mathrm{ord})$.
	\begin{proof}
		
		这是一个纯代数问题,设$A$是一维诺特局部环,设$f=a/b\in\mathrm{Frac}(A)$,其中$a,b\in A$.那么$a,b$是$A$的正则元,由于$A$是诺特一维环,导致$A/(a)$和$A/(b)$都是零维诺特环,也即阿廷环,于是它们的长度有限.假设$x,y$是$A$的两个正则元,那么$A/(x)\cong yA/(xy)$,考虑短正合列$0\to yA/(xy)\to A/(xy)\to A/(x)\to0$,就得到$l_A(A/(xy))=l_A(A/(x))+l_A(A/(y))$.这个等式可以同时说明$l_A(A/(a))-l_A(A/(b))$不依赖于$f=a/b$表达式的选取,以及这个映射是同态.最后如果$u\in A^*$,那么$A/(u)$是零模,它的长度自然是零.
	\end{proof}
	\item 设$C=\overline{\{\xi\}}$是余维数1的不可约闭子概型,如果$\mathscr{O}_{X,\xi}$是DVR,那么对包含点$\xi$的某个开集$U$上的亚纯函数$f$,就有$\mathrm{ord}_C(f)=v_C(f_{\xi})$,其中$v_C$是$\mathscr{O}_{X,\xi}$上的规范离散赋值.这件事是因为如果$v_C(a)=n\ge0$,那么$l_A(A/(a))=n$.
	\item 设$X$是诺特概形,设$f$是开集$U$上的亚纯函数,设$Y=\overline{\{\xi\}}$是素除子,设$\xi\in U$,如果$n=\mathrm{ord}_C(f)>0$就称$\xi$是$f$的$n$阶零点(zero),或者称$f$沿$C$有$n$阶零点;如果$n=-\mathrm{ord}_C(f)>0$就称$\xi$是$f$的$n$阶极点(pole),或者称$f$沿$C$有$n$阶极点.
	\item 我们来构造典范的$\mathrm{cyc}:\mathrm{CDiv}(X)\to\mathrm{WDiv}(X)$.给定Cartier除子$D=\{(U_i,f_i)\}$.对素除子$C=\overline{\{\xi\}}$,设指标$t$使得$\xi\in U_t$,那么在相差一个$\mathscr{O}_{X,\xi}^*$的意义下$f_t$不依赖于$(U_i,f_i)$的选取.我们定义$\mathrm{ord}_C(D)=\mathrm{ord}_C(f_t)\in\mathbb{Z}$.假设素除子$C$不在$\mathrm{Supp}(D)$中,按照定义有$\mathrm{ord}_C(D)=0$.我们解释过$\mathrm{Supp}(D)$的余维数$\ge1$,说明每个包含在$\mathrm{Supp}(D)$的素除子恰好是$\mathrm{Supp}(D)$的不可约分支,但是按照$X$是诺特的,导致$\mathrm{Supp}(D)$也是诺特的,于是它的不可约分支是有限的,这就说明$\sum_C\mathrm{ord}_C(D)C$是有限形式和,我们就定义$\mathrm{cyc}(D)=\sum_C\mathrm{ord}_C(D)C$.这是一个群同态.
	\item 主Weil除子.设$f\in\Gamma(X,\mathscr{K}_X)^*$是可逆亚纯函数,把$\mathrm{cyc}(\mathrm{div}(f))$简记作$\mathrm{cyc}(f)$,具有这样形式的Weil除子称为主Weil除子,换句话讲它具有形式$\sum\mathrm{ord}_C(f)C$.主除子构成了$\mathrm{WDiv}(X)$的子群,它们的商群称为Weil除子类群,记作$\mathrm{Cl}(X)$.于是$\mathrm{cyc}$诱导了同态$\mathrm{CaCl}(X)\to\mathrm{Cl}(X)$.但是通常这个映射未必是单射也未必是满射.
	\item 设$X$是局部诺特概形,考虑如下条件:
	\begin{enumerate}
		\item $X$是正则概形,即每个$\mathscr{O}_{X,x}$都是正则局部环.
		\item $X$是局部分解概形,即每个$\mathscr{O}_{X,x}$都是唯一分解整环.
		\item 对每个$x\in X$,局部环$\mathscr{O}_{X,x}$是整环,对每个余维数1的整闭子概型$C$,对每个点$x\in C$,那么$\mathrm{Spec}(\mathscr{O}_{X,x})$的闭子概型$\mathrm{Spec}(\mathscr{O}_{X,x})\cap C$都具有形式$V(f)$,其中$f\in\mathscr{O}_{X,x}$是正则元.
		\item $X$是正规概形.
		\item 对每个余维数1的整闭子概型$C$,有$\mathscr{O}_{X,C}$是DVR.
	\end{enumerate}
	
	那么(a)$\Rightarrow$(b)$\Leftrightarrow$(c)$\Rightarrow$(d)$\Rightarrow$(e).
	\item 设$X$是诺特概形.
	\begin{enumerate}
		\item 如果$X$是正规概形,那么$\mathrm{cyc}:\mathrm{CDiv}(X)\to\mathrm{WDiv}(X)$是单同态,特别的诱导的$\mathrm{cyc}:\mathrm{CaCl}(X)\to\mathrm{Cl}(X)$是单射.另外这个证明说明条件下有效除子的像是有效除子,有效除子的原像是有效除子.
		\begin{proof}
			
			明显有$\mathrm{CDiv}_+(X)\cap\left(-\mathrm{CDiv}_+(X)\right)=0$和$\mathrm{WDiv}_+(X)\cap\left(-\mathrm{WDiv}_+(X)\right)=0$.所以我们只需证明$\mathrm{cyc}^{-1}\left(\mathrm{WDiv}_+(X)\right)=\mathrm{CDiv}_+(X)$,就有$\mathrm{cyc}^{-1}(0)=\mathrm{cyc}^{-1}(\mathrm{WDiv}_+(X))\cap\mathrm{cyc}^{-1}(-\mathrm{WDiv}_+(X))=\mathrm{CDiv}_+(X)\cap\left(-\mathrm{WDiv}_+(X)\right)=0$.
			
			\qquad
			
			按照定义明显有Cartier有效除子在$\mathrm{cyc}$下的像是Weil有效除子.反过来设$D$是一个Cartier除子使得$\mathrm{cyc}(D)$是有效Weil除子.不妨设$X=\mathrm{Spec}A$是仿射的,那么$A$是正规整环,并且可设$D=\mathrm{CDiv}(f)$,那么条件就是对每个高度1素理想$\mathfrak{p}$有$f\in A_{\mathfrak{p}}$,但是对于正规整环$A$,当$\mathfrak{p}$取遍高度1素理想时$\cap A_{\mathfrak{p}}=A$,这说明$f\in A$.
		\end{proof}
		\item 如果$X$是局部分解概形,那么$\mathrm{cyc}:\mathrm{CDiv}(X)\to\mathrm{WDiv}(X)$是同构,特别的诱导的$\mathrm{cyc}:\mathrm{CaCl}(X)\to\mathrm{Cl}(X)$是同构.
		\begin{proof}
			
			上一条说明单射,下面只需证明满射.设$Z$是一个Weil素除子,它作为闭子概型唯一的对应于一个拟凝聚理想层$\mathscr{I}_Z$.按照局部分解概形的等价描述,对每个点$x\in X$,有$\mathscr{I}_{Z,x}$被单个元$f_x\in\mathscr{O}_{X,x}$生成.于是存在$x$的开邻域$U$和一个元$f\in\Gamma(U,\mathscr{O}_X)$,使得它在$x$处的stalk是$f_x$,并且有$\mathscr{I}\mid_U=\mathscr{O}_Uf$.于是我们找到了$X$的一个开覆盖$\{U_i\}$,以及元素$f_i\in\Gamma(U_i,\mathscr{O}_X)$使得$\mathscr{I}\mid_{U_i}=\mathscr{O}_{U_i}f_i$.因为两个主理想相同当且仅当生成元相差一个单位,所以$\{(U_i,f_i)\}$是一个Cartier除子$D$,明显有$\mathrm{cyc}(D)=Z$,这说明满射.
		\end{proof}
	\end{enumerate}
	\item 推论.对于诺特整局部分解概形$X$,函数域记作$K(X)$,有$\mathrm{WDiv}(X)\cong\mathrm{CDiv}(X)\cong\mathrm{Pic}(X)$.对Weil除子$\sum_Cn_CC$,它对应的可逆模层$\mathscr{L}$是$\mathscr{L}(U)=\{f\in K(X)\mid\mathrm{ord}_C(f)\ge -n_C,C\cap U\not=\emptyset,C\text{是素除子}\}$.
	\item 除子的限制.设$U\subseteq X$是开子集,如果$D$是Cartier除子,它在$U$上的限制就理解为它作为商模层截面在更小开集上的限制.并且这把主除子映射为主除子,于是限制映射诱导了同态$\mathrm{Cl}(X)\to\mathrm{Cl}(U)$.如果$X$是诺特概形,如果$\sum_Cn_CC$是Weil除子,它在开集$U$上的限制定义为$\sum_Cn_C(C\cap U)$.如果$C\cap U$非空,就有$\mathscr{O}_{U,C\cap U}=\mathscr{O}_{X,C}$.这说明有如下交换图表:
	$$\xymatrix{\mathrm{CDiv}(X)\ar[rr]^{\mathrm{cyc}}\ar[d]_{\mathrm{res}}&&\mathrm{WDiv}(X)\ar[d]^{\mathrm{res}}\\\mathrm{CDiv}(U)\ar[rr]^{\mathrm{cyc}}&&\mathrm{WDiv}(U)}$$
	
	对诺特概形就有如下交换图表:
	$$\xymatrix{\mathrm{Pic}(X)\ar[d]&\mathrm{CaCl}(X)\ar[l]_{\mathscr{L}(-)}\ar[r]^{\mathrm{cyc}}\ar[d]&\mathrm{Cl}(X)\ar[d]\\\mathrm{Pic}(U)&\mathrm{CaCl}(U)\ar[l]_{\mathscr{L}(-)}\ar[r]^{\mathrm{cyc}}&\mathrm{Cl}(U)}$$
	\item 设$X$是诺特概形,设$Z$是闭子概型,并且不包含$X$的不可约分支,设$Z$在$X$中余维数1的全部不可约分支为$\{Z_1,\cdots,Z_r\}$,记$U=X-Z$,记$\mathrm{res}:\mathrm{Cl}(X)\to\mathrm{Cl}(U)$是限制映射诱导的同态,那么有如下正合列:
	$$\xymatrix{\oplus_{i=1}^r\mathbb{Z}Z_i\ar[r]&\mathrm{Cl}(X)\ar[r]&\mathrm{Cl}(U)\ar[r]&0}$$
	\begin{proof}
		
		设$C_U$是$U$上的Weil素除子,它在$X$中的闭包是$X$的素除子,于是限制映射$\mathrm{res}:\mathrm{WDiv}(X)\to\mathrm{WDiv}(U)$是满射,于是诱导的$\mathrm{Cl}(X)\to\mathrm{Cl}(U)$也是满射.而$\mathrm{WDiv}(X)\to\mathrm{WDiv}(U)$的核明显是$\oplus_{i=1}^r\mathbb{Z}Z_i$.
	\end{proof}
	\item 推论.设$X$是诺特局部分解概形,设$Z\subseteq X$是余维数$\ge1$的闭子概型,设$U=X-Z$,那么$\mathrm{Pic}(X)\to\mathrm{Pic}(U)$,$\mathscr{L}\mapsto\mathscr{L}\mid_U$是满射.另外如果$Z$在$X$中的余维数$\ge2$,扣掉它不会影响余维数1的部分,所以此时这个限制映射是同构.
	\item 推论.设$X$是诺特局部分解概形,设$D$是有效Cartier除子,设它定义的局部主闭子概型为$Y$,那么有$\mathscr{I}_Y\cong\mathscr{L}(-D)$.
\end{enumerate}
\subsection{Cartier除子的逆像}
\begin{enumerate}
	\item 设$f:X\to Y$是概形之间的态射,设$D$是$Y$上的Cartier除子,也即$\mathscr{K}_Y^*/\mathscr{O}_Y^*$的截面.要想定义$D$的关于$f$的逆像除子,我们要有一个相应的层态射,这只能在添加条件下实现.如果层态射$\mathscr{O}_Y\to f_*\mathscr{O}_X$可以延拓为层态射$\mathscr{K}_Y\to f_*\mathscr{K}_X$,那么这诱导了层态射$\mathscr{K}_Y^*/\mathscr{O}_Y^*\to f_*\mathscr{K}_X^*/f_*\mathscr{O}_X^*\to f_*(\mathscr{K}_X^*/\mathscr{O}_X^*)$,取整体截面这诱导了同态$\Gamma(Y,\mathscr{K}_Y^*/\mathscr{O}_Y^*)\to\Gamma(X,\mathscr{K}_X^*/\mathscr{O}_X^*)$,这个同态记作$f^*$,就把$Y$上的Cartier除子$D$的像记作$f^*D$,称为$D$的逆像除子.
	\item $f^*:\mathrm{CDiv}(Y)\to\mathrm{CDiv}(X)$是群同态,也即$f^*(D_1+D_2)=f^*(D_1)+f^*(D_2)$.并且如果$D$是主除子,那么$f^*D$也是主除子.于是逆像诱导了同态$\mathrm{CaCl}(Y)\to\mathrm{CaCl}(X)$.另外如果把可逆层的回拉也记作$f^*:\mathrm{Pic}(Y)\to\mathrm{Pic}(X)$,那么有$f^*(\mathscr{L}(D))=\mathscr{L}(f^*D)$.
	\item 设$f:X\to Y$是概形之间的态射,如果如下两个命题的任一成立,那么典范层态射$\mathscr{O}_Y\to f_*\mathscr{O}_X$可以延拓为层态射$\mathscr{K}_Y\to f_*\mathscr{K}_X$.
	\begin{enumerate}
		\item $f$是平坦态射.
		\item $X$是局部诺特既约概形,并且$X$的每个不可约分支都支配了$Y$的某个不可约分支(不可约空间的连续像是不可约的).另外这一条可以改进成$X$是连通分支是局部有限集的既约概形,并且$X$的每个不可约分支都支配了$Y$的某个不可约分支(但是暂时不证这件事).
	\end{enumerate}
	\begin{proof}
		
		我们只要证明$\mathscr{O}_Y\to f_*\mathscr{O}_X$可以延拓为相应预层之间的态射$\mathscr{K}_Y'\to f_*\mathscr{K}_X'$,为此只需证明对任意仿射开子集$V\subseteq Y$和$U\subseteq X$使得$f(U)\subseteq V$,环同态$\varphi:A=\mathscr{O}_Y(V)\to B=\mathscr{O}_X(U)$可以延拓为全商环之间的同态,而这等价于讲$\varphi$把$A$的正则元映射为$B$的正则元.在(a)下$f$是平坦的,那么$\varphi$是平坦的,此时它把正则元映为正则元.在(b)下有$B$是诺特既约环,此时正则元恰好是不在极小素理想中的元.但是由于$U$的不可约分支支配了$V$的,导致$\varphi$把$B$的极小素理想拉回为$A$的极小素理想,这说明$\varphi$把正则元映为正则元.
	\end{proof}
	\item 设$f:X\to Y$是概形之间的态射满足上一条的两个条件之一,如果$D$是$Y$的有效Cartier除子,它对应于一个余维数1的正则闭子概型$Z$,我们断言$f^*(D)$也是$X$的有效Cartier除子,并且它对应的余维数1的正则闭子概型的底空间就是$f^{-1}(Z)$.
	\begin{proof}
		
		归结为仿射情况,设$Y=\mathrm{Spec}A$和$X=\mathrm{Spec}B$,设$Z=V(t)$,其中$t\in A$是正则元,设$f$对应的环同态是$\varphi:A\to B$,那么$\varphi(t)$也是$B$的正则元,于是如果记$Z$对应的有效Cartier除子为$D$,那么$f^{-1}(Z)=V(\varphi(t))$就是$f^*(D)$对应的正则闭子概型的底空间.
	\end{proof}
\end{enumerate}
\subsection{除子类群的例子}
\begin{enumerate}
	\item 设$A$是诺特正规整环,那么$A$是UFD当且仅当每个高度1素理想都是主理想,换句话讲每个素除子都是主除子,此即$\mathrm{Cl}(A)=0$.特别的对每个域$k$和自然数$n$都有:
	$$\mathrm{Pic}(\mathbb{A}_k^n)=\mathrm{CaCl}(\mathbb{A}_k^n)=\mathrm{Cl}(\mathbb{A}_k^n)=0$$
	\item 设$X=\mathbb{P}_k^n$是域$k$上的射影空间,它是整概形也是正则概形,它的余维数1的整闭子概型$Y$总可以唯一表示为$D_+(f)$,其中$f$是一个首一齐次正次不可约多项式,记$\deg Y=\deg f$.对每个Weil除子$D=\sum_in_iY_i$,记$\deg D=\sum_in_i\deg Y_i$.记$H$是$x_0=0$定义的超曲面(也是素除子),那么:
	\begin{enumerate}
		\item 如果$D$是$d$次除子,那么$D$和$dH$线性等价.
		\item 对每个有理函数$f$(我们解释过整概形上亚纯函数就是有理函数),有$\deg(f)=0$.
		\item $\deg$诱导了同构映射$\mathrm{Cl}(X)\cong\mathbb{Z}$.进而有$\mathrm{Pic}(X)\cong\mathrm{CaCl}(X)\cong\mathrm{Cl}(X)\cong\mathbb{Z}$.这里生成元$H=\{x_0=0\}$对应于$\mathrm{Pic}(X)$的生成元就是扭曲层$\mathscr{O}(1)$.于是特别的$\mathbb{P}_k^n$上的可逆模层总同构于某个$\mathscr{O}(d)$.
	\end{enumerate}
	\begin{proof}
		
		设$g$是$d$次齐次多项式,它可以分解为不可约因式的乘积$g=g_1^{n_1}\cdots g_r^{n_r}$,那么$g$诱导的主Weil除子$\mathrm{Div}(g)$恰好是$D=\sum n_iY_i$,其中$Y_i$是被$g_i=0$定义的余维数1整闭子概型,于是$\deg D=\deg g$.所以如果$f=g/h$是射影空间上的有理函数,那么$\deg g=\deg h$,就导致$\mathrm{Div}(f)=\mathrm{Div}(g)-\mathrm{Div}(h)=\deg g-\deg h=0$.这证明了(b).接下来如果$D$是$d$次除子,它可以表示为$D=D_1-D_2$,其中$D_1,D_2$都是有效除子,对于有效除子$\sum_in_iY_i$,系数不为零的素除子$Y_i$对应于一个不可约齐次多项式$p_i$,那么$\prod_ip_i^{n_i}$生成的主除子就是这个有效除子.设$D_1,D_2$分别是齐次多项式$g_1,g_2$诱导的,那么$D-dH$就是$g_1/x_0^dg_2$诱导的主除子,这证明了$(a)$.由于$\deg$明显是满同态(比方说,$dH$的次数就是$d$),它的核恰好是主除子,这说明诱导了同构$\mathrm{Cl}(X)\cong\mathbb{Z}$.下面考虑可逆层$\mathscr{O}(1)$,它对应的Cartier除子就是$\{(x_i/x_j,D_+(x_ix_j)),(x_j/x_i,D_+(x_ix_j))\}$,而$H=\{x_0=0\}$对应的Cartier除子是$\{(x_0/x_i,D_+(x_i))\}$,它们是线性等价的Cartier除子【】,于是$\mathscr{L}(\mathscr{O}_X(1))$和$H$是同构的.
	\end{proof}
	\item 如果$Y$是$\mathbb{P}_k^2$的$d$次不可约曲线,按照之前给出的正合列$\mathbb{Z}\to\mathrm{Cl}(X)\to\mathrm{Cl}(X-Y)\to0$,就得到$\mathrm{Cl}(X-Y)\cong\mathbb{Z}/d\mathbb{Z}$.
	\item 设$k$是域,设$A=k[x,y,z]/(xy-z^2)$,设$X=\mathrm{Spec}A$,我们断言$\mathrm{Cl}(X)=\mathbb{Z}/2\mathbb{Z}$,并且它被$y=z=0$定义的素除子$Y$生成.特别的,这里Weil除子$Y$就不是局部主除子,它就不对应Cartier除子.
	\begin{proof}
		
		我们解释过有$\mathbb{Z}\to\mathrm{Cl}(X)\to\mathrm{Cl}(X-Y)\to0$.其中第一个映射是$1\mapsto 1\cdot Y$.这里$Y$是$X$上的$V(\overline{y})$,所以$X-Y$是$\left(k[x,y,z]/(xy-z^2)\right)_y=k[x,y,y^{-1},z]/(x-y^{-1}z^2)=k[y,y^{-1},z]$的素谱,而这个环是UFD,所以$\mathrm{Cl}(X-Y)=0$.另外$k[x,y,z]/(xy-z^2)$上正则函数$\overline{y}$生成的主除子是$2Y$,这里$2$的由来是计算$\left(k[x,y,z]/(xy-z^2)\right)/(\overline{y})$的长度,也即2.至此我们证明了$\mathrm{Cl}(X)$被$Y$生成,并且$2Y=0$.最后只需证明$\mathrm{Cl}(X)\not=0$,但是$A=k[x,y,z]/(xy-z^2)$是正规整环,它的除子类群平凡当且仅当它是UFD,也等价于高度1素理想都是主理想,但是$A$的高度1素理想$(y,z)$不是主理想.
	\end{proof}
    \item 设$X$是余维数1正则的(此即任意维数为1的局部环$\mathscr{O}_{X,x}$是正则局部环)诺特分离整概形,那么$X\times\mathbb{A}^1$(不加角标指的是$\mathbb{Z}$上的仿射线和纤维积)满足相同的条件,并且有$\mathrm{Cl}(X)\cong\mathrm{Ck}(X\times\mathbb{A}^1)$.
    \begin{proof}
    	
    	明显$X\times\mathbb{A}^1$是诺特分离概形,它是整概形也是容易的:取$X$的仿射开覆盖$\{U_i\}$,它们都是不可约的,那么$\{U_i\times\mathbb{A}^1\}$是$X\times\mathbb{A}^1$的不可约开覆盖,并且按照$U_i$两两有交得到$\{U_i\times\mathbb{A}^1\}$两两有交,于是$X\times\mathbb{A}^1$是不可约的,它是既约是平凡的.下面证明它是余维数1正则的:设$x\in X\times\mathbb{A}^1$是余维数1的,那么它的像$y\in X$只能是余维数1或者余维数0(也即$X$的一般点).
    	\begin{itemize}
    		\item 称$x$是第一类的,如果$y$是余维数1的点,因为$X\times\mathbb{A}^1$的一般点要映为$X$的一般点,以及$x$的余维数是1,于是$x$是$\pi^{-1}(y)$的一般点,其中$\pi:X\times\mathbb{A}^1\to X$是投影态射.那么有$\mathscr{O}_x$是$\mathscr{O}_y[t]$的局部化.所以一旦$\mathscr{O}_y$是DVR就得到$\mathscr{O}_x$是DVR.
    		\item 称$x$是第二类的,如果$y$是一般点,那么$\mathscr{O}_x$是$K[t]$在某个极大理想处的局部化,其中$K$是$X$的函数域(这个局部化关系就是$(A[T])_{\mathfrak{q}}=(A_{\mathfrak{p}}[T])_{\mathfrak{q}}$),进而$\mathscr{O}_x$仍然是DVR.
    	\end{itemize}
    
        我们解释过这个条件下除子的回拉$\pi^*$有意义.也即$\pi^*:\mathrm{Cl}(X)\to\mathrm{Cl}(X\times\mathbb{A}^1)$为$D=\sum n_iY_i\mapsto \pi^*D=\sum n_i\pi^{-1}(Y_i)$.如果$f\in K^*$,那么$\pi^*((f))$就是把$f$视为$X\times\mathbb{A}^1$的函数域$K(t)$中的元的除子.所以$\pi^*$可以视为除子类群上的映射.
        \begin{itemize}
        	\item $\pi^*$是单射:取$X$上的除子$D$,使得$\pi^*D=(f)$,其中$f\in K(t)$.设$f=g/h$,其中$g,h\in K[t]$互素.如果$g,h$不全在$K$中,那么除子$(f)$就包含了第二类的点,但是按照定义$\pi^*D$不包含这种素除子,于是$f\in K$,于是$D=(f)$,得到$\pi^*$是单射.
        	\item $\pi^*$是满射:问题归结为证明第二类的素除子一定可以线性等价于第一类的素除子的线性组合.任取第二类的素除子$Z\subseteq X\times\mathbb{A}^1$,它在$X$的一般点做局部化得到$\mathrm{Spec}K[t]$的素除子,也即一个素理想$\mathfrak{p}\subseteq K[t]$,这是主理想,记作$(f)$,那么$f\in K(t)$,并且$\mathrm{Div}(f)$只包含了$Z$和一些第一类素除子,于是$Z$线性等价于一些第一类素除子的线性组合.
        \end{itemize}
    \end{proof}
\end{enumerate}

\subsection{曲线上的Cartier除子}

设$X/k$是域$k$上的有限型1维概形.
\begin{enumerate}
	\item 除子的次数.
	\begin{enumerate}[(1)]
		\item 设$A$是一维诺特局部环,任取正则元$f\in A$,有$A/fA$是零维环,于是长度$l_A(A/fA)$是有限数.并且如果$g\in A$是另一个正则元,那么有加法$l_A(A/fgA)=l_A(A/fA)+l_A(A/gA)$(这件事只要构造短正合列).于是长度可以延拓为$A$的全商环的乘法群$\mathrm{Frac}(A)^*\to\mathbb{Z}$的同态.并且$A^*$落在这个同态的核里,于是它诱导了群同态$\mathrm{mult}_A:\mathrm{Frac}(A)^*/A^*\to\mathbb{Z}$.
		\item 设$X$是局部诺特概形,设$D$是Cartier除子,任取余维数1的点$x\in X$,那么$D_x$落在$(\mathscr{K}_X^*/\mathscr{O}_X^*)_x=\mathrm{Frac}(\mathscr{O}_{X,x})^*/\mathscr{O}_{X,x}^*$中,我们定义$D$在点$x$处的重数就为$\mathrm{mult}_x(D)=\mathrm{mult}_{\mathscr{O}_{X,x}}(D_x)$.
		\item 设$X$是仿射诺特概形,设$D$是Cartier除子,设$x\in X$是余维数1的点,如果$\mathrm{nult}_x(D)\not=0$,那么$x\in E=\mathrm{Supp}D$,但是$x\in X$的特殊化的余维数为零,这一定落在$X-E$中,于是$x$是$E$的一般点.但是按照$X$是仿射诺特的,于是这样的一般点至多有有限个,换句话讲我们证明了局部诺特概形的任意仿射开子集上,只有有限个余维数1的点$x$满足$\mathrm{mult}_x(D)\not=0$.
		\item 现在设$X/k$是域$k$上的有限型1维概形,设$D$是Cartier除子,定义$D$的次数如下,上一条解释了这实际上是有限和:
		$$\deg D=\sum_{x\in X}\mathrm{mult}_x(D)[\kappa(x):k]$$
		\item 设$X/k$是域$k$上的有限型1维概形,它的素除子就是闭点,它的Weil除子形如$\sum_pn_p[p]$,次数定义为$\sum_pn_p[\kappa(p):k]$.于是有$\deg D=\deg\mathrm{cyc}(D)$.
	\end{enumerate}
    \item 设$X/k$是有限型1维概形,设$E$是有效Cartier除子,记$(E,\mathscr{O}_E)$为可逆理想层$\mathscr{L}(-E)\subseteq\mathscr{O}_X$定义的闭子概型.那么如果$D$是非零有效除子,就有$\deg_kD=\dim_k\mathrm{H}^0(D,\mathscr{O}_D)$.
    \begin{proof}
    	
    	因为$(D,\mathscr{O}_D)$是一维诺特空间的真闭子集,所以它是有限的离散空间.任取$x\in D$,有$\{x\}$是$D$的连通分支.记$A=\mathrm{H}^0(D,\mathscr{O}_D)$,那么有$A=\oplus_{x\in D}\mathscr{O}_{D,x}$.进而有$\dim_kA=\sum_{x\in D}\dim_k\mathscr{O}_{D,x}$.但是对$x\in D$有如下等式,并且在$x\not\in D$时$\mathrm{mult}_x(D)=0$,得证.
    	$$\mathrm{mult}_x(D)=l_{\mathscr{O}_{X,x}}(\mathscr{O}_{D,x})=[\kappa(x):k]^{-1}\dim_k\mathscr{O}_{D,x}$$
    \end{proof}
    \item 设$X$是一维诺特概形,设$D$是Cartier除子,那么存在两个非零有效Cartier除子$E,F$,满足$D=E-F$.
    \begin{proof}
    	
    	记$D=\{(U_i,f_i)\}$,其中$U_i=\mathrm{Spec}A_i$是仿射的,$f_i=a_i/b_i$是$A_i$上两个正则元的商.那么$V(b_i)$是$U_i$的真闭子集,于是它是有限集合.但是按照$X$是一维的,迫使$V(b_i)$是$X$的闭子集.记由$(X\backsim V(b_i),1),(U_i,b_i)$定义的Cartier除子为$D_i$,这是有效的,进而$F=\sum_iD_i$也是有效的.并且$E=D+F$也是有效的.如果$E,F$中有零除子,同时对$E,F$加上一个非零有效除子即可.
    \end{proof}
    \item 设$X/k$是有限型1维概形,设$K/k$是域扩张,设典范投影态射$p:X_K\to X$.
    \begin{enumerate}[(1)]
    	\item 对$X$上的除子$D$,记$D_K=p^*D$,那么$\deg_KD_K=\deg_kD$.
    	\item 设$k'$是$k$的子域,满足$[k:k']$有限.把$X$视为$k'$上的曲线时$D$就视为$X/k'$上的除子.此时有$\deg_{k'}D=[k:k']\deg_kD$.
    \end{enumerate}
    \begin{proof}
    	
    	(1):按照上一条,归结为设$D$是有效的.记$(D,\mathscr{O}_D)$是它对应的闭子概型.那么$(D_K,\mathscr{O}_{D_K})$是纤维积$D\times_kK$.按照$K$在$k$上平坦,我们有$\mathrm{H}^0(D_K,\mathscr{O}_{D_K})=\mathrm{H}^0(D,\mathscr{O}_D)\otimes_kK$.
    	
    	\qquad
    	
    	(2):对$x\in X$,$\mathrm{mult}_x(D)$不依赖于基域.所以这件事就来自于$[\kappa(x):k']=[\kappa(x):k][k:k']$.
    \end{proof}
    \item 设$\pi:Y\to X$是域$k$上有限型1维整概形之间的有限态射,那么对$Y$上任意除子$D$,有$\deg\pi^*D=[K(X):K(Y)]\deg D$.特别的,如果$\pi$是$Y$在$K(Y)$中的正规化态射,那么$\deg\pi^*D=\deg D$.
    \begin{proof}
    	
    	我们解释过在这个条件下$\mathscr{O}_Y\to f_*\mathscr{O}_X$可以延拓为$\mathscr{K}_Y\to f_*\mathscr{K}_X$.进而$\pi^*D$定义良性的.【Liuqing7.3.8】
    \end{proof}
\end{enumerate}
\newpage
\section{拟凝聚层}
\subsection{概形上的拟凝聚层}

设$(X,\mathscr{O}_X)$是概型.一个$\mathscr{O}_X$模层$\mathscr{F}$称为拟凝聚层,如果$X$存在仿射开覆盖$\{U_i=\mathrm{Spec}A_i\}$,使得每个$\mathscr{F}\mid U_i$都是$U_i$上的拟凝聚层.换句话讲存在$A_i$模$M_i$使得$\mathscr{F}\mid U_i=\widetilde{M_i}$.
\begin{enumerate}
	\item 引理.设$X=\mathrm{Spec}A$是仿射概型,设$D(f)$是一个主开集,设$\mathscr{F}$是$X$上的一个拟凝聚层.
	\begin{itemize}
		\item 如果$s$是$\mathscr{F}$的一个整体截面,使得限制在$D(f)$上为零,那么存在$n>0$使得$f^ns=0$.
		\item 给定$\mathscr{F}$在$D(f)$上的截面$t$,那么存在$n>0$使得$f^nt$可延拓为一个整体截面.
	\end{itemize}
    \begin{proof}
    	
    	按照定义存在仿射开覆盖使得$\mathscr{F}$限制在每个开集上是伴随模层.选取这样的仿射开子集$V=\mathrm{Spec}B$,选取它包含的主开集$D(g),g\in A$,那么包含映射$D(g)\subseteq V$对应于典范的环同态$B\to A_g$,于是有$\mathscr{F}\mid D(g)\cong\widetilde{M\otimes_BA_g}$.于是我们可以不妨设存在$X$的主开集覆盖,使得$\mathscr{F}$限制在每个开集上是伴随模层.另外仿射概型是拟紧的,于是这主开集覆盖可取有限的,记作$D(g_i),1\le i\le r$,记$\mathscr{F}$在$D(g_i)$上的限制为伴随模层$\widetilde{M_i}$.
    	
    	证明第一件事.设$s\in\Gamma(X,\mathscr{F})$,满足$s\mid D(f)=0$.记$s$在$D(g_i)$上的限制为$s_i\in M_i$.按照$D(f)\cap D(g_i)=D(fg_i)$,于是$\mathscr{F}\mid F(fg_i)=\widetilde{(M_i)_f}$.于是$s_i$在$(M_i)_f$中的像是零,于是存在$n_i$使得$f^{n_i}s_i=0$.取$n=\max_i\{n_i\}$,那么$f^ns_i=0$,于是$f^ns=0$.
    	
    	证明第二件事.设$t\in\Gamma(D(f),\mathscr{F})$,设它在$D(fg_i)$上的限制为$t_i'\in (M_i)_f$,那么存在正整数$n$使得存在$t_i\in M_i=\mathscr{F}(D(g_i))$满足$t_i\mid D(fg_i)=t_i'$.这里$n$本来是依赖于$i$的,但是$i$只有有限个,可选取其中最大的正整数.限制$t_i,t_j$限制在$D(fg_ig_j)\subseteq D(g_ig_j)$上是相同的,于是按照第一件事存在足够大的$m$使得$f^m(t_i-t_j)=0$在$D(g_ig_j)$上成立.最后这些$f^mt_i$粘合为一个整体截面$s$,满足它在$D(f)$上的限制是$f^{n+m}t$.
    \end{proof}
    \item 设$X$是概型,那么一个$\mathscr{O}_X$模层是拟凝聚层当且仅当对每个仿射开子集$U=\mathrm{Spec}A$,存在$A$模$M$使得$\mathscr{F}\mid U=\widetilde{M}$.换句话讲,拟凝聚层定义中的"存在一个仿射开覆盖"可改为"任意仿射开覆盖".特别的,这件事说明如果一个模层存在开覆盖,使得模层在每个开子集上的限制是拟凝聚层,那么这个模层是拟凝聚层.
    \begin{proof}
    	
    	引理的证明中我们解释了存在$X$的一组仿射基使得$\mathscr{F}$限制在每个仿射开子集上都是伴随模层.于是问题归结为不妨设$X$本身是仿射的,记作$\mathrm{Spec}A$.记$M=\Gamma(X,\mathscr{F})$,$M$上的恒等模同态诱导了模层态射$\alpha:\widetilde{M}\to\mathscr{F}$.设$X$被主开集$\{D(g_i)\}$覆盖,并且每个$\mathscr{F}\mid D(g_i)$是伴随模层$\widetilde{M_i}$.引理说明诱导的$\mathscr{F}(D(g_i))\to M_{g_i}$恰好是一个模同构,于是$\alpha$限制在$D(g_i)$上是同构,于是$\alpha$是同构.
    \end{proof}
    \item 按照拟凝聚层是一个局部性质,概型$X$上拟凝聚层之间态射的核,余核,像都是拟凝聚层.拟凝聚层的直和也是拟凝聚层.这说明拟凝聚层构成一个余完备阿贝尔范畴.但是一般概形上无限个拟凝聚层的直积未必还是拟凝聚层.不过概形上的拟凝聚层范畴自身上总有直积,只不过和$\mathscr{O}_X$模层范畴上的直积不一致.另外一般环空间上的拟凝聚层的无限直和未必是拟凝聚层.
    \item 设$X$是概型,给定$\mathscr{O}_X$模层的短正合列$0\to\mathscr{F}_1\to\mathscr{F}_2\to\mathscr{F}_3\to0$,那么$\mathscr{F}_i$中任意两个是拟凝聚层都得到第三个是拟凝聚层.
    \begin{proof}
    	
    	如果$\mathscr{F}_i$的前两个或者后两个是拟凝聚层,第三个模层可视为拟凝聚层之间态射的核或者余核,于是也是拟凝聚层.
    	
    	现在假设$\mathscr{F}_1,\mathscr{F}_3$是拟凝聚层.我们解释了拟凝聚层是局部性质,于是不妨设$X$是仿射的.记$\mathscr{F}_1=\widetilde{M_1}$和$\mathscr{F}_3=\widetilde{M_3}$,记$\Gamma(X,\mathscr{F}_2)=M_2$,我们解释了有短正合列$0\to M_1\to M_2\to M_3\to0$,于是作用函子$M\to\widetilde{M}$得到如下交换图表,按照短五引理得到$\widetilde{M_2}\to\mathscr{F}_2$是同构,于是$\mathscr{F}_2$是拟凝聚层.
    	$$\xymatrix{0\ar[r]&\widetilde{M_1}\ar@{=}[d]\ar[r]&\widetilde{M_2}\ar[r]\ar[d]&\widetilde{M_3}\ar[r]\ar@{=}[d]&0\\0\ar[r]&\mathscr{F}_1\ar[r]&\mathscr{F}_2\ar[r]&\mathscr{F}_3\ar[r]&0}$$
    \end{proof}
    \item 一般环空间$(X,\mathscr{O}_X)$上的拟凝聚层定义为一个模层$\mathscr{F}$,使得存在开覆盖$\{U_i\}$,满足对其中的每个开集$U_i$,都存在模层的正合列$\mathscr{O}_X^{(I)}\mid U_i\to\mathscr{O}_X^{(J)}\mid U_i\to\mathscr{F}\mid U_i\to0$.这里我们证明这个定义和概型上的定义一致.类似的,对局部诺特概型$X$,它上面的模层是凝聚层当且仅当存在开覆盖$\{U_i\}$使得在每个开集$U_i$上都有正合列$\mathscr{O}_X^{m}\mid U_i\to\mathscr{O}_X^{n}\mid U_i\to\mathscr{F}\mid U_i\to0$.
    \begin{proof}
    	
    	假设$\mathscr{F}$是概型意义下的拟凝聚层,那么存在$X$的开覆盖$\{U_i=\mathrm{Spec}A_i\}$,使得每个$\mathscr{F}\mid U_i=\widetilde{M_i}$,其中$M_i$是一个$A_i$模.我们知道模范畴上总有自由预解,存在正合列$A_i^{(I)}\to A_i^{(J)}\to M_i\to0$.按照$M\mapsto\widetilde{M}$是正合函子,得到$\widetilde{A_i}^{(I)}\to\widetilde{A_i}^{(J)}\to\mathscr{F}\mid U_i\to0$是正合列,于是这是环空间意义下的拟凝聚层.
    	
    	反过来如果存在开集$U$使得有模层的正合列$\mathscr{F}'\to\mathscr{F}''\to\mathscr{F}\mid U\to0$,这里$\mathscr{F}''$是$U$上的自由模层,于是它是概型意义下的拟凝聚层,而拟凝聚层的满射像是拟凝聚层,于是每个$\mathscr{F}\mid U$是拟凝聚层,按照这里$U$覆盖了整个$X$,得到$\mathscr{F}$是拟凝聚层.
    \end{proof}
    \item 设$\mathscr{F}$和$\mathscr{G}$是$\mathscr{O}_X$拟凝聚层.按照定义有商层$\mathscr{F}/\mathscr{G}$和张量积$\mathscr{F}\otimes_{\mathscr{O}_X}\mathscr{G}$都是相应预层的层化.我们断言如果$U$是仿射开子集,那么截面环是和预层一致的,即$(\mathscr{F}/\mathscr{G})(U)=\mathscr{F}(U)/\mathscr{G}(U)$和$(\mathscr{F}\otimes_{\mathscr{O}_X}\mathscr{G})(U)=\mathscr{F}(U)\otimes_{\mathscr{O}_X(U)}\mathscr{G}(U)$.
    \begin{proof}
    	
    	首先如果$X$是仿射概形$\mathrm{Spec}A$,设$\mathscr{F}=\widetilde{M}$和$\mathscr{G}=\widetilde{N}$,那么在主开集$D(f)$上这两个预层的截面环为$\mathscr{F}(D(f))/\mathscr{G}(D(f))=M_f/N_f=(M/N)_f$和$M_f\otimes_{A_f}N_f=(M\otimes_AN)_f$.它们都和局部化可交换,于是按照$M/N$和$M\otimes_AN$的伴随模层在主开集上构成层,就说明这两个预层在主开集上已经构成层.再设$X$是一般的概形,我们断言这两个预层已经在全体仿射开子集上构成层.为此,假设有仿射开子集$U$可以表示为仿射开子集的并$\cup_iU_i$,把每个$U_i$分解为$U$的主开集的并,那么仿射的情况就告诉我们满足粘合的存在性和唯一性.
    \end{proof}
    \item 回顾一下模层的一些性质.
    \begin{itemize}
    	\item 称$\mathscr{F}$被一族整体截面$\{s_i\mid i\in I\}\subseteq\Gamma(X,\mathscr{F})$生成,如果对每个$x\in X$有$\{s_{i,x}\mid i\in I\}$在$\mathscr{O}_{X,x}$上生成了整个$\mathscr{F}_x$.这等价于讲存在满态射$\mathscr{O}_X^{(I)}\to F$.
    	\item 称$\mathscr{F}$是有限生成模层,如果对每个点$x\in X$都存在开邻域$U$,和一个满态射:
    	$$\xymatrix{\mathscr{O}_X^n\mid_U\ar[r]&\mathscr{F}\mid_U\ar[r]&0}$$
    	\item 称$\mathscr{F}$是有限表示模层,如果对每个点$x\in X$都存在开邻域$U$,和一个正合列:
    	$$\xymatrix{\mathscr{O}_X^m\mid_U\ar[r]&\mathscr{O}_X^n\mid_U\ar[r]&\mathscr{F}\mid_U\ar[r]&0}$$
    \end{itemize}
    \begin{enumerate}[(1)]
    	\item 例如仿射概形上的伴随模层总能被整体截面生成,因为环上的模总存在生成元集.再例如取$X=\mathrm{Proj}S$,其中$S$是分次环,使得它被$S_1$在$S_0$上代数生成,那么$S_1$作为$\mathscr{O}_X(1)$的整体截面环生成了$\mathscr{O}_X(1)$.
    	\item 局部诺特概形上有限生成模层和有限表示模层一致.
    	\item 设$X=\mathrm{Spec}A$是仿射的,那么$X$上的拟凝聚层$\widetilde{M}$是有限生成模层或者有限表示模层当且仅当$M$是有限生成$A$模或者有限表示$A$模.
    	\item 设$X=\mathrm{Spec}A$是仿射概形,设$M$是$A$模,那么如下命题互相等价:
    	\begin{enumerate}
    		\item $\widetilde{M}$是局部自由的有限生成$\mathscr{O}_X$模层.
    		\item $M$是有限生成投射$A$模.(这里有限生成条件是必须的,存在这样的非有限的$A$模$M$,满足它是投射模,但是$\widetilde{M}$不是局部自由模层).
    		\item $M$是有限表示平坦$A$模.
    	\end{enumerate}
    \end{enumerate}
    \item 拟凝聚层的逆像.设$f:X\to Y$是概型的态射,如果$\mathscr{G}$是$Y$上的拟凝聚层,那么它的逆像$f^*\mathscr{G}$是$X$上的拟凝聚层.
    \begin{proof}
    	
    	逆像函子$f^*$满足$f^*(\mathscr{O}_Y)=\mathscr{O}_X$,另外逆像函子是左伴随函子,它是右正合的,于是如果$\mathscr{G}$是$Y$上的拟凝聚层,也即局部上它是自由模层之间态射的余核,于是作用逆像函子仍然得到一个余核,这说明$f^*\mathscr{G}$是$X$上的拟凝聚层.
    \end{proof}
    \item 拟凝聚层的正像.设$f:X\to Y$是概型的态射,设$f$是qcqs态射(拟紧拟分离,例如$X$诺特可以保证这个条件),那么如果$F$是$X$上的拟凝聚层,那么$f_*F$是$Y$上的拟凝聚层.一般情况即便是域上的有限型概形之间的态射,也不能保证拟凝聚层的正像还是拟凝聚层.
    \begin{proof}
    	
    	我们要证明$f_*F$是$\mathscr{O}_Y$拟凝聚层.问题是局部的,所以不妨设$Y$本身是仿射的.按照$f$是拟紧的得到$X$是拟紧的.所以$X$可以被有限个仿射开子集$\{U_i\}$覆盖.按照拟分离条件有$U_i\cap U_j$可被有限个仿射开子集覆盖,记作$\{U_{ijk}\}$.
    	
    	\qquad
    	
    	现在任取$F(f^{-1}(V))$中的元$s$,它等价于给定一组元$s_i\in F(f^{-1}(V)\cap U_i)$,使得它们在$f^{-1}(V)\cap U_{ijk}$上的限制都是相同的.这等价于层公理,所以有如下正合列:
    	$$\xymatrix{0\ar[r]&f_*F\ar[r]&\oplus_if_*(F\mid U_i)\ar[r]&\oplus_{i,j,k}f_*(F\mid U_{ijk})}$$
    	
    	现在每个$F\mid U_i$和$F\mid U_{ijk}$都是仿射概形上的拟凝聚的,于是$f_*(F\mid U_i)$和$f_*(F\mid U_{ijk})$都是仿射概形上的拟凝聚层.于是直和$\oplus_if_*(F\mid U_i)$和$\oplus_{i,j,k}f_*(F\mid U_{ijk})$都是拟凝聚层(这里用到了qcqs条件,如果$U_i$和$U_{ijk}$都是无限的,那么这个正合列中的两个直和要改为无限直积,但是拟凝聚层的无限直积未必是拟凝聚层).而拟凝聚层之间态射的核是拟凝聚层,这就说明$f_*F$是拟凝聚层.
    \end{proof}
    \item 考虑如下概形的交换图表:
    $$\xymatrix{X'\ar[rr]^{g'}\ar[d]_{f'}&&X\ar[d]^f\\Y'\ar[rr]_g&&Y}$$
    \begin{enumerate}
    	\item 设$\mathscr{F}$是$\mathscr{O}_X$模层,我们来构造一个自然的$\mathscr{O}_{Y'}$模层态射:
    	$$g^*(f_*\mathscr{F})\to f'_*(g'^*\mathscr{F})$$
    	
    	因为$g^*$左伴随于$g_*$,于是归结为构造模层态射$f_*\mathscr{F}\to g_*f'_*(g'^*\mathscr{F})=f_*g'_*g'^*\mathscr{F}$.按照伴随性,$g'^*\mathscr{F}$上的恒等态射对应了模层态射$\mathscr{F}\to g_*'g'^*\mathscr{F}$,进而作用函子$f_*$得到一个自然的模层态射.
    	\item 如果还有$\mathscr{G}'$是$\mathscr{O}_{Y'}$模层,那么有如下自然的模层态射,特别的如果取$\mathscr{G}'=\mathscr{O}_{Y'}$,这个模层态射就是(a).
    	$$\mathscr{G}'\otimes_{\mathscr{O}_{Y'}}g^*f_*\mathscr{F}\to f_*'(f'^*)\otimes_{\mathscr{O}_{Y'}}f_*'(g'^*\mathscr{F})\to f_*'(f'^*\mathscr{G}'\otimes_{\mathscr{O}_{X'}}g'^*\mathscr{F})$$
    \end{enumerate}

    我们断言如果$\mathscr{F}$和$\mathscr{G}'$都是拟凝聚层,如果上述交换图表是纤维积图表,如果如下两个条件有一个成立,那么(a)和(b)中的模层态射都是同构.
    \begin{enumerate}[(1)]
    	\item $f$是仿射态射.
    	\item $f$是qcqs态射,$g$是平坦的,$\mathscr{G}'$在$Y$上平坦(在(a)的情况下,也即$\mathscr{G}'=\mathscr{O}_{Y'}$的情况下,此即$g$是平坦态射).
    \end{enumerate}
    \begin{proof}
    	
    	先设$f$是仿射态射,问题是局部的,不妨设$Y=\mathrm{Spec}A$,$X=\mathrm{Spec}B$,还可以设$Y'=\mathrm{Spec}A'$也是仿射的.于是$X'=\mathrm{Spec}B\otimes_AA'$.进而可设$\mathscr{F}=\widetilde{N}$,其中$N$是$B$模,可设$\mathscr{G}'=\widetilde{M'}$,其中$M'$是$A'$模.那么(b)中的模层态射可以转化为模同态,它是同构因为:
    	\begin{align*}
    		M\otimes_{A'}(A'\otimes_AN)&\cong M'\otimes_AN\\&\cong(M'\otimes_{A'}B')\otimes_{B'}(N\otimes_BB')
    	\end{align*}
    
        下面设$f$是qcqs态射,并且$\mathscr{G}'$是$Y$上平坦拟凝聚层,我们要证明这个$\mathscr{O}_{Y'}$模层态射是同构,这是局部的问题,所以可设$Y=\mathrm{Spec}A$和$Y'=\mathrm{Spec}A'$.于是从$f$是qcqs态射得到$X$是qcqs概形,那么存在$X$的有限仿射开覆盖$\{U_i\mid i\in I\}$,并且每个$U_i\cap U_j$也存在有限仿射开覆盖$\{U_{ijk}\mid k\in K_{ij}\}$.设$\mathscr{F}_i=\mathscr{F}\mid_{U_i}$和$f_i=f\mid_{U_i}$.再记$U_{ij}=U_i\cap U_j$,其中$i,j\in I$.设$\mathscr{F}_{ijk}=\mathscr{F}\mid_{U_{ijk}}$和$f_{ijk}=f\mid_{U_{ijk}}$.那么我们有如下正合列:
        $$\xymatrix{0\ar[r]&f_*\mathscr{F}\ar[r]&\oplus_if_{i,*}\mathscr{F}_i\ar[r]&\oplus_{i,j,k}f_{ijk,*}\mathscr{F}_{ijk}}$$
        
        按照$g$是平坦的以及$\mathscr{G}'$是平坦的,我们得到如下正合列:
        $$\xymatrix{0\ar[r]&\mathscr{G}'\otimes_{\mathscr{O}_{Y'}}g^*f_*\mathscr{F}\ar[r]&\mathscr{G}'\otimes_{\mathscr{O}_{Y'}}\oplus_ig^*f_{i,*}\mathscr{F}_i\ar[r]&\oplus_{i,j,k}g^*f_{ijk,*}\mathscr{F}_{ijk}}$$
        
        我们再记$U_i'=g'^{-1}(U_i)$和$U_{ijk}'=g'^{-1}(U_{ijk})$.记$\mathscr{H}'=f'^*\mathscr{G}\otimes_{\mathscr{O}_{X'}}g'^*\mathscr{F}$以及$\mathscr{H}_i'=\mathscr{H}'\mid_{U_i'}$和$\mathscr{H}'_{ijk}=\mathscr{H}'\mid_{U_{ijk}}$.再记$f_i'$和$f_{ijk}'$分别是$f'$在$U_i'$和$U_{ijk}'$上的限制.于是我们得到如下交换图表,其中第一行是正合列我们解释过了,第二行是正合列还是因为$f$是qcqs态射.按照仿射情况我们的(b)是同构,就说明这里$u'$和$u''$都是同构,于是五引理告诉我们$u$是同构,这就得证.
        $$\xymatrix{0\ar[r]&\mathscr{G}'\otimes_{\mathscr{O}_{Y'}}g^*f_*\mathscr{F}\ar[d]_u\ar[r]&\mathscr{G}'\otimes_{\mathscr{O}_{Y'}}\oplus_ig^*f_{i,*}\mathscr{F}_i\ar[d]_{u'}\ar[r]&\oplus_{i,j,k}g^*f_{ijk,*}\mathscr{F}_{ijk}\ar[d]_{u''}\\0\ar[r]&f_*'\mathscr{H}'\ar[r]&\oplus_if_{i,*}'\mathscr{H}_i'\ar[r]&\oplus_{i,j,k}f_{ijk,*}\mathscr{H}_{ijk}'}$$
    \end{proof}
    \item 推论.设$X$是qcqs概形,设$Y=\mathrm{Spec}A$,设$f:X\to Y$是态射,设$A'$是平坦$A$代数,设$X'=X\times_AA'$和$p:X'\to X$是投影态射.那么对任意$\mathscr{O}_X$拟凝聚层$\mathscr{F}$,都有自然同构:
    $$\Gamma(X,\mathscr{F})\otimes_AA'\cong\Gamma(X',p^*\mathscr{F})$$
    \begin{proof}
    	
    	考虑如下纤维积图表,按照$Y$是仿射的,$f$是qcqs态射,我们有$f_*\mathscr{F}=\widetilde{\Gamma(X,\mathscr{F})}$和$f'_*(g'^*\mathscr{F})=\widetilde{\Gamma(X',g'^*\mathscr{F})}$.
    	$$\xymatrix{X'\ar[rr]^{g'}\ar[d]_{f'}&&X\ar[d]^f\\Y'=\mathrm{Spec}A'\ar[rr]^g&&Y=\mathrm{Spec}A}$$
    	
    	按照这里$f$是qcqs态射,$g$是平坦的,于是上一条说明有如下典范同构:
    	$$f_*\mathscr{F}\times_YY'\cong f'_*g'^*\mathscr{F}$$
    	
    	此即:
    	$$\Gamma(X,\mathscr{F})\otimes_AA'\cong\Gamma(X',p^*\mathscr{F})$$
    \end{proof}
    \item 关于张量积的补充.设$X,Y$是两个$S$概形,设$(Z=X\times_SY,p:Z\to X,q:Z\to Y)$是纤维积,设$\mathscr{F}$和$\mathscr{G}$分别是$X$和$Y$上的拟凝聚层.这两个拟凝聚层在$Z$上的张量积指的是$\mathscr{O}_Z$拟凝聚层$(p^*\mathscr{F})\otimes_{\mathscr{O}_Z}(q^*\mathscr{G})$,简记作$\mathscr{F}\otimes_S\mathscr{G}$.
    \begin{enumerate}[(1)]
    	\item 仿射情况.设$S=\mathrm{Spec}A$,$X=\mathrm{Spec}B$,$Y=\mathrm{Spec}C$都是仿射的,设拟凝聚层$\mathscr{F}=\widetilde{M}$和$\mathscr{G}=\widetilde{N}$,其中$M$和$N$分别是$B$和$C$模.那么$\mathscr{O}_Z$拟凝聚层$\mathscr{F}\otimes_S\mathscr{G}$就是$\widetilde{M\otimes_AN}$,其中$M,N$视为$A$模,$M\otimes_AN$视为$B\otimes_AC$模.
    	\item 设$f:T\to X$和$g:T\to Y$是两个$S$态射,设$\mathscr{F}$和$\mathscr{G}$分别是$\mathscr{O}_X$和$\mathscr{O}_Y$拟凝聚层.设$\{f,g\}$诱导的纤维积态射$h:T\to Z$,那么有$h^*(\mathscr{F}\otimes_S\mathscr{G})=(f^*\mathscr{F})\otimes_{\mathscr{O}_T}(g^*\mathscr{G})$.这件事是因为逆像和张量积可交换.
    	\item 推论.设$f:X\to X'$和$g:Y\to Y'$是$S$态射,设$\mathscr{F}'$和$\mathscr{G}'$分别是$\mathscr{O}_{X'}$和$\mathscr{O}_{Y'}$拟凝聚层,那么有:
    	$$(f\times_Sg)^*(\mathscr{F}'\otimes_S\mathscr{G}')=f^*\mathscr{F}'\otimes_Sg^*\mathscr{G}'$$
    	\item 推论.设$X$是$S$概形,那么任意拟凝聚$\mathscr{O}_X$模层$\mathscr{F}$都是$\mathscr{F}\otimes_S\mathscr{O}_S$在典范同构$X\cong X\times_SS$下的逆像.
    	\item 设$X$是$S$概形,设$\mathscr{F}$是拟凝聚$\mathscr{O}_X$模层,设$\varphi:S'\to S$是态射,记$X_{(S')}=X\times_SS'$和$\mathscr{F}_{(S')}=\mathscr{F}\otimes_S\mathscr{O}_{S'}$.那么有$\mathscr{F}_{(S')}=p^*\mathscr{F}$.并且如果$\varphi'':S''\to S'$是另一个态射,那么$(\mathscr{F}_{(S')})_{(S'')}$是$\mathscr{F}_{(S'')}$在典范同构$(X_{(S')})_{(S'')}\cong X_{(S'')}$下的逆像.
    	\item 设$f:X\to Y$是$S$态射,设$\varphi:S'\to S$是态射,对任意拟凝聚$\mathscr{O}_Y$模层$\mathscr{G}$,我们有:
    	$$(f_{(S')})^*(\mathscr{G}_{(S')})=(f^*\mathscr{G})_{(S')}$$
    	\item 设$X,Y$是$S$概形,设$\mathscr{F}$和$\mathscr{G}$分别是$\mathscr{O}_X$和$\mathscr{O}_Y$拟凝聚层.那么$(\mathscr{F}\otimes_S\mathscr{G})_{(S')}$是$(\mathscr{F}_{(S')})\otimes_{S'}(\mathscr{G}_{(S')})$在典范同构$(X\times_SY)_{(S')}\cong(X_{(S')})\times_{S'}(Y_{(S')})$下的逆像.
    	\item 茎.设$z\in Z=X\times_SY$,设$x=p(z)$和$y=q(z)$,那么$(\mathscr{F}\otimes_S\mathscr{G})_z$典范同构于$\left(\mathscr{F}_x\otimes_{\mathscr{O}_{X,x}}\mathscr{O}_{Z,z}\right)\otimes_{\mathscr{O}_{Z,z}}\left(\mathscr{G}_y\otimes_{\mathscr{O}_{Y,y}}\mathscr{O}_{Z,z}\right)$,也即$\mathscr{F}_x\otimes_{\mathscr{O}_{X,x}}\mathscr{O}_{Z,z}\otimes_{\mathscr{O}_{Y,y}}\mathscr{G}_y$.
    	\item 支集.设$\mathscr{F}$和$\mathscr{G}$都是有限型模层,那么有:
    	$$\mathrm{Supp}(\mathscr{F}\otimes_S\mathscr{G})=p^{-1}(\mathrm{Supp}\mathscr{F})\cap q^{-1}(\mathrm{Supp}\mathscr{G})$$
    	\begin{proof}
    		
    		因为逆像函子是右正合的,所以它把有限型模层映为有限型模层.于是按照张量积支集是分量支集的交,问题归结为证明$\mathrm{Supp}(p^*\mathscr{F})=p^{-1}(\mathrm{Supp}\mathscr{F})$.【】
    	\end{proof}
    \end{enumerate}
    \item 零化子.设$X$是概形,设$\mathscr{F}$是一个有限生成的拟凝聚层,我们解释过这个条件下支集$\mathrm{Supp}\mathscr{F}$总是闭集.我们要给这个闭集上赋予一个典范的闭子概型结构.
    \begin{enumerate}[(1)]
    	\item 设$X$是概形,设$\mathscr{F}$是一个有限生成的拟凝聚层.考虑模层态射$\mathscr{O}_X\to\mathrm{HOM}_{\mathscr{O}_X}(\mathscr{F},\mathscr{F})$为把$s\in\mathscr{O}_X(U)$映射为数乘$s$诱导的$\mathscr{F}\mid_U\to\mathscr{F}\mid_U$的模层态射.这个模层态射的核称为$\mathscr{F}$的零化子,记作$\mathrm{Ann}(\mathscr{F})$.它是拟凝聚理想层因为拟凝聚层之间模层态射的核一定是拟凝聚的.
    	\item 我们断言对仿射开子集$U$总有$\Gamma(U,\mathrm{Ann}(\mathscr{F}))=\mathrm{Ann}_{\mathscr{O}_X(U)}\Gamma(U,\mathscr{F})$(这里对$A$模$M$,有$\mathrm{Ann}_A(M)$表示$\{a\in A\mid aM=0\}$).并且这个拟凝聚理想层对应的闭子概型的底空间就是$\mathrm{Supp}\mathscr{F}$.
    	\begin{proof}
    		
    		所有问题都是局部的,不妨设$X=\mathrm{Spec}A$是仿射的,设$\mathscr{F}=\widetilde{M}$,其中$M$是有限$A$模.那么$A\to\mathrm{Hom}_A(M,M)$的核就是$\mathrm{Ann}(M)$.最后我们知道有限$A$模$M$的支集就是$V(\mathrm{Ann}_A(M))$,于是$\mathrm{Ann}_A(M)$对应的闭子概型$A/\mathrm{Ann}_A(M)$的底空间自然就是$V(\mathrm{Ann}_A(M))$.
    	\end{proof}
    	\item 依旧设$X$是概形,$\mathscr{F}$是一个有限生成的拟凝聚层,如果有拟凝聚理想层$\mathscr{I}\subseteq\mathrm{Ann}(\mathscr{F})$,它对应了闭嵌入$i:V(\mathscr{I})\to X$(这里$V(\mathscr{I})$表示$\mathscr{I}$对应的闭子概型),按照$\mathscr{I}\mathscr{F}=0$说明$\mathscr{F}$是$\mathscr{O}_X/\mathscr{I}$模层.于是我们有典范映射$\mathscr{F}\to i_*i^*\mathscr{F}$是同构.
    	\item 设$X$是概形,设$\mathscr{F}$是拟凝聚模层,设$\mathscr{E},\mathscr{E}'\subseteq\mathscr{F}$是两个拟凝聚$\mathscr{O}_X$子模层.我们定义$(\mathscr{E},\mathscr{E}')$是$\mathscr{O}_X$理想层,满足:
    	$$\Gamma(U,(\mathscr{E},\mathscr{E}'))=\{a\in\Gamma(U,\mathscr{O}_X)\mid\forall\text{开集}V\subseteq U,\forall m'\in\Gamma(V,\mathscr{E}'),\text{我们有}(a\mid_V)m'\in\Gamma(V,\mathscr{E})\}$$
    	
    	于是$(0:\mathscr{F})=\mathrm{Ann}(\mathscr{F})$.那么类似于零化子的情况,我们可以证明如果$\mathscr{E}'$是有限生成的拟凝聚层,那么$(\mathscr{E},\mathscr{E}')$是拟凝聚理想层.
    \end{enumerate}
\end{enumerate}
\subsection{概形上的凝聚层}

设$(X,\mathscr{O}_X)$是概型.一个$\mathscr{O}_X$模层$\mathscr{F}$称为凝聚层,如果它是有限生成模层(这是指对任意$x\in X$,存在开邻域$U$,存在正整数$n$,存在满同态$\mathscr{O}_X^n\mid_U\to\mathscr{F}\mid_U$),并且对任意同态$\mathscr{O}_X^n\mid_U\to\mathscr{F}\mid_U$,有它的核总是有限生成的.
\begin{enumerate}
	\item 按照定义,凝聚层总是拟凝聚层.
	\item 设$X$是概形,设$\mathscr{F}$是拟凝聚层$\mathscr{O}_X$模层.考虑如下三个条件,一般的我们有(a)$\Rightarrow$(b)$\Leftrightarrow$(c).如果$X$是局部诺特概形,那么这三个命题是等价的.
	\begin{enumerate}
		\item $\mathscr{F}$是凝聚层.
		\item $\mathscr{F}$是有限生成模层.
		\item 对任意仿射开子集$U$,我们有$\mathscr{F}(U)$是有限$\mathscr{O}_X(U)$模层.
	\end{enumerate}
    \begin{proof}
    	
    	(a)推(b)就是定义.(b)推(c):设$\mathscr{F}$是有限生成模层,任取仿射开子集$U\subseteq X$,那么$U$可以被有限个主开集$U_i$覆盖,使得存在正整数$n$以及一个满同态$\mathscr{O}_X^{n_i}\mid_{U_i}\to\mathscr{F}\mid_{U_i}$.因为有$\mathscr{F}(U_i)=\mathscr{F}(U)\otimes_{\mathscr{O}_X(U)}\mathscr{O}_X(U_i)$(比方说,验证典范同态的stalk总是同构),于是存在$\mathscr{O}_X(U)$模$\mathscr{F}(U)$的有限生成子模$M$使得$M\otimes_{\mathscr{O}_X(U)}\mathscr{O}_X(U_i)=\mathscr{F}(U_i)$对任意$i$成立【】.那么这诱导的$\widetilde{M}\to\mathscr{F}\mid_U$就是一个满同态,因为它在每个$U_i$上都是满同态.进而有$M\to\mathscr{F}(U)$是满射(仿射概形上一个模层的短正合列如果核是拟凝聚的,那么截面函子作用上仍然是短正合列),于是$M=\mathscr{F}(U)$,这得到(c).
    	
    	\qquad
    	
    	反过来(c)推(b)是平凡的.下面设$X$是局部诺特概形来证明(c)推(a):设$V\subseteq X$是非空开子集,设$\alpha:\mathscr{O}_X^n\mid_V\to\mathscr{F}\mid_V$是任意同态,我们要证明$\ker\alpha$是有限生成的.因为有限生成是一个局部性质,归结为设$V$是仿射的情况.但是此时有$\mathscr{O}_X(V)$是诺特环,所以$\mathscr{O}_X(U)^n$是诺特模,它的子模总是有限生成的.
    \end{proof}
    \item 如果$X$是局部诺特概形,那么一个$\mathscr{O}_X$模层是凝聚层当且仅当它在每个仿射开子集$\mathrm{Spec}A$上的限制是一个有限$A$模的伴随模层.
    \item 局部诺特概形上凝聚层之间态射的核,余核,像都是凝聚层.凝聚层的有限直和也是凝聚层.于是凝聚层构成一个阿贝尔范畴.
    \item 设$X$是局部诺特概形,给定$\mathscr{O}_X$模层的短正合列$0\to\mathscr{F}_1\to\mathscr{F}_2\to\mathscr{F}_3\to0$,那么$\mathscr{F}_i$中任意两个是凝聚层都得到第三个是凝聚层.
    \item 凝聚层的逆像.设$f:X\to Y$是概形之间的态射,设$\mathscr{G}$是拟凝聚层,设$V$是$Y$的仿射开子集,设$U$是$X$的仿射开子集使得$f(U)\subseteq V$,那么$f^*\mathscr{G}\cong\widetilde{g(V)\otimes_{\mathscr{O}_Y(V)}\mathscr{O}_X(U)}$.这说明如果$\mathscr{G}$是有限生成模层,那么$f^*\mathscr{G}$也是有限生成模层.特别的,如果$X,Y$都是局部诺特概形,那么$\mathscr{G}$是凝聚层可推出$f^*\mathscr{G}$是凝聚层.
    \item 凝聚层的正像.即便$X,Y$都是诺特概形,凝聚层的正像也未必是凝聚层.需要添加一些条件,例如$f$是有限态射或者紧合态射:设$f:X\to Y$是概形之间的有限态射,设$\mathscr{F}$是$X$上的有限生成拟凝聚层,那么$f_*\mathscr{F}$是$Y$上的有限生成拟凝聚层.
    \begin{proof}
    	
    	因为拟凝聚和有限生成都是局部性质,不妨设$Y=\mathrm{Spec}A$是仿射的,因为有限态射都是仿射态射,可设$X=\mathrm{Spec}B$是仿射的.那么我们解释过有$f_*\mathscr{F}=\widetilde{\mathscr{F}(X)}$,这里把$\mathscr{F}(X)$视为$A$模,这就说明$f_*\mathscr{F}$是拟凝聚层的.但是由于$B$模$\mathscr{F}(X)$是有限生成的,而$B$作为$A$模也是有限生成的,于是$\mathscr{F}(X)$作为$A$模也是有限生成的,于是$f_*\mathscr{F}$是有限生成的.
    \end{proof}
\end{enumerate}

\subsection{拟凝聚层上截面的延拓}

\begin{enumerate}
	\item (拟凝聚层上截面的延拓定理).设$L$是概形$X$上的可逆层(局部秩1自由模层),任取$f\in\Gamma(X,L)$,那么$X_f=\{x\in X\mid f_x\not\in m_xL_x\}$是$X$的开子集.设$F$是$X$上的拟凝聚层.
	\begin{itemize}
		\item 设$X$是拟紧的,设$s\in\Gamma(X,F)$是整体截面,并且在$X_f$上的限制为零.那么存在$n>0$使得$f^ns=0$,其中$f^ns$理解为$F\otimes L^{\otimes n}$上的整体截面.
		\item 进一步假设$X$存在有限仿射开覆盖$\{U_i\}$,使得$F\mid U_i$都是自由模层,并且每个$U_i\cap U_j$是拟紧的(例如拟紧拟分离满足这些条件).任取$t\in\Gamma(X_f,F)$,那么存在$n>0$使得$f^nt\in\Gamma(X_f,F\otimes L^{\otimes n})$可延拓为$F\otimes L^{\otimes n}$的整体截面.
		\item 如果$X$是qcqs概形,那么上两条的假设都成立.
	\end{itemize}
	\begin{proof}
		
		设$X$被有限个$U=\mathrm{Spec}A$覆盖,使得$L\mid U$是秩1自由的,于是可取$\psi:L\mid U\cong\mathscr{O}_U$.可设$F\mid U=\widetilde{M}$,其中$M$是$A$模.于是$s\in\Gamma(X,F)$限制在$U$上为一个元$s\in M$;$f\in\Gamma(X,L)$限制在$U$上为$g=\psi(f)\in A$.按照$X_f\cap U=D(g)$,从$s\mid X_f=0$得到$s$在$M_g$中为零,也即存在$n$使得在$M$中有$g^ns=0$.考虑同构$M\otimes_AA^{\otimes n}\cong M$,诱导了同构$F\otimes L^{\otimes n}\mid U\cong F\mid U$.在这个同构下就有$f^ns=0\in\Gamma(U,F\otimes L^{\otimes n})$.并且这件事不依赖于$\psi$的选取.对我们开头的有限个仿射开子集都选取合适的$n$,那么当$n$足够大的时候就有$f^ns=0$在$X$上成立.
	\end{proof}
    \item 推论.设$X$是qcqs概形,设$\mathscr{L}$是$X$上可逆层.任取$f\in\Gamma(X,\mathscr{L})$.设$\mathscr{F}$是$X$上拟凝聚层.记分次环$A_*=\Gamma_*(\mathscr{L})$和分次$A_*$模$M_*=\Gamma_*(\mathscr{L},\mathscr{F})$,那么有典范同构$\Gamma(X_f,\mathscr{F})\cong((M_*)_f)_0$.特别的,取$\mathscr{L}=\mathscr{O}_X$得到:设$X$是qcqs概形,记$A=\Gamma(X,\mathscr{O}_X)$,任取$f\in\Gamma(X,\mathscr{O}_X)$,记$M=\Gamma(X,\mathscr{F})$,那么$A_f$模$\Gamma(X_f,\mathscr{F})$典范同构于$M_f$.
    \item 设$X$是拟紧概形,设$\mathscr{F}$是有限生成拟凝聚$\mathscr{O}_X$模层,设$\mathscr{I}$是有限生成拟凝聚理想层,并且$\mathscr{F}$的支集包含在$\mathscr{O}_X/\mathscr{I}$的支集中,那么可以找到正整数$n$使得$\mathscr{I}^n\mathscr{F}=0$.
    \begin{proof}
    	
    	按照$X$可以被有限个仿射开子集覆盖,问题可以归结为设$X=\mathrm{Spec}A$和$\mathscr{F}=\widetilde{M}$,$\mathscr{I}=\widetilde{I}$,其中$M$是有限$A$模,$I$是有限生成$A$理想.记$I=\langle f_1,\cdots,f_r\rangle$.记$M$被$\{s_1,\cdots,s_m\}$生成.按照支集包含关系,就有整体截面$s_i$限制在每个$D(f_j)$上都是零,于是可以找到统一的正整数$h$使得对任意$i,j$都有$f_j^hs_i=0$.记$n=hr$,那么$I^nM=0$.
    \end{proof}
\end{enumerate}
\subsection{拟凝聚层的延拓}
\begin{enumerate}
	\item 典范延拓.设$X$是拓扑空间,$\mathscr{F}$是集合层/群层/环层,设$U\subseteq X$是开集,典范嵌入记作$i$.设$\mathscr{G}$是$\mathscr{F}\mid_U=i^{-1}\mathscr{F}$的子层.按照$i_*$是左正合的,就有$i_*\mathscr{G}$是$i_*i^{-1}\mathscr{F}$的子层.我们有典范层态射$\rho:\mathscr{F}:i_*i^{-1}\mathscr{F}$,记$\overline{\mathscr{G}}$表示$i_*\mathscr{G}$在$\rho$下的原像.换句话讲,对$X$的任意开子集$V$,有$\Gamma(V,\overline{\mathscr{G}})$由截面$s\in\Gamma(V,\mathscr{F})$构成,满足$s\mid_{V\cap U}$是$\mathscr{G}$在$V\cap U$上的一个截面.于是$\overline{\mathscr{G}}\mid_U=i^*\overline{\mathscr{G}}=\mathscr{G}$,并且$\overline{\mathscr{G}}$就是$\mathscr{F}$的在$U$上的限制为$\mathscr{G}$的最大的子层,它称为$\mathscr{F}\mid_U$的子层$\mathscr{G}$的典范延拓.
	\item 设$X$是概形,$U$是开子集,设典范开嵌入$j:U\to X$是拟紧的,那么:
	\begin{enumerate}[(1)]
		\item 对任意拟凝聚$\mathscr{O}_U$模层$\mathscr{G}$,都有$j_*\mathscr{G}$是拟凝聚$\mathscr{O}_X$模层.并且有$(j_*\mathscr{G})\mid_U=j^*j_*\mathscr{G}=\mathscr{G}$.
		\item 对任意拟凝聚$\mathscr{O}_X$模层$\mathscr{F}$,任取$\mathscr{F}\mid_U$的拟凝聚$\mathscr{O}_U$子模层$\mathscr{G}$,都有典范延拓$\overline{\mathscr{G}}$是$\mathscr{F}$的拟凝聚$\mathscr{O}_X$子模层.
	\end{enumerate}
    \item 推论.设$X$是概形,$U\subseteq X$是拟紧开集,设典范开嵌入$j:U\to X$是拟紧的.设任何拟凝聚$\mathscr{O}_X$模层都是它的的有限生成拟凝聚子模层的正向极限(例如$X$是仿射的).设$\mathscr{F}$是拟凝聚$\mathscr{O}_X$模层,设$\mathscr{G}$是$\mathscr{F}\mid_U$的一个有限生成拟凝聚$\mathscr{O}_U$子模层.那么存在$\mathscr{F}$的有限生成拟凝聚$\mathscr{O}_X$子模层$\mathscr{G}'$满足$\mathscr{G}'\mid_U=\mathscr{G}$.特别的,对任意有限生成拟凝聚$\mathscr{O}_U$模层$\mathscr{G}$,都可以找到一个有限生成拟凝聚$\mathscr{O}_X$模层$\mathscr{G}'$使得$\mathscr{G}'\mid_U=\mathscr{G}$.
    \begin{proof}
    	
    	按照上一条有$\overline{\mathscr{G}}$是拟凝聚层,并且有$\mathscr{G}=\overline{\mathscr{G}}\mid_U$.于是它可以表示为它的有限生成拟凝聚子模层$\{\mathscr{H}_i\}$的正向极限.进而$\mathscr{G}$是$\{\mathscr{H}_i\mid_U\}$的正向极限.于是按照$\mathscr{G}$是有限生成的,它就是某个$\mathscr{H}_i$.
    \end{proof}
    \item 设$X$是概形,设对任意仿射开子集$U$都有典范开嵌入$U\to X$是拟紧的.设对任意仿射开子集$U$,设$\mathscr{F}$是拟凝聚$\mathscr{O}_X$模层,设对任意仿射开子集$U$和任意$\mathscr{F}\mid_U$的有限生成拟凝聚$\mathscr{O}_U$子模层$\mathscr{G}$,都有$\mathscr{F}$的有限生成拟凝聚子模层$\mathscr{G}'$使得$\mathscr{G}'\mid_U=\mathscr{G}$(此即上一条的结论对固定的拟凝聚层$\mathscr{F}$和任意仿射开子集成立).那么$\mathscr{F}$是它全体有限生成拟凝聚子模层的正向极限.
    \begin{proof}
    	
    	对任意仿射开子集$U$,记$\Gamma(U,\mathscr{O}_X)=A$和$\mathscr{F}\mid_U=\widetilde{M}$,其中$M$是$A$模.按照$M$是它有限生成子模的正向极限,按照条件存在$\mathscr{F}$的有限生成子模层$\mathscr{G}_U$使得它限制在$U$上就是$\widetilde{M}$.那么$\{\mathscr{G}_U\}$的有限和构成的正向系统的极限就是$\mathscr{F}$.
    \end{proof}
    \item 引理.设$X$是概形,$L$是良序集,$\{V_{\lambda}\}_{\lambda\in L}$是$X$的仿射开覆盖.设$U\subseteq X$是开子集,对任意$\lambda\in L$,记$W_{\lambda}=\cup_{\mu<\lambda}V_{\mu}$.再设:
    \begin{enumerate}[(1)]
    	\item 对任意$\lambda\in L$有$V_{\lambda}\cap W_{\lambda}$是拟紧的.
    	\item 开浸入$U\to X$是拟紧的.
    \end{enumerate}

    那么对任意拟凝聚$\mathscr{O}_X$模层$\mathscr{F}$和$\mathscr{F}\mid_U$的任意有限生成的拟凝聚$\mathscr{O}_U$子模层$\mathscr{G}$,都可以找到$\mathscr{F}$的有限生成拟凝聚子模层$\mathscr{G}'$,使得$\mathscr{G}'\mid_U=\mathscr{G}$.
    \begin{proof}
    	
    	记$U_{\lambda}=U\cup W_{\lambda}$.我们来构造族$\{\mathscr{G}'_{\lambda}\}$,其中$\mathscr{G}'_{\lambda}$是$\mathscr{F}\mid_{U_{\lambda}}$的有限生成拟凝聚子模层,满足对$\mu<\lambda$有$\mathscr{G}'_{\lambda}\mid_{U_{\mu}}=\mathscr{G}'_{\mu}$和$\mathscr{G}'_{\lambda}\mid_U=\mathscr{G}$.一旦这成立,这族模层粘合而成的$\mathscr{F}$的拟凝聚子模层$\mathscr{G}'$就满足要求.
    	
    	\qquad
    	
    	用超限归纳构造这族模层.固定$\lambda\in L$,假设对任意$\mu<\lambda$已经构造了$\mathscr{G}'_{\mu}$满足上述条件.如果$\lambda$没有前继元,那么$\{U_{\mu},\mu<\lambda\}$是$U_{\lambda}$的开覆盖,所以这些$\{\mathscr{G}'_{\mu}\}$粘合为$\mathscr{F}\mid_{U_{\lambda}}$的唯一的满足要求的子模层$\mathscr{G}'_{\lambda}$.如果$\lambda$有前继元,设$\lambda=\mu+1$,那么$U_{\lambda}=U_{\mu}\cup V_{\mu}$.于是问题归结为找$\mathscr{F}\mid_{V_{\mu}}$的有限生成拟凝聚子模层$\mathscr{G}''_{\mu}$,使得$\mathscr{G}''_{\mu}\mid_{U_{\mu}\cap V_{\mu}}=\mathscr{G}'_{\mu}\mid_{U_{\mu}\cap V_{\mu}}$,再取粘合即可.但是这里$V_{\mu}$是仿射的,并且条件保证了$U_{\mu}\cup V_{\mu}$是拟紧的,于是上面推论保证了$\mathscr{G}''_{\mu}$存在.
    \end{proof}
    \item 推论.设$X$是qcqs概形,设$U\subseteq X$是开子集.那么对任意拟凝聚$\mathscr{O}_X$模层$\mathscr{F}$以及$\mathscr{F}\mid_U$的任意有限生成拟凝聚子模层$\mathscr{G}$,都可以找到$\mathscr{F}$的一个有限生成拟凝聚子模层$\mathscr{G}'$使得$\mathscr{G}'\mid_U=\mathscr{G}$.
    \item 推论.在上述条件下,任意有限生成的拟凝聚$\mathscr{O}_U$模层$\mathscr{G}$,都存在有限生成拟凝聚$\mathscr{O}_X$模层$\mathscr{G}'$满足$\mathscr{G}'\mid_U=\mathscr{G}$.
    \item 推论.在上述条件下,任意拟凝聚$\mathscr{O}_X$模层都是它有限生成拟凝聚子模层的正向极限.
    \item 推论.在上述条件下,如果拟凝聚$\mathscr{O}_X$模层$\mathscr{F}$的任意有限生成拟凝聚子模层都被整体截面生成,那么$\mathscr{F}$也被整体截面生成.
    \begin{proof}
    	
    	任取仿射开子集$U$,任取$s\in\Gamma(U,\mathscr{F})$,那么$s$生成了$\mathscr{F}\mid_U$的一个有限生成拟凝聚子模层$\mathscr{G}$.那么按照上述定理就有$\mathscr{F}$的有限生成拟凝聚子模层$\mathscr{G}'$使得$\mathscr{G}'\mid_U=\mathscr{G}$.按照条件$\mathscr{G}'$被整体截面生成.最后让$U$跑遍仿射开子集,让$s$跑遍局部截面,把这些整体截面取并即生成了整个$\mathscr{F}$.
    \end{proof}
\end{enumerate}
\subsection{拟凝聚代数层}

设$X$是环空间,一个$\mathscr{O}_X$代数模层也是$\mathscr{O}_X$模层.我们称一个代数模层是拟凝聚的,如果它作为模层是拟凝聚的.
\begin{enumerate}
	\item 对于仿射概形$X=\mathrm{Spec}A$,有$B\mapsto\widetilde{B}$是从$\textbf{Alg}(A)$到$\mathscr{O}_X$上拟凝聚代数层范畴的范畴等价.
	\item 设$X$是概形,设$\mathscr{B}$是拟凝聚$\mathscr{O}_X$代数层.那么一个$\mathscr{B}$模层$\mathscr{F}$是拟凝聚$\mathscr{B}$模层,当且仅当$\mathscr{F}$是一个拟凝聚$\mathscr{O}_X$模层.
	\begin{proof}
		
		问题是局部的,不妨设$X=\mathrm{Spec}A$是仿射的.那么$\mathscr{B}=\widetilde{B}$,其中$B$是一个$A$代数.如果$\mathscr{F}$是环空间$(X,\mathscr{B})$上的拟凝聚层,不妨设$\mathscr{F}$是某个$\mathscr{B}$模层态射$\mathscr{B}^{(I)}\to\mathscr{B}^{(J)}$的余核.这自然也是一个$\mathscr{O}_X$拟凝聚模层之间的态射,这就说明$\mathscr{F}$是拟凝聚$\mathscr{O}_X$模层.
		
		\qquad
		
		反过来设$\mathscr{F}$是拟凝聚$\mathscr{O}_X$模层,那么$\mathscr{F}=\widetilde{M}$,其中$M$是一个$A$模,并且它自然的具备一个$B$模结构.于是$M$是某个同态$B^{(I)}\to B^{(J)}$的余核,进而$\mathscr{F}$是拟凝聚$\mathscr{B}$模层.
	\end{proof}
    \item 特别的,如果$\mathscr{F},\mathscr{G}$是两个拟凝聚$\mathscr{B}$模层,那么$\mathscr{F}\otimes_{\mathscr{B}}\mathscr{G}$是拟凝聚$\mathscr{B}$模层.并且如果$\mathscr{G}$是有限表示$\mathscr{B}$模层,那么$\mathrm{HOM}_{\mathscr{B}}(\mathscr{F},\mathscr{G})$是拟凝聚$\mathscr{B}$模层.
    \item 称$\mathscr{O}_X$代数层$\mathscr{B}$是有限型代数层,如果存在$X$的仿射开覆盖$\{U_i\}$,使得$\Gamma(U_i,\mathscr{B})$总是有限型$\Gamma(U_i,\mathscr{O}_X)$代数.那么有限型$\mathscr{O}_X$代数层一定是拟凝聚的;局部诺特概形上的有限型代数层一定是凝聚层.
    \item 设$X$是qcqs概形,那么对任意有限型拟凝聚$\mathscr{O}_X$代数层$\mathscr{B}$,都可以找到一个有限生成拟凝聚$\mathscr{O}_X$模层$\mathscr{E}$,使得$\mathscr{E}$代数生成$\mathscr{B}$.特别的,任意拟凝聚$\mathscr{O}_X$代数层都是它自身的有限生成拟凝聚$\mathscr{O}_X$子模层的正向极限.
    \begin{proof}
    	
    	取$X$的有限仿射开覆盖$\{U_i\}$,使得$\Gamma(U_i,\mathscr{B})=B_i$是有限型$\Gamma(U_i,\mathscr{O}_X)=A_i$代数.设$E_i$是$B_i$的一个能够代数生成$B_i$的有限$A_i$子模.按照拟凝聚层的延拓定理,可以找到$\mathscr{B}$的有限生成拟凝聚$\mathscr{O}_X$子模层$\mathscr{E}_i$,使得它限制在$U_i$上就是$\widetilde{E_i}$.再取$\mathscr{E}$为这些子模$\{\mathscr{E}_i\}$的和即可.
    \end{proof}
\end{enumerate}






\subsection{函子$\mathscr{A}$}

我们要定义两种函子$\mathscr{A}$.首先是第一种:
\begin{enumerate}
	\item 
	\begin{itemize}
		\item 设$f:X\to S$是概形之间的态射,我们用$\mathscr{A}(X)$表示正像$f_*\mathscr{O}_X$,换句话讲它是$S$上的代数层,满足对$S$的开子集$U$有$\mathscr{A}(X)(U)=\mathscr{O}_X(f^{-1}(U))$.
		\item 再设$\mathscr{F}$是$X$上的模层,用$\mathscr{A}(\mathscr{F})$表示正像$f_*\mathscr{F}$.那么$\mathscr{A}(\mathscr{F})$不仅是$\mathscr{O}_S$模层,它还是$\mathscr{A}(X)$模层.
		\item 类似的如果$\mathscr{B}$是$\mathscr{O}_X$代数层,用$\mathscr{A}(\mathscr{B})$表示正像$f_*\mathscr{B}$.那么它不仅是$\mathscr{O}_S$代数层,它还是$\mathscr{A}(X)$代数层.
	\end{itemize}
    \item 函子性.设$g:Y\to S$是另一个$S$概形,设$h:X\to Y$是一个$S$态射,那么$h^{\#}:\mathscr{O}_Y\to h_*\mathscr{O}_X$是层态射,于是$g_*h^{\#}:g_*\mathscr{O}_Y\to g_*h_*\mathscr{O}_X=f_*\mathscr{O}_X$是模层之间的态射,我们把它记作$\mathscr{A}(h)$.于是$\mathscr{A}$是$\textbf{Sch}(S)\to\textbf{Mod}(\mathscr{O}_S)$的逆变函子.如果结构态射$X\to S$是仿射的,此时$\mathscr{A}(X)$是拟凝聚$\mathscr{O}_S$代数层,于是$\mathscr{A}$还可以限制为逆变函子$\textbf{AffSch}(S)\to\textbf{QCohAlg}(S)$.
    \item 如果$f:X\to S$是仿射态射,$g:Y\to S$是任意$S$概形,那么$h\mapsto\mathscr{A}(h)$诱导了如下双射.于是特别的,$\mathscr{A}$如果限制为$\textbf{AffSch}(S)\to\textbf{QCohAlg}(S)$就是一个完全忠实函子(后面会证明此时它是范畴等价函子).
    $$\mathrm{Hom}_S(Y,X)\cong\mathrm{Hom}(\mathscr{A}(X),\mathscr{A}(Y))$$
    \begin{proof}
    	
    	先设$\{S_i\}$是$S$的一个仿射开覆盖,按照$f$是仿射态射,就有$f^{-1}(S_i)$都是仿射的.任取$\mathscr{O}_S$代数层的态射$\omega:\mathscr{A}(X)\to\mathscr{A}(Y)$,则它可以限制为一族代数层的态射$\omega_i:\mathscr{A}(f^{-1}(S_i))\to\mathscr{A}(g^{-1}(S_i))$.倘若我们解决了$X,S$同时是仿射的情况,则这里每个$\omega_i$唯一的对应于一个$S_i$态射$h_i:g^{-1}(S_i)\to f^{-1}(S_i)$.所以只需验证这些态射可以粘合为一个$S$态射$h:Y\to X$,而这归结为证明对$S_i\cap S_j$中的仿射开子集$U$,我们有$h_i$和$h_j$在$g^{-1}(U)$上的限制是一致的,但是这两个限制都对应于$\omega$在$U$上的限制.所以这又归结到仿射情况.综上我们归结为设$S=\mathrm{Spec}A$和$X=\mathrm{Spec}B$的情况.
    	
    	\qquad
    	
    	我们要证明的是对任意$\mathscr{O}_S$代数层同态$\omega:f_*\mathscr{O}_X\to g_*\mathscr{O}_Y$,都存在唯一的$S$态射$h:Y\to X$满足$\mathscr{A}(h)=\omega$.对任意开集$U\subseteq S$,有$\omega$定义了一个$\mathscr{O}_S(U)$代数同态$\omega_U:\mathscr{O}_X(f^{-1}(U))\to\mathscr{O}_Y(g^{-1}(U))$.特别的对$U=S$,这给出了一个$\mathscr{O}_S(S)=A$代数同态$\mathscr{O}_X(X)=B\to\mathscr{O}_Y(Y)$,而它唯一的对应于一个$S$态射$h:Y\to X$.我们断言$\mathscr{A}(h)=\omega$:任取$S$的主开集$U=D(a)$,其中$a\in A$,那么$f^{-1}(U)=D(b)$,这里$b=\varphi(a)$,而$\varphi:A\to B$是$f$对应的环同态.那么$\varphi_U=\mathscr{A}(h)(U):\mathscr{O}_X(f^{-1}(U))=B_b\to\mathscr{O}_Y(g^{-1}(U))$就是$h$限制的$g^{-1}(U)\to f^{-1}(U)=D(b)$所对应的环同态.另一方面$\omega_U:\mathscr{O}_X(f^{-1}(U))\to\mathscr{O}_Y(g^{-1}(U))$.无论$\varphi_U$还是$\omega_U$都满足如下图表交换,按照分式化的泛性质,就有$\varphi_U=\omega_U$对$S$的任意主开集都成立,这导致$\mathscr{A}(h)=\omega$.
    	$$\xymatrix{B_b\ar[rr]^{\varphi_U}_{\omega_U}&&\mathscr{O}_Y(g^{-1}(U))\\B\ar[u]\ar[rr]_{\omega_S=\varphi_S}&&\mathscr{O}_Y(Y)\ar[u]}$$
    	
    	再说明$h$的唯一性,倘若有$h':Y\to X$使得$\mathscr{A}(h')=\omega$,那么${h'}^{\#}(X):\mathscr{O}_X(X)\to\mathscr{O}_Y(Y)$和$\mathscr{A}(h')(S):\mathscr{O}_X(X)\to\mathscr{O}_Y(Y)$是一致的,所以$h'$和$h$对应的环同态是一致的,但是终端为仿射概形的态射被它对应的环同态唯一决定,这导致$h'=h$.
    \end{proof}
    \item 推论.如果$X,Y$是$S$概形,使得结构态射都是仿射态射,那么一个$S$态射$h:X\to Y$是同构当且仅当$\mathscr{A}(h):\mathscr{A}(Y)\to\mathscr{A}(X)$是同构.
\end{enumerate}
    
下面是第二种函子$\mathscr{A}$:
\begin{enumerate}
	\item 设$\mathscr{F}$是$\mathscr{O}_X$模层,$\mathscr{G}$是$\mathscr{O}_Y$模层,那么一个$\mathscr{G}\to\mathscr{F}$的模层态射应该理解为一个$\mathscr{O}_Y$模层态射$u:\mathscr{G}\to h_*\mathscr{F}$.那么$g_*(u):g_*\mathscr{G}\to f_*\mathscr{F}$是一个$\mathscr{O}_S$模层态射$\mathscr{A}(\mathscr{G})\to\mathscr{A}(\mathscr{F})$,把它记作$\mathscr{A}(u)$.
	\item 用范畴的语言讲,设$S$是概形.
	\begin{itemize}
		\item 定义范畴$\textbf{C}_1'$:全体$(X,\mathscr{F})$作为对象,其中$X$是$S$概形,$\mathscr{F}$是$\mathscr{O}_X$模层,定义态射$(X,\mathscr{F})\to(Y,\mathscr{G})$为二元组$(h,u)$,其中$h:X\to Y$是一个$S$态射,而$u:\mathscr{G}\to\mathscr{F}$是模层的态射(它们是不同空间上的模层,它们之间的态射理解为一个$\mathscr{O}_Y$模层态射$u:\mathscr{G}\to h_*\mathscr{F}$,这依赖于$h$的选取).
		\item 定义范畴$\textbf{C}_1$是范畴$\textbf{C}_1'$的完全子范畴,它的对象定义为$(X,\mathscr{F})$,其中额外要求$X$作为$S$概形的结构态射是仿射态射.
		\item 定义范畴$\textbf{C}_2$:全体$(\mathscr{B},\mathscr{M})$作为对象,其中$\mathscr{B}$是拟凝聚$\mathscr{O}_S$代数层,$\mathscr{M}$是拟凝聚$\mathscr{B}$模层.定义态射$(\mathscr{B}_1,\mathscr{M}_1)\to(\mathscr{B}_2,\mathscr{M}_2)$是二元组$(k,v)$,其中$k:\mathscr{B}_1\to\mathscr{B}_2$是$\mathscr{O}_S$代数层之间的态射,而$v:\mathscr{M}_2\to\mathscr{M}_1$的态射(这也是不同环空间上模层之间的态射,理解为一个$\mathscr{B}_2$模层态射$v:\mathscr{M}_2\to k_*\mathscr{M}_1$,这是依赖于$k$的).
	\end{itemize}
	
	那么$\mathscr{A}$可以理解为$\textbf{C}_1'\to\textbf{C}_2$或者$\textbf{C}_1\to\textbf{C}_2$的函子$(X,\mathscr{F})\mapsto(\mathscr{A}(X),\mathscr{A}(\mathscr{F}))$.
	\item 设$X\to S$是仿射态射,$Y\to S$是任意态射,$\mathscr{F}$和$\mathscr{G}$分别是$\mathscr{O}_X$和$\mathscr{O}_Y$的拟凝聚模层,那么映射$(h,u)\mapsto(\mathscr{A}(h),\mathscr{A}(u))$诱导了如下双射.特别的如果$\mathscr{A}$限制为$\textbf{C}_1\to\textbf{C}_2$的函子就是一个完全忠实函子(同样的后文会证明它是一个范畴等价函子).
	$$\mathrm{Hom}_{\textbf{C}_1'}((X,\mathscr{F}),(Y,\mathscr{G}))\cong\mathrm{Hom}_{\textbf{C}_2}((\mathscr{A}(Y),\mathscr{A}(\mathscr{G})),(\mathscr{A}(X),\mathscr{A}(\mathscr{F})))$$
	\item 推论.如果$X,Y$是$S$概形,使得结构态射都是仿射态射,那么态射$(h,u):(X,\mathscr{F})\to(Y,\mathscr{G})$是同构当且仅当$(\mathscr{A}(h),\mathscr{A}(u))$是同构.
\end{enumerate}
\subsection{拟凝聚代数层的素谱}

设$\mathscr{B}$是概形$X$上的拟凝聚代数层,我们要构造的素谱$\mathrm{Spec}\mathscr{B}$是$X$上的一个概形.
\begin{enumerate}
	\item $\mathrm{Spec}\mathscr{B}$的粘合描述.设$X$是概形,设$\mathscr{B}$是$X$上的拟凝聚$\mathscr{O}_X$代数层,对任意仿射开集$U\subseteq X$,有$\mathscr{B}(U)$是$\mathscr{O}_X(U)$代数,于是这诱导了态射$\pi_U:\mathrm{Spec}\mathscr{B}(U)\to U$.我们断言当$U$跑遍$X$的仿射开集时$\pi_U$可以粘合为一个$X$概形$\pi:\mathrm{Spec}\mathscr{B}\to X$.并且这在$X$同构意义下是唯一的.
	\begin{proof}
		
		一旦这个粘合存在,则对任意仿射开集$U\subseteq X$有$\pi^{-1}(U)\cong\mathrm{Spec}\mathscr{B}(U)$.于是这个粘合必须是唯一的.另外由于这个构造在仿射情况吻合于我们定义的$\mathrm{Spec}\mathscr{B}$,所以我们只需验证这些$\pi_U$的确是可以粘合的,就得到它的确吻合于我们函子性定义的$\mathrm{Spec}\mathscr{B}$.
		
		\qquad
		
		先设$X=\mathrm{Spec}A$是仿射概形,那么$\mathscr{B}=\widetilde{B}$,其中$B$是$A$代数,我们断言$\mathrm{Spec}B$就是$\pi_U$的粘合.为此任取仿射开子集$U\subseteq X$,那么有$\mathrm{Spec}\mathscr{B}(U)=\mathrm{Spec}(\mathscr{B}(U))=\mathrm{Spec}(\mathscr{B}(X)\otimes_A\mathscr{O}_X(U))=\mathrm{Spec}B\times_XU=f^{-1}(U)$,其中$f:\mathrm{Spec}B\to\mathrm{Spec}A$是典范的$A\to B$诱导的态射.于是如果$X$是仿射的,那么这些$\pi_U$可以粘合,并且我们解释了粘合如果存在就是唯一的.
		
		\qquad
		
		再设$X$是一般的概形,取它的仿射开覆盖$\{U_i\}$,对每个$U_i\cap U_j$再取仿射开覆盖$\{U_{ijk}\}$.那么按照仿射的情况,对于所有包含在$U_i$中的仿射开子集$U$,有$\pi_U$可以粘合为一个态射,记作$\pi_i:\widetilde{X}_i\to U_i$.于是归结为证明这些$\{\pi_i\}$可以粘合,也即$U_{ijk}$在$\pi_i$和$\pi_j$的原像是典范同构的.但是这是因为这两个原像都是当$U$取遍包含在$U_{ijk}$中仿射开子集时$\pi_U$的粘合,仿射情况粘合是唯一的,就得到它们是典范同构的,于是$\pi_i$可以粘合,得证.
	\end{proof}
	\item $\mathrm{Spec}\mathscr{B}$的可表性描述.
	\begin{enumerate}
		\item 仿射情况.设$A$是环,记$X=\mathrm{Spec}A$,设$B$是$A$代数,那么我们解释过对任意$X$概形$T$有如下自然同构:
		$$\mathrm{Hom}_X(T,\mathrm{Spec}B)\cong\mathrm{Hom}_{\textbf{Alg}(A)}(B,\Gamma(T,\mathscr{O}_T))$$
		\item 一般的,设$X$是概形,设$\mathscr{B}$是拟凝聚$\mathscr{O}_X$代数层,那么函子:
		$$(\textbf{Sch}(X))^{\mathrm{op}}\to\textbf{Sets}$$
		$$(f:T\to X)\mapsto\mathrm{Hom}_{\textbf{Alg}(\mathscr{O}_X)}(\mathscr{B},f_*\mathscr{O}_T)$$
		
		是一个可表函子,它的表示对象是一个$X$概形,称为$\mathscr{B}$的素谱,记作$\mathrm{Spec}\mathscr{B}$.换句话讲,对任意$X$概形$f:T\to X$我们有如下自然双射:
		$$\mathrm{Hom}_X(T,\mathrm{Spec}\mathscr{B})\cong\mathrm{Hom}_{\textbf{Alg}(\mathscr{O}_X)}(\mathscr{B},f_*\mathscr{O}_T)$$
		\begin{proof}
			
			我们解释过一个逆变函子$\textbf{Sch}(X)\to\textbf{Sets}$的可表准则,这样的函子如果是Zariski层,并且存在由可表函子构成的开子函子覆盖,那么它是可表函子.我们这里的函子是Zariski层是因为$\mathrm{Hom}$和$f_*$都是左正合的;第二个条件归结为证明在仿射情况下这个函子是可表的,为此我们设$X=\mathrm{Spec}A$和$\mathscr{B}=\widetilde{B}$,那么有:
			\begin{align*}
				\mathrm{Hom}_X(T,\mathrm{Spec}B)&=\mathrm{Hom}_{\textbf{Alg}(A)}(B,\Gamma(T,\mathscr{O}_T))\\&=\mathrm{Hom}_{\textbf{Alg}(\mathscr{O}_X)}(\mathscr{B},f_*\mathscr{O}_T)
			\end{align*}
			
			这里第二个等号是因为(并不是因为$f_*\mathscr{O}_T$是拟凝聚层,这一般不是拟凝聚层)我们解释过如果$X=\mathrm{Spec}A$是仿射的,如果$\widetilde{M}$是拟凝聚层,如果$\mathscr{F}$是任意$\mathscr{O}_X$模层,那么有:
			$$\mathrm{Hom}_{\mathscr{O}_X}(\widetilde{M},\mathscr{F})=\mathrm{Hom}_A(M,\Gamma(X,\mathscr{F}))$$
		\end{proof}
	    \item 这件事还可以这样解读:我们记$\textbf{AffSch}(X)$为结构态射是仿射态射的$X$概形构成的范畴(这样的$X$概形之间的态射一定是仿射态射),那么$\mathrm{Spec}:\textbf{QCohAlg}(X)\to\textbf{AffSch}(X)$和$\mathscr{A}:\textbf{AffSch}(X)\to\textbf{QCohAlg}(X)$是互为伴随的逆变函子.但是我们解释过$\mathscr{A}$限制在结构态射是仿射态射的$X$概形上是完全忠实的,下面会解释$\mathrm{Spec}$也是完全忠实的,进而有它们实际上是互相伴随的逆变范畴等价函子.
	\end{enumerate}
	\item 等价描述.设$X$是概形,设$\mathscr{B}$是一个拟凝聚$\mathscr{O}_X$代数层,那么在$X$同构意义下存在唯一的$X$概形$Y$,满足结构态射$Y\to X$是仿射的,并且$\mathscr{A}(Y)=\mathscr{B}$.这个$X$同构意义下唯一的$Y$就是$\mathrm{Spec}B$.
	\begin{proof}
		
		如果$X=\mathrm{Spec}A$是仿射的,那么$X$上的拟凝聚代数层可以表示为$\widetilde{B}$,其中$B$是一个$A$代数,那么$\widetilde{B}$的素谱就是它作为环的素谱$\mathrm{Spec}B$.更一般的如果$X$是一般概形,把$\mathscr{O}_X$拟凝聚代数层$\mathscr{B}$的结构态射记作$h:\mathrm{Spec}\mathscr{B}\to X$,任取仿射开子集$U=\mathrm{Spec}A\subseteq X$,那么有$U$同构$h^{-1}(U)\cong\mathrm{Spec}\Gamma(U,\mathscr{B})$.这说明结构态射$\mathrm{Spec}\mathscr{B}\to X$总是仿射态射,并且总有$h_*\mathscr{O}_{\mathrm{Spec}\mathscr{B}}=\mathscr{B}$.换句话讲$\mathscr{A}(\mathrm{Spec}\mathscr{B})=\mathscr{B}$.
	\end{proof}
	\item 推论.按照上一条以及$\mathrm{Spec}$的可表性得到:如果$\mathscr{B}$和$\mathscr{B}'$是两个$\mathscr{O}_X$拟凝聚代数层,那么有如下自然双射.这说明$\mathrm{Spec}$是完全忠实函子(进而有我们解释的$\mathscr{A}$限制在仿射结构态射上和$\mathrm{Spec}$是互相伴随的逆变的范畴等价函子).
	$$\mathrm{Hom}_{\textbf{Alg}(\mathscr{O}_X)}(\mathscr{B},\mathscr{B}')\cong\mathrm{Hom}_X(\mathrm{Spec}\mathscr{B}',\mathrm{Spec}\mathscr{B})$$
	\item 推论.对概形$X$上的拟凝聚代数层$\mathscr{B}$,我们有典范$\mathscr{O}_X$代数层同构$\mathscr{A}(\mathrm{Spec}\mathscr{B})\cong\mathscr{B}$;对仿射结构态射的$X$概形$Y$,我们有典范$S$同构$\mathrm{Spec}\mathscr{A}(X)\cong X$.
    \item 设$g:X'\to X$是概形之间的态射,我们有如下关于位置$\mathscr{B}$的$X'$概形同构:
    $$\mathrm{Spec}g^*\mathscr{B}\cong(\mathrm{Spec}\mathscr{B})\times_XX'$$
    \begin{proof}
    	
    	对任意$X'$概形$f':T'\to X'$,我们有:
    	\begin{align*}
    		\mathrm{Hom}_{X'}(T',(\mathrm{Spec}\mathscr{B})\times_XX')&=\mathrm{Hom}_X(T',\mathrm{Spec})\\&=\mathrm{Hom}_{\textbf{Alg}(\mathscr{O}_X)}(\mathscr{B},g_*(f_*'\mathscr{O}_{T'}))\\&=\mathrm{Hom}_{\textbf{Alg}(\mathscr{O}_{X'})}(g^*\mathscr{B},f_*'\mathscr{O}_{T'})\\&=\mathrm{Hom}_{X'}(T',\mathrm{Spec}g^*\mathscr{B})
    	\end{align*}
    \end{proof}
    \item 
    \begin{enumerate}
    	\item 设$X\to S$是仿射态射,那么这是有限型态射当且仅当拟凝聚$\mathscr{O}_S$代数层$\mathscr{A}(X)$是有限型的.
    	\item 设$X\to S$是仿射态射,那么$X$是既约概形当且仅当拟凝聚$\mathscr{O}_S$代数层$\mathscr{A}(X)$是既约的.
    \end{enumerate}
    \item 设$Y$是$S$概形,设$X,X'$是$Y$概形,它们的结构态射都是仿射态射,于是$X,X'$在$S$上的结构态射也是仿射的.设$\mathscr{B}=\mathscr{A}(Y)$,$\mathscr{A}=\mathscr{A}(X)$和$\mathscr{A}'=\mathscr{A}(X')$,我们断言有$X\times_YX'\cong\mathrm{Spec}\mathscr{A}\otimes_{\mathscr{B}}\mathscr{A}'$.
    \begin{proof}
    	
    	首先拟凝聚代数层的张量积$\mathscr{A}\otimes_{\mathscr{B}}\mathscr{A}'$还是拟凝聚的.记$Z=\mathrm{Spec}\mathscr{A}\otimes_{\mathscr{B}}\mathscr{A}'$,记典范的$\mathscr{B}$模层同态$\mathscr{A}\to\mathscr{A}\otimes_{\mathscr{B}}\mathscr{A}'$和$\mathscr{B}$模层同态$\mathscr{A}'\to\mathscr{A}\otimes_{\mathscr{B}}\mathscr{A}'$唯一对应的$Y$态射分别为$p:Z\to X$和$p':Z\to X'$.我们只需证明$(Z,p,p')$就是纤维积$X\times_YX'$.
    	
    	\qquad
    	
    	为此我们可以归结为设$S=\mathrm{Spec}C$是仿射概形,那么此时$X,X',Y$都是仿射概形,对应的环分别记作$A,A',B$,那么$B$是$C$代数,而$A,A'$是$B$代数.并且有$\mathscr{B}=\widetilde{B}$,$\mathscr{A}=\widetilde{A}$和$\mathscr{A}'=\widetilde{A'}$.此时拟凝聚$\mathscr{B}$代数层$\mathscr{A}\otimes_{\mathscr{B}}\mathscr{A}'$就是$B$代数$A\otimes_BA'$的伴随代数层.此时$p,p'$就分别对应于典范映射$A\to A\otimes_BA'$和$A'\to A\otimes_BA'$,但是我们解释过仿射情况下纤维积就是张量积的素谱,问题得证.
    \end{proof}
\end{enumerate}
\subsection{拟凝聚模层的伴随模层}
\begin{enumerate}
	\item 粘合定义.设$S$是概形,设$\mathscr{B}$是拟凝聚$\mathscr{O}_S$代数层,设$\mathscr{M}$是拟凝聚$\mathscr{B}$模层.那么可记$X=\mathrm{Spec}\mathscr{B}$,记结构态射$p:X\to S$.任取仿射开子集$U=\mathrm{Spec}A\subseteq S$,那么$p^{-1}(U)=\mathrm{Spec}B\subseteq X$也是仿射的,并且有$\mathscr{B}(U)=B$.那么此时$\Gamma(U,\mathscr{M})$是一个$B$模,于是$\widetilde{\Gamma(U,\mathscr{M})}$是$p^{-1}(U)$上的拟凝聚层,我们断言当$U$跑遍$S$的仿射开子集时,有这些$\widetilde{\Gamma(U,\mathscr{M})}$可以粘合为$X$上的一个拟凝聚层,把它记作$\widetilde{\mathscr{M}}$,它称为$\mathscr{M}$的伴随模层,它在$\mathscr{O}_X$模层同构意义下是唯一的.
	\item 等价描述.条件同上一条,那么$\widetilde{\mathscr{M}}$是在同构意义下唯一的$\mathscr{O}_X$模层,使得$\mathscr{A}(\widetilde{\mathscr{M}})=\mathscr{M}$.
	\item 条件同上,那么函子$\textbf{QCoh}(\mathscr{B})\to\textbf{QCoh}(X)$,$\mathscr{M}\mapsto\widetilde{\mathscr{M}}$是加性正合函子,并且和归纳极限与直和可交换.
	\item 推论.条件同上,那么$\mathscr{M}$是有限$\mathscr{B}$模层当且仅当$\widetilde{\mathscr{M}}$是有限$\mathscr{O}_X$模层.
	\begin{proof}
		
		问题归结为仿射情况,记$S=\mathrm{Spec}A$,那么$\mathscr{B}=\widetilde{B}$,其中$B$是一个$A$代数,而$\mathscr{M}=\widetilde{M}$,其中$M$是一个$B$模.进而有$X=\mathrm{Spec}B$,而$\widetilde{\mathscr{M}}$就是$B$模$M$的伴随模层,所以它们有限型说的是一件事.
	\end{proof}
	\item 伴随模层的存在性告诉我们之前定义的函子$\mathscr{A}:\textbf{C}_1\to\textbf{C}_2$是本质满的,我们解释过它是完全忠实的,于是这实际上是逆变的范畴等价函子.
	\item 设$Y$是$S$概形,$X,X'$是$Y$概形,它们的结构态射都是仿射态射.再设$\mathscr{F}$和$\mathscr{F}'$分别是$\mathscr{O}_X$和$\mathscr{O}_{X'}$,这两个不同环空间上的拟凝聚层的张量积是这样定义的:记典范投影态射分别为$p:X\times_YX'\to X$和$p':X\times_YX'\to X'$,那么$\mathscr{F}$和$\mathscr{F}'$在$\mathscr{O}_Y$上的张量积定义为$X\times_YX'$上的拟凝聚层$p^*\mathscr{F}\otimes_{X\times_YX'}{p'}^*\mathscr{F}'$,它记作$\mathscr{F}\otimes_Y\mathscr{F}'$.这里我们断言有典范同构:
	$$\mathscr{A}(\mathscr{F}\otimes_Y\mathscr{F}')\cong\mathscr{A}(\mathscr{F})\otimes_{\mathscr{A}(Y)}\mathscr{A}(\mathscr{F}')$$
	\begin{proof}
		
		设结构态射$g:Y\to S$,$f:X\to Y$和$f':X'\to Y$.我们解释过此时$Z=\mathrm{Spec}\mathscr{A}(X)\otimes_{\mathscr{A}(Y)}\mathscr{A}(X')$以及典范态射$p:Z\to X$和$p':Z\to X'$就是纤维积$X\times_YX'$.于是结构态射$h:Z\to S$满足$h=g\circ f\circ p=g\circ f'\circ p'$.我们来构造如下典范态射:
		$$\mathscr{A}(\mathscr{F})\otimes_{\mathscr{A}(Y)}\mathscr{A}(\mathscr{F}')\to\mathscr{A}(\mathscr{F}\otimes_Y\mathscr{F}')$$
		
		对任意开集$U\subseteq S$,我们有典范的同态$\Gamma(f^{-1}(g^{-1}(U)),\mathscr{F})\to\Gamma(h^{-1}(U),p^*\mathscr{F})$如下:如果记$W=f^{-1}\circ g^{-1}(U)$,那么$p^{-1}(W)=h^{-1}(U)$,我们要构造的同态就是$F(W)\to\lim\limits{\substack{\rightarrow\\p(p^{-1}(W))\subseteq V}}\mathscr{F}(V)$(这里$V$可以取$W$)所诱导的$\mathscr{F}(W)\to p^*\mathscr{F}(p^{-1}(W))=\left(\mathscr{O}_Z\otimes_{p^{-1}\mathscr{O}_X}p^{-1}\mathscr{F}\right)(p^{-1}(W))$.同理我们可以构造典范的同态$\Gamma({f'}^{-1}({g'}^{-1}(U)),\mathscr{F}')\to\Gamma(h^{-1}(U),{p'}^*\mathscr{F}')$.进而它们诱导了张量积之间的同态:
		$$\Gamma(f^{-1}(g^{-1}(U)),\mathscr{F})\otimes_{\Gamma(g^{-1}(U),\mathscr{O}_Y)}\Gamma({f'}^{-1}({g'}^{-1}(U)),\mathscr{F}')$$
		$$\longrightarrow$$
		$$\Gamma(h^{-1}(U),p^*\mathscr{F})\otimes_{\Gamma(h^{-1}(U),\mathscr{O}_Z)}\Gamma(h^{-1}(U),{p'}^*\mathscr{F}')$$
		
		为了证明这是一个$\mathscr{O}_S$模层同构,归结为设$S=\mathrm{Spec}C$是仿射的,那么此时$Y=\mathrm{Spec}B$,$X=\mathrm{Spec}A$,$X'=\mathrm{Spec}A'$和$X\times_YX'=\mathrm{Spec}A\otimes_BA'$都是仿射的.再设$\mathscr{F}=\widetilde{M}$和$\mathscr{F}'=\widetilde{M'}$,其中$M$和$M'$分别是$A$模和$A'$模.那么此时$\mathscr{F}\otimes_Y\mathscr{F}'$对应于$A\otimes_BA'$模$M\times_BM'$.于是我们要证明的典范同构就是说在把$M,M',B$都视为$C$模时有$\mathscr{O}_S$模层同构$\widetilde{M\otimes_BM'}\cong\widetilde{M}\otimes_{\widetilde{B}}\widetilde{M'}$.
	\end{proof}
	\item 推论.
	\begin{itemize}
		\item 如果取$X=Y$和$\mathscr{F}'=\mathscr{O}_{X'}$得到:设$Y$是$S$概形,设$X$是$Y$概形,并且它们的结构态射都是仿射态射.如果$\mathscr{F}$是拟凝聚$\mathscr{O}_Y$模层,记$f:X\to Y$双射结构态射,那么有:
		$$\mathscr{A}(f^*\mathscr{F})\cong\mathscr{A}(\mathscr{F})\otimes_{\mathscr{A}(Y)}\mathscr{A}(X)$$
		\item 如果取$X=X'=Y$,我们得到:设$X$是$S$概形,并且结构态射是仿射的,设$\mathscr{F}$和$\mathscr{G}$是拟凝聚$\mathscr{O}_X$模层,那么有:
		$$\mathscr{A}(\mathscr{F}\otimes_{\mathscr{O}_X}\mathscr{G})\cong\mathscr{A}(\mathscr{F})\otimes_{\mathscr{A}(X)}\mathscr{A}(\mathscr{G})$$
		\item 类似模层的情况,设$X\to S$是仿射态射,设$\mathscr{F}$和$\mathscr{G}$是拟凝聚$\mathscr{O}_X$模层,其中$\mathscr{F}$是有限表示模层,那么我们有典范同构:
		$$\mathscr{A}(\mathrm{HOM}_X(\mathscr{F},\mathscr{G}))\cong\mathrm{HOM}_{\mathscr{A}(X)}(\mathscr{A}(\mathscr{F}),\mathscr{A}(\mathscr{G}))$$
	\end{itemize}
	\item 设$S$是概形,设$\mathscr{B}$是拟凝聚$\mathscr{O}_S$代数层,记$X=\mathrm{Spec}\mathscr{B}$,则对$\mathscr{B}$的任意拟凝聚理想层$\mathscr{I}$,有$\widetilde{\mathscr{I}}$是$\mathscr{O}_X$的拟凝聚理想层,并且$X$的由$\widetilde{\mathscr{I}}$定义的闭子概型典范同构于$\mathrm{Spec}(\mathscr{B}/\mathscr{I})$.换句话讲,如果$h:\mathscr{B}\to\mathscr{B}'$是拟凝聚$\mathscr{O}_S$代数层之间的满同态,那么$\mathscr{A}(h):\mathrm{Spec}\mathscr{B}'\to\mathrm{Spec}\mathscr{B}$是闭嵌入.
\end{enumerate}
\subsection{分次拟凝聚代数层的齐次谱}
\begin{enumerate}
	\item 关于分次模层的补充.
	\begin{enumerate}[(1)]
		\item 设$S$是概形,一个$\mathscr{O}_X$分次代数层$\mathscr{A}$是拟凝聚的当且仅当它的每个分次分支$\mathscr{A}_n$都是拟凝聚模层.类似的一个分次$\mathscr{A}$模层$\mathscr{M}$是拟凝聚的当且仅当它的分次分支$\mathscr{M}_n$都是拟凝聚模层.对自然数$r$,记$\mathscr{A}(r)=\oplus_{n\ge0}\mathscr{A}_r$和$\mathscr{M}(r)=\oplus_{n\ge0}\mathscr{M}_r$以及$\mathscr{A}^{(d)}=\oplus_{n\ge0}\mathscr{A}_{dn}$和$\mathscr{M}^{(d)}=\oplus_{n\ge0}\mathscr{M}_{dn}$.它们分别是$\mathscr{O}_X$分次代数层和$\mathscr{A}$分次模层.并且在$\mathscr{A}$或者$\mathscr{M}$是拟凝聚的时候有$\mathscr{A}(r),\mathscr{A}^{(d)}$或者$\mathscr{M}(r),\mathscr{M}^{(d)}$总是拟凝聚的.
		\item 一个分次拟凝聚$\mathscr{O}_S$代数层$\mathscr{A}$称为有限型的,如果存在$S$的仿射开覆盖$\{U_i\}$,使得每个$\Gamma(U_i,\mathscr{A})$都是有限型$\mathscr{O}_S(U_i)$代数.这等价于讲对任意$S$的仿射开子集$U$都有$\Gamma(U,\mathscr{A})$是有限型$\mathscr{O}_S(U)$代数.
		\item 一个分次拟凝聚$\mathscr{O}_S$代数层$\mathscr{A}$称为被$\mathscr{A}_1$生成的,如果$\mathscr{A}_1$生成的代数就是$\mathscr{A}$.这等价于讲存在$S$的仿射开覆盖$\{U_i\}$,使得每个$\Gamma(U_i,\mathscr{A})$都是$\Gamma(U_i,\mathscr{A}_1)$在$\mathscr{O}_S(U_i)$上代数生成的.这等价于讲对任意$S$的仿射开子集$U$都有$\Gamma(U,\mathscr{A})$是$\Gamma(U_i,\mathscr{A}_1)$在$\mathscr{O}_S(U)$上代数生成的.
	\end{enumerate}
	\item 设$S$是概形,设$\mathscr{A}$是分次拟凝聚$\mathscr{O}_S$代数层.对每个仿射开子集$U\subseteq S$,我们有$\mathscr{O}_S(U)$分次代数$\Gamma(U,\mathscr{A})=\oplus_{n\ge0}\Gamma(U,\mathscr{A}_n)$.这诱导了$U$上的射影概形$\pi_U:\mathrm{Proj}\Gamma(U,\mathscr{A})\to U$,并且这是分离的.当$U$跑遍$S$的所有仿射开子集时,$\pi_U$可以粘合为一个分离$S$概形,记作$\pi:\mathrm{Proj}\mathscr{A}\to S$.它称为$\mathscr{A}$的齐次谱.
	\begin{proof}
		
		这个和拟凝聚代数层的$\mathrm{Spec}$的粘合描述的证明是一样的,仿射情况要用一步如下结论:设$S$是分次环,设$B$是$S_0$代数,那么$S\otimes_{S_0}B=\oplus_{n\ge0}S_n\otimes_{S_0}B$也是分次环,并且有如下典范同构:
		$$\mathrm{Proj}(S\otimes_{S_0}B)\cong\mathrm{Proj}S\times_{\mathrm{Spec}S_0}\mathrm{Spec}B$$
	\end{proof}
    \item 设$S$是概形,设$\mathscr{A}$是分次拟凝聚$\mathscr{O}_S$代数层.那么$\mathrm{Proj}\mathscr{A}$总是分离概形,并且如果$\mathscr{A}$是有限型代数层的时候$\mathrm{Proj}\mathscr{A}\to S$是有限型态射.另外明显的$\mathrm{Proj}$和开子概型上的限制可交换.
    \item 函子性.如果$\mathscr{A},\mathscr{A}'$是两个拟凝聚$\mathscr{O}_X$分次代数层,记结构态射$p:Y=\mathrm{Proj}\mathscr{A}\to X$和$p':Y'=\mathrm{Proj}\mathscr{A}'\to X$.设$\varphi:\mathscr{A}'\to\mathscr{A}$是分次代数层之间的态射.任取仿射开子集$U\subseteq X$,那么$\varphi_U$是分次$\Gamma(U,\mathscr{O}_X)$代数同态$\Gamma(U,\mathscr{A}')\to\Gamma(U,\mathscr{A})$,那么我们解释过这诱导了从$p^{-1}(U)$的开集$G(\varphi_U)$到${p'}^{-1}(U)$的态射$\Phi_U$.如果$V\subseteq U$是另一个仿射开子集,那么$G(\varphi_V)=G(\varphi_U)\cap p^{-1}(U)$,并且$\Phi_V$就是$\Phi_U$在$G(\varphi_V)$上的限制.于是我们可以把$\Phi_U$全部粘合起来,得到一个仿射态射$\Phi:G(\varphi)\to Y'$.它称为$\varphi$的伴随态射,记作$\mathrm{Proj}(\varphi)$.我们之前解释过如果对任意$x\in X$都可以找到仿射开邻域$U$,使得$\Gamma(U,\mathscr{O}_X)$模$\Gamma(U,\mathscr{A}_+)$可以被$\varphi(\Gamma(U,\mathscr{A}'_+))$生成,那么$G(\varphi_U)=p^{-1}(U)$,于是此时$G(\varphi)=X$.
    \item 设$\psi:X'\to X$是概形态射,设$\mathscr{A}$是拟凝聚$\mathscr{O}_X$分次代数层,那么$\mathscr{A}'=\psi^*\mathscr{A}$是拟凝聚$\mathscr{O}_{X'}$分次代数层.我们有典范同构$\mathrm{Proj}\mathscr{A}'\cong\mathrm{Proj}\mathscr{A}\times_XX'$.
    \item 关于主开集$Y_f$.设$X$是概形,设$\mathscr{A}$是拟凝聚分次$\mathscr{O}_X$代数层.设结构态射$\varphi:Y=\mathrm{Proj}\mathscr{A}\to X$.
    \begin{enumerate}[(1)]
    	\item 设$d\ge1$,设$f\in\Gamma(X,\mathscr{A}_d)$是一个正次齐次整体截面.存在$S$的开子集$X_f$,满足对任意$S$的仿射开子集$U$,都有$\varphi^{-1}(U)\cap X_f=D_+(f\mid_U)$典范的等同于$X_U=\mathrm{Proj}\Gamma(U,\mathscr{A})$.概形$Y_f$在$X$上是仿射的,它就是$\mathrm{Spec}\left(\mathscr{A}^{(d)}/(f-1)\mathscr{A}^{(d)}\right)$.
    	\item 设$f\in\Gamma(X,\mathscr{A}_d)$和$g\in\Gamma(X,\mathscr{A}_e)$,其中$d,e\ge1$,那么有$X_{fg}=X_f\cap X_g$.
    	\item 设$\{f_i\}$是$\mathscr{A}$的一族正次数齐次整体截面,满足它们生成的理想层包含了次数足够大时的全部$\mathscr{A}_n$.那么$\cup_iX_{f_i}=X$.
    \end{enumerate}
    \item 设$X$是概形,设$\mathscr{A}$是拟凝聚$\mathscr{O}_X$分次代数层.
    \begin{enumerate}[(1)]
    	\item 对任意正整数$d\ge1$,存在典范$X$同构$\mathrm{Proj}\mathscr{A}\cong\mathrm{Proj}\mathscr{A}^{(d)}$.
    	\item 设$\mathscr{A}'=\mathscr{O}_X\oplus\mathscr{A}_1\oplus\mathscr{A}_2\oplus\cdots$,那么存在典范$X$同构$\mathrm{Proj}\mathscr{A}\cong\mathrm{Proj}\mathscr{A}'$.
    	\item 设$\mathscr{L}$是可逆$\mathscr{O}_X$模层,记$\mathscr{A}_{(\mathscr{L})}=\oplus_{d\ge0}\left(\mathscr{A}_d\otimes\mathscr{L}^{\otimes d}\right)$,那么存在典范$X$同构$\mathrm{Proj}\mathscr{A}\cong\mathrm{Proj}\mathscr{A}_{(\mathscr{L})}$.
    \end{enumerate}
    \begin{proof}
    	
    	问题是局部的,不妨设$X$是仿射的.那么前两条在射影概形那里已经得证了.对于第三条,按照$\mathscr{L}\cong\mathscr{O}_X$得到它们明显同构,但是我们这里需要构造一个典范同构.记$X=\mathrm{Spec}A$,记$\mathscr{A}=\widetilde{S}$,其中$S$是分次$A$代数,记$\mathscr{L}=\widetilde{L}$,其中$L$是秩1自由$A$模,设生成元为$c$.对$n\ge1$构造$S_n\to S_n\otimes_AL^{\otimes n}$为$x_n\mapsto x_n\otimes c^{\otimes n}$,这是一个$A$代数同构.进而得到$A$分次代数同构$\varphi_c:S\cong S_{(L)}=\oplus_{n\ge0}S_n\otimes L^{\otimes n}$.接下来解释这个同态不依赖于$c$的选取.为此任取正$d$次齐次元$f\in S_+$,任取可逆元$\varepsilon\in A^*$和$x\in S_{nd}$,那么有$(x\otimes c^{nd})/(f\otimes c^d)^n=(x\otimes(\varepsilon c)^{nd})/(f\otimes(\varepsilon c)^d)^n$,于是$\varphi_c$诱导的$S_{(f)}\cong(S_{(L)})_{(f\otimes c^d)}$是不依赖于$c$的选取的,
    \end{proof}
    \item 如果$X$是拟紧概形,设$\mathscr{A}$是有限型的拟凝聚$\mathscr{O}_X$分次代数层,那么存在正整数$d$,使得$\mathscr{A}^{(d)}$被$\mathscr{A}_d$生成,并且$\mathscr{A}_d$是有限生成$\mathscr{O}_X$模层.特别的,此时$\mathrm{Proj}\mathscr{A}$总是$X$同构于某个$\mathrm{Proj}\mathscr{A}'$,其中$\mathscr{A}'$被$\mathscr{A}_1'$生成并且$\mathscr{A}_1'$是有限生成$\mathscr{O}_X$模层.
    \item 既约性.
    \begin{enumerate}[(1)]
    	\item 设$\mathscr{A}$是一个拟凝聚$\mathscr{O}_X$分次代数,我们知道它的幂零根$\mathscr{N}$是拟凝聚的.记拟凝聚层$\mathscr{N}_+=\mathscr{N}\cap\mathscr{A}_+$称为$\mathscr{A}_+$的幂零根.任取$x\in X$,有$(\mathscr{N}_+)_x$是$(\mathscr{A}_+)_x$的幂零根.
    	\item 称$\mathscr{A}$是本质既约的,如果$\mathscr{N}_+=0$.这等价于讲对任意$x\in X$有$\mathscr{O}_{X,x}$代数$\mathscr{A}_x$是本质既约的,也即$\mathrm{nil}(\mathscr{A}_x)\cap(\mathscr{A}_x)_+=0$.特别的,对任意$\mathscr{O}_X$分次代数层$\mathscr{A}$有$\mathscr{A}/\mathscr{N}_+$是本质既约的.
    	\item 设$\mathscr{A}$是拟凝聚$\mathscr{O}_X$分次代数层,记$Y=\mathrm{Proj}\mathscr{A}$,那么$Y_{\mathrm{red}}$典范$X$同构于$\mathrm{Proj}(\mathscr{A}/\mathscr{N}_+)$.
    \end{enumerate}
    \item 整性.一个代数层$\mathscr{A}$称为整代数层,如果对任意$x\in X$有$\mathscr{A}_x$是整环,并且$(\mathscr{A}_+)_x=(\mathscr{A}_x)_+\not=0$.设$X$是整概形,设$\mathscr{A}$是拟凝聚$\mathscr{O}_X$分次代数层,满足$\mathscr{A}_0=\mathscr{O}_X$.
    \begin{enumerate}[(1)]
    	\item 如果$\mathscr{A}$是整代数层,那么$Y=\mathrm{Proj}\mathscr{A}$是整概形,并且结构态射$\varphi:Y\to X$是支配的.
    	\item 如果$\mathscr{A}$是本质既约的,如果$Y=\mathrm{Spec}\mathscr{A}$是整概形并且结构态射$\varphi:Y\to X$是支配的,那么$\mathscr{A}$是整代数层.
    \end{enumerate}
    \begin{proof}
    	
    	(1):问题归结为$X$仿射情况,因为整等价于既约和不可约,既约自然是一个局部性质,对于不可约性,设$\{U_i\}$是$X$的由仿射开子集构成的拓扑基,那么每个$\varphi^{-1}(U_i)$都是$Y$的不可约开子集,于是对任意指标$i,j$有$\varphi^{-1}(U_i)\cap\varphi^{-1}(U_j)=\varphi^{-1}(U_i\cap U_j)$是非空的,因为至少包含某个$\varphi^{-1}(U_k)$,于是$Y$是不可约的.
    	
    	\qquad
    	
    	设$X=\mathrm{Spec}A$是仿射的,其中$A$是整环,设$\mathscr{A}=\widetilde{S}$,其中$S$是$A$分次代数.条件要求$S_x$是整环并且$(S_x)_+\not=0$对任意$x\in X$成立.只需证明$S$是整环【EGAII2节】.任取$f,g\in S$,设$fg=0$,那么在每个$S_x$中就有$(f/1)(g/1)=0$,不妨设$S_x$中有$f/1=0$,也即存在$a\in A$,$a\not\in\mathfrak{p}_x$使得$af=0$.但是对任意$z\in X$,在$S_z$中有$(a/1)(f/1)=0$,而$a/1$在$A_z$中不会零因为$A$是整环,于是$f/1=0$,也即$f=0$.
    	
    	\qquad
    	
    	(2):问题仍然是仿射局部的,设$X=\mathrm{Spec}A$是整环的素谱,设$\mathscr{A}=\widetilde{S}$,其中$S$是$A$分次代数.条件是对任意$x\in X$有$(S_x)_+$是既约环,但是按照$A$是整环有$(S_0)_x=A_x$也是既约的,所以总有$S_x$是既约环.于是$S$也是既约环.但是$Y$是整的导致$S$是本质整的【EGAII2节】
    \end{proof}
    \item 关于闭子概型.设$X$是概形,设$\mathscr{A}$是拟凝聚$\mathscr{O}_X$分次代数层.我们要描述$Y=\mathrm{Proj}\mathscr{A}$的闭子概型.
    \begin{enumerate}[(1)]
    	\item 设$X$是概形,$\varphi:\mathscr{A}\to\mathscr{A}'$是拟凝聚$\mathscr{O}_X$分次代数层之间的零次态射,称$\varphi$是几乎满射/单射/双射,如果存在整数$N$使得$n\ge N$时总有$\varphi_n:\mathscr{A}_n\to\mathscr{A}_n'$是满射/单射/双射.如果我们仅考察态射$\mathrm{Proj}(\varphi)$,那么这个条件总可以归结为真正的满射/单射/双射,因为$\mathscr{A}$替换为$\mathscr{A}^{(d)}$不改变齐次谱.
    	\item 如果$\varphi:\mathscr{A}\to\mathscr{A}'$是几乎满射,那么对应的$\Phi=\mathrm{Proj}(\varphi)$在整个$Y'=\mathrm{Proj}\mathscr{A}'$上有定义,并且它是一个闭嵌入.记$\mathscr{I}=\ker\varphi$,那么这个闭嵌入对应的拟凝聚$\mathscr{O}_Y$理想层就是$\widetilde{\mathscr{I}}$.
    	\item 设$\mathscr{A}_0=\mathscr{O}_X$,设$\mathscr{A}$被$\mathscr{A}_1$有限生成.设$Y'$是$Y$的闭子概型,被拟凝聚理想层$\mathscr{I}$所定义.设$\Gamma_*(\mathscr{I})$在$\alpha:\mathscr{A}\to\Gamma_*(\mathscr{O}_Y)$下的逆像是$\mathfrak{a}$,那么分次代数层态射$\mathscr{A}\to\mathscr{A}/\mathfrak{a}$诱导的齐次谱态射就是该闭子概型.
    	\item 推论.设$\mathscr{A}$被$\mathscr{A}_1$生成,设$\varphi:\mathscr{A}\to\mathscr{A}'$是几乎满射,那么对任意整数$n$就有典范同构$\Phi^*(\mathscr{O}_Y(n))\cong\mathscr{O}_{Y'}(n)$.
    \end{enumerate}
\end{enumerate}
\subsection{分次拟凝聚模层的伴随模层}

设$X$是概形,设$\mathscr{A}$是拟凝聚$\mathscr{O}_X$分次代数层,设$\mathscr{M}$是拟凝聚$\mathscr{A}$分次模层.设结构态射$\varphi:Y=\mathrm{Proj}\mathscr{A}\to X$.对$X$的任意仿射开子集$U$,在$Y$的开子集$\varphi^{-1}(U)=\mathrm{Proj}\Gamma(U,\mathscr{A})$上取分次伴随模层$\widetilde{\Gamma(U,\mathscr{M})}$,那么这些模层可以粘合为$Y$上的拟凝聚模层,它称为$\mathscr{M}$的伴随模层,记作$\widetilde{\mathscr{M}}$.
\begin{enumerate}
	\item 设$f\in\Gamma(X,\mathscr{A}_d)$是一个正$d$次整体截面.我们之前解释过有典范同构:
	$$Y_f\cong\mathrm{Spec}\left(\mathscr{A}^{(d)}/(f-1)\mathscr{A}^{(d)}\right)$$
	
	那么$Y_f$上的拟凝聚层$\widetilde{M}\mid_{Y_f}$在这个同构下对应的就是后者上的伴随模层$\widetilde{\mathscr{M}^{(d)}/(f-1)\mathscr{M}^{(d)}}$.
	\item $\mathscr{M}\mapsto\widetilde{\mathscr{M}}$是$\textbf{GrQCoh}(\mathscr{A})\to\textbf{QCoh}(\mathscr{O}_Y)$的共变加性正合函子,并且和任意归纳极限和直和可交换.
	\begin{enumerate}[(a)]
		\item 如果$\mathscr{N}\subseteq\mathscr{M}$是拟凝聚分次$\mathscr{A}$子模层,那么$\widetilde{\mathscr{N}}$是$\widetilde{\mathscr{M}}$的拟凝聚$\mathscr{O}_Y$子模层.
		\item 对$\mathscr{A}$的任意拟凝聚分次理想层$\mathscr{I}$,有$\widetilde{\mathscr{I}}$是$\mathscr{O}_Y$的拟凝聚理想层.
		\item 如果$\mathscr{I}$是$\mathscr{O}_X$的拟凝聚理想层,那么$\mathscr{I}\mathscr{M}$是$\mathscr{M}$的拟凝聚分次$\mathscr{A}$子模层,并且有$\widetilde{\mathscr{I}\mathscr{M}}=\mathscr{I}\widetilde{\mathscr{M}}$.
	\end{enumerate}
	\item 设$\mathscr{M}$和$\mathscr{N}$是两个拟凝聚$\mathscr{A}$分次模层.
	\begin{enumerate}[(a)]
		\item 我们有如下典范态射,如果$\mathscr{A}$被$\mathscr{A}_1$生成,那么这是典范同构:
		$$\widetilde{\mathscr{M}}\otimes_{\mathscr{O}_Y}\widetilde{\mathscr{N}}\to\widetilde{\mathscr{M}\otimes_{\mathscr{A}}\mathscr{N}}$$
		\item 我们有如下典范态射,如果$\mathscr{A}$被$\mathscr{A}_1$生成,并且$\mathscr{M}$是有限表示模层,那么这是典范同构:
		$$\widetilde{\mathrm{HOM}_{\mathscr{A}}(\mathscr{M},\mathscr{N})}\to\mathrm{HOM}_{\mathscr{O}_Y}(\widetilde{\mathscr{M}},\widetilde{\mathscr{N}})$$
	\end{enumerate}
	\item 设$\mathscr{A}'$是另一个拟凝聚$\mathscr{O}_X$分次代数层,设$\varphi:\mathscr{A}'\to\mathscr{A}$是分次代数层态射,我们解释过它诱导了态射$\mathrm{Proj}(\varphi):G(\varphi)\to Y'$.
	\begin{enumerate}[(a)]
		\item 如果$\mathscr{M}$是拟凝聚$\mathscr{A}$分次模层,把它经$\varphi$视为$\mathscr{A}'$上拟凝聚分次模层记作$\mathscr{M}_{[\varphi]}$,那么我们有$\mathscr{O}_{Y'}$模层的典范同构:
		$$\Phi_*\left(\widetilde{M}\mid_{G(\varphi)}\right)\cong\widetilde{\mathscr{M}_{[\varphi]}}$$
		\item 如果$\mathscr{M}'$是拟凝聚$\mathscr{A}'$分次模层,那么我们如下函子性的$\mathscr{O}_Y\mid_{G(\varphi)}$模层的态射,并且如果$\mathscr{A}'$被$\mathscr{A}_1'$生成,那么这是自然同构:
		$$\Phi^*\widetilde{\mathscr{M}'}\to\widetilde{\mathscr{M}'\otimes_{\mathscr{A}'}\mathscr{A}}\mid_{G(\varphi)}$$
	\end{enumerate}
	\item 设$\psi:X'\to X$是概形之间的态射,设$\mathscr{A}$是拟凝聚$\mathscr{O}_X$分次代数层,那么$\mathscr{A}'=\psi^*\mathscr{A}$是拟凝聚$\mathscr{O}_{X'}$分次代数层.设$\mathscr{M}$是拟凝聚$\mathscr{A}$分次模层,那么$\mathscr{M}'=\psi^*\mathscr{M}$是拟凝聚$\mathscr{A}'$分次模层.我们有典范的$\mathscr{O}_{\mathrm{Proj}\mathscr{A}'}$模层同构$\widetilde{\mathscr{M}'}\cong\widetilde{M}\otimes_X\mathscr{O}_{X'}$.
    \item 扭曲层.设$f\in\Gamma(X,\mathscr{A}_d)$是正次整体截面,在开集$Y_f$上,对任意整数$n$,有$\mathscr{O}_{Y_f}$模层$\widetilde{\mathscr{A}(nd)}\mid_{Y_f}$都典范同构于$\mathscr{O}_{Y_f}$.特别的,如果$\mathscr{A}$被$\mathscr{A}_1$生成,那么对任意整数$n$有$\widetilde{\mathscr{A}(n)}$是可逆$\mathscr{O}_Y$模层,它记作$\mathscr{O}_Y(n)$.对任意$\mathscr{O}_Y$模层$\mathscr{F}$,依旧记$\mathscr{F}(n)=\mathscr{F}\otimes_{\mathscr{O}_Y}\mathscr{O}_Y(n)$.
    \begin{enumerate}[(1)]
    	\item 按照构造这明显和开子概型可交换,即对$X$的开子集$U$有$\mathscr{O}_{f^{-1}(U)}(n)=\widetilde{(\mathscr{A}\mid_U)(n)}=\mathscr{O}_Y(n)\mid_{f^{-1}(U)}$.
    	\item 如果$\mathscr{A}$被$\mathscr{A}_1$生成,那么有$\mathscr{O}_Y(n)=\mathscr{O}_Y(1)^{\otimes n}$.
    	\item 如果$\mathscr{A}$被$\mathscr{A}_1$生成,那么有$\widetilde{\mathscr{M}(n)}=\widetilde{\mathscr{M}}(n)$.
    	\item 记$\mathscr{A}'=\mathscr{O}_X\oplus\mathscr{A}_1\oplus\mathscr{A}_2\oplus\cdots$,我们解释过有典范同构$Y=\mathrm{Proj}\mathscr{A}\cong Y'=\mathrm{Proj}\mathscr{A}'$.那么在这个同构下$\mathscr{O}_Y(n)$对应于$\mathscr{O}_{Y'}(n)$.
    	\item 典范同构$Y=\mathrm{Proj}\mathscr{A}\cong Y^{(d)}=\mathrm{Proj}\mathscr{A}^{(d)}$下$\mathscr{O}_Y(dn)$对应于$\mathscr{O}_{X^{(d)}}(n)$.
    	\item 设$\mathscr{L}$是可逆$\mathscr{O}_X$模层,设$g$是我们之前构造的典范同构$Y_{(\mathscr{L})}=\mathrm{Proj}\mathscr{A}_{\mathscr{L}}\cong Y=\mathrm{Proj}\mathscr{A}$.那么在这个同构下$\mathscr{O}_{Y_{(\mathscr{L})}}(n)$对应于$\mathscr{O}_Y(n)\otimes_X\mathscr{L}^{\otimes n}$
    	\item 设$\psi:X'\to X$是概形之间的态射,设$\mathscr{A}$是拟凝聚$\mathscr{O}_X$分次代数层,那么$\mathscr{A}'=\psi^*\mathscr{A}$是拟凝聚$\mathscr{O}_{X'}$分次代数层.对任意整数$n$,我们有典范同构:
    	$$\mathscr{O}_{Y'}(n)\cong\mathscr{O}_Y(n)\otimes_X\mathscr{O}_{X'}$$
    \end{enumerate}
    \item 关于有限性条件.设$X$是概形,设$\mathscr{A}$是拟凝聚$\mathscr{O}_X$代数层,设$\mathscr{M}$是拟凝聚$\mathscr{A}$分次模层.
    \begin{itemize}
    	\item 称$\mathscr{A}$被$\mathscr{A}_1$有限生成,如果$\mathscr{A}_1$生成了$\mathscr{A}$,并且$\mathscr{A}_1$是有限生成模层.
    	\item 称$\mathscr{M}$几乎为零,如果存在整数$N$使得$n\ge N$时$\mathscr{M}_n=0$.
    	\item 称$\mathscr{M}$是几乎有限型,如果存在整数$N$使得$\oplus_{n\ge N}\mathscr{M}_n$是有限生成模层.
    \end{itemize}
    \begin{enumerate}[(1)]
    	\item 如果$\mathscr{A}$被$\mathscr{A}_1$有限生成,那么$Y=\mathrm{Proj}\mathscr{A}$是有限型$X$概形.
    	\item 如果$\mathscr{M}$几乎为零,那么$\widetilde{M}=0$.
    	\item 设$\mathscr{A}$被$\mathscr{A}_1$有限生成,如果$\mathscr{M}$是几乎有限型的,那么$\widetilde{\mathscr{M}}$是有限生成$\mathscr{O}_Y$模层.
    	\item 设$\mathscr{A}$被$\mathscr{A}_1$有限生成,如果$\mathscr{M}$是几乎有限型的,那么$\widetilde{\mathscr{M}}=0$当且仅当$\mathscr{M}$是几乎为零的.
    \end{enumerate}
\end{enumerate}
\subsection{函子$\Gamma_*$}
函子$\Gamma_*:\textbf{Mod}(\mathscr{O}_Y)\to\textbf{GrMod}(\mathscr{O}_X)$.
\begin{enumerate}
	\item 设$\mathscr{A}$被$\mathscr{A}_1$生成,对任意$\mathscr{O}_Y$模层$\mathscr{F}$,记:
	$$\Gamma_*(\mathscr{F})=\oplus_{n\in\mathbb{Z}}\varphi_*\left(\mathscr{F}(n)\right)$$
	
	一般的对环空间态射$p:Y\to X$,以及两个$\mathscr{O}_Y$模层$\mathscr{F}$和$\mathscr{G}$,我们有如下典范态射:
	$$p_*\mathscr{F}\otimes_{\mathscr{O}_X}p_*\mathscr{G}\to p_*(\mathscr{F}\otimes_{\mathscr{O}_Y}\mathscr{G})$$
	
	于是特别的这里有典范同态:
	$$\varphi_*(\mathscr{O}_Y(n))\otimes_{\mathscr{O}_Y}\varphi_*(\mathscr{O}_Y(m))\to \varphi_*(\mathscr{O}_Y(m+n))$$
	
	于是$\Gamma_*(\mathscr{O}_Y)$是$\mathscr{O}_X$分次代数层结构,并且$\Gamma_*(\mathscr{F})$是$\Gamma_*(\mathscr{O}_Y)$分次模层.$\Gamma_*$是$\textbf{Mod}(\mathscr{O}_Y)\to\textbf{GrMod}(\mathscr{O}_X)$的共变加性左正合函子.特别的,对$\mathscr{O}_Y$理想层$\mathscr{I}$,有$\Gamma_*(\mathscr{I})$是$\Gamma_*(\mathscr{O}_Y)$的分次理想层.
	\item 自然变换$\alpha$.设$\mathscr{A}$被$\mathscr{A}_1$生成,设$\mathscr{M}$是拟凝聚$\mathscr{A}$分次模层,我们来构造一个函子性的分次阿贝尔群层之间的态射$\alpha:\mathscr{M}\to\Gamma_*(\widetilde{\mathscr{M}})$:对$X$的仿射开子集$U$,我们构造过一个阿贝尔群同态$\alpha_{0,U}:\Gamma(U,\mathscr{M}_0)\to\Gamma(\varphi^{-1}(U),\widetilde{\mathscr{M}})$,这些同态和限制映射可交换,于是定义了一个阿贝尔群层态射$\alpha_0:\mathscr{M}_0\to\varphi_*\widetilde{\mathscr{M}}$.把这个态射用在$\mathscr{M}_n=(\mathscr{M}(n))_0$上,得到阿贝尔群层态射$\alpha_n:\mathscr{M}_n\to\varphi_*(\widetilde{\mathscr{M}}(n))$.进而得到态射阿贝尔群层的$\alpha:\mathscr{M}\to\Gamma_*(\widetilde{\mathscr{M}})$.
	\begin{enumerate}[(a)]
		\item 特别的,取$\mathscr{M}=\mathscr{A}$可以验证$\alpha$是一个$\mathscr{O}_X$分次代数层之间的态射.并且上述$\alpha:\mathscr{M}\to\Gamma_*(\widetilde{\mathscr{M}})$是相较于这个分次代数层态射的分次模层态射.
		\item $\alpha_n$对应的$\mathscr{O}_Y$模层态射$\varphi^*\mathscr{M}_n\to\widetilde{\mathscr{M}}(n)$恰好就是如下$\mathscr{A}$模层态射$\mathscr{M}_n\otimes_{\mathscr{O}_X}\mathscr{A}\to\mathscr{M}(n)$取伴随模层态射.
		\item 设$\mathscr{A}$被$\mathscr{A}_1$生成,对正次齐次整体截面$f\in\Gamma(X,\mathscr{A}_d)$,有$Y_f$由$Y$的那些使得$\alpha_d(f)\in\Gamma(Y,\mathscr{O}_Y(d))$不取零的点构成的集合.
	\end{enumerate}
	\item 自然变换$\beta$.设$\mathscr{A}$被$\mathscr{A}_1$生成.如果$\mathscr{F}$是拟凝聚$\mathscr{O}_Y$模层,并且$\varphi$是有限型态射,那么我们构造的$\Gamma_*(\mathscr{F})$是拟凝聚$\mathscr{O}_X$模层.于是$\widetilde{\Gamma_*(\mathscr{F})}$是拟凝聚$\mathscr{O}_Y$模层.对$X$的仿射开子集$U$,我们可以构造$\beta_U:\widetilde{\Gamma(U,\Gamma_*(\mathscr{F}))}\to\mathscr{F}\mid_{\varphi^{-1}(U)}$,这和限制映射可交换,于是得到一个拟凝聚$\mathscr{O}_Y$模层之间的态射$\beta:\widetilde{\Gamma_*(\mathscr{F})}\to\mathscr{F}$.
	\item $\mathscr{A}$被$\mathscr{A}_1$生成,设$\varphi$是有限型态射,设$\mathscr{F}$是拟凝聚$\mathscr{O}_X$模层,那么如下复合都是恒等态射:
	$$\xymatrix{\widetilde{\mathscr{M}}\ar[r]^{\widetilde{\alpha}}&\widetilde{\Gamma_*(\widetilde{\mathscr{M}})}\ar[r]^{\beta}&\widetilde{\mathscr{M}}}$$
	$$\xymatrix{\Gamma_*(\mathscr{F})\ar[r]^{\alpha}&\Gamma_*(\widetilde{\Gamma_*(\mathscr{F})})\ar[r]^{\Gamma_*(\beta)}&\Gamma_*(\mathscr{F})}$$
	\item 如果$\mathscr{N}$另一个拟凝聚$\mathscr{A}$分次模层,一个零次态射$u:\mathscr{M}\to\mathscr{N}$称为几乎满射/单射/双射,如果存在整数$N$使得$n\ge N$时$u_n:\mathscr{M}_n\to\mathscr{N}_n$都是满射/单射/双射.那么此时$\widetilde{u}$是满射/单射/双射.
	\item 如果$\mathscr{A}$被$\mathscr{A}_1$有限生成,那么对任意拟凝聚$\mathscr{O}_Y$模层$\mathscr{F}$,有$\beta:\widetilde{\Gamma_*(\mathscr{F})}\to\mathscr{F}$是同构.
	\item 推论.如果$\mathscr{A}$被$\mathscr{A}_1$有限生成,那么任意拟凝聚$\mathscr{O}_Y$模层都是某个拟凝聚$\mathscr{A}$分次模层的伴随模层;如果$X$是拟紧分离概形,那么任意有限生成的拟凝聚$\mathscr{O}_Y$模层都是某个有限生成的拟凝聚$\mathscr{A}$分次模层的伴随模层.
\end{enumerate}

\subsection{终端是齐次谱的态射}

设$Y$是概形,设$\mathscr{A}$是拟凝聚$\mathscr{O}_Y$分次代数层,我们期望构造任意概形$X$到$\mathrm{Proj}\mathscr{A}$的态射.为此任取概形态射$q:X\to Y$,任取$X$上可逆层$\mathscr{L}$,任取分次代数层态射$\psi:q^*\mathscr{A}\to\oplus_{n\ge0}\mathscr{L}^{\otimes n}$.我们会构造一个$Y$态射$r_{\mathscr{L},\psi}:G(\psi)\to\mathrm{Proj}\mathscr{A}$.然后给出它能定义在整个$X$上的准则,它是支配态射的准确,它是嵌入的准则.
\begin{enumerate}
	\item $\psi:q^*\mathscr{A}\to\oplus_{n\ge0}\mathscr{L}^{\otimes n}$按照伴随性等价于一个$\mathscr{O}_Y$分次代数层态射$\psi^{\mathrm{b}}:\mathscr{A}\to q_*\oplus_{n\ge0}\mathscr{L}^{\otimes n}$.这里$\mathrm{Proj}\oplus_{n\ge0}\mathscr{L}^{\otimes n}$典范等同于$X$,于是$\psi$诱导了$Y$态射$r_{\mathscr{L},\psi}:G(\psi)\to\mathrm{Proj}\mathscr{A}$,它称为$\mathscr{L}$和$\psi$的伴随态射.这个态射也是如下复合态射:
	$$\xymatrix{G(\psi)\ar[rr]^{\mathrm{Proj}(\psi)\qquad\quad}&&\mathrm{Proj}(q^*\mathscr{A})=\mathrm{Proj}(\mathscr{A})\times_YX\ar[rr]^{\qquad\pi_1}&&\mathrm{Proj}(\mathscr{A})}$$
	\item 设$Y=\mathrm{Spec}A$是仿射的,记$\mathscr{A}=\widetilde{S}$,其中$S$是分次$A$代数.对任意$f\in S_d$,都有$r_{\mathscr{L},\psi}^{-1}(D_+(f))=X_{\psi^{\mathrm{b}}(f)}$,其中$\psi^{\mathrm{b}}(f)\in\Gamma(X,\mathscr{L}^{\otimes d})$.并且态射$r_{\mathscr{L},\psi}$限制为$X_{\psi^{\mathrm{b}}(f)}\to\mathrm{Spec}S_{(f)}$对应于代数同态$\psi^{\mathrm{b}}_{(f)}:S_{(f)}\to\Gamma(X_{\psi^{\mathrm{b}}(f)},\mathscr{O}_X)$,即对任意$s\in S_{nd}$有:
	$$s/f^n\mapsto\left(\psi^{\mathrm{b}}(s)\mid_{X_{\psi^{\mathrm{b}}(f)}}\right)\left(\psi^{\mathrm{b}}(f)\mid_{X_{\psi^{\mathrm{b}}(f)}}\right)^{-n}$$
	\item 设$Y=\mathrm{Spec}A$是仿射的,记$\mathscr{A}=\widetilde{S}$,其中$S$是分次$A$代数.那么$r_{\mathscr{L},\psi}$在整个$X$上有定义当且仅当这些$X_{\psi^{\mathrm{b}}(f)}$覆盖了整个$X$,此即对任意$x\in X$存在正整数$n$和$f\in S_n$使得$t=\psi^{\mathrm{b}}(s)\in\Gamma(X,\mathscr{L}^{\otimes n})$在$x$处不取零.特别的如果$\psi$是几乎满射,那么$r_{\mathscr{L},\psi}$在整个$X$上有定义.
	\item 设$Y=\mathrm{Spec}A$是仿射的,记$\mathscr{A}=\widetilde{S}$,其中$S$是分次$A$代数.那么$r_{\mathscr{L},\psi}$是支配的当且仅当对任意正整数$n$和任意截面$s\in S_n$,只要$\psi^{\mathrm{b}}(s)\in\Gamma(X,\mathscr{L}^{\otimes n})$是局部幂零的,那么$s$是幂零的.
	\item 设$u:\mathscr{A}'\to\mathscr{A}$是拟凝聚$\mathscr{O}_Y$分次代数层之间的分次态射,设$\psi:q^*\mathscr{A}\to\oplus_{n\ge0}\mathscr{L}^{\otimes n}$是$\mathscr{O}_X$分次代数层之间的分次态射,记$\psi'=\psi\circ q^*(u)$.
	\begin{enumerate}[(1)]
		\item 如果$r_{\mathscr{L},\psi'}$定义在整个$X$上,那么$r_{\mathscr{L},\psi}$也定义在整个$X$上.
		\item 如果$u$是几乎满射,$r_{\mathscr{L},\psi'}$是支配态射,那么$r_{\mathscr{L},\psi}$也是支配态射.
		\item 如果$u$是几乎单射,并且$r_{\mathscr{L},\psi}$是支配态射,那么$r_{\mathscr{L},\psi'}$也是支配态射.
	\end{enumerate}
	\begin{proof}
		
		(1)是因为$G(\psi')\subseteq G(\psi)$.
		
		\qquad
		
		(2):如果$u$是几乎满射,那么$\mathrm{Proj}(u):\mathrm{Proj}\mathscr{A}\to\mathrm{Proj}\mathscr{A}'$定义在整个$X$上,并且是一个闭嵌入.于是明显有$\mathrm{Proj}(u)\circ r_{\mathscr{L},\psi}\mid_{G(\psi')}=r_{\mathscr{L},\psi'}$.于是$r_{\mathscr{L},\psi'}$是支配态射得到$r_{\mathscr{L},\psi}$也是支配态射.
		
		\qquad
		
		(3):如果$u$是几乎单射,那么$\mathrm{Proj}(u):G(u)\to\mathrm{Proj}\mathscr{A}'$是支配的.但是$G(\psi')$是$G(u)$在$r_{\mathscr{L},\psi}$下的原像,于是从$r_{\mathscr{L},\psi}$是支配态射得到$r_{\mathscr{L},\psi'}$也是支配态射.
	\end{proof}
    \item 定义在整个$X$上和归纳系统.设$Y$是拟紧概形,$q:X\to Y$是拟紧态射,$\mathscr{L}$是可逆$\mathscr{O}_X$模层,$\mathscr{A}$是拟凝聚$\mathscr{O}_Y$分次代数层,并且是某个拟凝聚$\mathscr{O}_Y$分次代数层归纳系统$\{\mathscr{A}^i\}$的极限.设$\varphi_i:\mathscr{A}^i\to\mathscr{A}$是典范态射,设$\psi:q^*\mathscr{A}\to\oplus_{n\ge0}\mathscr{L}^{\otimes n}$是$\mathscr{O}_X$分次代数层之间的态射,记$\psi_i=\psi\circ q^*(\varphi_i)$.
    \begin{enumerate}[(1)]
    	\item $r_{\mathscr{L},\psi}$定义在整个$X$上当且仅当存在指标$i$使得$r_{\mathscr{L},\psi_i}$定义在整个$X$上.此时对任意指标$j\ge i$都有$r_{\mathscr{L},\psi_j}$定义在整个$X$上.
    	\item 如果$\varphi_i$都是单射,那么从$r_{\mathscr{L},\psi}$是支配的得到每个$r_{\mathscr{L},\psi_i}$都是支配的.
    	\item 如果所有$r_{\mathscr{L},\psi_i}$都是支配的,那么$r_{\mathscr{L},\psi}$是支配的.
    \end{enumerate}
    \begin{proof}
    	
    	(1):充分性按照上一条得证.对于必要性,设$r_{\mathscr{L},\psi}$定义在整个$X$上.我们先解释下问题归结为$Y$是仿射的,因为一旦这成立,对任意仿射开子集$U\subseteq Y$,就能找到一个指标$i(U)$使得$r_{\mathscr{L},\psi_{i(U)}}$定义在整个$q^{-1}(U)$上.于是只要让$U$跑遍$Y$的一个有限仿射开覆盖,取对应指标的最大值即可.
    	
    	\qquad
    	
    	现在设$Y$是仿射的,对任意$x\in X$,我们可以找到某个$s^{(x)}\in\Gamma(Y,\mathscr{A}_n)$使得$t^{(x)}=\psi^{\mathrm{b}}(s^{(x)})$在$x$处不取零.于是存在$x$的开邻域$V_x$使得其中任意点$z$有$t^{(x)}(z)\not=0$(其中$t^{(x)}$视为$\mathscr{L}^{\otimes n}$的整体截面,$t^{(x)}(z)$是这个整体截面在$z$处的芽).取$\{V(x)\}$的有限开覆盖$\{V(x_1),\cdots,V(x_r)\}$,于是我们只要取一个足够大的指标$t$就保证这些$s^{(x_i)}$落在$\varphi_t(\mathscr{A}^t)$中,于是此时$r_{\mathscr{L},\psi_t}$满足定义在整个$X$上的终端是仿射概形的那个等价描述.
    	
    	\qquad
    	
    	(2)是上一条的特殊情况.(3):支配是一个终端局部性质,所以归结为设$Y$是仿射的.此时我们给出过等价描述.如果齐次元$s\in S=\Gamma(Y,\mathscr{A})$是局部幂零的,存在某个指标$i$使得$s=\varphi_i(s_i)$,所以$s_i$是幂零的,从而$s$是幂零的.
    \end{proof}
    \item 嵌入准则.设$Y$是仿射的.那么$r_{\mathscr{L},\psi}$定义在整个$X$上的嵌入当且仅当可以找到一族正次齐次整体截面$s_{\alpha}\in S_{n_{\alpha}}=\Gamma(Y,\mathscr{A}_{n_{\alpha}})$使得$f_{\alpha}=\psi^{\mathrm{b}}(s_{\alpha})$满足如下条件:
    \begin{enumerate}[(1)]
    	\item $\{X_{f_{\alpha}}\}$覆盖$X$.
    	\item 每个$X_{f_{\alpha}}$都是仿射开子集.
    	\item 对任意指标$\alpha$和任意$t\in\Gamma(X_{f_{\alpha}},\mathscr{O}_X)$,都可以找到一个正整数$m$和和一个$s\in S_{mn_{\alpha}}$,使得:
    	$$t=\left(\psi^{\mathrm{b}}(s)\mid_{X_{f_{\alpha}}}\right)\left(\psi^{\mathrm{b}}(s)\mid_{X_{f_{\alpha}}}\right)\left(f_{\alpha}\mid_{X_{f_{\alpha}}}\right)^{-m}$$
    \end{enumerate}
    \item 开嵌入准则.设$Y$是仿射的.那么$r_{\mathscr{L},\psi}$定义在整个$X$上的开嵌入当且仅当可以找到一族正次齐次整体截面$s_{\alpha}\in S_{n_{\alpha}}=\Gamma(Y,\mathscr{A}_{n_{\alpha}})$使得$f_{\alpha}=\psi^{\mathrm{b}}(s_{\alpha})$满足如下条件:
    \begin{enumerate}[(1)]
    	\item $\{X_{f_{\alpha}}\}$覆盖$X$.
    	\item 每个$X_{f_{\alpha}}$都是仿射开子集.
    	\item 对任意指标$\alpha$和任意$t\in\Gamma(X_{f_{\alpha}},\mathscr{O}_X)$,都可以找到一个正整数$m$和和一个$s\in S_{mn_{\alpha}}$,使得:
    	$$t=\left(\psi^{\mathrm{b}}(s)\mid_{X_{f_{\alpha}}}\right)\left(\psi^{\mathrm{b}}(s)\mid_{X_{f_{\alpha}}}\right)\left(f_{\alpha}\mid_{X_{f_{\alpha}}}\right)^{-m}$$
    	\item 对任意$n\ge1$和任意$s\in S_{nn_{\alpha}}$,只要$\psi^{\mathrm{b}}(s)\mid_{X_{f_{\alpha}}}=0$,就可以找到一个正整数$k$使得$s_{\alpha}^ks=0$.
    \end{enumerate}
    \item 闭嵌入准则.设$Y$是仿射的.那么$r_{\mathscr{L},\psi}$定义在整个$X$上的闭嵌入当且仅当可以找到一族正次齐次整体截面$s_{\alpha}\in S_{n_{\alpha}}=\Gamma(Y,\mathscr{A}_{n_{\alpha}})$使得$f_{\alpha}=\psi^{\mathrm{b}}(s_{\alpha})$满足如下条件:
    \begin{enumerate}[(1)]
    	\item $\{X_{f_{\alpha}}\}$覆盖$X$.
    	\item 每个$X_{f_{\alpha}}$都是仿射开子集.
    	\item 对任意指标$\alpha$和任意$t\in\Gamma(X_{f_{\alpha}},\mathscr{O}_X)$,都可以找到一个正整数$m$和和一个$s\in S_{mn_{\alpha}}$,使得:
    	$$t=\left(\psi^{\mathrm{b}}(s)\mid_{X_{f_{\alpha}}}\right)\left(\psi^{\mathrm{b}}(s)\mid_{X_{f_{\alpha}}}\right)\left(f_{\alpha}\mid_{X_{f_{\alpha}}}\right)^{-m}$$
    	\item $\{D_+(s_{\alpha})\}$覆盖$\mathrm{Proj}S$.
    \end{enumerate}
    \item 推论.设$u:\mathscr{A}'\to\mathscr{A}$是拟凝聚$\mathscr{O}_Y$分次代数层之间的态射,设$\psi:q^*\mathscr{A}\to\oplus_{n\ge0}\mathscr{L}^{\otimes n}$是分次代数层态射,设$\psi'=\psi\circ q^*(u)$.
    \begin{enumerate}[(1)]
    	\item 如果$r_{\mathscr{L},\psi'}$是定义在整个$X$上的嵌入,那么$r_{\mathscr{L},\psi}$也是如此.
    	\item 如果$u$是几乎满射,那么从$r_{\mathscr{L},\psi'}$是开嵌入/闭嵌入可以推出$r_{\mathscr{L},\psi}$是开嵌入/闭嵌入.
    \end{enumerate}
    \item 设$Y$是拟紧概形,$q:X\to Y$是有限型态射,$\mathscr{L}$是可逆$\mathscr{O}_X$模层,$\mathscr{A}$是拟凝聚$\mathscr{O}_Y$分次代数层,并且是某个拟凝聚$\mathscr{O}_Y$分次代数层归纳系统$\{\mathscr{A}^i\}$的极限.设$\varphi_i:\mathscr{A}^i\to\mathscr{A}$是典范态射,设$\psi:q^*\mathscr{A}\to\oplus_{n\ge0}\mathscr{L}^{\otimes n}$是$\mathscr{O}_X$分次代数层之间的态射,记$\psi_i=\psi\circ q^*(\varphi_i)$.那么$r_{\mathscr{L},\psi}$是定义在整个$X$上的嵌入当且仅当存在指标$i$使得$r_{\mathscr{L},\psi_i}$是定义在整个$X$上的嵌入.此时对任意$j\ge i$都有$r_{\mathscr{L},\psi_j}$是定义在整个$X$上的嵌入.
    \item 设$Y$是拟紧拟可分概形,设$q:X\to Y$是有限型态射.那么$r_{\mathscr{L},\psi}$是定义在整个$X$上的嵌入当且仅当存在正整数$n$和$\mathscr{A}_n$的有限型拟凝聚$\mathscr{O}_Y$子模层$\mathscr{E}$,满足:
    \begin{enumerate}[(1)]
    	\item 记$j_n:\mathscr{E}\to\mathscr{A}$是典范包含态射,那么$\psi_n\circ q^*(j_n):q^*\mathscr{E}\to\mathscr{L}^{\otimes n}$是满射.
    	\item 记$\mathscr{E}$在$\mathscr{A}$中生成的子代数层为$\mathscr{A}'$,记$j':\mathscr{A}'\to\mathscr{A}$是典范包含态射,记$\psi'=\psi\circ q^*(j')$,那么$r_{\mathscr{L},\psi'}$是定义在整个$X$上的嵌入.
    \end{enumerate}

    如果这些条件成立,则$\mathscr{A}_n$的任何包含$\mathscr{E}$的拟凝聚$\mathscr{O}_Y$子模层$\mathscr{E}'$都具有相同的性质,并且对任意$k\ge1$有$\mathscr{E}^{\otimes k}$在$\mathscr{A}_{kn}$中的像也满足相同的性质.
\end{enumerate}
\subsection{正规化}
\begin{enumerate}
	\item 设$(X,\mathscr{O}_X)$是环空间,设$\mathscr{A}$是(交换)$\mathscr{O}_X$代数层,设$f\in\Gamma(X,\mathscr{A})$,那么如下条件互相等价,此时称$f$是$\mathscr{A}$的整截面.
	\begin{enumerate}[(1)]
		\item $\{f^n,n\ge0\}$在$\mathscr{A}$中生成的子模层(此为这些整体截面诱导的$\mathscr{O}_X^{\oplus\mathbb{N}}\to\mathscr{A}$的像层)是有限生成模层.
		\item 存在$\mathscr{A}$的有限生成子模层$\mathscr{B}$,使得$f\in\Gamma(X,\mathscr{B})$.
		\item 对任意$x\in X$,有$f_x$在$\mathscr{O}_{X,x}$上整.
	\end{enumerate}
	\item 设$(X,\mathscr{O}_X)$是环空间,设$\mathscr{A}$是(交换)$\mathscr{O}_X$代数层.取$\mathscr{A}$的由整截面构成的子模层$\mathscr{O}_X'$,满足对任意$x\in X$有$\mathscr{O}_{X,x}'$由那些在$\mathscr{O}_{X,x}$上整的$\mathscr{A}_x$中的元构成.并且对任意开集$U\subseteq X$有$\Gamma(U,\mathscr{O}_X')$由那些在$\Gamma(U,\mathscr{O}_X)$上整的$\Gamma(U,\mathscr{A})$中的元构成.称$\mathscr{O}_X'$为$\mathscr{O}_X$在$\mathscr{A}$中的正规化.
	\item 设$X$是概形,$\mathscr{A}$是拟凝聚$\mathscr{O}_X$代数层,那么$\mathscr{O}_X$在$\mathscr{A}$中的正规化$\mathscr{O}_X'$也是拟凝聚的.此时称典范态射$X'=\mathrm{Spec}\mathscr{O}_X'\to X$为$X$在$\mathscr{A}$中的正规化.特别的,仿射情况下这个概念和环的正规化一致.
	\begin{proof}
		
		问题归结为仿射情况,而仿射情况成立是因为正规化和分式化可交换.
	\end{proof}
	\item 明显的代数层的正规化和在开子集上的限制可交换.于是特别的如果$\pi:X'\to X$是概形$X$在$\mathscr{A}$中的正规化,对任意开子集$U\subseteq X$,有$\pi^{-1}(U)\to U$是$U$在$\mathscr{A}\mid_U$中的正规化.
	\item 设$g=(\psi,\theta):(X,\mathscr{O}_X)\to(Y,\mathscr{O}_Y)$是环空间之间的态射.设$\mathscr{A}$和$\mathscr{B}$分别是$\mathscr{O}_X$和$\mathscr{O}_Y$的代数层.设$u:\mathscr{B}\to\mathscr{A}$是模层之间的$g$态射.记$\mathscr{O}_X$和$\mathscr{O}_Y$分别在$\mathscr{A}$和$\mathscr{B}$中的正规化分别是$\mathscr{O}_X'$和$\mathscr{O}_Y'$.那么$u$限制在$\mathscr{O}_Y'$上是到$\mathscr{O}_X'$上的$g$态射.
	\begin{proof}
		
		归结为证明$v=u^{\#}\circ g^*(j):g^*\mathscr{O}_Y'\to\mathscr{B}$映入$\mathscr{O}_X'$.而这是因为$(g^*\mathscr{O}_Y')_x=\mathscr{O}_{Y,\psi(x)}'\otimes_{\mathscr{O}_{Y,\psi(x)}}\mathscr{O}_{X,x}$明显在$\mathscr{O}_{X,x}$上整,从而它在$v_x$下的像也在$\mathscr{O}_{X,x}$上整.
	\end{proof}
	\item 设$X$只有有限个不可约分支$\{X_1,\cdots,X_r\}$,设$X_i$的一般点为$\eta_i$.记$\mathscr{R}_X$是有理函数层(我们解释过这是拟凝聚的,并且在$X$不可约的时候这是常值层).设$\mathscr{A}$是拟凝聚$\mathscr{R}_X$代数层,我们来描述$X$在$\mathscr{A}$中的正规化.我们解释过$\mathscr{A}$是$r$个拟凝聚$\mathscr{O}_X$代数层$\mathscr{A}_i$的直和,其中$\mathscr{A}_i$的支集包含在$X_i$中,并且$\mathscr{A}_i$在$X_i$上的回拉是常值层,对应的环记作$A_i$,这是$\mathscr{O}_{X,\eta_i}$代数.那么$\mathscr{O}_X$在$\mathscr{A}$中的正规化就是$\mathscr{O}_X'=\oplus_{1\le i\le r}\mathscr{O}_X^{(i)}$,其中$\mathscr{O}_X^{(i)}$是$\mathscr{O}_X$在$\mathscr{A}_i$中的正规化.于是$X'=\mathrm{Spec}\mathscr{O}_X'$就是$\mathrm{Spec}\mathscr{O}_X^{(i)}$的无交并.
	\item 设$X$是只有有限个不可约分支的既约概形,依旧设不可约分支为$\{X_1,\cdots,X_r\}$,依旧记$X_i$的一般点为$\eta_i$.此时既约条件就是说$A_i$都是既约的,它是域$\kappa(\eta_i)$上的代数.进而$X_i'$是既约概形.按照$X_i'\to X$是仿射态射得到它是分离的.另外$X'$也是$X_{\mathrm{red}}$在$\mathscr{A}$中的正规化.此时$A_i$是有限个域$K_{ij},1\le j\le s_i$的直和.如果记$\mathscr{K}_{ij}$是$\mathscr{A}_i$的对应于$K_{ij}$的子代数层,那么$\mathscr{O}_X^{(i)}$就是$\mathscr{O}_X$在$\mathscr{K}_{ij}$中的正规化$\mathscr{O}_X^{(ij)}$的直和.于是$X_i'$就是$X_{ij}'=\mathrm{Spec}\mathscr{O}_X^{(ij)}$的无交并.我们断言每个$X_{ij}'$都是正规整概形,并且函数域$K(X_{ij}')$恰好就是$\kappa(\eta_i)$在$K_{ij}$中的代数闭包.
	\begin{proof}
		
		问题归结为设$X$是整概形,此时$r=1$,可以不妨设$s_1=1$,此时$A_1$是域,问题归结为交换代数.
	\end{proof}
	\item 设$X$是既约概形,只有有限个不可约分支$\{X_1,\cdots,X_r\}$,设$X_i$的一般点是$\eta_i$.那么$X$在有理函数层$\mathscr{R}_X$中的正规化$X'$是$r$个正规整分离概形$X_i'$的无交并.记结构态射$f:X'\to X$,那么$f^{-1}(\eta_i)$只有一个点,就是$X_i'$的一般点$\eta_i'$,并且有$\kappa(\eta_i)=\kappa(\eta_i')$.
	\item 整概形的情况.设$X$是整概形,函数域记作$K$,取$K$的代数扩张$L$,定义$X$在$L$中的正规化就是在$L$诱导的常值层中的正规化$\pi:X'\to X$.它满足如下泛映射性质:$\pi:X'\to X$是一个源端为正规整概型的整支配态射,满足$K(X')=L$,并且对任意函数域为$L$的正规整概型$Y$和支配态射$Y\to X$,存在唯一的态射$Y\to\widetilde{X}$使得如下图表交换:
	$$\xymatrix{Y\ar[rr]\ar[dr]&&X'\ar[dl]\\&X&}$$
	\item 正规化的有限性定理.
	\begin{enumerate}
		\item 设$X$是局部诺特正规整概形,它的函数域记作$K$,取有限可分扩张$K\subseteq L$,那么$X$在这个扩张下的正规化$\widetilde{X}\to X$是有限态射.换句话讲,如果$A$是诺特正规整环,商域记作$K$,取有限可分扩张$K\subseteq L$,记$A$在$L$中的正规化为$B$,那么$B$是有限$A$模.
		\begin{proof}
			
			按照$K\subseteq L$是有限可分扩张,于是二次型$\langle-,-\rangle:L\times L\to K$,$\langle x,y\rangle=\mathrm{Tr}_{L/K}(xy)$是非退化的.记$[L:K]=n$,取$L$在$K$上的一组基,并且我们总可以乘以一个合适的$A-\{0\}$中的元使得这组基落在$B$中,把它们设为$b_1,\cdots,b_n$.那么按照这个二次型是非退化的,就可以设它的对偶基为$b_1^*,\cdots,b_n^*\in L$.按照$A$是整闭的,并且$B$在$A$上是整的,导致$B$中元满足的首一极小多项式的系数一定在$A$中,导致$\mathrm{Tr}_{L/K}(B)\subseteq A$.于是任取$b\in B$,取$\langle b,b_i\rangle=a_i$,那么$\langle b-\sum_ja_jb_j^*,b_i\rangle=0$,于是非退化迫使$b=\sum_ja_jb_j^*$,于是$B\subseteq\oplus_jAb_j^*$,但是这里$A$是诺特环,于是$B$作为有限$A$模的子模还是有限生成的.
		\end{proof}
		\item 设$X$是有限型整$k$概形,商域记作$K$,取有限扩张$K\subseteq L$,那么$X$在这个扩张下的正规化$\widetilde{X}\to X$是有限态射.换句话讲,如果$A$是有限型$k$整环,商域记作$K$,取有限扩张$K\subseteq L$,记$A$在$L$中的正规化为$B$,那么$B$是有限$A$模.
		\begin{proof}
			
			问题是仿射局部的,设$X=\mathrm{Spec}A$,那么诺特正规化引理说明存在有限单射$k[T_1,\cdots,T_n]\to A$,于是$A$在$L$中的整闭包也就是$k[T_1,\cdots,T_n]$在$L$中的整闭包.换句话讲归结为$A=k[T_1,\cdots,T_n]$,$K=k(T_1,\cdots,T_n)$的情况.【】
		\end{proof}
	    \item 设$X$是既约概形,是域或者$\mathbb{Z}$或者完备诺特局部环上的有限型分离概形,那么$X$在$\mathscr{R}_X$凝聚代数层$\mathscr{A}$中的正规化是有限态射(这件事是因为这种环在每个极小素理想处的商在自身商域的有限扩张中的正规化总是有限的).
	\end{enumerate}
	\item 设$X$是整概形,设正规化态射$\pi:\widetilde{X}\to X$是有限态射,那么$X$的所有正规点(此即局部环是正规整环的点)构成了一个开子集.于是特别的,对于域上的有限型整概形,它的所有正规点构成开集.
	\begin{proof}
		
		归结为仿射的情况.记$X=\mathrm{Spec}A$,其中$A$是整环,记$A$在商域中的整闭包是$A'$.按照整闭包和分式化可交换,任取$\mathfrak{p}\in\mathrm{Spec}A$,那么$A_{\mathfrak{p}}$的整闭包是$A'\otimes_AA_{\mathfrak{p}}$.于是$\mathfrak{p}$不是正规点等价于讲$(A'/A)\otimes_AA_{\mathfrak{p}}=0$,也即$\mathfrak{p}$落在$A'/A$的支集中.但是由于$A'/A$是有限$A$模,导致$A'/A$的支集就是$V(\mathrm{Ann}(A/A'))$,这是闭子集,于是补集是开集.
	\end{proof}
	\item 设$A$是戴德金整环,商域是$K$,取有限扩张$K\subseteq L$,设$B$是$A$在$L$中的整闭包,那么$B$也是戴德金整环,并且典范包含映射诱导的态射$f:\mathrm{Spec}B\to\mathrm{Spec}A$具有有限纤维(此即戴德金整环的有限扩张上素理想的提升只有有限个).
\end{enumerate}
\subsection{闭子概型和拟凝聚理想层}

设$Y$是$X$的闭子概型,即存在闭嵌入$i:Y\to X$,定义$Y$的理想层$\mathscr{I}_Y$是诱导的层态射$\mathscr{O}_X\to i_*\mathscr{O}_Y$的核.
\begin{enumerate}
	\item 引理.设$F$是$X$上的阿贝尔层,定义支集$\mathrm{Supp}F=\{x\in X\mid F_x\not=0\}$,这个集合一般不是闭集.设$M$是有限$A$模,那么$\mathrm{Spec}A$上的伴随模层$\widetilde{M}$的支集就是$V(\mathrm{Ann}_A(M))$.
	\begin{proof}
		
		一方面设某个素理想$p$使得$M_p=0$.记$M$被$\{x_1,\cdots,x_n\}$生成,那么每个$x_i/1=0$得到$s_i\not\in p$使得$s_ix_i=0$.取$s=s_1\cdots s_n$,那么$sx_i=0,\forall i$成立.于是$s\in\mathrm{Ann}_A(M)$.于是$\mathrm{Ann}_A(M)\not\subseteq p$.
		
		\qquad
		
		反过来设某个素理想$p$满足$\mathrm{Ann}_A(M)\not\subseteq p$,那么存在$s\in\mathrm{Ann}_A(M)$使得$s\not\in p$.于是对任意$x/t\in M_p$都有$x/t=sx/st=0$,于是$M_p=0$.
	\end{proof}
	\item 如果$X$是概形,$F$是拟凝聚层,并且$F$是有限生成的,那么$\mathrm{Supp}F$是$X$的闭子集.另外这个结论事实上对任意环空间上的有限生成模层都成立.
	\begin{proof}
		
		取$X$的仿射开覆盖$\{U_i=\mathrm{Spec}A_i\}$,那么可记$F\mid U_i=\widetilde{M_i}$,其中$M_i$是$A_i$有限模.于是引理说明$(\mathrm{Supp}F)\cap U_i=V(\mathrm{Ann}_{A_i}(M_i))$是$U_i$中的闭子集,于是$\mathrm{Supp}F$是闭集.
	\end{proof}
	\item 设$X$是概型,设$Y$是闭子概型,那么理想层$\mathscr{I}_Y$是拟凝聚理想层,如果$X$是局部诺特的那么它是凝聚层.反过来,任取$X$上的拟凝聚理想层$\mathscr{I}$则唯一的决定了$X$的一个闭子概型$(Y,\mathscr{O}_X/\mathscr{I})$,其中$Y$是$\mathscr{O}_X/\mathscr{I}$的支集.于是特别的,仿射概型$\mathrm{Spec}A$上的理想层恰好就是$A$的理想,于是仿射概型的闭子概型恰好一一对应于$A$的理想.
	\begin{proof}
		
		设$Y$是闭子概型,$\mathscr{O}_Y$本身是$Y$上的拟凝聚层,包含映射$j:Y\to X$是拟紧拟分离的,我们解释过这个条件下有$j_*\mathscr{O}_Y$是$X$上的拟凝聚层,于是理想层$\mathscr{I}_Y$是$X$上两个拟凝聚层之间态射的核,于是它是拟凝聚层.在诺特条件下诺特环的理想自然是有限生成的,于是此时它是凝聚层.
		
		\qquad
		
		反过来设$\mathscr{I}$是概型$X$上的一个拟凝聚理想层.取商模层$\mathscr{O}_X/\mathscr{I}$的支集$Y$,我们证明$(Y,\mathscr{O}_Y=\mathscr{O}_X/\mathscr{I})$是理想层为$\mathscr{I}$的闭子概型.问题是局部的,不妨设$X$本身是仿射的,记作$\mathrm{Spec}A$,那么$\mathscr{I}$可表示为$\widetilde{I}$,其中$I\subseteq A$是理想.那么$\mathscr{O}_X/\mathscr{I}=\widetilde{A/I}$,于是$\mathrm{Supp}(\mathscr{O}_X/\mathscr{I})=V(I)$.于是$(Y,\mathscr{O}_Y)$同构于闭子概型$(\mathrm{Spec}A/I,\mathscr{O}_{A/I})$.
		
		\qquad
		
		最后验证这两种对应互为逆映射.如果$\mathscr{I}$是$\mathscr{O}_X$的拟凝聚层理想层,设$\mathscr{O}_X/\mathscr{I}$的支集为$Y$,那么$\mathscr{O}_X$本身自然是$\mathscr{O}_X\to\mathscr{O}_X/\mathscr{I}$的核.反过来如果$(Y,\mathscr{O}_Y)$是$X$的一个闭子概型,设$j:Y\to X$是典范包含态射.那么如果$x\not\in j(Y)$则$(j_*\mathscr{O}_Y)_x=0$,如果$x=f(y)$则$(j_*\mathscr{O}_Y)_x=\mathscr{O}_{Y,y}$.于是考虑短正合列$0\to\mathscr{I}\to\mathscr{O}_X\to j_*\mathscr{O}_X\to0$.如果记$j^*\mathscr{O}_X$的支集为$Z$,那么$x\in Z$当且仅当$\mathscr{I}_x\not\mathscr{O}_{X,x}$,当且仅当$x\in f(Y)$.这说明我们有典范同构$(Y,\mathscr{O}_Y)\to(Z,\mathscr{O}_X/\mathscr{I})$,于是它们作为闭子概型是相同的.
	\end{proof}
\end{enumerate}
\newpage
\section{向量丛和射影丛}
\subsection{对称代数层}
\begin{enumerate}
	\item 模层的对称代数层.设$(X,\mathscr{O}_X)$是环空间,设$\mathscr{F}$是一个$\mathscr{O}_X$模层,定义预层$U\mapsto\mathrm{Sym}_{\mathscr{O}_X(U)}(\mathscr{F}(U))$(记号$\mathrm{Sym}_A(M)$表示$A$模$M$对应的对称代数,也即$M$的张量代数$\mathrm{T}(M)$模去由全体$m_1\otimes m_2-m_2\otimes m_1$生成的双边理想)的层化为$\mathscr{F}$的对称代数层,记作$\mathrm{Sym}(\mathscr{F})$.
	\item $\mathscr{F}\mapsto\mathrm{Sym}(\mathscr{F})$是从$\mathscr{O}_X$模层范畴到$\mathscr{O}_X$交换代数层的函子,按照对称代数的泛性质,这个函子左伴随于遗忘函子,换句话讲对任意$\mathscr{O}_X$交换代数层$\mathscr{A}$,我们有如下自然同构:
	$$\mathrm{Hom}_{\textbf{Alg}(\mathscr{O}_X)}(\mathrm{Sym}(\mathscr{F}),\mathscr{A})\cong\mathrm{Hom}_{\mathscr{O}_X}(\mathscr{F},\mathscr{A})$$
	
	换句话讲$\mathrm{Sym}(\mathscr{F})$具有如下泛性质:对任意$\mathscr{O}_X$交换代数层$\mathscr{B}$,和任意的$\mathscr{O}_X$模层态射$\mathscr{F}\to\mathscr{B}$,都存在唯一的$\mathscr{O}_X$代数层态射$\mathrm{Sym}(\mathscr{F})$使得预先给定的态射要经这个态射分解.
	$$\xymatrix{\mathrm{Sym}(\mathscr{F})\ar@{-->}[rr]^{\exists_!}&&\mathscr{B}\\\mathscr{F}\ar[u]\ar@/_1pc/[urr]&&}$$
	\item 设$M$是$A$模,设$B$是$A$代数,那么我们知道对称代数和张量积可交换,即$\mathrm{Sym}_A(M)\otimes_AB\cong\mathrm{Sym}_B(M\otimes_AB)$.特别的,这说明对任意$f\in A$有$\mathrm{Sym}_A(M)_f\cong\mathrm{Sym}_{A_f}(M_f)$.这件事说明:
	\begin{enumerate}
		\item 如果$\mathscr{F}=\widetilde{M}$是$X=\mathrm{Spec}A$上的拟凝聚层,那么对应的对称代数层$\mathrm{Sym}(\mathscr{F})=\widetilde{\mathrm{Sym}_A(M)}$.
		\item 如果$\mathscr{F}$是概形$X$上的拟凝聚层,那么$\mathrm{Sym}(\mathscr{F})$是$X$上的拟凝聚代数层.
		\item 如果$\mathscr{F}$是概形$X$上的拟凝聚层,尽管$\mathrm{Sym}(\mathscr{F})$的定义是一个预层的层化,但是在仿射开子集上层化和预层是一致的.
	\end{enumerate}
	\item 如果$\mathscr{F}$和$\mathscr{G}$是两个$\mathscr{O}_X$模层,那么有典范同构$\mathrm{Sym}(\mathscr{F}\oplus\mathscr{G})\cong\mathrm{Sym}(\mathscr{F})\otimes_{\mathscr{O}_X}\mathrm{Sym}(\mathscr{F})$.这件事是因为预层上有相应的性质.
	\item 如果$f:X\to Y$是概形之间的态射(事实上环空间之间态射也成立),如果$\mathscr{F}$是拟凝聚$\mathscr{O}_X$模层,那么有:
	$$f^*\mathrm{Sym}(\mathscr{F})\cong\mathrm{Sym}(f^*\mathscr{F})$$
	\begin{proof}
		\begin{align*}
			\mathrm{Hom}_{\textbf{Alg}(\mathscr{O}_X)}(f^*\mathrm{Sym}(\mathscr{F}),\mathscr{A})&=\mathrm{Hom}_{\textbf{Alg}(\mathscr{O}_Y)}(\mathrm{Sym}(\mathscr{F}),f_*\mathscr{A})\\&=\mathrm{Hom}_{\mathscr{O}_Y}(\mathscr{F},f_*\mathscr{A})\\&=\mathrm{Hom}_{\mathscr{O}_X}(f^*\mathscr{F},\mathscr{A})\\&=\mathrm{Hom}_{\textbf{Alg}(\mathscr{O}_X)}(\mathrm{Sym}(f^*\mathscr{F}),\mathscr{A})
		\end{align*}
	\end{proof}
\end{enumerate}
\subsection{向量丛}

设$X$是概形,设$\mathscr{F}$是拟凝聚$\mathscr{O}_X$模层,我们称$X$概形$\mathrm{Spec}(\mathrm{Sym}(\mathscr{F}))$是$\mathscr{F}$诱导的$X$上的向量丛,记作$\mathbb{V}(\mathscr{F})$.
\begin{enumerate}
	\item $\mathscr{F}\mapsto\mathbb{V}(\mathscr{F})$是$\textbf{QCoh}(\mathscr{O}_X)\to\textbf{AffSch}(X)$的逆变函子.例如取$X=\mathrm{Spec}A$是仿射的,取自由模层$\mathscr{F}=\mathscr{O}_X^n$,则$\mathrm{Sym}(\mathscr{F})$就是$X$上的代数层$\widetilde{A[T_1,\cdots,T_n]}$,进而有向量丛$\mathbb{V}(\mathscr{F})=\mathrm{Spec}A[T_1,\cdots,T_n]=\mathbb{A}_A^n$.
	\item 一些基本性质.
	\begin{enumerate}
		\item 函子$\mathbb{V}$是忠实函子,换句话讲给定两个拟凝聚$\mathscr{O}_X$模层$\mathscr{F},\mathscr{G}$,那么有如下典范单射:
		$$\mathrm{Hom}_{\mathscr{O}_X}(\mathscr{F},\mathscr{G})\to\mathrm{Hom}_X(\mathbb{V}(\mathscr{G}),\mathbb{V}(\mathscr{F}))$$
		
		我们称这种能被$\mathscr{O}_X$模层态射$\mathscr{F}\to\mathscr{G}$诱导的$X$概形态射$\mathbb{V}(\mathscr{G})\to\mathbb{V}(\mathscr{F})$是线性的态射.
		\item 如果$\mathscr{F}$是有限型$\mathscr{O}_S$模层,那么$\mathrm{Sym}(\mathscr{F})$是有限型$\mathscr{O}_S$代数层,进而有$\mathbb{V}(\mathscr{F})$在$S$上是有限型的.
		\item 设$g:X'\to X$是概形之间的态射,设$\mathscr{F}$是拟凝聚$\mathscr{O}_X$模层,那么有$\mathbb{V}(g^*\mathscr{F})\cong\mathbb{V}(\mathscr{F})\times_XX'$,特别的,取$g$是开嵌入得到,如果$U\subseteq X$是开子概型,那么$\mathbb{V}(\mathscr{F}\mid_U)=\mathbb{V}(\mathscr{F})\mid_U$.
		\begin{proof}
			\begin{align*}
				\mathbb{V}(g^*\mathscr{F})&=\mathrm{Spec}(\mathrm{Sym}(g^*\mathscr{F}))\\&=\mathrm{Spec}(g^*\mathrm{Sym}(\mathscr{F}))\\&=\mathrm{Spec}(\mathrm{Sym}(\mathscr{F}))\times_XX'\\&=\mathbb{V}(\mathscr{F})\times_XX'
			\end{align*}
		\end{proof}
		\item 如果有拟凝聚$\mathscr{O}_S$模层的满态射$\mathscr{F}\to\mathscr{G}$,那么诱导的$\mathrm{Sym}(\mathscr{F})\to\mathrm{Sym}(\mathscr{G})$也是满态射,进而诱导的$\mathbb{V}(\mathscr{G})\to\mathbb{V}(\mathscr{F})$是闭嵌入.
		\item 如果$\mathscr{F}$和$\mathscr{G}$都是拟凝聚$\mathscr{O}_S$模层,那么有:
		$$\mathbb{V}(\mathscr{F}\oplus\mathscr{G})\cong\mathbb{V}(\mathscr{F})\times_S\mathbb{V}(\mathscr{G})$$
	\end{enumerate}
	\item 按照拟凝聚代数层的素谱的可表性,以及对称代数层的泛性质,对任意$S$概形$X$,对任意拟凝聚$\mathscr{O}_S$模层$\mathscr{F}$,我们有如下典范一一对应:
	\begin{align*}
		\mathrm{Hom}_S(X,\mathbb{V}(\mathscr{F}))&\cong\mathrm{Hom}_{\textbf{Alg}(\mathscr{O}_S)}(\mathrm{Sym}(\mathscr{F}),\mathscr{A}(X))\\&\cong\mathrm{Hom}_{\mathscr{O}_S}(\mathscr{F},\mathscr{A}(X))
	\end{align*}
	\item 如果$X$是$S$概形,我们称$X$的$S$截面芽层指的是$S$上的一个层,它在开集$U\subseteq S$上取的是$\mathrm{Hom}_S(U,X)$,换句话讲,是全体$S$态射$U\to X$,使得复合$U\to X\to S$就是典范的开嵌入$U\to S$.下面设$\mathscr{F}$是$S$上的拟凝聚层,我们断言$\mathbb{V}(\mathscr{F})$的$S$截面芽层就是$\mathscr{F}^{\vee}=\mathrm{HOM}_{\mathscr{O}_S}(\mathscr{F},\mathscr{O}_S)$.
	\begin{proof}
		
		取$j:U\to S$是典范开嵌入,按照上一条,结合$j^*$和$j_*$是伴随的,我们有:
		\begin{align*}
			\mathrm{Hom}_S(U,\mathbb{V}(\mathscr{F}))&\cong\mathrm{Hom}_{\mathscr{O}_S}(\mathscr{F},j_*\mathscr{O}_U)\\&\cong\mathrm{Hom}_{\mathscr{O}_U}(j^*\mathscr{F},\mathscr{O}_U)\\&\cong\mathrm{Hom}_{\mathscr{O}_U}(\mathscr{F}\mid_U,\mathscr{O}_U)\\&\cong\mathrm{HOM}_{\mathscr{O}_S}(\mathscr{F},\mathscr{O}_S)(U)
		\end{align*}
		
		并且一个$S$态射$U\to\mathbb{V}(\mathscr{F})$在更小的开集$U'\subseteq U$上的限制就对应于模层态射$\mathscr{F}\mid_U\to\mathscr{O}_S\mid_U$在$U'$上的限制,这说明截面芽层和对偶模层是典范同构的.
	\end{proof}
	\item 设$S$是概形,设$\mathscr{F}$是拟凝聚$\mathscr{O}_S$模层,那么对任意$S$概形$h:X\to S$,我们有如下自然双射,换句话讲$\mathbb{V}(\mathscr{F})$是函子$X\mapsto\Gamma(X,(h^*\mathscr{F})^{\vee})$的表示对象.
	$$\mathrm{Hom}_S(X,\mathbb{V}(\mathscr{F}))\cong\Gamma(X,(h^*\mathscr{F})^{\vee})=\mathrm{Hom}_{\mathscr{O}_X}(h^*\mathscr{F},\mathscr{O}_X)$$
	\begin{proof}
		\begin{align*}
			\mathrm{Hom}_S(X,\mathbb{V}(\mathscr{F}))&=\mathrm{Hom}_{\textbf{Alg}(\mathscr{O}_S)}(\mathrm{Sym}(\mathscr{F}),h_*\mathscr{O}_X)\\&=\mathrm{Hom}_{\mathscr{O}_S}(\mathscr{F},h_*\mathscr{O}_X)\\&=\mathrm{Hom}_{\mathscr{O}_X}(h^*\mathscr{F},\mathscr{O}_X)
		\end{align*}
	\end{proof}
	\item 零截面.特别的,在典范对应$\mathrm{Hom}_S(S,\mathbb{V}(\mathscr{F}))=\mathrm{Hom}_{\mathscr{O}_S}(\mathscr{F},\mathscr{O}_S)$下,右侧的零态射对应的左侧的截面称为$\mathbb{V}(\mathscr{F})\to S$的零截面.另外按照分离态射的截面(右逆)总是闭嵌入,所以这里$\mathrm{Hom}_S(S,\mathbb{V}(\mathscr{F}))$中的态射都是闭嵌入.
	
	
	
	
	
	
	
	
	
	
	
	\item 例子.我们知道$h^*\mathscr{O}_X=\mathscr{O}_T$.所以如果我们取$\mathscr{F}=(\mathscr{O}_X^n)^{\vee}$,上述同构就是:
	$$\mathrm{Hom}_X(T,\mathbb{V}((\mathscr{O}_X^n)^{\vee}))\cong\Gamma(T,\mathscr{O}_T)^n=\mathrm{Hom}_X(T,\mathbb{A}_X^n)$$
	
	于是我们得到$\mathbb{V}((\mathscr{O}_X^n)^{\vee})\cong\mathbb{A}_X^n$.
\end{enumerate}




\subsection{几何向量丛和局部自由模层}
\begin{enumerate}
	\item 设$n\ge0$,概形$X$上的一个秩为$n$的几何向量丛是指一个$X$概形$V$,使得存在$X$的开覆盖$X=\cup_iU_i$,存在$U_i$概形同构$c_i:V\mid_{U_i}\cong\mathbb{A}_{U_i}^n$(这里我们用$V\mid_{U_i}$表示$U_i$在结构态射$V\to X$下的原像),满足对任意仿射开子集$U=\mathrm{Spec}A\subseteq U_i\cap U_j$,有$c_i\circ c_j^{-1}$限制为$\mathbb{A}_U^n=\mathrm{Spec}A[T_1,\cdots,T_n]$上的自同构是一个线性自同构,换句话讲它对应的$A$代数自同构$\varphi:A[T_1,\cdots,T_n]\to A[T_1,\cdots,T_n]$可以表示为$\varphi(T_i)=\sum_ja_{ji}T_j$,其中$(a_{ji})$构成$A$上的可逆矩阵.
	
	换句话讲,概形$X$上的秩$n$的几何向量丛指的是一个$X$概形$V$,使得存在开覆盖$X=\cup_iU_i$和$U_i$概形同构$c_i:V\mid_{U_i}\cong\mathbb{V}((\mathscr{O}_{U_i}^n)^{\vee})$,满足$c_i\circ c_j^{-1}$限制在$U_i\cap U_j$上是拟凝聚丛$\mathbb{V}((\mathscr{O}_{U_i\cap U_j}^n)^{\vee})$自身上的线性自同构(我们定义过这个概念,这是指这个拟凝聚丛同构是被模层同构诱导的).
	\item 如果$p:V\to X$是几何向量丛,那么$p$是相对光滑维数$n$的.
	\item 设$X$是概形,那么逆变函子$\mathscr{F}\mapsto\mathbb{V}(\mathscr{F})$是从秩$n$局部自由$\mathscr{O}_X$模层范畴到$X$上秩$n$几何向量丛范畴(约定态射是线性态射,也即被模层之间态射诱导的拟凝聚丛态射)的逆变范畴等价.
	\begin{itemize}
		\item 先设$\mathscr{F}$是秩$n$局部自由$\mathscr{O}_X$模层,那么使得$\mathscr{F}$限制后是秩$n$自由模层的开集构成了$X$的开覆盖.任取这样的两个开集$U,U'$,那么存在同构$c:\mathscr{F}\mid_U\cong(\mathscr{O}_U^n)^{\vee}$和$d:\mathscr{F}\mid_{U'}\cong(\mathscr{O}_{U'}^n)^{\vee}$.于是把$c\circ d^{-1}$限制在$(\mathscr{O}_{U\cap U'}^n)^{\vee}$上就诱导了$\mathbb{V}((\mathscr{O}_{U\cap U'}^n)^{\vee})$上的线性同构.
		\item 这个函子是完全忠实的,我们解释过$\mathscr{F}\mapsto\mathbb{V}(\mathscr{F})$作为终端是$X$上概形范畴的函子是忠实函子,所以这里作为终端是几何向量丛范畴的函子也是忠实函子,而几何向量丛范畴的态射定义就是被模层之间态射诱导的,所以这个函子自动是完全忠实的.
		\item 最后的问题是证明本质满.设$f:V\to X$是$X$上的秩$n$几何向量丛,定义$X$上的层$\mathscr{I}(V/X)$为在每个开集$U\mapsto\{s:U\to V\mid_U\mid f\circ s=\mathrm{id}_U\}$,这个预层的限制映射就是把态射限制在更小的开集上(这是一个层因为这里态射粘合的存在性和唯一性都是直接的).我们断言$\mathscr{I}(V/X)$可以作为$X$上的秩$n$局部自由$\mathscr{O}_X$模层.为此取几何向量丛定义中的开覆盖$X=\cup_iU_i$,我们有$V\mid_{U_i}\cong\mathbb{A}_{U_i}^n$,按照拟凝聚丛的可表性,对任意开集$U\subseteq U_i$,我们有:
		\begin{align*}
			\Gamma(U,\mathscr{I}(V/X)\mid_{U_i})&=\mathrm{Hom}_{U_i}(U,\mathbb{V}((\mathscr{O}_{U_i}^n)^{\vee}))\\&=\Gamma(U,\mathscr{O}_{U_i}^n)
		\end{align*}
		
		于是得到$\mathscr{I}(V/X)\mid_{U_i}=\mathscr{O}_{U_i}^n$.这说明$\mathscr{I}(V/X)$是$X$上的秩$n$局部自由$\mathscr{O}_X$模层.
		\item 我们构造的$\mathscr{I}(V/X)$和基变换可交换,换句话讲如果$f:X'\to X$是概形的态射,那么我们有典范的同构:
		$$f^*\mathscr{I}(V/X)\cong\mathscr{I}(V\times_XX'/X')$$
		\item 证明本质满,设$f:V\to X$是秩$n$几何向量丛,设$h:T\to X$是任意$X$概形,我们有:
		\begin{align*}
			\mathrm{Hom}_X(T,\mathbb{V}(\mathscr{I}(V/X))^{\vee})&=\Gamma(T,h^*\mathscr{I}(V/X))\\&=\Gamma(T,\mathscr{I}(V\times_XT/T))\\&=\mathrm{Hom}_T(T,V\times_XT)\\&=\mathrm{Hom}_X(T,V)
		\end{align*}
		
		于是我们有$\mathbb{V}(\mathscr{I}(V/X)^{\vee})\cong V$,于是本质满.
	\end{itemize}
\end{enumerate}
\subsection{向量丛的同构类和上同调}
\begin{enumerate}
	\item 设$(X,\mathscr{O}_X)$是环空间,定义$G=\mathrm{GL}_n(\mathscr{O}_X)$为开集$U\mapsto\mathrm{Aut}_{\mathscr{O}_U}(\mathscr{O}_X^n\mid_U)=\mathrm{GL}_n(\mathscr{O}_X(U))$的群层(未必是阿贝尔的).设$\mathscr{F}$和$\mathscr{G}$是$\mathscr{O}_X$模层,记$\mathrm{Isom}_{\mathscr{O}_X}(\mathscr{F},\mathscr{G})$为开集$U\mapsto$全体$\mathscr{F}\mid_U\to\mathscr{G}\mid_U$的模层同构,这明显是一个层因为同构总是可以唯一粘合的.现在设$\mathscr{F}$是$X$上的秩$n$局部自由$\mathscr{O}_X$模层,那么$\mathrm{GL}_n(\mathscr{O}_X)(U)$作用在$\mathrm{Isom}(\mathscr{O}_X^n,\mathscr{F})(U)$上为$(g,u)=u\circ g^{-1}$,并且这个作用是单可迁的(见挠子和非阿贝尔层上同调),并且如果设$X=\cup_iU_i$是开覆盖使得$\mathscr{F}\mid_{U_i}\cong\mathscr{O}_X^n\mid_{U_i}$,那么$\mathrm{Isom}(\mathscr{O}_X^n,\mathscr{F})(U_i)$总不是空集,这说明$\mathrm{Isom}(\mathscr{O}_X^n,\mathscr{F})$是$\mathrm{GL}_n(\mathscr{O}_X)$挠子.于是我们得到如下映射:
	$$\alpha:S=\{\text{秩}n\text{局部自由}\mathscr{O}_X\text{模层的同构类}\}\to\mathrm{H}^1(X,\mathrm{GL}_n(\mathscr{O}_X))$$
	$$\overline{\mathscr{F}}\mapsto\overline{\mathrm{Isom}(\mathscr{O}_X^n,\mathscr{F})}$$
	
	这里$\mathrm{H}^1(X,G)$是$X$上模层$G$的全体$G$挠子的等价类构成的带基点集合.我们断言这是一个保基点的双射.
	\begin{proof}
		
		记$\check{\mathrm{H}}^1(X,G)$是非阿贝尔的\v{C}ech上同调,我们解释过它和$\mathrm{H}^1(X,G)$是保基点典范同构的.我们来构造逆映射$\beta:\check{\mathrm{H}}^1(X,G)\to S$如下.取$\theta\in\check{\mathrm{H}}^1(X,G)$,这是一个上同调类,取它的一个1-余圈表示元为$\{g_{ij}\in G(U_{ij})\}$,其中$\{U_i\mid i\in I\}$是$X$的开覆盖,$U_{ij}=U_i\cap U_j$.记$\mathscr{F}_i=\mathscr{O}_{U_i}^n$,按照1-余圈定义有$g_{ij}:\mathscr{F}_j\mid_{U_{ij}}\cong\mathscr{F}_i\mid_{U_{ij}}$满足$g_{ik}=g_{ij}g_{jk}$,所以$\{(\mathscr{F}_i),(g_{ij})\}$构成粘合信息,它们粘合得到一个秩$n$局部自由$\mathscr{O}_X$模层$\mathscr{F}$,满足存在同构$t_i:\mathscr{F}\mid_{U_i}\cong\mathscr{F}_i$,使得$g_{ij}=t_i\circ t_j^{-1}$.那么$\mathscr{F}$所在的同构类不依赖于$\{U_i\}$和$(g_{ij})$的选取,并且它和$\alpha$互为逆映射.
	\end{proof}
	\item 设$\mathscr{U}=\{U_i\mid i\in I\}$是$X$的开覆盖,我们有如下双射:
	$$\{\text{秩}n\text{局部自由}\mathscr{O}_X\text{模层}\mathscr{F}\text{的同构类,使得}\mathscr{F}\mid_{U_i}\cong\mathscr{O}_{U_i}^n,\forall i\}\cong\check{\mathrm{H}}^1(\mathscr{U},\mathrm{GL}_n(\mathscr{O}_X))$$
	\item 秩1情况,Picard群.设$X$是环空间,$X$上的秩1局部自由模层也称为$X$上的线丛,或者$X$上的可逆层.全部可逆模层的同构类在张量积下构成一个交换群,其中幺元是$\mathscr{O}_X$,可逆层$\mathscr{F}$的逆元是$\mathscr{F}^{\vee}=\mathrm{HOM}_{\mathscr{O}_X}(\mathscr{F},\mathscr{O}_X)$.这个群称为环空间$X$上的Picard群,记作$\mathrm{Pic}(X)$.那么我们有:
	$$\mathrm{Pic}(X)\cong\mathrm{H}^1(X,\mathscr{O}_X^*)$$
	\begin{proof}
		
		首先把$\mathscr{O}_X$视为$\mathscr{O}_X$模层,有$\mathrm{Aut}_{\mathscr{O}_X}(\mathscr{O}_X)=\mathscr{O}_X(X)^*$.因为一方面任取$a\in\mathscr{O}_X(X)^*$,在每个开集$U$上数乘$a\mid_U$是$\mathscr{O}_X(U)$模$\mathscr{O}_X(U)$上的自同构.另一方面如果$\alpha=\{\alpha_U\}$是$\mathscr{O}_X$模层$\mathscr{O}_X$的自同构,那么$\alpha_X$是$\mathscr{O}_X$模$\mathscr{O}_X$的自同构,所以它是数乘某个元$a\in\mathscr{O}_X(X)^*$.按照如下交换图表说明每个$\alpha_U$就是数乘$a\mid_U$.
		$$\xymatrix{\mathscr{O}_X(X)\ar[rr]^a\ar[d]&&\mathscr{O}_X(X)\ar[d]\\\mathscr{O}_X(U)\ar[rr]&&\mathscr{O}_X(U)}$$
		
		下面任取$X$上的可逆模层$\mathscr{L}$,按照定义可取$X$的开覆盖$\mathscr{U}=\{U_i\}$,使得每个$\mathscr{O}_{U_i}\cong\mathscr{L}\mid_{U_i}$,固定一个这样的同构$\varphi_i$,那么$\varphi_i^{-1}\circ\varphi_j$是模层$\mathscr{O}_{U_i\cap U_j}$上的自同构,于是典范对应于$\mathscr{O}_X(U_i\cap U_j)^*$中的元记作$a_{ij}$.那么对指标$i,j,k$就有$a_{ij}a_{jk}/a_{ik}$是$\mathscr{O}_X(U_i\cap U_j\cap U_k)^*$的单位元.换句话讲$\{a_{ij}\in\mathscr{O}_X(U_i\cap U_j)^*\}$是\v{C}ech复形中$\mathrm{d}:\check{C}(\mathscr{U})^1(\mathscr{U},\mathscr{O}_X^*)\to\check{C}(\mathscr{U})^2(\mathscr{U},\mathscr{O}_X^*)$的核中,于是$\{a_{ij}\in\mathscr{O}_X(U_i\cap U_j)^*\}$对应了$\check{\mathrm{H}}^1(\mathscr{U},\mathscr{O}_X^*)$中的元.倘若我们选取不同的模层同构族$\psi_i:\mathscr{O}_{U_i}\cong\mathscr{L}\mid_{U_i}$,那么$\{\varphi_i^{-1}\psi_i\in\Gamma(U_i,\mathscr{O}_X^*)\}$对应了$\mathrm{d}\left(\check{C}(\mathscr{U})^0(\mathscr{U},\mathscr{O}_X^*)\right)$中,而$\check{\mathrm{H}}^1(\mathscr{U},\mathscr{O}_X^*)$模去了这个像,所以我们构造了$\mathrm{Pic}(X)\to\mathrm{H}^1(\mathscr{U},\mathscr{O}_X^*)$不依赖于同构的选取,并且的确定义在可逆模层的同构类上.容易验证这是群同态.进而复合上$\check{\mathrm{H}}^1(\mathscr{U},\mathscr{O}_X^*)\to\lim\limits_{\substack{\rightarrow\\\mathscr{V}}}\check{H}^1(\mathscr{V},\mathscr{O}_X^*)=\check{\mathrm{H}}^1(X,\mathscr{O}_X^*)\cong\mathrm{H}^1(X,\mathscr{O}_X^*)$.这就得到了映射$\mathrm{Pic}(X)\to\mathrm{H}^1(X,\mathscr{O}_X^*)$.
		
		\qquad
		
		反过来任取$\alpha\in\check{\mathrm{H}}^1(X,\mathscr{O}_X^*)$,按照定义存在$X$的开覆盖$\mathscr{V}=\{V_j\}$,存在$g_{ij}\in\mathscr{O}_X^*(V_i\cap V_j)$,满足$g_{ij}g_{jk}=g_{ik}$.取$\mathscr{F}_j=\mathscr{O}_{U_j}$,那么$g_{ij}$对应于一个同构$\mathscr{F}_j\mid_{U_i\cap U_j}\cong\mathscr{F}_i\mid_{U_i\cap U_j}$.这些信息就可粘合为$X$上的可逆模层$\mathscr{F}$.改变$\mathscr{V}$的选取以及$g_{ij}$的选取得到的可逆模层是同构的.于是这的确得到了$\check{\mathrm{H}}^1(X,\mathscr{O}_X^*)\to\mathrm{Pic}(X)$的群同态.验证它和上一段的同态互为逆映射即可.
	\end{proof}
	\item 域上仿射线$X=\mathbb{P}_k^1$上的向量丛.
	\begin{enumerate}
		\item 取$X$的仿射开覆盖$U_0,U_1$都同构于$\mathbb{A}_k^1$,记$X$的函数域为$k(T)$,记$U_0=\mathrm{Spec}k[T]$和$U_1=\mathrm{Spec}k[1/T]$,我们记$R^+=k[T]$,$R^-=k[1/T]$,$R^{\pm}=k[T,1/T]$.任取$X$上的局部秩$n$自由模层,它限制在$U_0$和$U_1$上是两个局部秩$n$自由模层,对应的模分解记作$M_0$和$M_1$,那么这两个模是有限投射模,但是$R^+$和$R^-$都是PID,它的投射模必然是自由模,这就说明$M_0$和$M_1$本身就是自由模,换句话讲尽管一般概形上局部秩$n$自由模层,取使得它自由的开覆盖是不确定的,但是在这里的情况它一定在开覆盖$\mathscr{U}=\{U_0,U_1\}$上是自由的.于是$X$上的局部秩$n$自由模层就和$\check{\mathrm{H}}^1(\mathscr{U},G)$一一对应,其中$G=\mathrm{GL}_n(\mathscr{O}_X)$.
		\item 计算这个\v{C}ech上同调.因为$\mathscr{U}$只由两个开集构成,所以$G$的关于$\mathscr{U}$的1-余圈就是单个元素$g=g_{01}\in G(U_0\cap U_1)=\mathrm{GL}_n(R^{\pm})$.两个这样的元素$g,g'$在相同的上同调类中当且仅当存在$h^+\in G(U_0)=\mathrm{GL}_n(R^+)$和$h^-\in G(U_1)=\mathrm{GL}_n(R^-)$使得$g'=h^+gh^-$.于是$\check{\mathrm{H}}^1(\mathscr{U},G)$等同于一个双陪集.进而有如下双射:
		$$\{X\text{上的秩}n\text{向量丛的同构类}\}\cong\mathrm{GL}_n(R^+)\backslash\mathrm{GL}_n(R^{\pm})/\mathrm{GL}_n(R^-)$$
		\item 设$(\mathbb{Z}^n)_+$表示序列$\alpha=(d_1,\cdots,d_n)$满足$d_i\in\mathbb{Z}$和$d_1\ge d_2\ge\cdots\ge d_n$.用$T^{\alpha}$表示$n$阶对角矩阵$\mathrm{diag}(T^{d_1},\cdots,T^{d_n})$,那么我们有如下双射:
		$$\tau:(\mathbb{Z}^n)_+\cong\mathrm{GL}_n(R^+)\backslash\mathrm{GL}_n(R^{\pm})/\mathrm{GL}_n(R^-)$$
		$$\alpha\mapsto\mathrm{GL}_n(R^+)T^{\alpha}\mathrm{GL}_n(R^-)$$
		\item 在$n=1$时$\mathscr{O}_X(d)$对应的矩阵就是$(T^d)$,这得到:任取$X$上的秩$n$向量丛$\mathscr{F}$,那么存在唯一的整数链$d_1\ge\cdots\ge d_n$,使得有:
		$$\mathscr{F}\cong\oplus_{i=1}^n\mathscr{O}_{\mathbb{P}_k^1}(d_i)$$
		\begin{proof}
			
			我们来解释下$\mathscr{O}_X(d)$对应的矩阵是$(T^d)$.设$S=k[X_0,X_1]$,齐次分解为$S=\oplus_{n\ge0}S_n$,设$X=\mathrm{Proj}S$,设$T=X_1/X_0$,设$U_0=D_+(X_0)=\mathrm{Spec}k[T]$和$U_1=D_+(X_1)=\mathrm{Spec}k[1/T]$.按照定义有$\mathscr{O}_X(d)=\widetilde{\oplus_{n\ge d}S_n}$.那么$\mathscr{O}_X(d)(D_+(X_0))=X_1^dk[T,1/T]$,同理$\mathscr{O}_X(d)(D_+(X_1))=X_0^dk[T,1/T]$,于是它们限制在$\mathscr{O}_X(d)(D_+(X_0X_1))$上的同构对应的矩阵就是$X_1^d/X_0^d=T^d$.
		\end{proof}
	\end{enumerate}
\end{enumerate}
\subsection{射影丛}

设$X$是概形,其上的射影丛是指同构于$\mathbb{P}(\mathscr{E})=\mathrm{Proj}\mathrm{Sym}\mathscr{E}$的$X$概形,其中$\mathscr{E}$是拟凝聚$\mathscr{O}_X$模层.如果$\mathscr{E}=\mathscr{O}_X^n$,把$\mathbb{P}(\mathscr{E})$记作$\mathbb{P}_X^{n-1}$.特别的$\mathbb{P}_X^0=X$.于是在$X=\mathrm{Spec}A$是仿射的情况,这吻合于我们的记号$\mathbb{P}_A^{n-1}$.
\begin{enumerate}
	\item 函子性.设$u:\mathscr{E}\to\mathscr{F}$是拟凝聚$\mathscr{O}_X$模层之间的态射,它诱导了拟凝聚$\mathscr{O}_X$分次代数层之间的分次态射$\mathrm{Sym}(u):\mathrm{Sym}(\mathscr{E})\to\mathrm{Sym}(\mathscr{F})$.如果$u$是满射,那么$\mathrm{Sym}(u)$也是满射,此时它诱导了闭嵌入$\mathbb{P}(u):\mathbb{P}(\mathscr{F})\to\mathbb{P}(\mathscr{E})$.
    \item 设$u$是满射,记$j=\mathbb{P}(u)$,那么对任意整数$n$有典范同构:
	$$j^*(\mathscr{O}_{\mathbb{P}(\mathscr{E})}(n))\cong\mathscr{O}_{\mathbb{P}(\mathscr{F})(n)}$$
	\item 设$\psi:X'\to X$是概形态射,设$\mathscr{E}'=\psi^*\mathscr{E}$,那么有典范同构$\mathrm{Sym}(\mathscr{E}')(n)\cong\psi^*\left(\mathrm{Sym}(\mathscr{E})(n)\right)$,进而有典范同构:
	$$\mathbb{P}(\psi^*\mathscr{E})\cong\mathbb{P}(\mathscr{E})\times_XX'$$
	$$\mathscr{O}_{\mathbb{P}(\mathscr{E}')}(n)\cong\mathscr{O}_{\mathbb{P}(\mathscr{E})}(n)\otimes_{\mathscr{O}_X}\mathscr{O}_{X'}$$
	\item 设$\mathscr{L}$是可逆$\mathscr{O}_X$模层,对任意拟凝聚$\mathscr{O}_X$模层$\mathscr{E}$,有典范$Y$同构$i_{\mathscr{L}}:\mathbb{P}(\mathscr{E})\cong\mathbb{P}(\mathscr{E}\otimes\mathscr{L})$.进而对任意整数$n$有典范自然同构$i_{\mathscr{L}}^*(\mathscr{O}_{\mathbb{P}(\mathscr{E}\otimes\mathscr{L})})\cong\mathscr{O}_{\mathbb{P}(\mathscr{E})}\otimes_{\mathscr{O}_X}\mathscr{L}^{\otimes n}$.
	\begin{proof}
			
		如果$A$是换,$E$是$A$模,$L$是秩1自由$A$模,定义典范$A$同构$\mathrm{Sym}_n*(E\otimes L)\to\mathrm{Sym}_n(E)\otimes L^{\otimes n}$为$(x_1\otimes y_1)\cdots(x_n\otimes y_n)\mapsto(x_1\cdots x_n)\otimes(y_1\otimes\cdots\otimes y_n)$.这定义了一个分次$A$代数同构$\mathrm{Sym}_A(E\otimes L)\mapsto\oplus_{n\ge0}\mathrm{Sym}_n(E)\otimes L^{\otimes n}$.进而定义了一个$\mathscr{O}_X$分次代数层同构:
		$$\mathrm{Sym}(\mathscr{E}\otimes_{\mathscr{O}_X}\mathscr{L})\cong\oplus_{n\ge0}\mathrm{Sym}_n(\mathscr{E})\otimes_{\mathscr{O}_X}\mathscr{L}^{\otimes n}=\mathrm{Sym}(\mathscr{E})_{(\mathscr{L})}$$
	\end{proof}
\end{enumerate}
\subsection{终端是射影丛的态射}

设$Y$是概形,$\mathscr{E}$是拟凝聚$\mathscr{O}_Y$模层,设$P=\mathbb{P}(\mathscr{E})$,设结构态射$p:P\to Y$.
\begin{enumerate}
	\item 我们之前构造的自然变换$\alpha$在1分量处是$\alpha_1:\mathscr{E}\to p_*(\mathscr{O}_P(1))$,按照伴随性对应了态射$\alpha_1^{\#}:p^*\mathscr{E}\to\mathscr{O}_P(1)$,那么这总是一个满射.这是因为它是模层态射$\mathscr{E}\otimes_{\mathscr{O}_X}\mathrm{Sym}(\mathscr{E})\to\mathrm{Sym}(\mathscr{E})(1)$取伴随得到的,而由于$\mathscr{E}$生成整个$\mathrm{Sym}(\mathscr{E})$,就得到这个态射是满射.
	\item 满态射$\varphi_r$和$\psi_r$.设$r:X\to P$是$Y$概形态射,也即有如下交换图表:
	$$\xymatrix{X\ar[rr]^r\ar[dr]_q&&P\ar[dl]^p\\&Y&}$$
	
	按照$r^*$是右正合的,于是有满态射$r^*(\alpha_1^{\#}):q^*\mathscr{E}\to\mathscr{L}_r=r^*(\mathscr{O}_P(1))$.这里$\mathscr{L}_r$是一个可逆$\mathscr{O}_X$模层,我们把上述满态射记作$\varphi_r$.它可以延拓为$\mathscr{O}_X$代数层态射$\psi_r:q^*\mathrm{Sym}(\mathscr{E})=\mathrm{Sym}(q^*\mathscr{E})\to\mathrm{Sym}(\mathscr{L}_r)=\oplus_{n\ge0}\mathscr{L}_r^{\otimes n}$.
	\item $Y$是仿射时我们来描述$\varphi_r$.记$Y=\mathrm{Spec}A$是仿射的,可记$\mathscr{E}=\widetilde{E}$,其中$E$是$A$模.对$f\in E$就有$r^{-1}(D_+(f))=X_{\varphi^{\mathrm{b}}_r(f)}$.设$V=\mathrm{Spec}B$是$r^{-1}(D_+(f))$的仿射开子集.记$S=\mathrm{Sym}_A(E)$,那么$r\mid_V:V\to D_+(f)$对应一个$A$代数同态$\omega:S_{(f)}\to B$.我们有$q^*\mathscr{E}\mid_V=\widetilde{E\otimes_AB}$和$\mathscr{L}_r\mid_V=\widetilde{L_r}$,其中$L_r=(S(1))_{(f)}\otimes_{S_{(f)}}B_{[\omega]}$,这里$B_{[\omega]}$表示$B$经$\omega$视为$S_{(f)}$模.于是$\varphi_r$限制在$q^*\mathscr{E}\mid_V$上就对应着$B$模同态$u:E\otimes_AB\to L_r$,它把$x\otimes b$映为$(f/1)\otimes(b\omega(x/f))$.
	\item 反过来,设$q:X\to Y$是概形态射,设$\mathscr{L}$是可逆$\mathscr{O}_X$模层,设$\varphi:q^*\mathscr{E}\to\mathscr{L}$是模层态射,那么它对应了拟凝聚$\mathscr{O}_X$代数层态射$\psi:\mathrm{Sym}(q^*\mathscr{E})=q^*\mathrm{Sym}(\mathscr{E})\to\oplus_{n\ge0}\mathscr{L}^{\otimes n}$.进而诱导了$Y$态射$r_{\mathscr{L},\psi}:G(\psi)\to\mathbb{P}(\mathscr{E})$.这个态射也记作$r_{\mathscr{L},\varphi}$.
	\item 给定态射$q:X\to Y$和拟凝聚$\mathscr{O}_Y$模层$\mathscr{E}$,映射$r\mapsto(\mathscr{L}_r,\varphi_r)$和$(\mathscr{L},\varphi)\mapsto r_{\mathscr{L},\varphi}$是全体$Y$态射$r:X\to P=\mathbb{P}(\mathscr{E})$构成的集合与全体二元组$(\mathscr{L},\varphi)$的全体等价类构成的集合之间的一一对应.这里二元组$(\mathscr{L},\varphi)$的定义是$\mathscr{L}$是可逆$\mathscr{O}_X$模层,$\varphi:q^*\mathscr{E}\to\mathscr{L}$是满态射.定义$(\mathscr{L},\varphi)$和$(\mathscr{L}',\varphi')$等价如果存在$\mathscr{O}_X$模层同构$\tau:\mathscr{L}\to\mathscr{L}'$使得$\varphi'=\tau\circ\varphi$.
	\item 在上一条中取$Y=X$,我们有:$\mathrm{Hom}_Y(Y,P)$和使得$\mathscr{E}/\mathscr{F}$是可逆$\mathscr{O}_Y$模层的$\mathscr{E}$的拟凝聚子模层$\mathscr{F}$是一一对应的.如果$Y=\mathrm{Spec}K$是域的素谱,$\mathscr{E}=\widetilde{E}$是有限秩的,那么命题中的$\mathscr{F}$就对应于$E$的超平面,而$P$的$Y$截面就是$P$的$K$值点.所以这个命题就是说$P$的$K$值点集合就是全体$E$的超平面构成的集合,后者是古典射影空间的定义.所以这个性质把$\mathbb{P}(\mathscr{E})$联系到固定射影空间.
	\item 推论.取$\mathscr{E}=\mathscr{O}_Y^n$,那么$Y$态射$r:X\to\mathbb{P}_Y^{n-1}$一定被可逆$\mathscr{O}_X$层$\mathscr{L}$和一个满态射$\varphi:\mathscr{O}_X^n\to\mathscr{L}$诱导,后者也即$\mathscr{L}$上$n$个整体截面生成了整个$\mathscr{L}$.特别的,如果$X/A$是概形,我们有:
	\begin{enumerate}
		\item 设$\varphi:X\to\mathbb{P}^n_A$是一个$A$态射,那么$\varphi^*(\mathscr{O}(1))$是$X$上的可逆层,并且它被整体截面$s_i=\varphi^*(x_i),0\le i\le n$生成.
		\item 反过来如果$\mathscr{L}$是$X$上的可逆层,如果$s_0,\cdots,s_n\in\Gamma(X,\mathscr{L})$是生成了整个$\mathscr{L}$的整体截面,那么存在唯一的$A$态射$\varphi:X\to\mathbb{P}_A^n$,满足$\mathscr{L}\cong\varphi^*(\mathscr{O}(1))$,并且$s_i=\varphi^*(x_i),\forall 0\le i\le n$.
	\end{enumerate}
	\item 推论.$\mathbb{P}_k^n$的自同构群就是$\mathrm{PGL}(n,k)=\mathrm{GL}(n+1,k)/k^*$.
	\begin{proof}
		
		首先给定$k$上$n+1$阶可逆矩阵$\left(a_{ij}\right)$,那么$x_i'=\sum_ja_{ij}x_j$提供了多项式环$k[x_0,\cdots,x_n]$的自同构,于是它诱导了$\mathbb{P}_k^n$上的自同构.对非零元$\lambda$,可逆矩阵$\left(\lambda a_{ij}\right)$诱导了相同的自同构(齐次素理想的原像当然还是不变的).如果可逆矩阵诱导的是$\mathbb{P}_k^n$上的恒等映射,分别考虑齐次素理想$(x_0),(x_1),\cdots,(x_n)$和$(x_0,\cdots,x_n)$的原像就得到该可逆矩阵是一个数量矩阵,换句话讲$\mathrm{PGL}(n,k)$在$\mathbb{P}_k^n$上的作用是忠实的.
		
		\qquad
		
		下面仅需验证$\mathbb{P}_k^n$的每个$k$自同构$\varphi$都在$\mathrm{PGL}(n,k)$中.我们在后面会解释$\mathrm{Pic}(\mathbb{P}_k^n)\cong\mathbb{Z}$,生成元可取$\mathscr{O}(1)$,而$\mathrm{P}_k^n$的自同构$\varphi$就要诱导$\mathrm{Pic}(\mathbb{P}_k^n)$上的自同构,于是$\varphi^*(\mathscr{O}(1))$必须也是Picard群的生成元,所以它只可能是$\mathscr{O}(1)$或者$\mathscr{O}(-1)$,但是后者没有非平凡的整体截面,这迫使$\varphi^*(\mathscr{O}(1))\cong\mathscr{O}(1)$.现在$\Gamma(\mathbb{P}_k^n,\mathscr{O}(1))$是以$\{x_0,\cdots,x_n\}$为基的$k$线性空间,所以$s_i=\varphi^*(x_i)$也要构成这个线性空间的一组基,记$s_i=\sum_ja_{ij}x_j$,上一条解释了$\varphi$被这些$s_i$唯一确定,于是可逆矩阵$\left(a_{ij}\right)$就诱导了这个自同构,完成证明.
	\end{proof}
	\item 推论.如果$\mathrm{Pic}(Y)$平凡(也即可逆模层都同构于$\mathscr{O}_Y$),记全体$\mathscr{E}\to\mathscr{O}_Y$的$\mathscr{O}_Y$态射构成的集合为$V$,满态射构成的子集记作$V^*$,记$A=\Gamma(Y,\mathscr{O}_Y)$,那么$V$和$V^*$都是$A$模,$A$的可逆元构成的集合为$A^*$,我们有$P=\mathbb{P}(\mathscr{E})$的$Y$截面构成的集合典范的对应于$V^*/A^*$.
	\begin{enumerate}[(1)]
		\item 如果$y\in Y$,记$Y'=\mathrm{Spec}\kappa(y)$,那么$p^{-1}(y)=\mathbb{P}(\mathscr{E})\times_YY'=\mathbb{P}(\mathscr{E}^y)$,其中$\mathscr{E}^y=\mathscr{E}_y\otimes_{\mathscr{O}_{Y,y}}\kappa(y)$视为$\kappa(y)$上线性空间.
		\item 如果$K/\kappa(y)$是域扩张,那么$p^{-1}(y)\times_{\kappa(y)}K=\mathbb{P}(\mathscr{E}^y\otimes_{\kappa(y)}K)$.
		\item 设$Y=\mathrm{Spec}A$是仿射的,设$\mathrm{Pic}(Y)$平凡,设$\mathscr{E}=\mathscr{O}_Y^n$,那么$V=A^n$,$V^*$中的元对应于一组$\{f_1,\cdots,f_n\}\subseteq A$使得它们生成单位理想,两组元对应了相同的满同态当且仅当它们相差一个$A$的可逆元.
	\end{enumerate}
	\item 
	\begin{enumerate}[(1)]
		\item 设$u:X'\to X$是$Y$态射.如果$Y$态射$r:X\to P$对应于$(\mathscr{L},\varphi)$,那么$r\circ u$就对应于$(u^*\mathscr{L},u^*\varphi)$.
		\item 设$v:\mathscr{E}\to\mathscr{F}$是拟凝聚$\mathscr{O}_Y$模层之间的满态射,那么$j=\mathbb{P}(v):\mathbb{P}(\mathscr{F})\to\mathbb{P}(\mathscr{E})$是闭嵌入.那么如果$Y$态射$r:X\to\mathbb{P}(\mathscr{F})$对应于$(\mathscr{L},\varphi)$,则$j\circ r$对应于:
		$$\xymatrix{q^*\mathscr{E}\ar[r]^{q^*v}&q^*\mathscr{F}\ar[r]^{\varphi}&\mathscr{L}}$$
		\item 设$\psi':Y'\to Y$是概形态射,记$\mathscr{E}=\psi^*\mathscr{E}$.如果$Y$态射$r:X\to P$对应于$\varphi:q^*\mathscr{E}\to\mathscr{L}$,那么$Y'$态射$r_{(Y')}:X_{(Y')}\to P'=\mathbb{P}(\mathscr{E}')$对应于$\varphi_{(Y')}$
	\end{enumerate}
    \item 终端为射影丛的嵌入.设$Y$是qcqs概形,设$q:X\to Y$是有限型态射,设$\mathscr{L}$是可逆$\mathscr{O}_X$模层.
    \begin{enumerate}[(1)]
    	\item 设$\mathscr{A}$是拟凝聚$\mathscr{O}_Y$分次代数层,设$\psi:q^*\mathscr{A}\to\oplus_{n\ge0}\mathscr{L}^{\otimes n}$是分次代数层态射,那么$r_{\mathscr{L},\psi}$是定义在整个$X$上的嵌入,当且仅当可以找到正整数$n$以及$\mathscr{A}_n$的有限生成拟凝聚子模层$\mathscr{E}$,记$j:\mathscr{E}\to\mathscr{A}_n$是包含态射,使得$\psi'=\psi_n\circ q^*(j):q^*\mathscr{E}\to\mathscr{L}^{\otimes n}=\mathscr{L}'$是满态射,并且$r_{\mathscr{L}',\psi'}:X\to\mathbb{P}(\mathscr{E})$是嵌入.
    	\item 设$\mathscr{F}$是拟凝聚$\mathscr{O}_Y$模层,设$\varphi:q^*\mathscr{F}\to\mathscr{L}$是满态射,那么$r_{\mathscr{L},\varphi}:X\to\mathbb{P}(\mathscr{F})$是嵌入当且仅当可以找到$\mathscr{F}$的拟凝聚有限生成子模层$\mathscr{E}$,记$j:\mathscr{E}\to\mathscr{F}$是包含态射,使得$\varphi'=\varphi\circ q^*(j):q^*\mathscr{E}\to\mathscr{L}$是满射,并且$r_{\mathscr{L},\varphi'}:X\to\mathbb{P}(\mathscr{E})$是嵌入.
    \end{enumerate}
    \begin{proof}
    	
    	(1):我们之前给出过终端是齐次谱的态射是嵌入的等价描述,在那里需要验证$\mathscr{E}$在$\mathscr{A}$中生成的子代数$\mathscr{A}'$要满足$r_{\mathscr{L},\psi'}$是定义在整个$X$上的嵌入.但是我们有满态射$\mathrm{Sym}(\mathscr{E})\to\mathscr{A}'$,于是$\mathrm{Proj}\mathscr{A}'\to\mathbb{P}(\mathscr{E})$是闭嵌入.于是二者等价.
    	
    	\qquad
    	
    	(2):只要用终端是齐次谱态射是嵌入和正向系统的那个判别法.
    \end{proof}
    \item 推论.设$\varphi:X\to\mathbb{P}_A^n$是环$A$上的态射,记$\mathscr{L}=\varphi^*(\mathscr{O}(1))$是可逆模层,记$s_i=\varphi^*(x_i),0\le i\le n$.那么$\varphi$是闭嵌入当且仅当如下两个条件成立:
    \begin{enumerate}
    	\item 每个$X_i=X_{s_i}$是仿射开子集.
    	\item 对每个指标$i$,环同态$A[y_0,\cdots,\widehat{y_i},\cdots,y_n]\to\Gamma(X_i,\mathscr{O}_{X_i})$,$y_j\mapsto s_j/s_i$都是满同态.
    \end{enumerate}
    \begin{proof}
    	
    	一方面如果$\varphi$是闭嵌入,那么每个$X_i=X\cap U_i$都是$U_i$的闭子概型,于是$X_i$都是仿射的,并且包含态射诱导的环同态是满同态.另一方面如果这两个条件成立,那么每个$X_i$都是$U_i$的闭子概型,按照$X_i=\varphi^{-1}(U_i)$,这些$X_i$覆盖了整个$X$,于是$X$是$\mathbb{P}_A^n$的闭子概型.
    \end{proof}
    \item 设$k$是代数闭域,设$X$是$k$上射影概形,记$\varphi:X\to\mathbb{P}_k^n$是一个$k$态射,记$\mathscr{L}=\varphi^*(\mathscr{O}(1))$和$s_i=\varphi^*(x_i)\in\Gamma(X,\mathscr{L})$.设$V\subseteq\Gamma(X,\mathscr{L})$是被$s_i$生成的$k$线性子空间,那么$\varphi$是闭嵌入当且仅当如下两个条件成立:
    \begin{enumerate}
    	\item $V$中的元素可以分离$X$中的点,具体地讲,如果$p,q\in X$是两个不同的闭点,那么可以找到$s\in V$,使得$s$恰好只落在$\mathfrak{m}_p\mathscr{L}_p$和$\mathfrak{m}_q\mathscr{L}_q$中的一个.
    	\item 对每个闭点$p\in X$,集合$\{s\in V\mid s_p\in\mathfrak{m}_p\mathscr{L}_p\}$生成了整个$k$线性空间$\mathfrak{m}_p\mathscr{L}_p/\mathfrak{m}_p^2\mathscr{L}_p$(此为$\mathscr{L}$在点$p$的切空间).
    \end{enumerate}
    \begin{proof}
    	
    	先设$\varphi$是闭嵌入,那么$X$可视为$\mathbb{P}_k^n$的闭子概型,那么此时$\mathscr{L}$就是$X$上的扭曲层$\mathscr{O}_X(1)$.给定$X$中的两个闭点$p\not=q$,设超平面$\sum a_ix_i=0,a_i\in k$只包含了$p$但不包含$q$,那么$s=\sum a_ix_i$在$X$上的限制就满足落在$\mathfrak{m}_p\mathscr{L}_p$中,但不落在$\mathfrak{m}_q\mathscr{L}_q$中.【】
    \end{proof}
\end{enumerate}
\subsection{Segre态射}
\begin{enumerate}
	\item 构造.设$Y$是概形,$\mathscr{E}$和$\mathscr{F}$是拟凝聚$\mathscr{O}_Y$模层,记$P_1=\mathbb{P}(\mathscr{E})$和$P_2=\mathbb{P}(\mathscr{F})$,结构态射分别记作$p_1,p_2$.记$Q=P_1\times_YP_2$,投影态射记作$q_1:Q\to P_1$和$q_2:Q\to P_2$.我们有如下可逆$\mathscr{O}_Q$模层:
	$$\mathscr{L}=\mathscr{O}_{P_1}(1)\otimes_{\mathscr{O}_Y}\mathscr{O}_{P_2}(1)=q_1^*\mathscr{O}_{P_1}(1)\otimes_{\mathscr{O}_Q}q_2^*\mathscr{O}_{P_2}(1)$$
	
	记$r=p_1\circ q_1=p_2\circ q_2:Q\to Y$,那么有:
	$$r^*(\mathscr{E}\otimes_{\mathscr{O}_Y}\mathscr{F})=q_1^*(p_1^*\mathscr{E})\otimes_{\mathscr{O}_Q}q_2^*(p_2^*\mathscr{F})$$
	
	我们知道函子$\alpha$诱导了如下典范满态射:
	$$p_1^*\mathscr{E}\to\mathscr{O}_{P_1}(1)$$
	$$p_2^*\mathscr{F}\to\mathscr{O}_{P_2}(1)$$
	
	进而有如下典范满态射:
	$$s:r^*(\mathscr{E}\otimes_{\mathscr{O}_Y}\mathscr{F})\to\mathscr{L}$$
	
	它诱导了如下态射,称为Segre态射:
	$$\zeta:\mathbb{P}(\mathscr{E})\times_Y\mathbb{P}(\mathscr{F})\to\mathbb{P}(\mathscr{E}\otimes_{\mathscr{O}_Y}\mathscr{F})$$
	\item 如果$Y=\mathrm{Spec}A$是仿射的,记$\mathscr{E}=\widetilde{E}$和$\mathscr{F}=\widetilde{F}$,那么:
	$$\mathscr{E}\otimes_{\mathscr{O}_Y}\mathscr{F}=\widetilde{E\otimes_AF}$$
	
	记$R=\mathrm{Sym}_A(E)$和$S=\mathrm{Sym}_A(F)$,设$f\in E$和$g\in F$,考虑$Q$的仿射开子集$D_+(f)\times_YD_+(g)=\mathrm{Spec}B$,其中:
	$$B=R_{(f)}\otimes_AS_{(g)}$$
	
	那么$\mathscr{L}$在这个仿射开子集上的限制是$\widetilde{L}$,其中:
	$$L=(R(1))_{(f)}\otimes_A(S(1))_{(g)}$$
	
	它作为$B$秩1自由模的一个生成元可以取为$c=(f/1)\otimes(g/1)$.满态射$s$在这个开子集上就对应于同态:
	$$(E\otimes_AF)\otimes_AB\to L$$
	$$(x\otimes y)\otimes b\mapsto b((x/1)\otimes(y/1))$$
	
	进而$\zeta$限制为$D_+(f)\times_YD_+(g)\to D_+(f\otimes g)$就对应于如下环同态,其中$T=\mathrm{Sym}_A(E\otimes_AF)$:
	$$\omega:T_{(f\otimes g)}\to R_{(f)}\otimes_AS_{(g)}$$
	$$(x\otimes y)/(f\otimes g)\mapsto(x/f)\otimes(y/g)$$
	\begin{enumerate}[(1)]
		\item 记$P=\mathbb{P}(\mathscr{E}\otimes_{\mathscr{O}_Y}\mathscr{F})$,我们有典范同构:
		$$\zeta^*(\mathscr{O}_P(1))\cong\mathscr{O}_{P_1}(1)\otimes_Y\mathscr{O}_{P_2}(1)$$
		\item 用$X_y$表示一个态射$f:X\to Y$在$y\in Y$处的概形纤维.那么我们总有:
		$$\zeta^{-1}(P_{x\otimes y})=(P_1)_x\times_Y(P_2)_y$$
	\end{enumerate}
	\item Segre态射是闭嵌入.
	\begin{proof}
		
		问题关于$Y$是仿射的,不妨设$Y=\mathrm{Spec}A$.因为$D_+(f\otimes g)$覆盖了整个$P$,以及$\zeta^{-1}(D_+(f\otimes g))=D_+(f)\times_YD_+(g)$.问题就归结为证明$\zeta$限制为$D_+(f)\times_YD_+(g)\to D_+(f\otimes g)$是闭嵌入,但是上一条已经描述了他对应的同态$\omega$,它是满射.
	\end{proof}
    \item 函子性.设$\mathscr{E}\to\mathscr{E}'$是拟凝聚$\mathscr{O}_Y$模层的满态射,那么有如下交换图表:
    $$\xymatrix{\mathbb{P}(\mathscr{E}')\times_Y\mathbb{P}(\mathscr{F})\ar[rr]^{j\times1}\ar[d]_{\zeta}&&\mathbb{P}(\mathscr{E})\times_Y\mathbb{P}(\mathscr{F})\ar[d]^{\zeta}\\\mathbb{P}(\mathscr{E}'\otimes_{\mathscr{O}_Y}\mathscr{F})\ar[rr]^{\mathbb{P}(j\otimes1)}&&\mathbb{P}(\mathscr{E}\otimes_{\mathscr{O}_Y}\mathscr{F})}$$
    \item 如果取概形态射$\psi:Y'\to Y$,记$\mathscr{E}'=\psi^*\mathscr{E}$和$\mathscr{F}'=\psi^*\mathscr{F}$.那么如下Segre态射就是Segre态射$\zeta$关于$Y'\to Y$的基变换.
    $$\zeta':\mathbb{P}(\mathscr{E}')\times_Y\mathbb{P}(\mathscr{F}')\to\mathbb{P}(\mathscr{E}'\otimes_{\mathscr{O}_Y}\mathscr{F}')$$
    \item 满态射$\mathscr{E}\oplus\mathscr{F}\to\mathscr{E}$和$\mathscr{E}\oplus\mathscr{F}\to\mathscr{F}$诱导了闭嵌入$\mathbb{P}(\mathscr{E})\to\mathbb{P}(\mathscr{E}\oplus\mathscr{F})$和$\mathbb{P}(\mathscr{F})\to\mathbb{P}(\mathscr{E}\oplus\mathscr{F})$.我们断言这两个闭子概型的底空间不交,从而诱导了闭嵌入$\mathbb{P}(\mathscr{E})\coprod\mathbb{P}(\mathscr{F})\to\mathbb{P}(\mathscr{E}\oplus\mathscr{F})$.
    \begin{proof}
    	
    	问题关于$Y$是局部的,不妨设$Y=\mathrm{Spec}A$,设$\mathscr{E}=\widetilde{E}$和$\mathscr{F}=\widetilde{F}$,设$\mathfrak{p}$是$\mathrm{Sym}_A(E)$的不包含无关理想的齐次素理想,那么它在$\mathrm{Sym}_A(E\oplus F)\to\mathrm{Sym}_A(E)$的原像是这样一个素理想,它包含了全部$\left(\mathrm{Sym}_A(F)\right)_+$,但是和$\mathrm{Sym}_A(E)$的交不能是整个$\left(\mathrm{Sym}_A(E)\right)_+$.这说明这两个底空间不交.
    \end{proof}
\end{enumerate}
\subsection{极丰沛层}

设$q:X\to Y$是概形的态射,称一个可逆$\mathscr{O}_X$模层$\mathscr{L}$是关于$q$-极丰沛层(very ample sheaf)或者$Y$-极丰沛层,如果存在拟凝聚$\mathscr{O}_Y$模层$\mathscr{E}$和一个$Y$嵌入$i:X\to P=\mathbb{P}(\mathscr{E})$使得$\mathscr{L}\cong i^*(\mathscr{O}_P(1))$.换句话讲可以找到拟凝聚$\mathscr{O}_Y$模层$\mathscr{E}$和一个满态射$\varphi:q^*\mathscr{E}\to\mathscr{L}$使得$r_{\mathscr{L},\varphi}$是嵌入.
\begin{enumerate}
	\item 如果$X$上存在$q$-极丰沛层,那么$q$一定是分离态射.
	\item 如果存在拟凝聚$\mathscr{O}_Y$分次代数层$\mathscr{A}$,被$\mathscr{A}_1$生成,并且存在$Y$嵌入$i:X\to P=\mathrm{Proj}\mathscr{A}$使得$\mathscr{L}\cong i^*(\mathscr{O}_p(1))$,那么$\mathscr{L}$是$q$-极丰沛层.
	\item 设$q:X\to Y$是拟紧态射,$\mathscr{L}$是可逆$\mathscr{O}_X$模层,那么$\mathscr{L}$是$q$-极丰沛层当且仅当$q_*\mathscr{L}$是拟凝聚的,并且典范态射$\sigma:q^*q_*\mathscr{L}\to\mathscr{L}$是满态射,并且$r_{\mathscr{L},\sigma}:X\to\mathbb{P}(q_*\mathscr{L})$是嵌入.
	\begin{proof}
		
		充分性是平凡的.必要性:此时$q$是qcqs态射,于是$q_*\mathscr{L}$是拟凝聚的.设有拟凝聚$\mathscr{O}_Y$模层$\mathscr{E}$和满态射$\varphi:q^*\mathscr{E}\to\mathscr{L}$使得$r_{\mathscr{L},\varphi}$是嵌入.我们知道$\varphi$有如下分解,于是$\sigma$是满态射:
		$$\xymatrix{q^*\mathscr{E}\ar[r]&q^*q_*\mathscr{L}\ar[r]^{\sigma}&\mathscr{L}}$$
		
		进而对应于代数层态射:
		$$\xymatrix{q^*\mathrm{Sym}(\mathscr{E})\ar[r]&q^*\mathrm{Sym}(q_*\mathscr{L})\ar[r]&\oplus_{n\ge0}\mathscr{L}^{\otimes n}}$$
		
		于是$r_{\mathscr{L},\varphi}$是嵌入就得到$r_{\mathscr{L},\sigma}$是嵌入.
	\end{proof}
	\item 推论.上一条中的等价描述在$Y$上是局部的,于是如果$q:X\to Y$是拟紧的,那么一个可逆$\mathscr{O}_X$层$\mathscr{L}$是$q$-极丰沛层是$Y$上的局部性质,也即当且仅当存在$Y$的开覆盖$\{U_i\}$,使得$\mathscr{L}\mid_{q^{-1}(U_i)}$是$U_i$-极丰沛层.
	\item 设$Y$是qcqs概形,设$q:X\to Y$是有限型态射,设$\mathscr{L}$是可逆$\mathscr{O}_X$模层,如下命题互相等价:
	\begin{enumerate}[(1)]
		\item $\mathscr{L}$是$q$-极丰沛层.
		\item 存在有限生成拟凝聚$\mathscr{O}_Y$模层$\mathscr{E}$和一个满态射$\varphi:q^*\mathscr{E}\to\mathscr{L}$使得$r_{\mathscr{L},\varphi}$是嵌入.
		\item 存在$q_*\mathscr{L}$的有限生成拟凝聚$\mathscr{O}_Y$子模层$\mathscr{E}$和一个满态射$q^*\mathscr{E}\to\mathscr{L}$,使得$r_{\mathscr{L},\varphi}$是嵌入.
	\end{enumerate}
    \begin{proof}
    	
    	这件事是因为我们给出过有限型条件下$r_{\mathscr{L},\varphi}$是嵌入的等价描述.
    \end{proof}
    \item 推论.设$Y$是qcqs概形,设$q:X\to Y$是有限型态射,如果$\mathscr{L}$是$q$-极丰沛层,那么可以找到一个拟凝聚$\mathscr{O}_Y$分次代数层$\mathscr{A}$,它被$\mathscr{A}_1$有限生成,并且有支配$Y$开嵌入$i:X\to P=\mathrm{Proj}\mathscr{A}$使得$\mathscr{L}\cong i^*(\mathscr{O}_P(1))$.
    \begin{proof}
    	
    	按照上一条,此时可以找到$q_*\mathscr{L}$的有限生成拟凝聚$\mathscr{O}_Y$子模层$\mathscr{E}$和满态射$\varphi:q^*\mathscr{E}\to\mathscr{L}$使得$j=r_{\mathscr{L},\varphi}:X\to P'=\mathbb{P}(\mathscr{E})$是嵌入.结构态射$p:P'\to Y$是有限型分离态射,于是从$Y$是qcqs概形得到$P'$也是.设嵌入$j:X\to P'$的概形闭包为$Z$,于是$j$可以分解为支配开嵌入$i:X\to Z$复合闭嵌入$k:Z\to P'$.于是$Z$可以表示为$\mathrm{Sym}(\mathscr{E})$某个商代数层$\mathscr{A}$的齐次谱,这里$\mathscr{A}$必然被$\mathscr{A}_1$有限生成(因为$\mathscr{E}$是有限生成的).进而$\mathscr{O}_Z(1)=k^*\mathscr{O}_{P'}(1)$,从而$\mathscr{L}=i^*(\mathscr{O}_Z(1))$.
    \end{proof}
    \item 设$q:X\to Y$是概形态射,设$\mathscr{L}$是$q$-极丰沛层,设$\mathscr{L}'$是可逆$\mathscr{O}_X$模层,并且存在拟凝聚$\mathscr{O}_Y$模层$\mathscr{E}'$和一个满态射$\varphi':q^*\mathscr{E}'\to\mathscr{L}'$.那么$\mathscr{L}\otimes_{\mathscr{O}_X}\mathscr{L}'$总是$q$-极丰沛层.
    \begin{proof}
    	
    	$Y$态射$r'=r_{\mathscr{L}',\varphi'}:X\to P'=\mathbb{P}(\mathscr{E}')$满足$\mathscr{L}'={r'}^*(\mathscr{O}_{P'}(1))$.并且存在拟凝聚$\mathscr{O}_Y$模层$\mathscr{E}$和满态射$\varphi:q^*\mathscr{E}\to\mathscr{L}$使得$r=r_{\mathscr{L},\varphi}:X\to P=\mathbb{P}(\mathscr{E})$是$Y$嵌入,并且$\mathscr{L}=r^*(\mathscr{O}_P(1))$.
    	
    	\qquad
    	
    	记$Q=\mathbb{P}(\mathscr{E}\otimes_{\mathscr{O}_Y}\mathscr{E}')$,考虑Segre态射$\zeta:P\times_YP'\to Q$,按照$r$是嵌入,就有$X\to P\times_YP'$也是嵌入(因为$X\to Y\to Z$是嵌入得到$X\to Y$是嵌入).按照$\zeta$是闭嵌入,就得到$r'':X\to P\times_YP'\to Q$是嵌入.我们知道$\zeta^*(\mathscr{O}_Q(1))\cong\mathscr{O}_P(1)\otimes_{\mathscr{O}_Y}\mathscr{O}_{P'}(1)$,于是有${r''}^*(\mathscr{O}_Q(1))\cong\mathscr{L}\otimes_{\mathscr{O}_X}\mathscr{L}'$.
    \end{proof}
    \item 推论.设$q:X\to Y$是态射.
    \begin{enumerate}[(1)]
    	\item 如果$\mathscr{L}$是可逆$\mathscr{O}_X$模层,$\mathscr{K}$是可逆$\mathscr{O}_Y$模层,那么$\mathscr{L}$是$q$-极丰沛层当且仅当$\mathscr{L}\otimes_{\mathscr{O}_X}q^*\mathscr{K}$是极丰沛层.
    	\item 设$\mathscr{L}$和$\mathscr{L}'$是两个$q$-极丰沛层,那么$\mathscr{L}\otimes_{\mathscr{O}_X}\mathscr{L}'$也是$q$-极丰沛层.特别的对任意$n\ge1$有$\mathscr{L}^{\otimes n}$都是$q$-极丰沛层.
    \end{enumerate}
    \begin{proof}
    	
    	上一条已经证明了(1)的必要性和(2).对于(1)的充分性就是$\mathscr{L}\cong\left(\mathscr{L}\otimes q^*\mathscr{K}\right)\otimes q^*(\mathscr{K}^{-1})$.
    \end{proof}
    \item 性质.
    \begin{enumerate}[(1)]
    	\item 对概形$X$,任意可逆层都是$1_X$-极丰沛层.
    	\item 设$f:X\to Y$是态射,$j:X'\to X$是嵌入,设$\mathscr{L}$是$f$-极丰沛层,那么$j^*\mathscr{L}$是$f\circ j$-极丰沛层.
    	\item 设$Z$是拟紧概形,设$f:X\to Y$是有限型态射,$g:Y\to Z$是拟紧态射,$\mathscr{L}$是$f$-极丰沛层,$\mathscr{K}$是$g$-极丰沛层,那么存在正整数$N$使得$n\ge N$时$\mathscr{L}\otimes f^*(\mathscr{K}^{\otimes n})$总是$g\circ f$-极丰沛层.
    	\item 设$f:X\to Y$和$g:Y'\to Y$是两个态射,记$X'=X_{(Y')}$,如果$\mathscr{L}$是$f$-极丰沛层,那么$\mathscr{L}'=\mathscr{L}\otimes_{\mathscr{O}_Y}\mathscr{O}_{Y'}$是$f_{(Y')}$-极丰沛层.
    	\item 设$f_i:X_i\to Y_i,i=1,2$是两个$S$态射,设$\mathscr{L}_i$是$f_i$-极丰沛层,$i=1,2$.那么$\mathscr{L}_1\otimes_{\mathscr{O}_S}\mathscr{L}_2$是$f_1\times_Sf_2$-极丰沛层.
    	\item 设$f:X\to Y$和$g:Y\to Z$是态射,如果$\mathscr{L}$是$g\circ f$-极丰沛层,那么$\mathscr{L}$是$f$-极丰沛层.
    	\item 设$f:X\to Y$是态射,$j:X_{\mathrm{red}}\to X$是典范嵌入.如果$\mathscr{L}$是$f$-极丰沛层,那么$j^*\mathscr{L}$是$f_{\mathrm{red}}$-极丰沛层.
    \end{enumerate}
    \begin{proof}
    	
    	(1),(2),(4),(5)都是定义或者顶多用到Segre态射.(7)是因为(2)和(6).
    	
    	\qquad
    	
    	(6):把$f:X\to Y$分解为$\xymatrix{X\ar[r]^{\Gamma_f}&X\times_ZY\ar[r]^{p_2}&Y}$.其中$p_2=(g\circ f)\times1_Y$.按照(5)知$\mathscr{L}\otimes_{\mathscr{O}_Z}\mathscr{O}_Y$是$p_2$-极丰沛层.这里$\Gamma_f$是嵌入,并且$\Gamma_f^*(\mathscr{L}\otimes_{\mathscr{O}_Z}\mathscr{O}_Y)=\mathscr{L}$,于是结论来自(2).
    	
    	\qquad
    	
    	(3):【】
    	
    \end{proof}
    \item 设$f:X\to Y$和$f':X'\to Y$是态射,设$f''=f'\coprod f':X''=X'\coprod X''\to Y$是无交并.设$\mathscr{L}$和$\mathscr{L}'$分别是$X$和$X'$上可逆层,那么$\mathscr{L}''=\mathscr{L}\coprod\mathscr{L}'$是$f''$-极丰沛层当且仅当$\mathscr{L}$和$\mathscr{L}'$分别是$f$-极丰沛层和$f'$-极丰沛层.
    \begin{proof}
    	
    	因为我们解释过有闭嵌入$\mathbb{P}(\mathscr{E})\coprod\mathbb{P}(\mathscr{E}')\to\mathbb{P}(\mathscr{E}\oplus\mathscr{E}')$.
    \end{proof}
\end{enumerate}
\subsection{丰沛层}

设$X$是概形,指定一个可逆层$\mathscr{L}$,在不引起歧义的前提下可以不指明这个可逆层.对任意$\mathscr{O}_X$模层$\mathscr{F}$,对任意整数$n$,记$\mathscr{F}(n)=\mathscr{F}\otimes\mathscr{L}^{\otimes n}$.记$S=\oplus_{n\ge0}\Gamma(X,\mathscr{L}^{\otimes n})$.记结构态射$p:X\to\mathrm{Spec}\mathbb{Z}$,那么$\mathrm{Hom}_{\textbf{GrAlg}(\mathscr{O}_X)}(p^*\widetilde{S})$和$\mathrm{End}_{\textbf{GrAlg}}(S)$之间存在一一对应,特别的我们把$1_S$对于的态射$\varepsilon:p^*\widetilde{S}\to\oplus_{n\ge0}\mathscr{L}^{\otimes n}$称为(关于$\mathscr{L}$的)典范代数层态射,它对应的概形态射$G(\varepsilon)\to\mathrm{Proj}S$称为典范态射.
\begin{enumerate}
	\item 设$X$是qcqs概形,设$\mathscr{L}$是可逆层,设$S=\oplus_{n\ge0}\Gamma(X,\mathscr{L}^{\otimes n})$,那么如下条件互相等价,在条件成立时称$\mathscr{L}$是丰沛层(ample sheaf).另外在条件成立时,设$\{f_i\}$是$S_+$中的一族正次齐次元,使得$X_{f_i}$都是仿射开子集,那么典范态射限制在$\cup_iX_{f_i}$是到$\cup_i(\mathrm{Proj}S)_{f_i}$的同构.
	\begin{enumerate}[(1)]
		\item 当$f$跑遍$S_+$的正次齐次元时$\{X_f\}$构成$X$的拓扑基.
		\item 当$f$跑遍$S_+$的正次齐次元时$\{X_f\}$中的那些仿射开子集构成$X$的开覆盖.
		\item 典范态射$G(\varepsilon)\to\mathrm{Proj}S$是定义在整个$X$上的支配开嵌入.
		\item 典范态射$G(\varepsilon)\to\mathrm{Proj}S$定义在整个$X$上,并且是拓扑嵌入.
		\item 对任意拟凝聚$\mathscr{O}_X$模层$\mathscr{F}$,把$\mathscr{F}(n)$的整体截面生成的子模层记作$\mathscr{F}_n$,那么$\mathscr{F}$是它的子模层族$\{\mathscr{F}_n(-n)\mid n\ge1\}$的和.
		\item 上一条中的模层改为理想层:对任意拟凝聚$\mathscr{O}_X$理想层$\mathscr{I}$,把$\mathscr{I}(n)$的整体截面生成的子模层记作$\mathscr{I}_n$,那么$\mathscr{I}$是它的子模层族$\{\mathscr{I}_n(-n)\mid n\ge1\}$的和.
		\item 对任意有限生成拟凝聚$\mathscr{O}_X$模层$\mathscr{F}$,可以找到整数$N$使得$n\ge N$时$\mathscr{F}(n)$都被整体截面生成.
		\item 对任意有限生成拟凝聚$\mathscr{O}_X$模层$\mathscr{F}$,可以找到两个正整数$m,k$使得$\mathscr{F}$是$\mathscr{L}^{\otimes(-n)}\otimes\mathscr{O}_X^k$的商模层.
		\item 上一条中的模层改为理想层:对任意有限生成拟凝聚$\mathscr{O}_X$理想层$\mathscr{I}$,可以找到两个正整数$m,k$使得$\mathscr{I}$是$\mathscr{L}^{\otimes(-n)}\otimes\mathscr{O}_X^k$的商模层.
	\end{enumerate}
    \begin{proof}
    	
    	【EGAII4.5节】
    	
    \end{proof}
    \item 设$X$是qcqs概形,如果$\mathscr{L}$是$\mathscr{O}_X$丰沛层,对任意开子集$U\subseteq X$都有$\mathscr{L}\mid_U$是$\mathscr{O}_U$丰沛层.这是因为等价描述中的(1).
    \item 如果$X$是仿射的,它的每个拟凝聚层都被整体截面生成,所以此时任何可逆层都是丰沛可逆层.
    \item 设$\mathscr{L}$是丰沛$\mathscr{O}_X$层,对$X$的任意有限子集$Z$和$Z$的任意开邻域$U$,都可以找到$n$和一个$f\in\Gamma(X,\mathscr{L}^{\otimes n})$,使得$Z\subseteq X_f\subseteq U$.
    \begin{proof}
    	
    	典范态射$\varphi:X\to\mathrm{Proj}S$是支配的开嵌入,所以问题归结为证明$\mathrm{Proj}S$的有限子集$Z$和开邻域$U$,都可以找到齐次元$f\in S_+$使得$Z\subseteq(\mathrm{Proj}S)_f\subseteq U$.记$U$的补集为$V_+(\mathfrak{a})$,其中$\mathfrak{a}$是$S_+$的分次真理想.$Z$由有限个没有包含$\mathfrak{a}$的齐次素理想构成,于是存在齐次元$f\in\mathfrak{a}$不包含在每个$Z$中的素理想里,它就是我们要找的$f$.
    \end{proof}
    \item 设$X$是qcqs概形,设$\mathscr{L}$是可逆$\mathscr{O}_X$层.
    \begin{enumerate}[(1)]
    	\item 对任意正整数$n$,有$\mathscr{L}$是丰沛层当且仅当$\mathscr{L}^{\otimes n}$是丰沛层.
    	\item 设$\mathscr{L}'$是可逆层,如果对任意$x\in X$都存在正整数$n$和${\mathscr{L}'}^{\otimes n}$的整体截面$s'$使得$s'(x)\not=0$.那么从$\mathscr{L}$是丰沛层得到$\mathscr{L}\otimes_{\mathscr{O}_X}\mathscr{L}'$是丰沛层.特别的,两个丰沛层的张量积仍然是丰沛层.
    	\item 设$\mathscr{L}$是丰沛层,设$\mathscr{L}'$是可逆层,那么存在正整数$N$使得$n\ge N$时$\mathscr{L}^{\otimes n}\otimes\mathscr{L}'$是丰沛层,并且可由整体截面生成.
    \end{enumerate}
    \begin{proof}
    	
    	(1)是因为丰沛层等价描述中的第一条,以及对任意齐次元$f\in S_+$有$X_f=X_{f^{\otimes n}}$.
    	
    	\qquad
    	
    	(2):对任意$x\in X$和$x$的任意开邻域$U$,可以找到$m\ge1$和$f\in\Gamma(X,\mathscr{L}^{\otimes m})$,使得$x\in X_f\subseteq U$.再设$f'\in\Gamma(X,{\mathscr{L}'}^{\otimes n})$满足$f'(x)\not=0$.取$s=f^{\otimes n}\otimes f'^{\otimes m}\in\Gamma(X,(\mathscr{L}\otimes\mathscr{L}')^{\otimes mn})$,那么$s(x)\not=0$,于是$x\in X_s\subseteq X_f\subseteq U$,于是按照丰沛层等价描述的第一条得到$\mathscr{L}\otimes\mathscr{L}'$是丰沛层.
    	
    	\qquad
    	
    	(3):按照丰沛层等价描述中的第七条,可以找到正整数$N'$使得$n\ge N'$时$\mathscr{L}^{\otimes n}\otimes\mathscr{L}'$被整体截面生成.特别的把它视为(2)中的$\mathscr{L}'$就满足前提条件,于是取$N=N'+1$就有$n\ge N$时有$\mathscr{L}^{\otimes n}\otimes\mathscr{L}'$是被整体截面生成的丰沛层.
    \end{proof}
    \item 设$\mathscr{L}$是qcqs概形$X$上的可逆层,那么如下条件互相等价:
    \begin{enumerate}[(1)]
    	\item $\mathscr{L}$是丰沛层.
    	\item 对每个$m>0$都有$\mathscr{L}^{\otimes m}$是丰沛层.
    	\item 存在某个$m>0$使得$\mathscr{L}^{\otimes m}$是丰沛层.
    \end{enumerate}
    \begin{proof}
    	
    	(a)$\Rightarrow$(b)$\Rightarrow$(c)都是平凡的.下面证明(c)$\Rightarrow$(a).假设正整数$m$使得$\mathscr{L}^{\otimes m}$是丰沛可逆层,任取有限生成的拟凝聚层$\mathscr{F}$,那么存在$n_0>0$,使得$n\ge n_0$时就有$\mathscr{F}\otimes\mathscr{L}^{\otimes mn}$被整体截面生成.对每个$1\le k\le m-1$,考虑凝聚层$\mathscr{F}\otimes\mathscr{L}^{\otimes k}$,说明存在$n_1,n_2,\cdots,n_{m-1}$,使得当$n\ge n_k$时有$\mathscr{F}\otimes\mathscr{L}^k\otimes\mathscr{L}^{\otimes mn}$被整体截面生成.所以取$N=m\cdot\max\{n_i\mid i=0,1,\cdots,m-1\}$,那么当$n\ge N$时就有每个$\mathscr{F}\otimes\mathscr{L}^n$是被整体截面生成的.
    \end{proof}
    \item 设$X/A$是环$A$上的qcqs概形,结构态射记作$q$,如果$\mathscr{L}$是$q$-极丰沛层,那么它是丰沛层.
    \begin{proof}
    	
    	存在$A$模$E$和一个满态射$\psi:q^*(\widetilde{\mathrm{Sym}_A(E)})\to\oplus_{n\ge0}\mathscr{L}^{\otimes n}$使得$i=r_{\mathscr{L},\psi}:X\to P=\mathbb{P}(\widetilde{E})$是定义在整个$X$上的嵌入.当$f$跑遍$\mathrm{Sym}_A(E)$的正次齐次元时$D_+(f)$构成$P$的拓扑基,于是$i^{-1}(D_+(f))=X_{\psi^{\mathrm{b}}(f)}$构成它的子空间$X$的拓扑基,于是$\mathscr{L}$是丰沛层.
    \end{proof}
    \item 推论.(Serre)设$X$是环$A$上的射影概形,设$\mathscr{O}(1)$是$X$上的极丰沛层,设$F$是有限生成的拟凝聚$\mathscr{O}_X$模层,那么存在$n_0$使得$n\ge n_0$时有$F(n)=F\otimes\mathscr{O}_X(1)^{\otimes n}$可被有限个整体截面生成.
    \begin{proof}
    	
    	这是等价定义的推论.下面的证明来自gtm52:设$i:X\to\mathbb{P}_A^r$是闭嵌入.使得$i^*(\mathscr{O}(1))=\mathscr{O}_X(1)$.按照$i$是有限态射,得到$i_*F$是$\mathbb{P}_A^r$上的凝聚层.另外我们证明过$i_*(F(n))=(i_*F)(n)$.另外还要$i_*(F(n))$和$F(n)$的整体截面环是相同的.上面这些事情说明了问题归结为把$X$替换为$\mathbb{P}_A^r$本身.
    	
    	\qquad
    	
    	$X$被$\{D_+(x_i),i=0,1,\cdots,r\}$覆盖,按照$F$是凝聚层,所以每个$F\mid D_+(x_i)\cong\widetilde{M_i}$,其中$M_i$是$B_i=A[x_0/x_i,\cdots,x_n/x_i]$有限模.所以可取有限个元素$\{s_{ij}\in M_i\}$生成了这个模.我们证明过可以选取足够大的$n$使得$x_i^ns_{ij}$延拓为$F(n)$上的整体截面$t_{ij}$对任意$i,j$成立.$F(n)$在$D_+(x_i)$上的限制对应于一个$B_i$模$M_i'$,并且$x_i^n:F\to F(n)$诱导了$M_i\to M_i'$的模同构.所以截面$x_i^ns_{ij}$生成了$M_i'$,所以整体截面$t_{ij}$生成了$F(n)$.
    \end{proof}
    \item 设$X/A$是环$A$上的有限型拟分离概形,结构态射记作$q$,设$\mathscr{L}$是$\mathscr{O}_X$可逆层.如下命题互相等价:
    \begin{enumerate}[(1)]
    	\item $\mathscr{L}$是丰沛层.
    	\item 存在正整数$N$使得$n\ge N$时$\mathscr{L}^{\otimes n}$是$q$-极丰沛层.
    	\item 存在某个正整数$n$使得$\mathscr{L}^{\otimes n}$是$q$-极丰沛层.
    \end{enumerate}
    \begin{proof}
    	
    	(2)推(3)推(1)是直接的.(3)推(2):设$\mathscr{L}^{\otimes n_0}$是极丰沛的,设$n\ge m_0$时$\mathscr{L}^{\otimes n}$被整体截面生成.记$N=n_0+m_0$,那么$n\ge n_0$时就有$\mathscr{L}^{\otimes n}=\mathscr{L}^{\otimes n_0}\otimes\mathscr{L}^{\otimes(n-n_0)}$.其中$\mathscr{L}^{\otimes(n-n_0)}$被整体截面生成,从而存在某个满态射$\mathscr{O}_X^r\to\mathscr{L}^{\otimes(n-n_0)}$,我们之前解释过此时它们的张量积是极丰沛层.
    	
    	\qquad
    	
    	(1)推(3):我们知道形如$\{X_f\}$的仿射开子集覆盖了整个$X$,取有限子覆盖$\{X_{s_i}\}$,其中$s_i\in\mathscr{L}^{n_i}(X)$.但是把$s_i$替换为次幂不改变主开集,于是可以取一个不依赖于$i$的正整数$n$使得所有$s_i\in\mathscr{L}^n(X)$.按照有限型概形的定义,有$\mathscr{O}_X(X_{s_i})=A[f_{ij}]$,这里$f_{ij}$只有有限个.按照拟凝聚层上截面的延拓定理,在拟紧拟可分条件下可取正整数$r\ge1$使得$X_{s_i}$上的截面$s_i^r\otimes f_{ij}$可以延拓为$\mathscr{L}^{nr}(X)$中的截面$s_{ij}$.我们可以选取一个足够大的$r$使得这个结论对每个$f_{ij}$都成立.我们断言$\mathscr{L}^{nr}$是关于$A$的极丰沛可逆层.
    	
    	\qquad
    	
    	因为$s_i$已经在$X_{s_i}$上生成了$\mathscr{L}^n$,于是整体截面族$\{s_i^r,\forall i\}$生成了$\mathscr{L}^{nr}$,特别的$\{s_i^r,s_{ij}\}_{i,j}$生成了$\mathscr{L}^{nr}$.于是存在态射$\pi:X\to\mathrm{Proj}A[S_i,S_{ij}]$.取终端的仿射开子集$U_i=D_+(S_i)$,那么$X_{s_i}=\pi^{-1}(U_i)$,并且$\mathscr{O}(U_t)\to\mathscr{O}_X(X_{s_t})$就是$A[S_i,S_{ij},1/S_t]\to A[f_{ij}]$把$S_{ij}/S_i$映为$f_{ij}$,于是这是满同态,于是$\pi$可以分解为$X\to U=\cup_iU_i\subseteq\mathrm{Proj}A[S_i,S_{ij}]$,前者是闭嵌入,后者是开嵌入.于是$\pi$是嵌入,于是$\mathscr{L}^{nr}$是关于$A$的极丰沛可逆层.
    \end{proof}
    \item 推论.设$Y$是仿射概形,设$q:X\to Y$是有限型分离态射,设$\mathscr{L}$是$\mathscr{O}_X$丰沛层,设$\mathscr{L}'$是$\mathscr{O}_X$可逆层,那么存在正整数$N$使得$n\ge N$时$\mathscr{L}^{\otimes n}\otimes\mathscr{L}'$总是$q$-极丰沛层.
    \begin{proof}
    	
    	首先可以找到$N'$使得$n\ge N'$时$\mathscr{L}^{\otimes n}\otimes\mathscr{L}'$总被整体截面生成.又可以找到$N''$使得$n\ge N''$时$\mathscr{L}^{\otimes n}$是$q$-极丰沛的.于是当$n\ge N'+N''$时就有$\mathscr{L}^{\otimes n}\otimes\mathscr{L}'$是$q$-极丰沛的.
    \end{proof}
    \item 设$X$是qcqs概形,$j:Z\to X$是闭子概型,设它对应的拟凝聚理想层$\mathscr{I}$是幂零的.那么可逆$\mathscr{O}_X$层$\mathscr{L}$是丰沛层当且仅当$\mathscr{L}'=j^*\mathscr{L}$是$\mathscr{O}_Z$丰沛层.
    \begin{proof}
    	
    	必要性:对$\mathscr{L}^{\otimes n}$的整体截面$f$,它在${\mathscr{L}'}^{\otimes n}=\mathscr{L}^{\otimes n}\otimes_{\mathscr{O}_X}(\mathscr{O}_X/\mathscr{I})$下的像记作$f'=f\otimes1$.我们有$X_f=Z_{f'}$,于是按照丰沛层等价描述的第一条就有$\mathscr{L}'$是丰沛层.
    	
    	\qquad
    	
    	充分性:按照$\mathscr{I}^r=0$,考虑闭子概型链$Z=X_0\to X_1\to\cdots\to X_{r-1}=X$,问题可以归结为设$\mathscr{I}^2=0$.设$g$是${\mathscr{L}'}^{\otimes n}$在$Z$上的整体截面,满足$Z_g$是仿射的,那么可以找到正整数$m$使得$g^{\otimes m}$是$\mathscr{L}^{\otimes mn}$在$X$上的某个整体截面$f$的典范像.为此考虑如下$\mathscr{O}_X$模的正合列:
    	$$\xymatrix{0\ar[r]&\mathscr{I}(n)\ar[r]&\mathscr{O}_X(n)=\mathscr{L}^{\otimes n}\ar[r]&\mathscr{O}_Z(n)={\mathscr{L}'}^{\otimes n}\ar[r]\ar[r]&0}$$
    	
    	它诱导了如下长正合列:
    	$$\xymatrix{0\ar[r]&\Gamma(X,\mathscr{I}(n))\ar[r]&\Gamma(X,\mathscr{L}^{\otimes n})\ar[r]&\Gamma(X,{\mathscr{L}'}^{\otimes n})\ar[r]^{\partial}&\mathrm{H}^1(X,\mathscr{I}(n))}$$
    	
    	按照$\mathscr{I}^2=0$有$\mathscr{I}$是拟凝聚$\mathscr{O}_Z$模层.并且$\mathscr{I}(n)=\mathscr{I}\otimes_{\mathscr{O}_X}\mathscr{L}^{\otimes n}=\mathscr{I}\otimes_{\mathscr{O}_Z}{\mathscr{L}'}^{\otimes n}$.对任意$s\in\Gamma(X,{\mathscr{L}'}^{\otimes k})$,张量$s$是$\mathscr{I}(n)\to\mathscr{I}(n+k)$的模层态射,进而诱导了上同调的同态$s:\mathrm{H}^i(X,\mathscr{I}(n))\to\mathrm{H}^i(X,\mathscr{I}(n+k))$.特别的,张量$g^{\otimes m}$就是$\mathrm{H}^1(X,\mathscr{I}(n))\to\mathrm{H}^1(X,\mathscr{I}(n+m))$的同态,我们断言对足够大的$m$有$\partial g$在这个同态下的像是零,也即$g^{\otimes m}\otimes\partial g=0$.
    	
    	\qquad
    	
    	按照$Z_g$是仿射的,有$\mathrm{H}^1(Z_g,\mathscr{I}(n))=0$.进而$g'=g\mid_{Z_g}$在$\partial:\Gamma(Z_g,{\mathscr{L}'}^{\otimes n})\to\mathrm{H}^1(Z_g,\mathscr{I}(n))$下的像$\partial g'=0$.我们知道一阶的时候\v{C}ech上同调和层上同调一致.$\partial g$具有如下描述:取$X$的有限仿射开覆盖$\{U_i\}$,对每个指标$i$取$g_i\in\Gamma(U_i,\mathscr{L}^{\otimes n})$使得它在$\Gamma(U_i,{\mathscr{L}'}^{\otimes n})$下的像就是$g\mid_{U_i}$.那么$\partial g$就是$(g_i-g_j)\mid_{U_{ij}}\in\v{C}^1(X,\mathscr{I}(n))$在$\mathrm{H}^1(X,\mathscr{I}(n))$下的像.于是$\partial g'=0$意味着,在适当把$\{U_i\}$替换为加细的有限仿射开覆盖后,存在$h_i\in\Gamma(U_i\cap Z_g,\mathscr{I}(n))$使得$(g_i-g_j)\mid_{U_{ij}\cap Z_g}=(h_i-h_j)\mid_{U_{ij}\cap Z_g}$.按照拟凝聚层上截面延拓定理,以及这里指标$i$是有限的,可以选取一个正整数$m$,使得对每个$i$有$g^{\otimes m}\otimes h_i$可以延拓为$t_i\in\Gamma(U_i,\mathscr{I}(n+nm))$.于是对任意指标$i,j$就有$g^{\otimes m}\otimes(g_i-g_j)\mid_{U_{ij}}=(t_i-t_j)\mid_{U_{ij}}$.此即$g^{\otimes m}\otimes\partial g$在一阶上同调群里平凡.
    	
    	\qquad
    	
    	另外依旧还原到\v{C}ech上同调可以得到对任意$s\in\Gamma(X,\mathscr{O}_Z(p))$和$t\in\Gamma(X,\mathscr{O}_Z(q))$都有$\partial(s\otimes t)=(\partial s)\otimes t+s\otimes(\partial t)$.于是归纳得到$\partial(g^{\otimes k})=(kg^{\otimes(k-1)})\otimes(\partial g)$,于是$\partial(g^{\otimes {m+1}})=0$,于是$g^{\otimes(m+1)}$就是$\mathscr{L}^{\otimes m(m+1)}$中某个整体截面的像.
    \end{proof}
    \item 例子.
    \begin{itemize}
    	\item 设$k$是域,设$X=\mathbb{P}_k^n$,那么扭曲层$\mathscr{O}(1)$本身当然是极丰沛可逆层.另外我们解释过如果记$S=k[x_0,\cdots,x_n]$,那么$\mathrm{Proj}S\cong\mathrm{Proj}S(d)\cong\mathbb{P}_k^n$,并且这个同构下$\mathscr{O}(1)$就对应于$\mathscr{O}(d)$.于是$\mathscr{O}(d)$也是极丰沛可逆层.而当$d<0$时$\mathscr{O}(d)$就没有非平凡的整体截面,于是$d\le0$时$\mathscr{O}(d)$就不会是丰沛的.于是我们证明了$\mathbb{P}_k^n$上$\mathscr{O}(d)$是丰沛可逆层$\Leftrightarrow$它是极丰沛可逆层$\Leftrightarrow d>0$.
    	\item 设$Q$是$\mathbb{P}_k^3$上被$xy=zw$定义的非奇异二次曲面,那么$Q\cong\mathbb{P}^1_k\times_k\mathbb{P}^1_k$,并且有$\mathrm{Pic}Q\cong\mathbb{Z}\otimes\mathbb{Z}$.这个同构是对任意整数$a,b$,记$\pi_1$和$\pi_2$是$\mathbb{P}^1_k\times_k\mathbb{P}^1_k$分别到第一分量和第二分量的投影态射,那么$(a,b)$对应的可逆层是第一分量上$\mathscr{O}(a)$在$\pi_1$下的逆像和第二分量上$\mathscr{O}(b)$在$\pi_2$下的逆像的张量积.如果$a,b>0$,有$a$重嵌入$\mathbb{P}^1\to\mathbb{P}^{n_1}$和$b$重嵌入$\mathbb{P}^1\to\mathbb{P}^{n_2}$,得到闭嵌入$\mathbb{P}^1\times\mathbb{P}^1\to\mathbb{P}^{n_1}\times\mathbb{P}^{n_2}\to\mathbb{P}^n$.这导致$(a,b)$对应的可逆层是极丰沛可逆层.另一方面如果$a<0$或者$b<0$,那么$(a,b)$代表的可逆层就不被整体截面生成,于是如果$a\le0$或者$b\le0$就有$(a,b)$不是丰沛层.于是我们证明了$Q$上具有形式$(a,b)$的可逆层是丰沛的$\Leftrightarrow$它是极丰沛的$\Leftrightarrow a,b>0$.
    \end{itemize}
\end{enumerate}
\subsection{相对丰沛层}

设$f:X\to Y$是拟紧态射,设$\mathscr{L}$是$\mathscr{O}_X$可逆层,称$\mathscr{L}$是$f$-丰沛层或者$Y$-丰沛层,如果存在$Y$的仿射开覆盖$\{U_i\}$,记$X_i=f^{-1}(U_i)$,那么每个$\mathscr{L}\mid_{U_i}$是$\mathscr{O}_{X_i}$丰沛层.
\begin{enumerate}
	\item 如果$X$上的$f$-丰沛层存在,那么$f$一定是分离态射.
	\item 设$f:X\to Y$是拟紧态射,如果$\mathscr{L}$是$f$-极丰沛层,那么它也是$f$-丰沛层.
	\item 设$f:X\to Y$是拟紧态射,设$\mathscr{L}$是可逆$\mathscr{O}_X$,设$\mathscr{A}=\oplus_{n\ge0}f_*(\mathscr{L}^{\otimes n})$是$\mathscr{O}_Y$分次代数层.如下命题互相等价:
	\begin{enumerate}[(1)]
		\item $\mathscr{L}$是$f$-丰沛层.
		\item $\mathscr{A}$是拟凝聚的,并且典范态射$\sigma:f^*\mathscr{A}\to\oplus_{n\ge0}\mathscr{L}^{\otimes n}$诱导的$r_{\mathscr{L},\sigma}:G(\sigma)\to P=\mathrm{Proj}\mathscr{A}$是定义在整个$X$上的支配开嵌入.
		\item $f$是分离的,并且典范态射$\sigma:f^*\mathscr{A}\to\oplus_{n\ge0}\mathscr{L}^{\otimes n}$诱导的$r_{\mathscr{L},\sigma}:G(\sigma)\to P=\mathrm{Proj}\mathscr{A}$是定义在整个$X$上的拓扑嵌入.
	\end{enumerate}

    在条件成立时,对任意整数$n$,典范态射$r_{\mathscr{L},\sigma}^*(\mathscr{O}_P(n))\to\mathscr{L}^{\otimes n}$是同构.并且对任意拟凝聚$\mathscr{O}_X$模层$\mathscr{F}$,记$\mathscr{M}=\oplus_{n\ge0}f_*(\mathscr{F}\otimes\mathscr{L}^{\otimes n})$诱导的态射$r_{\mathscr{L},\sigma}^*\widetilde{\mathscr{M}}\to\mathscr{F}$是同构.
    \item 推论.
    \begin{enumerate}[(1)]
    	\item $Y$-丰沛层是在$Y$上局部的性质:设$\{U_i\}$是$Y$的开覆盖,那么$\mathscr{L}$是$Y$-丰沛层当且仅当对每个$i$有$\mathscr{L}\mid_{X_i}$是$U_i$-丰沛层,其中$X_i=f^{-1}(U_i)$.
    	\item 设$\mathscr{K}$是$\mathscr{O}_Y$可逆层,那么$\mathscr{L}$是$Y$-丰沛层当且仅当$\mathscr{L}\otimes f^*\mathscr{K}$是$Y$-丰沛层.
    	\item 设$Y$是仿射的,那么$\mathscr{L}$是$Y$-丰沛层当且仅当$\mathscr{L}$是丰沛层.
    	\item 设$f:X\to Y$是拟紧态射,设有拟凝聚$\mathscr{O}_Y$模层$\mathscr{E}$和一个$Y$态射$g:X\to P=\mathbb{P}(\mathscr{E})$,使得$g$是拓扑嵌入,那么$\mathscr{L}=g^*(\mathscr{O}_P(1))$是$Y$-丰沛层.
    	\begin{proof}
    		
    		不妨设$Y=\mathrm{Spec}A$是仿射的,记$\mathscr{E}=\widetilde{E}$,其中$E$是$A$模.当$f$跑遍$S=\mathrm{Sym}_A(E)$的正次齐次元时,$D_+(f)$就构成$P=\mathbb{P}(\mathscr{E})$的拓扑基.这里$g=r_{\mathscr{L},\psi}$,其中$\psi:f^*\mathscr{E}\to\mathscr{L}$,那么有$g^{-1}(D_+(f))=X_{\psi^{\mathrm{b}}(f)}$,其中$\psi^{\mathrm{b}}(f)\in\Gamma(X,\mathscr{L}^{\otimes n})$.按照$X$是$P$的子空间,就有$\{X_{\psi^{\mathrm{b}}(f)}\}$构成$X$的拓扑基,于是$\mathscr{L}$是$Y$-丰沛层.
    	\end{proof}
    \end{enumerate}
    \item 设$X$是qcqs概形,设$f:X\to Y$是qcqs态射,设$\mathscr{L}$是可逆$\mathscr{O}_X$模层,如下命题互相等价:
    \begin{enumerate}[(1)]
    	\item $\mathscr{L}$是$f$-极丰沛层.
    	\item 对任意有限生成拟凝聚$\mathscr{O}_X$模层$\mathscr{F}$,存在正整数$n_0$使得$n\ge n_0$时典范态射$\sigma:f^*f_*(\mathscr{F}\otimes\mathscr{L}^{\otimes n})\to\mathscr{F}\otimes\mathscr{L}^{\otimes n}$是满态射.
    	\item 对任意有限生成拟凝聚$\mathscr{O}_X$理想层$\mathscr{I}$,存在正整数$n$,使得典范态射$\sigma:f^*f_*(\mathscr{I}\otimes\mathscr{L}^{\otimes n})\to\mathscr{I}\otimes\mathscr{L}^{\otimes n}$是满态射.
    \end{enumerate}
    \item 设$f:X\to Y$是拟紧态射,设$\mathscr{L}$是可逆$\mathscr{O}_X$模层.
    \begin{enumerate}[(1)]
    	\item 对任意正整数$n$,有$\mathscr{L}$是$f$-丰沛层当且仅当$\mathscr{L}^{\otimes n}$是$f$-丰沛层.
    	\item 设$\mathscr{L}'$是可逆$\mathscr{O}_X$层,如果存在正整数$n$使得典范态射$\sigma:f^*f_*({\mathscr{L}'}^{\otimes n})\to{\mathscr{L}'}^{\otimes n}$是满态射.那么从$\mathscr{L}$是$f$-丰沛层得到$\mathscr{L}\otimes\mathscr{L}'$是$f$-丰沛层.特别的,这说明两个$f$-丰沛层的张量积仍然是$f$-丰沛层.
    \end{enumerate}
    \item 设$Y$是拟紧概形,$f:X\to Y$是有限型态射,$\mathscr{L}$是可逆$\mathscr{O}_X$模层,那么如下命题互相等价:
    \begin{enumerate}[(1)]
    	\item $\mathscr{L}$是$f$-丰沛层.
    	\item 存在正整数$N$使得$n\ge N$时$\mathscr{L}^{\otimes n}$是$f$-极丰沛层.
    	\item 存在正整数$n$使得$\mathscr{L}^{\otimes n}$是$f$-极丰沛层.
    \end{enumerate}
    \item 推论.设$Y$是拟紧概形,$f:X\to Y$是有限型态射,$\mathscr{L}$和$\mathscr{L}'$是可逆$\mathscr{O}_X$模层.如果$\mathscr{L}$是$f$-丰沛层,那么存在$N$使得$n\ge N$时$\mathscr{L}^{\otimes n}\otimes\mathscr{L}'$是$f$-极丰沛层.
    \item 性质.
    \begin{enumerate}[(1)]
    	\item 对概形$X$,任意可逆层都是$1_X$-丰沛层.
    	\item 设$f:X\to Y$是拟紧态射,$j:X'\to X$是拟紧的拓扑嵌入态射,设$\mathscr{L}$是$f$-丰沛层,那么$j^*\mathscr{L}$是$f\circ j$-丰沛层.
    	\item 设$Z$是拟紧概形,设$f:X\to Y$和$g:Y\to Z$是拟紧态射,$\mathscr{L}$是$f$-丰沛层,$\mathscr{K}$是$g$-丰沛层,那么存在正整数$N$使得$n\ge N$时$\mathscr{L}\otimes f^*(\mathscr{K}^{\otimes n})$总是$g\circ f$-丰沛层.
    	\item 设$f:X\to Y$是拟紧态射,$g:Y'\to Y$是态射,记$X'=X_{(Y')}$,如果$\mathscr{L}$是$f$-丰沛层,那么$\mathscr{L}'=\mathscr{L}\otimes_{\mathscr{O}_Y}\mathscr{O}_{Y'}$是$f_{(Y')}$-丰沛层.
    	\item 设$f_i:X_i\to Y_i,i=1,2$是两个拟紧$S$态射,设$\mathscr{L}_i$是$f_i$-丰沛层,$i=1,2$.那么$\mathscr{L}_1\otimes_{\mathscr{O}_S}\mathscr{L}_2$是$f_1\times_Sf_2$-丰沛层.
    	\item 设$f:X\to Y$和$g:Y\to Z$是态射,满足$g\circ f$是拟紧的,如果$\mathscr{L}$是$g\circ f$-丰沛层,并且要么$g$是分离的要么$X$底空间是局部诺特的,那么$\mathscr{L}$是$f$-丰沛层.
    	\item 设$f:X\to Y$是拟紧态射,$j:X_{\mathrm{red}}\to X$是典范嵌入.如果$\mathscr{L}$是$f$-丰沛层,那么$j^*\mathscr{L}$是$f_{\mathrm{red}}$-丰沛层.
    \end{enumerate}
    \item 设$f:X\to Y$是拟紧态射,$\mathscr{I}$是局部幂零的拟凝聚$\mathscr{O}_X$理想层,它定义的闭子概型记作$j:Z\to X$.那么一个可逆$\mathscr{O}_X$模层$\mathscr{L}$是$f$-丰沛层当且仅当$j^*\mathscr{L}$是$f\circ j$-丰沛层.
    \item 推论.设$X$是局部诺特概形,设$f:X\to Y$是拟紧态射,那么一个可逆$\mathscr{O}_X$模层$\mathscr{L}$是$f$-丰沛层当且仅当它在$X_{\mathrm{red}}\to X$下的逆像$\mathscr{L}'$是$f_{\mathrm{red}}$-丰沛层.
    \item 设$f:X\to Y$和$f':X'\to Y$是态射,设$f''=f'\coprod f':X''=X'\coprod X''\to Y$是无交并.设$\mathscr{L}$和$\mathscr{L}'$分别是$X$和$X'$上可逆层,那么$\mathscr{L}''=\mathscr{L}\coprod\mathscr{L}'$是$f''$-丰沛层当且仅当$\mathscr{L}$和$\mathscr{L}'$分别是$f$-丰沛层和$f'$-丰沛层.
    \item 设$Y$是拟紧概形,$\mathscr{A}$是有限型拟凝聚$\mathscr{O}_Y$分次代数层,设$f:X=\mathrm{Proj}\mathscr{A}\to Y$是典范态射.那么$f$是有限型态射,并且存在正整数$d$使得$\mathscr{O}_X(d)$是$f$-丰沛层.
\end{enumerate}
\subsection{拟仿射态射}

称概形$X$是拟仿射概形,如果它同构于某个仿射概形的拟紧开子概型.称态射$f:X\to Y$是拟仿射的,如果存在$Y$的仿射开覆盖$\{U_i\}$,使得每个$f^{-1}(U_i)$都是拟仿射概形.
\begin{enumerate}
	\item 对概形$X$,记$A=\Gamma(X,\mathscr{O}_X)$,那么$1_A$诱导了态射$u:X\to\mathrm{Spec}A$,它就是丰沛层定义中取$\mathscr{L}=\mathscr{O}_X$所定义的典范态射(因为$\mathrm{Proj}A[T]=\mathrm{Spec}A$,并且此时典范态射一定是定义在整个$X$上的,这可设$X$仿射验证).设$X$是qcqs概形,记$A=\Gamma(X,\mathscr{O}_X)$,那么如下命题互相等价:
	\begin{enumerate}[(1)]
		\item $X$是拟仿射概形.
		\item 典范态射$u:X\to\mathrm{Spec}A$是(支配)开嵌入.
		\item 典范态射$u:X\to\mathrm{Spec}A$是拓扑嵌入.
		\item $\mathscr{O}_X$是自身的$u$-极丰沛层.
		\item $\mathscr{O}_X$是自身的丰沛层.
		\item 当$f$跑遍$A$中元时,$\{X_f\}$构成$X$的拓扑基.
		\item 当$f$跑遍$A$中元时,$\{X_f\}$的那些仿射集覆盖了整个$X$.
		\item 任何拟凝聚$\mathscr{O}_X$模层都可以被整体截面生成.
		\item 任何拟凝聚$\mathscr{O}_X$理想层都可以被整体截面生成.
	\end{enumerate}
    \item 推论.
    \begin{enumerate}[(1)]
    	\item 拟仿射态射是分离和拟紧的.仿射态射是拟仿射的.
    	\item 设$X$是拟紧概形,如果有终端仿射的态射$v:X\to Y$,使得拓扑上它是到开子空间的同胚,那么$X$是拟仿射概形.
    	\begin{proof}
    		
    		取$Y$的一组整体截面$g_i$使得$D(g_i)$是$v(X)$的拓扑基.记$v^{\#}(g_i)=f_i$,那么有$X_{f_i}=v^{-1}(D(g_i))$构成$X$的拓扑基,于是$X$是拟仿射的.
    	\end{proof}
        \item 设$X$是拟仿射概形,那么任何可逆模层都是极丰沛的,也是丰沛的.因为它总被整体截面生成.
    \end{enumerate}
    \item 设$f:X\to Y$是拟紧态射,如下命题互相等价:
    \begin{enumerate}[(1)]
    	\item $f$是拟仿射态射.
    	\item $\mathscr{O}_Y$代数层$f_*\mathscr{O}_X$是拟凝聚的,并且$f_*\mathscr{O}_X$上的恒等态射对应的概形态射$X\to\mathrm{Spec}f_*\mathscr{O}_X$是开嵌入.
    	\item $\mathscr{O}_Y$代数层$f_*\mathscr{O}_X$是拟凝聚的,并且$f_*\mathscr{O}_X$上的恒等态射对应的概形态射$X\to\mathrm{Spec}f_*\mathscr{O}_X$在拓扑上是拓扑嵌入.
    	\item $\mathscr{O}_X$是自身的$f$-极丰沛层.
    	\item $\mathscr{O}_X$是自身的$f$-丰沛层.
    	\item $f$是分离态射,并且对任意拟凝聚$\mathscr{O}_X$模层$\mathscr{F}$,典范态射$\sigma:f^*f_*\mathscr{F}\to\mathscr{F}$是满态射.
    \end{enumerate}
    \item 推论.
    \begin{enumerate}[(1)]
    	\item 如果$f:X\to Y$是拟仿射态射,那么任意可逆$\mathscr{O}_X$模层都是$f$-极丰沛层.
    	\item 拟仿射态射是终端局部性质:设$f:X\to Y$是拟紧态射,那么它是拟仿射态射当且仅当对任意(也可以改为存在)$Y$的开覆盖$\{V_i\}$,有$f$限制为$f^{-1}(V_i)\to V_i$是拟仿射态射.
    	\item 如果拟紧态射$f:X\to Y$在拓扑上是拓扑嵌入,那么$f$是拟仿射态射.
    	\item 拟仿射态射的复合仍然是拟仿射态射.
    	\item 拟仿射态射的基变换仍然是拟仿射态射.
    	\item 如果$X\to Y\to Z$是拟仿射态射,$Y\to Z$是分离态射或者$X$的底空间是局部诺特的,那么$f$是拟仿射的.
    	\item 如果$f$是拟仿射态射,那么$f_{\mathrm{red}}$是拟仿射态射.
    \end{enumerate}
    \item 设$f:X\to Y$是拟紧态射,$g:X'\to X$是拟仿射态射,设$\mathscr{L}$是$f$-丰沛层,那么$g^*\mathscr{L}$是$f\circ g$-丰沛层.
    \begin{proof}
    	
    	问题在$Y$上是局部的,不妨设$Y$是仿射的.按照$\mathscr{O}_{X'}$是自身上的$g$-极丰沛层,就存在正整数$n$使得$\mathscr{O}_{X'}\otimes_{\mathscr{O}_{X'}}g^*(\mathscr{L}^{\otimes n})=g^*(\mathscr{L}^{\otimes n})=(g^*(\mathscr{L}))^{\otimes n}$是$f\circ g$-丰沛层.从而$g^*\mathscr{L}$是$f\circ g$-丰沛层.
    \end{proof}
\end{enumerate}
\subsection{拟射影态射}

称有限型态射$f:X\to Y$是拟射影态射,如果$X$上存在$f$-丰沛层.
\begin{enumerate}
	\item 这不是一个关于$Y$的局部性质;拟射影态射一定是分离态射;另外如果$Y$是拟紧的,那么$f$是拟射影态射也等价于讲$f$是有限型态射,并且$X$上存在$f$-极丰沛层.
	\item 设$Y$是qcqs概形,设$X$是$Y$概形.如下命题互相等价:
	\begin{enumerate}[(1)]
		\item $X$是拟射影$Y$概形.
		\item $X$在$Y$上是有限型的,并且存在有限生成拟凝聚$\mathscr{O}_Y$模层$\mathscr{E}$,使得$X$是$Y$同构于$\mathbb{P}(\mathscr{E})$的子概型.
		\item $X$在$Y$上是有限型的,并且存在拟凝聚$\mathscr{O}_Y$分次代数层$\mathscr{A}$,它被$\mathscr{A}_1$有限生成,并且$X$是$Y$同构于$\mathrm{Proj}\mathscr{A}$的一个稠密开子集上的开子概型.
	\end{enumerate}
    \item 性质.
    \begin{enumerate}[(1)]
    	\item 有限型拟仿射态射总是拟射影的.
    	\item 如果$X\to Y$和$Y\to Z$是拟射影的,并且$Z$是拟紧的,那么$X\to Y\to Z$是拟射影的.
    	\item 拟射影态射的基变换总是拟射影的.
    	\item 如果$X\to Y\to Z$是拟射影的,$Y\to Z$是分离的,那么$X\to Y$是拟射影的.
    	\item 如果$f$是拟射影态射,那么$f_{\mathrm{red}}$是拟射影态射.这个逆命题不对.
    \end{enumerate}
\end{enumerate}
\subsection{射影态射}
\begin{enumerate}
	\item 设$X$是$Y$概形,称结构态射是射影态射,或者称$X$是射影$Y$概形,如果它满足如下等价描述中的任一条:
	\begin{enumerate}[(1)]
		\item $X$是$Y$同构于某个$\mathbb{P}(\mathscr{E})$的闭子概型,其中$\mathscr{E}$是有限生成拟凝聚$\mathscr{O}_Y$模层.
		\item 存在拟凝聚$\mathscr{O}_Y$分次代数层$\mathscr{A}$,它被$\mathscr{A}_1$有限生成,并且有$X$是$Y$同构于$\mathrm{Proj}\mathscr{A}$的.
	\end{enumerate}

    如果我们只考虑有限生成自由模层,也即$X$是$Y$同构于某个$\mathbb{P}_Y^n$的闭子概型,就称态射是H-射影态射(此为Hartshorne上定义的射影态射),如果把闭子概型改为子概型则称为H-拟射影态射.于是H-(拟)射影态射一定是(拟)射影态射.
    \item 闭嵌入本身是平凡的H-射影态射,因为我们可以取$n=0$:
    $$\xymatrix{X\ar[r]&\mathbb{P}^0_Y\ar@{=}[r]&Y}$$
    \item 关于复合.H-射影态射的复合是H-射影态射;如果$X\to Y$和$Y\to Z$是射影态射,并且$Z$是qcqs概形,那么它们的复合是射影态射.
    \begin{proof}
    	
    	设$X\to Y$和$Y\to S$都是H-射影态射.于是有态射的分解$X\to\mathbb{P}^m_Y\to Y$和$Y\to\mathbb{P}^n_S\to S$.其中左侧的态射都是闭嵌入,右侧的态射都是纤维积的典范投影映射.我们有:
    	$$(\mathbb{P}^m_S\times_S\mathbb{P}^n_S)\times_{\mathbb{P}_S^n}Y\cong\mathbb{P}_S^m\times_SY\cong\mathbb{P}^m_Y$$
    	
    	于是有如下交换图表,其中右上角是笛卡尔的:
    	$$\xymatrix{X\ar[dr]\ar[r]&\mathbb{P}_Y^m\ar[r]\ar[d]&\mathbb{P}_S^m\times_S\mathbb{P}_S^n\ar[d]\\&Y\ar[r]\ar[dr]&\mathbb{P}_S^n\ar[d]\\&&S}$$
    	
    	最后按照Segre嵌入把$\mathbb{P}_S^m\times_S\mathbb{P}_S^n$再嵌入到$\mathbb{P}_S^{(m+1)(n+1)-1}$即可.
    	
    	\qquad
    	
    	至于第二件事是因为此时射影态射是紧合的拟射影态射.
    \end{proof}
    \item 关于基变换.H-射影态射和射影态射都在基变换下不变.
    \begin{proof}
    	
    	因为闭嵌入的提升是闭嵌入,射影空间的结构态射的提升还是结构态射:
    	$$\xymatrix{X\times_YS\ar[r]\ar[d]&\mathbb{P}_S^n\ar[r]\ar[d]&S\ar[d]\\X\ar[r]&\mathbb{P}_Y^n\ar[r]&Y}$$
    \end{proof}
    \item 射影态射是拟射影的紧合态射,反过来如果$Y$是qcqs概形,那么拟射影紧合态射$X\to Y$总是射影态射;H-射影态射等价于H-拟射影的紧合态射.
    \begin{proof}
    	
    	射影态射是一个闭嵌入和一个结构态射的复合,我们解释过紧合态射是保复合的,并且闭嵌入是紧合态射,所以归结为证明结构态射是紧合态射.另外紧合态射在基变换下不变,所以归结为证明结构态射$\mathbb{P}_{\mathbb{Z}}^n\to\mathrm{Spec}\mathbb{Z}$是紧合态射,即验证它是有限型,分离,泛闭的态射.
    	
    	有限型:$D_+(x_i)\cong\mathrm{Spec}\mathbb{Z}[x_0,\cdots,x_n]_{(x_i)},0\le i\le n$构成了$\mathbb{P}_{\mathbb{Z}}^n$的仿射开覆盖,其中$\mathbb{Z}[x_0,\cdots,x_n]_{(x_i)}\cong\mathbb{Z}[x_0/x_i,\cdots,x_n/x_i]$是有限生成$\mathbb{Z}$代数,所以这个结构态射是有限型的.
    	
    	分离:我们要证明$\mathbb{P}_{\mathbb{Z}}^n\to\mathbb{P}_{\mathbb{Z}}^n\times\mathbb{P}_{\mathbb{Z}}^n$是闭嵌入.用$\{p^{-1}(D_+(x_i))\cap q^{-1}(D_+(x_j))\mid 0\le i,j\le n\}$覆盖整个$\mathbb{P}_{\mathbb{Z}}^n\times\mathbb{P}_{\mathbb{Z}}^n$.我们有:
    	$$\Delta^{-1}(p^{-1}(D_+(x_i))\cap q^{-1}(D_+(x_j)))=D_+(x_i)\cap D_+(x_j)=D_+(x_ix_j)\cong\mathrm{Spec}\mathbb{Z}[x_0,\cdots,x_n]_{(x_ix_j)}$$
    	$$p^{-1}(D_+(x_i))\cap q^{-1}(D_+(x_j))\cong D_+(x_i)\times D_+(x_j)\cong\mathrm{Spec}\left(\mathbb{Z}[x_0,\cdots,x_n]_{(x_i)}\otimes_{\mathbb{Z}}\mathbb{Z}[x_0,\cdots,x_n]_{(x_j)}\right)$$
    	
    	于是对角态射限制在$\Delta^{-1}(p^{-1}(D_+(x_i))\cap q^{-1}(D_+(x_j)))\to p^{-1}(D_+(x_i))\cap q^{-1}(D_+(x_j))$上是被如下满的环同态所诱导的,这说明对角态射是闭嵌入,所以结构态射是分离的.
    	$$\mathbb{Z}[x_0,\cdots,x_n]_{(x_i)}\otimes_{\mathbb{Z}}\mathbb{Z}[x_0,\cdots,x_n]_{(x_j)}\to\mathbb{Z}[x_0,\cdots,x_n]_{(x_ix_j)}$$
    	$$\frac{x_k}{x_i}\otimes\frac{x_l}{x_j}\mapsto\frac{x_kx_l}{x_ix_j}$$
    	
    	泛闭:需要证明对任意概形$Y$,结构态射$\pi:\mathbb{P}^n_Y\to Y$是闭映射.但是闭集是一个局部性质,所以归结为$Y$是仿射的情况,设$Y=\mathrm{Spec}A$.设$B=A[T_0,\cdots,T_n]$,取$\mathbb{P}^n_Y$中的闭子集$V_+(I)$,需要验证$Y-\pi(V_+(I))$是开集.任取$y\in Y$,我们有:
    	$$V_+(I)\cap\pi^{-1}(y)=\mathrm{Proj}(B/I)\times_{\mathrm{Spec}A}\mathrm{Spec}(B\otimes_A\kappa(y))=\mathrm{Proj}(B/I\otimes_A\kappa(y))=V_+(I\otimes_A\kappa(y))$$
    	
    	于是$y\in Y-\pi(V_+(I))$等价于讲$\pi^{-1}(y)\cap V_+(I)$是空集,等价于讲$B_+\otimes_A\kappa(y)\subseteq\sqrt{I\otimes_A\kappa(y)}$(因为$V_+(I)\subseteq V_+(J)$当且仅当$J\cap B_+\subseteq\sqrt{I}$).而这等价于讲存在某个$m$使得$B_m\otimes_A\kappa(y)\subseteq I\otimes_A\kappa(y)$.也等价于讲$(B/I)_m\otimes_A\kappa(y)=0$.
    	
    	现在选取$y\in Y-\pi(V_+(I))$,那么存在正整数$m$使得$(B/I)_m\otimes_A\kappa(y)=0$.这里$(B/I)_m$是有限$A$模,所以NAK引理导致$(B/I)_m\otimes_A\mathscr{O}_{Y,y}=0$.把$(B/I)_m$记作$M$,它是一个有限$A$模,这里$\mathscr{O}_{Y,y}$可表示为$A$的一个局部化$A_p$.于是$M\otimes_AA_p=M_p=0$.设$M$被$m_1,\cdots,m_r$生成,按照$m_i/1=0$说明存在$s_i\in A-p$使得$s_im_i=0$,那么$f=s_1s_2\cdots s_r\not=0$,并且有$fM=0$.于是$(B/I)_m\otimes_AA_f=0$,这说明$y\in D(f)$的每个点$y_0$都满足$(B/I)_m\otimes_A\mathscr{O}_{Y,y_0}=0$,所以$D(f)\subseteq Y-\pi(V_+(I))$,这说明$Y-\pi(V_+(I))$是开集.
    \end{proof}
    \item Chow引理.设$S$是qcqs概形,设$f:X\to S$是有限型分离态射,设$X$只有有限个不可约分支,那么存在$S$概形$g:X'\to S$和$S$态射$\pi:X'\to X$,满足$g$是拟射影态射,$\pi$是射影态射和满射.并且存在拟紧稠密开集$U\subseteq X$使得$\pi^{-1}(U)\subseteq X'$也是拟紧稠密开集,并且$\pi$可限制为同构$\pi^{-1}(U)\cong U$.还满足:
    \begin{enumerate}
    	\item $f$是紧合态射当且仅当$g$是射影态射.
    	\item 如果$X$是既约/不可约/整概形,那么$X'$可以选为既约/不可约/整概形.
    \end{enumerate}
    $$\xymatrix{X\ar[dr]_f&&X'\ar[ll]_{\pi}\ar[dl]^g\\&S&}$$
    \begin{proof}
    	
    	第一步,约化为$X$是不可约的情况.设$\{X_i\}$是$X$的全部有限个不可约分支,我们先来对$X_i$赋予闭子概型结构.因为这里不可约分支只有有限个,所以存在$X$的非空拟紧开子集$V_i\subseteq X_i$,使得$V_i$和其余$X_j,j\not=i$都是不交的.把$V_i$视为$X$的开子概型,按照$X$是拟分离的并且$V_i$也是拟紧的,对$X$的拟紧开子集$U$就有$U\cap V_i$是拟紧的,于是$V_i\to X$是拟紧态射.它的概形像是$X$的闭子概型,并且底空间是$\overline{V_i}=X_i$,就赋予$X_i$这个闭子概型结构.另外如果$X$本身是既约的,那么$V_i$是既约的,此时按照概形像的性质,就有$X_i$所赋予的闭子概型结构也是既约的.
    	
    	\qquad
    	
    	假设我们证明了存在不可约$S$概形$X_i'$以及态射$\pi_i:X_i'\to X_i$,以及稠密开集$U_i\subseteq X_i$满足结论中的一系列性质.设$X'$是所有$X_i'$的无交并,那么$X'$也是$S$上的拟射影态射(因为$\mathbb{P}_S^{n_1}\coprod\mathbb{P}_S^{n_2}$可以嵌入到$\mathbb{P}_S^{n_1+n_2+1}$中).粘合的$\pi:X'\to X$依旧是满射,它是射影态射是因为$X_i\to X$是闭嵌入从而是射影态射,所以射影态射的复合$X_i'\to X_i\to X$也是射影的,进而无交并粘合得到的$X'\to X$也是射影的.另外$U=\cup_i(U_i\cap V_i)$依旧是$X'$的稠密开集,并且$\pi$限制在$\pi^{-1}(U_i\cap V_i)$上依旧是同构.综上我们归结为设$X$本身是不可约的.
    	
    	\qquad
    	
    	第二步,构造$\pi$.取$S$的有限仿射开覆盖$\{S_j\}$,因为$X$也是拟紧的,可以取$X$的有限仿射开覆盖$\{X_1,\cdots,X_n\}$,使得每个$X_i$在$f$下被打到某个$S_j$内.那么$X_k\to S_j$是仿射和有限型态射,进而是拟射影态射(因为比方说$B=A[T_1,\cdots,T_n]/I$是有限型$A$代数,那么有$\mathrm{Spec}B\to\mathbb{A}_A^n\to\mathbb{P}_A^n$,前者是闭嵌入后者是开嵌入).另外由于$S$是分离的,得到$S_j\to S$是拟紧开嵌入于是它是拟射影态射【】.于是$X_k\to S$作为射影态射的复合也是射影态射.于是存在射影$S$概形$P_k$以及开嵌入$r_k:X_k\to P_k$(因为如果$X\to Y$是嵌入,那么它一定能分解为$X\to Z\to Y$,其中前者是开嵌入,后者是闭嵌入).下面设$U=\cap_kX_k$,因为$X$是不可约的,所以$U$是$X$的非空稠密开集.于是所有$r_k$限制在$U$上得到了一个拟紧【】
    	
    	
    \end{proof}
\end{enumerate}
\newpage
\section{爆破}
\subsection{爆破}

设$X$是概形,设$Z$是它的闭子概型,$X$的沿$Z$的爆破(blowup)指的是一个概形$\widetilde{X}$和一个态射$\pi:\widetilde{X}\to X$,使得$\pi^{-1}(Z)=Z\times_X\widetilde{X}$是$\widetilde{X}$的有效Cartier除子,并且它满足如下泛性质:倘若$\pi:\widetilde{X}'\to X$是任意态射使得$\pi'^{-1}(Z)$是有效Cartier除子,那么$\pi'$要经$\pi$唯一分解.如果$\pi:\widetilde{X}\to X$是$X$沿闭子概型$Z$的爆破,我们记$\widetilde{X}$为$\mathrm{Bl}_Z(X)$,称有效Cartier除子$\pi^{-1}(Z)$为爆破的异常除子(exceptional divisor),称$Z$是该爆破的中心(center).
\begin{enumerate}
	\item 回顾一下有效Cartier除子.按照定义它是可以表示为$(U_i,f_i)$,其中$f_i\in\mathscr{O}_X(U_i)$是正则元的Cartier除子.于是有效Carier除子一一对应于局部上被单个正则元生成的可逆模层,进而一一对应于余维数1的正则闭子概型.另外按照我们的定义,空集也是有效Cartier除子.
	\item 泛性质描述说明爆破如果存在则在同构意义下唯一.
	\item 如果$Z$本身就是$X$的有效Cartier除子(这包含了$Z$是空集的情况),则$1_X:X\to X$已经是爆破,换句话讲$Z$是$X$的有效Cartier除子当且仅当$X$沿$Z$的爆破$\mathrm{Bl}_Z(X)\to X$是同构.
	\item 函子性.设$f:X'\to X$是概形的态射,设$Z$是$X$的闭子概型,那么$Z'=f^{-1}(Z)=Z\times_XX'$是$X'$的闭子概型.我们断言存在唯一的态射$\mathrm{Bl}_Z(f):\mathrm{Bl}_{Z'}(X')\to\mathrm{Bl}_Z(X)$使得如下图表交换.并且如果$f$是平坦态射,则该图表是纤维积图表.
	$$\xymatrix{\mathrm{Bl}_{Z'}(X')\ar[rr]^{\mathrm{Bl}_Z(f)}\ar[d]_{\pi'}&&\mathrm{Bl}_Z(X)\ar[d]^{\pi}\\X'\ar[rr]^f&&X}$$
	\begin{proof}
		
		因为$(\pi'\circ f)^{-1}(Z)$是$\mathrm{Bl}_{Z'}(X')$的有效Cartier除子,于是$\pi:\mathrm{Bl}_Z(X)\to X$的泛性质告诉我们存在唯一的态射$\mathrm{Bl}_Z(f)$使得图表交换.
		
		\qquad
		
		下面设$(\widetilde{X}'=\mathrm{Bl}_Z(X)\times_XX',p,q)$是$f$和$\pi$的纤维积.设$f$是平坦态射.按照纤维积的泛性质,存在态射$r:\mathrm{Bl}_{Z'}(X')\to\widetilde{X}'$使得如下图表交换:
		$$\xymatrix{\mathrm{Bl}_{Z'}(X')\ar@/^1pc/[drr]^{\mathrm{Bl}_Z(f)}\ar[dr]^r\ar@/_1pc/[ddr]_{\pi'}&&\\&\widetilde{X}'\ar[r]^p\ar[d]_q&\mathrm{Bl}_Z(X)\ar[d]^{\pi}\\&X'\ar[r]^f&X}$$
		
		我们再断言存在虚线态射使得如下图表交换.设$E=\pi^{-1}(Z)$是异常除子.因为$f$是平坦态射,得到$p$也是平坦态射,我们解释过平坦态射下有效除子的回拉还是有效除子,于是$E'=p^{-1}(E)$是$\widetilde{X}'$的有效除子.但是按照实线图表的交换性,我们有$q^{-1}(Z')=q^{-1}\circ f^{-1}(Z)=p^{-1}\circ\pi^{-1}(Z)=E'$.于是按照$\pi'$的泛性质,就存在虚线态射$r'$满足$q=\pi'\circ r'$.进而有$\pi\circ\mathrm{Bl}_Z(f)\circ r'=\pi\circ p=h$,但是这个$h$满足$h^{-1}(Z)=E'$是有效除子,所以按照$\pi$的泛性质$h$要经$\pi$唯一分解,也即$p=\mathrm{Bl}_Z(f)\circ r'$.于是$r'$使得如下图表交换.
		$$\xymatrix{\widetilde{X}'\ar@/^1pc/[drr]^p\ar@{-->}[dr]^{r'}\ar@/_1pc/[ddr]_q&&\\&\mathrm{Bl}_{Z'}(X')\ar[r]^{\mathrm{Bl}_Z(f)}\ar[d]_{\pi'}&\mathrm{Bl}_Z(X)\ar[d]^{\pi}\\&X'\ar[r]^f&X}$$
		
		最后按照纤维积的泛性质我们有$r\circ r'=1_{\widetilde{X}'}$.按照爆破的泛性质我们有$r'\circ r=1_{\mathrm{Bl}_{Z'}(X')}$.于是有$\mathrm{Bl}_{Z'}(X')\cong\widetilde{X}'=\mathrm{Bl}_Z(X)\times_XX'$.
	\end{proof}
	\item 设$Z$是$X$的闭子概型,设爆破为$\pi:\mathrm{Bl}_Z(X)\to X$.设$U=X-Z$是开子概型,那么$\pi$限制为$\pi^{-1}(U)\to U$是一个同构.
	\begin{proof}
		
		设$i:U\to X$是开嵌入,那么$Z'=i^{-1}(Z)$是空集,于是$\mathrm{Bl}_{Z'}(U)=U$.又因为开嵌入是平坦的,于是上一条说明如下图表是纤维积图表.于是$U\cong U\times_X\mathrm{Bl}_Z(X)=\pi^{-1}(U)$,于是$\pi$限制为$\pi^{-1}(U)\to U$是同构.
		$$\xymatrix{U\ar[rr]\ar@{=}[d]&&\mathrm{Bl}_Z(X)\ar[d]^{\pi}\\U\ar[rr]^i&&X}$$
	\end{proof}
	\item 设$Z$是$X$的闭子概型,设对应的拟凝聚理想层为$\mathscr{I}$,如果对$X$的任意仿射开子集$V$,都要$\Gamma(V,\mathscr{I})$包含了$\mathscr{O}_X(V)$的正则元(例如如果$X$是整概形,而$\mathscr{I}\not=0$满足这个条件),那么$X$沿$Z$的爆破$\pi:\mathrm{Bl}_Z(X)\to X$是双有理的.
	\begin{proof}
		
		上一条已经解释了如果记$U=X-Z$,那么总有$\pi$可以限制为同构$\pi^{-1}(U)\to U$,这里只需说明$U$和$\pi^{-1}(U)$都是概形稠密开集.我们解释过如果$U\subseteq X$是开集,对任意仿射开集$V\subseteq X$,如果$U\cap V$总包含了$V$上的主开集$D(t)$,其中$t$是$V$上的正则元,那么$U$就是$X$的概形稠密开集.回到我们这里的命题,由于$\pi^{-1}(U)$本身是一个正则闭子概型$\pi^{-1}(Z)$的补集,所以仿射局部上$\pi^{-1}(U)$总包含了正则元对应的主开集,这导致$\pi^{-1}(U)$是概形稠密开集.而我们添加的条件就是要求了$U$在仿射局部上总包含了一个正则元对应的主开集,这导致$U$也是概形稠密开集.
	\end{proof}
\end{enumerate}
\subsection{爆破的具体构造}
\begin{enumerate}
	\item 设$X$是概形,设$Z$是闭子概型,设对应的拟凝聚理想层为$\mathscr{I}$,那么$\mathrm{Proj}\oplus_{d\ge0}\mathscr{I}^d$是$X$沿$Z$的爆破(当然$\mathscr{I}^0$约定为$\mathscr{O}_X$).
	\begin{proof}
		
		记$\mathscr{B}=\oplus_{d\ge0}\mathscr{I}^d$,这是一个分次拟凝聚$\mathscr{O}_X$代数层,并且被$\mathscr{B}_1=\mathscr{I}$生成.记$\widetilde{X}=\mathrm{Proj}\mathscr{B}$.记$\pi:\widetilde{X}\to X$是结构态射.我们来证明$\pi$是$X$沿$Z$的爆破.
		
		\qquad
		
		问题是局部的,不妨设$X=\mathrm{Spec}A$是仿射的.那么$Z=V(I)$,其中$I\subseteq A$是理想,并且有$\widetilde{I}=\mathscr{I}$和$B=\oplus_{d\ge0}B_d$,其中$B_d=I^d$,还有$\widetilde{B}=\mathscr{B}$.我们来描述$\widetilde{X}=\mathrm{Proj}B$在局部上的情况.设$f\in I$,记$A[If^{-1}]$表示$A_f$的由$\{x/f\mid x\in I\}$生成的$A$子代数.但是这里$x,f\in I$都是$B$的1次齐次元,于是$x/f$可视为$B_{(f)}$中的元.于是我们得到了$A$代数同态$A[If^{-1}]\to B_{(f)}$,它是一个同构.于是当$f$跑遍$I$的生成元集时有$D_+(f)=\mathrm{Spec}A[If^{-1}]$构成了$\widetilde{X}$的仿射开覆盖.因为$A[If^{-1}]$的理想$IA[If^{-1}]$是被$f/1$生成的,并且$f/1$是$A[If^{-1}]$的正则元.导致$V_A(I)$在$\widetilde{X}=\mathrm{Spec}B$上的回拉的确是一个有效除子.
		
		\qquad
		
		现在设$\pi':\widetilde{X}'\to\mathrm{Spec}A$是态射,使得$V_A(I)$在$\widetilde{X}'$上的回拉也是有效除子.我们要证明的是$\pi'$要经$\pi$唯一分解.我们知道$\widetilde{X}'$存在仿射开覆盖,使得$V_A(I)$在上面的回拉被单个正则元定义.不妨设$\mathrm{Spec}C$是满足这个条件的$\widetilde{X}'$的仿射开子集.设$\mathrm{Spec}C\to\widetilde{X}'\to\mathrm{Spec}A$对应的环同态是$\varphi:A\to C$.那么它满足$\varphi(I)C$被单个正则元$\varphi(f)$生成.我们要证明的是$\varphi:A\to C$要经$A\to B$唯一分解,由于$\varphi(f)$是$C$的正则元,有$D_C(\varphi(f))$是$\mathrm{Spec}C$的概形稠密开集.于是为说明$\mathrm{Spec}C\to\mathrm{Spec}A$唯一的经一个态射$\mathrm{Spec}C\to\mathrm{Proj}B$分解,只需验证$D_C(\varphi(f))\to\mathrm{Spec}A$唯一的经一个态射$D_C(\varphi(f))\to\mathrm{Proj}B$分解(因为$U\subseteq X$是概形稠密开集的一个推论是,如果$Y$是分离概形,如果$f,g:X\to Y$限制在$U$上一致,那么$f=g$,而这里$\mathrm{Proj}B$是分离的).归结为证明存在唯一的$\mathrm{Spec}A$态射$\mathrm{Spec}C_{\varphi(f)}\to\mathrm{Spec}B_{(f)}=\mathrm{Spec}A[If^{-1}]$.也即证明存在唯一的$A$代数同态$\psi:A[If^{-1}]\to C_{\varphi(f)}$.而这是因为对$x\in I$有$\psi(x/f)=\varphi(x)/\varphi(f)$,按照$\varphi(f)$是正则元我们知道存在唯一的$c\in C$使得$\varphi(f)c=\varphi(x)$.于是我们证明了在$\widetilde{X}'$的仿射开集上$\pi'$总要经$\pi$唯一分解,于是$\pi'$要经$\pi$唯一分解,证毕.
	\end{proof}
	\item 设$i:Z\to X$的闭子概型,对应的拟凝聚理想层为$\mathscr{I}$,记$X$沿$Z$的爆破是$\pi:\mathrm{Bl}_Z(X)\to X$,那么异常除子为$\mathrm{Proj}\oplus_{n\ge0}\mathscr{I}^n/\mathscr{I}^{n+1}$.
	\begin{proof}
		
		我们解释过如果$g:S'\to S$是态射,如果$\mathscr{A}$是$S$上的分次拟凝聚代数层,那么有$\mathrm{Proj}(g^*\mathscr{A})\cong(\mathrm{Proj}\mathscr{A})\times_SS'$.于是有:
		\begin{align*}
			\pi^{-1}(Z)&=Z\times_X\mathrm{Bl}_Z(X)\\&=Z\times_X\mathrm{Proj}\oplus_{n\ge0}\mathscr{I}^n\\&=\mathrm{Proj}i^*\oplus_{n\ge0}\mathrm{I}^n\\&=\mathrm{Proj}\oplus_{n\ge0}\mathscr{I}^n\otimes\mathscr{O}_X/\mathscr{I}\\&=\mathrm{Proj}\oplus_{n\ge0}\mathscr{I}^n/\mathscr{I}^{n+1}
		\end{align*}
	\end{proof}
	\item 例子.设$R$是环,我们来求$\mathbb{A}_R^n$在原点(闭点视为闭子概型)的爆破.记$A=R[T_1,\cdots,T_n]$和$I=(T_1,\cdots,T_n)$,记$B=\oplus_{n\ge0}B_n$,其中$B_n=I^n$,约定$B_0=A$.但是要区分$B$中$B_0$的$T_i$和$B_1$中的$T_i$.我们有分次满同态$A[X_1,\cdots,X_n]\to B$为把$X_i$映为$T_i\in I=B_1$.它的核是$T_iX_j-T_jX_i,\forall 1\le i,j\le n$.于是爆破为$\widetilde{X}=V_+(T_iX_j-T_jX_i;1\le i,j\le n)\subseteq\mathbb{P}_R^{n-1}\times_R\mathbb{A}_R^n$.其中$T_i$是$\mathbb{A}_R^n$的坐标,而$X_1,\cdots,X_n$是$\mathbb{P}_R^{n-1}$的齐次坐标.
	\item 例子.设$X$是局部诺特概形,设$x\in X$是$d$维正则闭点.设$X$沿$\{x\}$的爆破为$\pi:\widetilde{X}\to X$.那么$\pi$限制在$X-\{x\}$上是同构,而异常除子为如下,这里最后一个等式是因为$(\mathscr{O}_{X,x},\mathfrak{m}_x,k)$是$d$维正则局部环导致$\oplus_{n\ge0}\mathfrak{m}_x^n/\mathfrak{m}_x^{n+1}\cong k[T_1,\cdots,T_d]$.
	$$\pi^{-1}(x)=\mathrm{Proj}\oplus_{n\ge0}\mathfrak{m}_x^n/\mathfrak{m}_x^{n+1}\cong\mathbb{P}_k^{d-1}$$
	
	特别的,如果$X$是域$k$上的有限型概形,如果$x\in X$是一个剩余域为$k$的光滑闭点,那么$\{x\}$的异常除子可以理解为$x$的切空间的射影化.去掉光滑条件也有类似结论,此时异常除子是该点切锥的射影化.
	\item 推论.设$Z$是$X$的闭子概型,设对应的拟凝聚理想层为$\mathscr{I}$,设$X$沿$Z$的爆破为$\pi:\widetilde{X}\to X$.
	\begin{enumerate}
		\item 如果$\mathscr{I}$是有限生成的理想层,那么$\pi$是射影态射(从而是紧合态射),并且$\mathscr{I}\mathscr{O}_{\widetilde{X}}=\mathscr{O}_{\widetilde{X}}(1)$是关于$\pi$的极丰沛可逆层.
		\item 设$i:Y\to Z$是一个闭子概型,那么爆破图表中的$\mathrm{Bl}_Z(i):\mathrm{Bl}_{Y\cap Z}(Y)\to\mathrm{Bl}_Z(X)$仍然是闭嵌入.这称为$Y$关于$X$沿$Z$爆破的严变换(strict transform).
		$$\xymatrix{\mathrm{Bl}_{Y\cap Z}(Y)\ar[rr]\ar[d]&&\mathrm{Bl}_Z(X)\ar[d]\\Y\ar[rr]&&X}$$
		\item 即便爆破$\pi$本身是同构(也即$Z$本身是$X$的有效除子),也未必有$Y$的严变换同构于$Y$.但是倘若$Y-Z$是$Y$的概形稠密开集,那么结论成立.
	\end{enumerate}
	\begin{proof}
		
		(b):设$Y$对应的拟凝聚理想层是$\mathscr{a}$,那么$Y\cap Z$作为$Y$的闭子概型对应的拟凝聚理想层就是$(\mathscr{I}+\mathscr{a})/\mathscr{a}$.于是$\mathrm{Bl}_Z(i)$对应于$\oplus_{n\ge0}\mathscr{I}^n\to\oplus_{n\ge0}((\mathscr{I}+\mathscr{a})/\mathscr{a})^d$,这是满同态,于是诱导的是闭嵌入.
		
		【】
		
	\end{proof}
	\item 推论.设$X$是整概形,设$Z\subsetneqq X$是非空的闭子概型,设它对应的拟凝聚理想层是有限生成的.那么$\mathrm{Bl}_Z(X)$也是整概形,并且结构态射$\pi:\mathrm{Bl}_Z(X)\to X$是双有理,射影态射和满射.
	\begin{proof}
		
		我们解释过如果记$U=X-Z$,那么$\pi$可以限制为同构$\pi^{-1}(U)\cong U$,并且由于$\pi^{-1}(U)$是有效除子$\pi^{-1}(Z)$的补集,导致它是概形稠密开集.又因为$X$是整概形,导致非空开集$U$也是概形稠密开集,这说明$\pi$是双有理的.上一条解释了此时$\pi$是射影态射,特别的$\pi$是闭映射,所以像集$\pi(\mathrm{Bl}_Z(X))$是一个包含了稠密开集$U$的闭集,迫使像集是整个$X$,这说明$\pi$是满射.
		
		\qquad
		
		最后证明$\mathrm{Bl}_Z(X)$是整概形.首先因为$\pi^{-1}(U)\cong U$以及$U$是不可约的得到$\pi^{-1}(U)$是不可约的,又因为$\pi^{-1}(U)$在$\mathrm{Bl}_Z(X)$中稠密得到$\mathrm{Bl}_Z(X)$是不可约的.设$Z$对应的拟凝聚理想层为$\mathscr{I}$,因为$X$是整概形,说明对任意仿射开子集$V\subseteq X$有$B_V=\Gamma(V,\oplus_{n\ge0}\mathscr{I}^n)$是整环.于是对任意正次齐次元$f\in B_V$就有$D_+(f)\subseteq\mathrm{Proj}\Gamma(V,\oplus_{n\ge0}\mathscr{I}^n)$是整概形,而形如$D_+(f)$的仿射开子集覆盖了整个$\mathrm{Bl}_Z(X)$,于是有$\mathrm{Bl}_Z(X)$是既约概形,结合不可约就得到$\mathrm{Bl}_Z(X)$是整概形.
	\end{proof}
\end{enumerate}
\newpage
\section{上同调}
\subsection{仿射概形的层上同调和仿射准则}
\begin{enumerate}
	\item Serre消失定理.这个定理是说如果$X$是仿射概形,$F$是$X$上的拟凝聚层,那么$\mathrm{H}^q(X,F)=0,\forall q\ge1$成立.
	\begin{enumerate}
		\item 设$\mathscr{U}=\{U_0,\cdots,U_N\}$是仿射概形$X=\mathrm{Spec}A$上的有限主开集覆盖,设$F$是$X$上的拟凝聚层.那么有:
		$$\check{\mathrm{H}}^q(\mathscr{U},F)=\left\{\begin{array}{cc}0&q\ge1\\F(X)&q=0\end{array}\right.$$
		\begin{proof}
			
			首先我们证明过对于阿贝尔层$F$有$\check{H}^0(\mathscr{U},F)=F(X)$.下面设$F=\widetilde{M}$,按照定义复形$\check{C}(\mathscr{U},F)$为:
			$$\xymatrix{\oplus_{i=0}^NF(U_i)\ar[r]&\oplus_{i<j}F(U_i\cap U_j)\ar[r]&\cdots\ar[r]&F(U_0\cap\cdots\cap U_N)\ar[r]&0}$$
			
			记$U_i=D(f_i),0\le i\le N$按照仿射概形上拟凝聚层在主开集上的限制就是局部化,我们要证明这个复形在$\ge1$处是正合的等价于讲如下$A$模复形是正合的:
			$$\xymatrix{0\ar[r]&M\ar[r]&\oplus_{i=0}^NM_{f_i}\ar[r]&\oplus_{i<j}M_{f_if_j}\ar[r]&\cdots\ar[r]&M_{f_0\cdots f_N}\ar[r]&0}$$
			
			但是按照$(f_0,f_1,\cdots,f_N)=A$,这个复形是正合的等价于讲在每个$f_i$上的局部化是正合的,但是每个$f_i$的局部化明显是正合的.
		\end{proof}
		\item 引理.这个引理可以视为\v{C}ech上同调的Cartan引理的一种特殊形式:设$(X,\mathscr{O}_X)$是环空间,设$\mathscr{B}$是由拟紧开集构成的拓扑基,满足在有限交下封闭(严格讲这不要求包含空集,约定如果有限个其中的基元素的交非空,那么这个交也是基元素),设$\mathscr{C}$是由$\textbf{Mod}(\mathscr{O}_X)$的这样的模层$F$构成的完全子空间,满足对任意$U\in\mathscr{B}$和$U$的任意的由$\mathscr{B}$中开集构成的开覆盖$\mathscr{U}$,都有$\check{H}^p(\mathscr{U},F)=0,\forall p\ge1$.那么我们断言对每个$U\in\mathscr{B}$,有$\mathscr{C}$上的函子$\Gamma(U,-)$是右适应于(right adapted to)$\mathscr{C}$的.一个加性函子$T$是右适应于阿贝尔范畴$\mathscr{A}$的完全子范畴$\mathscr{A}'$的是指满足如下三件事:
		\begin{itemize}
			\item 对任意对象$E\in\mathscr{A}$,存在对象$E'\in\mathscr{A}'$使得存在单态射$E\to E'$.
			\item 如果有$\mathscr{A}$的如下短正合列,如果$E',E\in\mathscr{A}'$,那么有$E''\in\mathscr{A}'$.(这是右适应性的右的由来).
			$$\xymatrix{0\ar[r]&E'\ar[r]&E\ar[r]&E''\ar[r]&0}$$
			\item 如果有$\mathscr{A}'$中的短正合列,那么作用$T$仍然是短正合列.
			$$\xymatrix{0\ar[r]&E'\ar[r]&E\ar[r]&E''\ar[r]&0}$$
		\end{itemize}
		
		特别的,这件事可以说明(见导出范畴)对任意$U\in\mathscr{B}$和任意$F\in\mathscr{C}$,都有$\Gamma(U,F)\to R\gamma(U,F)$是同构,于是$\mathrm{H}^q(U,F)=0,\forall q\ge1$.
		\begin{proof}
			
			我们来验证$\mathscr{C}$满足这三个条件.为此先验证如下引理:如果有$\textbf{Mod}(\mathscr{O}_X)$中的如下短正合列:
			$$\xymatrix{0\ar[r]&F'\ar[r]&F\ar[r]&F''\ar[r]&0}$$
			
			并且$F'\in\mathscr{C}$,那么对任意$U\in\mathscr{B}$都有如下短正合列:
			$$\xymatrix{0\ar[r]&\Gamma(U,F')\ar[r]&\Gamma(U,F)\ar[r]&\Gamma(U,F'')\ar[r]&0}$$
			
			引理的证明.任取$U\in\mathscr{B}$的由$\mathscr{B}$中开集构成的开覆盖$\mathscr{U}$,我们解释过尽管\v{C}ech上同调不是上同调$\delta$函子,但是零阶和一阶的确构成长正合列,但是条件说明$\check{H}^1(\mathscr{U},F)=0$,并且我们解释过对任意阿贝尔层$F$总有$\check{H}^0(\mathscr{U},F)=F(U)$,就导致有结论中的短正合列.
			
			\qquad
			
			回到原命题的证明,我们仅需要验证$\mathscr{C}$满足命题中的三个条件.第一个条件是直接的,因为我们解释过松弛层都落在$\mathscr{C}$中,并且我们解释过拟凝聚层总存在到松弛层的嵌入(因为内射模层是松弛的).第三件事按照上一段的引理直接得证.最后验证第二件事,考虑如下复形的交换图表,按照上一段引理得到第一行是短正合列,按照短正合列的直积是短正合列,第二行也是短正合列,把这个图表放在同伦范畴中,前两列是拟同构得到第三列是拟同构.这得证.
			$$\xymatrix{0\ar[r]&F'(U)\ar[r]\ar[d]&F(U)\ar[r]\ar[d]&F''(U)\ar[r]\ar[d]&0\\0\ar[r]&\check{C}(\mathscr{U},F')\ar[r]&\check{C}(\mathscr{U},F)\ar[r]&\check{C}(\mathscr{U},F'')\ar[r]&0}$$
		\end{proof}
	\end{enumerate}
    \item 一些推论.
    \begin{enumerate}
    	\item 设$X$是分离概形,$\mathscr{F}$是其上层,那么对$X$的任意仿射开覆盖$\mathscr{U}$,都有$\check{\mathrm{H}}^p(\mathscr{U},\mathscr{F})\cong\mathrm{H}^p(X,\mathscr{F}),\forall p$成立.
    	\begin{proof}
    		
    		分离条件保证了$\mathscr{U}$中任意有限个开集$\{U_1,\cdots,U_n\}$的交仍然是仿射的.于是按照本节的定理就有$\check{H}^p(U_1\cap\cdots\cap U_n,\mathscr{F})=0$.但是按照\v{C}ech上同调的Leray定理,就说明结论成立.
    	\end{proof}
    	\item 设$f:X\to Y$是仿射态射,设$\mathscr{F}$是拟凝聚$\mathscr{O}_X$模层,那么对任意$p\ge1$,高阶前推函子满足$R^pf_*\mathscr{F}=0$.
    	\begin{proof}
    		
    		我们解释过层$R^pf_*\mathscr{F}$就是预层$V\mapsto\mathrm{H}^p(f^{-1}(V),\mathscr{F})$的层化.但是对仿射的$V\subseteq Y$,有$f^{-1}(V)$也是仿射的,按照本节的定理就有$\mathrm{H}^p(f^{-1}(V),\mathscr{F})=0,p\ge1$.最后仿射开子集构成了$Y$的一个开覆盖,就导致$R^pf_*\mathscr{F}=0$.
    	\end{proof}
    \end{enumerate}
    \item Serre仿射准则.设$X$是拟紧概形,那么如下命题互相等价(这里$X$是拟紧就够了,尽管其它书上包括EGA上都加上了分离条件或者甚至诺特的).
    \begin{enumerate}
    	\item $X$是仿射概形.
    	\item 对$X$上任意拟凝聚层$\mathscr{F}$,有$\mathrm{H}^p(X,\mathscr{F})=0$对任意$p\ge1$成立.
    	\item 对$X$上任意拟凝聚层$\mathscr{F}$,有$\mathrm{H}^1(X,\mathscr{F})=0$.
    	\item 对$X$上任意拟凝聚理想层$\mathscr{I}$,有$\mathrm{H}^1(X,\mathscr{I})=0$.
    \end{enumerate}
    \begin{proof}
    	
    	(a)推(b)是Serre消失定理,(b)推(c)推(d)是平凡的.最后证明(d)推(a).按照仿射准则,归结为证明存在$X$的形如$X_{f_1},\cdots,X_{f_n}$的仿射开覆盖.按照拟紧条件,归结为证明对任意点$x\in X$,有形如$X_f$的仿射开邻域.考虑闭子集$\overline{\{x\}}$,它是拟紧集的闭子集,所以仍然是拟紧的.我们解释过拟紧的$T_0$空间总有闭点,记作$y$,如果我们能证明$y$存在形如$X_f$的仿射开邻域,那么按照$y\in\overline{\{x\}}$得到$x\in X_f$.于是问题归结为设$x$本身是一个闭点.考虑$\{x\}$作为闭子集的既约闭子概型,它的理想层记作$\mathscr{I}$.任取$x$的仿射开邻域$U$,考虑$X-U$上的既约闭子概型,理想层记作$\mathscr{J}$.于是我们有拟凝聚层的短正合列$0\to\mathscr{I}\mathscr{J}\to\mathscr{J}\to\mathscr{J}/\mathscr{I}\mathscr{J}\to0$.只要$y\not=x$,就有$(\mathscr{J}/\mathscr{I}\mathscr{J})_y=0$.于是$\mathscr{J}/\mathscr{I}\mathscr{J}$是一个摩天大楼层,并且它在点$x$的stalk是$\mathscr{O}_{X,x}/\mathfrak{m}_x=\kappa(x)$.按照条件有$\mathrm{H}^1(X,\mathscr{I}\mathscr{J})=0$,于是考虑整体截面诱导的长正合列有$\mathscr{J}(X)\to(\mathscr{J}/\mathscr{I}\mathscr{J})(X)=\kappa(x)$是满射.于是存在$f\in\mathscr{J}(X)$满足$f_x\not\in\mathfrak{m}_x$.于是我们得到$x\in X_f\subseteq U$.于是有$X_f=D(f\mid_U)$,这得到$X_f$是仿射的.
    \end{proof}
    \item Chevalley仿射准则.
    \begin{enumerate}
    	\item 引理.设$X$是概形,设$X_0$是$X$的闭子概型,并且它们具有相同的底空间,设$X_0$对应的拟凝聚理想层$\mathscr{I}$是幂零的(比方说,如果$X$是诺特概形,那么幂零层$\mathscr{N}_X$就是幂零的,而$\mathscr{I}\subseteq\mathscr{N}_X$,于是此时$\mathscr{I}$是幂零的),那么$X$是仿射概形当且仅当$X_0$是仿射概形.
    	\begin{proof}
    		
    		如果$X$是仿射概形,那么它的闭子概型$X_0$自然也是仿射的.下面设$X_0$是仿射概形,设正整数$n$使得$\mathscr{I}^n=0$.我们定义$X_k$是被$\mathscr{I}^{k+1}$定义的闭子概型,那么$X_k$的底空间仍然是$X$的底空间.于是我们得到闭子概型链$X_0\le X_1\le\cdots\le X_n=X$,前一个总是后一个的闭子概型,并且$X_i$对应的拟凝聚理想层$\mathscr{I}^{k+1}$在$X_{i+1}$中满足平方为零,即$\mathscr{I}^{2k+2}$在$\mathscr{O}_X/\mathscr{I}^{k+2}$中为零,于是问题归结为设$n=2$的情况,也即$\mathscr{I}^2=0$的情况.
    		
    		\qquad
    		
    		因为$X$和$X_0$具有相同的底空间,于是从$X_0$是拟紧的得到$X$是拟紧的(这里$X$也是分离的,因为我们解释过$f$是分离态射当且仅当$f_{\mathrm{red}}$是分离态射,于是$X_0$分离得到$X_{\mathrm{red}}$分离,进而得到$X$是分离的,但是要用Serre仿射准则其实并不需要$X$是分离的).于是按照Serre仿射准则,归结为证明对任意拟凝聚$\mathscr{O}_X$模层$\mathscr{F}$都有$\mathrm{H}^1(X,\mathscr{F})=0$.因为层上同调实际上只依赖于拓扑空间,所以如果$\mathscr{G}$同时是$X$和$X_0$的拟凝聚模层,那么就有$\mathrm{H}^1(X,\mathscr{F})=\mathrm{H}^1(X_0,\mathscr{F})=0$.考虑如下短正合列:
    		$$\xymatrix{0\ar[r]&\mathscr{I}\mathscr{F}\ar[r]&\mathscr{F}\ar[r]&\mathscr{F}/\mathscr{I}\mathscr{F}\ar[r]&0}$$
    		
    		它诱导的长正合列的一部分为:
    		$$\xymatrix{\mathrm{H}^1(X,\mathscr{I}\mathscr{F})\ar[r]&\mathrm{H}^1(X,\mathscr{F})\ar[r]&\mathrm{H}^1(X,\mathscr{F}/\mathscr{I}\mathscr{F})}$$
    		
    		由于这里$\mathscr{I}\mathscr{F}$和$\mathscr{F}/\mathscr{I}\mathscr{F}$都被$\mathscr{I}$零化,所以它们都是$\mathscr{O}_X/\mathscr{I}$拟凝聚模层,于是从$X_0$是仿射的和Serre仿射消失定理得到$\mathrm{H}^1(X,\mathscr{I}\mathscr{F})=\mathrm{H}^1(X_0,\mathscr{I}\mathscr{F})=0$和$\mathrm{H}^1(X,\mathscr{F}/\mathscr{I}\mathscr{F}=\mathrm{H}^1(X_0,\mathscr{F}/\mathscr{I}\mathscr{F})=0$,这就说明$\mathrm{H}^1(X,\mathscr{F})=0$.于是$X$是仿射概形.
    	\end{proof}
        \item 诺特版本的Chevalley仿射准则.设$Y$是诺特概形,设$f:X\to Y$是有限态射也是满射,那么$X$是仿射概形当且仅当$Y$是仿射概形.
        \begin{proof}
        	
        	因为有限态射都是仿射态射,所以如果$Y$是仿射的自然有$X$是仿射的.反过来设$X$是仿射的.如果$Y'$是$Y$的闭子概型,那么基变换$f_{Y'}:f^{-1}(Y')\to Y'$也是有限态射和满射,并且$f^{-1}(Y')$作为闭嵌入的基变换也是仿射概形$X$的闭子概型,从而它也是仿射的.于是按照$Y$是诺特概形,按照诺特归纳原理,我们可以约定$Y$的每个真闭子概型都是仿射的,证明$Y$是仿射的.另外我们可以约定$X$是既约的,因为$X_{\mathrm{red}}\to X$仍然是有限态射和满射.那么$f:X\to Y$的像集落在闭子概型$Y_{\mathrm{red}}$中,我们解释过这个条件下$f$就要经$Y_{\mathrm{red}}$分解.那么此时$f_{\mathrm{red}}:X\to Y_{\mathrm{red}}$仍然是有限态射和满射(满射不用说,仍然是有限态射是因为我们解释过如果一个关于态射的性质P在复合与基变换下不变,并且闭嵌入满足P,那么$f$满足P总可以推出$f_{\mathrm{red}}$满足P).并且上一条引理告诉我们$Y$是仿射概形当且仅当$Y_{\mathrm{red}}$是仿射概形.于是不妨设$Y$也是既约的.按照Serre仿射准则(以及$Y$是诺特的),我们只需证明对$\mathscr{O}_Y$的每个凝聚理想层$\mathscr{F}$都有$\mathrm{H}^1(Y,\mathscr{F})=0$.
        	
        	\qquad
        	
        	这一段我们证明归结为约定$\mathscr{F}$的支集是整个$Y$.因为倘若$\mathscr{F}$是凝聚$\mathscr{O}_Y$层使得$\mathrm{Supp}\mathscr{F}$是$Y$的真闭子集,我们取$Y'$是零化子$\mathrm{Ann}(\mathscr{F})$对应的闭子概型(我们解释过这个闭子概型的底空间就是$\mathrm{Supp}(\mathscr{F})$,并且这些事实的前提是$\mathscr{F}$是有限生成的拟凝聚层).那么按照归纳假设有$Y'$是仿射的.设$i:Y'\to Y$是典范闭嵌入,我们解释过有$\mathscr{F}\cong i_*i^*\mathscr{F}$.于是我们有(这里倒数第二个等式是因为$i$是仿射态射,我们解释过对仿射态射总有这个结论):
        	$$\mathrm{H}^1(Y,\mathscr{F})=\mathrm{H}^1(Y,i_*i^*\mathscr{F})=\mathrm{H}^1(Y',i^*\mathscr{F})=0$$
        	
        	这一段我们证明归结为约定$Y$是不可约的,于是结合我们约定$Y$是既约的就得到可约定$Y$是整概形.倘若$Y$是可约空间,取一个不可约分支$Z$,赋予它整闭子概型,设$Z$的一般点为$\eta$,设$i:Z\to Y$是典范闭嵌入,我们有$\mathscr{O}_Y$模层的态射$u:\mathscr{F}\to i_*i^*\mathscr{F}$(它是从$i^*\mathscr{F}$上的恒等态射经$i_*$与$i^*$伴随性诱导的态射).于是$\ker u$和$\mathrm{im}u$都是凝聚层.于是我们得到正合列$\mathrm{H}^1(Y,\ker u)\to\mathrm{H}^1(Y,\mathscr{F})\to\mathrm{H}^1(Y,\mathrm{im}u)$.我们断言$\ker u$和$\mathrm{im}u$的支集都是$Y$的真闭子集,于是从上一段就得到它们在$Y$上的1阶层上同调都是零,进而从上面正合列得到$\mathrm{H}^1(Y,\mathscr{F})=0$.首先$\mathrm{Supp}\mathrm{im}u\subsetneqq Y$是简单的,因为$\mathrm{Supp}\mathrm{im}u\subseteq\mathrm{Supp}i_*i^*\mathscr{F}\subseteq Z$.我们断言$u_{\eta}$是同构,导致$\eta\not\in\mathrm{Supp}\ker u$说明$\mathrm{Supp}\ker u$是$Y$的真闭子集.首先设$\mathscr{G}$是$Z$上的凝聚层,那么当$x\not\in Z$时有$(i_*\mathscr{G})_x=0$,当$x\in Z$时有$(i_*\mathscr{G})_x=\mathscr{G}_x$.另外对$z\in Z$我们有$(i^*\mathscr{F})_z\cong\mathscr{O}_{Z,z}\otimes_{\mathscr{O}_{Y,z}}\mathscr{F}_z$.如果取$z=\eta$,取$\eta$的仿射开邻域$U=\mathrm{Spec}A$,那么$\mathscr{O}_{Y,\eta}=A_{\mathfrak{p}}$,其中$\mathfrak{p}$是$\eta$对应的$A$的极小素理想.而$Z=\mathrm{Spec}A/\mathfrak{p}$,于是$\mathscr{O}_{Z,\eta}=(A/\mathfrak{p})_{\mathfrak{p}}=A_{\mathfrak{p}}/\mathfrak{p}A_{\mathfrak{p}}$.但是由于我们约定了$Y$是既约的,所以这里$A$是既约环,导致$A_{\mathfrak{p}}$是域,也即$\mathscr{O}_{Z,\eta}=\mathscr{O}_{Y,\eta}$.综上得到$(i_*i^*\mathscr{F})_{\eta}=\mathscr{F}_{\eta}$,这说明$u_{\eta}$是同构,完成证明.
        	
        	\qquad
        	
        	这一段我们证明归结为约定$X$也是整概形.因为$f$是满射,所以存在$X$的不可约分支$X'$使得$f(X')$包含了$Y$的一般点(我们已经约定了$Y$是整概形),但是$f$是有限态射导致它是闭映射(整态射已经是闭映射),于是$f(X')=Y$,于是把$f:X\to Y$替换为$f':X'\to X\to Y$仍然是有限态射和满射.于是不妨约定$X$本身是整概形.
        	
        	\qquad
        	
        	因为$f$是有限态射,导致$\mathscr{B}=f_*\mathscr{O}_X$是有限$\mathscr{O}_Y$代数层.于是如果记$\eta$是$Y$的一般点,那么$\mathscr{B}_{\eta}$就是$\mathscr{O}_{Y,\eta}=K(Y)$(此为$Y$的函数域)上的有限维线性空间.按照一般自由性,存在$\eta$的仿射开邻域$V=\mathrm{Spec}A\subseteq Y$,使得$\mathscr{B}\mid_V$是有限自由$\mathscr{O}_V$模层.如果记$U=f^{-1}(V)=\mathrm{Spec}B=\Gamma(U,\mathscr{O}_X)=\Gamma(V,\mathscr{B})$,那么$B$就是有限自由$A$模.取一组基$\{b_1,\cdots,b_n\}\subseteq B$.我们可以把$b_1,\cdots,b_n$限制在一个包含在$U$内的$X$的主开集$U'$上,如果记$X=\mathrm{Spec}C$,那么存在$g\in C$使得$gb_i\mid_{U'}$都可以延拓为$s_i\in C$.但是我们还有$C=\Gamma(Y,\mathscr{B})$,于是$s_1,\cdots,s_n$定义了一个$\mathscr{O}_Y$模层态射$u:\mathscr{O}_Y^n\to\mathscr{B}=f_*\mathscr{O}_X$.并且按照构造有$u_{\eta}:\mathscr{O}_{Y,\eta}^n\to\mathscr{B}_{\eta}$是同构.因为$u$是诺特概形上局部有限型模层之间的态射,我们解释过它的stalk同构点一定存在开邻域是同构,于是存在非空开集$W\subseteq Y$使得$u\mid_W$是同构.那么对任意凝聚$\mathscr{O}_Y$模层$\mathscr{F}$,有复合$u$诱导了如下$\mathscr{O}_Y$模层态射,并且它在$W$上的限制是同构:
        	$$v:\mathscr{G}=\mathrm{HOM}_{\mathscr{O}_Y}(\mathscr{B},\mathscr{F})\to\mathrm{HOM}_{\mathscr{O}_Y}(\mathscr{O}_Y^n,\mathscr{F})=\mathscr{F}^n$$
        	
        	我们要证明的是对任意凝聚$\mathscr{O}_Y$理想层$\mathscr{F}$有$\mathrm{H}^1(Y,\mathscr{F})=0$,这等价于$\mathrm{H}^1(Y,\mathscr{F}^n)=0$.我们断言有$v$是单态射【】.记凝聚层$\mathscr{K}=\mathrm{coker}v$,于是有$\mathscr{O}_Y$凝聚层的短正合列$0\to\mathscr{G}\to\mathscr{F}^n\to\mathscr{K}\to0$.这诱导了长正合列的一部分是$\mathrm{H}^1(Y,\mathscr{G})\to\mathrm{H}^1(Y,\mathscr{F}^n)\to\mathrm{H}^1(Y,\mathscr{K})$.于是问题归结为证明两边的群上同调都是零.首先因为$v\mid_W$是同构,说明$\mathscr{K}$的支集是$Y$的真闭子集$Z$,那么按照归纳假设有$Z$赋予既约闭子概型后是仿射的,于是如果记$i:Z\to Y$是闭嵌入,那么$\mathrm{H}^1(Y,\mathscr{K})=\mathrm{H}^1(Z,i^*\mathscr{K})=0$.其次$\mathscr{G}$沿着第一分量可以作为$f_*\mathscr{O}_X$凝聚模层,我们解释过$f_*\mathscr{O}_X$拟凝聚模层范畴和$\mathscr{O}_X$拟凝聚模层范畴是等价的,于是存在拟凝聚$\mathscr{O}_X$模层$\mathscr{H}$使得$f_*\mathscr{H}\cong\mathscr{G}$,于是有$\mathrm{H}^1(Y,\mathscr{G})=\mathrm{H}^1(Y,f_*\mathscr{H})=\mathrm{H}^1(X,\mathscr{H})=0$.这就完成证明.
        \end{proof}
        \item 引理.设$f:X\to Y$是有限态射,其中$Y$是qcqs概形,那么存在$Y$闭嵌入$X\to Z$,其中$Z$是一个有限和有限表示的$Y$概形.
        \begin{proof}
        	
        	【】
        	
        \end{proof}
        \item 一般版本的Chevalley定理.设$X\to Y$是概形之间的有限态射并且是满射,那么$X$是仿射的当且仅当$Y$是仿射的.
        \begin{proof}
        	
        	【】
        \end{proof}
        \item 推论.取$X=\mathrm{red}_Y$,我们推广了第一条引理:设$X$是概形,设$X_0$是$X$的闭子概型,并且它们具有相同的底空间,那么$X$是仿射的当且仅当$X_0$是仿射的.
    \end{enumerate}
\end{enumerate}

\subsection{扭曲层的上同调}

设$A$是诺特环,设$X=\mathbb{P}^r_A,r\ge1$是$A$上的射影空间,它可以视为$\mathrm{Proj}S$,其中$S=A[x_0,\cdots,x_r]$.我们定义过扭曲层$\mathscr{O}_X(1)$是$\widetilde{S(1)}$,其中$S(1)=\oplus_{n\ge0}S_{n+1}$.对每个$\mathscr{O}_X$模层$\mathscr{F}$,我们定义过它的扭曲层$\mathscr{F}(n)=\mathscr{F}\otimes_{\mathscr{O}_X}\mathscr{O}_X(n)$.还定义过和$\mathscr{F}$伴随的分次模$\Gamma_*(\mathscr{F})=\oplus_{n\in\mathbb{Z}}\Gamma(X,\mathscr{F}(n))$.
\begin{enumerate}
	\item $\mathrm{H}^r(X,\mathscr{O}_X(-r-1))\cong A$.
    \item 对每个$n$,存在如下典范的双线性映射是完全对(perfect pairing).
    $$\mathrm{H}^0(X,\mathscr{O}_X(n))\times\mathrm{H}^r(X,\mathscr{O}_X(-n-r-1))\to\mathrm{H}^r(X,\mathscr{O}_X(-r-1))\cong A$$
    
    设$M,N,L$是$R$模,考虑双线性映射$\alpha:M\times N\to L$,它等价于一个$R$模同态$M\otimes_RN\to L$,这个双线性映射诱导了同态$\beta:M\to\mathrm{Hom}_R(N,L)$为$\beta(m)(n)=\alpha(m,n)$.称这个双线性映射是perfect pairing如果诱导的$\beta$是同构.
    \item 对每个$n\in\mathbb{Z}$和每个$0<i<r$,都有$\mathrm{H}^i(X,\mathscr{O}_X(n))=0$.
\end{enumerate}
\begin{proof}
	
	先记$\mathscr{F}=\oplus_{n\in\mathbb{Z}}\mathscr{O}_X(n)$,我们之前在扭曲层性质里解释过它就是拟凝聚层$\widetilde{S}$,按照$A$是诺特的,有$X$是诺特空间,所以$\mathrm{H}^p(X,\mathscr{F})=\oplus_{n\in\mathbb{Z}}\mathrm{H}^p(X,\mathscr{O}_X(n))$.现在对每个$i=0,1,\cdots,r$,记$U_i=D_+(x_i)$,那么每个$U_i$是$X$的仿射开子集,并且$\mathscr{U}=\{U_0,\cdots,U_r\}$是$X$的开覆盖,并且有$\mathscr{F}(U_{i_0,\cdots,i_p})=S_{x_{i_0}\cdots x_{i_p}}$.由于$X$是分离的,所以由$\mathscr{U}$定义的如下交错\v{C}ech复形的上同调吻合于层上同调$\mathrm{H}^p(X,\mathscr{F})$:
	$$\xymatrix{\prod S_{x_{i_0}}\ar[r]&\prod S_{x_{i_0}x_{i_1}}\ar[r]&\cdots\ar[r]&S_{x_0\cdots x_r}}$$
	
	考虑$\mathrm{H}^r(X,\mathscr{F})$,它是最后一个映射$d^{r-1}:\prod_kS_{x_0\cdots\widehat{x_k}\cdots x_r}\to S_{x_0\cdots x_r}$的余核.这里$S_{x_0\cdots x_r}$是由$\{x_0^{l_0}\cdots x_r^{l_r}\mid l_i\in\mathbb{Z}\}$作为一组基的自由$A$模.而$d^{r-1}$的像由那些至少有一个$x_i$的次幂$\ge0$的单项式生成.所以$\mathrm{H}^r(X,\mathscr{F})$是由$\{x_0^{l_0}\cdots x_r^{l_r}\mid l_i<0,\forall i\}$生成的自由$A$模.特别的,这些基元素里仅有的次数为$-r-1$的元是$x_0^{-1}\cdots x_r^{-1}$,这说明$\mathrm{H}^r(X,\mathscr{O}_X(-r-1))$是秩1自由的.这证明了第一条.
	
	\qquad
	
	如果$n<0$,按照我们之前证明的$S\cong\oplus_{n\in\mathbb{Z}}\mathrm{H}^0(X,\mathscr{O}_X(n))$,说明$\mathrm{H}^0(X,\mathscr{O}_X(n))=0$.另外按照上一段的证明,如果$n<0$也有$\mathrm{H}^r(X,\mathscr{O}_X(-n-r-1))=0$(不存在次数为$-n-r-1$的每个$x_i$次数为负的单项式).下面设$n\ge0$.这里$\mathrm{H}^0(X,\mathscr{O}_X(n))$的一组基可以表示为$\{x_0^{m_0}\cdots x_r^{m_r}\mid m_i\ge0,\sum m_i=n\}$.而$\mathrm{H}^r(X,\mathscr{O}_X(-n-r-1))$的一组基是$\{x_0^{l_0}\cdots x_r^{l_r}\mid l_i<0,\sum l_i=-n-r-1\}$.这个典范映射就取为$(x_0^{m_0}\cdots x_r^{m_r})\dot(x_0^{l_0}\cdots x_r^{l_r})=x_0^{m_0+l_0}\cdots x_r^{m_r+l_r}$,但是约定如果某个$m_i+l_i\ge0$就把右侧取为零.容易验证这是perfect pairing.这证明了第二条.
	
	\qquad
	
	下面证明第三件事.我们对$r$归纳.$r=1$没什么需要证的.设$r>1$.如果我们把前文给出的\v{C}ech复形在$x_r$处做局部化,得到的是$\mathscr{F}$限制在$U_r$上的\v{C}ech上同调,开覆盖则是$\{U_i\cap U_r\mid i=0,1,\cdots,r-1\}$,按照Leray定理,它的上同调仍然是$\mathscr{F}\mid_{U_r}$的层上同调,按照仿射情况的Serre消失定理,这个层上同调在$i>0$时是零.按照局部化是正合函子,综上我们证明了当$i>0$时总有$\mathrm{H}^i(X,\mathscr{F})_{x_r}=0$.换句话讲,$\mathrm{H}^i(X,\mathscr{F}),i>0$中的每个元都被$x_r$的某个次幂零化.为了证明我们的命题,我们只需证明对$0<i<r$,数乘$x_r$诱导了$\mathrm{H}^i(X,\mathscr{F})$上的双射.
	
	\qquad
	
	考虑数乘$x_r$诱导的短正合列$0\to S(-1)\to S\to S/(x_r)\to0$.于是有模层的短正合列$0\to\mathscr{O}_X(-1)\to\mathscr{O}_X\to\mathscr{O}_H\to0$,其中$H$是超平面$x_r=0$.对每个整数$n$用$\mathscr{O}_X(n)$张量这个短正合列,由于$S(1)$是平坦的,所以张量后仍然得到模层的短正合列,再取直和得到模层的短正合列$0\to\mathscr{F}(-1)\to\mathscr{F}\to\mathscr{F}_H\to0$,其中$\mathscr{F}_H=\oplus_{n\in\mathbb{Z}}\mathscr{O}_H(n)$.取上同调,我们得到长正合列$\cdots\to\mathrm{H}^i(X,\mathscr{F}(-1))\to\mathrm{H}^i(X,\mathscr{F})\to\mathrm{H}^i(X,\mathscr{F}_H)\to\cdots$.这里$H$同构于$\mathbb{P}^{r-1}_A$.并且有$\mathrm{H}^i(X,\mathscr{F}_H)=\oplus_{n\in\mathbb{Z}}\mathrm{H}^i(H,\mathscr{O}_H(n))$.按照归纳假设,在$0<i<r-1$时有$\mathrm{H}^i(X,\mathscr{F}_H)=0$.对于$i=0$有短正合列$0\to\mathrm{H}^0(X,\mathscr{F}(-1))\to\mathrm{H}^0(X,\mathscr{F})\to\mathrm{H}^0(X,\mathscr{F}_H)\to0$,这里$\mathrm{H}^0(X,\mathscr{F}_H)$就是$S/(x_r)$.对于$i=r$我们断言有短正合列$0\to\mathrm{H}^{r-1}(X,\mathscr{F}_H)\to\mathrm{H}^r(X,\mathscr{F}(-1))\to\mathrm{H}^r(X,\mathscr{F})\to0$:我们描述过$\mathrm{H}^r(X,\mathscr{F})$作为自由$A$模的一组基是$\{x_0,\cdots,x_r\}$的负次幂构成的单项式.所以这里$\mathrm{H}^r(X,\mathscr{F}(-1))\to\mathrm{H}^r(X,\mathscr{F})$,作为数乘$x_r$的映射,它的核由$\{x_0^{l_0}\cdots x_r^{l_r},l_r=-1,l_i<0,0\le i\le r-1\}$生成,而$\mathrm{H}^{r-1}(X,\mathscr{F}_H)$的一组基是$\{x_0^{l_0}\cdots x_{r-1}^{l_{r-1}},l_i<0\}$.所以它们同构,并且这里$\mathrm{H}^{r-1}(X,\mathscr{F}_H)\to\mathrm{H}^r(X,\mathscr{F}(-1))$是单射.综上每个$x_r:\mathrm{H}^i(X,\mathscr{F}(-1))\to\mathrm{H}^i(X,\mathscr{F})$都是双射,这就完成证明.
\end{proof}
\subsection{射影版本的Serre消失和有限性定理}

设$X$是诺特环$A$上的射影概形,设$\mathscr{O}_X(1)$是$X$的在$A$上的扭曲层.设$\mathscr{F}$是$X$上的凝聚模层.
\begin{enumerate}
	\item 有限性定理.对每个$p$,有$\mathrm{H}^p(X,\mathscr{F})$是有限$A$模.(如果$X$是$A$上的紧合概形则结论也成立).
	\item 消失定理.存在一个依赖于$\mathscr{F}$的整数$n_0$,使得对每个$p\ge1$和每个$n\ge N$,都有$\mathrm{H}^p(X,\mathscr{F}(n))=0$.
\end{enumerate}
\begin{proof}
	
	按照定义,$\mathscr{O}_X(1)$是$X$的在$A$上的极丰沛层是指存在$A$概形的嵌入$i:X\to\mathbb{P}^r_A$,射影情况下我们解释过这里的$i$实际上是闭浸入.另外这里$\mathscr{F}$是凝聚层,而$i$是闭浸入,它是诺特概形之间的有限映射,导致$i_*\mathscr{F}$仍是凝聚层(没有诺特条件或者不是有限映射的话,凝聚层的前推就未必还是凝聚层了).于是我们有$\mathrm{H}^p(X,\mathscr{F})=\mathrm{H}^p(Y,i_*\mathscr{F}),\forall p$.于是我们可以把问题约化到$X=\mathbb{P}_A^r$的情况.
	
	\qquad
	
	我们在上一节描述了扭曲层的上同调,所以这两个命题对$\mathscr{F}=\mathscr{O}_X(q)$总成立的.进一步这两个命题对形如$\mathscr{O}_X(q)$的模层的有限直和仍然成立.下面对一般的凝聚层$\mathscr{F}$,我们来对$i$下降归纳证明这两个命题.首先如果$i>r$,由于$X=\mathbb{P}_A^r$被$r+1$个仿射开集覆盖,它的上同调维数$\le r$,所以在$i>r$时对任意凝聚层$\mathscr{F}$就有$\mathrm{H}^i(X,\mathscr{F})=0$.
	
	\qquad
	
	设$\mathscr{F}$是$X$上的凝聚层,我们解释过存在模层$\mathscr{G}$,它是形如$\mathscr{O}_X(n_i)$的模层的有限直和,使得$\mathscr{F}$是$\mathscr{G}$的商层.也即存在短正合列$0\to\mathscr{R}\to\mathscr{G}\to\mathscr{F}\to0$.那么$\mathscr{R}$仍然是凝聚层.这个短正合列诱导了长正合列$\cdots\to\mathrm{H}^i(X,\mathscr{G})\to\mathrm{H}^i(X,\mathscr{F})\to\mathrm{H}^{i+1}(X,\mathscr{R})\to\cdots$.这里$\mathrm{H}^i(X,\mathscr{G})$按照我们的解释是有限$A$模.而$\mathrm{H}^{i+1}(X,\mathscr{R})$是有限$A$模是归纳假设的.由于$A$是诺特的,这就得到中间的$\mathrm{H}^i(X,\mathscr{F})$也是有限$A$模.这就证明了有限性定理.
	
	\qquad
	
	为证明消失定理.取扭曲层再取长正合列,得到$\cdots\to\mathrm{H}^i(X,\mathscr{G}(n))\to\mathrm{H}^i(X,\mathscr{F}(n))\to\mathrm{H}^{i+1}(X,\mathscr{R}(n))\to\cdots$.其中$\mathrm{H}^i(X,\mathscr{G}(n))$在$n$足够大时是零,因为$\mathscr{G}$是有限个$\mathscr{O}_X(n_i)$的直和.而$\mathrm{H}^{i+1}(X,\mathscr{R}(n))$按照归纳假设在$n$足够大时是零.所以对每个固定的$i$,在$n$足够大时就有$\mathrm{H}^i(X,\mathscr{F}(n))=0$.但是$i<0$或者$i>r$时这个上同调总是零,所以对每个$0\le i\le r$取足够大的$n_0$,就保证当$n\ge n_0$时总有$\mathrm{H}^i(X,\mathscr{F}(n))=0$.
\end{proof}
\subsection{丰沛层的上同调描述}

设$A$是诺特环,$X$是$A$上的紧合概形,设$\mathscr{L}$是$X$上的可逆模层,那么如下命题互相等价.并且在条件成立时有$f$是射影态射(我们解释过一个紧合态射是射影的当且仅当$X$存在关于这个态射的极丰沛可逆层).
\begin{enumerate}
	\item $\mathscr{L}$是丰沛可逆层.
	\item 对$X$上的每个凝聚层$\mathscr{F}$,都存在整数$n_0$,使得对任意$i\ge1$和任意$n\ge n_0$有$\mathrm{H}^i(X,\mathscr{F}\otimes\mathscr{L}^{\otimes n})=0$.
	\item 对$X$上的每个凝聚理想层$\mathscr{J}$,都存在整数$n_0$,使得对任意$n\ge n_0$有$\mathrm{H}^1(X,\mathscr{J}\otimes\mathscr{L}^{\otimes n})=0$.
\end{enumerate}
\begin{proof}
	
	1推2,如果$\mathscr{L}$是丰沛可逆层,我们解释过存在正整数$m$,使得$\mathscr{L}^{\otimes m}$是$X$的关于$\mathrm{Spec}A$的极丰沛可逆层.我们解释过$Y$上概形$X$是射影的当且仅当$Y$是$X$上的紧合概形,并且存在极丰沛可逆层.于是这里$X$是$\mathrm{Spec}A$上的射影概形.于是对每个$0\le k\le m-1$,可找$n_k$使得$n\ge n_k$时$\mathrm{H}^i(X,\mathscr{F}\otimes\mathscr{L}^{n+k})=0,\forall i$.选取$N=\max\{n_k\}$,那么$n\ge N$时就有$\mathrm{H}^i(X,\mathscr{F}\otimes\mathscr{L}^n)=0$.
	
	2推3是平凡的,最后证明3推1,任取凝聚层$\mathscr{F}$,任取闭点$p\in X$,记闭子概型$Y=\{p\}$的理想层为$\mathscr{I}_p$,按照定义它是层态射$i^{\#}:\mathscr{O}_X\to i_*\mathscr{O}_Y$的核.那么有模层的短正合列$0\to\mathscr{I}_p\mathscr{F}\to\mathscr{F}\to\mathscr{F}\otimes\mathscr{O}_X/\mathscr{I}_p\to0$.由于$\mathscr{L}^n$是局部秩1的,张量它不改变正合性,于是得到$0\to\mathscr{I}_p\mathscr{F}\otimes\mathscr{L}^{\otimes n}\to\mathscr{F}\otimes\mathscr{L}^{\otimes n}\to\mathscr{F}\otimes\mathscr{L}^{\otimes n}\otimes\mathscr{O}_X/\mathscr{I}_p\to0$.按照条件存在$n_0$,使得当$n\ge n_0$时有$\mathrm{H}^1(X,\mathscr{I}_p\mathscr{F}\otimes\mathscr{L}^n)=0$.于是典范映射$\Gamma(X,\mathscr{F}\otimes\mathscr{L}^{\otimes n})\to\Gamma(X,\mathscr{F}\otimes\mathscr{L}^{\otimes n}\otimes\mathscr{O}_X/\mathscr{I}_p)$在$n\ge n_0$时是满射.如果记$\mathscr{F}'=\mathscr{F}\otimes\mathscr{L}^{\otimes n}$,其中$n\ge n_0$,那么这个满射就是说$\mathscr{F}'(X)\to\mathscr{F}'_p/\mathfrak{m}_p\mathscr{F}'_p$是满射,其中$\mathfrak{m}_p$是$\mathscr{O}_{X,p}$的极大理想.按照NAK引理的一个版本,取$\mathscr{F}'_p/\mathfrak{m}_p\mathscr{F}'_p$的一组生成元在$\mathscr{F}'(X)$中的提升,那么它们生成了$\mathscr{F}'_p$.换句话讲$\mathscr{F}'_p$是被整体截面生成的.那么按照$\mathscr{F}'$是凝聚层,就存在$p$的开邻域$U$,使得对每个$q\in U$,都有$\mathscr{F}'_q$被整体截面生成.
	
	\qquad
	
	如果取$\mathscr{F}=\mathscr{O}_X$,说明存在一个正整数$n_1$,和$p$的开邻域$V$,使得$\mathscr{L}^{n_1}$在$V$上处处被整体截面生成.同理对每个$0\le r\le n_1-1$,存在$p$的开邻域$U_r$,使得$\mathscr{F}\otimes\mathscr{L}^{\otimes(n_0+r)}$在$U_r$上总被整体截面生成.取$U_p=V\cap U_0\cap\cdots\cap U_{n_1-1}$,那么对每个$n\ge n_0$,它可以表示为$n_0+mn_1+r$,其中$0\le r\le n_1-1$,按照$\mathscr{F}\otimes\mathscr{L}^{\otimes n}=\left(\mathscr{F}\otimes\mathscr{L}^{\otimes(n_0+r)}\right)\otimes(\mathscr{L}^{\otimes n_1})^{\otimes m}$,说明$\mathscr{F}\otimes\mathscr{L}^{\otimes n}$总可以在$U_p$上处处被整体截面生成.最后考虑$\{U_p,p\in X\}$的有限子覆盖,这些选定的闭点的$n_0$中最大的一个记作$N$,那么在$n\ge N$时$\mathscr{F}\otimes\mathscr{L}^{\otimes n}$就总是被整体截面生成.这说明$\mathscr{L}$是丰沛层.
\end{proof}
\subsection{延拓群和延拓层}
\begin{enumerate}
	\item 设$X$是诺特概形,设$\mathscr{F}$是$X$上凝聚层,设$\mathscr{G}$是$X$上任意模层,设$x\in X$任意,那么对每个$i\ge0$有典范同构:
	$$\mathrm{EXT}^i(\mathscr{F},\mathscr{G})_x\cong\mathrm{Ext}^i_{\mathscr{O}_{X,x}}(\mathscr{F}_x,\mathscr{G}_x)$$
	
	另外如果这里$\mathscr{F}$如果不要求凝聚层,那么即便$i=0$的时候也未必成立.
	\begin{proof}
		
		这个同构式右侧的$\mathrm{Ext}$是$A$模范畴上$\mathrm{Hom}_A(M,-)$的右导出函子.它也可以视为模层上$\mathrm{Ext}$的特例,即考虑单点空间,赋予的环层是全集上取$A$.我们的问题是局部的,所以可设$X$是仿射的,此时$\mathscr{F}$就有自由预解$\mathscr{L}_{\bullet}\to\mathscr{F}\to0$.对点$x\in X$,取stalk得到正合列$(\mathscr{L}_{\bullet})_x\to\mathscr{F}_x\to0$.我们解释过计算$\mathrm{EXT}^i(\mathscr{F},\mathscr{G})$可以通过取$\mathscr{F}$的局部自由预解计算.结合$\mathrm{HOM}(\mathscr{L}_{\bullet},\mathscr{G})_x=\mathrm{Hom}_{\mathscr{O}_{X,x}}((\mathscr{L}_{\bullet})_x,\mathscr{G}_x)$.按照茎函子是正合的,就得到这个典范同构.
	\end{proof}
    \item 设$X$是诺特环$A$上的射影概形,设$\mathscr{O}_X(1)$是一个极丰沛可逆层,设$\mathscr{F},\mathscr{G}$是$X$上的凝聚层,记$\mathscr{G}(n)=\mathscr{G}\otimes\mathscr{O}_X(1)^{\otimes n}$.那么对$i\ge0$,存在正整数$n_0$(它依赖于$i,\mathscr{F},\mathscr{G}$),使得在$n\ge n_0$时总有典范同构:
    $$\mathrm{Ext}^i(\mathscr{F},\mathscr{G}(n))\cong\Gamma(X,\mathrm{EXT}^i(\mathscr{F},\mathscr{G}(n)))$$
    \begin{proof}
    	
    	在$i=0$时这个命题对任意$n$成立,它是函子$\mathrm{HOM}$的定义.如果$\mathscr{F}=\mathscr{O}_X$,左侧是$\mathrm{H}^i(X,\mathscr{G}(n))$,我们证明过当$n$足够大时对每个$i\ge1$它都是零.另一方面我们解释过当$i\ge1$时$\mathrm{EXT}^i(\mathscr{O}_X,\mathscr{G})=0$.于是我们证明了$\mathscr{F}=\mathscr{O}_X$的情况.
    	
    	\qquad
    	
    	一般的在$\mathscr{L}$是有限秩局部自由模层的条件下,对任意模层$\mathscr{F}$和$\mathscr{G}$,我们解释过有$\mathrm{Ext}^i(\mathscr{F}\otimes\mathscr{L},\mathscr{G})\cong\mathrm{Ext}^i(\mathscr{F},\mathscr{L}^{\vee}\otimes\mathscr{G})$和$\mathrm{EXT}^i(\mathscr{F}\otimes\mathscr{L},\mathscr{G})\cong\mathrm{EXT}^i(\mathscr{F},\mathscr{L}^{\vee}\otimes\mathscr{G})$.这个结论结合$\mathscr{F}=\mathscr{O}_X$的情况就可以证明$\mathscr{F}$是有限秩局部自由模层的情况:
    	\begin{align*}
    		\mathrm{Ext}^i(\mathscr{F},\mathscr{G}\otimes\mathscr{O}(n))&\cong\mathrm{Ext}^i(\mathscr{O}_X,\mathscr{F}^{\vee}\otimes\mathscr{G}\otimes\mathscr{O}(n))\\&\cong\Gamma(X,\mathrm{EXT}^i(\mathscr{O}_X,\mathscr{F}^{\vee}\otimes\mathscr{G}\otimes\mathscr{O}(n)))\\&\cong\Gamma(X,\mathrm{EXT}^i(\mathscr{F},\mathscr{G}\otimes\mathscr{O}(n)))
    	\end{align*}
    	
    	最后设$\mathscr{F}$是任意凝聚层,那么它是一个局部自由模层$\mathscr{E}$的商,把核记作$\mathscr{R}$,它也是凝聚层,有短正合列$0\to\mathscr{R}\to\mathscr{E}\to\mathscr{F}\to0$.因为$\mathscr{E}$是局部自由的,就有$\mathrm{Ext}^1(\mathscr{E},\mathscr{G}(n))=0$.于是得到如下正合列:
    	$$\xymatrix{0\ar[r]&\mathrm{Hom}(\mathscr{F},\mathscr{G}(n))\ar[r]&\mathrm{Hom}(\mathscr{E},\mathscr{G}(n))\ar[r]&\mathrm{Hom}(\mathscr{R},\mathscr{G}(n))\ar[r]&\mathrm{Ext}^1(\mathscr{F},\mathscr{G}(n))\ar[r]&0}$$
    	$$\mathrm{Ext}^i(\mathscr{R},\mathscr{G}(n))\cong\mathrm{Ext}^{i+1}(\mathscr{F},\mathscr{G}(n)),\forall i\ge1$$
    	$$\xymatrix{0\ar[r]&\mathrm{HOM}(\mathscr{F},\mathscr{G}(n))\ar[r]&\mathrm{HOM}(\mathscr{E},\mathscr{G}(n))\ar[r]&\mathrm{HOM}(\mathscr{R},\mathscr{G}(n))\ar[r]&\mathrm{EXT}^1(\mathscr{F},\mathscr{G}(n))\ar[r]&0}$$
    	$$\mathrm{EXT}^i(\mathscr{R},\mathscr{G}(n))\cong\mathrm{EXT}^{i+1}(\mathscr{F},\mathscr{G}(n)),\forall i\ge1$$
    	
    	接下来我们用如下结论:如果$X$是诺特环$A$上的射影概形,设$\mathscr{F}_1\to\mathscr{F}_2\to\cdots\to\mathscr{F}_r$是$X$上凝聚层的有限长度的正合列,那么存在正整数$n_1$,当$n\ge n_1$时总有正合列$\Gamma(X,\mathscr{F}_1(n))\to\Gamma(X,\mathscr{F}_2(n))\to\cdots\to\Gamma(X,\mathscr{F}_r(n))$.假设正整数$n_0$使得命题对$i,\mathscr{R},\mathscr{G}$成立,我们选取上述结论用在$\mathrm{EXT}^i(\mathscr{R},\mathscr{G}(n))\cong\mathrm{EXT}^{i+1}(\mathscr{F},\mathscr{G}(n))$上的正整数$n_1$,选取$n_0'=\max\{n_0,n_1\}$,那么当$n\ge n_0'$时就有:
    	\begin{align*}
    		\mathrm{Ext}^{i+1}(\mathscr{F},\mathscr{G}(n))&\cong\mathrm{Ext}^i(\mathscr{R},\mathscr{G}(n))\\&\cong\Gamma(X,\mathrm{EXT}^i(\mathscr{R},\mathscr{G}(n)))\\&\cong\Gamma(X,\mathrm{EXT}^{i+1}(\mathscr{F},\mathscr{G}(n)))
    	\end{align*}
    \end{proof}
\end{enumerate}
\subsection{高阶前推函子}
\begin{enumerate}
	\item 设$f:X\to\mathrm{Spec}A$是概形的态射,其中$X$是诺特概形,设$\mathscr{F}$是$X$上的拟凝聚层,那么有:
	$$R^if_*(\mathscr{F})\cong\widetilde{\mathrm{H}^i(X,\mathscr{F})}$$
	\begin{proof}
		
		我们解释过$X$诺特的时候正像函子把拟凝聚层映成拟凝聚层,所以这里$f_*(\mathscr{F})$是$\mathrm{Spec}A$上的拟凝聚层,于是有$f_*\mathscr{F}\cong\widetilde{\Gamma(\mathrm{Spec}A,f_*\mathscr{F})}=\widetilde{\Gamma(X,\mathscr{F})}$.于是$i=0$时命题成立.接下来这个同构式的两侧都是同调$\delta$函子,并且$X$上的每个拟凝聚层都可以嵌入到一个松弛拟凝聚层,并且对于松弛拟凝聚层当$i\ge1$时同构式两侧都是零,此即这两个同调$\delta$函子都在$i\ge1$时是effaceable的,而它们在$i=0$时一致,于是它们同构.
	\end{proof}
    \item 推论.设$f:X\to Y$是概形的态射,设$X$是诺特的,对$X$上的任意拟凝聚层$\mathscr{F}$,对每个$i\ge0$有$R^if_*(\mathscr{F})$是$Y$上拟凝聚层.
    \item 设$f:X\to Y$是分离诺特概形之间的态射,设$\mathscr{F}$是$X$上拟凝聚层,设$\mathscr{U}=\{U_i\}$是$X$的仿射开覆盖,设$\mathscr{C}^{\bullet}(\mathscr{U},\mathscr{F})$是$\mathscr{F}$的\v{C}ech预解,那么计算$R^pf_*(\mathscr{F})$可以通过计算复形$f_*\mathscr{C}^{\bullet}(\mathscr{U},\mathscr{F})$(这里$\mathscr{C}^p(\mathscr{U},\mathscr{F})$是$X$上的层$\prod_{i_0<\cdots<i_p}f_*(\mathscr{F}\mid U_{i_1,\cdots,i_p})$),即对任意$p\ge0$有:
    $$R^pf_*(\mathscr{F})\cong\mathrm{H}^p\left(f_*\mathscr{C}^{\bullet}(\mathscr{U},\mathscr{F})\right)$$
    \begin{proof}
    	
    	首先对$Y$的仿射开子集$V$,我们有如下magic图表,这里按照$Y$是分离概形,有对角态射$Y\to Y\times Y$是闭嵌入,于是提升映射$U_i\times_YV\to U_i\times V$也是闭嵌入.这里$U_i\times V$是仿射的,于是$U_i\times_YV=U_i\cap f^{-1}(V)$是$U_i\times V$的闭子概型,于是它也是仿射的.所以我们可以把问题归结为$Y$是仿射诺特概形的情况.
    	$$\xymatrix{U_i\times_YV\ar[rr]\ar[d]&&U_i\times V\ar[d]\\Y\ar[rr]&&Y\times Y}$$
    	
    	因为$X$是诺特空间,这里$\mathscr{C}^p(\mathscr{U},\mathscr{F})$都是$X$上的拟凝聚层,它的正像是仿射概形$Y$上的拟凝聚层,所以有如下复形的同构:
    	$$f_*\mathscr{C}^{\bullet}(\mathscr{U},\mathscr{F})\cong\widetilde{\mathrm{C}^{\bullet}(\mathscr{U},\mathscr{F})}$$
    	
    	对左侧复形复形求同调就是$\mathrm{H}^p\left(f_*\mathscr{C}^{\bullet}(\mathscr{U},\mathscr{F})\right)$.对右侧复形求同调得到$\check{\mathrm{H}}^p(\mathscr{U},\mathscr{F})$的伴随模层,按照概形是分离的,我们解释过仿射开覆盖的\v{C}ech上同调和层上同调吻合,所以它典范同构于$\mathrm{H}^p(\mathscr{U},\mathscr{F})$的伴随模层.再按照上一条结论得到它典范同构于$R^if_*(\mathscr{F})$.
    \end{proof}
    \item 设$f:X\to Y$是诺特概形之间的射影态射,设$\mathscr{O}_X(1)$是$X$的关于$Y$的极丰沛可逆层,设$\mathscr{F}$是$X$上的凝聚层,那么:
    \begin{enumerate}
    	\item 对足够大的$n$,有$f^*f_*(\mathscr{F}(n))\to\mathscr{F}(n)$是满态射.这里$\mathscr{F}(n)=\mathscr{F}\otimes\mathscr{O}_X(1)^{\otimes n}$.
    	\item 对每个$i\ge0$,有$R^if_*(\mathscr{F})$是$Y$上的凝聚层.
    	\item 对每个$i\ge1$,当$n$足够大时有$R^if_*(\mathscr{F}(n))=0$.
    \end{enumerate}
    \begin{proof}
    	
    	$Y$是诺特概形,所以它是拟紧的,问题在$Y$上局部,所以不妨设$Y$是仿射的,记作$\mathrm{Spec}A$.那么(a)是在说$\mathscr{F}(n)$可被整体截面生成【?】,这是因为极丰沛可逆层是丰沛可逆层.按照前面的结论,$R^if_*(\mathscr{F})$已经是拟凝聚层,(b)是在说它对应的模是有限模,但是我们证明过诺特环上射影概形$X$,总有$\mathrm{H}^i(X,\mathscr{F})$是有限$A$模.最后我们也证明过诺特环上射影概形$X$,如果取极丰沛可逆层$\mathscr{F}$,对每个$i\ge1$,当$n$足够大时就有$\mathrm{H}^i(X,\mathscr{F}(n))=0$,这说明了(c).
    \end{proof}
    \item 引理.设$f:X\to\mathrm{Spec}A$是分离拟紧的态射,设$\mathscr{F}$是$X$上的拟凝聚层,设$M$是$A$模,用$\mathscr{F}\otimes_AM$表示对仿射开子集$U\mapsto\mathscr{F}(U)\otimes_AM$所唯一定义的拟凝聚层,那么对任意$p\ge0$我们有如下典范同态,如果$M$在$A$上平坦则这个典范同态是一个同构.
    $$\mathrm{H}^p(X,\mathscr{F})\otimes_AM\to\mathrm{H}^p(X,\mathscr{F}\otimes_AM)$$
    \begin{proof}
    	
    	取$X$的有限仿射开覆盖$\mathscr{U}=\{U_i\mid i\in I\}$.因为$f$是分离的,所以$U_{i_0,\cdots,i_p}=\cap_{0\le j\le p}U_{i_j}$是仿射的.于是$(\mathscr{F}\otimes_AM)(U_{i_0,\cdots,i_p})=\mathscr{F}(U_{i_0,\cdots,i_p})\otimes_AM$.因为$\mathscr{U}$是有限开覆盖,所以复形$\mathrm{C}(\mathscr{U},\mathscr{F})$的项都是有限直和,于是$\mathrm{C}(\mathscr{U},\mathscr{F})\otimes_AM=\mathrm{C}(\mathscr{U},\mathscr{F}\otimes_AM)$,如果复形$\mathrm{C}(\mathscr{U},\mathscr{F})$的微分是$\mathrm{d}$,那么$\mathrm{C}(\mathscr{U},\mathscr{F}\otimes_AM)$的微分是$\mathrm{d}\otimes\mathrm{id}_M$.一般的如果$\xymatrix{N_1\ar[r]^{\alpha}&N_2\ar[r]^{\beta}&N_3}$是$A$模的短正合列,那么它诱导了如下典范同态:
    	$$(\ker\beta/\mathrm{im}\alpha)\otimes_AM\to\ker(\beta\otimes\mathrm{id}_M)/\mathrm{im}(\alpha\otimes\mathrm{id}_M)$$
    	
    	并且在$M$是平坦$A$模时上述典范同态是同构.于是我们得到典范同态$\mathrm{H}^n(\mathscr{U},\mathscr{F})\otimes_AM\to\mathrm{H}^n(\mathscr{U},\mathscr{F}\otimes_AM)$,并且在$M$是平坦$A$模时它是同构.最后我们解释过借助Leray定理,分离概形上一个仿射开覆盖定义的\v{C}ech上同调和层上同调是一致的.这得到结论.
    \end{proof}
    \item 设$f:X\to\mathrm{Spec}A$是分离和拟紧的态射,设$B$是平坦$A$代数,记$X_B=X\times_AB$,记$\rho:X_B\to X$是典范投影态射,那么对$X$上任意拟凝聚层$\mathscr{F}$和任意$p\ge0$,我们有如下典范同构:
    $$\mathrm{H}^p(X,\mathscr{F})\otimes_AB\cong\mathrm{H}^p(X_B,\rho^*\mathscr{F})$$
    \begin{proof}
    	
    	取$X$的仿射开覆盖$\mathscr{U}=\{U_i\}$,那么$\mathscr{U}_B=\{(U_i)_B\}$是$X_B$的仿射开覆盖,并且有$\mathrm{C}(\mathscr{U}_B,\rho^*\mathscr{F})=\mathrm{C}(\mathscr{U},\mathscr{F}\otimes_AB)$.按照拟紧和分离态射在基变换下不变得到$X_B$是$B$上拟紧和分离概形,于是上述引理得到结论.
    \end{proof}
    \item 设$f:X\to Y$是分离和拟紧的态射,设$\mathscr{F}$是$X$上的拟凝聚层,$\mathscr{G}$是$Y$上的拟凝聚层,对任意$p\ge0$,我们有如下典范同态,并且如果$\mathscr{G}$是$Y$上平坦模层时这是典范同构.在后者情况下这个公式称为投影公式(projection formula).
    $$(\mathrm{R}^pf_*)\otimes_{\mathscr{O}_Y}\mathscr{G}\to\mathrm{R}^pf_*(\mathscr{F}\otimes_{\mathscr{O}_X}f^*\mathscr{G})$$
    \begin{proof}
    	
    	任取$Y$的仿射开邻域$V$,左侧是:
    	$$\mathrm{H}^p(f^{-1}(V),\mathscr{F}\mid_{f^{-1}(V)})\otimes_{\mathscr{O}_Y(V)}\mathscr{G}(V)$$
    	
    	右侧是:
    	$$\mathrm{H}^p(f^{-1}(V),\mathscr{F}\mid_{f^{-1}(V)}\otimes_{\mathscr{O}_Y(V)}\mathscr{G}(V))$$
    	
    	于是问题归结为我们的引理.
    \end{proof}
\end{enumerate}
\subsection{希尔伯特多项式}

\begin{enumerate}
	\item 欧拉示性数.设$X$是域$k$上的紧合概形,设$\mathscr{F}$是$X$上的凝聚层,那么紧合概形的有限性定理告诉我们$\mathrm{H}^i(X,\mathscr{F})$总是$k$上的有限维空间.记$h^i(X,\mathscr{F})=\dim_k\mathrm{H}^i(X,\mathscr{F})$.定义$\mathscr{F}$的欧拉示性数(Euler characteristic)为:
	$$\chi(X,\mathscr{F})=\sum_{i=0}^{\dim X}(-1)^ih^i(X,\mathscr{F})$$
	
	欧拉示性数关于正合列是可加的:如果如下正合列由$X$上凝聚层构成,那么有$\sum_{i=1}^n(-1)^i\chi(X,\mathscr{F}_i)=0$.
	$$\xymatrix{0\ar[r]&\mathscr{F}_1\ar[r]&\cdots\ar[r]&\mathscr{F}_n\ar[r]&0}$$
	\begin{proof}
		
		归结为对短正合列$\xymatrix{0\ar[r]&\mathscr{F}_1\ar[r]&\mathscr{F}_2\ar[r]&\mathscr{F}_3\ar[r]&0}$证明这件事,而这是因为,短正合列诱导了上同调的长正合列:
		$$\xymatrix{0\ar[r]&\mathrm{H}^0(X,\mathscr{F}_1)\ar[r]&\mathrm{H}^0(X,\mathscr{F}_2)\ar[r]&\mathrm{H}^0(X,\mathscr{F}_3)\ar[r]&\mathrm{H}^1(X,\mathscr{F}_1)\ar[r]&\cdots}$$
		
		进而有:
		$$0=h^0(X,\mathscr{F}_1)-h^0(X,\mathscr{F}_2)+h^0(X,\mathscr{F}_3)-h^1(X,\mathscr{F}_1)+\cdots$$
		
		此即$\chi(X,\mathscr{F}_1)-\chi(X,\mathscr{F}_2)+\chi(X,\mathscr{F}_3)=0$.
	\end{proof}
	\item 希尔伯特多项式.
	\begin{enumerate}
		\item 设$X$是射影$k$概形,取定一个极丰沛可逆层$\mathscr{O}_X(1)$(也即取定一个闭嵌入$X\to\mathbb{P}_k^n$),设$\mathscr{F}$是$X$上的凝聚层,数值函数$P(n)=\chi(\mathscr{F}(n))$是次数为$\dim\mathrm{Supp}\mathscr{F}$的多项式函数,称为$\mathscr{F}$关于$\mathscr{O}_X(1)$的希尔伯特多项式,记作$p_{\mathscr{F}}(n)$.
		\item 特别的,按照射影版本的Serre消失定理,固定拟凝聚层$\mathscr{F}$时,在$n$足够大的时候$\mathrm{H}^p(X,\mathscr{F}(n))=0,\forall p\ge1$,于是固定凝聚层$\mathscr{F}$的时候,在$n$足够大时$h^0(X,\mathscr{F}(n))$是一个次数为$\dim\mathrm{Supp}\mathscr{F}$的多项式,它恰好就是希尔伯特多项式$p_{\mathscr{F}}(n)$.
		\item 特别的,在$n$足够大时$h^0(X,\mathscr{O}_X(n))$是一个次数为$\dim X$的多项式函数,他称为射影$k$概形$X$的希尔伯特多项式,记作$p_X(n)$.
	\end{enumerate}
	\begin{proof}
		
		我们知道拟紧分离$k$概形上拟凝聚层的上同调和域扩张的基变换可交换,于是归结为设$k$是代数闭域,特别的这是一个无限域.于是可以找到一个$x\in\Gamma(X,\mathscr{O}_X(1))$使得它确定的超平面不包含$\mathscr{F}$的全部(有限个)伴随点.因为对于诺特环上的有限模,它的零因子集合恰好是全部伴随素理想的并,于是数乘$x$是$\mathscr{F}(-1)\to\mathscr{F}$的单态射,它的余核记作$\mathscr{G}$,是一个凝聚层.
		
		\qquad
		
		我们断言$\mathrm{Supp}\mathscr{G}=(\mathrm{Supp}\mathscr{F})\cap V(x)$:如果$x\not\in\mathfrak{p}$,那么数乘$x$诱导的$\mathscr{F}(-1)_{\mathfrak{p}}\to\mathscr{F}_{\mathfrak{p}}$是一个同构,于是此时$\mathfrak{p}\not\in\mathrm{Supp}\mathscr{G}$;如果$\mathfrak{p}\in V(x)$,那么$\mathscr{F}(-1)_{\mathfrak{p}}\to\mathscr{F}_{\mathfrak{p}}$是零映射,于是$\mathscr{F}_{\mathfrak{p}}\cong\mathscr{G}\mid_{\mathfrak{p}}$.
		
		\qquad
		
		我们知道射影空间的超曲面和每个正维数子空间都有交,于是$V(x)$和$\mathrm{Supp}\mathscr{F}$的每个正维数不可约分支有交,进而按照Krull主理想定理有$\dim\mathrm{Supp}\mathscr{G}=\dim\mathrm{Supp}\mathscr{F}-1$.我们有$p_{\mathscr{F}}(m)-p_{\mathscr{F}}(m-1)=p_{\mathscr{G}}(m)$.于是按照归纳假设,从$p_{\mathscr{G}}(m)$是$\dim\mathrm{Supp}(\mathscr{G})$次多项式得到$p_{\mathscr{F}}(m)$是$\dim\mathrm{Supp}(\mathscr{F})$次多项式.
	\end{proof}
    \item 设$i:X\to Y$是射影概形的闭嵌入,那么总有$p_X(m)\le p_Y(m)$.这是因为有如下正合列:
    $$\xymatrix{0\ar[r]&\mathscr{I}_{X/Y}(m)\ar[r]&\mathscr{O}_Y(m)\ar[r]&\mathscr{O}_X(m)\ar[r]&0}$$
    \item 如果射影概形$X$上的凝聚层$\mathscr{F}$满足希尔伯特多项式是零,那么$\mathscr{F}=0$.特别的,这说明如果$i:X\to Y$是射影概形的闭嵌入,满足$p_X=p_Y$,那么有$X=Y$.
    \begin{proof}
    	
    	这件事是因为当$m$足够大的时候$\mathscr{F}(m)$的整体截面平凡,但是它被整体截面生成,于是$\mathscr{F}(m)$是零层,再张量$\mathscr{O}_X(-m)$得到$\mathscr{F}$是零层.
    \end{proof}
	\item 射影$k$概形的次数.设$X$是一个$r$维的射影$k$概形,那么我们知道它的希尔伯特多项式是一个$r$次多项式,它的首系数$\times r!$称为$X$的次数(degree).例如取$X=\mathbb{P}_k^r$是射影空间本身,那么它的希尔伯特多项式为$p_X(n)=\left(\begin{array}{c}n+r\\r\end{array}\right)$,从而它的次数为1.
	\item 算术亏格.设$X$是$r$维射影$k$概形,它的算术亏格(arithmetic genus)定义为$p_a(X)=(-1)^r(\chi(X,\mathscr{O}_X)-1)=(-1)^r(p_X(0)-1)$.尽管希尔伯特多项式是在$n$足够大时不变的,但是当$n=0$时却仍然是$X$的一个重要不变量.
	\begin{enumerate}[(1)]
		\item 如果$X$是射影曲线,那么$p_a(X)=1-h^0(X,\mathscr{O}_X)+h^1(X,\mathscr{O}_X)$.
		\item 如果$X$是代数闭域上的整射影曲线,那么$h^0(X,\mathscr{O}_X)=1$(代数闭域上连通既约紧合概形的整体截面都是常值),于是此时$p_a(X)=h^1(X,\mathscr{O}_X)$.
	\end{enumerate}
    \item 算术亏格的例子.
    \begin{enumerate}[(1)]
    	\item $p_a(\mathbb{P}^n_K)=0$.
    	\item 设$C$是$d$次平面射影曲线,那么$p_a(C)=\frac{1}{2}(d-1)(d-2)$.
    	\begin{proof}
    		
    		记$S=k[T_0,T_1,T_2]$,考虑如下短正合列:
    		$$\xymatrix{0\ar[r]&S(-d)\ar[r]^F&S\ar[r]&S/(F)\ar[r]&0}$$
    		
    		记典范闭嵌入$i:C\to\mathbb{P}_k^2$,那么上述短正合列诱导了如下$\mathscr{O}_{\mathbb{P}_k^2}$模层的短正合列:
    		$$\xymatrix{0\ar[r]&\mathscr{O}_{\mathbb{P}_k^2}(-d)\ar[r]&\mathscr{O}_{\mathbb{P}_k^2}\ar[r]&i_*\mathscr{O}_Z\ar[r]&0}$$
    		
    		射影空间上扭曲层的上同调是已知的,所以长正合列诱导了如下两个同构:
    		$$k=\mathrm{H}^0(\mathbb{P}_k^2,\mathscr{O}_{\mathbb{P}_k^2})\cong\mathrm{H}^0(Z,\mathscr{O}_Z)$$
    		$$\mathrm{H}^1(Z,\mathscr{O}_Z)\cong\mathrm{H}^2(\mathbb{P}_k^2,\mathscr{O}_{\mathbb{P}_k^2}(-d))\cong S_{(d-3)}$$
    		
    		后者的维数是$\frac{1}{2}(d-1)(d-2)$.
    	\end{proof}
    	\item 更一般的,如果$H$是$\mathbb{P}_k^n$的$d$次超曲面,那么有$p_a(H)=\left(\begin{array}{cc}d-1\\n\end{array}\right)$.
    \end{enumerate}
\end{enumerate}
\newpage
\section{平展上同调}
\subsection{Grothendieck拓扑}

设$\mathscr{C}$是小范畴.
\begin{enumerate}
	\item 设$c$是一个对象,一个$c$上的筛(sieve)是指函子$h_c=\mathrm{Hom}_{\mathscr{C}}(-,c)$的子函子.换句话讲筛是$\mathscr{C}$中的一个终端为$c$的态射集合$\{\varphi_i:c_i\to c\}$,满足如果存在指标$i$和$\mathscr{C}$中的态射$c'\to c_i$,那么复合态射$c'\to c_i\to c$也落在这个集合中.
	\begin{enumerate}[(1)]
		\item 固定一个对象$c$,全体$c$上的筛构成集合$\mathrm{Si}(c)$.定义其上的偏序是$S\le S'$当且仅当对任意对象$c'$有$S(c)\subseteq S'(c)$.这个偏序下$h_c$是最大元;明显的筛的交和并都是筛,从而$\mathrm{Si}(c)$总是完备格;给定一个终端为$c$的态射集合$\{\varphi_i:c_i\to c\}$,它生成的筛或者说包含它的最小的筛就是全体终端为c的经某个$\varphi_i$分解的态射构成的集合(这个操作相当于把拓扑空间的一个开覆盖扩充为极大开覆盖,也即把每个出现的开子集的全部开子集都加进去).
		\item 回拉.设$f:c'\to c$是$\mathscr{C}$中的态射,设$S$是$c$上的筛,定义回拉$f^*S$由全体满足$f\circ g\in S$的态射$g$构成的集合,它是$c'$上的筛(这个操作对应拓扑空间一个开子集的开覆盖通过取交限制在更小的开子集上).
	\end{enumerate}
    \item 设$\mathscr{C}$是小范畴,其上的一个Grothendieck拓扑$T$是指对每个对象$c$都赋予一个由其上筛构成的集合$\mathrm{Cov}(c)$,其中的元素称为$c$的覆盖,它们满足如下三条公理.称赋予了Grothendieck拓扑的小范畴$(\mathscr{C},T)$为site.
    \begin{enumerate}[(a)]
    	\item (恒等)$\{1_c\}$生成的筛包含在$\mathrm{Cov}(c)$中.换句话讲$h_c$是$c$的覆盖.
    	\item (基变换)如果$S$是$c$的覆盖,如果$f:c'\to c$是任意态射,那么$f^*S$是$c'$的覆盖.
    	\item (局部)设$S$是对象$c$的覆盖,设$T$是$c$的任意筛,如果对任意$S$中的态射$f:c'\to c$,都有$f^*T$是$c'$的覆盖,那么$T$是$c$的覆盖(对应到拓扑空间就是说,如果一个开子集的开覆盖中的每个开子集都取一个开覆盖,那么全体这些开覆盖就构成最初开子集的开覆盖).
    \end{enumerate}
    \item 设$(\mathscr{C},T)$是一个site.其上的预层是指$F:\mathscr{C}\to\textbf{Sets}$的逆变函子,这和拓扑$T$无关.如果对任意对象$c$和其上的任意覆盖$S$,典范自然变换$S\to h_X$诱导的如下典范映射是单射/双射,就称$F$是可分预层/层.
    $$\mathrm{Nat}(h_X,F)\to\mathrm{Nat}(S,F)$$
    
    按照米田引理,$F$是层就等价于讲:记$S=\{f_i:c_i\to c\mid i\in I\}$,取定一组$\{s_i\in F(c_i)\}$,满足对任意态射$g:c_i\to c_j$都有$F(g)(s_j)=s_i$,那么存在唯一的对象$s\in F(c)$,满足$F(f_i)(s)=s_i$.
    \begin{enumerate}[(1)]
    	\item 如果把$\textbf{Sets}$改为$\textbf{Ab}$,定义的预层和层就称为阿贝尔预层和阿贝尔层.它们之间的态射就是作为函子的态射.$\textbf{Sh}(\mathscr{C},T)$是具有足够多内射对象的阿贝尔范畴.
    	\item 关于正合性.阿贝尔层的序列$\xymatrix{F\ar[r]^f&G\ar[r]^g&H}$称为正合的,如果对任意对象$c$和任意满足$g(s)=0$的$s\in G(c)$,都存在一个局部的$t$满足$f(t)=s$,换句话讲存在$c$的覆盖$S=\{f_i:c_i\to c\mid i\in I\}$,存在截面族$\{t_i\in F(c_i)\mid i\in I\}$,满足$f(t_i)=G(f_i)(s)$.
    \end{enumerate}
    \item 上同调.设小范畴$\mathscr{C}$具有终对象$X$,整体截面函子$\Gamma(X,-):\textbf{Sh}(\mathscr{C},T)\to\textbf{Ab}$是左正合的,它的右导出函子列记作$\mathrm{H}^p(X,-)$.其中$\mathrm{H}^1(X,\mathscr{F})$依旧是$\mathscr{F}$挠子(torsor)的同构类群.这里一个$\mathscr{F}$挠子是指一个层$\mathscr{G}$附带一个$\mathscr{F}$作用,使得局部上这个作用是$\mathscr{F}$上的平移作用.
\end{enumerate}
\subsection{平展拓扑上的层}

设$X$是概形,设$\mathscr{C}$是全体平展$X$概形构成的范畴,这里的$X$态射也是平展态射.定义$\mathscr{C}$上的平展覆盖为满足$Y=\cup_if_i(Y_i)$的筛$\{f_i:Y_i\to Y\}$,由此定义的Grothendieck拓扑称为平展拓扑.赋予$\mathscr{C}$平展拓扑称为平展site,记作$X_{\mathrm{et}}$.
\begin{enumerate}
	\item 【】回顾平展态射.环$A$上的代数$B$称为平展代数,如果它是有限表示代数,并且满足如下等价条件的任一:
	\begin{enumerate}[(a)]
		\item 对任意$A$代数$C$和$C$的任意满足$J^2=0$的理想$J$,有如下典范映射是双射:
		$$\mathrm{Hom}_{\textbf{Alg}(A)}(B,C)\to\mathrm{Hom}_{\textbf{Alg}(A)}(B,C/J)$$
		\item $B$是平坦$A$代数,并且有微分模$\Omega_{B/A}=0$.
		\item 记$B=A[T_1,\cdots,T_n]/I$,对$A[T_1,\cdots,T_n]$的任意包含$I$的素理想$\mathfrak{p}$,都存在多项式$P_1,\cdots,P_n\in I$,使得$I_{\mathfrak{p}}$被$P_1,\cdots,P_n$的像生成,并且$\det\left(\frac{\partial P_i}{\partial T_j}\right)\not\in\mathfrak{p}$.
	\end{enumerate}

    概形态射$f:X\to S$称为平展态射,如果对任意$x\in X$都存在$f(x)$仿射开邻域$U=\mathrm{Spec}A$以及$x$的仿射开邻域$V=\mathrm{Spec}B\subseteq f^{-1}(U)$使得$B$是平展$A$代数.例如域上的平展代数就是它有限格有限可分域扩张的乘积.再比如设$f:X\to S$是$\mathbb{C}$上有限型概形之间的态射,那么它是平展态射当且仅当$f^{\mathrm{an}}:X^{\mathrm{an}}\to S^{\mathrm{an}}$是局部同构.
    \item 关于模层.设$X$是概形,设$\mathscr{C}$是一些$X$概形构成的$\textbf{Sch}(X)$的完全子范畴,使得它在纤维积下封闭.设$S$是$X$的筛,$X$的$S$拟凝聚模层是指如下信息:
    \begin{enumerate}[(a)]
    	\item 对每个$U\in S$,取一个$U$的拟凝聚模层$E_U$.
    	\item 对任意$U\in S$和任意$\mathscr{C}$中的态射$\varphi:V\to U$,有同构$\rho_{\varphi}:E_V\cong\varphi^*E_U$.
    	\item 如果$\psi:W\to V$是另一个$\mathscr{C}$中的态射,那么有如下交换图表:
    	$$\xymatrix{E_W\ar[rr]^{\rho_{\varphi\circ\psi}}\ar[dr]_{\rho_{\psi}}&&\psi^*\varphi^*E_U\\&\psi^*E_V\ar[ur]_{\psi^*\rho_{\varphi}}&}$$
    \end{enumerate}
    
    如果$E$是$X$上的拟凝聚模层,对每个$(f:U\to X)\in S$,取$E_U=f^*E$,那么这些$\{E_U\}$构成$X$的$S$拟凝聚模层.如果$\{U_i\}\subseteq\mathscr{C}$是有限个平坦$X$概形构成的集合,使得它们的像集的并是整个$X$,设它们生成的筛是$S$.那么$E\mapsto\{E_U\}$是从$\textbf{Mod}(\mathscr{O}_X)$到$X$上$S$拟凝聚模层范畴的范畴等价.
    \begin{proof}
    	
    	问题归结为设$X=\mathrm{Spec}A$是仿射的,设$\{U_i\}=\{U\}$只由一个仿射$X$概形$U=\mathrm{Spec}B$构成,那么此时$B$是忠实平坦$A$代数.【】
    \end{proof}
    \item 例子.
    \begin{enumerate}[(1)]
    	\item 常值层.设$A$是阿贝尔群,设$X$是诺特概形,定义$A$的常值层为$\mathscr{F}(U)=A^{\pi_0(U)}$,其中$\pi_0(U)$表示$U$的全部有限个连通分支构成的集合.特别的如果取$A=\mathbb{Z}/n$,那么$\mathrm{H}^1_{\mathrm{et}}(X,\mathbb{Z}/n)$是全部$\mathbb{Z}/n$挠子的同构类,也即有$\mathrm{H}^1_{\mathrm{et}}(X,\mathbb{Z}/n)=\mathrm{Hom}(\pi_1(X,x_0),\mathbb{Z}/n)$.
    	\item 乘法层.定义$X_{\mathrm{et}}$上的乘法层$\mathbb{G}_{m,X}$为$U\mapsto\Gamma(U,\mathscr{O}_U^*)$.按照定义就有$\mathrm{H}^0_{\mathrm{et}}(X,\mathbb{G}_{m,X})=\mathrm{H}^0(X,\mathscr{O}_X)^*$.特别的如果$X$是代数闭域$k$上的既约连通紧合概形,那么有$\mathrm{H}^0_{\mathrm{et}}(X,\mathbb{G}_{m,X})=k^*$.我们断言有:
    	$$\mathrm{H}^1_{\mathrm{et}}(X,\mathbb{G}_{m,X})=\mathrm{Pic}(X)$$
    	\begin{proof}
    		
    		【】设$\mathscr{L}$是$X$上的可逆层,对任意平展态射$\varphi:U\to X$,取$\mathscr{L}^*(U)=\mathrm{Isom}_U(\mathscr{O}_U,\varphi^*\mathscr{L})$,那么$\mathscr{L}^*$是$X_{\mathrm{et}}$上的阿贝尔层,它是一个$\mathbb{G}_{m,X}$挠子.这个映射在挠子同构类上是双射.
    	\end{proof}
    	\item 单位根层.设$n$是正整数,$\mathbb{G}_{m,X}$上的$n$次幂同态的核称为$n$次单位根层,记作$\mu_n$.如果$X$是可分代数闭域$k$上的概形,并且$(n,\mathrm{char}k)=1$,那么取定$k$的一个$n$次单位根$\zeta$就定义了同构$\mathbb{Z}/n\cong\mu_n$,$i\mapsto\zeta^i$.于是$\mu_n$系数上同调和$\mathbb{G}_{m,X}$系数上同调就可以通过如下短正合列联系起来:
    	$$\xymatrix{0\ar[r]&\mu_n\ar[r]&\mathbb{G}_{m,X}\ar[r]&\mathbb{G}_{m,X}\ar[r]&0}$$
    \end{enumerate}
    \item 几何点的平展邻域.设$X$是概形,设$\overline{x}:\mathrm{Spec}\Omega\to X$是几何点(此即要求$\Omega$是可分代数闭域),$\overline{x}$的平展邻域是指如下交换图表,其中$U\to X$是平展态射:
    $$\xymatrix{&U\ar[d]\\\mathrm{Spec}\Omega\ar[r]_{\overline{x}}\ar[ur]&X}$$
    \item 严格局部环.定义$X$在$\overline{x}$处的严格局部环(strict localization)为$\mathscr{O}_{X,\overline{x}}=\varinjlim\Gamma(U,\mathscr{O}_U)$,其中$U$跑遍$\overline{x}$的平展邻域.这是一个严格汉森局部环(strict henselian local ring),它的剩余域是$\kappa(x)$在$\Omega$中的可分闭包.这两个概念充当平展拓扑下的茎和剩余域.
    \item 平展纤维.设$F$是$X_{\mathrm{et}}$的阿贝尔层,它在$X$的几何点$\overline{x}$处的纤维定义为$F_{\overline{x}}=\varinjlim F(U)$,其中$U$跑遍$\overline{x}$的平展邻域.
    \begin{enumerate}[(1)]
    	\item $X_{\mathrm{et}}$上的模层态射$F\to G$是单态射/满态射/同构当且仅当对任意几何点$\overline{x}$有诱导的$F_{\overline{x}}\to G_{\overline{x}}$是单射/满射/双射.
    	\item 如果$X/k$是代数闭域上的有限型概形,那么$X_{\mathrm{et}}$上的模层态射$F\to G$是单态射/满态射/同构当且仅当对任意有理点$x$有诱导的$F_x\to G_x$是单射/满射/双射.
    \end{enumerate}
    \item 正像.设$f:X\to Y$是概形态射,设$F$是$X_{\mathrm{et}}$上的阿贝尔层,对任意平展$Y$概形$V$,记$f_*F(V)=F(X\times_YV)$,于是$f_*$是$\textbf{Sh}(X_{\mathrm{et}})\to\textbf{Sh}(Y_{\mathrm{et}})$的左正合函子.它的右导出函子列记作$\mathrm{R}^qf_*$,称为$f$的高阶正像函子.如果$\overline{y}$是$Y$的几何点,严格局部环有如下表示,其中$V$跑遍$\overline{y}$的平展邻域:
    $$(\mathrm{R}^qf_*F)_{\overline{y}}=\varinjlim\mathrm{H}^q_{\mathrm{et}}(V\times_YX,F)$$
    \item 设$\overline{y}$是$Y$的几何点,设$\widetilde{Y}=\mathrm{Spec}\mathscr{O}_{Y,\overline{y}}$,设$\widetilde{X}=X\times_Y\widetilde{Y}$,那么$X_{\mathrm{et}}$上的阿贝尔层$F$可以延拓到$\widetilde{X}_{\mathrm{et}}$上:任取$\widetilde{X}$平展概形$\widetilde{U}$,那么存在$\overline{y}$的平展邻域$V$以及平展$X\times_YV$概形$U$使得$\widetilde{U}=U\times_V\widetilde{Y}$,我们取:
    $$F(\widetilde{U})=\varinjlim F(U\times_VV')$$
    
    其中$V'$跑遍$\overline{y}$的支配了$V$的平展邻域.按照定义就有:
    $$(\mathrm{R}^qf_*F)_{\overline{y}}=\mathrm{H}^q_{\mathrm{et}}(\widetilde{X},F)$$
    \item 逆像.$f_*$的左伴随函子记作$f^*:\textbf{Sh}(Y_{\mathrm{et}})\to\textbf{Sh}(X_{\mathrm{et}})$.任取$X$的几何点$\overline{x}$,那么有$(f^*F)_{\overline{x}}=F_{f(\overline{x})}$.于是$f^*$是正合函子.
    \item 借助谱序列,我们有:设$F$是$X_{\mathrm{et}}$上的阿贝尔层,设$\xymatrix{X\ar[r]^f&Y\ar[r]^g&Z}$是概形态射.如果$\mathrm{R}^qf_*F=0,\forall q\ge1$,那么当$p\ge0$时有$\mathrm{H}^p(Y,f_*F)=\mathrm{H}^p(X,F)$和$\mathrm{R^pg_*(f_*F)}=\mathrm{R}^p(gf)_*F$.
    \item 推论.设$f:X\to Y$是有限态射(取极限可以说明改为整态射也成立),设$F$是$X$上的阿贝尔层,那么对$q\ge1$有$\mathrm{R}^qf_*F=0$.
\end{enumerate}
\subsection{Galois上同调}

对于域的素谱$X=\mathrm{Spec}K$,它的平展上同调就是Galois上同调.
\begin{enumerate}
	\item 设$K$是域,设$G=\mathrm{Gal}(\overline{K}/K)$是绝对Galois群,那么$A\mapsto\mathrm{Hom}_K(A,\overline{K})$是从有限平展$K$代数范畴到附带连续$G$作用的有限集合范畴的逆变范畴等价.
	\item 固定$X=\mathrm{Spec}K$的几何点$\overline{x}:\mathrm{Spec}\overline{K}\to\mathrm{Spec}K$,那么$F\mapsto F_{\overline{x}}$是从$X$上的平展层范畴到附带连续$G$作用的有限集合范畴的范畴等价.
	
	\qquad
	
	设$G$连续作用在有限集合$E$上,那么这个拟逆函子很好构造:设有限平展$K$代数$A$满足$\mathrm{Hom}_K(A,\overline{K})$是$G$同构于$E$的.取$U=\mathrm{Spec}A$,那么$F(U)=\mathrm{Hom}_{G-\textbf{Sets}}(U(\overline{K}),F_{\overline{K}})$是对应的层.特别的有$F(X)=F_{\overline{K}}^G$.于是取导出函子列就得到平展上同调和群上同调一致:
	$$\mathrm{H}^q_{\mathrm{et}}(X_{\mathrm{et}},F)=\mathrm{H}^q(G,F_{\overline{K}})$$
\end{enumerate}
\subsection{Brauer群}














\newpage
\section{曲线}
\subsection{准曲线}

域$k$上的准曲线是指域$k$上的有限型分离等维数1概形.一个紧合准曲线通常也称为完备的准曲线.
\begin{enumerate}
	\item 因为维数的缘故,准曲线的一维不可约闭子集对应一般点,零维不可约闭子集对应闭点;不可约准曲线的真闭子集是由有限个闭点构成的.
	\item 设$f:X\to Y$是不可约$k$准曲线之间的$k$态射,那么只会出现如下两种情况之一:
	\begin{enumerate}[(1)]
		\item $f$不是支配的,等价于$f$是常值态射,此时像是$Y$的闭点(也即非一般点).
		\item $f$是支配的,等价于$f$具有有限纤维,也等价于$f$是拟有限的(因为这里它是有限型态射是自动的).
	\end{enumerate}
	
	特别的,不可约准曲线之间的紧合态射要么是常值的,要么是有限满射态射.
	\begin{proof}
		
		$\overline{f(X)}$是$Y$的不可约闭子集,如果它是单点闭集,归结为第一种情况.如果它不是单点集,那么$\overline{f(X)}=Y$,也即$f$是支配态射.此时任取闭点$y\in Y$,那么$f^{-1}(y)$是$X$的真闭子集,于是它由有限个闭点构成.于是$f$具有有限纤维,并且这不会出现在第一种情况中,得证.
	\end{proof}
    \item 准曲线上正则和正规等价,因为一维诺特环正则和正规等价.
    \item 设$X/k$是既约准曲线,设$\mathscr{A}$是$X$的既约凝聚$\mathscr{R}_X$代数层,其中$\mathscr{R}_X$是有理函数层,那么$X$在$\mathscr{A}$中的正规化$X'$是正规准曲线,并且典范态射$X'\to X$是有限的.
    \begin{proof}
    	
    	我们解释过$\mathscr{R}_X$是拟凝聚代数层.$X'\to X$有限在正规化中解释过了.于是$X'$是有限型分离$k$概形.它是等维数1的因为一般点之间的对应$\kappa(\eta_j')/\kappa(\eta_i)$是有限扩张,所以$X'$在一般点处的剩余域在$k$上的超越维数仍然是1.于是$X'/k$是准曲线.最后正规化里我们解释过$X'$是有限个正规整分离概形的无交并,于是$X'$是正规的.
    \end{proof}
    \item 既约准曲线$X/k$是紧合的当且仅当它在有理函数层$\mathscr{R}_X$中的正规化$X'$是紧合的.
    \begin{proof}
    	
    	这件事是因为$X'\to X\to\mathrm{Spec}k$的紧合性等价于$X\to\mathrm{Spec}k$的紧合性,因为$X'\to X$按照上一条是有限的.
    \end{proof}
    \item 设$X/k$是正规准曲线,设$Y/k$是紧合概形,那么$X\to Y$的任意$k$有理映射都在$X$上处处有定义,换句话讲它总是态射.
    \begin{proof}
    	
    	按照【EGAII7.3.7】,那些没定义的点构成的集合是余维数$\ge2$的,但是曲线时候这只能是空集.
    \end{proof}
    \item 正规准曲线$X/k$总是拟射影的.【事实上域上的准曲线总是拟射影的】
    \begin{proof}
    	
    	我们知道$X$是有限个正规整准曲线的无交并,所以问题归结为设$X$是正规整准曲线.按照$X$是诺特的,它被有限个仿射开子集$\{U_1,\cdots,U_r\}$覆盖,这里$U_i$都是$k$上有限型仿射概形,所以有$k$嵌入$f_i:U_i\to\mathbb{P}_k^{n_i}$.按照上一条这里$f_i$必然延拓到整个$X$上,记作$g_i:X\to\mathbb{P}_k^{n_i}$.进而得到纤维积态射$g:X\to\times\mathbb{P}_k^{n_i}$.但是我们知道源端不可约分离的嵌入是一个源端局部性质【EGAI8.2.8】,所以这里$g$也是嵌入.再取Segre态射$\mathbb{P}_k^{n_i}\to\mathbb{P}_k^N$,这是一个闭嵌入,于是我们构造了$X$到射影空间的嵌入,于是$X$是拟射影的.
    \end{proof}
    \item 任何正规准曲线$X$都同构于某个紧合正规准曲线$\widetilde{X}$的稠密开子概形,并且$\widetilde{X}$在同构意义下唯一.
    \begin{proof}
    	
    	按照正规准曲线到紧合准曲线的有理映射总是处处有定义的,所以两个紧合准曲线如果存在稠密开子集是同构的,这个同构就可以延拓为两个紧合准曲线之间的同构,这就得到唯一性.下面构造存在性,按照上一条有$X$是某个$\mathbb{P}_k^n$的子概型.取概形闭包$\overline{X}$.它就是$X$在$\mathbb{P}_k^n$中的拓扑闭包赋予既约闭子概型结构.$X$是$\overline{X}$的稠密开子集.因为$X$的一般点恰好就是$\overline{X}$的一般点,并且对应的一般点具有相同剩余域,所以$\overline{X}$是等维数1的,它有限型和分离都是平凡的,它是紧合的因为它是射影空间的闭子概型,所以$\overline{X}$是既约紧合准曲线.再取$\overline{X}$的正规化$\widetilde{X}$,这是紧合正规准曲线.记结构态射$h:\widetilde{X}\to\overline{X}$,按照$X$已经是正规的了,所以$h$限制在$h^{-1}(X)$上是到$X$的同构.最后按照$h^{-1}(X)$包含了$\widetilde{X}$的全部一般点,于是$X\cong h^{-1}(X)$在$\widetilde{X}$中稠密,得证.
    \end{proof}
    \item 设$X$是正规整准曲线,函数域记作$K$,设$Y$是整紧合准曲线,函数域记作$L$,那么存在从支配$k$态射$\mathrm{Hom}_k^{\mathrm{dom}}(X,Y)$到$\mathrm{Hom}_k(L,K)$的一一对应.
    \begin{proof}
    	
    	我们解释过有理映射集合$\mathrm{Rat}_k(X,Y)$和$(y,\varphi)$一一对应,其中$y\in Y$和$\varphi\in\mathrm{Hom}_k(L,K)$.但是我们还解释过源端正规终端紧合的时候有理映射总可以延拓为态射,所以$\mathrm{Rat}_k(X,Y)=\mathrm{Hom}_k(X,Y)$.当我们考虑支配态射时,对应的$y$恰好是$Y$的一般点,此时就有$\mathrm{Hom}_k^{\mathrm{dom}}(X,Y)$一一对应于$\mathrm{Hom}_k(L,K)$.
    \end{proof}
    \item 设$X/k$是整概形,设$\mathbb{P}^1_k$的一个无穷远点为$\infty$,那么存在从函数域$K=K(X)$到$\mathrm{Hom}_k(X,\mathbb{P}_k^1)-\{\infty\}$的一一对应,其中$\infty$理解为取值为$\infty$的常值态射.并且在这个对应下,一个元$a\in K$对应的态射$u$是支配的当且仅当$a$在$k$上超越.
    \begin{proof}
    	
    	我们解释过$\mathrm{Rat}_k(X,\mathbb{P}_k^1)$一一对应于$\mathrm{Hom}_k(K,\mathbb{P}_k^1)$,也即$\mathbb{P}_k^1$上的$K$值点.如果这个$K$值点在$\mathbb{P}_k^1$的一般点$\xi$之上,则这个有理映射是支配的.否则这个$K$值点在$\mathbb{P}_k^1$的某个闭点$s$之上,记这个$K$值点对应的有理映射$u$对应的极大定义域的态射为$f:U\to\mathbb{P}_k^1$.那么$f$把$U$映到闭点$s$,于是此时$u$不会是支配的,它对应于取值为$s$的常值态射.【】
    \end{proof}
    \item 取函数域是从$k$上紧合正规整准曲线和支配$k$态射构成的范畴到$k$上超越维数1的有限型域扩张范畴之间的逆变范畴等价.
    \begin{proof}
    	
    	只剩下证明本质满的.任取超越次数为1的有限型域扩张$K/k$.那么$K$是某个$k(T)$的有限扩张,其中$T\in K$是$k$上超越元.$k(T)$可以典范的视为$Y=\mathbb{P}_k^1$的函数域.设$X$是$Y$在$K$中的正规化,那么$X$是以$K$为函数域的正规整准曲线,它是紧合的因为$X\to Y$是有限的.
    \end{proof}
\end{enumerate}
\subsection{Riemann-Roch定理}

\begin{enumerate}
	\item 关于Serre对偶的补充.设$f:X\to Y$是终端仿射的紧合态射.设$\omega_f$是$f$的Grothendieck $r$-对偶模层.对拟凝聚层$\mathscr{F}$,我们有如下典范同构:
	$$\mathrm{H}^0(X,\mathscr{F}^{\vee}\otimes\omega_f)\cong\mathrm{Hom}_{\mathscr{O}_X}(\mathscr{F},\omega_f)\cong\mathrm{H}^r(X,\mathscr{F})^{\vee}$$
	
	设$f:X\to\mathrm{Spec}k$是射影曲线,设$\omega_f$是$f$的Grothendieck $1$-对偶模层.如果$f$是局部完全交概形(例如$X$是正则的),此时$\omega_f$恰好是典范层$\omega_{X/k}$;如果$X$是光滑的,此时$\omega_{X/k}=\Omega_{X/k}^1$是微分层.对任意可逆层$\mathscr{L}$有:
	$$\mathrm{H}^0(X,\mathscr{L}^{\vee}\otimes\omega_f)\cong\mathrm{Hom}_{\mathscr{O}_X}(\mathscr{L},\omega_f)\cong\mathrm{H}^1(X,\mathscr{L})^{\vee}$$
	\item 亏格.设$X/k$是射影曲线,我们定义过它的算术亏格$p_a(X)=1-\chi_k(\mathscr{O}_X)$.它的几何亏格定义为$p_g(X)=h^0(X,\omega_{X/k})$.
	\begin{enumerate}[(1)]
		\item 按照平坦基变换和上同调可交换,对任意域扩张$K/k$有$p_a(X_K)=p_a(X)$.
		\item 如果$X$是几何连通和几何既约的射影曲线,此时有$\mathrm{H}^0(X,\mathscr{O}_X)=k$,于是此时$p_a(X)=h^1(X,\mathscr{O}_X)$.
		\item 如果$X/k$是局部完全交的,按照Serre对偶,我们有:
		$$\mathrm{H}^0(X,\omega_{X/k})\cong\mathrm{H}^1(X,\mathscr{O}_X)^{\vee},\quad\mathrm{H}^0(X,\mathscr{O}_X)\cong\mathrm{H}^1(X,\omega_{X/k})^{\vee}$$
		
		进而有$\chi(\omega_{X/k})=-\chi(\mathscr{O}_X)$.于是如果$X$是光滑和几何连通的,那么有$p_a(X)=p_g(X)$(但是高维情况这两个亏格一般不一致).
	\end{enumerate}
	\item 回顾除子.设$X/k$是代数闭域上的整紧合光滑曲线,一个(Weil)除子指的是它的全部闭点生成的自由阿贝尔群中的元$D=\sum_in_iP_i$,除子的次数定义为$\deg D=\sum n_i$.两个除子称为线性等价的,如果它们的差是一个主除子,也即$X$上有理函数的除子.由于射影非奇异曲线上的主除子的次数总是零,于是除子的次数是定义在除子类群上的.这里$X$上的Weil除子类群,Cartier除子类群和皮卡群都是典范同构的.
	\item 空间$L(D)$.设$X/k$是射影曲线,设$D$是Cartier除子,定义$L(D)=\mathrm{H}^0(X,\mathscr{L}(D))$和$l(D)=\dim_kL(D)$.
	\begin{enumerate}[(1)]
		\item 如果$X$是整的,我们解释过:
		$$L(D)=\{f\in K(X)^*\mid\mathrm{div}(f)+D\ge0\}\cup\{0\}$$
		\item 如果$X$是正规的,此时Cartier除子和Weil除子一致,记$D=\sum_xn_x[x]$,那么有:
		$$L(D)=\{f\in K(X)^*\mid\mathrm{mult}_x(f)+n_x\ge0,\forall x\mbox{闭点}\}\cup\{0\}$$
	\end{enumerate}
	\item (Riemann-Roch)设$f:X\to\mathrm{Spec}k$是射影曲线,设$D$是$X$上的Cartier除子,那么有如下等式,其中$\chi$是欧拉示性数.
	$$\chi(\mathscr{L}(D))=\deg D+\chi(\mathscr{O}_X)$$
	\begin{enumerate}[(1)]
		\item 设$\omega_f$是$f$的Grothendieck 1-对偶模层.按照Serre对偶,对任意可逆层$\mathscr{L}$有:
		$$\mathrm{H}^0(X,\mathscr{L}^{\vee}\otimes\omega_f)\cong\mathrm{Hom}_{\mathscr{O}_X}(\mathscr{L},\omega_f)\cong\mathrm{H}^1(X,\mathscr{L})^{\vee}$$
		
		于是Riemann-Roch定理为:
		$$h^0(X,\mathscr{L}(D))-h^0(X,\omega_{X/k}\otimes\mathscr{L}(-D))=\deg D+1-p_a$$
		\item 如果$X/k$是局部完全交的,任意一个满足$\mathscr{L}(K)\cong\omega_{X/k}$的Cartier除子$K$称为典范除子,记作$K_{X/k}$,它只在线性等价意义下唯一(这个存在性是因为对诺特环上拟射影概形已经有典范同构$\mathrm{CaCl}(X)\cong\mathrm{Pic}(X)$).此时Riemann-Roch定理为:
		$$l(D)-l(K-D)=\deg D+1-p_a(X)$$
	\end{enumerate}
	\begin{proof}
		
		之前解释过$D=E-F$,其中$E,F$是两个非零有限除子.我们有如下短正合列:
		$$\xymatrix{0\ar[r]&\mathscr{L}(-F)\ar[r]&\mathscr{O}_X\ar[r]&\mathscr{O}_F\ar[r]&0}$$
		
		张量$\mathscr{O}_X(E)$仍然是正合的:
		$$\xymatrix{0\ar[r]&\mathscr{L}(D)\ar[r]&\mathscr{L}(E)\ar[r]&\mathscr{L}(E)\mid_F\ar[r]&0}$$
		
		因为$X$是一维的,这里$F$(有限除子对应的闭子概形)是有限概形,进而有$\mathscr{L}(E)\mid_F\cong\mathscr{O}_F$.另外零维空间的$\ge1$阶层上同调都平凡,并且我们解释过$\chi(\mathscr{O}_F)=h^0(F,\mathscr{O}_F)=\deg F$,于是我们有:
		$$\chi(\mathscr{L}(E))=\chi(\mathscr{L}(D))+\deg F$$
		
		把$D=0$带入,得到对任意非零有限除子$E$有$\chi(\mathscr{L}(E))=\chi(\mathscr{O}_X)+\deg E$.于是有$\chi(\mathscr{L}(D))=\chi(\mathscr{O}_X)+\deg E-\deg F=\deg D+\chi(\mathscr{O}_X)$.
	\end{proof}
    \item 线丛的次数.设$X/k$是射影曲线,设$\mathscr{L}$是可逆层,定义它的次数为$\deg_k\mathscr{L}=\chi_k(\mathscr{L})-\chi_k(\mathscr{O}_X)$.
    \begin{enumerate}[(1)]
    	\item 如果$\mathscr{L}\cong\mathscr{L}(D)$,其中$D$是Cartier除子,那么有$\deg D=\deg\mathscr{L}$.
    	\item 映射$\mathscr{L}\mapsto\deg\mathscr{L}$诱导了群同态$\mathrm{Pic}(X)\to\mathbb{Z}$.
    \end{enumerate}
    \item 推论.设$X/k$是射影曲线.
    \begin{enumerate}[(1)]
    	\item 对主除子$D$总有$\deg D=0$.
    	\item 如果$D'\le D$是两个Cartier除子,那么有:
    	$$l(D')\le l(D)\le l(D')+\deg(D-D')$$
    	
    	特别的如果$D$是有效的,那么:
    	$$l(D)\le\deg D+h^0(X,\mathscr{O}_X)$$
    	\item 设$X$是整的.对零次的除子$D$,它是主除子当且仅当整体截面非平凡(也即$l(D)\not=0$).
    	\item 设$X$是整的.对负次的除子$D$,它的整体截面平凡(也即$l(D)=0$).
    	\item 设$X$是整的.设$\mathscr{L}$是可逆层,那么$\mathscr{L}\cong\mathscr{O}_X$当且仅当$\deg L=0$并且$\mathrm{H}^0(X,\mathscr{L})\not=0$.
    	\item 如果$X/k$是局部完全交的射影曲线,那么$\deg\omega_{X/k}=2(p_a-1)$.
    	\item 如果$\deg D>2g-2$,那么$\deg(K-D)<0$,于是$l(K-D)=0$,于是有$l(D)=\deg D+1-p_a$.
    	\item 如果$X/k$是几何连通和几何既约的,那么$\mathrm{H}^0(X,\mathscr{O}_X)=k$,于是$p_a(X)=h^1(X,\mathscr{O}_X)$,按照Serre对偶,就有$p_a(X)=h^0(X,\omega_{X/k})=l(K)$.进而有$\deg K=2g-2$.
    \end{enumerate}
    \begin{proof}
    	
    	(2):按照$\mathscr{L}(D')$是$\mathscr{L}(D)$的子层,有$l(D')\le l(D)$.设$D=D'+E$,其中$E$是非零有效除子.我们有如下短正合列:
    	$$\xymatrix{0\ar[r]&\mathscr{L}(D')\ar[r]&\mathscr{L}(D)\ar[r]&\mathscr{O}_E\ar[r]&0}$$
    	
    	取整体截面得到正合列:
    	$$\xymatrix{0\ar[r]&L(D')\ar[r]&L(D)\ar[r]&\mathrm{H}^0(E,\mathscr{O}_E)}$$
    	
    	进而有:$\deg E=h^0(E,\mathscr{O}_E)\ge l(D)-\deg D$.
    	
    	\qquad
    	
    	(3),(4):如果$l(D)\not=0$,可取非零的整体截面$f\in L(D)$,那么$D$线性等价于$\mathrm{div}(f)+D\ge0$,于是$\deg D\ge0$.并且如果$\deg D=0$那么$\mathrm{div}(f)+D=0$,于是$D$是主除子.
    	
    	\qquad
    	
    	(5):必要性平凡.设$\deg\mathscr{L}=0$,并且有非零元$s\in\mathrm{H}^0(X,\mathscr{L})$.那么$\mathscr{L}'=\mathscr{L}\otimes(s\mathscr{O}_X)^{\vee}$包含在常值层$\mathscr{K}_{X/k}$中,于是它对应了有效除子$D$,但是$\deg D=\deg\mathscr{L}=0$,迫使$D=0$,从而$\mathscr{L}\cong\mathscr{O}_X$.
    \end{proof}
\end{enumerate}
\subsection{小亏格曲线}

\begin{enumerate}
	\item 算术亏格0.
	\begin{enumerate}[(1)]
		\item 设$X/k$是整局部完全交射影曲线,算术亏格0.对除子$D$有:
		$$l(D)=\left\{\begin{array}{cc}\deg D+1&\deg D\ge0\\0&\deg D<0\end{array}\right.$$
		\begin{proof}
			
			事实上如果$\deg D<0$,我们已经证明过$l(D)=0$.如果$\deg D\ge0$.首先有$\deg K=2(p_a-1)=-2$.所以此时$\deg(K-D)<0$,于是$l(K-D)=0$,于是Riemann-Roch定理说明$l(D)=\deg D+1$.
		\end{proof}
		\item 引理.设$X/k$是正规射影曲线,那么$X\cong\mathbb{P}_k^1$当且仅当它存在Cartier除子$D$使得$\deg D=1$和$l(D)\ge2$.此时$\mathscr{L}(D)$总是极丰沛的.
		\begin{proof}
			
			必要性是直接的,任取闭点$x\in X$,记$D=x$,那么$\deg D=1$和$l(D)=2$.充分性:设$D$是满足条件的除子.取非零元$g\in L(D)$,那么$D$线性等价于$D'=\mathrm{div}(g)+D\ge0$,那么$\deg D=\deg D'$和$l(D)=l(D')$.所以我们不妨设$D$本身是有效的.但是这迫使$D=[x_1]$,其中$x_1\in X(k)$.按照$l(D)\ge2$,可以取一个$f\in L(D)$使得$\mathrm{div}(f)+D$是有效除子,并且和$D$不线性等价,于是$\mathrm{div}(f)=[x_0]-[x_1]$,其中$x_0\in X(k)$和$x_1$不同.【liuqing7.3.12】说明$f$诱导了$X\cong\mathbb{P}_k^1$.
		\end{proof}
		\item 设$X/k$是几何整射影曲线,$p_a(X)\le0$,那么$X\cong\mathbb{P}_k^1$当且仅当$X(k)\not=\emptyset$.
		\begin{proof}
			
			先证明$X/k$总是光滑的.设$X'$是$X_{\overline{k}}$在$\overline{k}$上的正规化.那么$h^0(X',\mathscr{O}_{X'})=\overline{k}$,于是$p_a(X')\ge0$.按照【liuqing7.5.4】,有$p_a(X)=p_a(X')+\sum_{x\in X}\dim_{\overline{k}}\widetilde{\mathscr{O}_{X_{\overline{k}},x}}/\mathscr{O}_{X_{\overline{k}},x}$.迫使$X_{\overline{k}}=X'$.于是$X_{\overline{k}}$是正则的.我们知道$X/k$是光滑的当且仅当它基变换到代数闭域上是正则的,于是$X/k$是光滑概形.
			
			\qquad
			
			特别的此时$X/k$是正规射影曲线,取有理点$x\in X(k)$,那么取除子$D=[x]$满足上一条的条件,于是$X\cong\mathbb{P}_k^1$.
		\end{proof}
	    \item 考虑$\mathbb{P}^2_k$的被$x^2+y^2+z^2=0$定义的射影曲线$C$,它的算术亏格为零,它不同构于$\mathbb{P}^1_{\mathbb{R}}$因为没有有理点.另外这说明$C$上没有次数为1的线丛.
	    \item 设$X/k$是几何整射影曲线,$p_a(X)\le0$,那么$X$总是$\mathbb{P}^2_k$中的光滑圆锥曲线(二次光滑曲线).
	    \begin{proof}
	    	
	    	光滑性在上一条已经得证了.取典范除子$K$,那么$\deg(-K)=2$且$l(-K)=3$.我们断言$\mathscr{L}(-K)$是极丰沛的,一旦这成立,它和三个线性无关的整体截面就诱导了闭嵌入$X\to\mathbb{P}_k^2$,并且这是二次曲线.按照【liuqing.Ex.5.1.30】,归结为基变换到代数闭域上证明这件事,也即不妨设$k$是代数闭域.此时取$x_1\in X(k)$,那么$l(-K-2x_1)\ge1$,但是$-K-2x_1$是零次除子,它的整体截面非平凡导致它是主除子,于是$-K$线性等价于$2x_1$,于是$\mathscr{L}(-K)\cong\mathscr{L}(x_1)^{\otimes2}$,前面引理解释了$\mathscr{L}(x_1)$是极丰沛的,得到$\mathscr{L}(-K)$是极丰沛的.
	    \end{proof}
        \item 推论.上一条证明说明,在算术亏格0的光滑射影曲线上,次数为1和2的除子总是极丰沛的.进而其上任意正次除子都是极丰硕的.
	\end{enumerate}
    \item 极丰沛层和次数.
    \begin{enumerate}[(1)]
    	\item 引理.设$X/k$是射影曲线,设$D$是有效除子,它的支集中的点都是$X$的正则点.那么$\mathscr{L}(D)$被整体截面生成当且仅当对任意$x\in\mathrm{Supp}(D)$,都有$l(D-x)<l(D)$.
    	\begin{proof}
    		
    		因为$D\ge0$,所以$D=\mathrm{div}(1)+D\ge0$,于是$1\in L(D)$.于是如果$x\not\in\mathrm{Supp}(D)$,那么$\mathscr{L}(D)_x=\mathscr{O}_{X,x}$,于是此时$\mathscr{L}(D)_x$被整体截面生成.于是有$\mathscr{L}(D-x)=\mathfrak{m}_x\mathscr{L}(D)_x$【?】.
    		
    		\qquad
    		
    		如果$l(D-x)<l(D)$,那么可取$s\in L(D)\backslash L(D-x)$,那么$s_x\not\in\mathscr{L}(D-x)_x$【】
    		
    		
    		必要性.设$\mathscr{L}(D)$被整体截面生成.设$s\in L(D)$使得$s_x$是$\mathscr{L}(D)_x$的生成元,那么
    		
    	\end{proof}
        \item 引理.设$X/k$是代数闭域上的连通光滑射影曲线.设$D$是有效除子,满足对任意$p,q\in X(k)$都有$l(D-p-q)<l(D-p)<l(D)$.那么$\mathscr{L}(D)$是极丰沛的.
        \begin{proof}
        	
        	【】
        \end{proof}
        \item 定理.设$X/k$是算术亏格$g$的光滑几何连通射影曲线.设$\mathscr{L}$是可逆层.
        \begin{enumerate}[(a)]
        	\item 如果$\deg\mathscr{L}\ge2g$,那么$\mathscr{L}$被整体截面生成.
        	\item 如果$\deg\ge2g+1$,那么$\mathscr{L}$是极丰沛的.
        \end{enumerate}
        \begin{proof}
        	
        	因为可逆层被整体截面生成和是丰沛层这两个性质如果在经一个忠实平坦基变换回拉后成立,那么原始的可逆层也具有相同的性质.所以问题归结为设$k$是代数闭域.设Cartier除子$D$满足$\mathscr{O}_X(D)\cong\mathscr{L}$.按照Riemann-Roch定理,如果$\deg D\ge2g$,对任意次数$\le1$的有效除子$E$,都有$l(D-E)=l(D)-\deg E$;如果$\deg D\ge2g+1$,对任意次数$\le2$的有效除子$E$有相同的等式.特别的$l(D)\not=0$,所以任取$f\in L(D)$,$D$线性等价于一个有效除子,不妨设$D$本身是有效除子.于是上述等式结合上面两个引理得证.
        \end{proof}
        \item 推论.设$X/k$是算术亏格$g$的光滑几何连通射影曲线.
        \begin{enumerate}[(a)]
        	\item 一个除子$D$是丰沛的(这是指$\mathscr{L}(D)$是丰沛层)当且仅当$\deg D\ge1$.
        	\item 如果$g=0$,那么除子$D$是丰沛的当且仅当它是极丰沛的当且仅当$\deg D\ge1$.
        	\item 如果$g=1$,那么一个除子$D$是极丰沛的当且仅当$\deg D\ge3$.
        	\begin{proof}
        		
        		如果$\deg D=2$,那么$l(K-D)=0$,于是$\dim|D|=l(D)-1=1$,如果$D$是极丰沛的,那么$X$闭嵌入到$\mathbb{P}_k^1$中,但是$X$是一维的迫使$X=\mathbb{P}_k^1$,矛盾.
        	\end{proof}
        \end{enumerate}
    \end{enumerate}
    \item 算术亏格1.
    \begin{enumerate}[(1)]
    	\item 设$X/k$是整局部完全交射影曲线,那么$p_a(X)=1$当且仅当$\omega_{X/k}\cong\mathscr{O}_X$.此时对$\deg D>0$总有$l(D)=\deg D$.
    	\item 椭圆曲线.曲线$X/k$称为椭圆曲线,如果它同构于被如下齐次多项式定义的光滑平面射影曲线.特别的椭圆曲线的算术亏格为1.
    	$$Y^2Z+a_1XYZ+a_3YZ^2=X^3+a_2X^2Z+a_4XZ^2+a_6Z^3$$
    	\item 设$X/k$是光滑连通射影曲线,$p_a(X)=1$,那么它是椭圆曲线当且仅当存在$o\in X(k)$.此时总有$\mathscr{L}(3o)$是极丰沛层.另外存在这样的算术亏格1的光滑射影曲线不存在有理点.
    	\begin{proof}
    		
    		首先具有有理点和连通性保证$X$是几何连通的.取$L(2o)$的一组基$\{1,t\}$,记$L(3o)$的一组基$\{1,t,y\}$.考虑$\{1,t,y,t^2,ty,t^3,y^2\}\in L(6o)$,按照$\dim L(6o)=6$,就有不全为零的系数使得:
    		$$by^2+a_1ty+a_3y=a_0t^3+a_2t^2+a_4t+a_6$$
    		
    		按照$\{1,t,y,t^2\}$是$L(4o)$的一组基,$\{1,t,y,t^2,ty\}$是$L(5o)$的一组基,得到$a_0b\not=0$.适当换元$t,y$可不妨设$a_0=b=1$.那么$\mathscr{L}(3o)$和$\{1,t,y\}$就诱导了一个闭嵌入$X\to\mathbb{P}_k^2$.
    	\end{proof}
    \end{enumerate}
    \item 超椭圆曲线.一个算术亏格$\ge1$的光滑几何连通射影曲线$X/k$称为超椭圆曲线,如果存在次数2的有限可分态射(这里可分指的是函数域扩张是可分扩张)$X\to\mathbb{P}_k^1$.
    \begin{enumerate}[(1)]
    	\item 设$X/k$是算术亏格$\ge1$的光滑几何连通射影曲线,那么$X$是超椭圆曲线当且仅当存在Cartier除子$D$满足$l(D)=\deg D=2$.
    	\begin{proof}
    		
    		充分性:按照$l(D)\not=0$,不妨设$D$是有效除子.取$x\in\mathrm{Supp}(D)$,如果$\deg(D-x)<0$,我们知道负次数除子的整体截面平凡,也即$l(D-x)=0$;如果$\deg(D-x)=0$,我们知道零次除子整体截面非平凡当且仅当它是主除子,而主除子的整体截面是$\mathrm{H}^0(X,\mathscr{O}_X)$,这在光滑几何连通条件下恰好是基域$k$,于是此时$l(D-x)=0$或$1$;如果$\deg(D-x)=1$,按照$X$不是射影线(亏格不是0),迫使$l(D-x)\le1$.于是我们总有$l(D-x)<l(D)=2$,前面引理保证$\mathscr{L}(D)$被整体截面生成.按照$l(D)=2$,它就定义了一个次数为2的有限态射$\pi:X\to\mathbb{P}_k^1$.我们来证明它是可分的:如果$K(X)$是$k(t)$的二次纯不可分扩张.适当替换$t$使得$K(X)=k(\sqrt{t})$,我们知道正规射影曲线之间函数域的扩张对应了曲线之间的正规化态射【liuqing7.3.13】,于是$k(\sqrt{t})\cong K(t)$得到$X$双有理等价于$\mathbb{P}^1_k$,这导致它的算术亏格为零,矛盾.
    		
    		\qquad
    		
    		必要性:设$\pi:X\to\mathbb{P}_k^1$是次数为2的可分有限态射.取定有理点$y_0\in\mathbb{P}_k^1$.取$X$上的除子$D=\pi^*[y_0]$.那么我们知道$\deg D=\deg\pi\deg[y_0]=2$.按照$L(D)$包含$\mathrm{H}^0(\mathbb{P}_k^1,\mathscr{O}_{\mathbb{P}_k^1}(y_0))$,并且后者的维数是2,于是$l(D)\ge2$.另一方面取$x\in\mathrm{Supp}(D)$,那么有$l(D)\le\deg(D-x)+l(x)=2$,于是$l(D)=2$.
    	\end{proof}
        \item 设$X/k$是光滑几何连通射影曲线,设它是椭圆曲线或者算术亏格2,那么$X$总是超椭圆曲线.
        \begin{proof}
        	
        	先设$X$是椭圆曲线,取$o\in X(k)$,那么$\deg(2o)=2$和$l(2o)=2$.于是$X$是超椭圆曲线.再设$X$的算术亏格$g=2$,取典范除子$K$,那么$\deg K=2g-2=2$.另外条件下有$\mathrm{H}^0(X,\mathscr{O}_X)=k$,按照Serre对偶有$p_a(X)=1-h^0(X,\mathscr{O}_X)+h^1(X,\mathscr{O}_X)=h^0(X,\omega_{X/k})=l(K)$.于是$X$是超椭圆曲线.
        \end{proof}
        \item 设$X/k$是光滑几何连通射影曲线,$p_a(X)\ge1$.那么微分模$\Omega_{X/k}^1$被整体截面生成.此时$l(K)=g$,在选取$L(K)$一组基的前提下就诱导了态射$X\to\mathbb{P}_k^{g-1}$,称为典范态射.
        \begin{proof}
        	
        	我们知道被整体截面生成可以在忠实平坦基变换上验证,所以归结为设$k$是代数闭的.取典范除子$K$,那么有$l(K)=p_a(X)>0$,于是可以不妨设$K$是有效的.取$x\in X(k)$,因为$X$不可能是射影线,迫使$l(x)=1$.于是按照Riemann-Roch定理有$l(K-x)=\deg(K-x)+l(x)+1-g=g-1<l(K)$.这导致$\Omega_{X/k}$被整体截面生成.
        \end{proof}
        \item 设$X/k$是光滑几何连通射影曲线,$p_a(X)\ge2$.那么典范态射$X\to\mathbb{P}_k^{g-1}$是闭嵌入当且仅当$X_{\overline{k}}/\overline{k}$不是超椭圆曲线.
        \begin{proof}
        	
        	不妨设$k$是代数闭的.设$K$是$X$上的典范除子.如果$X$不是超椭圆曲线,取次数2的有效除子$E$,那么$l(E)=1$.Riemann-Roch定理说明$l(K-E)=l(E)+\deg(K-E)+1-g=g-2$.上一条已经证明了$l(K-x)=g-1$.所以我们证明了$l(K-p-q)<l(K-p)<l(K)$.于是$K$是极丰沛的,也即诱导了闭嵌入.【另一侧待补充】
        \end{proof}
    \end{enumerate}
\end{enumerate}
\subsection{Hurwitz公式}

\begin{enumerate}
	\item 回顾分歧性.设$f:X\to Y$是正规射影曲线之间的有限态射,那么$f$已经是平坦态射(例如【HartshorneIII.9.7】),此时一个点是非分歧的和是平展的一致.对闭点$x\in X$,用$e_x$表示DVR的同态$\mathscr{O}_{Y,f(x)}\to\mathscr{O}_{X,x}$的分歧指数.
	\begin{itemize}
		\item 称$x$是非分歧点,如果$e_x=1$并且$\kappa(x)/\kappa(f(x))$是(有限)可分扩张.
		\item 称$x$是温分歧点,如果剩余域扩张是可分扩张,并且在剩余域的特征$p>0$时有$p$不整除$e_x$.
		\item 全体分歧点构成的集合称为$f$的分歧中心(ramification locus),这个集合的像称为$f$的分枝中心(branch locus).
		\item 对任意闭点$y\in Y$总有:
		$$\deg f=\sum_{x\in f^{-1}(y)}e_{x/y}[\kappa(x):\kappa(y)]$$
	\end{itemize}
	\item (Hurwitz公式).设$f:X\to Y$是正规射影曲线之间的有限可分态射(可分指的是函数域扩张是可分的).那么有如下等式,其中求和让$x$跑遍$X$的闭点,$e_x'$是一个$\ge e_x$的整数,满足在$x$关于$f$温分歧的时候有$e_x'=e_x$.
	$$2p_a(X)-2=(\deg f)(2p_a(Y)-2)+\sum_x(e'_x-1)[\kappa(x):k]$$
	\begin{proof}
		
		记$f$的Grothendieck 1-对偶模层为$\omega_f$,那么有$\omega_{X/k}=\omega_f\otimes f^*\omega_{Y/k}$【liuqing6.4.26(b)和liuqing6.4.32】.于是有:
		\begin{align*}
			2(p_a(X)-2)&=\deg\omega_{X/k}\\&=\deg f^*\omega_{Y/k}+\deg\omega_f\\&=\deg f\deg\omega_{Y/k}+\deg\omega_f\\&=(\deg f)(p_a(Y)-2)+\deg\omega_f
		\end{align*}
		
		【】
	\end{proof}
    \item 关于纯不可分的情况.设$k$的特征$p>0$,记$F_k:\mathrm{Spec}k\to\mathrm{Spec}k$表示Frobenius态射,如果$X$是$k$上的概形,记$X^{(p^r)}$表示$X$关于$r$个$F_k$的复合的基变换.设$f:X\to Y$是正规$k$射影曲线之间的有限纯不可分态射.设$X/k$是光滑的,那么存在自然数$r\ge0$使得作为概形有$Y\cong X^{(p^r)}$(它们作为$\mathbb{F}_p$概形也是同构的,因为$F_k$是$\mathbb{F}_p$同构).特别的,此时总有$p_a(X)=p_a(Y)$.
    \begin{proof}
    	
    	记$[K(X):K(Y)]=p^r$,其中$r\ge0$.按照$X$是几何既约的(可分生成的),存在$k$上的超越元$T$和$k(T)$上的可分代数元$\theta$使得$K(X)=k(T,\theta)$.于是$K(X^{(p^r)})=k(T^{p^r},\theta^{p^r})$,并且它在$K(X)$中的指数是$p^r$【liuqing3.2.27】.但是我们有$k(T^{p^r},\theta^{p^r})\subseteq K(Y)$,迫使$K(Y)=K(X^{(p^r)})$.于是有$Y\cong X^{(p^r)}$.最后因为$F_k$是忠实平坦态射,基变换不改变上同调的维数,于是$p_a(X)=p_a(X^{(p^r)})$.
    \end{proof}
	\item 推论.
	\begin{enumerate}[(1)]
		\item 设$f:X\to Y$是正规$k$射影曲线之间的有限可分态射,满足$p_a(X)\ge0$,那么总有$p_a(X)\ge p_a(Y)$.
		\item 设$f:X\to Y$是光滑几何连通$k$射影曲线之间的有限平展态射,设$p_a(Y)=0$,那么$f$是同构.
		\begin{proof}
			
			按照Hurwitz公式,$2p_a(X)-2=-2\deg f$,从$p_a(X)\ge0$得到$\deg f=1$,于是$f$是同构.
		\end{proof}
	\end{enumerate}
    \item 一些简单应用.
    \begin{enumerate}[(1)]
    	\item 不存在从算术亏格2到算术亏格3的光滑整射影曲线之间的非常值态射.
    	\item 对任意域$k$有$\mathbb{P}_k^1$是单连通的,也即如果$f:X\to\mathbb{P}_k^1$是有限平展覆盖,其中$X$连通,那么$f$是一个同构:我们知道对于有限平展覆盖,终端的正则性能传递给源端,于是这里$X$是正则的,按照Hurwitz公式,有$2(p_a(X)-1)=-2\deg f$,迫使$\deg f=1$和$p_a(X)=0$,于是$f$是同构.
    	\item 证明L\"uroth定理:设$k$是域,设$L$是$k\subseteq k(t)$的中间域,那么要么$L=k$,要么$L/k$是纯超越扩张.
    	\begin{proof}
    		
    		设$L\not=k$,那么$L/k$的超越维数为1.取$Y/k$是正规射影曲线,使得$K(Y)=L$.那么域扩张$L\subseteq k(t)$就对应了一个有限态射$f:X=\mathbb{P}^1_k\to Y$.这个有限态射可以分解为可分态射复合纯不可分态射,于是$X\to Y$分解为有限纯不可分态射$X\to Z$复合上有限可分态射$Z\to Y$.我们已经解释了$p_a(Z)=p_a(X)=0$,并且$0=p_a(Z)\ge p_a(Y)$.于是$p_a(Y)=0$.按照$f(X(k))\subseteq Y(k)$知$Y$上存在有理点.于是$Y\cong\mathbb{P}^1_k$,于是$L=K(Y)$是$k$的纯超越扩张.
    	\end{proof}
    \end{enumerate}
\end{enumerate}
\subsection{超椭圆曲线}

一个光滑射影曲线$X/k$称为超椭圆曲线,如果它的算术亏格$g\ge1$,并且带一个次数2的有限可分态射$f:X\to\mathbb{P}_k^1$光滑射影曲线.
\begin{enumerate}
	\item 设$X/k$是算术亏格为$g$的超椭圆曲线,此为带一个次数2的有限可分态射$f:X\to\mathbb{P}_k^1$的$g\ge1$的光滑射影曲线,那么$f$诱导了如下信息.
	\begin{enumerate}[(1)]
		\item 函数域.我们有$K(X)=k(t)[y]$,这里$y$满足$y^2+Q(t)y+P(t)$,其中$P(t),Q(t)\in k[t]$,并且有:
		$$2g+1\le\max\{2\deg Q(t),\deg P(t)\}\le2g+2$$
		
		如果$\mathrm{char}(k)\not=2$,可设$Q(t)=0$.这个方程称为$X$的超椭圆方程,如果$X$是椭圆曲线,则称这个方程是椭圆方程.
		\item $X$是如下两个仿射开子概型的并,其中$Q_1(S)=Q(1/s)s^{g+1}$,$P_1(s)=P(1/s)s^{2g+2}$.这两个开子概型经$D(t)\cong D(s)$,$t=1/s,y=t^{g+1}z$粘合.
		$$U'=\mathrm{Spec}k[t,Y]/(Y^2+Q(t)Y-P(t))$$
		$$V'=\mathrm{Spec}k[s,Z]/(Z^2+Q_1(s)Z-P_1(s))$$
		\item $f$的分歧中心在$\deg(4P(t)+Q(t)^2)>2g+1$时是$V(4P(t)+Q(t)^2)\subseteq U'$;在$\deg(4P(t)+Q(t)^2)\le2g+1$时还要并上$\{s=0\}\in V'$.
	\end{enumerate}
    \begin{proof}
    	
    	记$Y=\mathbb{P}^1_k=U\cup V$,其中$U=\mathrm{Spec}k[t]$和$V=\mathrm{Spec}k[s]$,那么在$U\cap V$上就有$s=1/t$.记$U'=f^{-1}(U)$和$V'=f^{-1}(V)$,那么它们是仿射的,并且有$X=U'\cap V'$.按照Hurwitz公式,有$\deg\omega_f=2g+2$,进而有$h^0(X,\Omega_{X/Y}^1)=2g+2$.
    	
    	\qquad
    	
    	先设$\mathrm{char}(k)\not=2$.那么有$K(X)=k(t)[y]$,其中$y^2\in k(t)$,不妨给$y$添加一个倍数,可设$y^2=P(t)\in k[t]$,其中$P(t)$没有平方因子.$l[t,y]=k[t,Y]/(Y^2-P(t))$是正规整环,于是它就是$\mathscr{O}_Y(U)=k[t]$在$K(X)$中的正规化,于是$\mathscr{O}_X(U')=k[t,y]$.这里$P(t)$是可分多项式,记$d=\deg P$,$r=[(d+1)/2]$,那么$(y/t^r)^2=P_1(s)\in k[s]$没有平方因子.类似的有$\mathscr{O}_X(V')=k[s,z]$,其中$z=t^ry$.我们有:
    	$$\Omega_{U'/U}^1=k[t,y]\mathrm{d}y/k[t,y]2y\mathrm{d}y\cong k[t,y]/(y^2-P(t),y)=k[t]/(P(t))$$
    	
    	类似的有$\Omega_{V'/V}^1=k[s]/(P_1(s))$.按照$Y=U\cup\{s=0\}$,就有:
    	$$\mathrm{H}^0(X,\Omega_{X/Y}^1)=k[t]/(P(t))\oplus k[s]_{\mathfrak{m}}/(P_1(s))$$
    	
    	【】
    \end{proof}
    \item $X$的光滑性就等价于如下描述:如果$\mathrm{char}(k)\not=2$,那么等价于$4P(t)+Q(t)^2$是可分多项式;如果$\mathrm{char}(k)=2$,那么等价于$Q(t)$和$Q'(t)^2P(t)+P'(t)^2$互素.
    \item $\mathrm{H}^0(X,\Omega_{X/k}^1)$以$\left\{\omega_i=\frac{t^i\mathrm{d}t}{(2y+Q(t))}\mid0\le i\le g-1\right\}$为一组基(严格说这个表示是$U'$上的微分模截面,这个记号表示的是粘合得到的整体截面).
    \begin{proof}
    	
    	【】
    \end{proof}
    \item 超椭圆对合.$\mathrm{Gal}(K(X)/k(t))$的生成元$\sigma$诱导了$X$上的自同构,称为超椭圆对合(hyperelliptic involution).一个有理点$x\in X(k)$是$f$分歧的当且仅当$\sigma(x)=x$【liuqing7.4.27】.另外按照$\sigma(y)+y=-Q(t)$,就有$\sigma(\omega_i)=-\omega_i$,于是$\sigma$诱导了$\mathrm{H}^0(X,\Omega_{X/k}^1)$上的映射$-\mathrm{id}$.
    \item 设$X/k$是超椭圆曲线,设$\tau$是另一个超椭圆对合.如果$g\ge2$或者$\sigma,\tau$固定了一个相同的有理点$x_0\in X(k)$,那么$\sigma=\tau$.
    \begin{proof}
    	
    	先设$g\ge2$,那么$\{\omega_0,\omega_1\}$是$\mathrm{H}^0(X,\Omega_{X/k}^1)$的一组基,并且有$\tau(\omega_i)=-\omega_i$.我们有$-\omega_1=\tau(\omega_1)=\tau(t\omega_0)=-\tau(t)\omega_0$,于是$\tau(t)=t$,于是$\tau\in\mathrm{Gal}(K(X)/k(t))$,迫使$\tau=\sigma$.
    	
    	\qquad
    	
    	下面设$g=1$.设$f$对应的超椭圆对合是$\sigma$.不妨设$y_0=f(x_0)$是点$t=\infty$.我们有$L(2[x_0])=k+kt$,于是按照$\tau$固定$x_0$,就有$\tau(k+kt)=k+kt$,进而$\tau$诱导了$k[t]\to k[t]$的同构.如果$\tau(t)=t$,上一段的讨论证明了$\tau=\sigma$.如果$\tau(t)\not=t$,按照$\tau^2=\mathrm{id}$,迫使$\tau(t)=\lambda-t$,其中$\lambda\in k$.如果$\mathrm{char}(k)\not=2$,此时可不妨设$Q(t)=0$.于是$-\mathrm{d}t/2y=-\omega_0=\tau(\omega_0)=\mathrm{d}(\tau(t))/\tau(2y)=-\mathrm{d}t/2\tau(y)$.于是$\tau(y)=y$,进而$\tau(P(t))=P(t)$,这导致$P(t)$的次数是偶数.但是由于$f$在$\infty$分歧,有$P(t)$的次数是3,矛盾.
    	
    	\qquad
    	
    	最后设$\mathrm{char}(k)=2$.按照$\tau^2=\mathrm{id}$,有$\tau(y)=y+A(t)$,其中$A(t)\in k[t]$并且$\tau(A(t))=A(t)$.按照$-\omega_0=\tau(\omega_0)$展开,得到$\tau(Q(t))=Q(t)$,于是$\deg Q(t)$是偶数.但是$\deg Q(t)\le1$(因为$\infty$是分歧点导致$\deg(4P+Q^2)=3$,迫使$\deg Q\le1$),迫使$Q(t)$是常数.把$\tau$作用在$y^2+Q(t)y=P(t)$再加上这个式子,得到$(Q+A)A=P+\tau(P)$.这里$A$的次数是偶数.但是$P$的次数是3或者4,迫使$P$是4次的并且没有一次三次项,这导致$P'=0$和光滑性矛盾.
    \end{proof}
    \item 推论.上一条说明算术亏格$\ge2$的超椭圆曲线上的超椭圆对合是唯一的,于是此时定义中的次数2的有限可分态射$X\to\mathbb{P}_k^1$在同构意义下是唯一的.
    \item 推论.设$X/k$是算术亏格$g$的超椭圆曲线.设$y^2+Q(t)y=P(t)$和$v^2+R(u)v=S(u)$是$X$的两个超椭圆曲线.
    \begin{enumerate}[(1)]
    	\item 如果$g\ge2$,那么存在:
    	$$\left(\begin{array}{cc}a&b\\c&d\end{array}\right)\in\mathrm{GL}_2(k),e\in k^*,H(t)\in k[t],\deg H\le g+1$$
    	
    	使得:
    	$$u=\frac{at+b}{ct+d},\quad v=\frac{H(t)+ey}{(ct+d)^{g+1}}$$
    	\item 如果$X$是椭圆曲线,并且这两个方程是具有相同基点的椭圆方程,那么额外的还有$c=0,d=1$和$\deg H\le1$.
    \end{enumerate}
    \begin{proof}
    	
    	设方程$y^2+Qy=P$和$v^2+Rv=S$分别诱导了超椭圆对合$\sigma$和$\tau$.【】
    \end{proof}
    \item 设$X/k$是算术亏格$g\ge2$的超椭圆曲线,那么$\omega_{X/k}$诱导的典范态射$X\to\mathbb{P}_k^{g-1}$可分解为超椭圆曲线的典范态射$f:X\to\mathbb{P}_k^1$以及射影线的$g-1$-uple嵌入$h:\mathbb{P}_k^1\to\mathbb{P}_k^{g-1}$(此即$k[T_0,T_1,\cdots,T_{g-1}]\to k[S_0,S_1]$,$T_i\mapsto S_0^iS_1^{g-1-i}$诱导的态射).
    \begin{proof}
    	
    	【】
    \end{proof}
\end{enumerate}
\subsection{曲线的模型}



\newpage
\section{曲面}
\subsection{纤维曲面}

设$S$是戴德金概形(此为维数$\le1$的诺特正规整概形),$S$上的一个纤维曲面(fibered surface)指的是一个2维整射影平坦$S$概形$\pi:X\to S$.定义$S$纤维曲面之间的态射是它们之间的$S$态射.
\begin{enumerate}
	\item 按照终端是戴德金概形的平坦态射的描述,这里定义中的平坦等价于满射.
	\item 如果$\dim S=0$,那么$S$是域的素谱,此时$S$纤维曲面就是域上的整射影曲面,这是一个实际的曲面,称这种情况是几何情况;如果$\dim S=1$,那么$X$相对于$S$是一维曲线,称这种情况是算术情况:设$\dim S=1$,设$X\to S$是纤维曲面,那么对任意$s\in S$有$X_s$是射影$\kappa(s)$曲线,记$\eta$是$S$的一般点,那么$X_{\eta}$是整$K(S)$射影曲线,如果$X$是正规的,那么$X_{\eta}$是正规整$K(S)$射影曲线.
	\begin{proof}
		
		设$\xi$是$X$的一般点,按照$X\to S$是支配的,就有$\xi\in X_{\eta}$,于是$\xi$是$X_{\eta}$的一般点,于是$X_{\eta}$是不可约的.任取$x\in X_{\eta}$,我们有$\mathscr{O}_{X,x}=\mathscr{O}_{X_{\eta},x}$.于是$X_{\eta}$是整$K(S)$概形,并且在$X$正规的时候$X_{\eta}$是正规的.还要说明$\dim X_{\eta}=1$.因为这个条件下有$\dim X=\dim Y+e$,其中$e$是一般纤维的维数,于是$e=1$.最后按照【liuqing4.4.16】得到$X_s$总是等维数$\dim X_{\eta}$的.
	\end{proof}
	\item 设$\pi:X\to Y$是纤维曲面,其中$\dim S=1$,设$Y$的一般点为$\eta$,我们来描述$X$的非平凡(一维)不可约闭子集.
	\begin{enumerate}[(1)]
		\item 设$x$是一般纤维$X_{\eta}$的闭点,那么$\overline{\{x\}}$在$S$上有限满射.
		\item 设$D\subseteq X$是一维的不可约分支,那么要么$D$是某个闭纤维的不可约分支,要么$D=\overline{\{x\}}$,其中$x$是$X_{\eta}$的闭点.
		\item 设$x_0\in X$是闭点,那么$\dim\mathscr{O}_{X,x_0}=2$.
	\end{enumerate}

    设$D$是$X$的素除子,如果它的底空间满足$\pi(D)=S$,则称$D$是水平素除子;如果$\pi(D)$退化为单点,则称$D$是垂直素除子.一个除子称为水平/垂直素除子,如果它对应为Weil除子时,每个分量都是水平/垂直素除子.
    \begin{proof}
    	
    	(1):$D=\overline{\{x\}}$是不可约的,于是$\pi(D)$是包含$\eta$的不可约闭子集,迫使$\pi(D)=S$.按照$x$是不可约空间$X_{\eta}$的闭子集,所以$D\not=X$,所以$\dim D\le1$.任取$s\in S$,有$\dim X_s=1$.我们断言$\dim(D\cap X_s)=0$,因为否则的话它包含了$X_s$的不可约分支,迫使$D\cap X_s=X_s$,进而$D=X_s$这矛盾(如果$s=\eta$,按照$x$是$X_{\eta}$的闭点知$D\not=X_{\eta}$;如果$s\not=\eta$,那么$x\in D=X_s$和$x\in X_{\eta}$矛盾).于是$D\cap X_s$是有限集合,于是$\pi\mid_D:D\to S$是射影和拟有限的,进而它是有限态射.
    	
    	\qquad
    	
    	(2):$\pi(D)$是$S$的不可约闭子集,先设$\pi(D)=\{s\}$是一个单点集,那么$D\subseteq X_s$,按照维数就有$D$是$X_s$的不可约分支.如果$\pi(D)$没有退化为单点,那么$\pi(D)=S$(因为$\dim S=1$).于是$D\cap X_{\eta}$是$X_{\eta}$的非空闭子集,于是它包含了一个$X_{\eta}$的闭点$x$.于是$\overline{\{x\}}\subseteq D$.但是$\dim\overline{\{x\}}=1$,迫使$D=\overline{\{x\}}$.
    	
    	\qquad
    	
    	(3):在条件下,记$\pi(x_0)=s$,那么有$\dim\mathscr{O}_{X,x_0}=\dim\mathscr{O}_{S,s}+\dim X_{\eta}=2$.
    \end{proof}
    \item 设$\pi:X\to S$是纤维曲面,其中$\dim S=1$.设$s\in S$,设$\eta$是$S$的一般点.
    \begin{enumerate}[(1)]
    	\item 把$X_s$视为$\kappa(s)$射影曲线,那么总有$p_a(X_s)=p_a(X_{\eta})$.
    	\item 如果$X_{\eta}$是几何连通的,那么每个$X_s$都是几何连通的.
    	\item 如果$X_{\eta}$是几何整的,那么典范环层态射$\mathscr{O}_S\to\pi_*\mathscr{O}_X$是同构.
    	\item 如果$X$是正则的,那么$X\to S$和$X_s\to\mathrm{Spec}\kappa(s)$都是局部完全交态射,并且对偶层满足$\omega_{X_s/\kappa(s)}=\omega_{X/S}\mid_{X_s}$.
    \end{enumerate}
    \begin{proof}
    	
    	(1):见【liuqing5.3.28】.
    	
    	\qquad
    	
    	(2):任取闭点$s\in S$,取仿射开邻域和局部化,我们不妨设$S=\mathrm{Spec}A$是DVR的素谱,其中$s$是唯一闭点.先证明$X_s$是连通的:记$B=\mathscr{O}_X(X)$,那么$\pi$的Stein分解为射影态射$X\to\mathrm{Spec}B$和$\mathrm{Spec}B\to\mathrm{Spec}A$.按照Zariski连通性定理,前者的纤维都是连通的.另外按照$X_{\eta}$是$K=\mathrm{Frac}(A)$上的几何连通射影曲线,就有$L$是域并且是$K$的有限纯不可分扩张.迫使$\mathrm{Spec}B\to\mathrm{Spec}A$是双射,于是每个$X_s$都是连通的.任取$k=\kappa(s)$的有限单扩张$k'$,那么存在离散赋值环$A'$在$A$上有限,并且以$k'$为剩余域.那么$X_{A'}\to\mathrm{Spec}A'$满足相同的条件,上面一段已经证明了$X_{k'}$是连通的,这就得到$X_k$是几何连通的.
    	
    	\qquad
    	
    	(3):问题归结为设$S=\mathrm{Spec}A$是仿射的,证明$\mathscr{O}_X(X)=A$.因为$X/A$是紧合的,有$\mathscr{O}_X(X)$在$A$上整,并且按照$X_{\eta}$是几何整的得到$\mathscr{O}_X(X_{\eta})=K(S)$.但是$\mathscr{O}_X(X)\subseteq\mathscr{O}_X(X_{\eta})$,于是$\mathscr{O}_X(X)=A$.
    	
    	\qquad
    	
    	(4):【liuqing6.3.2节】.
    	
    \end{proof}
    \item 正规化.设$\pi:X\to S$是正规纤维曲面,设$\eta$是$X$的一般点.按照$X_{\eta}$是连通的,我们解释过$L=\mathscr{O}(X_{\eta})$是$K=K(S)$的有限扩张.设$\rho:S'\to S$是$S$在$L$中的正规化.
    \begin{enumerate}[(1)]
    	\item $\rho$是有限平坦态射,那么$X\to S$分解为$\pi':X\to S'$复合上$\rho$.此时$\pi'$使得$X$是$S'$上的正规纤维曲面,并且一般纤维典范等同于$X_{\eta}$.
    	\item 态射$\pi':X\to S'$的所有纤维都是几何连通的.
    	\item 设$s\in S$使得$\dim\mathscr{O}_{S,s}=1$,设$\rho^{-1}(s)=\{s_1,\cdots,s_n\}$.那么$X_s$是$n$个概形$\{X_i\mid1\le i\le n\}$的无交并,其中$X_i$是有限型$S'$概形,使得$X_{s_i}$是$X_i$的被某个次幂为零的理想层定义的闭子概型(也即$X_{s_i}$是$X_i$的具有相同底空间的闭子概型).
    \end{enumerate}
    \begin{proof}
    	
    	在这个条件下,取$\pi$的Stein分解$\xymatrix{X\ar[r]^{\pi'}&S'\ar[r]^{\rho}&S}$,其中$S'=\mathrm{Spec}\pi_*\mathscr{O}_X$,那么$\rho$恰好就是$S$在$L$中的正规化,于是$S'$也是戴德金概形.因为$S$是戴德金概形,其上模层无挠等价于平坦.按照条件有$\rho_*\mathscr{O}_{S'}=\pi_*\mathscr{O}_X$是$\mathscr{O}_S$上的凝聚无挠模层,进而有$\rho$是有限平坦的.这里$\pi'$自然也是平坦的(整概形到戴德金概形的非常值态射一定是平坦的),它是射影的是Stein分解的一部分.综上$\pi'$也是纤维曲面.一般纤维一致是因为如下纤维积图表,这就证明了(1).
    	$$\xymatrix{X_{\eta}\ar[r]\ar[d]&\mathrm{Spec}L\ar[r]\ar[d]&\mathrm{Spec}K\\X\ar[r]&S'\ar[r]&S}$$
    	
    	(2)就是Zariski连通性定理.(3):设$s\in S$满足$\dim\mathscr{O}_{S,s}=1$,那么$S'\times_S\mathrm{Spec}\kappa(s)$是$\mathrm{Spec}(\mathscr{O}_{S',s_i}/(t))$的无交并,其中$t$是$\mathscr{O}_{S,s}$的一致化参数.进而有$X_s$是$X_i=X\times_{S'}\mathrm{Spec}\mathscr{O}_{S',s_i}/(t)$的无交并.于是$X_{s_i}=X\times_{S'}\mathrm{Spec}\kappa(s_i)$就是$X_i$的具有相同底空间的闭子概型.
    \end{proof}
    \item 光滑性.设$\pi:X\to S$是纤维曲面.设一般纤维$X_{\eta}$是光滑的,那么存在$S$的非空开集$V$使得$\pi^{-1}(V)\to V$是光滑的.特别的此时只有有限个点$s\in S$使得$X_s$不是$\kappa(s)$光滑曲线.
    \begin{proof}
    	
    	我们知道终端是局部诺特概形的有限型态射的光滑点集是开集【liuqing6.2.12】,所以这里$X_{\mathrm{sm}}$是开集.于是$\pi(X\backsim X_{\mathrm{sm}})$是闭集,进而$V=S\backslash\pi(X\backslash X_{\mathrm{sm}})$是开集,非空因为包含$\eta$,它的补集有限是因为$\dim S\le1$,有$\pi^{-1}(V)\to V$是光滑的.
    \end{proof}
    \item 设$S=\mathrm{Spec}A$是仿射戴德金概形,设$\pi:X\to S$是紧合平坦态射,纤维都是1维的,并且$X$是正则的,那么$\pi$是射影态射.
    \begin{proof}
    	
    	$\dim S=0$的时候直接有紧合曲线和射影曲线等价.设$\dim S=1$,设$X$是连通的.正则条件保证了$X$上的Weil除子和Cartier除子一致.我们要构造一个水平有效Cartier除子$D$和每个纤维$X_s,s\in S$的不可约分支有交.一旦这成立,按照$X_s$是射影曲线,就有$\mathscr{L}(D)\mid_{X_s}$是丰沛层.我们知道终端是仿射戴德金概形的射影态射$f:X\to S$,如果$X$上的可逆层$\mathscr{L}$满足$\mathscr{L}_s$是丰沛层对任意闭点$s\in S$成立,那么有$\mathscr{L}$是丰沛层【liuqing5.3.24】.于是这里$\mathscr{L}(D)$是丰沛层,按照$\pi$是分离和有限型的,就有$\pi$是拟射影的,而我们知道拟射影紧合态射是射影的.这就得证.
    	
    	\qquad
    	
    	先取一个非零有效水平除子$D_0$,它的存在性我们解释过了,只要取$X_{\eta}$一个闭点的闭包即可.
    \end{proof}
\end{enumerate}



\newpage
\section{概形基本群}
\subsection{Galois范畴}

一个范畴$\mathscr{C}$称为Galois范畴,如果它具备一个共变函子$F:\mathscr{C}\to\textbf{FSets}$(有限集合范畴),使得$(\mathscr{C},F)$满足如下六条公理.这里函子$F$称为$\mathscr{C}$的纤维函子(fibre functor).
\begin{enumerate}[(G1)]
	\item 范畴$\mathscr{C}$具有终对象和全部二元纤维积(这一条等价于讲$\mathscr{C}$是有限完备范畴).
	\item 范畴$\mathscr{C}$具有全部有限余积(这包含了空集上余积的存在性,也即初对象的存在性),并且任意对象关于有限群作用总存在商对象.
	\item 对任意态射$u:X\to Y$都可以分解为复合$\xymatrix{X\ar[r]^{u'}&Y'\ar[r]^{u''}&Y}$,其中$u'$是一个严格满态射,$u''$是一个单态射,并且是到$Y$的某个直和项的同构.一般范畴中的一个态射$u:X\to Y$称为严格满态射,如果纤维积$X\times_YX$存在,并且$u:X\to Y$是两个投影态射$p_1,p_2:X\times_YX\to X$的余等化子.换句话讲,对任意态射$v:X\to Z$使得$v\circ p_1=v\circ p_2$,都存在唯一的态射$\widetilde{v}:Y\to Z$使得如下图表交换:
	$$\xymatrix{X\times_YX\ar@<.5ex>[r]^{p_1}\ar@<-.5ex>[r]_{p_2}&X\ar[r]^u\ar[dr]_v&Y\ar[d]^{\exists_!\widetilde{v}}\\&&Z}$$
	\item 函子$F$把终对象映为终对象,并且和纤维积可交换(这一条等价于讲$F$和全体有限极限可交换,也等价于讲$F$是左正合的).
	\item 函子$F$和有限余积可交换,和有限群作用的商概形可交换,并且把严格满态射映为严格满态射(但是有限集合范畴中严格满态射就是满态射,也就是满射).
	\item 函子$F$是反映同构的,换句话讲一个态射$u$是同构当且仅当$F(u)$是同构.
\end{enumerate}
\begin{enumerate}
	\item 我们再定义如下两个条件:
	\begin{enumerate}[(G2')]
		\item $\mathscr{C}$是有限余完备范畴.
	\end{enumerate}
	\begin{enumerate}[(G5')]
		\item 函子$F$是右正合的.
	\end{enumerate}
	
	那么有(G2')推出(G2),也有(G5')推出(G5).在给出Galois范畴主要定理后,我们就看到Galois范畴$(\mathscr{C},F)$的纤维函子$F$可以典范的分解为一个范畴等价$H:\mathscr{C}\to\mathscr{C}(\Pi)$和遗忘函子$\mathscr{C}(\Pi)\to\textbf{FSets}$.于是$\mathscr{G}$也是有限余完备的,并且$F$是右正合的.换句话讲这里(G1)-(G6)是等价于(G1),(G2'),(G3),(G4),(G5'),(G6)的.
	\item (G3)中的分解如果是存在的则在同构意义下是唯一的(当然这依赖于其它条件).具体地讲,如果$u$有两个分解$\xymatrix{X\ar[r]^{u_i'}&Y'\ar[r]^{u_i''}&Y\ar@{=}[r]&Y_i'\coprod Y_i''}$,$i=1,2$满足(G3),那么存在唯一的同构$\omega:Y_1'\cong Y_2'$满足$\omega\circ u_1'=u_2'$和$u_2''\circ\omega=u_1''$.
	$$\xymatrix{&Y_1'\ar[dr]^{u_1''}\ar[dd]^{\omega}&&\\X\ar[ur]^{u_1'}\ar[dr]_{u_2'}&&Y\ar@{=}[r]&Y_i'\coprod Y_i''\\&Y_2'\ar[ur]_{u_2''}&&}$$
	\begin{proof}
		
		设两个典范投影态射$X\times_{Y_1'}X\to X$分别为$p_1,p_2$.我们有$u_2''\circ u_2'\circ p_1=u\circ p_1=u\circ p_2=u_2''\circ u_2'\circ p_2$.但是按照$u_2''$是单态射,就得到$u_2'\circ p_1=u_2'\circ p_2$.于是按照$u_1'$是严格满态射,就存在态射$\omega:Y_1'\to Y_2'$满足$u_2'=\omega\circ u_1'$.进而按照$u_2''\circ\omega\circ u_1'=u_2''\circ u_2'=u=u_1''\circ u_1'$以及$u_1'$是满态射,就得到$u_2''\circ\omega=u_1''$.另外满足这两个等式$\omega\circ u_1'=u_2'$和$u_2''\circ\omega=u_1''$的$\omega$必然是唯一的,因为从$u_2''\circ\omega_1\circ u_1'=u_2''\circ\omega_2\circ u_1'$以及$u_1'$是满态射,$u_2''$是单态射就得到$\omega_1=\omega_2$.最后只要证明$\omega$是同构即可.按照(G6)这归结为证明$F(\omega):F(Y_1')\to F(Y_2')$是同构.但是我们有$F(u_2')=F(w)\circ F(u_1')$,这里$F(u_1')$和$F(u_2')$都是满射,于是$F(\omega)$是满射.但是按照对称性,我们也能构造$F(Y_2')\to F(Y_1')$的满射,于是$F(\omega)$是双射,得证.
	\end{proof}
	\item 引理.设$\mathscr{C}$是具有有限纤维积的范畴,设$u:X\to Y$是态射,那么$u$是单态射当且仅当它的对角态射是同构.特别的,如果一个函子的源端和终端都是具有有限纤维积的范畴,并且这个函子和纤维积可交换,那么它把单态射映为单态射;如果具有有限纤维积的范畴上的一个态射同时是单态射和严格满态射,那么它是同构.
	\begin{proof}
		
		我们只来证明最后一个断言.设$u:X\to Y$同时是单态射和严格满态射.设典范投影态射$p_1,p_2:X\times_YX\to X$,按照$u$是严格满态射,那么至少有$u\circ p_1=u\circ p_2$,于是再按照$u$是单态射得到$p_1=p_2$.于是按照严格满态射的泛性质,就有双射$\mathrm{Hom}_{\mathscr{C}}(Y,X)\cong\mathrm{Hom}_{\mathscr{C}}(Y,Y)$为$\alpha\mapsto u\circ\alpha$.于是特别的,存在态射$v:Y\to X$满足$u\circ v=1_Y$.进而有$u\circ v\circ u=u=u\circ1_X$,按照$u$是单态射就得到$v\circ u=1_X$,综上$u$是同构.
	\end{proof}
	\item Galois范畴总是阿廷的,这是指范畴的每个对象都是阿廷的.范畴上的一个对象称为阿廷的,如果它的子对象链总会终止.更具体地讲,一个对象$T_0$称为阿廷的,如果对如下任意态射链,其中每个$t_i$都是单态射,总存在正整数$N$使得$n>N$时$t_n$总是同构.
	$$\xymatrix{\cdots\ar[r]^{t_{n+1}}&T_n\ar[r]^{t_n}&\cdots\ar[r]^{t_2}&T_1\ar[r]^{t_1}&T_0}$$
	\begin{proof}
		
		设$(\mathscr{C},F)$是Galois范畴,设有上图的单态射链.按照(G6),归结为证明当$n$足够大的时候$F(t_{n+1}):F(T_{n+1})\to F(T_n)$是同构.按照我们的引理,每个$F(t_{n+1})$都是单射,但是最终端的$F(T_0)$是有限集,这就迫使当$n$足够大时$F(t_{n+1})$是同构.
	\end{proof}
	\item 设$(\mathscr{C},F)$是Galois范畴,对任意态射$u$,有$F(u)$是满射(这也是有限集合范畴中的严格满态射)当且仅当$u$是严格满态射;$F(u)$是单态射当且仅当$u$是单态射.
	\begin{proof}
		
		两个命题的充分性是容易的.下面设$F(u)$是满射,按照(G3)有$u:X\to Y$可以分解为$u=u''\circ u'$,其中$u'$是严格满态射,$u''$是单态射并且是到$Y$的直和项的同构.于是如果$F(u)$是满射,就得到$F(u'')$也是满射,但是按照$u''$是单态射得到$F(u'')$是单态射,于是$F(u'')$是同构,于是按照(G6)得到$u''$是同构,于是$u=u''\circ u'$是严格满态射.类似的可以证明$F(u)$是单态射则$u$是单态射.
	\end{proof}
	\item 设$(\mathscr{C},F)$是Galois范畴,设$X_0$是对象,那么$F(X_0)$是空集(此为有限集合范畴中的初对象)当且仅当$X_0$是$\mathscr{C}$的初对象;$F(X_0)$是单点集(此为有限集合范畴中的终对象)当且仅当$X_0$是$\mathscr{C}$的终对象.
	\begin{proof}
		
		我们先设$F(X_0)=\emptyset$.设$\emptyset_{\mathscr{C}}$是$\mathscr{C}$的初对象.对任意对象$X\in\mathscr{C}$,记唯一的态射$\emptyset_{\mathscr{C}}\to X$为$u_X$.我们有$u_{X_0}$是从$F(\emptyset_{\mathscr{C}})\to F(X_0)=\emptyset$的集合映射,但是这迫使$F(\emptyset_{\mathscr{C}})=\emptyset$.进而有$F(u_{X_0})=1_{\emptyset}$.于是特别的$F(u_{X_0})$是同构,进而按照(G6)得到$u_{X_0}$是同构,于是$X_0$是$\mathscr{C}$的初对象.
		
		\qquad
		
		再设$X_0=\emptyset_{\mathscr{C}}$,对任意对象$X\in\mathscr{C}$,我们有典范同构$(u_X,1_X):\emptyset_{\mathscr{C}}\coprod X\cong X$,于是$F((u_X,1_X)):F(\emptyset_{\mathscr{C}}\coprod X)\cong F(X)$也是同构.但是按照$F$和有限余积可交换,得到同构$F(\emptyset_{\mathscr{C}})\coprod F(X)\cong F(X)$,这迫使$F(\emptyset_{\mathscr{C}})$是空集.
		
		\qquad
		
		最后处理终对象的命题.设$*_{\mathscr{C}}$是$\mathscr{C}$中的终对象,那么对任意对象$X\in\mathscr{C}$有唯一的态射$v_X:X\to*_{\mathscr{C}}$.一方面$F(*_{\mathscr{C}})$是单点集$*$是因为(G4).反过来如果$F(X_0)=*$是单点集,那么$F(v_{X_0})$就是单点集之间的双射,于是(G6)得到$v_{X_0}$是同构,于是$X_0$是终对象.
	\end{proof}
\end{enumerate}
\subsection{Galois范畴对应的带基点范畴}

设$(\mathscr{C},F)$是一个Galois范畴,它的带基点范畴$\mathscr{C}^{\mathrm{pt}}$是这样定义的,对象是全体二元对$(X,\zeta)$,其中$X$是$\mathscr{C}$的对象,$\zeta\in F(X)$(这导致$F(X)$不能是空集,也即$X$不能是$\mathscr{C}$的初对象).一个$(X_1,\zeta_1)\to(X_2,\zeta_2)$的态射定义为$\mathscr{C}$中的一个态射$u:X_1\to X_2$,使得$F(u)(\zeta_1)=\zeta_2$.
\begin{enumerate}
	\item 我们有典范的遗忘函子$\textbf{For}:\mathscr{C}^{\mathrm{pt}}\to\mathscr{C}$,并且映射$\textbf{For}:\mathrm{Obj}(\mathscr{C}^{\mathrm{pt}})\to\mathrm{Obj}(\mathscr{C})$的截面(右逆)和集合(这里要求$\mathscr{C}$是小范畴)$\prod_{X\in\mathrm{Obj}(\mathscr{C})}F(X)$一一对应.我们考虑范畴$\mathscr{C}^{\mathrm{pt}}$的目的是相比$\mathscr{C}$它具有更多对象但是具有更少态射.
	\item 支配偏序.定义范畴的全体对象上的一个偏序为$X\ge Y$当且仅当$\mathrm{Hom}_{\mathscr{C}}(X,Y)$不是空集,此时也称$X$支配了$Y$.那么明显的$\mathscr{C}^{\mathrm{pt}}$的对象在支配偏序下是有向的(任取带基点范畴中的两个对象$(X,\xi)$和$(X',\xi')$,那么明显$(X\times X',(\xi,\xi'))$支配了这两个对象).
	\item 从Galois范畴总是阿廷的得到带基点范畴$\mathscr{C}^{\mathrm{pt}}$也总是阿廷的.
	\item 称对象$(X,\xi)$是$\mathscr{C}^{\mathrm{pt}}$中的最小元(minimial),如果态射$u:(Y,\eta)\to(X,\xi)$满足$u$作为$Y\to X$的态射是单态射,那么$u$作为$Y\to X$的态射一定是同构.我们会解释这样的最小元恰好描述了非平凡的连通对象.
\end{enumerate}
\subsection{Galois范畴上的连通对象}

我们始终在Galois范畴设$\mathscr{C}$中考虑.称一个对象$X$是连通对象,如果它不能写成两个对象的余积,使得它们都不是初对象.换句话讲如果$X=X_1\coprod X_2$,那么$X_1$和$X_2$至少有一个是初对象.例如对$G-\textbf{Sets}$范畴,它的连通对象就是可迁的$G$集合.
\begin{enumerate}
	\item 初对象本身是连通的,它称为平凡的连通对象.
	\item 对象$X_0$是连通对象当且仅当对任意不是初对象的对象$X$,任意单态射$X\to X_0$都是同构.
	\begin{proof}
		
		充分性.设有分解$X_0=X_0'\coprod X_0''$,不妨设$X_0'$不是初对象.我们解释过纤维函子$F$是反映单态射的,于是按照典范的$F(X_0')\to F(X_0)=F(X_0')\coprod F(X_0'')$明显是单射,得到典范的$X_0'\to X_0$是单态射.于是按照条件它是同构,于是$F(X_0'')$是空集,于是$X_0''$是初对象.
		
		\qquad
		
		必要性.不妨设$X_0$本身不是初对象(否则对单态射$X\to X_0$,有$F(X)\to F(X_0)$是单射,但是按照$F(X_0)$是空集导致$F(X)$是空集,进而$X$是初对象,从而$X\to X_0$是同构),设$X$不是初对象,设有单态射$i:X\to X_0$.于是按照(G3)有分解$i=i''\circ i'$,其中$i':X\to X_0'$是严格满态射,$i'':X_0'\to X_0$是单态射并且是到$X_0$的某个直和项的同构,也即有$X_0=X_0'\coprod X_0''$.但是按照$X_0$是连通的,就有要么$X_0'$是初对象,要么$X_0''$是初对象.如果$X_0'$是初对象,那么映射$F(X)\to F(X_0)$经一个空集分解,迫使$F(X)$是空集,进而$X$是初对象;如果$X_0''$是空集,那么按照(G6)有$i'':X_0'\to X_0$是同构,于是此时$X\to X_0$同时是单态射和严格满态射,我们解释过此时它是同构.
	\end{proof}
    \item 一个非初对象的对象$X$是连通对象当且仅当对任意(也等价的改为存在一个)$\xi\in F(X)$,有$(X,\xi)$是$\mathscr{C}^{\mathrm{pt}}$的最小元.
    \begin{proof}
    	
    	我们先设存在$\xi\in F(X)$使得$(X,\xi)$是$\mathscr{C}^{\mathrm{pt}}$的最小元.设有分解$X=X_1\coprod X_2$,于是有$\xi\in F(X)=F(X_1)\coprod F(X_2)$,不妨设$\xi\in F(X_1)$.那么我们有态射$j:(X_1,\xi)\to(X,\xi)$满足$j:X_1\to X$是单态射.于是按照$(X,\xi)$是最小元就有$j$是同构,于是$F(X_2)=\emptyset$,于是$X_2=\emptyset_{\mathscr{C}}$.
    	
    	反过来设$X$是非平凡的连通对象,任取$\xi\in F(X)$.假设存在态射$j:(Y,\eta)\to(X,\xi)$,满足$j:Y\to X$是单态射.按照(G3)我们做分解$j=j''\circ j'$,其中$j':Y\to X'$是严格满态射,而$j'':X'\to X$是单态射,并且是到某个直和项的同构,进而有分解$X=X'\coprod X''$.但是按照$j$是单态射得到$j'$也是单态射,于是$j'$同时是单态射和严格满态射,它就是同构.但是按照$X$是连通的,这迫使$X''=\emptyset_{\mathscr{C}}$,于是$j''$也是同构,于是$j=j''\circ j'$是同构,也即$(X,\xi)$是最小元.
    \end{proof}
	\item 对任意非初对象的对象$X$,它可以做分解$X=\coprod_{i=1}^rX_i$,使得每个$X_i$都是连通对象.并且这个分解在同构意义下是唯一的.我们称这些$\{X_i\}$就是$X$的连通分支.
	\begin{proof}
		
		分解的存在性.任取非初对象$X$,倘若它本身是连通的,则它本身已经是满足命题的分解.否则它有分解$X=X_1\coprod X_1'$,其中$X_1,X_1'$都不是初对象,这里$X_1\to X$是单态射是因为$F(X_1)\to F(X)=F(X_1)\coprod F(X_1')$是单射.接下来可不妨设$X_1$不是连通的,否则它俩都是连通的导致已经得到命题中的分解.归纳构造下去,我们得到一个链$\cdots\to X_2\to X_1\to X$.于是按照Galois范畴总是阿廷的,这个链必须终止.于是可以找到不是初对象的连通对象$X_1$,以及单态射$i_1:X_1\to X$.按照(G3)就有分解$i_1=i_1''\circ i_1'$,其中$i_1':X_1\to X'$是严格满态射,而$i_1'':X'\to X$是单态射并且是到直和项的同构,于是有分解$X=X'\coprod X''$.但是按照$i_1$和$i_1''$都是单态射,得到$i_1'$也是单态射,再结合它是严格满态射得到它是同构.于是分解$X=X'\coprod X''$中的$X'$是非初对象的连通对象.于是我们对$F(X)$的元素个数做归纳就得到分解的存在性.
		
		\qquad
		
		假设存在两种分解$X=\coprod_{i=1}^rX_i=\coprod_{i=1}^sY_i$,其中$X_i,Y_i$都是非初对象的连通对象.于是$\{F(X_i)\}$和$\{F(Y_i)\}$是集合$F(X)$的两个划分,于是对任意$X_i$,可以找到一个$Y_{\sigma(i)}$满足$F(X_i)\cap F(Y_{\sigma(i)})$是非空的.我们取如下纤维积图表:
		$$\xymatrix{X_i\times_XY_{\sigma(i)}\ar[rr]^p\ar[d]_q&&X_i\ar[d]^{i_{X_i}}\\Y_{\sigma(i)}\ar[rr]_{i_{Y_{\sigma(i)}}}&&X}$$
		
		并且这里$p,q$作为单态射的提升都是单态射.另外按照(G4)得到$F(X_i\times_XY_{\sigma(i)})=F(X_i)\cap F(Y_{\sigma(i)})$是非空的,于是$X_i\times_XY_{\sigma(i)}$不是初对象,于是按照上一条,从$p,q$都是源端非初对象终端连通对象的单态射,得到$p,q$都是同构,于是$X_i\cong Y_{\sigma(i)}$且$F(X_i)=F(Y_{\sigma(i)})$,这迫使$\sigma$是$\{1,2,\cdots,r\}\to\{1,2\cdots,s\}$的双射.
	\end{proof}
	\item 刚性(rigidity).设$X_0,X$是$\mathscr{C}$中的两个对象,设$\zeta_0\in F(X_0)$和$\zeta\in F(X)$,那么只要$X_0,X$都不是初对象,并且$X_0$是连通对象,那么在带基点范畴$\mathscr{C}^{\mathrm{pt}}$中$\mathrm{Hom}((X_0,\zeta_0),(X,\zeta))$至多有一个态射.
	\begin{proof}
		
		假设我们存在$\mathscr{C}^{\mathrm{pt}}$中的两个态射$u_1,u_2:(X_0,\zeta_0)\to(X,\zeta)$.那么这两个$u_1,u_2$是$\mathscr{C}$中$X_0\to X$的态射.按照(G1),可取这两个态射的等化子$i:K\to X_0$.于是按照(G4),在$\textbf{FSets}$中有$F(i):F(K)\to F(X_0)$是$F(u_1)$和$F(u_2)$的等化子.按照$F(u_i)(\zeta_0)=\zeta$,就有$\zeta_0\in F(K)$.于是特别的$F(K)$不是空集,进而$K$不是初对象.但是我们知道等化子一定是单态射,于是$i:K\to X_0$是源端非初对象,终端非初对象的连通对象的单态射,我们解释过这导致$i$一定是同构,进而有$u_1=u_2$.
	\end{proof}
	\item 推论.考虑$\mathscr{C}^{\mathrm{pt}}$的全体$(X,\xi)$构成的完全子范畴,其中$X$取遍连通对象,那么这个子范畴的全体对象在支配偏序下也是有向的.或者更具体地讲,对$\mathscr{C}^{\mathrm{pt}}$中的任意有限个对象$(X_i,\zeta_i),1\le i\le r$,一定存在对象$(X_0,\zeta_0)$,其中$X_0$连通,使得对任意$1\le i\le r$,至少存在一个态射$(X_0,\zeta_0)\to(X_i,\zeta_i)$.特别的,对任意对象$X\in\mathscr{C}$,总可以找到对象$(X_0,\zeta_0)\in\mathrm{C}^{\mathrm{pt}}$,使得$X_0$是连通的,并且映射$\mathrm{Hom}_{\mathscr{C}}(X_0,X)\to F(X)$,$\left(u:X_0\to X\right)\mapsto F(u)(\zeta_0)$是双射.
	\begin{proof}
		
		这件事如果从连通对象是带基点范畴在支配偏序下的最小元看是直接的:任取两个$(X,\xi)$,$(X',\xi')$,其中$X,X'$都是非平凡的连通对象,那么它们自然被$(X\times X',(\xi,\xi'))$支配.再按照$\mathscr{C}^{\mathrm{pt}}$也是阿廷的,就可以取到支配了$(X\times X',(\xi,\xi'))$的最小元$(Y,\eta)$,它同时支配了$(X,\xi)$和$(X',\xi')$,并且$Y$是连通对象.
		
		\qquad
		
		当然我们也可以从连通对象的初始定义出发证明这件事:一旦选取了有限个对象$(X_i,\zeta_i),1\le i\le r$,我们可以取$X=X_1\times\cdots\times X_r$和$\zeta=(\zeta_1,\cdots,\zeta_r)$,那么按照(G4)有典范投影态射$p_i:X\to X_i$满足$F(p_i)(\zeta)=\zeta_i$.综上问题归结为证明$r=1$的情况,即对任意对象$(X,\zeta)$,找一个对象$(X_0,\zeta_0)$使得$X_0$连通,并且有态射$(X_0,\zeta_0)\to(X,\zeta)$.
		
		\qquad
		
		倘若$X$本身是连通的,我们可以直接取$1_X:(X,\zeta)\to(X,\zeta)$.再设$X$不是连通的(于是它也不是初对象),那么有连通分支分解$X=\coprod_{i=1}^rX_i$.记余积的典范态射为$\alpha_i:X_i\to X$.按照(G2)有$F(X)=\coprod_{i=1}^rF(X_i)$,于是存在恰好一个指标$1\le i\le r$使得$\zeta\in F(X_i)$,那么$\alpha_i:(X_i,\zeta)\to(X,\zeta)$说明命题成立.最后一个命题只要对对象$X$,记$F(X)=\{\zeta_1,\cdots,\zeta_n\}$,取$\mathscr{C}^{\mathrm{pt}}$的有限个对象为$(X,\zeta_1),\cdots,(X,\zeta_n)$即可.
	\end{proof}
	\item 终端或者源端为连通对象的态射.
	\begin{enumerate}[(1)]
		\item 如果$X_0\in\mathscr{C}$是连通对象,设$X$不是初对象,那么$\mathscr{C}$中任意态射$u:X\to X_0$都是严格满态射.
		\begin{proof}
			
			按照(G3),有分解$u=u''\circ u'$,其中$u':X\to X_0'$是严格满态射,$u'':X_0'\to X_0$是单态射并且是到$X_0$直和项的同构,进而有分解$X_0=X_0'\coprod X_0''$.按照$X$不是初对象得到$X_0'$不是初对象(因为有映射$F(X)\to F(X_0')$,倘若终端是空集那必须源端也是空集,进而$X$是初对象).于是$u''$是一个源端非初对象,终端是非初对象连通对象的单态射,这导致$u''$是同构,于是$u$是严格满态射.
		\end{proof}
		\item 如果$u:X_0\to X$是严格满态射,并且$X_0$是连通对象,那么$X$也是连通对象.
		\begin{proof}
			
			倘若$X_0$是初对象,也即$F(X_0)$是空集,按照$F(u)$是满射导致$F(X)$是空集,于是$X$也是初对象,此时它平凡的是连通对象.下面设$X_0,X$都不是初对象.设有分解$X=X'\coprod X''$,不妨设$X'$不是初对象.任取$\zeta'\in F(X')$,由于$u$是严格满态射,得到$F(u)$是满态射,于是存在$\zeta_0\in F(X_0)$满足$F(u)(\zeta_0)=\zeta'$.考虑$\mathscr{C}^{\mathrm{pt}}$中的对象$(X_0,\zeta_0)$和对象$(X',\zeta')$,那么可以找到一个对象$(X_0',\zeta_0')$使得$X_0'$是连通对象,并且存在态射$p:(X_0',\zeta_0')\to(X_0,\zeta_0)$和$q:(X_0',\zeta_0')\to(X',\zeta')$.这里$X_0'$明显不是初对象(因为$\zeta_0'\in F(X_0')$导致$F(X_0')$不是空集),于是按照(1)得到$p:X_0'\to X_0$是严格满态射,于是$u\circ p$也是严格满态射.但是按照刚性,$(X_0',\zeta_0')\to(X,\zeta)$的态射至多一个,于是$i_{X'}\circ q=u\circ p$就也是严格满态射.但是这迫使$F(X')=F(X)$,进而$F(X'')=\emptyset$,于是$X''$是初对象,于是$X$是连通的.
		\end{proof}
		\item 如果$X_0$是连通对象,那么任意态射$u:X_0\to X_0$都是同构.
		\begin{proof}
			
			按照(G6)归结为证明$F(u):F(X_0)\to F(X_0)$是双射,又因为$F(X_0)$总是有限集合,于是归结为证明$F(u)$是满射.也即证明$u$是严格满态射.但是这就是(1).
		\end{proof}
	\end{enumerate}
    \item 引理.设$X_0,X_1,\cdots,X_r$是$\mathscr{C}$的非初对象的连通对象,记$X=\coprod_{i=1}^rX_i$,记典范态射$l_i:X_i\to X,1\le i\le r$.那么典范映射$\alpha:\coprod_{i=1}^r\mathrm{Hom}_{\mathscr{C}}(X_0,X_i)\to\mathrm{Hom}_{\mathscr{C}}(X_0,X)$是一个双射.
    \begin{proof}
    	
    	证明单射.首先$l_i$都是单态射因为$F(l_i)$都是单射.如果有态射$f:X_0\to X_i$和$g:X_0\to X_j$满足$l_i\circ f=l_j\circ g=\varphi$,那么$F(X_0)$在$F(\varphi)$下的像就同时落在$F(X_i)$和$F(X_j)$中,但是按照$X=\coprod_{1\le i\le r}X_i$是连通分支分解,这迫使$i=j$.于是此时有$l_i\circ f=l_i\circ g$,但是按照$l_i$是单态射就得到$f=g$.
    	
    	\qquad
    	
    	证明满射.任取态射$u:X_0\to X$,那么按照(G3)就有分解$u=u''\circ u'$,其中$u':X_0\to X'$是严格满态射,$u'':X'\to X$是单态射,并且是到直和项的同构,进而有分解$X=X'\coprod X''$.由于$X_0$是非初对象的连通对象,那么我们解释过这里$X'$也是非初对象的连通对象.那么$X'$必然是某个$X_i$,这得到满射.
    \end{proof}
\end{enumerate}
\subsection{Galois范畴上的Galois对象}

Galois对象.对任意非初对象的连通对象$X_0$,任取$\zeta_0\in F(X_0)$,按照刚性定理,我们有映射$e_{\zeta_0}:\mathrm{Aut}_{\mathscr{C}}(X_0)=\mathrm{Hom}_{\mathscr{C}}(X_0,X_0)\to F(X_0)$,$\left(u:X_0\cong X_0\right)\mapsto F(u)(\zeta_0)$是单射.如果一个非初对象的连通对象$X_0$满足对任意$\zeta_0\in F(X_0)$都有$e_{\zeta_0}$是双射,就称$X_0$是Galois对象.
\begin{enumerate}
	\item 等价描述.设$X_0$是非初对象的连通对象,有$\mathrm{Aut}_{\mathscr{C}}(X_0)$作用在$F(X_0)$上.我们有如下命题互相等价.其中最后一个等价描述说明Galois对象是一个和纤维函子$F$的选取无关的概念.
	\begin{enumerate}[(1)]
		\item $X_0$是Galois对象,也即$|\mathrm{Aut}_{\mathscr{C}}(X_0)|=|F(X_0)|$(于是虽然定义中要证明的是对任意$\zeta_0\in F(X_0)$都有$e_{\zeta_0}$是双射,但是实际上只要存在一个$\zeta_0\in F(X_0)$使得$e_{\zeta_0}$是双射就已经得到$X_0$是Galois对象了).
		\item $\mathrm{Aut}_{\mathscr{C}}(X_0)$在$F(X_0)$上的作用是可迁的.
		\item $\mathrm{Aut}_{\mathscr{C}}(X_0)$在$F(X_0)$上的作用是单可迁的(群$G$在集合$S$上的作用是单可迁的是指,对任意$x,y\in S$,恰好存在一个$g\in G$使得$gx=y$).
		\item 商对象$X_0/\mathrm{Aut}_{\mathscr{C}}(X_0)$是$\mathscr{C}$中的终对象.
	\end{enumerate}
	\begin{proof}
		
		按照连通对象的刚性,对任意$\zeta_0,\zeta_0'\in F(X_0)$,至多存在一个$\sigma\in\mathrm{Aut}_{\mathscr{C}}(X_0)$使得$F(\sigma)(\zeta_0)=\zeta_0'$.我们把这个性质称为$\mathrm{Aut}_{\mathscr{C}}(X_0)$在$F(X_0)$上的作用是单的.那么前三条的等价性就是一个形式上的性质:如果有限群$G$单作用在有限集合$S$上,那么作用是可迁的当且仅当作用是单可迁的,当且仅当$|G|=|S|$.
		
		\qquad
		
		我们知道(4)等价于讲$F(X_0/\mathrm{Aut}_{\mathscr{C}}(X_0))$是单点集,于是按照(G5)这等价于$F(X_0)/\mathrm{Aut}_{\mathscr{C}}(X_0)$是单点集,此即$\mathrm{Aut}_{\mathscr{C}}(X_0)$在$F(X_0)$上的作用是可迁的.
	\end{proof}
    \item 例如在$\mathscr{C}(\Pi)$中,一个对象是连通对象当且仅当$\Pi$作用是可迁的,于是此时该对象同构于一个开子群$U$的左陪集集合$\Pi/U$,而$\Pi$在上面的群作用是左乘.此时有$\mathrm{Aut}_{\Pi}(\Pi/U)=\mathrm{N}_{\Pi}(U)/U$,于是一个连通对象$\Pi/U$是Galois对象当且仅当$\mathrm{N}_{\Pi}(U)=\Pi$,也即$U$是开正规子群.
	\item Galois闭包.如果$X$是非初对象的连通对象,那么存在一个Galois对象$\widehat{X}$支配了$X$,并且在所有支配了$X$的Galois对象中存在最小元.这称为$X$的Galois闭包.特别的,这件事说明全体$(X,\xi)$构成的类,其中$X$跑遍Galois对象,是$\mathrm{Obj}(\mathscr{C}^{\mathrm{pt}})$在支配偏序下的共尾子类.
	\begin{proof}
		
		首先我们解释过可以找到$\mathscr{C}^{\mathrm{pt}}$中的对象$(X_0,\zeta_0)$,满足其中$X_0$是连通对象,并且映射$e_{\zeta_0}:\mathrm{Hom}_{\mathscr{C}}(X_0,X)\to F(X)$是双射.记$\mathrm{Hom}_{\mathscr{C}}(X_0,X)=\{u_1,\cdots,u_n\}$.再记$F(u_i)(\zeta_0)=\zeta_i,1\le i\le n$,于是$F(X)=\{\zeta_1,\cdots,\zeta_n\}$.再记$p_i:X^n\to X$表示第$i$个投影态射.于是按照积对象的泛性质,存在态射$u=(u_1,\cdots,u_n):X_0\to X^n$,换句话讲它满足$p_i\circ u=u_i,\forall1\le i\le n$.按照(G3)有分解$u=u''\circ u'$,其中$u':X_0\to\widehat{X}$是严格满态射,而$u'':\widehat{X}\to X^n$是单态射,并且是到$X^n$直和项的同构.于是有$X^n=\widehat{X}\coprod\widehat{X}'$.明显的$\widetilde{X}\ge X$(因为有态射$\widetilde{X}\to X^n\to X$),我们断言$\widehat{X}$是Galois对象,并且是所有支配了$X$的Galois对象中的最小元.
		
		\qquad
		
		先证明$\widehat{X}$是Galois对象.我们之前解释过按照$u':X_0\to\widehat{X}$是严格满态射以及$X_0$是非初对象的连通对象,就有$\widehat{X}$是非初对象的连通对象.下面记$\widehat{\zeta_0}=F(u')(\zeta_0)=(\zeta_1,\cdots,\zeta_n)\in F(\widehat{X})$.我们来证明$e_{\widehat{\zeta_0}}:\mathrm{Aut}_{\mathscr{C}}(\widehat{X})\to F(\widehat{X})$是双射,但是对于连通对象$\widehat{X}$这个映射已经是单射了,于是只需验证它是满射.也即证明对任意$\zeta\in\widehat{X}$,都可以找到$\omega\in\mathrm{Aut}_{\mathscr{C}}(\widehat{X})$使得$F(\omega)(\widehat{\zeta_0})=\zeta$.
		
		\qquad
		
		我们之前解释过可以找到对象$(\widetilde{X_0},\widetilde{\zeta_0})$满足$\widetilde{X_0}$是连通对象,使得$(\widetilde{X_0},\widetilde{\zeta_0})\ge(X_0,\zeta_0)$和$(\widetilde{X_0},\widetilde{\zeta_0})\ge(\widehat{X},\zeta)$,对任意$\zeta\in F(\widehat{X})$成立,并且我们解释过前面这个态射总是严格满态射.于是不妨用$(\widetilde{X_0},\widetilde{\zeta_0})$替换$(X_0,\zeta_0)$,那么分解$u=u''\circ u'$中$u'$作为两个严格满态射的复合就仍然是严格满态射,于是我们不妨设对任意$\zeta\in F(\widehat{X})$,都存在态射$\rho_{\zeta}:(X_0,\zeta_0)\to(\widehat{X},\zeta)$.
		
		\qquad
		
		于是能找到$\omega\in\mathrm{Aut}_{\mathscr{C}}(\widehat{X})$满足$F(\omega)(\widehat{\zeta_0})=\zeta$,等价于有$F(\omega\circ u')(\zeta_0)=F(\rho_{\zeta})(\zeta_0)$.但是按照刚性定理,这就等价于有$\omega\circ u'=\rho_{\zeta}$.我们断言这归结为证明有$\{u_1,\cdots,u_n\}=\{p_1\circ u''\circ\rho_{\zeta},\cdots,p_n\circ u''\circ\rho_{\zeta}\}$.因为一旦这成立,存在一个双射$\sigma\in S_n$(对称群),使得$p_{\sigma(i)}\circ u''\circ\rho_{\zeta}=p_i\circ u''\circ u',\forall1\le i\le n$.按照积对象的泛性质,可以找到同构仍然记作$\sigma:X^n\to X^n$,使得$p_i\circ\sigma=p_{\sigma(i)},\forall1\le i\le n$.于是有$p_i\circ u''\circ u'=p_i\circ\sigma\circ u''\circ\rho_{\zeta},\forall1\le i\le n$,进而得到$u=\sigma\circ u''\circ\rho_{\zeta}$.但是我们知道(G3)中的分解是同构意义下唯一的,于是存在自同构$\omega:\widehat{X}\to\widehat{X}$满足$\sigma\circ u''=u''\circ\omega$和$\omega\circ u'=\rho_{\zeta}$.
		
		\qquad
		
		综上为了证明$\widehat{X}$是Galois对象只差证明我们的断言,即有$\{u_1,\cdots,u_n\}=\{p_1\circ u''\circ\rho_{\zeta},\cdots,p_n\circ u''\circ\rho_{\zeta}\}$.左侧包含于右侧是明显的,因为右侧就是$\mathrm{Hom}_{\mathscr{C}}(X_0,X)$.于是问题归结为证明左侧中的元素两两不听.但是对$i\not=j$,按照$p_i\circ u''\circ u'=u_i\not=u_j=p_j\circ u''\circ u'$,就有$p_i\circ u''\not=p_j\circ u''$.又因为$\rho_{\zeta}:X_0\to\widehat{X}$是终端为非初对象的连通对象,源端非初对象,所以这是严格满态射,特别的它是满态射,于是必然有$p_i\circ u''\circ\rho_{\zeta}\not=p_j\circ u''\circ\rho_{\zeta}$.完成断言的证明.
		
		\qquad
		
		最后证明$\widehat{X}$是所有支配$X$的Galois对象中的最小元.假设还有Galois对象$Y$和一个态射$q:Y\to X$.那么$q$是严格满态射,于是$F(q)$是满射,于是可以找到$\eta_i\in F(Y)$使得$F(q)(\eta_i)=\zeta_i,\forall1\le i\le n$.又因为$Y$是Galois对象,于是可以找到$\omega_i\in\mathrm{Aut}_{\mathscr{C}}(Y)$使得$F(\omega_i)(\eta_1)=\eta_i,\forall1\le i\le n$.我们取积对象泛性质中的唯一态射$\kappa=(q\circ\omega_1,\cdots,q\circ\omega_n):Y\to X^n$.按照(G3)有分解$\kappa=\kappa''\circ\kappa'$,其中$\kappa':Y\to Z'$是严格满态射,$\kappa'':Z'\to X^n$是单态射,并且是到直和项的同构,于是有$X^n=Z'\coprod Z''$.那么我们之前解释过此时$Z'$也是连通对象.另外我们有$F(\kappa)(\eta_1)=(\zeta_1,\cdots,\zeta_n)=\widehat{\zeta_0}$.于是$Z'$就是$X^n$的包含$\widehat{\zeta_0}$的连通分支,并且$F(Z')$和$F(\widehat{X})$有交,这就迫使$Z'\cong\widehat{X}$.
	\end{proof}
    \item 推论.考虑$\mathscr{C}^{\mathrm{pt}}$的全体$(X,\xi)$构成的完全子范畴,其中$X$取遍Galois对象,那么这个子范畴的全体对象在支配偏序下也是有向的.
    \begin{proof}
    	
    	任取$(X,\xi)$和$(X',\xi')$,满足$X,X'$都是Galois对象,那么我们解释了存在一个非平凡的连通对象$Y$以及$\eta\in F(Y)$满足$(Y,\eta)$同时支配了$(X,\xi)$和$(X',\xi')$.接下来只要取$Y$的Galois闭包$\widehat{Y}$,任取态射$u:\widehat{Y}\to Y$,那么我们解释过$u$是严格满态射,于是$F(u)$是满射,于是存在$\widehat{\eta}\in F(\widehat{Y})$满足$F(u)(\widehat{\eta})=\eta$.于是$(\widehat{Y},\widehat{\eta})$支配了$(Y,\eta)$,从而同时支配了$(X,\xi)$和$(X',\xi')$.
    \end{proof}
	\item 设$X_0$是Galois对象,考虑这样的对象$X$,它的所有连通分支都被$X_0$支配,全体这样的对象$X$构成的$\mathscr{C}$的完全子范畴记作$\mathscr{C}^{X_0}$.纤维函子$F$在这个完全子范畴上的限制记作$F^{X_0}:\mathscr{C}^{X_0}\to\textbf{FSets}$.按照定义初对象不在这个子范畴中.我们有如下Galois对应定理:
	\begin{enumerate}[(1)]
		\item 对任意$\zeta_0\in F(X_0)$,映射$e_{\zeta_0}$诱导了函子的自然同构$\mathrm{Hom}_{\mathscr{C}}(X_0,-)\mid_{\mathscr{C}^{X_0}}\cong F^{X_0}$.此即对$\mathscr{C}^{X_0}$中的任意态射$u:X\to Y$有如下交换图表:
		$$\xymatrix{\mathrm{Hom}_{\mathscr{C}}(X_0,X)\ar[rr]^{u\circ-}\ar[d]_{e_{\zeta_0}(X)}^{\cong}&&\mathrm{Hom}_{\mathscr{C}}(X_0,Y)\ar[d]_{\cong}^{e_{\zeta_0}(Y)}\\F(X)\ar[rr]_{F(u)}&&F(Y)}$$
		
		特别的,这诱导了群同构(等号是米田引理):
		\begin{align*}
			u^{\zeta_0}:\mathrm{Aut}_{\textbf{Fun}}(F^{X_0})&\cong\mathrm{Aut}_{\textbf{Fun}}(\mathrm{Hom}_{\mathscr{C}}(X_0,-)\mid_{\mathscr{C}^{X_0}})=\mathrm{Aut}_{\mathscr{C}}(X_0)^{\mathrm{op}}\\
			u^{\zeta_0}(\theta)&=e_{\zeta_0}^{-1}\left(\theta(X_0)(\zeta_0)\right)
		\end{align*}
		\begin{proof}
			
			我们只要证明这里$e_{\zeta_0}(X):\mathrm{Hom}_{\mathscr{C}}(X_0,X)\to F(X)$的确是双射.按照$X_0$是连通的,我们已经知道$e_{\zeta_0}(X)$是单射.下面设$X$是$\mathscr{C}^{X_0}$中的对象.倘若$X$是非初对象的连通对象(初对象不在这个子范畴中,所以$X$一定非初对象),那么我们解释过任意态射$u:X_0\to X$一定是严格满态射.记$F(X)=\{\zeta_1,\cdots,\zeta_n\}$,那么按照$F(u)$是满射,就存在$\zeta_{0i}\in F(X_0)$满足$F(u)(\zeta_{0i})=\zeta_i,\forall1\le i\le n$成立.按照$X_0$是Galois对象,对每个$1\le i\le n$就可以取$\omega_i\in\mathrm{Aut}_{\mathscr{C}}(X_0)$满足$F(\omega_i)(\zeta_0)=\zeta_{0i}$.于是态射$u\circ\omega_i$就满足$F(u\circ\omega_i)(\zeta_0)=\zeta_i$,于是$e_{\zeta_0}(X)$是满射.最后如果$X$不是连通对象,那么按照上面引理就把一般情况约化到连通情况.
		\end{proof}
		\item 我们断言函子$F^{X_0}:\mathscr{C}^{X_0}\to\textbf{FSets}$有如下分解,其中$\mathscr{C}^{X_0}\to\mathscr{C}\left(\mathrm{Aut}_{\mathscr{C}}(X_0)^{\mathrm{op}}\right)$是范畴等价.这里对群$G$,范畴$\mathscr{C}(G)$表示的是全体赋予左$G$作用的有限集合构成的范畴,那么遗忘函子$\mathscr{C}(\Pi)\to\textbf{FSets}$作为纤维函子使得它是Galois范畴.
		$$\xymatrix{\mathscr{C}^{X_0}\ar[rr]^{F^{X_0}}\ar[d]&&\textbf{FSets}\\\mathscr{C}\left(\mathrm{Aut}_{\mathscr{C}}(X_0)^{\mathrm{op}}\right)\ar[urr]_{\textbf{遗忘函子}}&&}$$
		\begin{proof}
			
			记$G=\mathrm{Aut}_{\mathscr{C}}(X_0)$,按照(1)我们可以把函子$F^{X_0}$典范的视为可表函子$\mathrm{Hom}_{\mathscr{C}}(X_0,-)\mid_{\mathscr{C}^{X_0}}$.于是$G$右作用到每个$F^{X_0}(X)$上.于是$F^{X_0}$可经范畴$\mathscr{C}(G^{\mathrm{op}})$分解.我们把函子$\mathscr{C}^{X_0}\to\mathscr{C}(G^{\mathrm{op}})$仍然记作$F^{X_0}$.我们只需证明这是一个范畴等价.
			
			\qquad
			
			证明$F^{X_0}$是本质满的.任取一个对象$E\in\mathscr{C}(G^{\mathrm{op}})$.取它的连通分支分解,问题归结为设$E$是连通对象的情况.此即$E$赋予的右$G$作用是可迁的.于是对任意$e\in E$,我们有$G^{\mathrm{op}}-\textbf{Sets}$中的态射$p_e^0:G^{\mathrm{op}}\to E$为$\sigma\mapsto\sigma(e)$.再记满射$f_e=p_e^0\circ e_{\zeta_0}^{-1}:F(X_0)\to E$.那么对任意$s\in S_e=\{\sigma\in G^{\mathrm{op}}\mid\sigma(e)=e\}$和任意$\sigma\in G$,有:
			\begin{align*}
				f_e\circ F(s)(e_{\zeta_0}(\omega))&=p_e^0\circ e_{\zeta_0}^{-1}\circ e_{\zeta_0}(s\circ\omega)\\&=(s\circ\omega)e\\&=\omega(se)\\&=\omega e\\&=f_e\circ e_{\zeta_0}(\omega)
			\end{align*}
			
			于是$f_e$被$S_e$固定,于是$f_e$就要经商对象$F(X_0)/S_e$分解:
			$$\xymatrix{G^{\mathrm{op}}\ar[dr]_{p_e^0}\ar[r]^{e_{\zeta_0}}&F(X_0)\ar[r]\ar[d]_{f_e}&F(X_0)/S_e\ar[dl]^{\widetilde{f_e}}\\&E&}$$
			
			我们记$\mathscr{C}$中的商对象为$p_e:X_0\to X_0/S_e$,那么(G5)说明$F(p_e):F(X_0)\to F(X_0)/S_0$也是商对象(类似的讨论可以说明这里$\widetilde{f_e}$可以表示为某个$F(\widetilde{p_e})$,这要在证明$F^{X_0}$是完全函子用到).因为$X_0$是Galois对象,于是$|F(X_0)/S_e|=|F(X_0)|/|S_e|=|G|/|S_e|=[G:S_e]=|E|$(最后一个等式是轨道公式以及$E$上的$G$作用是可迁的).于是这里$\widetilde{f_e}$是双射$G$同态,于是它是$G$同构.这完成了本质满的证明.
			
			\qquad
			
			证明$F^{X_0}$是完全忠实的.首先我们解释下如果$X\in\mathscr{C}^{X_0}$是连通对象,那么$F(X)$也是连通的,换句话讲$\mathrm{Hom}_{\mathscr{C}}(X_0,X)$作为$\mathrm{Aut}_{\mathscr{C}}(X_0)^{\mathrm{op}}$的群作用集合是单轨道的.这件事是因为,如果任取态射$f,g:X_0\to X$,则它们都是严格满态射,任取$c\in F(X)$,那么按照$F(f)$和$F(g)$都是满射,就存在$a,b\in F(X_0)$满足$F(f)(a)=c$和$F(g)(a)=c$.又因为$X_0$是Galois对象,于是存在态射$h:X_0\to X_0$满足$F(h)$把$a$映为$b$.于是$g\circ h$和$f$在$F$作用下都把$a$映射为$c$,于是刚性定理说明$g\circ h=f$,也即$\mathrm{Aut}_{\mathscr{C}}(X_0)^{\mathrm{op}}$在$\mathrm{Hom}_{\mathscr{C}}(X_0,X)$上的作用是单轨道的.
			
			\qquad
			
			接下来任取对象$X,Y\in\mathscr{C}^{X_0}$,如果记$X=\coprod_iX_i$和$Y=\coprod_jY_j$都是连通分支分解,那么我们有如下交换图表(其中$G=\mathrm{Aut}_{\mathscr{C}}(X_0)^{\mathrm{op}}$,右侧的垂直分解依赖于我们上一段所解释的如果$X$连通则$F(X)$是$\textbf{Set}(G)$范畴中的连通对象,也即单轨道空间),于是为证明第一行是双射,归结为证明第二行的每个对应分量之间的映射是双射,也即不妨设$X,Y$都是$\mathscr{C}$中的连通对象.
			$$\xymatrix{\mathrm{Hom}_{\mathscr{C}}(X,Y)\ar[rr]\ar@{=}[d]&&\mathrm{Hom}_G(F(X),F(Y))\ar@{=}[d]\\\prod_i\coprod_j\mathrm{Hom}_{\mathscr{C}}(X_i,Y_j)\ar[rr]&&\prod_i\coprod_j\mathrm{Hom}_G(F(X_i),F(Y_j))}$$
			
			此时函子$F^{X_0}$是忠实的就是刚性定理.接下来证明它是完全的:任取一个$G^{\mathrm{op}}$同态$u:F(X)\to F(Y)$,固定一个元$e\in F(X)$.我们之前把群作用的稳定子记作$S_e$,那么有$S_e\subseteq S_{u(e)}$.于是$p_{u(e)}:X_0\to X_0/S_{u(e)}$就要经$X_0/S_e$分解:
			$$\xymatrix{X_0/S_e\ar[rr]^{p_{e,u(e)}}&&X_0/S_{u(e)}\\X_0\ar[u]_{p_e}\ar@/_1pc/[urr]_{p_{u(e)}}&&}$$
			
			那么按照本质满中所证明的,有典范同构$\widetilde{f_e}:F(X)\cong F(X_0/S_e)$和$\widetilde{f_{u(e)}}:F(X_0/S_{u(e)})\cong F(Y)$,进而按照商对象的泛性质有如下交换图表,这得到$u$可以表示为$F(\widetilde{p_{u(e)}}\circ p_{e,u(e)}\circ\widetilde{p_e}^{-1})$.完成证明.
			$$\xymatrix{&F(X_0)\ar[dl]_{F(p_e)}\ar[dr]^{F(p_{u(e)})}&\\F(X_0/S_e)\ar[d]_{F(\widetilde{p_e})}^{\cong}\ar[rr]^{F(p_{e,u(e)})}&&F(X_0/S_{u(e)})\ar[d]^{F(\widetilde{p_{u(e)}})}_{\cong}\\F(X)\ar[rr]_u&&F(Y)}$$
		\end{proof}
	\end{enumerate}
\end{enumerate}
\subsection{纤维函子的pro-可表性}
\begin{enumerate}
	\item pro-对象和pro-可表性.设$(\mathscr{C},F)$是一个Galois范畴,记$\textbf{Pro}(\mathscr{C})$是这样一个范畴,它的对象是$\mathscr{C}$上的逆向系统$\underline{X}=\{(X_i)_{i\in I},(\varphi_{ij}:X_j\to X_i)_{i\le j}\}$.两个pro-对象之间的态射是这样定义的:如果再记$\underline{X}'=\{(X_{i'}')_{i'\in I'},(\varphi_{i'j'}':X_{j'}'\to X_{i'}')_{i'\le j'}\}$.那么定义:
	$$\mathrm{Hom}_{\textbf{Pro}(\mathscr{C})}(\underline{X},\underline{X}')=\varprojlim\limits_{i'\in I'}\varinjlim\limits_{i\in I}\mathrm{Hom}_{\mathscr{C}}(X_i,X_{i'}')$$
	
	这样定义态射的目的是为了有如下性质:
	\begin{enumerate}[(1)]
		\item 如果把$\mathscr{C}$的对象$A$平凡的视为一个逆向系统,得到的典范函子$\alpha:\mathscr{C}\to\textbf{Pro}(\mathscr{C})$是完全忠实的.换句话讲$\mathscr{C}$典范的视为$\textbf{Pro}(\mathscr{C})$的子范畴.
		\item 如果记pro-对象$\underline{X}=\{(X_i)_{i\in I},(\varphi_{ij}:X_j\to X_i)_{i\le j}\}$,那么$\alpha(X_i)$和$\alpha(\varphi_{ij})$构成的$\textbf{Pro}(\mathscr{C})$中的逆向系统的极限恰好是$\underline{X}$本身.
		\item 对$\mathscr{C}$的对象$A$,有$\alpha(A)$是$\textbf{Pro}(\mathscr{C})$中的余紧对象,换句话讲关于对象$A$的逆变可表函子与逆向极限可交换:
		$$\mathrm{Hom}_{\textbf{Pro}(\mathscr{C})}(\varprojlim\limits_{i\in I}X_i,A)\cong\varinjlim\limits_{i\in I}\mathrm{Hom}_{\textbf{Pro}(\mathscr{C})}(X_i,A)$$
	\end{enumerate}
	
	一个函子$F:\mathscr{C}\to\textbf{FSets}$可以典范的延拓为函子$\mathrm{Pro}(F):\textbf{Pro}(\mathscr{C})\to\textbf{Pro}(\textbf{FSets})$.称函子$F:\mathscr{C}\to\textbf{FSets}$是pro-可表函子(pro-representable),如果存在$\mathscr{C}$上的一个pro-对象$\underline{X}=\{(X_i)_{i\in I},(\varphi_{ij}:X_j\to X_i)_{i\le j}\}$使得$F$自然同构于$\mathrm{Hom}_{\textbf{Pro}(\mathscr{C})}(\underline{X},-)\mid_{\mathscr{C}}$.称一个pro-可表函子$F$是严格pro-可表函子(strictly pro-representable),如果它的表示对象$\underline{X}$中的态射$\varphi_{ij}$都是满态射.
	\item 例如如果$\Pi$是射影有限群,记$\mathscr{C}(\Pi)$是$\Pi$的全体左连续作用有限离散集合(拓扑群$\Pi$在离散集合$X$上的作用称为连续的,如果作用对应的映射$\Pi\times X\to X$是连续的)构成的范畴,那么$\textbf{Pro}(\mathscr{C}(\Pi))$就是由$\Pi$的左连续作用的Hausdorff紧全不连通空间构成的范畴.【证明在SGA1的V.5.2.另外$\textbf{Ind}(\mathscr{C}(\Pi))$就是带一个$\Pi$左连续作用的全体集合(未必有限)构成的范畴】
	\item 设Galois范畴$\mathscr{C}$是本质小范畴,设$F:\mathscr{C}\to\textbf{FSets}$是纤维函子,那么它总是严格pro-可表函子.它的pro-表示对象可以这样构造:按照本质小范畴条件,我们可以选取一个指标集$I$,使得每个$S_i$都是非平凡的连通对象,满足$\mathscr{C}$的任意非平凡连通对象$X$,都能找到一个指标$t\in I$使得$X\cong S_t$.我们赋予$I$支配偏序(即$i\le j$当且仅当$S_i\le S_j$,也即$\mathrm{Hom}_{\mathscr{C}}(S_j,S_i)$是非空集).任取$\underline{\zeta}=(\zeta_i)_{i\in I}\in\prod_{i\in I}F(S_i)$,倘若$i\le j$,按照刚性定理,我们就取$\varphi_{ij}^{\underline{\zeta}}:S_j\to S_i$是唯一的(这里唯一性来自刚性定理)诱导的$F(S_j)\to F(S_i)$把$\zeta_j$映射为$\zeta_i$的态射.另外如果$i\le j\le k$,同样按照刚性定理得到余圈公式$\varphi_{ij}^{\underline{\zeta}}\circ\varphi_{jk}^{\underline{\zeta}}=\varphi_{ik}^{\underline{\zeta}}$.于是每取定一个$\underline{\zeta}$,就诱导了如下逆向系统.我们就断言这是$F$的pro-可表对象.
	$$\mathrm{S}^{\underline{\zeta}}=\{(S_i)_{i\in I},(\varphi_{ij}^{\underline{\zeta}})_{i\le j}\}$$
	\begin{proof}
		
		对任意非初对象的对象$X\in\mathscr{C}$,我们取如下典范映射,它把一个态射$u:S_i\to X$映射为$F(u)(\zeta_i)\in F(X)$.这是单射是因为$\{S_i\mid i\in I\}$在支配偏序下构成了有向集.这是满射是因为我们解释过任取$\xi\in F(X)$,那么存在连通对象$X_0$和$\xi_0\in F(X_0)$,以及一个态射$u:X_0\to X$满足$F(u)(\xi_0)=\xi$.最后我们解释过由于$S_i$是非初对象的连通对象,导致这里$\varphi_{ij}^{\underline{\zeta}}$总是严格满态射,这就得证.
	\end{proof}
    \item 反过来,如果纤维函子$F:\mathscr{C}\to\textbf{FSets}$被$\mathscr{C}$的一个逆向系统$P=\{(P_i)_{i\in I},(\varphi_{ij}:P_j\to P_i)_{i\le j}\}$表示,我们断言这里每个$P_i$都是非平凡的连通对象.
    \begin{proof}
    	
    	假设存在一个指标$t$满足$P_t=\emptyset_{\mathscr{C}}$是初对象,那么对任意满足$t\le s$的指标$s$,就存在态射$P_s\to P_t$,于是存在集合之间的映射$F(P_s)\to F(P_t)=\emptyset$,迫使$F(P_s)=\emptyset$,进而$P_s$也是初对象.那么我们有:
    	\begin{align*}
    		F(X)&=\mathrm{Hom}_{\textbf{Pro}(\mathscr{C})}(P,X)\\&=\varinjlim\mathrm{Hom}_{\mathscr{C}}(P_i,X)\\&=\mathrm{Hom}_{\mathscr{C}}(P_t,X)=\{*\}
    	\end{align*}
    
        这总是一个单点集.但是如果取$X=\emptyset_{\mathscr{C}}$就和$F(\emptyset_{\mathscr{C}})=\emptyset$矛盾.于是$P_i$总不是初对象.下面再设$P_t=A\coprod B$,那么按照(G5)有$F(P_t)=F(A)\coprod F(B)$.按照定义有$F(P_t)=\varinjlim\limits_i\mathrm{Hom}_{\mathscr{C}}(P_i,P_t)$.我们记右侧的恒等映射$1_{P_t}:P_t\to P_t$对应的$F(P_t)$中的元为$a_t$,那么$a_t\in F(A)\coprod F(B)$,可不妨设$a_t\in F(A)$.但是按照定义$F(A)=\varinjlim\mathrm{Hom}_{\mathscr{C}}(P_i,A)$,于是$a_t\in F(A)$意味着存在足够大的指标$j$使得$\varphi_{tj}:P_j\to P_t$要经典范单态射$A\to P_t$分解.那么按照$\varphi_{tj}$是严格满态射,就有典范的$F(P_j)\to F(A)\to F(P_t)=F(A)\coprod F(B)$是满射,这迫使$F(B)=\emptyset$,于是$B=\emptyset_{\mathscr{C}}$.从而$P_t$总是非平凡的连通对象.
    \end{proof}
    \item 我们之前解释过全体Galois对象构成了全体连通对象的共尾子类.于是取$(S_i)_{i\in I}$的由Galois对象构成的子逆向系统(不妨把指标集仍然记作$I$,尽管新的$I$应该是旧的$I$的子集)$\mathscr{G}^{\underline{\zeta}}=\{(P_i)_{i\in I},(\varphi_{ij}^{\underline{\zeta}}:P_j\to P_i)_{i\le j}\}$就仍然是纤维函子$F$的pro-可表对象.这里$\underline{\zeta}=(\zeta_i)_{i\in I}\in\prod_{i\in I}F(P_i)$.换句话讲纤维函子$F$的pro-可表对象我们可以要求由Galois对象构成.这样做所额外得到的性质是,典范映射:
    $$\mathrm{Aut}_{\mathscr{C}}(P_i)=\mathrm{Hom}_{\mathscr{C}}(P_i,P_i)\to F(P_i)=\mathrm{Hom}_{\textbf{Pro}(\mathscr{C})}(P,P_i)$$
    
    是一个双射.换句话讲对任意态射$u:P\to P_i$,恰好存在唯一的态射(自同构)$v:P_i\to P_i$,使得$u=v\circ\varphi_i$,这里$\varphi_i:P\to P_i$是逆向系统中的结构态射.
\end{enumerate}
\subsection{Galois范畴的基本群}

设$(\mathscr{C},F)$是Galois范畴,$F$在函子范畴中的自同构群(也即$F$自身上的全体自然同构构成的群)称为$\mathscr{C}$的关于基点$F$的基本群,记作$\pi_1(\mathscr{C};F)$.
\begin{enumerate}
	\item 经典拓扑例子.设$B$是连通,局部道路连通,局部单连通拓扑空间,记$\textbf{FCov}(B)$表示$B$的有限覆盖空间范畴,此时对$b\in B$,定义纤维函子$F_b$就是取点$b$纤维的函子:它把有限覆盖空间$p:\widetilde{X}\to X$映射为有限集合$p^{-1}(b)$.此时$(\textbf{FCov}(B),F_b)$构成Galois范畴,并且这里$F_b$称为基点是因为把它视为了点$b\in B$.如果记拓扑基本群为$\pi_1^{\textbf{Top}}(B,b)$,那么我们这里定义的基本群$\pi_1(\textbf{FCov}(B);F_b)$就是$\pi_1^{\textbf{Top}}(B,b)$的射影完备化.【】
	\item 例子.如果$\Pi$是一个射影有限群,记$\mathscr{C}(\Pi)$表示全体有限离散集合上赋予连续左$\Pi$作用构成的范畴.我们之前解释了这个范畴中的连通对象恰好是$\Pi/U$,其中$U$是开子群,它的群作用就是$\Pi$在左陪集上的正则作用;这个范畴中的Galois对象恰好是$\Pi/N$,其中$N$是开正规子群.取遗忘函子$F:\mathscr{C}(\Pi)\to\textbf{FSets}$,那么$(\mathscr{C}(\Pi),F)$是Galois范畴,并且此时基本群为$\pi_1(\mathscr{C}(\Pi),F)=\Pi$.
	\begin{proof}
		
		这件事是因为如果记$\Pi$是有限群范畴上的逆向系统$\{(P_i),(\varphi_{ij}:P_j\to P_i)_{i\le j}\}$的逆向极限,那么典范态射$\pi_i:\Pi\to P_i$使得$P_i$作为$\Pi$集合恰好是$\mathscr{C}(\Pi)$的全部(同构意义下的)Galois对象.于是下面定理告诉我们$\pi_1(\mathscr{C}(\Pi),F)=\Pi$.
	\end{proof}
	\item $\pi_1(\mathscr{C};F)$作为射影有限群.任取Galois范畴$(\mathscr{C},F)$,取$\Pi=\prod_X\mathrm{Aut}_{\textbf{FSets}}(F(X))$,其中$X$取遍对象(这里要求$\mathscr{C}$是本质小范畴,例如$\textbf{FSets}$是本质小范畴),如果对每个有限集$\mathrm{Aut}_{\textbf{FSets}}(F(X))$取离散拓扑,再取乘积拓扑,得到的$\Pi$就是射影有限群.我们有单同态$\pi_1(\mathscr{C};F)\to\Pi$为把$\theta$映射为$(\theta(X))_{X\in\mathscr{C}}$.如果对态射$\alpha:X\to Y$,记$S_{\alpha}=\{(\sigma_X)\in\Pi\mid\sigma_Y\circ F(\alpha)=F(\alpha)\circ\sigma_X\}$,那么这是$\Pi$的闭子集(我们要证明使得这个图表交换的$(\sigma_X)$构成乘积拓扑的闭子集,但是对于不为$X,Y$的对象$Z$,$\sigma_Z$的选取是任意的,而$\sigma_X$和$\sigma_Y$的选取是有限的,这一定是乘积拓扑的闭子集),并且$\pi_1(\mathscr{C};F)$视为$\Pi$的子集是$\cap S_{\alpha}$,其中$\alpha$取遍$\mathscr{C}$的态射(这里也要求$\mathscr{C}$是本质小范畴),于是$\pi_1(\mathscr{C};F)$是射影有限群$\Pi$的闭子群,所以它也是射影有限群.
	\item 引理.任取$X,Y\in\mathscr{G}$,设$X\le Y$,任取态射$\psi:Y\to X$,那么右复合$\psi$诱导了双射$\mathrm{Aut}_{\mathscr{C}}(X)\to\mathrm{Hom}_{\mathscr{C}}(Y,X)$.并且对任意态射$\varphi,\psi:Y\to X$和任意$\omega_Y\in\mathrm{Aut}_{\mathscr{C}}(Y)$,都存在唯一的态射$\omega_X\in\mathrm{Aut}_{\mathscr{C}}(X)$使得如下图表交换.
	$$\xymatrix{Y\ar[rr]^{\omega_Y}\ar[d]_{\psi}&&Y\ar[d]^{\varphi}\\X\ar[rr]_{\omega_X}&&X}$$
	
	于是我们定义了一个映射$r_{\varphi,\psi}:\mathrm{Aut}_{\mathscr{C}}(Y)\to\mathrm{Aut}_{\mathscr{C}}(X)$,$\omega_Y\to\omega_X$.如果任取$y\in F(Y)$,记$F(\psi)(y)=x\in F(x)$,那么这个映射有如下具体表示.特别的这说明这个映射总是满射,并且如果取$\varphi=\psi$,则$r_{\varphi}=r_{\varphi,\varphi}$是一个满的群同态.
	$$\xymatrix{\mathrm{Aut}_{\mathscr{C}}(Y)\ar[r]^{e_y}&F(Y)\ar[r]^{\varphi}&F(X)\ar[r]^{e_x^{-1}}&\mathrm{Aut}_{\mathscr{C}}(X)}$$
	\begin{proof}
		
		按照$X$是非初对象的连通对象,于是$\psi:Y\to X$自动是严格满态射,于是特别的右复合$\psi$诱导了一个单射$\mathrm{Aut}_{\mathscr{C}}(X)\to\mathrm{Hom}_{\mathscr{C}}(Y,X)$.按照Galois对应定理有$|\mathrm{Hom}_{\mathscr{C}}(Y,X)|=|F(X)|$,按照$X$是Galois对象又有$|F(X)|=|\mathrm{Aut}_{\mathscr{C}}(X)|$.于是这里右复合$\psi$得到的映射$\mathrm{Aut}_{\mathscr{C}}(X)\to\mathrm{Hom}_{\mathscr{C}}(Y,X)$是双射.特别的,存在唯一的态射$\omega_X$使得图表交换.
	\end{proof}
    \item 这一条我们证明基本群同构于Galois范畴的全体Galois对象(严格说是Galois对象每个同构类取一个元,用到本质有限范畴的条件)的自同构群的逆向极限.考虑我们之前定义的态射$\varphi_{X,Y}^{\underline{\zeta}}:Y\to X$,它是唯一的满足$F(\varphi_{X,Y}^{\underline{\zeta}})(\zeta_Y)=\zeta_X$的态射,我们解释过这些态射满足余圈条件.于是$\{\left(\mathrm{Aut}_{\mathscr{C}}(X)\right)_{X\in\mathscr{G}},(r_{\varphi_{X,Y}^{\underline{\zeta}}}:\mathrm{Aut}_{\mathscr{C}}(Y)\to\mathrm{Aut}_{\mathscr{C}}(X))_{X\le Y}\}$是有限群构成的逆向系统,它的极限记作$\Pi$.我们断言如下映射是射影有限群之间的同构(前面给出了$\pi_1(\mathscr{C};F)$的射影有限群结构).
    \begin{align*}
    	u^{\underline{\zeta}}:\pi_1(\mathscr{C};F)&\to\Pi^{\mathrm{op}}\\
    	\theta&\mapsto\left(e_{\zeta_X}^{-1}(\theta(X)(\zeta_X))\right)_{X\in\mathscr{G}}
    \end{align*}
    \begin{proof}
    	
    	我们知道射影有限群之间的连续双射同态就是同胚同构(因为射影有限群是紧Hausdorff空间,紧Hausdorff空间之间的连续双射一定是同胚),于是归结为证明这个$u^{\underline{\zeta}}$是双射并且连续.为了证明它是双射我们来构造逆映射.任取$\underline{\omega}=(\omega_X)_{X\in\mathscr{G}}\in\Pi$.对任意非初对象的连通对象$Z\in\mathscr{C}$,记$\widehat{Z}$为它的Galois闭包.我们定义$\theta_{\underline{\omega}}\in\pi_1(\mathscr{C};F)$如下:
    	$$\theta_{\underline{\omega}}(Z):\xymatrix{F(Z)\ar[r]^{e_{\widehat{Z}}^{-1}}&\mathrm{Hom}_{\mathscr{C}}(\widehat{Z},Z)\ar[r]^{\circ\omega_{\widehat{Z}}}&\mathrm{Hom}_{\mathscr{C}}(\widehat{Z},Z)\ar[r]^{e_{\zeta_{\widehat{Z}}}}&F(Z)}$$
    	
    	那么为了验证$\theta_{\underline{\omega}}$的确是$\pi_1(\mathscr{C};F)$中的元,任取态射$\alpha:X\to Y$,按照我们之前的【引理】,不妨设$X,Y$都是非初对象的连通对象,我们只需证明有如下交换图表:
    	$$\xymatrix{F(X)\ar[d]_{F(\alpha)}\ar[r]^{e_{\widehat{X}}^{-1}}&\mathrm{Hom}_{\mathscr{C}}(\widehat{X},X)\ar[r]^{\circ\omega_{\widehat{X}}}&\mathrm{Hom}_{\mathscr{C}}(\widehat{X},X)\ar[r]^{e_{\zeta_{\widehat{X}}}}&F(X)\ar[d]^{F(\alpha)}\\F(Y)\ar[r]^{e_{\widehat{Y}}^{-1}}&\mathrm{Hom}_{\mathscr{C}}(\widehat{Y},Y)\ar[r]^{\circ\omega_{\widehat{Y}}}&\mathrm{Hom}_{\mathscr{C}}(\widehat{Y},Y)\ar[r]^{e_{\zeta_{\widehat{Y}}}}&F(Y)}$$
    	
    	任取$x\in X$,记唯一的满足$\zeta_{\widehat{X}}\mapsto x$的态射$\widehat{X}\to X$为$\varphi$,记$F(\alpha)(x)=y$,再记唯一的满足$\zeta_{\widehat{Y}}\to y$的态射$\widehat{Y}\to Y$为$\psi$,那么这个交换图表等价于讲$F(\alpha)\left(\varphi(\omega_{\widehat{X}}(\zeta_{\widehat{X}}))\right)=\psi\left(\omega_{\widehat{Y}}(\zeta_{\widehat{Y}})\right)$.这件事是因为如下图表是交换的,其中下方小方格交换是因为这两个态射$\widehat{X}\to Y$都把$\zeta_{\widehat{X}}$映射为$y$,刚性定理保证它是交换的;上方小方格交换是$\underline{\omega}\in\Pi$的定义.
    	$$\xymatrix{\widehat{X}\ar[rr]^{\varphi_{\widehat{Y},\widehat{X}}^{\underline{\zeta}}}\ar[d]_{\omega_{\widehat{X}}}&&\widehat{Y}\ar[d]^{\omega_{\widehat{Y}}}\\\widehat{X}\ar[rr]^{\varphi_{\widehat{Y},\widehat{X}}^{\underline{\zeta}}}\ar[d]_{\varphi}&&\widehat{Y}\ar[d]^{\psi}\\X\ar[rr]_{\alpha}&&Y}$$
    	
    	最后证明$u_{\underline{\zeta}}$是连续的.按照构造$\Pi$是$\prod_{X\in\mathscr{G}}\mathrm{Aut}_{\mathscr{C}}(X)$的闭子集,所以我们只要任取一个$Y\in\mathscr{G}$,任取$\tau\in\mathrm{Aut}_{\mathscr{C}}(Y)$,只要证明$\left(\prod_{Y\not=X\in\mathscr{G}}\mathrm{Aut}(X)\right)\times\{\tau\}$在$u^{\underline{\zeta}}$下的原像是$\pi_1(\mathscr{C};F)\subseteq\prod_{X\in\mathscr{C}}F(X)$的开集.但是后者是离散有限空间的乘积拓扑.这个原像只涉及到分量$Y$,所以一定是开集.
    \end{proof}
\end{enumerate}
\subsection{Galois范畴主要定理}

这一节我们的Galois范畴总约定是本质满的.
\begin{enumerate}
	\item 设$(\mathscr{C},F)$是Galois范畴.对任意对象$X$,有$F(X)$典范的具备一个$\pi_1(\mathscr{C};F)$的左作用结构,于是我们有如下典范的分解.那么这里$H:\mathscr{C}\to\mathscr{C}(\pi_1(\mathscr{C};F))$是范畴等价.
	$$\xymatrix{\mathscr{C}\ar[rr]^F\ar[d]_{H}&&\textbf{FSets}\\\mathscr{C}(\pi_1(\mathscr{C};F))\ar[urr]_{\textbf{遗忘函子}}&&}$$
	\begin{proof}
		
		我们证明了有函子的自然同构$e_{\underline{\zeta}}:\varinjlim\mathrm{Hom}_{\mathscr{C}}(X,-)\mid_{\mathscr{C}}\cong F$.以及有射影有限群的同构$\pi_1(\mathscr{C};F)\cong\Pi^{\mathrm{op}}$.于是问题归结为证明$F^{\underline{\zeta}}=\mathrm{Hom}_{\textbf{Pro}(\mathscr{C})}(\underline{G}^{\underline{\zeta}},-)\mid_{\mathscr{C}}:\mathscr{C}\to\textbf{FSets}$要经一个范畴等价$\mathscr{C}\to\mathscr{C}(\Pi^{\mathrm{op}})$分解,把这个函子仍然记作$F^{\underline{\zeta}}$.我们只要证明这是范畴等价函子.但是这件事几乎就是Galois对应定理:
		\begin{itemize}
			\item $F^{\underline{\zeta}}$是本质满的:对任意对象$E\in\mathscr{C}(\Pi^{\mathrm{op}})$,如果取连通分解$E=\coprod_iE_i$,那么只要存在对象$X_i\in\mathscr{C}$满足$E_i\cong F(X_i)$,就有$F(\coprod_iX_i)=E$.于是不妨设$E$本身是连通对象,也即$\pi_1(\mathscr{C};F)$在$E$上的作用是可迁的.固定一个元$e\in E$,那么$\pi_1(\mathscr{C};F)$在$E$上的群作用就被满射$\pi_1(\mathscr{C};F)\to E$,$\sigma\mapsto\sigma(e)$所唯一决定.但是按照$E$是有限集合,就可以取到支配偏序下足够大的Galois对象$X$使得这个满射$\pi_1(\mathscr{C};F)\to E$可以分解为$\pi_1(\mathscr{C};F)\to\mathrm{Aut}_{\mathscr{C}}(X)\to E$.换句话讲$E$上的$\Pi^{\mathrm{op}}$作用等同于$\mathrm{Aut}_{\mathscr{C}}(X)^{\mathrm{op}}$作用.于是按照$\mathscr{C}^X\to\mathscr{C}(\mathrm{Aut}_{\mathscr{C}}(X)^{\mathrm{op}})$是本质满的,就得证.
			\item $F^{\underline{\zeta}}$是完全忠实的:任取$\mathscr{C}$中的对象$Z,Z'$,我们可以取一个Galois对象$X$满足$X\ge Z$和$X\ge Z'$,于是$Z,Z'$落在$\mathscr{C}^X$中,按照Galois对应定理有$F^{\underline{\zeta}}$限制在$\mathscr{C}^X$上是完全忠实的,这就得证.
		\end{itemize}
	\end{proof}
    \item 按照这个定理.对于Galois范畴$(\mathscr{C};F)$的很多问题,我们可以记范畴等价$\mathscr{C}\cong\mathscr{C}(\pi_1(\mathscr{C};F))$的拟逆函子为$G$,用$\mathscr{C}(\pi_1(\mathscr{C};F))$替代$\mathscr{C}$,用$F\circ G$替代$F$,于是总可以把问题归结为设$\mathscr{C}=\mathscr{C}(\Pi)$的情况.
    \item 推论.设$(\mathscr{C};F)$是Galois范畴,记$\Pi=\pi_1(\mathscr{C};F)$,任取对象$X\in\mathscr{C}$,那么$X$是连通对象当且仅当$\Pi$在$F(X)$上的作用是可迁的.
    \item 设$(\mathscr{C};F)$是Galois范畴,设$Q=(Q_i)_{i\in I}$是一个pro-对象.取它的pro-可表函子为$G:\mathscr{C}\to\textbf{Sets}$,$G(X)=\varprojlim\mathrm{Hom}_{\mathscr{C}}(Q_i,X)$.那么如下命题互相等价:
    \begin{enumerate}[(1)]
    	\item $G$和有限余积可交换.
    	\item $G$和二元余积可交换.
    	\item 每个$Q_i$都是非平凡的连通对象.
    	\item 把$F$分解的典范范畴等价仍然记作$F$,那么$F(Q)$同构于$\Pi/H$,其中$\Pi=\pi_1(\mathscr{C};F)$,而$H\subseteq\Pi$是一个闭子群.
    	\item 把$G$按照经$F$分解的范畴等价$\mathscr{C}\cong\mathscr{C}(\Pi)$视为$\mathscr{C}(\Pi)\to\textbf{Sets}$的函子,那么它就是$E\mapsto E^H$,其中$H\subseteq\Pi$是一个闭子群,$E^H$表示$E$的被$H$固定的子集.
    \end{enumerate}
    \begin{proof}
    	
    	不妨设$\mathscr{C}=\mathscr{C}(\Pi)$.(1)推(2)是平凡的,(2)推(3)和我们之前证明的纤维函子$F$的pro-可表对象的分量一定是非平凡连通对象是一样的.(3)推(4),按照这里$Q_i$都是非初对象,于是$\varprojlim Q_i$是非空集合(一般的,如果$\{(Q_i),(u_{ij}:Q_j\to Q_i)_{i\le j}\}$是有限集合范畴上以有向集作为指标集的逆向系统,记典范投影映射$\prod_iQ_i\to Q_i$为$\pi_i$,那么逆向极限就是$Q=\cap_{i\le j}\pi_j^{-1}\circ u_{ij}^{-1}(Q_i)\subseteq\prod_iQ_i$,赋予每个$Q_i$离散拓扑,那么Tychonoff定理说明$\prod_iQ_i$是紧空间,于是它的一族闭子集的交非空当且仅当这族闭子集满足有限交非空,但是按照指标集是有向集,这个闭子集族的有限交是非空的,这就导致$Q$是非空集合).于是可以取一个元$a\in\varprojlim Q_i$,那么有$\Pi$同态$\Pi\to\varprojlim Q_i$为把$\sigma\in\Pi$映为$(\sigma(a_i))_{i\in I}$.按照$Q_i$都是非平凡连通对象,有$Q_i$上的$\Pi$作用是可迁的,于是$\Pi\to\varprojlim Q_i\to Q_i$是满射.于是这个$\Pi\to\varprojlim Q_i$是满射.如果记$a$的固定子群为$H$,那么这是一个闭子群,并且有轨道同构$\Pi/H\cong\varprojlim Q_i$.(4)推(5)是因为$\mathrm{Hom}_{\textbf{Pro}(\mathscr{C}(\Pi))}(\Pi/H,E)=E^H$.而(5)推(1)是平凡的.
    \end{proof}
    \item 设$(\mathscr{C};F)$是Galois范畴,取$F$的一个pro-可表对象为$P$.再任取$\mathscr{C}$的一个pro-对象$P'=(P_i')_{i\in I}$,记函子$F':\mathscr{C}\to\textbf{Sets}$为$F'(X)=\varprojlim\mathrm{Hom}_{\mathscr{C}}(P_i',)$.那么如下命题互相等价:
    \begin{enumerate}[(1)]
    	\item $P'\cong P$或者$F'\cong F$.
    	\item $F'$是纤维函子.
    	\item $F'$和二元余积可交换,并且如果$X$不是初对象,就有$F(X)$不是空集.
    	\item $\mathscr{C}$的非平凡连通对象恰好是那些同构于某个$P_i'$的对象.
    \end{enumerate}

    设$\mathscr{C}$是Galois范畴.如果$F_1,F_2$是$\mathscr{C}$的两个纤维函子,那么$\pi_1(\mathscr{C};F_1)$和$\pi_1(\mathscr{C};F_2)$差一个非典范的内自同构.特别的,阿贝尔化$\pi_1(\mathscr{C};F)^{\mathrm{ab}}$在相差一个典范同构的意义下不依赖于基点$F$的选取.
    \begin{proof}
    	
    	【】
    	
    	我们只证明最后一个命题.设有两个纤维函子$F_1,F_2:\mathscr{C}\to\textbf{FSets}$.对$i=1,2$我们取定一个$\underline{\zeta^i}\in\prod_{X\in\mathscr{G}}F^i(X)$,那么它诱导了同构$e_{\underline{\zeta^i}}^{F_i}:\mathrm{Hom}_{\textbf{Pro}(\mathscr{C})}(\underline{G}^{\underline{\zeta^i}})\mid_{\mathscr{C}}\cong F_i$.于是问题归结为证明在$\textbf{Pro}(\mathscr{C})$中有$\underline{G}^{\underline{\zeta^1}}$和$\underline{G}^{\underline{\zeta^2}}$是同构的.按照定义,它们之间的态射集为:
    	$$\mathrm{Hom}_{\textbf{Pro}(\mathscr{C})}(\underline{G}^{\underline{\zeta^1}},\underline{G}^{\underline{\zeta^2}})=\varprojlim\limits_X\varinjlim\limits_Y\mathrm{Hom}_{\mathscr{C}}(Y,X)=\varprojlim\limits_X\varinjlim\limits_Y\mathrm{Aut}_{\mathscr{C}}(X)=\varprojlim\limits_X\mathrm{Aut}_{\mathscr{C}}(X)$$
    	
    	这里逆向系统里的结构映射是$r_{\varphi^1_{X,Y},\varphi^2_{X,Y}}:\mathrm{Aut}_{\mathscr{C}}(Y)\to\mathrm{Aut}_{\mathscr{C}}(X),X,Y\in\mathscr{G},X\le Y$.并且这个逆向极限中元素的逆元恰好在$\mathrm{Hom}_{\textbf{Pro}(\mathscr{C})}(\underline{G}^{\underline{\zeta^2}},\underline{G}^{\underline{\zeta^1}})$中.于是这其中的每个元都是同构,于是归结为证明这个逆向极限$\varprojlim\limits_X\mathrm{Aut}_{\mathscr{C}}(X)$非空.但是以有向集为指标集的有限非空集合构成的逆向系统的极限总是非空的.
    \end{proof}
\end{enumerate}
\subsection{Galois范畴的基本群胚}

一个Galois范畴$\mathscr{C}$的全部基本函子和它们之间的自然同构构成了一个连通群胚(一个范畴称为连通群胚,如果所有态射都是同构,并且任意两个对象之间的态射集是非空的),这称为$\mathscr{C}$的基本群胚.
\begin{enumerate}
	\item 如果我们称可以作为Galois范畴$\mathscr{C}$的纤维函子的pro-可表对象的pro-对象为基本pro-对象,那么全体基本pro-对象和它们之间的同构构成的范畴是和$\mathscr{C}$的基本群胚是逆变范畴等价的.
	\item 我们称$F_1\to F_2$的自然同构为$F_1$到$F_2$的道路(path),全体从$F_1$到$F_2$的道路构成的集合记作$\pi_1(\mathscr{C};F_1,F_2)$.并且$\pi_1(\mathscr{C};F,F)$就是我们的基本群$\pi_1(\mathscr{C};F)$.所以Galois范畴的基本群胚和拓扑上连通空间的基本群胚是一致的概念.
	\item 设$\mathscr{C}$是Galois范畴,记基本群胚为$\Gamma$.我们把对象$X\in\mathscr{C}$理解为基本群胚上的一个函子$E_X:\Gamma\to\textbf{FSets}$,$E_X(F)=F(X)$.函子$E_X$满足对任意纤维函子$F$有拓扑群$\mathrm{Aut}(F)$在有限集合$E_X(F)=F(X)$上是连续作用的.
	\begin{enumerate}
		\item 用$\mathscr{D}$表示全体$\Gamma\to\textbf{FSets}$的函子$E$构成的范畴,使得对任意纤维函子$F$都有拓扑群$\mathrm{Aut}(F)$在$E(X)$上的作用是连续的.那么我们断言$X\mapsto E_X$是$\mathscr{C}\to\mathscr{D}$的范畴等价.
		\begin{proof}
			
			任取一个纤维函子$F_0$,记它的基本群是$\pi_0$,我们定义函子$\mathscr{D}\to\mathscr{C}(\pi_0)$为把函子$E\in\mathscr{D}$映射为$E(F_0)$,按照$E$满足的额外条件保证了$E(F_0)$的确是一个$\pi_0$连续作用集合.这个函子是完全忠实的可以直接验证,它是本质满的也可以直接构造:任取一个$\pi_0$连续作用的有限集合$X$,我们构造$E\in\mathscr{D}$如下:取$E(F_0)=X$.存在一个对象$X'\in\mathscr{C}$使得$X'$在$F_0$所分解的范畴等价下映为$X$.对于纤维函子$F$就定义$E(F)=F(X')$即可.
		\end{proof}
	    \item 记$\widetilde{\mathscr{D}}$是全体函子$E:\Gamma\to\mathrm{Top}$构成的范畴,使得对任意纤维函子$F$都有$E(F)$是一个全不连通的紧空间,使得$\mathrm{Aut}(F)$在其上的作用是连续的.那么$\textbf{Pro}(\mathscr{C})$范畴等价于$\widetilde{\mathscr{D}}$的.
	\end{enumerate}
    \item 如果我们选取$\Gamma$的完全子范畴$\Gamma'$,记典范包含函子为$i:\Gamma'\to\Gamma$.记$\mathscr{D}'$是全体函子$E:\Gamma'\to\textbf{FSets}$构成的范畴,满足对任意$F\in\Gamma'$有$E(F)$是$\mathrm{Aut}(F)$连续作用有限集合.记$\widetilde{\mathscr{D}}'$是全体函子$E:\Gamma'\to\textbf{Top}$构成的范畴,满足对任意$F\in\Gamma'$有$E(F)$是全不连通的紧空间,并且$\mathrm{Aut}(F)$在其上连续作用.那么$X\mapsto E_X\circ i$提供了范畴等价$\mathscr{C}\cong\mathscr{D}'$和$\textbf{Pro}(\mathscr{C})\cong\widetilde{\mathscr{D}}'$.如果$\Gamma'=\Gamma$,这个结论就是上一条;如果$\Gamma'$由单个纤维函子构成,这个结论就是Galois范畴主要定理.
    \item 基本pro-群.考虑函子$f:\Gamma\to\textbf{Top}$为把纤维函子$F$映为$\mathrm{Aut}(F)=\pi_F$.这里拓扑空间$\pi_F$作为$\pi_F$连续作用空间的群作用取为内自同构.那么按照范畴等价$\textbf{Pro}(\mathscr{C})\cong\widetilde{\mathscr{D}}$,就有$f$对应了$\mathscr{C}$上的一个pro-对象.它甚至是一个pro-群【?】,称为$\mathscr{C}$的基本pro-群,记作$\Pi$.那么它恰好是唯一满足$F(\Pi)\cong\pi_F,\forall F\in\Gamma$的pro-群【?】.【见SGA1的V.5.10和11】
\end{enumerate}
\subsection{Galois范畴之间的基本函子}

设$\mathscr{C}$和$\mathscr{C}'$是两个Galois范畴,一个共变函子$H:\mathscr{C}\to\mathscr{C}'$称为基本函子(fundamental),如果存在$\mathscr{C}'$的纤维函子$F'$,使得$F'\circ H$是$\mathscr{C}$的纤维函子.
\begin{enumerate}
	\item 按照Galois范畴主要定理,同一个Galois范畴的不同纤维函子是同构的,于是$H:\mathscr{C}\to\mathscr{C}'$是基本函子也等价于讲对$\mathscr{C}'$的任意纤维函子$F'$,都有$F'\circ H$是$\mathscr{C}$的纤维函子.
	\item 设$H:\mathscr{C}\to\mathscr{C}'$是函子,如下命题互相等价.这个等价描述(3)导致基本函子也叫做正合函子(exact).
	\begin{enumerate}[(1)]
		\item $H$是正合函子.
		\item $H$是左正合函子,和有限余积可交换,把满态射映为满态射(特别的,这一条能推出如果$X\in\mathscr{C}$不是初对象,则$H(X)\in\mathscr{C}'$不是初对象).
		\item $H$是基本函子.换句话讲对任意$\mathscr{C}'$的纤维函子$F'$,都有$F=F'\circ H$是$\mathscr{C}$的纤维函子.
	\end{enumerate}
	\begin{proof}
		
		首先(1)推(2)任何范畴都成立.我们来解释(2)括号中的内容.首先一个对象$X\in\mathscr{C}$不是初对象当且仅当唯一的态射$u:X\to*_{\mathscr{C}}$是满态射(其中$*_{\mathscr{C}}$是终对象),因为一方面如果$X$是初对象,如果$X\to*_{\mathscr{C}}$是满态射,迫使$*_{\mathscr{C}}$还是终对象,但是初对象和终对象在纤维函子下的像一个是空集一个是单点集,这矛盾.必要性我们可以放在$\mathscr{C}=\mathscr{C}(\Pi)$上看,按照$X$不是初对象得到它是非空集合,那么从$v_1\circ u=v_2\circ u$得到$v_1,v_2$把$*_{\mathscr{C}}$唯一的点打到相同的点,这导致$v_1=v_2$.于是按照这个描述结合$H$把满态射映射为满态射就得到括号中的断言成立.(2)推(3):我们解释过由于这里$F=F'\circ H$和有限余积可交换,并且把非初对象映射为非初对象,就导致它是纤维函子.最后(3)推(1)是因为$F$是正合的并且$F'$是反映同构的.更具体地讲,如果$\varprojlim Q_i$是$\mathscr{C}$上的极限(余极限同理),于是有$F'(H(\varprojlim Q_i))=F(\varprojlim Q_i)\cong\varprojlim F(Q_i)=F'(\varprojlim H(Q_i))$,按照$F'$反映同构就得到典范同构$H(\varprojlim Q_i)\cong\varprojlim H(Q_i)$.
	\end{proof}
	\item 关于2-范畴.一个2-范畴$\mathscr{A}$是指如下信息:
	\begin{enumerate}[(1)]
		\item 一个类$|\mathscr{A}|$,其中的元素称为对象.
		\item 对任意对象$A,B$,取定一个小范畴$\mathscr{A}(A,B)$,其中的元素称为$A\to B$的1-态射.两个1-态射作为这个小范畴的对象之间的态射称为$\mathscr{A}$的2-态射.
		\item 对任意对象$A,B,C$,取定一个如下双函子,称为水平复合法则.
		$$c_{ABC}:\mathscr{A}(A,B)\times\mathscr{A}(B,C)\to\mathscr{A}(A,B)$$
		\item 记$\textbf{1}$表示小范畴构成的范畴中的终对象,换句话讲它由单个对象$\star$构成,并且$\mathrm{Hom}_{\textbf{1}}(\star,\star)$由$1_{\star}$构成.对任意对象$A$,取定一个如下函子,称为水平复合的幺元.
		$$u_A:\textbf{1}\to\mathscr{A}(A,A)$$
	\end{enumerate}

    这些信息满足如下两个公理:
    \begin{enumerate}[(1)]
    	\item 结合律.给定对象$A,B,C,D$,我们有如下交换图表:
    	$$\xymatrix{\mathscr{A}(A,B)\times\mathscr{A}(B,C)\times\mathscr{A}(C,D)\ar[rr]^{1\times c_{BCD}}\ar[d]_{c_{ABC}\times1}&&\mathscr{A}(A,B)\times\mathscr{A}(B,D)\ar[d]^{c_{ABD}}\\\mathscr{A}(A,C)\times\mathscr{A}(C,D)\ar[rr]_{c_{ACD}}&&\mathscr{A}(A,D)}$$
    	\item 幺元.给定对象$A,B$,我们有如下交换图表.
    	$$\xymatrix{\textbf{1}\times\mathscr{A}(A,B)\ar[d]_{u_A\times1}&\mathscr{A}(A,B)\ar[l]_{\cong}\ar[r]^{\cong}\ar@{=}[d]&\mathscr{A}(A,B)\times\textbf{1}\ar[d]^{1\times u_B}\\\mathscr{A}(A,A)\times\mathscr{A}(A,B)\ar[r]_{c_{AAB}}&\mathscr{A}(A,B)&\mathscr{A}(A,B)\times\mathscr{A}(B,B)\ar[l]^{c_{ABB}}}$$
    \end{enumerate}

    特别的,这说明$|\mathscr{A}|$和$\mathrm{Hom}(A,B)=|\mathscr{A}(A,B)|$构成一个1-范畴(常义范畴).
    \item 记如下两个2-范畴:
    \begin{enumerate}[(1)]
    	\item 2-范畴$\textbf{Gal}$:
    	\begin{itemize}
    		\item 对象:全体$(\mathscr{C};F)$,其中$\mathscr{C}$是Galois范畴,$F$是它的纤维函子.
    		\item 1-态射:Galois范畴之间的基本函子$H:(\mathscr{C};F)\to(\mathscr{C}';F')$,满足$H\circ F'=F$.
    		\item 2-态射:基本函子之间的自然同构.
    	\end{itemize}
        \item 2-范畴$\textbf{ProFinGrp}$:
        \begin{itemize}
        	\item 对象:全体射影有限群.
        	\item 1-态射:射影有限群之间的同态.
        	\item 2-态射:同态的内自同构,更具体地讲,两个同态$u_1,u_2:\Pi\to\Pi'$之间的内自同构$\theta:u_1\to u_2$是指一个元$\theta\in\Pi'$,满足$\theta\circ u_1(\bullet)\circ\theta^{-1}=u_2(\bullet)$.
        \end{itemize}
    \end{enumerate}

    我们断言这两个2-范畴是逆变范畴等价的.构造$\textbf{Gal}\to\textbf{ProFinGrp}$为:
    \begin{itemize}
    	\item 在对象上,把$(\mathscr{C};F)$映射为基本群$\pi_1(\mathscr{C};F)$,那么Galois主要定理解释了这是本质满的,它的拟逆是把射影有限群$\Pi$映射为$(\mathscr{C}(\Pi),\textbf{For})$,其中$\textbf{For}$是到$\textbf{FSets}$的遗忘函子.
    	\item 在1-态射上,任取基本函子$H:(\mathscr{C};F)\to(\mathscr{C}';F')$,那么诱导了基本群之间的典范同态$u_H:\pi_1(\mathscr{C}';F')\to\pi_1(\mathscr{C};F)$为把$F'$的自同构$\Theta'$映射为$\Theta'\circ H$,此为$F$的自同构.这个映射的拟逆是任取射影有限群之间的同态$u:\Pi'\to\Pi$,如果$E\in\mathscr{C}(\Pi)$,那么$E$按照同态$u$自然的是一个$\Pi'$作用集合,这诱导了一个基本函子$H_u:(\mathscr{C}(\Pi),\textbf{For})\to(\mathscr{C}(\Pi'),\textbf{For})$.这两个对应满足$u_{H_u}=u$以及如下交换图表(严格说这里的$F,F'$不是真正下纤维函子,指的是纤维函子所唯一分解的范畴等价):
    	$$\xymatrix{\mathscr{C}\ar[rr]^H\ar[d]_F&&\mathscr{C}'\ar[d]^{F'}\\\mathscr{C}(\Pi_1(\mathscr{C};F))\ar[rr]_{H_{u_H}}&&\mathscr{C}(\Pi_1(\mathscr{C}';F'))}$$
    	\begin{proof}
    		
    		先解释第一个等式$u_{H_u}=u$.任取射影有限群之间的同态$u:\Pi'\to\Pi$,则$H_u$是$(\mathscr{C}(\Pi),\textbf{For})\to(\mathscr{C}(\Pi'),\textbf{For}')$的函子,它把$\Pi$作用集合$X$映射为经$u$定义的$\Pi'$作用集合$X$,为了便于区分我们把后者记住$\widetilde{X}$.我们之前解释过$\pi_1(\mathscr{C}(\Pi');\textbf{For}')=\Pi'$,在这个典范对应下,函子$\textbf{For}'$的自同构$\Theta'$对应于一个元$a\in\Pi'$.如果记$\Pi'$是有限群族$\{\Pi'_j\}$的逆向极限,那么可记$a=(a_j)\in\prod_j\Pi'_j$.那么这里$a_j$可以这样描述:典范同态$\Pi'\to\Pi'_j$使得$\Pi'_j\in\mathscr{C}(\Pi')$,于是$\Theta(\Pi'_j)$是$\textbf{For}(\Pi'_j)\to\textbf{For}(\Pi'_j)$的映射,这个映射下,$\Pi'_j$的幺元的像就是$a_j$.接下来,为了证明$u_{H_u}=u$,也即证明$\textbf{For}$的自同构$\Theta'\circ H_u$对应的元素恰好是$u(a)$.为此我们设$\Pi$是一族有限群$\{\Pi_i\}$的逆向极限,我们只要证明$\Theta'\circ H_u(\Pi_i)=\Theta'(\widetilde{\Pi_i}):\textbf{For}(\widetilde{\Pi_i})\to\textbf{For}(\widetilde{\Pi_i})$把幺元映射为$u(a)$在第$i$分量上的$(u(a))_i$即可.为此注意到$\mathscr{C}(\Pi')$的Galois对象恰好就是全体$\{\Pi'_j\}$.不妨设$\widetilde{\Pi_i}$的Galois闭包是$\Pi'_j$,那么我们有:
    		$$\Theta'(\widetilde{\Pi_i})=\left(\xymatrix{\textbf{For}(\widetilde{\Pi_i})\ar[r]^{e_{\Pi'_j}}^{-1}&\mathrm{Hom}_{\mathscr{C}(\Pi')}(\Pi'_j,\widetilde{\Pi_i})\ar[r]^{\circ\Theta(\Pi'_j)}&\mathrm{Hom}_{\mathscr{C}(\Pi')}(\Pi'_j,\widetilde{\Pi_i})\ar[r]^{e_{\Pi'_j}}&\textbf{For}(\widetilde{\Pi_i})}\right)$$
    		
    		在这个映射下,$\Pi_i$的幺元先是映射为唯一的态射$\Pi'_j\to\widetilde{\Pi_i}$使得$\zeta_{\Pi'_j}$被映射到幺元1.再复合上$\Theta(\Pi'_j)$,也即左乘$a_j$.于是它最后被映射为$u(a)_i$.
    		
    		\qquad
    		
    		最后这个交换图表是容易的.任取$X\in\mathscr{C}$,那么它视为$\pi_1(\mathscr{C};F)$作用集合的作用就是对任意$\sigma\in\pi_1(\mathscr{C};F)$,取$\sigma(X):F(X)\to F(X)$.这个群作用在$H_{u_H}$下的像是这样的:任取$\Theta'\in\pi_1(\mathscr{C}';F')$,那么$\Theta'$在$F(X)=F'(H(X))$上的作用是$\Theta'(H(X))$,这就恰好是另一个轨迹的态射.
    	\end{proof}
        \item 在2-态射上,设$H_1,H_2:(\mathscr{C};F)\to(\mathscr{C}';F')$是两个基本函子,满足$H_i\circ F'=F$,设$\alpha:H_1\to H_2$是自然同构,那么它诱导了$u_{H_1}\to u_{H_2}$的内自同构.也即存在一个$u_{\alpha}\in\pi_1(\mathscr{C};F)$(我们会证明$u_{\alpha}(X)=F'(\alpha(X))$)满足$u_{\alpha}\circ u_{H_1}(\theta')=u_{H_2}(\theta')\circ u_{\alpha},\forall\theta'\in\Pi'$.
        \begin{proof}
        	
        	因为我们有$F=F'\circ H_1=F'\circ H_2$,结合$\alpha$是自然同构,于是$u_{\alpha}(X)=F'(\alpha(X))$是$F\to F$的自然同构.那么这个等式实际上就是如下交换图表:
        	$$\xymatrix{F(X)=F'(H_1(X))\ar[rr]^{\Theta'(H_1(X))}\ar[d]_{F'(\alpha(X))}&&F'(H_1(X))=F(X)\ar[d]^{F'(\alpha(X))}\\F(X)=F'(H_2(X))\ar[rr]_{\theta'(H_2(X))}&&F'(H_2(X))=F(X)}$$
        \end{proof}
    \end{itemize}
    \item 引理.取$\mathscr{C}=\mathscr{C}(\Pi)$和$\mathscr{C}'=\mathscr{C}(\Pi')$,取$u:\Pi'\to\Pi$是射影有限群之间的同态,那么$H=H_u$是$\mathscr{C}\to\mathscr{C}'$的基本函子.
    \begin{enumerate}[(1)]
    	\item 对任意开子群$U\subseteq\Pi$.
    	\begin{itemize}
    		\item $\mathrm{im}(u)\subseteq U$当且仅当在$(\mathscr{C}')^{\mathrm{pt}}$中有$(*_{\mathscr{C}'},*)\ge(H(\Pi/U),1)$(这件事也称作$(H(\Pi/U),1)$在$(\mathscr{C}')^{\mathrm{pt}}$中具有截面(has a section),一个范畴$\mathscr{C}$如果具有终对象$*_{\mathscr{C}}$,那么如果$*_{\mathscr{C}}$支配了一个对象$X$,就称$X$具有截面).
    		\begin{proof}
    			
    			这里$\mathscr{C}$和$\mathscr{C}'$的终对象就是单点集(上赋予唯一的平凡群作用),所以这个支配条件就是在讲映射$\varphi:\{*\}\to H(\Pi/U)$,把唯一点映射为$\Pi/U$的点$U$,是$\Pi'$同态.于是对任意$\theta'\in\Pi'$就有如下等式,这等价于$\mathrm{im}(u)\subseteq U$.
    			$$U=\varphi(*)=\varphi(\theta'\cdot*)=\theta'\varphi(*)=u(\theta')U$$
    		\end{proof}
    		\item 记$\Pi$的最小的包含$\mathrm{im}(u)$的正规子群为$\mathrm{K}_{\Pi}(\mathrm{im}(u))$.那么$\mathrm{K}_{\Pi}(\mathrm{im}(u))\subseteq U$当且仅当$H(\Pi/U)$在$\mathscr{C}'$中完全分裂(称一个对象$X$是完全分裂的(totally split),如果它同构于有限个终对象的余积,在这里的$\mathscr{C}(\Pi)$中一个对象完全分裂就是指$\Pi$在其上的作用是平凡的).
    		\begin{proof}
    			
    			我们知道包含$\mathrm{im}(u)$的最小的正规子群就是全体$g\mathrm{im}(u)g^{-1}$生成的子群.于是这个正规子群包含在$U$中当且仅当所有$g\mathrm{im}(u)g^{-1}$包含在$U$中.于是按照上一步,等价于讲$\varphi:\{*\}\to H(\Pi/gUg^{-1})$是$\Pi'$同态.这件事等价于讲对任意$g\in\Pi$都有$gU=u(\theta')gU$,等价于讲$\{*\}\to H(\Pi/U)$为$*\mapsto gU$是$\Pi'$同态.于是当$g$取遍$\Pi/U$的陪集代表元时我们有满射$\coprod_{g\in\Pi/U}\{*\}\to H(\Pi/U)$也是$\Pi'$同态.但是这两个集合的元素个数相同,于是这实际上是一个$\Pi'$同构,于是$H(\Pi/U)$是完全分裂的.
    			
    			\qquad
    			
    			反过来先设$H(\Pi/U)$是完全分裂的,那么我们有$\Pi'$同构$\coprod_{i\in I}\varphi_i:\coprod_i\{*\}\to H(\Pi/U)$.于是这里每个$\varphi:\{*\}\to H(\Pi/U)$,$*\mapsto gU$都是$\Pi'$同态,于是得到$gU=u(\theta')gU$,也即$g^{-1}\mathrm{im}(u)g\subseteq U$,进而$\mathrm{K}_{\Pi}(\mathrm{im}(u))\subseteq U$.
    		\end{proof}	
    		\item 特别的,$u:\Pi'\to\Pi$是平凡映射当且仅当对任意$X\in\mathscr{C}$都有$H(X)$是$\mathscr{C}'$的完全分裂对象.
    		\begin{proof}
    			
    			我们知道对于射影有限群,全部开子群的交是$\{1\}$.于是$u$是平凡映射当且仅当$\mathrm{im}(u)=\{1\}$,当且仅当$\mathrm{im}(u)$包含在任意开子群$U$中,当且仅当对任意开子群$U$有$H(\Pi/U)$是完全分裂对象.但是形如$\Pi/U$的对象恰好是全部连通对象.
    		\end{proof}
    	\end{itemize}
        \item 对任意开子群$U'\subseteq\Pi'$.
        \begin{itemize}
        	\item $\ker u\subseteq U'$当且仅当存在一个开子群$U\subseteq\Pi$满足在$(\mathscr{C}')^{\mathrm{pt}}$中有$(H(\Pi/U),1)_0\ge(\Pi'/U',1)$.这里对于一个一般的Galois范畴$(\mathscr{C};F)$,任取$(X,\xi)\in\mathscr{C}^{\mathrm{pt}}$,记号$(X,\xi)_0$表示$(X_0,\xi)$,其中$X_0$是$X$的满足$\xi\in F(X_0)$的连通分支.
        	\begin{proof}
        		
        		按照射影有限群的开子群恰好是有限指数闭子群,以及射影有限群之间的同态一定是闭映射,并且$u(U')$一定是$\mathrm{im}(u)$的有限指数子群,综上$u(U')$是$\mathrm{im}(u)$的开子群.于是存在开子群$U\subseteq\Pi$,满足$U\cap\mathrm{im}(u)\subseteq u(U')$.另外我们知道$1\in H(\Pi/U)$在$\mathscr{C}'$中的连通分支就是$\mathrm{im}(u)U/U\cong\mathrm{im}(u)/(U\cap\mathrm{im}(u))\cong\Pi'/u^{-1}(U)$.于是如果有$\Pi'$同态$\varphi:\Pi'/u^{-1}(U)\to\Pi'/U'$为把幺元映射到幺元,等价于讲对任意$\theta'\in\Pi'$都有$\varphi(\theta')=\theta'$,这等价于讲$u^{-1}(U)\subseteq U'$.于是这明显推出$\ker u\subseteq U'$.反过来如果$\ker u\subseteq U'$,那么有$u^{-1}(U)=u^{-1}(U\cap\mathrm{im}(u))\subseteq u^{-1}(u(U'))$.于是归结为证明$u^{-1}(u(U'))\subseteq U'$:倘若有$a\in\Pi'$满足存在$b\in U'$使得$u(a)=u(b)$,那么$ab^{-1}\in\ker u\subseteq U'$,于是$a=(ab^{-1})b\in U'$.
        	\end{proof}
            \item 如果$u$是满态射,那么$\ker(u)\subseteq U'$当且仅当存在开子群$U\subseteq\Pi$和一个同构$(H(\Pi/U),1)_0\cong(\Pi'/U',1)$.
            \begin{proof}
            	
            	延续上一步的证明,如果$u$是满射,此时$u(U')$在$\Pi$里是有限指数的,所以它本身就是$\Pi$的开子群,于是干脆就取$U=u(U')$.于是上一步的等价命题成立时这里的$\Pi'$同态$\Pi'/u^{-1}(U)\to\Pi'/U'$已经是同构了.
            \end{proof}
            \item 特别的,同态$u:\Pi'\to\Pi$是单态射当且仅当对任意连通对象$X'\in\mathscr{C}'$,都存在连通对象$X\in\mathscr{C}$和$H(X)$在$\mathscr{C}'$中的连通分支$H(X)_0$,使得在$\mathscr{C}'$中有$H(X)_0\ge X'$.
            \begin{proof}
            	
            	这件事依旧是因为在射影有限群中全体开子群的交是$\{1\}$.于是$u$是单态射等价于$\ker(u)=\{1\}$,等价于对$\Pi'$的任意开子群$U'$都有$\ker(u)\subseteq U'$,等价于讲对$\Pi'$的任意开子群$U'$,都存在开子群$U\subseteq\Pi$使得$(H(\Pi/U),1)_0\ge(\Pi'/U',1)$.并且我们解释过形如$\Pi/U$的对象恰好是全部连通对象.
            \end{proof}
            \item 特别的,同态$u:\Pi'\to\Pi$如果是满态射,那么它是同构当且仅当$H:\mathscr{C}\to\mathscr{C}'$是本质满的.
            \begin{proof}
            	
            	如果$u$是满态射,那么$\ker(u)=\{1\}$当且仅当对任意开子群$U'\subseteq\Pi'$,都存在开子群$U\subseteq\Pi$和一个同构$(H(\Pi/U),1)_0\cong(\Pi'/U',1)$.但是在连通对象上本质满等价于在所有对象上本质满,因为任意对象可以取连通分支分解.
            \end{proof}
        \end{itemize}
    \end{enumerate}
    \item 取$\mathscr{C}=\mathscr{C}(\Pi)$和$\mathscr{C}'=\mathscr{C}(\Pi')$,取$u:\Pi'\to\Pi$是射影有限群之间的同态,那么$H=H_u$是$\mathscr{C}\to\mathscr{C}'$的基本函子.
    \begin{enumerate}[(1)]
    	\item 如下命题互相等价:
    	\begin{enumerate}[i.]
    		\item $u:\Pi'\to\Pi$是满同态.
    		\item $H:\mathscr{C}\to\mathscr{C}'$把连通对象映射为连通对象.
    		\item $H:\mathscr{C}\to\mathscr{C}'$是完全忠实的.
    	\end{enumerate}
        \item 如下命题互相等价:
        \begin{enumerate}[i.]
        	\item $u:\Pi'\to\Pi$是单同态.
        	\item 对任意对象$X'\in\mathscr{C}'$,存在对象$X\in\mathscr{C}$和$H(X)$的连通分支$H(X)_0$满足$H(X)_0\ge X'$.
        \end{enumerate}
        \item 如下命题互相等价:
        \begin{enumerate}[i.]
        	\item $u:\Pi'\to\Pi$是同构.
        	\item $H:\mathscr{C}\to\mathscr{C}'$是范畴等价.
        \end{enumerate}
        \item 设$\xymatrix{\mathscr{C}\ar[r]^{H}&\mathscr{C}'\ar[r]^{H'}&\mathscr{C}''}$是Galois范畴之间的基本函子,对应的基本群同态记作$\xymatrix{\Pi&\Pi'\ar[l]_u&\Pi''\ar[l]_{u'}}$.那么:
        \begin{enumerate}[(a)]
        	\item $\ker(u)\supseteq\mathrm{im}(u')$当且仅当对任意$X\in\mathscr{C}$都有$H'(H(X))$是完全分裂对象.
        	\item $\ker(u)\subseteq\mathrm{im}(u')$当且仅当对任意连通对象$X'\in\mathscr{C}'$使得$H'(X')$在$\mathscr{C}''$中具有截面,都存在对象$X\in\mathscr{C}$和$H(X)$的一个连通分支$H(X)_0$满足$H(X)_0\ge X'$.
        \end{enumerate}
    \end{enumerate}
    \begin{proof}
    	
    	(2)在上一条引理中已经得证了,(3)是上一条引理结合(1).而(4)的(a)就是在描述$u\circ u'$是平凡同态,这在引理中已经解决了,(4)的(b)中,$\ker(u)\subseteq\mathrm{im}(u')$等价于讲对任意包含$\mathrm{im}(u')$的开子群$U'\subseteq\Pi'$,都有$\ker\subseteq U'$(因为在射影有限群中,一个闭子群就是包含它的全部开子群的交),借助上一条引理翻译一下这两句话就得到(b).综上我们只需证明(1).
    	
    	\qquad
    	
    	(i)推(ii):设$u:\Pi'\to\Pi$是满射,任取连通对象$X\in\mathscr{C}(\Pi)$,此即$\Pi$在$X$上的作用是可迁的,于是任取$x_1,x_2\in X$,就有$g\in\Pi$使得$gx_1=x_2$,按照$u$是满射有$g'\in\Pi'$使得$g=u(g')$,那么把$X$按照$u$视为$\Pi'$作用集合时就有$g'x_1=u(g')x_1=gx_1=x_2$,于是$H(X)$也是连通对象.
    	
    	\qquad
    	
    	(ii)推(i):我们知道如果取遍$\Pi$的开正规子群$U$,那么$\Pi/U$的逆向极限就是$\Pi$本身.现在$\Pi/U$作为$\Pi$作用集合当然是连通的,于是按照$H$把连通对象映射为连通对象,就有$\Pi'\to\Pi\to\Pi/U$总是满射,这就迫使诱导的$\Pi'\to\Pi$是满射.
    	
    	\qquad
    	
    	(i)推(iii):任取$\Pi$作用集合$X_1,X_2$,我们要证明的是$\mathrm{Hom}_{\mathscr{C}(\Pi)}(X_1,X_2)\to\mathrm{Hom}_{\mathscr{C}(\Pi')}(H(X_1),H(X_2))$是双射.这个映射在集合层面是不变的,所以它是单射.倘若有映射$f:X_1\to X_2$,满足对任意$g'\in\Pi'$和$x_1\in X_1$都有$f(g'x_1)=g'f(x_1)$,按照$u$是满射明显$f$也是$X_1\to X_2$的作为$\Pi$作用集合的同态.
    	
    	\qquad
    	
    	(iii)推(i):如果取开子集$U\subseteq\Pi$,满足$U\not=\Pi$,那么在$\mathscr{C}$中不存在$\{*\}\to\Pi/U$的$\Pi$同态.于是按照$H$是完全忠实的,在$\mathscr{C}'$中也不存在$\{*\}\to H(\Pi/U)$的$\Pi'$同态.于是按照上一条引理,这导致$\mathrm{im}(u)\not\subseteq U$.但是这件事对任意不是整个$\Pi$的开子群成立,这迫使$\mathrm{im}(u)=\Pi$.
    \end{proof}
    \item 设$(\mathscr{C};F)$是Galois范畴,取一个连通对象$S$,全体$S$对象和$S$态射构成的范畴记作$\mathscr{C}'$,那么这也是一个Galois范畴.函子$H:\mathscr{C}\to\mathscr{C}'$,$X\mapsto X\prod S$是一个正合函子.取定一个元$a\in F(S)$,取函子$F':\mathscr{C}'\to\textbf{FSets}$把$S$态射$X'\to S$映射为$a\in F(S)$在$F(X')\to F(S)$下的原像.那么有$F\cong F'\circ H$.于是$F'$是$\mathscr{C}'$的纤维函子.于是正合函子$H$对应了射影有限群之间的同态$u:\pi_{F'}\to\pi_F$,它把$F'$的任意自同构$\theta'$映为$F$的自同构$\theta'\circ H$.这是一个单同态,并且它的像集是$\pi_F$的固定$a\in F(S)$的元构成的子集,并且这是一个开子群.
    \begin{proof}
    	
    	先证明$u$的像固定$a\in F(S)$.设$\theta'$是$F'$的自同构,那么$\theta=\theta'\circ H$在$F(S)=F'(S\times\underline{S})$(这里$S\times S=S\times\underline{S}$表示的是它作为$S$对象是第二个分量的投影),按照定义它是第二个分量的投影映射$F(S\times\underline{S})=F(S)\times F(S)\to F(S)$在点$a\in F(S)$的纤维,也即$F(S)\times\{a\}$.考虑如下交换图表,其中$\Delta:S\to S\times\underline{S}$是对角态射.按照定义$F'(S)$是恒等态射$F(S)\to F(S)$在$a$的纤维,于是$F'(S)=\{a\}$,并且$F'(S)\to F'(S\times\underline{S})$是$F(S)\to F(S\times S)=F(S)\times F(S)$,$x\mapsto(x,x)$限制在$a\in F(S)$的纤维上的映射,于是图表交换迫使$F'(S\times\underline{S})=F(S)\times\{a\}\to F(S)\times\{a\}=F'(S\times\underline{S})$把$(a,a)$映为$(a,a)$.结合$\theta=\theta'\circ H$就得到$\theta(S):F(S)\to F(S)$固定了$a$.
    	$$\xymatrix{F'(S)\ar[rr]^{\theta'(S)}\ar[d]_{F'(\Delta)}&&F'(S)\ar[d]^{F'(\Delta)}\\F'(S\times\underline{S})\ar[rr]^{\theta'(S\times\underline{S})}&&F'(S\times\underline{S})}$$
    	
    	再证明如果$F$的自同构$\theta$在$F(S)$上固定了$a$,那么它一定可以表示为$\theta'\circ H$,其中$\theta'$是$F'$的一个自同构.任取$S$的对象$X$,那么$F'(X)$按照定义是$F(X)=F(X)\to F(S)$在$a\in F(S)$下的纤维.考虑如下交换图表,由于$\theta(S)$固定了$a$,于是$\theta(X)$可以限制为关于$a$的纤维之间的同构,把它记作$\theta'(X)$,这定义了一个$F'$的自同构,并且满足$\theta=\theta'\circ H$.
    	$$\xymatrix{F(X)\ar[rr]^{\theta(X)}\ar[d]_{F(X\to S)}&&F(X)\ar[d]^{F(X\to S)}\\F(S)\ar[rr]^{\theta(S)}&&F(S)}$$
    	
    	
    \end{proof}
\end{enumerate}









\subsection{群作用和商概形}

一般的,设$\mathscr{C}$是范畴,取对象$X$,一个群$G$在$X$上的右作用指的是一个群同态$G^{\mathrm{op}}\to\mathrm{Aut}(X)$.任取对象$Z$,一个态射$f:X\to Z$称为被$G$固定的,如果对任意$g\in G$都有$f\circ g=f$.把$X\to Z$的全体被$G$固定的态射构成的集合记作$\mathrm{Hom}_{\mathscr{C}}(X,Z)^G$.考虑函子$\mathscr{C}\to\textbf{Sets}$为$Z\mapsto\mathrm{Hom}_{\mathscr{C}}(X,Z)^G$,如果它是可表的,也即存在概形$X/G$和一个被$G$固定的态射$p:X\to X/G$,满足$f\mapsto f\circ p$是双射$\mathrm{Hom}_{\mathscr{C}}(X/G,Z)\cong\mathrm{Hom}_{\mathscr{C}}(X,Z)^G$.此时我们称$(X/G,p)$是$X$关于$G$的商对象.
\begin{enumerate}
	\item\label{群作用和商概形1} 如果商对象存在,那么它一定在同构意义下唯一.另外商对象换一种表述就是如下泛性质:对任意对象$Z$和任意被$G$固定的态射$f:X\to Z$,存在唯一的态射$\widetilde{f}:X/G\to Z$使得如下图表交换:
	$$\xymatrix{X/G\ar[rr]^{\exists_!\widetilde{f}}&&Z\\X\ar[u]^p\ar@/_1pc/[urr]_f&&}$$
	\item\label{群作用和商概形2} 我们先来证明环空间范畴中任意对象关于任意群的商对象总存在.并且一个局部环空间在局部环空间范畴中的商对象和它在环空间范畴中的商对象是一致的.
	\begin{enumerate}[(1)]
		\item 取$(X,\mathscr{O}_X)$是一个环空间,把$X$遗忘为集合,那么$G$也在集合$X$上有右作用,把这个作用的等价类集合记作$Y=X/G$.那么我们有典范的集合之间的满射$p:X\to X/G$.约定$X/G$上的拓扑是使得$p$是商映射的拓扑.换句话讲子集$V\subseteq X/G$是开集当且仅当$p^{-1}(V)$是$X$的开集.于是从$p$是满射得到$p$是开映射.
		\item 因为$p$在$G$下不变,于是$p^{-1}(V)$在$G$下不变.所以只要$V\subseteq Y$是开集,就有$G$诱导了环空间$p^{-1}(V)$上的右作用.特别的,$G$诱导了环$\mathscr{O}_X(p^{-1}(V))$上的左作用.我们就定义结构层为$\mathscr{O}_Y(V)=\mathscr{O}_X(p^{-1}(V))^G$.这个结构层可以记作$\mathscr{O}_Y=f_*(\mathscr{O}_X)^G$.
		\begin{proof}
			
			我们验证这的确是一个层.首先由于$g\in G$是环空间之间的态射,按照定义$g^{\#}$要和限制映射可交换,于是这说明$(X,\mathscr{O}_X)$上的限制$\mathrm{res}_{p^{-1}(V),p^{-1}(W)}:\mathscr{O}_X(p^{-1}(V))\to\mathscr{O}_X(p^{-1}(W))$是$G$同态,于是它可以限制为$\mathscr{O}_X(p^{-1}(V))^G\to\mathscr{O}_X(p^{-1}(W))^G$.这说明$\mathscr{O}_Y$是一个预层.接下来我们知道$A^G$是$A$的子环,所以从$\mathscr{O}_X$上具有粘合唯一性就得到$\mathscr{O}_Y$上也具有粘合唯一性.最后任取开集$V\subseteq Y$和开覆盖$V=\cup_iV_i$,如果选取$s_i\in\mathscr{O}_X(p^{-1}(V_i))$满足总有$s_i$和$s_j$在$p^{-1}(V_i\cap V_j)$上的限制相同,那么这些$\{s_i\}$至少可以粘合为$\mathscr{O}_X(p^{-1}(V))$中的元$a$.接下来任取$g\in G$,记$\mathrm{res}_i=\mathrm{res}_{p^{-1}(V),p^{-1}(V_i)}$,那么有$\mathrm{res}_i(g^{\#}(a))=g^{\#}(\mathrm{res}_i(a))=a_i$,于是按照粘合唯一性得到$g^{\#}(a)=a$,于是$a\in\mathscr{O}_X(p^{-1}(V))^G$.于是$\mathscr{O}_X$是层.
		\end{proof}
	    \item 如果$(X,\mathscr{O}_X)$是局部环空间,我们断言$(Y,\mathscr{O}_Y)$也是局部环空间.
	    \begin{proof}
	    	
	    	按照定义,我们有$\mathscr{O}_{X,x}=\varinjlim\limits_{x\in U}\mathscr{O}_X(U)$,其中$U$跑遍$x$的开邻域;有$\mathscr{O}_{Y,y}=\varinjlim\limits_{x\in p^{-1}(V)}\mathscr{O}_X(p^{-1}(V))$,其中$V$跑遍$Y$的开子集.于是第二个正向系统是第一个正向系统的一部分,于是我们有典范单射$\mathscr{O}_{Y,y}\to\mathscr{O}_{X,x}$.接下来$\mathscr{O}_{X,x}$的一个元$\{s_U\in\mathscr{O}_X(U)\}$是单位,当且仅当在指标足够大时(此即在开集$U$足够小)有$s_U$是$\mathscr{O}_X(U)$的单位.按照$\mathscr{O}_{X,x}$是局部环,所以它的单位集的补集就是它的唯一极大理想$\mathfrak{m}_x$,于是$\mathfrak{m}_x$由那些$s_U$,满足对任意$W\subseteq U$有$s_W$总不是$\mathscr{O}_X(W)$单位的元构成.于是$\mathscr{O}_{Y,y}$作为$\mathscr{O}_{X,x}$的子环,就以$\mathfrak{m}_y=\mathfrak{m}_x\cap\mathscr{O}_{Y,y}$为极大理想.于是$\mathfrak{m}_y$中的元也是由形如$s_{p^{-1}(V)}$的元构成,满足对任意$W\subseteq V$有$s_{p^{-1}(W)}$总不是$\mathscr{O}_X(p^{-1}(W))$的单位.那么$\mathfrak{m}_y$在$\mathscr{O}_{Y,y}$中的补集由形如$s_{p^{-1}(V)}\in\mathscr{O}_X(p^{-1}(V))^*$构成,于是这就是$\mathscr{O}_{Y,y}$的单位集,于是$\mathscr{O}_{Y,y}$以$\mathfrak{m}_y$为极大理想.
	    \end{proof}
        \item 构造$p^{\#}:\mathscr{O}_Y\to p_*\mathscr{O}_X$就是典范的包含映射$\mathscr{O}_X(p^{-1}(V))^G\to\mathscr{O}_X(p^{-1}(V))$.于是$(p,p^{\#}):(X,\mathscr{O}_X)\to(Y,\mathscr{O}_Y)$是环空间之间的态射.我们断言这就是$(X,\mathscr{O}_X)$关于$G$的商对象.
        \begin{proof}
        	
        	在这个证明里我们把态射都完整的写成连续映射部分和层态射部分:$f:(X,\mathscr{O}_X)\to(Z,\mathscr{O}_Z)$完整的记作$(f,f^{\#})$,$p$完整的记作$(p,p^{\#})$.先解释使得图表交换的连续映射$\widetilde{f}$是唯一的.这是因为$f$在$G$下不变,于是使得图表交换的集合映射$\widetilde{f}$只能是$\widetilde{f}([x])=f(x)$,其中$x\in X$,而$[x]=p(x)$.那么这个定义不依赖于$[x]$的代表元$x$的选取.它是连续映射是因为,对任意开集$W\subseteq Z$,有$p^{-1}(\widetilde{f}^{-1}(W))=f^{-1}(W)$是$X$的开集,于是按照$p$是商映射就得到$\widetilde{f}^{-1}(W)$是$Y$的开集.
        	
        	\qquad
        	
        	下面处理层态射的部分,任取开集$W\subseteq Z$,按照$\mathscr{O}_Y=f_*\mathscr{O}_X^G$,我们就是要找唯一的同态$\widetilde{f}^{\#}$使得有如下图表交换.但是这件事归结为证明这里$f^{\#}$的像集本身是关于$G$不变的.而这是因为对任意$g\in G$有$g^{\#}\circ f^{\#}=f^{\#}$.
        	$$\xymatrix{\mathscr{O}_Z(W)\ar[rr]^{\widetilde{f}^{\#}}\ar@/_1pc/[drr]_{f^{\#}}&&\mathscr{O}_X(f^{-1}(W))^G\ar[d]\\&&\mathscr{O}_X(f^{-1}(W))}$$
        \end{proof}
        \item 但是在概形范畴里商对象不是总存在的,而且即便存在也可能和局部环空间范畴中的商对象不一致.
	\end{enumerate}
    \item\label{群作用和商概形3} 仿射概形的情况.设$A$是环,设有限群$G$在$A$上左作用,记$B=A^G=\{a\in A\mid \sigma(a)=a,\forall\sigma\in G\}$.再记$X=\mathrm{Spec}A$和$Y=\mathrm{Spec}B$.记典范包含映射$A^G\subseteq A$诱导的态射为$p:X\to Y$.
    \begin{enumerate}[(1)]
    	\item $p$是一个整态射,也即$A^G\subseteq A$是环的整扩张(这里的确用到了$G$是有限群).
    	\item $p$是一个满射,它的纤维恰好就是$G$在$X$上作用的轨道,并且$Y$的拓扑恰好是$X$的商拓扑(此即$p$是商映射).
    	\item 取$x\in X$,记$y=p(x)$,记$G_x=\{\sigma\in G\mid\sigma(x)=x\}$是点$x$的稳定子.那么$\kappa(y)\subseteq\kappa(x)$总是正规扩张,并且典范群同态$G_x\to\mathrm{Gal}(\kappa(x)/\kappa(y))$是满射.
    	\item 此时有典范同构$\mathscr{O}_Y\cong p_*(\mathscr{O}_X)^G$,并且$(Y,p)$是$X$关于$G$的商概形,并且这也是环空间范畴中的商对象.
    \end{enumerate}
    \begin{proof}
    	
    	我们先说明$B=A^G\subseteq A$是整扩张,任取$a\in A$,那么$\prod_{\sigma\in G}(t-\sigma(a))$就在$G$下不变(这里用到了$G$是有限群),于是它落在$B[t]$中,于是$a$在$B$上整.但是整扩张诱导的概形态射一定是满射和闭映射,从而是商映射.接下来说明$p$的纤维恰好是$G$的轨道:设$\mathfrak{q}$是$B$的素理想,设$\mathfrak{p},\mathfrak{p}'$是它的两个提升素理想,则对任意$a\in\mathfrak{p}'$,我们有$\prod_{\sigma\in G}\sigma(a)\in\mathfrak{p}'\cap B=\mathfrak{q}\subseteq\mathfrak{p}$.于是存在某个$\sigma\in G$使得$\sigma(a)\in\mathfrak{p}$.但是这里$a\in\mathfrak{p}'$是任取的,所以有$\mathfrak{p}'\subseteq_{\sigma\in G}\sigma(\mathfrak{p})$,但是左侧是素理想,所以存在某个$\sigma\in G$使得$\mathfrak{p}'\subseteq\sigma(\mathfrak{p})$.另一方面$\sigma(\mathfrak{p})\cap B=\sigma(\mathfrak{p}\cap B)=\mathfrak{p}\cap B=\mathfrak{p}'\cap B$,但是整扩张的提升素理想满足不可比条件,于是$\sigma(\mathfrak{p})=\mathfrak{p}'$.这证明了$p$的同一个纤维中的两个点一定在相同轨道.反过来在相同轨道中的素理想差一个$G$作用,而这在$B$上是不变的,所以它们在相同纤维中.
    	
    	\qquad
    	
    	接下来证明(3):设$x$对应的$A$的素理想是$\mathfrak{p}$,那么$y$对应的$B$的素理想是$\mathfrak{q}=\mathfrak{p}\cap B$.我们不妨设$(B,\mathfrak{q})$都是局部环,否则可以用$A\otimes_BB_{\mathfrak{q}}$和$B_{\mathfrak{q}}$分别替代$A$和$B$,那么此时$\mathfrak{p}$是$A$的极大理想(因为$B\subseteq A$是整扩张).任取$a\in A$,它是$\prod_{\sigma\in G}(t-\sigma(a))\in B[t]$的根,并且这个多项式在$A$中分裂,于是$\kappa(y)\subseteq\kappa(x)$是一个正规扩张,并且每个元的扩张次数$\le|G|$,进而有$\kappa(y)$在$\kappa(x)$中的可分闭包$\kappa(y)_s$满足$[\kappa(y)_s:\kappa(y)]\le|G|$.这里$\kappa(y)\subseteq\kappa(y)_s$是有限可分扩张,我们记$a\in A$使得$\overline{a}\in\kappa(x)$是扩张的本原元.对任意$\sigma\in G-G_x$,有$\sigma(\mathfrak{p})$是不同于$\mathfrak{p}$的$A$的极大理想.按照中国剩余定理,我们可以取$a'\in A$使得$a'-a\in\mathfrak{p}$,而对任意$\sigma\in G-G_x$有$a'\in\sigma(\mathfrak{p})$.我们取$f(t)=\prod_{\sigma\in G}(t-\overline{\sigma(a')})$,那么这是$k(y)[t]$中的元,并且满足$f(\overline{a'})=0$.下面任取$\tau\in\mathrm{Gal}(\kappa(x)/\kappa(y))$,那么$\tau(\overline{a'})$仍然是$f(t)$的根,于是存在某个$\sigma\in G$使得$\tau(\overline{a'})=\overline{\sigma(a')}$.
    	\begin{itemize}
    		\item 倘若这里$\sigma\not\in G_x$,那么按照$a'$的选取,就有$\sigma(a')\in\mathfrak{p}$,进而有$\overline{\sigma(a')}=0$.于是此时$\tau(\overline{a'})=0$,这迫使$\overline{a'}=0$,进而$\overline{a}=\overline{a'}=0$,那么此时$\kappa(y)_s=\kappa(y)$并且$\mathrm{Gal}(\kappa(x)/\kappa(y))=\{e\}$,此时$G_x\to\mathrm{Gal}(\kappa(x)/\kappa(y))$是平凡成立的.
    		\item 倘若$\sigma\in G_x$,那么此时$\tau$恰好是$\sigma$在典范映射$G_x\to\mathrm{Gal}(\kappa(x)/\kappa(y))$的像,所以无论哪种情况都有这个典范映射是满射.
    	\end{itemize}
    	
    	证明典范同构$\mathscr{O}_Y\cong p_*(\mathscr{O}_X)^G$,这归结为在主开集上证明有典范同构,也即有典范同构$(S^{-1}A)^G=S^{-1}(A^G)$,也即$A\mapsto A^G$和局部化可交换,但是它和任意平坦基变换都可交换.最后按照$\mathscr{O}_Y\cong p_*(\mathscr{O}_X)^G$和$p$是商映射(这依赖于$G$是有限的),类似环空间范畴上的证明,得到$(Y,p)$是$X$关于$G$的商概形.
    \end{proof}
    \item\label{群作用和商概形4} 设$X$是概形,设有限群$G$在$X$上有右作用,我们称这个群作用是容许的(admissible),如果存在仿射态射$\pi:X\to Y$,满足在$G$下不变(此即对任意$\sigma\in G$有$\pi\circ\sigma=\pi$),并且满足$\mathscr{O}_Y=\pi_*(\mathscr{O}_X)^G$.这个情况就是仿射情况的推广,于是此时$\pi$仍然满足上一条中的全部结论.特别的,此时商概形总是存在的.
    \item\label{群作用和商概形5} 容许作用下商概形是局部性质.设有限群$G$在概形$X$上的右作用是容许的,设商概形为$Y$,那么对任意开集$U\subseteq Y$,有$G$也作用在$\pi^{-1}(U)$上,并且限制态射$\pi:\pi^{-1}(U)\to U$也是容许的,于是$\pi^{-1}(U)$关于$G$的商概形就是$U$.
    \item\label{群作用和商概形6} 设有限群$G$在$X$上是容许作用.如果$X$是$Z$概形,如果$G$在$X$上的自同构都是$Z$自同构,那么结构态射$X\to Z$是被$G$固定的态射.于是$X$关于$G$的商概形$Y$就能唯一的作为$Z$概形.我们断言:
    \begin{itemize}
    	\item 如果$X\to Z$是仿射态射或者分离态射当且仅当$Y\to Z$也是如此.
    	\item 特别的$X$是仿射概形或者分离概形当且仅当$Y$也是如此.
    	\item 如果$X\to Z$是有限型态射,那么$X\to Y$是有限态射.
    	\item 如果$X\to Z$是有限型态射,并且$Z$是局部诺特概形,那么$Y\to Z$是有限型态射.
    \end{itemize}
    \begin{proof}
    	
    	因为$X\to Y$是仿射态射,特别的它是分离态射,于是从$Y\to Z$是仿射态射或者分离态射,就得到$X\to Z$也是如此.反过来如果$X\to Z$是仿射态射,为了证明$Y\to Z$是仿射态射,这是一个局部性质,所以不妨设$Z$是仿射概形,那么此时$X$也是仿射概形,此时我们解释过商概形$Y$也是仿射概形,于是$Y\to Z$是仿射的.接下来设$X\to Z$是分离态射.我们有如下交换图表,其中$p$是整态射也是满射,于是它是满射闭映射,进而$p\times p$也是满射闭映射.于是我们有$\Delta_{Y/Z}(Y)=\Delta_{Y/Z}(p(X))=(p\times p)(\Delta_{X/Z}(X))$是闭集,于是$Y\to Z$是分离态射.
    	$$\xymatrix{X\ar[rr]^p\ar[d]_{\Delta_{X/Z}}&&Y\ar[d]^{\Delta_{Y/Z}}\\X\times_ZX\ar[rr]_{p\times p}&&Y\times_ZY}$$
    	
    	接下来设$X\to Z$是有限型态射.那么$X\to Y$是局部有限型态射,但是$X\to Y$还是整态射,于是$X\to Y$是有限态射.再设$X\to Z$是有限型态射且$Z$是局部诺特概形,我们要证明$Y\to Z$是有限型态射,这是终端局部性质,并且容许作用下商概形也是局部性质,于是不妨设$Z$是仿射的.那么此时$X$是拟紧概形,按照$p$是满射就得到$Y$也是拟紧概形.于是$Y$有有限仿射开覆盖.按照商概形是局部性质,归结为设$Y$本身是仿射的,那么此时$X$也是仿射的.于是问题转化为代数问题:如果$C$是诺特环,$A$是有限型$C$代数,那么$B=A^G$也是有限型$C$代数.而这件事又是因为,我们设$A=C[a_1,\cdots,a_n]$,取每个$a_i$在$B$中满足的首一多项式的全部系数,生成的$C$代数$B$的有限型子代数记作$B'$,那么$A$也在$B'$上整.另外按照$C$是诺特的得到$B'$也是诺特的.于是$B'$模$B$作为$B'$模$A$的子模就是有限模,于是$B$是有限型$B'$代数,进而$B$是有限型$C$代数.
    \end{proof}
    \item\label{群作用和商概形7} 设$X$是概形,设有限群$G$在$X$上有右作用,我们断言如下命题互相等价:
    \begin{enumerate}[(1)]
    	\item 这个右作用是容许的.
    	\item $X$是一族仿射开子集$\{U_i\}$的并,使得$G$中的元总可以限制为$U_i$上的自同构.
    	\item $G$的每个轨道都包含在某个仿射开子集中.
    \end{enumerate}
    \begin{proof}
    	
    	(1)推(3):记典范态射$\pi:X\to X/G$,设$\{U_i\}$是$X/G$的仿射开覆盖,按照$\pi$是仿射的,就有$\{\pi^{-1}(U_i)\}$是$X$的仿射开覆盖,按照$\pi$是在$G$下不变的,导致$G$的每个轨道都落在某个$\pi^{-1}(U_i)$中.
    	
    	\qquad
    	
    	(3)推(2):设$S\subseteq X$是$G$的一条轨道,设$U\subseteq X$是一个包含$S$的仿射开子集,那么$S\subseteq\cap_{\sigma\in G}\sigma(U)\subseteq U$.我们断言从$U$是仿射的并且$S$是有限的,能说明存在仿射开子集$V$满足$S\subseteq V\subseteq\cap_{\sigma\in G}\sigma(U)$.事实上,记$U=\mathrm{Spec}A$,记$S=\{\mathfrak{p}_1,\cdots,\mathfrak{p}_n\}$,这里$\cap_{\sigma\in G}\sigma(U)$是$U$的开子集,所以它可以表示为某个闭集$V(\mathfrak{a})$的补集,那么$\mathfrak{a}\not\subseteq$每个$\mathfrak{p}_i$,进而得到$\mathfrak{a}\not\subseteq\cup_i\mathfrak{p}_i$,任取$f\in\mathfrak{a}-\cup_i\mathfrak{p}_i$,那么$V=D(f)$就满足要求,这完成断言的证明.接下来按照$V\subseteq\cap_{\sigma\in G}\sigma(U)$,得到对任意$\sigma\in G$有$\sigma(V)\subseteq U$,并且这里$\sigma(V)\subseteq U$总是仿射的,按照$U$是分离的,得到$\cap_{\sigma\in G}\sigma(V)$是仿射开子集,并且它在$G$下不变,并且包含$S$,最后这里轨道$S$是任取的,所以这样的在$G$下不变的仿射开子集可以覆盖整个$X$.
    	
    	\qquad
    	
    	(2)推(1):设$X$关于$G$的拓扑轨道空间为$Y$,记典范连续映射$\pi:X\to Y$.那么只需说明$(Y,\pi_*(\mathscr{O}_X)^G)$是概形,并且它是$X$关于$G$的商概形.为此我们取仿射开覆盖$X=\cup_iU_i$,其中$U_i=\mathrm{Spec}A_i$满足在$G$下不变,则此时$Y$就是$\{\mathrm{Spec}A_i^G\}$的粘合,所以它是概形.
    \end{proof}
    \item\label{群作用和商概形8} 推论.设有限群$G$右作用在概形$X$上.如果$G$容许的作用在$X$上,那么$G$的任意子群都容许的作用在$X$上.
    \item\label{群作用和商概形9} 推论.设$X\to Z$是仿射态射,设$G$在$X$上的右作用要求都是$X$的$Z$自同构,那么$G$在$X$上的作用是容许的.如果$X$表示为$\mathrm{Spec}\mathscr{A}$,其中$\mathscr{A}$是$Z$上的拟凝聚代数层,那么$Y=\mathrm{Spec}\mathscr{A}^G$.
    \item\label{群作用和商概形10} 平坦基变换.设有限群$G$右作用在概形$X$上,设态射$X\to Y$是$G$不变的,对任意态射$Y'\to Y$,有$G$按基变换作用在$X'=X\times_YY'$上.
    \begin{enumerate}
    	\item 基变换.如果$G$在$X$上的作用是容许的,如果$X\to Y$恰好是关于$G$的商概形,如果$Y'\to Y$是平坦态射,那么$G$在$X'$上的作用也是容许的,并且有$Y'\cong X'/G$.
    	\item 下降.如果$Y'\to Y$是拟紧忠实平坦态射,如果$G$作用在$X'$上是容许的,并且$Y'\cong X'/G$,那么$G$作用在$X$上的容许的,并且有$Y\cong X/G$.
    \end{enumerate}
    \begin{proof}
    	
    	证明(a):我们知道容许作用的结构态射$X\to Y$是仿射的,所以不妨设$X=\mathrm{Spec}A$,其中$G$在$A$上有典范的左作用.记$B=A^G$,那么$Y=\mathrm{Spec}B$.我们要验证的容许作用以及$Y'\cong X'/G$都是局部的性质,并且平坦是源端局部的性质,所以不妨设$Y'=\mathrm{Spec}B'$也是仿射的,并且其中$B'$是平坦$B$代数.我们有如下正合列:
    	$$\xymatrix{0\ar[r]&B\ar[r]&A\ar[r]^{\varphi}&\prod_{\sigma\in G}}$$
    	
    	其中$\varphi$为$a\mapsto(a-\sigma(a))_{\sigma\in G}$.按照$B'$是平坦$B$代数,于是张量$B'$就得到正合列:
    	$$\xymatrix{0\ar[r]&B'\ar[r]&A\otimes_BB'\ar[r]&\prod_{\sigma\in G}(A\otimes_BB')}$$
    	
    	这说明$B'\cong(A\otimes_BB')^G$,换句话讲$G$在$X'$上的作用是容许的,并且$Y'=X'/G$.
    	
    	\qquad
    	
    	证明(b):同样不妨设$Y=\mathrm{Spec}B$和$Y'=\mathrm{Spec}B'$,这里$B'$是一个忠实平坦$B$代数.按照下降理论,这里$X\to Y$同样是仿射态射,于是可设$X=\mathrm{Spec}A$.按照条件我们有正合列:
    	$$\xymatrix{0\ar[r]&B'\ar[r]&A\otimes_BB'\ar[r]&\prod_{\sigma\in G}(A\otimes_BB')}$$
    	
    	按照$B'$是忠实平坦$B$代数,就得到正合列:
    	$$\xymatrix{0\ar[r]&B\ar[r]&A\ar[r]^{\varphi}&\prod_{\sigma\in G}}$$
    	
    	于是$B\cong A^G$,于是$G$在$X$上的作用是容许的,并且$Y=X/G$.
    \end{proof}
\end{enumerate}    
\subsection{分解群和惯性群}

设有限群$G$右作用在概形$X$上,设$x\in X$,把它的稳定子$\{g\in G\mid g(x)=x\}$称为点$x$的分解群(decomposition subgroup),记作$G_d(x)$.于是$g\in G_d(x)$诱导了$\mathscr{O}_{X,x}$上的自同构,进而诱导了$\kappa(x)$上的自同构.把$G_d(x)$在$\kappa(x)$上的这个左作用的核称为点$x$的惯性群(inertia subgroup),也即$G_d(x)$的作用在$\kappa(x)$平凡的元构成的子群,惯性群记作$G_i(x)$.
\begin{enumerate}
	\item\label{分解群和惯性群1} 几何点.设$X$是概形,我们通常预先取定一个代数闭域或者可分闭域$\Omega$,然后称$X(\Omega)$中的点为$X$的几何点.对点$x\in X$,把它视为一个几何点等价于讲约定一个域扩张$\kappa(x)\subseteq\Omega$,也等价于取定一个态射$\gamma\in\mathrm{Hom}(\mathrm{Spec}\Omega,X)$(当然这个前提条件是$\kappa(x)$到$\Omega$要有域扩张).关于点$x\in X$的几何点通常记作$\overline{x}$.如果$f:X\to S$是态射,如果$\overline{x}$是$X$的关于点$x$的几何点,那么$f(\overline{x})$理解为$S$上关于点$s=f(x)$的几何点$\overline{s}:\mathrm{Spec}\Omega\to X\to S$.另外此时惯性群$G_i(x)$恰好就是$G$在$\mathrm{Hom}_{\textbf{Sch}}(\mathrm{Spec}\Omega,X)$上作用的核$\{\sigma\in G\mid\sigma\circ\gamma=\gamma\}$.
	\item\label{分解群和惯性群2} 惯性群在基变换下不变.设$X$是$Z$概形,设有限群$G$右作用在$X$上,设$Z'\to Z$是任意态射,记$X'=X\times_ZZ'$,设$G$按照基变换右作用在$X'$上,任取$x'\in X'$,设它在$X$中的像是$x$,那么我们总有$G_i(x)\cong G_i(x')$.
	\begin{proof}
		
		我们选取$\Omega$足够大使得它是$\kappa(x)$和$\kappa(x')$的域扩张,选取$x\in X$对应的一个几何点为$\gamma\in\mathrm{Hom}(\mathrm{Spec}\Omega,X)$,记$x'$对应的一个几何点为$\gamma'\in\mathrm{Hom}(\mathrm{Spec}\Omega,X')$.那么按照纤维积的泛性质,就有$p\circ\gamma'=\gamma$,这里$p:X'\to X$是典范投影态射.于是我们有:
		\begin{align*}
			G_i(x)&=\{\sigma\in G\mid\sigma\circ\gamma=\gamma\}\\&=\{\sigma\in G\mid\sigma\circ \gamma'=\gamma'\}\\&=G_i(x')
		\end{align*}
	\end{proof}
    \item\label{分解群和惯性群3} 设有限群$G$右作用在概形$X$上,并且设这个作用是容许的.对任意子群$H\subseteq G$,我们解释过$X/H$也存在,并且按照商概形的泛性质,有$X\to X/H\to X/G$就是典范态射$X\to X/G$.对$x\in X$,记它在$X'=X/H$中的像是$x'$,再记$x'$在$Y=X/G$中的像是$y$.设$Y$是局部诺特概形,再设$X\to Y$是有限态射.
    \begin{enumerate}
    	\item 如果$G_d(x)\subseteq H$,那么典范局部环同态$\mathscr{O}_{Y,y}\to\mathscr{O}_{X',x'}$诱导的完备化之间的同态是同构.
    	\item 如果$G_i(x)\subseteq H$,那么典范局部环同态$\mathscr{O}_{Y,y}\to\mathscr{O}_{X',x'}$是平展的,也即$X'\to Y$在点$x'$平展.
    \end{enumerate}
    \begin{proof}
    	
    	记$Y_1=\mathrm{Spec}\widetilde{\mathscr{O}_{Y,y}}$,我们知道诺特环上理想的完备化是正合的,于是典范态射$Y_1\to Y$是平坦态射.$y$在$Y_1\to Y$的原像是$Y_1$的唯一闭点,记作$y_1$.另外有$X_1\to Y_1$是有限态射.按照\ref{群作用和商概形10},把$G$按照基变换作用在$X_1=X\times_YY_1$上,那么商概形就是$Y_1$.另外按照$\kappa(y)=\kappa(y_1)$,有$\kappa(y_1)\otimes_{\kappa(y)}\kappa(x)$是域,所以存在唯一的$x_1\in X$满足它在$X$和$Y_1$中的像分别是$x$和$y_1$.记$X_1'=X_1/H=X'\times_YY_1$,记$x_1\in X_1$在$X_1'$中的像是$x_1'$.那么按照如下交换图表,就有$x_1'$在$X_1'\to X'$下的像恰好就是$x'$.
    	$$\xymatrix{X_1\ar[r]\ar[d]&X_1'\ar[r]\ar[d]&Y_1\ar[d]\\X\ar[r]&X'\ar[r]&Y}$$
    	
    	按照$X'\to Y$是有限型态射,可以验证$\mathscr{O}_{X',x'}\to\mathscr{O}_{X_1',x_1'}$诱导了完备化之间的同构【这件事似乎是因为,如果$A,B$都是局部环,$B$是有限型$A$代数,那么$B\to B\otimes_A\widetilde{A}$诱导了完备化之间的同构】.另外明显$Y_1\to Y$诱导的局部环同态$\mathscr{O}_{Y,y}\to\widetilde{\mathscr{O}_{Y,y}}$可以延拓为完备化之间的同构.于是问题归结为证明$X_1'\to Y_1$在点$y_1$诱导的局部环同态可以延拓为完备化之间的同构.换句话讲问题归结为设$Y=\mathrm{Spec}B$是一个完备局部环的仿射概形.于是按照作用是容许的,就有$X=\mathrm{Spec}A$也是仿射的,并且$A$在$B$上有限.【】
    \end{proof}
    \item\label{分解群和惯性群4} 推论.设有限群$G$右容许作用在概形$X$上,设$Y=X/G$是局部诺特概形,设典范态射$X\to Y$是有限的.如果$G_i(x)=\{e\}$,则$X\to Y$在点$x$是平展的.于是如果$G_i(x)=\{e\}$对任意$x\in X$成立,那么$X\to Y$是平展态射.
    \item\label{分解群和惯性群5} 推论.设有限群$G$容许和忠实的右作用在概形$X$上,设$X$是连通概形,设$Y=X/G$是局部诺特概形,设典范态射$X\to Y$是有限的.那么典范态射$p:X\to Y$是平展态射当且仅当所有惯性群$G_i(x)$都是平凡的.并且在这个条件满足时,$G$典范的等同于$X$在范畴$\textbf{Sch}(Y)$中的自同构群.另外这个结论是依赖于$G$是忠实作用的和$X$是连通的,比方说取$X=Y\times E$,其中$E$是有限离散空间,那么只要$G$是$E$的可迁置换群,就有$(T\times E)/G=Y\times(E/G)=Y$,所以我们只要取$G$是严格小于$E$上对称群的可迁置换群,那么存在$X$的$Y$自同构不包含在$G$中.
    \begin{proof}
    	
    	充分性是上一条.下面证明必要性,我们有如下引理【SGA1的I.5.4】:设$X,Y$是$S$概形,其中$Y\to S$是非分歧分离态射,$X$是连通概形,设$f,g:X\to Y$是两个$S$态射,如果存在$x\in X$满足$f(x)=g(x)=y$,并且$f,g$诱导的$\kappa(y)\to\kappa(x)$是一致的,那么$f,g$是一致的.
    	
    	\qquad
    	
    	设$X\to Y$是平展态射,按照条件它还是有限态射,于是特别的它是非分歧的分离态射.任取$g\in G_i(x)$,那么$g(x)=x=1_X(x)$,并且$g$和$1_X$诱导的$\kappa(x)\to\kappa(x)$都是恒等映射,按照引理就有$g=1_X$,于是$G_i(x)$总是平凡的.这证明了第一个断言.
    	
    	\qquad
    	
    	任取一个$Y$自同构$u:X\to X$,任取$x\in X$,那么$x$和$u(x)$落在$X\to Y$的相同纤维中,我们解释过$X\to Y$的纤维恰好就是$G$作用的轨道,所以存在$g\in G$使得$g(x)=u(x)$记作$y$.另外按照\ref{群作用和商概形3}的(3),我们还可以选取$h\in G$使得$h(x)=x$,并且$h$和$g^{-1}\circ u$诱导的$\kappa(y)\to\kappa(x)$是相同的.于是再用一次我们的引理【SGA1的I.5.4】得到$g^{-1}\circ u=h$,于是$u=g\circ h\in G$.再结合这个群作用是忠实的,就得证.
    \end{proof}
\end{enumerate}
\subsection{有限平展覆盖范畴}

概形$Y$上的一个有限平展覆盖指的是一个有限平展态射$X\to Y$,有时我们也称$X$(作为$Y$概形)是$Y$的有限平展覆盖空间.设$S$是局部诺特连通概形,用$\textbf{FEt}(S)$表示$\textbf{Sch}(S)$的由有限平展覆盖构成的完全子范畴.取定$S$的一个几何点$\overline{s}:\mathrm{Spec}\Omega\to S$,其中$\Omega$是一个代数闭域.对有限平展覆盖$f:X\to S$,我们把概形$X_{\overline{s}}=X\times_{f,S,\overline{s}}\mathrm{Spec}\Omega$的底集合记作$X_{\overline{s}}^{\mathrm{set}}$,这是一个有限点集.我们有函子$F_{\overline{s}}:\textbf{FEt}(S)\to\textbf{FSets}$为把有限平展覆盖$f:X\to S$映射为$X_{\overline{s}}^{\mathrm{set}}$,把有限平展覆盖之间的态射$h:X\to Y$映射为关于$\overline{s}:\mathrm{Spec}\Omega\to S$的基变换$X_{\overline{s}}\to Y_{\overline{s}}$在集合层面上的映射.本节的目标就是证明$(\textbf{FEt}(S),F_{\overline{s}})$是一个Galois范畴.
\begin{enumerate}
	\item 仿射情况的有限平展覆盖.设$A$是域$k$上的有限维代数,那么如下条件互相等价.在条件成立时称$A$是域$k$上的有限平展代数.
	\begin{enumerate}[(1)]
		\item 有$k$代数同构$A\cong\prod_{i=1}^nL_i$,其中每个$L_i$都是$k$的有限可分扩张.
		\item 有$\overline{k}$代数同构$A\otimes_k\overline{k}\cong\prod_{1\le i\le m}\overline{k}$.
		\item $A\otimes_k\overline{k}$是既约代数.
		\item 微分模$\Omega_{A/k}=0$.
	\end{enumerate}
	\begin{proof}
		
		我们先证明如下引理:域$k$上的有限维代数$A$是既约的当且仅当它是有限个$k$的有限的域扩张的直积.一旦这件事得证,直接得到(2)和(3)的等价性.
		
		\qquad
		
		引理的证明:充分性是显然的,有限个域的直积当然没有非平凡的幂零元.对于必要性,因为$A$是诺特的,它的素谱只有有限个连通分支,我们设有连通分支分解$A=\prod_{i=1}^rA_i$,我们断言这里每个$A_i$实际上就是$k$的一个有限的域扩张.换句话讲问题归结为设$\mathrm{Spec}A$本身就是连通的,证明任意$a\in A-\{0\}$总是$A$的单位元.由于$A$是$k$上有限维代数,于是$A$是阿廷环,于是存在足够大的正整数$n$使得$Aa^n=Aa^{n+1}$,于是存在$b\in A$使得$a^n=ba^{n+1}$.但是我们有$a^n=ba(a^n)=ba(baa^n)=\cdots=b^na^{2n}$,于是$a^nb^n$是$A$的一个幂等元,按照$A$是连通代数,它只有平凡的幂等元,于是$a^nb^n=0$或者1.倘若$a^nb^n=0$,那么$a^n=b^{2n}a^n=0$,按照$A$是既约代数就有$a=0$矛盾.于是$a^nb^n=1$,于是$a$是$A$的单位.这证明了$A$是域,又因为它是$k$上的有限维代数,得到它是$k$的有限扩张.
		
		\qquad
		
		(2)推(1):记$\overline{A}=A/\sqrt{0}$,那么这是既约环.于是按照引理有$k$代数同构$\overline{A}\cong\prod_{i=1}^rK_i$,其中每个$K_i$都是$k$的有限扩张.我们有$\mathrm{Hom}_{\textbf{Alg}(k)}(A,\overline{k})=\mathrm{Hom}_{\textbf{Alg}(k)}(\overline{A},\overline{k})\cong\prod_{i=1}^r\mathrm{Hom}_{\textbf{Alg}(k)}(K_i,\overline{k})$,我们记$N=|\mathrm{Hom}_{\textbf{Alg}(k)}(A,\overline{k})|=\sum_{i=1}^r|\mathrm{Hom}_{\textbf{Alg}(k)}(K_i,\overline{k})|$.我们知道$|\mathrm{Hom}_{\textbf{Alg}(k)}(K_i,\overline{k})|$恰好是$k\subseteq K_i$的可分维数,这不超过扩张维数,于是有$|\mathrm{Hom}_{\textbf{Alg}(k)}(K_i,\overline{k})|\le[K_i:k]$,取等号当且仅当$k\subseteq K_i$是有限可分扩张.于是我们有$N\le\dim_k(\overline{A})\le\dim_k(A)$,并且$N=\dim_k(A)$当且仅当$A=\overline{A}$并且$k\subseteq K_i$总是有限可分扩张.
		
		\qquad
		
		下面按照张量积的泛性质有$\mathrm{Hom}_{\textbf{Alg}(k)}(A,\overline{k})=\mathrm{Hom}_{\textbf{Alg}(\overline{k})}(A\otimes_k\overline{k},\overline{k})$.于是$N=|\mathrm{Hom}_{\textbf{Alg}(\overline{k})}(A\otimes_k\overline{k},\overline{k})|$.于是按照条件2,结合上一段最后一句话的充分性,得到$N=\dim_{\overline{k}}(A\otimes_k\overline{k})$,进而有$N=\dim_k(A)$.再按照上一段最后一句话的必要性得到(1)成立.
		
		\qquad
		
		(1)推(4):记$A=\prod_{i=1}^rK_i$,其中每个$K_i$都是$k$的有限可分扩张.于是$A$的所有极大理想恰好是$\mathfrak{m}_i=\{0\}\times\prod_{j\not=i}K_j,\forall1\le i\le r$.按照零模是一个局部性质,我们归结为证明对每个指标$i$都有$(\Omega_{A/k})_{\mathfrak{m}_i}=\Omega_{K_i/k}=0$.而这是因为$k\subseteq K_i$是有限可分扩张,记本原元为$a$,记它的极小多项式为$p(T)\in k[T]$,那么$0=\mathrm{d}p(a)=p'(a)\mathrm{d}a$,按照$p$是可分多项式得到$p'(a)\not=0$,于是$\mathrm{d}a=0$,于是$\Omega_{K_i/k}=0$.
		
		\qquad
		
		(4)推(3):按照微分模和基变换可交换,我们有$\Omega_{A\otimes_k\overline{k}/\overline{k}}=\Omega_{A/k}\otimes_k\overline{k}=0$.于是不妨设$k$本身是代数闭域.我们要证的是从$\Omega_{A/k}=0$直接推出$A$是既约代数.因为$A$是阿廷环,它只有有限个素理想,并且这些素理想恰好都是极大理想,把它们记作$\{\mathfrak{m}_1,\cdots,\mathfrak{m}_r\}$.按照中国剩余定理,我们有$A/\sqrt{0}\cong\prod_{i=1}^rA/\mathfrak{m}_i$.由于$A$在$k$上有限维,得到这里每个域$A/\mathfrak{m}_i$都在$k$上有限维,但是这里$k$是代数闭域,就导致每个$A/\mathfrak{m}_i=k$.
		
		\qquad
		
		我们断言可以找到$e_1,\cdots,e_r\in A$满足如下三个条件:
		\begin{enumerate}[i.]
			\item $e_i$在$A/\mathfrak{m}_j$中的像是$\delta_{ij}$(此即如果$j=i$则取1,$j\not=i$则取0).
			\item $e_ie_j\in(\sqrt{0})^2,\forall i\not=j$.
			\item $e_i-e_i^2\in(\sqrt{0})^2,\forall i$.
		\end{enumerate}
		
		一旦找到满足这三个条件的$r$个元,我们记$A\to A/\mathfrak{m}_i\cong k$为$\lambda_i$,构造$\lambda:A\to A$为$a\mapsto\sum_{i=1}^r\lambda_i(a)e_i$.那么$a-\lambda(a)$明显在$\prod_{i=1}^rA/\mathfrak{m}_i$中的像是零,这导致$a-\lambda(a)\in\sqrt{0}$.那么构造$\mathrm{d}:A\to\sqrt{0}/(\sqrt{0})^2$为$a\mapsto(a-\lambda(a))\mathrm{mod}(\sqrt{0})^2$是一个$k$导数:
		\begin{align*}
			\mathrm{d}(ab)-a\mathrm{d}(b)-b\mathrm{d}(a)&=-ab+a\sum_{i=1}^r\lambda_i(b)e_i+b\sum_{i=1}^r\lambda_i(a)e_i-\sum_{i=1}^r\lambda_i(a)\lambda_i(b)e_i\\&=-(a-\sum_{i=1}^r\lambda_i(b)e_i)(b-\sum_{i=1}^r\lambda_i(a)e_i)\left(\mathrm{mod}(\sqrt{0})^2\right)\\&=0\left(\mathrm{mod}(\sqrt{0})^2\right)
		\end{align*}
		
		于是按照条件(4)得到这个$k$导数为零.那么任取$a\in\sqrt{0}$,就有$\lambda(a)=0$,于是就有$a\in(\sqrt{0})^2$,这导致$\sqrt{0}=(\sqrt{0})^2$.但是按照$A$是诺特环得到$\sqrt{0}$是有限生成的,于是NAK引理导致$\sqrt{0}=0$.于是$A$是既约代数.最后我们来解释能找到满足这三个条件的$\{e_1,\cdots,e_r\}$.首先按照中国剩余定理,我们可以找到一组$\{e_1,\cdots,e_r\}\subseteq A$满足(i).此时在$i\not=j$的时候$e_ie_j\in\sqrt{0}$,于是只要把每个$e_i$替换为$e_i^2$就可以同时满足(i)和(ii).接下来按照$A$是阿廷环,所以降链$Ae_i\supseteq Ae_i^2\supseteq\cdots$一定终止,于是我们可以找到一个统一的足够大的正整数$n$,使得对任意指标$i$都有$Ae_i^{2n}=Ae_i^n$,于是对任意指标$i$都可以找到一个元$a_i\in A$满足$a_ie_i^{2n}=e_i^n$.我们取$\varepsilon_i=(a_ie_i^n)^2=a_ie_i^n$.那么$\{\varepsilon_1,\cdots,\varepsilon_r\}$满足全部三个条件.
	\end{proof}
	\item $X_{\overline{s}}^{\mathrm{set}}$中的点恰好就是在点$s\in S$上的$X$的取值在$\Omega$中的几何点,换句话讲$X_{\overline{s}}^{\mathrm{set}}$中的点和使得如下图表交换的态射$\mathrm{Spec}\Omega\to X$是一一对应的.
	$$\xymatrix{\mathrm{Spec}\Omega\ar[rr]\ar[d]&&X\ar[d]\\\mathrm{Spec}\kappa(s)\ar[rr]&&S}$$
	\begin{proof}
		
		按照$X\to S$是仿射的,就有$X_{\overline{s}}=\mathrm{Spec}A$是仿射的,并且这里$A$是$\Omega$上的有限代数,于是$A$是阿廷环,它的点都是闭点,再按照$\Omega$是代数闭域,就有$X_{\overline{s}}$的点和它的$\Omega$有理点一一对应.再按照纤维积的泛性质,$X_{\overline{s}}$中的点就和满足这个图表交换的态射$\mathrm{Spec}\Omega\to X$一一对应.
		$$\xymatrix{\mathrm{Spec}\Omega\ar@/^1pc/@{=}[drr]\ar@{-->}[dr]\ar@/_1pc/@{-->}[ddr]&&\\&X\times_S\mathrm{Spec}\Omega\ar[r]\ar[d]&\mathrm{Spec}\Omega\ar[d]\\&X\ar[r]&S}$$
	\end{proof}
	\item 引理.如果$\varphi:X\to S$是有限态射,那么$\varphi$是平坦态射当且仅当$\varphi_*\mathscr{O}_X$是一个局部自由$\mathscr{O}_S$模层.
	\begin{proof}
		
		首先充分性是平凡的.下面证明必要性.因为问题是局部的,不妨设$X=\mathrm{Spec}B$和$S=\mathrm{Spec}A$都是仿射的,那么这里$A$是诺特环.记$\varphi:X\to S$对应于环同态$\varphi^{\#}:A\to B$.那么$B$是平坦$A$模当且仅当对任意$\mathfrak{p}\in S$都有$B_{\mathfrak{p}}$是平坦$A_{\mathfrak{p}}$模.但是按照$A_{\mathfrak{p}}$是诺特局部环,于是$B_{\mathfrak{p}}$是平坦$A_{\mathfrak{p}}$模当且仅当$B_{\mathfrak{p}}$是自由$A_{\mathfrak{p}}$模.于是对任意$\mathfrak{p}\in S$可记$B_{\mathfrak{p}}=\oplus_{i=1}^rA_{\mathfrak{p}}\frac{b_i}{s}$,其中$s\in A-\mathfrak{p}$.取环同态$\alpha:A_s^r\to B_s$为$(a_1,\cdots,a_r)\mapsto\sum_{i=1}^r\frac{a_ib_i}{s}$,它的核与余核分别记作$K$核$Q$.按照$A_s$是诺特的,得到$K$是有限生成$A_s$模,于是$\mathrm{Supp}(K)$是$\mathrm{Spec}A_s$的闭子集$V(\mathrm{Ann}(K))$.类似的从$B_s$是有限$A_s$模得到$Q$也是有限$A_s$模,于是$Q$的支集$\mathrm{Supp}(Q)$是$\mathrm{Spec}A_s$的闭子集$V(\mathrm{Ann}(Q))$.于是取$U_{\mathfrak{p}}$为$\mathrm{Supp}(K)\cap\mathrm{Supp}(Q)$的补集,它是$\mathfrak{p}\in S$的开邻域,在其上满足$\varphi_*\mathscr{O}_X\mid_{U_{\mathfrak{p}}}\cong\mathscr{O}_{U_{\mathfrak{p}}}^r$,得证.
	\end{proof}
	\item 关于秩函数.设$\varphi:X\to S$是有限平坦态射,上一条解释了$\varphi_*\mathscr{O}_X$是一个局部自由$\mathscr{O}_S$模层.我们定义$\varphi$的秩函数$S\to\mathbb{Z}_{\ge0}$为:
	\begin{align*}
		r_s(\varphi)&=\mathrm{rank}_{\mathscr{O}_{S,s}}((\varphi_*\mathscr{O}_X)_s)\\
		&=\dim_{\kappa(s)}(\mathscr{O}_{X_s}(X_s))=\dim_{\kappa(s)}((\varphi_*\mathscr{O}_X)_s\otimes_{\mathscr{O}_{S,s}}\kappa(s))\\&=\dim_{\Omega}(\mathscr{O}_{X_s}(X_s)\otimes_{\kappa(s)}\Omega)
	\end{align*}
    \begin{itemize}
    	\item 如果我们赋予$\mathbb{Z}_{\ge0}$离散拓扑,那么一般的秩函数总是上半连续函数.倘若它的取值是有限的,那么秩函数是连续函数.
    	\item 如果$S$是局部诺特的连通概形,那么这个秩函数是常值函数,把这个常值记作$r(\varphi)$称为$\varphi$的秩.
    	\item 如果$\varphi$是有限平展态射,按照域上有限平展代数的结构定理,就有$r_s(\varphi)=|X_{\overline{s}}|$,以及$\varphi^{-1}(s)$的元素个数不超过$r_s(\varphi)$.特别的$r_s(\varphi)=0$当且仅当$s$不在集合$\varphi(X)$中.
    	\item 如果$S$是局部诺特连通概形,那么一个有限平展态射$\varphi$是同构当且仅当$r(\varphi)=1$.
    \end{itemize}
	\begin{proof}
		
		证明最后一句话.必要性是平凡的.下面证明充分性.从$r(\varphi)=1$得到纤维$X_s=\mathrm{Spec}\kappa(s)$,这说明$\varphi$已经是一个双射.但是它是连续映射而且是开映射(局部有限表示平坦态射总是开映射),说明它是同胚.于是为证明它是同构只需证明对任意$s\in S$都有同构$\varphi_s^{\#}:\mathscr{O}_{S,s}\cong(\varphi_*\mathscr{O}_X)_s$.这件事归结为讲如果$A\to B$是忠实平坦环同态(于是这是单射,$A$视为$B$的子环),并且作为$A$模有$B=Ab$,那么$b\in A$:首先取$a\in A$使得$ab=1$,按照$B$在$A$上有限,于是$b$满足一个$A$系数的首一多项式$b^n+a_1b^{n-1}+\cdots+a_n=0$,对这个等式两边乘以$a^{n-1}$,得到$b\in A$得证.
	\end{proof}
    \item Galois范畴$\textbf{FEt}(S)$中的连通对象恰好是那些有限平展覆盖$X\to S$满足$X$本身是拓扑连通的.于是按照Galois范畴中对象总可以分解成有限个连通分支分解,得到有限平展覆盖$X\to S$的$X$总是只有有限个连通分支.
	\item $\textbf{FEt}(S)$上并不是所有纤维函子都具有形式$F_{\overline{s}}$.例如选取代数闭域$\Omega$和一个态射$\varphi:\mathbb{P}^1_{\Omega}\to S$,那么函子$F_{\varphi}:\textbf{FEt}(S)\to\textbf{FSets}$把有限平展态射$f:X\to S$映射为$\pi_0(X\times_{f,S,\varphi}\mathbb{P}^1_{\Omega})$也是一个纤维函子.【】
	\item 我们就把基本群$\pi_1(\textbf{FEt}(S);F_{\overline{s}})$称为$S$关于基点$\overline{s}$的平展基本群(\'etale fundamental group),记作$\pi_1(S;\overline{s})$.对任意两个基点$\overline{s_i}:\mathrm{Spec}\Omega_i\to S,i=1,2$,记$\pi_1(S;\overline{s_1},\overline{s_2})=\pi_1(\textbf{FEt}(S);F_{\overline{s_1}},F_{\overline{s_2}})$其中的元素称为从$\overline{s_1}$到$\overline{s_2}$的平展道路(\'etale path).
	\item 于是我们解释过$\pi_1(S;\overline{s_1},\overline{s_2})$总是非空的(尽管$\Omega_1$和$\Omega_2$可能特征都不同),并且$\pi_1(S;\overline{s_1})$和$\pi_1(S;\overline{s_2})$相差一个非典范的内自同构.
	\item 设$f:S'\to S$是局部诺特连通概形之间的态射.取一个几何点$\overline{s}':\mathrm{Spec}\Omega\to S'$,那么关于$f$的基变换就诱导了一个函子$f^*:\textbf{FEt}(S)\to\textbf{FEt}(S')$,如果记$f\circ\overline{s}'=\overline{s}$,那么有$F_{\overline{s}'}\circ f^*=F_{\overline{s}}$,于是$f^*$是$(\textbf{FEt}(S);\overline{s})\to(\mathscr{C}_{S'};\overline{s}')$的基本函子.于是它对应了一个射影有限群之间的同态$\pi_1(f):\pi_1(S';\overline{s}')\to\pi_1(S;\overline{s})$.
	$$\xymatrix{\textbf{FEt}(S)\ar[rr]^{f^*}\ar[dr]_{\overline{s}}&&\textbf{FEt}(S')\ar[dl]^{\overline{s}'}\\&\textbf{FSets}&}$$
	
	我们已经解释过诱导的基本群同态具有如下性质:设$f:(S,\overline{s})\to(S',\overline{s}')$和$g:(S',\overline{s}')\to(S'',\overline{s}'')$是连通概形之间的两个态射.则它们诱导了基本群之间的如下映射序列:
	$$\xymatrix{\pi_1(S,\overline{s})\ar[r]^{\pi_1(f)}&\pi_1(S',\overline{s}')\ar[r]^{\pi_1(g)}&\pi_1(S'',\overline{s}'')}$$
	\begin{enumerate}[(1)]
		\item $\pi_1(f)$是满射当且仅当对任意连通的有限平展覆盖$X\to S$(这里连通就指的是$X$是连通空间),有基变换$X'=X\times_SS'$仍然是连通空间.
		\item $\pi_1(f)$是单射当且仅当对任意连通的有限平展覆盖$X\to S$,存在有限平展覆盖$Y\to S$,存在$Y'=Y\times_{S'}S$的连通分支$Y_i'$,使得有$S$态射$Y'_i\to X$.
		\item 特别的,如果对任意连通的有限平展覆盖$X\to S$,都有$X$恰好是某个有限平展覆盖$Y\to S'$的基变换$X=Y\times_{S'}S$,那么$\pi_1(f)$是单射.
		\item 有$\pi_1(g)\circ\pi_1(f)=0$当且仅当对任意连通(从而去掉连通也是等价的)有限平展覆盖$Z\to S''$,基变换$Z\times_{S''}S$是$S$的平凡覆盖(此即可以写成$\coprod_{1\le i\le n}S\to S$).
		\item $\ker\pi_1(g)\subseteq\mathrm{im}\pi_1(f)$当且仅当,如果连通的有限平展覆盖$Y\to S'$满足$Y\times_{S'}S$具有截面(此为有$S$态射$S\to Y\times_{S'}S$),则存在一个(这里可以加上连通条件)有限平展覆盖$Z\to S''$,使得存在$Z\times_{S''}S'$的某个连通分支$(Z\times_{S''}S')_0$,满足有$S'$态射$(Z\times_{S''}S')_0\to Y$.
	\end{enumerate}
    \item 关于非连通的情况.设$S$是未必连通的局部诺特概形,我们先解释它的连通分支一定是开的,为此只需证明$S$是局部连通空间,而这是因为任取$s\in S$,可取仿射开邻域$U=\mathrm{Spec}A$,由于这里$A$是诺特环,于是$U$只有有限个连通分支,于是这些连通分支是既开又闭的,于是它们也是$S$的连通开子集,这就说明$s$存在连通的开邻域.于是$S$的连通分支分解$S=\coprod_iS_i$就是概形上的无交并.现在取一个有限平展覆盖$X\to S$,那么$X$就也是局部诺特概形,于是$X$的连通分支分解就也是无交并$X=\coprod_jE_j$,由于连通空间的连续像还是连通的,于是这里每个$E_j$都被打到某个$S_i$,把那些打到固定的$S_i$的$E_j$的无交并记作$X_i$,那么$X_i\to S_i$就也是有限平展覆盖,给定一个有限平展覆盖$X\to S$恰好等价于给定一族有限平展覆盖$X_i\to S_i$.于是我们有范畴等价$\textbf{FEt}(S)\cong\prod_i\textbf{FEt}(S_i)$.于是如果选取一族几何点$\overline{s_i}:\mathrm{Spec}\Omega\to S_i$,那么$X\to S$的基本群理应定义为$\pi_1(S,(\overline{s_i}))=\prod_i\pi_1(S_i,\overline{s_i})$.为此我们可以定义一种"多重Galois"范畴使得它是等价于一族Galois范畴直积的范畴,用这样的范畴来描述$S$的基本群.
\end{enumerate}
\begin{enumerate}[(G1)]
	\item 范畴$\textbf{FEt}(S)$具有终对象和全部二元纤维积.
	\begin{proof}
		
		这个终对象就是恒等态射$1_S:S\to S$,两个有限平展覆盖空间关于第三个有限平展覆盖空间的纤维积,此为$\textbf{Sch}(S)$中的纤维积,它也就是$\textbf{Sch}$中的纤维积,这仍然是$S$的有限平展覆盖空间.
	\end{proof}
	\item 范畴$\textbf{FEt}(S)$具有全部有限余积(这包含了空集上余积的存在性,也即初对象的存在性),并且任意对象关于有限群作用总存在商对象.
	\begin{enumerate}[(1)]
		\item 首先初对象就是空集.两个有限平展覆盖的余积就是无交并,由于平展是源端茎局部的性质,所以无交并仍然是有限平展覆盖.任取有限平展覆盖$X\to S$,设有限群$G$在$X$上的作用都是$S$同构,那么\ref{群作用和商概形9}说明此时群作用是容许的,商概形$X/G$一定存在,并且$X/G\to S$仍然是仿射态射,它是有限态射是因为在仿射局部上,$S=\mathrm{Spec}C$,$X=\mathrm{Spec}A$,那么$X/G=\mathrm{Spec}A^G$.由于$A$是诺特环$C$上的有限模,于是$A^G$作为$A$的子模也是$C$上的有限模.综上验证这一条公理只剩下证明此时$X/G\to S$仍然是平展态射.
		\item 按照SGA1上的做法.设$S$是局部诺特概形,设$X\to S$是平展分离有限型态射,设有限群$G$右作用在$X$上,并且其中的元都是$S$同构.那么$G$的作用是容许的,并且商概形$X/G$也在$S$上平展.
		\begin{proof}
			
			我们知道有限型平展态射一定是拟有限态射,于是按照【EGAIV8.11.2】,有$X\to Y$是拟射影态射.我们要证明$G$在$X$上的$Y$作用是容许的,这件事关于$Y$是局部的,于是我们不妨设$Y=\mathrm{Spec}A$是仿射的,那么$X$就是某个射影空间$\mathbb{P}_A^n$的子概型.按照\ref{群作用和商概形7}的(3),我们只需证明$X$在$G$下的每条轨道都被某个仿射开子集覆盖.这件事是因为,由于$X\to Y$的纤维都是有限集,导致$X$在$G$下的轨道都是有限集.但是射影空间中的有限点集一定被一个仿射开子集覆盖,因为一定可以取一个超曲面不包含这有限个点的每一个,再取补集就是一个仿射开子集并且覆盖了这有限个点.
			
			\qquad
			
			要证明$X/G\to S$是平展的,这是一个终端局部性质,所以不妨设$S$本身是诺特环的素谱.那么按照$X\to S$是有限型态射,得到$X$也是诺特概形,它只有有限个连通分支,所以每个连通分支都是既开又闭子集,并且$X$就是这有限个连通开子概型$\{X_i\}$的无交并.此时$X_i\to X\to S$仍然是平展分离有限型的,并且容许作用下商概形是局部性质.综上问题归结为设$X$是连通概形.另外如果记$G\to\mathrm{Aut}_{\textbf{Sch}(Y)}(X)$的核是$H$,记$G'=G/H$,那么$X/G$和$X/G'$是典范同构的.于是不妨设$G$在$X$上的作用是忠实的.我们先来说明$X\to X/G$是平展的,按照\ref{分解群和惯性群5}这件事等价于讲$G$在$X$上作用的所有惯性群都是平凡的.而这件事是因为,假设存在$\sigma\in G_d(x)$满足如下图表交换,按照【SGA1的I.5.4】就得到$\sigma=1_X$.这说明惯性群都平凡.于是我们证明了$X\to X/G$是平展的.最后归结为证明如下引理:设$X\to X'\to Y$都是有限型态射,设$Y$是局部诺特概形,设$x\in X$,它在$X'$和$Y$中的像分别记住$x'$和$y$.如果$X\to Y$和$X\to X'$都在点$x$平展,证明$X'\to Y$在$x'$平展:这两个态射诱导了局部环之间的同态$\mathscr{O}_{Y,y}\to\mathscr{O}_{X',x'}\to\mathscr{O}_{X,x}$.于是诱导的$\widehat{\mathscr{O}_{Y,y}}\to\widehat{\mathscr{O}_{X,x}}$和$\widehat{\mathscr{O}_{X',x'}}\to\widehat{\mathscr{O}_{X,x}}$是同构,于是第三个诱导的同态$\widehat{\mathscr{O}_{Y,y}}\to\widehat{\mathscr{O}_{X',x'}}$自然也是同构.
		\end{proof}
	    \item 另一种做法.我们先证明这样一个引理:设$S$是局部诺特概形,一个仿射满射态射$\varphi:X\to S$是有限平展覆盖当且仅当存在一个有限忠实平坦$f:S'\to S$,使得$\varphi$关于$f$的基变换$\varphi':X'=S'\times_{f,S,\varphi}X\to S'$是一个完全分裂的有限平展覆盖(此即它可以表示为有限个$1_{S'}:S'\to S'$的无交并$\coprod S'\to S'$).换句话讲仿射满射态射$\varphi:X\to S$是有限平展覆盖当且仅当它关于有限忠实平坦态射的Grothendieck拓扑是局部平凡的.然后我们再用这件事证明一个有限平展覆盖$\varphi:X\to S$关于一个有限群$G$的商概形仍然是有限平展覆盖.
	    \begin{proof}
	    	
	    	先证明引理的充分性.我们要证明$\varphi$是有限平展态射.
	    	\begin{itemize}
	    		\item $\varphi$是平坦态射:因为$f:S'\to S$是有限忠实平坦态射,于是对任意$s\in S$,都存在它的仿射开邻域$U=\mathrm{Spec}A$,使得$f$限制为$f^{-1}(U)\to U$对应的环的单同态$f^{\#}:A\to A'$使得$A'$是有限秩自由$A$模.又因为$\varphi:X\to S$是仿射满射态射,记$\varphi$限制为$\varphi^{-1}(U)\to U$对应的环的单同态为$\varphi^{\#}:A\to B$.按照完全分裂条件,可记有$A'$代数同构$B\otimes_AA'\cong{A'}^s$.于是有$A$模同构$B\otimes_AA'\cong A^{rs}$.另外我们有$B$模同构$B\otimes_AA''\cong B\otimes_AA^r\cong B^r$,于是这也是$A$模同构.特别的,$B$作为$A$模是自由模$A^{rs}$的直和项,导致$B$在$A$上平坦.于是我们证明了$\varphi$是平坦态射.
	    		\item $\varphi$是有限态射:按照$A$是诺特的,那么$B$作为$A^{rs}$的子模也是有限模,于是$\varphi:X\to S$是有限态射.
	    		\item $\varphi$是非分歧态射:按照微分模和基变换的相容性,有${f'}^*\Omega_{X/S}=\Omega_{X'/S'}=0$(这里$f'$是$f$的提升),于是对任意$x'\in X'$都有$({f'}^*\Omega_{X/S})_{x'}=\Omega_{X/S,f'(x')}=0$.按照$f'$是满射的基变换,它也是满射,这导致$\Omega_{X/S}=0$.于是$\varphi$是非分歧态射.
	    	\end{itemize}
	    	
	    	证明引理的必要性.为此来对$\varphi$的秩$r(\varphi)$做归纳.如果$r(\varphi)=1$,按照引理有$\varphi$是同构,此时直接取$f=1_S$即可.下面设$r(\varphi)>1$,按照$\varphi$是有限非分歧态射,导致对角态射$\Delta_{X/S}:X\to X\times_SX$同时是开嵌入和闭嵌入.于是$X\times_SX$可以表示为概形余积$X\coprod X'$.于是$X'\to X\times_SX$也同时是开嵌入和闭嵌入,特别的它也是一个有限平展态射.另外投影态射$p_1:X\times_SX\to X$作为$\varphi$的基变换也是有限平展态射.于是复合态射$\varphi':X'\to X\times_SX\to X$也是有限平展态射.但是按照$X\times_SX=X\coprod X'$,就有$r(\varphi')=r(p_1)-1=r(\varphi)-1$.于是按照归纳假设,存在有限忠实平坦态射$f:S'\to X$满足$S'\times_{f,X,\varphi'}X'\to S'$是$S'$上的完全分裂的有限平展覆盖.那么$\varphi\circ f:S'\to S$仍然是有限忠实平坦态射.$\varphi$关于这个态射的基变换是完全分裂的:首先按照如下交换图表得到$S'\times_{\varphi\circ f,S,\varphi}X=S'\times_{f,X,p_1}(X\times_SX)$.
	    	$$\xymatrix{X\times_{\varphi,S,f\circ\varphi}S'\ar[r]\ar[d]&X\times_SX\ar[r]\ar[d]&X\ar[d]^{\varphi}\\S'\ar[r]^f&X\ar[r]^{\varphi}&S}$$
	    	
	    	于是这是完全分裂的:
	    	\begin{align*}
	    		S'\times_{\varphi\circ f,S,\varphi}X&=S'\times_{f,X,p_1}(X\times_SX)\\
	    		&=S'\times_{f,X,p_1}(X\coprod X')\\
	    		&=(S'\times_{f,X,1_X}X)\coprod(S'\times_{f,X,p_1}X')\\
	    		&=S'\coprod(S'\times_{f,X,p_1}X')
	    	\end{align*}
	        
	        接下来我们借助引理证明我们的断言.我们要证明的是如果$X\to S$是有限平展覆盖,如果有限群$G$右作用在这个覆盖空间上,那么商概形$X/G$也是$S$上的有限平展覆盖空间.这里商概形的存在性我们已经解释过了,并且商概形是局部性质(因为作用是容许的),有限平展也是终端局部性质,并且这些态射都是仿射的,于是问题归结为仿射情况.即证明如果$B$是有限平展$A$代数,如果有限群$G$在$\textbf{Alg}(A)$上左作用在$B$上,那么$B^{G^{\mathrm{op}}}$也是有限平展$A$代数.那么按照我们的引理,等价于证明存在一个忠实平坦环同态$A\to A'$,使得$B^{G^{\mathrm{op}}}\otimes_AA'$是$A'$代数同构于${A'}^{\prod n}$的.而这件事是因为,首先按照$B$是有限平展$A$代数,再用一次引理就存在一个忠实平坦$A$代数$A'$,使得有$A'$代数同构$B\otimes_AA'\cong{A'}^n$.考虑如下$A$代数的正合列:
	        $$\xymatrix{0\ar[r]&B^{G^{\mathrm{op}}}\ar[r]&B\ar[rr]^{\sum_{g\in G^{\mathrm{op}}}(1_B-g\cdot)}&&\bigoplus_{g\in G^{\mathrm{op}}}B}$$
	        
	        张量平坦$A$代数$A'$仍然保证正合性,于是得到$A'$代数的正合列:
	        $$\xymatrix{0\ar[r]&B^{G^{\mathrm{op}}}\otimes_AA'\ar[r]&B\otimes_AA'\ar[rr]^{(\sum)\otimes1_{A'}}&&\oplus_{g\in G^{\mathrm{op}}}}B\otimes_AA'$$
	        
	        于是得到:
	        $$B^{G^{\mathrm{op}}}\otimes_AA'=(B\otimes_AA')^{G^{\mathrm{op}}}=({A'}^n)^{G^{\mathrm{op}}}={A'}^{\prod E_0}$$
	        
	        其中最后一个等式这个含义:如果记$E=\{1,\cdots,n\}$,那么$G^{\mathrm{op}}$在${A'}^n$上的作用无非就是对$E$做置换(因为$G^{\mathrm{op}}$现在是在$\textbf{Alg}(A')$范畴中作用在${A'}^n$的),所以如果记$E$在$G$下固定的子集是$E_0$,那么就有$({A'}^n)^{G^{\mathrm{op}}}={A'}^{\prod E_0}$.于是我们验证了引理的条件,进而得到$B^{G^{\mathrm{op}}}$也是有限平展$A$代数.
	    \end{proof}
	\end{enumerate}
	\item 如果$X,X'$是$S$的两个平展覆盖空间,那么任意$S$态射$f:X\to X'$可以分解为两个$S$态射的复合$X\o X''\to X'$,其中$X\to X''$是$\textbf{FEt}(S)$范畴中的严格满态射,$X''\to X'$是一个单态射,并且它是到$X'$的某个直和项的同构.
	\begin{enumerate}[(1)]
		\item 从$X\to S$和$X'\to S$都是有限平展态射得到$f$也是有限平展态射.于是它是开映射(平展态射是开映射)也是闭映射(有限态射是闭映射).于是只要取$X''=f(X)$,赋予它$X'$上的开子概型结构,那么$X\to X''$是有限平展满射,而$X''\to X'$是典范开嵌入(这是单态射),并且$X'$是$X''$和$X'-X''$的概形范畴意义下的余积.问题归结为证明有限平展满射是$\textbf{FEt}(S)$中的严格满态射.
		\item 引理.Amitsur复形.设$f:A\to A'$是环同态,定义:
		$$A''=A'\otimes_AA',A'''=A'\otimes_AA'\otimes_AA',\cdots$$
		
		再定义环同态$p_1,p_2:A\to A''$分别为:
		$$p_1(a')=a'\otimes1,p_2(a')=1\otimes a'$$
		
		再定义环同态$p_{12},p_{13},p_{23}:A''\to A'''$分别为:
		$$p_{12}(a'_1\otimes a'_2)=a'_1\otimes a_2'\otimes1,p_{13}(a'_1\otimes a'_2)=a'_1\otimes1\otimes a_2',p_{23}(a'_1\otimes a'_2)=1\otimes a'_1\otimes a_2'$$
		
		接下来取:
		\begin{align*}
			\partial_0&=p_1-p_2\\
			\partial_1&=p_{12}-p_{13}+p_{23}\\
			\partial_2&=p_{123}-p_{124}+p_{134}-p_{234}\\
			\cdots
		\end{align*}
		
		那么我们得到如下复形,它称为$f$的Amitsur复形,记作$A'/A$.
		$$\xymatrix{A\ar[r]^f&A'\ar[r]^{\partial_0}&A''\ar[r]^{\partial_1}&A'''\ar[r]^{\partial_2}&A''''\ar[r]&\cdots}$$
		
		我们断言如果$f:A\to A'$是忠实平坦同态,那么Amitsur复形是零调的,具体地讲:
		\begin{itemize}
			\item $A\cong\mathrm{H}^0(A'/A)$.
			\item $\mathrm{H}^q(A'/A)=0,\forall q\ge1$(但是实际下面的证明用不到这一条).
		\end{itemize}
		\begin{proof}
			
			对任意的忠实平坦$A$代数$B$,要验证这个零调结论,等价于验证$A'/A$在张量$B$后得到的$B'/B$满足$B\cong\mathrm{H}^0(B'/B)$和$\mathrm{H}^q(B'/B)=0,\forall q\ge1$.特别的,如果我们取$A$代数$B$就是$A$代数$A'$,那么这里$B\to B'$就是$A'\to A'\otimes_AA'$为$a'\mapsto1\otimes a'$,于是$B\to B'$存在左逆.换句话讲,我们可以不妨设$f:A\to A'$存在一个左逆环同态$\sigma:A'\to A$.这导致这个复形是零伦的:
			$$\xymatrix{0\ar[r]&A\ar[r]^f\ar@{=}[d]&A'\ar[r]^{\partial_0}\ar@{=}[d]\ar[dl]_{\sigma}&A''\ar[r]^{\partial_1}\ar@{=}[d]\ar[dl]_{1\otimes\sigma}&A'''\ar[r]^{\partial_2}\ar@{=}[d]\ar[dl]_{1\otimes1\otimes\sigma}&\cdots\\0\ar[r]&A\ar[r]_f&A'\ar[r]_{\partial_0}&A''\ar[r]_{\partial_1}&A'''\ar[r]_{\partial_2}&\cdots}$$
			
			于是$q\ge1$时$\mathrm{H}^q(A'/A)=0$.接下来如果$a'\in\mathrm{H}^0(A'/A)=\ker\partial_0$,此即$p_1(a')=p_2(a')$,作用$1\otimes\sigma$就得到$A'$中有$a'=f(\sigma(a'))$,于是$a'\in f(A)\cong A$(忠实平坦是单射).而另一侧的包含关系$A\subseteq\ker\partial_0$是平凡的.
		\end{proof}
		\item 我们来证明更强的结论:在范畴$\textbf{Sch}$中,如果$S$是局部诺特概形,那么有限型的忠实平坦态射$f:S'\to S$总是严格满态射.
		\begin{proof}
			
			我们先处理仿射情况.设$S=\mathrm{Spec}A$和$S'=\mathrm{Spec}A'$,那么$f$对应的环同态$\varphi:A\to A'$仍然是忠实平坦同态.我们要证明的是对任意概形$Y$,有如下序列是集合范畴中的正合列:
			$$\xymatrix{\mathrm{Hom}_{\textbf{Sch}}(S,Y)\ar[r]&\mathrm{Hom}_{\textbf{Sch}}(S',Y)\ar@<.5ex>[r]\ar@<-.5ex>[r]&\mathrm{Hom}_{\textbf{Sch}}(S'\times_SS',Y)}$$
			
			如果$Y=\mathrm{Spec}B$,那么上述序列就是如下环上同态的序列,这个正合性我们在上面引理中已经得证.
			$$\xymatrix{\mathrm{Hom}_{\textbf{Rings}}(B,A)\ar[r]&\mathrm{Hom}_{\textbf{Rings}}(B,A')\ar@<.5ex>[r]\ar@<-.5ex>[r]&\mathrm{Hom}_{\textbf{Rings}}(B,A'\otimes_AA')}$$
			
			下面设$Y$是任意概形.我们先来证明$\circ f:\mathrm{Hom}_{\textbf{Sch}}(S,Y)\to\mathrm{Hom}_{\textbf{Sch}}(S',Y)$是单射.任取态射$\varphi_1,\varphi_2$满足$\varphi_1\circ f=\varphi_2\circ f$.按照$f$是满射,就有$\varphi_1$和$\varphi_2$在集合层面上是相同的映射.下面任取$s\in S$,记$s'\in S'$满足$f(s')=s$.再记$y=\varphi_1(s)=\varphi_2(s)$.取$y$的仿射开邻域$\mathrm{Spec}B$.那么可以取$s\in S$的仿射开邻域$\mathrm{Spec}A_{\theta}$,其中$\theta\in A$,并且满足$\varphi_i(\mathrm{Spec}A_{\theta})\subseteq\mathrm{Spec}B$.再记$\theta'=\varphi(\theta)\in A'$.于是$s'\in\mathrm{Spec}A'_{\theta'}$,并且有$f(\mathrm{Spec}A'_{\theta'})\subseteq\mathrm{Spec}A_{\theta}$.于是按照仿射情况,这里$\varphi_1$和$\varphi_2$限制在$\mathrm{Spec}A_{\theta}$上是相同的态射.这得到在整个$S$上都有$\varphi_1=\varphi_2$作为态射是相同的.
			
			\qquad
			
			下面任取态射$\varphi':S'\to Y$满足$\varphi'\circ p_1=\varphi'\circ p_2$,其中$p_1,p_2$是$S''=S'\times_SS'\to S'$的两个典范投影态射.需要找到一个态射$\psi:S\to Y$满足$\varphi'=\psi\circ f$.为此我们只要局部上找到这样的$\psi$即可,因为上一段证明的单射保证了这些局部上构造的$\psi$在相交的开集上的限制是相同的态射,从而它们可以粘合为一个整体态射.
			
			\qquad
			
			任取$s\in S$,记$s'\in S'$满足$f(s')=s$.选取$\varphi'(s')$在$Y$中的仿射开邻域$V$,那么$s'$的开邻域${\varphi'}^{-1}(V)$是$S'$的关于$f$的饱和子集(一般的,如果$f:A\to B$是集合映射,一个子集$A_0\subseteq A$称为饱和的(saturated),如果$f^{-1}(f(A_0))=A_0$,换句话讲和$A_0$中的点的像相同的点都在$A_0$中):如果$x_1\in{\varphi'}^{-1}(V)$和$x_2\in S'$满足$f(x_1)=f(x_2)$,那么按照纤维积的性质有$s''\in S''=S'\times_SS'$满足$p_1(s'')=x_1$和$p_2(s'')=x_2$.于是有$\varphi'(x_1)=\varphi'(p_1(s''))=\varphi'(p_2(s''))=\varphi'(x_2)$,于是$x_2\in{\varphi'}^{-1}(V)$.
			
			\qquad
			
			按照$f$是局部有限表示的平坦态射,得到它是开映射.于是$f({\varphi'}^{-1}(V))$是$s\in S$的开邻域.我们可以选取$\theta\in A$使得$s\in\mathrm{Spec}A_{\theta}\subseteq f({\varphi'}^{-1}(V))$.再记$\theta'=\varphi(\theta)\in A'$,那么有$f(\mathrm{Spec}A'_{\theta'})=\mathrm{Spec}A_{}\theta$.并且按照${\varphi'}^{-1}(V)$关于$f$饱和得到$\mathrm{Spec}A'_{\theta'}=f^{-1}(\mathrm{Spec}A_{\theta})\subseteq{\varphi'}^{-1}(V)$.综上我们得到如下情况,这按照仿射情况就得到$\varphi'$就要经$f$分解.至此完成了$S,S'$都是仿射概形的情况的证明.
			$$\xymatrix{\mathrm{Spec}(A'_{\theta'}\otimes_{A_{\theta}}A'_{\theta'})\ar@<.5ex>[r]^{p_1}\ar@<-.5ex>[r]_{p_2}&\mathrm{Spec}A'_{\theta'}\ar[r]^f\ar[dr]_{\varphi'}&\mathrm{Spec}A_{\theta}\ar@{-->}[d]^{\psi}\\&&V}$$
			
			最后我们设$S,S'$是一般概形.我们可以不妨设$S$是仿射的.按照$S'\to S$是有限型态射,于是$S'$存在有限仿射开覆盖$\{S_i'\mid1\le i\le n\}$.记$S^*=\coprod_{i=1}^nS_i'$.那么$S^*$是仿射的,并且态射$S^*\to S$仍然是有限型忠实平坦态射.任取概形$Y$,我们就有如下交换图表.这里第二行是正合列源自于我们证明的$S',S$都是仿射的情况.中间垂直映射是单射,于是第二行的正合性得到第一行的正合性.
			$$\xymatrix{\mathrm{Hom}_{\textbf{Sch}}(S,Y)\ar[r]\ar@{=}[d]&\mathrm{Hom}_{\textbf{Sch}}(S',Y)\ar@<.5ex>[rr]\ar@<-.5ex>[rr]\ar[d]&&\mathrm{Hom}_{\textbf{Sch}}(S'',Y)\ar[d]\\\mathrm{Hom}_{\textbf{Sch}}(S,Y)\ar[r]&\mathrm{Hom}_{\textbf{Sch}}(S^*,Y)\ar@<.5ex>[rr]\ar@<-.5ex>[rr]&&\mathrm{Hom}_{\textbf{Sch}}(S^*\times_SS^*,Y)}$$
		\end{proof}
	\end{enumerate}
	\item 函子$F_{\overline{s}}$把终对象映为终对象,并且和纤维积可交换.
	\begin{proof}
		
		明显的,$\textbf{FEt}(S)$的终对象是恒等态射$1_S:S\to S$,它在$F_{\overline{s}}$下打成单点集合$|\mathrm{Spec}\Omega|$,是集合范畴的终对象.接下来任取两个有限平展态射$f:X\to S$和$g:Y\to S$,把它们的纤维积图表做关于$\overline{s}:\mathrm{Spec}\Omega$,有:
		$$\xymatrix{(X\times_SY)\times_S\mathrm{Spec}\Omega\ar[rr]\ar[d]&&Y\times_S\mathrm{Spec}\Omega\ar[d]\\X\times_S\mathrm{Spec}\Omega\ar[rr]&&\mathrm{Spec}\Omega}$$
		
		一般的,对于如下纤维积图表,点对$(x,y)\in X\times Y$满足它们在$S$中的像相同当且仅当存在$z\in Z$满足它在$X,Y$中的像分别是$x$和$y$.这件事用在上面纤维积图表就得到集合$|(X\times_SY)\times_S\mathrm{Spec}\Omega|$是集合$|X\times_S\mathrm{Spec}\Omega|$和集合$|Y\times_S\mathrm{Spec}\Omega|$的笛卡尔积.
		$$\xymatrix{Z\ar[rr]\ar[d]&&Y\ar[d]\\X\ar[rr]&&S}$$
	\end{proof}
	\item 函子$F_{\overline{s}}$和有限余积可交换,和有限群作用的商概形可交换,并且把严格满态射映为有限集合范畴中的满射.
	\begin{enumerate}
		\item $F_{\overline{s}}$和有限余积可交换是容易的,因为有$(X\coprod Y)\times_S\mathrm{Spec}\Omega=(X\times_S\mathrm{Spec}\Omega)\coprod(Y\times_S\mathrm{Spec}\Omega)$.
		\item $F_{\overline{s}}$把严格满态射映射为满射,我们只需证明这样一件事:设$f:X\to S$和$g:Y\to S$是两个有限平展态射,设$h:X\to Y$是$S$态射,那么我们解释过$h$也是有限平展态射.我们断言$h$是$\textbf{FEt}(S)$中的满态射就已经推出$h$是满射.一旦这件事得证,$h$的基变换$h_{\overline{s}}:X_{\overline{s}}\to Y_{\overline{s}}$就也是满射.
		\begin{proof}
			
			记$Y_0=\{y\in Y\mid r_y(h)=0\}$,那么这是$Y$的既开又闭子集(这里$Y$未必连通导致$h$的秩函数未必是常值的,一般的秩函数是上半连续的,而这里$h$的秩取值有限,所以这是既开又闭的),并且它恰好是$Y$的在$h$下没有原像的点集.记$Y=Y_0\coprod Y_1$.于是$h$限制为$X\to Y_1$是满射.考虑两个典范态射$u_1,u_2:Y=Y_0\coprod Y_1\to Y_0\coprod Y_0\coprod Y_1$分别是把$Y_0$映入右侧第一个分量和第二个分量,那么$u_1\circ h=u_2\circ h$,这导致$u_1=u_2$,迫使$Y_0=\emptyset$,于是$h$是满射.
		\end{proof}
	    \item $F_{\overline{s}}$和有限群作用的商概形可交换.具体地讲,设$X\to S$是有限平展覆盖,设有限群$G$在$\textbf{FEt}(S)$上右作用在$X\to S$上.那么$\sigma\in G$右作用在$F_{\overline{s}}$上为把$\mathrm{Spec}\Omega\to X$映射成$\xymatrix{\mathrm{Spec}\Omega\ar[r]&X\ar[r]^{\sigma}&X}$.记典范态射$p:X\to X/G$,这是一个有限忠实平坦态射,于是$F_{\overline{s}}(p)$是满射.另外$F_{\overline{s}}(p)$是$G$作用下不变的(因为对任意$\sigma\in G$和$\left(u:\mathrm{Spec}\Omega\to X\right)\in F_{\overline{s}}(X)$有$p\circ\sigma\circ u=p\circ u$).于是按照集合范畴上商集合的泛性质,$F(p)$就要经一个$\widetilde{p}$分解.我们就断言这个$\widetilde{p}$是双射.
	    $$\xymatrix{F(X)/G\ar@{-->}[rr]^{\exists_!\widetilde{p}}&&F(X/G)\\F(X)\ar[u]^{F(p)}\ar@/_1pc/[urr]&&}$$
	    \begin{proof}
	    	
	    	我们已经说明了$\widetilde{p}$是满射,只需再说明它是单射,换句话讲考虑如下序列,其中$u_1,u_2\in F(X)/G$,满足$p\circ u_1=p\circ u_2$,要证明的是$u_1=u_2$.我们知道一个态射$\mathrm{Spec}\Omega\to X$等同于一个二元对$(x,l)$,其中$x\in X$是这个态射的集合像,而$l$是$\kappa(x)\to\Omega$的嵌入.如果记$u_1,u_2$对应的二元对分别是$(x_1,l_1)$和$(x_2,l_2)$.我们要证明的就是$x_1=x_2$(记作$x$)和$l_1=l_2:\kappa(x)\to\Omega$.对于$x_1=x_2$,按照$p\circ u_1=p\circ u_2$,就有$p(x_1)=p(x_2)=y$,于是归结为证明$G$在$p:X\to X/G$的任何纤维上的作用是可迁的;对于$l_1=l_2$,按照$p\circ u_1=p\circ u_2$,如果记$p$诱导的典范映射$\kappa(y)\to\kappa(x)$为$\alpha$,那么有$l_1\circ\alpha=l_2\circ\alpha$.我们断言$\kappa(y)\subseteq\kappa(x)$是一个有限Galois扩张,并且$x$的分解群$G_d(x)=\{\sigma\in G\mid\sigma(x)=x\}$到$\mathrm{Gal}(\kappa(x)/\kappa(y))$是满射.这就导致存在一个$\sigma\in G_i(x)$使得$l_1=l_2\circ\sigma$,但是按照$u_2$在$G$下不变,导致$l_2\circ\sigma=l_2$,于是得到$l_1=l_2$.我们的问题都是局部的,不妨设$X=\mathrm{Spec}A$,那么有$X/G=\mathrm{Spec}A^G$.
	    	
	    	\qquad
	    	
	    	证明$G$在$p$的纤维上的作用总是可迁的.假设这个作用不是可迁的,那么存在$A$的两个素理想$\mathfrak{P}_1,\mathfrak{P}_2$,满足$\mathfrak{P}_1$在$G$下的像总不是$\mathfrak{P}_2$,并且满足$\mathfrak{P}_1\cap A^G=\mathfrak{P}_2\cap A^G=\mathfrak{p}$.我们先断言不妨设$\mathfrak{P}_1$和$\mathfrak{P}_2$都是极大理想,否则我们可以用$S^{-1}A$替换$A$,其中$S=A^G-\mathfrak{p}$,按照分式化是正合的,就有$S^{-1}(A^G)=(S^{-1}A)^G$,那么此时$\mathfrak{p}$是$A^G$的极大理想,但是$A^G\subseteq A$是整扩张,导致$\mathfrak{P}_1$和$\mathfrak{P}_2$都是$A$的极大理想,完成断言的证明(这个断言的证明换句话讲就是做基变换$\mathrm{Spec}\mathscr{O}_{Y,y}\to Y$).于是可以找到$a\in\mathfrak{P}_2$,满足$a\not\in\sigma\mathfrak{P}_1,\forall\sigma\in G$(此为中国剩余定理).于是$b=\prod_{\sigma\in G}\sigma(a)\not\in\mathfrak{P}_1$.但是$b\in A^G$,于是$b\in A^G\cap\mathfrak{P}_1=A^G\cap\mathfrak{P}_2$,这就矛盾.
	    	
	    	\qquad
	    	
	    	证明$\kappa(y)\subseteq\kappa(x)$是有限Galois扩张.因为$A^G\subseteq A$是有限整扩张,得到$A^G/\mathfrak{p}=\kappa(y)\subseteq A/\mathfrak{P}=\kappa(x)$是有限可分扩张,于是存在本原元$\theta\in A$使得$\kappa(x)=\kappa(y)(\overline{\theta})$.考虑多项式$f=\prod_{\sigma\in G}(T-\sigma(\theta))\in A^G[T]$,在$\mathrm{mod}\mathfrak{p}$下这个多项式落在$\kappa(y)[T]$中,以$\overline{\theta}$为根并且在$\kappa(x)$上完全分裂,这说明$\kappa(y)\subseteq\kappa(x)$是正规扩张,综上它是有限Galois扩张.
	    	
	    	\qquad
	    	
	    	证明分解群$G_d(x)$到$\mathrm{Gal}(\kappa(x)/\kappa(y))$的典范映射是满射.这里$x=\mathfrak{P}$是$A$的素理想.对任意$\sigma\not\in G_d(\mathfrak{P})$都有$\sigma(\mathfrak{P})\not=\mathfrak{P}$.于是按照中国剩余定理就存在$\theta_1\in A$满足$\theta_1\equiv\theta(\mathrm{mod}\mathfrak{P})$(其中$\theta\in A$是上一段证明中有限可分扩张$\kappa(y)\subseteq\kappa(x)$的本原元在$A$上的提升)和$\theta_1\equiv0(\mathrm{mod}\sigma^{-1}(\mathfrak{P}))$对任意$\sigma\not\in G_d(\mathfrak{P})$成立.于是第一个同余条件保证了$\kappa(x)=\kappa(y)(\overline{\theta_1})$.取$\overline{g}=\prod_{\sigma\in G}(T-\overline{\sigma}(\overline{\theta_1}))$.由于$\overline{\theta_1}$是$\overline{g}$的根,于是对任意$\varphi\in\mathrm{Gal}(\kappa(x)/\kappa(y))$都有$\varphi(\overline{\theta})$是$\overline{g}$的根.于是就存在一个$\sigma\in G$满足$\varphi(\overline{\theta_1})=\overline{\sigma}(\overline{\theta_1})$.我们不妨设$\overline{\theta_1}\not=0$,否则Galois群是平凡的,这时候$G_d(x)$到Galois群当然是满射.于是$\varphi(\overline{\theta_1})\not=0$.但是$\overline{\sigma}(\overline{\theta_1})\not=0$只能有$\sigma\in G_d(\mathfrak{P})$,于是得证.
	    \end{proof}
	\end{enumerate}
	\item 一个态射$u$是同构当且仅当$F_{\overline{s}}(u)$是同构.换句话讲,设$S$是连通局部诺特概形,设$s\in S$,设$\Omega$是一个包含$\kappa(s)$的代数闭域.设$X,Y$是$S$的两个平展覆盖空间,如果$S$态射$u:X\to Y$使得诱导的$X_{\overline{s}}\to Y_{\overline{s}}$是同构,那么$u$也是同构.
	\begin{proof}
		
		我们要证明$u$是同构是一个终端局部性质,所以不妨设$S$是仿射的,进而$Y$也是仿射的,并且它是一个诺特环的素谱,于是$Y$可以表示为有限个连通分支的无交并,综上我们不妨设$Y$是连通的.此时$u$是满射(因为$u$是既开又闭映射,而$Y$是连通的).并且$u$的秩函数是常值的.任取$y\in Y$,考虑如下纤维积图表,其中$\overline{y}:\mathrm{Spec}\Omega\to Y$是一个几何点,从$X_{\overline{s}}\to Y_{\overline{s}}$是同构得到$X_{\overline{y}}$是单点集,导致$r(u)=1$,于是按照我们之前的引理有$u$是同构.
		$$\xymatrix{X_{\overline{y}}\ar[r]\ar[d]&\mathrm{Spec}\Omega\ar[d]&\\X_{\overline{s}}\ar[r]\ar[d]&Y_{\overline{s}}\ar[r]\ar[d]&\mathrm{Spec}\Omega\ar[d]\\X_s\ar[d]\ar[r]&Y_s\ar[d]\ar[r]&\mathrm{Spec}\kappa(s)\ar[d]\\X\ar[r]^u&Y\ar[r]&S}$$
	\end{proof}
\end{enumerate}
\subsection{域的基本群}
\begin{enumerate}
	\item 设$k$是域,取定包含$k$的一个代数闭域$k\subseteq\Omega$,记$k^s\subseteq\Omega$是$k$的可分闭包.记$\Gamma_k=\mathrm{Aut}_k(k^s)$为绝对Galois群.转化为几何语言,记$S=\mathrm{Spec}k$,那么取定的包含$k$的代数闭域$k\subseteq\Omega$相当于取定$S$上的一个几何点$\overline{s}:\mathrm{Spec}\Omega\to S$.
	\item 连通对象.一个有限平展覆盖$X\to S$一定有$X=\mathrm{Spec}A$,其中$A$是域$k$上的有限平展代数,于是$A$是$k$的有限个有限可分扩张的直积$A=\prod_{1\le i\le r}L_i$.这导致概形上$X$是一个有限离散空间.特别的$X\to S$是连通对象当且仅当$X=\mathrm{Spec}L$,其中$k\subseteq L$是一个有限可分扩张.
	\item $F_{\overline{s}}(X)$.如果$X$是连通对象,那么$F_{\overline{s}}(X)=\mathrm{Hom}_k(L,\Omega)=\{\sigma_1,\cdots,\sigma_m\}$.于是如果记$k\subseteq L$的本原元$t$的极小多项式为$p(T)$,它的全部根为$\{a_1,\cdots,a_m\}$,其中$a_i=\sigma_i(t)$,那么$F_{\overline{s}}(X)=\{a_1,\cdots,a_m\}$.并且$\mathrm{Aut}_k(L)$在其上的作用就是$\tau(a_i)=\tau\sigma_i(t)=a_j$.
	\item Galois对象.如果一个连通对象$\mathrm{Spec}L\to S$是Galois对象,等价于讲$\mathrm{Aut}_k(L)$在$\mathrm{Hom}_k(L,\Omega)$上的作用是可迁的,于是对任意$\alpha\in\mathrm{Hom}_k(L,\Omega)$,存在$\sigma\in\mathrm{Aut}_k(L)$使得$\alpha\circ\sigma=1_L$,这件事等价于讲$k\subseteq L$是正规扩张.于是一个连通对象$\mathrm{Spec}L\to S$是Galois对象当且仅当$k\subseteq L$是有限Galois扩张(Galois对象不依赖于纤维函子的选取).
	\item 记$k$在$\Omega$中的代数闭包为$\overline{k}$,那么几何点$\overline{s}$可以经几何点$\overline{t}:\mathrm{Spec}\overline{k}\to S$分解.我们有函子的自然同构$F_{\overline{t}}\cong F_{\overline{s}}$.这是因为任取一个连通对象$\mathrm{Spec}L\to S$,我们有典范同构$\mathrm{Hom}_k(L,\Omega)\cong\mathrm{Hom}_k(L,\mathrm{Spec}\overline{k})$,因为按照$k\subseteq L$是有限可分扩张,导致一个$k$同态$L\to\Omega$一定要经$L\to\overline{k}$分解.
	\item 我们有典范映射$\mathrm{Aut}_k(k^s)\to\mathrm{Aut}(F_{\overline{t}})$为$\sigma\in\mathrm{Aut}_k(k^s)$在$F_{\overline{t}}(\mathrm{Spec}L)=\{a_1,\cdots,a_m\}$上的双射定义为$a_i\mapsto\sigma(a_i)$,其中$a_1,\cdots,a_m$是$k\subseteq L$的本原元的极小多项式在$\overline{k}$中的全部根.我们断言这是一个同构,进而有射影有限群的典范同构:
	$$c_{\overline{s}}:\Gamma_k=\mathrm{Aut}_k(k^s)\cong\pi_1(S;\overline{t})\cong\pi_1(S;\overline{s})$$
	\begin{proof}
		
		单射:因为如果$\sigma\in\Gamma_k$满足对任意有限可分扩张$k\subseteq L$有$\sigma$在$F_{\overline{t}}(\mathrm{Spec}L)$上的作用是恒等映射,那么$\sigma$在$L$上是恒等作用,但是$k^s$本身是全部有限可分扩张的复合,于是$\sigma:k^s\to k^s$是恒等映射.
		
		\qquad
		
		满射:任取一个$\tau\in\mathrm{Aut}(F_{\overline{t}})$,任取可分元$t\in k^s$,任取包含$t$的有限可分扩张$k\subseteq L$,那么$t\in F_{\overline{t}}(\mathrm{Spec}L)$,那么$\sigma(t)=\tau(\mathrm{Spec}L)(t)$不依赖于$L$的选取(因为如果$L_1,L_2$是两个包含$t$的$k$的有限可分扩张,那么可以选取一个更大的有限可分扩张$k\subseteq L$同事包含了$L_1,L_2$,于是自然变换满足的交换图表告诉我们$\tau(\mathrm{Spec}L_1)(t)=\tau(\mathrm{Spec}L)(t)=\tau(\mathrm{Spec}L_2)(t)$).再验证$\sigma\in\Gamma_k$即可.
		
		\qquad
		
		连续:为证它是连续的,按照$\mathrm{Aut}(F_{\overline{t}})$是$\prod_L\mathrm{Aut}_{\textbf{FSets}}(F_{\overline{t}}(\mathrm{Spec}L))$的子空间,只需证明$\mathrm{Aut}_k(k^s)\to\prod_L\mathrm{Aut}_{\textbf{FSets}}(F_{\overline{t}}(\mathrm{Spec}L))$是连续的,其中$L\subseteq k^s$取遍$k$的有限可分扩张.而这等价于证明对每个有限可分扩张$L\subseteq k^s$都有$\mathrm{Aut}_k(k^s)\to\mathrm{Aut}_{\textbf{FSets}}(F_{\overline{t}}(\mathrm{Spec}L))$是连续的.取$k\subseteq L$的本原元为$t$,那么终端就是$t$的全体共轭元构成的离散空间.固定一个共轭元为$\sigma(t)$,其中$\sigma\in\mathrm{Hom}_k(L,\overline{k})$.我们要证明这个映射连续,就是证明全体满足$\tau(t)=\sigma(t)$的$\tau$构成了$\Gamma_k$的既开又闭子集.但是这样的$\tau$构成的集合就是陪集$\sigma\mathrm{Aut}_L(k^s)$,它是既开又闭子群$\mathrm{Aut}_L(k^s)$的平移,所以是既开又闭的.
	\end{proof}
	
	
\end{enumerate}
\subsection{局部诺特正规整概形的基本群}

设$S$是局部诺特连通概形,那么它是正规概形当且仅当它是整概形(因为对于局部诺特概形,它的不可约分支两两不交当且仅当不可约分支恰好是连通分支).我们总设$S$是局部诺特正规整概形,用$K(S)$表示$S$的函数域.
\begin{enumerate}
	\item 正规化的有限性引理.对任意有限可分扩张$K(S)\subseteq L$,有$S$在$K(S)\subseteq L$中的正规化是有限的.这里正规化指的是对任意仿射开子集$U=\mathrm{Spec}A\subseteq S$,取$A$在$L$中的正规化$\widetilde{A}$,再把这些$\mathrm{Spec}\widetilde{A}$粘合起来得到的$S$概形.等价的讲,如果取$\mathscr{O}_S$在$S$上常值代数层$\underline{L}$的正规化代数层为$\mathscr{A}$,那么$S$在$K(S)\subseteq L$中的正规化就是$\mathrm{Spec}\mathscr{A}\to S$.【这件事前面证过了】
	\item 标准平展态射引理.设$S=\mathrm{Spec}A$是仿射的,设$P\in A[T]$是一个首一多项式,满足$P'\not=0$.取$B=A[T]/PA[T]$,再取$b\in B$,使得$P'(t)$在$C=B_b$中是单位,其中$t$是$T$在$B$中的像.那么$\mathrm{Spec}C\to S$是一个平展态射.这样的平展态射称为标准平展态射.
	\begin{enumerate}[(1)]
		\item 设$f:Y\to X$在点$y\in Y$的附近是非分歧的,那么存在$y$和$f(y)$分别的仿射开邻域$V$和$U$,使得$f(V)\subseteq U$,并且$f$限制为$V\to U$可以分解为一个嵌入$V\to\mathrm{Spec}C$和一个标准平展态射$\mathrm{Spec}C\to U$的复合.
		\item 如果把上一条中的$f$在$y$附近非分歧改为在$y$附近平展,那么结论中的$V\to\mathrm{Spec}C$是嵌入可以改为开嵌入.于是如果把$V$选取的更小,可以要求$V\to U$本身是标准平展态射.
	\end{enumerate}
	\begin{proof}
		
		【Milne etale cohomology不是讲义是书3.14和3.15】
	\end{proof}
	\item 引理.设$A$是诺特正规局部环,商域记作$K$,记$S=\mathrm{Spec}A$,设$\varphi:X\to S$是非分歧的,那么对任意$x\in X$,存在仿射开邻域$U$和如下分解,使得$U\to\mathrm{Spec}C$是一个嵌入,$\mathrm{Spec}C\to S$是一个标准平展态射,并且标准平展态射中的$B=A[T]/PA[T]$可以要求首一多项式$P$是$K[T]$中的不可约多项式.如果把$\varphi$是非分歧态射改为平展态射,那么结论中的$U\to\mathrm{Spec}C$是嵌入可以改为是开嵌入(于是如果把$U$选取的更小,可以要求$U\to S$本身是标准平展态射).
	$$\xymatrix{U\ar[rr]\ar[d]_{\varphi}&&\mathrm{Spec}C\ar[dll]\\S&&}$$
	\begin{proof}
		
		记$A$的唯一极大理想为$\mathfrak{m}$,它对应的$S$的唯一闭点记作$s$.按照上一条,我买已经有$\varphi:X\to S$可以经一个$A\to B_b$分解,其中$B=A[T]/PA[T]$,$b\in B$满足$P'(t)$在$B_b$中可逆.现在按照$A$是整闭的,导致$P$在$K[T]$中唯一分解的每个首一不可约因式都是$A$系数的.任取$x\in X_s$,可设$P$的不可约分支$Q$在$\kappa(x)$中为零.记$A[T]$中有分解$P=QR$.由于$Q$没有重根,导致在$\kappa(s)[T]$中就有$\overline{Q}$和$\overline{R}$是互素的,于是它们两个生成了单位理想$\kappa(s)[T]=A[T]/\mathfrak{m}[T]$.但是按照$M=A[T]/(Q,R)$是有限$A$模,于是按照NAK引理得到$A[T]=(Q,R)$.于是按照中国剩余定理得到$A[T]/PA[T]\cong A[T]/QA[T]\times A[T]/RA[T]$.记$B_1=A[T]/QA[T]$,记$b$在$B_1$中的像是$b_1$.那么开嵌入$U_1=\mathrm{Spec}(B_1)_{b_1}\to X$包含了点$x$,并且$U_1\to X\to S$是满足条件的标准平展态射.
	\end{proof}
    \item 设$\varphi:X\to S$是有限平展覆盖,那么$X$是正规概形,特别的$X$可以写成有限个不可约分支的无交并.另外给定$X$的一个连通分支$X_0$,那么有限平展覆盖$X_0\to S$就是$S$关于有限可分域扩张$K(S)\to K(X_0)$的正规化.
    \begin{proof}
    	
    	首先一旦我们证明了$X$是正规概形,导致它的局部环都是整环,于是$X$的不可约分支两两不交,但是从$S$局部诺特和$\varphi$是有限态射得到$X$也是局部诺特概形,我们解释过局部诺特概形上如果不可约分支两两不交,那么它的连通分支分解恰好就是不可约分支分解,而Galois范畴的形式性质告诉我们$X$只有有限个连通分支,于是此时$X$是有限个不可约分支的无交并.
    	
    	\qquad
    	
    	我们要证明的$X$是正规概形是一个茎局部性质,所以问题归结为设$S=\mathrm{Spec}A$是诺特正规局部环的素谱的情况.再按照上一条引理,还可以设$\varphi:X\to S$本身是一个标准平展态射.于是可记$X=\mathrm{Spec}C$,其中$C=B_b$,$B=A[T]/PA[T]$,其中$P$是$K[T]$中的不可约多项式,而$b\in B$满足$P'(t)$在$B_b$中是单位.我们要证明的就是$C$是正规的.
    	
    	\qquad
    	
    	记$A$的商域(也即$K(S)$)为$K$,那么$L=C\otimes_AK=K[T]/PK[T]$就是$K$是一个有限可分扩张(从而这证明了后面的$K(S)\to K(X_0)$一定是有限可分扩张).再记$A$在$L$中的正规化为$\widetilde{A}$.按照$B$在$A$上整(因为$P$是首一$A$系数多项式),就有$A\subseteq B\subseteq\widetilde{A}\subseteq L$.进而有$B_b\subseteq(\widetilde{A})_b=\widetilde{(\widetilde{A})_b}\subseteq L$.所以只要证明$\widetilde{A}\subseteq B_b$,就得到$\widetilde{(\widetilde{A})_b}=B_b$,于是$C$在$K\subseteq L$中是整闭的,就得到$C$是正规的.
    	
    	\qquad
    	
    	下面取$\alpha\in\widetilde{A}$,那么按照$\alpha\in L=K[t]$,就有$\alpha=\sum_{i=0}^{n-1}a_it^i$,其中$a_i\in K$,$n=\deg P$.按照$K\subseteq L$是$n$次可分扩张,就恰好存在$n$个不同的$K$代数同态$\varphi_i:L\to\overline{K},1\le i\le n$.那么有:
    	$$\left(\begin{array}{c}\varphi_1(\alpha)\\\varphi_2(\alpha)\\\vdots\\\varphi_n(\alpha)\end{array}\right)=\left(\begin{array}{cccc}1&\varphi_1(t)&\cdots&\varphi_1^{n-1}(t)\\1&\varphi_2(t)&\cdots&\varphi_2^{n-1}(t)\\\vdots&\vdots&\ddots&\vdots\\1&\varphi_n(t)&\cdots&\varphi_n^{n-1}(t)\end{array}\right)\left(\begin{array}{c}a_1\\a_2\\\vdots\\a_n\end{array}\right)$$
    	
    	记$V_n(t)$表示$\varphi_1(t),\cdots,\varphi_n(t)$的范德蒙矩阵,它的伴随矩阵记作$\mathrm{adj}(V_n(t))$,那么有$\mathrm{adj}(V_n(t))(\varphi_1(\alpha),\cdots,\varphi_n(\alpha))^t=|V_n(t)|(a_1,\cdots,a_n)^t$.按照$\varphi_i(\alpha)$和$\varphi_i(t)$都是$A$上的整元,得到每个$|V_n(t)|a_i$都是$A$上的整元,按照$A$是正规的就得到这些元都落在$A$中.这个$|V_n(t)|^2=\prod_{1\le i<j\le n}(\varphi_i(t)-\varphi_j(t))^2=(-1)^{\frac{n(n-1)}{2}}\mathrm{N}_{L/K}(f'(t))$.按照$f'(t)$是$B_b\subseteq(\widetilde{A})_b$的单位就得到$|V_n(t)|\in A_b\subseteq B_b$是单位,这就得到每个$a_i\in B_b$,从而$\alpha\in B_b$.
    	
    	\qquad
    	
    	最后设$X$本身是连通正规概形,从而它是整概形(这依赖于$X$是局部诺特的).任取仿射开子集$U=\mathrm{Spec}A\subseteq S$,记$\varphi^{-1}(U)=\mathrm{Spec}B$,那么$B$是正规整环,并且在$A$上整,这就导致如果记$A,B$的商域是$K,L$,那么$B$就是$A$在$L$中的正规化(如果记$A$在$L$中的正规化是$\widetilde{A}$,按照$B$在$A$上整得到$B\subseteq\widetilde{A}$,但是按照$\widetilde{A}$在$B$上整和$B$是整闭的,导致$B=\widetilde{A}$).
    \end{proof}
    \item 反过来如果给定一个有限可分域扩张$K(S)\subseteq L$,设$S$在$K(S)\subseteq L$中的正规化$\varphi:X\to S$是非分歧的,那么$\varphi$是一个有限平展覆盖.
    \begin{proof}
    	
    	按照正规化的有限性引理,有$\varphi:X\to S$是有限态射.下面证明它是平坦态射,也即对任意$x\in X$证明$\mathscr{O}_{S,\varphi(x)}\subseteq\mathscr{O}_{X,x}$是平坦扩张(这是单射因为$A\subseteq B$是环扩张,设$\mathfrak{q}\subseteq B$是素理想,记$\mathfrak{p}=\mathfrak{q}\cap A$,那么$A/\mathfrak{p}\to B/\mathfrak{q}$明显是单射).按照我们的引理,这个同态可以分解为$\mathscr{O}_{S,\varphi(x)}\to C\to\mathscr{O}_{X,x}$,其中$\mathscr{O}_{S,\varphi(x)}\to C$是标准平展态射,并且$C\to\mathscr{O}_{X,x}$是满射(就是标准平展态射分解中的嵌入分解的闭嵌入).按照$\mathscr{O}_{S,\varphi(x)}\to K(S)$是忠实平坦映射【没道理啊,它平坦是显然的,但是这诱导的态射能是满射吗】,于是从$\mathscr{O}_{S,\varphi(x)}\to\mathscr{O}_{X,x}$是单射得到如下图表的第一行是单射,这导致第二行是单射,再按照$\mathscr{O}_{S,\varphi(x)}\to K(S)$是忠实平坦映射就得到$C\to\mathscr{O}_{X,x}$是单射,但是它还是满射,于是它是同构,于是$\mathscr{O}_{S,\varphi(x)}\to\mathscr{O}_{X,x}$就是平坦的.
    	$$\xymatrix{\mathscr{O}_{S,\varphi(x)}\otimes_{\mathscr{O}_{S,\varphi(x)}}\ar@{->>}[d]\ar[rr]&&\mathscr{O}_{X,x}\otimes_{\mathscr{O}_{S,\varphi(x)}}K(S)\ar@{=}[d]\\C\otimes_{\mathscr{O}_{S,\varphi(x)}}K(S)\ar[rr]&&\mathscr{O}_{X,x}\otimes_{\mathscr{O}_{S,\varphi(x)}}K(S)}$$
    \end{proof}
    \item 刻画$\textbf{FEt}(S)$.
    \begin{enumerate}[(1)]
    	\item 设$S$是局部诺特正规整概形,那么存在函子$\textbf{FEt}(S)\to(\textbf{FEAlg}(K(S)))^{\mathrm{op}}$为把有限平展覆盖$X\to S$映射为$K(S)$有限平展代数$R(X)=\prod_{X_0\in\pi_0(X)}K(X_0)$.按照这个映射把连通对象映成连通对象,得到这是一个完全忠实函子.
    	\item 另外按照$(\textbf{FEAlg}(K(S)))^{\mathrm{op}}\cong\textbf{FEt}(\mathrm{Spec}K(S))$,这个函子实际上就是$f:\mathrm{Spec}K(S)\to S$诱导的基变换函子$f^*:\textbf{FEt}(S)\to\textbf{FEt}(\mathrm{Spec}K(S))$.于是特别的,这是一个基本函子,从而它是正合函子.
    	\begin{proof}
    		
    		验证这件事只要注意到如果$X\to S$是连通有限平展覆盖,那么$X\times_S\mathrm{Spec}K(S)=\mathrm{Spec}K(X)$,而这件事只要在仿射情况下验证,也即如果$A$是正规整环,商域记作$K$,设$K\subseteq L$是有限可分扩张,设$A$在$L$中的正规化是$B$(因为我们解释过这里$X\to S$是连通有限平展覆盖的时候$X$就是$S$在$K(S)\to K(X)$中的正规化),那么有$B\otimes_AK=L$:如果记$S=A-\{0\}$,那也就是证明$S^{-1}B=L$,一方面$S^{-1}B\subseteq L$,另一方面记$K\subseteq L$的本原元是$t\in B$,那么$t$满足一个首一极小$A$系数多项式$t^n+a_1t^{n-1}+\cdots+a_n=0$,那么$t^{-1}=-\frac{1}{a_n}\left(t^{n-1}+a_1t^{n-2}+\cdots+a_{n-1}\right)\in S^{-1}B$,进而对任意$L$中的元可以写成$\lambda_{n-1}t^{n-1}+\cdots+\lambda_0$,其中$\lambda_i\in K=S^{-1}A\subseteq S^{-1}B$,于是$L\subseteq S^{-1}B$.
    	\end{proof}
    	\item 如果记$\mathscr{D}$表示$\textbf{FEAlg}(K(S))$的完全子范畴,由那些满足$S$在每个$K(S)\subseteq L_i$中的正规化是非分歧态射的$K(S)$的有限平展代数$R=\prod_{1\le i\le n}L_i$构成,那么我们构造的这个函子就是$\textbf{FEt}(S)\to\mathscr{D}$的逆变范畴等价.它的拟逆函子是把$R=\prod_{1\le i\le n}L_i\in\mathscr{D}$映射为$\coprod_iX_i\to S$,其中$X_i\to S$是$S$在$K(S)\subseteq L_i$中的正规化,这可以称为正规化函子.
    \end{enumerate}
    \item 基本群$\pi_1(S,\overline{\eta})$.把$S$的一般点记作$\eta$,取一个包含$\kappa(\eta)=K(S)$的代数闭域$\Omega$,记$\overline{s}:\mathrm{Spec}\Omega\to\mathrm{Spec}\kappa(\eta)$和$\overline{\eta}:\mathrm{Spec}\Omega\to\mathrm{Spec}\kappa(\eta)\to S$.上一条告诉我们的是$f:\mathrm{Spec}\kappa(\eta)\to S$的基变换函子$f^*:\textbf{FEt}(S)\to\textbf{FEt}(\mathrm{Spec}\kappa(\eta))$是完全忠实的,于是它诱导的基本群同态$\pi_1(f):\pi_1(\mathrm{Spec}\kappa(\eta),\overline{s})\to\pi_1(S,\overline{\eta})$是满射.它的核是域扩张$\kappa(\eta)\subseteq M$的绝对Galois群,这里$M$是$\kappa(\eta)\subseteq\Omega$的所有使得$S$在$\kappa(\eta)\subseteq L$其中的正规化是非分歧的,的有限可分扩张$\kappa(\eta)\subseteq L$的合成.进而有典范同构$\pi_1(S,\overline{\eta})\cong\mathrm{Aut}_{K(S)}(\overline{K(S)})/\mathrm{Aut}_M(\overline{M})\cong\mathrm{Gal}(M/K(S))$.
    \begin{proof}
    	
    	我们先解释下$K=K(\eta)\subseteq M$的确是Galois扩张:它当然是可分扩张,因为是有限可分扩张的复合.任取有限可分扩张$K\subseteq L$,使得$S$在$L$中的正规化是非分歧的,任取一个$K$代数同态$\sigma L\to\Omega$,那么$K\subseteq\sigma(L)$也是有限可分扩张,并且$S$也在$\sigma(L)$中的正规化是非分歧的,于是$M$包含每个元的共轭元,于是$K\subseteq M$是正规扩张.
    	
    	\qquad
    	
    	对于有限维$K$代数$B$,称$M$分裂了$B$,如果有$M$代数同构$B\otimes_KM\cong M\times M\cdots\times M$.特别的这样的$B$一定是有限平展$K$代数.考虑$\textbf{FEt}(\mathrm{Spec}K)$的由被$M$分裂的有限维$K$代数构成的完全子范畴记作$\mathscr{E}$,那么我们构造的函子$\textbf{FEt}(S)\to\textbf{FEt}(\mathrm{Spec}K)$的像落在$\mathscr{E}$中.我们来证明$\mathscr{E}$与范畴$\mathscr{C}(\Pi)$是范畴等价的,其中$\Pi=\mathrm{Gal}(M/K)^{\mathrm{op}}$.我们来构造一个范畴等价函子$\mathscr{E}\to\mathscr{C}(\Pi)$:
    	\begin{itemize}
    		\item 对象:设$B$是被$M$分裂的有限$K$代数,它对应的$\Pi$右连续作用集合取为$\mathrm{Spec}B\otimes_KM$.群作用定义为对$\sigma\in\Pi$,它在$\mathrm{Spec}B\otimes_KM$上的双射为$1_B\otimes\sigma:B\otimes_KM\to B\otimes_KM$诱导的双射.
    		\item 态射:任取$K$代数同态$\alpha:B_1\to B_2$,那么它诱导了同态$\alpha\otimes1:B_1\otimes_KM\to B_2\otimes_KM$,进而诱导了映射$\mathrm{Spec}B_2\otimes_KM\to\mathrm{Spec}B_1\otimes_KM$.这明显是一个$\Pi$同态.
    		\item 本质满:任取$\mathscr{C}(\Pi)$中的连通对象,它同构于陪集集合$\Pi/U$,按照无限Galois定理,这对应于一个有限可分扩张$K\subseteq L$,就取$B=L$,那么有$L\otimes_KM=K[T]/(f(T))\otimes_KM=M[T]/(f(T))\cong\prod_{1\le i\le n}M$.
    		\item 我们至此还没用到$\mathscr{E}$中的对象要被$M$分裂这个条件.这个条件保证了这样一件事:任取$B_1,B_2\in\mathscr{E}$,那么$\mathrm{Spec}B_i\otimes_KM\cong\mathrm{Spec}\prod M$导致对任意集合映射$\mathrm{Spec}B_2\otimes_KM\to\mathrm{Spec}B_1\otimes_KM$,都能被一个$\prod M\to\prod M$诱导,换句话讲任意集合映射$\mathrm{Spec}B_2\otimes_KM\to\mathrm{Spec}B_1\otimes_KM$都是被某个$K$代数同态$B_1\otimes_KM\to B_2\otimes_KM$诱导的.
    		\item 忠实:设有两个$K$代数同态$\alpha_1,\alpha_2:B_1\to B_2$诱导了相同的$\Pi$同态$\mathrm{Spec}B_2\otimes_KM\to\mathrm{Spec}B_1\otimes_KM$.设这个$\Pi$同态被$K$代数同态$\beta:B_1\otimes_KM\to B_2\otimes_KM$诱导的.那么对任意$b\in B_1$就有$\beta(b\otimes1)=\alpha_1(b)\otimes1=\alpha_2(b)\otimes1$,这迫使$\alpha_1(b)=\alpha_2(b)$,也即$\alpha_1=\alpha_2$.
    		\item 完全:任取$\Pi$同态$\mathrm{Spec}B_2\otimes_KM\to\mathrm{Spec}B_1\otimes_KM$,那么它被一个$K$代数同态$\beta:B_1\otimes_KM\to B_2\otimes_KM$诱导.于是任取$b\in B_1$,取$B_2$作为$K$线性空间的一组基$\{\omega_1,\cdots,\omega_n\}$,那么$\beta(b\otimes1)=\sum_i\lambda_i\omega_i\otimes m_i$,其中$\lambda_i\in K,m_i\in M$(并且这样的分解是唯一的,因为这是域上代数的张量积).那么按照$\beta$是$\Pi$同态,任取$\sigma\in\Pi$就导致$\sum_i\lambda_i\omega_i\otimes m_i=\sum_i\lambda_i\omega_i\otimes\sigma(m_i)$,于是对任意$\sigma\in\Pi$就有$\sigma(m_i)=m_i$,按照Galois定理就有$m_i\in K$,于是$\varphi(b\otimes1)$一定可以表示为$b'\otimes1$的形式,那么这个表示是唯一的,取$\alpha:B_1\to B_2$为$\alpha(b)=b'$,那么$\alpha$是$K$代数同态,并且它诱导了$\beta$.
    	\end{itemize}
    	
    	于是我们构造的函子$H:\textbf{FEt}(S)\to\mathscr{E}$是Galois范畴之间的纤维函子.它诱导了基本群之间的同态$u_H:\mathrm{Gal}(M/K)\to\pi_1(S,\overline{\eta})$.由于$H$把连通对象映为连通对象,导致这是一个满射.下面证明它是单射.为此我们来验证单射的等价描述:任取$\mathscr{E}$中的一个连通对象$X'=\mathrm{Spec}L$,那么$K\subseteq L$是一个有限可分扩张,并且$L\subseteq M$.按照$M$的定义,就可以找到有限个有限可分扩张$K\subseteq L_i,1\le i\le r$,使得$S$在每个$L_i$中的正规化$X_i$都是非分歧的,并且$L\subseteq L_1\cdots L_r$.那么$X=X_1\times_S\cdots\times_SX_r\to S$是有限平展覆盖,并且有$H(X)=\mathrm{Spec}(L_1\otimes_K\cdots\otimes_KL_r)$(这件事是因为$H$是正合函子,从而和有限积可交换).我们有满同态$L=L_1\otimes_K\cdots\otimes_KL_r\to L_1\cdots L_r$为$x_1\otimes\cdots\otimes x_r\mapsto x_1\cdots x_r$,它的核是$L$的一个素理想$\mathfrak{p}$,但是因为$L$是$K$上的有限维代数,导致$\mathrm{Spec}L$是一个有限离散空间,于是$\mathrm{Spec}(L_1\cdots L_r)$是$H(X)$的一个连通分支.我们有$\mathscr{E}$中的态射$\mathrm{Spec}(L_1\cdots L_r)\to\mathrm{Spec}L$.至此我们验证了$u_H$是单射的等价描述.
    \end{proof}
    \item 一维情况.设$S$是一维局部诺特正规整概形.此时$S$的每个仿射开子集都是一个戴德金整环的素谱(这里戴德金整环的定义是维数$\le1$的诺特正规整环,涵盖了域的情况).如果$A$是戴德金整环,商域记作$K$,取有限扩张$K\subseteq L$,取$A$在$L$中的正规化是$B$,那么$B$也是一个戴德金整环,并且$\mathrm{Spec}B\to\mathrm{Spec}A$是非分歧态射等价于讲对$A$的任意素理想$\mathfrak{p}$,有$\mathfrak{p}B$在$B$中做素理想分解中的次数都是1次的.特别的,由于$S$只有闭点和一般点,并且它的一般点处对应的是局部仿射开子集上的零理想,此时素理想分解是平凡的非分歧的.于是$S$在有限可分扩张$K=K(S)\subseteq L$中的正规化是非分歧的只需验证$S$的全部闭点处的离散赋值环在$L$中的正规化是非分歧的.
    \item 一维函数域的补充.设$K$是域,设$t$是$K$的一个超越元.
    \begin{enumerate}[(1)]
    	\item 加性赋值.任取不可约多项式$f\in K[t]$,我们可以定义$K(t)$上的加性赋值$v_f:K(t)^*\to\mathbb{Z}$为,对任意$g\in K[t]$,可记$g=f^nh$,其中$(f,h)=1$,定义$v_f(g)=n$,再延拓定义$v_f(g/h)=v_f(g)-v_f(h)$.再定义加性赋值$v_{\infty}:K(t)^*\to\mathbb{Z}$为$v_{\infty}(g/h)=\deg h-\deg g$.那么$K(t)$上的加性赋值一定等价于某个$v_f$或者$v_{\infty}$.
    	\item 任取$K(t)$上的加性赋值$v$,任取一个有限可分域扩张$K(t)\subseteq F$,那么$v$可以延拓为$F$上的加性赋值记作$w$(但是延拓未必唯一).我们称扩张$w/v$是温分歧的(tamely ramified)如果剩余域扩张$\kappa(v)\subseteq\kappa(w)$(剩余域是赋值环的剩余域)是可分的,并且分歧指数$e(w/v)$不被$\mathrm{char}(K)$整除;如果扩张$w/v$是温分歧的,我们称它是非分歧的,如果分歧指数$e(w/v)=1$;称$v$是关于$K(t)\subseteq F$的温分歧或者非分歧的,如果对它任意的到$F$上的延拓$w$都有$w/v$是温分歧或者非分歧的.
    	\item 设$K(t)\subseteq F$是有限可分扩张,满足$F-K$中的元在$K$上都是超越的.再设$v_{\infty}$关于$F$是温分歧的,并且对任意不可约多项式$f\in K[t]$有$v_f$关于$F$是非分歧的.那么这个域扩张是平凡的,也即$F=K(t)$.
    	\begin{proof}
    		
    		任取$K(t)$上的加性赋值$v$,任取它在$F$上的延拓赋值$w$.把域关于赋值的完备化分别记作$K(t)_v$和$F_w$.我们定义$w$的微分指数(differential exponent)$m(w)$是最大的整数$m$,满足对任意满足$e(w/v)w(x)\ge-m$的$x\in F_w$都有$v(\mathrm{Tr}_{F_w/K(t)_v}(x))\ge0$(这里$e(w/v)w$就是和$w$等价的规范加性赋值,也即取值是整个$\mathbb{Z}$).我们有$m(w)\ge e(w/v)-1$,并且这个不等式取等号当且仅当$w/v$是温分歧扩张,并且$m(w)=0$当且仅当$w/v$是非分歧的【见chevalley的introduction to the theory of algebraic functions of one variable】.【后面证明见galois theory for schemes 6.20】
    		
    		
    	\end{proof}
        \item 推论.设$K(t)\subseteq F$是有限可分扩张.如果$v_{\infty}$关于$F$是温分歧的,并且全部$v_f$关于$F$都是非分歧的.那么存在有限可分扩张$K\subseteq L$使得$F=L(t)$.
        \begin{proof}
        	
        	我们只要取$L$为$K\subseteq L$的代数闭包(全部代数元构成的中间域),那么$t$也是$L$的超越元,此时$L(t)\subseteq F$的赋值扩张满足相应条件,于是按照上一条得到$F=L(t)$.最后按照$K(t)\subseteq L(t)$是有限可分扩张,得到$K\subseteq L$是有限可分扩张.
        \end{proof}
    \end{enumerate}
    \item 一些例子.
    \begin{enumerate}[(1)]
    	\item 取$S=\mathrm{Spec}\mathbb{Z}_p$(这是$p$-adic数环,它明显有离散赋值,所以是一个DVR).它的函数域就是$K=\mathbb{Q}_p$.那么$M$就是$K$的极大非分歧扩张.我们有数论中的熟知结论$\mathrm{Gal}(M/K)\cong\mathrm{Gal}(\overline{\mathbb{F}_p}/\mathbb{F}_p)\cong\widehat{\mathbb{Z}}$(此为$\varprojlim_{n\ge1}\mathbb{Z}/n\mathbb{Z}$).于是我们有:
    	$$\pi_1(\mathrm{Spec}\mathbb{Z}_p)\cong\widehat{\mathbb{Z}}$$
    	\item 取$S=\mathrm{Spec}\mathbb{Z}$,它的函数域是$\mathbb{Q}$,在数论中有熟知结论:$\mathbb{Q}$不存在非平凡的非分歧扩张,于是$M=\mathbb{Q}$,并且有$\pi_1(\mathrm{Spec}\mathbb{Z})$是平凡的.
    	\item 取$K$是域,取$S=\mathbb{P}^1_K$.它的函数域是$K(t)$,其中$t$是$K$上超越元.它的全部闭点对应的加性赋值就是全部$v_f$和$v_{\infty}$.如果$K(t)\subseteq F$是有限可分扩张,使得$S$在$F$中是非分歧的,那么全部$v_f$和$v_{\infty}$都是非分歧的,于是上一条的补充告诉我们$F\subseteq K^s(t)$,其中$K^s$是一个预先选定的$K$的可分闭包.反过来任取一个有限可分扩张$K\subseteq L$,那么$S$在$L(t)$中是非分歧的【?】.于是我们证明了$M=K^s(t)$,于是我们有如下典范同构.特别的如果$K$本身是可分闭的(此即$K=K^s$),那么$\pi_1(S)$是平凡的.
    	$$\pi_1(\mathbb{P}_K^1)\cong\mathrm{Gal}(K^s(t)/K(t))\cong\mathrm{Gal}(K^s/K)\cong\Gamma_K=\pi_1(\mathrm{Spec}K)$$
    	\item 取$K$是域,取$S=\mathbb{A}_K^1$.它的函数域依旧是$K(t)$,但是此时$S$闭点上的所有加性赋值只有全体$v_f$,其中$f\in K[t]$是不可约多项式.
    	\begin{itemize}
    		\item 如果$\mathrm{char}K=0$,此时$K(t)$上的任意赋值都在任意扩张$K(t)\subseteq F$上温分歧【?】,特别的$v_{\infty}$是温分歧的.于是按照我们的补充,依旧有$M=K^s(T)$.于是此时有:
    		$$\pi_1(\mathbb{A}_k^1)=\Gamma_K=\pi_1(\mathrm{Spec}K)$$
    		\item 如果$\mathrm{char}K=p>0$,此时我们依旧有满同态$\pi_1(\mathbb{A}_K^1)\to\pi_1(\mathrm{Spec}K)$,但是它未必是单射.【?】
    	\end{itemize}
        \item 设$A$是有限环,那么$A$是阿廷环,所以它是有限个有限局部环的直积,不妨设$S=\mathrm{Spec}A$是连通的,也即$(A,\mathfrak{m})$是有限局部环,它的剩余域$k$就是一个有限域.环同态$A\to k$诱导了射影有限群之间的同态$\pi_1(\mathrm{Spec}k)\to\pi_1(\mathrm{Spec}A)$.我们断言这是一个同构.
        \begin{proof}
        	
        	满射:只要证明$A\to k$诱导的基变换函子把连通对象映射为连通对象.换句话讲如果$A$代数$B$是连通代数(没有非平凡幂等元),那么$B\otimes_Ak=B/\mathfrak{m}B$也是连通代数.注意这里$\mathfrak{m}$是幂零理想(因为$A$只有有限个理想,所以一定有$\mathfrak{m}^n=\mathfrak{m}^{n+1}$,再用NAK引理即可),导致$B/\mathfrak{m}B$从概形上看和$B$在拓扑上是一样的,于是它们的连通性是一致的.
        	
        	\qquad
        	
        	单射:我们来验证基本函子诱导射影有限群之间单射的等价描述的条件.任取$\textbf{FEt}(\mathrm{Spec}k)$的一个连通对象$\mathrm{Spec}L$,那么$k\subseteq L$是有限可分扩张,可记$L=k[X]/(f(X))$,其中$f(X)$是一个可分不可约多项式.取$f(X)$在$A[X]$中的一个提升$g(X)$,那么$g$的首系数是$A$的一个单位(因为$A$是局部环).取$B=A[X]/(g(X))$,这是一个自由$A$模从而平坦,并且$\mathrm{Spec}B\to\mathrm{Spec}A$是非分歧的(等价于纤维态射$\mathrm{Spec}B\otimes_Ak\to\mathrm{Spec}k$是非分歧的,但是这是因为$B\otimes_Ak=L$是$k$的有限可分扩张).于是$\mathrm{Spec}B\to\mathrm{Spec}A$是一个有限平展覆盖,并且有$B\otimes_Ak\cong L$,于是在$\textbf{FEt}(\mathrm{Spec}k)$上存在态射$\mathrm{Spec}B\otimes_Ak\to\mathrm{Spec}k$.至此验证了单射的等价描述.
        \end{proof}
    \end{enumerate}
\end{enumerate}
\subsection{第一同伦列}
\begin{enumerate}
	\item 一些预备补充【】.
	\begin{enumerate}[(1)]
		\item Stein分解定理.设$S$是局部诺特概形,设$f:X\to S$是紧合态射,那么$f_*\mathscr{O}_X$是一个凝聚$\mathscr{O}_S$代数层,它定义了一个$S$概形$p:S'=\mathrm{Spec}(f_*\mathscr{O}_X)$,并且这是有限态射(一般的,如果$S$是局部诺特概形,一个仿射态射$p:Y\to S$是有限态射当且仅当$p_*\mathscr{O}_Y$是凝聚层【GEA2的6.1.3】).那么$f$可经这个$p$分解,并且分解得到的$f'$也是紧合态射(这从$f$紧合与$p$分离直接得到).这个分解称为紧合态射$f$的Stein分解.
		$$\xymatrix{X\ar[rr]^f\ar[dr]_{f'}&&S\\&\mathrm{Spec}f_*\mathscr{O}_X\ar[ur]_p&}$$
		\item Zariski连通性定理.设$S$是局部诺特概形,设$f:X\to S$是紧合态射,记Stein分解为$f=p\circ f'$.那么对任意$s'\in S'$,有纤维${f'}^{-1}(s')$总是非空的并且是几何连通的(特别的这个$f'$是满射).
		\item 推论.在上述条件下,任取$s\in S$,由于$p:S'\to S$是有限态射,导致$S'_s$是有限离散空间.于是按照Zariski连通性定理得到$f^{-1}(s)={f'}^{-1}(S'_s)$的连通分支分解就是${f'}^{-1}(s'),s'\in S'_s$.换句话讲我们证明了$X_s$的连通分支恰好一一对应于$|S'_s|$中的点.类似的按照${f'}^{-1}(s')$是几何连通的,就得到对任意几何点$\overline{s}:\mathrm{Spec}\Omega\to S$,都有$X_{\overline{s}}$的连通分支和$|S'_{\overline{s}}|$是一一对应的.
		\item separable态射.称一个态射$X\to S$是separable态射,如果它是平坦的,并且对任意$s\in S$都有$X_s=X\times_S\mathrm{Spec}\kappa(s)$是几何既约$\kappa(s)$概形,
	\end{enumerate}
    \item 引理.设$S$是局部诺特概形,设$f:X\to S$是紧合态射和separable态射.记它的Stein分解为$\xymatrix{X\ar[r]^{f'\qquad}&S'=\mathrm{Spec}(f_*\mathscr{O}_X)\ar[r]^{\qquad p}&S}$.那么这里$p$是一个有限平展覆盖.
    \item 引理.设$S$是局部诺特概形,设$X\to S$是紧合态射,满足$f_*\mathscr{O}_X=\mathscr{O}_S$.那么如果$S$是连通的,则$X$也是连通的.
    \begin{proof}
    	
    	按照Zariski连通性定理,条件下有$f:X\to S$的所有纤维都是几何连通的.另外$f$是紧合态射导致它是闭映射.下一段会证明$f$是满射.在这三个条件下,如果有开集的无交并$X=X_1\coprod X_2$,任取$s\in S$,那么$f^{-1}(s)$是连通的导致它只能整个落在$X_1$中或者$X_2$中.这说明闭子集$S_1=f(X_1)$和$S_2=f(X_2)$是无交的,按照$f$是满射得到$S=S_1\coprod S_2$是两个闭子集的无交并.于是如果$S$是连通的,就导致比方说$S_2=\emptyset$,进而$X_2=\emptyset$,这说明$X$是连通的.
    	
    	\qquad
    	
    	这一段我们证明$f$是满射,为此只需证明$f(X)=\mathrm{Supp}(f_*\mathscr{O}_X)$.一旦这成立,结合$f_*\mathscr{O}_X=\mathscr{O}_S$以及$\mathrm{Supp}(\mathscr{O}_S)=S$就导致$f(X)=S$.一方面,按照$f$是闭映射,有$f(X)$是$S$的闭子集,于是对$S-f(X)$中的任意点$s$,都存在开邻域$V$满足$f^{-1}(V)=\emptyset$,于是此时$(f_*\mathscr{O}_X)_s=0$,也即有$\mathrm{Supp}(f_*\mathscr{O}_X)\subseteq f(X)$.另一方面任取$s=f(x)\in f(X)$,我们只要证明$((f_*\mathscr{O}_X))_s\not=0$即可.我们断言$(f_*\mathscr{O}_X)_s$只和$x$的开邻域有关,一旦这成立,我们可以约化到$X,S$都是仿射的情况,但是对于仿射情况,记$S=\mathrm{Spec}A$和$X=\mathrm{Spec}B$,如果记$x$对应于$\mathfrak{q}\in\mathrm{Spec}B$,那么$s$对应于$\mathfrak{p}=\mathfrak{q}\cap A\in\mathrm{Spec}A$,此时有$(f_*\mathscr{O}_X)_s=B_{\mathfrak{p}}$(此为$B$关于乘性闭子集$A-\mathfrak{p}$的分式化)明显不是零环.综上问题归结为证明$(f_*\mathscr{O}_X)_s$只和$x$的开邻域有关.按照定义这个环是全体$\mathscr{O}_X(f^{-1}(V))$的正向极限,其中$V$跑遍$S$的使得$x\in f^{-1}(V)$的开集.任取$x$的开邻域$W$,我们断言一定可以找到$S$的开邻域$V'$使得$f^{-1}(V')\subseteq f^{-1}(V)\cap W$,一旦这成立,那么包含在$W$中的形如$f^{-1}(V')$的$x$的开邻域构成了所有形如$f^{-1}(V)$的$x$的开邻域的共尾子集,我们知道一般的一个正向系统和它的共尾子集具有相同的极限,就得证.而最后这件事是因为:按照$f$是闭映射,有$V'=S-f(X-f^{-1}(V)\cap W)$是开集.那么如果$x\in f^{-1}(V')$,就有$f(x)\in V'=S-f(X-f^{-1}(V)\cap W)$,于是$f(x)\not\in f(X-f^{-1}(V)\cap W)$,于是$x\not\in X-f^{-1}(V)\cap W$,于是$x\in f^{-1}(V)\cap W$.也即$f^{-1}(V')\subseteq f^{-1}(V)\cap W$.
    \end{proof}
    \item 引理.设$\xymatrix{X'\ar[r]^g&X\ar[r]^f&S}$是概形之间的态射,这个复合记作$h$,其中$S$是局部诺特概形,$f$是紧合态射和separable态射(从而$X$也是局部诺特概形),$g$是有限平展覆盖.我们断言$h$是separable态射.
    \begin{proof}
    	
    	首先$h$是平坦态射的复合所以是平坦态射.还需要证明$h$关于任意一个$\mathrm{Spec}K\to Y$的基变换是既约的.但是此时$X$关于这个态射的基变换仍然是既约的,所以我们归结为设$S=\mathrm{Spec}K$本身是域的素谱.要证明的是从$X$是既约概形推出$X'$也是既约概形.按照既约是一个茎局部性质,归结为设$X=\mathrm{Spec}A$是仿射的情况,于是此时$X'=\mathrm{Spec}A'$也是仿射的.设$A$的全部极小素理想为$\{\mathfrak{p}_1,\cdots,\mathfrak{p}_n\}$,这是有限的是因为$X$是局部诺特概形,导致$A$是诺特环.那么按照$A$既约得到典范映射$A\to\prod_{i=1}^n(A/\mathfrak{p}_i)$是单射.于是它张量平坦$A$代数$A'$就也是一个单射$A'\to\prod_{i=1}^n(A'/\mathfrak{p}_iA')$.于是归结为证明$A'/\mathfrak{p}_iA'$是既约环.于是我们不妨用$A/\mathfrak{p}_i$替换$A$,此时$A/\mathfrak{p}_i\to A'/\mathfrak{p}_iA'$作为$A\to A'$的基变换仍然是有限平展的.换句话讲我们不妨设$A$是整环和$\mathfrak{p}_i$是零理想.取$\mathrm{Spec}A$的唯一一般点为$\eta$,那么$\kappa(\eta)$是$A$的商域,记作$L$.点$\eta$在$\mathrm{Spec}A'\to\mathrm{Spec}A$下的纤维是$\mathrm{Spec}(A'\otimes_AK)$.由于这个纤维态射是终端为域的平展态射,导致$A'\otimes_AK=\prod_{i=1}^rL_i$,其中每个$L_i$都是$L$的有限可分扩张.于是$A'\otimes_AK$是既约环.于是对任意$x'\in X'$有$A'\otimes_AK$的分式化$\mathscr{O}_{X',x'}\otimes_{\mathscr{O}_{S,g(x')}}K$是既约环,特别的它的子环$\mathscr{O}_{X',x'}$是既约环,这就得到$A'$是既约环.
    \end{proof}
    \item 引理.设$f:X\to S$是紧合态射和separable态射,设$S$是局部诺特概形,那么$f_*\mathscr{O}_X=\mathscr{O}_S$当且仅当$f$的所有纤维都是几何连通的(并且这里必要性是不需要separable态射条件的).
    \begin{proof}
    	
    	一方面如果$f_*\mathscr{O}_X=\mathscr{O}_S$,那么$f$的stein分解是平凡的,此时Zariski连通性定理保证了$f$的所有纤维都是几何连通的.反过来如果$f$的所有纤维都是几何连通的,于是$X_{\overline{s}}$总是连通的,于是Zariski连通性定理保证了$|S_{\overline{s}}'|=1$,于是有限平展覆盖$p:S'\to S$满足$r(p)=1$,于是$p$是同构,也即$f_*\mathscr{O}_X=\mathscr{O}_S$.
    \end{proof}
    \item 第一同伦列定理.设$S$是局部诺特连通概形,设$f:X\to S$紧合态射,满足$f_*\mathscr{O}_X=\mathscr{O}_S$.取$x\in X$,记$s=f(x)$,取定$X$上的关于点$x$的一个几何点$\eta:\mathrm{Spec}\Omega\to X$,它在$f$下的像记作$\overline{s}$,那么纤维积的泛性质保证了它们诱导了$X_{\overline{s}}$上的一个几何点$\overline{\eta}$:
    $$\xymatrix{\mathrm{Spec}\Omega\ar@{=}@/^1pc/[drr]\ar@/_1pc/[ddr]_{\eta}\ar@{-->}[dr]^{\overline{\eta}}&&\\&X_{\overline{s}}\ar[r]\ar[d]&\mathrm{Spec}\Omega\ar[d]^{\overline{s}}\\&X\ar[r]^f&S}$$
    
    于是投影态射和$f$是态射链$\xymatrix{(X_{\overline{s}},\overline{\eta})\ar[r]&(X,\eta)\ar[r]&(S,\overline{s})}$.这诱导了射影有限群的如下同态链.第一同伦列定理是说,这里$p$是一个满射,并且有$\mathrm{im}(i)\subseteq\ker(p)$,并且如果$f$还是一个separable态射(于是这里完整的条件是$f$是紧合态射和separable态射并且满足$f_*\mathscr{O}_X=\mathscr{O}_S$,我们解释过这个条件等价于讲$f$是平坦紧合态射,并且所有纤维都是几何既约和几何连通的),那么这是一个正合列.
    $$\xymatrix{\pi_1(X_{\overline{s}},\overline{\eta})\ar[r]^i&\pi_1(X,\eta)\ar[r]^p&\pi_1(S,\overline{s})\ar[r]&1}$$
    \begin{proof}
    	\begin{enumerate}[(a)]
    		\item $p$是满射:只需验证基变换函子$f^*$把连通对象映射为连通对象.任取连通的有限平展覆盖$\varphi:S'\to S$,它关于$f$的基变换是$\varphi':X'=X\times_SS'\to X$.
    		$$\xymatrix{X'\ar[rr]^{\varphi'}\ar[d]_{f'}&&X\ar[d]^f\\S'\ar[rr]_{\varphi}&&S}$$
    		
    		我们有$f'_*(\mathscr{O}_{X'})=f'_*({\varphi'}^*\mathscr{O}_X)=\varphi^*(f_*\mathscr{O}_X)=\varphi^*\mathscr{O}_S=\mathscr{O}_{S'}$(其中第二个等号依赖于$f$是qcqs态射和$\varphi$是平坦态射),第三个等号是条件要求的.最后结合$f':X'\to S'$也是紧合态射和$S'$是连通的,就得到$X'$是连通的.
    		\item $p\circ i=0$:按照Zariski连通性定理,这里$f:X\to S$的所有纤维都是几何连通的.我们来验证之前给出的$p\circ i=0$的等价描述,也即任取一个连通的有限平展覆盖$\varphi:S'\to S$,证明$S'\times_SX_{\overline{\eta}}$是$X_{\overline{\eta}}$的平凡覆盖.但是我们知道由于$S'\to S$是有限态射,所以它的几何点的纤维是离散空间,于是有:
    		\begin{align*}
    			S'\times_SX_{\overline{\eta}}&=S'\times_S(\mathrm{Spec}\Omega\times_SX)\\
    			&=(S'\times_S\mathrm{Spec}\Omega)\times_SX\\
    			&=(\coprod\mathrm{Spec}\Omega)\times_SX\\
    			&=\coprod X_{\overline{\eta}}
    		\end{align*}
    	    \item $\ker(p)\subseteq\mathrm{im}(i)$:按照我们的等价描述,要取一个连通的有限平展覆盖$\varphi:X'\to X$,设它关于$\overline{s}:\mathrm{Spec}\Omega\to S$的基变换$\widetilde{\varphi}:\widetilde{X'}\to\widetilde{X}$作为$\widetilde{X}$态射具有截面,记作$\sigma:\widetilde{X}\to\widetilde{X'}$.只需证明$\varphi:X'\to X$是一个连通有限平展覆盖$S'\to S$关于$X\to S$的基变换.
    	    $$\xymatrix{\widetilde{X'}\ar[r]^{\widetilde{\varphi}}\ar[d]&\widetilde{X}\ar[r]\ar[d]&\mathrm{Spec}\Omega\ar[d]\\X'\ar[r]^{\varphi}&X\ar[r]^f&S}$$
    	    
    	    由于$\varphi$是有限平展覆盖,并且$f:X\to S$是紧合态射和separable态射(后者是因为$f_*\mathscr{O}_X=\mathscr{O}_S$和Zariski连通性定理),我们的引理说明$g=f\circ\varphi:X'\to S$是紧合态射和separable态射.取这个态射的Stein分解为$\xymatrix{X'\ar[r]^{g'}&S'\ar[r]^p&S}$,那么我们的引理说明$p$是有限平展覆盖.于是按照纤维积的泛性质,存在虚线态射$\alpha$使得如下图表交换:
    	    $$\xymatrix{X'\ar@{-->}[dr]\ar@/^1pc/[drr]^{\varphi}\ar@/_1pc/[ddr]_{g'}&&\\&X''\ar[r]^{p_X}\ar[d]_{f'}&X\ar[d]^f\\&S'\ar[r]^p&S}$$
    	    
    	    于是最后只剩下证明这个$\alpha$是一个同构.我们之前解释过只要证明这里$\alpha$是有限平展覆盖,$X''$是连通的,并且$r(\alpha)=1$就能得到$\alpha$是同构.
    	    \begin{itemize}
    	    	\item 先证明$\alpha$是一个有限平展覆盖:$p_X$作为有限平展覆盖$p$的基变换仍然是有限平展覆盖(按照引理这里$X$是连通的).于是从$\varphi=p_X\circ\alpha$和$p_X$都是有限平展覆盖得到$\alpha$也是有限平展覆盖.
    	    	\item $X''$是连通的:我们在证明$p$是满射中已经得到了基变换函子$f^*$把连通对象映射为连通对象,于是这里从$S'$是连通的就得到$X''$是连通的.而$S'$连通是因为按照Zariski连通性定理,有$g':X'\to S'$是满射,于是从$X'$是连通的就得到满射像$S'$是连通的.
    	    	\item 证明$r(\alpha)=1$:由于这里$r(\alpha)$是$S$上的常值函数,所以只要证明存在一个点$x''\in X''$满足$r_{x''}(\alpha)=1$即可.我们知道做基变换$\mathrm{Spec}\Omega\to S$不改变$\alpha$的秩,所以我们可以把所有东西都做关于这个几何点的基变换.把做基变换后的概形或者态射都记作$\widetilde{\bullet}$.只要证明$r(\widetilde{\varphi})$在某个点是1即可.
    	    	
    	    	\qquad
    	    	
    	    	我们接下来是思路是这样的,由于$\widetilde{\alpha}$是有限平展态射,它把$\widetilde{X'}$的连通分支映射为$\widetilde{X''}$的连通分支.另外由于$\alpha:X'\to X''$的源端和终端都是连通的,所以$\alpha(X')$是$X''$的既开又闭子集,于是得到$\alpha$是满射.于是$\widetilde{\alpha}$恰好是$\widetilde{X'}$的全部连通分支集合到$\widetilde{X''}$的全部连通分支集合的满射.一旦我们证明了二者的连通分支个数相同,那么$\widetilde{\alpha}$就是它们连通分支之间的双射.接下来我们证明$\widetilde{\alpha}$限制在$\widetilde{X'}$的某个连通分支上是到$\widetilde{X''}$的某个连通分支$\widetilde{X''}_0$的同构,这就导致$\widetilde{X''}_0$上的元素的秩为1,于是$r(\widetilde{\alpha})$在某个点的秩是1,就导致$r(\alpha)$在某个点的秩是1,从而得证.
    	    	
    	    	\qquad
    	    	
    	    	我们来证明$\widetilde{X'}$和$\widetilde{X''}$的连通分支个数相同.由于$\widetilde{S'}\to\mathrm{Spec}\Omega$是终端为代数闭域的有限平展覆盖,于是有$\widetilde{S'}=\coprod_{1\le i\le n}\widetilde{S}_i$,其中$\widetilde{S}_i=\widetilde{S}=\mathrm{Spec}\Omega$.于是有$\widetilde{X''}=\widetilde{X}\times_{\widetilde{S}}\widetilde{S'}=\coprod_{1\le i\le n}\widetilde{X}_i$,其中$\widetilde{X}_i=X$.另一方面Zariski连通性定理(的推论)告诉我们$\widetilde{X'}$的连通分支个数恰好是$|S'\times_S\mathrm{Spec}\Omega|=|\widetilde{S'}|=n$.
    	    	
    	    	\qquad
    	    	
    	    	最后证明存在$\widetilde{X'}$的某个连通分支,使得$\widetilde{\alpha}$在其上的限制是到$\widetilde{X''}$的某个连通分支的同构.回顾态射$\sigma:\widetilde{X}\to\widetilde{X'}$满足$\widetilde{\varphi}\circ\sigma=1_{\widetilde{X}}$,于是从$1_{\widetilde{X}}$和$\widetilde{\varphi}$都是有限平展覆盖得到$\sigma$也是有限平展覆盖.按照$\widetilde{X}$是连通的(因为$f:X\to S$的纤维态射是几何连通的),于是$Z=\sigma(\widetilde{X})$是$\widetilde{X'}$的既开又闭子集而且是连通的,于是$Z$是$\widetilde{X'}$的连通分支.类似的有$\widetilde{\alpha}(Z)$是$\widetilde{X}''$的连通分支,于是它恰好是某个$\widetilde{X}_i$.由于$p_X$是完全分裂的,这里$\widetilde{p_X}$限制为$\widetilde{X}_i$到$\widetilde{X}$的恒等映射,所以我们干脆就设这个$\widetilde{X}_i=\widetilde{X}$.最后我们来说明$\widetilde{\alpha}\mid_Z:Z\to\widetilde{X}_i$和$\sigma\circ\widetilde{p_X}\mid_{\widetilde{X}}:\widetilde{X}_i\to Z$互为逆映射.这件事归结为证明$\sigma$诱导了$\widetilde{X}$到$Z$的同构,而这是因为,$\sigma$视为$\widetilde{X}\to Z$的态射(严格说是一个态射$f:X\to Y$的像集如果落在$Y$的开子集$U$中,那么$f$肯定可以经开嵌入$U\to Y$唯一分解)是同胚闭嵌入(因为它是截面态射,而且是满射),是平坦的.所以局部上它是$A\to A/I$诱导的,但是同胚和诺特条件保证$I$是幂零理想,平坦保证$I$是幂等理想,于是$I=0$.
    	    \end{itemize}
    	\end{enumerate}
    \end{proof}
    \item 如果我们去掉条件$f_*\mathscr{O}_X=\mathscr{O}_S$(这个条件主要是结合Zariski连通性定理得到$X_{\overline{s}}$是连通的)和$S$连通的条件,只要求$f:X\to S$是紧合态射和separable态射,以及$S$是局部诺特概形,以及$f$是满射(这个是确保诱导的连通分支之间的映射是满射),那么第一同伦列变为如下形式,其中$\pi_0(S,\overline{s})$表示$S$的全部连通分支构成的带基点集合,其中基点取为几何点$\overline{s}$的像所在的连通分支.另外这里我们约定在$S$非连通的时候,选取一个关于点$s\in S$的几何点$\overline{s}$,那么$\pi_1(S,\overline{s})$理解为$\pi_1(S_0,\overline{s})$,其中$S_0$是$s\in S$所在的连通分支(局部诺特概形是局部连通的,所以它的连通分支一定是开子集,概形结构就取开子概型).
    $$\xymatrix{\pi_1(X_{\overline{s}},\overline{\eta})\ar[r]^i&\pi_1(X,\eta)\ar[r]^p&\pi_1(S,\overline{s})\ar[r]&\pi_0(X_{\overline{s}},\overline{\eta})\ar[r]&\pi_0(X,\eta)\ar[r]&\pi_0(S,\overline{s})\ar[r]&1}$$
    \begin{proof}
    	
    	先设$X,S$都是连通概形,也即只去掉条件$f_*\mathscr{O}_X=\mathscr{O}_S$,那么此时$X_{\overline{\eta}}$未必是连通的.取$f$的Stein分解为$f=p\circ f'$,其中$p:S'=\mathrm{Spec}(f_*\mathscr{O}_X)\to S$是有限平展覆盖,$f':X\to S'$仍然是紧合态射.此时$f'$满足$f'_*\mathscr{O}_X=\mathscr{O}_S$.另外这里$X_{\overline{\eta}}$的连通分支恰好是全部$X_{\overline{s_i}'}$,其中$\overline{s_i}':\mathrm{Spec}\Omega\to S'_{\overline{s}}$是一个几何点,它们也恰好是$S'_{\overline{s}}$的全部点(这里$S'_{\overline{s}}$的几何点恰好都是实际的点,因为$S'\to S$是有限态射,导致$S'_{\overline{s}}$上点的剩余域都是$\Omega$).所以如果记几何点$\eta:\mathrm{Spec}\Omega\to X$在$X\to S'$下的像是$\overline{s'}$,那么$\pi_1(X_{\overline{s}},\overline{\eta})$应该理解为$\pi_1(X_{\overline{s'}},\overline{\eta})$.于是把上面版本的第一同伦列用在$f'$上,就得到正合列:
    	$$\xymatrix{\pi_1(X_{\overline{s}},\overline{\eta})\ar[r]&\pi_1()&&}$$
    	
    	由于$S'\to S$是有限平展覆盖,导致$\textbf{FEt}(S')$是$\textbf{FEt}(S)$的由$S'\to S$对象和态射构成的完全子范畴,我们解释过此时$\pi_1(S',\overline{s'})$就可以典范的视为$\pi_1(S,\overline{s})$的开子群,进而有左陪集集合$\pi_1(S,\overline{s})/\pi_1(S',\overline{s'})$同构于$|S'_{\overline{s}}|$,也即$X_{\overline{s}}$的全部连通分支,于是我们得到带基点集合的同构$\pi_1(S,\overline{s})/\pi_1(S',\overline{s'})\cong\pi_0(X_{\overline{s}},\overline{\eta})$,综上得到正合列:
    	$$\xymatrix{\pi_1(X_{\overline{s}},\overline{\eta})\ar[r]&\pi_1(X,\eta)\ar[r]&\pi_1(S,\overline{s})\ar[r]&\pi_0(X_{\overline{s}},\overline{\eta})\ar[r]&1}$$
    	
    	如果我们再去掉$X,Y$连通的条件,那么仍然有正合列:
    	$$\xymatrix{\pi_1(X_{\overline{s}},\overline{\eta})\ar[r]&\pi_1(X,\eta)\ar[r]&\pi_1(S,\overline{s})\ar[r]&\pi_0(X_{\overline{s}},\overline{\eta})}$$
    	
    	此时$\pi_1(S,\overline{s})/\pi_1(S',\overline{s'})$不再是$S'$的所有$s$上的几何点,而是$S'_0=\mathrm{Spec}f_*\mathscr{O}_X\mid_{S_0}$上的全部$s$上的几何点,其中$S_0$是$s$所在的连通分支.进而我们有如下正合列:
    	$$\xymatrix{\pi_1(S,\overline{s})\ar[r]&\pi_0(X_{\overline{s}},\overline{\eta})\ar[r]&\pi_0(X,\eta)\ar[r]&\pi_0(S,\overline{s})}$$
    \end{proof}
    \item (K\"unneth公式)设$k$是代数闭域,设$X,Y$是$k$上是局部诺特连通概形,其中$X$还是紧合$k$概形,我们断言$X\times_kY$是连通概形.选取两个几何点$\overline{x}:\mathrm{Spec}\Omega\to X$和$\overline{y}:\mathrm{Spec}\Omega\to Y$,其中$\Omega$是一个包含了$k$的代数闭域,它们诱导了唯一的$X\times_kY$上的几何点记作$(\overline{x},\overline{y})$.我们断言被$p_X:X\times_kY$和$p_Y:X\times_kY\to Y$诱导的射影有限群之间的如下同态是一个同构:
    $$\pi_1(X\times_kY,(\overline{x},\overline{y}))\to\pi_1(X,\overline{x})\times\pi_1(Y,\overline{y})$$
    \begin{proof}
    	
    	一般的,一个泛同胚的整态射$f:X\to Y$所诱导的$f^*:\textbf{FEt}(Y)\to\textbf{FEt}(X)$是范畴等价,从而诱导了基本群同构【似乎要下降理论证明】.这里$X_{\mathrm{red}}\to X$和基变换的$X_{\mathrm{red}}\times_kY\to X\times_kY$都是泛同胚的整态射,于是不妨设$X$本身是既约概形.
    	$$\xymatrix{\mathrm{Spec}k\ar@{=}@/^1pc/[drr]\ar@/_1pc/[ddr]_{(\overline{x},\overline{y})}\ar[dr]^{\overline{x}}&&\\&X\ar[r]\ar[d]_{\mathrm{id}\times\overline{y}}&\mathrm{Spec}k\ar[d]^{\overline{y}}\\&X\times_kY\ar[r]^{p_Y}&Y}$$
    	
    	先设$\Omega=k$.此时$X\times_kY\to Y$是紧合态射和平坦态射,它关于几何点$\overline{y}$的基变换是$X\times_kY\times_Yk=X$是几何连通的也是几何既约的(代数闭域上概形的连通或者既约等价于几何连通或者几何既约),它对应的几何点是$\overline{x}$(见上图),于是对$p_Y$用第一同伦列得到:
    	$$\xymatrix{\pi_1(X,\overline{x})\ar[r]^{\pi_1(\mathrm{id}_X\times\overline{y})}&\pi_1(X\times_kY)\ar[r]^{\pi_1(p_Y)}\ar[r]&\pi_1(Y,\overline{y})\ar[r]&1}$$
    	
    	但是这里$\xymatrix{X\ar[r]^{\mathrm{id}\times\overline{y}}&X\times_kY\ar[r]^{p_X}&X}$是同构,所以$\pi_1(p_X)\circ\pi_1(\mathrm{id}\times\overline{y})$也是基本群$\pi_1(X,\overline{x})$上的同构,这迫使$\pi_1(\mathrm{id}\times\overline{y})$是单射,并且下述短正合列分裂,从而得到$\pi_1(X\times_kY,(\overline{x},\overline{y}))\cong\pi_1(X,\overline{x})\times\pi_1(Y,\overline{y})$.
    	$$\xymatrix{1\ar[r]&\pi_1(X,\overline{x})\ar[r]^{\pi_1(\mathrm{id}_X\times\overline{y})}&\pi_1(X\times_kY)\ar[r]^{\pi_1(p_Y)}\ar[r]&\pi_1(Y,\overline{y})\ar[r]&1}$$
    	
    	最后如果$\Omega$是任意的包含$k$的代数闭域,那么上面的结论证明了如果取$\overline{x},\overline{y}$和$(\overline{x},\overline{y})$在$X\times_k\Omega$,$Y\times_k\Omega$和$X\times_k\times_k\Omega$中合适的提升的几何点$\overline{x'}$和$\overline{y'}$,那么有$\pi_1(X\times_kY\times_k\Omega,(\overline{x'},\overline{y'}))\cong\pi_1(X\times_k\Omega,\overline{x'})\times\pi_1(Y\times_k\Omega,\overline{y'})$.于是问题转化为【下面结论】.
    \end{proof}
    \item 引理.有限平展态射和逆向极限的兼容性.取诺特概形$S_0$,取$S_0$上拟凝聚代数层构成的正向系统$\{\mathscr{A}_i\}$,记$S_i=\mathrm{Spec}\mathscr{A}_i$,那么$\{S_i\}$构成$S_0$概形上的逆向系统,如果记$\mathscr{A}=\varinjlim\mathscr{A}_i$,那么我们解释过$S=\mathrm{Spec}\mathscr{A}$就是$\{S_i\}$的逆向极限.现在取一个$S_0$态射$X_0\to Y_0$,约定$X_0,Y_0$都是有限表示$S_0$概形,记$X_i=X_0\times_{S_0}S_i$和$Y_i=Y_0\times_{S_0}S_i$,再记$f_i=f_0\times1_{S_i}$,同理定义$X=X_0\times_{S_0}S$,$Y=Y_0\times_{S_0}S$和$f=f_0\times1_S$.如果$Y_i,Y$都是局部诺特概形,如果$f:X\to Y$是有限平展覆盖,那么存在指标$i$使得$i\le j$的时候总有$f_j$是有限平展覆盖.
    \item 设$k$是一个代数闭域,设$X$是$k$上的紧合且连通的概形,取一个包含$k$的代数闭域$\Omega$,固定一个几何点$\overline{x}:\mathrm{Spec}\Omega\to X_{\Omega}=X\times_k\Omega$,记态射$\mathrm{Spec}\Omega\to X_{\Omega}\to X$的像是$x$(这是$X$上的一个真正的点).我们断言$X_{\Omega}\to X$诱导的基本群同态是一个同构:
    $$\pi_1(X_{\Omega},\overline{x})\cong\pi_1(X,x)$$
    \begin{proof}
    	
    	满射:取一个有限平展覆盖$\varphi:Y\to X$,只需证明$Y\times_k\Omega$是连通的,这是因为代数闭域上的连通概形自动是几何连通的.单射:归结为任取连通的有限平展覆盖$\varphi':Y'\to X_{\Omega}$,证明$Y'$是被某个$X$上的有限平展覆盖$Y\to X$基变换.
    	
    	\qquad
    	
    	$\Omega$可以写成它的全体有限表示$k$子代数的正向极限,于是按照有限平展覆盖和逆向系统的兼容性,就可以找到$\Omega$的一个有限表示$k$子代数$R$和一个有限平展覆盖$\varphi_R:Y_R\to X_R$,使得它的基变换恰好就是预先取定的$\varphi':Y'\to X_{\Omega}$.并且由于$Y'$是连通的,得到$Y_R$也是连通的(这里$Y_R$连通是因为,取它的有限连通分支分解,按照概形上纤维积和无交并可交换,归结为说明$Y_R$的连通分支做$\mathrm{Spec}\Omega\to\mathrm{Spec}R$的基变换不会是空集,而这是因为$\mathrm{Spec}\Omega\to\mathrm{Spec}R$把唯一点$x_1$映成零理想,而$Y_R\to\mathrm{Spec}R$如果限制到某个连通分支上就还是一个满射,于是这个连通分支也存在一个点$x_2$映成$R$的零理想,于是这个连通分支和$\mathrm{Spec}\Omega$在$R$上的纤维积中就存在一个点分别在这个连通分支和$\mathrm{Spec}\Omega$中的像是$x_2$和$x_1$).进而$Y_R$是几何连通的,所以如果记$\mathrm{Spec}R$的唯一一般点(唯一因为$R$是整环)为$\eta$,那么$Y_{\kappa(\eta)}$也是连通的.
    	
    	\qquad
    	
    	记$S=\mathrm{Spec}R$,这是代数闭域$k$上的一个有限型概形,所以它一定存在闭点$s$,并且$s$的剩余域一定是$k$本身.于是按照K\"unneth公式就有:
    	$$\pi_1(X_R,(x,z))\cong\pi_1(X,x)\times\pi_1(S,s)$$
    	
    	记连通的有限平展覆盖$\varphi_R:Y_R\to X_R$对应于$\pi_1(X_R,(x,s))$的开子群$U$,那么按照上述同构,存在$\pi_1(X,x)$和$\pi_1(S,s)$分别的开子群$U_X$和$U_S$满足$U_X\times U_S\subseteq U$.分别记$U_X$和$U_S$对应的连通有限平展覆盖为$\psi_X:\widetilde{X}\to X$和$\psi_S:\widetilde{S}\to S$.严格说这是指有$\pi_1(X,x)$集合同构$F_x(\psi_X)\cong\pi_1(X,x)/U_X$和$\pi_1(S,s)$集合同构$F_s(\psi_S)\cong\pi_1(S,s)/U_S$.于是$F_{(x,s)}(\psi_X\times\psi_S)=F_x(\psi_X)\times F_s(\psi_S)\cong\pi_1(X,x)/U_X\times\pi_1(S,s)/U_S\cong\pi_1(X\times_kS,(x,s))/U_X\times U_S$,于是$\psi_X\times\psi_S:\widetilde{X}\times_k\widetilde{S}\to X\times_kS$对应于$\pi_1(X\times_kS,(x,s))$的开子群$U_X\times U_S$.再按照$U_X\times U_S\subseteq U$,于是$\psi_X\times\psi_S$要经$\varphi_R$分解,于是纤维积的泛性质保证了有如下虚线态射使得图表交换:
    	$$\xymatrix{\widetilde{X}\times_k\widetilde{S}\ar@/^1pc/[drr]\ar@/_1pc/[ddr]\ar@{-->}[dr]&&\\&\widetilde{Y_R}\ar[r]\ar[d]&Y_R\ar[d]\\&X\times_k\widetilde{S}\ar[r]&X\times_kS}$$
    	
    	我们来证明$\widetilde{Y_R}$是连通的.按照$\kappa(\eta)\subseteq\Omega$,就有$\eta$在$\widetilde{S}$中的任意提升(原像)$\widetilde{\eta}$有$\kappa(\widetilde{\eta})\subseteq\Omega$.对有限平展覆盖$\widetilde{Y_R}\to X\times_k\widetilde{S}$做基变换$X\times_k\widetilde{S}\times_{\widetilde{S}}\times\Omega\to X\times_k\widetilde{S}$,就得到有限平展覆盖$Y'\Omega\to X_{\Omega}$(见下图),这里$Y'$是连通的导致$\widetilde{Y_R}$是连通的(和前面的理由一样,因为概形上纤维积和无交并可交换,并且一般的如果$x\in X$和$y\in Y$在$X\to S$和$Y\to S$的像相同,那么存在$X\times_SY$中的点$z$在$X$和$Y$中的像分别是$x$和$y$).另外按照$\widetilde{X}\times_k\widetilde{S}\to\widetilde{Y_R}\to X\times_k\widetilde{S}$和$\widetilde{Y_R}\to X\times_k\widetilde{S}$都是有限平展覆盖,得到$\widetilde{X}\times_k\widetilde{S}\to\widetilde{Y_R}$也是有限平展覆盖,于是$\widetilde{Y_R}\to X\times_k\widetilde{S}$对应于$\pi_1(X\times_k\widetilde{S})=\pi_1(X,x)\times U_S$的开子群$V$,并且它包含了$U_X\times U_S$.于是$V$具有形式$V=U\times U_S$,其中$U$是满足$U_X\subseteq U\subseteq\pi_1(X)$的开子群.这说明$\widetilde{Y_R}\to X\times_k\widetilde{S}$具有形式$\widetilde{Y}\times_k\widetilde{S}\to X\times_k\widetilde{S}$,其中$\widetilde{Y}\to X$是有限平展覆盖.
    	$$\xymatrix{Y'\ar[r]\ar[d]&\widetilde{Y_R}\ar[r]\ar[d]&Y_R\ar[d]\\X_{\Omega}\ar[r]&X\times_k\widetilde{S}\ar[r]&X_R}$$
    \end{proof}
\end{enumerate}
\subsection{阿贝尔簇的基本群}
\begin{enumerate}
	\item 【】回顾阿贝尔簇.域$k$上的阿贝尔簇$A$指的是一个连通紧合$k$群概形.
	\begin{enumerate}[(1)]
		\item 域$k$上的阿贝尔簇总是交换的群概形,并且总是射影概形.
		\item 设$n$是整数,用$[n_A]$表示$A(k)$上数乘$n$的态射,它的核记作$A[n](k)$.如果$n$和$\mathrm{char}k=p$互素,那么$A[n](k)$是秩为$2g$的自由$\mathbb{Z}/n\mathbb{Z}$模,其中$g=\dim A$;如果$n=p$,那么$A[p](k)$是一个自由$\mathbb{Z}/p\mathbb{Z}$模,它的秩$\gamma$称为阿贝尔簇$A$的$p$-秩,可以证明$0\le\gamma\le g$,并且可以证明对任意$0\le\gamma\le g$,都存在维数是$g$的域$k$上的阿贝尔簇使得$p$-秩恰好为$\gamma$.
		\item 对任意素数$\ell$,定义$A$的$\ell$-adic Tate模为$\mathrm{T}_{\ell}(A)=\varprojlim_nA[\ell^n](k)$,这是一个$\mathbb{Z}_{\ell}$模.如果$\ell\not=\mathrm{char}(k)=p$,那么$\mathrm{T}_{\ell}(A)$的秩是$2g$;如果$\ell=p$,那么$\mathrm{T}_p(A)$的秩是$A$的$p$-秩$\gamma$.
		\item 对任意有限子群$N\subseteq A(k)$,它左平移作用在$A$上,由于$A$是$k$上的射影概形,导致商概形$B=A/N$总存在,并且此时$B$自然的具备一个阿贝尔簇结构,使得典范投影态射$A\to B$是一个$k$群概形态射.
	\end{enumerate}
    \item 设$A$是代数闭域$k$上的阿贝尔簇,设$\overline{x}$是$A$的一个几何点,那么$\pi_1(A,\overline{x})$是交换群.
    \begin{proof}
    	
    	不妨设$\overline{x}$就是零元$0_A$.考虑$A$上的乘法态射$m:A\times_kA\to A$,它诱导了基本群之间的态射:
    	$$\pi_1(m):\pi_1(A\times_kA,(0_A,0_A))\to\pi_1(A,0_A)$$
    	
    	按照K\"unneth公式有典范同构:
    	$$\pi_1(A\times_kA,(0_A,0_A))\cong\pi_1(A,0_A)\times\pi_1(A,0_A)$$
    	
    	于是上述基本群同态变成:
    	$$\pi_1(A,0_A)\times\pi_1(A,0_A)\to\pi_1(A,0_A)$$
    	
    	我们断言这个同态就是$(g_1,g_2)\mapsto g_1g_2$.取态射$i_1=1_A\times0_A:A\to A\times_kA$,那么从K\"unneth公式的证明中得到$\pi_1(i_1):\pi_1(A,0_A)\to\pi_1(A\times_kA,(0_A,0_A))$是单射,并且在同构$\pi_1(A\times_kA,(0_A,0_A))\cong\pi_1(A,0_A)\times\pi_1(A,0_A)$下$\pi_1(i_1)$就是$\pi_1(A,0_A)\to\pi_1(A,0_A)\times\pi_1(A,0_A)$,$g\mapsto(g,1)$.于是从$p_1\circ i_1=1_A$就得到$\pi_1(m)(g,1)=g$,同理考虑$i_2$得到$\pi_1(m)(1,g)=g$,于是就有$\pi_1(m)(g_1,g_2)=\pi_1(m)(g_1,1)\cdot\pi_1(m)(1,g_2)=g_1g_2$.于是得到交换性:
    	\begin{align*}
    		g_1g_2&=\pi_1(m)(g_1,g_2)\\
    		&=\pi_1(m)((g_1,1)(1,g_2))\\
    		&=\pi_1(m)((1,g_2)(g_1,1))\\
    		&=\pi_1(m)(1,g_2)\cdot\pi_1(m)(g_1,1)\\
    		&=g_2g_1
    	\end{align*}
    \end{proof}
    \item (Serre-Lang)设$A$是代数闭域$k$上的阿贝尔簇,设$f:B\to A$是一个连通有限平展覆盖,设$e\in B$是$0_A$的一个提升.那么存在$B$上的一个典范$k$阿贝尔簇结构,使得$e$是零元,并且$f$是阿贝尔簇之间的同态(阿贝尔簇之间的概形态射$f:B\to A$如果满足$m_A\circ(f\times f)=f\circ m_B$就称为阿贝尔簇之间的同态,可以证明阿贝尔簇之间的态射一定是同态的一个平移,也即保幺元的态射一定是同态【Mumford.P41】).
    \begin{proof}
    	
    	设连通有限平展覆盖$f:(B,e)\to(A,0_A)$对应于$\pi_1(A,0_A)$的开子群$U$,那么$F_{0_A}(B)$同构于$\pi_1(A,0_A)$集合$M=\pi_1(A,0_A)/U$.由于$\pi_1(A,0_A)$是阿贝尔群,所以$U$是正规子群,所以$\mu:M\times M\to M$,$(g_1U,g_2U)\mapsto g_1g_2U$是定义良性的.记$A$上的乘法为$m:A\times_kA\to A$,那么$M$可以经$\pi_1(m)$视为一个$\pi_1(A,0_A)\times\pi_1(A,0_A)\cong\pi_1(A\times_kA,(0_A,0_A))$集合,此时它对应于$A\times_kA$上的有限平展覆盖$B\times_{f,A,m}(A\times_kA)$.于是上述$\mu$是一个$\pi_1(A,0_A)\times\pi_1(A,0_A)$同态,它对应于连通有限平展覆盖之间的态射$\mu_B:B\times_kB\to B\times_{f,A,m}(A\times_kA)$,按照纤维积泛性质这等同于一个态射$m_B:B\times_kB\to B$使得如下实线图表交换.类似的从$M\to M$,$gU\mapsto g^{-1}U$可以定义出一个$i_B:B\to B$.最后验证$(B,m_B,i_B,e)$构成一个代数群,结合$B$连通和$B\to A$紧合就得到$B$是$k$上的阿贝尔簇.
    	$$\xymatrix{B\times_kB\ar@/^1pc/[drr]^{m_B}\ar@/_1pc/[ddr]_{f\times f}\ar@{-->}[dr]^{\mu_B}&&\\&B\times_{f,A,m}(A\times_kA)\ar[r]\ar[d]&B\ar[d]^f\\&A\times_kA\ar[r]^m&A}$$
    \end{proof}
    \item 引理.设$f:B\to A$是$k$阿贝尔簇之间的有限平展同源,那么$N=\ker(f)$是有限集,记$|N|=n$,我们断言$[n_A]:A\to A$一定经$f$分解(此为对偶同源):
    $$\xymatrix{B\ar[rr]^f&&A\\&&A\ar[u]_{[n_A]}\ar[ull]^g}$$
    \begin{proof}
    	
    	我们有$N\subseteq B[n](k)$,于是$[n_B]:B\to B$就要经$B\to B/N\cong A$分解:
    	$$\xymatrix{B\ar[rr]^f\ar[d]_{[n_B]}&&A\ar[dll]^g\\B&&}$$
    	
    	于是有$f\circ g\circ f=f\circ[n_B]=[n_A]\circ f$,按照$f$是满射就得到$f\circ g=[n_A]$.
    \end{proof}
    \item 设$A$是代数闭域$k$上的阿贝尔簇,我们有如下典范同构:
    $$\pi_1(A,0_A)\cong\varprojlim_nA[n](k)\cong\prod_{\ell}\mathrm{T}_{\ell}(A)$$
    
    于是特别的,$\pi_1(A,0_A)$被$A$的维数和$p$-秩完全决定.
    \begin{proof}
    	
    	一般的对阿贝尔簇之间的同源$g:A\to B$,它可以分解为两个同源的复合$A\to A/(\ker g)^0\to B$,其中$(\ker g)^0$表示$\ker g$的幺元所在的连通分支,这是一个既开又闭子群(它是子群是因为:设$G$是$k$群概形,设$G^0$是幺元所在的连通分支,任取$a,b\in G^0$,那么$ab^{-1}G^0$和$G^0$具有公共元$a$,于是它们的并也是连通子集,于是$ab^{-1}G^0\subseteq G^0$,特别的$ab^{-1}\in G^0$).其中$A\to A/(\ker g)^0$是一个纯不可分同源(一个同源是纯不可分的等价于核是一个连通子群),而$A/(\ker g)^0\to B$是一个可分同源也即有限平展同源(一个同源是可分的等价于核是平展子群,这里的核是$\ker g/(\ker g)^0$,一般的对一个有限$k$群概形$G$,有$G/G^0\cong\overline{\omega_0}(G)$是平展$k$群概形,它是连通分支构成的群).
    	
    	\qquad
    	
    	下面取同源$[n_A]:A\to A$,它的核记作$A[n]$,那么$[n_A]$可以分解为$A\to A/A[n]^0\to A$,其中有限平展同源$A/A[n]^0\to A$记作$f_n$.我们断言这些$\{f_n\}$是$\textbf{FEt}(A)$的共尾对象集:任取连通有限平展覆盖$f:B\to A$,那么$B$上存在$k$上阿贝尔簇结构使得$f$是同源,如果记$n=|\ker f|$,那么引理告诉我们$[n_A]$经$f$分解,分解得到的对偶同源记作$g:B\to A$.把$B$和$[n_A]$分别做上一段的可分纯不可分分解,按照同源的可分纯不可分分解在同构下唯一,就得到$f_n$经$f$分解.
    	
    	\qquad
    	
    	最后如果$p\not\mid n$,那么$\mathrm{Aut}(f_n)\cong A[n](k)$(或者证明$f_n$是Galois覆盖)【?】;如果$p\mid n$,纯不可分的部分不改变自同构,于是仍然有$\mathrm{Aut}(f_n)\cong A[n](k)$,综上得到:
    	$$\pi_1(A,0_A)\cong\varprojlim_n\mathrm{Aut}_A(n_A)=\varprojlim_nA[n](k)\cong\prod_{\ell}\mathrm{T}_{\ell}(A)$$
    \end{proof}
    \item 例子.如果$k=\mathbb{C}$,此时复阿贝尔簇具有形式$A=\mathbb{C}^g/\Lambda$,其中$\Lambda\subseteq\mathbb{C}^g$是格(lattice),此时$\mathbb{C}^g\to A$是万有覆盖,于是拓扑基本群为$\pi_1^{\mathrm{top}}(A(\mathbb{C}),0_A)\cong\Lambda$.另一方面,我们有:
    \begin{align*}
    	\mathrm{T}_{\ell}(A)&=\varprojlim_AA[\ell^n](\mathbb{C})\\
    	&=\varprojlim_n\frac{1}{\ell^n}\Lambda/\Lambda\\
    	&=\varprojlim_n\Lambda/\ell^n\Lambda\\
    	&=\Lambda^{(\ell)}
    \end{align*}

    其中$\Lambda^{\ell}$表示$\Lambda$的$p$-射影完备化,于是有:
    $$\pi_1(A,0_A)=\prod_{\ell}\mathrm{T}_{\ell}(A)=\prod_{\ell}\pi_1^{\mathrm{top}}(A(\mathbb{C}),0_A)^{(\ell)}=\widehat{\pi_1^{\mathrm{top}}(A(\mathbb{C}),0_A)}$$
\end{enumerate}
\subsection{几何连通有限型概形的基本群}

【Arithmetic of étale fundamental groups-jltong】

\subsection{第二同伦列}
\begin{enumerate}
	\item 【】回顾形式几何中的比较定理和存在性定理.
	\begin{enumerate}[(1)]
		\item 设$S$是诺特概形,设$p:X\to S$是紧合态射.设$\mathscr{I}\subseteq\mathscr{O}_S$是凝聚理想层,那么凝聚理想层链$0\subseteq\cdots\subseteq\mathscr{I}^{n+1}\subseteq\mathscr{I}^n\subseteq\cdots\subseteq\mathscr{I}$就对应于闭子概型链$S_0\to S_1\to\cdots\to S_n\to\cdots\to S$.取如下基变换$X_i$:
		$$\xymatrix{X_0\ar[r]\ar[d]_{p_0}&X_1\ar[r]\ar[d]_{p_1}&\cdots\ar[r]&X_n\ar[r]\ar[d]_{p_n}&X\ar[d]_{p}\\S_0\ar[r]&S_1\ar[r]&\cdots\ar[r]&S_n\ar[r]&S}$$
		
		对$n\ge0$记闭嵌入$i_n:X_n\to X$.对任意凝聚$\mathscr{O}_X$模层$\mathscr{F}$,记$i_n^*\mathscr{F}=\mathscr{F}\otimes_{\mathscr{O}_X}\mathscr{O}_{X_n},\forall n\ge0$.于是$\mathscr{F}_n$是一个凝聚$\mathscr{O}_{X_n}$模层.用$\mathrm{R}^qp_*$表示前推函子$p_*$(这总是左正合的)的右导出函子.于是$\mathscr{O}_X$模层态射$\mathscr{F}\to\mathscr{F}_n$就诱导了$\mathscr{O}_S$模层态射:
		$$\mathrm{R}^qp_*\mathscr{F}\to\mathrm{R}^qp_*\mathscr{F}_n,\forall q\ge0$$
		
		进而诱导了$\mathscr{O}_{S_n}$模层态射:
		$$\left(\mathrm{R}^qp_*\mathscr{F}\right)\otimes_{\mathscr{O}_S}\mathscr{O}_{S_n}\to\mathrm{R}^qp_*\mathscr{F}_n,\forall q\ge0$$
		
		进而取逆向极限得到态射:
		$$\varprojlim\left(\mathrm{R}^qp_*\mathscr{F}\right)\otimes_{\mathscr{O}_S}\mathscr{O}_{S_n}\to\varprojlim\mathrm{R}^qp_*\mathscr{F}_n,\forall q\ge0$$
		\item 比较定理【EGAIII4.1.5】是说,上述态射总是一个同构.例如取$S=\mathrm{Spec}A$时$\mathscr{I}$对应于$A$的理想$I$,如果记$A$关于$I$的完备化是$\widehat{A}_I$,那么比较定理就是如下同构:
		$$\mathrm{H}^q(X,\mathscr{F})\otimes_A\widehat{A}_I\cong\varprojlim\mathrm{H}^q(X_n,\mathscr{F}_n),\forall q\ge0$$
		\item 存在性定理【EGAIII5.1.4】.在上述条件下,如果额外要求$S=\mathrm{Spec}A$是仿射的并且$A$是关于$\mathscr{I}=\widetilde{I}$是adic完备的.如果对任意$n\ge0$取一个凝聚$\mathscr{O}_{X_n}$模层$\mathscr{F}_n$,使得对任意$n\ge0$有$\mathscr{F}_{n+1}\otimes_{\mathscr{O}_{X_{n+1}}}\mathscr{O}_{X_n}\cong\mathscr{F}_n$,那么存在一个凝聚$\mathscr{O}_X$模层$\mathscr{F}$,使得对任意$n\ge0$有$\mathscr{F}\otimes_{\mathscr{O}_X}\mathscr{O}_{X_n}\cong\mathscr{F}_n$.
	\end{enumerate}
	\item (第二同伦列)设$S=\mathrm{Spec}A$是一个完备诺特局部环的仿射概形,设$s_0$是$S$的唯一闭点,设$f:X\to S$是紧合态射,取$X$的关于点$x_0$的几何点$\overline{x_0}:\mathrm{Spec}\Omega\to X$,使得它在$f$下的像是关于$s_0$的一个几何点$\overline{s_0}$.记$X_{\overline{s_0}}=X\times_{f,S,\overline{s_0}}\mathrm{Spec}\Omega$,那么有如下短正合列:
	$$\xymatrix{1\ar[r]&\pi_1(X_{\overline{s_0}},\overline{x_0})\ar[r]^{i_0}&\pi_1(X,\overline{x_0})\ar[r]^{p_0}&\pi_1(S,\overline{s_0})\ar[r]&1}$$
	
	并且我买有典范同构$\Gamma_{\kappa(s_0)}\cong\pi_1(S,\overline{s_0})$.特别的如果$x_0\in X(\kappa(s_0))$,那么有典范同构$\pi_1(X_{s_0},\overline{x_0})\cong\pi_1(S,\overline{s_0})$.
	\item 我们先解释下在上述条件下$X$是连通的.事实上我们只需要$f:X\to S=\mathrm{Spec}A$是闭映射,$X$是局部诺特的,$A$是局部环,$X_{\overline{s_0}}=X\times_{f,S,\overline{s_0}}\mathrm{Spec}\Omega$是连通的即可.
	\begin{proof}
		
		设$A$的唯一闭点是$s_0$,按照$\kappa(s_0)$概形$X_{s_0}=X\times_S\kappa(s_0)$到代数闭域上的基变换$X_{\overline{s_0}}$是连通的,于是$X_{s_0}=f^{-1}(s_0)$是连通的,所以它包含在$X$的某个连通分支$U$中.按照$X$局部诺特得到$U$是开集(因为局部诺特保证$X$是局部连通空间),倘若$E=X-U$是非空的,那么$f(E)$是$\mathrm{Spec}A$的闭子集,但是仿射概形的闭子集一定包含闭点,这里$S$只有唯一闭点$s_0$,于是$E$和$f^{-1}(s_0)$的交非空,这矛盾,于是$X=U$是连通的.
	\end{proof}
	\item 先证明第二同伦列在$A$是阿廷局部环时成立.
	\begin{proof}
		
		设$A$是阿廷局部环,那么它的唯一极大理想$\mathfrak{m}$就是它的唯一素理想.记$k=A/\mathfrak{m}$,那么$\mathrm{Spec}k\to\mathrm{Spec}A$是泛同胚也是单态射(泛同胚是因为它是整态射,泛单的和满射,它是单态射因为对角态射是同构).于是$X\times_Ak$和$X$同胚,$X_{\overline{s_0}}=X\times_S\Omega=X\times_k\Omega$.进而把$A,X$替换为$k,X\times_Ak$不改变命题中的三个基本群.特别的,此时$\Gamma_k\cong\pi_1(S,\overline{s_0})$是我们已经证过的结论.记$k$在$\Omega$中的纯不可分闭包为$k^i$,记$X_{s_0}^i=X\times_kk^i$,那么我们有如下纤维积图表:
		$$\xymatrix{X_{\overline{s_0}}\ar[r]\ar@{=}[d]&X_{s_0}^i\ar[r]\ar[d]&\mathrm{Spec}k^i\ar[d]\\X_{\overline{s_0}}\ar[r]&X\ar[r]^f&S}$$
		
		这诱导了基本群的如下交换图表,并且由于上图表中的垂直态射都是泛同胚,所以下图表中的垂直同态都是同构.于是我们要证明的下一行的正合性就等价于证明下一行的正合性,也即设$k$本身是完全域(即在$\Omega$中没有纯不可分扩张).
		$$\xymatrix{\pi_1(X_{\overline{s_0}},\overline{x_0})\ar[r]\ar[d]_{\cong}&\pi_1(X_{s_0}^i,x_0^i)\ar[r]\ar[d]_{\cong}&\pi_1(\mathrm{Spec}k^i,\overline{s_0}^i)\ar[d]_{\cong}\\\pi_1(X_{\overline{s_0}})\ar[r]&\pi_1(X,x_0)\ar[r]&\pi_1(S,\overline{s_0})}$$
		
		记$k$在$\Omega$中的代数闭包是$\overline{k}$,那么按照$k$是完全域就有$k\subseteq\overline{k}$是可分扩张,于是这个扩张可以写成它的全部有限Galois子扩张$k\subseteq k_i,i\in I$的正向极限.仍把$\overline{x_0}$在$X_{k_i}$上的像记作$\overline{x_0}$,记$\overline{x_0}$在$\overline{k}$中的像是$\overline{\eta}$,那么按照$X_{\overline{k}}\cong\varprojlim X_{k_i}$,就得到典范同构$\pi_1(X_{\overline{s_0}},\overline{x_0})\cong\pi_1(X_{\overline{k}},\overline{\eta})\cong\varprojlim\pi_1(X_{k_i},\overline{x_0})$.
		
		\qquad
		
		我们知道$\pi_1(X_{k_i},\overline{x_0})$是$\pi_1(X,\overline{x_0})$的子群,它恰好就是那些固定了几何点$\overline{x_0}$的元素构成的子群,于是有$\pi_1(X,\overline{x_0})/\pi_1(X_{k_i},\overline{x_0})\cong F(X_{\overline{k_i}}\to X)\cong\mathrm{Aut}_k(k_i)$.于是我们有如下短正合列:
		$$\xymatrix{1\ar[r]&\pi_1(X_{k_i},\overline{x_0})\ar[r]&\pi_1(X,\overline{x_0})\ar[r]&\mathrm{Aut}_k(k_i)\ar[r]&1}$$
		
		由于逆向极限函子在射影有限群上是正合函子,对$i$取逆向极限就得到:
		$$\xymatrix{1\ar[r]&\pi_1(X_{\overline{s_0}},\overline{x_0})\ar[r]&\pi_1(X,\overline{x_0})\ar[r]&\Gamma_k\ar[r]&1}$$
	\end{proof}
    \item 在闭嵌入$i_{s_0}:X_{s_0}\to X$诱导的函子$H$是范畴等价$\textbf{FEt}(X)\to\textbf{FEt}(X_{s_0})$(我们前面证明了$X$是连通的),于是这个闭嵌入诱导了基本群的同构$\pi_1(X_{s_0},\overline{x_0})\cong\pi_1(X,\overline{x_0})$.
    \begin{proof}
    	
    	先来证明函子$H$是完全忠实的,换句话讲对任意有限平展覆盖$p:Z\to X$和$p':Z'\to X$,有如下典范双射:
    	$$\mathrm{Hom}_X(Z,Z')\cong\mathrm{Hom}_{X_{s_0}}(Z\times_XX_{s_0},Z'\times_XX_{s_0})$$
    	
    	由于$p,p'$是有限平展态射,于是$\mathscr{A}(Z)$和$\mathscr{A}(Z')$都是$\mathscr{O}_X$上的凝聚局部自由$\mathscr{O}_X$代数层.并且按照$\mathrm{Spec}$的泛性质有:
    	$$\mathrm{Hom}_X(Z,Z')\cong\mathrm{Hom}_{\textbf{Alg}(\mathscr{O}_X)}(\mathscr{A}(Z'),\mathscr{A}(Z))$$
    	
    	记$A$的极大理想为$\mathfrak{m}$,对任意$n\ge0$,记$A_n=A/\mathfrak{m}^{n+1}$,$X_n=X\times_AA_n$,$Z_n=Z\times_XX_n$和$Z_n'=Z'\times_XX_n$.特别的$X_{s_0}=X_0$.那么按照形式几何中的比较定理,就有:
    	\begin{align*}
    		\mathrm{Hom}_{\mathscr{O}_X}&=\Gamma(X,\mathrm{HOM}_{\mathscr{O}_X}(\mathscr{A}(Z'),\mathscr{A}(Z)))\\&\cong\varprojlim_n\Gamma(X_n,\mathrm{HOM}_{\mathscr{O}_X}(\mathscr{A}(Z'),\mathscr{A}(Z))\otimes_AA_n)
    	\end{align*}
    
        按照$\mathscr{A}(Z')$和$\mathscr{A}(Z)$是凝聚局部自由模层,于是:
        $$\mathrm{HOM}_{\mathscr{O}_X}(\mathscr{A}(Z'),\mathscr{A}(Z))\otimes_AA_n\cong\mathrm{HOM}_{\mathscr{O}_{X_n}}(\mathscr{A}(Z_n'),\mathscr{A}(Z_n))$$
        
        综上我们得到如下同构:
        \begin{align*}
        	\mathrm{Hom}_{\mathscr{O}_X}(\mathscr{A}(Z'),\mathscr{A}(Z))&\cong\varprojlim_n\mathrm{Hom}_{\mathscr{O}_{X_n}}(\mathscr{A}(Z_n'),\mathscr{A}(Z_n))\\&\cong\varprojlim_n\mathrm{Hom}_{X_n}(Z_n,Z_n')
        \end{align*}
        
        但是这里$A/I^{n+1}\to A/I$是泛同胚,导致$Z_0\to Z_n$和$Z_0'\to Z_n'$都是泛同胚,于是有:
        \begin{align*}
        	\varprojlim_n\mathrm{Hom}_{X_n}(Z_n,Z_n')&\cong\varprojlim_n\mathrm{Hom}_{X_0}(Z_0,Z_0')\\&\cong\mathrm{Hom}_{X_0}(Z_0,Z_0')
        \end{align*}
        
        再证明$H$是本质满的,换句话讲对任意有限平展覆盖$p_0:Z_0\to X_0$,我们要找到一个有限平展覆盖$p:Z\to X$,使得有$X_0$同构$Z\times_{p,X,i_0}X_0\cong Z_0$.记$\mathscr{B}_0=\mathscr{A}(Z_0)$,那么这是一个凝聚局部自由$\mathscr{O}_{X_0}$代数层.按照$j:X_n\to X_{n-1}$是泛同胚的,于是它诱导的基本群同态是一个同构,于是$j^*:\textbf{FEt}(X_{n-1})\to\textbf{FEt}(X_n)$是范畴等价,于是我们可以归纳构造凝聚局部自由$\mathscr{O}_{X_n}$代数层$\mathscr{B}_n$满足$\mathscr{B}_n\otimes_{\mathscr{O}_{X_n}}\mathscr{O}_{X_{n-1}}\cong\mathscr{B}_{n-1}$.那么按照形式几何中的存在性定理,就有一个凝聚$\mathscr{O}_X$代数层$\mathscr{B}$,使得对任意$n\ge0$都有$\mathscr{B}\otimes_{\mathscr{O}_X}\mathscr{O}_{X_n}\cong\mathscr{B}_n$.取$Z=\mathrm{Spec}\mathscr{B}$,我们断言$g:Z\to X$是有限平展覆盖,并且有$Z\times_XX_0\cong Z_0$.注意由于$\mathscr{B}$是凝聚代数层,已经有$g$是有限态射,于是归结为证明$g$是平展的.
        
        \qquad
        
        证明$Z\to X$是平展态射归结为证明$X_0=f^{-1}(s_0)$的每个点$x_0$都是平展的.这是因为无论平坦还是非分歧都是一个局部性质,所以一旦这成立,我们就证明了存在$X_0$的一个开邻域$U$,使得$g$限制为$g^{-1}(U)\to U$是平展态射.但是$X$的包含$X_0$的开邻域必然是整个$X$(见$X$连通的证明).另外我们还可以归结为验证$X_0$的闭点处都是平展的,因为倘若$X_0$的全部闭点处任取的开邻域不能覆盖整个$X_0$,那么这些开邻域并集的补集是$X_0$的一个非空闭子集,这必然存在闭点(拟紧$T_0$空间存在闭点),矛盾.
        
        \qquad
        
        任取闭点$x_0\in X_0$,这也是$X$的闭点,取它的仿射开邻域$U=\mathrm{Spec}C$使得$\mathscr{B}_0$限制在$U\cap X_0$上是有限自由的.这里$C$是一个$(A,\mathfrak{m})$代数,记$J=\mathfrak{m}C\subseteq C$.对任意$n\ge0$,有$C_n=C/J^{n+1}\cong\Gamma(U\cap X_n,\mathscr{O}_{X_n})$.类似的如果记$D=\Gamma(U,\mathscr{B})$,那么对任意$n\ge0$有$D_n=D/J^{n+1}D=\Gamma(U\cap X_n,\mathscr{B}_n)$.我们刚才设了$D_0=\Gamma(U\cap X_0,\mathscr{B}_0)$是有限自由$C_0=\Gamma(U\cap X_0,\mathscr{O}_{X_0})$模.选取$D_0=D/JD$在$C_0$上的一组基$\overline{d_1},\cdots,\overline{d_r}$,把它们提升到$D$上记作$d_1,\cdots,d_r$.记$L=\oplus_{1\le i\le r}Cd_i$是自由$C$模,取$C$模同态$\varphi:L\to D$为$(c_1d_1,\cdots,c_rd_r)\mapsto\sum_ic_id_i$,于是我们有如下$C$模的正合列:
        $$\xymatrix{0\ar[r]&\ker\varphi\ar[r]&L\ar[r]^{\varphi}&D\ar[r]&\mathrm{coker}\varphi\ar[r]&0}$$
        
        对任意$n\ge0$记$L_n=L/J^{n+1}L$,记$\varphi:L\to D$诱导的典范$C_n$同态$\varphi_n:L_n\to D_n$.我们来证明这些$\varphi_n$都是同构.首先$\varphi_0$是同构是平凡的.下面设$\varphi_0,\cdots,\varphi_{n-1}$都是同构.我们有如下$C_n$模的短正合列:
        $$\xymatrix{0\ar[r]&J^n/J^{n+1}\ar[r]&C_n\ar[r]&C_{n-1}\ar[r]&0}$$
        
        因为我们构造的$\mathscr{B}_n$都是$\mathscr{O}_{X_n}$平坦的,于是这里$D_n$总是平坦$C_n$模.自由模$L_n$当然也总是平坦$C_n$模.于是我们得到如下短正合列之间的同态:
        $$\xymatrix{0\ar[r]&J^n/J^{n+1}\otimes_{C_n}D_n\ar[d]_{\alpha}\ar[r]&D_n\ar[r]\ar[d]_{\varphi_n}&D_{n-1}\ar[r]\ar[d]_{\varphi_{n-1}}&0\\0\ar[r]&J^n/J^{n+1}\otimes_{C_n}D_n\ar[r]&D_n\ar[r]&D_{n-1}\ar[r]&0}$$
        
        因为$A$是完备诺特局部环,于是$A$作为$A_0=A/\mathfrak{m}$代数被$\mathfrak{m}/\mathfrak{m}^2$中有限个元$\{x_1,\cdots,x_s\}\subseteq\mathfrak{m}$生成,于是$\mathfrak{m}^n/\mathfrak{m}^{n+1}$作为$A_0$模被$\underline{x}^i,i\in I$生成,其中$i$是次数和为$n$的重指标.于是我们有$A_0$模同构$\mathfrak{m}^n/\mathfrak{m}^{n+1}\cong\left(A/\mathfrak{m}\right)^{\oplus I}$,这两边都被$\mathfrak{m}$零化,所以这也是$A$模同构.进而得到:
        $$J^n/J^{n+1}\cong\mathfrak{m}^n/\mathfrak{m}^{n+1}\otimes_AC\cong\left(C_0\right)^{\oplus I}$$
        $$J^n/J^{n+1}\otimes_{C_n}D_n\cong\left(D_0\right)^{\oplus I}$$
        $$J^n/J^{n+1}\otimes_{C_n}L_n\cong\left(L_0\right)^{\oplus I}$$
        
        进而上述短正合列的同态中第一个垂直同态和第三个垂直同态都是同构,这就得到$\varphi_n$是同构.完成归纳.如果我们用$\widehat{C},\widehat{L},\widehat{D}$分别表示$C,L,D$的$J$-adic完备化.那么这些$\varphi_n$都是同构告诉我们$\widehat{\varphi}:\widehat{L}\to\widehat{D}$是一个$\widehat{C}$模同构.由于这里$x_0$是$\mathrm{Spec}C$的闭点,它对应于$C$的极大理想$\mathfrak{n}$,我们知道一个诺特环关于极大理想的局部化和完备化是交换的,于是$\widehat{\varphi_{\mathfrak{n}}}=(\widehat{\varphi})_{\mathfrak{n}}$是同构,于是下面正合列告诉我们$\widehat{\ker\varphi_{\mathfrak{n}}}=\widehat{\mathrm{coker}\varphi_{\mathfrak{n}}}$.但是对于诺特局部环$C_{\mathfrak{n}}$,它的有限模$M$总有$M\to\widehat{M}$是单射,于是$\ker\varphi_{\mathfrak{n}}=\mathrm{coker}\varphi_{\mathfrak{n}}=0$,于是我们得到了$\varphi_{\mathfrak{n}}$是同构,于是$D_{\mathfrak{n}}\cong L_{\mathfrak{n}}$是$C_{\mathfrak{n}}$自由模.
        $$\xymatrix{0\ar[r]&\widehat{\ker\varphi_{\mathfrak{n}}}\ar[r]&\widehat{L_{\mathfrak{n}}}\ar[r]^{\widehat{\varphi_{\mathfrak{n}}}}&\widehat{D_{\mathfrak{n}}}\ar[r]&\widehat{\mathrm{coker}\varphi_{\mathfrak{n}}}\ar[r]&0}$$
        
        最后验证$X_0$上的点的非分歧的,但是我们知道$Z_0\to X_0$是非分歧的,而$Z_0=g^{-1}(X_0)$是$Z$的闭子概型,$g:Z\to X$限制在闭子概型上的态射$g\mid_{Z_0}:Z_0\to X_0$在$Z_0$上的点诱导的局部环同态和$Z_0\to X_0$是相同的,于是$Z\to X$在$Z_0$上总是非分歧的.
    \end{proof}
    \item 证明第二同伦列定理.
    \begin{proof}
    	
    	按照上一条,我们有闭嵌入$X_0=X_{s_0}\to X$诱导了基本群同构$\pi_1(X,\overline{x_0})\cong\pi_1(X_{s_0},\overline{x_0})$,再把上一条用在$X=S$上还得到闭嵌入$\{s_0\}\to S$诱导了基本群同构$\pi_1(S,\overline{s_0})\cong\Gamma_{\kappa(s_0)}$.于是我们有如下交换图表,其中第二行是短正合列,这就得到第一行是短正合列.
    	$$\xymatrix{1\ar[r]&\pi_1(X_{\overline{s_0}},\overline{x_0})\ar[r]\ar@{=}[d]&\pi_1(X,\overline{x_0})\ar[r]\ar[d]^{\cong}&\pi_1(S,\overline{s_0})\ar[r]\ar[d]^{\cong}&1\\1\ar[r]&\pi_1(X_{\overline{s_0}})\ar[r]&\pi_1(X_{s_0},\overline{x_0})\ar[r]&\Gamma_{\kappa(s_0)}\ar[r]&1}$$
    \end{proof}
\end{enumerate}
\subsection{特殊化态射}
\begin{enumerate}
	\item 终端是完备诺特局部环的特殊化态射(specialization morphism).设$S=\mathrm{Spec}A$是一个完备诺特局部环的素谱,设$f:X\to S$是一个紧合态射.任取$X$上的两个几何点$\overline{x_0},\overline{x_1}$,它们在$f$下的像分别记作$\overline{s_0},\overline{s_1}$,这是$S$的两个几何点,它们对应的实际的点分别记作$s_0,s_1$,要求$s_0$是$S$的唯一闭点.把$\overline{x_i}$和$\overline{s_i}$按照纤维积的泛性质诱导的$X_{\overline{s_i}}$的几何点仍记作$\overline{x_i}$.设$X_{\overline{s_0}}$是连通的.我们知道改变几何点后基本群只差一个同构,于是存在同构$\alpha,\beta$构成如下图表:
	$$\xymatrix{&\pi_1(X_{\overline{s_1}},\overline{x_1})\ar@{-->}[d]^{\mathrm{sp}}\ar[r]^{i_1}&\pi_1(X,\overline{x_1})\ar[r]^{p_1}\ar[d]^{\beta}&\pi_1(S,\overline{s_1})\ar[r]\ar[d]^{\alpha}&1\\1\ar[r]&\pi_1(X_{\overline{s_0}},\overline{x_0})\ar[r]^{i_0}&\pi_1(X,\overline{x_0})\ar[r]^{p_0}&\pi_1(S,\overline{s_0})\ar[r]&1}$$
	
	其中第一行按照第一同伦列有$\mathrm{im}(i_1)\subseteq\ker p_0$,第二行按照第二同伦列是短正合列.这个图表右边的小方格未必是交换的,但是由于$\alpha\circ p_1$和$p_0\circ\beta$对应的函子是自然同构的,于是这两个基本群同态差一个$\pi_1(S,\overline{s_0})$的内自同构,所以仍然从$\alpha\circ p_1\circ i_1=0$得到$p_0\circ\beta\circ i_1=0$.于是我们可以定义唯一的虚线同态$\mathrm{sp}:\pi_1(X_{\overline{s_1}},\overline{x_1})\to\pi_1(X_{\overline{s_0}},\overline{x_0})$使得左边这个小方格交换,但是这个态射是依赖于同构$\beta$的选取的,如果去掉$\beta$的选取,那么$\mathrm{sp}$就是一个在差一个$\pi_1(X,\overline{s_0})$内自同构意义下唯一的(换句话讲这里定义的态射$\mathrm{sp}$不是唯一的,但是对于不同的$\mathrm{sp}$,总有$i_0\circ\mathrm{sp}$只差一个$\pi_1(X,\overline{s_0})$中元素的内自同构),这个态射称为特殊化态射.
	
	\item 一般情况的特殊化态射.设$S$是局部诺特概形,设$f:X\to S$是一个紧合态射.任取$X$上的两个几何点$\overline{x_0},\overline{x_1}$,它们在$f$下的像分别记作$\overline{s_0},\overline{s_1}$,这是$S$的两个几何点,它们对应的实际的点分别记作$s_0,s_1$,要求$s_0$是$s_1$的特殊化,也即$s_0\in\overline{\{s_1\}}$.把$\overline{x_i}$和$\overline{s_i}$按照纤维积的泛性质诱导的$X_{\overline{s_i}}$的几何点仍记作$\overline{x_i}$.设$X_{\overline{s_0}}$是连通的.此时仍然可以定义特殊化态射.
	
	\qquad
	
	为此我们考虑$S'=\mathrm{Spec}\widehat{\mathscr{O}_{S,s_0}}$,它的唯一闭点记作$\widehat{s_0}$,有$s_1\in\mathrm{Spec}\mathscr{O}_{S,s_0}$,并且由于$\mathrm{Spec}\widehat{\mathscr{O}_{S,s_0}}\to\mathrm{Spec}\mathscr{O}_{S,s_0}$是忠实平坦态射,可取$s_1\in\mathrm{Spec}\mathscr{O}_{S,s_0}$在$S'$中的提升$\widehat{s_1}$.几何点$\overline{s_1}:\mathrm{Spec}\Omega\to S$经$\mathrm{Spec}\mathscr{O}_{S,s_1}$分解,把$\mathrm{Spec}\Omega\to\mathrm{Spec}\mathscr{O}_{S,s_0}$记作$\overline{\eta}$.此时$X\to S$关于$S'\to S$的基变换$X'=X_{\widehat{\mathscr{O}_{S,s_0}}}\to S'$仍然是紧合态射,并且此时$\widehat{s_0}$是$S'$的唯一闭点,并且有$X'_{\eta}=X_{\overline{s_1}}$,于是把上一条的构造用在这个紧合态射上,就得到特殊化态射$\pi_1(X_{\overline{s_1}},\overline{x_1})\to\pi_1(X_{\overline{s_0}},\overline{x_0})$.
	\item 基本群的半连续定理.仍然设$S=\mathrm{Spec}A$是一个完备诺特局部环的素谱,设$f:X\to S$是一个紧合态射.任取$X$上的两个几何点$\overline{x_0},\overline{x_1}$,它们在$f$下的像分别记作$\overline{s_0},\overline{s_1}$,这是$S$的两个几何点,它们对应的实际的点分别记作$s_0,s_1$,满足$s_0$是$s_1$的特殊化.把$\overline{x_i}$和$\overline{s_i}$按照纤维积的泛性质诱导的$X_{\overline{s_i}}$的几何点仍记作$\overline{x_i}$.设$X_{\overline{s_0}}$是连通的.半连续定理就是说在$f$还是separable态射的前提下这里的典范特殊化态射是满射.事实上我们会证明在这些条件下有$f_*\mathscr{O}_X=\mathscr{O}_S$,此时第一同伦列定理告诉我们下面图表的第一行也是正合列,于是此时从$\alpha,\beta$是同构得到这里构造的特殊化态射$\mathrm{sp}$总是满射.
	$$\xymatrix{&\pi_1(X_{\overline{s_1}},\overline{x_1})\ar@{-->}[d]^{\mathrm{sp}}\ar[r]^{i_1}&\pi_1(X,\overline{x_1})\ar[r]^{p_1}\ar[d]^{\beta}&\pi_1(S,\overline{s_1})\ar[r]\ar[d]^{\alpha}&1\\1\ar[r]&\pi_1(X_{\overline{s_0}},\overline{x_0})\ar[r]^{i_0}&\pi_1(X,\overline{x_0})\ar[r]^{p_0}&\pi_1(S,\overline{s_0})\ar[r]&1}$$
	\begin{proof}
		
		我们来证明$f_*\mathscr{O}_X=\mathscr{O}_S$.取紧合态射$f$的Stein分解为$f=p\circ f'$.那么我们之前解释过这里$p:S'\to S$是一个有限平展覆盖.按照Zariski连通性定理,有$X_{\overline{s_0}}$的连通分支恰好一一对应于集合$|S'_{\overline{s_0}}|$中的元素,于是从$X_{\overline{s_0}}$连通就得到$|S'_{\overline{s_0}}|=r_{s_0}(\varphi)=1$.我们解释过如果这里$S$是诺特连通概形(连通是因为局部环的素谱总是连通的),那么一个有限平展覆盖$p:S'\to S$是同构当且仅当常值函数$r(\varphi)=1$.这就得到$S'\cong S$,从而$f_*\mathscr{O}_X=\mathscr{O}_S$.
	\end{proof}
    \item Abhyankar引理.设$R$是一个DVR,商域记作$K$,设$L/K$和$M/K$是两个非分歧的有限Galois扩张,设$A$关于$M/K$的惯性群的阶数$n$整除$A$关于$L/K$的惯性群的阶数$m$.那么对$A^L$的任意极大理想$\mathfrak{m}_L$,都有$LM$在$\mathscr{O}_{\mathfrak{m}_L}^L$上非分歧.
    \begin{proof}
    	
    	【murre.Appendix】
    	
    \end{proof}
    \item 设$S$是局部诺特概形,设$X\to S$是光滑紧合态射,并且所有纤维都是几何连通的.设$s_0,s_1\in S$满足$s_0\in\overline{\{s_1\}}$.取$X$上的几何点$\overline{x_0},\overline{x_1}$使得它们在$f$下的像是$S$上的几何点$\overline{s_0},\overline{s_1}$,并且分别对应于实际点$s_0,s_1$.设$G$是有限群,并且阶数和$\kappa(s_0)$的特征$p$互素.那么对任意射影有限群之间的满同态$\varphi_1:\pi_1(X_{\overline{s_1}})\to G$,它总要经预先取定的特殊化态射$\mathrm{sp}:\pi_1(X_{\overline{s_1}},\overline{x_1})\to\pi_1(X_{\overline{s_0}},\overline{x_0})$分解.换句话讲总存在射影有限群的满同态$\varphi_0:\pi_1(X_{\overline{s_0}},\overline{x_0})\to G$满足$\varphi_0\circ\mathrm{sp}=\varphi_1$.
    \begin{proof}
    	
    	证明思路是这样的,先证明问题可以归结为设$S=\mathrm{Spec}A$,其中$A$是一个完备离散赋值环,并且可设它的剩余域是代数闭域.这允许我们把$\overline{s_0}$就取为实际的点$s_0$.另外按照纯不可分域扩张是泛同胚的,就有$\pi_1(X_{\overline{s_1}})\cong\pi_1(X_{s_1^s})$,其中$s_1^s$是$\overline{s_1}$在$K=\mathrm{Frac}(A)$的可分闭包$K^s$上的几何点.命题中的群同态$\varphi_0$存在当且仅当$\ker\mathrm{sp}\subseteq\ker\varphi_1$,其中$\ker\varphi_0$是$\pi_1(X_{s_1^s},\overline{x_1})$的开子群,前面解释过在$\mathrm{sp}$是满同态的时候,这个包含关系成立当且仅当存在$\pi_1(X_{s_0},\overline{x_0})$的开子集$V$,使得$H_{\mathrm{sp}}(\pi_1(X_{s_0},\overline{x_0})/V)\cong\pi_1(X_{s_1^s},\overline{x_1})$.翻译为覆盖的语言就是说,如果有连通有限平展覆盖$Y'\to X_{s_1^s}$,那么存在连通有限平展覆盖$Y\to X$满足$Y'$就是$Y\to X$关于$X_{s_1^s}\to S$的基变换.我们构造$Y\to X$的方法是自上而下构造如下三个小纤维积图表:
    	$$\xymatrix{Y'\ar[rr]\ar[d]&&X^{K^s}\ar[d]\\{Y'}^L\ar[rr]\ar[d]&&X^L\ar[d]\\Y^L\ar[rr]\ar[d]&&X\times_SS^L\ar[d]\\Y\ar[rr]&&X}$$
    	\begin{itemize}
    		\item 上方纤维积图表:我们知道$K\subseteq K^s$是全部有限子扩张$K\subseteq L$的正向极限,于是$X^{K^s}=X\times_SK^s$是$\{X^L=X\times_SL\}$的逆向极限.前面证明过当$L$足够大时存在有限平展覆盖${Y'}^L\to X^L$构成这个纤维积图表.
    		\item 中间纤维积图表:取$A$在$K\subseteq L$中的正规化为$A^L$,记$S^L=\mathrm{Spec}A^L$,我们会证明当$L$足够大时$X\times_SS^L$在$K(X\times_SS^L)=K(X^L)\to K({Y'}^L)$中的正规化$Y^L$是非分歧的,于是此时$Y^L\to X\times_SS^L$是有限平展覆盖.并且按照正规化的泛性质有这个纤维积图表.
    		\item 下方纤维积图表:我们会证明$X\times_SS^L\to X$诱导的基本群同态总是同构(这里要用到$\kappa(s_1)$是代数闭域),这就导致存在有限平展覆盖$Y\to X$构成这个纤维积图表.
    	\end{itemize}
    	\begin{enumerate}[(1)]
    		\item 先证明问题可以约化到$S=\mathrm{Spec}A$,其中$A$是一个完备离散赋值环,并且可设$A$的剩余域是代数闭域.为此取$s_0=t_0,t_1,\cdots,t_r=s_1\in S$,满足$t_i\in\overline{\{t_{i+1}\}}$,并且$\mathscr{O}_{\overline{t_{i+1}},t_i}$是一维的(闭子集$\overline{t_i}$上的闭子概型结构就取既约的,特别的此时$\mathscr{O}_{\overline{t_{i+1}},t_i}$是整环),这总可以在有限步实现是因为所有$t_i$都是在$\mathrm{Spec}\mathscr{O}_{X,s_0}$中选取的,而这是诺特局部环的素谱所以维数有限.于是我们有如下特殊化态射链,这就允许我们不妨约定$\mathscr{O}_{\overline{s_1},s_0}$是一维的.接下来取$\mathscr{O}_{\overline{s_1},s_0}$的正规化,这是一维诺特正规局部环,所以它是DVR,再取它的严格Hensel化记作$R$,这也是DVR(因为DVR的严格Hensel化还是DVR【见Neron Models2.3】).于是$R$的完备化$\widehat{R}$是一个剩余域$\kappa(\widehat{R})$是完全域的完备DVR.接下来如果$\widehat{R}$是等特征的,那么它同构于$\kappa[[T]]$,此时它可以嵌入到剩余域是代数闭域的完备离散赋值环$\overline{\kappa}[[T]]$;如果$\widehat{R}$是混合特征的,由于$\kappa$是完全域,有$\widehat{R}$同构于Witt环$W(\kappa)$,于是它可以嵌入到剩余域是代数闭域的完备离散赋值环$W(\overline{\kappa})$.综上我们总有一个剩余域是代数闭域的完备离散赋值环$R$,和一个态射$\mathrm{Spec}R\to S$,它把$R$的唯一闭点映为$s_0$,唯一一般点映为$s_1$.把$X\to S$基变换到$\mathrm{Spec}R$上不会改变特殊化态射.
    		\item 做上述约化后,此时$\kappa(s_0)$是代数闭域,于是可以不妨设$\overline{s_0}$就是实际的点$s_0$.有$\kappa(s_1)=\mathrm{Frac}(A)=K$,不妨设$\overline{s_1}$是$\mathrm{Spec}\overline{K}\to S$.用$K^s$表示$K$在$\overline{K}$中的可分闭包.可设$\overline{s_1}$经几何点$s_1^s:\mathrm{Spec}(K^s)\to S$分解.那么由于$\mathrm{Spec}(\overline{K})\to\mathrm{Spec}(K^s)$是泛同胚,于是$X_{\overline{s_1}}\to X_{s_1^s}$也是泛同胚,于是它诱导了基本群同构$\pi_1(X_{\overline{s_1}},\overline{x_1})\cong\pi_1(X_{s_1^s},\overline{x_1})$.于是我们可以把$X_{\overline{s_1}}$替换为$X_{s_1^s}$.
    		\item 命题中的群同态$\varphi_0$存在当且仅当$\ker\mathrm{sp}\subseteq\ker\varphi_1$,其中$\ker\varphi_0$是$\pi_1(X_{s_1^s},\overline{x_1})$的开子群,前面解释过在$\mathrm{sp}$是满同态的时候,这个包含关系成立当且仅当存在$\pi_1(X_{s_0},\overline{x_0})$的开子集$V$,使得$H_{\mathrm{sp}}(\pi_1(X_{s_0},\overline{x_0})/V)\cong\pi_1(X_{s_1^s},\overline{x_1})$.翻译为覆盖的语言就是说,如果有连通有限平展覆盖$Y'\to X_{s_1^s}$,那么存在连通有限平展覆盖$Y\to X$满足$Y'$就是$Y\to X$关于$X_{s_1^s}\to S$的基变换.我们构造$Y\to X$的方法是自上而下构造如下三个小纤维积图表:
    		$$\xymatrix{Y'\ar[rr]\ar[d]&&X^{K^s}\ar[d]\\{Y'}^L\ar[rr]\ar[d]&&X^L\ar[d]\\Y^L\ar[rr]\ar[d]&&X\times_SS^L\ar[d]\\Y\ar[rr]&&X}$$
    		\item 上方纤维积图表:我们知道$K\subseteq K^s$是全部有限子扩张$K\subseteq L$的正向极限,于是$X^{K^s}=X\times_SK^s$是$\{X^L=X\times_SL\}$的逆向极限.前面证明过当$L$足够大时存在有限平展覆盖${Y'}^L\to X^L$构成这个纤维积图表.
    		\item 任取有限扩张$K\subseteq L\subseteq K^s$,记$A$在$L$中的正规化为$A^L$,记$S^L=\mathrm{Spec}A^L$.我们来证明$X\times_SS^L$是局部诺特正则整概形.它是局部诺特的因为$X\times_SS^L\to S^L$是紧合态射$X\to S$的基变换,并且$A^L$也是诺特环.它是正则的因为$X\times_SS^L\to S^L$作为光滑态射$X\to S$的基变换仍然是光滑的,而$S^L$是正则的,于是$X\times_SS^L$是正则的.它是连通的是因为$X\times_SS^L\to S^L$作为满射$X\to S$(这如果不是满射就取不到几何点了)的基变换还是满射,并且它是紧合态射导致它是闭映射,最后按照$X\to S$具有几何连通纤维,导致$X\times_SS^L$的纤维都是连通的,而$S^L$作为局部环的素谱是连通的.综上$X\times_SS^L\to S^L$是一个闭映射,满射,终端连通,纤维都连通的映射,这迫使$X\times_SS^L$是连通的.最后正则局部诺特和连通三个条件可以推出不可约,综上$X\times_SS^L$是整概形.
    		\item 证明存在一个有限可分扩张$K\subseteq L\subseteq K^s$,使得$X\times_SS^L$在$K(X\times_SS^L)=K(X^L)\to K({Y'}^L)$中的正规化是非分歧的.由于这个正规化态射是有限满射(因为扩张$K(X^L)\subseteq K({Y'}^L)$是有限扩张),并且$Y^L$是正规整概型,$X\times_SS^L$是正则概形,于是这个态射满足Zariski-Nagata纯粹定理,于是它的非平展点只能在$Y^L$的余维数1的点取到,但是按照$Y^L\to X\times_SS^L\to X$都是整扩张,于是$Y^L$的余维数1的点在$X$中的像也是余维数1的.但是$X$的余维数1的点要么是闭纤维的一般点$\eta$,要么在一般纤维中,归结为构造有限可分扩张$K\subseteq L$使得$Y^L\to X\times_SS^L$在$\eta$的在$X\times_SS^L$中的提升的点是非分歧的即可【?】.
    		
    		\qquad
    		
    		为此我们设$R$的uniformizer为$\pi$,它也是$\mathscr{O}_{X,\eta}$的uniformizer【?】.取$L=K[T]/(T^n-\pi)$,按照Kummer理论,有$K(X^L)=K(X)L=K(X)[T]/(T^n-\pi)$【?】是$K(X)$的关于$\mathscr{O}_{X,\eta}$的$n$维温分歧扩张.于是此时$K(Y)$【Anna Cadoret10.2】
    		
    		
    		
    		\item 中间纤维积图表:选取上一条中的有限可分扩张$K\subseteq L$,前面解释过此时正规化$Y^L\to X\times_SS^L$是有限平展覆盖,于是对任意包含$L$的$K$的有限可分扩张$L'$,都有$Y^{L'}\to X\times_SS^{L'}$作为有限平展覆盖的基变换还是有限平展覆盖.另外按照正规化的泛性质,以及局部诺特正规整概形上有限平展覆盖描述为正规化,就得到中间纤维积图表成立.
    		\item 下方纤维积图表:只需证明$X\times_SS^L\to X$诱导的基本群同态总是同构,这就导致存在有限平展覆盖$Y\to X$构成这个纤维积图表.按照第二同伦列定理,我们有如下短正合列的交换图表,由于$S$和$S^L$的一般点的剩余域都是代数闭域,导致这里$\pi_1(S)=\pi_1(S^L)=1$,另外有$X_{s_0}=(X\times_SS^L)_{s_0^L}$,于是两边的垂直同态都是同构,这就得到中间垂直同态是同构.
    		$$\xymatrix{1\ar[r]&\pi_1((X\times_SS^L)_{s_0^L})\ar[r]\ar[d]&\pi_1(X\times_SS^L)\ar[r]\ar[d]&\pi_1(S^L)\ar[r]\ar[d]&1\\1\ar[r]&\pi_1(X_{s_0})\ar[r]&\pi_1(X)\ar[r]&\pi_1(S)\ar[r]&1}$$
    	\end{enumerate}
    \end{proof}
    \item 上一条告诉我们的是,特殊化态射$\mathrm{sp}:\pi_1(X_{\overline{s_1}},\overline{x_1})\to\pi_1(X_{\overline{s_0}},\overline{x_0})$的核落在$\pi_1(X_{\overline{s_1}},\overline{x_1})$的所有满足对应的商是阶数和$p=\mathrm{char}\kappa(s_0)$互素的开正规子群中.对射影有限群$\Pi$,对素数$p$,记$\Pi^{(p)'}$表示$\Pi$的所有阶数和$p$互素的有限商群的逆向极限,它也就是$\Pi$模掉它的全部Sylow-$p$子群生成的闭正规子群得到的商.那么特殊化态射诱导了如下同构:
    $$\pi_1(X_{\overline{s_1}},\overline{x_1})^{(p)'}\cong\pi_1(X_{\overline{s_0}},\overline{x_0})^{(p)'}$$
    \item Grothendieck引入特殊化态射是为了证明如下结论(这件事特征零也是成立的,引入特殊化态射是为了解决特征$>0$的情况,具体见曲线的基本群一节):设$k$是特征$p>0$的代数闭域,设$X$是$k$上的一条非奇异紧合连通亏格为$g$的曲线.那么有$\pi_1(X)^{(p)'}\cong\Pi^{(p)'}$,其中$\Pi$是由生成元集合$\{a_1,b_1,\cdots,a_g,b_g\}$和一个关系$[a_1,b_1]\cdots [a_g,b_g]=1$所确定的群.
\end{enumerate}
\subsection{基本群的双有理不变性}
\begin{enumerate}
	\item 【SGA2.X.3.4】Zariski-Nagata纯粹定理.设$X,Y$是整概型,其中$X$是正规概形,$Y$是正则概形.设$f:X\to Y$是一个拟有限支配态射,设$Z_f\subseteq X$是全部在$f$下非平展的点构成的闭子集,那么要么$Z_f=X$,要么$Z_f$是等余维数1的(此为对任意一般点$\eta\in Z_f$都有$\dim\mathscr{O}_{X,\eta}=1$).
	\item 设$X$是连通正则局部诺特概形,设$i_U:U\to X$是一个开子集$U$的典范开嵌入,如果$X-U$的余维数$\ge2$,那么$i_U$就诱导了范畴等价$i_U^*:\textbf{FEt}(X)\to\textbf{FEt}(U)$.进而任取$U$上的几何点$\overline{s}:\mathrm{Spec}\Omega\to U$,就有$i_U$诱导了基本群的同构:
	$$\pi_1(i_U):\pi_1(U)\cong\pi_1(U,\overline{s})\cong\pi_1(X,\overline{s})$$
	\begin{proof}
		
		先证明函子$i_U^*$是完全忠实的,这等价于讲任取$X$的连通有限平展覆盖$p:Y\to X$,证明$Y\times_XU=p^{-1}(U)$是连通的,这等价于证明$\mathscr{O}_Y(p^{-1}(U))$没有非平凡幂等元.由于$p$是平坦态射,于是$p_*\mathscr{O}_Y$是$\mathscr{O}_X$上的局部自由模层.我们解释过局部自由模层仍然满足Hargos定理,于是限制映射$\mathscr{O}_Y(Y)=p_*\mathscr{O}_X(X)\mathscr{O}\top_*\mathscr{O}_X(U)=\mathscr{O}_Y(f^{-1}(U))$是同构,于是从$Y$连通得到$\mathscr{O}_Y(Y)$没有非平凡幂等元,进而$\mathscr{O}_Y(f^{-1}(U))$也没有非平凡幂等元.
		
		\qquad
		
		还剩下证明$i_U^*$是本质满的,也即对任意连通有限平展覆盖$p_U:V\to U$,存在$X$的一个有限平展覆盖$p:Y\to X$,满足$p_U$是$p$经$i_U:U\to X$的基变换.我们之前解释过由于这俩$U$是正规整局部诺特概形,所以连通有限平展覆盖$V\to U$恰好就是$U$在有限可分域扩张$K(X)=K(U)\to K(V)$中的正规化.记$X$在$K(X)\to K(V)$中的正规化是$p:Y\to X$,我们来证明$Y\times_XU=p^{-1}(U)\to U$满足$U$在$K(U)\subseteq K(V)$中正规化的泛性质.首先由于$K(Y)=K(V)$,于是$Y$的非空开子集$p^{-1}(U)$就也是一个函数域为$k(V)$的正规整概形.接下来任取一个函数域为$K(V)$的正规整概型$M$和一个支配态射$M\to U$,那么$M\to U\to X$就也是一个支配态射,于是按照正规化$Y\to X$的泛性质,就存在态射$M\to Y$使得如下实线图表交换,于是按照纤维积的泛性质存在唯一的虚线态射使得图表交换,这说明有$U$同构$Y\times_XU=V$.
		$$\xymatrix{M\ar@/^1pc/[drr]\ar@/_1pc/[ddr]\ar@{-->}[dr]&&\\&Y\times_XU\ar[r]\ar[d]&Y\ar[d]^p\\&U\ar[r]^{i_U}&X}$$
		
		于是我们还剩下证明$Y\to X$是有限平展覆盖,由于这个态射是$X$在某个有限可分扩张中的正规化,所以它已经是有限态射.按照$p:Y\to X$限制为$p^{-1}(U)\to U$是平展的,于是$p$在$p^{-1}(U)=Y-p^{-1}(X-U)$上非分歧.按照$X-U$上的点的余维数$\ge2$,以及$p$是有限态射,于是$p^{-1}(X-U)$中的点和它在$X-U$中的像所对应的局部环同态是整扩张,于是它们具有相同维数,这使得$p^{-1}(X-U)$中的点余维数$\ge2$,于是按照Zariski-Nagata纯粹定理,$p^{-1}(X-U)$中的点也都是$p$的平展的,这就得到$p$是有限平展覆盖.
	\end{proof}
    \item 引理.设$X,Y$是域$k$上的紧合概形,其中$X$是正规整概形,那么有理映射$f:X\leadsto Y$在余维数1的点总是有定义的.
    \begin{proof}
    	
    	任取$x\in X$是余维数1的点,那么$\mathscr{O}_{X,x}$是一维正规诺特局部环,于是它是DVR,它的商域记作$K$.按照条件有$Y\to\mathrm{Spec}k$是紧合的,于是按照紧合态射的赋值准则,就存在唯一的虚线态射使得如下图表交换(其中上行的态射是把$\mathrm{Spec}\mathscr{O}_{X,x}$的唯一一般点,这也是$X$的唯一一般点,映射到$Y$,因为$f$的极大可定义开子集$U_f$一定包含一般点,所以这个态射是确定的).这说明存在点$x$的一个开邻域$U$和一个态射$g:U\to Y$.我们还要说明$f$对应的实际态射$f\mid_{U_f}:U_f\to Y$和$f$在$U\cap U_f$上是一致的,为此只要任取$U\cap U_f$中余维数1的点$x'$把下图表中的$\mathscr{O}_{X,x}$替换为$\mathscr{O}_{X,x'}$,按照虚线态射的唯一性,就得到$f\mid_{U_f}$和$g$在$x'$附近是一致的.
    	$$\xymatrix{\mathrm{Spec}K\ar[d]\ar[rr]&&Y\ar[d]\\\mathrm{Spec}\mathscr{O}_{X,x}\ar@{-->}[urr]\ar[rr]&&\mathrm{Spec}k}$$
    \end{proof}
	\item 【?】(基本群的双有理不变性)设$k$是域,设$X,Y$是域$k$上的局部诺特连通正则紧合概形,设$f:X\leadsto Y$是一个双有理映射,记$f$的极大可定义开子集为$U_f\subseteq X$,那么上一条引理保证了$X-U_f$和$Y-U_{f^{-1}}$都是余维数$\ge2$的.特别的$f$限制为$U_f\to U_{f^{-1}}$是一个同构,于是如果选取$U_f$上的几何点$\overline{x}$,记它在$f$下的像是$U_{f^{-1}}$上的几何点$\overline{y}$,那么我们有如下典范同构:
	$$\xymatrix{\pi_1(X,\overline{x})\ar@{=}[rr]^{\pi_1(i_{U_f})^{-1}}&&\pi_1(U_f,\overline{x})\ar@{=}[rr]^{\pi_1(f:U_f\cong U_{f^{-1}})}&&\pi_1(U_{f^{-1}},\overline{y})\ar@{=}[rr]^{\pi_1(i_{U_{f^{-1}}})}&&\pi_1(Y,\overline{y})}$$
\end{enumerate}
\subsection{曲线的基本群}

设$X$是亏格为$g\ge0$的复射影光滑连通曲线,按照GAGA原理,它对应的解析空间$X^{\mathrm{an}}$是一个亏格为$g$的连通紧黎曼曲面,它的拓扑基本群的结构是熟知的,是由$2g$个元$\{a_1,b_1,\cdots,a_g,b_g\}$和一条关系$[a_1,b_1]\cdots[a_g,b_g]=1$确定的群,记作$\Pi_g$.于是按照GAGA原理,$X$的平展基本群为$\pi_1(X,\overline{x})=\widehat{\Pi_g}$,这是一个拓扑有限群(这是指它是一个有限生成子群的闭包,一个群$\Pi$总是它射影完备化$\widehat{\Pi}$的稠密子集).特别的,所有亏格为$g$的复射影光滑连通曲线具有同构的平展基本群.本节的目标就是把这件事推广到任意特征零的代数闭域上,并且证明特征$p$的代数闭域上的类似结论.
\begin{enumerate}
	\item 拓扑有限群的补充.一个拓扑群称为拓扑有限群,如果他有一个稠密的有限生成子群.例如由于一个群$G$在它的射影完备化中稠密,于是一个有限生成群的射影完备化是拓扑有限群.
	\begin{enumerate}[(1)]
		\item 设$G$是射影有限群,如下两个条件互相等价,并且如果$G$是拓扑有限群时这个条件成立.
		\begin{enumerate}[(a)]
			\item 对任意给定的正整数$n$,$G$只有有限个开子群具有指数$n$.
			\item 对任意给定的正整数$n$,$G$只有有限个开正规子群具有指数$n$.
		\end{enumerate}
		\begin{proof}
			
			线证明这两个命题互相等价.假设(b)成立,任取指数为$n$的开子群$H$,它所包含的$G$的正规子群的极大元是$H'=\cap_{g\in G}g^{-1}Hg$,并且按照$G/H'$忠实的左平移作用在左陪集集合$G/H$上,于是$[G:H']\le n!$.固定$G$的一个正规子群$G_0$,按照$G/G_0$的子群是有限的,特别的所有满足$H'=G_0$的开子群$H$也是有限的.
			
			\qquad
			
			设$G$被$\langle g_1,\cdots,g_n\rangle$拓扑生成,固定一个正整数$n$,那么$n$阶有限群$H$的同构类个数是有限的.$G$的阶数$n$的有限商群对应于群同态$G\to H$,而这被$\{g_1,\cdots g_n\}$的像决定,这也是有限的.
		\end{proof}
		\item 对射影有限群$G$,用$\mathrm{im}(G)$表示$G$的全体有限商群(的同构类)构成的集合.设射影有限群$G_1$满足对任意正整数$N$只存在有限个开正规子群具有指数$N$.设$G_2$是任意另一个射影有限群,那么如果$\mathrm{im}(G_1)=\mathrm{im}(G_2)$,则$G_1\cong G_2$;并且任意满同态$G_1\to G_2$或者$G_2\to G_1$都是同构.
		\begin{proof}
			
			对正整数$N$,用$U_N$和$V_N$分别表示$G_1$和$G_2$的全体指数至多为$N$的开正规子群的交.按照条件就有$[G_1:U_N]$有限.设$V$是$G_2$的一些指数至多为$N$的开正规子群的交,按照条件就可以找到$G_1$的开正规子群$U$满足$G_1/U\cong G_2/V$.于是按照$V_N$【】
		\end{proof}
	\end{enumerate}
    \item 引理.设$k$是代数闭域,设$X/k$是连通有限型概形,设$k\subseteq K$是一个包含$K$的代数闭域,如果$\pi_1(X)$是拓扑有限群,那么典范同态$\pi_1(X_K)\to\pi_1(X)$是同构.
    \begin{proof}
    	
    	首先$X$上的连通平展覆盖基变换到$X_K$上仍然是连通的,这导致$\pi_1(X_K)\to\pi_1(X)$是满射,于是按照拓扑有限群的补充,归结为证明$\mathrm{im}(\pi_1(X))=\mathrm{im}(\pi_1(X_K))$.但是由于$\pi_1(X)$已经是$\pi_1(X_K)$的商群,于是$\pi_1(X)$的有限商群一定是$\pi_1(X_K)$的有限商群,于是问题归结为证明$\mathrm{im}(\pi_1(X_K))\subseteq\mathrm{im}(\pi_1(X))$.
    	
    	\qquad
    	
    	任取一个有限商群$\pi_1(X_K)\to G$,这对应于$X_K$的一个Galois覆盖$Z\to X_K$.只需找$X$的一个对应的群为$G$的Galois覆盖$Y\to X$(之前证明单射是需要找到$X$的Galois覆盖$Y\to X$使得基变换是$Z\to X_K$).由于$X_K/K$是有限表示的,并且$Z\to X_K$是有限态射,于是可以找到$K$的有限型$k$子代数$R$,和一个连通有限平展覆盖$Z'\to X_R=X\times_kR$【补充】,满足如下纤维积图表.最后由于$S$是代数闭域$k$上的有限型概形,一定存在闭点$p$,于是此时纤维上的态射$Z'_p\to X_R\times_Rk=X$是对应于群$G$的有限平展覆盖【Bertini定理?】.
    	$$\xymatrix{Z\ar[rr]\ar[d]&&X_K\ar[d]\\Z'\ar[rr]&&X_R}$$
    \end{proof}
	\item 设$k$是特征为零的代数闭域,设$X/k$是亏格为$g$的光滑射影连通曲线,那么有$\pi_1(X,\overline{x})\cong\widehat{\Pi_g}$,特别的这是一个拓扑有限群.
	\begin{proof}
		
		按照$X/k$是有限表示的,可以找到$k$的子域$k_0$,使得它是有限型$\mathbb{Q}$代数,并且存在$k_0$概形$X_{0,k_0}$,使得它基变换到$k$上就是$X$.按照lefschetz原理,$k_0$一定同构于$\mathbb{C}$的某个子域,于是如果记$k_0$在$k$中的代数闭包为$\overline{k_0}$,那么存在嵌入$\overline{k_0}\to\mathbb{C}$.把$X_{0,k_0}$基变换到$\overline{k_0}$和$\mathbb{C}$上分别记作$X_{0,\overline{k_0}}$和$X_{0,\mathbb{C}}$,于是有如下纤维积图表:
		$$\xymatrix{X_{0,\mathbb{C}}\ar[r]\ar[d]&X_{0,\overline{k_0}}\ar[d]&X\ar[l]\ar[d]\\\mathrm{Spec}\mathbb{C}\ar[r]&\mathrm{Spec}\overline{k_0}&\mathrm{Spec}k\ar[l]}$$
		
		这诱导了基本群的如下两个满同态(满射因为几何连通性).这里$\pi_1(X_{0,\mathbb{C}})=\widehat{\Pi_g}$是拓扑有限群,于是按照我们的引理$\alpha$,$\beta$都是同构.
		$$\xymatrix{\pi_1(X_{0,\mathbb{C}})\ar[r]^{\alpha}&\pi_1(X_{0,\overline{k_0}})&\pi_1(X)\ar[l]_{\beta}}$$
	\end{proof}
    \item (形变理论的补充).【SGA1.III.7.3】.设$A$是完备诺特局部环,剩余域记作$k$.设$X_0/k$是射影光滑概形,满足:
    $$\mathrm{H}^2(X_0,\mathrm{T}_{X_0/k})=\mathrm{H}^2(X_0,\mathscr{O}_{X_0})=0$$
    
    其中$\mathrm{T}_{X_0/k}=\mathrm{HOM}_{\mathscr{O}_X}(\Omega_{X_0/k}^1,\mathscr{O}_{X_0})$是微分模层的对偶模层,也即$X_0$的切丛.那么存在一个射影光滑$A$概形$X$,满足$X$的闭纤维同构于$X_0/k$.
    \item 于是如果$k$是特征为$p\not=0$的代数闭域,设$X_0/k$是光滑连通射影曲线(对于曲线,射影和紧合是等价的),那么它的所有二阶层上同调都为零(Grothendieck vanishing theorem),如果取Witt环$A=W(k)$(这是完备离散赋值环,混合特征,剩余域为$k$),记$S=\mathrm{Spec}A$,那么按照上述定理,就存在射影光滑概形$X/S$满足$X\times_S\mathrm{Spec}k\cong X$.于是$X$的闭纤维$X_0$是几何连通的,我们断言$X$的一般纤维$X_1$也是几何连通的光滑射影曲线.最后在第二同伦列的证明中我们证明了如果$S=\mathrm{Spec}A$是完备诺特局部环的仿射概形,$X\to S$是紧合态射,那么闭纤维$X_0$对应的闭嵌入$i:X_0\to X$诱导了基本群的同构$\pi_1(X_0)\cong\pi_1(X)$.于是求基本群$\pi_1(X_0)$就约化到求一个混合特征的完备离散赋值环$A$上射影光滑概形$X$的基本群$\pi_1(X)$.
    \begin{proof}
    	
    	我们来证明$X$的一般纤维$X_1$也是几何连通的.这件事归结为证明如果$(A,\mathfrak{m},k)$是剩余域为代数闭域的完备离散赋值环,如果$A\to B$是一个有限平展代数,满足闭纤维是单点集,那么这是一个同构.进而做$X\to S$的stein分解就得到$X$的纤维都是几何连通的.
    	
    	\qquad
    	
    	首先诺特局部环$A$的有限平展代数必然具有形式$B=A[T]/(F)$【EGAIV.18.4.5】,其中$F$是一个首一多项式(其实是可分的,也即$(F,F')=A[T]$,但是用不到).那么$A\to B$的闭纤维是$\mathrm{Spec}k[T]/(\overline{F})$,按照这是几何连通的,就有$\overline{F}$是$k[T]$上的不可约多项式,于是$k[T]/(\overline{F})$是$k$的有限扩张,但是$k$是代数闭域,迫使$\overline{F}$是一次多项式.
    	
    	\qquad
    	
    	再说明$X_1$的确是一维的【liuqing4.4.16】:任取$X$的和$X_0$有交的仿射开子集$U$,那么$U$和$X_1$也有交(因为$X$是整概型,于是$X\to S$把一般点映为一般点,而非空集合$U$肯定包含$X$的一般点).于是按照$X_0$和$X_1$都是整概形,就有$\dim X_1=\dim X_1\cap U$和$\dim X_0=\dim X_0\cap U=1$.换句话讲问题归结为设$X=\mathrm{Spec}B$本身是仿射的.
    	
    	\qquad
    	
    	记$A$的uniformizer为$\pi$,那么按照$A\to B$是有限型的,于是$B/\pi B$是域$k=A/\pi A$上的有限型代数,由于$\dim X_0=1$,按照诺特正规化引理就有一个有限的$A$代数同态$\overline{\varphi}:k[T]\to B/\pi B=B\otimes_Ak$.记$\overline{\varphi}(T)\in B/\pi B$在$B$中的一个提升为$b$,取$A$代数同态$A[T]\to B$为$T\mapsto b$,按照NAK引理这使得$B$是有限$A[T]$模,它对应于一个有限态射$f:X\to\mathbb{A}_S^1$.于是诱导的函数域之间的扩张$K(\mathbb{A}_S^1)=K(T)\subseteq K(X)$是有限扩张,于是它们在$K$上的超越维数一致,于是$\dim X_1=\dim X=1$.
    \end{proof}
    \item 记$S$的闭点为$a_0$,一般点为$a_1$,记$X\to S$的一般纤维$X_K=X\times_SK$,其中$K=\mathrm{Frac}(A)$,再记几何纤维$\overline{X_K}=X\times_S\overline{K}$,其中$\overline{K}$是$K$的代数闭包.再记$\overline{a_0}\in X_k=X\times_Sk$是$a_0$的一个几何点,记$\overline{a_1}\in\overline{X_K}$.那么我们有特殊化态射(其中同构号仍然是我们在第二同伦列中证明的):
    $$\pi_1(\overline{X_K},\overline{a_1})\to\pi_1(X_k,\overline{a_0})\cong\pi_1(X)$$
    
    由于这里$X\to S$是紧合光滑的,并且纤维都是几何连通和几何既约的,就有这个特殊化态射是满射,特别的,$\pi_1(X)$总是$\widehat{\Pi_g}$的商,于是它是拓扑有限群.另外按照特殊化态射那里我们证明的结论,这个特殊化态射的核落在$\pi_1(X_k)$的所有满足对应的商是阶数和$p$互素的开正规子群中.另外特殊化态射诱导了如下极大和$p$互素商之间的同构:
    $$\pi_1(X)^{(p)'}\cong\pi_1(X_k)^{(p)'}$$
\end{enumerate}
\subsection{紧合概形的基本群}

本节我们主要证明的是代数闭域$k$上的紧合连通概形的基本群总是拓扑有限群.
\begin{enumerate}
	\item 断言1.对任意自然数$d\ge0$,如果命题对$k$上的所有$\le d$维连通射影正规概形成立,那么命题对$k$上的所有$d$维紧合连通概形成立.
	\begin{proof}
		
		设$X$是代数闭域$k$上的$d$维紧合连通概形.按照Chow引理,我们有一个射影$k$概形$X'$和一个满射$k$态射$X'\to X$,并且它自动是紧合态射.不妨设$X$是既约的,这不改变$X$的基本群,并且${X'}^{\mathrm{red}}\to X^{\mathrm{red}}$仍然是紧合满射.取$X'$的正规化$\widetilde{X'}$,那么$\widetilde{X'}\to X'$是有限态射,这导致$\widetilde{X'}\to\mathrm{Spec}k$仍然是射影态射(因为紧合态射$f:\widetilde{X'}\to X'$是射影态射当且仅当它是拟射影的,也即$\widetilde{X'}$存在$f$丰沛层.而这里按照$X'$是射影$k$概形,它就存在极丰沛层也是丰沛层$\mathscr{L}$,所以我们只要证明丰沛层在有限态射下的回拉仍是丰沛层:设$f:X\to Y$是诺特概形之间有限态射,设$\mathscr{L}$是$Y$上丰沛层,任取$X$上凝聚层$\mathscr{F}$,那么$f_*\mathscr{F}$是$Y$上凝聚层,于是存在正整数$n_0$使得$n\ge n_0$的时候$f_*\mathscr{F}\otimes\mathscr{L}^{\otimes n}$的整体截面是有限生成的,于是有满态射$\mathscr{O}_Y^r\to f_*\mathscr{F}\otimes\mathscr{L}^{\otimes n}$.按照$f^*$是右正合的,得到满态射$\mathscr{O}_X^r\to f^*f_*\mathscr{F}\otimes f^*\mathscr{L}^{\otimes n}$,但是按照$f$是仿射的,有$f^*f_*\mathscr{F}\to\mathscr{F}$是满态射,这就得到满态射$\mathscr{O}_X^r\to\mathscr{F}\otimes f^*\mathscr{L}^{\otimes n}$,于是$f^*\mathscr{F}$是丰沛层).
		
		\qquad
		
		于是我们不妨设Chow引理得到的分解中$X'$是正规射影$k$概形.此时$X'\to X$仍然是紧合满射,于是它是有限平展覆盖范畴作为纤维范畴的有效下降态射.这导致$X$的基本群可以被$X'$和$X''$和$X'''$描述,并且只要$X'$的所有连通分支上的基本群都是拓扑有限群,就得到$X$的基本群是拓扑有限群【补充】.另外按照Chow引理这里的$p:X'\to X$可以保证存在$X$的一个开子集$U$使得$p$限制为$p^{-1}(U)\to U$是同构(不改变$p$是紧合满射的事实).这导致$\dim X'=\dim p^{-1}(U)=\dim U\le\dim X=d$.
	\end{proof}
    \item 引理.设$X/k$是紧合概形,设$f:X\to\mathbb{P}_k^n$是态射,设$H\subseteq\mathbb{P}_k^n$是超平面,如果$X$是不可约的并且$\dim f(X)\ge2$,那么$f^{-1}(H)=H\times_{\mathbb{P}_k^n}X$是非空且连通的.
    \item 断言2.设$X$是$d\ge2$维射影正规连通$k$概形,那么存在一个紧合连通$\le d-1$维$k$概形$Y$和一个基本群之间的满同态$\pi_1(Y)\to\pi_1(X)$.
    \begin{proof}
    	
    	(域上的射影态射和H-射影态射是等价的)按照定义有闭嵌入$i:X\to\mathbb{P}_k^n$.选取一个超平面$H\subseteq\mathbb{P}_k^n$使得$X\not\subseteq H$.取它们的交$Y=X\cap H$赋予既约闭子概型结构.由于$X$是正规概形,它的不可约分支两两不交,结合$X$是诺特概形和连通的,就得到$X$是不可约的,结合$f(X)\cong X$的维数$\ge2$,按照引理就得到$Y$是非空且连通的.最后还剩下证明闭嵌入$Y\to X$诱导的$\pi_1(Y)\to\pi_1(X)$是满射,而这依旧是因为引理:任取$X$的连通有限平展覆盖$Z\to X$,按照$X$是正规的得到$Z$也是正规的,于是$Z$也是不可约诺特概形,于是把引理用在$Z\to X\to\mathbb{P}_k^n$上就得到$Z\times_XY$是连通的.
    \end{proof}
    \item 证明定理.
    \begin{proof}
    	
    	设$X$是域$k$上的紧合连通$d$维概形,我们对$d$归纳证明它的基本群是拓扑有限群.首先$d=0$时$X$是单点空间,所以它是阿廷局部环的素谱,考虑基本群可以不妨设它是既约的,于是它是域的素谱,但是由于是有限型态射,所以对应的域扩张是有限扩张,结合$k$是代数闭域就导致此时$X=\mathrm{Spec}k$,那么它的基本群是平凡群.$d=1$的时候紧合与射影是等价的,我们在曲线的基本群中已经解释了代数闭域上的射影光滑连通曲线的基本群是拓扑有限群(一维的时候正规等价于正则,又因为域是代数闭域,正则等价于光滑).
    	
    	\qquad
    	
    	下面设$d\ge2$.假设命题对$\le d-1$均成立.按照断言1,归结为证明$d$维正规射影连通$k$概形$X$的基本群是拓扑有限群.但是按照断言2,存在一个$\le d-1$维紧合连通$k$概形$Y$和一个满射$\pi_1(Y)\to\pi_1(X)$,按照归纳假设有$\pi_1(Y)$是拓扑有限群,进而它的商$\pi_1(X)$就也是拓扑有限群,完成归纳.
    \end{proof}
\end{enumerate}
\subsection{Van Kampen定理}

设$X$是离散赋值环$R$上的连通概形,特殊纤维记作$X_k$,设有开覆盖$X_k=U_k\cup V_k$,那么可以找到$X$的闭子概型使得特殊纤维就是$X-U_k$(因为$X_k$是闭子集),于是可以找到$X$的开子集$U,V$使得它们的特殊纤维分别是$U_k,V_k$.设$\widehat{U},\widehat{V}$分别是$U,V$沿各自闭纤维的形式完备化.
\begin{enumerate}
	\item 设$G$是有限群,设$X$是整概形,一个态射$f:Y\to X$称为$G$-Galois分歧覆盖,如果$f$是有限态射,并且存在群同态$G\to\mathrm{Aut}_X(Y)$使得$G$在几何一般纤维集上的作用是单可迁的,名字中的分歧指的是它可能是分歧的.对概形$S$,用$\textbf{GCohAlg}(\mathscr{O}_S)$表示那些对应的态射是$G$-Galois分歧覆盖的凝聚$\mathscr{O}_S$代数层构成的范畴.
	\item 设$H\le G$是有限群的子群,设$f:Y\to X$是$H$-Galois覆盖,那么这个覆盖的$[G:H]$个无交并,赋予诱导的左陪集$G$作用是一个不连通的$G$-Galois覆盖,它称为$H$-Galois覆盖在$G$中的诱导覆盖,记作$\mathrm{Ind}_G^H(Y)\to X$.
	\item 设$X/R$是紧合概形,那么有如下典范的范畴等价:
	$$\textbf{Coh}(X)\cong\textbf{Coh}(\widetilde{U})\times_{\textbf{Coh}(\widetilde{U}\times_X\widetilde{V})}\textbf{Coh}(\widetilde{V})$$
	$$\textbf{GCohAlg}(X)\cong\textbf{GCohAlg}(\widetilde{U})\times_{\textbf{GCohAlg}(\widetilde{U}\times_X\widetilde{V})}\textbf{GCohAlg}(\widetilde{V})$$
	\begin{proof}
		
		如果存在$\widehat{U}$和$\widehat{V}$上的凝聚层,使得它们回拉到$\widehat{U}\times_X\widehat{V}$上是同构的,那么它们可以粘合为$\widehat{X}$上的凝聚层,再按照Grothendieck存在性定理对应到$X$上的凝聚层.
	\end{proof}
    \item (Van Kampen定理)设$\psi_1:A\to G_1$和$\psi_2:A\to G_2$是射影有限群之间的两个同态,它们的amalgamated积$G_1*_AG_2$定义为自由积$G_1*G_2$模去$\{\psi_1(a)\psi_2(a)^{-1}\mid a\in A\}$生成的正规子群的闭包.我们有如下典范同构:
    $$\pi_1(\widetilde{U})*_{\pi_1(\widetilde{U}\times_X\widetilde{V})}\pi_1(\widetilde{V})\cong\pi_1(X)$$
    \begin{proof}
    	
    	任取有限群$G$,假设$U'\to\widehat{U}$和$V'\to\widehat{V}$是两个$G$-Galois覆盖,并且回拉到$\widehat{U}\times_X\widehat{V}$上是同构的.按照上一条就有这两个覆盖唯一的对应了一个$X$上的$G$-Galois覆盖.这说明命题中的这两个群的有限商的同构类集合是一致的.我们之前解释过紧合连通概形的基本群是拓扑有限群,并且终端为拓扑有限群的满同态,如果终端和源端的射影有限群的有限商的同构类一致,那么这个满同态是同构.
    \end{proof}
\end{enumerate}
\subsection{覆盖的形式/刚性修补}

\begin{enumerate}
	\item 设$X$是不可约概形,一个有限覆盖$f:Y\to X$称为mock覆盖,如果它限制在$Y$的每个不可约分支上都是同构.设$G$是有限群,设$R$是完备离散赋值环,$X/R$是紧合概形,使得一般纤维$X_K$是$K=\mathrm{Frac}(R)$上的曲线,那么存在不可约分歧$G$-Galois覆盖$\varphi:Y\to X$使得它的闭纤维是一个mock覆盖.
	\begin{proof}
		
		问题归结为$X=\mathbb{P}^1_R$的情况,因为不妨取闭嵌入$X_K\to\mathbb{P}^1_K$,基变换得到闭嵌入$X\to\mathbb{P}^1_R$.倘若我们构造了覆盖$Y'\to\mathbb{P}^1_R$满足条件,取它的基变换就得到满足条件的覆盖$Y\to X$.
		
		\qquad
		
		先设$G$是阶数为$q=p^n$的循环群.此时命题基本上就是拿域扩张的Galois理论构造覆盖:先设$K$包含$p$次单位根,那么分$p=\mathrm{char}(K)$和$p\not=\mathrm{char}(K)$的情况,分别用Kummer理论和Artin-Schreier理论构造相应的覆盖,并且我们可以控制它的分歧中心是某个固定的多项式$f$的零点集.接下来如果$K$不包含$p$次单位根,可以先取扩张$K(\zeta_p)$,构造它的相应覆盖,再降到$K$上即可.
		
		\qquad
		
		接下来设$G$是一般的有限群,对它的阶数是素数幂的生成元个数做归纳,归结为设$G$被两个元$g_1,g_2$生成,并且$g_i$的阶数$q_i$是素数幂.记$H=\langle g_i\rangle$,那么我们可以按照上面构造$X$的$H_i$-Galois覆盖,并且可以要求它们的分歧中心$B_1,B_2$是不交的.再取这两个覆盖在$G$上诱导的覆盖记作$Y_i\to X$.再记$U_k=X_k-B_1$和$V_k=X_k-B_2$,记$W_k=U_k\cap V_k$.由于$Y_i\to X$的闭纤维都是mock覆盖,所以$Y_{i,k}$在$W_k$上的限制是同构的.记$U,V,W\subseteq X$是开子集使得它们的闭纤维分别是$U_k,V_k,W_k$.由于平展覆盖的形变是唯一的,于是$Y_1$限制在$\widehat{U}$和$Y_2$限制在$\widehat{V}$上这两个覆盖限制在$\widehat{W}$上是同构的.于是按照Van Kampen定理,它们唯一的对应了$X$上的一个覆盖$Y\to X$.【】
	\end{proof}
	\item 推论.设$K$是完备离散赋值域或者代数闭域,设$X/K$是曲线,那么对任意有限群$G$,都存在不可约分歧$G$-Galois覆盖(但是这个结论并没有控制分歧中心).
	\begin{proof}
		
		完备离散赋值域的情况是上一条的推论.下面设$K$是代数闭域,取$R=K[[t]]$,那么存在不可约$G$-Galois覆盖$Y\to X_R$【这个覆盖下降为一族平坦$X_V$覆盖?】
		
	\end{proof}
	\item 设$X$是一维诺特概形,设$Z\subseteq X$是某些闭点构成的有限集合.设$U=X-Z$是仿射的.设$\xi\in X$是闭点,记$X_{\xi}=\mathrm{Spec}(\widetilde{\mathscr{O}_{\xi}})$,它的全商环记作$K_{\xi}$.下面记$X'=\coprod_{\xi\in Z}X_{\xi}$和$U'=\coprod_{\xi\in Z}\mathrm{Spec}K_{\xi}$,那么有$U'=U\times_XX'$.再记典范包含态射$i:X'\to X$和$i':U'\to U$.我们有如下典范的范畴等价:
	$$\textbf{Coh}(\mathscr{O}_X)\cong\textbf{Coh}(\mathscr{O}_U)\times_{\textbf{Coh}(\mathscr{O}_{U'})}\textbf{Coh}(\mathscr{O}_{X'})$$
	\begin{proof}
		
		
		
	\end{proof}
    \item 设$R$是离散赋值环,单值化参数记作$\pi$,设$X/R$是紧合概形,使得一般纤维$X_K$是紧合曲线.设$Z\subseteq X$是若干闭点构成的有限集,记$U_k=X_k-Z$,记$X'=\coprod_{\xi\in Z}\mathrm{Spec}\widehat{\mathscr{O}_{X,\xi}}$.取开子集$U\subseteq X$使得闭纤维就是$U_k$,再记$U'=U\times_XX'$.再记$U,U',X'$沿闭纤维的形式完备化分别为$\widehat{U},\widehat{U'},\widehat{X'}$.那么有$\widehat{U'}_k=\widehat{X'}_k-Z$.设$S/R$是概形,对任意整数$n\ge1$,记$S$的$\mathrm{mod}(\pi^n)$为$S_n$,那么有$\widehat{S_n}=S_n$.我们断言有如下典范的范畴等价:
    $$\textbf{Coh}(\mathscr{O}_X)\cong\textbf{Coh}(\mathscr{O}_{\widehat{U}})\times_{\textbf{Coh}(\mathscr{O}_{\widehat{U'}})}\textbf{Coh}(\mathscr{O}_{\widehat{X'}})$$
    $$\textbf{GCohAlg}(\mathscr{O}_X)\cong\textbf{GCohAlg}(\mathscr{O}_{\widehat{U}})\times_{\textbf{GCohAlg}(\mathscr{O}_{\widehat{U'}})}\textbf{GCohAlg}(\mathscr{O}_{\widehat{X'}})$$
    \begin{proof}
    	
    	按照上一条,对任意正整数$n$有典范范畴等价:
    	$$\textbf{Coh}(\mathscr{O}_{X_n})\cong\textbf{Coh}(\mathscr{O}_{U_n})\times_{\textbf{Coh}(\mathscr{O}_{U'_n})}\textbf{Coh}(\mathscr{O}_{X'_n})$$
    	
    	逆向系统和这个同构是可交换的,于是取极限得到:
    	$$\textbf{Coh}(\mathscr{O}_{\widehat{X}})\cong\textbf{Coh}(\mathscr{O}_{\widehat{U}})\times_{\textbf{Coh}(\mathscr{O}_{\widehat{U'}})}\textbf{Coh}(\mathscr{O}_{\widehat{X'}})$$
    	
    	最后按照Grothendieck存在性定理有:
    	$$\textbf{Coh}(\mathscr{O}_{\widehat{X}})\cong\textbf{Coh}(\mathscr{O}_X)$$
    	
    \end{proof}
    \item 应用.设$k$是特征$p>0$的代数闭域.
    \begin{enumerate}[(1)]
    	\item 设$\xi$是$\mathbb{P}^1_k$的闭点,设$G$是$\pi_1(\mathbb{P}^1_k-\{\xi\})$的有限商,那么存在$\mathbb{P}_k^1$的不可约正则分歧$G$-Galois覆盖,使得它只在点$\xi$分歧,并且对应的惯性群是$G$的Sylow-$p$子群.
    	\begin{proof}
    		
    		首先存在不可约正则$G$-Galois覆盖$f_0:Y_0\to\mathbb{P}_k^1$,它在一个点$\eta_0\in f_0^{-1}(\xi)$的惯性群具有形式【】$I=P\rtimes C_m$,其中$P$是一个有限$p$群,$C_m$是阶数为和$p$互素的$m$的循环群.
    		
    		
    	\end{proof}
    \end{enumerate}
    
\end{enumerate}
\subsection{Artin-Schreier覆盖}

设$k$是特征$p>0$的代数闭域.
\begin{enumerate}
	\item 考虑仿射线$\mathbb{A}_k^1$上的代数群态射$\mathscr{P}:t\mapsto t^p-t$.这是有限平展态射(非分歧因为微分是$-1$,平坦是因为它是曲线之间的有限态射).按照$t^p-t=\prod_{i=0}^{p-1}(t-i)$,得到$\ker\mathscr{P}=\mathrm{Spec}\left(\frac{k[t]}{t^p-t}\right)=(\mathbb{F}_p)_k$(这里对有限群$G$和概形$X$,记$G_X=\coprod_{\sigma\in G}X_{\sigma}$,其中$X_{\sigma}=X$)
\end{enumerate}



记号.设$G$是射影有限群,设$p$是素数,用$p(G)$表示$G$的全部Sylow-$p$子群生成的子群(它自动是正规子群).如果$p(G)=G$,就称$G$是拟$p$群.

\newpage
\section{皮卡概形}
\subsection{皮卡函子}
\begin{enumerate}
	\item 绝对皮卡函子.设$X$是$S$概形,绝对皮卡函子$\mathrm{Pic}_X$是$\textbf{Sch}(S)\to\textbf{Ab}$的函子,它把$S$概形$T$映射为皮卡群$\mathrm{Pic}_X(T)=\mathrm{Pic}(X_T)$.
	\begin{enumerate}
		\item 任给一族$S$概形$\{T_i\}$,有$\mathrm{Pic}_X(\coprod_iT_i)=\prod_i\mathrm{Pic}_X(T_i)$.
		\item $\mathrm{Pic}_X$在Zariski拓扑下总不是一个可分预层,从而它总不是层.我们知道可表函子在Zariski拓扑下总是层,所以绝对皮卡函子总不是可表的.
	\end{enumerate}
    \item 相对皮卡函子.设$X$是$S$概形,相对皮卡函子$\mathrm{Pic}_{X/S}$是$\textbf{Sch}(S)\to\textbf{Ab}$的函子,它把$S$概形$T$映射为阿贝尔群$\mathrm{Pic}_{X/S}(T)=\mathrm{Pic}(X_T)/\mathrm{Pic}(T)$.这个函子在Zariski拓扑,\'etale拓扑,fppf拓扑下对应的预层(后文会证明是层)分别记作$\mathrm{Pic}_{X/S,\mathrm{zar}}$,$\mathrm{Pic}_{X/S,\text{\'et}}$,$\mathrm{Pic}_{X/S,\mathrm{fppf}}$
\end{enumerate}








