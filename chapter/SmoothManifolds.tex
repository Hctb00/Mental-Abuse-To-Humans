\chapter{微分流形}
\section{流形和光滑结构}

内蕴的讲,流形就是曲线曲面的高维推广.它是局部欧式的,这是说流形的每个点都存在一个开邻域和欧式空间同胚,这个要求允许在流形上局部的运用坐标系进行计算,于是例如微分点导数切空间等等概念都可以推广到流形上.现在给出的流形的定义在历史上并不是被单一数学家给出的,很多基本的数学概念都是经过很多年的讨论和取舍最后才得到最佳的定义.Gauss于1827年发表的论文上已经可以自由的运用曲面上的局部坐标系,这说明他当时已经具备了坐标卡的概念.并且他是最早的把曲面独立的看作一个个体进行探究,而非依赖于所在的大空间.在1854年Riemann于哥廷根的将以给出了高维微分几何的基础,他用于描述他所探究的对象的德语名词翻译过来变得到了manifold一词.这个词在19世纪晚期被庞加莱在他的同调理论里继续使用.在19世纪末到20世纪初点集拓扑得到了发展,直到1931年才得出了基于点集拓扑的现代的流形定义.

\subsection{流形的拓扑性质}

一个$n$维拓扑流形$M$是一个拓扑空间满足如下三个条件:
\begin{itemize}
  \item $M$是第二可数的
  \item $M$是hausdorff空间
  \item $M$是局部同胚于$\mathbb{R}^n$,即对任意点$p\in M$,存在邻域$U$到$R^n$上开集$V$
  同胚$\varphi$,称$(\varphi,U,V)$是一个坐标卡,其中$U$称为坐标邻域,$\varphi$称为坐标映射.坐标卡有时也会记作$(\varphi,U)$.
\end{itemize}
\begin{enumerate}
	\item 第三个条件中的开集$V$可以改为开球或者$\mathbb{R}^n$,因为我们总可以适当缩小坐标邻域.
	\item 这三个条件是互不依赖的:$\mathbb{R}^2$上两条相交直线是第二可数和hausdorff,但是非局部$\mathbb{R}^n$,
	the long line是hausdorff和局部$\mathbb{R}^n$,但是非第二可数;带两个原点的实直线是第二可数和局部$\mathbb{R}^n$,但是非hausdorff.
	\item 设$(\varphi,U)$是坐标卡,如果$\varphi(U)$是$\mathbb{R}^n$中的开球,就称$U$是坐标球.如果$p\in U$使得$\varphi(p)=0\in\mathbb{R}^n$,就称坐标邻域$U$以点$p$为中心.
	\item 为了使得流形维数的定义是有意义的,需要证明当$n\not=m$的时候$\mathbb{R}^n$中的非空开集总不会同胚于$\mathbb{R}^m$的非空开集,这件事我们会借助同调论证明.另外当我们给流形添加光滑结构后解决这件事就变得容易了.
\end{enumerate}

拓扑性质.
\begin{enumerate}
	\item 拓扑流形$M$具有预紧(precompact)坐标开球构成的可数拓扑基.这里预紧是指闭包是紧集.
	\begin{proof}
		
		先设$M$具有整体坐标卡,于是它本身同胚于$\mathbb{R}^n$的开子集$V$.选取全体开球$B_r(x)$,满足$x$是有理点,$r$是有理数,并且$\overline{B_r(x)}\subset V$.这些开球构成了$V$上的可数的预紧拓扑基,按照同胚得到$M$上的预紧坐标开球构成的拓扑基.
		
		\qquad
		
		再设$M$是任意的$n$维流形.按照第二可数条件,$M$可被可数个坐标邻域$\{U_i\}$覆盖.这里每个坐标邻域按照上一段存在预紧坐标开球构成的拓扑基.这些拓扑基的并就构成了$M$上的坐标开球构成的可数拓扑基.最后如果$V\subset U_i$是其中一个坐标开球,我们知道$V$在$U_i$中的闭包是紧致的.按照$M$是Hausdorff空间,它的紧子集都是闭的,于是$V$在$U_i$中的闭包是闭的,这导致$V$在$U_i$中的闭包和在$M$中的闭包一致,这说明$V$都在$M$中预紧.
	\end{proof}
    \item 连通性.拓扑流形的第三个条件保证它是局部道路连通的,这个条件下总有空间的连通分支和道路连通分支一致(特别的,此时空间连通当且仅当道路连通),这个条件下还有连通分支总是开集.结合流形是第二可数的,说明流形至多有可数个连通分支,每个连通分支都是它的开子流形.
    \item 拓扑流形总是仿紧的(paracompact),即它的每个开覆盖都有局部有限的开加细覆盖.另外这里加细中的每个开集还可以要求是预紧的.
    \begin{proof}
    	
    	称一个拓扑空间$X$具有exhaustion,如果存在一族可数个开集$\{X_i,i\ge1\}$使得:
    	\begin{enumerate}
    		\item 每个$X_i$是预紧的
    		\item $\forall i\ge1$,有$\overline{X_{i}}\subset X_{i+1}$
    		\item $X=\cup_{i\ge1}X_i$
    	\end{enumerate}
    	
    	我们先来证明流形具有exhaustion.先取一组预紧的可数基$\{Y_i,i\ge1\}$.取$X_1=Y_1$,
    	则$\overline{X_1}$紧致,倘若已经定义$X_n$使得$\overline{X_n}$紧致,由于$\{Y_i,i\ge1\}$是开覆盖,存在有限个$Y_i$使得$\overline{X_n}\subset Y_{i_1}\cup\cdots\cup Y_{i_r}$,取$X_{n+1}=Y_{i_1}\cup\cdots\cup Y_{i_r}\cup Y_n$,则有
    	$\overline{X_{n+1}}\subset \overline{Y_{i_1}}\cup\cdots\cup \overline{Y_{i_r}}\cup \overline{Y_n}$后者紧致,于是
    	$\overline{X_{n+1}}$紧致并且$\overline{X_n}\subset X_{n+1}$.最后由于$X=\cup Y_i,Y_i\subset X_i$,于是$X=\cup X_i$.
    	
    	\qquad
    	
    	我们接下来证明如果$X$是具有exhaustion的Hausdorff空间,则$X$是仿紧致的.任取开覆盖$\{U_{\alpha}:\alpha\in J\}$,记exhaustion为$\{X_i,i\ge1\}$,记$K_1=\overline{X_2},K_n=\overline{X_{n+1}}\backslash X_{n},n\ge2$,则每个$K_i$是紧集.记$V_1=X_3,V_n=X_{n+2}\backslash\overline{X_{n-1}},n\ge2$,则每个$V_i$是开集,并且$X=\cup V_i=\cup K_i$.对于任意一个$K_n$,有$\{U_{\alpha}\cap V_n:\alpha\in J\}$构成一个开覆盖,由紧性可取有限子覆盖,$n$取遍全体正整数,得到可数个开集$\mathscr{A}$,它们覆盖整个空间,每个开集位于某个紧集$\overline{V_n}$中,于是它是预紧的.最后说明$\mathscr{A}$是局部有限的,这是因为,对于任意一点$p\in X$,不妨设$p\in K_n$,$\mathscr{A}$中每个开集位于某个$V_n$之中,并且每个$V_n$之中只有$\mathscr{A}$中有限个开集.于是取$p$的开邻域$V_n$只和$\mathscr{A}$中有限个开集相交.
    \end{proof}
    \item 按照Smirnov可度量化定理,从流形是仿紧Hasudorff并且局部可度量化的,说明流形都是度量空间.
    \item 拓扑流形$M$的基本群总是可数的.
    \begin{proof}
    	
    	我们已经解释过$M$存在坐标开球构成的拓扑基$\mathscr{B}$,对任意两个基元素$B,B'$,它们的交$B\cap B'$至多有可数个连通分支,并且每个分支都是道路连通的.让$B$和$B'$跑遍全部基元素(可以有$B=B'$),取$B\cap B'$每个分支中的一个点,这些点构成了一个可数集$\chi$.对每个基元素$B$,如果$x,x'\in\chi$使得$x,x'\in B$,那么这两个点之间存在落在$B$中的道路,选取这样一条从$x$到$x'$的道路记作$h_{x,x'}^B$.
    	
    	\qquad
    	
    	我们知道空间的道路分支上的基本群在同构(事实上是共轭的)意义下唯一,于是不妨选取一个基准点$p\in\chi$.称$p$为基准点的一个特殊圈是指能表示成有限个形如$h_{x,x'}^B$的道路的乘积.按照$\chi$可数,全体$h_{x,x'}^B$可数,于是全体特殊圈可数.于是问题归结为证明基本群$\pi_1(M,p)$中每个元可以被某个特殊圈代表.
    	
    	\qquad
    	
    	假设$f:[0,1]\to M$是基准点在$p$的一个圈.当$B$跑遍$\mathscr{B}$时,$f^{-1}(B)$是$[0,1]$的开覆盖,于是可取有限子覆盖.于是存在划分$0=a_0<a_1<\cdots<a_k=1$,使得每个闭区间$[a_{i-1},a_i]$落在某个$f^{-1}(B)$中.设$f$在$[a_{i-1},a_i]$上限制,再经重新选取坐标表示使得从$[0,1]$起始得到的道路为$f_i$.设基元素$B_i$包含了$f_i$的像集.于是对每个指标$i$就有$f(a_i)\in B_i\cap B_{i+1}$.于是存在$x_i\in\chi$使得它落在$B_i\cap B_{i+1}$中并且和$f(a_i)$落在相同的连通分支中.选取$g_i$是从$x_i$到$f(a_i)$的道路.约定$x_0=x_k=p$,约定$g_0,g_k$是$p$上的平凡圈.那么$g_i^{-1}\circ g_i$同伦于$x_i$处的平凡圈,导致有:
    	\begin{align*}
    	f&\sim f_1\circ f_2\circ\cdots\circ f_k\\&\sim g_0\circ f_1\circ g_1^{-1}\circ g_1\circ f_2\circ g_2^{-1}\circ\cdots\circ g_{k-1}\circ f_k\circ g_k^{-1}\\&\sim\widetilde{f_1}\circ\widetilde{f_2}\circ\cdots\circ\widetilde{f_k}
    	\end{align*}
    	
    	其中$\widetilde{f_i}=g_{i-1}\circ f_i\circ g_i^{-1}$.每个$\widetilde{f_i}$是$B_i$中的从$x_{i-1}$到$x_i$的道路,按照$B_i$是单连通的,导致$\widetilde{f_i}$同伦于$h_{x_{i-1},x_i}^{B_i}$,于是$f$同伦于一条特殊圈,完成证明.
    \end{proof}
\end{enumerate}

带边流形.
\begin{itemize}
	\item 定义$n$维上半闭欧氏空间为$\mathbb{H}^n=\{(x^1,x^2,\cdots,x^n)\in\mathbb{R}^n\mid x^n\ge0\}$,赋予$\mathbb{R}^n$的子空间拓扑.记定义$\mathrm{Int}\mathbb{H}^n=\{(x^1,x^2,\cdots,x^n)\in\mathbb{R}^n\mid x^n>0\}$和边界$\partial\mathbb{H}^n=\{(x^1,x^2,\cdots,x^n)\in\mathbb{R}^n\mid x^n=0\}$.
	\item 一个带边$n$维拓扑流形$M$是指一个第二可数Hausdorff空间$M$,并且每个点都存在开邻域和$\mathbb{H}^n$的开子集同胚.于是流形总是带边流形,但是反过来未必成立.
	\item 一个坐标卡$(U,\varphi)$称为内部坐标卡,如果$\varphi(U)\cap\partial\mathbb{H}^n=\emptyset$.否则称为边界坐标卡.如果$p\in M$存在边界坐标卡$(U,\varphi)$使得$\varphi(U)$是以$\varphi(p)\in\partial\mathbb{H}^n$为圆心的半球,就称它是坐标半球.
	\item 设$M$是$n$维带边流形,如果$p\in M$被某个内部坐标卡覆盖,就称它是$M$的(流形)内点;如果$p\in M$被某个边界坐标卡覆盖并且坐标映射把$p$打到$\partial\mathbb{H}^n$上,就称它是$M$的(流形)边界点.$M$的全体(流形)内点构成的子集记作$\mathrm{Int}M$,全体(流形)边界点构成的子集记作$\partial M$.
\end{itemize}
\begin{enumerate}
	\item 一个注解.在点集拓扑中已经有内部,内点,边界,边界点的概念,我们用拓扑内点,拓扑边界,流形内部,流形边界点等名字加以区分,如果未加说明总是指这里新定义的流形意义上的概念.另外当我们证明带边流形上的点不能同时是内点和边界点后,可以验证我们定义的$\mathrm{Int}\mathbb{H}^n$和$\partial\mathbb{H}^n$符合于把$\mathbb{H}^n$视为带边流形后的内部和边界.
	\item 按照定义$\mathbb{H}^0=\mathbb{R}^0=\{0\}$,此时$\partial\mathbb{H}^0=\emptyset$.
	\item 带边流形$M$是内部和边界的无交并.于是特别的带边流形是流形当且仅当边界是空集.特别的拓扑流形之间的同胚必须把内点映为内点,边界点映为边界点.
	\begin{proof}
		
		先说明$p\in M$要么是内点要么是边界点.如果它不是边界点,那么要么它被内部坐标卡覆盖,此时$p$是内点,要么$p$被边界坐标卡$(U,\varphi)$覆盖但是$\varphi(p)$不在$\partial\mathbb{H}^n$中.于是$U\cap\varphi^{-1}(\mathrm{Int}\mathbb{H}^n)$是一个覆盖$p$的内部坐标卡,于是$p$是内点.
		
		借助同调论可以验证$\mathrm{Int}M$和$\partial M$是不交的.【】
	\end{proof}
	\item 如果$M$是$n$维的边界非空的带边流形,那么它的内部是$n$维不带边流形,它的边界是$n-1$维不带边流形.
	\item 带边流形$M$同样有如下拓扑性质:
	\begin{itemize}
		\item $M$具有由坐标球和坐标半球构成的可数预紧拓扑基.
		\item $M$是局部紧的.
		\item $M$是仿紧的.
		\item $M$局部道路连通.
		\item $M$具有至多可数个连通分支,每个分支都是开子集,并且是连通带边流形.
		\item $M$的基本群具有至多可数个元.
	\end{itemize}
\end{enumerate}

\subsection{光滑结构}

定义.设$M$是一个带边或不带边拓扑流形.
\begin{itemize}
	\item 如果$A\subset\mathbb{R}^n$是未必开的非空子集,一个映射$f:A\to\mathbb{R}^k$称为光滑的,如果每个点$x\in A$都存在足够小的$\mathbb{R}^n$的开邻域$U$,使得存在$U\to\mathbb{R}^k$的光滑映射$\widetilde{f}$满足$\widetilde{f}\mid U\cap A=f\mid U\cap A$.(对此有这样一个结论,如果$U$是$\mathbb{H}^n$的开子集,映射$F:U\to\mathbb{R}^k$是光滑的当且仅当它是连续的,$F\mid\mathrm{Int}\mathbb{H}^n\cap U$是常义光滑的,并且$F\mid\mathrm{H}^n\cap U$的全部任意阶偏导数都可以连续延拓到整个$U$).
	\item 称两个坐标卡$(\varphi_i,U_i,V_i),i=1,2$是光滑相容的,如果
	映射$\varphi_{12}=\varphi_2\circ\varphi_1^{-1}:\varphi_1(U_1\cap U_2)\to\varphi_2(U_1\cap U_2)$
	是微分同胚.这个映射称为这两个坐标卡的过渡映射.
	\item 一组坐标卡$(\varphi_i,U_i)$称为一个图册(atlas),如果$\cup_iU_i=M$.称为光滑图册,如果还满足任意两个坐标卡都是相容的.两个光滑图册称为相容的,如果它们的坐标卡互相也是相容的,或者换句话讲它们的并还是一个光滑图册.相容是光滑图册上的等价关系.每个等价类中存在唯一一个包含序下的极大元,称为极大光滑图册.
	\begin{proof}
		
		验证光滑相容是光滑图册上的等价关系.自反性和对称性是直接的.传递性:如果$S_1,S_2,S_3$是$M$上的三个光滑图册,满足$S_1$和$S_2$相容,$S_2$和$S_3$相容.现在任取$S_1$中的坐标卡$(\varphi,U)$,任取$S_3$中的坐标卡$(\psi,V)$.需验验证$\psi\circ\varphi^{-1}:\varphi(U\cap V)\to\psi(U\cap V)$是微分同胚.按照对称性仅需验证它是光滑的.任取$p\in U\cap V$,记$x=\varphi(p)$,按照图册定义,可取$S_2$中的坐标卡$(\theta,W)$使得$p\in W$.按照$S_1$和$S_2$相容,说明$\theta\circ\varphi^{-1}$在点$x$是光滑的,记$\theta\circ\varphi^{-1}(x)=y$,按照$S_2$和$S_3$相容说明$\psi\circ\theta^{-1}$在点$y$是光滑的,于是它们的复合映射$\psi\circ\varphi^{-1}$在点$x$光滑,于是$\psi\circ\varphi^{-1}$是光滑的.
	\end{proof}
	\item 拓扑流形上的一个光滑结构是指一个极大光滑图册,或者等价的讲是指赋予一个光滑图册的等价类.赋予光滑结构的流形称为光滑流形.另外(尽管这里我们不会用到)如果我们仅要求过渡映射是$\mathrm{C}^k$的,那么相容的图册称为$\mathrm{C}^k$结构,赋予这样结构的流形称为$\mathrm{C}^k$流形.在这个描述下$\mathrm{C}^0$流形就是拓扑流形.另外还可以类似定义实解析流形,复流形等.今后我们提及流形或者带边流形如果不加说明总是指光滑的情况.
\end{itemize}

一些构造和例子.
\begin{enumerate}
	\item 开子流形.设$M$是一个$n$维光滑流形,取非空开子集$U\subset M$,选取$M$极大图册中那些落在$U$中的坐标卡,它们构成了$U$上的一个极大图册.它视为$U$上的标准光滑结构,此时称$U$为$M$的开子流形.
	\item 零维流形$M$就是一个至多可数的离散空间.它的坐标开集只能是单点子集,因为$\mathbb{R}^0$是单点集,于是此时流形上的全部坐标卡平凡的构成一个极大光滑图册.此时流形仅有唯一的光滑结构.
	\item 欧氏空间$\mathbb{R}^n$上选取单个整体坐标卡$(\mathrm{id}_{\mathbb{R}^n},\mathbb{R}^n)$是一个光滑结构,它称为欧氏空间上的标准光滑结构.但是还存在不同的光滑结构,例如在实直线$\mathbb{R}$上选取整体坐标卡$(\psi,\mathbb{R})$,其中$\psi(x)=x^3$.这个坐标卡并不和标准光滑结构相容,因为$\mathrm{id}_{\mathbb{R}}\circ\psi^{-1}(y)=y^{1/3}$并不在原点光滑.于是这个坐标卡定义出来的光滑结构和标准的不同.
	\item 设$V$是有限维实线性空间,它的范数诱导了拓扑,并且这个拓扑不依赖于范数的选取.这个拓扑使得$V$成为一个$n$维拓扑流形.给定有序基$(E_1,E_2,\cdots,E_n)$可定义一个同构同胚映射$E:\mathbb{R}^n\to V$为$E(x)=\sum_{1\le i\le n}x^iE_i$,那么$(V,E^{-1})$是一个坐标卡.如果给出另一组基,两个坐标卡的过渡映射是线性同构,于是它是微分同胚,据此我们得到了$V$上一个不依赖于基选取的光滑结构,称为有限维实线性空间上的标准光滑结构.
	\item 我们把矩阵空间$\mathrm{M}(m\times n,\mathbb{R})$等同于$\mathbb{R}^{mn}$.此时一般线性群$\mathrm{GL}(n,\mathbb{R})$是$\mathrm{M}(n\times n,\mathbb{R})$的开子集,它作为开子流形的光滑结构视为它的标准光滑结构.类似的考虑$M(m\times n,\mathbb{R}),m<n$,那么矩阵秩$m$就是在说存在$m$阶的非零子式,或者写作全体$m$阶子式的平方和非零,于是全体秩$m$矩阵构成开子集,它作为开子流形的光滑结构视为它的标准光滑结构.
	\item 设$U\subset\mathbb{R}^n$是非空开子集,设$f:U\to\mathbb{R}^k$是光滑映射.它的图像是指空间$\Gamma(f)=\{(x,y)\in\mathbb{R}^n\times\mathbb{R}^k\mid x\in U,y=f(x)\}$.我们有连续的投影映射$\pi:\Gamma(f)\to\mathbb{R}^n,(x,y)\mapsto x$,它有连续的逆映射$x\mapsto(x,f(x))$,于是它是同胚,于是$(\pi,\Gamma(f))$是一个整体坐标卡.它确定光滑结构称为图像上的标准光滑结构.
	\item 考虑$\mathbb{R}^{n+1}$的单位球面$\mathbb{S}^n$.记$U_i^+$是第$i$分量大于0的点构成的$\mathbb{R}^{n+1}$的开子集,类似定义$U_i^-$,这里$i=1,2,\cdots,n+1$.考虑函数$f:\mathbb{D}^n\to\mathbb{R}$为$f(u)=\sqrt{1-|u|^2}$,那么$U_i^+\cap\mathbb{S}^n$是连续函数$x^i=f(x^1,\cdots,\hat{x^i},\cdots,x^{n+1})$的图像.类似的$U_i^-\cap\mathbb{S}^n$是连续函数$x^i=-f(x^1,\cdots,\hat{x^i},\cdots,x^{n+1})$的图像.这样我们得到了$\mathbb{S}^n$上$2n+2$个坐标卡$(\varphi_i^{\pm},U_i^{\pm})$使得它成为$n$维拓扑流形.验证这些坐标卡相容是直接的,如果$i\not=j$,不妨设$i<j$,那么我们有:
	$$\varphi_i^{\pm}\circ(\varphi_j^{\pm})^{-1}(u^1,u^2,\cdots,u^n)=\left(u^1,\cdots,\hat{u^i},\cdots,\pm\sqrt{1-|u|^2},\cdots,u^n\right)$$
	
	如果$i=j$此时$\varphi_i^+\circ(\varphi_i^-)^{-1}=\varphi_i^-\circ(\varphi_i^+)^{-1}=\mathrm{id}_{\mathbb{D}^n}$.于是这些坐标卡定义了一个光滑结构,视为球面上的标准光滑结构.
	\item 水平集.设$U\subset\mathbb{R}^n$是非空开集,设$\Phi:U\to\mathbb{R}^n$是光滑函数.对每个点$c\in\mathbb{R}$,纤维$M=\Phi^{-1}(c)$通常称为$\Phi$的水平集(level set).设$M$非空,设$\Phi$在$M$中每个点的全微分都非零,那么对每个点$a\in M$,至少存在一个偏导数$\frac{\partial\Phi}{\partial x^i}(a)\not=0$,那么隐函数定理说明存在$a$的足够小的开邻域$U_0$使得$M\cap U_0$可以表示为一个形如$x^i=f(x^1,\cdots,\hat{x^i},\cdots,x^n)$的光滑函数的图像.这使得$M$成为一个$n-1$维拓扑流形.并且这些$(f,U_0)$定义了一个光滑结构.
\end{enumerate}

光滑流形的坐标卡粘合引理.设$M$是一个集合,存在一族子集$\{U_i\}$和一族映射$\varphi_i:U_i\to\mathbb{R}^n$满足如下条件,那么$M$具有唯一的光滑结构使得$(\varphi_i,U_i)$是一个光滑图册.
\begin{itemize}
	\item $\varphi_i$总是$U_i\to\varphi_i(U_i)\subset\mathbb{R}^n$的双射.
	\item 对指标$i,j$,总有$\varphi_i(U_i\cap U_j)$是$\mathbb{R}^n$的开子集.
	\item 映射$\varphi_j\circ\varphi_i^{-1}:\varphi_i(U_i\cap U_j)\to\varphi_j(U_i\cap U_j)$总是光滑的.
	\item $\{U_i\}$存在可数开覆盖.
	\item 设$p,q\in M$是两个不同点,那么要么存在同一个$U_i$覆盖了这两个点,要么存在不交的$U_i,U_j$分别覆盖这两个点.
\end{itemize}
\begin{proof}
	
	全体形如$\varphi_i^{-1}(V)$,其中$i$跑遍指标,$V$跑遍$\mathbb{R}^n$开子集,构成了$M$上的拓扑基:首先它们覆盖了整个$M$,另外有$\varphi_i(V)\cap\varphi_j(W)=\varphi_i^{-1}\left(V\cap(\varphi_j\circ\varphi_i^{-1})^{-1}(W)\right)$.这个拓扑定义为$M$上的拓扑.于是$M$是局部欧氏空间的.第四条保证第二可数性,第五条保证Hausdorff条件,第三条保证这些坐标卡构成光滑图册.
\end{proof}

光滑函数和光滑映射.设$M,N$是带边或不带边光滑流形,设$U\subset M$是非空开子集.
\begin{itemize}
	\item $U$上的实值函数$f$称为光滑函数,如果对任意光滑坐标卡$(\varphi,V)$,都有$f\circ\varphi^{-1}:\varphi(U\cap V)\to\mathbb{R}$是欧氏空间上的光滑函数,这个函数会称为$f$关于这个坐标卡的坐标表示.按照光滑坐标卡之间的过渡映射总是光滑的,这个定义中的任意光滑坐标卡可以改为任取一组覆盖了$U$的光滑坐标卡.开集$U$上的全体光滑函数构成一个$\mathbb{R}$代数,记作$\mathrm{C}^{\infty}(U)$.
	\item 映射$F:M\to N$称为光滑映射,如果对任意$M$上的光滑坐标卡$(\varphi,U)$和任意$N$上的光滑坐标卡$(\psi,V)$,总有$\psi\circ F\circ\varphi$是$\varphi(U)\to\psi(V)$的欧氏空间上的光滑映射,这个应该会称为$F$关于这两个坐标卡的坐标表示.类似的按照光滑坐标卡之间的过渡映射总是光滑的,定义中的任意光滑坐标卡可以改为任取一组覆盖了整个空间是光滑坐标卡.
	\item 恒等映射总是光滑的,光滑映射的复合总是光滑的,于是光滑流形和光滑映射构成一个范畴.它的同构称为微分同胚,这吻合于欧氏空间中微分同胚的定义.
\end{itemize}
\begin{enumerate}
	\item 我们可以定义函数或者映射在一点处的光滑性,以函数为例,设$f$是开集$U\subset M$上的实函数,称它在点$p\in U$处是光滑的,如果存在(按照坐标卡过渡映射总是光滑的,这里也等价于讲任取$p$的坐标邻域均有如下事情成立)$p$的坐标邻域$U$,使得$f\circ\varphi^{-1}$是$\varphi(U)\to\mathbb{R}$的光滑函数.那么函数$f$是$U$上的光滑函数当且仅当它在$U$上任意点光滑.这说明光滑是一个局部性质:光滑映射或者光滑函数在开子集上的限制总是光滑的;一个整体函数或者映射如果在一个开覆盖的每个开子集上的限制都是光滑的,那么整体上它是光滑的.
	\item 把欧氏空间视为光滑流形,我们的光滑函数和光滑映射的定义吻合于欧氏空间的情况.另外如果把实直线视为光滑流形,那么光滑函数就是终端为实直线的光滑映射.
	\item 光滑映射和光滑函数自然都是连续的,因为局部上都是连续映射的复合.
	\item 常值函数总是光滑的,恒等映射总是光滑的,开子集的包含映射总是光滑的,光滑映射的复合总是光滑的.按照定义坐标映射总是微分同胚.
	\item 积对象.一组光滑流形$M_1,M_2,\cdots,M_r$的积对象就是它们的笛卡尔积$M=M_1\times M_2\times\cdots\times M_r$,它的拓扑是积拓扑,相容光滑坐标卡就是分量坐标卡的笛卡尔积.另外如果$N$是光滑流形,一个映射$F:N\to M$是光滑的当且仅当每个限制映射$F_i=\pi_i\circ F:N\to M_i$都是光滑的.但是由于$\mathbb{H}^n\times\mathbb{H}^m$并不是某个欧氏空间的闭上半空间,边界非平凡的带边流形就没法定义积空间.对此的一个解决办法是引入带角流形.
	\item 我们解释过拓扑流形上可能存在多个光滑结构,一个更有趣的问题是是否存在互相不微分同胚的光滑结构.例如我们给出过实直线上的两个光滑结构分别由整体坐标卡$(x,\mathbb{R})$和$(x^3,\mathbb{R})$提供,但是它们实际上是经$F:x\mapsto x^{1/3}$微分同胚的.事实上我们会证明实直线上在微分同胚意义下只有唯一的光滑结构.另外我们还有一些稍微不平凡的结论:
	\begin{itemize}
		\item 维数不超过3的拓扑流形上在微分同胚意义下存在唯一的光滑结构.
		\item $\mathbb{R}^n$在$n\not=4$时在微分同胚意义下存在唯一的光滑结构.离奇的是$n=4$时即便在微分同胚意义下存在不可数个光滑结论.
		\item $\mathbb{S}^7$上在微分同胚意义下存在15个不同的光滑结构.
		\item 维数大于3的紧拓扑流形上不存在光滑结构.
	\end{itemize}
\end{enumerate}

\subsection{单位分解}

光滑碰撞函数(smooth bump function).记$C^{\infty}_0(M)$是$M$上全体具有紧支集的光滑函数,它是$C^{\infty}(M)$的理想.$M$是光滑流形,对于任意一对开集$U$和所包含的紧集$K$,称相伴的光滑碰撞函数是指$\varphi\in C^{\infty}_0(M)$使得$0\le \varphi\le1$,在$K$上恒为1,并且$\mathrm{supp}(\varphi)\subset U$.下面证明存在性.
\begin{proof}
	
	构造$R$上函数$h(x)$使得它在$|x|\ge2$取0,在$|x|\le1$取1,在剩下的两个区间上光滑连接,使得$0\le h(x)\le1$ 恒成立.为此可以考虑$f(x)$在$x>0$取$e^ {-1/x}$,在$x\le0$ 取0,则它是$R$上光滑函数,取$g(x)=\frac{f(x)} {f(x)+f(1-x)}$,再取$h(x)=g(2-|x|)$即可.
	
	\qquad
	
	对于任意$p\in K$,取坐标卡$(\varphi_p,U_p,B(0,3))$,使得$U_p\subset U$,取$V_p=\varphi_p^{-1}(B(0,1))$,令:
	$$f_p(x)=\left\{\begin{array}{cc}
	h(\varphi_p(x))   & x\in U_p\\
	0  &  x\not\in U_p\end{array}\right.$$
	
	那么$f_p\in C^{\infty}_0(M),\mathrm{supp}(f_p)\subset U_p$并且$f_p$在$V_p$上恒为1.下面$\{V_p,p\in K\}$是紧集$K$的开覆盖,于是存在有限字覆盖$V_ {q_1},\cdots,V_{q_r}$,那么取$\theta(x)=\sum_{i=1}^rf_{p_i}(x)\in C^{\infty}_0(M)$,则$\theta(x)$在$K$上$\ge1$,紧支集在$U$内,最后取$g(\theta(x))$为所求.
\end{proof}

单位分解定理.$M$是带边或不带边光滑流形,对于它的任意开覆盖$\{U_{\alpha}:\alpha\in J\}$,存在关于它的单位分解,即一族同指标集的$M$上光滑函数$\{\rho_{\alpha}:\alpha\in J\}$满足:
\begin{itemize}
	\item $0\le\rho_{\alpha}\le1,\alpha\in J$.
	\item $\mathrm{supp}(\rho_{\alpha})\subset U_{\alpha},\alpha\in J$.
	\item $\{\mathrm{supp}(\rho_{\alpha}):\alpha\in J\}$是局部有限的.
	\item $\sum_{\alpha\in J}\rho_{\alpha}(p)=1,\forall p\in M$.第三条保证对每个点$p$这个求和总是只有至多有限项非零,于是求和有意义.
\end{itemize}
\begin{proof}
	
	对于任意一个开覆盖$\{U_{\alpha}:\alpha\in J\}$,我们证明过它存在可数的局部有限的开加细覆盖$\{W_i,i\ge1\}$使得每个开集是预紧的,由于$M$局部同胚$\mathbb{R}^n$并且第二可数,存在可数的开覆盖$\{V_i,i\ge1\}$使得每个$V_i$是预紧的并且$\overline{V_i}\subset W_i$.由光滑碰撞函数存在性,对于每个$i\ge1$,存在$C^{\infty}_0(M)$中的函数$\varphi_i$使得它取值$0\le\varphi_i\le1$,在$\overline{V_i}$上恒为1, 并且$\mathrm{supp}(\varphi_i)\subset W_i$.由于$\{W_i,i\ge1\}$是局部有限的,从而可以定义光滑的函数$\varphi=\sum_{i=1}^{\infty}\varphi_i$,再取$\psi_i=\frac{\varphi_i}{\varphi}$,则它光滑并且$\sum_ {i=1}^{\infty}\psi_i=1$.
	
	\qquad
	
	最后,对每个$i\ge1$固定$J$中的指标$\alpha(i)$使得$W_i\subset U_{\alpha(i)}$,定义$\rho_{\alpha}=\sum_{\alpha(i)=\alpha}\psi_i$则$\{\psi_{\alpha}:\alpha\in J\}$为所求.
\end{proof}

单位分解定理的一些应用.
\begin{enumerate}
	\item 我们之前给出的碰撞函数是要求$K$是紧集,借助单位分解定理可以证明仅要求$K$是闭集就已经存在碰撞函数.
	\begin{proof}
		
		设$U\subset M$是开集,$K\subset U$是闭集.记$U_0=U,U_1=K^c$,这是$M$的开覆盖,于是存在单位分解$\{\psi_0,\psi_1\}$.那么这里$\psi_0$在$K$上的取值是$1-\psi_1=1$,在$U$以外的取值是$0$,满足要求.
	\end{proof}
    \item 光滑函数延拓引理.设$M$是光滑流形,设$A\subset M$是非空闭子集,设$f:A\to\mathbb{R}^k$是一个光滑映射(这里闭集为源端的光滑映射$f$是指存在这个闭集的每个点都存在某个开邻域$U$上的光滑映射$g$使得$g$在$U\cap A$上的限制是$f\mid U\cap A$).那么对任意$A$的开邻域$U$,存在光滑映射$g:M\to\mathbb{R}^k$使得$g\mid A=f$并且$\mathrm{Supp}g\subset U$.
    \begin{proof}
    	
    	考虑$A\subset U$上的光滑碰撞函数$\psi$,那么$\psi$在$A$上恒为1,并且支集落在$U$中.设$A$存在开邻域$V$使得存在$V$上光滑函数$f_0$满足$f_0\mid A=f$,不妨设$V\subset U$,否则可以把$f_0$限制在$U\cap V$上.这里$f_0\psi$可以延拓到整个$M$,只要在$N-\mathrm{Supp}\psi$上取零即可.此时$A$上有$f=f\psi$.
    \end{proof}
    \item 光滑exhaustion函数.如果$M$是拓扑空间,$M$上的一个exhaustion函数是指$M\to\mathbb{R}$的连续函数$f$,满足对任意$c\in\mathbb{R}$,都有$f^{-1}((-\infty,c])$是紧集.紧流形上的任意连续函数都是exhaustion函数,所以这个概念只在非紧流形上有实际意义.这里我们断言光滑流形上总存在正的光滑exhaustion函数.
    \begin{proof}
    	
    	设$M$是光滑流形,取$M$的由预紧开子集构成的可数开覆盖$\{V_j\}$,取这个开覆盖的单位分解$\{\psi_j\}$,定义函数$f(p)=\sum_{j\ge1}j\psi_j(p)$.由于每个点都存在开邻域,使得这个求和只有有限项非零,所以这个求和有意义,并且是光滑的.这个函数恒正是因为$f(p)\ge\sum_{j\ge1}\psi_j(p)=1$.最后说明它是exhaustion函数,任取$c\in\mathbb{R}$,记正整数$N>c$,如果$p\not\in\cup_{j=1}^N\overline{V_j}$,那么有$\psi_j(p)=0,\forall 1\le j\le N$.导致:
    	$$f(p)=\sum_{j=N+1}^{+\infty}j\psi_j(p)\ge N\sum_{j=N+1}^{+\infty}\psi_j(p)=N\sum_{j=1}^{+\infty}\psi_j(p)=N>c$$
    	
    	于是$f(p)\le c$导致$p\in\cup_{j=1}^N\overline{V_j}$,于是$f^{-1}((-\infty,c])$是紧子集$\cup_{i=1}^N\overline{V_j}$的闭子集,于是它是紧集.
    \end{proof}
\end{enumerate}

\newpage
\section{切空间}
\subsection{切空间和切丛}

茎,切向量,微分的定义.设$M$是光滑流形,设$p\in M$.
\begin{itemize}
	\item 考虑全体对$(U,f)$,其中$U$是$p$的开邻域,$f\in\mathrm{C}^{\infty}(U)$.两个对视为等价的$(U,f)\sim(V,g)$,如果存在$p$的更小的开邻域$W\subset U\cap V$,使得$f\mid W=g\mid W$.这是全体对上的一个等价关系,等价类称为点$p$处的芽(germ),全体芽构成一个$\mathbb{R}$代数,记作$\mathrm{C}^{\infty}_p(M)$,称为$M$在点$p$处的茎(stalk).
	\item 点$p$处的一个切向量(或者称为方向导数,或者点导数),是指一个$\mathbb{R}$线性映射$D_p:\mathrm{C}^{\infty}_p(M)\to\mathbb{R}$,满足莱布尼兹法则$D_p(fg)=f(p)D_p(g)+g(p)D_p(f)$.点$p$的全体切向量构成了一个$\mathbb{R}$线性空间,记作$\mathrm{T}_pM$.
	\item 设$F:M\to N$是光滑映射,对每个点$p\in M$,它诱导了切空间之间的线性映射$\mathrm{d}F_p:\mathrm{T}_pM\to\mathrm{T}_{F(p)}N$,为对每个切向量$v\in\mathrm{T}_pM$,对每个$F(p)$某个开邻域上的光滑函数$f$,有$\mathrm{d}F_p(v)(f)=v(f\circ F)$.
	\item 函子性.光滑流形和光滑映射构成的范畴记作$\textbf{Diff}$,它的带基点范畴记作$\textbf{Diff}_*$(这是指对象是光滑流形上约定一个点为基点,态射是把基点映射为基点的光滑映射),那么把带基点光滑流形对应为基点的切空间,把带基点光滑流形之间的光滑映射对应为基点切空间之间的线性变换是一个函子,换句话讲有如下结论:设$M,N,P$是光滑流形,设$F:M\to N$和$G:N\to P$是光滑映射,设$p\in M$,那么:
	\begin{enumerate}
		\item 链式法则:$\mathrm{d}(G\circ F)_p=\mathrm{d}G_{F(p)}\circ\mathrm{d}F_p$.
		\item $\mathrm{d}(\mathrm{id}_M)_p=\mathrm{id}_{\mathrm{T}_pM}$.
		\item 如果$F$是微分同胚,那么它诱导的切空间之间的映射是同构.
	\end{enumerate}
\end{itemize}
\begin{enumerate}
	\item 切向量的基本性质.设$M$是光滑流形,设$p\in M$,设$v\in\mathrm{T}_pM$,设$f,g$是点$p$局部上的两个光滑函数.
	\begin{itemize}
		\item 如果$f$是常值函数,那么$v(f)=0$.
		\item 如果$f(p)=g(p)=0$,那么$v(fg)=0$.
	\end{itemize}
    \item 切空间是局部性质,即任取$p\in M$的开邻域$U$,那么有$\mathrm{T}_pM=\mathrm{T}_pU$.
    \item 关于切空间的定义这里有个技术性的小事情.任取开集$p\in U\subset M$上的光滑函数$f$,那么存在$M$整体上的光滑函数$g$使得它们落在点$p$的同一个芽中.这件事导致我们可以把切向量定义为$\mathrm{C}^{\infty}(M)\to\mathbb{R}$的满足莱布尼兹法则的线性函数.
    \begin{proof}
    	
    	我们可以选取$p$的开邻域$B$使得$\overline{B}\subset U$(这是拓扑空间上分离公理中的正则性,但是我们解释过拓扑流形总是度量空间,它满足所有分离性公理).按照光滑函数的延拓引理,就存在$M$上的整体光滑函数$g$使得$\mathrm{Supp}g\subset U$并且$g$在$\overline{B}$上的限制和$f$相同,导致$(f,U)\sim(g,M)$.
    	
    	\qquad
    	
    	另外为了说明这两种定义相同我们还要说明如果$M$上两个整体光滑函数$f,g$在$p$的某个开邻域上相同,如果$D$是这里新定义的切向量,那么它们满足$D(f)=D(g)$.设$f-g=h$,那么$h$在$p$的某个开邻域$U$上相同,考虑$\overline{\mathrm{Supp}(h)}\subset M-\{p\}$上的碰撞函数$\psi$,即$\psi$在$\mathrm{Supp}h$上恒为1,并且支集落在$M-\{p\}$中,这导致$h=h\psi$,导致$D(h)=D(h\psi)=0$.
    \end{proof}
    \item 光滑函数的微分就是给每个点赋予一个线性函数,也就是一个余向量,所以光滑函数的微分总是一个光滑1形式.
    \item 欧氏空间上的切空间.取点$p\in\mathbb{R}^n$,传统的切空间是纯粹几何定义的$\{p\}\times\mathbb{R}^n$,这里暂时称它为几何切空间,我们断言它是和这里定义的切空间同构的.特别的,这说明欧氏空间$\mathbb{R}^n$上点的(这里新定义的)切空间的维数总是$n$,并且在我们新的切空间观点下,$n$个偏导数就是切空间的一组基,每个切向量可以表示为$n$个偏导数的$\mathbb{R}$线性组合.
    \begin{proof}
    	
    	取几何切空间中的一个向量$v$,定义一个方向导数$D_{v,p}$为,对点$p$局部上的光滑函数$f$,有$D_{v,p}(f)=\frac{\mathrm{d}}{\mathrm{d}t}\mid_{t=0}f(a+tv)=\sum_iv^i\frac{\partial f}{\partial x^i}(a)$.这是线性映射并且满足莱布尼兹法则.于是我们得到了几何切空间到切空间的线性映射.
    	
    	\qquad
    	
    	验证单射,因为如果几何切向量$v$使得$D_{v,p}=0$,把它作用在坐标函数$x^i$上得到$0=D_{v,p}(x^i)=v^i$,于是$v=0$.验证满射,任取方向导数$D$,设$D(x^i)=v^i$,记$v=(v^1,v^2,\cdots,v^n)$,我们断言$D_v=D$.任取$p$局部上的光滑函数$f$,按照泰勒展开有:
    	$$f(x)=f(p)+\sum_{i=1}^n\frac{\partial f}{\partial x^i}(p)(x^i-p^i)+\sum_{i,j=1}^n(x^i-p^i)(x^j-p^j)\int_0^1(1-t)\frac{\partial^2f}{\partial x^i\partial x^j}(p+t(x-p))\mathrm{d}t$$
    	
    	这里最后一个和式在方向导数的作用下为零.第一项常数项在方向导数的作用下也是零,于是最后只需验证:
    	\begin{align*}
    	Df&=\sum_{i=1}^nD(\frac{\partial f}{\partial x^i}(p)(x^i-p^i))\\&=\sum_{i=1}^n\frac{\partial f}{\partial x^i}(p)D(x^i)\\&=\sum_{i=1}^n\frac{\partial f}{\partial x^i}(p)v^i=D_v(f)
    	\end{align*}
    \end{proof}
    \item 切空间的维数.设$M$是一个$n$维光滑流形,任取点$p\in M$,那么$\mathrm{T}_pM$是一个$n$维实线性空间.这是因为取点$p$的坐标邻域是到$\mathbb{R}^n$某个开子集的微分同胚,导致它们对应点的切空间同构,但是欧氏空间$\mathbb{R}^n$的切空间的维数总是$n$.另外如果选取点$p$的坐标邻域$(U,\varphi)$,那么点$p$切空间的一组基可以表示为坐标偏导数在切空间同构下的原像.
    \item Jacobian矩阵.设$F:M\to N$是光滑映射,选取$p\in M$和$F(p)\in N$的光滑坐标卡$(U,\varphi),(V,\psi)$,那么$\psi\circ F\circ\varphi^{-1}$是欧氏空间上的光滑映射,它在点$\varphi(p)$处的Jacobian矩阵称为$F$关于这两个坐标卡的Jacobian矩阵.这个定义依赖于坐标卡的选取,但是选取不同坐标卡的过渡映射都是微分同胚,微分同胚诱导了切空间的线性同构,于是不同坐标卡得到的Jacobian是相抵的(两个方阵$M,N$相抵如果存在可逆矩阵$P,Q$使得$M=PNQ$),于是Jacobian矩阵的秩是不依赖于坐标卡选取的概念,称为$F$在一点处的秩.
    \item 有限维实线性空间$V$上的切空间.对每个点$p\in V$,对每个向量$v\in V$,我们定义$D_{V,p}:\mathrm{C}^{\infty}_p(M)\to\mathbb{R}$为$D_{V,p}(f)=\frac{\mathrm{d}}{\mathrm{d}t}\mid_{t=0}f(p+tv)$.这里右侧导数有定义是因为赋范空间上极限有定义.这里$v\mapsto D_{v,p}$是从$V$到$\mathrm{T}_pV$的典范同构,并且它具有函子性,即对任意有限实维线性空间之间的线性映射$L:V\to W$,都有如下交换图表:
    $$\xymatrix{V\ar[rr]^{\cong}\ar[d]_L&&\mathrm{T}_pV\ar[d]^{\mathrm{d}L_p}\\W\ar[rr]_{\cong}&&\mathrm{T}_{L(a)}W}$$
    \item 积对象上的切空间.如果$M_1,M_2,\cdots,M_r$是光滑流形,记$M=M_1\times M_2\times\cdots\times M_r$,记投影映射$\pi_i:M\to M_i$,那么有同构$\mathrm{T}_pM\cong\oplus_{i=1}^r\mathrm{T}_{p_i}M_i$为$v\mapsto\left((\mathrm{d}\pi_1)_p(v),(\mathrm{d}\pi_2)_p(v),\cdots,(\mathrm{d}\pi_r)_p(v)\right)$.
    \item 切空间的等价描述.设$M$是光滑流形,其上一条光滑曲线是指从开区间$J$到$M$的光滑映射$\gamma$,这条曲线在点$t_0\in J$的速度定义为$\gamma'(t_0)=\mathrm{d}\gamma\left(\frac{\mathrm{d}}{\mathrm{d}t}\mid_{t_0}\right)\in\mathrm{T}_{\gamma(t_0)}M$.对$\gamma(t_0)$附近的光滑函数$f$,按照定义有$\gamma'(t_0)(f)=(f\circ\gamma)'(t_0)$.我们断言每个切向量都可以表示为过该点的某条光滑曲线的速度.据此可以证明点$p$的切空间可以描述为全体过点$p$的光滑曲线在如下等价关系下的等价类集合:两条曲线$\gamma_1,\gamma_2$,约定$\gamma_1(0)=\gamma_2(0)=p$,它们等价定义为存在$p$附近的某个光滑函数$f$使得$(f\circ\gamma_1)'(0)=(f\circ\gamma_2)'(0)$.
\end{enumerate}

切丛定义.设$M$是光滑流形,我们构造一个$2n$维光滑流形称为切丛,记作$\mathrm{T}M$.
\begin{itemize}
	\item $M$的切丛$\mathrm{T}M$作为集合是$M$上所有切空间的无交并$\coprod_{p\in M}\mathrm{T}_pM$.它的点可以表示为$(p,v)$,其中$p\in M$,$v\in\mathrm{T}_pM$.我们有典范的投影映射$\pi:\mathrm{T}M\to M$为$(p,v)\mapsto p$.
	\item 对$M$上每个光滑坐标卡$(U,\varphi)$,我们有从$\mathrm{T}U\to\varphi(U)\times\mathbb{R}^n\subset\mathbb{R}^{2n}$的双射为$\Phi:(p,v=\sum_iv^i\partial_i)\mapsto(\varphi(p),v^1,v^2,\cdots,v^n)$.就定义$\mathrm{T}M$上的拓扑是使得这些坐标卡$(\mathrm{T}U,\Phi)$总是同胚的拓扑.为了验证定义良性还需要说明在坐标卡相交的地方过渡映射是同胚,这是因为任取两个坐标卡$(U_1,\varphi_1)$和$(U_2,\varphi_2)$,如果记$p\in U_1\cap U_2$,记$\varphi_1(p)=(x^1,x^2,\cdots,x^n)=x$,记$\varphi_2(p)=(y^1,y^2,\cdots,y^n)$.那么有:
	$$\Phi_2\circ\Phi^{-1}(x^1,x^2,\cdots,x^n,v^1,v^2,\cdots,v^n)=\left(\varphi_2\circ\varphi_1(x),\sum_j\frac{\partial y^1}{\partial x^j}v^j,\cdots,\sum_j\frac{\partial y^n}{\partial x^j}v^j\right)$$
	
	这过渡映射甚至是欧氏空间中的微分同胚(因为一般性告诉我们逆映射还是光滑的).
	\item 验证切丛的光滑流形条件.上面构造已经说明$\mathrm{T}M$局部同胚于欧氏空间$\mathbb{R}^{2n}$.如果我们取$M$上可数个光滑坐标卡,对应的切丛上可数个坐标卡覆盖了整个切丛,于是切丛是第二可数的.Hausdorff条件是因为如果取切丛中的两个点$(p,v),(q,w)$,如果$p=q$那么这两个点被同一个坐标卡覆盖,所以它们可以被开集分离;如果$p\not=q$那么按照$M$上的Hausdorff条件存在开集分离点$p$和$q$,导致切丛上存在两个不交开集分离这两个点$(p,v),(q,w)$.这些坐标卡的光滑相容性上一条已经验证,于是综上得到切丛上的$2n$维光滑流形结构,这视为切丛上的标准光滑流形结构.
\end{itemize}
\begin{enumerate}
	\item 投影映射$\pi$是一个光滑映射.
	\item 如果$M$是$n$维光滑流形,具有整体光滑坐标卡,那么$\mathrm{T}M$微分同胚于$M\times\mathbb{R}^n$.
	\item (整体)微分.如果$F:M\to N$是光滑映射,构造映射$\mathrm{d}F:\mathrm{T}M\to\mathrm{T}N$为,在每个$\mathrm{T}_pM$上的限制就是(局部)微分映射$\mathrm{d}_pF$.它也称为微分映射,为了区分可称为整体微分映射.这具有函子性,即它保复合,保恒等映射,保同构.
\end{enumerate}

\subsection{光滑映射的秩}

给定光滑映射$F:M\to N$,我们之前解释过$F$在一点$p\in M$的Jacobian矩阵是依赖于光滑坐标卡的选取,但是它的秩是和坐标卡无关的概念,把这个秩称为$F$在点$p$的秩.它就是线性变换$\mathrm{d}F_p:\mathrm{T}_pM\to\mathrm{T}_{F(p)}N$的秩.
\begin{itemize}
	\item 如果$F$在某个开集上的秩总为$r$,就称它是这个开集上的常秩$r$映射,并记$\mathrm{rank}F\mid U=r$.
	\item 我们知道光滑映射$F$在一点$p$的秩的最大值是$\min\{\dim M,\dim N\}$,如果$F$在某点取到最大值,就称$F$在该点是满秩的,如果$F$处处满秩,就称$F$是满秩的.
	\item 两种特殊的满秩映射.如果$F$的微分处处是满射,也即$\mathrm{rank}F=\dim N$,则称$F$是光滑浸没(smooth submersion);如果$F$的微分处处是单射,也即$\mathrm{rank}F=\dim M$,则称$F$是光滑浸入(smooth immersion).
\end{itemize}
\begin{enumerate}
	\item 如果$F$在一点$p$处满秩,那么存在点$p$的开邻域上$F$处处满秩.特别的,这件事说明如果$F$在一点处的微分是满射,那么它在该点的附近(即某个开邻域上)是光滑浸没;如果$F$在一点处的微分是单射,那么它在该点的附近是光滑浸入.
	\begin{proof}
		
		选取相应坐标卡使得问题转移到欧氏空间上,从源端到$\mathrm{M}(m\times n,\mathbb{R})$的每个点取Jacobian矩阵的映射是连续的,但是$m\times n$矩阵空间上的满秩矩阵构成开子集,按照连续性,在满秩点的附近总是满秩的.
	\end{proof}
    \item 局部微分同胚.一个光滑映射$F:M\to N$称为局部微分同胚,如果每个点$p\in M$都存在开邻域$U$使得$F(U)$是开集,并且$F$在上面的限制是微分同胚.一些基本性质:
    \begin{itemize}
    	\item 局部微分同胚的复合还是局部微分同胚.
    	\item 局部微分同胚总是局部同胚和开映射.
    	\item 局部微分同胚在源端的开子集上的限制还是局部微分同胚.
    	\item 微分同胚自然是局部微分同胚,反过来双射的局部微分同胚总是微分同胚.
    \end{itemize}
    \item 流形上的逆映射定理.设$F:M\to N$是光滑映射,如果它在点$p$的微分$\mathrm{d}F_p$是可逆的,那么$F$在点$p$是局部微分同胚,即存在$p$和$F(p)$的开邻域$U_0,V_0$使得$F$在$U_0$上的限制是$U_0\to V_0$的微分同胚.这个逆命题自然也是成立的,即如果$F$在点$p\in M$是局部微分同胚,那么它在点$p$处的微分是可逆的.
    \item 上一条说明$F$是局部微分同胚当且仅当它同时是光滑浸入和光滑浸没.另外运用一点线性代数,如果$\dim M=\dim N$,如果$F$是光滑浸入或者光滑浸没,那么$F$是局部微分同胚.
    \item 秩定理.如果$M,N$是秩分别为$m,n$的光滑流形,设$F:M\to N$是常秩$r$的光滑映射.那么对每个点$p\in M$,存在以$p$为中心的坐标卡$(U,\varphi)$和以$F(p)$为中心的坐标卡$(V,\psi)$,使得$F(U)\subset V$,并且有坐标表示:
    $$\hat{F}(x^1,x^2,\cdots,x^r,x^{r+1},\cdots,x^m)=(x^1,x^2,\cdots,x^r,0,\cdots,0)$$
    
    特别的,如果$F$是光滑浸没,那么坐标表示为:
    $$\hat{F}(x^1,x^2,\cdots,x^n,x^{n+1},\cdots,x^m)=(x^1,x^2,\cdots,x^n)$$
    
    如果$F$是光滑浸入,那么坐标表示为:
    $$\hat{F}(x^1,x^2,\cdots,x^m)=(x^1,x^2,\cdots,x^m,0,\cdots,0)$$
    \item 推论.如果$F:M\to N$是光滑映射,并且$M$是连通的,那么$F$是局部常秩的当且仅当它在每个点都存在坐标表示是线性的.
    \item 推论.按照秩定理,光滑浸没总是一个开映射.
    \item 整体秩定理.设$F:M\to N$是常秩光滑映射.
    \begin{itemize}
    	\item 如果$F$是满射,那么它是光滑浸没.
    	\item 如果$F$是单射,那么它是光滑浸入.
    	\item 如果$F$是双射,那么它是微分同胚.
    \end{itemize}
    \begin{proof}
    	
    	设$\dim M=m$和$\dim N=n$,设$F$的秩为$r$.先设$F$是满射,假设它不是光滑浸没,那么有$r<n$.按照秩定理,每个点$p$和$F(p)$分别存在中心坐标卡$(U,\varphi)$和$(V,\psi)$,满秩$F(U)\subset V$,并且坐标映射是投影映射.适当缩小$U$,我们可以约定$\overline{U}$落在一个有界的坐标邻域里,并且$F(\overline{U})\subset V$,这导致$F(\overline{U})$是紧集,并且它落在子集$S=\{y\in V\mid y^{r+1}=\cdots=y^n=0\}$,这导致$F(\overline{U})$是$N$的闭子集,并且没有内点.按照第二可数条件,可取$M$的可数个坐标卡$\{(U_i,\varphi_i)\}$,记相应的$N$上坐标卡为$\{V_i,\psi_i\}$.但是这里$F(M)$是可数个无处稠密子集的并,按照Baire纲定理,$F(M)$在$N$中没有内点,导致$F$不会是满射,这矛盾.
    	
    	\qquad
    	
    	再设$F$是单射.这个是容易的,如果$F$不是光滑浸入,那么$r<m$,导致秩定理中的局部坐标表示局部上不是单射,这自然和单射矛盾.最后如果$F$是双射,前两条结论导致$F$是局部微分同胚,但是我们解释过双射时候局部微分同胚就是微分同胚.
    \end{proof}
\end{enumerate}

光滑嵌入.一个光滑浸入$F:M\to N$如果还满足它是一个拓扑嵌入,即$F$是从$M$到$F(M)$的同胚,就称$F$是一个光滑嵌入.
\begin{enumerate}
	\item 设$F:M\to N$是单的光滑浸入,那么$F$是光滑嵌入等价于如下任一条件成立:
	\begin{itemize}
		\item $F$是开或闭映射.
		\item $F$是真映射(紧集的原像紧).
		\item $M$是紧的.
		\item $\dim M=\dim N$.
	\end{itemize}
    \item 设$F:M\to N$是光滑映射,那么$F$是光滑浸入当且仅当$M$的每个点局部上都是光滑嵌入.即对每个点$p\in M$,存在开邻域$U$使得$F\mid U:U\to N$是光滑嵌入.
    \begin{proof}
    	
    	必要性是直接的,局部上是光滑嵌入说明每个点的微分都是单射,导致$F$是光滑浸入.充分性,设$F$是光滑浸入,任取点$p\in M$,按照秩定理存在开邻域$U$使得$F$在其上限制是单射,并且局部坐标表示为投影映射.再选取预紧开邻域$U_1$使得$\overline{U_1}\subset U$,于是$F$在$U_1$上的限制是拓扑嵌入:$U_1$的闭子集$E$是$\overline{U_1}$的闭子集,紧集的闭子集紧,于是$F(E)$是$N$中紧集,但是Hausdorff空间的紧子集是闭的,于是$F(E)$是闭集,于是$F\mid U_1$是闭映射,而$F\mid U_1$还是单的光滑浸入,于是$F\mid U_1$是拓扑嵌入.
    \end{proof}
\end{enumerate}

\subsection{向量场}

向量场的定义.
\begin{itemize}
	\item 设$M$是光滑流形,它的向量场$X$是指切丛的典范映射$\pi:\mathrm{T}M\to M$的连续截面映射,换句话讲有$\pi\circ X=\mathrm{id}_X$.换句话讲向量场$X$就是对每个点$p$赋予一个相应的切向量$X_p$.如果这个截面映射还是光滑的,就称它是光滑向量场.这里我们提及向量场如果不加说明总是光滑向量场.
	\item 向量场$X$的支集是指$\{p\in M\mid X_p\not=0\}$的闭包,如果这个支集是紧集,就称$X$是紧支撑的.
	\item 设$A\subset M$是光滑流形的任意子集,一个沿$A$的连续/光滑向量场是指以$A$为源端的$\pi:\mathrm{T}M\to M$的连续/光滑截面.
	\item $M$上的全体光滑向量场构成一个$\mathrm{C}^{\infty}(M)$模,记作$\chi(M)$.
\end{itemize}
\begin{enumerate}
	\item 给定$M$上的甚至未必连续的向量场$X$,任取$M$上的光滑坐标卡$(U,(x^i))$,那么在$U$上$X$可以表示为$X_p=\sum_iX^i(p)\partial_i\mid_p$.这里$X^i:U\to\mathbb{R}$称为$X$关于该坐标卡的分量函数.我们断言$X$在$U$上光滑当且仅当每个分量函数都是光滑的.
	\begin{proof}
		
		设$(U,(x^i))$对应的切丛上的光滑坐标卡为$(\pi^{-1}(U),(x^i,v^i))$.那么向量场$X:M\to\mathrm{T}M$在$U$上的坐标表示为$(x^1,x^2,\cdots,x^n)\mapsto(x^1,x^2,\cdots,x^n,X^1(x),\cdots,X^n(x))$.这说明$X$的光滑性等价于坐标分量$X^i$的光滑性.
	\end{proof}
    \item 向量场的延拓引理.设$A\subset M$是闭子集,如果$X$是沿$A$的光滑向量场,任取包含$A$的开子集$U$,那么存在$M$上的整体光滑向量场$Y$使得$Y\mid A=X$,并且$\mathrm{Supp}Y\subset U$.
    \begin{proof}
    	
    	只要选取$A\subset U$的光滑碰撞函数$f$,构造$Y=f(p)X$即可.
    \end{proof}
    \item 特别的,上一条延拓引理说明,如果任取$p\in M$,任取$v\in\mathrm{T}_pM$,那么存在$M$上的整体光滑向量场$X$使得$X_p=v$.证明只要在延拓定理中取$A=\{p\}$,$U=M$.
\end{enumerate}

标架.设$M$是$n$维光滑流形.
\begin{itemize}
	\item 一个有序$k$个沿子集$A$的向量场$(X_1,X_2,\cdots,X_k)$称为线性无关的,如果对每个$p\in A$都有$\{X_1\mid_p,X_2\mid_P,\cdots,X_k\mid_p\}$都是$\mathrm{T}_pM$中的线性无关组.称这个有序$k$个沿子集$A$的向量场生成了切丛,如果对每个$p\in A$都有$\{X_1\mid_p,X_2\mid_P,\cdots,X_k\mid_p\}$生成了$\mathrm{T}_pM$.
	\item $M$上的一个局部标架是指某个非空开集$U\subset M$上的有序$n$个向量场$(X_1,X_2,\cdots,X_n)$,使得它们线性无关并且生成了切丛.当$U=M$成立时称它们是整体标架.如果流形存在整体标架,则称它是可平行化的(parallelizable).
\end{itemize}
\begin{enumerate}
	\item 如果光滑流形的维数是$n$,那么它的标架中的向量场的个数必须是$n$.为了验证$n$个向量场的确是标架,只需验证它们是线性无关的或者生成了切丛.
	\item 如果$(X_1,X_2,\cdots,X_k)$是开集$U\subset M$上的线性无关的光滑向量场,对每个点$p\in U$,存在开邻域$V\subset U$使得存在$V$上向量场$X_{k+1},\cdots,X_n$使得$\{X_1,X_2,\cdots,X_n\}$是$V$上的光滑局部标架.
	\item 如果$\{v_1,v_2,\cdots,v_k\}$是$\mathrm{T}_pM$中的$k$个线性无关向量.那么存在$p$某个开邻域上的局部标架$\{X_1,X_2,\cdots,X_n\}$,使得$X_i\mid_p=v_i,i=1,2,\cdots,k$.
	\item 如果$\{X_1,X_2,\cdots,X_n\}$是$M$上沿闭子集$A$的线性无关的光滑向量场,那么存在$A$的某个开邻域上的局部标架$\{Y_1,Y_2,\cdots,Y_n\}$,使得每个$Y_i\mid A=X_i$.
\end{enumerate}

向量场和导数.我们之前解释过切向量恰好就是$\mathrm{C}^{\infty}_p\to\mathbb{R}$的满足莱布尼兹法则的线性算子.这里我们证明向量场也是满足莱布尼兹法则的线性算子.
\begin{itemize}
	\item $M$上的导数是指一个$\mathbb{R}$线性映射$D:\mathrm{C}^{\infty}(M)\to\mathrm{C}^{\infty}(M)$,满足莱布尼兹法则$D(fg)=fD(g)+gD(f)$.
	\item 给定光滑向量场$X$,给定开集$U$上的光滑函数$f$,我们定义$(Xf)(p)=X_pf$,这使得$Xf$是$U$上的函数.
\end{itemize}
\begin{enumerate}
	\item 设$X:M\to\mathrm{T}M$是甚至未必连续的向量场,那么如下条件互相等价:
	\begin{itemize}
		\item $X$是光滑向量场.
		\item 对每个$f\in\mathrm{C}^{\infty}(M)$都有$Xf$是$M$上的光滑函数.
		\item 对每个开集$U\subset M$和$U$上每个光滑函数$f$,都有$Xf$是$U$上的光滑函数.
	\end{itemize}
    \begin{proof}
    	
    	1推2,假设$X$是光滑向量场,任取$M$上的光滑函数$f$,任取$p\in M$,选取$p$的光滑坐标卡$(U,(x^i))$,那么有如下等式,这里$f$的偏导数和$X^i$都是光滑的,所以$Xf$在点$p$光滑,所以$Xf$处处光滑.
    	$$Xf(x)=\left(\sum_iX^i(x)\partial_i\mid_x\right)f=\sum_iX^i(x)\frac{\partial f}{\partial x^i}(x)$$
    	
    	2推3,任取非空开集$U\subset M$,任取点$p\in U$,取$\{p\}\subset U$的光滑碰撞函数$\psi$,取$g=\psi f$,那么$g$可以零延拓到$M-\mathrm{Supp}\psi$上,于是按照条件$Xg$是光滑的,但是$Xf$和$Xg$在点$p$足够小的开邻域上相同,于是$Xf$在$U$上每个点光滑.
    	
    	3推1,选取光滑坐标卡$(U,(x^i))$,那么有$Xx^i=X^i$是光滑的,于是$X$的分量函数处处是光滑的,于是$X$是光滑的.
    \end{proof}
    \item 给定光滑向量场$X$,容易验证我们定义的$\mathrm{C}^{\infty}(M)$上的算子$f\mapsto Xf$是$\mathbb{R}$线性并且满足莱布尼兹法则.这里我们证明$M$上的导数必然是被某个光滑向量场诱导的.
    \begin{proof}
    	
    	设$D$是$M$上的导数.任取点$p\in M$,那么$f\mapsto (Df)(p)$是$p$处的一个切向量,把它记作$X_p$,据此我们得到一个不确定光滑性的向量场$X$,但是它满足$Xf=Df$总是光滑的,上面等价描述导致$X$是光滑向量场.
    \end{proof}
\end{enumerate}

切向量和向量场的前推.
\begin{enumerate}
	\item 设$F:M\to N$是光滑映射,设$X$是$M$上的向量场,那么对每个点$p\in M$,都确定了一个向量$\mathrm{d}F_p(X_p)\in\mathrm{T}_{F(p)}N$.这称为该切向量的前推.但是我们没法定义向量场的前推,换句话讲一般这些前推切向量不会构成$N$上的向量场.如果$F$不是满射,那么不存在$N-F(M)$中点的向量;如果$F$不是单射,那么可能出现同一个点上出现多个向量的情况.不过如果$N$上的向量场$Y$满足$\forall p\in M$都有$\mathrm{d}F_p(X_p)=Y_{F(p)}$,我们就称向量场$X,Y$是$F$相关的.也称$Y$是$X$的前推,记作$Y=F_*X$.
	\item 如果$F:M\to N$是光滑映射,$X,Y$分别是$M,N$上的向量场,那么它们是$F$相关的当且仅当对每个$N$上某个开子集上的光滑函数$f$,总有$X(f\circ F)=(Yf)\circ F$.
	\begin{proof}
		
		任取点$p\in M$,任取$F(p)$附近的光滑函数$f$,那么有$X(f\circ F)(p)=X_p(f\circ F)=\mathrm{d}F_p(X_p)f$.另一方面$(Yf)\circ F(p)=Y_{F(p)}f$.于是这是充要的.
	\end{proof}
    \item 如果$F:M\to N$是微分同胚,对$M$上的每个向量场$X$,都存在$N$上唯一的向量场$Y$满足它们是$F$相关的,换句话讲微分同胚的情况下向量场总存在前推向量场.
    \item 另一个前推向量场总存在的例子是李群同态,后文会证明.
\end{enumerate}

子流形上的向量场.设$S\subset M$是浸入或者嵌入子流形,任给$M$上的向量场$X$,那么$X$在$S$上的限制未必是$S$上的向量场,因为对$p\in S$未必有$X_p$落在$\mathrm{T}_pS\subset\mathrm{T}_pM$中.如果这个条件成立,就称$M$上的向量场$X$在点$p$是和$S$相切的.如果对每个$p\in S$,向量场$X$都在点$p$和$S$相切,就称向量场$X$是和$S$相切的.
\begin{enumerate}
	\item 如果$S\subset M$是正则子流形,设$X$是$M$上的向量场,那么$X$和$S$相切当且仅当对每个限制在$S$上恒为零的$M$上的光滑函数$f$,都有$(Xf)\mid S=0$.这个结论是因为我们解释过正则子流形上的切空间有这样一个类似描述.
	\item 设$S\subset M$是浸入子流形,设$Y$是$M$上的向量场,如果存在$S$上的向量场和$Y$是$i$相关的,这里$i:S\subset M$是包含映射,那么自然有$Y_p=\mathrm{d}i_p(X_p)$,于是$Y$自然是和$S$相切的.我们断言这个逆命题成立:如果$S\subset M$是浸入子流形,$i:S\subset M$是包含映射,如果$M$上的向量场$Y$和$S$相切,那么存在$S$上唯一的向量场和$Y$是$i$相关的.
	\begin{proof}
		
		$Y$和$S$相切意味着对每个$p\in S$,都有向量$X_p\in\mathrm{T}_pS$,使得$Y_p=\mathrm{d}i_p(X_p)$.按照$\mathrm{d}i_p$是单射,说明这里$X_p$是唯一的.于是我们定义了一个$S$上的粗糙向量场(即甚至不知道连续性的向量场),记作$X$.如果我们证明了$X$是光滑的,那么它自然是唯一的和$Y$是$i$相关的向量场.
		
		任取$p\in S$,我们解释过光滑浸入局部上都是光滑嵌入,于是存在$p$在$S$中的开邻域$V$使得它是$M$的嵌入子流形.取$V$的以$p$为中心的一个切片光滑坐标卡为$(U,(x^i))$,那么$V\cap U$是$U$的满足$x^{k+1}=\cdots=x^n=0$的点构成的子集.这里$(x^1,x^2,\cdots,x^k)$是$S$在$V\cap U$的局部坐标.在$U$的坐标下可记$Y=\sum_{1\le i\le n}Y^i\partial x^i$,那么$X$在$U\cap V$坐标下可记$X=\sum_{1\le i\le k}Y^i\partial x^i$,这说明$X$在$V\cap U$上光滑,于是点$p$的一般性说明$X$是光滑向量场.
	\end{proof}
\end{enumerate}

\newpage
\section{子流形}
\subsection{嵌入和浸入子流形}

$M$是光滑流形,它的嵌入子流形或者正则子流形,是指一个非空子集$S\subset M$,赋予子空间拓扑后是拓扑流形,并且具备的光滑结构使得包含映射$S\subset M$是光滑嵌入.我们把维数差$\dim M-\dim S$称为这个正则子流形的余维数.余维数1的正则子流形也称为正则超曲面.一个正则子流形成为真(proper)正则子流形,如果包含映射是真映射,也即它把紧集拉回为紧集.
\begin{enumerate}
	\item 设$M$是光滑流形,它的余维数0的正则子流形恰好就是它的非空开子流形.
	\begin{proof}
		
		如果$U\subset M$是非空开子集,我们解释过$\dim M=\dim U$,另外包含映射自然是一个光滑嵌入,于是非空开子流形是余维数0的正则子流形.反过来如果$U$是余维数0的正则子流形,那么包含映射是一个光滑嵌入,于是它是一个局部微分同胚,于是它是开映射,于是$U$的像集是$M$的开子集.双射的局部微分同胚是微分同胚,于是$U$微分同胚于开子流形.
	\end{proof}
    \item 正则子流形是光滑嵌入的像集.如果$F:N\to M$是光滑嵌入,那么$S=F(N)$在$M$的子空间拓扑下是一个拓扑流形,并且它具有唯一的光滑结构使得它成为$M$的正则子流形并且$F$是微分同胚.于是特别的$\dim S=\dim N$
    \begin{proof}
    	
    	光滑嵌入有要求$N$和$S$是同胚的,这里$S$赋予的是子空间拓扑,这导致$S$是拓扑流形.赋予$S$光滑图册为全体$(F(U),\varphi\circ F^{-1})$,这里$(U,\varphi)$是$N$上的光滑坐标卡.$N$上光滑坐标卡的相容性传递给这个图册.在这个光滑结构下,$F$是$N\to S$的微分同胚.并且这是唯一的使得$F$是微分同胚的光滑结构.最后包含映射$S\to N\to M$是两个光滑嵌入的复合,于是这是光滑嵌入,于是$S$是$M$的正则子流形.
    \end{proof}
    \item 图像作为正则子流形.设$M,N$是维数分别为$m,n$的光滑流形,设$U\subset M$是开子流形,设$f:U\to N$是光滑映射,那么图像$\Gamma(f)=\{(x,y)\in M\times N\mid x\in U,y=f(x)\}$是$M\times N$的$m$维正则子流形.并且这是一个真正则子流形.
    \begin{proof}
    	
    	定义$F:U\to M\times N$为$x\mapsto(x,f(x))$,这是像集为$\Gamma(f)$的光滑映射,并且它是一个拓扑嵌入,因为逆映射是第一个分量的投影映射$\pi:M\times N\to M$.于是按照上一条结论我们仅需验证它是光滑浸入.按照$\pi\circ F$是$U$上的恒等映射,得到$\mathrm{d}(\pi)_{(x,f(x))}\circ\mathrm{d}F_x$是$\mathrm{T}_xM$上的恒等映射对任意$x\in U$成立.导致$\mathrm{d}F_x$是单射对任意$x\in U$成立,于是$F$是光滑浸入.
    	
    	\qquad
    	
    	为证明它是真正则子流形,需验证明映射$F:U\to M\times N$是真映射.但是我们有如下一般结论:如果$F:X\to Y$是连续映射,$Y$是Hausdorff的,并且$F$有连续左逆,那么$F$是真映射.我们这里映射$F$具有光滑左逆$\pi:M\times N\to M$,即有$\pi\circ F$是$U$上恒等映射.这说明$F$是真映射.
    \end{proof}
    \item 设$S\subset M$是一个正则子流形,那么它是真正则子流形当且仅当$S$是$M$的闭子集.
    \begin{proof}
    	
    	必要性是因为流形之间的真映射是闭映射.充分性是因为,任取$M$的紧子集$E$,那么$i^{-1}(E)=E\cap S$是紧集的闭子集,自然也是紧子集.
    \end{proof}
    \item 紧的正则子流形总是真正则子流形.因为Hausdorff空间上的紧子集是闭的.
\end{enumerate}

切片坐标卡和正则子流形.
\begin{enumerate}
	\item 如果$U\subset\mathbb{R}^n$是开子集,设$0\le k\le n$,一个$k$-切片是指$U$的固定$\{x^{k+1},\cdots,x^n\}$分量取值的子集.设$M$是光滑流形,如果$(U,\varphi)$是一个坐标卡,如果子集$S\subset U$使得$\varphi(S)$是$\varphi(U)$的$k$-切片,就称$S$是$U$的$k$-切片.称$M$的子集$S$满足局部$k$-切片条件,如果$S$的每个点都存在$M$的光滑坐标卡$(U,\varphi)$覆盖它,使得$S\cap U$是$U$的$k$-切片.这样的坐标卡称为$S$在$M$中的切片坐标卡.
	\item 设$M$是$n$维光滑流形,如果$S\subset M$是一个$k$维正则子流形,那么$S$满足局部$k$-切片条件.反过来如果子集$S\subset M$满足局部$k$-切片条件,那么在子空间拓扑下$S$是一个$k$维拓扑流形,并且存在光滑结构使得它成为$M$的$k$维正则子流形.
\end{enumerate}

正则和临界.设$F:M\to N$是光滑映射.
\begin{itemize}
	\item 点$p\in M$称为$F$的正则点(regular point),如果该点处$F$的微分是满射;点$p\in M$称为$F$的临界点(critical point),如果该点处$F$的微分不是满射.
	\item 点$c\in N$称为$F$的正则值(regular value),如果水平集$F^{-1}(c)$中每个点都是正则点,注意这包含了水平集为空的情况;点$c\in N$称为$F$的临界值(critical value),如果水平集$F^{-1}(c)$中至少存在一个临界点.
	\item 如果$c\in N$是正则值,称$F^{-1}(c)$是一个正则水平集.
\end{itemize}
\begin{enumerate}
	\item 光滑函数的临界点就是存在(或等价的改为任意)坐标表示在该点的每个偏导数都为零的点.
	\item 设$F:M\to N$是常秩光滑映射,那么$F$的每个非空水平集都是$M$的余维数$r$的真正则子流形.特别的,如果$F$是光滑浸没,那么每个非空水平集都是余维数恰好是$\dim N$的真正则子流形.
	\begin{proof}
		
		记$\dim M=m$,$\dim N=n$,记$k=m-r$.取$c\in N$使得水平集$S=F^{-1}(c)$非空.任取$p\in S$,按照秩定理,选取$p$和$c$的光滑坐标卡$(U,\varphi)$和$(V,\psi)$,使得$F$的坐标表示为投影映射$(x^1,\cdots,x^r,x^{r+1},\cdots,x^m)\mapsto(x^1,x^2,\cdots,x^r,0,\cdots,0)$.这导致$S\cap U$是$k$-切片$\{(0,0,\cdots,0,x^{r+1},\cdots,x^m)\in U\}$.于是$S$满足局部$k$-切片条件,于是$S$是$M$的$k$维正则子流形.最后$F$的连续性保证$S$是闭子集,于是$S$是$k$维的真正则子流形.
	\end{proof}
	\item 正则水平集定理.光滑映射的正则水平集总是一个真正则子流形,并且它的余维数就是这个正则映射终端的维数.
	\begin{proof}
		
		设$F:M\to N$是光滑映射,设$c\in N$是一个正则值,我们解释过微分的秩为$\dim N$的点构成开子集,记作$U$.那么有$F^{-1}(c)\subset U$.并且$F\mid U$是光滑浸没.于是上一条说明$F^{-1}(c)$是$U$的正则子流形.于是复合$F^{-1}(c)\subset U\subset M$是光滑嵌入,于是$F^{-1}(c)$是$M$的正则子流形,它是真正则子流形因为$F$的连续性保证$F^{-1}(c)$是闭子集.
	\end{proof}
    \item 例如$\mathbb{S}^n$是$\mathbb{R}^{n+1}$的真正则子流形:$\mathbb{S}^n$是光滑函数$f:\mathbb{R}^{n+1}\to\mathbb{R}$,$x\mapsto|x|^2$在点1处的水平集.这里1是正则值是因为$\mathrm{d}f_x(v)=2\sum_ix^iv^i$只在原点处不是满射.
\end{enumerate}

浸入子流形.设$M$是光滑流形,它的一个浸入子流形是指一个子集$S\subset M$,赋予了未必是子空间拓扑的拓扑使得它成为一个拓扑流形,并且赋予一个光滑结构使得包含映射$S\subset M$是光滑浸入.浸入子流形的余维数一样定义为$\dim M-\dim S$.
\begin{enumerate}
	\item 一件平凡的事情:正则子流形总是浸入子流形.
	\item 设$F:N\to M$是单的光滑浸入,设$S=F(N)$,那么$S$上具有唯一的拓扑结构和光滑结构使得它成为$M$的浸入子流形,并且$F$是$N$到像集的微分同胚.
	\begin{proof}
		
		定义$S$上拓扑和$N$同胚,换句话讲$U\subset S$是开集当且仅当$F^{-1}(U)$是$N$中开集.定义光滑坐标卡为$(F(U),\varphi\circ F^{-1})$,其中$(U,\varphi)$是$N$上的光滑坐标卡.这自然是唯一的使得$F$是微分同胚的光滑流形结构.最后包含映射$S\to M$可视为两个光滑浸入的复合$S\to N\to M$,于是这是光滑浸入.
	\end{proof}
    \item 设$S\subset M$是浸入子流形,那么$S$作为子空间的开集也是它作为子流形的开集.这个逆命题成立当且仅当$S$是嵌入子流形.
    \item 设$S\subset M$是浸入子流形,如果如下任一条件成立,那么$S$是嵌入子流形.
    \begin{itemize}
    	\item $S$在$M$中具有余维数0.
    	\item 包含映射$S\subset M$是真映射.
    	\item $S$是紧集.
    \end{itemize}
    \item 我们之前证明过光滑浸入局部上看是光滑嵌入,于是浸入子流形局部上看是嵌入子流形:如果$S\subset M$是浸入子流形,任取$p\in S$则存在在$S$中的开邻域$U$使得$U$是$M$的嵌入子流形.
    \item 光滑函数的延拓.设$S\subset M$是浸入子流形,设$f$是$S$上的光滑函数.
    \begin{itemize}
    	\item 如果$S$是正则子流形,那么存在$S$在$M$中的开邻域$U$使得存在$U$上的光滑函数在$S$上的限制就是$f$.
    	\item 如果$S$是真正则子流形,那么上一条中的$U$可取为整个$M$.
    \end{itemize}
\end{enumerate}

子流形的切空间.如果$S\subset M$是浸入或者嵌入子流形,那么包含映射$i:S\to M$诱导的切映射总是单射,我们把$\mathrm{T}_pS$视为$\mathrm{T}_pM$的子空间.
\begin{enumerate}
	\item 设$S\subset M$是浸入或者嵌入子流形,向量$v\in\mathrm{T}_pM$落在$\mathrm{T}_pS$中当且仅当存在光滑曲线$\gamma:J\to M$使得它的像集完全落在$S$中,并且$0\in J$,$\gamma(0)=p$,$\gamma'(0)=v$.
	\item 设$S\subset M$是嵌入子流形,那么$\mathrm{T}_pS$由$\mathrm{T}_pM$中这样的向量$v$构成:它满足只要存在$M$上的光滑函数$f$使得$f\mid S=0$,就有$vf=0$.
	\begin{proof}
		
		先设$v\in\mathrm{T}_pS$.严格讲这是指存在$w\in\mathrm{T}_pS$使得$v=\mathrm{d}i_p(w)$,其中$i:S\to M$是包含映射.如果$f$是$M$上的光滑函数并且在$S$上恒为零,那么有$vf=\mathrm{d}i_p(w)(f)=w(f\circ i)=0$.
		
		\qquad
		
		反过来设$v\in\mathrm{T}_pM$满足这个条件.我们需要证明存在向量$w\in\mathrm{T}_pS$使得$v=\mathrm{d}i_p(w)$.记$S$在点$p$的切片坐标卡是$(U,\varphi)$,那么$\varphi(U\cap S)$是$\varphi(U)$的后$n-k$个坐标映射的分量为零.并且包含映射$i:S\cap U\to M$的坐标表示为$i(x^1,x^2,\cdots,x^k)=(x^1,x^2,\cdots,x^k,0,\cdots,0)$.这说明$\mathrm{d}i_p(\mathrm{T}_pS)$是由$\partial x^i\mid_p,1\le i\le k$生成.如果记$v=\sum_{1\le i\le n}v^i\partial x^i\mid_p$那么那么$v\in\mathrm{T}_pS$当且仅当$i>k$时有$v^i=0$.
		
		\qquad
		
		选取$\{p\}\subset U$的碰撞函数$\varphi$,换句话讲$\varphi$的紧支集在$U$中,并且在$p$附近恒为1.选取指标$j>k$,构造函数$f(x)=\varphi(x)x^j$,它可以零延拓到$M-\mathrm{Supp}\varphi$上.那么$f$在$S$上恒为0,导致$0=vf=v^j,\forall j>k$,于是$v\in\mathrm{T}_pS$.
	\end{proof}
\end{enumerate}

光滑映射在子流形上的限制.设$F:M\to N$是光滑映射.
\begin{enumerate}
	\item 如果$S\subset M$是浸入子流形,那么$F$在$S$上的限制是$S\to N$的光滑映射.这是因为$F\mid S=F\circ i$,其中$i:S\subset M$是包含映射.
	\item 如果$S\subset N$是浸入子流形,使得$F(M)\subset S$,那么$F$限制为$M\to S$未必是光滑映射.不过如果$F$作为$M\to S$的映射是连续的,那么它限制为$M\to S$就是光滑的.
	\begin{proof}
		
		任取$p\in M$,记$q=F(p)\in S$.我们证明过浸入子流形是局部嵌入的,于是存在$q$在$S$中的开邻域$V$使得$V$是$N$的嵌入子流形.于是$V$在$M$中满足局部$k$-切片条件,其中$k=\dim S$.于是存在$N$的覆盖$q$的光滑坐标卡$(W,\psi)$,使得$W\cap V$是$V$的$k$-切片.那么记$V_0=W\cap V$和$\widetilde{\psi}=\pi\circ\psi$,其中$\pi:\mathbb{R}^n\to\mathbb{R}^k$是到前$k$个分量的投影映射,就有$(V_0,\widetilde{\psi})$是$V$的覆盖$p$的光滑坐标卡.现在$V_0$是$V$的开子集(因为$V_0$是包含映射$V\subset M$下$W$的原像),于是$V_0$也是$S$的开子集,于是$(V_0,\widetilde{\psi})$是$S$中的一个光滑坐标卡.
		
		记$U=F^{-1}(V_0)$,它是$M$的覆盖点$p$的开子集(这里用到了条件).选取覆盖点$p$的包含在$U$中的$M$的光滑坐标卡$(U_0,\varphi)$.那么$F:M\to S$的关于$(U_0,\varphi)$和$(V_0,\widetilde{\psi})$的坐标表示为$\widetilde{\psi}\circ F\circ\varphi^{-1}=\pi\circ(\psi\circ F\circ\varphi^{-1})$是光滑的.完成证明.
	\end{proof}
	\item 特别的,如果$S\subset N$是嵌入子流形,满足$F(M)\subset S$,那么$F$限制为$M\to S$的映射是自动连续的,于是此时$F:M\to S$自动是光滑映射.
	\item 一个浸入子流形$H\subset M$称为弱嵌入,如果光滑映射$F:N\to M$的像落在$H$中,那么它视为$N\to H$的像也是光滑的.于是上一条就是说嵌入子流形总是弱嵌入子流形.
	\item 如果$H\subset M$是弱嵌入子流形,那么$H$上存在唯一的拓扑结构和光滑结构使得它是$M$的浸入子流形.
\end{enumerate}

\subsection{Sard定理}

零测集.设$M$是$n$维光滑流形,它的一个子集$A$称为零测子集,如果对每个光滑坐标卡$(U,\varphi)$,都有$\varphi(A\cap U)\subset\mathbb{R}^n$是勒贝格零测集.
\begin{enumerate}
	\item 如果仅仅是存在一组覆盖了整个$M$的光滑坐标卡$\{(U_i,\varphi_i)\}$满足$\varphi_i(A\cap U_i)$都是$\mathbb{R}^n$中的零测集,就已经得到$A$是零测集.
	\begin{proof}
		
		任取光滑坐标卡$(V,\psi)$,需要证明$\psi(A\cap V)$是零测集.选取$\{U_i\}$的可数子覆盖,按照零测集的可数并还是零测集,我们仅需验证$\psi(A\cap V\cap U_i)=(\psi\circ\varphi_i^{-1})\circ\varphi_i(A\cap V\cap U_i)$是零测集.这里$\varphi_i(A\cap V\cap U_i)$是零测集$\varphi_i(A\cap V)$的子集,于是它也是零测集.于是问题归结为如下引理:如果$A\subset\mathbb{R}^n$是零测集,如果$F:A\to\mathbb{R}^n$是光滑映射,那么$F(A)$也是零测集.
	\end{proof}
    \item 光滑流形上可数个零测集的并仍然是零测集.
    \item 如果$A\subset M$是光滑流形的零测集,那么$M-A$在$M$中稠密.这件事是容易的,因为若否则零测集$A$包含了一个非空开集,适当选取光滑坐标卡,映射到欧氏空间中将不会是零测集.
    \item 设$F:M\to N$是光滑映射,如果$A\subset M$是零测集,那么$F(A)\subset N$也是零测集.
    \begin{proof}
    	
    	选取$M$的可数光滑坐标卡$\{(U_i,\varphi_i)\}$覆盖了全空间.问题归结为证明对$N$的每个光滑坐标卡$(V,\psi)$,总有$\psi(F(A)\cap N)$是$\mathbb{R}^n$中的零测集.但是这个集合可以表示为可数个形如$\psi\circ F\circ\varphi_i^{-1}(\varphi_i(A\cap U_i\cap F^{-1}(V)))$的并,我们解释过这样形式的集合是零测集.
    \end{proof}
\end{enumerate}

Sard定理.设$F:M\to N$是光滑映射,那么$F$的临界值构成了$N$的零测集.
\begin{proof}
	
	设$\dim M=m$,$\dim N=n$,我们来对$m$做归纳.如果$m=0$,倘若$n=0$,那么$F$没有临界值;倘若$n>0$,那么$F$的像集是可数集,必然是零测集.现在设$m\ge1$,并且假设维数小于$m$的情况已经得证.选取可数光滑坐标卡覆盖,按照可数个零测集的并还是零测的,我们不妨设$F$是从$\mathbb{R}^m$的某个开子集到$\mathbb{R}^n$的光滑映射.
	
	\qquad
	
	记$C\subset U$是$F$的临界点集合.我们构造子集列$C\supset C_1\supset C_2\supset\cdots$,使得$C_k$是$C$的由$F$的从1到$k$阶的所有偏导数为零的点$x$构成的子集.那么$F$的连续性说明$C,C_k$都是$U$的闭子集.我们来分三步证明$F(C)$是零测集.
	
	\qquad
	
	第一步,证明$F(C-C_1)$是零测集.因为$C_1$是$U$的闭子集,于是不妨设$C_1$是空集,否则可以用开集$U-C_1$代替$U$.任取$a\in C$,那么$\partial_1F(a)\not=0$.不妨设$\partial_1F^1(a)\not=0$,否则可以适当重新选择一个坐标卡.取$a$的某个足够小的开邻域$V_a\subset U$到$\mathbb{R}^m$的光滑映射为$(u,v)=(u,v^2,\cdots,v^m)$,其中$u=F^1$,$v^i=x^i,i\ge2$.因为这个光滑映射在点$a$处的Jacobian矩阵是非奇异的,所以该点是局部微分同胚,所以选取足够小的$V_a$可以使得这是一个坐标映射.现在适当缩小$V_a$,我们可以约定$\overline{V_a}\subset U$是紧子集,并且$V_a$上的坐标映射可以光滑的延拓至$\overline{V_a}$.现在$F$的坐标表示为$F(u,v^2,\cdots,v^m)=(u,F^2(u,v),\cdots,F^n(u,v))$,它的Jacobian矩阵为:
	$$\left(\begin{array}{cc}1&0\\\bullet&\frac{\partial F^i}{\partial v^j}\end{array}\right)$$
	
	于是$C\cap\overline{V_a}$恰好是矩阵$(\partial F^i/\partial v^j)_{(n-1)\times(m-1)}$的秩小于$n-1$的点构成的.一旦说明$F(C\cap\overline{V_a})$是$\mathbb{R}^n$的零测集,按照$U$可以被至多可数个$\overline{V_a}$所覆盖,导致$F(C\cap U)$被至多可数个零测集覆盖,这就证明了第一步.
	
	\qquad
	
	我们有这样一个引理:$\mathbb{R}^n$的一个紧子集$A$如果每个切片都是$n-1$维零测集,此即每个$\{c\}\times\mathbb{R}^{n-1}\cap A$都是$n-1$维零测集,那么$A$是$n$维零测集.按照这个引理,我们只要证明$F(C\cap\overline{V_a})$和任意超平面$y^1=c$的交是$n-1$维零测集.为此记$B_c=\{v\mid(c,v)\in\overline{V_a}\}\subset\mathbb{R}^{m-1}$,构造$F_c:B_c\to\mathbb{R}^{n-1}$为$v\mapsto(F^2(c,v),\cdots,F^n(c,v))$.由于$F(c,v)=(c,F_c(v))$,说明$F\mid\overline{V_a}$的落在$y^1=c$的临界值恰好对应于$F_c$的临界值$w$.于是按照归纳假设,$F_c$的临界值是$n-1$维零测集,这就证明了第一步.
	
	\qquad
	
	第二步,证明每个$F(C_k-C_{k+1})$是零测集.类似上一步,这里$C_{k+1}$在$U$中是闭集,于是不妨设$C_{k+1}$是空集,否则以$U-C_{k+1}$代替$U$.换句话讲我们不妨设$C_k$中每个点,都存在$F$的某个$k+1$阶偏导数不为零.
	
	\qquad
	
	任取$a\in C_k$,设$y$是$F$的某个分量的某个$k$阶偏导数,使得$y$至少存在某个一阶偏导数在点$a$处不为零.那么$a$就是光滑函数$y$的正则点,于是存在$a$的一个开邻域$V_a$仅由$y$的正则点构成.记$y$在$V_a$中的零点集为$Y$,按照正则水平集定理,$Y$是一个光滑超曲面.按照定义$y$理应在$C_k$上恒为零,于是$C_k\cap V_a\subset Y$.任取$p\in C_k\cap V_a$,那么$\mathrm{d}F_p$不是满射,导致$\mathrm{d}(F\mid Y)_p=(\mathrm{d}F_p)\mid\mathrm{T}_pY$不是满射,于是$F(C_k\cap V_a)$包含在$F\mid Y:Y\to\mathbb{R}^n$的临界值集合中,于是按照归纳假设是零测集.最后按照$U$可以被可数个形如$V_a$的开子集覆盖,于是$F(C_k-C_{k+1})$被可数个形如$F(C_k\cap V_a)$的零测集覆盖,于是它是零测集.
	
	\qquad
	
	第三步,我们证明$k>m/n-1$时$F(C_k)$是零测集,这三步就证明了整个$F(C)$是零测集.对每个$a\in U$,可以选取一个闭方体$E$满足$a\in E\subset U$.按照$U$可以被可数个这样的方体所覆盖,归结为证明$F(C_k\cap E)$是零测集,其中$E$是包含于$U$的闭方体.设$F$的全部$k+1$阶偏导数的绝对值在紧集$E$上有统一的上界$A$.记$E$的边长为$R$,设$K$是一个待定的正整数.现在把$E$等分为$K^m$个边长为$R/K$的方体,这些方体编号为$E_1,E_2,\cdots,E_{K^m}$.如果$a_i\in C_k\cap E_i$,那么Taylor公式说明对每个$x\in E_i$就有$|F(x)-F(a_i)|\le A'|x-a_i|^{k+1}$,这里$A'$仅依赖于$A,k,m$.于是$F(E_i)$落在半径为$A'(R/K)^{k+1}$的圆盘中,于是$F(C_k\cap E)$包含在$K^m$个圆盘的并中,而它们的$n$维体积不超过$K^m(A')^n(R/K)^{n(k+1)}=A''K^{m-nk-n}$,其中$A''=(A')^nR^{n(k+1)}$,按照$k>m/n-1$,得到$m-nk-n<0$,于是当$K$足够大时这个体积足够小,于是$F(C_k\cap E)$是零测集.
\end{proof}

推论.
\begin{itemize}
	\item 如果$F:M\to N$是光滑映射,并且$\dim M<\dim N$,那么这时$M$中每个点都是$F$的临界点,于是Sard定理说明此时$F(M)$是$N$的零测集.
	\item 如果$S\subset M$是浸入子流形,并且$\dim S<\dim M$,那么$S$在$M$中是零测集.
\end{itemize}

\subsection{Whitney嵌入定理和逼近定理}

我们的最终目标是证明每个$n$维光滑流形都微分同胚于$\mathbb{R}^{2n+1}$的一个真正则子流形.
\begin{enumerate}
	\item 首先我们说明一个从$n$维光滑流形$M$到某个欧氏空间$\mathbb{R}^N$的单光滑浸入,如果$N>2n+1$,那么总可以转化为$M$到更低维欧氏空间的单光滑浸入.具体的讲:如果$M\subset\mathbb{R}^N$是一个$n$维浸入子流形,对每个$v\in\mathbb{R}^N-\mathbb{R}^{N-1}$(这里我们把$\mathbb{R}^{N-1}$视为$\mathbb{R}^N$的最后一个分量取零的子集),记$\pi_v:\mathbb{R}^N\to\mathbb{R}^{N-1}$是核为$\mathbb{R}v$的投影映射(此即把$x=(x^1,x^2,\cdots,x^N)$映射为$x-x^Nv/v^N$).如果$N>2n+1$,那么存在由向量$v\in\mathbb{R}^N-\mathbb{R}^{N-1}$构成的稠密集,使得$\pi_v\mid M$是$M$到$\mathbb{R}^{N-1}$的单光滑浸入.
	\begin{proof}
		
		$\pi_v\mid M$是单射等价于讲不存在$M$中的两个不同点$p,q$使得$\vec{pq}$和$v$平行.而$\pi_v\mid M$是光滑浸入等价于讲对每个$p\in M$都有$\mathrm{T}_pM$和$\ker\mathrm{d}(\pi_v)_p$的交只有零向量.但是这里$\pi_v$是线性映射,它的微分是相同的线性映射(这里把$\mathrm{T}_p\mathbb{R}^N$等同于$\mathbb{R}^N$),于是这个条件等价于要求$\mathrm{T}_pM$不能包含平行于$v$的向量.
		
		\qquad
		
		记$\Delta_M=\{(p,p)\mid p\in M\}\subset M\times M$.记$M_0=\{(p,0)\in\mathrm{T}M\mid p\in M\}$,考虑两个到实射影空间$\mathbb{R}\mathbb{P}^{N-1}$的映射:$\kappa:M\times M-\Delta M\to\mathbb{R}\mathbb{P}^{N-1},(p,q)\mapsto[p-q]$和$\tau:\mathrm{T}M-M_0\to\mathbb{R}\mathbb{P}^{N-1},(p,w)\mapsto[w]$.它们都是光滑映射因为它们都是到欧氏空间的光滑映射和光滑映射$\mathbb{R}^N-\{0\}\to\mathbb{R}\mathbb{P}^{N-1}$的复合.并且$\pi_v\mid M$是单光滑浸入等价于讲$[v]$不在$\kappa$和$\tau$的像集.但是按照$\kappa$和$\tau$的源端的维数都是$2n$并且$<N-1=\dim\mathbb{R}\mathbb{P}^{N-1}$,按照Sard定理有这两个映射的像集都是零测集,于是它们的并是零测集,而零测集的补集是稠密集,这就得证.
	\end{proof}
    \item 如果光滑流形$M$存在到某个欧氏空间的单光滑浸入,如果$M$是紧集,我们解释过此时单光滑浸入也是一个光滑嵌入.
    \item 如果$n$维光滑流形$M$存在到$\mathbb{R}^N$的光滑嵌入,那么存在从$M$到$\mathbb{R}^{2n+1}$的真光滑嵌入.
    \begin{proof}
    	
    	对$\mathbb{R}^N$的一维线性子空间$S$,对正实数$R$,我们定义管子$T_R(S)=\{x\in\mathbb{R}^N\mid\exists y\in S, |x-y|<R\}$,这是以$S$为中心的半径为$R$的管子.设$F:M\to\mathbb{R}^N$是一个光滑嵌入,取微分同胚$G:\mathbb{R}^N\to\mathbb{B}^N$,取光滑exhaustion函数$f$(此即$f:M\to\mathbb{R}$使得每个$c\in\mathbb{R}$都有$f^{-1}((-\infty,c])$是$M$的紧子集,这个名字的由来是因为$n$取正整数的时候$f^{-1}((-\infty,n])$构成了$M$上的一个exhaustion).
    	
    	\qquad
    	
    	构造$\psi:M\to\mathbb{R}^N\times\mathbb{R}$为$p\mapsto(G\circ F(p),f(p))$.按照$G\circ F$是光滑嵌入,导致$\psi$是一个单射,并且$\mathrm{d}\psi_p$是单射,于是$\psi$是一个单光滑浸入.$\psi$是一个真映射因为每个紧子集的原像都是某个$f^{-1}((-\infty,c])$的闭子集.我们解释过单光滑浸入如果是真映射那么它是光滑嵌入.于是$\psi$是光滑嵌入.按照我们的构造,$\psi$的像集落在$\mathbb{B}^N\times\mathbb{R}$中.于是我们证明了存在从$M$到$\mathbb{R}^{N+1}$的真光滑嵌入,使得它的像集落在某个管子$T_R(S)$中.换句话讲,$M$可视为某个$\mathbb{R}^{N+1}$的落在某个管子中的真正则子流形.
    	
    	\qquad
    	
    	如果$N+1>2n+1$,我们之前证明的结论说明存在向量$v\in\mathbb{R}^{N+1}-\mathbb{R}^N$使得$\pi_v\mid M$是从$M$到$\mathbb{R}^N$的单光滑浸入.另外我们可以选取$v$使得它不落在一维线性子空间$S$中.于是$\pi_v(S)$是$\mathbb{R}^N$的一维线性子空间,并且$\pi_v(M)$落在以一维线性子空间$\pi_v(S)$为中心的管子中($\pi_v$是有界线性映射).我们最后只要证明$\pi_v\mid M$是真映射,就导致$\pi_v(M)$是$\mathbb{R}^N$的真正则子流形.反复做这一段和上一段的事情,就把$M$经一个真光滑嵌入映射到$\mathbb{R}^{2n+1}$中.
    	
    	\qquad
    	
    	任取紧集$K\subset\mathbb{R}^N$,那么$K$落在某个以原点为圆心,半径$R_1$的开球中.对每个点$x\in\pi_v^{-1}(K)$,按照定义有$c\in\mathbb{R}$使得$\pi_v(x)=x-cv$.按照$|\pi_v(x)|<R_1$,这导致$x$落在以$\mathbb{R}v$为中心,半径为$R_1$的管子中.现在$M\cap\pi_v^{-1}(K)$就落在两个管子中,一个以$S$为中心,一个以$\mathbb{R}v$为中心.这两个一维线性子空间不是平行的,导致这两个管子的交是有界的.于是$M\cap\pi_v^{-1}(K)$是紧集,于是$\pi_v\mid M$是真映射.完成证明.
    \end{proof}
    \item (弱)Whitney嵌入定理.每个$n$维光滑流形都可以经真光滑嵌入映射到$\mathbb{R}^{2n+1}$中.
    \begin{proof}
    	
    	按照前面做的工作,这里只要验证$M$可以光滑嵌入到任一欧氏空间即可.先设$M$是紧的,那么有有限开覆盖$\{B_1,B_2,\cdots,B_m\}$,其中每个$B_i$都是正则坐标球(称$B$是正则坐标球,如果存在一个坐标球$B'$使得$\overline{B}\subset B'$).于是存在坐标球$B_i'$使得$\overline{B_i}\subset B_i'$,并且$B_i'$上的坐标映射$\varphi_i$限制在$\overline{B_i}$上是到一个紧集的微分同胚.对每个指标$i$,选取$\overline{B_i}\subset B_i'$的光滑碰撞函数$\rho_i$.换句话讲$\rho_i$在$\overline{B_i}$上恒为1,并且它支集的闭包在$B_i'$中.构造$F:M\to\mathbb{R}^{nm+m}$为($\varphi_i$是到$\mathbb{R}^n$的映射,所以是$mn+n$):
    	$$F(p)=\left(\rho_1(p)\varphi_1(p),\cdots,\rho_m(p)\varphi_m(p),\rho_1(p),\cdots,\rho_m(p)\right)$$
    	
    	这里$\rho_i\varphi_i$依旧零延拓到$\rho_i$支集以外的点.我们来证明$F$是单光滑浸入,按照$M$的紧性就得到$F$是光滑嵌入.先证明$F$是单射:如果有$F(p)=F(q)$,选取指标$i$使得$p\in B_i$,那么$\rho_i(p)=1$,导致$\rho_i(q)=1$,于是$q\in\mathrm{Supp}\rho_i\subset B_i'$.于是有$\varphi_i(q)=\rho_i(q)\varphi_i(q)=\rho_i(p)\varphi_i(p)=\varphi_i(p)$,按照$\varphi_i$是单射得到$p=q$.再证$F$是光滑浸入:任取$p\in M$,选取指标$i$使得$p\in B_i$,按照$\rho_i$在$p$的某个开邻域上恒为1,于是有$\mathrm{d}(\rho_i\varphi_i)_p=\mathrm{d}(\varphi_i)_p$是单射.这说明$\mathrm{d}F_p$是单射。
    	
    	\qquad
    	
    	再设$M$是非紧的.选取光滑exhaustion函数$f:M\to\mathbb{R}$.Sard定理说明对每个非负整数$i$都存在正则值$a_i,b_i$使得$i<a_i<b_i<i+1$.我们记$D_i,E_i\subset M$为:
    	$$D_0=f^{-1}((-\infty,1]),E_0=f^{-1}((-\infty,a_1])$$
    	$$D_i=f^{-1}([i,i+1]),E_i=f^{-1}([b_{i-1},a_{i+1}]),i\ge1$$
    	
    	我们有$D_i\subset\mathrm{Int}(E_i)$,并且$M=\cup_iD_i$,帮其$E_i\cap E_j=\emptyset$除非$j=i-1$或者$i$或者$i+1$.我们之前证明的紧性情况说明了每个$i$都存在光滑嵌入$\varphi_i:E_i\to\mathbb{R}^{2n+1}$.取$D_i\subset\mathrm{Int}(E_i)$上的光滑碰撞函数$\rho_i:M\to\mathbb{R}$,定义映射$F:M\to\mathbb{R}^{2n+1}\times\mathbb{R}^{2n+1}\times\mathbb{R}$为:
    	$$F(p)=\left(\sum_{i\text{为偶数}}\rho_i(p)\varphi_i(p),\sum_{i\text{为奇数}}\rho_i(p)\varphi_i(p),f(p)\right)$$
    	
    	$F$是光滑的因为对于固定的点,这些无穷和中只存在一个分量是非零的.这里$F$是真映射因为$f$是真映射(紧集的闭子集紧).我们来证明$F$是单光滑浸入,结合它是真映射得到它是光滑嵌入.先证明单射:如果$F(p)=F(q)$,不妨设$p\in D_j$,那么从$f(q)=f(p)$得到$q\in D_j$,不妨设$j$是偶数(奇数同理),得到$\varphi_j(p)=\sum_{i\text{为偶数}}\rho_i(p)\varphi_i(p)=\sum_{i\text{为偶数}}\rho_i(q)\varphi_i(q)=\varphi_i(q)$,从$\varphi_i$是单射得到$p=q$.最后证明$F$是光滑浸入:任取$p\in M$,设$p\in D_j$,那么$\rho_j$在$p$的一个开邻域上恒为1,不妨设$j$是偶数,那么对于这个开邻域上的每个元$q$都有$F(q)=(\varphi_j(q),\cdots)$,从$\varphi_j$的切映射是单射就得到$\mathrm{d}F_p$是单射.
    \end{proof}
    \item (弱)Whitney浸入定理.每个$n$维光滑流形都微分同胚于$\mathbb{R}^{2n}$的某个浸入子流形.等价的讲每个$n$维光滑流形$M$都存在到$\mathbb{R}^{2n}$的光滑浸入.
    \item (强)Whitney嵌入定理.如果$n>0$,每个$n$维光滑流形都存在到$\mathbb{R}^{2n}$的光滑嵌入.
    \item (强)Whitney浸入定理.如果$n>1$,每个$n$维光滑流形都存在到$\mathbb{R}^{2n-1}$的光滑浸入.
\end{enumerate}

Whitney逼近定理.
\begin{enumerate}
	\item 函数版本.如果$M$是光滑流形,$F:M\to\mathbb{R}^k$是连续映射,任给正连续函数$\delta:M\to\mathbb{R}$,那么存在光滑映射$G:M\to\mathbb{R}^k$使得$|F-G|<\delta$.如果$F$在闭集$A\subset M$上光滑,那么$G$还可以要求满足$G\mid A=F$.
	\begin{proof}
		
		如果$F$在闭子集$A$上光滑,按照光滑延拓定理,存在光滑映射$F_0:M\to\mathbb{R}^k$使得$F_0\mid A=F$.取$U_0=\{y\in M\mid|F_0(y)-F(y)|<\delta(y)\}$,这是一个包含$A$的开子集.如果$A$是空集就取$A=U_0=\emptyset$,下述操作都是相同的.
		
		\qquad
		
		我们断言存在可数个点$\{x_i,i\ge1\}\subset M-A$,存在$x_i$在$M-A$中的开邻域$U_i$,使得$\{U_i\}$是$M-A$的开覆盖,并且对$y\in U_i$总有$|F(y)-F(x_i)|<\delta(y)$.一旦这件事得证,取$M$的开覆盖$\{U_0,U_i\}$的光滑单位分解$\{\varphi_0,\varphi_i\}$,定义$G:M\to\mathbb{R}^k$为$G(y)=\varphi_0(y)F_0(y)+\sum_{i\ge1}\varphi_i(y)F(x_i)$.单位分解保证这个无穷和在具体点上是有限和,另外$G$是光滑的,并且在$A$上和$F$一致.最后对任意$y\in M$,我们有:
		\begin{align*}
		|G(y)-F(y)|&=\left|\varphi_0(y)F_0(y)+\sum_{i\ge1}\varphi_i(y)F(x_i)-\left(\varphi_0(y)+\sum_{i\ge1}\varphi_i(y)\right)F(y)\right|\\&\le\varphi_0(y)|F_0(y)-F(y)|+\sum_{i\ge1}\varphi_i(y)|F(x_i)-F(y)|\\&<\varphi_0(y)\delta(y)+\sum_{i\ge1}\varphi_i(y)\delta(y)=\delta(y)
		\end{align*}
		
		最后证明我们的断言.对每个点$x\in M-A$,选取$x$的落在$M-A$中的开邻域$U_x$,我们要求它足够小使得$\delta(y)>\frac{1}{2}\delta(x)$和$|F(y)-F(x)|<\frac{1}{2}\delta(x)$成立.这就导致$y\in U_x$时有$|F(y)-F(x)|<\delta(y)$.完成证明.
	\end{proof}
    \item 推论.设$\delta:M\to\mathbb{R}$是正连续函数,那么存在光滑函数$e:M\to\mathbb{R}$使得$0<e(x)<\delta(x),\forall x\in M$成立.这只要在函数版本的Whitney逼近定理中考虑$|e(x)-\frac{1}{2}\delta(x)|<\frac{1}{2}\delta(x)$即可.
    \item 映射版本.设$F:N\to M$是光滑流形之间的连续映射,那么$F$(连续)同伦于一个光滑映射.如果$F$已经在某个闭子集$A\subset N$上光滑,那么这个同伦可以取为相对于$A$的.
    \item 光滑延拓引理.设$A\subset N$是闭子集,一个光滑映射$f:A\to M$可以光滑延拓到整个$N$上当且仅当它可以连续延拓到整个$N$上.
    \begin{proof}
    	
    	如果$f$延拓为连续映射$F:N\to M$,那么Whitney逼近定理保证存在光滑映射$G$相对于$A$同伦于$F$,特别的就有$G$在$A$上的限制是$f$.
    \end{proof}
    \item 光滑同伦.类似于连续同伦我们可以定义光滑同伦.光滑同伦依旧是两个光滑流形之间光滑映射集上的等价关系.另外两个源端相同,终端相同的光滑映射是相对于源端的某个闭子集是连续同伦的当且仅当是相对于源端该闭子集光滑同伦的.
\end{enumerate}

\newpage
\section{张量}
\subsection{向量丛}

定义.
\begin{enumerate}
	\item 设$M$是拓扑空间,$M$上的秩为$k$的实向量丛是指一个拓扑空间$E$和一个连续满射$\pi:E\to M$,满足如下条件.这里$E$称为全空间,$M$称为底空间,$\pi$称为投影映射.
	\begin{itemize}
		\item 对每个$p\in M$,纤维$E_p=\pi^{-1}(p)$具备一个$k$维实线性空间结构.
		\item 对每个点$p\in M$,存在$p$在$M$中的开邻域$U$,以及一个称为局部平凡化(local trivialization)的同胚映射$\Phi:\pi^{-1}(U)\to U\times\mathbb{R}^k$,这是指它满足:$\pi_U\circ\Phi=\pi$,其中$\pi_U:U\times\mathbb{R}^k\to U$是投影映射;对每个$q\in U$,映射$\Phi$在$E_q$的限制是到$\{q\}\times\mathbb{R}^k\cong\mathbb{R}^k$的线性同构.
	\end{itemize}
    \item 如果$M,E$都是光滑流形,$\pi$是光滑映射,局部平凡化是微分同胚,那么称$(E,\pi)$是一个光滑向量丛.
    \item 秩1向量丛称为线丛.类似可定义复向量丛.
    \item 如果向量丛上的一个局部平凡化是定义在整个$M$上的,就称它是整体平凡化,此时$E$微分同胚于$M\times\mathbb{R}^k$,称为平凡丛.
\end{enumerate}
\begin{enumerate}
	\item 我们之前定义的切丛是一个向量丛.
	\begin{proof}
		
		首先每个纤维$\pi^{-1}(p)$的确具备$\mathbb{R}$线性空间结构,即该点切空间.现在选取$M$上的坐标卡$(U,\varphi)$,我们只需证明切丛定义中的映射$\Phi:\pi^{-1}(U)\to U\times\mathbb{R}^n$是局部平凡化.它的坐标表示维$\Phi(\sum_iv^i\partial_i\mid_p)\mapsto(p,v^1,v^2,\cdots,v^n)$,这满足$\pi_U\circ\Phi=\pi$.另外如下映射复合是微分同胚,于是$\Phi$是微分同胚:
		$$\xymatrix{\pi^{-1}(U)\ar[r]^{\Phi}&U\times\mathbb{R}^n\ar[r]^{\varphi\times\mathrm{id}}&\varphi(U)\times\mathbb{R}^n}$$
	\end{proof}
    \item 设$\pi:E\to M$是一个秩为$k$的光滑向量丛,设$\Phi:\pi^{-1}(U)\to U\times\mathbb{R}^k$和$\Psi:\pi^{-1}(V)\to V\times\mathbb{R}^k$是两个局部平凡化,满足$U\cap V$非空.那么存在光滑映射$\tau:U\cap V\to\mathrm{GL}(k,\mathbb{R})$,使得$\Phi\circ\Psi^{-1}:(U\cap V)\times\mathbb{R}^k\to(U\cap V)\times\mathbb{R}^k$为$(p,v)\mapsto(p,\tau(p)v)$.这里$\tau$称为两个局部平凡化之间的过渡映射.
    \begin{proof}
    	
    	$$\xymatrix{(U\cap V)\times\mathbb{R}^k\ar[drr]_{\pi_1}&&\pi^{-1}(U\cap V)\ar[ll]_{\Psi}\ar[d]_{\pi}\ar[rr]^{\Phi}&&(U\cap V)\times\mathbb{R}^k\ar[dll]^{\pi_1}\\&&U\cap V&&}$$
    	
    	考虑如上交换图,有$\pi_1\circ(\Phi\circ\Psi^{-1})=\pi_1$,于是有$\Phi\circ\Psi^{-1}(p,v)=(p,\sigma(p,v))$.这里$\sigma:(U\cap V)\times\mathbb{R}^k\to\mathbb{R}^k$是光滑映射.但是固定$p\in U\cap V$有$v\mapsto\sigma(p,v)$是$\mathbb{R}^k$上的可逆线性变换(局部平凡化的要求),所以存在可逆线性映射$\tau(p)$使得$\sigma(p,v)=\tau(p)v$.最后要验证$\tau$是光滑映射【】.
    \end{proof}
    \item 向量丛的粘合验证引理.设$M$是光滑流形,固定非负整数$k$,对每个点$p\in M$赋予一个$k$维实线性空间$E_p$,记$E=\coprod_{p\in M}E_p$,记$\pi:E\to M$是投影映射,把$E_p$中的元映射为$p$.如果如下条件均成立,那么$E$上存在唯一的拓扑结构和光滑结构使得它成为$M$上的一个光滑向量丛,并且这里$\{(U_i,\Phi_i)\}$是光滑局部平凡化.
    \begin{itemize}
    	\item 存在$M$的开覆盖$\{U_i\}$.
    	\item 对每个指标$i$,存在双射$\Phi_i:\pi^{-1}(U_i)\to U_i\times\mathbb{R}^k$,它限制在每个$E_p$上是到$\{p\}\times\mathbb{R}^k\cong\mathbb{R}^k$的线性同构.
    	\item 对每对指标$i,j$,如果$U_i\cap U_j$非空,存在光滑映射$\tau_{i,j}:U_i\cap U_j\to\mathrm{GL}(k,\mathbb{R})$,使得$(U_i\cap U_j)\times\mathbb{R}^k$的到自身的映射$\Phi_i\circ\Phi_j^{-1}$可以表示为$(p,v)\mapsto(p,\tau_{i,j}(p)v)$.
    \end{itemize}
\end{enumerate}

向量丛上的截面.设$\pi:E\to M$是向量丛.
\begin{itemize}
	\item 它的截面是指一个映射$\sigma:M\to E$使得$\pi\circ\sigma=\mathrm{id}_M$,换句话讲截面$\sigma$就是对每个点$p\in M$赋予$E_p$中的一个向量.如果截面没有要求连续性,就称它是粗糙(rough)截面;如果截面作为映射是连续或者光滑的,就称它是连续或者光滑截面.这里我们不加强调时提及截面总是光滑的.
	\item 一个局部截面是指$M$的开子集$U$到$E$的光滑映射$\sigma$使得$\pi\circ\sigma=\mathrm{id}_U$.如果这里$U=M$就称它是整体截面.
	\item 截面$\sigma$的支集是指$M$的子集$\{p\in M\mid\sigma(p)\not=0\}$的闭包.
\end{itemize}
\begin{enumerate}
	\item $E$上的零截面是指截面$\sigma(p)=0_p\in E_p$.零截面总是光滑的.
	\item $E$上全体截面构成一个$\mathrm{C}^{\infty}(M)$模,这里数乘结构定义为$(f\sigma)(p)=f(p)\sigma(p)$.
	\item 截面的延拓定理.设$\pi:E\to M$是光滑向量从,如果$A\subset M$是闭子集,并且$\sigma:A\to E$是满足$\pi\circ\sigma=\mathrm{id}_E$的光滑映射,这里光滑性是指对每个点$p\in A$,都存在$p$在$E$上的开邻域以及开邻域到$E$的光滑映射,使得它在$A$上的限制和$\sigma$在上面限制一致.那么我们断言都没个包含$A$的开子集$U\subset M$,都存在$E$上的整体截面$\sigma_0$使得$\sigma_0\mid A=\sigma$,并且$\mathrm{Supp}\sigma_0\subset U$.
\end{enumerate}

标架.设$\pi:E\to M$是向量丛.设$U\subset M$是非空开子集,其上有序的$k$个截面$\{\sigma_1,\sigma_2,\cdots,\sigma_k\}$称为是$U$上的一个局部标架,如果对每个$p\in U$都有$\{\sigma_1(p),\sigma_2(p),\cdots,\sigma_k(p)\}$是$E_p$的一组基.如果$U=M$就称它们是一个整体标架.
\begin{enumerate}
	\item 类似切丛有如下完备性定理.设$\pi:E\to M$是秩为$k$的向量丛.
	\begin{itemize}
		\item 如果$\{\sigma_1,\sigma_2,\cdots,\sigma_m\}$是开集$U\subset M$上的$m$个线性无关的截面,那么对每个$p\in U$存在它在$U$中的开邻域$V$上的截面$\sigma_{m+1},\cdots,\sigma_k$,使得$\{\sigma_1,\sigma_2,\cdots,\sigma_k\}$构成了$V$上的局部标架.
		\item 如果$\{v_1,v_2,\cdots,v_m\}$是$E_p$中线性无关的$m$个向量,那么存在$p$的某个开邻域上的局部标架$\{\sigma_1,\sigma_2,\cdots,\sigma_k\}$使得$\sigma_i(p)=v_i,1\le i\le m$.
		\item 如果$A\subset M$是闭子集,并且$\{\tau_1,\tau_2,\cdots,\tau_k\}$是$E\mid A$上的光滑截面,那么存在$A$的某个开邻域上的局部标架$\{\sigma_1,\sigma_2,\cdots,\sigma_k\}$,使得$\sigma_i\mid A=\tau_i$.
	\end{itemize}
    \item 局部平凡化诱导的局部标架.设$\pi:E\to M$是向量丛,设$\Phi:\pi^{-1}(U)\to U\times\mathbb{R}^k$是一个局部平凡化,定义$\sigma_i(p)=\Phi^{-1}(p,e_i)$,那么这是一个光滑映射,并且从$\pi_1\circ\Phi=\pi$得到$\pi\circ\sigma_i(p)=\pi\circ\Phi^{-1}(p,e_i)=\pi_1(p,e_i)=p$说明$\sigma_i$是截面.最后要说明$\{\sigma_i(p)\}$是$E_p$的一组基,按照$\Phi(\sigma_i(p))=(p,e_i)$,从$\{e_i\}$构成$\{p\}\times\mathbb{R}^k\cong\mathbb{R}^k$的一组基得证.我们称构造的局部标架$\{\sigma_i\}$是和局部平凡化$\Phi$相伴的.
    \item 反过来,每个局部标架必然是某个局部平凡化诱导的.
    \begin{proof}
    	
    	设$\pi:E\to M$是向量丛,设$\{\sigma_i\}$是开集$U\subset M$上的局部标架.定义$\Psi:U\times\mathbb{R}^k\to\pi^{-1}(U)$为$(p,v^i)\mapsto\sum_iv^i\sigma_i(p)$.这是一个双射.记$\widetilde{e_i}:U\to U\times\mathbb{R}^k$是映射$p\mapsto(p,e_i)$,其中$e_i\in\mathbb{R}^k$是第$i$分量为1其余分量为0的元.那么有$\sigma_i=\Psi\circ\widetilde{e_i}$.于是一旦我们证明$\Psi$是微分同胚,那么$\Psi^{-1}$是诱导了$\{\sigma_i\}$的局部平凡化.
    	
    	按照$\Psi$是双射,为了证明它是微分同胚仅需验证它是局部微分同胚.任取$q\in U$,选取$q$在$M$中的开邻域$V$使得其上存在局部平凡化$\Phi:\pi^{-1}(V)\to V\times\mathbb{R}^k$.按照$\Phi$是微分同胚,如果证明了$\Phi\circ\Psi\mid V\times\mathbb{R}^k$是$V\times\mathbb{R}^k\to V\times\mathbb{R}^k$的微分同胚,那么就得到$\Psi$限制在$V\times\mathbb{R}^k$上是微分同胚.
    	$$\xymatrix{V\times\mathbb{R}^k\ar[rr]^{\psi\mid V\times\mathbb{R}^k}\ar[drr]_{\pi_1}&&\pi^{-1}(V)\ar[rr]^{\Phi}\ar[d]_{\pi}&&V\times\mathbb{R}^k\ar[dll]^{\pi_1}\\&&V\times\mathbb{R}^k&&}$$
    	
    	对每个截面$\sigma_i$,复合映射$\Phi\circ\sigma_i\mid V:V\to V\times\mathbb{R}^k$是光滑的,于是存在光滑函数$\sigma_i^j:V\to\mathbb{R}$使得$\Phi\circ\sigma_i(p)=(p,\sigma_i^1(p),\cdots,\sigma_i^k(p))$.于是有$\Phi\circ\Psi(p,v^1,v^2,\cdots,v^k)=(p,\sum_iv^i\sigma_i^1(p),\cdots,\sum_iv^i\sigma_i^k(p))$是光滑的.
    	
    	再证明$(\Phi\circ\Psi)^{-1}$是光滑的,我们有$\left(\sigma_i^j(p)\right)$对每个$p\in V$都是可逆矩阵,它的逆矩阵记作$\left(\tau_i^j(p)\right)$,因为取逆矩阵是$\mathrm{GL}(k,\mathbb{R})$上的光滑映射,于是每个$\tau_i^j$都是光滑函数,于是$(\Phi\circ\Psi)^{-1}(p,w^1,w^2,\cdots,w^k)=(p,\sum_iw^i\tau_i^1(p),\cdots,\sum_iw^i\tau_i^k(p))$是光滑映射.
    \end{proof}
    \item 推论.一个光滑向量丛是平凡丛当且仅当它存在整体光滑标架.
    \item 设$\pi:E\to M$是秩$k$的向量丛,设$(V,\varphi)$是$M$上的光滑坐标卡,记坐标映射为$(x^i)$,设$V$上有局部标架$\{\sigma_i\}$,定义$\widetilde{\varphi}:\pi^{-1}(V)\to\varphi(V)\times\mathbb{R}^k$为$\sum_iv^i\sigma_i(p)\mapsto(x^1(p),\cdots,x^n(p),v^1,\cdots,v^k)$.那么$(\pi^{-1}(V),\widetilde{\varphi})$是$E$上的一个光滑坐标卡.
    \item 设$\{\sigma_i\}$是开集$U\subset M$上的局部标架,如果$\tau$是$U$上的粗糙截面,那么$\tau$在$U$上每个点可以表示为$\tau(p)=\sum_i\tau^i(p)\sigma_i(p)$,这里$\tau_i$称为$\tau$关于这组局部标架的分量函数.我们断言$\tau$在$U$上光滑当且仅当全部分量函数是光滑的.
    \begin{proof}
    	
    	设$\Phi:\pi^{-1}(U)\to U\times\mathbb{R}^k$是和局部标架$\{\sigma_i\}$相伴的局部平凡化,按照$\Phi$是微分同胚,说明$\tau$在$U$上光滑当且仅当$\Phi\circ\tau$在$U$上光滑.但是我们有$\Phi\circ\tau(p)=(p,\tau^1(p),\cdots,\tau^k(p))$,于是$\tau$光滑当且仅当全部$\tau^i$光滑.
    \end{proof}
\end{enumerate}

丛同态.
\begin{itemize}
	\item 设$\pi:E\to M$和$\pi':E'\to M'$是两个向量丛,一个连续或者光滑的映射$F:E\to E'$称为连续或者光滑的丛同态,如果存在映射$f:M\to M'$,使得如下图表交换,并且对每个$p\in M$,限制映射$F\mid E_p$是$E_p\to E_{f(p)}'$的线性映射.这个条件称为$F$覆盖了$f$.当我们提及丛同态时未加说明时总是指光滑的丛同态.
	$$\xymatrix{E\ar[rr]^F\ar[d]_{\pi}&&E'\ar[d]^{\pi'}\\M\ar[rr]_f&&M'}$$
	\item 如果$\pi:E\to M$和$\pi':E'\to M$是同一个光滑流形上的两个向量丛,一个丛同态$F:E\to E'$是指使得如下图表交换的连续或者光滑映射,同样的不加说明时提及丛同态总是指光滑的.如果两个向量丛是同一个光滑流形上的,我们通常把丛同态称为$M$上的丛同态.
	$$\xymatrix{E\ar[rr]^F\ar[dr]_{\pi}&&E'\ar[dl]^{\pi'}\\&M&}$$
\end{itemize}
\begin{enumerate}
	\item 如果$F$覆盖了$f$,那么$f$被$F$唯一决定,另外如果$F$是连续或者光滑的,有$f$也是连续或者光滑的.这件事只要注意到$f=\pi'\circ F\circ\delta$,其中$\delta:M\to E$是零截面.
    \item 全体(光滑)向量丛和(光滑)丛同态构成范畴,$M$上的全体向量丛和丛同态也构成范畴.
    \item 例子.
    \begin{itemize}
    	\item 设$F:M\to N$是光滑映射,那么整体微分$\mathrm{d}F:\mathrm{T}M\to\mathrm{T}N$是覆盖了$F$的丛同态.
    	$$\xymatrix{\mathrm{T}M\ar[rr]^{\mathrm{d}F}\ar[d]_{\pi_M}&&\mathrm{T}N\ar[d]^{\pi_N}\\M\ar[rr]_F&&N}$$
    	\item 设$S\subset M$是浸入子流形,设$\pi:E\to M$是向量丛,那么包含映射$E\mid S\subset E$是覆盖了包含映射$S\subset M$的丛同态.
    	$$\xymatrix{E\mid S\ar[rr]^{\subset}\ar[d]_{\pi\mid S}&&E\ar[d]^{\pi}\\S\ar[rr]_{\subset}&&M}$$
    \end{itemize}
    \item 如果$E,E'$是$M$上的两个向量丛,并且$F:E\to E'$是一个双射丛同态,那么它的逆映射自动是一个丛同态,换句话讲双射丛同态是丛同构.
    \item $M$上的秩为$k$的丛$E$是平凡丛当且仅当它丛同构于平凡丛$M\times\mathbb{R}^k$.
\end{enumerate}

丛同态诱导的截面.
\begin{enumerate}
	\item 设$\pi:E\to M$和$\pi':E'\to M$是$M$上的两个向量丛,设$F:E\to E'$是丛同态,它诱导了截面之间的映射$\widetilde{F}:\Gamma(E)\to\Gamma(E')$为$\sigma\mapsto F\circ\sigma$.这是一个$\mathrm{C}^{\infty}(M)$模同态,即对$u_1,u_2\in\mathrm{C}^{\infty}(M)$和$\sigma_1,\sigma_2\in\Gamma(E)$,有$\widetilde{F}(u_1\sigma_1+u_2\sigma_2)=u_1\widetilde{F}(\sigma_1)+u_2\widetilde{F}(\sigma_2)$,因为丛同态在纤维上的限制是线性同态.
	\item 这个逆命题也是成立的:设$\pi:E\to M$和$\pi':E'\to M$是两个向量丛,设$\widetilde{F}$是$\Gamma(E)\to\Gamma(E')$的$\mathrm{C}^{\infty}(M)$模同态,那么$\widetilde{F}$必然是被某个丛同态$F:E\to E'$诱导的,换句话讲对每个$\sigma\in\Gamma(E)$,都有$\widetilde{F}(\sigma)=F\circ\sigma$.
	\begin{proof}
		
		我们先证明如果在某个非空开子集$U\subset M$上截面满足$\sigma_1\equiv\sigma_2$,那么在$U$上就有$\widetilde{F}(\sigma_1)=\widetilde{F}(\sigma_2)$.记$\tau=\sigma_1-\sigma_2$,按照$\widetilde{F}$的线性,只需验证如果$\tau$在$U$上恒为零,那么$\widetilde{F}(\tau)$也在$U$上恒为零.任取$p\in U$,取$\{p\}\subset U$上的光滑碰撞函数$\psi$,按照$\psi\tau$在$M$上恒为零,得到$0=\widetilde{F}(\psi\tau)=\psi\widetilde{F}(\tau)$.考虑这个等式在点$p$处的取值,就有$0=\psi(p)\widetilde{F}(\tau)(p)=\widetilde{F}(\tau)(p)$,而这里$p$是任意的,就得到$\widetilde{F}(\tau)$在$U$上恒为零.
		
		再证明如果$\sigma_1(p)=\sigma_2(p)$,那么有$\widetilde{F}(\sigma_1)(p)=\widetilde{F}(\sigma_2)(p)$.同样的,按照$\widetilde{F}$的线性,只需验证如果对某个$p\in U$有$\tau(p)=0$,那么$\widetilde{F}(\tau)(p)=0$.选取点$p$在$U$中的某个开邻域上的局部标架$\{\sigma_1,\sigma_2,\cdots,\sigma_k\}$.在这组标架下把$\tau$记作$\sum_iu^i\sigma_i$.那么条件$\tau(p)=0$意味着每个$u^i(p)=0$.按照截面的延拓定理,存在$E$上的整体截面$\{\widetilde{\sigma_i}\}$使得每个$\widetilde{\sigma_i}$和$\sigma_i$在$p$的附近一致.我们还可以取$M$上的整体光滑函数$\{\widetilde{u}^i\}$使得$\widetilde{u}^i$和$u_i$在$p$的附近一致.于是在$p$的附近有$\tau=\sum_i\widetilde{u}^i\widetilde{\sigma_i}$,于是按照上一段结论就有$\widetilde{F}(\tau)(p)=\widetilde{F}(\sum_i\widetilde{u}^i\widetilde{\sigma_i})(p)=\widetilde{u}^i(p)\widetilde{F}(\widetilde{\sigma_i})(p)=0$.
		
		现在构造丛同态$F:E\to E'$如下.对每个$p\in M$和$v\in E_p$,定义$F(v)=\widetilde{F}(\widetilde{v})(p)\in E'_p$,这里$\widetilde{v}$是满足$\widehat{v}(p)=v$的$E$上的整体截面.我们前一段证明的事情就是解释了这个定义不依赖于$\widetilde{v}$这样整体延拓的选取.这个映射$F$自然满足$\pi'\circ F=\pi$,它在纤维上是线性的因为$\widetilde{F}$是线性的.另外按照定义它满足$F\circ\sigma(p)=\widetilde{F}(\sigma)(p)$.现在为了证明原问题我们仅需验证$F$的光滑性.
		
		任取$p\in M$,设$\{\sigma_i\}$是$E$在点$p$附近的局部标架.按照延拓引理,存在$E$上的整体截面$\{\widetilde{\sigma_i}\}$使得每个$\widetilde{\sigma_i}$和$\sigma_i$都在$p$的某个固定的足够小的开邻域$U$上一致.适当缩小$U$,我们可以约定$E'$在$U$上也存在局部标架$\{\sigma_j'\}$.于是存在$U$上的光滑函数$A_i^j$使得$\widetilde{F}(\widetilde{\sigma_i})\mid U=\sum_jA_i^j\sigma_j^i$.现在任取$q\in U$和$v\in E_q$,可记$v=\sum_iv^i\sigma_i(q)$,其中$v^i$是实数,那么有$F(v)=\sum_iv^i\widetilde{F}(\widetilde{\sigma_i})(q)-\sum_{i,j}v^iA_i^j(q)\sigma_j'(q)$.再设$E$和$E'$的局部标架$\{\sigma_i\}$和$\{\sigma_i'\}$相伴的局部平凡化分别为$\Phi,\Phi'$.那么就有$F$的坐标表示$\Phi'\circ F\circ\Phi^{-1}:U\times\mathbb{R}^k\to U\times\mathbb{R}^k$为$(q,v^1,v^2,\cdots,v^k)\mapsto(q,\sum_iA_i^1(q)v^i,\cdots,\sum_iA_i^m(q)v^i)$.这是光滑的,于是$F$是光滑映射.
	\end{proof}
\end{enumerate}

子向量丛.设$\pi_E:E\to M$是$M$上的向量丛.它的一个子丛是一个向量丛$\pi_D:D\to M$,满足$D$是$E$的一个拓扑子空间,并且对每个$p\in M$都有$D_p=D\cap E_p$是$E_p$的线性子空间(这要求了全部这样的子空间$D_p$的维数相同),并且$\pi_D$是$\pi_E$在$D$上的限制.
\begin{enumerate}
	\item 如果$D$是$E$的子丛,那么包含映射$i:D\subset E$总是光滑映射.
	\item 设$\pi:E\to M$是光滑向量丛,如果对每个$p\in M$选取$E_p$的一个$m$维子空间$D_p$,那么$D=\cup_{p\in M}D_p\subset E$是光滑子丛当且仅当如下条件满足:$M$中的每个点都存在开邻域$U$使得存在$U\to E$的光滑截面$\sigma_1,\sigma_2,\cdots,\sigma_m$满足对每个$q\in U$有$\sigma_1(q),\sigma_2(q),\cdots,\sigma_m(q)$构成了$D_q$的一组基.
	\begin{proof}
		
		必要性,假设$D$是光滑子丛,对每个点$p\in M$存在开邻域$U$上存在局部平凡化,我们解释过局部平凡化一一对应于局部标架,于是$U$上存在一个$D$的局部标架,记作$\{\tau_1,\tau_2,\cdots,\tau_m\}$(它们都是$U\to D$的截面),它们在点$p\in U$上总构成一组基.把它们复合上光滑的包含映射$i:D\to E$即得到要求的光滑截面.
		
		充分性,设$\pi:E\to M$是秩为$k$的向量丛,设$D\subset E$满足命题中的要求.我们来证明$D$是$E$的嵌入子流形,并且$\pi$在$D$上的限制使得$D$成为$M$上的一个光滑向量丛.
		
		先证明$D\subset E$是嵌入子流形.只需验证每个点$p\in M$存在开邻域$U$使得$D\cap\pi^{-1}(U)$是$\pi^{-1}(U)$的嵌入子流形.而这是因为我们可以选取$p$的开邻域$U$使得其上存在满足命题条件的光滑截面$\{\sigma_1,\cdots,\sigma_m\}$.我们解释过完备性允许我们适当缩小$U$后把这些$\{\sigma_1,\cdots,\sigma_m\}$延拓为局部标架$\{\sigma_1,\cdots,\sigma_k\}$.而这个局部标架对应于一个局部平凡化$\Phi:\pi^{-1}(U)\to U\times\mathbb{R}^k$为$\sum_is^i\sigma_i(q)\mapsto(q,s^1,s^2,\cdots,s^k)$.而$\Phi$把$D\cap\pi^{-1}(U)$映射为$U\times\mathbb{R}^k$的子集$\{(q,s^1,s^2,\cdots,s^m,0,\cdots,0)\}$,这是$U\times\mathbb{R}^k$的嵌入子流形.于是这是嵌入子流形.最后定义$\Psi:D\cap\pi^{-1}(U)\to U\times\mathbb{R}^m$为$\sum_is^i\sigma_i(q)\mapsto(q,s^1,s^2,\cdots,s^m)$,这是$D$上的局部平凡化,于是$D$自身是一个向量丛.
	\end{proof}
    \item 设$F:E\to E'$是$M$上的向量丛之间的丛同态,称它是常秩的如果对每个$p\in M$,$F$限制为$E_p\to E'_p$的线性变换的秩都是相同的.如果$F$是常秩的丛同态,那么它的核是$E$的光滑子丛,它的像是$E'$的光滑子丛.
\end{enumerate}

最后我们给出纤维丛的定义,它是向量丛的推广.设$M$是拓扑空间,它的一个纤维丛是指$(E,F,\pi:E\to M)$,其中$E,F$都是拓扑空间,$F$称为纤维模型,$E$称为全空间,$M$称为底空间,$\pi$是满连续映射.满足对每个$x\in M$,都存在$x$的开邻域$U$和一个同胚映射$\Phi:\pi^{-1}(U)\to U\times F$,它们满足$\pi_1\circ\Phi=\pi$,这里$\pi_1:U\times F\to U$是连续的投影映射.这样的$\Phi$称为$E$在$U$上的一个局部平凡化.

\subsection{余切丛}

余切向量.
\begin{enumerate}
	\item 余切向量和坐标表示.设$M$是光滑流形,点$p$处切空间的对偶空间$\mathrm{T}^*_pM$称为点$p$的余切空间,它的元素是切空间上的线性函数,称为点$p$的与切向量.给定光滑坐标卡$(U,(x^i))$,对点$p\in U$,它的切空间上有基$\{\partial/\partial x^i\mid_p\}$,这组基的对偶基记作$\{\mathrm{d}x^i\mid_p\}$(这个记号的实际含义我们放在后面解释).于是点$p$的每个余切向量可以表示为$\omega=\sum_i\omega_i\mathrm{d}x^i$.
	\item 余切向量坐标表示的变换公式.如果选取两个光滑坐标卡$(U,(x^i))$和$(V,(y^i))$,设$U\cap V$非空,任取其中一个元$p$,取$p$的一个余切向量$\omega$,那么它在这两个坐标卡下有两种表示$\omega=\sum_ia_i\mathrm{d}x^i=\sum_ib_i\mathrm{d}y^i$.那么有:
	$$a_i=\omega\left(\frac{\partial}{\partial x^i}\mid_p\right)=\omega\left(\sum_j\frac{\partial y^j}{\partial x^j}(p)\frac{\partial}{\partial y^j}\mid_p\right)=\sum_j\frac{\partial y^j}{\partial x^i}(p)b_j$$
\end{enumerate}

余切丛.
\begin{enumerate}
	\item 设$M$是光滑流形,它的余切丛是它的全体余切空间的无交并$\mathrm{T}^*M=\coprod_{p\in M}\mathrm{T}^*_pM$,投影映射$\pi:\mathrm{T}^*M\to M$就定义为$(p,\omega)\mapsto p$.对每个$M$上的光滑坐标卡$(U,(x^i))$,定义局部平凡化为$\Phi:\pi^{-1}(U)\to U\times\mathbb{R}^n$为$\sum_ia_i\mathrm{d}x^i\mapsto(p,(a_i))$.这些局部平凡化作为光滑坐标卡使得$\mathrm{T}^*M$是一个$2n$维光滑流形,并且$\pi:\mathrm{T}^*M\to M$是秩$n$的向量丛.
	\item 类似的我们可以定义向量丛的对偶丛的概念.如果$\pi:E\to M$是向量丛,取$E^*=\coprod_{p\in M}E_p^*$,其中$E_p^*$是$E_p$的对偶空间.它具备$M$上光滑向量丛结构,称为$E$的对偶丛.
\end{enumerate}

余切向量场(微分1形式).
\begin{enumerate}
	\item 余切丛的光滑截面称为余切向量场或者微分1形式.当我们提及余切向量场或者微分1形式的时候默认是光滑的,但是我们也会用粗糙余切向量场特指不确定连续性的截面.类似向量场的情况,一个微分1形式$\omega$在点$p$的取值记作$\omega_p$.另外如果选取一个光滑坐标卡$(U,(x^i))$,那么即便一个粗糙微分1形式$\omega$也可以在$U$上表示为$\sum_i\omega_i\mathrm{d}x^i$,其中$\omega_i$是$U$上的实函数,它们也称为$\omega$关于这个坐标卡的分量函数.另外$\omega$在$U$上光滑当且仅当所有分量函数$\omega_i$都是光滑的.
	\item 微分1形式在向量场上的作用.给定$M$上的粗糙微分1形式$\omega$和向量场$X$,我们定义$\omega(X):M\to\mathbb{R}$为$\omega(X)(p)=\omega_p(X_p),p\in M$.如果记坐标表示$\omega=\sum_i\omega_i\mathrm{d}x^i$和$X=\sum_jX^j\partial/\partial x^j$,那么有$\omega(X)=\sum_i\omega_iX^i$.于是如果$\omega$和$X$都是光滑的,那么$\omega(X)$是光滑函数.
	\item 设$\omega$是$M$上的粗糙微分1形式,它的光滑性具有如下等价描述:
	\begin{itemize}
		\item $\omega$关于每个光滑坐标卡的分量函数都是光滑的.
		\item 存在$M$的一个光滑坐标卡覆盖使得$\omega$对其中每个坐标卡的分量函数都是光滑的.
		\item 对每个光滑向量场$X$,都有$\omega(X)$是光滑的.
		\item 对每个开集$U\subset X$和$U$上的每个光滑向量场$X$,都有$\omega(X):U\to\mathbb{R}$是光滑的.
	\end{itemize}
\end{enumerate}

余标架.余切丛上开集或全集上的标架称为局部标架或整体标架.
\begin{enumerate}
	\item 给定开集$U\subset M$上的关于切丛的粗糙标架$\{E_1,E_2,\cdots,E_n\}$,它在每个点$p\in U$的切空间的基的对偶基构成了$U$上的一个粗糙余标架,记作$\{\varepsilon_1,\varepsilon_2,\cdots,\varepsilon_n\}$,它称为对偶于之前标架的余标架.那么我们断言一个粗糙标架是光滑的当且仅当它的对偶标架是粗糙余标架.
	\begin{proof}
		
		我们需验证明每个点$p\in U$有$\{E_i\}$光滑当且仅当$\{\varepsilon_i\}$光滑.取点$p$的光滑坐标卡$(V,(x^i))$,那么有$E_i=\sum_ka_i^k\frac{\partial}{\partial x^k}$和$\varepsilon^j=\sum_lb_l^j\mathrm{d}x^l$.我们证明过标架$\{E_i\}$光滑当且仅当$a_i^k$光滑,而余标架$\{\varepsilon^j\}$光滑当且仅当$b_l^j$光滑.但是$a_i^k$和$b_l^j$分别构成的矩阵是互逆的,于是它们的光滑性是互相等价的.
	\end{proof}
    \item 我们在向量丛的一般情况下证明过,给定局部标架,该开集上的一个粗糙微分1形式是光滑的当且仅当它关于这组标架的分量函数都是光滑的.
\end{enumerate}

函数的微分,记号$\mathrm{d}x^i$的意义.给定开集$U$上的光滑函数$f$,它的微分$\mathrm{d}f$定义为$U$上的一个微分1形式,满足$\mathrm{d}f_p(v)=v(f),v\in\mathrm{T}_pM$.
\begin{enumerate}
	\item 光滑函数的微分总是光滑的微分1形式.
	\begin{proof}
		
		容易验证$\mathrm{d}f_p$是线性的,于是它的确是$p$的一个余切向量.于是$\mathrm{d}f$的确是一个粗糙微分1形式.为了证明它的光滑性,我们之前证明过等价于验证对每个$U$内的局部光滑向量场$X$都有$\mathrm{d}f(X)=X(f)$光滑,但是这是明显成立的.
	\end{proof}
    \item 光滑函数微分的局部表示.选取光滑$f$定义域中的一个光滑坐标卡$(U,(x^i))$,那么局部上有$\mathrm{d}f_p=\sum_ir_i(p)\mathrm{d}x^i$,那么有$r_i(p)=\mathrm{d}f_p\left(\frac{\partial}{\partial x^i}\right)=\frac{\partial f}{\partial x^i}(p)$.于是记号$\mathrm{d}x^i$恰好是选取坐标卡后坐标函数$x^i$的微分.并且此时我们有微积分中的标准公式$\mathrm{d}f=\sum_i\frac{\partial f}{\partial x^i}\mathrm{d}x^i$.
    \item 微分的基本性质.设$f,g\in\mathrm{C}^{\infty}(M)$.
    \begin{itemize}
    	\item 如果$a,b\in\mathbb{R}$,那么有$\mathrm{d}(af+bg)=a\mathrm{d}f+b\mathrm{d}g$.
    	\item $\mathrm{d}(fg)=f\mathrm{d}g+g\mathrm{d}f$.
    	\item 如果$g\not=0$,那么$\mathrm{d}(f/g)=(g\mathrm{d}f-f\mathrm{d}g)/g^2$.
    	\item 如果$f$是常数那么$\mathrm{d}f=0$.
    \end{itemize}
    \item 设$f$是$M$上的光滑函数,那么$\mathrm{d}f\equiv0$当且仅当$f$在$M$的每个连通分支(流形是局部道路连通的,于是连通分支和道路连通分支是一致的)上是常值函数.
    \begin{proof}
    	
    	归结为证明如果$M$是连通流形,其上满足$\mathrm{d}f=0$的光滑函数$f$都有$f$是常值的.任取点$p\in M$,考虑非空集合$S=\{q\in M\mid f(q)=f(p)\}$,按照$f$的连续性这是闭集.任取$q\in S$,设$U$是以$q$为中心的坐标球,设坐标为$(x^i)$,那么对每个$i$和每个$U$都有$\partial f/\partial x^i\equiv0$,于是初等微积分说明$f$在$U$上是常值函数,于是$S$还是$M$的闭子集,最后连通性得到$S=M$.
    \end{proof}
    \item 设$f$是$M$上的光滑函数,$\gamma:J\to M$是$M$上的光滑曲线,那么$f$经曲线$\gamma$的导数恰好为$(f\circ\gamma)'(t)=\mathrm{d}f_{\gamma(t)}(\gamma'(t))$.
    \begin{align*}
    \mathrm{d}f_{\gamma(t_0)}(\gamma'(t_0))&=\gamma'(t_0)f\\&=\mathrm{d}\gamma_{t_0}\left(\frac{\mathrm{d}}{\mathrm{d}t}\mid_{t_0}\right)f\\&=\frac{\mathrm{d}}{\mathrm{d}t}\mid_{t_0}(f\circ\gamma)\\&=(f\circ\gamma)'(t_0)
    \end{align*}
\end{enumerate}

微分1形式的回拉.设$F:M\to N$是光滑映射,
\begin{itemize}
	\item 任取点$p\in M$,微分映射$\mathrm{d}F_p:\mathrm{T}_pM\to\mathrm{T}_{F(p)}N$是线性映射,它的对偶映射$\mathrm{d}F^*_p:\mathrm{T}^*_{F(p)}N\to\mathrm{T}^*_pM$称为$F$在点$p$的回拉.按照定义,对$\omega\in\mathrm{T}_{F(p)}N$和$v\in\mathrm{T}_pM$就有:
	$$\mathrm{d}F^*_p(\omega)(v)=\omega(\mathrm{d}F_p(v))$$
	\item 设$\omega$是$N$上的微分1形式,它关于$F$的回拉$F^*\omega$定义为$M$上的粗糙1形式$(F^*\omega)_p=\mathrm{d}F_p^*(\omega_{F(p)})$.它作用在向量$v\in\mathrm{T}_pM$为$(F^*\omega)_p(v)=\omega_{F(p)}(\mathrm{d}F_p(v))$.我们在下面会证明光滑1形式的回拉总是光滑的.
\end{itemize}
\begin{enumerate}
	\item 如果$F:M\to N$是光滑映射,$u$是$N$上的连续实函数,$\omega$是$N$上的光滑1形式,那么有$F^*(u\omega)=(u\circ F)F^*\omega$.如果$u$是光滑函数,那么$F^*\mathrm{d}u=\mathrm{d}(u\circ F)$.另外第二件事说明恰当形式的回拉总是恰当形式.
	\begin{proof}
		
		第一件事是因为:
		\begin{align*}
		(F^*(u\omega))_p&=\mathrm{d}F_p^*\left((u\omega)_{F(p)}\right)\\&=\mathrm{d}F_p^*\left(u(F(p))\omega_{F(p)}\right)\\&=u(F(p))\mathrm{d}F_p^*(\omega_{F(p)})\\&=u(F(p))(F^*\omega)_p\\&=\left((u\circ F)F^*\omega\right)_p
		\end{align*}
		
		第二件事是因为:任取$p\in M$和$v\in\mathrm{T}_pM$,那么
		\begin{align*}
		(F^*\mathrm{d}u)_p(v)&=\mathrm{d}u_{F(p)}\left(\mathrm{d}F_p(v)\right)\\&=v(u\circ F)=\mathrm{d}(u\circ F)_p(v)
		\end{align*}
	\end{proof}
    \item 如果$F:M\to N$是光滑映射,$\omega$是$N$上的连续余向量场,那么$F^*\omega$是$M$上的连续余向量场;如果$\omega$是$N$上的光滑余向量场,那么$F^*\omega$是$M$上的光滑余向量场.
    \begin{proof}
    	
    	任取点$p\in M$,任取$F(p)$的光滑坐标卡$(V,(y^j))$,那么$U=F^{-1}(V)$是$p$的开邻域.把$\omega$写作局部表示$\sum_j\omega_j\mathrm{d}y^j$,那么$\omega_j$都是$V$上的连续函数.按照上一条的公式得到如下等式,于是$\omega$的连续性和光滑性都可以传递给$F^*\omega$.
    	$$F^*\omega=F^*(\sum_j\omega_j\mathrm{d}y^j)=\sum_j(\omega_j\circ F)F^*\mathrm{d}y^j=\sum_j(\omega_j\circ F)\mathrm{d}(y^j\circ F)$$
    \end{proof}
    \item 余切向量场在子流形上的回拉.设$S\subset M$是浸入子流形,包含映射记作$i$.任取$M$上的余切向量场$\omega$,那么它关于$i$的回拉$i^*\omega$是$S$上的余切向量场,并且它恰好就是$\omega$在$S$的切空间上的限制.这称为$\omega$在子流形$S$上的回拉.注意$\omega$在子流形$S$上的限制为零是指对每个$p\in S$,都有$\omega_p$是$\mathrm{T}_pM$上的零映射,但是$\omega$在子流形$S$上的回拉为零是指对每个$p\in S$,都有$\omega_p$限制在$\mathrm{T}_pS$上是零映射.前者是比后者更强的条件.
\end{enumerate}

曲线积分.
\begin{enumerate}
	\item 光滑曲线和分段光滑曲线.光滑流形$M$上的光滑曲线定义为从带边光滑流形$[a,b]\subset\mathbb{R}$到$M$的光滑映射$\gamma$的像.它称为分段光滑曲线如果存在有限个小区间$a=a_0<a_1<\cdots<a_k=b$,使得$\gamma$在每个$[a_{i-1},a_i]$上的限制都是光滑曲线.
	\item 首先是最基本的情况.如果$[a,b]\subset\mathbb{R}$是紧区间,设$\omega$是$[a,b]$上的光滑1形式.如果我们记$\mathbb{R}$上的标准坐标函数为$t$,那么有$\omega_t=f(t)\mathrm{d}t$.我们定义$\omega$在$[a,b]$上的积分为$\int_{[a,b]}\omega=\int_a^bf(t)\mathrm{d}t$.
	\item 如果$\varphi:[c,d]\to[a,b]$是单调递增的微分同胚,那么微积分的换元公式说明有:
	$$\int_{[c,d]}\varphi^*\omega=\int_{[a,b]}\omega$$
	\item 如果$M$是连通光滑流形,那么它的任意两个点存在分段光滑曲线相连.
	\begin{proof}
		
		任取$p\in M$,记全体能经分段光滑曲线和$p$相连的点构成的集合为$S$,这是非空的因为$p\in S$,我们只需验证$S=M$,而这只需验证$S$是开闭子集.
		
		$S$是开集:任取$q\in S$,取以$q$为中心的光滑坐标球$U$,如果$q'\in U$,那么在微分同胚下有直线段连接$q$和$q'$,于是$q'\in S$,于是$U\subset S$,于是$S$是开集.
		
		$S$是闭集:任取$S$的聚点$q'$,选取以$q'$为中心的光滑坐标球$U$,那么在$U\cap S$中存在异于$q'$的点$q$,同样有$q'$和$q$在微分同胚到欧氏空间中存在直线段相连,于是$q'\in S$,于是$S$是闭集.
	\end{proof}
    \item 定义.设$\gamma:[a,b]\to M$是光滑曲线,设$\omega$是$M$上的微分1形式,定义$\omega$沿$\gamma$的曲线积分为$\int_{\gamma}\omega=\int_{[a,b]}\gamma^*\omega$.如果$\gamma$是分段光滑的,记分段区间为$a=a_0<a_1<\cdots<a_k=b$,定义$\int_{\gamma}\omega=\sum_{i=1}^k\int_{[a_{i-1,a_i}]}\gamma^*\omega$.
    \item 曲线积分的一些基本性质.设$\gamma:[a,b]\to M$是分段光滑曲线,设$\omega_i,\omega$都是$M$上的微分1形式.
    \begin{itemize}
    	\item 对实数$c_1,c_2\in\mathbb{R}$有:
    	$$\int_{\gamma}(c_1\omega_1+c_2\omega_2)=c_1\int_{\gamma}\omega_1+c_2\int_{\gamma}\omega_2$$
    	\item 如果$\gamma$是常值映射,那么$\int_{\gamma}\omega=0$.
    	\item 如果$\gamma_1=\gamma\mid[a,c]$和$\gamma_2=\gamma\mid[c,b]$,其中$a\le c\le b$,那么有:$$\int_{\gamma}\omega=\int_{\gamma_1}\omega+\int_{\gamma_2}\omega$$
    	\item 如果$F:M\to N$是光滑映射,$\eta$是$N$上的微分1形式,那么有:$$\int_{\gamma}F^*\eta=\int_{F\circ\gamma}\eta$$
    \end{itemize}
    \item 给定两条分段光滑曲线$\gamma_1:[a,b]\to M$和$\gamma_2:[c,d]\to M$,它们互相称为正的重参数化,如果存在单调递增的微分同胚$\varphi:[c,d]\to[a,b]$使得$\gamma_2=\gamma_1\circ\varphi$.如果微分同胚$\varphi$是单调递减的就称它们互为负的重参数化.按照曲线积分的基本性质,互为正的重参数化的曲线上同一个微分1形式的积分是相同的,互为负的重参数化的曲线上同一个微分1形式的积分差一个负号.
    \item 如果$\gamma:[a,b]\to M$是分段光滑曲线,设$\omega$是$M$上的微分1形式,那么曲线积分可以用经典积分计算:$$\int_{\gamma}\omega=\int_a^b\omega_{\gamma(t)}(\gamma'(t))\mathrm{d}t$$
    \begin{proof}
    	
    	不妨设$\gamma$是光滑曲线,否则只需考虑有限个光滑的分段上积分相加.按照$[a,b]$是紧集,我们还可以不妨设$[a,b]$的像包含在$M$的一个光滑坐标卡中(否则按照勒贝格数引理可以选取足够小的分段分支).现在在这个坐标卡上有局部表示$\gamma(t)=(\gamma^1(t),\gamma^2(t),\cdots,\gamma^n(t))$和$\omega=\sum_i\omega_i\mathrm{d}x^i$,那么有:
    	$$(\gamma^*\omega)_t=\sum_i(\omega_i\circ\gamma)(t)(\mathrm{d}\gamma^i)_t=\sum_i\omega_i(\gamma(t))(\gamma^i)'(t)\mathrm{d}t=\sum_i\omega_i(\gamma(t))\mathrm{d}x^i(\gamma'(t))\mathrm{d}t=\omega_{\gamma(t)}(\gamma'(t))\mathrm{d}t$$
    \end{proof}
    \item 曲线积分基本定理.如果$f$是$M$上的光滑函数,$\gamma:[a,b]\to M$是分段光滑曲线,那么有如下公式. 特别的,如果$\gamma$是闭曲线,即$\gamma(a)=\gamma(b)$,那么$\mathrm{d}f$的曲线积分总是0.
    $$\int_{\gamma}\mathrm{d}f=f(\gamma(b))-f(\gamma(a))$$
    \begin{proof}
    	
    	不妨设$\gamma$本身是光滑的,否则在每个光滑小区间上做这个等式再求和即可,我们有:
    	$$\int_{\gamma}\mathrm{d}f=\int_a^b\mathrm{d}f_{\gamma(t)}(\gamma'(t))\mathrm{d}t=\int_a^b(f\circ\gamma)'(t)\mathrm{d}t=f(\gamma(b))-f(\gamma(a))$$
    \end{proof}
\end{enumerate}

恰当形式.光滑流形$M$上的微分1形式称为恰当形式,如果它可以表示为一个光滑函数的微分$\mathrm{d}f$.
\begin{enumerate}
	\item 一个微分1形式称为保守的,如果它在任意分段光滑闭曲线上的积分总是零.那么一个微分1形式$\omega$是保守的当且仅当它的曲线积分和路径无关,换句话讲如果$\gamma_1$和$\gamma_2$是源端相同终端相同的两条分段光滑曲线,那么有$\int_{\gamma_1}\omega=\int_{\gamma_2}\omega$.
	\item 光滑流形$M$上的微分1形式是恰当的当且仅当它是保守的.
	\begin{proof}
		
		曲线积分基本定理已经说明了恰当微分1形式总是保守的.反过来假设$\omega$是$M$上的保守微分1形式.固定一个点$p_0\in M$,定义函数$f:M\to\mathbb{R}$为$f(q)=\int_{p_0}^q\omega$,这里曲线取为任意的从$p_0$连接到$q$的分段光滑曲线.我们只需验证$f$是光滑的并且$\mathrm{d}f=\omega$.
		
		任取$q_0\in M$,选取以$q_0$为中心的光滑坐标卡$(U,(x^i))$,在这个坐标下记$\omega=\sum_i\omega_i\mathrm{d}x^i$,于是只需验证对每个指标$j$有$\frac{\partial f}{\partial x^j}(q_0)=\omega_j(q_0)$.这既证明了$f$的光滑性也证明了$\mathrm{d}f=\omega$.
		
		对固定的指标$j$,选取光滑曲线$\gamma:[-\varepsilon,\varepsilon]\to U$为$\gamma(t)=(0,\cdots,t,\cdots,0)$,其中$t$在第$j$分量,其余分量都是0.并且要求$\varepsilon$足够小使得$\gamma$的像集落在$U$中.记$p_1=\gamma(-\varepsilon)$,定义$\widetilde{f}:M\to\mathbb{R}$为$q\mapsto\int_{p_1}^q\omega$.按照积分和路径无关,说明$f(q)-\widetilde{f}(q)=\int_{p_0}^{p_1}\omega$和$q$无关,于是$f$和$\widetilde{f}$相差一个常数,但是常数的偏导数都是0,于是归结为证明$\frac{\partial\widetilde{f}}{\partial x^j}(q_0)=\omega_j(q_0)$.按照$\gamma$定义有$\gamma'(t)=\frac{\partial}{\partial x^j}\mid_{\gamma(t)}$,于是有:$$\omega_{\gamma(t)}(\gamma'(t))=\sum_i\omega_i(\gamma(t))\mathrm{d}x^i\left(\frac{\partial}{\partial x^j}\mid_{\gamma(t)}\right)=\omega_j(\gamma(t))$$
		
		这里$\gamma$限制在$[-\varepsilon,t]$上仍然是光滑曲线,得到$\widetilde{f}\circ\gamma(t)=\int_{p_1}^{\gamma(t)}\omega=\int_{-\varepsilon}^t\omega_{\gamma(s)}(\gamma'(s))\mathrm{d}s=\int_{-\varepsilon}^t\omega_j(\gamma(s))\mathrm{d}s$.于是按照基本定理得到:
		\begin{align*}
		\frac{\partial\widetilde{f}}{\partial x^j}(q_0)&=\gamma'(0)\widetilde{f}=\frac{\mathrm{d}}{\mathrm{d}t}\mid_{t=0}\widetilde{f}\circ\gamma(t)\\&=\frac{\mathrm{d}}{\mathrm{d}t}\mid_{t=0}\int_{-\varepsilon}^t\omega_j(\gamma(s))\mathrm{d}s\\&=\omega_j(\gamma(0))=\omega_j(q_0)
		\end{align*}
	\end{proof}
\end{enumerate}

闭形式.一个微分1形式$\omega$称为闭形式,如果它关于任意光滑坐标卡的分量函数$\{\omega_i\}$都满足$\frac{\partial\omega_j}{\partial x^i}=\frac{\partial\omega_i}{\partial x^j}$.
\begin{enumerate}
	\item 当我们介绍微分形式上的外导数(微分)$\mathrm{d}$时,一个闭形式就是满足$\mathrm{d}\omega=0$的微分形式.
	\item 恰当形式总是闭形式.因为对光滑函数$f$总有$\frac{\partial^2f}{\partial x^i\partial x^j}=\frac{\partial^2f}{\partial x^j\partial x^i}$.
	\item 设$\omega$是$M$上的微分1形式,如下条件互相等价:
	\begin{itemize}
		\item $\omega$是闭形式.
		\item 存在$M$的一个光滑坐标卡覆盖,使得其中每个坐标卡下该微分1形式的坐标函数$\{\omega_i\}$都满足$\frac{\partial\omega_j}{\partial x^i}=\frac{\partial\omega_i}{\partial x^j}$.
		\item 对每个开集$U$和$U$上的任意光滑切向量场$X,Y$,都有:
		$$X(\omega(Y))-Y(\omega(X))=\omega([X,Y])$$
	\end{itemize}
    \begin{proof}
    	
    	1推2是平凡的.2推3,任取$U$中的光滑坐标卡$(V,(x^i))$,记$\omega=\sum_i\omega_i\mathrm{d}x^i$,再记$X=\sum_iX_i\partial/\partial x^i$和$Y=\sum_iY_i\partial/\partial y^i$.那么有$X(\omega(Y))=\sum_iX(\omega_iY^i)=\sum_i(Y^iX\omega_i+\omega_iXY^i)=\sum_{i,j}Y^iX^j\frac{\partial\omega_i}{\partial x^j}+\sum_i\omega_iXY^i$.于是按照条件得到:
    	$$X(\omega(Y))-Y(\omega(X))=\sum_i\omega_i(XY^i-YX^i)=\omega([X,Y])$$
    	
    	3推1,任取坐标卡,取$X=\partial/\partial x^i$和$Y=\partial/\partial x^j$,那么$[X,Y]=0$,于是条件说明$X(\omega(Y))=Y(\omega(X))$,此即$\frac{\partial\omega_j}{\partial x^i}=\frac{\partial\omega_i}{\partial x^j}$.
    \end{proof}
    \item 设$F:M\to N$是一个局部微分同胚,那么$N$上的闭形式的回拉是闭形式.
    \begin{proof}
    	
    	设$(U,\varphi)$是$N$上的一个光滑坐标卡,那么对$F^{-1}(U)$中的每个点,存在足够小的开邻域使得$\varphi\circ F$是一个光滑坐标卡,在这些坐标下$F$的坐标表示是恒等映射,于是如果$\omega$在$U$上满足$\frac{\partial\omega_j}{\partial x^i}=\frac{\partial\omega_i}{\partial x^j}$,那么$F^*\omega$也在$F^{-1}(U)$上满足$\frac{\partial\omega_j}{\partial x^i}=\frac{\partial\omega_i}{\partial x^j}$.
    \end{proof}
\end{enumerate}

关于闭形式和恰当形式的关系.$\mathbb{R}^n$的星形集$S$是指存在一点$c\in S$,使得每个点$s\in S$和$c$的直线段都落在$S$中.注意凸集总是星形集.
\begin{enumerate}
	\item 设$U\subset\mathbb{R}^n$是一个星形开集,那么$U$上的每个闭形式$\omega$都是恰当形式.
	\begin{proof}
		
		设$c\in U$使得$U$中每个点和$c$的直线段都在$U$中.欧氏空间开子集上总有整体标架,于是可记$\omega=\sum_i\omega_i\mathrm{d}x^i$.我们的思路和证明保守形式是恰当形式类似,希望用以$c$为源端的曲线积分构造$\omega$的原函数$f$.但是这里我们没有积分和路径无关的条件,所以我们要选取特殊的路径.
		
		不妨设$c=0$,否则我们可以对星形集$S$和闭形式$\omega$做一些平移.任取$x\in U$,定义光滑曲线$\gamma_x:[0,1]\to U,t\mapsto tx$为从$0$到$x$的直线段.星形集条件保证$\gamma_x$的像集都落在$U$中.我们就定义$f:U\to\mathbb{R}$为$f(x)=\int_{\gamma_x}\omega$.下面仅需验证$\partial f/\partial x^j=\omega_j$对任意指标$j$成立.首先按照基本定理有:
		$$f(x)=\int_0^1\omega_{\gamma_x(t)}(\gamma_x'(t))\mathrm{d}t=\int_0^1\sum_i\omega_i(tx)x^i\mathrm{d}t$$
		
		按照光滑性,求偏导和积分可交换,于是有:
		$$\frac{\partial f}{\partial x^j}(x)=\int_0^1\sum_i\left(t\frac{\partial\omega_i}{\partial x^j}(tx)x^i+\omega_j(tx)\right)\mathrm{d}t$$
		
		按照$\omega$是恰当形式,有$\frac{\partial\omega_j}{\partial x^i}=\frac{\partial\omega_i}{\partial x^j}$,于是有:
		$$\frac{\partial f}{\partial x^j}(x)=\int_0^1\sum_i\left(t\frac{\partial\omega_j}{\partial x^i}(tx)x^i+\omega_j(tx)\right)\mathrm{d}t=\int_0^1\frac{\mathrm{d}}{\mathrm{d}t}(t\omega_j(tx))\mathrm{d}t=\omega_j(x)$$
	\end{proof}
    \item 闭形式是局部恰当的.设$\omega$是$M$上的闭形式,那么每个点$p\in M$都存在开邻域使得$\omega$在其上是恰当形式.这是因为每个点附近可以取光滑坐标球,而欧氏空间中的开球是星形集,闭形式在坐标球上的限制还是闭形式,上一条结论说明这个限制是恰当形式.
\end{enumerate}

\subsection{张量}

线性空间上的共变和逆变张量.
\begin{enumerate}
	\item 一个引理.设$V_1,V_2,\cdots,V_k$是有限维实线性空间,记$\mathrm{L}(V_1,V_2,\cdots,V_k;\mathbb{R})$是$V_1\times\cdots\times V_k\to\mathbb{R}$的全体多重线性函数构成的$\mathbb{R}$线性空间,那么我们有不依赖于基选取的典范同构:
	$$V_1^*\otimes\cdots\otimes V_k^*\cong\mathrm{L}(V_1,\cdots,V_k;\mathbb{R})$$
	\item 设$V$是有限维实线性空间,它上面的一个共变$k$张量$\alpha$是指$k$个对偶空间的张量积$V^*\otimes\cdots\otimes V^*=\mathrm{T}^k(V^*)$中的元.按照引理中的典范同构,这等价于$k$个$V$的笛卡尔积上的一个多重线性函数.这里的$k$称为$\alpha$的秩.共变0张量就约定为实数.
	\item 设$V$是有限维实线性空间,它上面的一个逆变$k$张量$\alpha$是指$k$个$V$的张量积$\mathrm{T}^k(V)=V\otimes\cdots\otimes V$中的元.按照引理中的典范同构,以及典范同构$V\cong V^{**}$,这等价于$k$个$V^*$的笛卡尔积上的一个多重线性函数.0张量同样约定为实数.
	\item 设$V$是有限维实线性空间,设$k,l$是非负整数,$V$上面的一个$(k,l)$型张量是指$k$个$V$和$l$个$V^*$的张量积$\mathrm{T}^{(k,l)}$中的元.如果记$\dim V=n$,那么这个张量积是$n^{k+l}$维线性空间,如果记$\{E_i\}$是$V$的一组基,记对偶基是$\{\varepsilon^j\}$,那么这个张量积的一组基为:
	$$\{E_{i_1}\otimes\cdots\otimes E_{i_k}\otimes\varepsilon^{j_1}\otimes\cdots\otimes\varepsilon^{j_l}\}$$
	
	这里$E_{i_1}\otimes\cdots\otimes E_{i_k}\otimes\varepsilon^{j_1}\otimes\cdots\otimes\varepsilon^{j_k}$对应的是$V^*\times\cdots\times V^*\times V\times\cdots\times V$上的多重线性函数为$(\varepsilon^{u_1},\cdots,\varepsilon^{u_k},E_{v_1},\cdots,E_{v_l})\mapsto\varepsilon^{u_1}(E_{i_1})\cdots\varepsilon^{u_k}(E_{i_k})\varepsilon^{j_1}(E_{v_1})\cdots\varepsilon^{j_l}(E_{v_l})$
	\item 存在从$\mathrm{T}^{(k,l+1)}V$到形如$(V^*)^{\times l}\times(V)^{\times k}\to V$的多重线性映射的自然同构(自然指和基选取无关的同构).特别的,一个$V\to V$的线性变换可以视为$V^*\times V\to\mathbb{F}$的多重线性函数.对于$l=0,k=1$的情况此即$\mathrm{Hom}(V,V)\cong V^*\otimes V$.
	\item 张量的迹(trace,也称为contration).我们熟知的迹映射是$\mathrm{Hom}(V,V)\to\mathbb{F}$的线性映射.按照上一条此即$V^*\otimes V\to\mathbb{F}$的线性映射.一般的我们定义迹是$\mathrm{T}^{(k+1,l+1)}V\to\mathrm{T}^{(k,l)}V$的线性映射,对$T\in\mathrm{T}^{(k+1,l+1)}V$,定义$\mathrm{Tr}(T)\in\mathrm{T}^{(k,l)}V$为:
	$$x_1\otimes\cdots\otimes x_{k+1}\otimes y_1^*\otimes\cdots\otimes y_{l+1}^*\mapsto\sum_{i,j}y_i^*(x_j)x_1\otimes\cdots\otimes\widehat{x_j}\otimes\cdots\otimes x_{k+1}\otimes y_1^*\otimes\cdots\otimes\widehat{y_j^*}\otimes\cdots\otimes y_{l+1}^*$$
\end{enumerate}

对称张量(symmetric tensors)和对称算子.
\begin{itemize}
	\item 一个共变$k$张量称为对称张量,如果它作为多重线性函数$f$是对称的,换句话讲交换任意两个分量的函数值不变.全体对称$k$张量构成了$\mathrm{T}^k(V^*)$的$\mathbb{R}$线性子空间,记作$\Sigma^k(V^*)$.
	$$f(v_1,\cdots,v_i,\cdots,v_j,\cdots,v_k)=f(v_1,\cdots,v_j,\cdots,v_i,\cdots,v_k),\forall i,j$$
	\item 对称算子$\mathrm{Sym}:\mathrm{T}^k(V^*)\to\Sigma^k(V^*)$.给定共变$k$张量$\alpha$,定义它的对称张量为:
	$$(\mathrm{Sym}\alpha)(v_1,v_2,\cdots,v_k)=\frac{1}{k!}\sum_{\sigma\in S_k}\alpha(v_{\sigma(1)},v_{\sigma(2)},\cdots,v_{\sigma(k)})$$
\end{itemize}
\begin{enumerate}
	\item 一个共变$k$张量$f$是对称的等价于讲,对$k$元对称群$S_k$中的任意元$\sigma$都有$f(v_1,v_2,\cdots,v_k)=f(v_{\sigma(1)},v_{\sigma(2)},\cdots,v_{\sigma(k)})$.
	\item $\mathrm{Sym}\alpha$总是一个对称张量.事实上任取$\tau\in S_k$,那么:
	\begin{align*}
	(\mathrm{Sym}\alpha)(v_{\tau(1)},v_{\tau(2)},\cdots,v_{\tau(k)})&=\frac{1}{k!}\sum_{\sigma\in S_k}\alpha(v_{\sigma\tau(1)},v_{\sigma\tau(2)},\cdots,v_{\sigma\tau(k)})\\&=\frac{1}{k!}\sum_{\eta\in S_k}\alpha(v_{\eta(1)},v_{\eta(2)},\cdots,v_{\eta(k)})\\&=(\mathrm{Sym}\alpha)(v_1,v_2,\cdots,v_k)
	\end{align*}
	\item 上一条说明$\alpha$是对称$k$张量当且仅当$\mathrm{Sym}\alpha=\alpha$.
	\item 两个对称张量的张量积未必是对称的,因为来自不同张量的分量交换未必保证对应多重线性函数的函数值不变.不过我们可以定义两个对称张量$\alpha,\beta$的对称积为$\alpha\beta=\mathrm{Sym}(\alpha\otimes\beta)$.具体的写就是:
	$$\alpha\beta(v_1,v_2,\cdots,v_{k+l})=\frac{1}{(k+l)!}\sum_{\sigma\in S_{k+l}}\alpha(v_{\sigma(1)},v_{\sigma(2)},\cdots,v_{\sigma(k)})\beta(v_{\sigma(k+1)},v_{\sigma(k+2)},\cdots,v_{\sigma(k+l)})$$
	\item 对称积是交换和结合的,另外对余切向量$A,B$对称积总满足$AB=\frac{1}{2}(A\otimes B+B\otimes A)$.
\end{enumerate}

交错张量(alternating tensors)和交错算子.
\begin{itemize}
	\item 一个共变$k$张量称为交错张量,如果它作为多重线性函数$f$是交错的,即交换任意两个不同分量使得函数值加一个负号.交错$k$张量也称为(线性空间上的)外形式或者$k$余切向量.全体交错$k$张量构成了$\mathrm{T}^k(V^*)$的$\mathbb{R}$线性子空间,记作$\bigwedge^k(V^*)$.
	$$f(v_1,\cdots,v_i,\cdots,v_j,\cdots,v_k)=-f(v_1,\cdots,v_j,\cdots,v_i,\cdots,v_k),\forall i\not=j$$
	\item 交错算子$\mathrm{Alt}:\mathrm{T}^k(V^*)\to\bigwedge^k(V^*)$.给定共变$k$张量$\alpha$,定义它的交错张量为:
	$$(\mathrm{Alt}\alpha)(v_1,v_2,\cdots,v_k)=\frac{1}{k!}\sum_{\sigma\in S_k}(\mathrm{sgn}\sigma)\alpha(v_{\sigma(1)},v_{\sigma(2)},\cdots,v_{\sigma(k)})$$
\end{itemize}
\begin{enumerate}
	\item 设$f$是共变$k$张量,那么如下条件互相等价:
	\begin{itemize}
		\item 对每个$\sigma\in S_k$都有$f(v_1,v_2,\cdots,v_k)=(\mathrm{sgn}\sigma)f(v_{\sigma(1)},v_{\sigma(2)},\cdots,v_{\sigma(k)})$.
		\item 只要$\{v_1,v_2,\cdots,v_k\}$是线性无关的,就有$f(v_1,v_2,\cdots,v_k)=0$.
		\item 只要$v_1,v_2,\cdots,v_k$中有两个向量相同,就有$f(v_1,v_2,\cdots,v_k)=0$.
	\end{itemize}
	\item 类似对称算子,一个共变张量的交错张量总是交错的,另外$\alpha$是交错的当且仅当$\mathrm{Alt}\alpha=\alpha$.
	\item 初等交错张量.设$V$是有限维实线性空间,设$\{\varepsilon^1,\cdots,\varepsilon^n\}$是$V^*$的任意一组基,对$1\le i_1,i_2,\cdots,i_k\le n$,我们称$I=(i_1,i_2,\cdots,i_k)$是长度为$k$的复指标.对任意这样的复指标,我们定义一个共变$k$张量如下,按照行列式的性质这是一个交错张量,它称为一个初等交错张量.
	$$\varepsilon^I(v_1,v_2,\cdots,v_k)=\det\left(\begin{array}{ccc}\varepsilon^{i_1}(v_1)&\cdots&\varepsilon^{i_1}(v_k)\\\vdots&\ddots&\vdots\\\varepsilon^{i_k}(v_1)&\cdots&\varepsilon^{i_k}(v_k)\end{array}\right)=\det\left(\begin{array}{ccc}v_1^{i_1}&\cdots&v_k^{i_1}\\\vdots&\ddots&\vdots\\v_1^{i_k}&\cdots&v_k^{i_k}\end{array}\right)$$
	\item 设$\{E_i\}$是有限维实线性空间$V$的一组基,设它的对偶基为$\{\varepsilon^i\}$,设$I$是一个复指标,设$\varepsilon^I$是一个初等交错张量.
	\begin{itemize}
		\item 如果$I$中存在重复的指标,那么$\varepsilon^I=0$.
		\item 如果$\sigma\in S_k$,记复指标$I=(i_1,\cdots,i_k)$,那么记$I_{\sigma}$为$(i_{\sigma(1)},\cdots,i_{\sigma(k)})$.如果记$J=I_{\sigma}$,那么有$\varepsilon^I=(\mathrm{sgn})\varepsilon^I$.
		\item 我们有$\varepsilon^I(E_{j_1},\cdots,E_{j_k})=\delta_J^I$.这里$\delta_J^I$的定义如下,另外我们可以证明如果$I,J$都没有重复指标并且$J=I_{\sigma}$,那么$\delta_J^I=\mathrm{sgn}\sigma$;如果$I$或$J$之一存在重复指标,或者$J$不是$I$的某个置换,那么$\delta_J^I=0$.
		$$\delta_J^I=\det\left(\begin{array}{ccc}\delta_{j_1}^{i_1}&\cdots&\delta_{j_k}^{i_1}\\\vdots&\ddots&\vdots\\\delta_{j_1}^{i_k}&\cdots&\delta_{j_k}^{i_k}\end{array}\right)$$
	\end{itemize}
	\item $\bigwedge^k(V^*)$的维数和基.设$\dim V=n$,一个长度$k$的重指标$I=(i_1,i_2,\cdots,i_k)$称为递增的,如果有$1\le i_1<i_2<\cdots<i_k\le n$.那么我们断言当$I$取遍递增重指标时全体初等交错张量$\varepsilon^I$构成了$\bigwedge^k(V^*)$的一组基.于是特别的它的维数是:
	$$\dim\bigwedge^k(V^*)=\left\{\begin{array}{cc}\left(\begin{array}{c}n\\k\end{array}\right)=\frac{n!}{k!(n-k)!}&k\le n\\0&k>n\end{array}\right.$$
	\begin{proof}
		
		如果$k>n$,任取$k$个$V$中向量总是线性相关的,于是交错$k$张量总是零张量.现在设$k\le n$,记重指标递增的全体初等交错张量构成的集合是$S$,我们需要证明$S$线性生成了全部交错$k$张量,并且$S$是线性无关的.
		
		验证$S$线性生成了$\bigwedge^k(V^*)$:任取交错$k$张量$\alpha$,任取重指标$I=(i_1,i_2,\cdots,i_k)$,记$\alpha_I=\alpha(E_{i_1},\cdots,E_{i_k})$,容易验证$\alpha=\sum_I\alpha_I\varepsilon^I$,但是按照$\alpha$是交错的,这里指标重复的项为零,指标不为递增的项可经一个置换转化为递增的(按照$J=I_{\sigma}$时有$\varepsilon^I=(\mathrm{sgn}\sigma)\varepsilon^J$).这说明$S$线性生成了$\alpha$.
		
		验证$S$是线性无关的:如果有$\sum_I\alpha_I\varepsilon^I=0$,其中$I$取遍不同的递增重指标,任取递增的重指标$J=(j_1,\cdots,j_k)$,把这个等式作用在$(E_{j_1},\cdots,E_{j_k})$上得到$\alpha_J=0$,这说明线性无关性.
	\end{proof}
    \item 设$V$是$n$维实线性空间,设$\omega\in\bigwedge^n(V^*)$,如果$F$是$V$上的线性变换,对任意向量$v_1,v_2,\cdots,v_n\in V$,总有:
    $$\omega(Fv_1,Fv_2,\cdots,Fv_n)=(\det F)\omega(v_1,v_2,\cdots,v_n)$$
    \begin{proof}
    	
    	选取$V$上一组基$\{E_i\}$,它的对偶基记作$\{\varepsilon^i\}$,设$F$在这组基下的矩阵表示为$(F_i^j)$,再记$F_i=F(E_i)=\sum_jF_i^jE_j$.按照我们证明了$\bigwedge^n(V^*)$是一维线性空间,可记$\omega=c\varepsilon^{(1,2,\cdots,n)}$.
    	
    	现在我们要证明的等式两侧都是交错$k$张量,于是我们只需验证它们在$(E_1,E_2,\cdots,E_n)$的取值相同.右侧的取值为$(\det F)c\varepsilon^{(1,2,\cdots,n)}(E_1,E_2,\cdots,E_n)=c\det F$.左侧的取值为$c\varepsilon^{(1,2,\cdots,n)}(T_1,T_2,\cdots,T_n)=c\det(\varepsilon^j(T_i))=c\det(T_i^j)$.它们相同,这就得证.
    \end{proof}
\end{enumerate}

外积和外代数.
\begin{enumerate}
	\item 设$\omega$和$\eta$分别是交错$k$张量和交错$l$张量.它们的外积定义为一个交错$k+l$张量:
	$$\omega\wedge\eta=\frac{(k+l)!}{k!l!}\mathrm{Alt}(\omega\otimes\eta)$$
	\item 引理.设$V$是$n$维实线性空间,设$\{E_i\}$是$V$的一组基,$\{\varepsilon^i\}$是对偶基,取两个重指标$I=(i_1,i_2,\cdots,i_k)$和$J=(j_1,j_2,\cdots,j_l)$,那么有$\varepsilon^I\wedge\varepsilon^J=\varepsilon^{IJ}$,其中$IJ=(i_1,\cdots,i_k,j_1,\cdots,j_l)$.
	\item 外积的基本性质,设$\omega,\omega',\eta,\eta',\xi$是有限维实线性空间$V$上的交错张量.
	\begin{itemize}
		\item 双线性.对每个$a,a'\in\mathbb{R}$有:
		$$(a\omega+a'\omega')\wedge\eta=a(\omega\wedge\eta)+a'(\omega'\wedge\eta)$$
		$$\eta\wedge(a\omega+a'\omega')=a(\eta\wedge\omega)+a'(\eta\wedge\omega')$$
		\item 结合律:$\omega\wedge(\eta\wedge\xi)=(\omega\wedge\eta)\wedge\xi$.
		\begin{proof}
			
			按照外积是双线性的,归结为证明$\omega,\eta,\xi$都是初等交错张量时候的结合律,但是我们证明过$\varepsilon^I\wedge\varepsilon^J=\varepsilon^{IJ}$,这得到结合律.
		\end{proof}
		\item 反对称性:如果$\omega$和$\eta$分别是$k$和$l$次的,那么有:
		$$\omega\wedge\eta=(-1)^{kl}\eta\wedge\omega$$
		\begin{proof}
			
			按照外积是双线性的,归结为证明$\omega=\varepsilon^I$和$\eta=\varepsilon^J$的情况.如果记$IJ$到$JI$的置换为$\tau$,那么$\mathrm{sgn}\tau=(-1)^{kl}$.于是有:
			$$\varepsilon^I\wedge\varepsilon^J=\varepsilon^{IJ}=(\mathrm{sgn}\tau)\varepsilon^{JI}=(-1)^{kl}\varepsilon^J\wedge\varepsilon^I$$
		\end{proof}
		\item 如果$\{\varepsilon^i\}$是$V^*$的一组基,设重指标$I=(i_1,i_2,\cdots,i_k)$,那么有:
		$$\varepsilon^{i_1}\wedge\cdots\wedge\omega^{i_k}=\varepsilon^I$$
		\item 对余切向量$\omega^1,\cdots,\omega^k$和切向量$v_1,\cdots,v_k$,有:
		$$\omega^1\wedge\cdots\wedge^k(v_1,v_2,\cdots,v_k)=\det\left(\omega^j(v_i)\right)$$
	\end{itemize}
    \item 外代数.设$V$是$n$维实线性空间,记$\bigwedge(V^*)=\oplus_{k\ge0}\bigwedge^k(V^*)$,这是一个$2^n$维实线性空间.外积使得这个线性空间构成一个非对称的分次代数,称为$V$上的外代数.
\end{enumerate}

内乘(interior multiplication).设$V$是有限维实线性空间,任取$v\in V$,定义$v$的内乘映射$i_v:\bigwedge^k(V^*)\to\bigwedge^{k-1}(V^*)$为:如果$k=0$则定义$i_v=0$,如果$k\ge1$则定义:$$i_v\omega(v_1,v_2,\cdots,v_{k-1})=\omega(v,v_1,\cdots,v_{k-1})$$
\begin{enumerate}
	\item $i_v$是一个线性映射,并且满足$i_v\circ i_v=0$.
	\item 对余切向量$\omega^1,\cdots\omega^k$,有:
	$$i_v\omega(\omega^1\wedge\cdots\wedge\omega^k)=\sum_{i=1}^k(-1)^{i-1}\omega^i(v)\omega^1\wedge\cdots\wedge\hat{\omega^i}\wedge\cdots\wedge\omega^k$$
	\item 如果$\omega$是交错$k$张量,$\eta$是交错$l$张量,那么有:
	$$i_v(\omega\wedge\eta)=(i_v\omega)\wedge\eta+(-1)^k\omega\wedge(i_v\eta)$$
\end{enumerate}

张量丛和张量场.设$M$是光滑流形.
\begin{itemize}
	\item $M$上的共变$k$张量丛定义为:$$\mathrm{T}^k\mathrm{T}^*M=\coprod_{p\in M}\mathrm{T}^k(\mathrm{T}_p^*M)$$
	\item $M$上的逆变$k$张量丛定义为:$$\mathrm{T}^k\mathrm{T}M=\coprod_{p\in M}\mathrm{T}^k(\mathrm{T}_pM)$$
	\item $M$上的$(k,l)$型张量丛定义为:$$\mathrm{T}^{(k,l)}\mathrm{T}M=\coprod_{p\in M}\mathrm{T}^{(k,l)}(\mathrm{T}_pM)$$
	\item 共变$k$张量场,逆变$k$张量场,$(k,l)$型张量场分别指上述三种张量丛作为向量丛的光滑截面.例如逆变1张量场就是切向量场,共变1张量场就是微分1形式,而0张量就是实数,于是(光滑)0张量场就是光滑函数.
\end{itemize}
\begin{enumerate}
	\item 按照定义有:
	$$\mathrm{T}^{(0,0)}\mathrm{T}M=\mathrm{T}^0\mathrm{T}M=\mathrm{T}^0\mathrm{T}^*M=M\times\mathbb{R}$$
	\item 选取$M$上的一个坐标卡$(U,(x^i))$,那么这三种张量场局部上有如下表示,并且张量场在$U$上光滑当且仅当这里全部分量函数都是光滑的.
	$$\alpha=\left\{\begin{array}{cc}\sum_{i_1,\cdots,i_k}a_{i_1,\cdots,i_k}\mathrm{d}x^1\otimes\cdots\otimes\mathrm{d}x^{i_k}&\alpha\text{是共变$k$张量场}\\\sum_{j_1,\cdots,j_k}a^{j_1,\cdots,j_k}\frac{\partial}{\partial x^{j_1}}\otimes\cdots\otimes\frac{\partial}{\partial x^{j_k}}&\alpha\text{是逆变$k$张量场}\\\sum_{i_1,\cdots,i_k,j_1,\cdots,j_l}a_{i_1,\cdots,i_k}^{j_1,\cdots,j_k}\frac{\partial}{\partial x^{j_1}}\otimes\cdots\otimes\frac{\partial}{\partial x^{j_k}}\otimes\mathrm{d}x^{i_1}\otimes\cdots\otimes\mathrm{d}x^{j_l}&\alpha\text{是$(k,l)$型张量场}\end{array}\right.$$
	\item 张量场是光滑的等价于取$M$上任意相应切向量场和余切向量场带入张量场后得到的是$M$上的光滑函数(张量场是给每个点赋予一个混合多重线性函数,带入切向量场和余切向量场就得到一个实数,于是这是$M$上的函数),也等价于对任意非空开集$U\subset M$,取其上任意相应切向量场和余切向量场带入张量场后得到的是$U$上的光滑函数.
	\item 如果$\alpha,\beta$分别是$M$上的共变$k$张量场和共变$l$张量场,如果$f$是$M$上光滑函数,那么$(f\alpha)$是$M$上共变$k$张量场,它的分量函数就是$fa_{i_1,\cdots,i_k}$;$\alpha\otimes\beta$是$M$上共变$k+l$张量场,它的分量函数是$a_{i_1,\cdots,i_k}b_{i_{k+1},\cdots,i_{k+l}}$.
	\item $M$上的对称张量场是指一个共变对称张量.
\end{enumerate}

共变张量场的回拉.设$F:M\to N$是光滑映射,设$\alpha$是$N$上的共变$k$张量场,它的回拉$F^*\alpha$定义为:
$$(F^*\alpha)_p(v_1,v_2,\cdots,v_k)=\alpha_{F(p)}(\mathrm{d}F_p(v_1),\mathrm{d}F_p(v_2),\cdots,\mathrm{d}F_p(v_k))$$
\begin{enumerate}
	\item 如果$\alpha$是光滑张量场,那么回拉$F^*\alpha$总是光滑的.
	\item 一些基本性质,如果$F:M\to N$是光滑映射,$G:N\to P$是光滑映射,$\alpha,\beta$是$N$上的共变张量场,$f$是$N$上的光滑函数,那么:
	\begin{itemize}
		\item $F^*(f\beta)=(f\circ F)F^*\beta$.
		\item $F^*(\alpha\otimes\beta)=F^*\alpha\otimes F^*\beta$.
		\item $F^*(\alpha+\beta)=F^*\alpha+F^*\beta$.
		\item $(G\circ F)^*\beta=F^*(G^*\beta)$.
		\item $(\mathrm{id}_N)^*\beta=\beta$.
		\item $F^*(\sum_{i_1,\cdots,i_k}\mathrm{d}y^{i_1}\otimes\cdots\otimes\mathrm{d}y^{i_k})=\sum_{i_1,\cdots,i_k}(B_{i_1,\cdots,i_k}\circ F)\mathrm{d}(y^{i_1}\circ F)\otimes\cdots\otimes\mathrm{d}(y^{i_k}\circ F)$.
	\end{itemize}
    \item 对称张量场的回拉仍然是对称张量场.另外对称积和回拉可交换,换句话讲如果$F:M\to N$是光滑映射,如果$\alpha,\beta$是$N$上的两个对称张量场,那么$F^*(\alpha\beta)=(F^*\alpha)(F^*\beta)$.
\end{enumerate}
\subsection{微分形式}

微分$k$形式.考虑交错张量构成的丛$\bigwedge^k\mathrm{T}^*M=\coprod_{p\in M}\bigwedge^k(\mathrm{T}_p^*M)$,这是$\mathrm{T}^k\mathrm{T}^*M$的秩为$\left(\begin{array}{c}n\\k\end{array}\right)$的子丛.这个丛的光滑截面称为微分$k$形式,或者简称$k$形式.其中$k$称为微分形式的次数.$M$上全体微分$k$形式构成的实线性空间记作$\Omega^k(M)$.
\begin{enumerate}
	\item 记$\Omega(M)=\oplus_{k\ge0}\Omega^k(M)$,两个微分形式的外积约定为$(\omega\wedge\eta)_p=\omega_p\wedge\eta_p$,这个外积使得$\Omega(M)$成为一个反对称分次代数.
	\item 在每个光滑坐标卡下,一个粗糙$k$形式$\omega$可以表示为$\sum_I\omega_I\mathrm{d}x^{i_1}\wedge\cdots\wedge\mathrm{d}x^{i_k}=\sum_I\omega_I\mathrm{d}x^I$,这里$I$取遍递增的复指标.我们证明过粗糙$k$形式是光滑的当且仅当这里分量函数$\omega_I=\omega(\partial x^{i_1},\cdots,\partial x^{i_k})$都是光滑函数.
	\item 给定光滑映射$F:M\to N$,设$\omega$是$N$上的微分形式,那么回拉$F^*\omega$定义为$M$上的一个微分形式:$$(F^*\omega)_p(v_1,v_2,\cdots,v_k)=\omega_{F(p)}(\mathrm{d}F_p(v_1),\mathrm{d}F_p(v_2),\cdots,\mathrm{d}F_p(v_k))$$
	\begin{itemize}
		\item $F^*:\Omega^k(N)\to\Omega^k(M)$是线性映射.
		\item 回拉和外积可交换:$F^*(\omega\wedge\eta)=F^*\omega\wedge F^*\eta$.
		\item 在每个光滑坐标卡下有:
		$$F^*\left(\sum_I\omega_I\mathrm{d}y^{i_1}\wedge\cdots\wedge\mathrm{d}y^{i_k}\right)=\sum_I(\omega_I\circ F)\mathrm{d}(y^{i_1}\circ F)\wedge\cdots\wedge\mathrm{d}(y^{i_k}\circ F)$$
	\end{itemize}
    \item 设$F:M\to N$是两个$n$维光滑流形之间的光滑映射,如果$(x^i)$和$(y^i)$是两个光滑坐标邻域$U\subset M$和$V\subset N$上的坐标函数,设$u$是$V$上的实函数,那么如下等式在$U\cap F^{-1}(V)$上成立,其中$\mathrm{D}F$表示$F$在这些坐标卡下的Jacobian矩阵.
    $$F^*(u\mathrm{d}y^1\wedge\cdots\wedge\mathrm{d}y^n)=(u\circ F)(\det\mathrm{D}F)\mathrm{d}x^1\wedge\cdots\wedge\mathrm{d}x^n$$
    \begin{proof}
    	
    	这是因为我们证明过$F^*(u\omega)=(u\circ F)F^*\omega$以及回拉和外积可交换.
    \end{proof}
    \item 如果$(U,(x^i))$和$(V,y^i)$是$M$上两个相交的光滑坐标卡,那么在$U\cap V$上有坐标变换公式:
    $$\mathrm{d}y^1\wedge\cdots\wedge\mathrm{d}y^n=\det\left(\frac{\partial y^j}{\partial x^i}\right)\mathrm{d}x^1\wedge\cdots\wedge\mathrm{d}x^n$$
    \item 内乘映射.取$M$上的切向量场$X$和$M$上的微分$k$形式$\omega$,定义一个$k-1$形式为$(i_X\omega)_p=i_{X_p}\omega_p$,这是光滑的,于是$i_X$是$\Omega^k(M)\to\Omega^{k-1}(M)$的$\mathrm{C}^{\infty}(M)$模同态.我们解释过截面之间的同态对应于丛同态,于是$i_X$对应于一个丛同态$\bigwedge^k\mathrm{T}^*M\to\bigwedge^{k-1}\mathrm{T}^*M$.
\end{enumerate}

欧氏空间上的外导数(微分).给定欧氏空间上的微分$k$形式$\omega=\sum_J\omega_J\mathrm{d}x^J$,其中$J$取遍递增的重指标,它的外导数$\mathrm{d}\omega$定义为一个微分$k+1$形式:
$$\mathrm{d}(\sum_J\omega_J\mathrm{d}x^J)=\sum_J\mathrm{d}\omega_J\wedge\mathrm{d}x^J=\sum_J\sum_i\frac{\partial\omega_J}{\partial x^i}\mathrm{d}x^i\wedge\mathrm{d}x^{j_1}\wedge\cdots\wedge\mathrm{d}x^{j_k}$$
\begin{enumerate}
	\item 特别的,如果$\omega=\sum_j\omega_j\mathrm{d}x^j$是1形式,那么它的外导数为:
	\begin{align*}
	\mathrm{d}(\sum_j\omega_j\mathrm{d}x^j)&=\sum_{i,j}\frac{\partial\omega_j}{\partial x^i}\mathrm{d}x^i\wedge\mathrm{d}x^j\\&=\sum_{i<j}\frac{\partial\omega_j}{\partial x^i}\mathrm{d}x^i\wedge\mathrm{d}x^j+\sum_{i>j}\frac{\partial\omega_j}{\partial x^i}\mathrm{d}x^i\wedge\mathrm{d}x^j\\&=\sum_{i<j}\left(\frac{\partial\omega_j}{\partial x^i}-\frac{\partial\omega_i}{\partial x^j}\right)\mathrm{d}x^i\wedge\mathrm{d}x^j
	\end{align*}
	\item 特别的,我们知道0形式就是一个光滑函数,并且0形式和一个$k$形式的外积就是$f\wedge\omega=f\omega$.于是0形式上的外导数吻合于我们微分的定义,于是外导数也称为微分或者微分算子.
	$$\mathrm{d}f=\sum_{i=1}^n\frac{\partial f}{\partial x^i}\mathrm{d}x^i$$
	\item $\mathrm{d}$是$\Omega(\mathbb{R}^n)$上的$\mathbb{R}$线性映射.
	\item 如果$\omega$和$\eta$分别是定义在开集$U\subset\mathbb{R}^n$上的$k$形式和$l$形式,那么它们满足如下反导数条件:
	$$\mathrm{d}(\omega\wedge\eta)=(\mathrm{d}\omega)\wedge\eta+(-1)^k\omega\wedge\mathrm{d}\eta$$
	\begin{proof}
		
		等式两边都是关于$\omega$和$\eta$线性的,于是归结为证明$\omega=u\mathrm{d}x^I$和$\eta=v\mathrm{d}x^J$的情况:
		\begin{align*}
		\mathrm{d}(\omega\wedge\eta)&=\mathrm{d}\left(uv\mathrm{d}x^I\wedge\mathrm{d}x^J\right)\\&=(v\mathrm{d}u+u\mathrm{d}v)\wedge\mathrm{d}x^I\wedge\mathrm{d}x^J\\&=(\mathrm{d}u\wedge\mathrm{d}x^I)\wedge(v\mathrm{d}x^J)+(-1)^k(u\mathrm{d}x^I)\wedge(\mathrm{d}v\wedge\mathrm{d}x^J)\\&=\mathrm{d}\omega\wedge\eta+(-1)^k\omega\wedge\mathrm{d}\eta
		\end{align*}
	\end{proof}
    \item $\mathrm{d}\circ\mathrm{d}=0$.
    \begin{proof}
    	
    	首先如果$u$是一个零形式,即一个光滑函数:
    	\begin{align*}
    	\mathrm{d}(\mathrm{d}u)&=\mathrm{d}\left(\sum_j\frac{\partial u}{\partial x^j}\mathrm{d}x^j\right)=\sum_{i,j}\frac{\partial^2u}{\partial x^i\partial x^j}\mathrm{d}x^i\wedge\mathrm{d}x^j\\&=\sum_{i<j}\left(\frac{\partial^2u}{\partial x^i\partial x^j}-\frac{\partial^2u}{\partial x^j\partial x^i}\right)\mathrm{d}x^i\wedge\mathrm{d}x^j=0
    	\end{align*}
    	
    	再设$\omega$是一个$k$形式,结合反导数条件得到:
    	\begin{align*}
    	\mathrm{d}(\mathrm{d}\omega)&=\mathrm{d}\left(\sum_J\mathrm{d}\omega_J\wedge\mathrm{d}x^{j_1}\wedge\cdots\wedge\mathrm{d}x^{j_k}\right)\\&=\sum_J\mathrm{d}(\mathrm{d}\omega_J)\wedge\mathrm{d}x^{j_1}\wedge\cdots\wedge\mathrm{d}x^{j_k}+\sum_J\mathrm{d}\omega_J\wedge\mathrm{d}(\mathrm{d}x^J)=0
    	\end{align*}
    \end{proof}
    \item 微分和回拉可交换:如果$F:U\to V$是某些欧氏空间的开子集之间的光滑映射,设$\omega$是$V$上的微分形式,那么有:
    $$F^*(\mathrm{d}\omega)=\mathrm{d}(F^*\omega)$$
    \begin{proof}
    	
    	按照等式两边的线性,不妨设$\omega=u\mathrm{d}x^{i_1}\wedge\cdots\wedge\mathrm{d}x^{i_k}$,那么有:
    	\begin{align*}
    	F^*\left(\mathrm{d}(u\mathrm{d}x^{i_1}\wedge\cdots\wedge\mathrm{d}x^{i_k})\right)&=F^*\left(\mathrm{d}u\wedge\mathrm{d}x^{i_1}\wedge\cdots\wedge\mathrm{d}x^{i_k}\right)\\&=\mathrm{d}(u\circ F)\wedge\mathrm{d}(x^{i_1}\circ F)\wedge\cdots\wedge\mathrm{d}(x^{i_k}\circ F)
    	\end{align*}
    	\begin{align*}
    	\mathrm{d}\left(F^*(u\mathrm{d}x^{i_1}\wedge\cdots\wedge\mathrm{d}x^{i_k})\right)&=\mathrm{d}\left((u\circ F)\mathrm{d}(x^{i_1}\circ F)\wedge\cdots\wedge\mathrm{d}(x^{i_k}\circ F)\right)\\&=\mathrm{d}(u\circ F)\wedge\mathrm{d}(x^{i_1}\circ F)\wedge\cdots\wedge\mathrm{d}(x^{i_k}\circ F)
    	\end{align*}
    \end{proof}
\end{enumerate}

流形上的外导数(微分).给定光滑流形$M$上的微分$k$形式$\omega$,定义$\mathrm{d}\omega$为,在每个光滑坐标卡$(U,\varphi)$上,都有$\mathrm{d}\omega=\varphi^*\mathrm{d}((\varphi^{-1})^*\omega)$.换句话讲如果记$\varphi=(x^i)$,记$\omega=\sum_J\omega_J\mathrm{d}x^J$,其中$J$取遍递增的重指标,都有:
$$\mathrm{d}\omega=\sum_J\mathrm{d}\omega_J\wedge\mathrm{d}x^J=\sum_J\sum_i\frac{\partial\omega_J}{\partial x^i}\mathrm{d}x^i\wedge\mathrm{d}x^{j_1}\wedge\cdots\wedge\mathrm{d}x^{j_k}$$
\begin{enumerate}
	\item 验证这个定义良性,换句话讲不会出现选取不同的光滑坐标卡定义出不同$\mathrm{d}\omega$的情况:如果$(V,\psi)$是另一个光滑坐标卡,那么$\varphi\circ\psi^{-1}$是欧氏空间中某些开集的微分同胚.按照我们证明的欧氏空间上微分和回拉可交换,得到:
	$$(\varphi\circ\psi^{-1})^*\mathrm{d}((\varphi^{-1})^*\omega)=\mathrm{d}((\varphi\circ\psi^{-1})^*(\varphi^{-1})^*\omega)=\mathrm{d}((\varphi^{-1}\circ\varphi\circ\psi^{-1})^*\omega)=\mathrm{d}((\psi^{-1})^*\omega)$$
	
	于是得到$\varphi^*\mathrm{d}((\varphi^{-1})^*\omega)=\psi^*\mathrm{d}((\psi^{-1})^*\omega)$.
	\item 于是按照欧氏空间上微分的性质,光滑流形上的微分满足如下四条,这里我们断言我们构造的微分映射是唯一的满足如下四件事的映射.
	\begin{itemize}
		\item $\mathrm{d}:\Omega^k(M)\to\Omega^{k+1}(M)$对每个自然数$k$都是一个$\mathbb{R}$线性映射.
		\item 如果$\omega$和$\eta$分别是次数为$k,l$的微分形式,那么有反导数条件:
		$$\mathrm{d}(\omega\wedge\eta)=(\mathrm{d}\omega)\wedge\eta+(-1)^k\omega\wedge\mathrm{d}\eta$$
		\item $\mathrm{d}\circ\mathrm{d}=0$.
		\item 对每个光滑函数$f$(也即微分0形式),有$\mathrm{d}f$是$f$的微分,即$\mathrm{d}f(X)=Xf$.
	\end{itemize}
    \begin{proof}
    	
    	设$d$是满足这四件事的算子,我们先证明如果两个$k$形式在开集$U$上满足$\omega_1=\omega_2$,那么有$\mathrm{d}\omega_1=\mathrm{d}\omega_2$:任取$p\in U$,记$\eta=\omega_1-\omega_2$,取$\{p\}\subset U$的光滑碰撞函数$\psi$.那么$\psi\eta$在整个$M$上恒为零,于是条件导致$0=\mathrm{d}(\psi\eta)=\mathrm{d}\psi\wedge\eta+\psi\mathrm{d}\eta$.把这个等式作用在点$p$上,按照$\psi(p)=1$和$\mathrm{d}\psi_p=0$,得到$\mathrm{d}\eta_p=0$,也即$\forall p\in U$有$\mathrm{d}\omega_1\mid_p=\mathrm{d}\omega_2\mid_p$.
    	
    	现在任取$\omega\in\Omega^k(M)$,任取$M$上的光滑坐标卡$(U,\varphi)$,那么$\omega$有坐标表示$\sum_I\omega_I\mathrm{d}x^I$,这里$I$取遍递增的重指标.任取$p\in U$,按照光滑碰撞函数的存在性,我们可以构造$M$上的$\widetilde{\omega_I}$和$\widetilde{x}^i$,使得它们在$p$的附近和$\omega_I$与$x^i$一致.那么按照$\mathrm{d}$的这些条件以及上一段,说明在点$p$有$\mathrm{d}(\sum_J\omega_J\mathrm{d}x^J)=\sum_J\mathrm{d}\omega_J\wedge\mathrm{d}x^J$,这说明了唯一性.
    \end{proof}
    \item 上一条的证明说明如果两个微分形式$\omega_1,\omega_2$在光滑流形$M$的开子集$U$上相同,那么在$U$上就有$\mathrm{d}\omega_1=\mathrm{d}\omega_2$.
    \item 关于反导数条件.给定分次代数$A=\oplus_{n\ge0}A^n$,它自身的一个线性算子$T$称为反导数,如果对$x\in A^k,y\in A^l$总有$T(xy)=(Tx)y+(-1)^kx(Ty)$.于是我们之前证明的结论可以描述为:我们熟知的光滑函数上的微分映射可以唯一的延拓为一个次数$+1$的,平方为零的$\Omega(M)$上的反导数.
    \item 回拉和微分可交换(回拉是德拉姆复形之间的链映射).如果$F:M\to N$是光滑流形之间的光滑映射,对每个$N$上的$k$形式$\omega$,总有:
    $$F^*(\mathrm{d}\omega)=\mathrm{d}(F^*\omega)$$
    \begin{proof}
    	
    	任取$M,N$上的光滑坐标卡$(U,\varphi)$和$(V,\psi)$,我们已经证明过欧氏空间上的回拉和微分可交换,于是这里$\psi\circ F\circ\varphi^{-1}$的回拉和微分可交换,于是:
    	\begin{align*}
    	F^*(\mathrm{d}\omega)&=F^*\psi^*\mathrm{d}((\psi^{-1})^*\omega)\\&=\varphi^*\circ(\psi\circ F\circ\varphi^{-1})^*\mathrm{d}((\psi^{-1})^*\omega)\\&=\varphi^*\mathrm{d}((\psi\circ F\circ\varphi^{-1})^*(\psi^{-1})^*\omega)\\&=\varphi^*\mathrm{d}((\varphi^{-1})^*F^*\omega)\\&=\mathrm{d}(F^*\omega)
    	\end{align*}
    \end{proof}
\end{enumerate}

外导数在向量场下的取值公式.给定一个$k$形式$\omega$,给定$k$个向量场$X_1,X_2,\cdots,X_k$,我们解释过$\omega(X_1,X_2,\cdots,X_k)$可以视为一个光滑函数,即:
$$\omega(X_1,\cdots,X_k)(p)=\omega_p(X_1(p),\cdots,X_k(p))$$
\begin{enumerate}
	\item 1形式的情况.如果$\omega$是微分1形式,$X,Y$是两个向量场,那么有:
	$$\mathrm{d}\omega(X,Y)=X(\omega(Y))-Y(\omega(X))-\omega([X,Y])$$
	\begin{proof}
		
		按照等式两边都是线性的,不妨设$\omega=u\mathrm{d}v$.那么有:
		\begin{align*}
		\mathrm{LHS}&=\mathrm{d}(u\mathrm{d}v)(X,Y)\\&=\mathrm{du}\wedge\mathrm{d}v(X,Y)\\&=\mathrm{d}u(X)\mathrm{d}v(Y)-\mathrm{d}v(X)\mathrm{d}u(Y)\\&=X(u)Y(v)-X(v)Y(u)
		\end{align*}
		\begin{align*}
		\mathrm{RHS}&=X(u\mathrm{d}v(Y))-Y(u\mathrm{d}v(X))-u\mathrm{d}v([X,Y])\\&=X(uY(v))-Y(uX(v))-u[X,Y](v)\\&=(X(u)Y(v)+uX(Y(v)))-(Y(u)X(v)+uY(X(v)))-u(XY-YX)(v)\\&=X(u)Y(v)-X(v)Y(u)
		\end{align*}
	\end{proof}
    \item 一般情况.如果$\omega\in\Omega^k(M)$,对$M$上的$k+1$个向量场$X_1,\cdots,X_{k+1}$,就有:
    $$\mathrm{d}\omega(X_1,X_2,\cdots,X_{k+1})=\sum_{1\le i\le k+1}(-1)^{i-1}X_i\left(\omega(X_1,\cdots,\hat{X_i},\cdots,X_{k+1})\right)$$
    $$+\sum_{1\le i<j\le k+1}(-1)^{i+j}\omega\left([X_i,X_j],X_1,\cdots,\hat{X_i},\cdots,\hat{X_j},\cdots,X_{k+1}\right)$$
\end{enumerate}

\newpage
\section{积分}
\subsection{定向}

线性空间上的定向.粗略的讲,流形上的定向就相当于在曲面上约定什么是顺时针和逆时针方向,或者说当我们站在曲面上时怎么约定右手边和左手边.用数学语言描述,这是指在每个点局部上我们怎么约定切空间的基的顺序.据此我们抽象出如下概念:一个有限维数$n\ge1$的实线性空间上的两组有序基$\{E_i\}$,$\{\widetilde{E_i}\}$称为定向相同的,如果它们的过渡矩阵具有正的行列式,否则称它们定向不同.
\begin{enumerate}
	\item 定向相同是$V$的有序基上的一个等价关系.取定一个有序基作为基准,和它定向相同的有序基称为正定向的,否则称为负定向的,容易验证全体正定向和全体负定向分别构成这个等价关系的等价类,这个等价关系仅有这两个等价类.
	\item 线性空间$V$上有序基在定向相同下的等价类称为它的定向.选取一个定向的线性空间称为定向线性空间.有序基$\{E_i\}$所在的等价类我们记作$[E_i]$.如果$V$是定向线性空间,我们约定定义中所选取的定向称为正定向,另一个等价类称为负定向,正负定向有时也会分别记作$+1$和$-1$.
	\item 例如$\mathbb{R}^n$自身存在一个不依赖于基的选取的定向,即$[e_1,e_2,\cdots,e_n]$,其中$e_i$是第$i$分量取1其余取0的向量.这个定向称为欧氏空间上的标准定向.
	\item 定向和交错张量.设$V$是维数$n\ge1$的有限维实线性空间,每个非零元$\omega\in\bigwedge^n(V^*)$确定了一个定向:全体满足$\omega(E_1,E_2,\cdots,E_n)>0$的有序基构成的等价类.另外这个定义可以形式的定义在$n=0$的情况:此时$\omega$就是一个实数,如果它$>0$和$<0$对应仅有的两个定向.
	\begin{proof}
		
		这件事很容易,设$n\ge1$,如果$\{\widetilde{E_i}\}$和$\{E_i\}$是两个有序基,设过渡矩阵为$B$,那么有:
		$$\omega(\widetilde{E_1},\cdots,\widetilde{E_n})=(\det B)\omega(E_1,\cdots,E_n)$$
	\end{proof}
    \item 当$V$是定向线性空间时,一个$n$余切向量$\omega$如果定义了相同的定向,就称它是正定向$n$余切向量,否则称为负定向余切向量.
\end{enumerate}

流形上的定向的多种定义.
\begin{enumerate}
	\item 用线性空间上的定向定义.
	\begin{itemize}
		\item 设$M$是$n$维光滑流形,它的一个定向$\mathscr{O}$是指对每个点$p\in M$的切空间$\mathrm{T}_pM$上赋予一个定向$\mathscr{O}_p$,使得$M$上每个点都被某个定向光滑局部标架覆盖.这里定向光滑坐标卡是指开集$U$上的一个光滑局部标架$\{E_1,E_2,\cdots,E_n\}$,使得每个$p\in U$都有$\{E_1\mid_p,E_2\mid_p,\cdots,E_n\mid_p\}$都是切空间$\mathrm{T}_pM$的正定向有序基.一个赋予定向的光滑流形就称为定向流形.
		\item 零维情况.在零维情况下局部定向标架的条件是没有用处的,此时定向的选取就是对每个点选取$\pm1$中的某个定向,这时候任意零维光滑流形都是定向的.
		\item 一个引理.如果$M$是定向$n$维光滑流形,它的连通开集上的局部标架要么是正定向的要么是负定向的.
	\end{itemize}
    \item 用$n$形式定义.
    \begin{itemize}
    	\item 设$M$是一个$n$维光滑流形,$M$上的一个处处非零的$n$形式$\omega$唯一的决定了$M$上的一个定向.反过来给定$n$维定向光滑流形$M$,那么存在一个处处非零的$n$形式$\omega$处处是正定向的.
    	\begin{proof}
    		
    		给定处处非零的$n$形式$\omega$,它在点$p\in M$出的定向约定为使得$\omega(E_1,E_2,\cdots,E_n)>0$的有序基$\{E_i\}$构成的线性空间的定向.我们需要验证局部定向标架的存在性.首先$n=0$时局部定向标架存在的条件是平凡的.下设$n\ge1$,任取$p\in M$,选取它的连通开邻域$U$上的局部标架$\{E_i\}$.记对偶标架为$\{\varepsilon^i\}$,那么$\omega$在这个对偶标架下局部可以表示为$\omega=f\varepsilon^1\wedge\cdots\wedge\varepsilon^n$.按照$\omega$是处处非零的,这里$f$是$U$上处处非零的光滑函数.导致$\omega(E_1,E_2,\cdots,E_n)=f\not=0$在$U$上处处成立.但是$U$是连通的,所以这个取值要么在$U$上恒正要么在$U$上恒负.所以这个局部标架要么是正定向的要么是负定向的,如果它是负定向的就把$E_1$换为$-E_1$.这就得到每个点都存在开邻域上有定向局部标架.
    		
    		如果$M$是$n$维定向光滑流形,【单位分解】
    	\end{proof}
        \item 按照这个性质,我们把$n$维光滑流形$M$上处处非零的$n$形式称为定向形式.如果$M$是$n$维定向流形,诱导出这个定向的定向形式称为正定向形式.那么如果$\widetilde{\omega}$和$\omega$都是定向流形$M$上的正定向形式,就存在严格恒为正的光滑函数$f$使得$\widetilde{\omega}=f\omega$.
    \end{itemize}
    \item 用光滑图册定义.
    \begin{itemize}
    	\item 设$M$是定向光滑流形,它的一个光滑坐标卡$(U,(x^i))$称为正定向的,如果$\{\partial/\partial x^i\}$是一个定向局部标架.光滑流形$M$上的一个光滑图册$\{(U_i,\alpha_i)\}$称为定向光滑图册,如果对任意指标$i,j$,都有$\varphi_j\circ\varphi_i^{-1}$在$\varphi_i(U_i\cap U_j)$上的Jacobian矩阵的行列式总具有正行列式.
    	\item 设$M$是正维数的光滑流形,给定定向光滑图册,那么唯一存在$M$上的定向使得该图册中的每个光滑坐标卡对应的局部标架是正定向的.反过来如果$M$是定向光滑流形,那么所有正定向光滑坐标卡构成一个定向光滑图册.
    	\begin{proof}
    		
    		先设$M$存在定向光滑图册,其中每个光滑坐标卡定义了它定义域中每个点上的定向.当两个坐标卡相交时,过渡映射的Jacobian矩阵的行列式为正,于是该图册中不同的坐标卡定义了相同的定向,而其中每个坐标卡对应了一个定向局部标架,这得到$M$上唯一的定向.
    		
    		再设$M$是定向光滑流形,每个点附近存在光滑坐标卡,如果这个光滑坐标卡不是正定向的,我们就把坐标映射的第一个分量取负,这说明正定向光滑坐标卡覆盖了整个$M$,于是它们构成一个图册.现在任取两个正定向光滑坐标卡,那么它们的交上的坐标映射的Jacobian矩阵具有正的行列式,导致这个光滑图册是定向光滑图册.
    	\end{proof}
        \item 一个连通流形上恰好存在两个定向.特别的如果连通流形上两个定向在一点相同,那么它们是相同的定向.
        \begin{proof}
        	
        	设$\alpha$和$\beta$是连通光滑流形$M$上的两个定向.对每个点$p\in M$,有$\alpha_p$和$\beta_p$是$p$处切空间的定向,定义函数$f:M\to\{\pm1\}$使得:
        	$$f(p)=\left\{\begin{array}{cc}1&\alpha_p=\beta_p\\-1&\alpha_p=-\beta_p\end{array}\right.$$
        	
        	正定向光滑图册的存在性保证这是一个连续映射,其中$\{\pm1\}$赋予的是离散拓扑.但是$M$上的连通性导致$f^{-1}(1)$和$f^{-1}(-1)$中至少一个是全集,也即要么$\alpha=\beta$要么$\alpha=-\beta$.这说明连通流形上恰好存在两个定向.并且这两个定向处处互为负定向.
        \end{proof}
    \end{itemize}
\end{enumerate}

一些构造和性质.
\begin{enumerate}
	\item 定向的积.如果$M_1,M_2,\cdots,M_k$都是定向流形,任取$M_i$上的正定向形式$\omega_i$,那么$\pi_1^*\omega_1\wedge\cdots\wedge\pi_k^*\omega_k$是$M_1\times M_2\times\cdots\times M_k$上的定向,并且这不依赖于正定向形式$\omega_i$的选取.这个定向称为积空间上的乘积定向.
	\item 设$M$是定向流形,设$D\subset M$是余维数0的浸入子流形,那么$M$上的定向限制在$D$上是$D$上的定向.如果$\omega$是$M$上的一个定向形式,那么$i^*\omega$是$D$上的定向形式,这里$i:D\to M$是包含映射.
\end{enumerate}

保定向映射.如果$M,N$是定向流形,一个局部微分同胚$F:M\to N$称为保定向映射,如果对每个点$p\in M$,同构$\mathrm{d}F_p$都是保定向的线性同构.
\begin{enumerate}
	\item 一个局部微分同胚$F:M\to N$是保定向映射等价于如下任一条成立:
	\begin{itemize}
		\item 对任意的$M,N$的定向光滑坐标卡,$F$的坐标映射的Jacobian行列式都是正的.
		\item 对$N$上的任意正定向形式$\omega$,总有$F^*\omega$是$M$上的正定向形式.
	\end{itemize}
    \item 回拉定向.如果$M,N$是流形,$F:M\to N$是局部微分同胚,如果$N$是定向流形,那么存在$M$上唯一的定向,使得$F$是保定向映射.这个$M$上的定向称为回拉定向.
\end{enumerate}

超曲面上的定向.
\begin{enumerate}
	\item 回顾.设$S\subset M$是浸入子流形,我们定义过$M$上沿$S$的向量场是一个光滑映射$N:S\to\mathrm{T}M$,使得$N_p\in\mathrm{T}_pM$.
	\item 如果$M$是一个定向$n$维流形,$N$是沿$M$的浸入超曲面$S$的向量场,并且处处和$S$不相切(即处处不落在$\mathrm{T}_pS$中).那么有:
	\begin{itemize}
		\item $S$上具有唯一的定向,使得对$p\in S$,$\{E_1,E_2,\cdots,E_{n-1}\}$是$\mathrm{T}_pS$上的定向基当且仅当$\{N_p,E_1,E_2,\cdots,E_{n-1}\}$是$\mathrm{T}_pM$的定向基.
		\item 我们之前定义过内乘映射$i_N:\Omega^n(S)\to\Omega^{n-1}(S)$.如果$\omega$是$M$上的定向形式,那么$i_N(\omega)$是$S$关于上一条定向的定向形式.
	\end{itemize}
	\begin{proof}
		
		如果$\omega$是$M$上的定向形式,那么$\sigma=i_N(\omega)$是$S$上的$n-1$形式.为证明它是定向形式归结为证明它处处非零.给定$\mathrm{T}_pS$上的一组基$\{E_1,E_2,\cdots,E_{n-1}\}$,按照$N_p$不在$\mathrm{T}_pS$中,导致$\{N_p,E_1,E_2,\cdots,E_{n-1}\}$是$\mathrm{T}_pM$的一组基.按照$\omega$是处处非零的形式,得到$\sigma_p(E_1,E_2,\cdots,E_{n-1})=\omega_p(N_p,E_1,E_2,\cdots,E_{n-1})\not=0$.另外$\sigma_p(E_1,E_2,\cdots,E_{n-1})>0$当且仅当$\omega_p(N_p,E_1,\cdots,E_{n-1})>0$,说明$\sigma$诱导的定向是唯一的满足命题中条件的定向.
	\end{proof}
    \item 例子.球面$\mathbb{S}^n$是$\mathbb{R}^{n+1}$的超曲面.向量场$N=\sum_ix^i\partial/\partial x^i$是球面处处不相切的向量场,这个向量场按照我们这里的命题诱导了球面上的定向,称为球面上的标准定向.(但是不是所有超曲面都存在处处不相切的向量场).
\end{enumerate}

带边流形边界上的定向——Stokes定向.
\begin{enumerate}
	\item 如果$M$是边界非空的流形,记$p\in\partial M$,称$\mathrm{T}_p\partial M\subset\mathrm{T}_pM$中的向量为和$\partial M$相切的向量.如果$v\in\mathrm{T}_pM-\mathrm{T}_p\partial M$:如果存在$\varepsilon>0$,存在光滑曲线$\gamma:[0,\varepsilon)\to M$使得$\gamma(0)=p$和$\gamma'(0)=v$,则称$v$是内向(inward-pointing)的;如果存在$\varepsilon>0$,存在光滑曲线$(-\varepsilon,0]\to M$满足$\gamma(0)=p$和$\gamma'(0)=v$,则称$v$是外向(outward-pointing)的.
	\item 内向外向相切的等价描述.如果$M$是边界非空的$n$维流形,取点$p\in\partial M$,设$(x^i)$是点$p$附近的坐标映射,那么$\mathrm{T}_pM$中的和边界相切向量恰好是在这个坐标卡下$x^n$分量为零的切向量;外向切向量恰好是在这个坐标卡下$x^n$分量为负的切向量;内向切向量恰好是在这个坐标卡下$x^n$分量为正的切向量.于是$\mathrm{T}_pM$恰好可以分为这三种切向量的无交并.
	\item 如果$M$是边界非空的流形,那么$\partial M$是它$n-1$维(边界非空的)真嵌入子流形.
	\item 如果$M$是边界非空的流形,那么存在$M$上的整体向量场,它在$\partial M$上的限制是处处内向的;也存在$M$上的整体向量场,它在$\partial M$上的限制是处处外向的.
	\item 按照边界非空的流形$M$的边界是超曲面,并且按照上一条我们可以取处处外向的整体向量场,按照超曲面上定向的定理,就说明$\partial M$上存在诱导的定向.但是我们还要说明这个定向不依赖于处处外向的整体向量场的选取,一旦这得证,就诱导了$\partial M$上的典范定向,它称为Stokes定向.
	\begin{proof}
		
		任取$p\in\partial M$,任取$p$附近的坐标映射$(x^i)$,设$N,\widetilde{N}$是$M$上的两个沿$\partial M$的处处外向的向量场.如果记这两个向量场在$x^n$分量的光滑函数分别是$N^n$和$\widetilde{N}^n$,这两个向量场在边界处处外向说明$N^n(p)$和$\widetilde{N}^n(p)$都$<0$.现在$\{N_p,\partial/\partial x^1\mid_p,\cdots,\partial/\partial x^n\mid_p\}$和$\{\widetilde{N}_p,\partial/\partial x^1\mid_p,\cdots,\partial/\partial x^{n-1}\mid_p\}$都是$\mathrm{T}_pM$的基,并且它们的过渡矩阵的行列式为正(依赖于$N^n(p)/\widetilde{N}^n(p)>0$).这说明它们定义了$\mathrm{T}_p\partial M$上相同的定向.
	\end{proof}
    \item 例子.$\mathbb{S}^n$作为$\mathbb{R}^{n+1}$单位闭球的边界,外向向量场诱导的定向恰好是$\mathbb{S}^n$上的标准定向.
\end{enumerate}
\subsection{流形上的积分}

欧氏空间上微分形式的积分.欧氏空间上的容许集是指边界具有勒贝格测度零的有界子集.容许集实际上就是Jordan可测集,它满足上面的连续函数总是黎曼可积的.这个概念就是充当积分取域用的.
\begin{enumerate}
	\item 设$D\subset\mathbb{R}^n$是容许集,设$\omega$是$\overline{D}$上的连续或者光滑$n$形式,那么它可以具体的表示为$\omega=f\mathrm{d}x^1\wedge\cdots\wedge\mathrm{d}x^n$,其中$f:\overline{D}\to\mathbb{R}$是连续或者光滑的实值函数.我们定义$n$形式在$D$上的积分为(最后一个积分是黎曼积分):
	$$\int_D\omega=\int_Df\mathrm{d}x^1\wedge\cdots\wedge\mathrm{d}x^n=\int_Df\mathrm{d}x^1\mathrm{d}x^2\cdots\mathrm{d}x^n=\int_Df\mathrm{d}V$$
	\item 如果$U$是$\mathbb{R}^n$或者$\mathbb{H}^n$的开子集,如果$\omega$是被$U$紧支撑(即$\omega$的支集落在$U$中)的(光滑)$n$形式,我们定义$\int_U\omega=\int_D\omega'$,其中$D$是$\mathbb{R}^n$或者$\mathbb{H}^n$的包含$\mathrm{Supp}\omega$的容许集,而$\omega'$是$\omega$零延拓至$D$上的$n$形式.这个定义不依赖于满足条件的容许集$D$的选取.
	\item 积分在微分同胚下不变性:如果$D,E$是$\mathbb{R}^n$或者$\mathbb{H}^n$的容许集,设$G:\overline{D}\to\overline{E}$是光滑映射,它限制在$D$上是到$E$的微分同胚,设$\omega$是$\overline{E}$上的$n$形式,那么:
	$$\int_DG^*\omega=\left\{\begin{array}{cc}\int_E\omega& G\text{是保定向的微分同胚}\\-\int_E\omega&G\text{是反向保定向的微分同胚}\end{array}\right.$$
	\begin{proof}
		
		以$G$是保定向的微分同胚为例,记$E$上的标准坐标函数为$\{y^1,y^2,\cdots,y^n\}$,记$D$上的标准坐标函数为$\{x^1,x^2,\cdots,x^n\}$,可记$\omega=f\mathrm{d}y^1\wedge\cdots\wedge\mathrm{d}y^n$,按照黎曼积分上换元积分公式以及$n$形式的回拉公式,得到:
		\begin{align*}
		\int_E\omega&=\int_Ef\mathrm{d}V=\int_D(f\circ G)|\det DG|\mathrm{d}V=\int_D(f\circ G)(\det DG)\mathrm{d}V\\&=\int_D(f\circ G)(\det DG)\mathrm{d}x^1\wedge\cdots\wedge\mathrm{d}x^n=\int_DG^*\omega
		\end{align*}
	\end{proof}
    \item 如果$U,V$是$\mathbb{R}^n$或者$\mathbb{H}^n$的开子集,并且$G:U\to V$是保定向或者反定向的微分同胚,如果$\omega$是被$V$紧支撑的$n$形式,那么有如下公式,其中如果$G$是保定向的那么右侧符号取正,反定向的则符号取负.$$\int_V\omega=\pm\int_UG^*\omega$$
    \begin{proof}
    	
    	如果存在容许开集$E$满足$\mathrm{Supp}\omega\subset E\subset\overline{E}\subset V$,按照微分同胚把拓扑内点映为拓扑内点,拓扑边界点映为拓扑边界点,把零测集映为零测集,于是$D=G^{-1}(E)$就是包含$\mathrm{Supp}G^*\omega$的容许集.
    	
    	于是问题归结为证明,如果$U$是$\mathbb{R}^n$或者$\mathbb{H}^n$的开子集,$K\subset U$是紧集,那么存在容许集$D$满足$K\subset D\subset\overline{D}\subset U$:任取$p\in K$,那么存在$p$为圆心的开球或者半开球,满足闭包落在$U$中.这些开球或者半开球构成了$K$的开覆盖,可取有限子覆盖$B_1,B_2,\cdots,B_m$.但是开球的边界是余维数1的,半开球的边界包含在两个余维数1的子流形中,我们证明过维数严格下降的子流形作为子集总是零测的,这说明每个$B_i$都是容许集,于是$D=B_1\cup\cdots\cup B_m$是容许集,并且满足条件.
    \end{proof}
\end{enumerate}

流形上微分形式的积分定义.
\begin{enumerate}
	\item 设$M$是定向$n$流形,设$\omega$是$M$上的被某个正定向或负定向光滑坐标卡$(U,\varphi)$紧支撑的$n$形式,定义$\omega$在$M$上的积分如下,其中如果坐标卡是正定向右侧符号就取正,负定向右侧符号就取负.这里$(\varphi^{-1})^*\omega$是被开集$\varphi(U)\subset\mathbb{R}^n$或者$\mathbb{H}^n$紧支撑的$n$形式,于是按照我们之前的讨论这是可积的.但是我们还要说明这个定义不依赖于包含$\mathrm{Supp}M$的(正或负)定向坐标卡的选取.
	$$\int_M\omega=\pm\int_{\varphi(U)}(\varphi^{-1})^*\omega$$
	\begin{proof}
		
		设$(U,\varphi)$和$(V,\psi)$是两个包含$\mathrm{Supp}\omega$的正或负定向坐标卡,不妨设它们同为正定向或者同为负定向,此时$\psi\circ\varphi^{-1}:\varphi(U\cap V)\to\psi(U\cap V)$是保定向的微分同胚(如果这两个坐标卡一个正定向一个负定向,那么$\psi\circ\varphi^{-1}$就是反定向的微分同胚,接下来的证明是完全类似的).于是有:
		\begin{align*}
		\int_{\psi(V)}(\psi^{-1})^*\omega&=\int_{\psi(U\cap V)}(\psi^{-1})^*\omega\\&=\int_{\varphi(U\cap V)}(\psi\circ\varphi^{-1})^*(\psi^{-1})^*\omega\\&=\int_{\varphi(U\cap V)}(\varphi^{-1})^*\omega
		\end{align*}
	\end{proof}
    \item 设$M$是定向$n$流形,设$\omega$是$M$上的具有紧支集的$n$形式.设$\{U_i\}$是$\mathrm{Supp}\omega$的由正定向或者负定向光滑坐标卡构成的有限开覆盖,设$\{U_i\}$的单位分解为$\{\psi_i\}$,我们定义$\omega$在$M$上的积分如下.这里右侧积分有意义是因为每个$\psi_i\omega$都被$U_i$紧支撑,于是上一条定义了该积分.但是这里我们还要说明这个定义不依赖于有限开覆盖或者单位分解的选取.
    $$\int_M\omega=\sum_i\int_M\psi_i\omega$$
    \begin{proof}
    	
    	假设$\{V_j\}$是$\mathrm{Supp}\omega$的另一个由正定向或者负定向光滑坐标卡构成的有限开覆盖,设$\{\varphi_j\}$是$\{V_j\}$对应的单位分解.对每个指标$i$有:
    	$$\int_M\psi_i\omega=\int_M(\sum_j\varphi_j)\psi_i\omega=\sum_j\int_M\varphi_j\psi_i\omega$$
    	
    	对$i$求和,得到:$$\sum_i\int_M\psi_i\omega=\sum_{i,j}\int_M\varphi_j\psi_i\omega$$
    	
    	同理有另一个等式,这说明两种积分是相同的.
    \end{proof}
    \item 零维情况.如果$M$是零维定向流形,它的一个0形式就是指一个实值函数$f$,这个零形式具有紧支集等价于讲$f$至多只在$M$的一个有限集上不为零.此时我们定义的积分如下,其中每个$f(p)$前面的符号取决于$f$在该点定向是正还是负.
    $$\int_Mf=\sum_{p\in M}\pm f(p)$$
    \item 如果$S\subset M$是$k$维的定向浸入子流形,如果$\omega$是$M$上的$k$形式,我们定义子流形积分如下,其中$i:S\to M$是包含映射.$$\int_S\omega=\int_Si^*\omega$$
\end{enumerate}

流形积分的一些性质.设$M,N$是定向$n$维流形,设$\omega$和$\eta$是$M$上的具有紧支集的$n$形式.
\begin{enumerate}
	\item 线性.如果$a,b\in\mathbb{R}$,那么有:
	$$\int_Ma\omega+b\eta=a\int_M\omega+b\int_M\eta$$
	\item 改变定向.如果$-M$表示$M$赋予相反的定向,那么有:
	$$\int_{-M}\omega=-\int_M\omega$$
	\item 正定性.如果$\omega$是正定的定向形式(即在每个点处对应的多重线性函数是正定的),那么有$\int_M\omega>0$.
	\item 微分同胚下不变.如果$F:N\to M$是保定向或者反定向的微分同胚,那么有如下等式成立,其中右侧符号取正或负取决于$F$是保定向还是反定向的.
    $$\int_M\omega=\pm\int_NF^*\omega$$
\end{enumerate}

Stokes定理.
\begin{enumerate}
	\item 定理内容和证明.如果$M$是带边的$n$维定向流形,设$\omega$是$M$上的具有紧支集的$n-1$形式,那么有如下公式,其中右侧的$\partial M$赋予的是Stokes定向,而右侧的$\omega$应该理解为$i^*\omega$,其中$i:\partial M\to M$是包含映射.
	$$\int_M\mathrm{d}\omega=\int_{\partial M}\omega$$
	\begin{proof}
		
		先设$M=\mathbb{H}^n$.按照$\omega$具有紧支集,可设正实数$R$满足$\mathrm{Supp}\omega\subset A=[-R,R]^{n-1}\times[0,R]$.在标准整体标架下可记$\omega=\sum_{i=1}^n\omega_i\mathrm{d}x^1\wedge\cdots\wedge\hat{\mathrm{d}x^i}\wedge\cdots\wedge\mathrm{d}x^n$.于是有:
		\begin{align*}
		\mathrm{d}\omega&=\sum_{i=1}^n\mathrm{d}\omega_i\wedge\mathrm{d}x^1\wedge\cdots\wedge\hat{\mathrm{d}x^i}\wedge\cdots\wedge\mathrm{d}x^n\\&=\sum_{i,j=1}^n\frac{\partial\omega_i}{\partial x^j}\mathrm{d}x^j\wedge\mathrm{d}x^1\wedge\cdots\wedge\hat{\mathrm{d}x^i}\wedge\cdots\wedge\mathrm{d}x^n\\&=\sum_{i=1}^n(-1)^{i-1}\frac{\partial\omega_i}{\partial x^i}\mathrm{d}x^1\wedge\cdots\wedge\mathrm{d}x^n
		\end{align*}
		
		于是欲证等式左侧写成黎曼积分就是:
		\begin{align*}
		\int_{\mathbb{H}^n}\mathrm{d}\omega&=\sum_{i=1}^n(-1)^{i-1}\int_A\frac{\partial\omega_i}{\partial x^i}\mathrm{d}x^1\wedge\cdots\wedge\mathrm{d}x^n\\&=\sum_{i=1}^n(-1)^{i-1}\int_0^R\int_{-R}^R\cdots\int_{-R}^R\frac{\partial\omega_i}{\partial x^i}(x)\mathrm{d}x^1\cdots\mathrm{d}x^n
		\end{align*}
		
		我们可以选取足够大的$R$使得当$x^i=\pm R,i\not=n,x^n=R$时总有$\omega=0$.于是换元积分得到:
		\begin{align*}
		\int_{\mathbb{H}^n}\mathrm{d}\omega&=\sum_{i=1}^n(-1)^{i-1}\int_0^R\int_{-R}^R\frac{\partial\omega_i}{\partial x^i}(x)\mathrm{d}x^1\cdots\mathrm{d}x^n\\&=(-1)^{n-1}\int_{[-R,R]^{n-1}}[\omega_n(x)]\mid^{x^n=R}_{x^n=0}\mathrm{d}x^1\cdots\mathrm{d}x^{n-1}\\&+\sum_{i=1}^{n-1}(-1)^{i-1}\int_0^R\int_{-R}^R\cdots\int_{-R}^R[\omega_i(x)]\mid^{x^i=R}_{x^i=-R}\mathrm{d}x^1\cdots\hat{\mathrm{d}x^i}\cdots\mathrm{d}x^n\\&=(-1)^n\int_{[-R,R]^{n-1}}\omega_n(x^1,\cdots,x^{n-1,0})\mathrm{d}x^1\cdots\mathrm{d}x^{n-1}
		\end{align*}
		
		再考虑欲证等式右侧,我们有如下事实:如果$f$是$M$上的光滑函数,$S\subset M$是一个浸入子流形,那么$\mathrm{d}(f\mid S)=i^*(\mathrm{d}f)$,于是这里由于$x^n$在$\partial\mathbb{H}^n$上恒为零,导致$\mathrm{d}x^n$的回拉恒为零.于是有如下等式.另外按照Stokes定向的定义,这里如果$n$是偶数那么$(x^1,\cdots,x^{n-1})$是$\partial\mathbb{H}^n$的正定向,如果$n$是奇数则是负定向,这说明$M=\mathbb{H}^n$的情况成立.
		\begin{align*}
		\int_{\partial\mathbb{H}^n}\omega&=\sum_i\int_{A\cap\partial\mathbb{H}^n}\omega_i(x^1,\cdots,x^{n-1},0)\mathrm{d}x^1\wedge\cdots\wedge\hat{\mathrm{d}x^i}\wedge\cdots\wedge\mathrm{d}x^n\\&=\int_{A\cap\partial\mathbb{H}^n}\omega_n(x^1,\cdots,x^{n-1},0)\mathrm{d}x^1\wedge\cdots\wedge\mathrm{d}x^{n-1}
		\end{align*}
		
		现在考虑$M=\mathbb{R}^n$的情况.此时$\omega$的支集包含在某个方体$A=[-R,R]^n$中,和上面类似的讨论可以说明等式两侧的积分都是零.
		
		现在设$M$是任意的带边流形,设$\omega$是$M$上的具有紧支集的$n-1$形式,设它被某个正定向或者负定向坐标卡$(U,\varphi)$支撑.不妨设它是正定向的,不妨设$U$是内部坐标卡,其它情况都是类似的.这里$\partial\mathbb{H}^n$赋予Stokes定向,这里$\mathrm{d}\varphi$把$\partial M$的外向向量映为$\mathbb{H}^n$的外向向量.于是$\varphi$限制在$U\cap\partial M$上是到$\varphi(U)\cap\partial\mathbb{H}^n$的保定向微分同胚,于是我们有如下等式:
		$$\int_M\mathrm{d}\omega=\int_{\mathbb{H}^n}(\varphi^{-1})^*\mathrm{d}\omega=\int_{\mathbb{H}^n}\mathrm{d}\left((\varphi^{-1})^*\omega\right)=\int_{\partial\mathbb{H}^n}(\varphi^{-1})^*\omega=\int_{\partial M}\omega$$
		
		最后设$\omega$是任意的具有紧支集的$n-1$形式,选取$\mathrm{Supp}\omega$的由正定向或者负定向坐标卡$\{U_i\}$构成的有限开覆盖,选取对应的单位分解$\{\psi_i\}$,那么有如下等式成立,这完成证明.
		\begin{align*}
		\int_{\partial M}\omega&=\sum_i\int_{\partial M}\psi_i\omega=\sum_i\int_M\mathrm{d}(\psi_i\omega)=\sum_i\int_M\mathrm{d}\psi_i\wedge\omega+\psi_i\mathrm{d}\omega\\&=\int_M\mathrm{d}(\sum_i\psi)\wedge\omega+\int_M(\sum_i\psi_i)\mathrm{d}\omega=\int_M\mathrm{d}\omega
		\end{align*}		
	\end{proof}
	\item 一些推论.
	\begin{itemize}
		\item 曲线积分基本定理.如果$M$是光滑流形,$\gamma:[a,b]\to M$是光滑嵌入,那么$S=\gamma([a,b])$是一个边界非空的1维嵌入子流形,如果$f$是$M$上的光滑函数,Stokes定理说明如下结论,此即曲线积分基本定理.
		$$\int_{\gamma}\mathrm{d}f=\int_{[a,b]}\gamma^*\mathrm{d}f=\int_S\mathrm{d}f=\int_{\partial S}f=f(\gamma(b))-f(\gamma(a))$$
		\item 恰当形式的积分.如果$M$是边界为空的紧定向光滑流形,那么$M$上的每个恰当形式的积分为零$\int_M\mathrm{d}\omega=0$.
		\item 闭形式的积分.如果$M$是带边的紧定向流形,如果$\omega$是$M$上的闭形式,那么有$\int_{\partial M}\omega=0$.
		\item 经典微积分中的Green公式.如果$D$是$\mathbb{R}^2$中的紧容许集,如果$P,Q$是$D$上的光滑实值函数,那么有:$$\int_D\left(\frac{\partial Q}{\partial x}-\frac{\partial P}{\partial y}\right)\mathrm{d}x\mathrm{d}y=\int_{\partial D}P\mathrm{d}x+Q\mathrm{d}y$$
	\end{itemize}
\end{enumerate}

\newpage
\section{德拉姆理论}
\subsection{德拉姆上同调}

德拉姆(De Rham)上同调的定义.
\begin{itemize}
	\item 设$M$是光滑流形,对非负整数$p$,我们定义过$p$形式构成的是线性空间$\Omega^p(M)$,我们定义过微分算子$\mathrm{d}:\Omega^p(M)\to\Omega^{p+1}(M)$,它满足$\mathrm{d}\circ\mathrm{d}=0$,于是这些信息构成了一个复形,称为$M$上的德拉姆复形:
	$$\xymatrix{0\ar[r]&\Omega^0(M)\ar[r]^{\mathrm{d}}&\Omega^1(M)\ar[r]^{\mathrm{d}}&\Omega^2(M)\ar[r]^{\cdots}&\cdots}$$
	\item 对每个非负整数$p$,定义闭$p$形式构成的子空间$Z^p(M)$和恰当$p$形式构成的子空间$B^p(M)$分别为:
	$$Z^p(M)=\ker\left(\mathrm{d}:\Omega^p(M)\to\Omega^{p+1}(M)\right)$$
	$$B^p(M)=\mathrm{im}\left(\mathrm{d}:\Omega^{p-1}(M)\to\Omega^p(M)\right)$$
	\item 按照$\mathrm{d}\circ\mathrm{d}=0$,每个恰当形式都是闭形式,定义$M$的$p$次德拉姆上同调群为:
	$$\mathrm{H}_{\mathrm{dR}}^p(M)=\frac{Z^p(M)}{B^p(M)}$$
	
	对$p<0$和$p>\dim M$,都有$\mathrm{H}^p_{\mathrm{dR}}(M)=0$,于是德拉姆上同调真正有信息的部分只在$0\le p\le n$.对$n$维流形$M$,我们把它的$n$次德拉姆上同调群称为最高次上同调群.另外$\mathrm{H}^p_{\mathrm{dR}}(M)=0$等价于讲所有闭$p$形式都是恰当$p$形式.另外如果流形的维数是$n$,那么$n$形式总是闭形式.
	\item 闭$p$形式$\omega$在同调群(作为商群)的像称为它的同调类,记作$[\omega]$.两个$p$形式的像相同就称它们在同一个同调类中.
\end{itemize}

函子性.我们要构造丛光滑流形到同调群列的反变函子.
\begin{enumerate}
	\item 给定光滑映射$F:M\to N$,我们之前证明过回拉和微分可交换.换句话讲对每个非负整数$p$,回拉$F^*:\Omega^p(N)\to\Omega^p(M)$诱导了德拉姆复形之间的链映射,即回拉满足如下图表交换:
	$$\xymatrix{0\ar[r]&\Omega^0(N)\ar[r]^{\mathrm{d}}\ar[d]_{F^*}&\Omega^1(N)\ar[r]^{\mathrm{d}}\ar[d]_{F^*}&\Omega^2(N)\ar[r]^{\mathrm{d}}\ar[d]_{F^*}&\cdots\\0\ar[r]&\Omega^0(M)\ar[r]^{\mathrm{d}}&\Omega^1(M)\ar[r]^{\mathrm{d}}&\Omega^2(M)\ar[r]^{\mathrm{d}}&\cdots}$$
	\item 链复形映射必然把$Z^p(N)$打到$Z^p(M)$中,必然把$B^p(N)$打到$B^p(M)$中.于是光滑映射$F$诱导了同调群列之间的同态,同样记作$F^*$.
	\item 对每个非负整数$p$,把光滑流形$M$打到$\mathrm{H}^p_{\mathrm{dR}}(M)$,把光滑映射$F$打到$F^*$是一个逆变函子.特别的微分同胚诱导了同调群之间的同构,于是微分同胚的光滑流形具有相同的同调群列.
\end{enumerate}

一些基本例子.
\begin{enumerate}
	\item 无交并的上同调.如果$\{M_j\}$是至多可数个$n$维流形,记$M=\coprod_jM_j$,那么对每个非负整数$p$,包含映射$l_j:M_j\to M$诱导了同构$\mathrm{H}^p_{\mathrm{dR}}(M)\cong\prod_j\mathrm{H}^p_{\mathrm{dR}}(M_j)$.另外我们知道流形上连通分支总是无交并拓扑,于是流形的德拉姆上同调就是连通分支的德拉姆上同调的直积.这允许我们主要关注连通流形上的德拉姆上同调.
	\begin{proof}
		
		这些包含映射的回拉诱导了映射$\Omega^p(M)\to\prod_j\Omega^p(M_j)$为$\omega\mapsto(l_1^*\omega,l_2^*\omega,\cdots)$,其中$l_j^*\omega=\omega\mid M_j$.这个映射明显是同构,而同构的链复形诱导同构的同调群列.
	\end{proof}
    \item 如果$M$是连通流形,那么$\mathrm{H}^0_{\mathrm{dR}}(M)\cong\mathbb{R}$,于是它是一维实线性空间.结合上一条说明流形$M$的零维德拉姆上同调群就是以全体连通分支为指标集的$\mathbb{R}$的直积.
    \begin{proof}
    	
    	按照定义有$B^0(M)=0$,于是$\mathrm{H}^0_{\mathrm{dR}}(M)=Z^0(M)$.但是一个闭0形式就是$M$上的一个实光滑函数$f$,满足$\mathrm{d}f=0$.由于$M$是连通的,这等价于$f$是常值实函数,得证.
    \end{proof}
    \item 在$\mathbb{R}^3$的开子集上外微分算子就是梯度旋度和散度.
    $$\xymatrix{
    	\text{光滑0形式}\ar[r]^{\text{外微分}}\ar[d]_{\sim}&\text{光滑1形式}\ar[r]^{\text{外微分}}\ar[d]_{\sim}&\text{光滑2形式}\ar[r]^ {\text{外微分}}
    	\ar[d]_{\sim}&\text{光滑3形式}\ar[d]_{\sim}\\
    	\text{光滑函数空间}\ar[r]_{\text{梯度}}&\text{光滑向量场空间}\ar[r]_ {\text{旋度}}&\text{光滑向量场空间}\ar[r]_{\text{散度}}&\text{光滑函数空间}}$$
\end{enumerate}

德拉姆上同调的同伦不变性.
\begin{enumerate}
	\item 设$M,N$是光滑流形,设$F,G:M\to N$是两个光滑映射,那么$F^*,G^*$是德拉姆复形之间的链映射,它们的德拉姆复形之间的次数$-1$的链映射$h:\Omega^p(N)\to\Omega^{p-1}(M),\forall p$称为$F,G$的链同伦映射(或者有时叫上链同伦),如果对$M$上每个$p$形式$\omega$,都有$\mathrm{d}(h\omega)+h(\mathrm{d}\omega)=G^*\omega-F^*\omega$.同伦是链映射上的等价关系.容易验证同伦的链复形诱导了相同的上同调群同态$F^*=G^*:\mathrm{H}^p_{\mathrm{dR}}(N)\to\mathrm{H}^p_{\mathrm{dR}}(M)$.
	\item 引理.设$M$是光滑流形,记$I=[0,1]$,记$i_t:M\to M\times I$为$i_t(x)=(x,t)$.那么两个链映射$i_0^*,i_1^*:\Omega^*(M\times I)\to\Omega^*(M)$是同伦的.(注意这里如果$M$边界为空那么$M\times I$是带边流形,但是如果$M$本身边界非空,那么$M\times I$需要理解为带角流形,一样可以定义带角流形的上同调).
	\begin{proof}
		
		记$\mathbb{R}$上的标准坐标为$s$,记$M\times\mathbb{R}$上的向量场$S$为$S_{(q,s)}=(0,\partial/\partial s\mid_s)$(这里把$\mathrm{T}_{(q,s)}M\times I$同构为$\mathrm{T}_qM\times\mathrm{T}_s\mathbb{R}$).任取$M\times I$上的$p$形式$\omega$,我们定义$h\omega=\int_0^1i_t^*(i_S\omega)\mathrm{d}t$.这里$i_S\omega$是$\omega$关于向量场$S$的内乘,它得到的是一个$p-1$形式,它的定义是$(i_S\omega_p)(v_1,v_2,\cdots,v_{p-1})=\omega_p(S_p,v_1,v_2,\cdots,v_{p-1})$.于是有:【】
		\begin{align*}
		h(\mathrm{d}\omega)+\mathrm{d}(h\omega)&=\int_0^1\left(i_t^*(i_S(\mathrm{d}\omega))+\mathrm{d}(i_t^*(i_S\omega))\right)\mathrm{d}t\\&=\int_0^1\left(\right)
		\end{align*}
	\end{proof}
    \item 如果$F,G:M\to N$是同伦的光滑映射,那么它们诱导的德拉姆复形之间的链映射是同伦的.于是特别的同伦的光滑映射诱导了相同的德拉姆上同调群之间的同态.这说明同调实际上是光滑流形范畴的同伦商范畴为源端的函子.于是特别的光滑同伦等价的流形具有相同的德拉姆同调群列.
    \begin{proof}
    	
    	按照Whitney逼近定理,两个光滑映射是同伦的则它们是光滑同伦的,于是可取从$F$到$G$的光滑同伦映射$H:M\times I\to N$.(此即$H(x,0)=F(x)$,$H(x,1)=G(x)$).于是$F=H\circ i_0$和$G=H\circ i_1$,结合上一条引理得到有链映射满足:
    	$$F^*=(H\circ i_0)^*=i_0^*\circ H^*\sim i_1^*\circ H^*=(H\circ i_1)^*=G^*$$
    \end{proof}
    \item 同伦不变性.如果$M,N$是互相(拓扑)同伦等价的光滑流形(这是指存在$M\to N$和$N\to M$的光滑映射,使得两个方向的复合分别同伦于$\mathrm{id}_M$和$\mathrm{id}_N$).那么它们的德拉姆上同调群列是同构的,并且任意光滑同伦等价都可以诱导它们的同构.
    \begin{proof}
    	
    	如果$F:M\to N$和$G:N\to M$是一对同伦等价,按照Whitney逼近定理,存在光滑映射$\widetilde{F}$和$\widetilde{G}$分别同伦于$F$和$G$,那么有$\widetilde{F}\circ\widetilde{G}\sim F\circ G\sim\mathrm{id}_N$,同理$\widetilde{G}\circ\widetilde{F}\sim\mathrm{id}_M$.而它们是光滑映射之间的同伦,于是等价于光滑同伦,于是$M,N$是光滑同伦等价的.于是它们的德拉姆上同调列是同构的,并且可由光滑同伦等价诱导.
    \end{proof}
\end{enumerate}

同伦不变性的一些推论.
\begin{enumerate}
	\item 同伦不变性说明德拉姆上同调仅仅依赖于空间的拓扑,这比较神奇因为德拉姆上同调明明是借助微分形式定义出来的.特别的这说明同一个拓扑流形上多个光滑结构诱导的德拉姆上同调群必须是相同的.
	\item 如果$M$是可缩光滑流形(可缩是指和单点空间同伦等价),那么$\mathrm{H}^p_{\mathrm{dR}}(M)=0,\forall p\ge1$.
	\item 庞加莱引理.如果$U$是$\mathbb{R}^n$或者$\mathbb{H}^n$的星形开子集,那么$\mathrm{H}^p_{\mathrm{dR}}(M)=0,\forall p\ge1$.特别的$\mathrm{H}^p_{\mathrm{dR}}(M)=0,\forall p\ge1$.
	\item 一次上同调和基本群.设$M$是连通流形,对每个$q\in M$,有实线性同构$\Phi:\mathrm{H}^1_{\mathrm{dR}}(M)\cong\mathrm{Hom}_{\textbf{Grp}}(\pi_1(M,q),\mathbb{R})$定义为,对每个上同调类$[\omega]\in\mathrm{H}^1_{\mathrm{dR}}(M)$,定义$\Phi[\omega]:\pi_1(M,q)\to\mathbb{R}$为$[\gamma]\mapsto\int_{\widetilde{\gamma}}\omega$,其中$\widetilde{\gamma}$是和$\gamma$在相同道路同伦类的分段光滑曲线.
	\begin{proof}
		
		先验证$\Phi$的定义良性,为此需要先说明如果$\gamma_0,\gamma_1:[0,1]\to M$是道路同伦的分段光滑曲线,那么对$M$上的闭1形式$\omega$总有$\int_{\gamma_0}\omega=\int_{\gamma_1}\omega$.这个结论需要带角流形上的Stokes定理.
		
		验证定义良性还需要验证结果这个积分不依赖于同调类$[\omega]$中的代表元$\omega$的选取,而这是因为如果$\omega,\widetilde{\omega}$在同一个同调类,那么存在光滑函数$f$使得$\widetilde{\omega}-\omega=\mathrm{d}f$,就导致:
		$$\int_{\widetilde{\gamma}}\widetilde{\omega}-\int_{\widetilde{\gamma}}\omega=\int_{\widetilde{\gamma}}\mathrm{d}f=f(q)-f(q)=0$$
		
		验证$\Phi$是单射.如果$\Phi[\omega]$是零同态,那么对任意以点$q$为起点的分段光滑闭曲线$\widetilde{\gamma}$,都有$\int_{\widetilde{\gamma}}\omega=0$.任取点$q'\in M$起始的分段光滑闭曲线$\sigma$,选取从$q$到$q'$的分段光滑闭曲线$\alpha$,按照如下等式说明$\omega$在其上的积分总是零,这说明$\omega$是恰当形式,于是$[\omega]=0$,于是$\Phi$是单射.
		$$0=\int_{\alpha\cdot\sigma\cdot\overline{\alpha}}\omega=\int_{\alpha}\omega+\int_{\sigma}\omega-\int_{\alpha}\omega=\int_{\sigma}\omega$$
		
		【】
	\end{proof}
    \item 一个小推论.按照有限群到$\mathbb{R}$的群同态只有平凡同态,说明如果$M$是连通的基本群有限的流形,那么$\mathrm{H}^1_{\mathrm{dR}}(M)=0$.类似的基本群是挠群(存在一个固定整数使得群中每个元素的阶都整除它)的连通流形$M$满足一阶德拉姆上同调群平凡.
\end{enumerate}

Mayer-Vietoris序列.
\begin{enumerate}
	\item 定理内容和证明.如果$M$是光滑流形,$U,V$是覆盖了$M$的两个开子集,那么对每个非负整数$p$存在线性映射$\delta:\mathrm{H}^p_{\mathrm{dR}}(U\cap V)\to\mathrm{H}^{p+1}_{\mathrm{dR}}(M)$使得存在如下正合列,其中$k:U\to M$,$l:V\to M$,$i:U\cap V\to U$,$j:U\cap V\to V$都是包含映射.这个正合列称为Mayer-Vietoris序列:
	$$\xymatrix{\cdots\ar[r]^{\delta\quad}&\mathrm{H}^p_{\mathrm{dR}}(M)\ar[r]^{k^*\oplus l^*\qquad}&\mathrm{H}^p_{\mathrm{dR}}(U)\oplus\mathrm{H}^p_{\mathrm{dR}}(V)\ar[r]^{\quad i^*-j^*}&\mathrm{H}^p_{\mathrm{dR}}(U\cap V)\ar[r]^{\delta}&\mathrm{H}^{p+1}_{\mathrm{dR}}(M)\ar[r]&\cdots}$$
	\begin{proof}
		
		只需证明有如下链复形的短正合列,这样的短正合列就诱导出命题中的长正合列:
		$$\xymatrix{0\ar[r]&\Omega^*(M)\ar[r]^{k^*\oplus l^*\qquad}&\Omega^*(U)\oplus\Omega^*(V)\ar[r]^{\quad i^*-j^*}&\Omega^*(U\cap V)\ar[r]&0}$$
		
		归结为证明有如下短正合列,因为光滑映射诱导的回拉自动构成链映射:
		$$\xymatrix{0\ar[r]&\Omega^p(M)\ar[r]^{k^*\oplus l^*\qquad}&\Omega^p(U)\oplus\Omega^p(V)\ar[r]^{\quad i^*-j^*}&\Omega^p(U\cap V)\ar[r]&0}$$
		
		证明$\ker(k^*\oplus l^*)=0$:如果$\sigma\in\Omega^p(M)$满足$(k^*\oplus l^*)(\sigma)=(\sigma\mid U,\sigma\mid V)=0$,即$\omega$在$U,V$上的限制都是0,但是$U\cup V=M$,这说明$\sigma=0$.
		
		证明$\ker(i^*-j^*)=\mathrm{im}(k^*\oplus l^*)$:按照复形定义以及有$\mathrm{im}(k^*\oplus l^*)\subset\ker(i^*-j^*)$.反过来如果$(\eta,\eta')\in\ker(i^*-j^*)$,其中$\eta$是$U$上的$p$形式,$\eta'$是$V$上的$p$形式,此即$\eta\mid U\cap V=\eta'\mid U\cap V$.那么$\eta$和$\eta'$可以唯一的粘合为$M$上的一个整体$p$形式$\sigma$,于是$(k^*\oplus l^*)(\sigma)=(\sigma\mid U,\sigma\mid V)=(\eta,\eta')$,于是$\ker(i^*-j^*)\subset\mathrm{im}(k^*\oplus l^*)$.
		
		证明$i^*-j^*$是满射,这是唯一稍微不平凡的部分:任取$\omega$是$U\cap V$上的$p$形式,需要验证存在$U$上的$p$形式$\eta$和$V$上的$p$形式$\eta'$满足$\omega=\eta\mid U\cap V-\eta'\mid U\cap V$.记$\{U,V\}$的单位分解为$\{\varphi,\psi\}$.定义$\eta$为在$U\cap V$上取$\psi\omega$,再在$U-\mathrm{Supp}\psi$上恒取零;定义$\eta'$为在$U\cap V$上取$-\varphi\omega$,再在$V-\mathrm{Supp}\varphi$上恒取零.于是有如下等式,这完成证明.
		$$\eta\mid U\cap V-\eta'\mid U\cap V=\psi\omega-(-\varphi\omega)=(\psi+\varphi)\omega=\omega$$
	\end{proof}
    \item 这里我们来描述一下MV序列中的连接映射$\delta:\mathrm{H}^p_{\mathrm{dR}}(U\cap V)\to\mathrm{H}^{p+1}_{\mathrm{dR}}(M)$:对$U\cap V$上的每个闭$p$形式$\omega$,存在$U$和$V$上分别的$p$形式$\eta$和$\eta'$,使得$\omega=\eta\mid U\cap V-\eta'\mid U\cap V$,那么$U$上的$\mathrm{d}\eta$和$V$上的$\mathrm{d}\eta'$可粘合为$M$上整体$p+1$形式$\sigma$,那么有$\delta[\omega]=[\sigma]$.如果选取$\{U,V\}$上的单位分解$\{\varphi,\psi\}$,那么这里的$\eta$和$\eta'$可分别取为$\psi\omega$和$-\varphi\omega$(分别零延拓到$\psi$和$\varphi$的支集以外).
\end{enumerate}

Mayer-Vietoris序列的一些应用.
\begin{enumerate}
	\item 球面的德拉姆上同调.对$n\ge1$有:
	$$\mathrm{H}^p_{\mathrm{dR}}(\mathbb{S}^n)=\left\{\begin{array}{cc}\mathbb{R}&p=0,n\\0&0<p<n\end{array}\right.$$
	
	另外球面的最高次上同调群作为一维实线性空间可被任意定向形式生成.
	\begin{proof}
		
		按照球面在维数$\ge1$时总是连通的,它的零次上同调群是常值函数群,也即$\mathbb{R}$.另外我们证明过一次上同调群同构于$\mathrm{Hom}_{\textbf{Grp}}(\pi_1(\mathbb{S}^1,1),\mathbb{R})\cong\mathbb{R}$.于是我们解决了$n=1$的情况.
		
		对$n\ge2$,取$U,V$分别是$\mathbb{S}^n$扣去北极点和南极点得到的开子集,那么$U,V$都同胚于$\mathbb{R}^n$,它们的次数$\ge2$的上同调群都平凡,而$U\cap V$同胚于$\mathbb{R}^n$扣去一个点,这同伦等价于$\mathbb{S}^{n-1}$.按照MV序列得到对每个$p\ge2$和每个$n\ge2$都有$\mathrm{H}^p_{\mathrm{dR}}(\mathbb{S}^n)\cong\mathrm{H}^{p-1}_{\mathrm{dR}}(\mathbb{S}^{n-1})$.从$n=1$的情况就归纳得到一般情况.
		
		最后定向形式的积分肯定是非零的,导致定向形式总不会是恰当形式,于是任意定向形式都线性生成了整个最高次上同调群.
	\end{proof}
    \item 记$M=\mathbb{R}^n-\{x\}$,那么$M$上一个闭$n-1$形式$\eta$是恰当的当且仅当存在某个(于是等价于对任意的)以$x$为中心的$n-1$维球面$S\subset M$有$\int_S\eta=0$.
    \begin{proof}
    	
    	包含映射$i:S\to M$是一个同伦等价,于是诱导了德拉姆上同调群的同构$i^*:\mathrm{H}^p_{\mathrm{dR}}(M)\cong\mathrm{H}^p_{\mathrm{dR}}(S)$.如果$\eta$是$M$上的闭$n-1$形式,那么$\eta$是恰当的当且仅当$i^*\eta$是$S$上恰当形式.结合如下引理这等价于$\int_S\eta=\int_Si^*\eta=0$:
    	
    	引理.$\mathbb{S}^n$上的一个闭$n$形式$\eta$是恰当的当且仅当$\int_{\mathbb{S}^n}\eta=0$:我们解释过$n$次德拉姆上同调群被任意定向形式线性生成,也即这里的$\eta$可以表示为定向形式的某个倍数加上某个恰当形式,但是定向形式的积分是严格为正的,恰当形式的积分为零,导致这里定向形式的某个倍数必然是零,于是$\eta$是恰当的.
    \end{proof}
    \item 如果$n\ge2$,$U\subset\mathbb{R}^n$是非空开子集,$x\in U$,那么$\mathrm{H}^{n-1}_{\mathrm{dR}}(U-\{x\})$非平凡.
    \begin{proof}
    	
    	因为$U$是开集,可选取$x$为中心的$n-1$维球面$S$使得$S\subset U-\{x\}$,取包含映射$i:S\subset U-\{x\}$,取射线投影映射$r:U-\{x\}\to S$,它们都是光滑映射,并且有$r\circ i=\mathrm{id}_S$,导致有同调群上同态的关系$i^*\circ r^*=\mathrm{id}$,这说明$r^*:\mathrm{H}^{n-1}_{\mathrm{dR}}(S)\to\mathrm{H}^{n-1}_{\mathrm{dR}}(U-\{x\})$是单射,但是源端非平凡的,所以终端非平凡.
    \end{proof}
    \item 拓扑维数不变性.如果一个$n$维拓扑流形同胚于一个$m$维拓扑流形,那么必须有$m=n$.
    \begin{proof}
    	
    	如果拓扑流形$M$同时是$m$维和$n$维的,不妨设$m>n$,假设$n=0$,那么作为零维流形它是至多可数集,但是作为$m>0$维流形它是不可数集,这矛盾.下面设$m>n\ge1$.按照$M$是$m$维流形,可取开子集$V$同胚于$\mathbb{R}^m$.任取$x\in V$,又可取开邻域$x\in U\subset V$使得$U$同胚于$\mathbb{R}^n$.经这个同胚可定义$U$上的一个光滑结构,此时有$\mathrm{H}^{m-1}_{\mathrm{dR}}(U-\{x\})=0$.但是从$V$同胚于$\mathbb{R}^m$的某个开子集,又可定义其上的光滑结构使得$\mathrm{H}^{m-1}_{\mathrm{dR}}(U-\{x\})\not=0$.这样我们在同一个拓扑流形上定义了两个不同的光滑结构,而它们的德拉姆上同调不同.但是我们证明过同胚的光滑流形就一定具有相同的德拉姆上同调,这个矛盾说明必须有$m=n$.
    \end{proof}
\end{enumerate}

紧支撑的德拉姆上同调.设$M$是流形,设$\Omega_c^p(M)$上全体紧支撑$p$形式构成的实线性空间,按照紧支撑微分形式的微分仍然是紧支撑的,说明外微分算子限制在$\Omega_c^*(M)$上构成德拉姆复形的一个子复形,它称为紧支撑德拉姆复形,它的上同调群列称为紧支撑的德拉姆上同调群列.
$$\mathrm{H}_c^p(M)=\frac{\ker\left(\mathrm{d}:\Omega_c^p(M)\to\Omega_c^{p+1}(M)\right)}{\mathrm{im}\left(\mathrm{d}:\Omega_c^{p-1}(M)\to\Omega_c^p(M)\right)}$$
\begin{enumerate}
	\item 明显的,如果$M$本身是紧流形,那么紧支撑德拉姆上同调和常规德拉姆上同调是一致的.
	\item 引理.设$n\ge p\ge1$,设$\omega$是紧支撑的$\mathbb{R}^n$上的闭$p$形式,在$p=n$时额外添加条件$\int_{\mathbb{R}^n}\omega=0$.那么对任意$n\ge p\ge1$,总存在$\mathbb{R}^n$上的紧支撑的$n-1$形式$\eta$满足$\mathrm{d}\eta=\omega$.
	\begin{proof}
		
		如果$n=p=1$,可记$\omega=f\mathrm{d}x$(因为在欧氏空间中,有整体坐标卡),其中$f$是具有紧支集的光滑映射.定义$F:\mathbb{R}\to\mathbb{R}$为$F(x)=\int_{-\infty}^xf(t)\mathrm{d}t$.按照微积分基本定理,就有$\mathrm{d}F=F'\mathrm{d}x=f\mathrm{d}x=\omega$.设足够大的正实数$R$满足$\mathrm{Supp}f\subset [-R,R]$.那么如果$x<-R$,有$F(x)=0$;如果$x>R$,按照$\int_{\mathbb{R}}\omega=0$得到$F(x)=\int_{-\infty}^xf(t)\mathrm{d}t=\int_{-\infty}^{+\infty}f\mathrm{d}t=0$.于是有$\mathrm{Supp}F\subset[-R,R]$.这证明了$n=p=1$的情况.
		
		下设$n\ge2$,可取以原点为圆心的开球$B,B'\subset\mathbb{R}^n$使得$\mathrm{Supp}\omega\subset B\subset\overline{B}\subset B'$.按照欧氏空间上的庞加莱引理,存在$\mathbb{R}^n$上的$p-1$形式$\eta_0$使得$\mathrm{d}\eta_0=\omega$.于是特别的$\mathrm{d}\eta_0$在$\mathbb{R}^n-\overline{B}$上为零.接下来分情况考虑:
		
		情况1,如果$p=1$.此时$\eta_0$是一个光滑函数,由于在$n\ge2$时$\mathbb{R}^n-\overline{B}$是连通的,说明$\eta_0$在其上是一个常值函数$c$.设$\eta=\eta_0-c$,那么$\eta$是一个具有紧支集的0形式,并且满足$\mathrm{d}\eta=\omega$.
		
		情况2,如果$1<p<n$.考虑$\eta_0$在$\mathbb{R}^n-\overline{B}$上的限制,这是一个闭$p-1$形式(因为我们证明过扣去单点的欧氏空间上有$\mathrm{H}_{\mathrm{dR}}^{p-1}(\mathbb{R}^n-\overline{B})$,而扣去单点的欧氏空间$\mathbb{R}^n-\{x\}$是和$\mathbb{R}^n-\overline{B}$同伦等价的).于是存在$\mathbb{R}^n-\overline{B}$上的$p-2$形式$\gamma$使得$\mathrm{d}\gamma=\eta_0$.取$\mathbb{R}^n-B'\subset\mathbb{R}^n-\overline{B}$上的光滑碰撞函数$\psi$,那么有$\eta=\eta_0-\mathrm{d}(\psi\gamma)$是$\mathbb{R}^n$上的光滑形式,并且有$\mathrm{d}\eta=\mathrm{d}\eta_0=\omega$.最后在$\mathbb{R}^n-B'$上有$\mathrm{d}(\psi\gamma)=\mathrm{d}\gamma=\eta_0$,说明在其上有$\eta=0$,于是$\eta$是一个紧支撑的形式.
		
		情况3,如果$p=n$.此时$\mathrm{H}^{n-1}_{\mathrm{dR}}(\mathbb{R}^n-\overline{B})\not=0$所以不能用情况2的方法.但是按照如下等式有$\eta_0$在$\partial B'$上的积分为零,我们之前证明过的一个定理说明这导致$\eta_0$在$\mathbb{R}^n-\overline{B}$上正合,剩下的操作就和情况2相同.
		$$0=\int_{\mathbb{R}^n}\omega=\int_{\overline{B'}}\omega=\int_{\overline{B'}}\mathrm{d}\eta_0=\int_{\partial B'}\eta_0$$
	\end{proof}
    \item 对$n\ge1$,欧氏空间$\mathbb{R}^n$的紧支撑德拉姆上同调群列为:
    $$\mathrm{H}_c^p(\mathbb{R}^n)=\left\{\begin{array}{cc}0&0\le p<n\\\mathbb{R}&p=n\end{array}\right.$$
    \item 函子性.对于一般的光滑映射,它未必把紧支撑微分形式拉回为紧支撑微分形式.不过一个真(proper)映射满足这个性质.于是对于真光滑映射$F:M\to N$,它总诱导了紧支撑上同调群列之间的同态$F^*:\mathrm{H}_c^p(N)\to\mathrm{H}_c^p(M),\forall p\ge0$.
\end{enumerate}

最高次紧支集上同调群上的积分泛函.设$M$是$n$维定向流形,那么存在实线性泛函$I:\Omega_c^n(M)\to\mathbb{R}$为$I(\omega)=\int_M\omega$.按照恰当$n$形式的积分为零,说明这个线性泛函也可以视为$\mathrm{H}_c^n(M)$为源端的.
\begin{enumerate}
	\item 如果$M$是连通定向$n$流形(不带边),那么积分泛函$I:\mathrm{H}_c^n(M)\to\mathbb{R}$是一个同构.于是特别的此时$\mathrm{H}^n_c(M)$是一维的实线性空间.
	\begin{proof}
		
		$n=0$的时候,连通零维流形就是一个单点空间,此时结论平凡成立.下设$n\ge1$,任取$M$上的定向光滑坐标卡$(U,(x^i))$,设$f$是被$U$支撑的光滑碰撞函数.定义整体$n$形式$\theta_0$在$U$上为$f\mathrm{d}x^1\wedge\cdots\wedge\mathrm{d}x^n$,在$U$以外为零,那么这是一个紧支撑的形式,并且有$I(\theta_0)>0$.这说明积分泛函$I$是一个满射.为证$I$是单射需要证明如果$\omega$是$M$上具有紧支集的$n$形式,并且满足$\int_M\omega=0$,那么存在一个具有紧支集的$n-1$形式$\eta$使得$\omega=\mathrm{d}\eta$.(这件事我们之前证明了欧氏空间上的情况,这里是一般流形上).
		
		取$M$的可数个光滑坐标卡$\{U_i\}$,这里每个$U_i$都是微分同胚于$\mathbb{R}^n$的.记$M_k=U_1\cup U_2\cdots\cup\cdots\cup U_k$.按照$M$是连通的,我们可以重排这些$U_i$的指标使得对每个指标$k$都有$M_k\cap U_{k+1}\not=\emptyset$.具有紧支集的微分形式$\omega$肯定落在某个$M_k$中,于是问题归结为:对每个指标$k$,如果$\omega\in\Omega_c^n(M_k)$的积分为零,那么存在$\eta\in\Omega_c^{n-1}(M_k)$使得$\omega=\mathrm{d}\eta$.我们对$k$归纳说明这件事.
		
		$k=1$的情况.此时$M_1=U_1$微分同胚于$\mathbb{R}^n$,欧氏空间的情况我们已经证明过了.下设命题对某个整数$k\ge1$成立,假设$\omega$是$M_{k+1}=M_k\cup U_{k+1}$上的具有紧支集的$n$形式,并且满足$\int_{M_{k+1}}\omega=0$.
		
		借助光滑碰撞函数可构造$\theta\in\Omega_c^n(M_{k+1})$并且被$M_k\cap U_{k+1}$支撑,并且满足$\int_{M_{k+1}}\theta=1$.选取$M_{k+1}$的开覆盖$\{M_k,U_{k+1}\}$的单位分解$\{\varphi,\psi\}$.记$c=\int_{M_{k+1}}\varphi\omega$.那么$\varphi\omega-c\theta$被$M_k$紧支撑,并且在$M_k$上的积分为零.于是按照归纳假设存在被$M_k$紧支撑的$n-1$形式$\alpha$,使得$\mathrm{d}\alpha=\varphi\omega-c\theta$.类似的,$\psi\omega+c\theta$被$U_{k+1}$紧支撑,并且在$M_{k+1}$上的积分为零.但是这里$U_{k+1}$微分同胚于$\mathbb{R}^n$,而欧氏空间上本结论已经得证过,于是存在被$U_{k+1}$紧支撑的$n-1$形式$\beta$,满足$\mathrm{d}\beta=\psi\omega+c\theta$.把$\alpha$和$\beta$都零延拓到整个$M_{k+1}$上,得到$\mathrm{d}(\alpha+\beta)=\omega$,得证.
	\end{proof}
    \item 特别的,按照紧流形上的紧支撑上同调就是德拉姆上同调,得到:如果$M$是紧连通定向光滑$n$流形,那么$\mathrm{H}^n_{\mathrm{dR}}(M)$是1维实空间,并且被任意定向形式的同调类生成.
    \item 如果$M$是非紧连通定向光滑$n$流形,那么$\mathrm{H}^n_{\mathrm{dR}}(M)=0$.
    \begin{proof}
    	
    	选取$M$上的光滑exhaustion函数$f$,不妨设$\inf_Mf=0$,否则可以对$f$减去一个常数.那么$M$的非紧性和连通性就保证$f(M)=[0,\infty)$.对每个正整数$i$,记$V_i=f^{-1}((i-2,i))$,那么$\{V_i,i\ge1\}$就是$M$的由预紧开子集构成的开覆盖.并且满足$V_i\cap V_j$非空当且仅当$j=i-1$或$i$或$i+1$.取这组开覆盖对应的单位分解为$\{\psi_i\}$.对每个$i$,选取$\theta_i\in\Omega_c^n(M)$使得它是被$V_i\cap V_{i+1}$紧支撑的满足$\int_M\theta_i=1$的形式.
    	
    	任取$M$上的$n$形式$\omega$,记$\omega_i=\psi_i\omega$,那么每个$\omega_i\in\Omega_c^n(V_i)$.记$c_1=\int_{V_1}\omega_1$,那么$\omega_1-c_1\theta_1$是被$V_1$紧支撑的形式,并且积分为零.按照紧情况下本定理的结论,存在$\eta_1\in\Omega_c^n(V_1)$使得$\mathrm{d}\eta_1=\omega_1-c_1\theta_1$.再取实数$c_2$使得$\int_{V_2}\left(\omega_2+c_1\theta_1-c_2\theta_2\right)=0$,于是存在$\eta_2\in\Omega_c^n(V_2)$使得$\mathrm{d}\eta_2=\omega_2+c_1\theta_1-c_2\theta_2$.归纳构造下去,存在$c_j\in\mathbb{R}$和$\eta_j\in\Omega_c^n(V_j)$使得$\mathrm{d}\eta_j=\omega_j+c_{j-1}\theta_{j-1}+c_j\theta_j$.记$\eta=\sum_{j=1}^{+\infty}\eta_j$,这里每个$\eta_j$都零延拓到$M-V_j$上.对每个$V_j$,这个无穷和中至多存在三个项(并且是相邻的)非零,于是这个无穷和有意义并且是一个光滑$n$形式,并且有$\mathrm{d}\eta=\sum_j\omega_j=\omega$.
    \end{proof}
    \item 如果$M$是连通非定向光滑$n$流形,那么有$\mathrm{H}^n_c(M)=\mathrm{H}_{\mathrm{dR}}^n(M)=0$.
    \begin{proof}
    	
    	【】
    \end{proof}
\end{enumerate}

光滑映射的映射度.设$M,N$是紧连通定向$n$流形,我们证明过这个条件下的最高次德拉姆上同调群经积分泛函同构于$\mathbb{R}$.如果$F:M\to N$是光滑映射,它诱导了上同调群之间的同态$\mathrm{H}^n_{\mathrm{dR}}(N)\to\mathrm{H}^n_{\mathrm{dR}}(M)$.经积分泛函诱导的同构,这个上同调群之间的同态也就是一个$\mathbb{R}\to\mathbb{R}$的实线性映射,这样的同态被它在1处的取值唯一决定,这个取值称为$F$的映射度,记作$\deg F$.我们会证明映射度总是一个整数.按照定义,光滑映射$F$的映射度$k$就是满足如下条件的实数:对$N$上每个$n$形式$\omega$都有:$$\int_MF^*\omega=k\int_N\omega$$
\begin{enumerate}
	\item 如果$M,N$是紧连通定向$n$维流形,设$F:M\to N$是光滑映射,如果$q\in N$是$F$的正则值(即$F^{-1}(q)$中每个点的微分都是满射),那么有如下等式,其中如果$\mathrm{d}F_x$是保定向的那么$\mathrm{sgn}(x)=+1$;如果$\mathrm{d}F_x$是反定向的那么$\mathrm{sgn}(x)=-1$.特别的,这件事说明映射度总是一个整数.
	$$\deg F=\sum_{x\in F^{-1}(q)}\mathrm{sgn}(x)$$
    \begin{proof}
    	
    	记$\deg F=k$.任取$F$的正则值$q\in N$,那么$F^{-1}(q)$是$M$的零维真嵌入子流形,于是这是有限子集.设$F^{-1}(q)$非空,可记$\{x_1,x_2,\cdots,x_m\}$.按照反函数定理,对每个指标$i$存在$x_i$的开邻域$U_i$,使得$F$在其上的限制是到$q$的开邻域$W_i$的微分同胚.适当缩小$U_i$,我们不妨设这些$U_i$是两两不交.记$K=M-(U_1\cup\cdots\cup U_m)$,这是紧空间$M$的闭子集,于是它自身也是紧的.于是$F(K)$是$N$中的闭子集(紧集的连续像紧,Hausdorff空间的紧子集是闭的),并且$q\not\in F(K)$.设$W_1\cap\cdots\cap W_m\cap(N-F(K))$的包含点$q$的连通分支维$W$,记$V_i=F^{-1}(W)\cap U_i$.那么$W$是$q$的连通开邻域,并且它在$F$下的原像是$V_1,V_2,\cdots,V_m$的无交并,并且$F$在每个$V_i$的限制都是到$W$的微分同胚.于是每个$V_i$是连通的,导致$F$在$V_i$上的限制必须是保定向或者反定向的.
    	
    	设$\omega$是$N$上的被$W$紧支撑的$n$形式,并且满足$\int_N\omega=\int_W\omega=1$.我们之前证明了$\int_MF^*\omega=k$.按照$F^*\omega$被$F^{-1}(W)$紧支撑,得到$\int_MF^*\omega=\sum_{i=1}^m\int_{V_i}F^*\omega$.而我们证明过$\int_{V_i}F^*\omega=\pm\int_W\omega=\pm1$,符号取决于$F$是保定向还是反定向,至此我们证明了当$F^{-1}(q)$非空时$k$满足第二条.
    	
    	最后假设$F^{-1}(q)$是空集,我们约定对空集的求和是零.按照$F(M)$是$N$的闭子集,此时$q$存在开邻域$W$包含在$N-F(M)$中.任取被$W$紧支撑的$N$上的$n$形式$\omega$,那么$\int_MF^*\omega=0$,于是$k=0$,完成证明.
    \end{proof}
    \item 映射度的一些基本性质.设$M,N,P$都是$n$维的紧连通定向流形.
    \begin{itemize}
    	\item 如果$F:M\to N$,$G:N\to P$都是光滑映射,那么有$\deg(G\circ F)=(\deg G)(\deg F)$.
    	\item 如果$F$是保定向的微分同胚,那么$\deg F=+1$;如果$F$是反定向的微分同胚,那么$\deg F=-1$.
    	\item 如果两个光滑映射$F_0,F_1:M\to N$同伦,那么它们具有相同的次数.
    \end{itemize}
    \item 如果$M,N$仍然是$n$维紧连通定向流形,我们可以定义连续映射$F:M\to N$的映射度:按照Whitney逼近定理,存在光滑映射$G:M\to N$和$F$同伦,我们就定义$\deg F=\deg G$.这个定义不依赖于$G$的选取是因为同伦的光滑映射的映射度相同.
    \item 设$N$是紧连通定向$n$流形,设$X$是紧定向$n+1$带边流形,并且有连通的边界,如果$f:\partial X\to N$是连续映射,并且可以连续延拓到$X$上,那么$\deg f=0$.
    \begin{proof}
    	
    	设$f$可连续延拓为$F:X\to N$.按照Whitney逼近定理,存在光滑映射$\widetilde{F}$同伦于$F$,于是我们不妨设$F$和$f$本身就是光滑的.任取$N$上的$n$形式$\omega$,那么有$\mathrm{d}\omega=0$,于是按照Stokes定理有如下等式,这说明$\deg f=0$.
    	$$\int_{\partial X}f^*\omega=\int_{\partial X}F^*\omega=\int_X\mathrm{d}(F^*\omega)=\int_XF^*\mathrm{d}\omega=0$$
    \end{proof}
    \item Brouwer不动点定理.$\overline{B}^n$到自身的每个连续映射都存在不动点.
    \begin{proof}
    	
    	假设连续映射$F:\overline{B}^n\to\overline{B}^n$不存在不动点,我们可以定义$G:\overline{B}^n\to\mathbb{S}^{n-1}$为:$$G(x)=\frac{x-F(x)}{|x-F(x)|}$$
    	
    	记$g$为$G$在边界$\mathbb{S}^{n-1}$上的限制.那么上一条结论说明$\deg g=0$.但是我们可以构造$\mathbb{S}^{n+1}$上的恒等映射到$g$的同伦为如下式子.这里分母不为零是因为,如果$t=1$那么分母不为零是$F$不存在不动点的假设,如果$t<1$那么$|x-tF(x)|>|x|-|tF(x)|\ge 1-t$.于是按照同伦的映射存在相同的映射度,得到$\mathbb{S}^{n-1}$上恒等映射的映射度为0,但是恒等映射的映射度理应是$+1$,这矛盾.
    	$$H(x,t)=\frac{x-tF(x)}{|x-tF(x)|}$$
    \end{proof}
\end{enumerate}

\subsection{德拉姆定理}

既然德拉姆上同调只和流形上的拓扑有关,它就应该可以纯拓扑上计算,所谓的德拉姆定理就是指对于光滑流形,德拉姆上同调和$\mathbb{R}$系数的奇异上同调是一致的.为了证明这件事我们先引入光滑奇异同调.设$M$是光滑流形:
\begin{itemize}
	\item $M$上的一个光滑奇异$p$单形是指一个光滑映射$\sigma:\Delta_p\to M$,这里$\Delta_p$是标准的几何$p$单形$[e_0,e_1,\cdots,e_p]=\{\sum_{i=0}^pt_ie_i\mid 0\le t_i\le1,\sum_{i=0}^pt_i=1\}\subset\mathbb{R}^p$($e_0=0$,$e_i,i\ge1$是第$i$分量为1其余为零的向量).这里$\sigma$的光滑性是指$\sigma$在$\Delta_p\subset\mathbb{R}^p$的每个点,都可以光滑延拓到某个(在$\mathbb{R}^p$中的)开邻域上.
	\item 全体光滑奇异$p$单形生成的自由阿贝尔群记作$\mathrm{C}_p^{\infty}(M)$,它是全体奇异$p$单形生成的自由阿贝尔群$\mathrm{C}_p(M)$的子群.
	\item 几何$p$单形$[v_0,v_1,\cdots,v_p]$的第$i$个面($0\le i\le p$)定义为几何$p-1$单形$[v_0,\cdots,\hat{v_i},\cdots,v_p]$.如果记$\Delta_{p-1}=[w_0,w_1,\cdots,w_{p-1}]$,定义从$\Delta_{p-1}$到$\Delta_p$的第$i$个面的满足$w_0\mapsto e_0$,$\cdots$,$w_{i-1}\mapsto e_{i-1}$,$w_i\mapsto e_{i+1}$,$\cdots$,$w_{p-1}\mapsto e_p$的唯一的仿射映射为$F_{i,p}$.如果$\sigma:\Delta_p\to M$是一个奇异$p$单形,那么$\partial\sigma=\sum_{i=0}^p(-1)^i\sigma\circ F_{i,p}$是$\mathrm{C}_{p-1}(M)$中的元,这个映射$\partial$就是$\mathrm{C}_p(M)\to\mathrm{C}_{p-1}(M)$的同态,称为奇异边界算子.
	\item 奇异边界算子$\partial$满足$\partial\circ\partial=0$,于是$(\mathrm{C}_*(M),\partial)$构成一个复形,称为$M$的奇异复形.它在$\mathrm{C}^{\infty}_p(M)$上的限制是到$\mathrm{C}^{\infty}_{p-1}(M)$的同态,于是$(\mathrm{C}_*^{\infty}(M),\partial)$是一个复形,称为$M$的光滑奇异复形,它的同调群称为光滑奇异同调群:
	$$\mathrm{H}_p^{\infty}(M)=\frac{\ker\left(\partial:\mathrm{C}_p^{\infty}(M)\to\mathrm{C}_{p-1}^{\infty}(M)\right)}{\mathrm{im}\left(\partial:\mathrm{C}_{p+1}^{\infty}(M)\to\mathrm{C}_p^{\infty}(M)\right)}$$
\end{itemize}

包含映射$i:\mathrm{C}_p^{\infty}(M)\to\mathrm{C}_p(M)$和奇异边界算子可交换,于是包含映射是链映射,它诱导了同调群之间的同态$i_*:\mathrm{H}_p^{\infty}(M)\to\mathrm{H}_p(M)$为$i_*[c]=[i(c)]$.我们断言这实际上是同调群之间的同构.
\begin{proof}
	
	【】
\end{proof}

德拉姆定理.
\begin{enumerate}
	\item 设$M$是光滑流形,设$\omega$是$M$上的闭$p$形式,设$\sigma$是$M$上的光滑奇异$p$单形,定义$\omega$在$\sigma$上的积分为$\int_{\sigma}\omega=\int_{\Delta_p}\sigma^*\omega$.如果$p=1$这个定义吻合于曲线积分.如果$c\in\mathrm{C}_p^{\infty}(M)$,那么它可以唯一表示为$c=\sum_{i=1}^kc_i\sigma_i$,其中$\sigma_i$都是$M$上的光滑奇异$p$单形,$c_i\in\mathbb{R}$.那么定义$\omega$在光滑奇异$p$链$c$上的积分为:
	$$\int_c\omega=\sum_{i=1}^kc_i\int_{\sigma_i}\omega=\sum_{i=1}^kc_i\int_{\Delta_p}\sigma_i^*\omega$$
	\item 引理(我们会在带角流形上使用这个引理).设$M$是$n$维定向流形,设$\omega$是$M$上的具有紧支集的$n$形式,设$D_1,\cdots,D_k$是$\mathbb{R}^n$上的$k$个开容许集,并且对每个指标$i$都存在光滑映射$F_i:\overline{D}_i\to M$,满足如下三个条件:
	\begin{itemize}
		\item $F_i$限制在每个$D_i$上都是到$W_i\subset M$的保定向或者反定向微分同胚.
		\item 对$i\not=j$总有$W_i\cap W_j=\emptyset$.
		\item $\mathrm{Supp}\omega\subset\overline{W_1}\cup\overline{W_2}\cup\cdots\cup\overline{W_k}$.
	\end{itemize}

    那么有$\int_M\omega=\sum_{i=1}^k\mathrm{sgn}(i)\int_{D_i}F_i^*\omega$.其中如果$F_i$保定向那么$\mathrm{sgn}(i)=+1$,如果$F_i$反定向那么$\mathrm{sgn}(i)=-1$.
    \begin{proof}
    	
    	按照流形上积分的定义,问题归结为设$\omega$被单个定向坐标卡$(U,\varphi)$紧支撑,并且我们不妨设$U$是预紧的,并且$Y=\varphi(U)$是$\mathbb{R}^n$的容许开子集,并且不妨设$\varphi$可延拓为$\overline{U}\to\overline{Y}$的微分同胚.
    	
    	对每个指标$i$,记$A_i=F_i^{-1}(U\cap W_i)$,$B_i=U\cap W_i=F_i(A_i)$,$C_i=\varphi(B_i)=\varphi_i\circ F_i(A_i)$.那么有$A_i\subset D_i$,$B_i\subset W_i$,$C_i\subset Y$.
    	
    	从$\overline{D_i}$是紧集得到$\partial W_i\subset F_i(\partial D_i)$,于是$\partial W_i$是零测集(零测集在微分同胚下的像是零测的).同样有$\partial C_i=\varphi(\partial B_i)$是零测集.于是$(\varphi^{-1})^*\omega$的支集落在$\overline{C_1}\cup\cdots\cup\overline{C_k}$,并且这些并的任意两不同像的交只在它们的拓扑边界上,于是这些交总是零测的,于是得到:
    	$$\int_M\omega=\int_Y(\varphi^{-1})^*\omega=\sum_{i=1}^k\int_{C_i}(\varphi^{-1})^*\omega$$
    	
    	最后我们证明过如果$G:\overline{D}\to\overline{E}$是光滑映射,限制在$D\to E$上是保定向或者反定向的微分同胚,那么有$\int_DG^*\omega=\pm\int_E\omega$,符号取决于这个限制是保定向还是反定向的.把这个结论用在$\varphi\circ F_i:A_i\to C_i$上得到:
    	$$\int_{C_i}(\varphi^{-1})^*\omega=\int_{A_i}(\varphi\circ F_i)^*(\varphi^{-1})^*\omega=\int_{A_i}F_i^*\omega=\int_{D_i}F_i^*\omega$$
    \end{proof}
	\item 链上的Stokes定理.如果$c$是流形$M$上的光滑$p$链,设$\omega$是$M$上的$p-1$形式,那么有:
	$$\int_{\partial c}\omega=\int_c\mathrm{d}\omega$$
	\begin{proof}
		
		按照线性,归结为证明$c$本身是一个光滑奇异$p$单形$\sigma$的情况.按照带角流形上的Stokes定理,有:
		$$\int_{\sigma}\mathrm{d}\omega=\int_{\Delta_p}\sigma^*\mathrm{d}\omega=\int_{\Delta_p}\mathrm{d}\sigma^*\omega=\int_{\partial\Delta_p}\sigma^*\omega$$
		
		考虑面映射$F_{i,p}:\Delta_{p-1}\to\Delta_p$,即把$[e_0,e_1,\cdots,e_p]$映射为$[e_0,\cdots,\hat{e_i},\cdots,e_p,e_i]$,这是保定向的当且仅当$(e_0,\cdots,\hat{e_i},\cdots,e_p,e_i)$是$(e_0,\cdots,e_p)$的偶置换,当且仅当$p-i$是偶数,于是当$i$是偶数时$F_{i,p}$是保定向的,当$i$是奇数时$F_{i,p}$是反定向的.于是按照引理得到:
		$$\int_{\partial\Delta_p}=\sigma^*\omega=\sum_{i=0}^p(-1)^i\int_{\Delta_{p-1}}F_{i,p}^*\sigma^*\omega=\sum_{i=0}^p(-1)^i\int_{\Delta_{p-1}}(\sigma\circ F_{i,p})^*\omega=\sum_{i=0}^p(-1)^i\int_{\sigma\circ F_{i,p}}\omega=\int_{\partial\sigma}\omega$$
	\end{proof}
    \item 德拉姆同态.对每个非负整数$p$,定义实线性映射$J:\mathrm{H}^p_{\mathrm{dR}}(M)\to\mathrm{H}^p(M;\mathbb{R})$(后者是$\mathbb{R}$系数的奇异上同调,即$\mathrm{Hom}_{\textbf{Grp}}(\mathrm{H}_p(M),\mathbb{R})$)为,对每个$[\omega]\in\mathrm{H}^p_{\mathrm{dR}}(M)$和每个$[c]\in\mathrm{H}_p(M)\cong\mathrm{H}_p^{\infty}(M)$,定义$J[\omega][c]=\int_{\widetilde{c}}\omega$,其中$\widetilde{c}$是任意的落在同调类$[c]$的光滑$p$圈.下面验证定义良性和它的确是同态:
    \begin{itemize}
    	\item 定义不依赖于$\widetilde{c}$的选取:如果$\widetilde{c}$和$\widetilde{c}'$是两个落在相同同调类的$p$圈,那么存在$p+1$奇异链$\widetilde{b}$使得$\widetilde{c}-\widetilde{c}'=\partial\widetilde{b}$,导致:
    	$$\int_{\widetilde{c}}\omega-\int_{\widetilde{c}'}\omega=\int_{\partial\widetilde{b}}\omega=\int_{\widetilde{b}}\mathrm{d}\omega$$
    	\item 如果$\omega$是恰当形式那么:$$\int_{\widetilde{c}}\omega=\int_{\widetilde{c}}\mathrm{d}\eta=\int_{\partial\widetilde{c}}\eta=0$$
    	\item $J$是群同态因为$\int_{\widetilde{c}}\omega$关于$\widetilde{c}$和$\omega$都是加性的.
    \end{itemize}
    \item 德拉姆同态的自然性.设$M$是光滑流形,设$p$是非负整数,设$J:\mathrm{H}^p_{\mathrm{dR}}(M)\to\mathrm{H}^p(M;\mathbb{R})$是德拉姆同态.
    \begin{itemize}
    	\item 如果$F:M\to N$是光滑映射,那么有如下交换图表:
    	$$\xymatrix{\mathrm{H}_{\mathrm{dR}}^p(N)\ar[rr]^{F^*}\ar[d]_J&&\mathrm{H}_{\mathrm{dR}}^p(M)\ar[d]^J\\\mathrm{H}^p(N;\mathbb{R})\ar[rr]_{F^*}&&\mathrm{H}^p(M;\mathbb{R})}$$
    	\item 如果$U,V$是$M$的开子集并且覆盖了整个$M$,那么有如下交换图表,其中$\delta$和$\partial^*$分别是德拉姆MV序列和实系数奇异MV序列的连接映射(这里$\mathbb{R}$系数奇异上同调满足MV定理是因为$\mathbb{R}$是内射群,于是$\mathrm{Hom}_{\textbf{Grp}}(-,\mathbb{R})$是逆变的正合函子,于是奇异同调的MV序列作用这个正合函子得到实系数奇异上同调的MV定理成立).
    	$$\xymatrix{\mathrm{H}_{\mathrm{dR}}^{p-1}(U\cap V)\ar[rr]^{\delta}\ar[d]_J&&\mathrm{H}_{\mathrm{dR}}^p(M)\ar[d]^J\\\mathrm{H}^{p-1}(U\cap V;\mathbb{R})\ar[rr]_{\partial^*}&&\mathrm{H}^p(M;\mathbb{R})}$$
    \end{itemize}
    \begin{proof}
    	
    	第一个图表是容易的,任取$M$上的光滑奇异$p$单形$\sigma$,任取$N$上的$p$形式$\omega$,那么有:
    	$$\int_{\sigma}F^*\omega=\int_{\Delta_p}\sigma^*F^*\omega=\int_{\Delta_p}(F\circ\sigma)^*\omega=\int_{F\circ\sigma}\omega$$
    	
    	于是得到:
    	$$J(F^*[\omega])[\sigma]=J[\omega][F\circ\sigma]=J[\omega](F_*[\sigma])=F^*(J[\omega])[\sigma]$$
    	
    	第二个图表.任取$[\omega]\in\mathrm{H}_{\mathrm{dR}}^{p-1}(U\cap V)$,任取$[e]\in\mathrm{H}_p(M)$,那么我们要证的交换图表为$J(\delta[\omega])[e]=J[\omega](\partial_*[e])$.设$\sigma$是落在同调类$\delta[\omega]$中的光滑$p$形式,设$c$是落在同调类$\partial_*[e]$中的光滑$p-1$链.那么我们要证的等式为$\int_e\sigma=\int_c\omega$.
    	
    	按照奇异同调MV序列连接映射的描述,可记$c=\partial f$,其中$c$在$U,V$上的限制分别是$\partial f=-\partial f'$,其中$f,f'$分别是$U,V$上的$p$链,并且$f+f'$和$e$落在相同的同调类中.再按照德拉姆上同调的MV序列连接映射的描述,存在$\eta\in\Omega^{p-1}(U)$和$\eta'\in\Omega^{p-1}(V)$,使得$\omega=\eta\mid U\cap V-\eta'\mid U\cap V$,那么$U$上的$\mathrm{d}\eta$和$V$上的$\mathrm{d}\eta'$可粘合为$M$上的整体$p$形式$\sigma$.
    	
    	于是按照$\partial f+\partial f'=\partial e=0$和$\mathrm{d}\eta\mid U\cap V-\mathrm{d}\eta'\mid U\cap V=\mathrm{d}\omega=0$,得到:
    	\begin{align*}
    	\int_c\omega&=\int_{\partial f}\omega=\int_{\partial f}\eta-\int_{\partial f}\eta'=\int_{\partial f}\eta+\int_{\partial f'}\eta'\\&=\int_f\mathrm{d}\eta+\int_{f'}\mathrm{d}\eta'=\int_f\sigma+\int_{f'}\sigma=\int_e\sigma
    	\end{align*}
    \end{proof}
    \item 德拉姆定理.设$M$是光滑流形,对每个非负整数$p$,德拉姆同态$J:\mathrm{H}^p_{\mathrm{dR}}(M)\to\mathrm{H}^p(M;\mathbb{R})$都是同构.
    \begin{proof}
    	
    	如果一个光滑流形满足对每个非负整数$p$,都有德拉姆同态是同构,我们暂且称它为德拉姆流形.按照$J$和光滑映射诱导的上同调同态可交换,说明和德拉姆流形微分同胚的光滑流形仍然是德拉姆流形.设$M$光滑流形,一个开覆盖$\{U_i\}$称为德拉姆覆盖,如果每个$U_i$是德拉姆流形,并且这些开子集任意的有限交$U_{i_1}\cap U_{i_2}\cap\cdots\cap U_{i_k}$都是德拉姆流形.如果流形上的一组拓扑基作为开覆盖还是德拉姆覆盖,就称它是德拉姆基.我们的目标就是证明每个光滑流形都是德拉姆流形.
    	
    	第一步,证明至多可数个德拉姆流形$\{M_i\}$的无交并仍然是德拉姆流形.我们知道奇异同调函子和直积可交换,按照$\mathrm{Hom}_{\textbf{Grp}}(-,\mathbb{R})$是正合函子,实系数的奇异上同调函子和无交并可交换.我们之前还证明过德拉姆上同调函子也和无交并可交换.于是按照$J$的自然性得到$\coprod_iM_i$是德拉姆流形:
    	$$\xymatrix{\mathrm{H}^p_{\mathrm{dR}}(\coprod_iM_i)\ar[rr]^{\sim}\ar[d]_J&&\prod_i\mathrm{H}^p_{\mathrm{dR}}(M_i)\ar[d]^J_{\sim}\\\mathrm{H}^p(\coprod_iM_i;\mathbb{R})\ar[rr]^{\sim}&&\prod_i\mathrm{H}^p(M_i;\mathbb{R})}$$
    	
    	第二步,证明$\mathbb{R}^n$的凸开集$U$总是德拉姆流形.按照庞加莱引理有$p\not=0$时$\mathrm{H}^p_{\mathrm{dR}}(U)=0$.按照$U$同伦等价于单点空间,它的实系数奇异上同调依旧满足$p\not=0$时$\mathrm{H}^p(U;\mathbb{R})=0$.零次的时候,我们证明过$\mathrm{H}^0_{\mathrm{dR}}(U)$是由实常值函数构成的一维实空间.而对于奇异上同调有$\mathrm{H}^0(U;\mathbb{R})=\mathrm{Hom}(\mathrm{H}_0(U),\mathbb{R})\cong\mathbb{R}$是被任意奇异0单形生成的一维实空间(任意奇异0单形光滑因为零维流形总是光滑的).而如果$\sigma:\Delta_0\to M$是一个光滑奇异0单形(光滑),而$f$是恒为1的常值函数,那么有如下等式,于是$p=0$时德拉姆同态$J$不是零映射,但是它是两个1维实空间之间的实线性映射,这不是零映射就必然是同构.
    	$$J[f][\sigma]=\int_{\Delta_0}\sigma^*f=(f\circ\sigma)(0)=1$$
    	
    	第三步,证明存在有限德拉姆覆盖的光滑流形是德拉姆流形.设$M=U_1\cup\cdots\cup U_k$是有限开覆盖,并且这里每个$U_i$和这些$U_i$的任意交都是德拉姆流形.我们来对$k$归纳.$k=1$没什么需要证的,下证明$k=2$成立.设$M$有有限德拉姆覆盖$\{U,V\}$.按照这里的两种上同调都存在MV序列,以及德拉姆同态的自然性,得到如下图表交换:
    	$$\xymatrix{\mathrm{H}^{p-1}_{\mathrm{dR}}(U)\oplus\mathrm{H}^{p-1}_{\mathrm{dR}}(V)\ar[r]\ar[d]&\mathrm{H}^{p-1}_{\mathrm{dR}}(U\cap V)\ar[r]\ar[d]&\mathrm{H}^p_{\mathrm{dR}}(M)\ar[r]\ar[d]&\mathrm{H}^p_{\mathrm{dR}}(U)\oplus\mathrm{H}^p_{\mathrm{dR}}(V)\ar[r]\ar[d]&\mathrm{H}^p_{\mathrm{dR}}(U\cap V)\ar[d]\\\mathrm{H}^{p-1}(U;\mathbb{R})\oplus\mathrm{H}^{p-1}(V;\mathbb{R})\ar[r]&\mathrm{H}^{p-1}(U\cap V;\mathbb{R})\ar[r]&\mathrm{H}^p(M;\mathbb{R})\ar[r]&\mathrm{H}^p(U;\mathbb{R})\oplus\mathrm{H}^p(V;\mathbb{R})\ar[r]&\mathrm{H}^p(U\cap V;\mathbb{R})}$$
    	
    	按照$U,V,U\cap V$都是德拉姆流形,这个交换图表的前两个和后两个垂直映射都是同构.按照短五引理得到$\mathrm{H}^p_{\mathrm{dR}}(M)\to\mathrm{H}^p(M;\mathbb{R})$是同构,这得到$k=2$成立.
    	
    	最后假设命题对正整数$k$成立.任取$M$的德拉姆覆盖$\{U_1,U_2,\cdots,U_{k+1}\}$,记$U=U_1\cup\cdots\cup U_k$和$V=U_{k+1}$,按照归纳假设有$U$也是德拉姆流形.而$U\cap V$上存在长度$k$的有限德拉姆覆盖$\{U_1\cap U_{k+1},\cdots,U_k\cap U_{k+1}\}$,归纳假设说明$U\cap V$也是德拉姆流形,于是$\{U,V\}$是$M$的德拉姆覆盖,于是$M$是德拉姆流形,完成归纳.
    	
    	第四步,证明如果$M$存在德拉姆基,那么它是德拉姆流形.设$M$存在德拉姆基$\{U_i\}$,取$M$上的exhaustion函数$f:M\to\mathbb{R}$.对每个整数$m$,记$A_m=f^{-1}([m,m+1])$和$A_m'=f^{-1}((m-\frac{1}{2},m+\frac{3}{2}))$.对每个点$q\in A_m$,都存在基元素覆盖点$q$并且包含在$A_m'$中.这些基元素就构成了$A_m$的开覆盖.按照$f$是exhaustion函数,这些$A_m$都是紧集,于是每个$A_m$存在由这种基元素构成的有限覆盖,这些有限个基元素的并记作$B_m$,按照第三步说明$B_m$是德拉姆流形.
    	
    	另外有$B_m\subset A_m'$,说明$B_n$和$B_m$有交只能发生在$n=m-1,m,m+1$的情况.记$m$取奇数时$B_m$的并为$U$;$m$取偶数时$B_m$的并为$V$.那么$U$和$V$都是德拉姆流形的无交并,第一步说明它们都是德拉姆流形.下面每个$B_m\cap B_{m+1}$都存在由形如$U_i\cap U_j$的开集构成的有限德拉姆覆盖,这里$U_i$任取用来定义$B_m$的那有限个基元素,$U_j$任取用来定义$B_{m+1}$的那有限个基元素.这说明每个$B_m\cap B_{m+1}$都是德拉姆流形.而$U\cap V$是$B_m\cap B_{m+1},m\in\mathbb{Z}$的无交并,第一步就说明$U\cap V$是德拉姆流形.按照第三步就说明$M=U\cup V$是德拉姆流形.
    	
    	第五步,证明欧氏空间的开子集都是德拉姆流形.任取开子集$U\subset\mathbb{R}^n$,它存在由开球构成的拓扑基,因为开球和有限个开球的交都是凸集,说明这个拓扑基是德拉姆基,说明第四步说明$U$是德拉姆流形.
    	
    	第六步,证明任意光滑流形都是德拉姆流形.任取光滑流形$M$的由光滑坐标卡构成的拓扑基,每个坐标开集都微分同胚于欧氏空间的开子集,有限个坐标开集的交也微分同胚于欧氏空间的开子集,于是第五步说明它们都是德拉姆流形,于是光滑坐标卡构成的拓扑基是德拉姆基,第四步就得到$M$是德拉姆流形.
    \end{proof}
\end{enumerate}
\newpage
\section{积分曲线和积分流形}
\subsection{积分曲线和流}

积分曲线.设$V$是流形$M$上的向量场,$V$的一条积分曲线是指一个可微曲线$\gamma:J\to M$,使得$\forall t\in J$有$\gamma'(t)=V_{\gamma(t)}$.
\begin{enumerate}
	\item 为了求$V$上的积分曲线$\gamma$,选取一个光滑坐标卡$U\subset M$,那么$\gamma$在局部坐标下记作$\gamma(t)=(\gamma^1(t),\gamma^2(t),\cdots,\gamma^n(t))$,那么$\gamma'(t)=V_{\gamma(t)}$在坐标下可以写作$\sum_i(\gamma^i)'(t)\frac{\partial}{\partial x^i}\mid_{\gamma(t)}=\sum_iV^i(\gamma(t))\frac{\partial}{\partial x^i}\mid_{\gamma(t)}$.也即如下常微分方程组:
	$$\left\{\begin{array}{c}(\gamma^1)'(t)=V^1(\gamma^1(t),\gamma^2(t),\cdots,\gamma^n(t))\\\vdots\\(\gamma^n)'(t)=V^1(\gamma^1(t),\gamma^2(t),\cdots,\gamma^n(t))\end{array}\right.$$
	
	按照常微分方程基本定理,给定流形上的一个点$p\in M$,总存在过该点的关于$V$的积分曲线.并且这样的曲线具有唯一性:如果额外约定固定初始条件$t_0\in\mathbb{R},\gamma(t_0)=p\in M$,如果有两条过点$p$关于$V$的积分曲线,那么在它们公共定义域上是相同的.
	\item (相同定义区间上)积分曲线的唯一性.如果$\gamma,\widetilde{\gamma}:J\to M$是$M$上向量场$V$的两条积分曲线,并且存在某个$t_0\in J$使得$\gamma(t_0)=\widetilde{\gamma}(t_0)$.那么有$\gamma=\widetilde{\gamma}$.
	\begin{proof}
		
		设$S$表示全部满足$\gamma(t)=\widetilde{\gamma(t)}$的点$t\in J$.那么$S$非空的,并且连续性保证$S$是$J$的闭子集.另一方面,任取$t_1\in S$,那么在点$p=\gamma(t_1)$附近的坐标卡上,$\gamma$和$\widetilde{\gamma}$是同一个常微分方程组的解.按照常微分方程组的唯一性定理,在$t_1$的附近这两个解是相同的,这说明$S$是$J$的开子集.按照$J$是区间,它是连通的,于是$S=J$.这说明有$\gamma=\widetilde{\gamma}$.
	\end{proof}
	\item 设$V$是流形$M$上的向量场,设$\gamma:J\to M$是关于$V$的一条积分曲线.
	\begin{itemize}
		\item 任取$a\in\mathbb{R}$,那么$\widetilde{\gamma}=\gamma(at):\widetilde{J}=\{t\mid at\in J\}\to M$是$aV$的积分曲线.
		\item 任取$b\in\mathbb{R}$,那么$\widetilde{\gamma}=\gamma(t+b):\widetilde{J}=\{t\mid t+b\in J\}\to M$也是$V$的积分曲线.
	\end{itemize}
	\item 积分曲线的自然性.设$F:M\to N$是光滑映射,设$X,Y$分别是$M,N$上的向量场.那么$X,Y$是$F$相关的当且仅当$F$把$X$上的积分曲线映为$Y$上的积分曲线.
	\begin{proof}
		
		必要性是直接的.任取$X$上的积分曲线$\gamma:J\to M$,那么有:
		$$(F\circ\gamma)'(t)=\mathrm{d}F_{\gamma(t)}(\gamma'(t))=\mathrm{d}F_{\gamma(t)}(X_{\gamma(t)})=Y_{F(\gamma(t))}$$
		
		充分性.假设$F$把$X$上积分曲线映为$Y$上积分曲线,任取$p\in M$,任取$X$的积分曲线$\gamma:(-\varepsilon,\varepsilon)\to M$满足$\gamma(0)=p$,那么$F\circ\gamma$是$Y$上积分曲线,于是有:
		$$Y_{F(p)}=(F\circ\gamma)'(0)=\mathrm{d}F_p(\gamma'(0))=\mathrm{d}F_p(X_p)$$
	\end{proof}
\end{enumerate}

整体流(global flows).设$M$是流形.
\begin{enumerate}
	\item 我们先体会一下流的概念.假设$V$是$M$上的向量场,假设对每个点$p\in M$可取到以该点为起始点(即$\gamma(0)=p$)的关于$V$的积分曲线,并且假设积分曲线的定义域可以取为整个$\mathbb{R}$,记做$\theta^{(p)}:\mathbb{R}\to M$.那么我们可以定义一个作用$\theta:\mathbb{R}\times M\to M$为$(t,p)\mapsto\theta^{(p)}(t)$.这里$t\in\mathbb{R}$可以视为这个积分曲线族的统一时间.那么$\theta(t,-)$就是在时间$t$时这个积分曲线族的一个切片.我们之前解释过$t\mapsto\theta^{(p)}(t+s)$是一个以$q=\theta^{(p)}(s)$为起始点的积分曲线,于是积分曲线的唯一性导致$\theta^{(q)}(t)=\theta^{(p)}(t+s)$,也即$\theta(t,\theta(s,p))=\theta(t+s,p)$.这说明我们构造的$\theta$是$\mathbb{R}$在$M$上的群作用.
	\item $M$上的一个连续整体流定义为$\mathbb{R}$在$M$上的一个连续左作用.具体的讲整体流是一个连续映射$\theta:\mathbb{R}\times M\to M$,满足对任意$s,t\in\mathbb{R}$和$p\in M$,都有:
	$$\theta(t,\theta(s,p))=\theta(t+s,p)$$
	$$\theta(0,p)=p$$
	
	约定当我们提及整体流时特指这个映射是光滑的.
	\item 对每个$t\in\mathbb{R}$,记连续映射$\theta_t:M\to M$为$\theta_t(p)=\theta(t,p)$.于是它满足如下群的性质:$$\theta_t\circ\theta_s=\theta_{t+s};\theta_0=\mathrm{id}_M$$
	
	于是特别的每个$\theta_t:M\to M$都是同胚.如果整体流还具有光滑性那么每个$\theta_t$都是微分同胚.
	\item 对每个$p\in M$,记曲线$\theta^{(p)}:\mathbb{R}\to M$为$\theta^{(p)}(t)=\theta(t,p)$.那么这条曲线的像就是点$p$在群作用下的轨道.
	\item 设$\theta$是$M$上的整体流,对每个点$p\in M$取$V_p=(\theta^{(p)})'(0)\in\mathrm{T}_pM$,它们构成的向量场$V$是一个光滑向量场,并且每个$\theta^{(p)}$都是$V$的积分曲线.我们称$V$是$\theta$的无穷小生成元.换句话讲整体流肯定是按照第一条的方式从积分曲线族定义出来的.
	\begin{proof}
		
		为证明$V$是光滑的,任取开集$U\subset M$上的光滑函数$f$,只需验证$Vf$是$U$上的光滑函数.任取$p\in U$有:$$Vf(p)=V_pf=(\theta^{(p)})'(0)f=\frac{\mathrm{d}}{\mathrm{d}t}\mid_{t=0}f(\theta^{(p)}(t))=\frac{\partial}{\partial}\mid_{(0,p)}f(\theta(t,p))$$
		
		这里$f(\theta(t,p))$是光滑的,所以它的偏导数也是光滑的.接下来要证明每条$\theta^{(p)}$都是$V$的积分曲线,也即证明$(\theta^{(p)})'(t)=V_{\theta^{(p)}(t)}$对任意$t\in\mathbb{R}$和任意$p\in M$成立.任取$t_0\in\mathbb{R}$,记$q=\theta^{(p)}(t_0)=\theta_{t_0}(p)$,于是只需证明$(\theta^{(p)})'(t_0)=V_q$.对任意$t$我们有:
		$$\theta^{(q)}(t)=\theta_t(q)=\theta_t(\theta_{t_0}(p))=\theta_{t+t_0}(p)=\theta^{(p)}(t+t_0)$$
		
		于是对任意的$q$附近的光滑函数$f$,就有:
		\begin{align*}
		V_qf&=(\theta^{(q)})'(0)f=\frac{\mathrm{d}}{\mathrm{d}t}\mid_{t=0}f(\theta^{(q)}(t))=\frac{\mathrm{d}}{\mathrm{d}t}\mid_{t=0}f(\theta^{(p)}(t+t_0))\\&=(\theta^{(p)})'(t_0)f
		\end{align*}
	\end{proof}
\end{enumerate}

局部流(local flows).
\begin{enumerate}
	\item 我们已经证明了每个光滑整体流对应于一个向量场,也即它的无穷小生成元.反过来我们希望证明每个光滑向量场都是某个整体流的无穷小生成元.但是这未必总成立,因为存在这样的向量场,它的某些积分曲线不是定义在整个$\mathbb{R}$上的.为此我们要缩小流的定义域.
	\item 定义.设$M$是流形,它的一个流定义域是指积空间的一个开子集$D\subset\mathbb{R}\times M$,满足对任意$p\in M$,几何$D^{(p)}=\{t\in\mathbb{R}\mid(t,p)\in D\}$是包含0的$\mathbb{R}$上的开区间.$M$上的一个连续/光滑局部流或者简称流,是指一个连续/光滑映射$\theta:D\to M$,其中$D\subset\mathbb{R}\times M$是一个流定义域,满足:
	\begin{itemize}
		\item 对每个$p\in M$有$\theta(0,p)=p$.
		\item 对任意$s\in D^{(p)}$和$t\in D(\theta(s,p))$,使得$s+t\in D^{(p)}$,都有:
		$$\theta(t,\theta(s,p))=\theta(t+s,p)$$
	\end{itemize}
	
	如果$\theta$是流,记$\theta_t(p)=\theta^{(p)}(t)=\theta(t,p)$,其中$(t,p)\in D$.再对每个$t\in\mathbb{R}$记$M_t=\{p\in M\mid(t,p)\in D\}$.于是$p\in M_t\Leftrightarrow t\in D^{(p)}\Leftrightarrow(t,p)\in D$.最后我们约定提及流的时候总是指光滑的,除非加以说明.
	\item 如果$\theta:D\to M$是流,其中$D$是流定义域,那么$D\subset\mathbb{R}\times M$在$M$分量的投影$U$是$M$的开子集.对每个$p\in U$,记$V_p=(\theta^{(p)})'(0)$,那么$V$是$U$上的光滑向量场,并且每个$\theta^{(p)}$都是$V$的积分曲线.我们称这里的$V$为流$\theta$的无穷小生成元.
	\item 极大积分曲线和极大流.点$p\in M$为起始点的积分曲线称为极大的,如果它不能延拓为定义域更大的积分曲线.一个流称为极大的,如果它不能延拓为流定义域更大的流.
	\item 流的基本定理.设$V$是$M$上的整体向量场,那么存在唯一的极大流$\theta:D\to M$,满足它的无穷小生成元是$V$(此时$D$在$M$分量的投影肯定是整个$M$).这个$\theta$称为被$V$生成的流,或者称为$V$的流.关于$V$的流$\theta$我们还有如下结论:
	\begin{itemize}
		\item 对每个$p\in M$,有$\theta^{(p)}:D^{(p)}\to M$是唯一的以$p$为起始点的关于$V$的极大积分曲线.
		\item 如果$s\in D^{(p)}$,那么$D^{\theta(s,p)}$就是区间$D^{(p)}-s=\{t-s\mid t\in D^{(p)}\}$.
		\item 对每个$t\in\mathbb{R}$都有$M_t$是$M$的开子集,并且$\theta_t:M_t\to M_{-t}$是微分同胚,它的逆映射是$\theta_{-t}$.
	\end{itemize}
	\begin{proof}
		
		我们已经解释过任取点$p\in M$都存在以该点为起始点的积分曲线.对每个点$p\in M$,我们把所有以$p$为起始点的积分曲线的定义区间并起来,得到一个开区间记作$D^{(p)}$.按照积分曲线的唯一性,以$p$为起始点的积分曲线在公共定义区间上是相同的.于是这些以$p$为起始点的积分曲线可以粘合为$D^{(p)}$上的一条以$p$为起始点的积分曲线,记作$\theta^{(p)}$.它明显是以$p$为起始点的关于$V$的极大积分曲线.
		
		现在构造$D=\{(t,p)\in\mathbb{R}\times M\mid t\in D^{(p)}\}$.构造$\theta:D\to M$为$\theta(t,p)=\theta^{(p)}(t)$.我们要证明$D$是$\mathbb{R}\times M$的开子集,并且$\theta$是光滑的.考虑这样的点$(t,p)\in D$,使得它存在一个乘积开邻域$J\times U\subset D$,其中$t$是包含0的开区间,而$U$是$p$在$M$中的一个开邻域,满足$\theta$在这个乘积开邻域上的限制是光滑的,这些点$(t,p)$构成的$D$的子集记作$W$.那么$W$是$\mathbb{R}\times M$的开子集,并且$\theta$在$W$上的限制是光滑的.于是问题归结为证明$D=W$.假设这不成立,可以取$(\tau,p_0)\in D-W$,不妨设$\tau>0$,当$\tau<0$时证明是类似的.
		
		记$t_0=\inf\{t\in\mathbb{R}^+\mid(t,p_0)\not\in W\}$.按照$\theta$肯定在$(0,p_0)$附近光滑,说明$t_0\not=0$.于是有$0<t_0\le\tau$.但是$D^{(p_0)}$要包含点$\tau$和0,导致$t_0\in D^{(p_0)}$.记$q_0=\theta^{(p_0)}(t_0)$.存在$\varepsilon>0$和$q_0$的开邻域$U_0$,使得$(-\varepsilon,\varepsilon)\times U_0\subset W$.我们来把$\theta$延拓到$(t_0,p_0)$附近,从而和$t_0$的选择矛盾.从而证明$W=D$.
		
		选取$t_1<t_0$使得$t_1+\varepsilon>t_0$和$\theta^{(p_0)}(t_1)\in U_0$.按照$t_0$的极小性得到$(t_1,p_0)\in W$,于是存在乘积开邻域$(t_1-\delta,t_1+\delta)\times U_1\subset W$.按照$W$的定义,说明$\theta$在$[0,t_1+\delta)\times U_1$上光滑.按照$\theta(t_1,p_0)\in U_0$,可适当缩小$U_1$使得$\theta(\{t_1\}\times U_1)\subset U_0$.最后我们构造$\widetilde{\theta}:[0,t_1+\varepsilon)\times U_1\to M$为:
		$$\widetilde{\theta}(t,p)=\left\{\begin{array}{cc}\theta_t(p)&p\in U_1,0\le t<t_1\\\theta_{t-t_1}\circ\theta_{t_1}(p)&p\in U_1,t_1-\varepsilon<t<t_1+\varepsilon\end{array}\right.$$
		
		$\theta$满足的群法则(在下面证明)保证了这两个分段相交的时候相同.而这个映射本身是光滑的.并且我们解释过$t\mapsto\widetilde{\theta}(t,p)$是$V$的积分曲线.这就说明$\widetilde{\theta}$是$\theta$的延拓.这和$t_0$的定义矛盾,于是$W=D$.
		
		验证$\theta$满足群法则.固定点$p\in M$和$s\in D^{(p)}$,记$q=\theta(s,q)=\theta^{(p)}(s)$.曲线$\gamma(t)=\theta^{(p)}(t+s):D^{(p)}-s\to M$是以$q$为起始点的,并且它也是积分曲线.于是唯一性保证了$\gamma$和$\theta$在公共定义域是相同的.也即$\theta(t,\theta(s,p))=\theta(t+s,p)$.群法则的另一条$\theta(0,p)=p$直接从定义得到.
		
		现在验证命题中的三件事.第一件事我们已经解释过了,并且据此说明$\theta$本身是一个极大流,否则如果它还能延拓,构成它的某个积分曲线就应该能继续延拓,但是这些积分曲线本身已经都是极大的了.第二件事,按照积分曲线$\theta^{(q)}$的极大性,按照$\gamma(t)=\theta^{(p)}(t+s):D^{(p)}-s\to M$和它在公共定义域上相同,得到$D^{(p)}-s\subset D^{(q)}$.从$0\in D^{(p)}$得到$-s\in D^{(q)}$.群法则得到$\theta^{(q)}(-s)=p$.类似得到$D^{(q)}\subset D^{(p)}-s$.这证明了第二件事.最后证明第三件事:从$D$是开集得到每个$M_t$是开集.第二件事说明,任取$p\in M_t$,那么$t\in D^{(p)}$,于是$D^{\theta_t(p)}=D^{(p)}-t$,于是$-t\in D^{\theta_t(p)}$,于是$\theta_t(p)\in M_{-t}$.这说明$\theta_t$把$M_t$映入$M_{-t}$.群法则保证$\theta_{-t}\circ\theta_t$和$\theta_t\circ\theta_{-t}$分别是$M_t$和$M_{-t}$上的恒等映射,这说明该映射是微分同胚.
	\end{proof}
	\item 设$F:M\to N$是光滑映射,设$X,Y$分别是$M$和$N$上的整体向量场,设$\theta$和$\eta$分别是$X$和$Y$的流.如果$X,Y$是$F$相关的,那么对每个$t\in\mathbb{R}$,都有$F(M_t)\subset N_t$和$\eta_t\circ F=F\circ\theta_i$.换句话讲有如下交换图表:
	$$\xymatrix{M_t\ar[rr]^F\ar[d]_{\theta_t}&&N_t\ar[d]^{\eta_t}\\M_{-t}\ar[rr]^F&&N_{-t}}$$
	\begin{proof}
		
		我们解释过对每个$p\in M$有$F\circ\theta^{(p)}$是$Y$的积分曲线,并且以$F\circ\theta^{(p)}(0)=F(p)$为起始点.按照积分曲线的唯一性,极大积分曲线$\eta^{(F(p))}$至少在$D^{(p)}$上有意义,并且在其上满足$F\circ\theta^{(p)}=\eta^{(F(p))}$.于是有:
		$$p\in M_t\Rightarrow t\in D^{(p)}\Rightarrow t\in D^{(F(p))}\Rightarrow F(p)\in N_t$$
		
		此即$F(M_t)\subset N_t$.另外对每个$t\in D^{(p)}$有$F(\theta^{(p)}(t))=\eta^{(F(p))}(t)$就得到$\eta_t\circ F(p)=F\circ\theta_t(p),\forall p\in M_t$.
	\end{proof}
	\item 特别的上一条说明,如果$F:M\to N$是微分同胚,如果$X$是$M$上向量场,$\theta$是$X$的流,那么$F_*X$的流就是$\eta_t=F\circ\theta_t\circ F^{-1}$,并且对每个$t\in\mathbb{R}$满足$N_t=F(M_t)$.
	\item 流和子流形,Flowout定理.设$S\subset M$是$k$维嵌入子流形,设$V$是$M$上的向量场,并且处处不相切于$S$.设$\theta:D\to M$是$V$的流,记$O=(\mathbb{R}\times S)\cap D$,记$\Phi=\theta\mid O$.
	\begin{itemize}
		\item $\Phi:O\to M$是浸入.
		\item $\partial/\partial t$是$O$上的和$V$是$\Phi$相关的向量场.
		\item 存在正光滑函数$\delta:S\to\mathbb{R}$,使得$\Phi$在$O_{\delta}$上的限制是单射.这里$O_{\delta}=\{(t,p)\in O\mid|t|<\delta(p)\}$是一个流定义域.于是$\Phi(O_{\delta})$是包含了$S$的$M$的浸入子流形,它称为$S$沿$V$的flowout,另外$V$和这个子流形相切.
		\item 如果$S$具有余维数1,那么$\Phi\mid O_{\delta}$是到$M$的某个开子集的微分同胚.
	\end{itemize}
    \begin{proof}
    	
    	先证明第二件事,固定$p\in S$,选取$\sigma:D^{(p)}\to\mathbb{R}\times S$为$t\mapsto(t,p)$.那么有$\Phi\circ\sigma(t)=\theta(t,p)$是$V$上的积分曲线.于是对任意$t_0\in D^{(p)}$就有:
    	$$\mathrm{d}\Phi_{(t_0,p)}\left(\frac{\partial}{\partial t}\mid_{(t_0,p)}\right)=(\Phi\circ\sigma)'(t_0)=V_{\Phi(t_0,p)}$$
    	
    	证明第一件事,首先$\Phi$在$\{0\}\times S$上的限制是微分同胚$\{0\}\times S\cong S$和嵌入$S\subset M$的复合,于是这个限制映射是一个光滑嵌入.所以$\mathrm{d}\Phi_{(0,p)}$在$\mathrm{T}_pS$(视为$\mathrm{T}_{(0,p)}O\cong\mathrm{T}_0\mathbb{R}\oplus\mathrm{T}_pS$的子空间)上的限制是包含映射$\mathrm{T}_pS\subset\mathrm{T}_pM$.记$\mathrm{T}_pS$的一组基为$\{E_1,E_2,\cdots,E_k\}$,那么$\mathrm{T}_{(0,p)}O$的一组基为$\{\partial/\partial t\mid_{(0,p)},E_1,\cdots,E_k\}$.微分$\mathrm{d}\Phi_{(0,p)}$把这组基映射为$\{V_p,E_1,\cdots,E_k\}$.按照$V_p$和$S$不相切,说明这$k+1$个向量是线性无关的,于是我们证明了$\mathrm{d}\Phi_{(0,p)}$是单射.下证$\mathrm{d}\Phi$总是单射.任取$(t_0,p_0)\in O$,取$\tau_{t_0}:O\to\mathbb{R}\times S$是平移映射$\tau_{t_0}(t,p)=(t+t_0,p)$.那么在映射有意义的前提下,按照$\theta$满足群法则说明有如下交换图表:
    	$$\xymatrix{O\ar[rr]^{\tau_{t_0}}\ar[d]_{\Phi}&&O\ar[d]^{\Phi}\\M\ar[rr]_{\theta_{t_0}}&&M}$$
    	
    	这里两个水平的映射都是局部微分同胚,所以取微分后它们都变成同构.我们证明了左侧垂直映射是单射,于是右侧也是单射.这说明$\mathrm{d}\Phi$总是单射.
    	$$\xymatrix{\mathrm{T}_{(0,p_0)}O\ar[rr]^{\mathrm{d}(\tau_{t_0})_{(0,p_0)}}\ar[d]_{\mathrm{d}\Phi_{(0,p_0)}}&&\mathrm{T}_{(t_0,p_0)}O\ar[d]^{\mathrm{d}\Phi_{(t_0,p_0)}}\\\mathrm{T}_{p_0}M\ar[rr]_{\mathrm{d}(\theta_{t_0})_{p_0}}&&\mathrm{T}_{\Phi(t_0,p_0)}M}$$
    	
    	证明第三件事.给定点$p_0\in S$,选取以$p_0$为中心的切片坐标卡$(U,(x^i))$,此即有$U\cap S$是$U$的满秩$x^{k+1}=\cdots=x^n=0$的点构成的子集,其中$n=\dim M$.按照$V$不和$S$相切,所以$V_{p_0}$后$n-k$个分量中必然有某个非零,记作$V^j(p_0)$不是零.适当缩小$U$,我们可以约定存在正实数$c$使得$|V^j(p)|>c,\forall p\in U$.按照$\Phi^{-1}(U)$是$\mathbb{R}\times S$的开子集,可选取$\varepsilon_{p_0}>0$和$p_0$在$S$中的开邻域$W_{p_0}$,使得$(-\varepsilon_{p_0},\varepsilon_{p_0})\times W_{p_0}\subset O$,并且这个开集在$\Phi$下的像落在$U$中.记$\Phi(t,p)$的坐标表示为$(\Phi^1(t,p),\Phi^2(t,p),\cdots,\Phi^n(t,p))$.因为$\Phi$是$V$的流在$S$上的限制,所以这些分量函数$\Phi^j$都满足$\frac{\partial\Phi^j}{\partial t}(t,p)=V^j(\Phi(t,p)),\Phi^j(0,p)=0$.按照微积分基本定理,从$|V^j(p)|\ge c$就得到$|\Phi^j(t,p)|\ge c|t|$.于是对于$(t,p)\in(-\varepsilon_{p_0},\varepsilon_{p_0})\times W_{p_0}$就有$\Phi(t,s)\in S$当且仅当$t=0$.
    	
    	选取$S$上开覆盖$\{W_p,p\in S\}$的单位分解$\{\psi_p,p\in S\}$,构造光滑正函数$f:S\to\mathbb{R}$为$f(q)=\sum_{p\in S}\varepsilon_p\psi_p(q)$.对每个$q\in S$,只存在有限个$p\in S$使得$\psi_p(q)\not=0$,选取这有限个点中使得$\varepsilon_{p}$最大的$p_0$,那么有$f(q)\le\varepsilon_{p_0}$.所以只要$(t,q)\in O$,并且$|t|<f(q)$,就有$(t,q)\in(-\varepsilon_{p_0},\varepsilon_{p_0})\times W_{p_0}$.于是我们证明了$\Phi(t,q)\in S$当且仅当$t=0$.
    	
    	取$\delta=f/2$,我们来证明$\Phi\mid O_{\delta}$是单射.假设有$(t,q),(t',q')\in O_{\delta}$使得$\Phi(t,q)=\Phi(t',q')$.不妨设$f(q')\le f(q)$.取值相同的条件即$\theta_t(q)=\theta_{t'}(q')$.而$\theta$的群法则导致$\theta_{t-t'}(q)=q'\in S$.从$(t,q),(t',q')\in O_{\delta}$导致$|t-t'|\le|t|+|t'|<\frac{1}{2}f(q)+\frac{1}{2}f(q')\le f(q)$.所有$t=t'$,所有$q=q'$.
    	
    	最后证明第四件事.如果$S$具有余维数1,那么此时$\Phi\mid O_{\delta}$是相同维数的流形之间的单光滑浸入,所有它是嵌入,并且是到一个开子流形的微分同胚.
    \end{proof}
\end{enumerate}

完备向量场.我们解释过了不是每个向量场都生成了一个整体流.如果这个条件恰好成立,就称这个向量场是完备的.
\begin{enumerate}
	\item 引理.如果$V$是$M$上的整体向量场,设$\theta$是它的流.如果存在正实数$\varepsilon$满足对任意$p\in M$有$\theta^{(p)}$包含了$(-\varepsilon,\varepsilon)$,那么$V$是完备的.
	\begin{proof}
		
		假设$V$不是完备的,换句话讲存在某个$p\in M$使得$\theta^{(p)}$的定义域$D^{(p)}$不是整个$\mathbb{R}$,那么不妨设它是上有界的(下有界是类似的).记$b=\sup D^{(p)}$,记$t_0>0$满足$b-\varepsilon<t_0<b$.记$q=\theta^{(p)}(t_0)$.按照条件有$\theta^{(q)}(t)$至少在$(-\varepsilon,\varepsilon)$上有定义.定义曲线$\gamma:(-\varepsilon,t_0+\varepsilon)\to M$为:
		$$\gamma(t)=\left\{\begin{array}{cc}\theta^{(p)}(t)&-\varepsilon<t<b\\\theta^{(q)}(t-t_0)&t_0-\varepsilon<t<t_0+\varepsilon\end{array}\right.$$
		
		这两个分段的交相同是因为$\theta^{(q)}(t-t_0)=\theta_{t-t_0}(q)=\theta_{t-t_0}\circ\theta_{t_0}(p)=\theta_t(p)=\theta^{(p)}(t)$.但是这里$t_0+\varepsilon>b$,这矛盾.
	\end{proof}
	\item 紧支撑向量场总是完备的.特别的,紧光滑流形上的向量场总是完备的.
	\begin{proof}
		
		设$V$是$M$上的紧支撑向量场.记$K=\mathrm{Supp}V$.任取点$p\in K$,存在它的开邻域$U_p$,和一个正实数$\varepsilon_p$使得$V$的流至少能定义在$(-\varepsilon_p,\varepsilon_p)\times U_p$.按照紧性,可取$U_{p_1},U_{p_2},\cdots,U_{p_k}$覆盖了$K$.记$\varepsilon=\min\{\varepsilon_{p_1},\cdots\varepsilon_{p_k}\}$.这导致以$K$中点为起始点的极大积分曲线至少要定义在$(-\varepsilon,\varepsilon)$.在$K$以外恒有$V=0$,于是以$M-K$中点为起始点的积分曲线是常值的,于是自然可以把定义域延拓为整个$\mathbb{R}$.于是按照上述引理得到$V$是完备的.
	\end{proof}
	\item 李群上的左不变向量场总是完备的.
	\begin{proof}
		
		设$G$是李群,设$X$是左不变向量场,设$\theta:D\to G$是$X$的流.那么存在$\varepsilon>0$使得$\theta^{(e)}$在$(-\varepsilon,\varepsilon)$上定义.现在任取$g\in G$,由于$X$是左平移不变的,也即$X$和自身是$L_g$相关的,于是$L_g\circ\theta^{(e)}$是$X$的以$g$为起始点的积分曲线.于是只能等同于$\theta^{(g)}$.于是每个$\theta^{(g)}$都至少定义在$(-\varepsilon,\varepsilon)$上.按照引理说明$X$是完备的.
	\end{proof}
\end{enumerate}

向量场上的正则点和奇异点.如果$V$是流形$M$上的向量场,一个点$p\in M$称为$V$的奇点或者奇异点(singular points),如果$V_p=0$,否则就称为正则点(regular points).
\begin{enumerate}
	\item 以正则点和奇点为起始点的积分曲线的行为是非常不同的.设$V$是流形$M$上的向量场,设$V$的流为$\theta:D\to M$.如果点$p\in M$是$V$的奇点,那么有$D^{(p)}=\mathbb{R}$,并且$\theta^{(p)}$是一条常值曲线$\theta^{(p)}(t)\equiv p$.如果$p\in M$是正则点,那么$\theta^{(p)}:D^{(p)}\to M$是光滑浸入(从而不可能是常值曲线).
	\begin{proof}
		
		第一件事没什么好证的,如果$V_p=0$,那么$\gamma:\mathbb{R}\to M,\gamma(t)\equiv p$是以$p$为起始点的$V$的极大积分曲线,按照极大积分曲线的唯一性就得到$\theta^{(p)}=\gamma$.
		
		假设$p$使得极大积分曲线$\theta^{(p)}$不是一个浸入,我们来证明$p$必须是一个奇点.存在某个$s\in D^{(p)}$使得$(\theta^{(p)})'(s)=0$.记$q=\theta^{(p)}(s)$,那么$\theta^{(q)}$是一条常值曲线,于是$D^{(q)}=\mathbb{R}$.但是我们证明过$D^{(q)}=D^{(p)}-s$,导致$D^{(p)}=\mathbb{R}$.并且对每个$t\in\mathbb{R}$有:
		$$\theta^{(p)}(t)=\theta_t(p)=\theta_{t-s}(\theta_s(p))=\theta_{t-s}(q)=q$$
		
		取$t=0$,得到$p=q$,导致$\theta^{(p)}$是常值曲线,并且$V_p=(\theta^{(p)})'(0)=0$.
	\end{proof}
	\item 正则点附近的标准型.设$V$是流形$M$上的向量场,设$p\in M$是$V$的正则点.存在$p$附近的坐标映射$(s^i)$,使得$V$具有非常简单的坐标表示$\partial/\partial s^1$.如果$S\subset M$是嵌入超曲面,使得$p\in S$且$V_p\not\in\mathrm{T}_pS$,那么还可以让坐标满足$s^1$是局部定义$S$的函数.
	\begin{proof}
		
		选取以$p$为中心的坐标卡$(U,(x^i))$.按照$p$是正则点,在这个坐标下$V$存在某个分量满足$V^j(p)\not=0$,记$S\subset U$s $x^j=0$定义的超曲面.所以命题的前半部分可以由后半部分推出.
		
		按照$V_p\not\in\mathrm{T}_pS$,可以适当缩小$S$使得$V$处处和$S$不相切.于是按照Flowout定理,存在流定义域$O_{\delta}\subset\mathbb{R}\times S$,使得$V$的流在$O_{\delta}$上的限制是一个到$M$的某个包含$S$的开子集$W$的微分同胚,这个限制记作$\Phi$.存在$(0,p)的在$$O_{\delta}$中的乘积开邻域$(-\varepsilon,\varepsilon)\times W_0$.选取局部参数化$X:\Omega\to S$(设$S\subset M$是一个$k$维浸入子流形,它的一个局部参数化是指一个连续拓扑嵌入$X:U\to M$,其中$U\subset\mathbb{R}^k$是一个开子集,它的像是$S$的开子集),这里$\Omega$是$\mathbb{R}^{n-1}$的开子集,把坐标记作$\{s^2,\cdots,s^n\}$.考虑映射$\Psi:(-\varepsilon,\varepsilon)\times\Omega\to M$为$\Psi(t,s^2,\cdots,s^n)=\Phi(t,X(s^2,\cdots,s^n))$.这是到$p\in M$某个开邻域的微分同胚.这个微分同胚下$\partial/\partial t$的前推是自身,并且$\Phi_*(\partial/\partial t)=V$,导致$\Psi^{-1}$是一个使得$V$具有坐标表示$\partial/\partial t$的坐标映射.
	\end{proof}
\end{enumerate}


李导数.在欧氏空间上我们可以定义向量场的方向导数,结果仍然是一个向量场.即对$\mathbb{R}^n$上的向量场$W$,对方向$v$,定义$W$在方向$v$的导数$D_vW$是$D_vW(p)=\lim_{t\to0}\frac{W_{p+tv}-W_p}{t}=\sum_iD_vW^i(p)\frac{\partial}{\partial x^i}\mid_p$.对于一般流形,一个看起来合理的思路是把$p+tv$替换为起始点为$p$,起始点的微分为$v$的曲线$\gamma$.但是现在$W_{\gamma(t)}$和$W_{\gamma(0)}$是不同切空间中的向量,没办法做减法,欧氏空间的情况合理是因为任一点的切空间都是典范同构于$\mathbb{R}^n$的.所以这个方案不行.解决办法是不仅考虑$p$处的一个切向量$v$,要考虑一整个向量场$V$,借助流来定义向量场$W$关于$V$的导数,称为李导数.
\begin{enumerate}
	\item 设$W$是向量场,设$V$是向量场,并且$V$的流记作$\theta$,定义$W$关于$V$的李导数$\mathscr{L}_VW$为如下向量场.$(\mathscr{L}_VW)_p$可以视为$\mathrm{T}_pM$中依赖时间的向量场$\mathrm{d}(\theta_{-t})_{\theta_{t}(p)}(W_{\theta_t(p)})$在$t=0$处的导数.
	\begin{align*}
	(\mathscr{L}_VW)_p&=\frac{\mathrm{d}}{\mathrm{d}t}\mid_{t=0}\mathrm{d}(\theta_{-t})_{\theta_{t}(p)}(W_{\theta_t(p)})\\&=\lim_{t\to0}\frac{\mathrm{d}(\theta_{-t})_{\theta_t(p)}(W_{\theta_t(p)})-W_p}{t}
	\end{align*}
	\item 要说明李导数的存在性和光滑性.设$M$是边界为空的流形(如果$M$边界非空,还要添加条件$V$相切于$\partial M$),设$V,W$是$M$上的两个向量场,那么对任意$p\in M$,$(\mathscr{L}_VW)_p$总存在的,并且$\mathscr{L}_VW$是一个光滑向量场.
	\begin{proof}
		
		设$\theta$是$V$的流.对任意$p\in M$,选取覆盖$p$的光滑坐标卡$(U,(x^i))$.选取包含0的实开区间$J_0$,以及一个覆盖点$p$的开集$U_0\subset U$,使得$\theta$把$J_0\times U_0$映入$U$.对每个$(t,x)\in J_0\times U_0$,把$\theta$的坐标表示记作$(\theta^1(t,x),\theta^2(t,x),\cdots,\theta^n(t,x))$.那么$\mathrm{d}(\theta_{-t})_{\theta_t(x)}:\mathrm{T}_{\theta_t(x)}M\to\mathrm{T}_xM$的矩阵可以表示为$\left(\frac{\partial\theta^i}{\partial x^j}(-t,\theta(t,x))\right)$.于是有:
		$$\mathrm{d}(\theta_{-t})_{\theta_t(x)}(W_{\theta_t(x)})=\sum_{i,j}\frac{\partial\theta^i}{\partial x^j}(-t,\theta(t,x))W^j(\theta(t,x))\frac{\partial}{\partial x^i}\mid_x$$
		
		这里$W^j$和$\theta^i$都是光滑函数.于是得到存在性和光滑性.
	\end{proof}
	\item 向量场的流一般不容易计算,不过事实上李导数有如下简单表达式.如果$M$是边界为空的流形,如果$V,W$是两个向量场,那么$\mathscr{L}_VW=[V,W]=VW-WV$.
	\begin{proof}
		
		把$V$的所有正则点(即$V_p\not=0$的全部$p$)记作$\mathscr{R}(V)$,连续性保证$\mathscr{R}(V)$是$M$的开子集,并且它的闭包就是$\mathrm{Supp}V$.我们来对$p\in M$分情况证明$(\mathscr{L}_VW)_p=[V,W]_p$.
		
		情况1:$p\in\mathscr{R}(V)$.我们证明过此时存在$p$附近的坐标卡$(U,(u^i))$使得$V$具有坐标表示$V=\partial/\partial u^1$.此时$V$的流的坐标表示为$\theta_t(u)=(u^1+t,u^2,\cdots,u^n)$.此时$\mathrm{d}(\theta_{-t})_{\theta_t(x)}$的Jacobian矩阵处处是单位矩阵.所以对每个$u\in U$就有:
		$$\mathrm{d}(\theta_{-t})_{\theta_t(u)}(W_{\theta_t(u)})=\sum_jW^j(u^1+t,u^2,\cdots,u^n)\frac{\partial}{\partial u^j}\mid_u$$
		
		于是按照李导数定义就有如下等式,而这恰好是李括号$[\partial/\partial u^1,W]$的坐标表示.$$(\mathscr{L}_VW)_u=\sum_j\frac{\mathrm{d}}{\mathrm{d}t}\mid_{t=0}W^j(u^1+t,u^2,\cdots,u^n)\frac{\partial}{\partial u^j}\mid_u=\sum_j\frac{\partial W^j}{\partial u^1}(u^1,\cdots,u^n)\frac{\partial}{\partial u^j}\mid_i$$
		
	    情况2:$p\in\mathrm{Supp}V$.因为$\mathrm{Supp}V$是$\mathscr{R}(V)$的闭包,按照$(\mathscr{L}_VW)_p$和$[V,W]_p$都是关于$p$连续的,从它们在$\mathscr{R}(V)$上相等就得到在闭包$\mathrm{Supp}(V)$上也相等.
	    
	    情况3:$p\in M-\mathrm{Supp}V$.此时$V$在$p$的某个开邻域上恒为零.所以一方面$\theta_t$应该在$p$的某个开邻域上恒为恒等映射,这里$t$是任取的,所以就有$\mathrm{d}(\theta_{-t})_{\theta_t(p)}(W_{\theta_t(p)})=W_p$.所以$(\mathscr{L}_VW)_p=0$.另一方面按照李括号的坐标表示也有$[V,W]_p=0$.
	\end{proof}
	\item 李导数的一些基本性质.设$V,W,X$是流形$M$上的三个向量场.
	\begin{itemize}
		\item $\mathscr{L}_VW=-\mathscr{L}_WV$.
		\item $\mathscr{L}_V[W,X]=[\mathscr{L}_VW,X]+[W,\mathscr{L}_VX]$.
		\item $\mathscr{L}_{[V,W]}X=\mathscr{L}_V\mathscr{L}_WX-\mathscr{L}_W\mathscr{L}_VX$.
		\item 如果$g\in\mathrm{C}^{\infty}(M)$,那么$\mathscr{L}_V(gW)=(Vg)W+g\mathscr{L}_VW$.
		\item 如果$F:M\to N$是微分同胚,那么$F_*(\mathscr{L}_VX)=\mathscr{L}_{F_*V}F_*X$.
	\end{itemize}
	\item 设$M$是边界为空的流形(如果$M$边界非空,还要添加条件$V$相切于$\partial M$),设$V,W$是$M$上的向量场,设$V$的流为$\theta$.对为于$\theta$定义域中的任意点$(t_0,p)$,总有:
	$$\frac{\mathrm{d}}{\mathrm{d}t}\mid_{t=t_0}\mathrm{d}(\theta_{-t})_{\theta_{t}(p)}(W_{\theta_t(p)})=\mathrm{d}(\theta_{-t_0})\left((\mathscr{L}_VW)_{\theta_{t_0}(p)}\right)$$
	\begin{proof}
		
		任取$p\in M$,记极大积分曲线$\theta^{(p)}$的定义域为$D^{(p)}$.记映射$X:D^{(p)}\to\mathrm{T}_pM$为$t\mapsto\mathrm{d}(\theta_{-t})_{\theta_{t}(p)}(W_{\theta_t(p)})$.在李导数存在性中我们证明了$X$是切空间$\mathrm{T}_pM$中的光滑曲线.做变量换元$t=t_0+s$,那么有:
		\begin{align*}
		X'(t_0)&=\frac{\mathrm{d}}{\mathrm{d}s}\mid_{s=0}X(t_0+s)=\frac{\mathrm{d}}{\mathrm{d}s}\mid_{s=0}\mathrm{d}(\theta_{-t_0-s})_{\theta_{t_0+s}(p)}(W_{\theta_{s+t_0}}(p))\\&=\frac{\mathrm{d}}{\mathrm{d}s}\mid_{s=0}\mathrm{d}(\theta_{-t_0})_{\theta_{t_0}(p)}\circ\mathrm{d}(\theta_{-s})_{\theta_{s}(p)}(W_{\theta_s(\theta_{t_0}(p))})\\&=\mathrm{d}(\theta_{-t_0})_{\theta_{t_0}(p)}\left(\frac{\mathrm{d}}{\mathrm{d}s}\mid_{s=0}\mathrm{d}(\theta_{-s})_{\theta_s(p)}(W_{\theta_s(\theta_{t_0}(p))})\right)
		\end{align*}
		
		其中最后一个等式因为$\mathrm{d}(\theta_{-t_0})_{\theta_{t_0}(p)}$是和$s$无关的$\mathrm{T}_{\theta_{t_0}(p)}M\to\mathrm{T}_pM$的线性变换.最后这里最后一个式子就是欲证等式的右侧.
	\end{proof}
\end{enumerate}

可交换向量场.设$V,W$是流形$M$上的两个向量场,它们称为可交换的,如果$[V,W]\equiv0$.流形$M$上的一个局部标架$\{E_1,E_2,\cdots,E_n\}$称为可交换标架,如果对任意$i,j$总有$[E_i,E_j]\equiv0$.
\begin{enumerate}
	\item 设$W$是$M$上的向量场,$\theta$是$M$上的流.称$W$是$\theta$不变的,如果对每个$t$都有$W$是$\theta_t$相关于自身的.换句话讲对每个$t$都有$W\mid M_t$和$W\mid M_{-t}$是$\theta_t$相关的.或者换句话讲对任意$(t,p)\in D$(此即$\theta$的流定义域),都有$\mathrm{d}(\theta_t)_p(W_p)=W_{\theta_t(p)}$.
	\item 设$V,W$是$M$上的向量场,那么如下条件互相等价.
	\begin{itemize}
		\item $V,W$是可交换的.
		\item $W$在$V$的流下不变.
		\item $V$在$W$的流下不变.
	\end{itemize}
	\begin{proof}
		
		如果2成立.记$V$的流为$\theta$,那么只要$(t,p)$落在$\theta$的定义域$D$中,就有$W_{\theta_t(p)}=\mathrm{d}(\theta_t)_p(W_p)$.把$\mathrm{d}(\theta_{-t})_{\theta_t(p)}$作用在等式两侧,得到$\mathrm{d}(\theta_{-t})_{\theta_t(p)}(W_{\theta_t(p)})=W_p$.按照李导数的定义有$[V,W]=\mathscr{L}_VW=0$.于是2推1成立.对偶的3推1成立.
		
		如果1成立.任取$p\in M$,记$X(t)=\mathrm{d}(\theta_{-t})_{\theta_t(p)}(W_{\theta_t(p)}),t\in D^{(p)}$.我们之前证明过$X'(t_0)=\mathrm{d}(\theta_{-t_0})((\mathscr{L}_VW)_{\theta_{t_0}(p)})$.这里$\mathscr{L}_VW=0$,导致恒有$X'(t)\equiv0$.而$X(0)=W_p$,于是$X(t)\equiv W_p,\forall t\in D^{(p)}$.再把$\mathrm{d}(\theta_t)_p$作用在等式两侧就得到$W$在$\theta$下不变.这得到1推2成立.对偶的1推3成立.于是三个条件互相等价.
	\end{proof}
	\item 推论.每个向量场在自身的流下不变.
	\item 流的可交换性.我们接下来要给出向量场可交换的一个更深的描述:当且仅当它们对应的流可交换.这一条我们来定义流的可交换性.如果两个向量场$V,W$是完备的,那么它们的流$\theta,\psi$的可交换性定义非常简单,即对任意$s,t\in\mathbb{R}$都有$\theta_t\circ\psi_s=\psi_s\circ\theta_t$.但是如果$\theta,\psi$不是整体流,我们能做到的最佳是只要$s,t$使得这个等式两侧都有定义,那么等式成立.但是这个条件太强了,在这个可交换意义下存在可交换的向量场但是它们对应的流不是可交换的.流的可交换性要减弱为:如果$\theta,\psi$是$M$上的两个流,对任意点$p\in M$,只要$J,K$是两个包含0的开区间,使得$\theta_t\circ\psi_s(p)$和$\psi_s\circ\theta_t(p)$在$(s,t)\in J\times K$上存在,那么这两个式子都是存在的并且相等,满足这样的条件就称$\theta,\psi$是可交换的流.对于整体流形的情况这就是要求对任意$s,t\in\mathbb{R}$总有$\theta_t\circ\psi_s=\psi_s\circ\theta_t$.
	\item 两个向量场可交换当且仅当它们的流可交换.
	\begin{proof}
		
		设$V,W$是流形$M$上的向量场,它们的流分别记作$\theta,\psi$.先设$V,W$可交换.取$p\in M$,取$J,K$是包含0的开区间,使得$\psi_s\circ\theta_t(p)$对全部$(s,t)\in J\times K$有意义.按照$V,W$可交换说明$V$是在$\psi$下不变的.固定$s\in J$,考虑曲线$\gamma:K\to M$为$\gamma(t)=\psi_s\circ\theta_t(p)=\psi_s(\theta^{(p)}(t))$.这条曲线满足$\gamma(0)=\psi_s(p)$,并且它在$t\in K$处的速度为:
		$$\gamma'(t)=\frac{\mathrm{d}}{\mathrm{d}t}(\psi_s(\theta^{(p)}(t)))=\mathrm{d}(\psi_s)((\theta^{(p)})'(t))=\mathrm{d}(\psi_s)(V_{\theta^{(p)}}(t))=V_{\gamma(t)}$$
		
		于是$\gamma$是$V$的一条以$\psi_s(p)$为起始点的积分曲线.按照积分曲线的唯一性定理得到$\gamma(t)=\theta^{\psi_s(p)}(t)=\theta_t(\psi_s(p))$.这说明另一侧的复合在$J\times K$上有意义并且二者相同.
		
		反过来假设两个流$\theta,\psi$可交换.任取$p\in M$,选取$\varepsilon>0$使得$\psi_s\circ\theta_t(p)$在$|s|,|t|<\varepsilon$上总存在.那么可交换性保证$\psi_s\circ\theta_t(p)=\theta_t\circ\psi_s(p)$对$|s|,|t|<\varepsilon$总成立.也即有:
		$$\psi^{\theta_t(p)}(s)=\theta_t(\psi^{(p)}(s))$$
		
		对这个等式两侧对$s$求导得到:
		$$W_{\theta_t(p)}=\frac{\mathrm{d}}{\mathrm{d}s}\mid_{s=0}\psi^{\theta_t(p)}(s)=\frac{\mathrm{d}}{\mathrm{d}s}\mid_{s=0}\theta_t(\psi^{(p)}(s))=\mathrm{d}(\theta_t)_p(W_p)$$
		
		等式两侧作用$\mathrm{d}(\theta_{-t})_{\theta_t(p)}$,得到$\mathrm{d}(\theta_{-t})_{\theta_t(p)}(W_{\theta_t(p)})=W_p$.再对$t$求导,按照李导数的定义式就得到$(\mathscr{L}_VW)_p=0$,于是$\mathscr{L}_VW\equiv0$,也即$V,W$可交换.
	\end{proof}
	\item 可交换向量场的标准型.设$M$是$n$维流形,设$\{V_1,V_2,\cdots,V_k\}$是开集$W\subset M$上的线性无关的可交换向量场.对每个点$p\in W$,存在以$p$为中心的光滑坐标卡$(U,(s^i))$,使得每个$V_i$的坐标表示就是$\partial/\partial s^i,1\le i\le k$.如果$S\subset W$是一个余维数$k$的嵌入子流形,并且$p\in S$,并且$\mathrm{T}_pS$是$\{V_1\mid_p,\cdots,V_k\mid_p\}$生成的子空间的补(即这两个子空间的和是全空间).那么这个坐标卡还可以满足$S\cap U$恰好由$s^1=\cdots=s^k=0$定义.
\end{enumerate}

\subsection{积分流形}

我们要把向量场上的积分曲线做一个高维推广.为此要先把向量场做一个高维推广.
\begin{enumerate}
	\item 流形$M$上的一个秩$k$的分布(distribution,容易和分析中广义函数的那个分布混淆)是指切丛$\mathrm{T}M$的秩$k$的子丛.如果它是光滑子丛就称它是光滑分布.不过由于我们遇到的分布几乎都是光滑的,这里我们提及分布总是特指光滑的分布,我们称粗糙分布是指连续性和光滑性都不确定的分布.
	\item $M$上的一个秩$k$粗糙分布$D$就是指对每个点$p\in M$取定一个$k$维子空间$D_p\subset\mathrm{T}_pM$,并且$D=\cup_{p\in M}D_p$.按照子丛光滑性的局部标架准则,这个粗糙分布$F$是光滑的当且仅当$M$的每个点都存在开邻域$U$,使得其上有光滑向量场$X_1,X_2,\cdots,X_k:U\to\mathrm{T}M$,使得$\{X_1\mid_q,\cdots,X_k\mid_q\}$是$D_q$的一组基对任意$q\in U$成立.此时我们称$D$在$U$上被$X_1,X_2,\cdots,X_k$生成.
	\item 设$D\subset\mathrm{T}M$是一个秩$k$的分布,一个非空的$k$维浸入子流形$N\subset M$称为$D$的积分流形,如果对每个$p\in N$都有$\mathrm{T}_pN=D_p$.
	\item 向量场和积分曲线的情况.如果$V$是$M$上处处非零的向量场,那么$V$本身生成了一个秩1的分布$D$,此时积分曲线的像(因为积分曲线我们定义成了一种映射)是$D$的积分流形.
	\item 积分流形不存在的例子.考虑$\mathbb{R}^3$上被向量场$X=\frac{\partial}{\partial x}+y\frac{\partial}{\partial z}$和$Y=\frac{\partial}{\partial y}$生成的分布,它不存在积分流形.
\end{enumerate}

可积分布和对合分布.和向量场上的积分曲线不同,分布上未必总存在积分流形.如果流形$M$上的分布$D$满足$M$的每个点都被某个积分流形经过(就是覆盖),就称$D$是可积分布(integrable).如果$M$上的分布$D$满足对任意开集$U\subset M$,任意$U$上的$D$的截面$X,Y$(即$U$上的向量场$X,Y$,使得$X_p,Y_p\in D_p,\forall p\in U$),都有$[X,Y]$也是$D$的截面(即$[X_p,Y_p]\in D_p,\forall p\in U$),那么称$D$是对合分布(involutive).
\begin{enumerate}
	\item 设$D\subset\mathrm{T}M$是一个分布,记$\Gamma(D)$表示$D$的所有整体截面构成的整体向量场空间$\chi(M)$的子空间,那么$D$是对合分布当且仅当$\Gamma(D)$是$\chi(M)$的李子代数.
	\begin{proof}
		
		$\Gamma(D)$已经是$\chi(M)$的子空间,如果$D$是对合分布,按照定义就有$\Gamma(D)$在李括号下封闭,于是它是$\chi(M)$的李子代数.反过来假设$\Gamma(D)$是$\chi(M)$的李子代数,任取$D$的在开子集$U\subset M$的局部截面$X,Y$.任取$p\in U$,取$\{p\}\subset U$的光滑碰撞函数$\psi$,那么$\psi X$和$\psi Y$都是$D$的整体截面,所以它们的李括号也是$D$的整体截面.这个李括号就是$[\psi X,\psi Y]=\psi^2[X,Y]+\psi(X\psi)Y-\psi(Y\psi)X$,而它在$p$的附近恒为$[X,Y]$,所以有$[X,Y]_p\in D_p,\forall p\in U$,这说明$D$是对合分布.
	\end{proof}
    \item 可积分布总是对合分布.
    \begin{proof}
    	
    	设$D\subset\mathrm{T}M$是可积分布,设$X,Y$是开集$U\subset M$上的$D$的局部截面,设$p\in U$,设$N$是经过点$p$的$D$的积分流形.按照$X,Y$是$D$的截面,说明它们和$N$相切.我们证明过和浸入子流形相切的两个向量场的李括号还是和该子流形相切的,此即$[X,Y]_p\in D_p,\forall p\in U$成立,于是$D$是对合分布.
    \end{proof}
    \item 对合性的局部标架准则.这一条说明验证对合条件不需要对分布$D$的所有局部截面验证,只需要在每个点附近验证一个局部标架即可:设$D\subset\mathrm{T}M$是分布,如果$M$的每个点附近都存在一个局部标架$\{V_1,\cdots,V_k\}$,使得$[V_i,V_j],\forall i,j$都是$D$的局部截面,那么$D$是对合分布.
    \begin{proof}
    	
    	设$X,Y$是$D$的两个定义在相同开集$U$上的局部截面.任取$p\in U$,适当缩小$U$使得它仍然是$p$的开邻域,并且它满足命题中的条件,于是可以选取$U$上的局部标架$\{V_1,\cdots,V_k\}$满足条件.记$X=\sum_iX^iV_i$和$Y=\sum_iY^iV_i$,那么就有:
    	$$[X,Y]=[\sum_iX^iV_i,\sum_jY^jV_j]=\sum_{i,j}\left(X^iY^j[V_i,V_j]+X^i(V_iY^j)V_j-Y^j(V_jX^i)V_i\right)\in D_p$$
    \end{proof}
\end{enumerate}

微分形式提供了描述分布与对合分布的另一种方法.
\begin{enumerate}
	\item 分布光滑性的1形式准则.设$M$是$n$维流形,设$D\subset\mathrm{T}M$是秩$k$的粗糙分布,那么$D$是光滑的当且仅当每个点$p\in M$都存在开邻域$U$,使得其上存在光滑1形式$\{\omega^1,\cdots,\omega^{n-k}\}$,满足对任意$q\in U$都有:$$D_q=\ker\omega^1\mid_q\cap\cdots\cap\ker\omega^{n-k}\mid_q$$
	
	当开集$U$上的1形式$\{\omega^1,\cdots,\omega^{n-k}\}$满足这个条件时我们称它们是局部定义$D$的,或者在$U$上定义$D$的.
	\begin{proof}
		
		充分性,如果$U$上的1形式$\{\omega^1,\cdots,\omega^{n-k}\}$满足命题中的条件,考虑线性映射$\mathrm{T}_qM\to\mathbb{R}^{n-k}$为$x\mapsto(\omega^1(x),\cdots,\omega^{n-k}(x))$,按照条件它的核是$D_q$,具有维数$k$,按照维数公式有这个线性变换的秩是$n$减去核的维数,也即$n-k$,这说明$\{\omega^1,\cdots,\omega^{n-k}\}$在$U$上是线性无关的.那么按照局部余标架的完备性,它可以延拓为$U$上的余标架$\{\omega^1,\cdots\omega^n\}$.设它的对偶标架为$\{E_1,E_2,\cdots,E_n\}$,那么$\{E_{n-k+1},\cdots,E_n\}$就是在$U$上生成$D$的局部标架.按照光滑性的局部标架准则说明$D$是光滑的.
		
		必要性,假设$D$是光滑的.任意点$p\in M$,存在开邻域$U$,使得存在$U$上的标架$\{Y_1,\cdots,Y_k\}$在$U$上生成了$D\mid U$.按照局部标架的完备性,我们可以把这个标架延拓为$U$上的标架$\{Y_1,Y_2,\cdots,Y_n\}$.记对偶标架为$\{\varepsilon^1,\cdots\varepsilon^n\}$,那么$\varepsilon^{k+1},\cdots,\varepsilon^n$就满足命题中的要求.
	\end{proof}
    \item 设$0\le p\le n$,一个$p$形式$\omega\in\Omega^p(M)$称为零化分布$D$,如果对$D$的任意局部截面$X_1,\cdots,X_p$,总有$\omega(X_1,\cdots,X_p)=0$.
    \item 设$M$是$n$维流形,设$D$是$M$上的秩$k$的分布,设$\{\omega^1,\cdots,\omega^{n-k}\}$是开集$U\subset M$上定义$D$的局部1形式.$U$上的一个$p$形式$\eta$零化$D$当且仅当存在$U$上的$p-1$形式$\beta^1,\cdots,\beta^{n-k}$,使得有$\eta=\sum_{i=1}^{n-k}\omega^i\wedge\beta^i$.对于$p=1$的情况,光滑0形式就是光滑函数,光滑函数与其它形式的外积就是相乘,所以此时命题就是在讲$\eta$是$\{\omega^1,\cdots\omega^{n-k}\}$的光滑函数系数的线性组合.
    \begin{proof}
    	
    	明显的按照$\omega^i$都在$U$上零化$D$,得到$\sum_i^{n-k}\omega^i\wedge\beta^i$在$U$上零化$D$.反过来假设$\eta$在$U$上零化$D$.在$p\in U$的包含在$U$内的某个开邻域$U_p$上可以选取定义$D$的余标架$\{\omega^1,\cdots,\omega^{n-k}\}$,把它延拓为$U$上的完备余标架$\{\omega^1,\cdots,\omega^n\}$.记对偶标架为$\{E_1,\cdots,E_n\}$,那么$D$在$U_p$上是被$E_{n-k+1},\cdots,E_n$生成的.在这组完备余标架下,每个$p$形式都可以唯一的表示为$\eta=\sum_I\eta_I\omega^{i_1}\wedge\cdots\wedge\omega^{i_p}$,其中这里$I$取遍单调递增的复指标.那么这里$\eta_I=\eta(E_{i_1},\cdots,E_{i_p})$.所以$\eta$在$U_p$上零化$D$当且仅当对$n-k+1\le i_1<\cdots<i_p\le n$时有$\eta_I=0$.所以$\eta$可以在$U_p$上表示为:
    	$$\eta=\sum_{I,i_1\le n-k}\eta_I\omega^{i_1}\wedge\cdots\wedge\omega^{i_p}=\sum_{i_1=1}^{n-k}\omega^{i_1}\wedge\left(\sum_{I'}\eta_{i_1,I'}\omega^{i_2}\wedge\cdots\wedge\omega^{i_p}\right)$$
    	
    	其中$I$取遍单调递增的复指标,而$I'=(i_2,\cdots,i_p)$.这个表示对每个$U_p,p\in U$都可构造.选取$\{U_p\}$的单位分解$\{\psi_p\}$,把这些表示乘以$\psi_p$相加得到$\eta$在$U$上具有该表示.
    \end{proof}
    \item 对合分布的1形式准则.设$D\subset\mathrm{T}M$是分布,如下命题互相等价:
    \begin{itemize}
    	\item $D$是对合的.
    	\item 如果$\eta$是开集$U\subset M$上零化$D$的1形式,那么$\mathrm{d}\eta$也是$U$上零化$D$的形式.
    \end{itemize}
    \begin{proof}
    	
    	先设$D$是对合分布,设$\eta$是在开集$U\subset M$上零化$D$的1形式,任取$U$上$D$的局部截面$X,Y$,那么就有:$$\mathrm{d}\eta(X,Y)=X(\eta(Y))-Y(\eta(X))=\eta([X,Y])=0$$
    	
    	反过来,设$D$满足命题中第二个条件.任取开集$U\subset M$上$D$的局部截面$X,Y$,设$\{\omega^1,\cdots,\omega^{n-k}\}$是$U$内局部定义$D$的余标架,就得到:
    	$$\omega^i([X,Y])=X(\omega^i(Y))-Y(\omega^i(X))-\mathrm{d}\omega^i(X,Y)=0$$
    	
    	所以$[X,Y]_p\in\cap_i\ker\omega^i=D_p$中,这得到$D$是对合的.
    \end{proof}
    \item 我们之前解释过对合分布的李括号条件只需验证每个点附近的一组局部标架满足李括号条件即可.这里我们给出了对合条件的外导数等价描述,这里也是只需验验证每个点附近的定义$D$的局部余标架即可:设$D$是$n$维流形$M$上的秩$k$分布,设$\{\omega^1,\cdots,\omega^{n-k}\}$是在开集$U\subset M$上定义$D$的余标架,那么如下条件互相等价:
    \begin{itemize}
    	\item $D$在$U$上是对合分布.
    	\item $\mathrm{d}\omega^1,\cdots\mathrm{d}\omega^{n-k}$都零化$D$.
    	\item 存在光滑1形式$\{\alpha_j^i\mid i,j=1,2,\cdots,n-k\}$,使得对任意$i=1,2,\cdots,n-k$都有:
    	$$\mathrm{d}\omega^i=\sum_{j=1}^{n-k}\omega^j\wedge\alpha_j^i$$
    \end{itemize}
    \item 对合分布的一种优雅描述.设$M$是$n$维流形,记$\Omega(M)=\oplus_{0\le p\le n}\Omega^p(M)$是它的外代数.如果$D$是$M$上的分布,记$J^p(D)\subset\Omega^p(M)$是所有零化$D$的$p$形式构造的子空间,那么$J(D)=\oplus_{0\le p\le n}J^p(D)$是代数$\Omega(M)$的理想.那么$D$是对合分布当且仅当$J(D)$是微分理想,也即满足$\mathrm{d}(J(D))\subset J(D)$.
\end{enumerate}

我们接下来要证明本节的核心定理,Frobenius定理.定理断言对合分布与可积分布实际上是等价的.
\begin{enumerate}
	\item 设$D\subset\mathrm{T}M$是秩$k$的分布,$M$上的一个坐标卡$(U,\varphi)$称为对$D$平坦的,如果$\varphi(U)$是$\mathbb{R}^n$的一个方体,并且对任意$p\in U$,都有$D_p$被前$k$个坐标向量场$\partial/\partial x^1,\cdots,\partial/\partial x^k$生成.在这样的坐标卡内,固定常数$c^{k+1},\cdots,c^n$,都有$x^{k+1}=c^{k+1},\cdots,x^n=c^n$是$D$的积分流形.这是针对可积流形的最佳条件.如果对$D$平坦的光滑坐标卡能覆盖整个$M$,我们就称$D$是完全可积分布.
	\item Frobenius定理.明显的我们有完全可积$\Rightarrow$可积$\Rightarrow$对合,这里我们断言三个条件实际上是互相等价的.
	\begin{proof}
		
		只需证明对合分布总是完全可积分布.我们给出过可交换向量场的标准型定理,所以如果一个分布局部上是被线性无关的可交换向量场生成,适当缩小坐标邻域使得像是一个方体,那么就有该分布是完全可积的.于是问题归结为证明对合分布局部上都是被线性无关的可交换向量场生成的.
		
		设$D$是$n$维流形$M$上的秩$k$的对合分布,任取$p\in M$,按照完全可积是一个局部性质,我们不妨把$p$的坐标邻域取代$M$,再把微分同胚像取代$M$,所以归结为设$M$本身是$\mathbb{R}^n$的一个开子集.选取$p$附近生成$D$的标架$\{X_1,\cdots,X_k\}$.适当重排指标,我们可以设$\{\partial/\partial x^{k+1}\mid_p,\cdots,\partial/\partial x^n\mid_p\}$生成的$\mathrm{T}_p\mathbb{R}^n$的子空间是$D_p$的补.
		
		考虑对前$k$个分量的投影映射$\pi:\mathbb{R}^n\to\mathbb{R}^k$,它诱导了光滑的丛同态$\mathrm{d}\pi:\mathrm{T}\mathbb{R}^n\to\mathrm{T}\mathbb{R}^k$为$\sum_{i=1}^nv^i\frac{\partial}{\partial x^i}\mid_q\mapsto\sum_{i=1}^kv^i\frac{\partial}{\partial x^i}\mid_{\pi(q)}$.那么$\mathrm{d}\pi\mid D$是$\mathrm{d}\pi$和$D\subset\mathrm{T}U$的复合,所以它也是一个光滑丛同态.所以$\mathrm{d}\pi\mid D_q$关于基$\{X_i\mid_q\}$和$\{\partial/\partial x^j\mid_{\pi(q)}\}$的矩阵表示的通项是关于$q$的光滑函数.
		
		按照我们的构造有$\ker\mathrm{d}\pi_p$是$D_p\subset\mathrm{T}_p\mathbb{R}^n$的补,所以$\mathrm{d}\pi_p$在$D_p$上的限制是双射.按照连续性,在$p$的附近中的元$q$就总有$\mathrm{d}\pi_q$在$D_q$上的限制是双射.于是在$p$的附近就有$(\mathrm{d}\pi\mid D_q)^{-1}:\mathrm{T}_{\pi(q)}\mathbb{R}^k\to D_q$的通项也是关于$q$的光滑函数.定义$p$附近局部定义$D$的标架$\{V_1,\cdots,V_k\}$为$V_i\mid_q=(\mathrm{d}\pi\mid D_q)^{-1}\frac{\partial}{\partial x^i}\mid_{\pi(q)}$.我们断言有$[V_i,V_j]=0,\forall i,j$.
		
		按照$\partial/\partial x^i\mid_{\pi(q)}=(\mathrm{d}\pi\mid D_q)V_i\mid_q=\mathrm{d}\pi_q(V_i\mid_q)$,得到$V_i$和$\partial/\partial x^i$是$\pi$相关的.于是得到$[V_i,V_j]$和$[\partial/\partial x^i,\partial/\partial x^j]$是$pi$相关的,也即$\mathrm{d}\pi_q([V_i,V_j]_q)=[\partial/\partial x^i,\partial/\partial x^j]_{\pi(q)}=0$.按照$D$是对合分布,说明$[V_i,V_j]_q\in D_q$,所以按照$\mathrm{d}\pi\mid D_q$是单射就得到$[V_i,V_j]_q=0$,所以$[V_i,V_j]\equiv0,\forall i,j$,完成证明.
	\end{proof}
    \item 可积流形的局部结构定理.设$D$是流形$M$上的秩$k$对合分布,设$(U,(x^i))$是一个关于$D$的平坦坐标卡.设$H$是$D$的积分流形,那么$H\cap U$是至多可数个$U$的$k$维平行开切片的无交并,每个都是$H$的开子集,并且每个都是$M$的嵌入子流形.
    \begin{proof}
    	
    	$H$是子流形,所以包含映射$i:H\subset M$是连续的,所以$H\cap U=i^{-1}(U)$是$H$的开子集,所以它是至多可数个连通分支的无交并,并且每个分支都是$H$的开子集.
    	
    	任取其中一个连通分支记作$V$,我们先证明$V$包含在某个切片中.按照$U$是平坦坐标卡,说明在$U$上$\{\mathrm{d}x^{k+1},\cdots,\mathrm{d}x^n\}$是定义了$D$的余标架,那么这些1形式回拉到$V$上是恒为零的,因为$D_p$只由前$k$个坐标向量场生成.而$V$本身是连通的,就导致坐标函数$x^{k+1},\cdots,x^n$在$V$上是常值的,也即$V$落在单个切片$S$中.
    	
    	明显$S$是$M$的嵌入子流形,我们解释过此时包含映射$V\subset M$如果视为$V\subset S$仍然是光滑的.于是$V\subset S$是同维数流形之间的单光滑浸入,所以它是局部微分同胚,是开映射,所以是到$S$的一个开子集的同胚.所以$V\subset S$是光滑嵌入,它和光滑嵌入$S\subset M$的复合$V\subset M$是光滑嵌入.
    \end{proof}
    \item 我们之前给出过弱嵌入的概念.这里我们断言对合分布的积分流形总是弱嵌入的.
    \begin{proof}
    	
    	设$M$是$n$维流形,设$D$是$M$上的秩$k$对合分布,设$H\subset M$是$D$的积分流形.设$F:N\to M$是光滑映射,使得$F(N)\subset H$.任取$p\in N$,记$q=F(p)\in H$.任取$q$附近的平坦坐标卡$(U,(y^i))$,取$p$的连通坐标卡$(B,(x^i))$,使得$F(B)\subset U$.记$F$在这些坐标卡下的的坐标表示为$(F^1(x),\cdots,F^n(x))$.按照$F(B)\subset H\cap U$和$B$是连通的,说明$F^{k+1},\cdots,F^n$都是常值函数.所以$F(B)$落在一个切片$S\subset U$中.那么$S\cap H$就是$H$的一个开子集,并且是$M$的嵌入子流形,于是有$F\mid B:B\to S\cap H$是光滑的,于是映射的复合$F\mid B:B\to S\cap H\subset H$就是光滑的.
    \end{proof}
\end{enumerate}

叶理.设$M$是$n$维流形,设$\mathscr{F}$是一族$k$维浸入子流形,一个光滑坐标卡$(U,\varphi)$称为关于$\mathscr{F}$平坦的,如果$\varphi(U)$是$\mathbb{R}^n$的一个方体,并且$\mathscr{F}$中的每个子流形和$U$的交要么是空集,要么是至多可数个维数$k$的具有形式$x^{k+1}=c^{k+1},\cdots,x^n=c^n$的切片的并.定义$M$上的$k$维叶理(foliation)是一族不交连通非空$k$维浸入子流形,它们称为叶理的叶,满足这些子流形的并是$M$,并且每个点附近都存在关于$\mathscr{F}$平坦的坐标卡.
\begin{enumerate}
	\item 关于叶理的核心定理是流形上的叶理与对合分布是一一对应的,首先是相对简单的一侧对应:如果$\mathscr{F}$是流形$M$上的叶理,那么$\mathscr{F}$的所有叶的所有切空间构成$M$上的一个对合分布.
	\item 引理.设$D\subset\mathrm{T}M$是一个对合分布,设$\{N_{\alpha}\}_{\alpha\in A}$是一族$D$的连通积分流形,并且它们包含了某个公共点,那么$N=\cup_{\alpha}N_{\alpha}$上具有唯一的光滑结构使得它成为$D$的一个积分流形.
	\begin{proof}
		
		如果我们能够构造$N$上的拓扑结构和光滑结构使得它成为$D$的积分流形,按照积分流形都是弱嵌入的,并且弱嵌入子流形上存在唯一的拓扑结构和光滑结构使得它是浸入子流形,就证明了该引理.
		
		先证明对任意$\alpha,\beta\in A$都有$N_{\alpha}\cap N_{\beta}$在$N_{\alpha}$和$N_{\beta}$中都是开子集.任取$q\in N_{\alpha}\cap N_{\beta}$,选取$q$附近的关于$D$的平坦坐标卡$W$,分别记$V_{\alpha}$和$V_{\beta}$为$N_{\alpha}\cap W$和$N_{\beta}\cap W$的包含$q$的分支.按照可积流形的局部结构定理(和证明),它们在子空间拓扑下分别是$N_{\alpha}\cap W$和$N_{\beta}\cap W$的开子集,并且都落在某个切片中,但是它们都包含同一个点$q$,导致它们都落在同一个切片$S$中.所以$V_{\alpha}\cap V_{\beta}$是$S$的开子集,并且还落在$N_{\alpha}$和$N_{\beta}$中,这得到$N_{\alpha}\cap N_{\beta}$是开集.
		
		定义$N$上的拓扑为,子集$U\subset N$约定为开集当且仅当$U\cap N_{\alpha}$是$N_{\alpha}$的开子集对任意$\alpha\in A$成立.那么上一段说明每个$N_{\alpha}$都是$N$的开子集,于是$N_{\alpha}$的所有开子集都是$N$的开子集.这个拓扑是局部$k$维欧氏空间的:任取$q\in N$,它必然被某个$N_{\alpha}$覆盖,选取$p$在$N_{\alpha}$中的坐标卡即可.包含映射$N\subset M$是连续映射:任取开集$U\subset M$,那么每个$U\cap N_{\alpha}$在$N_{\alpha}$中是开子集,所以$U\cap N$是$N$的开子集.$N$是Hausdorff空集:任取$q,q'\in N$,在$M$中存在不交开集$U,U'$分别包含这两个点,那么$N\cap U$和$N\cap U'$是$N$中不交开集分别包含这两个点.
		
		$N$是第二可数空间:取$M$的由关于$D$的平坦坐标卡构成的至多可数的开覆盖,记作$\{W_i\}$,只需验证每个$N\cap W_i$包含在至多可数个切片的并中,因为这样的话单个切片的开子集都是第二可数的,至多可数个第二可数空间的并是第二可数的.【】
	\end{proof}
	\item 整体Frobenius定理.如果$D$是一个对合分布,它的所有极大连通积分流形构成了$M$的一个叶理.
	\begin{proof}
		
		任取点$p\in M$,记$L_p$表示所有经过点$p$的关于$D$的连通积分流形的并,按照上面引理这个$L_p$是经过点$p$的关于$D$的连通积分流形.所以这自然是极大的.如果两个$L_p$和$L_{p'}$的交非空,那么按照引理有$L_p\cup L_{p'}$也是连通积分流形,极大性就保证$L_p=L_{p'}$.所以这些极大的连通积分流形要么不交要么相同.所以它们的并是整个$M$.
		
		设$(U,\varphi)$是关于$D$的平坦坐标卡,按照积分流形的局部结构定理,有$L_p\cap U$是至多可数个切片的开子集的并.对每个这样的切片$S$,如果$L_p\cap S$非空并且不是整个$S$,那么$L_p\cup S$是一个严格包含$S$的连通积分流形,这和极大性矛盾,所以$L_p\cap U$恰有由至多可数个切片的并构成.所以$U$就是关于这族极大积分流形的平坦坐标卡,这说明是一个叶理.
	\end{proof}
    \item 设$M$是流形,设$\Phi:M\to M$是微分同胚.$M$上的一个分布$D$称为是$\Phi$不变的,如果$\mathrm{d}\Phi(D)=D$.换句话讲如果对任意$x\in M$有$\mathrm{d}\Phi_x(D_x)=D_{\Phi(x)}$.类似的,$M$上的叶理$\mathscr{F}$称为$\Phi$不变的,如果对它的任意叶$L$都有$\Phi(L)$也是$\mathscr{F}$的叶.
    \item 如果$M$是流形,$\Phi:M\to M$是微分同胚,如果$D$是$M$上的对合分布,$\mathscr{F}$是由它生成的叶理,那么$D$是$\Phi$不变的当且仅当$\mathscr{F}$是$\Phi$不变的.
\end{enumerate}

\newpage
\section{李群和李代数}
\subsection{基本概念}

定义.
\begin{itemize}
	\item 李群是指一个(边界为空的)光滑流形$G$同时是一个群,满足它作为群的二元运算$G\times G\to G,(g,h)\mapsto gh$和逆元映射$G\to G,g\mapsto g^{-1}$都是光滑映射.
	\item 如果$G,H$都是李群,映射$F:G\to H$称为李群同态,如果它同时是群同态和光滑映射.一个李群同态称为同构,如果它同时是群同构和微分同胚.
\end{itemize}
\begin{enumerate}
	\item 如果$G$是光滑流形并且具备群结构,那么它是李群等价于映射$G\times G\to G,(g,h)\mapsto gh^{-1}$是光滑的.
	\item 李群自然都是拓扑群.
	\item 开子李群.如果李群$G$的开子集$H$还是一个子群,那么它自动是一个李群,称为$G$的开子李群.
	\item 离散李群.一个元素至多可数的群如果是光滑流形,就称为是离散李群.
	\item 有限个李群的积空间是李群.
	\item 左右平移映射.设$G$是李群,对$g\in G$,定义左平移映射和右平移映射$L_g,R_g:G\to G$是左乘$g$和右乘$g$的映射.它们都是光滑映射,因为按照李群定义有$h\mapsto(g,h)\mapsto gh$是光滑映射的复合.而$L_g$的逆映射是$L_{g^{-1}}$,$R_g$的逆映射是$R_{g^{-1}}$.于是李群上的左平移映射和右平移映射都是微分同胚.
	\item 共轭映射.设$G$是李群,$g\in G$,定义关于$g$的共轭映射为$C_g:G\to G,h\mapsto ghg^{-1}$.这是一个李群同构,它的逆映射是$C_{g^{-1}}$.
	\item 李群同态作为光滑映射是常秩映射.
	\begin{proof}
		
		设$F:G\to H$是李群同态,记$G,H$的幺元分别为$e_G,e_H$.任取$g_0\in G$,我们只需验证$\mathrm{d}F_{g_0}$和$\mathrm{d}F_{e_G}$具有相同的秩.
		
		按照$F$是群同态,说明对任意$g\in G$有$F(L_{g_0}(g))=F(g_0g)=F(g_0)F(g)=L_{F(g_0)}(F(g))$,也即$F\circ L_{g_0}=L_{F(g_0)}\circ F$.在这个等式两侧取微分,得到:
		$$\mathrm{d}F_{g_0}\circ\mathrm{d}(L_{g_0})_{e_G}=\mathrm{d}(L_{F(g_0)})_{e_H}\circ\mathrm{d}F_{e_G}$$
		
		我们解释过左平移映射是微分同胚,于是它的微分是同构,这就得到$\mathrm{d}F_{g_0}$和$\mathrm{d}F_{e_G}$的秩相同.
	\end{proof}
    \item 一个李群同态是李群同构当且仅当它是一个双射.
    \begin{proof}
    	
    	我们知道群同态是群同态当且仅当它是双射.而整体秩定理说明常秩的光滑映射是微分同胚等价于双射.这得证.
    \end{proof}
\end{enumerate}

李子群.设$G$是李群,它的一个子集$H$称为李子群,如果群层面是$G$的子群,光滑流形层面是$G$的浸入子流形,并且$H$的群结构和光滑流形结构构成李群.这里做一个约定,当我们提李群的子群时,是指只考虑李群作为群时的子群,和光滑结构无关,而当我们提及李群的李子群时是这里定义的同时依赖于光滑结构和群结构的概念.
\begin{enumerate}
	\item 设$G$是李群,如果$H\subset G$在群层面是$G$的子群,在光滑流形层面是$G$的嵌入子流形,那么$H$的群结构和光滑流形结构自动构成李群.此时我们称$H$是嵌入李子群,它是不需要验证李子群定义中的最后一条的.
	\begin{proof}
		
		我们要证明群运算$(h_1,h_2)\mapsto h_1h_2$和$h\mapsto h^{-1}$都是光滑映射.以前者为例,首先条件说明$G\times G\to G,(g_1,g_2)\mapsto g_1g_2$是光滑的,按照我们给出过的光滑映射的限制条件,这个映射限制为$H\times H\to G$仍然是光滑的.而这个限制映射的像落在$G$的嵌入子流形$H$中,我们证明过这个条件下再把映射限制为$H\times H\to H$仍然是光滑的.
	\end{proof}
    \item 我们之前定义的开李子群是嵌入李子群,另外开李子群总是闭子集.
    \begin{proof}
    	
    	开子集自然是嵌入子流形,于是开李子群是嵌入李子群.下面仅需验证它是闭子集:事实上这件事对拓扑群已经成立了,即拓扑群的开子群总是闭的,因为开子群的所有左陪集都是开的(左平移是同胚),而开子群自身是全部非平凡左陪集的并的补集,所以是闭的.
    \end{proof}
    \item 设$G$是李群,设$W\subset G$是幺元的任意开邻域.
    \begin{itemize}
    	\item $W$作为群$G$的子集生成(纯代数意义的生成)的子群是$G$的开子群.
    	\item 如果$W$是连通的,那么它生成(纯代数意义的生成)的是连通开子群.
    	\item 如果$G$本身是连通的,那么$W$生成(纯代数意义的生成)的是整个$G$.
    \end{itemize}
    \begin{proof}
    	
    	设$W$纯代数意义生成的$G$的子群为$H$,这里纯代数意义是指$H$是$G$的所有包含$W$的代数意义的子群的交.这里约定下记号,如果$A,B$是$G$的子集,我们记$AB=\{ab\mid a\in A,b\in B\}$和$A^{-1}=\{a^{-1}\mid a\in A\}$.对每个正整数$k$,记$W_k$表示$G$中的所有可以表示为$W\cup W^{-1}$的不超过$k$个元素乘积的元构成的子集.那么必然有$\cup_{k\ge1}W_k=H$.现在$W_1=W\cup W^{-1}$是开集的并,所以是开集.而$W_k=W_1W_{k-1}=\cup_{g\in W_1}L_g(W_{k-1})$也是若干开集的并,于是所有$W_k$是开集,于是$H$是开集.
    	
    	如果$W$还是连通的,那么$W^{-1}$也是连通的,并且这两个集合都包含了幺元,我们知道交非空的两个连通子集的并也是连通的,于是$W_1=W\cup W^{-1}$是连通的.记群二元运算映射为$m:G\times G\to G$,那么$W_k=m(W_1\times W_{k-1})$,对$k$归纳结合连通空间的连续像是连通的,就得到每个$W_k$都是连通的,于是$H=\cup_kW_k$也是连通的.
    	
    	最后如果$G$本身是连通的,我们解释过$H$是开子群的时候也是闭子群,连通性就保证$H=G$.
    \end{proof}
    \item 设$F:G\to H$是李群同态,那么$\ker F$是$G$的真嵌入李子群,它的余维数是$F$的秩(我们证明过李群同态是常秩映射).
    \begin{proof}
    	
    	我们证明过对于常秩映射$F$,它的非空水平集都是$G$的余维数是$\mathrm{rank}F$的真嵌入子流形.特别的$F^{-1}(e_H)$是一个非空水平集.
    \end{proof}
    \item 设$F:G\to H$是单李群同态,$\mathrm{im}F$上存在唯一的光滑结构使得$F(G)$是$H$的李子群,并且使得$F:G\to F(G)$是李群同构.
    \begin{proof}
    	
    	按照李群同态是常秩映射,我们证明过单的常秩映射是光滑浸入.我们还证明过单光滑浸入$F:N\to M$的像集$S=F(N)$上存在唯一的拓扑结构和光滑结构使得它成为$M$的浸入子流形,并且$F$视为$N\to S$时是微分同胚.在这里$G$同构$F(G)$是$H$的子群,所以$F(G)$是李群,$F:G\to F(G)$是李群同态且是双射,所以它是李群同构.
    \end{proof}
    \item 设$G$是李群,$H\subset G$是李子群,那么$H$在$G$中闭当且仅当它是嵌入李子群.我们在后文会证明一个更强的定理,闭子群定理:李群的子群如果拓扑上是闭的,那么它就是一个闭嵌入李子群.
    \begin{proof}
    	
    	先设$H$是$G$的嵌入李子群.任取$g\in\overline{H}$,那么存在$H$中的一个点列$\{h_i\}$在$G$中收敛到$g$.任取$H$的包含幺元的切片坐标卡$U$,再取幺元在$H$中更小的开邻域$W$满足$\overline{W}\subset U$.还可以取幺元在$H$中更小的开邻域$V$使得$VV^{-1}\subset W$.
    	
    	由于$h_ig^{-1}\to e$,不妨删去序列$\{h_i\}$的前足够多有限项,使得$h_ig^{-1}\in V,\forall i$.这导致$h_ih_j^{-1}=(h_ig^{-1})(h_jg^{-1})^{-1}\in W$对任意$i,j$成立.固定$j$令$i\to\infty$,得到$gh_j^{-1}\in\overline{W}\subset U$.按照切片坐标卡的定义有$H\cap U$是$U$的闭子集,于是有$gH_j^{-1}\in H$,于是$g\in H$.这说明$H$是闭子集.
    	
    	【】
    \end{proof}
\end{enumerate}

李括号.设$X,Y$是流形$M$上的向量场,定义它们的李括号为$[X,Y]:\mathrm{C}^{\infty}(M)\to\mathrm{C}^{\infty}(M)$,$[X,Y]f=XYf-YXf$.(我们证明过向量场的一个等价描述是流形上光滑函数空间到自身的线性映射,满足莱布尼兹法则).
\begin{enumerate}
	\item 向量场的李括号仍然是向量场.
	\begin{proof}
		
		归结为证明向量场的李括号满足莱布尼兹法则:
		\begin{align*}
		[X,Y](fg)&=X(Y(fg))-Y(X(fg))\\&=X(fYg+gYf)-Y(fXg+gXf)\\&=XfYg+fXYg+XgYf+gXYf-YfXg-fYXg-YgXf-gYXf\\&=fXYg+gXYf-fYXg-gYXf\\&=f[X,Y]g+g[X,Y]f
		\end{align*}
	\end{proof}
    \item 李括号的坐标表示.设$X,Y$是向量场,在某个坐标卡上它们分别的坐标表示为$X=\sum_iX^i\partial/\partial x^i$和$Y=\sum_jY^j\partial/\partial x^j$.那么李括号在这个坐标卡上可以表示为:
    $$[X,Y]=\sum_{i,j}\left(X^i\frac{\partial Y^j}{\partial x^i}-Y^i\frac{\partial X^j}{\partial x^i}\right)\frac{\partial}{\partial x^j}$$
    \item 一个平凡的情况是$[\partial/\partial x^i,\partial/\partial x^j]=0,\forall i,j$.
    \item 李括号的一些基本性质:设$X,Y,Z$是$M$上的向量场.
    \begin{itemize}
    	\item 双线性.对任意$a,b\in\mathbb{R}$有:
    	$$[aX+bY,Z]=a[X,Z]+b[Y,Z]$$
    	$$[Z,aX+bY]=a[Z,X]+b[Z,Y]$$
    	\item 反对称性.$$[X,Y]=-[Y,X]$$
    	\item Jacobi恒等式.$$[X,[Y,Z]]+[Y,[Z,X]]+[Z,[X,Y]]=0$$
    	\item 对$f,g\in\mathrm{C}^{\infty}(M)$有:$$[fX,gY]=fg[X,Y]+(fXg)Y-(gYf)X$$
    \end{itemize}
    \item 设$F:M\to N$是光滑映射,设$X_1,X_2$是$M$上向量场,$Y_1,Y_2$是$N$上向量场,满足$X_i$和$Y_i$是$F$相关的(是指对每个$p\in M$有$\mathrm{d}F_p(X_p)=Y_p$),这里$i=1,2$.那么$[X_1,X_2]$和$[Y_1,Y_2]$是$F$相关的.
    \begin{proof}
    	
    	我们证明过$X$和$Y$是$F$相关的等价于讲对$N$上每个光滑函数$f$都有$X(f\circ F)=(Yf)\circ F$.于是任取$N$上的光滑函数$f$,那么有:
    	\begin{align*}
    	[X_1,X_2](f\circ F)&=X_1(X_2(f\circ F))-X_2(X_1(f\circ F))\\&=X_1((Y_2f)\circ F)-X_2((Y_1f)\circ F)\\&=(Y_1Y_2f)\circ F-(Y_2Y_1f)\circ F\\&=([Y_1,Y_2]f)\circ F
    	\end{align*}
    \end{proof}
    \item 李括号的前推.如果$F:M\to N$是微分同胚,并且$X_1,X_2$是$M$上的向量场,那么有$F_*[X_1,X_2]=[F_*X_1,F_*X_2]$.证明只要在上一条中取$F$为该微分同胚,取$Y_i=F_*X_i$.
    \item 如果$S\subset M$是浸入子流形,如果$Y_1,Y_2$是$M$上的和$S$相切的向量场,那么$[Y_1,Y_2]$也是和$S$相切的向量场.
    \begin{proof}
    	
    	我们证明过从$Y_i$和$S$相切推出存在唯一的$S$上的向量场$X_i$,使得它和$Y_i$是$i$相关的,其中$i:S\subset M$是包含映射.于是前面的定理说明$[X_1,X_2]$和$[Y_1,Y_2]$也是$i$相关的,于是$[Y_1,Y_2]$和$S$相切.
    \end{proof}
\end{enumerate}

李代数和一些基本定义.
\begin{enumerate}
	\item 一个(实)李代数是指一个(实)线性空间$\mathfrak{g}$,具备一个二元运算称为李括号$[-,-]:\mathfrak{g}\times\mathfrak{g}\to\mathfrak{g}$,满足:
	\begin{enumerate}
		\item 双线性,即$[aX+bY,Z]=a[X,Z]+b[Y,Z];[X,aY+bZ]=a[X,Y]+b[X,Z]$.
		\item 反对称性,即$[Y,X]=-[X,Y]$.
		\item Jacobi恒等式,即$[X,[Y,Z]]+[Y,[Z,X]]+[Z,[X,Y]]=0$.
	\end{enumerate}
    \item 李代数$\mathfrak{g}$的李子代数是指它的一个线性子空间,满足在李括号下封闭.此时李子代数本身也是一个李代数.
    \item 两个李代数之间的李代数同态$A$定义为一个线性变换,使得它保李括号:$A[X,Y]=[AX,AY]$.一个李代数同态称为李代数同构,如果它存在逆映射,并且逆映射也是一个李代数同态,而这等价于仅要求李代数同态是双射.当两个李代数之间存在李代数同构时,我们称它们作为李代数是同构的.
    \item 如果$A:\mathfrak{g}\to\mathfrak{h}$是一个线性变换,由于已经有了线性,为了验证它是李代数同态,只需任取$\mathfrak{g}$的一组基$\{E_1,E_2,\cdots,E_n\}$,验证对任意$i,j$总有$A[E_i,E_j]=[AE_i,AE_j]$即可.
    \item 如果李括号是平凡的,即对任意$X,Y$都有$[X,Y]=0$,则称该李代数是交换的.
\end{enumerate}

李群的李代数.
\begin{enumerate}
	\item 左不变向量场.设$G$是李群,其上一个向量场$X$如果对每个左平移映射都相关于自身,就称它是左不变向量场.换句话讲$X$要满足对任意$g,h\in G$都有$\mathrm{d}(L_g)_h(X_h)=X_{gh}$.$G$是全体左不变向量场构成向量场空间的一个线性子空间.
	\item 如果$X,Y$都是李群$G$上的左不变向量场,那么$[X,Y]$也是左不变向量场.于是$G$上全体左不变向量场构成一个李代数,称为李群$G$的李代数,记作$\mathrm{Lie}(G)$.
	\begin{proof}
		
		任取$g\in G$,有$(L_g)_*X=X$和$(L_g)_*Y=Y$.于是有$(L_g)_*[X,Y]=[(L_g)_*X,(L_g)_*Y]=[X,Y]$.这说明$[X,Y]$也是$G$上左不变向量场.
	\end{proof}
    \item 设$G$是李群,映射$\varepsilon:\mathrm{Lie}(G)\to\mathrm{T}_eG$,$\varepsilon(X)=X_e$是一个线性同构.于是特别的$\mathrm{Lie}(G)$是有限维的,并且维数就是$\dim G$.
    \begin{proof}
    	
    	$\varepsilon$明显是一个线性变换.它是单射因为如果$\varepsilon(X)=X_e=0$,那么左不变性导致$X_g=\mathrm{d}(L_g)_e(X_e)=0$对任意$g\in G$成立,于是$X=0$.下面验证满射性.任取$v\in\mathrm{T}_eG$,定义$G$上的粗糙向量场$v^L$为$v^L\mid_g=\mathrm{d}(L_g)_e(v)$.
    	
    	先验证$v^L$是光滑的.为此只需验证对$G$上的每个光滑函数$f$都有$v^Lf$是光滑函数.选取$\gamma:(-\delta,\delta)\to G$满足$\gamma(0)=e$和$\gamma'(0)=v$,那么对每个$g\in G$就有:
    	\begin{align*}
    	(v^Lf)(g)&=v^L\mid_gf=\mathrm{d}(L_g)_e(v)f=v(f\circ L_g)=\gamma'(0)(f\circ L_g)\\&=\frac{\mathrm{d}}{\mathrm{d}t}\mid_{t=0}(f\circ L_g\circ\gamma)(t)
    	\end{align*}
    	
    	定义$\varphi:(-\delta,\delta)\times G\to\mathbb{R}$为$(t,g)\mapsto f\circ L_g\circ\gamma(t)=f(g\gamma(t))$.于是上面计算说明$(v^Lf)(g)=\partial\varphi/\partial t(0,g)$.这里$\varphi$是光滑的,导致它的偏导数也是光滑的,于是$v^Lf$是光滑的.
    	
    	再验证$v^L$是左不变的,这是因为对任意$g,h\in G$都有:
    	$$\mathrm{d}(L_h)_g(v^L\mid_g)=\mathrm{d}(L_h)_g\circ\mathrm{d}(L_g)_e(v)=\mathrm{d}(L_{hg})_e(v)=v^L\mid_{hg}$$
    	
    	于是$v^L$是$G$上的一个左不变向量场,而$\varepsilon(v^L)=v^L\mid_e=v$,这说明$\varepsilon$是满射.
    \end{proof}
    \item 上一条的证明中实际证明了李群上的任意粗糙左不变向量场都是光滑的:任取李群$G$上的左不变光滑向量场$X$,记$v=X_e$,左不变条件保证$X=v^L$,而我们证明过$v^L$是光滑的.
    \item 李群上的一个局部或者整体标架如果由左不变向量场构成,就称它是左不变标架.李群上总存在整体左不变标架,于是特别的李群总是可平行化的.
    \begin{proof}
    	
    	事实上李代数$\mathrm{Lie}(G)$的一组基就构成左不变整体标架.
    \end{proof}
\end{enumerate}

李群的李代数的例子.
\begin{enumerate}
	\item $\mathbb{R}^n$在加法运算下视为李群.任取$b\in\mathbb{R}^n$,映射$L_b(x)=b+x$的微分是标准标架下的恒等矩阵,于是一个向量场$X=\sum_iX^i\partial/\partial x^i$是左不变向量场当且仅当$X^i$都是常值函数.此时李括号是平凡的,于是$\mathrm{Lie}(\mathbb{R}^n)\cong\mathbb{R}^n$,李括号是平凡的.
	\item $\mathrm{M}(n,\mathbb{R})$是全体$n\times n$的实矩阵构成的$n^2$维线性空间,定义李括号$[A,B]=AB-BA$构成的李代数记作$\mathfrak{gl}(n,\mathbb{R})$.
	\item 考虑李群$\mathrm{GL}(n,\mathbb{R})$,我们证明过它的李代数线性同构于$\mathrm{GL}(n,\mathbb{R})$在单位矩阵$I_n$处的切空间.但是$\mathrm{GL}(n,\mathbb{R})$是$\mathfrak{gl}(n,\mathbb{R})$的开子集,于是它们在单位矩阵的切空间是相同的.而线性空间的切空间就是自身,于是我们得到了一个典范线性同构$\mathrm{Lie}(n,\mathbb{R})\cong\mathfrak{gl}(n,\mathbb{R})$.这个典范同构还是一个李代数同构.
	\begin{proof}
		
		以矩阵的通项$X_j^i$作为$\mathrm{GL}(n,\mathbb{R})\subset\mathfrak{gl}(n,\mathbb{R})$上的整体坐标,那么典范同构$\mathrm{T}_{I_n}\mathrm{GL}(n,\mathbb{R})\cong\mathfrak{gl}(n,\mathbb{R})$具体写出来就是$\sum_{i,j}A_j^i\frac{\partial}{\partial X_j^i}\mid_{I_n}\mapsto(A_j^i)$.
		
		把$\mathrm{GL}(n,\mathbb{R})$的李代数记作$\mathfrak{g}$.任取$A=(A_j^i)\in\mathfrak{gl}(n,\mathbb{R})$,它对应了一个左不变向量场$A^L\in\mathfrak{g}$为:
		$$A^L\mid_X=\mathrm{d}(L_X)_{I_n}(A)=\mathrm{d}(L_X)_{I_n}\left(\sum_{i,j}A_j^i\frac{\partial}{\partial X_j^i}\mid_{I_n}\right)$$
		
		这里$L_X$就是$\mathfrak{gl}(n,\mathbb{R})$上的映射$A\mapsto XA$在$\mathrm{GL}(n,\mathbb{R})$上的限制,它的微分还是自身这个线性变换,于是有:
		$$A^L\mid_X=\sum_{i,j,k}X_j^iA_k^j\frac{\partial}{\partial X_k^i}\mid_X$$
		
		为证明该同构是李代数同构只需验证对任意$A,B\in\mathfrak{gl}(n,\mathbb{R})$都有$[A^L,B^L]=[A,B]^L$.我们有:
		\begin{align*}
		[A^L,B^L]&=[\sum_{i,j,k}X_j^iA_k^j\frac{\partial}{\partial X_k^i},\sum_{p,q,r}X_q^pB_r^q\frac{\partial}{\partial X_r^p}]\\&=\sum_{i,j,k,p,q,r}\left(X_j^iA_k^j\frac{\partial}{\partial X_k^i}(X_q^pB_r^q)\frac{\partial}{\partial X_r^p}-X_q^pB_r^q\frac{\partial}{\partial X_r^p}(X_j^iA_k^j)\frac{\partial}{\partial X_k^i}\right)\\&=\sum_{i,j,k,r}X_j^iA_k^jB_r^k\frac{\partial}{\partial X_r^i}-\sum_{p,q,r,k}X_q^pB_r^qA_k^r\frac{\partial}{\partial X_k^p}\\&=\sum_{i,j,k,r}\left(X_j^iA_k^jB_r^k-X_j^iB_k^jA_r^k\right)\frac{\partial}{\partial X_r^i}
		\end{align*}
		
		于是有:$$[A^L,B^L]_{I_n}=\sum_{i,k,r}\left(A_k^iB_r^k-B_k^iA_r^k\right)\frac{\partial}{\partial X_r^i}\mid_{I_n}$$
		
		此即对应于$[A,B]$的切向量.按照左不变性质,$[A^L,B^L]$被它在幺元处的取值决定,这导致$[A^L,B^L]=[A,B]^L$.
	\end{proof}
    \item 线性空间版本.如果$V$是有限维实空间,如下典范映射的复合是李代数同构.
    $$\mathrm{Lie}(\mathrm{GL}(V))\to\mathrm{T}_{\mathrm{id}}\mathrm{GL}(V)\to\mathfrak{gl}(V)$$
\end{enumerate}

李代数的函子性.
\begin{enumerate}
	\item 设$G,H$是李群,相应的李代数记作$\mathfrak{g}$和$\mathfrak{h}$.如果$F:G\to H$是一个李群同态,对每个左不变向量场$X\in\mathfrak{g}$,存在$\mathfrak{h}$中的唯一向量场和$X$是$F$相关的,它记作$F_*X$,这样定义的$F_*:\mathfrak{g}\to\mathfrak{h}$是一个李代数同态.它称为$F$诱导的李代数同态.
	\begin{proof}
		
		如果存在左不变向量场$Y\in\mathfrak{h}$使得它和$X$是$F$相关的.那么有$Y_e=\mathrm{d}F_e(X_e)$.那么按照左不变性必然有$Y=(\mathrm{d}F_e(X_e))^L$.这说明唯一性.下证$Y$和$X$是$F$相关的.先按照$F$是李群同态得到:
		\begin{align*}
		F(gg')=F(g)F(g')&\Rightarrow F(L_gg')=L_{F(g)}F(g')\\&\Rightarrow F\circ L_g=L_{F(g)}\circ F\\&\Rightarrow\mathrm{d}F\circ\mathrm{d}(L_g)=\mathrm{d}(L_{F(g)})\circ\mathrm{d}F
		\end{align*}
		
		于是有:
		$$\mathrm{d}F(X_g)=\mathrm{d}F(\mathrm{d}(L_g)(X_e))=\mathrm{d}(L_{F(g)})(\mathrm{d}F(X_e))=\mathrm{d}(L_{F(g)})(Y_e)=Y_{F(g)}$$
		
		也即$X$和$Y$是$F$相关的.最后按照我们证明过的$F_*[X,Y]=[F_*X,F_*Y]$说明$F_*$是李代数同态.
	\end{proof}
    \item 函子性.
    \begin{itemize}
    	\item 李群$G$上恒等映射诱导的李代数同态是恒等映射.
    	\item 设$F_1:G\to H$和$F_2:H\to K$都是李群同态,那么诱导的李代数同态满足:
    	$$(F_2\circ F_1)_*=(F_2)_*\circ(F_1)_*$$
    	\item 李群之间的同构诱导了李代数之间的同构.
    \end{itemize}
\end{enumerate}

\subsection{李对应}

李子群和李子代数.
\begin{enumerate}
	\item 设$G$是李群,$H\subset G$是李子群,我们自然期望$H$的李代数恰好是$G$的李代数的李子代数:记$i:H\subset G$是包含映射,那么存在$\mathrm{Lie}(G)$的李子代数$\mathfrak{h}=i_*(\mathrm{Lie}(H))=\{X\in\mathrm{Lie}(G)\mid X_e\in\mathrm{T}_eH\}$,典范同构于$\mathrm{Lie}(H)$.
	\begin{proof}
		
		这里只要说明$i_*$是单射,而这是因为$\mathrm{d}i_e$是$\mathrm{T}_eH$上的单射.
	\end{proof}
	\item 例子.$\mathrm{O}(n)$是$\mathrm{GL}(n,\mathbb{R})$的李子群,它的李代数$\mathfrak{o}(n)$由全体$n$阶斜对称实矩阵构成.
	\item 设$D$是李群$G$上的分布,称它维左不变的,如果它对每个左平移映射都是左不变的,换句话讲对任意$g\in G$都有$\mathrm{d}(L_g)(D)=D$.
	\item 设$G$是李群,设$\mathfrak{h}$是$\mathrm{Lie}(G)$的李子代数,那么$D=\cup_{g\in G}D_g\subset\mathrm{T}G$,其中$D_g=\{X_g\mid X\in\mathfrak{h}\}\subset\mathrm{T}_gG$,是$G$上的一个左不变对合分布.
	\begin{proof}
		
		左不变性:按照$X\in\mathfrak{h}$是$G$上的左不变向量场,所以对任意$g,g'\in G$,都有微分$\mathrm{d}(L_{g'g^{-1}})$限制为$D_g\to D_{g'}$的同构.所以对所有$g$有$D_g$的维数都是相同的,并且$D$本身是左不变的.
		
		对合性:任取$\mathfrak{h}$上的一组基$\{X_1,\cdots,X_k\}$,它也是生成$D$的整体光滑标架,所以$D$是光滑的.另外按照$[X_i,X_j]\in\mathfrak{h},\forall i,j$,按照我们之前给出的对合性的局部标架准则,就得到$D$是对合分布.
	\end{proof}
	\item 李子群一定是某个对合分布的积分流形,于是特别的李子群总是弱嵌入子流形.
	\begin{proof}
		
		设$G$是李群,$H\subset G$是李子群.我们解释过$H$的李代数可以典范的视为$G$的李代数的李子代数.即视为子代数$\mathfrak{h}=i_*(\mathrm{Lie}(H))\subset\mathrm{Lie}(G)$,其中$i:H\subset G$是包含映射.记这个李子代数按照上一条所决定的对合分布为$D\subset\mathrm{T}G$.于是对每个$h\in H$,都有$\mathrm{T}_hH=D_h$,所以自然有$H$是$D$的积分流形.
	\end{proof}
	\item 李子代数对应的李子群.设$G$是李群,$\mathfrak{g}$是它的李代数,设$\mathfrak{h}$是$\mathfrak{g}$的李子代数,那么存在$G$的唯一的连通李子群$H$使得它的李代数就是$\mathfrak{h}$.
	\begin{proof}
		
		记$D\subset\mathrm{T}G$是由李子代数$\mathfrak{h}$所确定的$M$上的左不变对合分布,也即定义$D_g=\{X_g\mid X\in\mathfrak{h}\}$.再记$\mathscr{H}$是$D$确定的叶理.对每个$g\in G$,记$\mathscr{H}_g$表示包含$g$的叶.按照$D$是左不变的,我们证明过这等价于叶理$\mathscr{H}$是左不变的,也即对任意$g,g'\in G$,都有$L_g(\mathscr{H}_{g'})=\mathscr{H}_{gg'}$.
		
		取$H=\mathscr{H}_e$,这是连通的积分子流形,我们来证明这就是满足结论的李子群.首先验证它是子群,任取$h,h'\in H$,那么有$hh'=L_h(h')\in L_h(H)=L_h(\mathscr{H}_e)=\mathscr{H}_h=H$.类似的有$h^{-1}=h^{-1}e\in L_{h^{-1}}(\mathscr{H}_e)=L_{h^{-1}}(\mathscr{H}_h)=\mathscr{H}_e=H$.
		
		验证$H$是李群,需验验证$\mu:H\times H\to H,(h,h')\mapsto h(h')^{-1}$是光滑的.这是因为把它视为$G\times G\to G$的映射是光滑的,所以限制为$H\times H\to G$的映射也是光滑的.又因为$H$是积分流形,它是弱嵌入的,所以这个映射限制为$H\times H\to H$也是光滑的.
		
		按照$H$在幺元处的切空间为$D_e=\{X_e\mid X\in\mathfrak{h}\}$,说明$H$的李代数恰好就是$\mathfrak{h}$.最后我们要说明唯一性.假设$\widetilde{H}$是另一个具有相同李代数的连通李子群.那么它也是$D$的积分流形,按照$H=\widetilde{H}_e$的极大性,必须有$\widetilde{H}\subset H$.另一方面如果$U$是关于$D$的幺元附近的平坦坐标邻域,那么积分流形的局部结构定理表明$\widetilde{H}\cap U$是若干切片的开子集的并.其中包含幺元的那个切片是$H$的开子集,所以$\widetilde{H}$包含了幺元在$H$中的一个开邻域.我们证明过连通李群上幺元的任意开邻域生成了整个李群,就导致$\widetilde{H}=H$,完成证明.
	\end{proof}
\end{enumerate}

单参数子群(one-parameter subgroups).李群$G$的单参数子群定义为一个李群同态$\gamma:\mathbb{R}\to G$,其中$\mathbb{R}$典范的视为李群.这不是一个李子群.
\begin{enumerate}
	\item 设$G$是李群,它的单参数子群恰好就是左不变向量场上的以幺元为起始点的极大积分曲线.
	\begin{proof}
		
		先设$\gamma$是$G$上某个左不变向量场的极大积分曲线.我们证明过左不变向量场都是完备的,于是$\gamma$定义在整个$\mathbb{R}$上.对任意$g\in G$,都有$X$和自身是$L_g$相关的,我们解释过此时$L_g$把积分曲线映射为积分曲线.取一个点$g=\gamma(s)$,那么曲线$t\mapsto L_{\gamma(s)}(\gamma(t))$是一条以$\gamma(s)$为起始点的积分曲线.另一方面$t\mapsto\gamma(s+t)$自然也是一条以$\gamma(s)$为起始点的积分曲线,唯一性就说明二者相同,也即有$\gamma(s)\gamma(t)=\gamma(s+t)$.于是$\gamma$是单参数子群.
		
		反过来设$\gamma:\mathbb{R}\to G$是单参数子群,记$X=\gamma_*(\mathrm{d}/\mathrm{d}t)$是$G$上的左不变向量场,其中$\mathrm{d}/\mathrm{d}t$是$\mathbb{R}$上的左不变向量场.按照$\gamma(0)=e$,我们来证明$\gamma$是$X$的积分曲线,而这是因为:
		$$\gamma'(t_0)=\mathrm{d}\gamma_{t_0}\left(\frac{\mathrm{d}}{\mathrm{d}t}\mid_{t_0}\right)=X_{\gamma(t_0)}$$
	\end{proof}
    \item 按照左不变向量场被它在幺元处的取值唯一决定,结合上一条结论,说明$G$上的单参数子群一一对应于$G$上的左不变向量场,一一对应于幺元处的切空间$\mathrm{T}_eG$.我们称左不变向量场$X$对应的单参数子群为$X$生成的单参数子群.
    \item $\mathrm{GL}(n,\mathbb{R})$的单参数子群.任取左不变向量场$A\in\mathfrak{gl}(n,\mathbb{R})$,考虑如下级数,它收敛到一个可逆矩阵$e^A\in\mathrm{GL}(n,\mathbb{R})$,并且$A$生成的单参数子群为$\gamma(t)=e^{tA}$.
    $$e^A=\sum_{k=0}^{\infty}\frac{1}{k!}A^k=I_n+A+\frac{1}{2}A^2+\cdots$$
    \begin{proof}
    	
    	先验证这个矩阵级数的收敛性.矩阵$A=(a_{ij})$上的Frobenius范数为$|A|=\sqrt{|\sum_{ij}a_ij^2|}$.它是范数满足$|AB|\le|A||B|$,所以有$|A^k|\le|A|^k$,所以$|\sum_{k\ge0}\frac{1}{k!}A^k|\le\sum_{k\ge0}\frac{1}{k!}|A|^k=e^{|A|}$.按照Weierstrass判别法有这个矩阵级数收敛.
    	
    	我们之前解释过给定$n$阶矩阵$A$,它对应于一个左不变向量场$A^L$,即$A^L$在$n$阶可逆矩阵$X$处取的向量为$\sum_{i,j,k}X_j^iA_k^j\frac{\partial}{\partial X_k^i}\mid_X$.于是$A$生成的单参数子群就是左不变向量场$A^L$的积分曲线$\gamma$.于是它要满足$\gamma'(t)=A^L\mid_{\gamma(t)}$,$\gamma(0)=I_n$.按照$A^L$的表达式就得到这个条件就是$(\gamma_k^i)'(t)=\sum_j\gamma_j^i(t)A_k^j$.写成矩阵形式就是$\gamma'(t)=\gamma(t)A$.
    	
    	下面我们验证$\gamma(t)=e^{tA}$是光滑的,并且满足这个方程,并且每个$\gamma(t)$是可逆方阵,按照$\gamma(0)=I_n$就得到它是$A^L$的以单位矩阵为起始点的积分曲线,这就说明$\gamma$是$A$生成的单参数子群.首先验证它满足这个方程,这是因为$\gamma'(t)=\sum_{k\ge1}\frac{k}{k!}t^{k-1}A^k=\left(\sum_{k\ge1}\frac{1}{(k-1)!}(tA)^{k-1}\right)A=\gamma(t)A$.并且这个方程说明$\gamma(t)$本身是光滑的.
    	
    	最后验证$\gamma(t)$总是可逆的.记$\sigma(t)=\gamma(t)\gamma(-t)$,那么$\sigma$是$\mathrm{gl}(n,\mathbb{R})$上一条光滑曲线,并且有$\sigma'(t)=(\gamma(t)A)\gamma(-t)-\gamma(t)(A\gamma(-t))=0$,所以有$\sigma(t)\equiv\sigma(0)=I_n$.于是$\gamma(t)\gamma(-t)=I_n$,于是每个$\gamma(t)$是可逆矩阵.
    \end{proof}
    \item 李子群上的单参数子群.设$G$是李群,设$H\subset G$是李子群,那么$H$上的单参数子群恰好是$G$的初始速度落在$\mathrm{T}_eH$中的单参数子群.
    \begin{proof}
    	
    	设$\gamma:\mathbb{R}\to H$是$H$上的一条单参数子群,那么复合映射$\mathbb{R}\to H\subset G$也是李群同态,所以是$G$的单参数子群,此时自然有$\gamma'(0)\in\mathrm{T}_eH$.
    	
    	反过来假设$\gamma:\mathbb{R}\to G$是一条单参数子群,使得$\gamma'(0)\in\mathrm{T}_eH$.记$\widetilde{\gamma}:\mathbb{R}\to H$是单参数子群,使得$(\widetilde{\gamma})'(0)=\gamma'(0)\in\mathrm{T}_eH$(这可实现因为我们解释过单参数子群一一对应于左不变向量场,一一对应于幺元处的向量).把$\widetilde{\gamma}$复合上包含映射$H\subset G$,这样得到初始速度相同的$G$上的单参数子群,所以它们必然是相同的.
    \end{proof}
\end{enumerate}

指数映射.设$G$是子群,设$\mathfrak{g}$是它的李代数,它的指数映射(exponential map)为$\exp:\mathfrak{g}\to G$,$X\mapsto\gamma(1)$,其中$\gamma$是$X$生成的单参数子群.
\begin{enumerate}
	\item 设$G$是李群,任取左不变向量场$X$,那么$\gamma(s)=\exp sX$就是由$X$生成的单参数子群.
	\begin{proof}
		
		设$\gamma:\mathbb{R}\to G$是由$X$生成的单参数子群,我们解释过此即$\gamma$是$X$的以幺元$e$为起始点的积分曲线.于是对固定的$s\in\mathbb{R}$,就有$\widetilde{\gamma}(t)=\gamma(st)$是$sX$的以$e$为起始点的积分曲线.所以有$\exp sX=\widetilde{\gamma}(1)=\gamma(s)$.
	\end{proof}
    \item 例如在李群$\mathrm{GL}(n,\mathbb{R})$上指数映射即$\exp A=e^A\in\mathrm{GL}(n,\mathbb{R})$.再例如$\mathbb{R}^n$作为李群时左不变向量场等同于$\mathbb{R}^n$本身,选取一个左不变向量场$s\in\mathbb{R}^n$,那么它生成的单参数子群就是$\gamma(x)=xs$,此时指数映射即$\exp:\mathbb{R}^n\to\mathbb{R}^n$为$s\mapsto s$.
    \item 指数映射是$\mathfrak{g}\to G$的光滑映射.
    \begin{proof}
    	
    	对$X\in\mathfrak{g}$,我们记它的流为$\theta_X$,所以我们要证明$\theta_X^e(1)$关于$X$是光滑的.流的基本定理并没有覆盖这一内容,所以这里我们需要给出证明.定义乘积流形$G\times\mathfrak{g}$上的向量场$V$为$V_{(g,X)}=(X_g,0)\in\mathrm{T}_gG\oplus\mathrm{T}_x\mathfrak{g}\cong\mathrm{T}_{(g,x)}(G\times\mathfrak{g})$.取$\{X_1,\cdots,X_k\}$是$\mathfrak{g}$的一组基,再记$(x^i)$表示它对应的$\mathfrak{g}$的整体标架(即坐标$(x^i)$对应于向量场$\sum_ix^iX_i$).任取$G$的局部坐标$(w^i)$,如果$f$是$G\times\mathfrak{g}$上任意的光滑函数,那么局部上有坐标表示$Vf(w^i,x^i)=\sum_jx^jX_jf(w^i,x^i)$,右侧是光滑函数,所以$V$是光滑向量场.那么$V$的流$O$是$O_t(g,X)=(\theta_X(t,g),X)$.按照流的基本定理得到$O$是光滑的.按照$\exp X=\pi_G(\theta_1(e,X))$,其中$\pi_G:G\times\mathfrak{g}\to G$是投影映射,这就得到$\exp$是光滑映射.
    \end{proof}
    \item 指数映射的一些基本性质.设$X\in\mathfrak{g}$和$s,t\in\mathbb{R}$.
    \begin{itemize}
        \item $\exp(s+t)X=\exp sX\exp tX$.
        \item $(\exp X)^{-1}=\exp(-X)$.
        \item $(\exp X)^n=\exp nX$.
    \end{itemize}
    \item 微分映射$(\mathrm{d}\exp)_0:\mathrm{T}_0\mathfrak{g}\to\mathrm{T}_eG$是恒等映射,其中$\mathrm{T}_0\mathfrak{g}$和$\mathrm{T}_eG$都典范的等同于$\mathfrak{g}$本身.特别的,结合反函数定理说明指数映射可以限制为$0\in\mathfrak{g}$到$e\in G$的某个开邻域之间的微分同胚.
    \begin{proof}
    	
    	任取$X\in\mathfrak{g}$,考虑曲线$\sigma:\mathbb{R}\to\mathfrak{g}$为$t\mapsto tX$,那么$\sigma'(0)=X$.于是有:
    	$$(\mathrm{d}\exp)_0(X_e)=(\mathrm{d}\exp)_0(\sigma'(0))=(\exp\circ\sigma)'(0)=\frac{\mathrm{d}}{\mathrm{d}t}\mid_{t=0}\exp tX=\gamma'(0)=X_e$$
    \end{proof}
    \item 如果$\Phi:G\to H$是李群同态,那么有如下交换图表:
    $$\xymatrix{\mathfrak{g}\ar[rr]^{\Phi_*}\ar[d]_{\exp}&&\mathfrak{h}\ar[d]^{\exp}\\G\ar[rr]_{\Phi}&&H}$$
    \begin{proof}
    	
    	我们要证明的是对任意$X\in\mathfrak{g}$都有$\exp(\Phi_*X)=\Phi(\exp X)$.这里我们来证明对任意实数$t$都有$\exp(t\Phi_*X)=\Phi(\exp tX)$.我们解释过这里左侧就是向量场$\Phi_*X$生成的单参数子群.所以如果记$\sigma(t)=\Phi(\exp tX)$,只需验证$\sigma:\mathbb{R}\to H$是李群同态并且$\sigma'(0)$.它是李群同态因为它是两个李群同态$\Phi$和$t\mapsto\exp tX$的复合,所以是李群同态.它的初始速度为:
    	$$\sigma'(0)=\frac{\mathrm{d}}{\mathrm{d}t}\mid_{t=0}\Phi(\exp tX)=\mathrm{d}\Phi_0\left(\frac{\mathrm{d}}{\mathrm{d}t}\mid_{t=0}\exp tX\right)=\mathrm{d}\Phi_0(X_e)=(\Phi_*X)_e$$
    \end{proof}
    \item 左不变向量场$X$的流$\theta$满足$\theta_t=R_{\exp tX}$,即右乘$\exp tX$的映射.
    \begin{proof}
    	
    	我们知道左乘$g$的映射$L_g$把$X$的积分曲线映射为$X$的积分曲线,所以$t\mapsto L_g(\exp tX)$是$X$的以$g$为起始点的积分曲线,所以它就是$\theta_X^g(t)$.那么有:
    	$$R_{\exp tX}(g)=g\exp tX=L_g(\exp tX)=\theta_X^g(t)=(\theta_X)_t(g)$$
    \end{proof}
    \item 李子群的李子代数的另一种描述.设$G$是李群,设$H\subset G$是李子群,我们解释过$\mathrm{Lie}(H)$可视为$\mathrm{Lie}(G)$的李子代数.那么$H$上的指数映射恰好就是$G$上指数映射在$\mathrm{Lie}(H)$上的限制.并且有:$$\mathrm{Lie}(H)=\{X\in\mathrm{Lie}(G)\mid\exp tX\in H,\forall t\in\mathbb{R}\}$$
    \begin{proof}
    	
    	$H$上的指数映射在某个$X\in\mathrm{Lie}(H)$上的取值就是$X$对应的单参数子群在1处的取值,但是我们解释过$H$上的单参数子群恰好就是$G$的满足初始速度落在$\mathrm{T}_eH$中的单参数子群.这解决了第一个命题.对于第二个命题,假设$G$上的左不变向量场$X$满足$X_e\in\mathrm{T}_eH$,那么$X$生成的单参数子群恰好也是$H$的单参数子群,导致$\exp tX\in H,\forall t\in\mathbb{R}$.反过来假设$X$是$G$上的左不变向量场,满足对任意$t\in\mathbb{R}$都有$\exp tX\in H$.由于李子群都是弱嵌入子流形,所以$t\mapsto\exp tX$视为$\mathbb{R}\to H$的映射也是光滑的.并且有$X_e=\gamma'(0)=\mathrm{T}_eH$.这说明了第二个命题.
    \end{proof}
\end{enumerate}

闭子群定理.我们之前证明过李群的一个李子群是嵌入子流形当且仅当它拓扑上是闭的.这里我们借助指数映射证明可以把条件李子群弱化为子群.
\begin{enumerate}
	\item 设$G$是李群,设$\mathfrak{g}$是它的李代数,任取$X,Y\in\mathfrak{g}$,存在光滑函数$Z:(-\varepsilon,\varepsilon)\to\mathfrak{g}$满足对任意$t\in(-\varepsilon,\varepsilon)$都有:
	$$(\exp tX)(\exp tY)=\exp(t(X+Y)+t^2Z(t))$$
	\begin{proof}
		
		我们解释过指数映射在$\mathfrak{g}$的某个开邻域上的限制是微分同胚.所以存在足够小的$\varepsilon>0$使得$\varphi:(-\varepsilon,\varepsilon)\to\mathfrak{g}$定义为$t\mapsto\exp^{-1}(\exp tX\exp tY)$是光滑的.明显有$\varphi(0)=0$和$\exp tX\exp tY=\exp\varphi(t)$.把$\varphi$视为如下映射链的复合:
		$$\xymatrix{\mathbb{R}\ar[r]^{e_X\times e_Y}&G\times G\ar[r]^m&G\ar[r]^{\exp^{-1}}&\mathfrak{g}}$$
		
		其中$e_X(t)=\exp tX$和$e_Y(t)=\exp tY$.按照$\mathrm{d}m_{(e,e)}(X,Y)=X+Y$.就得到$\varphi(0)=((\mathrm{d}\exp)_0)^{-1}(e'_X(0)+e'_Y(0))=X+Y$.所以泰勒定理就保证存在光滑函数$Z$使得$\varphi(t)=t(X+Y)+t^2Z(t)$.
	\end{proof}
    \item 闭子群定理.设$G$是李群,设$H\subset G$是子群(没有李子群条件),并且拓扑上$H$是闭的,那么$H$是$G$的嵌入李子群.
    \begin{proof}
    	
    	【】
    \end{proof}
    \item 推论.设$G$是李群,设$H$是任意子群,那么如下条件互相等价:
    \begin{itemize}
    	\item $H$拓扑上是闭的.
    	\item $H$是$G$的嵌入子流形.
    	\item $H$是$G$的嵌入李子群.
    \end{itemize}
\end{enumerate}

李对应定理.由于李群到李代数是一个函子,所以同构的李群具有同构的李代数.但是反过来这是不成立的,例如$\mathbb{R}^n$和$\mathbb{T}^n$是不同构的李群,但是它们的李代数都是$n$维的阿贝尔李代数(阿贝尔李代数是指李括号恒取零的李代数,阿贝尔李群的李代数总是阿贝尔的).不过如果我们考虑的是单连通李群那么这件事是成立的.
\begin{enumerate}
	\item 
	
	
	
\end{enumerate}





为什么把李群的子群定义为immersed子流形?这是因为在这个定义下可以陈述李群理论的一个基本定理:一个李群的连通子群和它的李代数的子代数是一一对应的.如果把定义改为正则子流形将不会有这样的一一对应.

给定域$k$上的李代数$V$,称$V$上的一个李导数是指$V$上的一个$k$线性映射$D$,满足:
$$D[Y,Z]=[DY,Z]+[Y,DZ],\forall Y,Z\in V$$

注意如果记$ad_X(Y)=[X,Y]$,那么按照Jacobi恒等式,$ad_X$总是一个李导数.





【流形上的覆盖空间】
