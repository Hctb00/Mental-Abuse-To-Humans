\chapter{分析学基础}
\section{实数的构造}

按照集合论的处理,自然数集的定义为最小的归纳集.可对自然数集添加加法逆元得到整数集$\mathbb{N}$,它具有环结构和序结构.从$\mathbb{N}$构造有理数集$\mathbb{Q}$需要代数中一种称为分式化的手段.此时$\mathbb{Q}$具有域结构和序结构.人们最初对数字的理解是,能够存在的数字是能够度量几何上长度的数字.古希腊时期人们已经意识到有些存在的数字不能表示为有理数,即在由"存在"的数字构成的连续的一条直线上,有许多不能被有理数填充的缝隙.这些缝隙就是所谓的无理数.从$\mathbb{Q}$出发构造实数集通常被称为完备化,这里的完备指的就是填充这些缝隙.

本节的目标就是详细给出从有理数集构造实数集的两种的具体做法:戴德金的分割构造和康托的柯西序列构造.最后证明两种构造得到的实数集合是一致的.

第一个构造是戴德金分割(cuts).称$\mathbb{Q}$的一个戴德金分割是指$\mathbb{Q}$的一个子集$A$,满足如下三个要求:$B=\mathbb{Q}-A$非空;$A$中的每个有理数都严格小于$B$中每个有理数;并且$A$作为$\mathbb{Q}$的全序子集不存在最大元.定义实数集$\mathbb{R}$为全体戴德金分割构成的集合.在给出两种实数构造的等价性之前,我们先把戴德金构造的实数集记作$\mathbb{R}_D$以便区分.

现在来对$\mathbb{R}_D$上定义加法,序结构和乘法,并说明它具备全序域结构.
\begin{enumerate}
	\item 加法.若$r,s\in\mathbb{R}_D$,定义加法为$r+s=\{p+q\mid p\in r,q\in s\}$.验证它的确是一个戴德金分割.它满足结合律与交换律.注意全体负有理数构成一个戴德金分割,它就是加法零元,为了和有理数0区分,把它记作$0'$.给定戴德金分割$r$,它的加法逆元是$-r=\{q\in\mathbb{Q}\mid \exists p>q:-p\in\mathbb{Q}\backslash r\}$.综上得到$\mathbb{R}_D$的加法群结构.
	\item 序结构.若$r,s\in\mathbb{R}_D$,定义$r<s$当且仅当$r\subsetneqq s$.需要验证对任意的$r,s\in\mathbb{R}_D$,有$r<s,s<r,s=r$三者恰好存在一个成立.为此假设$r\not=s$,那么$r\backslash s$和$s\backslash r$至少有一个非空,假设$r\backslash s$非空,任取一个有理数$q$,那么小于$q$的有理数都在$r$中,而$q\not\in s$说明$s$中的有理数都小于$q$,这说明$s\subsetneqq r$.综上得到$\mathbb{R}_D$上的全序结构.称大于$0'$的实数为正实数,小于$0'$的实数为负实数.加法和序结构是兼容的,即$r<s,p<q$可得$r+p<s+q$.
	\item 乘法.给定两个非负实数$r,s$,定义乘法为$rs=\{pq\mid p\in r\backslash 0',q\in s\backslash 0'\}\cup 0'$.可定义绝对值函数$|r|$,借助绝对值定义任意两个实数的乘积的符号.全体小于1的有理数构成了一个分割,记作$1'$,它就是乘法幺元.乘法逆元可以直接构造出来:如果$s>0$,则$s^{-1}=\{q\in\mathbb{Q}\backslash\{0\}\mid\exists p\in\mathbb{Q}\backslash s:p<q^{-1}\}\cup 0'\cup\{0\}$;如果$s<0$则$s^{-1}=-|s|^{-1}$.乘法同样满足交换律和结合律.最后乘法和加法是兼容的,即有分配律;乘法和序结构也是兼容的,即不等式两侧乘以正实数不变号负实数变号.这就使得$\mathbb{R}_D$是一个全序域.
\end{enumerate}

任意全序域都可以定义戴德金分割.在这个实数构造下,有理数集的不完备性在于存在戴德金分割,它作为有理数集的子集没有上确界,例如$0'$这个戴德金分割.一个全序域称为戴德金完备的,就是指任意戴德金分割都有上确界.于是实数域的完备性可以描述为:$\mathbb{R}_D$是戴德金完备的.
\begin{proof}
	
	设$A$是$\mathbb{R}_D$的一个戴德金分割.那么$A$中的每个元都是$\mathbb{Q}$上的戴德金分割.记这些有理数子集的并为$a$,我们来说明$a$是$\mathbb{Q}$上的戴德金分割,这就说明了$a\in\mathbb{R}_D$.再证明$a$就是$A$的上确界,这就完成了证明.
	
	按照定义可取一个$\mathbb{Q}$上的戴德金分割$b\in\mathbb{R}_D\backslash A$.于是对每个$c\in A$,有$c\subsetneqq b$.于是$a=\cup A\subset b$.于是$a\subsetneqq b+1$,这说明$\mathbb{Q}\backslash a$非空.
	
	任取有理数$q\in a$,任取有理数$p\in\mathbb{Q}\backslash a=\cap_{c\in A}(\mathbb{Q}\backslash c)$.记小于$p$的全部有理数构成的集合为$p'$,那么$p'\in\mathbb{R}_D$.由于$p\not\in c$对任意$c\in A$成立,说明$c\subsetneqq p'$,于是$a\subset p'$,于是$q\in p'$,于是$q<p$.
	
	假设$a$有上确界$s\in\mathbb{Q}$.于是$s$同样是每个$c\in A$的上界,导致$s\not\in c,\forall c\in A$,这说明$c\not\in a$矛盾.至此说明了$a$的确是$\mathbb{Q}$上的一个分割,即$a\in\mathbb{R}_D$.
	
	最后说明$a$是$A$的上确界.已经有$a$是$A$的上界,假设$A$还存在上界$b\in\mathbb{R}_D$.如果$b<a$,即$b\subsetneqq a$,取有理数$t\in a\backslash b$,得到$b\subsetneqq t'\subsetneqq a$,于是$t\in A$,得到$b<t$,这和$b$是$A$的上界就矛盾.于是$a$是$A$的上确界.
\end{proof}

一个比戴德金完备性更为常用的完备性描述是确界原理.即全序域满足确界原理当且仅当,它的任一个非空子集如果有上界就有上确界.在全序域上确界原理自然可以推出戴德金完备.这里我们来证明$\mathbb{R}_D$还满足确界原理.
\begin{proof}
	
	取$\mathbb{R}_D$的非空子集$A$,假设它存在上界.记$B=\{x\in\mathbb{R}_D\mid\exists y\i A:x<y\}$.验证$B$是$\mathbb{R}_D$上的戴德金分割,并且$B$的上确界就是$A$的上确界.
\end{proof}

现在我们介绍第二个构造.记$\mathbb{Q}$上的全体柯西列构成的集合为$C$.那么$C$是一个交换环,它的幺元为恒取1的柯西列,记作$(1)$.考虑$C$中的极限为0的序列构成的子集$I$,按照柯西列有界,说明$I$是交换环$R$的理想.商环$C/I$实际上是一个域,这只要注意到如果$(d_n)\in C$不是极限为0的序列,说明存在正整数$N$使得$n>N$时$d_n\not=0$,于是只要取$(e_n)$为当$d_n\not=0$时$e_n=d_n^{-1}$,当$d_n=0$时$e_n=0$,得到$(d_n)(e_n)-(1)\in I$,于是$C/I$每个非0元都是单位,于是它是域.

注意这个操作等价于在$C$上定义一个等价关系$\cong$为,$(a_n)\cong(b_n)$当且仅当$\lim_{n\to+\infty}(a_n-b_n)=0$.$C/I$就是这个等价关系的全体等价类构成的集合.但是如果这样处理还需要验证$C/I$上的域结构,不如直接借助$C$的交换环结构直接运用商环的操作.

定义域$C/I$是实数域,为了和戴德金构造区分,将它记作$\mathbb{R}_C$.现在还需要说明它具有和域结构兼容的序结构.我们把$(a_n)\in C$在商环$C/I$中的对应为$[a_n]$.把恒取一个固定有理数$q$的柯西列记作$(q)$,它在商环中的对应记作$[q]$.现在定义$[a_n]<[b_n]$当且仅当存在正有理数$\delta$使得存在正整数$N$满足$n>N$时候有$a_n+\delta<b_n$.首要说明的则是这个定义良性,和域结构兼容可直接验证,这就说明了$\mathbb{R}_C$是全序域.
\begin{proof}
	
	假设$(a_n),(a_n'),(b_n),(b_n')\in C$,满足$(a_n)\cong(a_n')$和$(b_n)\cong(b_n')$.存在正整数$N_1$和正有理数$\delta$使得$n>N_1$时候恒有$a_n>b_n+\delta$.那么按照等价关系的定义,可取正整数$N_2$使得$n>N_2$的时候有$a_n'>a_n-\frac{\delta}{3}$和$b_n+\frac{\delta}{3}>b_n'$.于是当$n>\max\{N_1,N_2\}$时候就有如下不等式,完成证明.
	$$a_n'>a_n-\frac{\delta}{3}>b_n+\frac{2}{3}\delta>b_n'+\frac{\delta}{3}$$
\end{proof}

接下来仍要描述完备性.一个全序域称为柯西完备的,如果每个柯西列都是收敛列.那么$\mathbb{R}_C$是柯西完备的.
\begin{proof}
	
	记$b_m=(a^m_n)$是$\mathbb{Q}$上的柯西列,约定$(b_m)$是$\mathbb{R}_C$上的柯西列.于是存在正整数$L_m$使得$n,n'>L_m$的时候恒有$|a^m_n-a^m_{n'}<\frac{1}{m}|$.不妨约定$\{L_m\}$是严格递增的正整数序列,构造一个$\mathbb{Q}$上的序列为$(c_m)=(a^m_{L_m})$.我们断言$[c_m]$是实数并且它就是$[b_m]$的极限.
	
	先验证$(c_m)$的确是$\mathbb{Q}$上的柯西列.先取正实数$\varepsilon>0$,可取足够大的正整数$N$使得$m,m'>N$时有$\mathbb{R}_C$上的不等式关系$|b_m-b_{m'}|<\frac{(\varepsilon)}{2}$.这说明存在正整数$M$使得$n>M$的时候有$|a^m_n-a^{m'}_n|<\frac{\varepsilon}{2}$.取正整数$K\ge\max\{N,M,\frac{2}{\varepsilon}\}$,那么$k>k'>K$的时候恒有:
	$$|c_k-c_{k'}|=|a_{L_k}^k-a_{L_{k'}}^{k'}|\le|a_{L_k}^k-a_{L_k}^{k'}|+|a_{L_k}^{k'}-a_{L_{k'}}^{k'}|<\frac{\varepsilon}{2}+\frac{1}{K}<\varepsilon$$
	
	这就得到$(c_m)$是$\mathbb{Q}$上的柯西列.最后验证$(b_m\in\mathbb{R}_C)$趋于$(c_m)\in\mathbb{R}_C$.固定$m$,需要估计$|(a^m_n)-(c_m)|$,其中$|a^m_n-a^m_{L_n}|<\frac{2}{n}-\frac{1}{n}$对任意$n\ge L_n$成立,说明$|(a^m_n)-(c_m)|<\frac{2}{n}$.让$n\to\infty$就得到$b_m$的极限是$(c_m)$,完成证明.
\end{proof}

这里总结一下上述两种构造.戴德金分割的构造是先在在有理数集上定义戴德金分割,然后把实数定义为全体戴德金分割,其中完备性描述的就是实数集上的每个戴德金分割被唯一一个实数所表示,它也就是戴德金分割的上确界,于是存在实数集上戴德金分割到实数集自身的一一对应.康托的柯西数列的构造是先定义有理柯西列上的等价关系,把全体等价类构成的集合定义为实数集.此时完备性描述的是实数上的柯西列被一个实数唯一决定,它也就是柯西列收敛到的极限,此时实数集上的柯西列到实数集自身是一一对应的.

关于多种构造的等价性.在上述构造中我们介绍了一共三种完备性,即戴德金完备,确界原理,柯西完备(柯西收敛准则).事实上任意两个全序域如果分别满足三种完备性的任一个,则它们是序同构的全序域.这个证明可见Mickael Spivak的<Calculus>.

还存在一些涉及到实数完备性的命题,它们通常也叫做实数连续性命题.有的教材会指出它们是彼此等价的命题,实际上把它们理解为$\mathbb{R}$上能够揭示完备性本质的并且根源自实数构造本身的命题更为准确.
\begin{enumerate}
	\item 确界原理:$\mathbb{R}$上的非空子集如果存在上界,则存在上确界.
	\item 确界原理的数列描述:$\mathbb{R}$上的有上界的单调增点列必然收敛.
	\item 闭区间套定理:设$\mathbb{R}$上递减的闭区间序列$[a_1,b_1]\supset[a_2,b_2]\supset\cdots$满足$b_n-a_n$趋于0,那么存在恰好一个实数$x_0$落在$\cap_{n\ge1}[a_n,b_n]$中.
	\item Heine-Borel有限覆盖定理:$\mathbb{R}$的有界闭区间$[a,b]$是紧集.
	\item Weierstrass聚点定理:$\mathbb{R}$的有界无限点集必然有聚点.
	\item Bolzano-Weierstrass定理:$\mathbb{R}$的有界数列必有收敛子列.
	\item 柯西收敛定理:$\mathbb{R}$的柯西列必然是收敛列.
\end{enumerate}
\newpage
\section{微分学}
\subsection{可微性}

当提及实函数时,总是指$\mathbb{R}$的某个子集到$\mathbb{R}$的映射,若不加说明,这个子集总是指开集.当提及实值函数的时候,总是指某个实赋范空间的某个子集到$\mathbb{R}$的函数,若不加说明同样这个子集约定为开集.当提及赋范空间时总是指实的赋范空间.

给定两个实赋范空间$X,Y$,取$X$的一个开子集$U$,设$f,g$是从$U$到$Y$的两个映射.称$f,g$在内点$x_0\in U$相切,如果满足:
$$\lim_{x\to x_0,x\not=x_0}\frac{\| f(x)-g(x)\|_Y}{\| x-x_0\|_X}=0$$

注意$f,g$在$x_0$相切蕴含了$f(x_0)=g(x_0)$.另外相切这个概念只依赖于赋范空间上的拓扑,如果把$X,Y$上的范数分别改为等价的范数,不会改变相切的关系.另外相切是函数空间上的等价关系,即如果$f,g$和$g,h$都在$x_0\in U$处相切,那么按照$\|f(x)-h(x)\|\le\|f(x)-g(x)\|+\|g(x)-h(x)\|$就得到$f,h$也是相切的.

在所有的在$x_0$处和连续函数$f$相切的函数中,最多只能有一个函数具有形式$x\mapsto f(x_0)+u(x-x_0)$,其中$u$是$X\to Y$的线性映射.这是因为若否,那么存在$x\mapsto f(x_0)+u_1(x-x_0)$和$x\mapsto f(x_0)+u_2(x-x_0)$在$x_0$处相切,这导致,如果记$v=u_1-u_2$是线性函数,那么有$\lim_{y\to0,y\not=0}\frac{\|v(y)\|}{\|y\|}=0$,这说明对于任意的$\varepsilon>0$,存在$r>0$使得只要$\|y\|\le r$,就有$\|v(y)\|\le\varepsilon\|y\|$.但是如果把$y$替换为$rx/\|x\|$,看出实际上这个不等式对任意的$y\not=0,y\in X$成立,再结合$\varepsilon$的任意性,得到对每个$y\in X$有$v(y)=0$,就得到$v_1=v_2$.

称一个从$U$到$Y$的映射$f$在点$x_0\in U$是可微的,如果存在一个从$X$到$Y$的连续线性映射$u$使得$f$和$x\mapsto f(x_0)+u(x-x_0)$在$x_0$处相切.那么已经证明了如果$u$存在则必然唯一,称线性映射$u$为$f$在$x_0$处的导数或全导数,记作$f'(x_0)$或者$Df(x_0)$.如果映射$f:U\to Y$是处处可微的,那么可以定义一个$U\to L(X,Y)$的映射$f':x\mapsto f'(x)$,称为映射$f$的导函数.

这里指出,如果$f$在一个点$x\in U$处可微,把导数理解为线性变换,是从点$x$的切空间$TX_x$,到$f(x)$的切空间$TY_{f(x)}$的线性变换,只不过$TX_x\simeq X$,$TY_{f(x)}\simeq Y$,因而简单说是从$X\to Y$的线性映射.

特别的,在有限维的情况下,即如果$U\subset\mathbb{R}^m$是开集,称$f:U\to\mathbb{R}^n$在$x\in U$处是可微的,如果存在线性映射$L:\mathbb{R}^m\to\mathbb{R}^n$满足$f(x+h)-f(x)=L(x)h+o(h)$,其中当$|h|\to0$的时候有$|o(h)|\to0$.注意有限维上线性变换总是连续的.满足这里的线性变换$L(x)$如果存在则是唯一的,称为$f$在$x$处的导数.另外知道$\mathbb{R}^m\to\mathbb{R}^n$的线性变换可以表示为一个$n\times m$的实矩阵.这称为$f$在$x$处的Jacobi矩阵.

这个定义初看似乎和实直线上的情况不同,事实上对于一维的情况,$\mathbb{R}\to\mathbb{R}$的线性变换具有形式$x\mapsto\lambda x$,其中$\lambda\in\mathbb{R}$.因而,可以把一维上的线性变换等同于实数本身,严格说就是总有$L(\mathbb{R},X)\simeq X$.这导致一维情况对可微的定义就是:给定$\mathbb{R}$上开集$U$到$\mathbb{R}$的映射$f$,称$f$在$x_0\in U$处可微,如果存在一个$\mathbb{R}$上的线性变换$L$,或者等价于说存在一个实数$L$,使得$f(x_0+h)-f(x_0)=L(h)+o(h),h\to0$.如果$f$在$U$上处处可微,那么这时候的导函数比较特殊,它还是一个$U\to\mathbb{R}$的映射.由于一维情况的特殊性,往往把实函数的可微性称为可导性.

按照的定义,可微需要存在连续线性映射,这说明了可微点必然是连续点.关于$\mathbb{R}$上的连续性是否说明可导曾经困扰数学界很长时间,第一个发表的反例是1872年发表的Weierstrass函数,这个函数甚至是处处不可导的连续函数.

这里总结下导数的一些法则.
\begin{enumerate}
	\item 积空间的导数1.设$Y=Y_1\times Y_2\times\cdots\times Y_m$是赋范空间的积,设$f=(f_1,\cdots,f_m)$是从赋范空间$X$的某个开子集$U$到$Y$的连续函数.那么$f$在点$x_0\in U$处可微的充要条件是,每个$f_i$都在$x_0$处可微,并且有$f'(x_0)=(f_1'(x_0),\cdots,f_m'(x_0))$,其中把$L(X,Y)$同构于$\prod_{1\le i\le m}L(X,Y_i)$.
	\item 导数的链式法则.给定三个赋范空间$E,F,G$,设$A$是某个$x_0\in E$的开邻域,设$f$是从$A$到$F$的连续映射,记$y_0=f(x_0)\in F$,取$y_0\in F$的一个开邻域$B$,设$g$是从$B$到$G$的连续映射.那么如果$f$在$x_0$处可微,$g$在$y_0$处可微,那么复合映射$h=g\circ f$在$x_0$处可微,并且满足$h'(x_0)=g'(y_0)\circ f'(x_0)$.
	\begin{proof}
		
		按照定义,对任意的$1>\varepsilon>0$,存在$r>0$使得对$\|s\|\le r$和$\|t\|\le r$,有:
		$$f(x_0+s)=f(x_0)+f'(x_0)s+o_1(s)$$
		$$g(y_0+t)=g(y_0)+g'(y_0)t+o_2(t)$$
		
		其中$\|o_1(s)\|\le\varepsilon\|s\|$且$\|o_2(t)\|\le\varepsilon\|t\|$.另外按照导数是有界线性映射,存在正数$a,b$,对任意$s,t$满足:
		$$\|f'(x_0)s\|\le a\|s\|;\|g'(y_0)t\|\le b\|t\|$$
		
		于是对任意$\|s\|\le r$,有:
		$$\|f'(x_0)s+o_1(s)\|\le(a+1)\|s\|$$
		
		那么对$\|s\|\le\frac{r}{a+1}$,就得到:
		$$\|o_2(f'(x_0)s+o_1(s))\|\le(a+1)\varepsilon\|s\|;\|g'(y_0)o_1(s)\|\le b\varepsilon\|s\|$$
		
		这得到:
		$$h(x_0+s)=g(f(x_0)+f'(x_0)s+o_1(s))=g(y_0)+g'(y_0)(f'(x_0)s)+o_3(s)$$
		
		其中:
		$$\|o_3(s)\|\le(a+b+1)\varepsilon\|s\|$$
		
		这就得证.
		
	\end{proof}
	\item 导数的加法和数乘.可以借助上述法则或者利用定义证明,$f,g$如果在$x_0$处可微,那么$(f+g)'(x_0)=f'(x_0)+g'(x_0)$,并且对任意的系数$\alpha$有$(\alpha f)'(x_0)=\alpha f'(x_0)$.如果借助上述法则,需要把$f+g$看作从$F\times F$到$F$的连续映射$(u,v)\mapsto u+v$和从$A\subset E\to F$的连续映射$x\mapsto(f(x),g(x))$的复合,前者是线性映射,后者是可微映射的积映射,于是它们都可微,于是复合是可微的,后一个映射在$x_0$处的导数是$(f'(x_0),g'(x_0))$,前一个映射是线性映射,导数是自身,于是它们的复合就是$f'(x_0)+g'(x_0)$.数乘是同理的,可以看作可微映射$f(x)$和$F$上的线性映射$y\mapsto\alpha y$的复合,前者导数是$f'(x_0)$,后者是自身,于是复合是$\alpha f'(x_0)$.
	\item 逆映射的导数.给定两个赋范空间$E,F$,如果$A$是$E$的开子集,$B$是$F$的开子集,如果存在$A\to B$的同胚$f$,如果$f$在$x_0\in A$处可微,并且$f'(x_0)$是$E\to F$的线性同胚,那么$f$的逆映射$g$在$y_0=f(x_0)$处可微,并且$g'(y_0)$是$f'(x_0)$的逆线性映射.
	\begin{proof}
		
		映射$s\mapsto f(x_0+s)-f(x_0)$是从$E$的0元的某个开邻域$V$到$F$的0元的某个开邻域$W$的同胚,并且具有逆连续映射$z:t\mapsto g(y_0+t)-g(y_0)$.按照条件,$f'(x_0)$具有连续的逆映射$u$.于是存在$c>0$使得$\|u(t)\|\le c\|t\|,\forall t\in F$.
		
		现在对任意的$\varepsilon>0$,不妨设$\varepsilon\le\frac{1}{2c}$,那么存在$r>0$使得,如果记$f(x_0+s)-f(x_0)=f'(x_0)s+o_1(s)$,从$\|s\|\le r$可以推出$\|o_1(s)\|\le\varepsilon\|s\|$.现在取足够小的正数$r'$使得$\|t\|\le r'$包含在$W$中,并且这个闭球在$t\mapsto g(y_0+t)-g(y_0)$下的像包含在$\|s\|\le r$中.那么如果$\|t\|\le r'$,就有$t=f(x_0+z)-f(x_0)$,并且由于$\|z\|\le r$,就有$t=f'(x_0)z+o_1(z)$,其中$\|o_1(z)\|\le\varepsilon\|z\|$,于是:
		$$ut=u(f'(x_0)z)+uo_1(z)=z+uo_1(z)$$
		
		注意到$\|uo_1(z)\|\le c\|o_1(z)\|\le c\varepsilon\|z\|\le\frac{\|z\|}{2}$,于是得到$\|ut\|\ge\|z\|-\frac{\|z\|}{2}=\frac{\|z\|}{2}$.于是$\|z\|\le2\|ut\|\le 2c\|t\|$,得到$\|uo_1(z)\|\le c\varepsilon\|z\|\le2c^2\varepsilon\|t\|$.综上证明了存在$r'$使得$\|t\|\le r'$时候推出$\|g(y_0+t)-g(y_0)-ut\|\le2c^2\varepsilon\|t\|$.于是说明$u$是$g$在$y_0$处导数.
		
	\end{proof}
	
	这里强调,如果$f$仅仅作为$A\to B$的可微同胚,甚至不能保证$f'(x_0)$是满射,例如$\mathbb{R}\to\mathbb{R}^3$的映射$x\mapsto x^3$.
	\item 如果取$Y=\mathbb{R}$,那么可以定义映射的乘法和除法,它们有法则$(fg)'=f'g+g'f$,$(\frac{f}{g})'=\frac{gf'-g'f}{g^2}$.
\end{enumerate}

这里来举一些例子.对于常值函数,它处处可微,并且导数就是零线性映射,也即$L(E,F)$中的0元.对于$E\to F$的连续线性映射$u$,按照$u(x)=u(x_0)+u(x-x_0)$得到它处处可微,并且每一点的导数都是$u$本身.最后,连续的多重线性映射是处处可微的.设$f$是从赋范空间的积$E_1\times E_2\times\cdots\times E_m=E$到赋范空间$F$的连续多重线性映射,通过分解$f(x+h)-f(x)$,得到$f$在点$x=(x_1,x_2,\cdots,x_m)$处的导数为$f'(x)h=\sum_{k=1}^{m}f(x_1,\cdots,h_k,\cdots,x_m)$.或者写作$Df(x_1,\cdots,x_m)=\sum_{k=1}^{m}Df(x_1,\cdots,Dx_k,\cdots,x_m)$.

对于赋范空间之间的映射$f:U\subset X\to Y$,期望探究$X,Y$是积空间情况下,可微性是否被分量的可微性所刻画.这样做不光可以简化对可微性问题,而且它可以完全描述有限维赋范空间之间映射的可微性.如果$Y$是积赋范空间,这个情况已经在前面讨论过了,此时$f$可以记作$(f_1,f_2,\cdots,f_m)$,$f$在一个点处可微等价于每个$f_i$在该点可微.现在反过来,如果约定$X$是积空间,则此时对可微性的探究需要偏导数这个概念.

偏导数.给定点$a=(a_i)\in X=X_1\times\cdots\times X_m$的一个开邻域$U=U_1\times\cdots\times U_m$,取从$U\to Y$的映射$f$.现在如果固定除$a_j$以外的全部$a_i$,那么可以得到一个$X_j\to Y$的映射$\phi_j(x_j)=f(a_1,\cdots,x_j,\cdots,a_m)$.把$\phi_j$称为关于分量$x_j$的偏映射,如果$\phi_j$在$a_j$处可导,这个导数称为$f$在$a$处关于$x_j$的偏导数.偏导数具有多种记号:
$$\partial_jf(a);D_jf(a);\frac{\partial f}{\partial x_j}(a);f'_j(a)$$

为了区别于偏导数,会把$f(x)$的导数称为全导数.如果$f$在$a=(a_i)$处可微,那么首先,全导数存在可以保证偏导数存在,事实上取$h\in X$的分量表示为$(0,\cdots,v,\cdots,0)$,其中只有分量$i$取了非零向量,那么$\partial_if(a)$要满足$f(a+h)-f(a)=\partial_if(a)v+o(v)$.现在注意到如果有$f$在$a$处可微,那么$f(a+h)-f(a)=f'(a)h+o(h)$,于是$f'(a)h+o(h)=\partial_if(a)v+o(v)$,注意到$\|v\|=\|h\|$,于是对任意$\varepsilon>0$,存在$A>0$使得$\|h\|=\|v\|<A$的时候有$\|f'(a)\frac{h}{\|h\|}-\partial_if(a)\frac{v}{\|v\|}\|<\varepsilon$,于是看到对任意的$X_i$中的单位向量$v$和对应的$X$中的单位向量$h$,总有$\|f'(a)h-\partial_if(a)v\|<\varepsilon$对任意的正数$\varepsilon$成立,导致对任意的单位向量$v\in X_i$总有$f'(a)h=\partial_if(a)v$,进而对任意的向量$t$这个等式成立.至此看到,如果全导数存在,那么每个偏导数都存在,并且第$i$个偏导数就是$f'(a)$限制在积空间$X$的第$i$个分量$X_i$上得到的线性映射.另外,看到如果偏导数全部存在,那么尽管这时候未必全导数存在,但是如果全导数存在,那么它具有如下形式,这就是所谓的全微分公式:
$$df(a)=\partial_1f(a)dx_1+\partial_2f(a)dx_2+\cdots+\partial_mf(a)dx_m$$

这里$dx_i$表示的是第$i$个分量的增值,有时也记作$\Delta_i$.微分是一个线性映射,全微分公式就是这个线性映射的表达式.另外,在实函数的情况下,知道$f'(a)$代表的就是数乘以$f'(a)$这个线性变换,因而此时有$df=f'dx$.

注意,尽管偏导数理应是$X_j\to Y$的线性映射,但是为了便于把偏导数理解为特殊的方向导数,会把偏导数看作$X$到$Y$的线性映射,尽管它在$X_j,j\not=i$上是零映射.

例子.来举例说明偏导数存在不能说明可微.考虑$\mathbb{R}^2$上的实值函数,它在$x,y$轴上取0,其余的点取1,那么在$(0,0)$处的两个偏导数都存在,都是0,但是这个点甚至不是连续的.

至此得到可微的必要条件,即全部偏导数存在,为了进一步探究可微的充分条件,需要中值定理这个工具.
\newpage
\subsection{中值定理}

先来从$\mathbb{R}$上的中值定理讲起.给定$\mathbb{R}$的开集$U$上的实函数$f$,称点$x_0\in U$是$f$的极大值点,如果满足存在$x_0$的一个在$U$中的开邻域$V$,使得对$x\in V$总有$f(x)\le f(x_0)$.这时候称实数$f(x_0)$是$f$的一个极大值.对偶的可以定义极小值点和极小值.如果还要求存在的开邻域$V$满足如果$x\in V$,$x\not=x_0$就有$f(x)<f(x_0)$,就称$x_0\in U$是$f$的严格极大值点,称实数$f(x_0)$是$f$的严格极大值.对偶的可以定义严格极小值点和严格极小值.注意这个概念是局部的.

把极大值和极小值统称为极值,开集$U\subset\mathbb{R}$上的实值函数的极值集合是一个至多可数集.这只要分别证明极大值集合和极小值集合都是至多可数集,注意到$f$的极大值就是$-f$的极小值,于是只要证明其中一个集合是至多可数的,另一个也必然是至多可数的.核心思路是证明存在从极大值集合到$\mathbb{Q}^2$的单射,于是按照后者是可数集就完成证明.对每个极大值$y_0$,知道至少存在一个极大值点$x_0\in U$使得$y_0=f(x_0)$.按照定义,存在$x_0$的一个开邻域$V\subset U$使得在$V$上恒有$f(x)\le y_0$.取闭区间$x_0\in(a,b)\subset[a,b]\subset V$,并且要求$a,b$是有理数,然后构造$y_0\mapsto (a,b)$为极大值集合到$\mathbb{Q}^2$的映射.注意到这个映射的定义依赖于选择公理.断言这个映射是单射,事实上如果存在另一个实数$y_1$对应于$(a,b)$,那么在区间$(a,b)$上存在点$x_1$使得恒有$f(x)\le f(x_1)=y_1$.这导致$y_1=y_0$,就完成了单射的证明.

Fermat定理.一个实函数的极值点处如果可导,那么它的导数是0.事实上按照可导的定义,有$\frac{f(x_0+h)-f(x_0)}{h}=f'(x_0)+o(1)$,当要求$h$从$x_0$的两个不同方向趋于0时,看到左侧的表达式是非负的或者非正的,按照极限的保号性,看到$f'(x_0)$同时满足非负和非正,这就得到$f'(x_0)=0$.

这里强调两点.首先,Fermat定理提供的是内点作为极值点要满足的必要条件,例如一个紧区间$[a,b]$上的函数可以在边界点$a$或$b$取到极值,但是导数非0;另外导数为0的点称为函数的驻点(stationary point),驻点说明不了该点是极值点,例如$x\mapsto x^3$在$x=0$处.第二,从图像角度看这个命题的结论是符合直观的,因为极值点处的切线必须是水平的.

Rolle定理.如果实函数$f:[a,b]\to\mathbb{R}$是连续的,并且在$(a,b)$上可导,并且满足$f(a)=f(b)$,那么存在一个$\xi\in(a,b)$满足$f'(\xi)=0$.事实上按照$f$在闭区间上连续,知道它必然可以取到最大值或者最小值,倘若两个值相同,那么此时函数是常函数,于是任意的$\xi\in(a,b)$总有$f'(\xi)=0$.如果两个值不同,那么至少其中一个值和$f(a)=f(b)$不同,那么这个值是一个内点所取到的极值,按照Fermat定理,这个点的导数为0.

这个证明本身容易让产生一种误解,即满足条件的函数,它的导数为0的点只能由极值处提供,实际上这是不对的.

Lagrange中值定理.如果实函数$f:[a,b]\to\mathbb{R}$是连续的,并且在$(a,b)$上可导,那么存在$\xi\in(a,b)$满足$\frac{f(b)-f(a)}{b-a}=f'(\xi)$.证明只要构造$F(x)=f(x)-\frac{f(b)-f(a)}{b-a}(x-a)$运用Rolle定理.另外,Lagrange中值定理可以加以改进,可以要求$[a,b]$上有至多个数个点集的可导性是未知的:

改进的$\mathbb{R}$上中值定理.如果$I=[a,b]$是$\mathbb{R}$的紧区间,$f$是$I\to\mathbb{R}$的连续函数,它在$I$的除去一个可数子集$D$上处处可导,并且$f'(x)$在$I-D$上恒有$m\le f'(x)\le M$,那么有$m(b-a)\le f(b)-f(a)\le M(b-a)$.另外,这个不等式在单侧取等号当且仅当它是一个一次函数.
\begin{proof}
	
	设$D$中的点排列为$p_n,n\ge1$.对任意$\varepsilon>0$,来证明$f(b)-f(a)\le M(b-a)+\varepsilon(b-a+1)$,按照$\varepsilon$的任意性就得证.另一侧的不等式是同理的.记集合$A$表示这样的$I$的子集,它由满足如下条件的点$y$构成:对任意的满足$a\le x<y$的点$x$,有:
	$$f(x)-f(a)\le (M+\varepsilon)(x-a)+\varepsilon\sum_{p_n<\varepsilon}2^{-n}$$
	
	那么集合$A$是非空的,因为$a\in A$.并且如果$y\in A$,那么有$[a,y]\subset A$.于是如果记$c=\sup A$,那么$A=[a,c)$或者$[a,c]$.但是如果是前者,按照连续性,只要对上面不等式取极限,就得到$c\in A$,于是实际上$A=[a,c]$.于是只要证明$c=\sup A=b$.假设$c<b$,如果$c\not\in D$,那么按照导数的定义,存在足够小的正数$d$使得$c\le x<c+d$总有$f(x)-f(c)-f'(c)(x-c)\le\varepsilon(x-c)$,这导致:
	$$f(x)-f(c)\le|f'(c)|(x-c)+\varepsilon(x-c)
	\le (M+\varepsilon)(x-c)$$
	
	但是有
	$$f(c)-f(a)\le (M+\varepsilon)(c-a)+\varepsilon\sum_{p_n<x}2^{-n}$$
	
	相加导致$x\in A$和$c$的极大性矛盾.于是必然有$c$在$D$中.现在设$c=p_m$,按照$f$的连续性可设$d$足够小使得$[c,c+d]\subset I$,并且满足对任意的$c\le x<x+d$,有$f(x)-f(c)\le\varepsilon2^{-m}$,于是得到:
	$$f(x)-f(a)=f(x)-f(c)+f(c)-f(a)\le (M+\varepsilon)(c-a)+\varepsilon\sum_{p_n\le c}2^{-n}\le (M+\varepsilon)(x-a)+\varepsilon\sum_{p_n<x}2^{-n}$$
	
	这又导致$x\in A$和$c$的极大性矛盾,于是$c=b$,完成证明.现在来证明最后一个命题,倘若有$f(b)-f(a)=m(b-a)$,注意到$f(x)-f(a)-m(x-a)$是单调增的,并且$a,b$处都是0,就导致是恒为0的.
	
\end{proof}

Lagrange中值定理可以提供实函数的导数和单调性的关系,以及常函数的一个准则.即如下三条:一个可导函数在开区间上导数处处非负,等价于可导函数是单调增函数;一个可导函数在开区间上导数处处非负,并且导数为0的点不包含一个子区间,那么等价于可导函数是严格增函数;一个可导函数如果处处导数为0,则等价于函数是常值函数.不过既然证明了改进的中值定理,就可以推出如下命题:若$f$是从闭区间$[a,b]$到$\mathbb{R}$的连续函数,在除去一个可数集上是可导的,
\begin{enumerate}
	\item 如果$f'(x)=0$,那么$f$是常值函数.
	\item 如果$f'(x)\ge0$,那么$f$是单调增函数.
	\item 如果$f'(x)>0$,那么$f$是严格增函数.
\end{enumerate}

注意这个命题是不能随便把可数改为零测的,Cantor函数就是一个非常值的$[0,1]\to[0,1]$的连续函数,它在一个零测集以外导数处处为0.如果把可数改为零测,需要对连续函数添加条件才能使得上述命题1成立,例如绝对连续性.

对上述命题再给出一个不直接依赖中值定理的证明,注意到1和3都可以从2推出,来证明命题2:如果连续函数$f:[a,b]\to\mathbb{R}$在除去一个可数集以外处处导数非负,那么$f(a)\le f(b)$.
\begin{proof}
	
	先假设连续函数在除去一个可数集以外处处导数大于0,因为一旦证明了这个命题,就可以推出原命题:对一个除去一个可数集以外处处导数$\ge0$的函数$f$,取一个正数$\varepsilon$,那么记$g(x)=f(x)+\varepsilon x$,就得到$g$的导数在除去一个可数集以外处处大于0,于是按照之前的命题,得到$f(a)+\varepsilon a\le f(b)+\varepsilon b$对任意的正数$\varepsilon$成立,让$\varepsilon\to0$就说明了$f$是单调增的.
	
	现在假设结论不成立,即有$f(a)>f(b)$.对任意的$y\in[f(b),f(a)]$,定义集合$S_y=\{x\in[a,b]\mid f(x)=y\}$,由于$f$连续,看到每个$S_y$都是非空闭集.记$\sup S_y=s_y$,那么如果$y_1\not=y_2$,必然有$s_{y_1}\not=s_{y_2}$.于是从$[f(b),f(a)]$不可数,得到集合$\{s_y\in[a,b]\mid y\in[f(b),f(a)]\}$是不可数的,这导致必然存在某个$s_{y'}$是可导点,注意到对$x>s_{y'}$必然有$f(x)<y'$,否则按照连续函数的介值性,必然存在$[x,b]$中的某个点的函数值为$y'\in[f(b),f(x)]$,这和$s_{y'}$的极大性矛盾.于是,看到这个点的导数是非正的,这就矛盾:
	$$f'(s_{y'})=\lim_{x\to s_{y'}^+}\frac{f(x)-f(s_{y'})}{x-s_{y'}}\le0$$
	
\end{proof}

现在回到中值定理.来介绍实直线情况下最后一个中值定理,Cauchy中值定理.如果$f,g$是$[a,b]$上的两个实连续函数,都在$(a,b)$上可导,那么存在$\xi\in(a,b)$满足:$f'(\xi)(g(b)-g(a))=g'(\xi)(f(b)-f(a))$.如果要求$g'(x)$在$(a,b)$上总不取0,那么$g$必然在$[a,b]$上是单射,此时就存在$\xi\in(a,b)$满足$\frac{f(b)-f(a)}{g(b)-g(a)}=\frac{f'(\xi)}{g'(\xi)}$.证明同样利用Rolle定理.

现在开始步入高维情况.首先对于实值函数的情况.如果$f:U\subset X\to\mathbb{R}$是赋范空间$X$上开区域$U$上的实值函数,称$U$中两个点$x,x+h$确定的线段$S$是$\{x+\theta h\mid 0\le\theta\le1\}$.如果$S\subset U$,满足$f$在$S$上连续,并且在这个线段除去端点的每个点上可微,那么存在$S$中不是端点的点$\xi$满足$f(x+h)-f(x)=f'(\xi)h$.这个证明只要构造$F(t)=f(x+th)$为$[0,1]\to\mathbb{R}$的映射,按照实函数上的Lagrange中值定理就得证.

最后来讨论一般赋范空间上的中值定理.这时候不再得到一个等式,而是一个不等式.设$I=[a,b]$是$\mathbb{R}$的一个紧区间,设$f$是从$I$到赋范空间$F$的连续映射,如果对$(a,b)$中任一点$x$满足$\|f'(x)\|\le M$,那么对任意的$x_1,x_2\in[a,b]$有$\|f(x_2)-f(x_1)\|\le M|x_1-x_2|$.
\begin{proof}
	
	若否,则存在$a\le x_1<y_1\le b$使得$\|f(y_1)-f(x_1)\|>M(y_1-x_1)$,那么记$c=\frac{y_1+x_1}{2}$,则$x_1,c$和$c,y_1$必然存在其中一组,使得同样原命题中的不等式不成立,否则把两个不等式相加就得到$\|f(y_1)-f(x_1)\|\le M(y_1-x_1)$成立,把这对数构成的区间记作$[x_2,y_2]$,归纳构造$[x_n,y_n]$,那么有$y_n-x_n=\frac{y_1-x_1}{2^{n-1}}$,于是闭区间套$[x_n,y_n]$的交是一个点$c$,于是总有$\frac{\|f(x_n)-f(c)-f'(c)(x_n-c)\|}{|x_n-c|}>M-f'(c)>0$,但是取$n\to+\infty$和导数定义矛盾.
	
\end{proof}

对于内积诱导的赋范空间,甚至可以证明在上述条件下,存在一个$\xi\in(a,b)$使得$\|f(b)-f(a)\|\le\|f'(\xi)\|(b-a)$.
\begin{proof}
	
	记$\phi(t)=(f(b)-f(a),f(t))$,这是内积定义的函数,曾经给出过多重线性映射的可微性,于是$\phi(t)$在$[a,b]$连续,在$(a,b)$可微,导数为$\phi'(t)=(f(b)-f(a),f'(t))$.于是由实函数的中值定理,得到存在$\xi\in(a,b)$满足$\phi(b)-\phi(a)=(b-a)(f(b)-f(a),f'(\xi))$.另外有$\phi(b)-\phi(a)=\|f(b)-f(a)\|^2$,于是从Cauchy不等式就得到下式,整理就证毕.
	$$\|f(b)-f(a)\|^2\le(b-a)\|f(b)-f(a)\|\cdot\|f'(\xi)\|$$
	
\end{proof}

一般赋范空间上的中值定理.如果$f$是从$X$的某个点$a,b$之间线段$S$的邻域(或者我们要求定义域是凸开集也能保证这个要求)到一个赋范空间$Y$上的连续映射,满足$f$在$S$中每个点上可微,那么有$\|f(b)-f(a)\|\le\|b-a\|\sup_{x\in S}\|f'(x)\|$.另外对于内积空间,就有存在$\xi\in S$不是端点,满足$\|f(b)-f(a)\|\le\|b-a\|\cdot\|f'(\xi)\|$.这两个证明都只要考虑$I=[0,1]$到$F$的函数$g(t)=f(a+t(b-a))$再运用之前版本的中值定理.

中值定理的应用.首先,如果$U$是赋范空间$X$的一个开连通子集,$f$是从$U$到一个赋范空间$Y$的连续映射,如果$f$在$U$上处处导数为0,那么$f$是一个常值函数.为此,取定$A$中一个点$x_0$,设集合$A$为$U$的这样一个子集,它是$f$下的函数值为$f(x_0)$的点构成的集合.那么按照中值定理,利用开球邻域总是一个凸开集,看到$B$和$A-B$都是$A$的开子集,从连通性就得到$A$必然为整个$B$.另外可以得到,从赋范空间一个凸紧集到另一个赋范空间的连续可微的映射,满足Lipschitz条件,即$\|f(x_2)-f(x_1)\|\le M\|x_2-x_1\|$.

来利用中值定理给出可微性的一个充分条件.曾证明过,无论一维还是高维的情况,映射的可微性和连续性的关系是一致的,即可微推出连续但反之不成立.但是对于偏导数的连续性和可微的关系,一维和高维之间是不同的,高维情况下偏导数在一点全部连续可以得到可微,但是一维情况这是不成立的.或者可以理解为,一维情况下本质上讲并没有偏导数.

全导数与偏导数的连续性.给定赋范空间$X$的一个开子集$U$到一个赋范空间$Y$的映射$f$,如果存在$U$的开子集$V$使得$f$在$V$上可微,那么全导数$f'$是从$V$到$L(X,Y)$的映射.知道$L(X,Y)$本身也是一个赋范空间,于是可以谈论$f'$的连续性,如果它在点$x\in V$是连续的,就称$f$在$x_0$处是连续可导的.如果$f'$在$V$上处处连续,就称$f$在$V$上连续可导,记作$f\in C^{(1)}(V,Y)$.注意,对连续性的定义的前提是这个点存在一个开邻域上有意义,因此,当谈及$f$在一个点处连续可导时,实际上默认了这个点存在一个开邻域上是可导的.对于一维的情况,给定实可导函数$f:U\subset\mathbb{R}\to\mathbb{R}$,它的导函数$f'$仍然是$U\to\mathbb{R}$的函数,于是对于一维的情况,连续可导就是指这个函数是连续的.

可微的充分条件,偏导数连续.给定赋范空间的积空间$X=X_1\times\cdots\times X_m$的一个点$x=(x_i)$的开邻域$U$,设$f:U\to Y$是到赋范空间$Y$的映射,如果$f$在$U$上关于全部分量的偏导数处处存在(或者在$x$的一个开邻域上存在),并且偏导函数在$x$处都连续(注意,这个连续是指$f'_x(x,y)$在$(x_0,y_0)$连续,而不是指$f'_x(x,y_0)$在$x=x_0$连续),那么$f$在$x$处是可微的.
\begin{proof}
	
	只证明$m=2$的情况.记$Lh=\partial_1f(x)h_1+\partial_2f(x)h_2$,这是关于$h=(h_1,h_2)$的线性映射,来证明它就是$f$在$x$处的全微分.为此,注意到:
	$$f(x+h)-f(x)-Lh=f(x_1+h_1,x_2+h_2)-f(x_1,x_2)-\partial_1f(x)h_1-\partial_2f(x)h_2=F_1(x,h)+F_2(x,h)$$
	$$F_1(x,h)=f(x_1+h_1,x_2+h_2)-f(x_1,x_2+h_2)-\partial_1f(x_1,x_2)h_1$$
	$$F_2(x,h)=f(x_1,x_2+h_2)-f(x_1,x_2)-\partial_2f(x_1,x_2)h_2$$
	
	按照中值定理,看到:
	$$|F_1(x,h)|\le\sup_{0<\theta_1<1}\|\partial_1f(x_1+\theta_1h_1,x_2+h_2)-\partial_1f(x_1,x_2)\|\cdot|h_1|$$
	$$|F_2(x,h)|\le\sup_{0<\theta_2<1}\|\partial_2f(x_1,x_2+\theta_2h_2)-\partial_2f(x_1,x_2)\|\cdot|h_2|$$
	
	按照偏导数在$x$处的连续性,以及$\max\{|h_1|,|h_2|\}\le|h|$,得到上面两个表达式是趋于0的,于是就证明了$f$在$x$处的可微性.
	
\end{proof}

特别的,看到从$X=X_1\times\cdots\times X_m$的开子集$U$到$Y$的映射$f$是在$U$上连续可导的,当且仅当$f$的全部偏导函数在$U$上连续.
\newpage
\subsection{高阶导数}

高阶导数.给定赋范空间$X$的开集$U$,设$f$是从$U$到赋范空间$Y$的映射,如果要求$f$在$U$上可导,那么就有$f'$是从$U$到$Y_1=L(X,Y)$的映射.由于$Y_1$是赋范空间,还可以继续谈论$f'$的可导性.如果$f'$可导,那么它的导数$(f')'$是$U$到$L(U,Y_1)$的映射,称为$f$的二阶导数或者二阶微分,记作$f''$或者$f^{(2)}$.来给出高阶导数的归纳定义:对一个正整数$n$,称$f$在$x\in U$是$n$阶可导的,如果$f'$在这个点是$n-1$阶可导的,这个导数记作$f^{(n)}(x)$.它是从$X$到$Y_n$的连续线性映射.另外,可以讨论高阶的连续可导性.称$f:U\to Y$是$n$阶连续可导的,如果$f^{(n)}$在$U$上处处存在并且是连续映射,记作$f\in C^{(n)}(U,Y)$.按照可导推出连续,看到一个在$U$上$n$阶可导的映射,对每个$1\le k<n$,有$f$总是$k$阶连续可导的.由于一维的特殊性,看到实函数的高阶导数总是一个实函数.

关于实函数乘积的高阶导数有Leibniz公式.如果$f,g$是开集$U\subset\mathbb{R}\to\mathbb{R}$的实函数,那么有公式$(fg)^{(n)}=\sum_{k=0}^{n}C_n^kf^{(k)}g^{(n-k)}$.这个证明只要对$n$归纳.

这里给出实函数的高阶导数的几个应用.首先,已经看到一阶导数描述了函数的单调性.这里讨论二阶导数如何描述函数的凹凸性.开区间$(a,b)\subset\mathbb{R}$上的实函数称为凸(convex)函数,如果对任意的$x_1,x_2\in(a,b)$和任意的$\lambda\in[0,1]$有:
$$f(\lambda x_1+(1-\lambda)x_2)\le\lambda f(x_1)+(1-\lambda)f(x_2)$$

如果约定这个不等式取等号当且仅当$x_1=x_2$或者$\lambda=1$或0,就说它是严格凸函数.如果把不等号改变方向,就可以对偶的定义凹(concave)函数和严格凹函数.先来把凸函数定义写作适合探究的形式,一个函数$f$在$(a,b)$上是凸函数,当且仅当对任意的区间$(a,b)$中的三个点$x_1<x<x_2$,有$\frac{f(x)-f(x_1)}{x-x_1}\le\frac{f(x_2)-f(x_1)}{x_2-x_1}\le\frac{f(x_2)-f(x)}{x_2-x}$.事实上这个不等式的任意两项组成的不等式已经和原定义是等价的了.把这个不等式称为三弦不等式,它的几何描述是直观的,这个不等式是说,对$x_1<x_2$,无论把$x_1,x_2$如何变大,只要依然满足$x_1<x_2$,那么$\frac{f(x_1)-f(x_2)}{x_1-x_2}$就是变大的.

首先来证明,开区间上的凸函数实际上是连续函数.事实上,从三弦不等式看到对任意$x_0\in (a,b)$,那么当$x\to x_0^+$的时候,$\frac{f(x)-f(x_0)}{x-x_0}$的单调递减并且有界的,其中下界只要取$s,t\in(a,b)$,$s<t<x_0$,那么总有$\frac{f(s)-f(t)}{s-t}\le\frac{f(x)-f(x_0)}{x-x_0}$.于是$x_0$处总存在右导数,对偶的,总存在左导数,于是看到三弦不等式推出$(a,b)$上处处存在单侧导数.但是,知道单侧导数可以保证单侧连续性,于是$f$在$(a,b)$上是处处左右连续的,这就保证了$f$是处处连续的.

注意上一段的证明里实际上证明了凸函数处处存在左右导数,并且每一点的左导数小于等于右导数.

如果要求$f$在$(a,b)$上可导,那么凸条件可以说明$f'$是单调增函数.事实上在三弦不等式中让$x\to x_1$和$x\to x_2$,那么有$f'(x_1)\le\frac{f(x_2)-f(x_1)}{x_2-x_1}\le f'(x_2)$.而严格凸条件可以说明$f'$是严格增函数.这只要注意到三弦不等式写成整式取等号当且仅当$x,x_1,x_2$中有俩相等.另外这两个逆命题都是成立的,于是得到:开区间上的可导函数是凸函数当且仅当它的导函数是单调增函数,是严格凸函数当且仅当它的导函数是严格增函数.对偶的有凹函数的相关命题.

另外如果$f$在$(a,b)$上甚至是二阶可导的,那么知道$f'$单调增当且仅当$f''\ge0$.但是$f''>0$仅仅可以推出$f'$严格增,反过来不成立.于是一个二阶可导的$(a,b)$上函数$f$,是凸函数等价于$f''\ge0$,而$f''>0$可以推出严格凸条件,严格凸条件的等价描述是,$f''\ge0$并且$f''=0$的点不会构成一个区间.

这一段来证明.$\mathbb{R}$上既凹又凸的函数必然的线性函数,即$f(x)=sx+t$.既凹又凸告诉对任意$x,y\in(a,b)$和任意$\lambda\in[a,b]$,有$f(\lambda x+(1-\lambda)y)=\lambda f(x)+(1-\lambda)f(y)$.不妨设$f(0)=0$,否则以$g(x)=f(x)-f(0)$代替$f(x)$.取$\lambda=\frac{1}{2}$可归纳证明$f(2^n)=2^nf(1)$对任意整数$n$成立.接下来对任意整数$n$,对任意$x\in[2^{n-1},2^n]$,有$x=\frac{x}{2^n}2^n+(1-\frac{x}{2^n})0$,这导致$f(x)=\frac{x}{2^n}f(2^n)=xf(1)$.于是看到$f(x)=xf(1)$对$x>0$恒成立.对偶的得到$f(x)=xf(-1)$对$x<0$恒成立,最后取$x=1,y=1,\lambda=\frac{1}{2}$,得到$f(1)=-f(-1)$,于是看到对任意实数$x$有$f(x)=xf(1)$.

关于凸函数还有Jensen不等式.对开区间$(a,b)$上的凸实函数$f$,满足对任意的$x_1,x_2,\cdots,x_n\in (a,b)$和一列和为$1$的正实数$\lambda_1,\lambda_2,\cdots,\lambda_n$,满足:
$$f(\sum_{k=1}^{n}\lambda_kx_k)\le\sum_{k=1}^{n}\lambda_kf(x_k)$$
\begin{proof}
	
	来对$n$归纳.对于$n=2$就是凸函数的定义.现在假设对$n-1$成立,来证明对$n$成立.任取$(a,b)$中的$n$个数$x_1,x_2,\cdots,x_n$,任取和为$1$的$n$个正实数$\lambda_1,\lambda_2,\cdots,\lambda_n$.那么按照凸函数的定义,得到:$$f(\sum_{k=1}^{n}\lambda_kx_k)=f((1-\lambda_n)\sum_{k=1}^{n-1}\frac{\lambda_k}{1-\lambda_n}x_k+\lambda_nx_n)
	\le(1-\lambda_n)f(\sum_{k=1}^{n-1}\frac{\lambda_k}{1-\lambda_n}x_k)+\lambda_nf(x_n)$$
	
	再按照归纳假设,有$f(\sum_{k=1}^{n-1}\frac{\lambda_k}{1-\lambda_n}x_k)\le\sum_{k=1}^{n-1}\frac{\lambda_kx_k}{1-\lambda_n}$,带入就完成归纳.
	
\end{proof}

称函数的点$x_0$是拐点,如果存在$x_0$的左右邻域$(a,x_0)$和$(x_0,b)$上一个是凹函数一个是凸函数.那么如果$f$是二阶可导的,$x_0$是拐点就要求存在$x_0$的左右邻域$(a,x_0)$和$(x_0,b)$上一阶导数一个是单调增一个单调减的,这导致$x_0$是$f'$的一个极值点,于是必然有$f''(x_0)=0$.但是反过来,二阶导数为0的点未必是拐点.例如$f(x)=x^4$,在$x=0$处二阶导数为0,但是它的任意足够小的左右邻域上函数都是凸的.

第二个应用即L'Hopital法则.给定开区间$(a,b)$上的两个可导函数$f,g$,满足$g'$在$(a,b)$上总不为0,当$f,g$满足如下两个条件任一时,从$\lim_{x\to a^+}\frac{f'(x)}{g'(x)}=A$,其中$-\infty\le A\le+\infty$可以推出$\lim_{x\to a^+}\frac{f(x)}{g(x)}=A$.
\begin{enumerate}
	\item $\lim_{x\to a^+}f(x)=\lim_{x\to a^+}g(x)=0$,即$\frac{0}{0}$型.
	\item $\lim_{x\to a^+}g(x)=\infty$,即$\frac{*}{\infty}$型.
\end{enumerate}

这里给出高阶导数在实函数上的最后一个应用.来利用二阶导数来给出极值点的更加精细的判断.如果$f$在$(a,b)$上二阶可导,那么为了判断$x_0\in (a,b)$是否是极值点,我们已经知道首先需要看它是否是驻点,即导数是否为0,如果不为0那么必然不会是极值点.现在假设$f'(x_0)=0$,那么,注意到$\lim_{x\to x_0}\frac{f'(x)}{x-x_0}=f''(x_0)$,于是如果$f''(x_0)>0$,则说明$f'$在$x_0$的足够小的右邻域上正,足够小的左邻域上负,于是$x_0$处是$f$的极小值点.同理如果$f''(x_0)<0$那么$x_0$是$f$的极大值点.在介绍Taylor展开后这个判断会进一步加细.

在有限维空间上,来把偏导数推广为方向导数.首先如果$f$是开集$U\subset\mathbb{R}^m\to\mathbb{R}^n$的映射,那么这时候偏导数是$\mathbb{R}\to\mathbb{R}^n$的线性映射,于是按照$\mathrm{Hom}(\mathbb{R},\mathbb{R}^n)\simeq\mathbb{R}^n$,这个线性映射可以等价的写作$\mathbb{R}^n$中的一个向量.如果记$h\in\mathbb{R}^m$为第$i$个分量为实数$1$,其余为0的向量.那么这时候偏导数可以表示为:
$$\partial_if(a)=\lim_{t\to0}\frac{f(a+th)-f(a)}{t}$$

据此,定义$f$在点$x$关于$\mathbb{R}^m$上的一个向量$h$的方向导数为$\lim_{t\to0}\frac{f(x+th)-f(x)}{t}$所定义的$\mathbb{R}^n$中的向量,记作$D_hf(x)$,它是从$\mathbb{R}$到$\mathbb{R}^n$的线性映射.于是偏导数就是关于$\mathbb{R}^m$的标准正交基中的向量的方向导数.

从定义可以直接看出,对实数$\lambda$总有$D_{\lambda h}f(x)=\lambda D_h(x)$.另外如果$f$在$x\in U$处已经可微,那么直接有$D_hf(x)=f'(x)h$.最后方向导数关于方向向量是线性的,即$D_{\lambda_1h_1+\lambda_2h_2}f(x)=\lambda_1D_{h_1}f(x)+\lambda_2D_{h_2}f(x)$.

方向导数的表达式.给定向量$h=(h_1,\cdots,h_m)$.那么一方面,按照定义有$f(x+th)-f(x)=D_hf(x)t+o(t)$,另一方面,如果$f$在点$x$是可微的,就有$f(x+th)-f(x)=f'(x)th+o(th)$.于是看到$D_hf(x)+o(t)/t=f'(x)h+o(th)/t$.让$t\to0$,按照$f'(x)$的表达式,得到:
$$D_hf(x)=f'(x)h=\partial_1f(x)h_1+\partial_2f(x)h_2+\cdots+\partial_mf(x)h_m$$

高阶偏导数.给定$U\subset\mathbb{R}^m\to\mathbb{R}^n$的映射$f$,如果$f$在$U$上具有对$i$分量的偏导数$\partial_if(x)$,那么偏导数是$\mathbb{R}^m\to\mathbb{R}^n$中的映射.现在假设$\partial_if(x)$在$U$上还具有对$j$分量的偏导数,把这个偏导数称为$f$的关于$(i,j)$的二阶偏导数,记作$\partial_{ji}f(x)$或者$\frac{\partial^2f}{\partial x^j\partial x^i}(x)$,这仍然是一个从$\mathbb{R}^m\to\mathbb{R}^n$的映射.归纳的,可以定义$k$阶偏导数,即:
$$\partial_{i_1,i_2,\cdots,i_k}f(x);\frac{\partial^kf}{\partial x^{i_k}\cdots\partial x^{i_1}}(x)$$

关于高阶偏导数一个自然的问题是,如果改变求偏导的顺序,是否会影响结果?对此有:如果$f:U\to\mathbb{R}^n$具有偏导数$\partial_{ij}f(x)$和$\partial_{ji}f(x)$.如果这两个偏导数都在一个点$x$处连续,那么两个偏导数是相同的.据此,得到,如果$f$是$k$次连续可导的映射,那么改变求偏导顺序不会影响它的$k$阶偏导数的取值.
\begin{proof}
	
	取$x=(x^1,\cdots,x^m)$的一个开球$B=B(x,r)\subset U$.由于只有两个变量$x^i,x^j$是变动的,于是不妨设$f$本身就是一个二元函数.现在记映射:
	$$F(h^1,h^2)=f(x^1+h^1,x^2+h^2)+f(x^1,x^2)-f(x^1,x^2+h^2)-f(x^1+h^1,x^2)$$
	
	约定$h=(h^1,h^2)$只在范数足够小的开球上取值,使得$x+h\in B$.现在记$\phi(t)=f(x^1+th^1,x^2+h^2)-f(x^1+th^1,x^2)$,那么$F(h^1,h^2)=\phi(1)-\phi(0)$,于是按照中值定理,得到存在$\theta_1\in(0,1)$使得:
	$$F(h^1,h^2)=\phi'(\theta_1)=\left(\partial_1f(x^1+\theta_1h^1,x^2+h^2)
	-\partial_1f(x^1+\theta_1h^1,x^2)\right)h^1$$
	
	再利用中值定理,得到$F(h^1,h^2)=\partial_{21}f(x^1+\theta_1h^1,x^2+\theta_2h^2)h_1h_2$,对偶的,如果设$\psi(t)=f(x^1+h^1,x^2+th^2)-f(x^1,x^2+th^2)$,从$F(h^1,h^2)=\psi(1)-\psi(0)$同样两次中值定理得到一个等式,于是看到存在$(0,1)$中的$\theta_1,\theta_2,\delta_1,\delta_2$,满足:
	$$\partial_ {21}f(x^1+\theta_1h^1,x^2+\theta_2h^2)h_1h_2=\partial_ {12}f(x^1+\delta_1h^1,x^2+\delta_2h^2)h_1h_2$$
	
	最后约去$h_1h_2$,让$h=(h_1,h_2)$趋于0,就得到两个偏导数在$(x^1,x^2)$处取值相同.
	
\end{proof}

高阶导数作为方向导数的复合.给定映射$f:U\subset X=\mathbb{R}^m\to Y=\mathbb{R}^s$,如果$f$在$U$上具有直至$n$阶导数,来把$f^{(n)}(x_0),x_0\in U$描述为一个多重线性映射.首先如果$n=1$,那么$f'(x_0)h=D_hf(x_0),h\in X$,即$f'(x_0)$是$X\to Y$的线性映射.现在如果$n=2$,那么$f^{(2)}(x_0)\in L(X,L(X,Y))$,于是对于$X$中的向量$h_1$,有$(f^{(2)}(x_0)h_1)\in L(X,Y)$,于是,定义$f^{(2)}(x_0)(h_1,h_2)=(f^{(2)}(x_0)h_1)h_2$,这是$X\times X\to Y$的双线性映射.并且可以表示为$f^{(2)}(x_0)(h_1,h_2)=D_{h_1}D_{h_2}f(x_0)$.类似的,得到,$f^{(n)}(x_0)$可以定义为$X^n$到$Y$的$n$重线性映射:
$$f^{(n)}(x_0)(h_1,h_2,\cdots,h_n)=D_{h_1}D_{h_2}\cdots D_{h_n}f(x_0)$$

并且,只要$f^{(n)}$在$x_0$处有定义,那么把它看作上述多重线性映射的时候,把右侧的求方向导数的次序改变不会影响结果.【】另外,注意到在有限维的情况下,如果把$h_i$都取为标准正交基中的单位向量,那么$f^{(n)}(x_0)(h_1,\cdots,h_n)$表示的是一个$n$次的偏导数.另外,按照多重线性,可以把一般的$(h_1,h_2,\cdots,h_n)$表示为标准正交基的线性组合,这就导致$f^{(n)}(x_0)$总可以表示为偏导数的线性组合.于是,对于有限维空间之间的映射$f$,它是$n$次连续可导的,等价于要求全部$n$次偏导数是连续的.
\newpage
\subsection{Taylor公式}

从实函数的Taylor公式讲起.至此已经看到,如果两个实函数在一个点处越多的高阶导数相同(约定0阶导数就是函数自身),那么两个函数在这个点的邻域上的行为越加接近.知道对于一个$n$次多项式$p_n(x)$对定义域中的点$x_0$,可以唯一的表示为$p_n(x,x_0)=c_0+c_1(x-x_0)+\cdots+c_n(x-x_n)^n$,其中$c_k=\frac{p_n^{(k)}(x_0)}{k!}$.于是对一个在点$x_0$处直至$n$阶可导的函数$f$,可以构造多项式$p_n(x)=f(x_0)+\frac{f'(x_0)}{1!}(x-x_0)+\cdots+\frac{f^{(n)}(x_0)}{n!}(x-x_0)^n$.于是,如果一个实函数在点$x_0$具有直至$n$阶导数,把所构造的这个多项式称为$f$在点$x_0$的$n$次Taylor多项式.

为了探究Taylor多项式如何在$x_0$附近逼近原函数,需要探究余项$f(x)-p_n(x,x_0)=r_n(x,x_0)$.那么首先,如果$f$在$x_0$处有直至$n$次的导数,那么余项可以表示为$r_n(x,x_0)=o((x-x_0)^n)$,这称为Peano余项.证明只要做$n-1$次L'Hopital法则求极限即可.于是此时$f(x)$可以表示为带Peano余项的Taylor公式:
$$f(x)=f(x_0)+\frac{f'(x_0)}{1!}(x-x_0)+\cdots+\frac{f^{(n)}(x_0)}{n!}(x-x_0)^n+o((x-x_0)^n)$$

但是注意,如果$f$在$x_0$附近可以表示为带Peano余项的Taylor公式,这不代表对应的多项式部分就是相应的Taylor公式.例如考虑Dirichlet函数$D(x)$,取$f(x)=x^{n+1}D(x)$,那么有$f(x)=o(x^n)$,但是$f$并不存在0附近的高阶导数.

来介绍更一般的余项公式.如果$f$在以$x,x_0$为端点的闭区间上连续,在对应开区间上具有直至$n+1$阶导数,那么对任意的在这个闭区间上连续,在对应开区间可导并且导数不取0的函数$\phi$,存在$x_0$和$x$之间的一个实数$\xi$,满足余项为:
$$r_n(x,x_0)=\frac{\phi(x)-\phi(x_0)}{\phi'(\xi)n!}f^{(n+1)}(\xi)(x-\xi)^n$$
\begin{proof}
	
	把$x_0$换作变换$t$,记$F(t)=f(x)-\left(f(t)+\frac{f'(t)}{1!}(x-t)+\cdots+\frac{f^{(n)}(t)}{n!}(x-t)^n\right)$.那么就有$F'(t)=-\frac{f^{(n+1)}(t)}{n!}(x-t)^n$.于是按照柯西中值定理,有:
	$$\frac{F(x)-F(x_0)}{\phi(x)-\phi(x_0)}=\frac{F'(\xi)}{\phi'(\xi)}$$
	
	整理下就得证.
	
\end{proof}

特别的,如果取特殊的$\phi(x)$,得到如下两个余项:
\begin{enumerate}
	\item Lagrange余项的Taylor公式.取$\phi(t)=(x-t)^{n+1}$得到:如果$f$在$x,x_0$确定的闭区间上连续,在对应开区间上有直至$n+1$阶的导函数,那么有$x$和$x_0$之间的一个数$\xi$满足:
	$$f(x)=f(x_0)+\frac{f'(x_0)}{1!}(x-x_0)+\cdots+\frac{f^{(n)}(x_0)}{n!}(x-x_0)^n+\frac{f^{(n+1)}(\xi)}{(n+1)!}(x-x_0)^{n+1}$$
	\item Cauchy余项的Taylor公式.取$\phi(t)=x-t$得到:如果$f$在$x,x_0$确定的闭区间上连续,在对应开区间上有直至$n+1$阶的导函数,那么有$x$和$x_0$之间的一个数$\xi$满足:
	$$f(x)=f(x_0)+\frac{f'(x_0)}{1!}(x-x_0)+\cdots+\frac{f^{(n)}(x_0)}{n!}(x-x_0)^n+\frac{f^{(n+1)}(\xi)}{n!}(x-\xi)^n(x-x_0)$$
\end{enumerate}

这里利用Taylor公式进一步讨论极值点问题.设$f$在$x_0$附近有直至$n+1$阶连续导数,并且满足$f'(x_0)=f''(x_0)=\cdots=f^{(n)}(x_0)=0$,并且$f^{(n+1)}(x_0)\not=0$.那么按照Peano余项公式,看到$f(x)-f(x_0)=\frac{f^{n+1}(x_0)}{(n+1)!}(x-x_0)^{n+1}+o((x-x_0)^{n+1})$.于是有:
$$f(x)-f(x_0)=\left(\frac{f^{(n+1)}(x_0)}{(n+1)!}+\frac{o((x-x_0)^{n+1})}{(x-x_0)^{n+1}}\right)(x-x_0)^{n+1}$$

注意到括号中第二项极限为0,所以在足够小的$x_0$的邻域内它不会影响括号这一项的正负性,这个正负性就是前一项的正负性.于是,如果$n$是偶数,看到$f(x)-f(x_0)$在$x_0$的左右邻域上变号,这导致$x_0$不会是极值点.如果$n$是奇数,那么如果$f^{(n+1)}(x_0)>0$,就有$x_0$是$f$的严格极小值点;如果$f^{(n+1)}(x_0)<0$,就有$x_0$是$f$的严格极大值点.

现在给出一般赋范空间上的Taylor公式.如果$f:U\subset X\to Y$是开集$U$上的具有直至$n-1$阶导数的映射,并且在点$x_0\in U$处具有$n$阶导数,那么有:
$$f(x_0+h)=f(x_0)+f'(x_0)h+\cdots+\frac{1}{n!}f^{(n)}(x_0)h^n+o(|h|^n),|h|\to0$$

注意这里的$h^k$是$X^k$中的一个元,曾经说明过一个$k$阶导数在一个点处是一个多重线性映射.【证明】.特别的,对于有限维空间上,并且为了便于描述不妨设$Y=\mathbb{R}$,那么有:
$$f(x^1+h^1,\cdots,x^m+h^m)-f(x^1,\cdots,x^m)=\sum_{k=1}^{n-1}\frac{(h^1\partial_1+\cdots+h^m\partial_m)^k}{k!}f(x)+o(|h|^n),|h|\to0$$

Taylor公式的应用,多元实值函数的极值.给定一个多元函数$f:U\subset\mathbb{R}^m\to\mathbb{R}$,类似于一维的情况,可以定义点$x_0\in U$是$f$的一个极大/小值点,如果存在$U$中的$x_0$的一个开邻域$V$,使得$f$在$V$上满足$f(x)\le f(x_0)$或者$f(x)\ge f(x_0)$.如果还要求取等号当且仅当$x=x_0$,就分别称为严格极大值点和严格极小值点.那么首先,如果一个点$x_0=(x^1,x^2,\cdots,x^m)$是$f$的极大/小值,固定除$x^i$以外的任意$m-1$个,那么此时函数$f$变成偏函数,$x^i$会是这个偏函数的极值点,于是得到,如果$f$在极值点可微,那么这个点处的全部偏导数都为0.

现在,假设$f$在$(x^1,x^2,\cdots,x^m)$处的全部偏导数都为0.为了进一步判断它是否是极值点,假设$f$在$x_0$附近是二阶连续可导的,那么做Taylor展开得到:
$$f(x^1+h^1,x^2+h^2,\cdots,x^m+h^m)-f(x^1,x^2,\cdots,x^m)=\frac{1}{2!}\sum_{i,j=1}^m\frac{\partial^2f}{\partial x^i\partial x^j}(x_0)h^ih^j+o(\|h\|^2)$$

注意右侧的第一项是一个二次型.他对应的矩阵是由$\frac{\partial^2f}{\partial x^i\partial x^j}(x_0)$作为项构成的对称实矩阵,这称为$f$的Hesse矩阵.那么按照二次型理论,知道如果这个Hesse矩阵是正定或者负定的,那么$x_0$会是$f$的严格极小值或者严格极大值.
\newpage
\subsection{隐函数定理与逆映射定理}

先占坑...
\newpage
\subsection{原函数}

这里来给出实函数的微分学的最后一个内容,原函数.给定开区间$(a,b)$上的函数$f(x)$,称另一个$(a,b)$上的函数$F(x)$是$f(x)$的原函数,如果满足$F'(x)=f(x)$.那么首先,原函数如果存在则不是唯一的.事实上如果$f$存在两个原函数$F_1,F_2$,那么按照求导法则看到$F_1(x)-F_2(x)$的导数是0,证明过导数恒为0的函数是常函数,于是同一个函数$f$如果存在一个原函数$F(x)$,那么它的全部原函数可以表示为$F(x)$加上一个常数,这个函数族记作$F(x)+C$,$C$代表constant,即常数.

这里插入一个内容.并不是所有的函数都存在原函数.这可以从导函数需要满足某些特定性质推出.Darboux定理,导函数满足介质性:即如果$f$在$[a,b]$上可导,其中边界点是存在单侧导数,那么对$f_+'(a)$和$f'_-(b)$之间的任意实数$k$,存在$\xi\in[a,b]$使得$f'(\xi)=k$.
\begin{proof}
	
	取$F(x)=f(x)-kx$,那么$F(x)$在$[a,b]$上可导,其中边界点是单侧可导.那么就有$F_+'(a)F_-'(b)=(f_+'(a)-k)(f_-'(b))\le0$.如果这个数是0,那么左侧两个数至少有一个是0,于是这点就满足$f$的导数是$k$;如果这个数是负数,那么可以不妨设$F_+'(a)>0,F_-'(b)<0$,那么在$a$的足够小的右邻域内$F(x)>F(a)$,在$b$的足够小左邻域内$F(x)<F(b)$,导致连续函数$F$必然在$(a,b)$内取到$[a,b]$的最大值.于是这个点作为极大值点,导数为0.
	
\end{proof}

于是,$f'(x)$尽管未必连续的,看到它必然满足介质性,于是并不是所有的函数都能有原函数.另外,关于导函数还有导数极限定理:如果$f$在$[x_0,b]$上连续,在$(x_0,b)$内可导,并且导数在$x_0$处存在右极限,那么$f$在$x_0$处存在右导数,并且数值和导数的右极限相同.证明是直接的,按照中值定理,有$\frac{f(x)-f(x_0)}{x-x_0}=f'(\xi(x))$,现在让$x\to x_0^+$,则$\xi(x)\to x_0^+$,就得证.导数极限定理和Darboux定理告诉导函数不存在第一类间断点:如果导函数$f'$在一个点$x_0$的某开邻域内有定义,并且导函数在$x_0$处有左右极限,那么按照导数极限定理,得到$f$在$x_0$处是左右导数和这个左右极限是对应相同的.按照$f$在$x_0$可导,看到这两个数必然相同,于是$f'$在$x_0$处的左右极限是相同的.现在假设这个相同的数是$s$,它不等于$f'(x_0)=t$,即$x_0$是$f$的可去间断点.那么按照导函数的连续性,看到存在$x_0$的足够小的开邻域上的取值可以不取$t$和$\frac{s+t}{2}$中的全部元,但是按照Darboux定理,这个小区间里必须要取到$t$到$\frac{s+t}{2}$中的元,这就矛盾.综上得到了存在第一类间断点的函数必然不会存在原函数.

按照求导的性质,直接得到原函数的一些性质:
\begin{enumerate}
	\item 如果$u,v$有原函数,那么它们的线性组合有原函数:
	$$\int\left(\alpha u(x)+\beta v(x)\right)\mathrm{d}x=\alpha\int u(x)\mathrm{d}x+\beta\int v(x)\mathrm{d}x$$
	\item 分部积分:如果$u,v$有原函数$U,V$,那么只要$u(x)V(x)$和$U(x)v(x)$有一个有原函数,那么另一个有原函数,并且满足:
	$$UV=\int(UV)'\mathrm{d}x=\int U(x)v(x)\mathrm{d}x+\int u(x)V(x)\mathrm{d}x$$
	\item 换元积分:如果$\phi(x)$为$[a,b]\to[\alpha,\beta]$的可导函数,$g$是$[\alpha,\beta]$上有原函数$G$,那么$f(x)=g(\phi(x))\phi'(x)$在$[a,b]$上有原函数$G(\phi(x))$,即:
	$$\int f(x)\mathrm{d}x=\int g(\phi(x))\phi'(x)\mathrm{d}x=G(\phi(x))+C$$
\end{enumerate}
\newpage
\section{黎曼积分}
\subsection{基本定义与性质}

先来定义$\mathbb{R}^n$中的闭矩体,即形如$I=\{x\in\mathbb{R}^n\mid a^i\le x^i\le b^i,1\le i\le n\}$的点集.如果记$a=(a^1,a^2,\cdots,a^n)$和$b=(b^1,b^2,\cdots,b^n)$.那就把这个闭矩体记作$I_{a,b}$.特别的,一元情况就是熟知的闭区间$[a,b]$.

对闭矩体$I_{a,b}$约定它的体积,或者称为测度,为$|I_{a,b}|=\prod_{1\le i\le n}(b^i-a^i)$.关于体积,有如下基本性质:体积的齐次性,即如果对正实数$\lambda$定义$\lambda I_{a,b}=I_{\lambda a,\lambda b}$,那么有$|\lambda I_{a,b}|=\lambda^n|I_{a,b}|$;体积的加性,即如果$I$可以分解为有限个小闭矩体$I_i$的并,并且任意两个小闭矩体的交不会包含各自的内点,那么$I$的体积是这些小闭矩体的体积和,即$|I|=\sum|I_i|$;如果$I$被有限个闭矩体$\{I_k\}$的并覆盖,那么有$|I|\le\sum|I_k|$.

黎曼积分.给定闭矩体$I=I_{a,b}$,把每个$[a^i,b^i]$分割为若干小区间,这使得$I$被分割为若干小闭矩体的并.这称为闭矩体$I$的一个划分,会用字母$P$表示一个划分.如果划分$P$的小闭矩体记作$I_k,1\le k\le m$,记$d(I_k)$表示闭矩体$I_k$的直径,记$d(P)$表示$\max_{1\le k\le m}d(I_k)$,称为这个划分的直径.现在,对$I$到某个赋范空间上的函数$f$,在每个$I_k$中取一个点$\xi_k$,记$\xi=(\xi_1,\cdots,\xi_m)$,称$(P,\xi)$是赋点划分,称$\sigma(f,P,\xi)=\sum_{k=1}^{m}f(\xi_k)|I_k|$为映射$f$关于赋点划分$(P,\xi)$的黎曼和.

如果$\lim_{d(P)\to0}\sigma(f,P,\xi)$极限存在,就称$f$在$I$上黎曼可积,把这个极限称为$f$在$I$上的黎曼积分,记作$\int_If(x)\mathrm{d}x$.换句话说,存在赋范空间中的元$I$,对任意的$\varepsilon>0$,都存在正数$\delta>0$使得,只要划分$P$的直径小于$\delta$,对这个划分的任意黎曼和$\sigma(f,P,\xi)$,总有$\left|I-\sigma(f,P,\xi)\right|<\varepsilon$.在一维情况下,闭矩体就是一个闭区间$[a,b]$,这时候把映射$f$的积分记作$\int_a^bf(x)\mathrm{d}x$.

黎曼可积必有界.给定闭矩体$I$上的实值函数$f$,那么如果$f$在$I$上黎曼可积,则它必有界.事实上,按照黎曼积分定义,存在一个实数$I$,和一个正数$\delta$,使得只要划分$P$的直径小于$\delta$,对任意的赋点划分$(P,\xi)$,总有$\left|\sum_{k=1}^{m}f(\xi_k)|I_k|\right|<1$.但是按照$f$在$I$上无界,导致必须存在一个小闭矩体$I_t$使得$f$在$I_t$上无界,于是可以取$\xi_t\in I_t$满足$|f(\xi_t)||I_t|>1+|\sum_{j\not=t}f(\xi_j)|I_j||+|J|$.但是由之前的式子得到如下不等式,这就矛盾:
$$|f(\xi_t)||I_t|=|\sum_ {k=1}^{m}f(\xi_k)|I_k|-I-\sum_{j\not=t}f(\xi_j)|I_j|+I|\le1+|\sum_{j\not=t}f(\xi_j)|I_j||+|I|$$

现在始终假设$f$在闭矩体$I$上是有界的.对任意划分$P$,记$S(P,f)=\sum_{k=1}^{m}\sup_{x\in I_k}f(x)|I_k|$为$f$关于划分$P$的Darboux上和,类似定义$s(P,f)=\sum_{k=1}^{m}\inf_{x\in I_k}f(x)|I_k|$为$f$关于划分$P$的Darboux下和.那么首先,Darboux上下和就是关于划分$P$的所有黎曼和中的上下确界.如果对闭矩体$I$上一个划分$P$,把每个$[a^i,b^i]$上的划分进行加细,即添加新的分割点,那么看到加细后的Darboux上和不增,下和不减.现在,如果给定两个分割$P,P'$,记$P+P'$表示两个分割的并构成的分割,那么$P+P'$同时是$P$和$P'$的加细,于是得到$s(P')\le s(P+P')\le S(P+P')\le S(P)$,于是上和永远不小于下和,特别的,看到上和与下和总以$m|I|$和$M(I)$作为上下界,其中$m=\inf_{x\in I}f(x)$,$M=\sup_{x\in I}f(x)$,按照$f$在$I$上有界看到$m,M$都是有限的.由此,定义$f$在$I$上的上积分是全部Darboux上和的下确界,定义下积分是全部Darboux下和的上确界.

上下积分的等价定义.上积分$S$满足,对任意的$\varepsilon>0$,总存在$\delta>0$,使得只要分割$P$的直径小于$\delta$,那么就有$|S(P,f)-S|<\varepsilon$.对偶的有关于下积分的命题.为了证明这个内容,需要引理,如果对分割$P$,在每个$[a^i,b^i],1\le i\le n$上新添加若干个分割点,得到的新分割记作$P'$,那么有$S(P')\ge S(P)-(M-m)Pd(P)$,其中$P$是只依赖于新增分割点的一个固定正数.现在,对任意$\varepsilon>0$,按照上积分$S$的定义,存在分割$P'$使得$S(P')<S+\frac{\varepsilon}{2}$.对任意的另一个分割$P$,按照引理,约定$P+P'$相比$P'$,在每个$[a^i,b^i]$上新增了$p_i$个分割点.那么得到$S(P)-(M-m)Pd(P)\le S(P+P')$.于是当$d(P)<\delta=\frac{\varepsilon}{2(M-m+1)P}$时,就得到$S\le S(P)\le S(P')+(M-m)Pd(P)<\varepsilon$,完成证明.

至此给出黎曼可积的第一个等价描述,Darboux准则.一个闭矩体$I$上的实值有界函数$f$是黎曼可积的,当且仅当它在$I$上的上下积分相等,此时黎曼积分就是这个相同的上下积分.
\begin{proof}
	
	一方面,如果$f$在$I$上黎曼可积,设积分值为$J$,那么对于任意的$\varepsilon>0$,存在$\delta>0$,使得只要$I$上的划分$P$的直径小于$\delta$,就有黎曼和总满足$J-\varepsilon<\sum_{k=1}^{m}f(\xi)|I_k|<J+\varepsilon$.于是,得到$J-\frac{\varepsilon}{4}\le s(P)\le S(P)\le J+\frac{\varepsilon}{4}$对任意的直径小于$\delta$的分割$P$成立,按照上下积分是Darboux上下和的确界,结合加细时上和不增下和不减,得到上下积分满足$J-\varepsilon\le s\le S\le J+\varepsilon$,令$\varepsilon\to0$得证.另一方面,如果$S=s$,记作$J$,按照上下积分的等价描述,对任意$\varepsilon>0$,存在$\delta>0$只要分割$P$的直径小于$\delta$,就有:$J-\varepsilon<s(P)\le\sum_{k=1}^{m}f(\xi_k)|I_k|\le S(T)<J+\varepsilon$,这就说明$f$在$I$上可积并且积分值是$J$.
	
\end{proof}

另外,在实际应用时,这个等价描述可以简化.一个闭矩体$I$上的有界实值函数$f$是黎曼可积的,当且仅当存在$I$上一列分割$P_j$,使得$\lim_{j\to\infty}S(P_j)-s(P_j)=0$.事实上,一方面如果上下积分相同,按照上下积分的等价描述,可以找到这样的一列分割$P_j$.另一方面,如果存在这样的一列$P_j$,注意到上下积分满足$s\ge s(T_j)$和$S\le S(T_j)$,导致$S-s\le S(T_j)-s(T_j)$,条件就保证了$S=s$.

并且于是黎曼积分的定义也是可以简化的.闭矩体$I$上有界实值函数$f$黎曼可积,当且仅当存在实数$J$使得,对任意$\varepsilon>0$,存在$I$的某个分割$P$,使得关于$P$的任意黎曼和$\sigma(P,f,\xi)$总有$\left|\sigma(P,f,\xi)-J\right|<\varepsilon$.

可以把上下积分相等写作另一种等价形式.给定$I$上的划分$P$,对每个小闭矩体$I_k$,记$\sup_{x\in I_k}f(x)-\inf_{x\in I_k}f(x)$为$\omega_k(f)$,称$f$关于分割$P$的振幅和为$S(T)-s(T)=\sum_{k=1}^{m}\omega_k(f)|I_k|$.那么对有界实值函数,上下积分和相同等价于说振幅和是趋于0的,即,对任意$\varepsilon>0$,存在$\delta>0$,只要分割$P$的直径小于$\delta$,那么振幅和就小于$\varepsilon$.据此,又得到了黎曼可积的一个等价描述:闭矩体$I$上的有界实值函数$f$可积,当且仅当对任意$\varepsilon>0$,存在某个分割$P$的振幅和是小于$\varepsilon$的.

这里再给出一个黎曼可积的等价描述.闭矩体$I$上的有界实值函数$f$是黎曼可积的,当且仅当,对任意$\varepsilon>0$和任意$\eta>0$,存在$I$的一个分割$P$使得,振幅大于等于$\varepsilon$的那些小矩体$I_k$的体积和是小于$\eta$的.事实上,一方面如果$f$黎曼可积,那么对$\varepsilon\eta>0$,存在一个分割$P$,使得振幅和$\sum_{k=1}^{m}\omega_k(f)|I_k|<\varepsilon\eta$.于是得到$\sum_{k,\omega_k(f)\ge\varepsilon}\varepsilon|I_k|<\varepsilon\eta$,就得到这个条件.反过来,对任意$\varepsilon'>0$,设$T$表示$f$在$I$上的振幅,取条件里的$\varepsilon=\frac{\varepsilon'}{2|I|},\eta=\frac{\varepsilon
}{2T}$,于是存在分割$P$使得$\sum_{\omega_k(f)\ge\varepsilon}|I_k|<\eta$.于是,得到$\sum_{k=1}^{m}\omega_k(f)|I_k|=\sum_{\omega_k\ge\varepsilon}\omega_k(f)|I_k|+\sum_{\omega_k<\varepsilon}\omega_k(f)|I_k|
<\varepsilon'$.就得到$f$是黎曼可积的.

把之前给出的等价描述统称为Darboux准则,它们都是通过探究黎曼和的上下确界,给出的黎曼可积的等价描述.接下来要给出的第二个准则Lebesgue准则,从完全不同的角度刻画了黎曼可积,描述这个命题需要涉及到一点点实分析内容.给定$\mathbb{R}^n$的子集$E$,称它为零测集,如果对任意的$\varepsilon>0$,总存在至多可数个开矩体$I_k$,满足$E\subset\cup_kI_k$并且$\sum_k|I_k|<\varepsilon$.那么,至多可数点集总是一个零测集.并且零测集的可数并还是零测集.最后,一个非退化的闭矩体$I_{a,b}$,即每个$[a^i,b^i]$不会退化为一个单点,必然不是零测集.称闭矩体$I$上的实值函数是几乎处处连续的,如果它在$I$上的不连续点构成一个零测集.

黎曼可积的勒贝格准则.$f$在闭矩体$I$上黎曼可积当且仅当,它是有界的并且在$I$上几乎处处连续.
\begin{proof}
	
	先证必要性.知道函数$f$在一个点$x_0$连续等价于振幅$\lim_{r\to0^+}\left(\sup_{x\in B(x_0,r)}f(x)-\inf_{x\in B(x_0,r)}f(x)\right)$为0,对正数$\delta$记$D_{\delta}$表示$f$在$I$上振幅大于$\delta$的点构成的集合,那么如果证明了$D_{\delta}$是零测集,按照不连续点集$D$可以表示为$\cup_{n\ge1}D_{\frac{1}{n}}$,就立马得到不连续点集是一个零测集.
	
	现在,对任意的$\varepsilon>0$,按照上一个等价描述,看到可以取一个分割使得振幅大于$\delta$的那些小矩体的体积和可以要求小于$\varepsilon$.知道如果$I$中一个点是某个$I_k$的内点,那么它振幅大于等于$\delta$必然有,这个小闭矩体$I_k$的振幅大于等于$\delta$,于是证明了,不连续点集和全部$I_k$内部的交这个点集是零测集.但是不连续点还有可能出现在小闭矩体的边界上,注意到小闭矩体的边界总是一个零测集,全部有限个小闭矩体的边界就是一个零测集.于是得到了$D_{\delta}$总是零测集.
	
	下面证充分性,来从不连续点集零测和有界推出,对任意预先给定的$\varepsilon>0$,总存在一个振幅和小于$\varepsilon$.记$D$为$f$在$I$上的不连续点集,那么存在开矩体集$\{U_i,i\ge1\}$,使得$D\subset\cup_{i\ge1}U_i$并且$\sum_{i\ge1}|U_i|<\frac{\varepsilon}{2M+1}$,这里$M$表示$f$在$I$上的振幅.现在对任意的$x\in I-\cup_{i\ge1}U_i$,有$f$在$x$处连续.于是可以取到$x$的一个开邻域$U_x$满足,对任意$u\in U_x\cap I$,有$|f(u)-f(x)|<\frac{\varepsilon}{2|I|}$.现在注意到全体$\{U_x\mid x\in I-\cup_{i\ge1}U_i\}$以及全体$\{U_1,U_2,\cdots\}$构成了$I$的一个开覆盖,按照$I$是紧集,得到它有有限子覆盖,记作$\{U_{x_1},\cdots,U_{x_q},U_{i_1},\cdots,U_{i_p}\}$.按照紧致度量空间上的Lebesgue数定理,存在一个正数$\lambda$,使得只要$I$的某个子集的直径小于$\lambda$,那么这个子集必然落在某个$U_{x_j}$或者$U_{i_k}$中.于是,当对$I$做的分割$P$的直径小于$\lambda$时,每个小闭矩体$\{I_1,I_2,\cdots,I_m\}$都会落在某个选出的$U_{x_j}$或者$U_{i_k}$中,这导致振幅和总有放缩:
	\begin{align*}
	\sum_{k=1}^{m}\omega_k(f)|I_k|&\le \sum_{I_k\subset U_{i_s}}\omega_k(f)|I_k|+\sum_{I_k\subset}\omega_k(f)|I_k|  \\
	&\le M\sum_{I_k\subset U_{i_s}}|I_k|+\frac{\varepsilon}{2|I|}\sum_{k=1}^{m}|I_k| \\
	&\le M\sum_{i\ge1}|U_i|+\frac{\varepsilon}{2} \\
	&\le \frac{M\varepsilon}{2M+1}+\frac{\varepsilon}{2}<\varepsilon
	\end{align*}
	
\end{proof}

至此完成了对闭矩体上实值函数的黎曼可积的定义,以及两个判断可积的准则.现在把闭矩体这个条件推广到更为广泛的一类点集上.称$\mathbb{R}^n$中的一个子集为容许集,如果这个子集有界,并且边界点集是勒贝格零测集.回顾一下边界点集的一些基本性质:点集$E$的边界点集$\partial E$总是闭集;对任意点集$E_1,E_2$,$E_1\cup E_2$,$E_1\cap E_2$,$E_1\setminus E_2$这三个点集的边界点集都落在$\partial E_1\cup\partial E_2$中.于是,容许集的有限交;有限并;差都是容许集.

现在给定$\mathbb{R}^n$上的一个有界子集$E$,假设有一个$E$上的实值函数$f$,一个自然并且合理的定义$f$在$E$上的黎曼积分的方式是,取定一个包含$E$的闭矩体$I$,把$f$延拓到$I$上,使得在$I\setminus E$上处处取值为0,那么如果$f$在$I$上是黎曼可积的,就称$f$在$E$上黎曼可积,并且积分值定义为$\int_Ef(x)\mathrm{d}x=\int_I\hat{f}(x)\mathrm{d}x$,其中$\hat{f}$表示延拓函数.等价的,也可以取$E$的特征函数$\chi_E$,即对$x\in E$有$\chi_E(x)=1$,对$x\not\in E$有$\chi_E(x)=0$,那么定义$\int_Ef(x)\mathrm{d}x=\int_If(x)\chi_E(x)\mathrm{d}x$.

接下来要做的自然就是证明这个定义的良性,即定义不依赖于包含$E$的闭矩体$I$的选取.倘若两个闭矩体$I_1,I_2$都包含了$E$,那么$I_1\cap I_2$同样是个包含$E$的闭矩体,并且$f(x)\chi_E(x)$在$E_1\setminus E_1\cap E_2$和$E_2\setminus E_1\cap E_2$上处处连续,所以$f(x)\chi_E(x)$限制在$I_1$和限制在$I_2$上的不连续点都只能出现在$E_1\cap E_2$中,于是根据勒贝格准则,这就保证了这两个定义域不同的函数是同时可积和同时不可积的.另外,按照$I_1\cap I_2$以外的点都取0,于是关于$I_i$上分割的黎曼和,实际上就是一个关于$I_1\cap I_2$上分割的黎曼和,于是必然有$\int_{I_i}f(x)\chi_E(x)\mathrm{d}x=\int_{I_1\cap I_2}f(x)\chi_E(x)\mathrm{d}x$,看出可积时二者积分相同.

现在给定有界集$E$上的实值函数$f$,那么$f$的不连续点可以分为两类:是$E$边界点的不连续点,和是$E$内点的不连续点.当考虑延拓函数$f(x)\chi_E(x)$时,$E$的内点的连续性不会改变,并且不连续的边界点仍然是不连续点,但是,由于新增了函数趋于边界点的一种方式,即0值趋近,导致可能存在原本连续的边界点变为不连续的边界点,也就是那些取值不为0的边界连续点.分析一般有界集的边界点集的结构往往是十分复杂的,于是,如果我们约定子集是容许集,那么尽管边界点集上会新增不连续点,但是这顶多增加零测的不连续点集,按照勒贝格准则这不影响函数的黎曼可积性.于是,对于容许集上的实值函数,它黎曼可积当且仅当这个函数的不连续点集是勒贝格零测集.

容许集的等价定义.一个有界子集$E\subset\mathbb{R}^n$是容许集,当且仅当黎曼积分$\int_E1\cdot\mathrm{d}x$存在.事实上按照定义这个积分存在当且仅当对任意的包含$E$的闭矩体$I$,有$\chi_E(x)$限制在$I$上是黎曼可积函数,但是$\chi_E(x)$的不连续点集就是边界点集$\partial E$,于是,集合的特征函数黎曼可积恰好就是所定义的容许集.

现在来探讨$\mu(E)$的几何意义.对于容许集$E$,按照Darboux准则,$\mu(E)$等于$\chi_E(x)$的上下Darboux和的极限.但是按照定义,$\chi_E(x)$的下Darboux和就是对闭矩体$I\supset E$的分割,的那些包含于$E$的小矩体的体积和;上Darboux和就是对闭矩体$I$的分割,的那些和$E$有交的小矩体的体积和.换句话说,面都和标准平面平行的$\mathbb{R}^n$中的多面体,分别从$E$的内部和外部做极限,那么多面体体积的极限相同.于是$\mu(E)$可以充当高维体积这一角色.

详细探究这个几何意义就会得到Jordan可测的概念.首先这个特殊的多面体就是有限个闭矩体的并,称为简单集.那么对于简单集,总可以把它划分为有限个两两的交不会包含内点的闭矩体的并,并且对于不同的这种分解,小矩体的体积和不变,称为这个简单集$S$的体积,记作$\mu(S)$.接下来,定义一个有界集的$E$的Jordan内测度$\mu_*(E)$为$\sup_{S\subset B}\mu(S)$,其中$S$取简单集.定义有界集$E$的Jordan外测度$\mu^*(E)$为$\inf_{E\subset S}\mu(S)$,这里$S$同样取简单集.注意取确界的这个$\mu(S)$,其中$S$包含于$E$或者包含着$E$,分别就是某个分割下,$\chi_E(x)$的Darboux下和与Darboux上和,于是按照Darboux定理,有界集$E$的Jordan内外测度,就是特征函数$\chi_E$的上下积分.定义Jordan可测集是有界的内外测度相同的点集,并且把这个相同的数值称为Jordan测度.于是Jordan可测集恰好就是特征函数是黎曼可积的函数,也恰好就是容许集.另外Jordan测度就是对应特征函数的黎曼积分.

关于Jordan可测集做一点注解.首先,定义里的闭矩体可以改为开矩体,也可以改为半开矩体,可以证明这些定义是等价的.另外,Jordan测度实际上不满足现代术语中测度的定义,因为全体Jordan可测集没有构成一个$\sigma$代数,例如,单点集总是Jordan可测的,于是如果可测集构成$\sigma$代数,那么有界可数集理应是Jordan可测的,但是$[0,1]$上的全体有理数构成的子集,它的Jordan内测度是0,外测度却是1.因此,有时不把它称为Jordan测度,而是称为Jordan Content.

这一段给出Jordan内外测度和勒贝格测度的关系.对于有界集$E$,以$\overline{E}$和$E^{\circ}$分别表示$E$的闭包和内点集,以$m(-)$表示勒贝格测度,注意到开集和闭集总是勒贝格可测的,那么有$\mu^*(E)=\mu^*(\overline{E})=m(\overline{E})$,$\mu_*(E)=\mu_*(E^{\circ})=m(E^{\circ})$.
\begin{proof}
	
	先来证明Jordan外侧度的等式.一方面,按照外侧度的单调性有$\mu^*(E)\le\mu^*(\overline{E})$,另一方面,对任意$\varepsilon>0$,按照定义,存在闭矩体的有限集$\{I_1,\cdots,I_m\}$,使得这些闭矩体的并覆盖了$E$,并且有$|\cup_{i\ge1}I_k|\le mu^*(E)+\varepsilon$.但是按照这些闭矩体的并还是一个闭集,它覆盖了$\overline{E}$,于是这些闭矩体的体积和不小于$\mu^*(\overline{E})$,就得到$\mu^*(\overline{E})\le\mu^*(E)+\varepsilon$,再由$\varepsilon$的任意性,得到$\mu^*(\overline{E})\le\mu^*(E)$.
	
	接下来证明如果$E$是有界闭集,也就是紧集,就有$\mu^*(E)=m(E)$.一方面,按照有限覆盖总是可数覆盖,得到$\mu^*(E)\ge m(E)$.另一方面,对任意的$\varepsilon>0$,按照勒贝格测度的定义,存在可数个闭矩体$\{I_i\}_{i\ge1}$覆盖了$E$,并且满足$|\cup_{1\le i\le r}I_i|< m(E)+\varepsilon$.现在,按照$E$是紧集,存在$\{I_i\}$中的有限个闭矩体$\{I_{t_1},\cdots,I_{t_r}\}$覆盖了$E$,于是得到$\mu^*(E)\le\sum_{k=1}^{r}|I_{t_k}|\le\sum_{i\ge1}|I_i|<m(E)+\varepsilon$,再由$\varepsilon$的任意性,得到$\mu^*(E)\le m(E)$.
	
	现在证明Jordan内测度的等式.一方面,同样按照单调性得到$\mu_*(E)\ge\mu_*(E^{\circ})$.另一方面,对任意的$\varepsilon>0$,按照内测度的定义,有$E$中的有限个闭矩体$U_i$,它们的并记作$U$,使得$\mu_*(E)<|U|+\frac{\varepsilon}{2}$.现在对每个$U_i$,将它的每个边长收缩足够小的$\delta$长,得到$U_i'$,使得每个$U_i'$包含在$U_i$的内点集中,并且$U_i'$的并$U'$满足$|U|-|U'|<\frac{\varepsilon}{2}$.这就导致$\mu_*(E^{\circ})\ge|U'|>|U|-\frac{\varepsilon}{2}>\mu_*(E)-\varepsilon$.按照$\varepsilon$的任意性就得到$\mu_*(E)=\mu_*(E^{\circ})$.
	
	最后证明当$E$是一个有界开集时,有$\mu_*(E)=m(E)$.【】
	
\end{proof}

把Jordan测度为0的点集称为Jordan零测集,那么在黎曼可积的勒贝格准则中,实际上证明了每个$D_{\delta}$是一个Jordan零测集,由此得到勒贝格准则的另一等价形式:一个闭矩体$I$上的有界实值函数是黎曼可积的当且仅当,它的不连续点集是可数个Jordan零测集的并.

按照边界点集就是闭包扣去内点集,结合上述定理,也得到了有界集Jordan可测当且仅当边界点集是勒贝格零测的.另外,边界点集总是闭集,结合Jordan可测集的定义要有界集,得到边界点集是紧集.而容易验证紧集上Jordan零测是等价于勒贝格零测的,于是一个有界集是Jordan可测的也等价于边界点集是Jordan零测的.

现在开始给出黎曼可积的性质.给定有界集合$E\subset\mathbb{R}^n$,那么全体在$E$上黎曼可积的实值函数构成了一个$\mathbb{R}$线性空间$\mathscr{R}(E)$.那么这时$E$上的黎曼积分就是线性空间$\mathscr{R}(E)$上的线性泛函.

按照黎曼积分的定义,如果一个函数$f$在容许集$E\subset\mathbb{R}$上黎曼可积,并且$f$在$E$上几乎处处为0,注意到零测集的补集是稠密集,于是$f$的黎曼积分$\int_Ef(x)\mathrm{d}x=0$.如果两个$E$上黎曼可积的函数是几乎处处相同的,就有它们的黎曼积分是相同的.由此,黎曼积分作为线性泛函,诱导了一个商空间上的泛函,即两个$E$上黎曼可积函数等价定义为几乎处处相同,此时黎曼积分这个线性泛函可以定义在商空间上.反过来,如果容许集$E$上一个非负函数$f$是可积的,并且积分为0,那么容易证明连续点必然不能取0,否则积分大于0.于是,不连续点只能取非0数,导致这样的函数不取0的点是零测集,取0的点是稠密集.或者,从Darboux准则可以证明弱化版本的勒贝格准则:黎曼可积函数连续点是稠密的,也可以说明取0的点是稠密集.

一个Jordan零测集上的有界函数总是黎曼可积的,并且积分是0.这只要注意到覆盖这个Jordan零测集的简单集的体积可以任意足够小.

给定$\mathbb{R}^n$中的两个容许集$E_1,E_2$,那么$E_1\cup E_2$也是容许集,如果实值函数定义在$E_1\cup E_2$上,那么$f$在$E_1\cup E_2$上黎曼可积当且仅当$f$分别在$E_1,E_2$上黎曼可积.这个证明只要利用勒贝格准则,并且注意到容许集的边界点集零测,导致只要讨论内点的连续性,于是$f$在$E_1\cup E_2$上的不连续的内点集必然包含在$E_1$和$E_2$的不连续的内点集并上$E_1$和$E_2$的边界点集.另外,如果还要求$E_1\cap E_2$是Jordan零测的,那么有如下积分等式.为此只要注意到$\chi_{E_1\cup E_2}=\chi_{E_1}+\chi_{E_2}-\chi_{E_1\cap E_2}$.
$$\int_{E_1\cup E_2}f(x)\mathrm{d}x=\int_{E_1}f(x)\mathrm{d}x+\int_{E_2}f(x)\mathrm{d}x$$

这里给出一些可积函数的例子.首先,按照勒贝格准则,一个闭矩体上的连续函数自然是黎曼可积的.这一事实还可以用一致连续定理,从达布准则证明.另外,对于$\mathbb{R}$的闭区间上的单调函数总是黎曼可积的.同样的这一事实可以从勒贝格准则和达布准则两个角度证明.

这里来给出一些积分的估计式.
\begin{enumerate}
	\item 如果$f\in\mathscr{R}(E)$,那么按照$|f|$的不连续点集包含于$f$的不连续点集,得到$|f|\in\mathscr{R}(E)$.另外,按照$f$的黎曼和与$|f|$的黎曼和的不等式关系,取极限得到:
	$$\left|\int_Ef(x)\mathrm{d}x\right|\le\int_E|f(x)|\mathrm{d}x$$
	\item 对非负函数$f\in\mathscr{R}(E)$,按照黎曼和总是非负的,取极限得到积分总是非负的.据此看到如果两个黎曼可积函数$f,g\in\mathscr{R}(E)$,并且满足$E$上处处有$f(x)\le g(x)$,那么有积分不等式$\int_Ef(x)\mathrm{d}x\le\int_Eg(x)\mathrm{d}x$.最后,如果黎曼可积函数$f$满足$m\le f(x)\le M$,那么总有$m \mu(E)\le\int_Ef(x)\mathrm{d}x\le M\mu(E)$.
\end{enumerate}

积分中值定理.
\begin{enumerate}
	\item 一维情况,给定闭区间$[a,b]$上的连续函数,那么按照最值定理,$f$在$[a,b]$上取到最大值$M$和最小值$m$,于是按照积分不等式,看到$m(b-a)\le\int_a^bf(x)\mathrm{d}x\le M(b-a)$.现在,按照闭区间上连续函数的介值定理,可以取到一个$\xi\in[a,b]$满足
	$$\int_a^bf(x)\mathrm{d}x=f(\xi)(b-a)$$
	\item 一维情况,给定闭区间$[a,b]$上的连续函数$f,g$,如果约定$g$在$[a,b]$上不变号,即$g$要么恒非负,要么恒非正.那么不妨设$g(x)\ge0$,同样按照最值定理,有$f$在$[a,b]$上最小值$m$和最大值$M$,于是得到$m\int_a^bg(x)\mathrm{d}x\le\int_a^bf(x)g(x)\mathrm{d}x\le M\int_a^bfg(x)\mathrm{d}x$.于是只要$g$不是恒为0的映射,那么就可以找到一个$\xi\in[a,b]$满足:
	$$\int_a^bf(x)g(x)\mathrm{d}x=f(\xi)\int_a^bg(x)\mathrm{d}x$$
	\item 高维一般情况,给定Jordan可测集$\Omega\subset\mathbb{R}^n$,设他是连通紧致集,设$f,g:\Omega\to\mathbb{R}$是连续函数,并且$g$在$\Omega$上不变号.那么$f,g,fg$都是$\Omega$上的黎曼可积函数,按照紧空间上连续函数的最值定理有$f$在$\Omega$上取到最大值$M$和最小值$m$,再结合连通集上连续函数的介值定理,就得到,必然存在一个$\xi\in\Omega$使得:
	$$\int_{\Omega}fg=f(\xi)\int_{\Omega}g$$
\end{enumerate}

在一维情况下还有如下的第二积分中值定理:设$f$在$[a,b]$上黎曼可积,
\begin{enumerate}
	\item 如果$g$在$[a,b]$上单调减,并且$g(x)\ge0,x\in[a,b]$,那么存在$\xi\in[a,b]$满足:$$\int_a^bf(x)g(x)\mathrm{d}x=g(a)\int_a^{\xi}f(x)\mathrm{d}x$$
	\begin{proof}
		
		不妨设$g(a)>0$,否则如果$g(a)=0$,就恒有$g(x)\equiv0,x\in[a,b]$,那么此时取任意的$\xi\in[a,b]$都成立.现在按照$f(x)$黎曼可积说明它有界,于是函数$F(x)=\int_a^xf(x)\mathrm{d}x$是连续函数,于是可以取到最大值$M$和最小值$m$,的思路是证明不等式$m\le\frac{\int_a^bf(x)g(x)\mathrm{d}x}{g(a)}\le M$,于是按照$F(x)$的介值定理就得证.
		
		设$f$在$[a,b]$上满足$|f|<L$,按照$g$在$[a,b]$上黎曼可积,存在$[a,b]$的分割$P:a=x_0<x_1<\cdots<x_n=b$,满足振幅和$\sum_{1\le k\le n}\omega_k(g)|I_k|<\frac{\varepsilon}{L}$.于是得到:
		\begin{align*}
		\int_a^bf(x)g(x)\mathrm{d}x &=\sum_{k=1}^{n}\int_{x_{k-1}}^{x_k}f(x)g(x)\mathrm{d}x \\
		&=\sum_{k=1}^{n}\int_{x_{k-1}}^{x_k}\left(g(x)-g(x_{k-1})\right)f(x)\mathrm{d}x+\sum_{k=1}^{n}g(x_{k-1})\int_{x_{k-1}}^{x_k}f(x)\mathrm{d}x \\
		&\le\sum_{k=1}^{n}\left|g(x)-g(x_{k-1})\right|\cdot\left|f(x)\right|\mathrm{d}x+\sum_{k=1}^{n}g(x_{k-1})\left(F(x_k)-F(x_{k-1})\right) \\
		&\le L\sum_{k=1}^{n}\omega_k(f)|I_k|+\sum_{k=1}^{n-1}F(x_k)\left(g(x_{k-1})-g(x_k)\right)+F(b)g(x_{n-1}) \\
		&<L\cdot\frac{\varepsilon}{L}+M\sum_{k=1}^{n-1}\left(g(x_{k-1})-g(x_k)\right)+Mg(x_{n-1})=\varepsilon+Mg(a)
		\end{align*}
		
		现在按照$\varepsilon$的任意性,就得到$\int_a^bf(x)g(x)\mathrm{d}x\le Mg(a)$,同理得到另一侧的不等式,完成证明.
	\end{proof}
	\item 如果$g$在$[a,b]$上是单调增的,并且$g(x)\ge0,x\in[a,b]$,那么存在$\eta\in[a,b]$使得如下不等式成立,这个证明和第一个是类似的.
	$$\int_a^bf(x)g(x)\mathrm{d}x=g(b)\int_{\eta}^bf(x)\mathrm{d}x$$
	\item 从如上两种情况得到一般情况:若$g$是单调函数,那么存在$\xi\in[a,b]$满足:
	$$\int_a^bf(x)g(x)\mathrm{d}x=g(a)\int_a^{\xi}f(x)\mathrm{d}x+g(b)\int_{\xi}^bf(x)\mathrm{d}x$$
\end{enumerate}
\newpage
\subsection{微积分学基本定理}

给定闭区间$[a,b]$上的黎曼可积函数$f$,定义函数$F(x)=\int_a^xf(x)\mathrm{d}x$为$f$的变上限积分函数.那么由于$f$在$[a,b]$可积得到$f$在$[a,x]$可积,于是定义良性.

变上限积分函数总是满足Lipschitz条件的函数,特别的,它总是一致连续的,也是连续的.这是因为,按照黎曼可积必有界,存在正数$M$满足$|f(x)|\le M,x\in[a,b]$,于是,得到$|F(x)-F(y)|=\left|\int_x^yf(x)\mathrm{d}x\right|\le M|x-y|$.另外,对于$f$的连续点,必然是$F$的可导点,并且$F'(x_0)=f(x_0)$:
\begin{proof}
	
	如果$f$在$x_0\in[a,b]$处连续,那么对任意$\varepsilon>0$,存在$\delta>0$使得$|h|<\delta$时候有$\left|f(x_0+h)-f(x_0)\right|<\varepsilon$,那么从:
	$$F(x_0+h)-F(x_0)=\int_{x_0}^{x_0+h}f(x)\mathrm{d}x=f(x_0)h+\int_{x_0}^{x_0+h}\left(f(x)-f(x_0)\right)\mathrm{d}x$$
	
	看到最后一项是$o(h)$的,这就得证.
	
\end{proof}

于是立刻得到,闭区间上的连续函数$f$总存在原函数,即$F(x)=\int_a^xf(x)\mathrm{d}x+C$.

据此可以得到\textbf{微积分学第一基本定理}:如果$f$在闭区间$[a,b]$上连续,设$F$为$f$在$[a,b]$上的任一原函数,那么有:
$$\int_a^bf(x)\mathrm{d}x=F(b)-F(a)$$

但是这个条件实际上可以减弱,即\textbf{微积分学第二基本定理}:如果$f$在闭区间$[a,b]$上可积,并且有原函数$F(x)$,那么就有公式:
$$\int_a^bf(x)\mathrm{d}x=F(b)-F(a)$$
\begin{proof}
	
	设$f$在$[a,b]$上的原函数为$F(x)$,那么$F(x)$在$[a,b]$连续,在$(a,b)$可导.做分割$a=x_0<x_1<\cdots<x_n=b$,于是得到$F(b)-F(a)=\sum_{k=1}^{n}\left(F(x_{k})-F(x_{k-1})\right)$.按照中值定理,可以找到$c_i\in[x_{i-1},x_i]$,满足$F(x_i)-F(x_{i-1})=f(c_i)(x_i-x_{i-1})$,于是,得到$F(b)-F(a)=\sum_{k=1}^{n}f(c_i)(x_i-x_{i-1})$.这说明,对任意分割,总存在黎曼和取值恰好为$F(b)-F(a)$,按照条件已经有$f$是黎曼可积的,于是这个积分只能是$F(b)-F(a)$,因为如果积分是异于$F(b)-F(a)=p$的实数$q$,那么存在一个正数$\delta$使得分割的直径小于$\delta$时候有黎曼和与$q$的差总是小于$\frac{|p-q|}{2}$的,但是这就导致这个分割上的黎曼和总是不能取到$p$的,这就矛盾.于是得证.
	
\end{proof}

注意这个证明实际上只需要$F$在有限个$[a,b]$中的点以外有$F'(x)=f(x)$即可.
\newpage
\subsection{黎曼积分的计算}

首先介绍一元积分的分部积分.
\begin{enumerate}
	\item 如果$u,v$是$[a,b]$上的连续可导函数,那么$u'(x)v(x)$和$v'(x)u(x)$都是连续函数,于是按照微积分学基本定理,得到:
	$$\int_a^bu(x)v'(x)\mathrm{d}x=u(x)v(x)\mid_a^b-\int_a^bu'(x)v(x)\mathrm{d}x$$
	\item 如果$u,v$是$[a,b]$上的可导函数,如果$u'(x)v(x)$和$v'(x)u(x)$都是可积函数,并且其中一个有原函数,不妨设$u'(x)v(x)$有原函数,那么按照微积分学基本定理,得到$u(x)v'(x)$有原函数,并且有:
	$$\int_a^bu(x)v'(x)\mathrm{d}x=u(x)v(x)\mid_a^b-\int_a^bu'(x)v(x)\mathrm{d}x$$
	\item 如果$u(x),v(x)$在$[a,b]$上有直至$n+1$阶连续导数,那么反复运用第一条,得到:
	$$\int_a^bu(x)v^{(n+1)}(x)\mathrm{d}x$$
	$$=\left(u(x)v^ {(n)}(x)-u'(x)v^{(n-1)}(x)+\cdots+(-1)^nu^{(n)}(x)v(x)\right)\mid_a^b+(-1)^{n+1}\int_a^bu^{(n+1)}(x)v(x)\mathrm{d}x$$
\end{enumerate}

借助分部积分,可以得到Taylor公式的积分余项.如果$f$在$x_0$的某个开邻域$U$内有直至$n+1$阶连续导数,设$x\in U$,那么按照上述分部积分第三条,得到$\int_{x_0}^x(x-t)^nf^{(n+1)}(t)\mathrm{d}t=n!R_n(x)$,即:
$$f(x)-\sum_{k=0}^{n}\frac{f^{(k)}(x_0)}{k!}(x-x_0)^k=R_n(x)=\frac{1}{n!}\int_{x_0}^xf^{(n+1)}(t)(x-t)^n\mathrm{d}t$$

注意到$f^{(n+1)}(t)$连续并且$(x-t)^n$在区间$x$到$x_0$上同号,于是按照第一积分中值定理,得到$R_n(x)=\frac{1}{n!}f^{(n+1)}(\xi)\int_{x_0}^x(x-t)^n\mathrm{d}t=\frac{f^{(n+1)}(\xi)}{(n+1)!}(x-x_0)^{n+1}$,这就是Lagrange余项.如果直接用积分中值定理,得到$R_n(x)=\frac{1}{n!}f^{(n+1)}(\xi)(x-\xi)^n\int_{x_0}^x\mathrm{d}x=\frac{1}{n!}f^{(n+1)}(\xi)(x-\xi)^n(x-x_0)$,这就是Cauchy余项.

一元积分的换元积分.
\begin{enumerate}
	\item 设$f(x)$在$[a,b]$上连续,$x=\phi(t)$在$[\alpha,\beta]$上连续可导,并且$\phi([\alpha,\beta])\subset[a,b]$,$\phi(\alpha)=a$,$\phi(\beta)=b$,那么有:
	$$\int_a^bf(x)\mathrm{d}x=\int_{\alpha}^{\beta}f(\phi(t))\phi'(t)\mathrm{d}t$$
	\begin{proof}
		
		按照$f$连续说明它有原函数$F(x)$,于是$F(\phi(t))$是$f(\phi(t))\phi'(t)$的原函数,于是按照微积分学基本定理,得到上式.
		
	\end{proof}
	\item 设$f(x)$在$[a,b]$上黎曼可积,$x=\phi(t)$连续可导,并且当$t$从$\alpha$到$\beta$时$\phi(t)$严格递增取值从$a$到$b$,那么有:
	$$\int_a^bf(x)\mathrm{d}x=\int_{\alpha}^{\beta}f(\phi(t))\phi'(t)\mathrm{d}t$$
	\begin{proof}
		
		对任意的$\varepsilon>0$,按照$\phi'(t)$连续,于是它在$[a,b]$上一致连续,于是存在$\delta_1>0$,使得$t',t''\in[\alpha,\beta]$并且$|t'-t''|<\delta_1$时有$|\phi'(t')-\phi'(t'')|<\frac{\varepsilon}{(3M+1)(\beta-\alpha)}$,其中$M=\sup_{x\in[a,b]}|f(x)|$,按照$f$可积知这个上确界是有限的.现在按照$f$可积,知存在$\delta_2>0$使得只要分割$T$的直径小于$\delta_2$,就有振幅和小于$\frac{\varepsilon}{3}$,并且黎曼和满足$\left|\sum_{k=1}^nf(\xi_k)|I_k|-\int_a^bf(x)\mathrm{d}x\right|<\frac{\varepsilon}{3}$,$\xi_k\in[x_{k-1},x_k]$.现在对$[\alpha,\beta]$的任一分割$W:a=t_0<t_1<\cdots<t_n=\beta$,设$x_k=\phi(t_k)$,那么有$x_k-x_{k-1}=\phi(t_k)-\phi(t_{k-1})=\phi'(\tau_k)(t_k-t_{k-1})$,$\tau_k\in[t_{k-1},t_k]$.于是当分割$W$的直径小于$\delta=\min\{\delta_1,\frac{\delta_2}{K+1}\}$时,其中$K=\max_{[\alpha,\beta]}|\phi'(t)|$,就得到$T$的直径小于$\delta_2$.于是:
		\begin{align*}
		\left|\sum_{k=1}^{n}f(\phi(\tau_k'))\phi'(\tau_k')-\int_a^bf(x)\mathrm{d}x\right| &\le\sum_{k=1}^{n}|f(\phi(\tau_k'))|\cdot|\phi'(\tau_k')-\phi'(\tau_k)|(t_k-t_{k-1})\\
		&\quad +\sum_{k=1}^{n}|f(\phi(\tau_k'))-f(\phi(\tau_k))|\phi'(\tau_k)(t_k-t_{k-1})\\
		&\quad +\left|\sum_{k=1}^{n}f(\phi(\tau_k))\phi'(\tau_k)-\int_a^bf(x)\mathrm{d}x\right|\\
		&\le M\frac{\varepsilon}{3M+1}+\frac{\varepsilon}{3}+\frac{\varepsilon}{3}<\varepsilon
		\end{align*}
		
		完成证明.
	\end{proof}
\end{enumerate}

Fubini定理,二重积分和累次积分的转化.给定$\mathbb{R}^m$和$\mathbb{R}^n$的闭矩体$X,Y$,那么$X\times Y$是$\mathbb{R}^{m+n}$上的闭矩体.设$f$是$X\times Y$上的实值函数,对每个$x\in X$,积分$\int_Yf(x,y)\mathrm{d}x$尽管未必存在,但是存在上下积分.先断言,如果$X\times Y$上的重积分$f$可积,那么上述上下积分函数$\phi(x)$和$\psi(x)$总是$X$上的可积函数,并且积分值都是$\int_{X\times Y}f$.
\begin{proof}
	
	对$X$做分割$T_x:X_1,X_2,\cdots,X_p$,对$Y$做分割$T_y:Y_1,Y_2,\cdots,Y_q$.于是得到了$X\times Y$的分割$T:X_i\times Y_j,1\le i\le p,1\le j\le q$.按照$f$是可积函数,于是对任意的$\varepsilon>0$,总存在$\delta>0$,使得只要分割$T$的直径小于$\delta$,就有:
	$$\int_{X\times Y}f-\frac{\varepsilon}{2}<\sum_{i=1}^{p}\sum_{j=1}^{q}f(\xi_i,\eta_j)|X_i||Y_j|<\int_{X\times Y}f+\frac{\varepsilon}{2}$$
	
	现在按照$m,n$固定,可以取$\delta_1,\delta_2>0$,使得只要$T_x$的直径小于$\delta_1$,$T_y$的直径小于$\delta_2$,那么就有$T$的直径小于$\delta$,于是,在$T_x,T_y$直径分别小于$\delta_1,\delta_2$的前提下,有:
	$$\int_{X\times Y}f-\varepsilon<\sum_{i=1}^{p}\sum_{j=1}^{q}\inf f(\xi_i,\eta_j)|X_i||Y_j|\le\sum_{i=1}^{p}\sum_{j=1}^{q}\sup f(\xi_i,\eta_j)|X_i||Y_j|<\int_{X\times Y}f+\varepsilon$$
	
	注意到$\sum_{j=1}^{q}\inf f(\xi_i,\eta_j)|Y_j|$和$\sum_{j=1}^{q}\sup f(\xi_i,\eta_j)|Y_j|$分别是$f(\xi_i,y)$在$Y$上的上下Darboux和,于是得到函数$\phi(x)$和$\psi(x)$的黎曼和夹在$\int_{X\times Y}f-\varepsilon$和$\int_{X\times Y}f+\varepsilon$之间,这就说明上下积分函数$\phi(x)$和$\psi(x)$都是$X$上的可积函数,并且积分值就是$\int_{X\times Y}f$.
	
\end{proof}

现在,按照上述定理,看到上下积分函数满足非负函数$\phi(x)-\psi(x)$在$X$上的积分为0,这说明这个函数不取0的点是一个零测集.于是上下积分函数是几乎处处相同的.也就是说几乎处处有积分$\int_Yf(x,y)\mathrm{d}y$和$\int_Xf(x,y)\mathrm{d}x$存在.

现在讨论多元积分上的还原积分.给定$\mathbb{R}^n$中的一个点集$D_x$,和其上的一个黎曼可积函数$f$,并且存在一个映射$\phi:D_t\to D_x$,期望在知晓$f$和$\phi$的前提下,寻找一个$D_t$上的函数$\psi$,使得有$\int_{D_x}f(x)\mathrm{d}x=\int_{D_t}\psi(t)\mathrm{d}t$.也就是把积分域$D_x$换作$D_t$.在给出定理前先来说明下思路.

首先假设$D_t$是$\mathbb{R}^n$中的一个闭矩体,并且$\phi:I\to D_x$是一个微分同胚.于是对$I$的每一种分割$I_1,I_2,\cdots,I_k$,它对应着$D_x$的一个分割$\phi(I_1),\phi(I_2),\cdots,\phi(I_k)$.如果这些点集都是Jordan可测的并且它们任意两个的交集是Jordan零测的,那么按照积分的加性,得到$\int_{D_x}f(x)\mathrm{d}x=\sum_{i=1}^{k}\int_{\phi(I_i)}f(x)\mathrm{d}x$.现在如果$f$在$D_x$上连续,那么按照中值定理,得到$\int_{\phi(I_i)}f(x)\mathrm{d}x=f(\xi_i)\mu(\phi(I_i))$,其中$\xi_i\in\phi(I_i)$,于是可以写作$f(\xi_i)=f(\phi(\tau_i))$.于是只要试图用$|I_i|$来表示$\mu(\phi(I_i))$.现在假设$\phi$是一个线性变换,那么从解析几何里知道此时$\phi(I_i)$会是一个平行六面体.它的体积为$|\det\phi'|\cdot|I_i|$,但是知道微分同胚局部上是线性变换,于是应该有在足够小的开邻域上有$\mu(\phi(I_i))$近似于$|\det\phi'(\tau_i)|\cdot|I_i|$,于是得到:$\sum_{i=1}^{k}\int_{\phi(I_i)}f(x)\mathrm{d}x$近似于$\sum_{i=1}^{k}f(\phi(\tau_i))|\det\phi'(\tau_i)|\cdot|I_i|$,取极限,那么换元积分公式就应该为:
$$\int_{D_x}f(x)\mathrm{d}x=\int_{D_t}f(\phi(t))|\det\phi'(t)|\mathrm{d}t$$

这里先来陈述定理内容,再给出一些引理最后证明这个定理.换元积分定理:如果$\phi:D_t\to D_x$是$\mathbb{R}^n$上有界开集之间的微分同胚(这里要求导数连续).设$f$是$D_x$上的具有紧支集的黎曼可积函数,其中支集包含于$D_x$,那么$f\circ\phi|\det\phi'|$是$D_t$上的黎曼可积函数,并且满足:
$$\int_{D_x=\phi(D_t)}f(x)\mathrm{d}x=\int_{D_t}f\circ\phi(t)|\det\phi'(t)|\mathrm{d}t$$

三个引理.设$\phi:D_t\to D_x$是$\mathbb{R}^n$上两个有界开集之间的微分同胚(要求导数连续),那么:
\begin{enumerate}
	\item 如果$E_t\subset D_t$是勒贝格零测集,那么它的像$\phi(E_t)\subset D_x$也是勒贝格零测集
	\item 如果集合$E_t\subset D_t$,并且$\overline{E_t}$是Jordan零测的,那么像$\phi(E_t)=E_x\subset D_x$,并且像的闭包也具有Jordan测度零.
	\item 如果$E_t\subset D_t$是Jordan可测集,并且$\overline{E_t}\subset D_t$,那么它的像$E_x=\phi(E_t)$也是Jordan可测的并且$\overline{E_x}\subset D_x$.
\end{enumerate}
\begin{proof}
	
	首先说明$\mathbb{R}^n$中的任意开集可以表示为可数个闭矩体的并,这些闭矩体两两的交不含内点.为此,先取$d=1$,然后把坐标轴按照长度$d=1$做分割,取定那些包含在开集内的闭矩体,接下来取$d=\frac{1}{2}$,把之前的分割长度再平分,在新的分割下,取出新的落于开集中的边长为$\frac{1}{2}$的闭矩体,再继续对$d=\frac{1}{2^2}$,这样归纳构造闭矩体列,它们两两的交不含内点,并且并是整个开集.
	
	于是,按照勒贝格零测集的至多可数并是勒贝格零测集,对于第一个引理只要证明$E_t$落在某个一个闭矩体$I$中时,它的像是勒贝格零测集.按照$\phi$连续可导,说明$\phi$满足Lip条件,即存在正数$M$满足$|x_2-x_1|\le M(t_2-t_1),x_i=\phi(t_i)$.现在取$E_t$的一个闭矩体覆盖$\{I_i\}$,使得$\sum_{i}|I_i|<\varepsilon$,不妨约定$I_i\subset I$.现在$\{\phi(I_i)\}$是$E_x=\phi(E_t)$的覆盖,设$t_i$是闭矩体$I_i$的中心,记$x_i=\phi(t_i)$,设以$x_i$为中心,以$I_i$相应平行的边长度为$M$倍的闭矩体为$\widetilde{I_i}$,那么按照刚刚给的Lipschitz条件,就得到$\phi(E_t\cap I)$被$\widetilde{I_i}$覆盖,于是$E_x=\phi(E_t)$被$\{\widetilde{I_i}$覆盖,并且$\sum_{i}|\widetilde{I_i}|<M^n\varepsilon$.这就证明了引理1.
	
	至于引理2只要注意到一个紧集是Jordan零测集等价于是勒贝格零测的,另外在$\phi$是同胚的前提下$\phi(\overline{A})=\overline{\phi(A)}$,再结合引理1得证.最后引理3只要注意到Jordan可测集的等价定义是边界是Jordan零测的,并且同胚把内点映射到内点,这导致$E_x=\phi(E_t)$的边界就是$\phi$下$E_t$边界的像.
	
\end{proof}

至此可以说明定理的等式中右侧的积分是存在的.由于在$D_t$上$|\det\phi'(t)|$不为0,于是$f\circ\phi|\det\phi'|$的支集是$f\circ\phi$的支集,也就是$\phi^{-1}\left(\mathrm{supp}f\right)$.这是一个包含在$D_t$中的紧集,于是$f\circ\phi|\det\phi'|\chi_{D_t}$在$\mathbb{R}^n$中的不连续点集和$\chi_{D_t}$无关,而是包含在$f$不连续点集在$\phi$下的原像中.按照$f$可积说明$f$在$D_x$内的不连续点集$E_x$是零测集,按照上述引理1得到$E_t=\phi^{-1}(E_x)$是$D_t$中的零测集,于是说明了$f\circ\phi|\det\phi'|\chi_{D_t}$是黎曼可积的.

注意,按照一般换元积分定理,一元的情况应该是$\int_{I_x}f(x)\mathrm{d}x=\int_{I_t}f\circ\phi(t)|\phi'(t)|\mathrm{d}t$,之所以之前的描述中$\phi'$没有绝对值,是因为积分记号$\int_{I_x}$本身约定了定向,但是写作上下限形式积分时候,$\int_a^b$可能出现$b<a$的情况.

一个值得指出的事实是,在一维情况下,对于闭区间之间的微分同胚$\phi:I_t\to I_x$,对于任意有界函数,关于上下积分,换元积分公式总是成立的,即:
$$\overline{\int_{I_x}}f(x)\mathrm{d}x=\overline{\int_{I_t}}f\circ\phi(t)|\phi'(t)|\mathrm{d}t$$
$$\underline{\int_{I_x}}f(x)\mathrm{d}x=\underline{\int_{I_t}}f\circ\phi(t)|\phi'(t)|\mathrm{d}t$$
\begin{proof}
	
	先假设$f$是非负的有界函数,设一个上界为$M$.对$I_t$的每个分割$P_t$,记分割点依次排列为$t_i$,记$\phi(t_i)=x_i$,$x_i$作为分割点所构成的$I_x$的分割记作$P_x$,【】
	
\end{proof}

现在继续证明一般的换元积分定理.首先利用已经证明过的一维情况证明初等微分同胚的情况.称微分同胚$\phi:D_t\to D_x$是初等微分同胚,如果$\phi=(\phi_1,\cdots,\phi_n)$满足$\phi_i$中有$n-1$个满足$\phi_i(t_1,\cdots,t_n)=t_i$.断言,换元积分定理对$\phi$取初等微分同胚是成立的.
\begin{proof}
	
	不妨设前$n-1$个分量都是相应的投影,记第$n$个分量函数为$\phi_n$,记$\widetilde{x}=(x_1,x_2,\cdots,x_{n-1})$,记$\widetilde{t}=(t_1,t_2,\cdots,t_{n-1})$,记$D_x'(\widetilde{x_0})=\{(\widetilde{x_0},x_n)\in D_x\mid x_n\in\mathbb{R}\}$,记$D_t'(\widetilde{t_0})=\{(\widetilde{t_0},t_n)\in D_t\mid t_n\in\mathbb{R}\}$.于是$D_x'(\widetilde{x})$和$D_t'(\widetilde{t})$分别表示的是$D_x$和$D_t$的和第$n$个坐标轴平行的一维截面.设$I_x$是$\mathbb{R}^n$中的包含了$D_x$的闭矩体,并记$I=I_{\widetilde{x}}\times I_{x_n}$是一个$n-1$维闭矩体和1维闭矩体的笛卡儿积.类似的定义$I_t=I_{\widetilde{t}}\times I_{t_n}$.按照积分的定义和Fubini定理,以及$\det\phi'=\frac{\partial\phi_n}{\partial t_n}$,得到:
	\begin{align*}
	\int_{D_x}f(x)\mathrm{d}x &=\int_{I_x}f\cdot\chi_{D_x}(x)\mathrm{d}x=\int_{I_{\widetilde{x}}}\mathrm{d}\widetilde{x}\int_{I_{x_n}}f\cdot\chi_{D_x}(\widetilde{x},x_n)\mathrm{d}x_n\\
	&=\int_{I_{\widetilde{x}}}\mathrm{d}\widetilde{x}\int_{D_x'(\widetilde{x})}f(\widetilde{x},x_n)\mathrm{d}x_n\\
	&=\int_{I_{\widetilde{t}}}\mathrm{d}\widetilde{t}\int_{D_t'(\widetilde{t})}f(\widetilde{t},\phi_n(\widetilde{t},t_n))\left|\frac{\partial\phi_n}{\partial t_n}\right|(\widetilde{t},t_n)\mathrm{d}t_n \\
	&=\int_{I_{\widetilde{t}}}\mathrm{d}\widetilde{t}\int_{I_{t_n}}\left(f\circ\phi|\det\phi'|\chi_{D_t}\right)(\widetilde{t},t_n)\mathrm{d}t_n\\
	&=\int_{I_t}\left(f\circ\phi|\det\phi'|\chi_{D_t}\right)(\widetilde{t},t_n)\mathrm{d}t_n=\int_{D_t}\left(f\circ\phi|\det\phi'|\right)(t)\mathrm{d}t
	\end{align*}
\end{proof}

最后,注意到微分同胚必然是至多$n$个初等微分同胚的复合,于是为证明一般情况,还要用到如下引理:如果$\psi:D_{\tau}\to D_t$,$\phi:D_t\to D_x$是两个微分同胚满足换元积分公式,那么复合映射$\alpha=\phi\circ\psi:D_{\tau}\to D_x$也满足换元积分公式:
\begin{align*}
\int_{D_x}f(x)\mathrm{d}x &=\int_{D_t}f\circ\phi(t)|\det\phi'(t)|\mathrm{d}t \\
&=\int_{D_{\tau}}f\circ\phi\circ\psi(x)|\det\phi'\circ\psi(\tau)|\cdot|\det\psi'(\tau)|\mathrm{d}\tau \\
&=\int_{D_{\tau}}f\circ\alpha(\tau)|\det\alpha'(\tau)|\mathrm{d}\tau
\end{align*}

现在来证明换元积分定理.
\begin{proof}
	
	记紧集$K_t=\mathrm{supp}f\circ\phi|\det\phi'|$,于是$K_t\subset D_t$.对$K_t$中的任意一个点$t$,取【】
	
\end{proof}

现在来给出换元积分的一些推论.首先,定理可以推广至Jordan可测集上:给定有界开集之间的微分同胚$\phi:D_t\to D_x$,设$E_t$和$E_x$分别是$D_t$和$D_x$的Jordan可测子集,并且满足$E_t,E_x$的闭包分别包含于$D_t,D_x$.设$E_x=\phi(E_t)$,现在假设$f$是$E_x$上的黎曼可积函数,那么$f\circ\phi|\det\phi'|$是$E_t$上的黎曼可积函数,并且满足:
$$\int_{E_x}f(x)\mathrm{d}x=\int_{E_t}\left(f\circ\phi|\det\phi'|\right)(t)\mathrm{d}t$$

知道有界点集上的黎曼积分的定义是把函数零延拓至包含这个点集的闭矩体上.闭矩体的定义是依赖于坐标轴的选取的.所以一个自然的问题是对于欧式空间中的任意点集,它的黎曼积分是否依赖于坐标轴的选取?借助换元积分定理,按照坐标系的改变对应于一个正交同构$\phi$和一个平移的复合,它的导数的行列式是$\pm1$,于是行列式的绝对值是1.所以当旋转,对称交换,或者平移坐标轴,不会改变点集上函数的黎曼积分值.
\newpage
\subsection{反常积分}

从一元的反常积分讲起.黎曼积分要求函数有界,还要求被积分点集是有界点集.反常积分就是黎曼积分在这两个不同角度上的合理推广.

无穷积分.如果函数$f$在区间$[a,+\infty)$上有定义,并且在任意有界区间$[a,u]$上可积,把极限式$\lim_{u\to\infty}\int_a^uf(x)\mathrm{d}x$记作$\int_a^{+\infty}f(x)\mathrm{d}x$,如果这个极限存在(有限实数),就称$\int_a^{+\infty}f(x)\mathrm{d}x$收敛,并且把这个数值称为$f$在$[a,+\infty)$上的无穷积分.如果不存在,就称$\int_a^{+\infty}f(x)\mathrm{d}x$是发散的.

瑕积分.设$f$在$(a,b]$上有定义,在$a$的任一右开邻域上无界,但在任意内闭区间$[v,b]$上黎曼可积,这时候称$a$为函数$f$的瑕点.把极限式$\lim_{v\to a^+}\int_v^bf(x)\mathrm{d}x$记作$\int_a^bf(x)\mathrm{d}x$,如果极限存在(有限实数),就称$\int_a^bf(x)\mathrm{d}x$收敛,并且把这个数值称为$f$在$(a,b]$上的瑕积分.如果不存在就称$\int_a^bf(x)\mathrm{d}x$是发散的.

瑕积分和广义积分统称为反常积分.按照极限存在的定义,瑕积分和广义积分的可积性只依赖于瑕点或者无穷远点的开邻域上函数的性质.严格说就是,无论$w$是无穷大还是$f$的瑕点,设$f$在$[a,w)$上有定义,并且在任意的内闭子区间$[a,u]$上黎曼可积,那么对任意的$b\in[a,w)$,有$\int_a^wf(x)\mathrm{d}x$和$\int_b^wf(x)\mathrm{d}x$的敛散性一致.另外,对于瑕点出现在闭区间内部的情况,即如果$f$在$[a,c)\cup(c,b]$上有定义,并且$c$的任意单侧去心开邻域内$f$都无界,这时候当写下式子$\int_a^bf(x)\mathrm{d}x$时,它表示的是两个极限式$\lim_{u\to c^-}\int_a^uf(x)\mathrm{d}x$和$\lim_{v\to c^+}\int_v^bf(x)\mathrm{d}x$的和.类似的,可以定义符号$\int_{-\infty}^{+\infty}f(x)\mathrm{d}x$的含义.

至此,记号$\int$的定义范围扩大了,它的被积分区间可以是一个无穷区间,它的被积函数可以在有限个点的附近无界.把这种扩大定义范围的积分称为广义积分,当考虑闭区间$[a,b]$时,对于有些函数$f(x)$,积分$\int_a^bf(x)\mathrm{d}x$不含有瑕点,此时积分的含义就是常义的,但是对于另一些函数,$f(x)$可能在$[a,b]$上存在有限个瑕点,此时就要按照上一段把它理解为若干瑕积分的和.

按照反常积分就是常义积分的极限,容易得到一些基本性质,这里约定$b$是函数的瑕点或者$+\infty$:
\begin{enumerate}
	\item 如果$f,g$都在$[a,b)$上广义可积,那么它们的线性组合$\lambda_1f+\lambda_2g$也在$[a,b)$上广义可积,并且满足:
	$$\int_a^b\left(\lambda_1f(x)+\lambda_2g(x)\right)\mathrm{d}x=\lambda_1\int_a^bf(x)\mathrm{d}x+\lambda_2\int_a^bg(x)\mathrm{d}x$$
	\item 换元积分.如果$\phi:[\alpha,\beta)\to[a,b)$是严格递增的光滑单调函数,并且$\phi(\alpha)=a$,$\lim_{x\to\beta}\phi(x)=b$.那么$f\circ\phi(t)\phi'(t)$在$[\alpha,\beta)$上反常积分收敛,并且满足:
	$$\int_a^bf(x)\mathrm{d}x=\int_{\alpha}^{\beta}f\circ\phi(t)\cdot\phi'(t)\mathrm{d}t$$
	\item 分部积分.如果$f,g$在$[a,b)$上连续可导,并且极限$\lim_{x\to b^-}f(x)g(x)$存在,那么$f(x)g'(x)$和$f'(x)g(x)$同时可积也同时不可积,并且在可积的情况下满足:
	$$\int_a^bf(x)g'(x)\mathrm{d}x=\lim_{x\to b^-}f(x)g(x)-f(a)g(a)-\int_a^bf'(x)g(x)\mathrm{d}x$$
\end{enumerate}

接下来讨论反常积分的敛散性准则.首先既然是实数的极限问题,自然有柯西收敛准则.
\begin{enumerate}
	\item 无穷积分情况.设$f(x)$在$[a,+\infty)$上有定义,并且内闭黎曼可积(即在每个内闭子区间上黎曼可积),那么无穷积分$\int_a^{+\infty}f(x)\mathrm{d}x$收敛当且仅当,对任意的$\varepsilon>0$,存在$\delta>0$,使得只要$u_1,U_2>\delta$,就有:
	$$\left|\int_{u_1}^{u_2}f(x)\mathrm{d}x\right|<\varepsilon$$
	\item 瑕积分情况.设$f(x)$在$[a,b)$上有定义,在$b$的每个左侧的去心开邻域上无界,并且内闭黎曼可积,那么无穷积分$\int_a^bf(x)\mathrm{d}x$收敛当且仅当,对于任意的$\varepsilon>0$,存在$\delta>0$,使得只要$u_1,u_2$满足$b-u_i<\delta$,就有:
	$$\left|\int_{u_1}^{u_2}f(x)\mathrm{d}x\right|<\varepsilon$$
\end{enumerate}

反常积分与原函数.
\begin{enumerate}
	\item 无穷积分的情况.如果$f(x)$在$[a,+\infty)$上有原函数$F(x)$,并且$f$在$[a,+\infty)$上内闭黎曼可积,那么$f$在$[a,+\infty)$上反常积分收敛当且仅当,$F(+\infty)=\lim_{x\to+\infty}F(x)$存在并有限,并且此时积分为$F(+\infty)-F(a)$.
	\item 瑕积分的情况.如果$f(x)$在$[a,b)$上有原函数$F(x)$,并且$f$在$[a,b)$上内闭黎曼可积,那么$f$在$[a,b)$上反常积分收敛当且仅当$F(b-)=\lim_{x\to b^-}F(x)$存在并有限,并且此时积分为$F(b-)-F(a)$.
\end{enumerate}

绝对收敛与条件收敛.给定$[a,b)$上的反常积分$\int_a^bf(x)\mathrm{d}x$,如果绝对值函数$\int_a^b|f(x)|\mathrm{d}x$收敛,那么称原反常积分绝对收敛.按照柯西收敛准则以及绝对值不等式,有$f$在$[a,b)$上条件收敛必然导致$f$在$[a,b)$上反常积分收敛.如果上述绝对值函数的反常积分发散,但是原反常积分收敛,就称$f$在$[a,b)$上是条件收敛的.

按照柯西收敛准则可以得到如下比较判别法.
\begin{enumerate}
	\item 无穷积分的标准形式.设$f,g$是定义在$[a,+\infty)$上的非负函数,并且内闭黎曼可积,满足$f(x)\le g(x),x\in[a,+\infty)$,那么$\int_a^{+\infty}g(x)\mathrm{d}x$收敛则$\int_a^{+\infty}f(x)\mathrm{d}x$收敛;$\int_a^{+\infty}f(x)\mathrm{d}x$发散则$\int_a^{+\infty}g(x)\mathrm{d}x$发散.
	\item 无穷积分的极限形式.设$f,g$是定义在$[a,+\infty)$上的非负函数,并且$g(x)$不取0,如果存在极限$\lim_{x\to+\infty}\frac{f(x)}{g(x)}=c$,那么如果$c$是非0有限正数,则两个广义积分$\int_a^{+\infty}f(x)\mathrm{d}x$和$\int_a^{+\infty}g(x)\mathrm{d}x$同敛散性;如果$c=0$,那么$\int_a^{+\infty}g(x)\mathrm{d}x$收敛蕴含$\int_a^{+\infty}f(x)\mathrm{d}x$收敛;如果$c=+\infty$那么$\int_a^{+\infty}g(x)\mathrm{d}x$发散蕴含$\int_a^{+\infty}f(x)\mathrm{d}x$发散.
	\item 瑕积分的标准形式.设$f,g$是定义在$(a,b]$上的两个非负函数,都以$a$为瑕点,并且都在$(a,b]$上内闭黎曼可积,并且满足$f(x)\le g(x)$.那么如果$\int_a^bg(x)\mathrm{d}x$收敛,则$\int_a^bf(x)\mathrm{d}x$收敛;如果$\int_a^bf(x)\mathrm{d}x$发散,那么$\int_a^bg(x)\mathrm{d}x$发散.
	\item 瑕积分的极限形式.条件同上,并且约定$g(x)$不取0,那么如果$\lim_{x\to a^+}\frac{f(x)}{g(x)}=c$,当$c$是有限正实数时,$\int_a^bf(x)\mathrm{d}x$和$\int_a^bg(x)\mathrm{d}x$同敛散性;当$c=0$时,$\int_a^bg(x)\mathrm{d}x$收敛蕴含$\int_a^bf(x)\mathrm{d}x$收敛;当$c=+\infty$时,$\int_a^bg(x)\mathrm{d}x$发散蕴含$\int_a^bf(x)\mathrm{d}x$发散.
\end{enumerate}

无穷积分的Abel-Dirichlet判别法.设两个函数$f,g$定义在$[a,+\infty)$上并且内闭黎曼可积
\begin{enumerate}
	\item Dirichlet判别法.如果$g$在$[a,+\infty)$上单调趋于0,并且$F(u)=\int_a^uf(x)\mathrm{d}x$在$[a,+\infty)$上有界,那么$\int_a^{+\infty}f(x)g(x)\mathrm{d}x$收敛.
	\item Abel判别法.如果$g$在$[a,+\infty)$上单调,并且$\int_a^{+\infty}f(x)\mathrm{d}x$收敛,那么$\int_a^{+\infty}f(x)g(x)\mathrm{d}x$收敛.
\end{enumerate}
\begin{proof}
	
	这个证明依赖于第二积分中值定理.证明Dirichlet判别法,Abel判别法是类似的.设正数$M$满足$\left|\int_a^uf(x)\mathrm{d}x\right|<M,\forall u>a$.对任意$\varepsilon>0$,存在$\delta>\max\{a,0\}$,满足只要$u>\delta$,就有$|g(u)|<\frac{\varepsilon}{4M}$.于是按照第二积分中值定理,当$u_1,u_2>\delta$时,存在介于$x_1,x_2$的$\xi$满足:
	\begin{align*}
	\left|\int_{u_1}^{u_2}f(x)g(x)\mathrm{d}x\right| &=\left|g(u_1)\int_{u_1}^{\xi}f(x)\mathrm{d}x+g(u_2)\int_{\xi}^{u_2}f(x)\mathrm{d}x\right| \\
	&\le|g(u_1)|\cdot\left|\int_{u_1}^{\xi}f(x)\mathrm{d}x\right|+|g(u_2)|\cdot\left|\int_{\xi}^{u_2}f(x)\mathrm{d}x\right| \\
	&<\frac{\varepsilon}{4M}\cdot2M+\frac{\varepsilon}{4M}\cdot2M=\varepsilon
	\end{align*}
\end{proof}

现在来讨论高维情况.给定$\mathbb{R}^m$中的一个点集$E$(没有要求它有界),称它的一个exhaustion为一列Jordan可测子集$\{E_n\}$,满足递增性,即$E_n\subset E_{n+1}\subset E,\forall n\ge1$,并且有$\cup_{n\ge1}E_n=E$.

那么首先,用exhaustion逼近Jordan可测集是保Jordan测度和积分的,即,如果$\{E_n\}$是Jordan可测集$E$的exhaustion,那么:
\begin{enumerate}
	\item $\lim_{n\to+\infty}\mu(E_n)=\mu(E)$.
	\item 对任意的$E$上黎曼可积函数$f$,有$f$在$E_n$上的限制也是黎曼可积的,并且有:
	$$\lim_{n\to+\infty}\int_{E_n}f(x)\mathrm{d}x=\int_Ef(x)\mathrm{d}x$$
\end{enumerate}
\begin{proof}
	
	一方面按照测度的单调性,得到$\mu(E_n)\le\mu(E_{n+1})\le\mu(E)$.按照单调有界定理得到$\lim_{n\to+\infty}\mu(E_n)$存在,并且极限$A$满足$A\le\mu(E)$.为了证明$A\ge\mu(E)$,按照$E$是Jordan可测集,它的边界$\partial E$是Jordan零测集,于是对任意的$\varepsilon>0$,存在有限个开矩体的覆盖,满足这些开矩体的体积和小于$\varepsilon$.记这些开矩体的并为$U$,那么$U\cup E$是一个开集,并且满足$\mu(E\cup U)<\mu(E)+\varepsilon$.现在,对任意的$E_n$,按照刚刚的构造,存在开集$U_n$覆盖了$E_n$的边界,并且满足$\mu(E_n\cup U_n)<\mu(E_n)+\frac{\varepsilon}{2^n}$.注意$E_n\cup U_n$是开集,并且$\{U,E_n\cup U_n\}$构成了紧集$\overline{E}$的一个开覆盖,于是存在有限子覆盖记作$\{U,E_{i_1}\cup U_{i_1},E_{i_2}\cup U_{i_2},\cdots,E_{i_s}\cup U_{i_s},i_1<i_2<\cdots<i_s\}$,按照$E_i$是递增的,得到$\{U,U_{i_1},U_{i_2},\cdots,U_{i_s},E_{i_s}\}$覆盖了整个$\overline{E}$,于是得到$\mu(E)\le\mu(\overline{E})\le\mu(E_{i_s})+\mu(U)+\mu(U_1)+\cdots+\mu(U_{i_s})<\mu(E_{i_s})+2\varepsilon$.再由$\varepsilon$的任意性就完成证明.
	
	首先按照勒贝格准则得到$f$在$E_n$上黎曼可积.设$f$在$E$上满足$|f(x)|\le M$,那么就有:
	$$\left|\int_ {E_n}f(x)\mathrm{d}x-\int_Ef(x)\mathrm{d}x\right|\le M\mu(E\setminus E_n)\to0$$
	
\end{proof}

给定集合$E\subset\mathbb{R}^m$,给定$E$上有定义的实值函数$f$,要求$f$在$E$的每个Jordan可测子集上黎曼可积 ,如果对任意的$E$的exhaustion$\{E_n\}$,有$\int_{E_n}f(x)\mathrm{d}x$极限存在,并且极限值不依赖于exhaustion的选取,就称$f$在集合$E$上广义可积,积分值为这个极限.

按照exhaustion逼近Jordan可测集的结论,如果定义里的$E$本身取了Jordan可测集,并且$f$在$E$上黎曼可积,那么实际上上述定义和常义黎曼可积是吻合的,黎曼积分值就是广义积分值.因而,广义积分放宽了黎曼可积对积分点集和对积分函数的要求,扩大了可积的情况.

对于一般的点集$E$,它的全体exhaustion是非常多的,自然不打算验证全部情况都收敛到同一个数.对于非负函数$f:E\subset\mathbb{R}^m\to\mathbb{R}$,如果对$E$的每个exhaustion$\{E_n\}$,都有$\int_{E_n}f(x)\mathrm{d}x$收敛,那么全体对不同的exhaustion的选取,这些极限都是相同的.
\begin{proof}
	
	给定点集$E$的任意两个exhaustion,记作$\{E_n\}$和$\{E_n'\}$.记$E_{k,n}=E_k'\cup E_n,\forall k,n\ge1$,那么当固定$k$的时候,$\{E_{k,n},n\ge1\}$构成了$E_k'$的一个exhaustion,于是有:
	$$\int_{E_k'}f(x)\mathrm{d}x=\lim_{n\to+\infty}\int_{E_{k,n}}f(x)\mathrm{d}x\le\lim_{n\to+\infty}\int_{E_n}f(x)\mathrm{d}x$$
	
	结合$\{E_k'\}$是递增的点集列,得到$\lim_{k\to+\infty}\int_{E_k'}f(x)\mathrm{d}x$存在并且满足:
	$$\lim_ {k\to+\infty}\int_{E_k'}f(x)\mathrm{d}x\le\lim_ {n\to+\infty}\int_{E_n}f(x)\mathrm{d}x$$
	
	类似得到另一侧的不等式,完成证明.
	
\end{proof}

非负函数的广义积分.给定点集$E$上的非负函数$f$,那么$f$在$E$上广义可积当且仅当,对$E$的全体Jordan可测子集,$f$在其上可积,并且积分值对于全部Jordan可测子集的选取,有上界.



广义积分的比较准则.给定点集$E$上的两个实值函数$f,g$满足$|f(x)|\le g(x)$,并且都在$E$的Jordan可测子集上可积,如果广义积分$\int_Eg(x)\mathrm{d}x$收敛,那么广义积分$\int_Ef(x)\mathrm{d}x$是绝对收敛的.
\begin{proof}
	
	按照勒贝格准则得到$|f(x)|$在$E$的每个Jordan可测子集上黎曼可积.据此得到:
	$$\int_{E_{n+k}\setminus E_n}|f(x)|\mathrm{d}x\le\int_{E_{n+k}\setminus E_n}g(x)\mathrm{d}x$$
	
	按照柯西收敛准则,就得到$\int_{E_n}|f(x)|\mathrm{d}x$是收敛的.
	
\end{proof}

绝对收敛蕴含收敛.如果实值函数$f$定义在点集$E$上,并且对每个可测子集,$f$限制其上都是黎曼可积的,如果还满足$|f(x)|$在$E$上广义可积,那么$f(x)$在$E$上也是广义可积的.为此,记$f^+=\frac{|f|+f}{2}$,$f^-=\frac{|f|-f}{2}$,那么$f^+-f^-=f,F^++f^-=|f|$,现在注意到$f^+$和$f^-$都是非负函数,并且满足$f^+\le|f|$和$f^-\le|f|$,于是按照上述定理,得到$f^+$和$f^-$都是$E$上广义可积的,这就保证了$f=f^+-f^-$也是在$E$上广义可积的.

最后,广义积分的换元积分公式和正常版本的是一样的,因为它只是常义积分取了极限,这里不再赘述.
\newpage
\section{级数论}
\subsection{数列项级数}

给定一实数列$\{a_n\}$,定义它的第$n$个部分和为$S_n=\sum_{k=1}^{n}a_k$,注意如果这个实数列的指标从0开始,那么上述求和的定义也是从0开始的.称极限$\lim_{n\to+\infty}S_n$为无穷级数,记作$\sum_{n\ge1}a_n$.如果$S_n$作为数列是收敛的,就称对应的无穷级数$\sum_{n=1}^{+\infty}a_n$是收敛的,并记$\sum_{n=1}^{+\infty}=\lim_{n\to+\infty}S_n$.如果$\lim_{n\to+\infty}S_n$不存在,就称对应的无穷级数是发散的.

级数和序列实际上是同一个东西的两种不同表述.对于数列$\{a_n\}$,它对应的级数是$\sum_{n=1}^{+\infty}\left(a_{n+1}-a_n\right)$,这个数列和对应的级数同敛散,并且收敛时收敛值是相同的.对于级数$\sum_{n=1}^{+\infty}a_n$,它对应的数列就是部分和数列$\{S_n\}$,级数和所对应的数列同敛散,并且收敛时收敛值相同.于是序列收敛的命题都存在针对级数的对偶命题.

级数收敛的柯西准则.给定实级数$\sum_{n=1}^{+\infty}a_n$,那么级数收敛当且仅当,对任意的$\varepsilon>0$,存在正整数$N$,使得只要$m>n>N$,就有$\left|\sum_{k=n+1}^{m}\right|<\varepsilon$.

级数收敛的一个必要条件.级数$\sum_{n=1}^{+\infty}a_n$如果收敛,那么就是说部分和数列收敛,于是$a_n=S_{n}-S_{n-1}$收敛于0,即级数收敛则通项$a_n$收敛于0.

级数收敛的简单性质.收敛级数做加减法做数乘都是收敛级数,即如果$\sum_{n=1}^{+\infty}a_n$和$\sum_{n=1}^{+\infty}b_n$都收敛,那么级数$\sum_{n=1}^{+\infty}\left(a_n\pm b_n\right)$和$\sum_{n=1}^{+\infty}ca_n$都是收敛级数,并且如果记$\sum_{n=1}^{+\infty}a_n=a$,$\sum_{n=1}^{+\infty}b_n=b$,那么$\sum_{n=1}^{+\infty}\left(a_n\pm b_n\right)=a\pm b$,$\sum_{n=1}^{+\infty}ca_n=ca$.另外,收敛级数去掉或加上或改变有限项,不改变级数的敛散性,这只要注意到柯西准则并不受前有限项的影响.

关于级数的项归组成为新级数.一个收敛级数$\sum_{n=1}^{+\infty}a_n$,如果把级数的项任意归组,但不改变先后顺序,即约定$b_1=a_1+\cdots+a_{n_1}$,$b_2=a_{n_1+1}+\cdots+a_{n_2},\cdots$,那么$\sum_{n=1}^{+\infty}b_n$是收敛的.但是注意这个逆命题是不成立的,例如$\sum_{n=1}^{+\infty}(-1)^{n-1}$是一个发散级数,因为它的部分和序列是$\{1,0,1,0,\cdots\}$,并不收敛.但是如果把项两两归组成为新的级数,得到的是$\sum_{n=1}^{+\infty}(1-1)$收敛于0.尽管如此,如果加上一个条件,这个逆命题是成立的:如果级数$\sum_{n=1}^{+\infty}a_n$按照上述定义的$b_n$把项归组,如果每个$b_n$中的原始项的符号是相同的(0视为任意符号),那么$\sum_{n=1}^{+\infty}b_n$收敛得到$\sum_{n=1}^{+\infty}a_n$收敛.
\begin{proof}
	
	记$\{b_n\}$的部分和为$S_n'$,设$S_n'$的极限为$A$,于是对任意的$\varepsilon>0$,存在$N_1$使得$k>N_1$的时候,$\left|S_n'-A\right|<\frac{\varepsilon}{2}$.并且按照$\{S_n'\}$是柯西列,存在$N_2$使得$k_1,k_2>N_2$的时候,$\left|S_{k_1}'-S_{k_2}'\right|<\frac{\varepsilon}{2}$.按照条件,归组的项的符号相同或者为0,说明对每个正整数$k$,有$S_{n_k},S_{n_k+1},\cdots,S_{n_{k+1}}$是夹在$S_{n_k}=S_{k}'$和$S_{n_{k+1}}=S_{k+1}'$之间的.于是,如果$m$满足$m>n_N$,其中$N=\max\{N_1,N_2\}$,那么可设$m$夹在严格递增的正整数列$\{n_1,n_2,\cdots\}$的$n_{t}$和$n_{t+1}$之间,那么有:
	$$\left|S_m-A\right|\le\left|S_m-S_t'\right|+\left|S_t'-A\right|\le\left|S_{t+1}'-S_t'\right|+\left|S_t'-A\right|<\varepsilon$$
	
\end{proof}

正项级数.如果一个无穷级数的项都是非负的,就称它是正项级数.那么正项级数对应的部分和数列是一个递增数列,于是,正项级数收敛当且仅当,它的部分和数列是有界的.接下来我们给出正项级数的一些收敛判别法.

比较判别法.设$\sum_{n=1}^{+\infty}a_n$和$\sum_{n=1}^{+\infty}b_n$是两个正项级数.
\begin{enumerate}
	\item 比较判别法的一般形式.如果存在正整数$N$使得$n>N$的时候总有$a_n\le b_n$,那么$\sum b_n$收敛推出$\sum a_n$收敛;$\sum a_n$发散推出$\sum b_n$发散.
	\item 比较判别法的极限形式.如果存在极限$\lim_{n\to+\infty}\frac{a_n}{b_n}=l$.那么如果$l$是非0实数,则$\sum a_n$和$\sum b_n$同敛散性;如果$l=0$,那么$\sum b_n$收敛推出$\sum a_n$收敛;如果$l=+\infty$,那么$\sum a_n$发散推出$\sum b_n$发散.
\end{enumerate}

如果把比较叛变法应用到一些常见的级数,几何级数$\sum q^n$,以及收敛得更慢的两个级数$\sum\frac{1}{n^{a}},a>1$,$\sum_{n\ge2}\frac{1}{n\ln^an},a>1$,就依次得到如下几种特殊的判别法.

Cauchy判别法.设$\sum_{n=1}^{+\infty}a_n$是正项级数.
\begin{enumerate}
	\item 一般形式.如果存在$q\in[0,1)$满足存在某个正整数$N$使得$n>N$时候总有$\sqrt[n]{a_n}\le q$.那么就有$\sum_{n\ge1}a_n$收敛.
	\item 极限形式.设$\limsup_{n\to+\infty}\sqrt[n]{a_n}=q$,如果$0\le q<1$,那么$\sum_{n\ge1}a_n$收敛;如果$q>1$,则$\sum_{n\ge1}a_n$发散.
\end{enumerate}

达朗贝尔(d'Alembert)判别法.设$\sum_{n\ge1}a_n$是严格的正项级数,即不存在项为0.
\begin{enumerate}
	\item 一般形式.如果存在$q\in(0,1)$满足存在某个正整数$N$使得$n>N$的时候总有$\frac{a_{n+1}}{a_n}\le q$,那么$\sum_{n\ge1}a_n$收敛;如果$\frac{a_{n+1}}{a_n}\ge1$,那么自然$\sum_{n\ge1}a_n$发散.
	\item 极限形式.如果$\limsup_{n\to+\infty}\frac{a_{n+1}}{a_n}=q\in(0,1)$,那么$\sum_{n\ge1}a_n$是收敛的;如果$\liminf_{n\to+\infty}\frac{a_{n+1}}{a_n}=q'>1$,那么$\sum_{n\ge1}a_n$是发散的.
\end{enumerate}

事实上柯西判别法是强于达朗贝尔判别法的.因为有如下结论:对于正项数列$\{a_n\}$,有$$\liminf_{n\to+\infty}\frac{a_{n+1}}{a_n}\le\liminf_{n\to+\infty}\sqrt[n]{a_n}\le\limsup_{n\to+\infty}\sqrt[n]{a_n}\le\limsup_{n\to+\infty}\frac{a_{n+1}}{a_n}$$

Raabe判别法可以看作是达朗贝尔判别法的延续.按照达朗贝尔判别法,如果$\lim_{n\to+\infty}\frac{a_{n+1}}{a_n}=1$那么无法判断严格正项级数$\sum_{n\ge1}a_n$的敛散性.在这种情况下就有Raabe判别法作为补充.给定严格正项级数$\sum_{n\ge1}a_n$.
\begin{enumerate}
	\item 一般形式.如果存在$r>1$,存在正整数$N$,使得$n>N$的时候总有$n\left(\frac{a_{n}}{a_{n+1}}-1\right)\ge r$.那么$\sum_{n\ge1}a_n$收敛;如果存在正整数$N$使得$n>N$时候始终有$n\left(\frac{a_n}{a_{n+1}}-1\right)\le1$,那么$\sum_{n\ge1}a_n$发散.
	\item 极限形式.如果$\liminf_{n\to+\infty}n\left(\frac{a_n}{a_{n+1}}-1\right)=l>1$,则$\sum_{n\ge1}a_n$收敛;如果$\limsup_{n\to+\infty}n\left(\frac{a_n}{a_{n+1}}-1\right)=l'<1$,则$\sum_{n\ge1}a_n$发散.
\end{enumerate}
\begin{proof}
	
	只证明一般形式,极限形式可以归结到一般形式.取定$a\in(1,r)$,那么有$\lim_{n\to+\infty}\frac{\left(1+\frac{1}{n}\right)^a-1}{\frac{1}{n}}=a<r$,于是存在正整数$N$使得$n>N$的时候总有$\left(1+\frac{1}{n}\right)^a-1<\frac{r}{n}$.结合条件,存在一个正整数$N'$使得$n>N'$的时候总有$\frac{a_n}{a_{n+1}}\ge1+\frac{r}{n}>(1+\frac{1}{n})^a$,这得到$n^aa_n>(n+1)^aa_n$,于是$\{n^aa_n\}_{n\ge1}$是一个单调递减的正项级数,于是它上有界$M$,导致$a_n<\frac{M}{n^a}$,按照$a>1$得到$\sum_{n\ge1}\frac{1}{n^a}$收敛,从比较判别法得到$\sum_{n\ge1}a_n$收敛.
	
	如果存在$N$使得$n>N$的时候$n\left(\frac{a_n}{a_{n+1}}-1\right)\le1$,也就是$n>N$的时候有$na_n\le(n+1)a_{n+1}$,于是$n>N$的时候$a_n\ge\frac{Na_{N}}{n}$,按照$\sum\frac{1}{n}$发散就说明$\sum_{n\ge1}a_n$发散.
	
\end{proof}

Gauss判别法.Gauss判别法可以理解为Raabe判别法的延续,设$\sum_{n\ge1}a_n$为严格正项级数,满足$\frac{a_n}{a_{n+1}}=1+\frac{1}{n}+\frac{\delta}{n\ln n}+o(\frac{1}{n\ln n}),n\to+\infty$,如果$\delta>1$,则$\sum_{n\ge1}a_n$收敛;如果$\delta<1$,则$\sum_{n\ge1}a_n$发散;在$\delta=1$的时候无法判断.

现在开始讨论通项可正可负的一般级数.首先对于符号交替的一般级数有Leibniz判别法:设$\{a_n\}$是单调递减趋于0的数列,那么交错级数$\sum_{n\ge1}(-1)^{n-1}a_n$是收敛的.
\begin{proof}
	
	设部分和序列$\{S_n\}$,按照$a_n$单调递减,得到$S_{2k}=S_{2k-2}+(a_{2k-1}-a_{2k})\ge S_{2k-2}$,同理得到$S_{2k+1}\le S_{2k-1}$.再注意到$S_{2k+1}=S_{2k}+a_{2k+1}\ge S_{2k}$,于是得到部分和序列是偶数列递增,奇数列递减的序列:
	$$S_2\le S_4\le\cdots\le S_{2k}\le S_{2k+1}\le S_{2k-1}\le\cdots\le S_1$$
	
	于是按照单调有界定理,$S_n$的奇偶序列是分别收敛的,最后按照$a_n\to0$和$S_{2k+1}=S_{2k}+a_{2k+1}$,得到奇偶序列的极限相同,这就说明$\{S_n\}$是收敛的.
	
\end{proof}

在给出另外两个判别法之前,先来介绍几个引理.
\begin{enumerate}
	\item 记$S_k=\sum_{i=1}^{k}a_i$,那么有$\sum_{k=1}^{n}a_kb_k=\sum_{k=1}^{n-1}S_k(b_k-b_{k+1})+S_nb_n$.
	\item 若$b_1\ge b_2\ge\cdots\ge b_n\ge0$,并且$m\le S_k\le M,k=1,2,\cdots,n$,那么有$b_1m\le\sum_{k=1}^{n}a_kb_k\le b_1M$.
	\item 若$\{b_k\}$是单调的,并且$|S_k|$有上界$M$,那么$\sum_{k=1}^{n}a_kb_k\le M\left(|b_1|+2|b_n|\right)$.
\end{enumerate}

\begin{enumerate}
	\item Dirichlet判别法.设$\{b_n\}$单调趋于0,设$\{a_n\}$的部分和序列有界$|S_n|\le M,n\ge1$,那么$\sum_{n\ge1}a_nb_n$收敛.
	\item Abel判别法.设$\{b_n\}$单调有界,设$\sum_{n\ge1}a_n$收敛,那么$\sum_{n\ge1}a_nb_n$是收敛的.注意Abel判别法可以由Dirichlet判别法得出.
\end{enumerate}

绝对收敛与条件收敛.给定一般级数$\sum_{n\ge1}a_n$,如果对应的绝对值级数$\sum_{n\ge1}|a_n|$是收敛的,按照Cauchy准则容易得到$\sum_{n\ge1}a_n$是收敛的.如果$\sum_{n\ge1}|a_n|$收敛,就称$\sum_{n\ge1}a_n$是绝对收敛的.如果$\sum_{n\ge1}a_n$收敛但$\sum_{n\ge1}|a_n|$发散,就称$\sum_{n\ge1}a_n$是条件收敛的.

这里给出绝对值级数的等价描述.给定一般级数$\sum_{n\ge1}a_n$,记$a_n^+=\max\{a_n,0\}$,$a_n^-=\max\{-a_n,0\}$,那么有$a_n^+-a_n^-=a_n$,并且$a_n^++a_n^-=|a_n|$.于是$\sum_{n\ge1}a_n^+$和$\sum_{n\ge1}a_n^-$是两个正项级数.如果$\sum a_n$绝对收敛,那么按照$a_n^+\le |a_n|$和$a_n^-\le a_n$,说明这两个正项级数是收敛的.反过来按照$\sum |a_n|=\sum a_n^+-\sum a_n^-$说明如果这两个正项级数收敛则级数绝对收敛.于是证明了,$\sum a_n$绝对收敛当且仅当,$\sum a_n^+$和$\sum a_n^-$都是收敛的级数.现在假设$\sum a_n$是条件收敛的,那么$\sum |a_n|$发散,于是$\sum a_n^+$和$\sum a_n^-$至少有一个是趋于无穷的.事实上可以说明它们两个都是趋于无穷的,因为如果只有一个趋于无穷,那么$\sum a_n=\sum a_n^+-\sum a_n^-$也是趋于某个无穷的,这矛盾.但是注意,从$\sum a_n^+$和$\sum a_n^-$都趋于无穷并不能推出$\sum a_n$是条件收敛的,例如$\sum (-1)^n$.

绝对收敛级数的特殊性质.绝对收敛级数任意交换项的次序所得的新级数仍然绝对收敛,并且收敛和不变.严格的说,给定绝对收敛级数$\sum_{n\ge1}a_n$,设$\mathbb{Z}^+$自身的一个双射是$\sigma$,记$b_n=a_{\sigma(n)}$,那么新的级数$\sum_{n\ge1}b_n$是绝对收敛的,并且$\sum_{n\ge1}b_n=\sum_{n\ge1}a_n$.
\begin{proof}
	
	首先假设$\sum_{n\ge1}a_n$是正项级数.设$\{a_n\}$的部分和数列是$\{S_n\}$,设$\{b_n\}$的部分和数列是$S_n'$,那么按照$S_n'\le S_{\sigma(n)}$,以及$\lim_{n\to+\infty}\sigma(n)=+\infty$,得到$S_n'\le\sum_{n\ge1}a_n<+\infty$,于是$\sum_{n\ge1}b_n$收敛.另外$\sum_{n\ge1}a_n$也可看作$\sum_{n\ge1}b_n$交换次序得到的级数,于是$S_n\le\sum_{n\ge1}b_n$,于是二者收敛和相同.
	
	现在设$\sum_{n\ge1}a_n$是一般级数.那么有$\sum_{n\ge1}a_n=\sum_{n\ge1}a_n^+-\sum_{n\ge1}a_n^-$,现在设$\sum a_n$交换次序得到$\sum b_n$,于是$\sum b_n^+$和$\sum b_n^-$分别是由$\sum a_n^+$和$\sum a_n^-$交换次序得到的,于是按照上一段得到$\sum_{n\ge1}a_n^+=\sum_{n\ge1}b_n^+$,$\sum_{n\ge1}b_n^-=\sum_{n\ge1}a_n^-$,于是得到$\sum b_n$绝对收敛,并且收敛和与交换次序前相同.
	
\end{proof}

条件收敛级数的特殊性质.设$\sum a_n$是条件收敛级数,那么适当交换项的次序,可以使它收敛于任一预先设定实数,也可以使它发散.
\begin{proof}
	
	首先设预先设定的实数为$S$,如果$S$是无穷大证明是类似的.按照$\sum a_n$条件收敛,得到$\sum a_n^+$和$\sum a_n^-$都是趋于无穷的级数.于是,可取正整数$n_1$满足:
	$$a_1^++a_2^++\cdots+a_{n_1-1}^+\le S<a_1^++a_2^++\cdots+a_{n_1}$$
	$$0<(a_1^++a_2^++\cdots+a_{n_1}^+)-S\le a_{n_1}^+$$
	
	再取$n_1'$满足:
	$$a_1^++\cdots+a_{n_1}^+-a_1^--\cdots-a_{n_1'}^-<S\le
	a_1^++\cdots+a_{n_1}^+-a_1^--\cdots-a_{n_1'-1}^-$$
	
	继续取$n_2,n_2',n_3,n_3',\cdots$,得到对每个正整数$l$有:
	$$0<\left(\sum_{k=1}^{n_1}a_k^+-\sum_{k=1}^{n_1'}a_k^-+\cdots+\sum_{k=n_{l-1}+1}^{n_l}\right)-S\le a_{n_l}^+$$
	$$0<S-\left(\sum_{k=1}^{n_1}a_k^+-\sum_{k=1}^{n_1'}a_k^-+\cdots-\sum_{k=n_{l-1}'+1}^{n_l'}\right)\le a_{n_l'}^-$$
	
	按照$\lim_{n\to+\infty}a_n=0$得到$\lim_{n\to+\infty}a_n^+=\lim_{n\to+\infty}a_n^-=0$.于是得到如下级数和为$S$,由于括号中的项具有相同符号,于是可以除去括号,并且把$a_n\not=0$但是$a_n^+$和$a_n^-$为0的那个划去,如果$a_n=0$则可以划去$a_n^+$和$a_n^-$中的任一个,这得到的级数为原级数的重排,完成证明.
	$$\left(a_1^++\cdots+a_{n_1}^+\right)-\left(a_1^-+\cdots+a_{n_1'}^-\right)+\left(a_{n_1+1}^++\cdots+a_{n_2}^+\right)-\cdots$$
	
\end{proof}

级数的乘法.给定两个收敛级数$\sum a_n=A$和$\sum b_n=B$,按照有限和情况乘积的展开,级数的乘法$\sum a_n\sum b_n$应该理解为一个右侧和下侧无穷长的二维数表的求和.但是为了要定义有意义,需要约定乘积级数的通项,这时有两种选择,一种是沿方块相加,即$d_1=a_1b_1,d_2=a_1b_2+a_2b_2+a_2b_1,\cdots$,另一种是沿对角线相加,即$c_1=a_1b_1,c_2=a_1b_2+a_2b_1$,$c_3=a_1b_3+a_2b_2+a_3b_1,\cdots$.用得比较多的是沿对角线相加,今后提及级数乘法总约定新级数是沿对角线相加.那么一个自然的问题是,何时可以有$\sum c_n=AB$?对于一般的级数,两个收敛级数的乘积级数甚至可以发散.但是如果加上一些绝对收敛条件,就能得到$\sum c_n=AB$.

\begin{enumerate}
	\item 如果$\sum a_n$和$\sum b_n$都是绝对收敛级数,其和分别为$A,B$,那么乘积级数$\sum c_n$也是绝对收敛级数,并且$\sum c_n=AB$.
	\item 如果级数$\sum a_n$绝对收敛,级数$\sum b_n$收敛,和分别记作$A,B$,那么乘积级数$\sum c_n$收敛并且收敛和为$AB$.
\end{enumerate}
\begin{proof}
	
	先证第一个.设$\sum a_n$和$\sum b_n$和$\sum c_n$的部分和序列是$\{A_n\}$和$\{B_n\}$和$\{C_n\}$.设对应的绝对值级数的部分和分别为$\{A_n'\}$和$\{B_n'\}$和$\{C_n'\}$,前两个极限分别为$A',B'$,那么有$C_n'\le A_N'B_N'\le A'B'$,于是得到$\sum c_n$是绝对收敛得到.记极限是$C$.
	
	事实上,按照$C_n'\le A'B'$,得到的是级数$a_1b_1+a_1b_2+a_2b_1+\cdots$是绝对收敛的,于是可以任意交换次序求和,于是可以把乘积级数按照方块相加求和,这和之前的收敛和相同,于是$C=AB$.
	
	现在证明第二个.同样设$\sum a_n$,$\sum b_n$,$\sum c_n$的部分和数列分别为$\{A_n\},\{B_n\},\{C_n\}$.设$\beta_n=B-\beta_n$,那么$\lim_{n\to+\infty}\beta_n=0$.那么有$C_n=A_nB-\sum_{k=1}^{n}a_k\beta_{n+1-k}$,前者趋于$AB$,于是只要证明后者趋于0.
	
	即归结为如下问题:如果$\sum a_n$绝对收敛$\lim_{n\to+\infty}b_n=0$,那么$\lim_{n\to+\infty}\sum_{k=1}^{n}a_kb_{n+1-k}=0$.为此,先按照$b_n$收敛得到它有界,设$|b_n|<M,\forall n\ge1$.对于任意的$\varepsilon>0$,存在$N_1$使得$n>N_1$的时候有$|b_n|<\frac{\varepsilon}{2A'+1}$,其中$A'=\sum |a_n|$.再按照$\sum a_n$绝对收敛,存在$N_2$使得$n>N_2$的时候有$|a_{n+1}+\cdots+a_{n+p}|<\frac{\varepsilon}{2M}$.至此,如果$n>N_1+N_2$,就有$n-N_1>N_2$,于是:
	\begin{align*}
	|a_1b_n+a_2b_{n-1}+\cdots+a_nb_1| &\le|a_1b_n|+\cdots+|a_{n-N_1}b_{N_1+1}|+|a_{n-N_1+1}b_{N_1}|+\cdots+|a_nb_1| \\
	&<(|a_1|+|a_2|+\cdots+|a_{n-N_1}|)\frac{\varepsilon}{2A'+1}+M\left(|a_{n-N_1+1}|+\cdots+|a_n|\right)  \\
	&<A'\cdot\frac{\varepsilon}{2A'+1}+M\cdot\frac{\varepsilon}{2M}<\varepsilon
	\end{align*}
\end{proof}

无穷乘积.给定一个数列$\{p_n\}$,称极限式$\lim_{n\to+\infty}\prod_{k=1}^np_k$为无穷乘积,记作$\prod_{n\ge1}p_n$.记$P_n=\prod_{k=1}^np_k$为部分积.如果部分积收敛,就称无穷乘积收敛,否则称无穷乘积发散.

无穷乘积的一个必要条件.无穷乘积$\prod_{n\ge1}p_n$收敛则$\lim_{n\to+\infty}p_n=1$.

如果约定无穷乘积的通项$p_n>0$,那么无穷乘积和无穷级数是可以互相转化的,即$\prod_{n\ge1}p_n$和$\sum_{n\ge1}\ln p_n$的敛散性相同,如果前者收敛积为$P$,那么后者收敛和为$\ln P$,$p>0$.如果$P=0$,那么$\sum_{n\ge1}\ln p_n=-\infty$.因此,往往对于$\prod_{n\ge1}p_n=0$称为无穷乘积发散于0.

现在约定无穷乘积$\prod_{n\ge1}p_n$的通项$p_n>0$,记$p_n=1+a_n$,那么$a_n>-1$,无穷乘积可以记作$\prod_{n\ge1}(1+a_n)$.这时无穷乘积收敛的必要条件是$\lim_{n\to+\infty}a_n=0$.如果恒有$a_n\ge0$或者恒有$0\ge a_n>-1$,那么无穷乘积$\prod_{n\ge1}(1+a_n)$和$\sum_{n\ge1}a_n$是同敛散性的.这只要利用比较判别法的极限形式,注意有$\lim_{n\to+\infty}\frac{\ln(1+a_n)}{a_n}=1$.如果$a_n$不保号,有如下结论:
\begin{enumerate}
	\item 如果$\sum_{n\ge1}a_n^2$收敛,那么$\prod_{n\ge1}(1+a_n)$和$\sum_{n\ge1}a_n$同敛散.
	\item 如果$\sum_{n\ge1}a_n$收敛,$\sum_{n\ge1}a_n^2=+\infty$,那么$\prod_{n\ge1}(1+a_n)$发散于0.
\end{enumerate}
\begin{proof}
	
	无论第一条还是第二条,都会得到$\lim_{n\to+\infty}a_n=0$,于是有$\lim_{n\to+\infty}\frac{a_n-\ln(1+a_n)}{a_n^2}=\frac{1}{2}$.于是,如果$\sum_{n\ge1}a_n^2$收敛得到$\sum_{n\ge1}\left(a_n-\ln(1+a_n)\right)$收敛,于是$\sum_{n\ge1}a_n$和$\sum_{n\ge1}\ln(1+a_n)$同敛散.如果$\sum_{n\ge1}a_n^2=+\infty$.按照$\sum_{n\ge1}\left(a_n-\ln(1+a_n)\right)$是正项级数,它发散于$+\infty$.再由$\sum_{n\ge1}a_n$收敛得到$\sum_{n\ge1}\ln(1+a_n)$发散于$-\infty$.于是$\prod_{n\ge1}(1+a_n)=0$.
	
\end{proof}

无穷乘积的绝对收敛与条件收敛.设$1+a_n>0$,那么$\prod_{n\ge1}(1+|a_n|)$收敛蕴含$\prod_{n\ge1}(1+a_n)$收敛.事实上,前者收敛等价$\sum_{n\ge1}\ln(1+|a_n|)$收敛,等价$\sum_{n\ge1}|a_n|$收敛,按照比较判别法这等价于$\sum_{n\ge1}|\ln(1+a_n)|$收敛,于是推出$\prod_{n\ge1}(1+a_n)$收敛.注意这个逆命题不成立.据此约定一个无穷乘积$\prod_{n\ge1}(1+a_n)$是绝对收敛的,如果$\prod_{n\ge1}(1+|a_n|)$收敛,此时原无穷乘积收敛.称无穷乘积是条件收敛的,如果$\prod_{n\ge1}(1+|a_n|)$发散,并且原无穷级数收敛.

无穷乘积的绝对收敛与条件收敛特性.这只要取对应的无穷级数归结到无穷级数的特性:通项大于$-1$的绝对收敛的无穷乘积,可以任意改变项的次序不英雄收敛性和收敛积.条件收敛的无穷乘积,可以适当交换项的次序使得它收敛到任意预先设定的正数,也可以使它发散于$+\infty$或0.
\newpage
\subsection{函数项级数}

给定一列函数$\{f_n\}_{n\ge1}$,其中所有函数有公共的定义域$X\subset\mathbb{R}^m$,称函数列在点$x_0\in X$处收敛,如果数列$\{f_n(x_0)\}$是收敛的.如果函数列在点集$X$上处处收敛,那么可以定义一个函数$f$为,$\forall x_0\in X$,定义$f(x_0)=\lim_{n\to+\infty}f_n(x_0)$,这时候称函数列在$X$上逐点收敛于$f(x)$.记作在$X$上有$f_n\to f$.

对于函数列主要关注的问题是,函数列的公共的分析性质能否传递给极限函数.即,连续函数列的极限函数是否是连续的?可导函数列的极限函数是否是可导的?可积函数列的极限函数是否是可积的?然而对于一般的函数列这三个问题的回答都是否.设$f_n=x^n$定义在$[0,1]$上,那么$f_n$的极限函数$f$为,在$[0,1)$上取0,在$x=1$处取1,于是这存在不连续点;设$f_n(x)=\frac{\sin n^2x}{n}$,那么它的极限函数是0,但是导函数列为$\{f_n'(x)=n\cos n^2x\}$一般是不收敛的,于是并不总收敛到极限函数的导数;设$f_n(x)=2(n+1)x(1-x^2)^n$,那么$f_n$在$[0,1]$上的极限函数是0,每个$f_n$的定积分$\int_0^1f_n(x)\mathrm{d}x=1$,但是极限函数在$[0,1]$上的定积分是0.

为了让函数列的公共的分析性质传递给极限函数,需要对函数列加条件,这个条件就是一致收敛性.首先,给定$X\times T$上的二元函数$(x,t)\mapsto F(x,t)$,对每个$t_0\in T$,固定它,就得到一个$X$上的函数$F(x,t_0)$,在这个表述下,称这个二元函数是参数集为$T$的$X$上的函数族.特别的,如果取$T$是正整数集,那么参数集为$T$的函数族$F(x,t)$就是函数列的定义.往往把函数族的参数写在函数的右下角,即$\{f_t:X\to\mathbb{R}\mid t\in T\}$.

当赋予参数集$T$拓扑基$\mathscr{B}$的时候,可以讨论函数族关于参数集$T$的极限函数.给定点$x_0\in X$,称函数族$\{f_t\mid t\in T\}$在点$x_0$处收敛,如果存在实数$A$,使得对任意的$\varepsilon>0$,存在拓扑基中的基元素$U$,使得对任意的$t\in U$,有$|f_t(x)-A|<\varepsilon$.在正整数集的情况下,也就是函数列的情况下,拓扑基取为$\{[n,+\infty),n\ge1\}$.于是此时极限的定义吻合于函数列的情况.如果函数族在$X$上处处收敛,于是可以定义极限函数$f:X\to\mathbb{R}$,此时我们称函数族在$X$上逐点收敛于极限函数$f$.

一致收敛性.给定函数族$\{f_t:X\to\mathbb{R}\mid t\in T\}$,赋予$T$拓扑基,如果函数族已经在$X$上逐点收敛到$f$,称极限$f_t\to f$是在$X$上一致收敛的,或者说$f_t$是在$X$上一致收敛于$f$的,如果对任意的$\varepsilon>0$,已经提前存在了一个开集$U$,使得对任意的$x\in X$,只要$t\in U$就有$|f(x)-f_t(x)|<\varepsilon$.特别的,在函数列的情况下,就是要求,对任意的$\varepsilon>0$,提前存在一个正整数$N$,满足,对任意的$x\in X$,只要$n>N$,就有$|f(x)-f_n(x)|<\varepsilon$.会把逐点收敛记作$f_t\to f$,把一致收敛记作$f_t\rightrightarrows f,x\in X$.

一致收敛的柯西准则.给定关于参变量集$t\in T$的函数族$\{f_t:X\to\mathbb{R}\}$,赋予参变量集一个拓扑基$\mathscr{B}$,那么函数族在$X$上一致收敛于$f$的充要条件是,对任意的$\varepsilon>0$,存在一个基元素$B\in\mathscr{B}$,满足对任意的$t_1,t_2\in B$,总有$|f_{t_1}(x)-f_{t_2}(x)|<\varepsilon$对任意的$x\in X$成立.
\begin{proof}
	
	一方面,如果函数族已经一致收敛,那么存在基元素$B$使得存在$X$上函数$f(x)$满足$t\in B$时有$|f_t(x)-f(x)|<\frac{\varepsilon}{2}$对任意$x\in X$成立,于是对任意的$t_1,t_2\in B$,就有$|f_{t_1}(x)-f_{t_2}(x)|\le|f_{t_1}(x)-f(x)|+|f(x)-f_{t_2}(x)|<\varepsilon$.
	
	另一方面,如果条件成立,对任意的$\varepsilon>0$,存在基元素$B$满足只要$t_1,t_2\in B$就有$|f_{t_1}(x)-f_{t_2}(x)|<\varepsilon$,于是按照柯西收敛准则,对每个$x_0\in X$,点集族$\{f_t(x_0)\}_{t\in T}$关于拓扑基$\mathscr{B}$是收敛的,于是可以定义收敛函数为$f(x):X\to\mathbb{R}$.最后在$|f_{t_1}(x)-f_{t_2}(x)|<\frac{\varepsilon}{2}$中让$t_2$取关于拓扑基的极限,就得到$f_t(x)\rightrightarrows f$.
	
\end{proof}

一致收敛的另一充要条件.函数列$\{f_n\}$在$X$上一致收敛于函数$f$的充要条件是,$\lim_{n\to+\infty}\sup_{x\in X}|f_n(x)-f(x)|=0$.

之前解释过序列和级数是同一个东西的两种不同形式,现在开始转而运用级数语言描述函数列,这就是函数项级数.给定一列$X$上的函数列$\{f_n\}$,把部分和$S_n=\sum_{k=1}^{n}f_k$的极限记作$\sum_{n\ge1}f_n(x)$,称它为函数项级数,如果部分和函数列是收敛的或者一致收敛的,就称函数项级数是收敛的或者一致收敛的.

级数一致收敛的柯西准则.给定函数项级数$\sum_{n\ge1}f_n(x)$,那么它在公共定义域的子集$X$上一致收敛当且仅当,对任意的$\varepsilon>0$,存在正整数$N$,满足只要$m\ge n>N$,就有$|f_n(x)+f_{n+1}(x)+\cdots+f_m(x)|<\varepsilon$对任意的$x\in X$成立.

级数一致收敛的一个必要条件.级数$\sum_{n\ge1}f_n(x)$一致收敛,则必然有$f_n$在相同点集上一致收敛于0.证明只要在柯西准则中取$m=n$.

级数一致收敛的优级数判别法(Weierstrass判别法).给定级数$\sum_{n\ge1}f_n(x)$和$\sum_{n\ge1}g_n(x)$,如果存在正整数$N$使得只要$n>N$,那么对任意的$x\in X$都满足$|f_n(x)|\le g_n(x)$,那么$\sum_{n\ge1}g_n(x)$一致收敛蕴含$\sum_{n\ge1}f_n(x)$是绝对收敛也是一致收敛的.特别的,如果存在收敛的正项级数$\sum_{n\ge1}a_n$,满足存在正整数$N$使得只要$n>N$就有$|f_n(x)|\le a_n$对任意的$x\in X$成立,那么$\sum_{n\ge1}f_n(x)$是一致收敛和绝对收敛的.

Abel-Dirichlet准则.首先,关于参变量$t\in T$的函数族$\{f_t:X\to\mathbb{R}\}$称为一致有界的,如果存在一个正数$M$,满足对任意的$t\in T,x\in X$有$|f(x)|\le M$.称一个函数列$\{f_n:X\to\mathbb{R}\}$是递增的或者递减的,如果对任意的$x_0\in X$,$\{f_n(x_0)\}_{n\ge1}$作为数列是递增的或者递减的.关于一致收敛的Abel-Dirichlet准则是指:
\begin{enumerate}
	\item Dirichlet准则.如果$\{g_n(x)\}_{n\ge1}$是单调的,并且一致收敛于0;$\sum_{n\ge1}f_n(x)$的部分和在$X$上一致有界,那么$\sum_{n\ge1}f_n(x)g_n(x)$在$X$上一致收敛.
	\item Abel准则.如果$\{g_n(x)\}_{n\ge1}$是单调的并且一致有界;$\sum_{n\ge1}f_n(x)$是一致收敛的,那么$\sum_{n\ge1}f_n(x)g_n(x)$在$X$上一致收敛.
\end{enumerate}
\begin{proof}
	
	首先证明一个引理,如果$\{b_n,b_{n+1},\cdots,b_m\}$是单调的数列,那么有$\left|\sum_{k=n}^ma_kb_k\right|\le 4\max_{n-1\le k\le m}|A_k|\cdot\max\{|b_n|,|b_m|\}$.
	
	于是有$\left|\sum_{k=n}^mf_n(x)g_n(x)\right|\le 4\max_{n-1\le k\le m}|f_n(x)|\cdot\max\{|g_n(x)|,|g_m(x)|\}$.对于Dirichlet准则,最后一式的前者有界,后者按照$g_n(x)\rightrightarrows0$说明趋于0;对于Abel准则,最后一式的前者按照$\sum_{n\ge1}f_n(x)$一致收敛说明趋于0,后者有界.这就完成证明.
	
\end{proof}

现在开始来讨论,在一致收敛的条件下,函数列的公共性质如何传递给极限函数.无论是连续性,可导性还是可积性,本质的讲,函数列公共的这种性质能否传递给极限函数,就是探究何时下面极限符号可交换:
$$\lim_{x\to x_0}\left(\lim_{y\to y_0}f(x,y)\right)=\lim_{y\to y_0}\left(\lim_{x\to x_0}f(x,y)\right)$$

而一致收敛条件的确可以保证这个极限符号可交换.这里要强调,这时的一致收敛性可以定义两种,第一种是说,极限$\lim_{y\to y_0}f(x,y)$,把$y$看参数,是关于$x\in X$一致收敛的,另一种是说,极限$\lim_{x\to x_0}f(x,y)$,把$x$看作参数,是关于$y\in Y$一致收敛的.而这两种情况都可以保证上述两个极限符号是可交换的.这两种一致收敛在本节不会都遇到.在含参变量积分那里会再遇到一次.以取$Y$为正整数集,$y_0$为正无穷大为例,上述等式变为:
$$\lim_{x\to x_0}\left(\lim_{n\to+\infty}f_n(x)\right)=\lim_{n\to +\infty}\left(\lim_{x\to x_0}f_n(x)\right)$$

首先记$\lim_{x\to x_0}f_n(x)$为$c_n$,那么先断言$\{c_n\}$是收敛的,为此,按照一致收敛条件,对任意的$\varepsilon>0$,存在$N$使得只要$m,n>N$,就有$|f_m(x)-f_n(x)|<\frac{\varepsilon}{2}$,让$x\to x_0$,得到$|c_n-c_m|\le\frac{\varepsilon}{2}<\varepsilon$.这就说明$\{c_n\}$是一个柯西列,于是它收敛.现在设$\lim_{n\to+\infty}c_n=c$,需要证明$\lim_{x\to x_0}f(x)=c$,为此,回顾极限定义,对任意的$\varepsilon>0$,存在正整数$N_1$使得$n>N_1$的时候有$|c_n-c|<\frac{\varepsilon}{3}$.再按照$f_n(x)\rightrightarrows f$,得到正整数$N_2$使得$n>N_2$的时候有$|f_n(x)-f(x)|<\frac{\varepsilon}{3}$对任意的$x\in X$成立,于是当$N=\max\{N_1,N_2\}+1$的时候,按照$\lim_{x\to x_0}f_N(x)=c_N$,得到一个$\delta>0$,使得只要$0<|x-x_0|<\delta$并且$x\in X$,就有$|f_N(x)-c_N|<\frac{\varepsilon}{3}$,于是得到如下不等式,就完成证明.
$$|f(x)-c|\le|f(x)-f_N(x)|+|f_N(x)-c_N|+|c_N-c|<\frac{\varepsilon}{3}\cdot3=\varepsilon$$

据此得到连续性传递给极限函数的结论.在函数列的表述下,给定函数列$\{f_n\}$一致收敛于$f$,如果存在一点$x_0\in X$使得每个$f_n$都在$x_0$处连续,那么极限函数$f$在$x_0$处连续.特别的,一致收敛的连续函数列,极限函数必然是连续函数.级数的表述下,给定一致收敛的函数项级数$\sum_{n\ge1}f_n(x)$,如果每个$f_n(x)$在点$x_0$处连续,那么和函数$S(x)$在点$x_0$处连续.特别的,如果一致收敛的函数项级数的通项都是连续函数,那么和函数是连续函数.

Dini定理.给定紧集$K$上的连续函数列$\{f_n\}$,如果它对每个点$x_0\in K$,数列$\{f_n(x_0)\}$都是单调趋于极限的,极限函数是一个连续函数$f$,那么$\{f_n(x)\}$在这个紧集$K$上一致收敛于$f$的.这个定理的级数角度描述是,如果函数项级数$\sum_{n\ge1}f_n(x)$定义在紧集$K$上,它的每个函数项$f_n(x)$都是非负函数,那么如果级数的和函数是连续函数,就有这个收敛是一致收敛的.
\begin{proof}
	
	来证明函数列描述的命题,级数角度的描述是等价的。不妨设$\{f_n\}$在每个点上单调递增趋于连续函数$f$.先固定任意的正数$\varepsilon$,对紧集$K$中任意一个点$x$,可以找到正整数$n_x$满足$0\le f(x)-f_{n_x}(x)<\varepsilon$,按照$f$和$f_{n_x}$都是连续函数,于是可以取$x$的一个开邻域$U(x)$上还满足这个不等式成立.于是紧集$K$被开集族$\{U_x\mid x\in K\}$覆盖,于是存在有限子覆盖$\{U(x_1),U(x_2),\cdots,U(x_k)\}$.下面取$n_{x_1},\cdots,n_{x_k}$中的最大元$N$,于是只要$n>N$,就有$0\le f(t)-f_n(t)<\varepsilon$对任意的$t\in K$成立.
	
\end{proof}

现在讨论可积性传递给极限函数.给定一个Jordan可测集$X$上的可积函数列$\{f_n(x)\}$,那么每个$f_n$的不连续点集$D_n$和$X$内点集的交是一个零测集,于是$\cup_{n\ge1}D_n\cap X$是一个零测集,于是$X$的内点集除了一个零测集以外全部点都是$\{f_n\}$的公共连续点,于是只要这个可积函数列是一致收敛的,就有极限函数也是可积的.期望有关系$\lim_{n\to+\infty}\int_Xf(x)\mathrm{d}x=\int_Xf(x)\mathrm{d}x$.为此设$X$的Jordan测度为$d$,那么对任意的$\varepsilon>0$,可以取正整数$N$使得$n>N$的时候有$|f_n(x)-f(x)|<\frac{\varepsilon}{d}$对任意的$x\in X$成立,于是当$n>N$时得到如下不等式,完成证明.

$$\left|\int_Xf_n(x)\mathrm{d}x-\int_Xf(x)\mathrm{d}x\right|\le\int_X|f_n(x)-f(x)|\mathrm{d}x<\varepsilon$$

转化为级数语言描述这个结论,就是逐项积分:设函数项级数$\sum_{n\ge1}f_n(x)$在紧集$K$上一致收敛于和函数$S(x)$,那么$S(x)$在$K$上可积,并且积分就是原级数逐项积分的收敛和:
$$\int_XS(x)\mathrm{d}x=\sum_{n\ge1}\int_Xf_n(x)\mathrm{d}x$$

最后讨论可导性传递给极限函数.给定有界凸集合$X$上的可导函数列$\{f_n:X\to\mathbb{R}\}$,如果它满足如下两个要求:导函数列$\{f_n'\}$是一致收敛到某个函数$g(x)$的;原本的函数列$\{f_n\}$至少在$X$上一点是收敛的,那么$\{f_n\}$在$X$上一致收敛于某个可导函数$f(x)$,并且有$f'(x)=g(x)$.
\begin{proof}
	
	先来证明$\{f_n\}$是一致收敛的.设$\{f_n\}$收敛的那个点为$x_0\in X$.先注意到:
	$$|f_n(x)-f_m(x)|\le|(f_n(x)-f_m(x))- (f_n(x_0)-f_m(x_0))|+|f_n(x_0)-f_m(x_0)|\le$$
	$$\sup_{t\in[x_0,x]}|f'_n(t)-f'_m(t)|\cdot|x-x_0|+|f_n(x_0)-f_m(x_0)|$$
	
	把最后一式记作$\Delta(x,m,n)$.按照$\{f_n'\}$是一致收敛的,结合$X$是有界集,于是存在正整数$N$使得$m,n>N$的时候总有$\Delta(x,m,n)<\varepsilon$对任意$x\in X$成立.于是,按照柯西准则说明$\{f_n(x)\}$是一致收敛的,记极限函数为$f$.接下来继续运用中值定理,得到:
	$$\left|\left(f_n(x+h)-f_n(x)-f'_n(x)h\right)-\left(f_m(x+h)-f_m(x)-f'_m(x)h\right)\right|=$$
	$$\left|(f_n(x+h)-f_m(x+h))-(f_n(x)-f_m(x))-(f'_n(x)-f'_m(x))h\right|\le$$
	$$\sup_{0<\theta<1}\left|f'_n(x+\theta h)-f'_m(x+\theta h)\right|\cdot|h|+\left|f_n'(x)-f_m'(x)\right|\cdot|h|$$
	
	其中$x,x+h\in X$.接下来继续按照$\{f_n'\}$的一致收敛性,得到函数列$\{F_n(h)\}$其中$h$满足$x+h\in X$的点集上是一致收敛的:
	$$F_n(h)=\frac{f_n(x+h)-f_n(x)-f'_n(x)h}{|h|}$$
	
	但是$\lim_{h\to0}F_n(h)=0$对每个$n\ge1$都成立,结合$f_n'(x)\to g(x)$,就得到$f'(x)=g(x)$成立.
	
\end{proof}

转化为级数语言描述这个结论,就是逐项求导:设函数项级数$\sum_{n\ge1}f_n(x)$定义在有界凸集$X$上,每个$f_n$在$X$上可导,并且导函数作为级数$\sum_{n\ge1}f_n'$在$X$上一致收敛,并且原级数$\sum_{n\ge1}f_n(x)$至少在一点收敛,那么$\sum_{n\ge1}f_n$是一致收敛的可导函数,并且它的导数可以通过逐项求导来得到:
$$\left(\sum_{n\ge1}f_n\right)'(x)=\sum_{n\ge1}f_n'(x)$$

上一个定理具有如下值得注意的推论:给定开区间上的一列具有原函数的函数列$\{f_n\}$,如果$\{f_n\}$一致收敛于$g(x)$,那么$g(x)$也具有原函数.事实上它就是,在对每个$f_n$选取特定原函数构成的函数列至少在一点收敛的前提下,新的函数列会一致收敛于一个可导函数$f(x)$,并且满足$f'(x)=g(x)$.在函数未必黎曼可积的情况下,利用函数列手段可以有效的证明原函数的存在性.
\newpage
\subsection{幂级数}

幂级数是指形如$\sum_{n\ge0}a_n(x-x_0)^n$的函数项级数.关于幂级数第一个要处理的问题是它的收敛域问题.为此记$a=\limsup_{n\to+\infty}\sqrt[n]{|a_n|}$,记$R=\frac{1}{a}$,这里约定当$a=0$的时候$R$取$+\infty$;当$a=+\infty$时$R=0$.运用柯西判别法,当$R=0$时,有$\limsup_{n\to+\infty}\sqrt[n]{|a_n(x-x_0)^n|}=\frac{|x-x_0|}{R}$,于是在$R=0$时幂级数只在点$x=x_0$处收敛;当$R=+\infty$时,有$\limsup_{n\to+\infty}\sqrt[n]{|a_n(x-x_0)^n|}=0<1$,于是幂级数在实直线上处处绝对收敛;最后当$R$是一个非0有限正数的时候,有$\limsup_{n\to+\infty}\sqrt[n]{|a_n(x-x_0)^n|}=\frac{|x-x_0|}{R}$,于是得到幂级数在$(x_0-R,x_0+R)$上收敛,在$[x_0-R,x_0+R]$补集处处发散,至于端点情况是不一定的.据此,把$R$称为幂级数的收敛半径.另外可以看出,如果幂级数在一点$x_1$处收敛,那么只要$x$满足$|x-x_0|<|x_1-x_0|$,那么幂级数在$x$处绝对收敛.

把$(x_0-R,x_0+R)$称为收敛区间,注意这个定义并不是说它是使得幂级数收敛的点集构成的区间,因为幂级数还可能在端点上收敛.

幂级数的一致收敛性.为了方便起见记幂级数的中心点$x_0=0$,即幂级数表达式为$\sum_{n\ge0}a_nx^n$.若幂级数$\sum_{n\ge0}a_nx^n$的收敛半径为$R>0$,那么对任意的$r\in(0,R)$,该幂级数在$[-r,r]$上一致收敛.事实上这由Weierstrass判别法,对$x\in[-r,r]$总有$|a_nx^n|\le|a_n|r^n$直接得出.

内闭一致收敛.所谓在开区间$(a,b)$内闭一致收敛,是指函数项级数或者函数列,在$(a,b)$的任意闭子区间$[\alpha,\beta]$上都是一致收敛的.这个性质的用处在于,当讨论单点的性质时,如连续性可导性,相比在$(a,b)$上一致收敛,内闭一致收敛条件已经足够.考虑函数列$\{f_n\}$,设它在开区间$(a,b)$内闭一致收敛,那么首先,极限函数在$(a,b)$上处处存在,因为对任意的$x_0\in(a,b)$就有闭区间包含了这个点,于是按照内闭一致收敛性,说明在这个点上函数列收敛.另外,如果每个$f_n$都在$x_0\in(a,b)$连续,就得到极限函数$f$也在$x_0$处连续,同样是因为内闭一致收敛已经保证了$x_0$附近一个包含在$(a,b)$内的闭区间上是一致收敛的.幂级数在收敛半径的开区间内是内闭一致收敛的.

幂级数的分析性质.设幂级数$\sum_{n\ge0}a_nx^n$的收敛半径为$R$,设和函数为$S(x)$,那么,按照每个$a_nx^n$的连续性,结合幂级数是内闭一致收敛的,就得到$S(x)$在$(-R,R)$上是连续的;按照$\sum_{n\ge0}na_nx^{n-1}$的收敛半径$1/\limsup_{n\to+\infty}\sqrt[n]{n|a_n|}$仍然是$R$,于是幂级数总在收敛半径确定的开区间内可导,并且求导数可以逐项求导;事实上求导的操作可以继续下去,幂级数存在任意阶导数,并且幂级数的收敛半径如果是$R$,那么任意阶导数的收敛半径也是$R$,求高阶导数同样可以逐项求导;最后对任意的$[\alpha,\beta]\subset(a,b)$,有$S(x)$总在$[\alpha,\beta]$上可积,并且求积分可以逐项积分.

这里来处理下收敛区间的端点问题.首先有如下Abel定理:设幂级数$\sum_{n\ge0}a_nx^n$的收敛半径为$R$,如果幂级数在$x=R$处收敛,那么和函数$S(x)$在$x=R$处左连续.类似的有另一端点$-R$处的结论.
\begin{proof}
	
	事实上按照$\sum_{n\ge0}a_nR^n$收敛,可以把幂级数写作$\sum_{n\ge0}a_nR^n\cdot\left(\frac{x}{R}\right)^n$,按照$\sum_{n\ge0}a_nR^n$收敛,它已经不包含变量$x$于是自然对$x\in[0,R]$上一致收敛,另外$\left(\frac{x}{R}\right)^n$是一致有界的,因为以1为界,于是$\sum_{n\ge0}a_nx^n$在$[0,R]$上一致收敛,于是在$x=R$处左连续.
	
\end{proof}

Abel定理的逆定理要想成立需要添加条件,对此介绍两个定理,首先,在添加$a_n\ge0$的条件下逆定理成立:设幂级数$\sum_{n\ge0}a_nx^n$收敛半径为$R$,并且恒有$a_n\ge0$,如果和函数$S(x)$在$x=R$处有左极限$\lim_{x\to R^-}S(x)=A\in\mathbb{R}$,那么幂级数在$x=R$处收敛,并且收敛和就是$A$.
\begin{proof}
	
	一方面,按照$\sum_{n\ge0}a_nx^n\le\sum_{n\ge0}a_nR^n$,于是得到$A\le\sum_{n\ge0}a_nR^n$.另一方面,对任意的正整数$N$,有$\sum_{n=0}^Na_nR^n=\lim_{x\to R^-}\sum_{n=0}^Na_nx^n\le\lim_{x\to R^-}\sum_{n\ge0}a_nx^n=A$,就得到$\sum_{n\ge0}a_nR^n\le A$,得到结论.
	
\end{proof}

第二个逆命题定理是Tauber定理:设$\sum_{n\ge0}a_nx^n$的收敛半径为$R$,并且$\lim_{x\to R^-}\sum_{n\ge0}a_nx^n=A\in\mathbb{R}$,并且$\frac{a_n}{R^n}=o(\frac{1}{n})$,那么有幂级数在$x=R$处收敛,收敛和就是$A$.
\begin{proof}
	
	设$\frac{a_n}{R^n}=b_n$,$\frac{x}{R}=y$,那么幂级数表示为$\sum_{n\ge0}b_ny^n$,它的收敛半径为1,并且有$\lim_{x\to1^-}\sum_{n\ge0}b_ny^n=A$,并且$b_n=o(\frac{1}{n})$.也就是说不妨就设$R=1$.
	
	按照条件有$\lim_{n\to+\infty}na_n=0$,记$\delta_n=\sup_{k\ge n}\{|ka_k|\}$,那么$\{\delta_n\}$单调递减趋于0.按照$\lim_{x\to1^-}S(x)=A$,对任意的$\varepsilon>0$,存在$\eta>0$,使得只要$0<1-x<\eta$,就有$\left|S(x)-A\right|<\varepsilon$.现在取足够大的正整数$N_0$使得$\frac{\varepsilon}{N_0}<\eta$,并且$\delta_{N_0+1}<\varepsilon^2$,于是对于正整数$N>N_0$,总有$\frac{\varepsilon}{N}<\frac{\varepsilon}{N_0}<\eta$和$\delta_{N+1}\le\delta_{N_0+1}<\varepsilon^2$.现在再取$x=1-\frac{\varepsilon}{N}$,那么$0<1-x<\frac{\varepsilon}{N}<\eta$,于是得到$\left|S(x)-A\right|<\varepsilon$,并且还有:
	$$\left|\sum_{n=0}^Na_n(1-x^n)\right|\le\sum_{n=0}^{N}|a_n||1-x^n|\le\sum_{n=0}^{N}n|a_n|(1-x)\le\delta_0N(1-x)=\delta_0\varepsilon$$
	$$\left|\sum_{n\ge N+1}a_nx^n\right|\le\sum_{n\ge N+1}\frac{|na_n|}{n}\cdot x^n\le\frac{\delta_{N+1}}{N}\cdot\frac{x^{N+1}}{1-x}\le\frac{\delta_{N+1}}{N(1-x)}<\varepsilon$$
	
	综上得到了:
	$$\left|\sum_{n=0}^{N}a_n-A\right| \le\left|\sum_{n=0}^Na_n(1-x^n)\right|+\left|\sum_{n\ge N+1}a_nx^n\right|+\left|S(x)-A\right|<(2+\delta_0)\varepsilon$$
	
	这就完成证明.
	
\end{proof}

关于两个幂级数的加减法与乘法.给定两个幂级数$\sum_{n\ge0}a_nx^n$和$\sum_{n\ge0}b_nx^n$,分别记收敛半径为$R_1,R_2$,记$R=\min\{R_1,R_2\}$,那么加减法得到的函数项级数$\sum_{n\ge0}\left(a_n\pm b_n\right)x^n$在$(-R,R)$内收敛.并且按照幂级数是绝对收敛的,得到两个幂级数的乘积$\left(\sum_{n\ge0}a_nx^n\right)(\sum_ {n\ge0}b_nx^n)$在$(-R,R)$上收敛到$\sum_{n\ge0}c_nx^n$,其中$c_n=\sum_{k=0}^{n}a_kb_{n-k}$.

按照Abel定理可以得到这样一个结论:如果$\sum_{n\ge0}a_n$和$\sum_{n\ge0}b_n$收敛,如果柯西乘积级数$\sum_{n\ge0}c_n$也收敛,那么有等式$\sum_{n\ge0}c_n=\sum_{n\ge0}a_n\sum_{n\ge0}b_n$.事实上,把三个级数对应的幂级数写出来$\sum_{n\ge0}a_nx^n$,$\sum_{n\ge0}b_nx^n$,$\sum_{n\ge0}c_nx^n$.那么按照$x=1$处都收敛,说明这三个幂级数的收敛半径都不小于1.于是按照Abel定理,可以将$x\to1^-$,就得到上述等式.

已经探究了幂级数的诸多性质.接下来讨论一个函数在什么情况下,可以在某个点附近展开为幂级数.首先幂级数展开具有唯一性,即,如果函数$f(x)$在$(x_0-R,x_0+R)$上可以展开为幂级数$f(x)=\sum_{n\ge0}a_n(x-x_0)^n$,那么知道$f(x)$在这个区间上存在任意阶导数,于是求导可得$a_n=\frac{f^{(n)}(x_0)}{n!},n\ge0$.

现在给定函数$f(x)$,约定它在点$x=x_0$处存在任意阶导数,于是如果它能在$x=x_0$附近展开为幂级数,幂级数的形式为$\sum_{n\ge0}\frac{f^{(n)}(x_0)}{n!}(x-x_0)^n$.把这个形式幂级数称为$f(x)$在$x=x_0$处的Taylor级数.当$x_0=0$的时候称为麦克劳林(Maclaurin)级数.

注意,尽管可以形式的构造出光滑函数(存在任意阶导数)的形式Taylor级数,但是这个级数是未必收敛的,并且即便它收敛,也未必收敛于原本的函数.倘若在点$x_0$处存在的确一个开区间$(x_0-R,x_0+R)$上Taylor级数$\sum_{n\ge0}\frac{f^{(n)}(x_0)}{n!}(x-x_0)^n$收敛于$f(x)$,称光滑函数$f$在点$x_0$处实解析.如果光滑函数在$(a,b)$上处处实解析,就称它是$(a,b)$上的实解析函数.那么实解析函数必然是光滑函数.

在一点处实解析实际上不是一个单点的性质,而是局部的性质.如果光滑函数$f(x)$在点$a$的开区间$(a-R,a+R)$可展开为收敛于自身的幂级数,那么$f(x)$在$(a-R,a+R)$中的每个点$x_0$,都有$f(x)$在点$x_0$的开区间$(x_0-r,x_0+r)$可展开为收敛于自身的幂级数,这里$r=R-|x_0-a|$.
\begin{proof}
	
	首先对任意的$x\in(x_0-r,x_0+r)$,都有$|x-x_0|+|x_0-a|<R$,于是级数$\sum_{n\ge0}\frac{f^{(n)}(a)}{n!}\left(|x-x_0|+|x_0-a|\right)^n$绝对收敛.也就是说如下级数是绝对收敛的:
	\begin{align*}
	\sum_{n\ge0}\left|\frac{f^{(n)}(a)}{n!}\right|\left(|x-x_0|+|x_0-a|\right)^n &=\sum_{n\ge0}\left|\frac{f^{(n)}(a)}{n!}\right|\sum_{k=0}^{n}C_n^k|x-x_0|^k|x_0-a|^{n-k} \\
	&=\sum_{k\ge0}\left(\sum_{n\ge k}\left|\frac{f^{(n)}(a)}{n!}\right|C_n^k|x_0-a|^{n-k}\right)|x-x_0|^k
	\end{align*}
	
	于是原级数交换求和次序不改变收敛和,就得到:
	\begin{align*}
	f(x)&=\sum_{n\ge0}\frac{f^{(n)}(a)}{n!}\left((x-x_0)+(x_0-a)\right)^n \\
	&=\sum_{n\ge0}\frac{f^{(n)}(a)}{n!}\sum_{k=0}^nC_n^k(x-x_0)^k(x_0-a)^{n-k} \\
	&\sum_{k\ge0}\left(\sum_{n\ge k}\frac{f^{(n)}(a)}{n!}C_n^k(x_0-a)^{n-k}\right)(x-x_0)^k \\
	&=\sum_{k\ge0}\frac{1}{k!}\left(\sum_{n\ge k}\frac{(f^{(k)})^{(n-k)}(a)}{(n-k)!}(x_0-a)^{n-k}\right)(x-x_0)^k \\
	&=\sum_{k\ge0}\frac{f^{(k)}(x_0)}{k!}(x-x_0)^k
	\end{align*}
\end{proof}

至此还是没有给出如何判断一个函数实解析的准则.首先指出了这样的函数必须是光滑的,另外,之前在微分学介绍过Taylor公式和余项估计,在那里对某个点$x_0$存在直至$n+1$阶导数的函数构造了相应的Taylor多项式.于是,一个$(x_0-R,x_0+R)$上的光滑函数是实解析的,当且仅当在这个开区间上,余项$R_n(x)=f(x)-\sum_{k=0}^{n}\frac{f^{(k)}(x_0)}{k!}(x-x_0)^k$是收敛于0的.

但是一般来讲,由于函数自身的复杂性,判断余项趋于0是很复杂的事情.不过有一些简单的充分条件,例如,如果各阶导函数在开区间上一致有界,那么容易估计出余项是趋于0的.
\newpage
\subsection{连续函数空间的稠密子集}

先给出几个概念.一个度量空间$X$的子集$Y$称为完全有界的,如果对任意给定的正数$\varepsilon$,都存在$X$中有限个半径为$\varepsilon$的圆覆盖了$Y$.给定两个度量空间$X,Y$,称一族$X\to Y$的映射族$\mathscr{F}$是等度连续的,如果对任意的$\varepsilon>0$,存在$\delta>0$使得对任意的满足$d_X(x_1,x_2)<\delta$的$x_1,x_2\in X$和任意的$f\in\mathscr{F}$,都有$d_Y(f(x_1),f(x_2))<\varepsilon$.那么等度连续说明映射族$\mathscr{F}$的每个映射都是一致连续的.

这些新概念和一致收敛性的联系在于:给定两个度量空间$K,Y$,其中$K$是紧空间,那么$K\to Y$的连续函数列$\{f_n\}$是一致收敛的,的必要条件是,函数列是完全有界和等度连续的.【】

Arzela-Ascoli定理.给定紧度量空间$X$到度量空间$Y$的函数族$\mathscr{F}$,那么存在函数族的一个可数子集构成的函数列是一致收敛的,当且仅当,这个函数族是等度连续和完全有界的.【】

来解释下这个定理的意义.给定紧度量空间$K$和任一度量空间$Y$,考虑全体$K\to Y$的连续函数构成的空间$C(K,Y)$,那么这个空间上可以自然的定义一个度量为,$d(f,g)=\max_{x\in K}d_Y(f(x),g(x))$.这个度量称为一致收敛度量,因为容易验证$d(f_n,f)\to0$当且仅当,在$K$上有$f_n\rightrightarrows f$.再结合一致收敛的柯西准则,得到$C(K,Y)$关于一致收敛度量是完备度量空间.

称一个度量空间$X$的子空间$Y$是预紧子空间(这个称法不同教材不唯一),如果$Y$上的任意点列包含一个柯西子列.那么Arzela-Ascoli定理实际上给出了$C(K,Y)$的预紧子空间的完全刻画.

Weierstrass定理.如果$f\in C([a,b],\mathbb{R})$,那么存在一列多项式$\{p_n\}$,满足在$[a,b]$上$p_n\rightrightarrows f$.几何语言就是说,多项式集合是$C([a,b],\mathbb{R})$在一致收敛度量下的稠密子集.
\begin{proof}
	
	首先注意到,如果$f,g:[a,b]\to\mathbb{R}$可以被多项式函数一致逼近,那么$f+g,fg$和$\lambda f$都可以被多项式函数逼近.先来说明在$[-1,1]$上,绝对值函数$f(x)=|x|$可以被多项式函数一致逼近.为此,首先证明对$\alpha>0$,有如下展开式:$$(1+x)^{\alpha}=1+\frac{\alpha}{1!}x+\frac{\alpha(\alpha-1)}{2!}+\cdots+\frac{\alpha(\alpha-1)\cdots(\alpha-n+1)}{n!}x^n+\cdots$$
	
	右侧幂级数就是左侧函数的形式Taylor级数,利用Raabe判别法,这个级数至少在$x=1$处收敛,于是按照Abel定理,在$[0,1]$上幂级数一致收敛.另外注意到$(1+x)^{\alpha}$这个函数的各阶导函数在$[0,1]$上一致有界,于是它的形式Taylor级数就收敛到自身.于是,取$x=-t^2$,就得到当$t\in[-1,1]$时有下式,其中右侧的级数在$[-1,1]$上一致收敛于左侧函数:
	$$(1-t^2)^{\alpha}=1+\frac{\alpha}{1!}t^2+\frac{\alpha(\alpha-1)}{2!}t^4-\cdots+(-1)^n\frac{\alpha(\alpha-1)\cdots(\alpha-n+1)}{n!}t^{2n}+\cdots$$
	
	取$\alpha=\frac{1}{2}$,取$t^2=1-x^2$,就得到下式右侧级数在$[-1,1]$上一致收敛于左侧函数:
	$$|x|=1-\frac{1/2}{1!}(1-x^2)+\frac{1/2(1/2-1)}{2!}(1-x^2)^2-\cdots$$
	
	于是只要记右侧部分和函数为$S_n(x)$,那么多项式列$\{p_n=S_n(x)-S_n(0)\}$就一致收敛到$|x|$.
	
	如果$\{p_n\}$在$[-1,1]$上一致收敛于$|x|$,那么$Mp_n(x/M)$就在$[-M,M]$上一致收敛于$|x|$.现在任取连续函数$f:[a,b]\to\mathbb{R}$,记$\max_{[a,b]}|f(x)|=M$.如果有$||y|-\sum_{k=1}^nc_ky^k|<\varepsilon$,其中$|y|\le M$,那么就有$x\in[a,b]$时有$||f(x)|-\sum_{k=1}^{n}c_nf^k(x)|<\varepsilon$.于是如果闭区间上的连续函数$f$可以被多项式函数一致逼近,那么它的绝对值函数$|f|$也可以被多项式函数一致逼近.
	
	另外,如果$f,g:[a,b]\to\mathbb{R}$可以被多项式函数一致逼近,那么$\max\{f,g\}=\frac{f+g+|f-g|}{2}$和$\min\{f,g\}=\frac{f+g-|f-g|}{2}$都可以被多项式函数一致逼近.
	
	最后,按照闭区间上连续函数的一致连续性,知道分段线性连续函数(折线函数)可以一致逼近任意的连续函数,于是只要证明折线函数可以被多项式函数一致逼近.为此,注意到折线函数就是对若干线性函数取最大最小值函数,线性函数本身已经是多项式函数了,这就导致折线函数总可以被多项式函数一致逼近.
	
\end{proof}

本节最后要介绍的定理是Stone定理,为此还需要引入一些概念.给定一个具有公共定义域的实值函数构成的族$A$,称它是一个代数,如果只要$f,g\in A$,那么就有$f+g\in A,fg\in A,\lambda f\in A,\forall \lambda\in\mathbb{R}$.例如,多项式函数构成的族是一个代数.称集合$X$上的一族实值函数$S$分离$X$中的点,如果对任意$X$中两个不同的点$x_1,x_2$,存在$S$中的函数$f$满足$f(x_1)\not=f(x_2)$.例如,多项式函数族就分离闭区间$[a,b]$中的点,只要考虑多项式$f(x)=x$.最后,称集合$X$上的实值函数族$S$不在$X$上消失,如果对任意的$x_0\in X$,存在$S$中的函数$f$满足$f(x)\not=0$.那么多项式函数族也满足这个性质.

引理.如果集合$X$上的一个实值函数族$A$,满足是代数,分离$X$中的点,并且不在$X$上消失,那么对$X$中任意两个不同点$x_1,x_2$,和任意两个实数$c_1,c_2$,存在$A$中的函数$f$,满足$f(x_1)=c_1,f(x_2)=c_2$.
\begin{proof}
	
	为证这个引理成立,只要证明对$c_1=0,c_2=1$和$c_1=1,c_2=0$分别成立即可.再结合对称性,只要证明其中一个成立即可,不妨要求$c_1=1,c_2=0$.首先存在$A$中的函数$s$,满足$s(x_1)\not=s(x_2)$.不妨要求$f(x_1)\not=0$,否则,按照条件可以取到一个函数$g\in A$,满足$g(x_1)\not=0$,选取适当的非0实数$\lambda$,就可以保证$(f+\lambda g)(x_1)\not=0$并且$(f+\lambda g)(x_1)\not=(f+\lambda g)(x_2)$.最后取$f(x)=\frac{s^2(x)-s(x_2)s(x)}{s^2(x_1)-s(x_1)s(x_2)}$,按照代数性质,它在$A$中,并且满足$f(x_1)=1,f(x_0)=0$.
	
\end{proof}

Stone定理.给定紧集$K$上的实值连续函数族$A$,要求它是代数,并且分离$K$中的点,并且不在$K$上消失,那么$A$是度量空间$C(K,\mathbb{R})$上的稠密子集.【】
\newpage
\subsection{含参变量积分}

这一节介绍一种特殊的函数族,即含参变量积分.含参变量积分是指形如$F(t)=\int_{E_t}f(x,t)\mathrm{d}x$形式的积分,$t$就是所谓的参变量,对每个定义域中取定的参变量$t$,$E_t$是一个点集,$\phi_t(x)=f(x,t)$是在$E_t$上黎曼可积的函数,于是此时$F(t)$是有意义的.含参变量积分定义了关于参变量的函数,因而可以讨论这个函数的连续性,可导性,可积性等问题.为了方便起见,主要讨论一维的情况.

闭矩体上含参变量常义积分的性质.给定$\mathbb{R}^2$中的闭矩体$P=[a,b]\times[c,d]$.设函数$f$为$P$上的函数,记含参变量积分$F(y)=\int_a^bf(x,y)\mathrm{d}x$,那么有如下结论.
\begin{enumerate}
	\item 连续性.如果函数$f$为$P$上的连续函数,那么它在紧集$P$上一致连续,按照一致连续性得到含参变量积分$F(y)$是在$[c,d]$上连续的.
	\item 可导性.如果$f$是$P$上的连续函数,并且对分量$y$在$P$上存在连续的偏导函数,同样按照一致连续性得到含参变量积分$F(y)$在$[c,d]$上可导,并且有:
	$$F'(y)=\int_a^b\frac{\partial f}{\partial y}(x,y)\mathrm{d}x$$
	\item 可积性.如果$f$是$P$上的连续函数,按照第一条$F(y)$是$[c,d]$上连续函数,于是它可积,此时按照重积分和累次积分的关系,有积分符号交换结论:
	$$\int_c^dF(y)\mathrm{d}y=\int_c^d\mathrm{d}y\int_a^bf(x,y)\mathrm{d}x=\int_a^b\mathrm{d}x\int_c^df(x,y)\mathrm{d}y$$
\end{enumerate}

上下积分界为函数形式的含参变量常义积分的性质.设$\mathbb{R}^2$中的闭矩体为$P=[a,b]\times[c,d]$,设$\alpha(y)$和$\beta(y)$都是$[c,d]\to[a,b]$的函数,设二元函数$f$定义在$P$上,记变上下界的含参变量积分为$F(y)=\int_{\alpha(y)}^{\beta(y)}f(x,y)\mathrm{d}x$,那么:
\begin{enumerate}
	\item 连续性.只要$f$是$P$上连续函数,$\alpha,\beta$都是连续函数,就有$F(y)$总是$[c,d]$上连续函数.
	\item 可导性.如果$f$在$P$上连续,并且对分量$y$的偏导数在$P$上也连续,并且$\alpha,\beta$都是可导函数,那么$F(y)$是$[c,d]$上的可导函数,并且有公式:
	$$F'(y)=\left(\int_{\alpha(y)}^{\beta(y)}f'_y(x,y)\mathrm{d}x\right)+f(\beta(y),y)\beta'(y)-f(\alpha(y),y)\alpha'(y)$$
\end{enumerate}

重点关注的是含参变量反常积分的连续性,可导性和可积性.之前的证明中主要运用的是有界闭矩体上连续函数是一致连续的,现在如果要积分区域变为无界点集或者函数不再有界.给定含参变量反常积分$\int_a^wf(x,y)\mathrm{d}x$,这里$w$要么是无穷大要么是瑕点,那么对一个固定的$y$取值,可以讨论反常积分的敛散性,这个敛散性就是$F(b,y)=\int_a^bf(x,y)\mathrm{d}x$在$b\to w$的敛散性.于是合理的思路是探究函数$F(b,y)$.和之前函数族的情况一样,为了探究$\lim_{b\to w}F(b,y)$的连续性可导性以及可积性,本质上是要探究极限符号的交换法则:
$$\lim_{b\to w}\left(\lim_{y\to y_0}F(b,y)\right)=\lim_{y\to y_0}\left(\lim_{b\to w}F(b,y)\right)$$

这是在函数列一节里讨论过的东西!能够保证极限可交换的条件就是一致收敛性!强调过存在两种一致收敛性的定义,这里把$b$作为参数的那种一致收敛性约定为,含参变量广义积分的一致收敛性.也就是说,称含参变量$y$的反常积分$\int_a^wf(x,y)\mathrm{d}x$关于$y$的一个取值集$E$是一致收敛的,如果对任意的$\varepsilon>0$,存在一个$w$的单侧开区间$U$,使得对任意$y\in E$,任意$b\in U$,都有$\left|\int_b^wf(x,y)\mathrm{d}x\right|<\varepsilon$.于是,这个定义就是之前对函数族的一致收敛性定义的特殊情况.

含参变量广义积分一致收敛的柯西准则.给定含参变量广义积分$F(y)=\int_a^wf(x,y)\mathrm{d}x$,这里$w$要么是正无穷大要么是瑕点.那么这个积分函数在集合$y\in E$上一致收敛的充要条件是,对任意的$\varepsilon>0$,存在$w$的左侧的一个开邻域$U$,使得只要$b_1,b_2\in U$,就有$\left|\int_{b_1}^{b_2}f(x,y)\mathrm{d}x\right|<\varepsilon$对任意d$y\in E$成立.

如果二元函数$f(x,y)$是$[a,w)\times[c,d]$上的连续函数,含参变量广义积分对$y\in(c,d)$都收敛,但是如果对$y=c$或$y=d$是发散的,那么在$y\in(c,d)$上就不是一致收敛的.问题出在,由于$\left|\int_{b_1}^{b_2}f(x,c)\mathrm{d}x\right|>\varepsilon_0$,按照连续性,把$c$代替为它足够小的右邻域上的元也会有这个不等式成立.

由于含参变量广义积分的一致收敛性是函数族一致收敛性的特殊情况,它们拥有相同的如下若干准则:给定二元函数$f(x,y)$和$g(x,y)$,约定对任意$y\in Y$有$f,g$都在$[a,w)$的闭子区间上可积.其中$w$是一个瑕点或者是无穷大.
\begin{enumerate}
	\item Weierstrass准则.如果对任意$x\in[a,w)$和任意$y\in Y$有$|f(x,y)|\le g(x,y)$,并且积分$\int_a^wg(x,y)\mathrm{d}x$在$y\in Y$上一致收敛,那么有$\int_a^wf(x,y)\mathrm{d}x$绝对收敛并且在$Y$上一致收敛.
	\item Dirichlet准则.如果$f$对任意$y\in Y$和任意$[a,w)$的闭子区间上的积分一致有界$\left|\int_a^bf(x,y)\mathrm{d}x\right|<M$,并且对任意$y\in Y$有$g(x,y)$关于$f$单调而且$g(x,y)$在$x\to w$的时候关于$y\in Y$一致收敛于0,那么$\int_a^wf(x,y)g(x,y)\mathrm{d}x$在$y\in Y$上一致收敛.
	\item Abel准则.如果$\int_a^wf(x,y)\mathrm{d}x$关于$y\in Y$一致收敛,并且对每个$y\in Y$有$g(x,y)$关于$x$在$[a,w)$上单调,而且$g(x,y)$一致有界,那么$\int_a^wf(x,y)g(x,y)\mathrm{d}x$是关于$y\in Y$一致收敛的.
\end{enumerate}

现在借助一致收敛来讨论含参变量广义积分函数的连续性,可导性与可积性.为此先来讨论极限.给定二元函数$f(x,y)$,其中$y$取一个集合$Y$,约定对每个$y\in Y$,$f(x,y)$在区间$[a,w)$上可积,这里$w$可能是无穷大,也可能是有限数,在有限数的情况下,积分可能对某些$y$的取值时,$w$是$f(x,y)$的瑕点,也可能对其他的某些$y$的取值$w$未必是瑕点,此时就可以把$f(x,y)$延拓定义到$[a,w]$上成为一个常义积分.约定的对每个$y\in Y$有$f(x,y)$在$[a,w)$上可积包含了上述全部可能出现的情况.现在赋予$Y$上一个局部拓扑基.

那么极限符号与广义积分号交换,由于广义积分号本质上就是一个极限号,于是这个交换法法则实际上就是两个极限符号的交换法则:如果对任意的$b\in[a,w)$,有$f(x,y)$在$[a,b]$上关于$Y$上的基一致收敛于一个函数$\phi(x)$,这个条件是说,对任意的$\varepsilon>0$,存在$Y$中一个基元素$U$,使得只要$y\in U$,$x\in[a,b]$,就总有$|f(x,y)-\phi(x)|<\varepsilon$.第二个条件是,积分$\int_a^wf(x,y)\mathrm{d}x$是关于$y\in Y$上一致收敛的,那么极限函数$\phi(x)$在$[a,w)$上可积或者广义可积,并且满足:
$$\lim_{B_Y}\int_a^wf(x,y)\mathrm{d}x=\int_a^w\phi(x)\mathrm{d}x$$
\begin{proof}
	
	按照第二个条件,对任意的$\varepsilon>0$,可以取到$w$左侧的一个开区间$(t,w)$,满足只要$A_1,A_2\in(t,w)$,就有$\left|\int_{A_1}^{A_2}f(x,y)\right|<\frac{\varepsilon}{2}$对任意的$y\in Y$成立.于是按照常义含参变量积分下,积分号和极限号可交换法则,就得到$\left|\int_{A_1}^{A_2}\phi(x)\mathrm{d}x\right|\le\frac{\varepsilon}{2}<\varepsilon$,于是由柯西收敛准则证明了$\phi(x)$在$[a,w)$上可积.
	
	现在对任意$\varepsilon>0$,按照第二个条件,可以取到$w$左侧一个开区间$(t,w)$,满足只要$A\in(t,w)$,就有$\left|\int_Af(x,y)\mathrm{d}x\right|<\frac{\varepsilon}{3}$.再按照上一段所证,也存在$w$左侧的一个开区间$(t',w)$满足只要$A\in(t',w)$,就有$\left|\int_A^w\phi(x)\mathrm{d}x\right|<\frac{\varepsilon}{3}$.取$A$是同时满足$A\in(t,w)$和$A\in(t',w)$的数.现在按照第一个条件,在区间$[a,A]$上$f(x,y)$关于$Y$一致收敛于$\phi(x)$,也就是说,存在$Y$中的一个开集$U$,满足只要$y\in U$,就有$|f(x,y)-\phi(x)|<\frac{\varepsilon}{3(A-a)}$,综上,就得到只要$y\in U$,就有:
	\begin{align*}
	\left|\int_a^wf(x,y)\mathrm{d}x-\int_a^w\phi(x)\mathrm{d}x\right| &\le\int_a^A|f(x,y)-\phi(x)|\mathrm{d}x+\left|\int_A^wf(x,y)\mathrm{d}x\right|+\left|+\int_A^w\phi(x)\mathrm{d}x\right| \\
	&<\frac{\varepsilon}{3(A-a)}\cdot(A-a)+\frac{\varepsilon}{3}+\frac{\varepsilon}{3}=\varepsilon
	\end{align*}
\end{proof}

特别的,如果取$Y$是正整数集,把$Y$上的拓扑基取为全体$[n,+\infty)\cap Y$,那么就得到函数列极限符号和广义积分符号的交换法则:设函数列$\{f_n\}$在$[a,w)$上收敛于$f(x)$,如果满足如下两个条件,第一,对每个$b\in[a,w)$,有$\{f_n\}$在$[a,b]$上关于$n\to+\infty$一致收敛于$f(x)$,第二,广义积分$\int_a^wf_n(x)\mathrm{d}x$在$n\ge1$一致收敛.那么$\int_a^wf(x)\mathrm{d}x$收敛,并且有等式:
$$\lim_{n\to+\infty}\int_a^wf_n(x)\mathrm{d}x=\int_a^wf(x)\mathrm{d}x$$

特别的,如果取$Y$是一个开区间$(c,d)$,把$Y$上局部拓扑基取为某个点$y_0\in(c,d)$的局部基,即全体包含$y_0$的$(c,d)$的开子区间,那么得到二元函数极限符号核广义积分符号的交换法则:设函数$f(x,y)$定义在$[a,w)\times(c,d)$上,有$\lim_{y\to y_0}f(x,y)=\phi(x),\forall x\in[a,w)$,如果满足如下两个条件,第一,对每个$b\in[a,w)$,有$f(x,y)$在$[a,b]$上关于$y\to y_0$一致收敛于$\phi(x)$,第二,广义积分$\int_a^wf(x,y)\mathrm{d}x$在$y\in(c,d)$一致收敛.那么$\int_a^w\phi(x)\mathrm{d}x$收敛,并且有等式:
$$\lim_{y\to y_0}\int_a^wf(x,y)\mathrm{d}x=\int_a^w\phi(x)\mathrm{d}x$$

于是得到含参变量广义积分函数的连续性定理.若$f(x,y)$在$[a,w)\times[c,d]$上连续,并且$F(y)=\int_a^wf(x,y)\mathrm{d}x$在$y\in[c,d]$上一致收敛.那么$F(y)$在$[c,d]$上是连续函数.

现在讨论可微性.给定函数$f(x,y)$定义在$[a,w)\times[c,d]$上,满足如下三个条件:\begin{enumerate}
	\item 函数$f(x,y)$和$f'_y(x,y)$都是$[a,w)\times[c,d]$上的连续函数.
	\item 积分$G(y)=\int_a^wf'_y(x,y)\mathrm{d}x$在$y\in[c,d]$上一致收敛.
	\item 积分$F(y)=\int_a^wf(x,y)\mathrm{d}x$至少在$[c,d]$中一个点$y_0$收敛
\end{enumerate}

那么$F(y)=\int_a^wf(x,y)\mathrm{d}x$实际上在$[c,d]$上一致收敛,并且$F(y)$可导,导函数就是$G(y)$.
\begin{proof}
	
	按照第一个条件,知道对任意的$b\in[a,w)$,函数$F_b(y)=\int_a^bf(x,y)\mathrm{d}x$是在$[c,d]$上可导的并且导函数为$(F_b)'(y)=\int_a^bf'_y(x,y)\mathrm{d}x$.按照第二个条件,关于变量$b$的函数族$\{(F_b)'(y)\}$当$b\to w$时在$[c,d]$上一致收敛.按照条件3,至少存在一个点$y_0\in[c,d]$使得$F_b(y_0)$在$b\to w$时收敛.
	
	于是,按照级数论里对函数族一致收敛导函数的定理,就得到结论.
	
\end{proof}

最后是含参变量广义积分函数的可积性,这分为常义可积性和广义可积性两种.首先,如果$f(x,y)$在$[a,w)\times[c,d]$上连续,并且积分$F(y)=\int_a^wf(x,y)\mathrm{d}x$在$y\in[c,d]$上一致收敛,那么$F(y)$就是$[c,d]$上的可积函数,并且积分符号之间可交换:
$$\int_c^d\mathrm{d}y\int_a^wf(x,y)\mathrm{d}x=\int_a^w\mathrm{d}x\int_c^df(x,y)\mathrm{d}y$$

现在要考虑两个广义积分符号的交换法则.给定函数$f(x,y)$在$[a,w)\times[c,u)$上连续,两个函数$F(y)=\int_a^wf(x,y)\mathrm{d}x$和$G(x)=\int_c^uf(x,y)\mathrm{d}y$分别关于$[c,u)$和$[a,w)$的任意闭子区间上一致收敛,并且,要求$|f(x,y)|$的两个累次广义积分至少有一个收敛.那么就有如下两个积分存在并且相等:
$$\int_c^u\mathrm{d}y\int_a^wf(x,y)\mathrm{d}x=\int_a^w\mathrm{d}x\int_c^uf(x,y)\mathrm{d}y$$
\begin{proof}
	
	不妨设收敛的累次积分是$\int_a^w\mathrm{d}x\int_c^u|f(x,y)|\mathrm{d}y$.按照$f(x,y)$连续以及$F(y)$在$[c,u)$的任意闭子区间上一致收敛,于是,按照常义积分号与广义积分号的交换法则,就得到,对任意的$d\in[c,u)$,有:
	$$\int_c^d\mathrm{d}y\int_a^wf(x,y)\mathrm{d}x=\int_a^w\mathrm{d}x\int_c^df(x,y)\mathrm{d}y$$
	
	这个等式左侧取极限$d\to u$得到$\int_c^u\mathrm{d}y\int_a^wf(x,y)\mathrm{d}x$.于是需要证明的是右侧取极限$d\to u$得到$\int_a^w\mathrm{d}x\int_c^uf(x,y)\mathrm{d}y$.现在记$G_d(x)=\int_c^df(x,y)\mathrm{d}y$,于是函数定义在$[c,u)$上,并且由于$f$连续得到每个$G_d(x)$也是连续的.而按照第二个条件的一致收敛性,得到$G_d(x)\rightrightarrows G(x)$在$[a,w)$的任意闭子区间$[a,b]$上都成立.
	
	而由于$|G_d(x)|\le\int_c^u|f(x,y)|\mathrm{d}y$,右侧函数在$[a,w)$上可积(这是条件3),于是说明了$\int_a^wG_d(x)\mathrm{d}x$实际上是关于$d\in[c,u)$上一致收敛的,这就保证了$\lim_{d\to u}\int_a^wG_d(x)\mathrm{d}x=\int_a^w\mathrm{d}x\int_c^uf(x,y)\mathrm{d}y$.
	
\end{proof}
\newpage
\subsection{Beta函数与Gamma函数}

Beta函数与Gamma函数就是第一类和第二类Eulerian积分,它们的定义如下:
$$B(\alpha,\beta)=\int_0^1x^{\alpha-1}(1-x)^{\beta-1}\mathrm{d}x$$
$$\Gamma(\alpha)=\int_0^{+\infty}x^{\alpha-1}e^{-x}\mathrm{d}x$$

先来讨论Beta函数的性质.首先要解决的定义域问题,为了使$B(\alpha,\beta)$在$0$附近积分收敛,需要$\alpha>0$,为了使$B(\alpha,\beta)$在$1$附近积分收敛,需要$\beta>0$,于是Beta函数的定义域为$\alpha,\beta>0$.

\begin{enumerate}
	\item 对称性.一步换元积分公式告诉我们$B(\alpha,\beta)=B(\beta,\alpha)$.
	\item 化简公式:$B(\alpha,\beta)=\frac{\alpha-1}{\alpha+\beta-1}B(\alpha-1,\beta)$.
	\item 另一积分表述(做代换$x=\frac{y}{1+y}$):$B(\alpha,\beta)=\int_0^{+\infty}\frac{y^{\alpha-1}}{(1+y)^{\alpha+\beta}}$
\end{enumerate}

Gamma函数.同样首先需要解决的是定义域问题.考虑0附近的积分的收敛性,就得到需要$\alpha>0$.Gamma函数的第一个性质是它具有光滑性:Gamma函数$\Gamma(\alpha)$是任意次可导的,并且有求导公式:
$$\Gamma^{(n)}(\alpha)=\int_0^{+\infty}x^{\alpha-1}\ln^nxe^{-x}\mathrm{d}x$$
\begin{proof}
	
	首先来证明对$(0,+\infty)$的每个闭子区间$[a,b]$,上述等式右侧对每个取值$n$,广义积分$\Gamma(\alpha)$都在$\alpha\in[a,b]$上一致收敛.为此,【】
	
\end{proof}
\begin{enumerate}
	\item 简化公式:$\Gamma(\alpha+1)=\alpha\Gamma(\alpha)$.
	\item Euler-Gauss公式:$\Gamma(\alpha)=\lim_{n\to+\infty}n^{\alpha}\frac{(n-1)!}{\alpha(\alpha+1)\cdots(\alpha+n-1)}$
	\item 余元公式:$\Gamma(\alpha)\Gamma(1-\alpha)=\frac{\pi}{\sin\pi\alpha},0<\alpha<1$.
\end{enumerate}

最后给出Beta函数和Gamma函数的联系:
$$B(\alpha,\beta)=\frac{\Gamma(\alpha)\Gamma(\beta)}{\Gamma(\alpha+\beta)}$$
\newpage
\subsection{卷积和广义函数}

给定两个函数$u,v:\mathbb{R}\to\mathbb{C}$,定义它们的卷积为下式,如果前提是右侧的广义积分对每个$x\in\mathbb{R}$都成立:
$$(u\ast v)(x)=\int_{\mathbb{R}}u(y)v(x-y)\mathrm{d}y$$

首要问题是,什么条件可以保证卷积的存在性.给定实值或者复值函数$f:G\to\mathbb{C}$,这里$G$是$\mathbb{R}$上的一个开集.称$f$在$G$上局部可积的,如果对每个点$x\in G$,存在一个开邻域$U(x)\subset G$满足$f$限制在$U(x)$上是黎曼可积的.那么如果$G=\mathbb{R}$,这个局部可积条件实际上就是说,对任意的闭区间$[a,b]$有$f$在这个闭区间上可积.把$G\to\mathbb{C}$的存在直至$m$阶$(0\le m\le+\infty)$连续导数的函数集合记作$C^m(G)$,把它的子集,具有紧支集的函数构成的子集记作$C_0^m(G)$.给定两个局部可积函数$u,v:\mathbb{R}\to\mathbb{C}$,那么如下三个条件中的任意一个,都可以保证卷积$u\ast v$的存在性:
\begin{enumerate}
	\item 函数$|u|^2$和$|v|^2$在$\mathbb{R}$上可积.
	\item 函数$|u|$和$|v|$其中一个在$\mathbb{R}$上可积,另一个在$\mathbb{R}$上有界.
	\item 函数$u,v$至少有一个是具有紧支集的函数.
\end{enumerate}




记全体实值或者复值的$\mathbb{R}$线性空间为$V$,对每个$t_0\in\mathbb{R}$,定义$V$上的平移算子$T_{t_0}$为,$(T_{t_0}f)(t)=f(t-t_0)$.如果$V$上线性变换$A$满足,对任意$t_0\in\mathbb{R}$和任意$f\in V$,有$A(T_{t_0}f)=T_{t_0}(Af)$,或者等价的说就是$Af(t-t_0)=(Af)(t-t_0)$,那么就称$A$为平移不变算子.
\newpage
\section{曲面积分}