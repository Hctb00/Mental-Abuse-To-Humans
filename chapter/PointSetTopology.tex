\chapter{点集拓扑}

拓扑指的是空间上的一种结构,它在拉伸压皱弯曲(但非撕破或粘合)等连续形变下保持不变.近代数学中拓扑思想被广泛使用.拓扑自身的成形理论主要包括点集拓扑,代数拓扑,微分拓扑等.点集拓扑是拓扑学理论的基础,它通过集合论构造拓扑学的基本概念和性质.代数拓扑则是用代数学的方法给出基本群同调群同伦群等拓扑不变量.微分拓扑主要是处理微分流形的拓扑性质.

一种说法是拓扑学最早的萌芽来自欧拉.欧拉在1736年的一篇论文中讨论了如今我们称为七桥问题的内容.欧拉意识到他在处理一种和距离概念无关的几何问题,这可以说是摆脱传统度量束缚的萌芽.这篇论文不光证明了一次经过全部七座桥的走法,还证明了,用我们现在的语言讲,一个简单图如果存在一条道路经过每条边恰好一次,当且仅当恰好存在两个顶点的次数是奇数.

随后欧拉在1750年的一封写给哥德巴赫的信中给出了著名的关于多面体的欧拉公式$v-e+f=2$.这里$v$是多面体的顶点数,$e$是边数,$f$是面数.有趣的是这一事实被处理过大量多面体问题的阿基米德和笛卡尔等人忽略.这或许是在欧拉之前人们无法跳出传统度量思考几何性质的一种佐证.关于这个公式欧拉随后在1752年发表了两篇论文给出了证明.值得注意的是欧拉要求了这个多面体是一个凸集,即多面体任表面上两点的连线都落在多面体内.

1813年,一个不太著名的数学家Antoine-Jean Lhuilier发表了一篇重要的论文,他注意到欧拉公式对于带有洞的多面体是错误的.如果洞的个数是$g$,那么欧拉公式应该为$v-e+f=2-2g$.这也是历史上人们发现的第一个拓扑不变量.

1865年莫比乌斯发表了他的莫比乌斯带的描述.他尝试把这种一面的性质描述为不能被一种带条件的三角形覆盖.如今我们知道这是定向和非定向的概念.

Johann Benedict Listing是第一个使用"topology"这个词的人.他的大多数拓扑思想都来自于高斯.(事实上高斯本人没有发表任何他所做的,涉及到我们今天称为拓扑思想的工作).Listing在1847年发表了论文《Vorstudien zur Topologie》.这篇论文没有特别重要,不过它有提出了复形这个基本概念.之后1861年Listing发表了一篇非常重要的论文,在这篇论文里他描述了莫比乌斯带,还探究了曲面的分支和连通性.

不过Listing不是第一个提出连通性的人.黎曼在1851年已经探究了这个概念,并且在1857年当他提出黎曼曲面的时候继续探究了这个性质.Jordan则提出了另一种检验连通性的方法,他通过定义一种不自交不退化为点的闭曲线,给出了涉及到连通性的一种曲面上的拓扑不变量.

Listing检验了三维欧氏空间上的连通性.后来Betti延续了他是思想证明了一般维欧氏空间的连通性.但是始终连通性的定义并没有一个公认完整的描述.Betti本人对连通性的定义就被Heegaard批判过.

庞加莱是第一个给出令人接受的连通性的描述.庞加莱在1895年提出了同调的概念,并给出了比Betti本人提出的更精确的对Betti数的定义.欧拉的对凸多面体的公式也被庞加莱描述为更为完整和一般化的表述.

促进拓扑学发展的另一条路线是人们尝试对收敛概念的推广.康拓在1872年提出了导集的概念,或者说极限点的概念.他定义实直线的闭子集是包含自身导集的子集.康拓还提出了开集的基本概念.

维尔斯特拉斯在1877年的未发表的讲义中给出了Bolzano-Weierstrass定理的严格证明.在这个证明中开邻域的概念被首次提出.希尔伯特在1902年也开始使用邻域的概念来处理他自己思考的问题.

1906年Fr\'echet提出了紧致性的概念,他称一个空间是紧致空间,如果无穷有界子集总包含极限点.之后Fr\'echet成功把这个概念从欧氏空间推广到了一般的度量空间上.他还给出了如何将康拓的开闭集概念自然的延拓到一般度量空间上.

Riesz在一篇1909年发表的论文中提出了关于拓扑的一种新公理定义.这个定义只依赖于对集合上极限点的定义,不需要引入距离的概念.几年后豪斯多夫在1914年用四条不涉及距离概念的公理重新定义了开邻域这个概念.Riesz和豪斯多夫的工作使得一般点集上可以定义拓扑结构.可以说,至此人们终于摆脱了长达千年的传统度量概念对几何思想的束缚,人们对几何结构的理解得以达到新的高度.
\section{拓扑空间}

在二十世纪头十年,许多数学家提出了拓扑空间的不同定义,但是这些定义很快过眼烟云,经过了很长时间数学家们终于确定了最合适的定义,尽管最后确定的定义看起来有些抽象.

集合$X$上的一个拓扑是指$X$的子集构成的一个族$\mathscr{R}$,它满足如下三个条件.称一个指定了拓扑的集合为拓扑空间.即拓扑空间是一个对$(X,\mathscr{R})$,当拓扑比较明确的时候往往忽略不写.称拓扑中的元素是开集.
\begin{enumerate}
	\item 空集和全集包含在$\mathscr{R}$中.
	\item $\mathscr{R}$中任意子族的元素的并在$\mathscr{R}$中.
	\item $\mathscr{R}$中任意有限子族的元素的交在$\mathscr{R}$中.
\end{enumerate}

一些例子.称点集上的平凡拓扑,是指包含空集和全集的拓扑.称点集上的离散拓扑,是指点集的整个幂集作为的拓扑.点集上的有限补拓扑是指补是整个空间或者有限集的那些集合构成的拓扑.点集上的可数补是把上述有限均改为可数.

同一个集合上可以存在多个拓扑,倘若存在两个拓扑$\mathscr{R}$和$\mathscr{R'}$ 有:$\mathscr{R}\subset\mathscr{R'}$,就称后者是细于前者的.这样的描述是形象的,前者的开集相对于后者比是大块的,需要将大块打碎才能得到后者.于是集合上最粗的拓扑就是平发拓扑,最细的拓扑就是离散拓扑.当然一般来讲同一个集合上的两个拓扑可能不存在包含关系.

给定拓扑空间$X$,任取一个点$x$,如果开集$U$包含点$x$,就称$U$是点$x$的一个开邻域.对于$X$的任意子集$V$,称$x\in V$是$V$的一个内点,如果存在点$x$的一个开邻域$V_x\subset V$.于是,一个子集$V$是开集当且仅当,$V$的每个点都是自己的内点.点$x\in X$的邻域定义为$X$的子集$V$,以$x$为内点.有些书上把邻域定义为这里的开邻域,也就是必须是开集,不用这个定义.

现在来考虑如何生成一个拓扑,这便需要引入基和子集族生成拓扑的概念.称一个由点集的子集构成的族$\mathrm{B}$是一个拓扑基,如果它满足:
\begin{enumerate}
	\item 对每个$x\in X$,存在$\mathrm{B}$中的元素$B$包含$x$.
	\item 倘若$X$被两个$\mathrm{B}$中的元素包含,不妨记作$B_1,B_2$,那么存在$\mathrm{B}$中的一个元素$B_3$满足:$x\in B_3\subset B_1\cap B_2$.
\end{enumerate}

拓扑基中的元素称为基元素.拓扑基生成的拓扑定义为拓扑基的任意子族的并再加上空集(如果约定对指标集的空子集的并为空集,那么不需要加上空集),容易验证这的确是一个拓扑.

注意到同一个集合上的一族拓扑的交仍然是一个拓扑,由此可以定义子集族生成的拓扑为包含子集族的全部拓扑之交.上述拓扑基生成的拓扑就吻合于生成的含义.更一般的,给定一族子集,它的全部有限交构成一个拓扑基,这个拓扑基生成的拓扑就是这个子集族生成的拓扑.也即,一族子集族生成的拓扑在构造角度看就是它的全部有限交的任意并.

下面通过拓扑基这个工具,给全序集上定义一种具体的拓扑,称为序拓扑.设$X$是一个全序集,现在定义它的标准拓扑,称为标准拓扑的目的在于,今后若提及全序集上的拓扑,若未指明何种拓扑,那么均指的是这个标准拓扑.按照序关系,可以给出四种类型的集合:$(a,b),(a,b],[a,b),[a,b]$.注意到对于一般的全序集,即便$a<b$,$(a,b)$仍然有可能是空集.全序集上如下三种类型的集合构成了一个拓扑基:
\begin{enumerate}
	\item 所有开区间$(a,b)$.
	\item 如果存在最小元$a_0$,那么所有形如$[a_0,b)$的区间.
	\item 如果存在最大元$b_0$,那么所有形如$(a,b_0]$的区间.
\end{enumerate}

这个拓扑基生成的拓扑称为全序集上的序拓扑.按照这个定义,$R$作为全序集,全体$(a,b)$生成的拓扑称为它的标准拓扑.

一个拓扑空间上开集的补集称为闭集.将看到尽管开集闭集仅仅是取补的关系,但是利用邻域语言他们具有非常不同的等价刻画.注意到所说的补是指开集在拓扑空间中的补,并不是指拓扑在空间的幂集中的补,拓扑空间中的子集未必不是开集就是闭集,事实上可以存在很多非开非闭的子集,另外对于某些拓扑空间也会存在既开又闭的子集.按照开集的定义,看到空集和全集是闭集,并且闭集的任意交,有限并均为闭集.

闭包.给定拓扑空间$X$的一个子集$A$,全体包含$A$的闭集的交称为子集$A$的闭包.那么按照闭集公理闭包必然是一个闭集.它就是包含$A$的最小的闭集,即每个包含$A$的闭集都要包含$A$的闭包.另外,一个子集是闭集当且仅当,它就是自身的闭包.把子集$A$的闭包记作$\overline{A}$.

关于闭包的最简单观察是,如果子集$A\subset B$,那么包含$B$的闭集必然也包含$A$,于是$\overline{A}\subset\overline{B}$.另外,闭包中的点具有如下等价描述:一个点$x$在子集$A$的闭包$\overline{A}$中,当且仅当,对$x$的任意开邻域,都和$A$的交非空.事实上一方面,如果$x\in\overline{A}$,假如$x$存在开邻域$U$不和$A$相交,那么补集$U^c$是一个闭集,并且不含点$x$,导致全体包含$A$的闭集的交不含点$x$,这和$x$在$\overline{A}$中矛盾.另一方面,假如$x$的任意开邻域和$A$相交,并且$x$不在$A$的闭包中,那么存在包含$A$的闭集$F$不含点$x$,导致$F$的补集$U$是$x$的一个开邻域并且和$A$不交,这矛盾.注意这个等价条件还可以写作:$x\in\overline{A}$当且仅当,存在$x$的一个局部基,每个基元素都和$A$有交.

闭包的这个等价描述告诉,如果$U$是开集并且$U\cap A=\emptyset$,那么$U\cap\overline{A}=\emptyset$.特别的,如果$U,V$是不交开集,那么$U\cap\overline{V}=V\cap\overline{U}=\emptyset$.称空间$X$的两个子集$A,B$是分离的,如果满足$A\cap\overline{B}=B\cap\overline{A}=\varnothing$.换句话讲,两个子集是分离的当且仅当它们不交并且任一不包含另一个的聚点.于是两个不交开集总是分离的.

另外,按照这个等价描述,给定子集$A$,不在$\overline{A}$中的点可以这样描述:它是这样的点$x$,存在开邻域$U$和$A$不交.把这样的点称为$A$的外点.于是子集$A$的外点构成的集合恰好就是$A$闭包的补集.

对于有限并,闭包运算和并集运算可交换,即$\overline{A\cup B}=\overline{A}\cup\overline{B}$.一方面$\overline{A}\subset\overline{A\cup B}$和$\overline{B}\subset\overline{A\cup B}$得到$\overline{A}\cup\overline{B}\subset\overline{A\cup B}$.另一方面,从$A\subset\overline{A}$和$B\subset\overline{B}$得到$A\cup B\subset\overline{A}\cup\overline{B}$,右侧是闭集,就得到$\overline{A\cup B}\subset\overline{A}\cup\overline{B}$.

和闭包概念对偶的,有内点集这个概念.闭包是包含着点集的最小闭集,内点集是包含于点集的最大开集.记子集$A$的内点集为$\mathrm{Int}(A)$,那么$x\in\mathrm{Int}(A)$等价于说$x$是$A$的内点,于是内点集的确符合它的名字,它就是$A$的全体内点构成的子集.和闭包对偶的,子集是开集当且仅当它等于自身的内点集,还有取内点集这个运算和有限交运算是可交换的,即$\mathrm{Int}(A\cap B)=\mathrm{Int}(A)\cap\mathrm{Int}(B)$.

于是子集$A$的外点集就是补集$A^c$的内点集.至此,给定拓扑空间$X$的一个子集$A$,存在两种对全集的划分,第一种是$A$的闭包和$A$的外点集,第二种是$A$的内点集和$A$补集的闭包.

已经讨论了子集的闭包和内点集,现在来看第三种相关的构造,即边界.给定子集$A$,定义它的边界为$\mathrm{Bd}(A)=\overline{A}\cap\overline{A^c}=\overline{A}-\mathrm{Int}(A)$.称边界中的点是$A$的边界点,那么按照定义,$A$的边界点恰好就是对每个开邻域$U$,即和$A$有交,又和$A^c$有交.粗略的讲边界就是同时能被$A$和$A^c$逼近的点构成的集合.

按照定义得出的边界的性质.首先边界必然是闭集.给定子集$A$,那么全集可以表示为$A$的内点集,外点集和边界的无交并.另外子集$A$和$A$的补集的边界是相同的.一个子集$A$的闭包或者内点集的边界包含于这个子集的边界.最后,子集$A$的闭包可以划分为$A$的边界和$A$的内点集.

为了详细讨论边界中的点引入两种新的点.点$x\in X$称为子集$A$的聚点,如果$x$的任意开邻域都包含异于$x$的$A$中点,等价的说,就是满足$x\in\overline{A-\{x\}}$.全体$A$的聚点构成的点集称为$A$的导集,记作$A^d$.点$x\in X$称为子集$A$的孤立点,如果存在$x$的某个开邻域$U$使得$U\cap A=\{x\}$.用后面子空间的话说,就是赋予子集$A$子空间拓扑后,构成单点开集的点称为孤立点.注意聚点未必是$A$中的点,但是孤立点一定是$A$中的点.

按照定义,内点集包含于导集包含于闭包.$A$的闭包可以表示为$A$和导集的并,这导致$A$是闭集当且仅当它包含自身的导集.导集运算和闭包运算一样,和有限并运算是可交换的.另外注意集合$A$可以表示为$A\cap\mathrm{Bd}(A)$和$\mathrm{Int}(A)$的无交并.而$A$的边界也可以划分为两种点集的无交并:第一种为不是内点的聚点,第二种是孤立点.

稠密性.给定拓扑空间$X$的子集$A$,称$A$在$X$中稠密,如果$\overline{A}=X$.称$A$在$X$中余稠密,如果$A^c$在$X$中稠密.称$A$在$X$中无处稠密,如果$\overline{A}$在$X$中余稠密.最后称$A$在自身上稠密,如果$A\subset A^d$.稠密子集的意义在于,给定一个$X\to Y$的连续函数$f$,如果给$Y$添加一些额外的性质(Hausdorff条件),$f$会被它在$X$的某个稠密子集上的取值完全确定.

首先,$A$在$X$中稠密当且仅当,对$X$的任意非空开集$U$,$U\cap A$是非空的.这导致,$A$在$X$中余稠密当且仅当,对$X$的任意非空开集$U$,$U$和$A$补集的交是非空的.$A$在$X$中无处稠密当且仅当,对$X$的任意非空开集$U$,包含了$X$的一个非空开集$V$和$A$是不交的.

开集的可数交未必是开集,闭集的可数并也未必是闭集.这个反例可以直接考虑$\mathbb{R}$上的开区间或者闭区间.尽管如此,这种可以被开集或者闭集通过可数交,可数并或者补集运算得到的集合,从拓扑意义上看,仍然具有某种特殊的规范形式.一个集合的子集族如果满足包含全集和空集,在可数交和可数并以及补集运算下封闭,就称它是一个$\sigma$代数.同一个集合上的$\sigma$代数的任意交仍然是一个$\sigma$代数,于是可以定义一个子集族生成的$\sigma$代数就是包含这个子集族的全部$\sigma$代数的交.称一个拓扑空间上开集族生成的$\sigma$代数为Borel代数,其中的集合称为拓扑空间的Borel集.那么上述开集的可数交和闭集的可数并是两种特殊的Borel集合,称为$G_{\delta}$集和$F_{\sigma}$集.

已经介绍了关于拓扑空间的足够信息,现在来考虑两个拓扑空间之间的映射.模仿欧式空间上的连续映射的定义,给定两个拓扑空间之间的映射$f:X\to Y$,称$f$在点$x_0\in X$是连续的,如果存在$x_0$和$y_0=f(x_0)$分别的局部基$B(x_0)$和$D(y_0)$,使得对每个$V\in D(y_0)$,存在$U\in B(x_0)$,满足$f(U)\subset V$.这个定义符合于欧式空间上映射连续性的定义,在那里把局部基取为开球,这样连续定义也就是$\varepsilon-\delta$语言.如果$f$在$X$的每个点都连续,就称$f:X\to Y$是连续映射.

对于全局上的连续映射,它有如下简洁的等价描述:映射$f:X\to Y$是连续映射当且仅当,对$Y$的每个开集,它在$f$下的原像是$X$中的开集.这也等价于要求$Y$的每个闭集的原像是$X$中的闭集.另外连续映射还有一个等价描述是,对任意$A\subset X$,有$f(\overline{A})\subset\overline{f(A)}$.

连续映射的复合仍然是连续映射,一个拓扑空间上的恒等映射是连续映射,这些保证了拓扑空间和连续映射构成了一个范畴.这个范畴中的同构称为同胚,它是一个双射,并且自身和逆映射都是连续的.在同胚下不变的性质称为拓扑性质,点集拓扑便是给出分离性,可数性,连通性,紧致性,可度量化性等拓扑性质.这里需要说明,这些性质对于判断任意两个空间同胚还是远远不够的,事实上在点集拓扑范围内甚至不能判断$\mathbb{R}^n$和$\mathbb{R}^m$是不同胚的.($n\not=m$).

这里介绍两种特殊的映射,开映射和闭映射.开映射是指把开集映射为开集的映射,闭映射同理定义.注意开闭映射未必是连续映射,例如取集合$Y$为离散拓扑,那么无论怎么定义$X$上的拓扑,任意映射$f:X\to Y$都是开闭映射,但是容易取特定的$X$上的拓扑使得$f$不连续.因此来关注连续的开映射和闭映射.

一个连续映射$f:X\to Y$是闭(开)映射,当且仅当对任意的满足$f^{-1}(B)\subset A$的子集$B\subset Y$和开(闭)集$A\subset X$,存在开(闭)集$B\subset C\subset Y$满足$f^{-1}(B)\subset f^{-1}(C)\subset A$.
\begin{proof}
	
	来证明闭的情况,对于开的情况证明是类似的.首先假设$f$是连续闭映射,取$B\subset Y$,取开集$A\subset X$满足$f^{-1}(B)\subset A$,那么集合$C=Y-f(X-A)$是一个开集,并且包含了$B$,并且满足$f^{-1}(C)=X-f^{-1}f(X-A)\subset X-(X-A)=A$.反过来,如果连续映射$f$满足题目中的条件,现在任取$X$中闭集$F$,那么$A=X-F$是开集,取$B=Y-f(F)$,有$f^{-1}(B)\subset A$,于是存在开集$C\subset Y$包含了$B$并且$f^{-1}(C)\subset A$,也就是说$f^{-1}(C)\cap F=\emptyset$,于是得到$C\cap f(F)=\emptyset$,也就是$B\subset C\subset Y-f(F)$,这导致$f(F)=Y-C$,于是$f(F)$是闭集.
\end{proof}

一个连续映射$f:X\to Y$是闭映射,当且仅当对每个$y\in Y$,和每个包含$f^{-1}(y)$的$X$中开集$U$,存在$Y$中点$y$的开邻域$V$满足$f^{-1}(V)\subset U$.事实上只要验证和上一个命题的等价性,任取$B\subset Y$和开集$A\subset X$满足$f^{-1}(B)\subset A$,对每个点$y\in B$,取条件中的开邻域$V_y\subset Y$,使得$f^{-1}(V_y)\subset A$,那么开集$C=\cup_{y\in B}V_y$满足$B\subset C$并且$f^{-1}(C)\subset A$.

一个连续映射$f:X\to Y$是开映射,当且仅当存在$X$的一个拓扑基,其中每个基元素的像都是$Y$中的开集.

同胚必然是开闭映射.注意一个连续双射未必是同胚,这个反例可以考虑一个点集上的两个不同拓扑,其中一个严格细于另一个,那么这个点集上的恒等映射,作为两个不同拓扑之间的映射,从细的到粗的恒等映射是连续的,但是从粗的到细的恒等映射不是连续的.一个连续双射是同胚当且仅当,它是开映射或者闭映射.

收敛概念.对一列点集$\{x_n\}$,称点$x_0$是一个极限点,如果对于$x_0$的任意一个开邻域,存在一个正整数$N$,使得$n>N$时$x_n$均在这个邻域当中.在一般的拓扑空间中点列的极限点如果存在,未必是一个点,例如取平凡拓扑,其上任意一个点列都收敛于任何极限.一个比较常见的使得每个点列至多收敛于一个点的条件是第二分离性公理$T_2$,也即Hausdorff条件,它的内容是任意两个点存在两个各自的开邻域使得它们不交.如果同一个点列收敛到两个点,它在项数足够大的时候会同时存在于两个不交开集中,这矛盾.
\newpage
\section{构造新的拓扑空间}

本节介绍一些从旧的拓扑空间构造新的拓扑空间的手段,包括子空间,拓扑空间的积和余积,商空间,拓扑空间的逆向系统与逆向极限.

先来定义子空间.取拓扑空间$X$的子集$A$,期望能在$A$上定义一个由大空间$X$上的拓扑传递过来的拓扑,一个直观的定义是,$A$上的这个拓扑使得从$A$到$X$的包含映射连续的最粗拓扑.把这个拓扑称为$A$作为拓扑空间$X$赋予的子空间拓扑,也可以直接称具备了子空间拓扑的$X$的子集$A$为子空间.

子空间拓扑具有一个同样直观的等价描述.子空间$A\subset X$上的拓扑恰好就是全体$U\cap A$构成的集族,其中$U$跑遍$X$中开集.即,子空间中的开集就是原空间的开集和子集的交.对偶的,子空间中的闭集就是原空间中的闭集和子集的交.

现在给定拓扑空间$X$和一个子空间$A$,当考虑$A$的子集$B$的时候,$B$可以同时看作空间$A$的子集和空间$X$的子集.$B$作为不同空间上的子集,它的闭包一般是不同的.对此有这样的结论:条件同上,此时$B$在空间$A$中的闭包,就是$B$在$X$中的闭包和$A$的交.

如果子空间$A\subset X$作为$X$的子集是闭子集,就称$A$是$X$的闭子空间.注意对于闭子空间$A$,$A$的闭子集也是$X$的闭子集,于是$A$的子集$B$在$A$中和在$X$中的闭包是相同的.

如果$f:X\to Y$是连续映射,取$X$的子空间$A$,那么$f$限制在$A$上是到$Y$上的连续映射.反过来,如果知道一个映射在足够多的子空间上是连续的,那么可以把这些局部的连续映射粘合成全空间$X$上的连续映射,这就是粘合引理:设$f:X\to Y$,若$X$是有限个闭集的并或者任意个开集的并,并且$f$在其中每个闭集/开集中是连续函数,那么$f$本身是一个连续映射.换一种说法,如果$X$可以写作有限个闭集或者任意个开集的并$X=\cup_{i\in I}C_i$,在每个$C_i$上存在到$Y$的连续映射$f_i$,并且对任意的$i,j\in I$,只要$C_i\cap C_j\not=\emptyset$,那么$f_i$和$f_j$在$c_i\cap C_j$上的限制是相同的映射,则存在$X\to Y$的连续映射$f$,满足它在每个$C_i$上的限制就是$f_i$.

将介绍的第二种构造是拓扑空间范畴中的余积,也称为拓扑空间的和或者无交并或者拓扑和.给定一族拓扑空间$\{X_s\}_{s\in S}$,取它的无交并$X=\coprod_{s\in S}$,期望赋予$S$上一个拓扑,粗略的讲能够整体的考虑这些不同空间$X_s$的拓扑,尽管这些不同的拓扑空间之间看起来无关的.为此,和子空间的情况一样,考虑典范的映射,然后要求这些映射连续来确定$X$上该赋予什么样的拓扑.从每个$X_s$存在到无交并$X$的嵌入映射$\phi_s$,要求$X$上赋予的拓扑使得全体$\phi_s$都连续的最细拓扑.把$\{X_s\}_{s\in S}$的和记作$\oplus_{s\in S}X_s$.

等价的,拓扑和$X$的一个子集$A$是$X$的开集,当且仅当$A$和每个$X_s$的交都是$X_s$中的开集.对偶的,$X$子集$A$是闭集当且仅当$A$和每个$X_s$的交都是$X_s$中的闭集.据此,得出每个$X_s$作为$X$的子集都是既开又闭的.

反过来,如果空间$X$可以表示为两两不交的开集的并$X=\cup_{i\in I}X_i$,那么$\cup_{j\not=i}X_j$是开集,导致每个$X_i$还是闭集,于是此时$X_i$实际上是两两不交的既开又闭的子集.那么集合$X$和$\oplus_{i\in I}X_i$不仅作为集合是一致的,而且作为拓扑空间也是一致的.即存在典范双射恰好是同胚.

综上两段,看到一个空间可以写作子空间的拓扑和,当且仅当这些子空间恰好是两两不交的开子空间.并且此时实际上这些开子空间也都是闭子空间.

给定一族拓扑空间$\{X_s\}_{s\in S}$,对每个$X_s$取子空间$A_s$,那么对于$A_s$无交并$A$这个集合,有两种方法赋予拓扑,第一种是作为$X_s$拓扑和的子空间,第二种是作为$A_s$的拓扑和.这两种拓扑是一致的.

最后验证下和的确是拓扑空间范畴上的余积.给定一族拓扑空间$\{X_s\}_{s\in S}$,记拓扑和为$X$,任取拓扑空间$Y$和一族连续映射$f_s:X_s\to Y$,取典范的包含映射为$\phi_s:X_s\to X$.那么$(X,\phi_s)$满足泛映射性质等价于说,存在唯一的连续映射$f:X\to Y$使得如下图表对任意$i,j\in S$都交换.事实上满足图表交换的映射只有$f$把$X_s$中的点$x_s$映射到$f_s(x_s)$.而这个映射的连续性由$f_s$的连续性保证.于是拓扑和就是余积

$$\xymatrix{
	&Y&\\
	X_i\ar[r]_{\phi_i}\ar[ur]^{f_i}&X\ar[u]^f&X_j\ar[l]^{\phi_j}\ar[ul]_{f_j}
}$$

按照拓扑和的泛映射性质,如果一个空间可以拆开为若干子空间的和,那么从空间出发的映射是连续的当且仅当它在每个这样的子空间上的限制是连续的.严格写就是,如果一个拓扑空间$X$可以表示为一族两两不交的开子集$\{X_s\}_{s\in S}$的无交并,那么有$X=\oplus_{s\in S}X_s$.如果映射$f:\oplus_{s\in S}X_s\to Y$是从拓扑和到空间$Y$的映射,那么$f$连续当且仅当,$f$在每个$X_s$上的限制映射$f\circ\phi_s$是连续的.

接下来介绍第三个构造,即拓扑空间笛卡尔积上的拓扑.给定一族拓扑空间$X_i,i\in I$,考虑这个集族的笛卡儿积$X=\prod_{i\in I}X_i$,记典范的投影映射为$\rho_i:X\to X_i$.即把$x=(x_i)\in X$映为$i$处的分量$x_i$.称$X$上的积拓扑是使得全部$\rho_i$连续的最粗拓扑.

这里给出积拓扑上开集的描述.首先,全部$\rho_i$连续说明形如$\rho_i^{-1}(U_i)$的$X$的子集都是$X$中的开集,其中$U_i\subset X_i$是开集.$X$上最粗的使得全部$\rho_i$连续的拓扑就是这些开集生成的最小拓扑,也就是这些集合的有限交作为拓扑基生成的拓扑.于是积拓扑中的开集具有形式$\prod_{i\in I}U_i$,其中$U_i$是$X_i$的开子集,并且$U_i$当中至多有有限个不等于对应的$X_i$.

容易验证积拓扑是拓扑空间范畴上的积.泛映射性质告诉,如果取拓扑空间的笛卡尔积上的积拓扑$X=\prod_{i\in I}X_i$,那么拓扑空间之间的映射$f:Y\to X$是连续的,当且仅当,每个$\rho_i\circ f:Y\to X\to X_i$都是连续的.

给定一族拓扑空间$\{X_s\}_{s\in S}$,对每个$X_s$取子空间$A_s$,考虑笛卡尔积$\prod_{s\in S}A_s$,那么有两种方式定义这个积上的拓扑,一个是拓扑空间族$\{A_s\}$的积拓扑,另一个是$\prod_{s\in S}A_s$作为$\prod_{s\in S}X_s$子空间的拓扑,这两个拓扑是一致的.

闭包运算和积拓扑运算是可交换的.即如果$A_s$是$X_s$的子空间,$s\in S$.那么$\overline{\prod_ {s\in S}X_s}=\prod_{s\in S}\overline{A_s}$.事实上$x=(x_s)\in\overline{\prod_{s\in S}A_s}$当且仅当$x$在$\prod_{s\in S}X_s$的每个开邻域中存在$\prod_{x\in S}A_s$中的元,而这等价于对$x_s$在$X_s$的每个开邻域中存在$A_s$中的元,也就是$x\in\prod_{s\in S}\overline{A_s}$.

上一段告诉,如果$\prod_{s\in S}A_s\subset\prod_{s\in S}X_s$不是空集,那么它是闭集当且仅当,每个$A_s$都是$X_s$的闭子集.另外如果$\prod_{s\in S}A_s$不是空集,则它是$\prod_{s\in S}X_s$中的稠密集当且仅当每个$A_s$都是$X_s$的稠密子集.

在笛卡尔积上还可以赋予第二种拓扑,称为箱拓扑.即$\prod_{s\in S}X_s$中的开集是全体$\prod_{s\in S}U_s$,其中$U_s$是$X_s$的开集.那么这个拓扑会比积拓扑粗,或者说笛卡尔积上的恒等映射是箱拓扑到积拓扑的连续映射.对于有限个拓扑空间的笛卡尔积,其上的积拓扑和箱拓扑是一致的.

注意箱拓扑并不是拓扑空间范畴上的积(至少在无限分量的情况下),所以积拓扑的很多自然的性质,箱拓扑并不满足.举例来讲,对$\mathbb{R}$赋予常规拓扑,考虑可数个$\mathbb{R}$的笛卡儿积$X$,在其上赋予箱拓扑,考虑$f:\mathbb{R}\to X$为每个分量映射都是恒等映射,取$X$的开集$(-1,1)\times(-\frac{1} {2},\frac{1}{2})\times(-\frac{1}{3},\frac{1}{3})\times\cdots$
它在映射下的原像是单点$\{0\}$,这明显不是一个开集,于是$f$不是连续函数.

第四个构造为商空间.在拓扑空间$X$上定义一个等价关系$\sim$,这相当于给出一个划分,记全体等价类构成的集合是$\widetilde{X}=X/\sim$,那么存在$X\to\widetilde{X}$的自然映射$p$为把$x\in X$映射为$x$所在等价类$[x]$的映射.约定$X$关于等价关系$\sim$的商空间为$\widetilde{X}$上的最细的拓扑使得$p$是一个连续函数.这相当于约定,对于$\widetilde{X}$中的子集$U$,它是开集当且仅当$p^{-1}(U)$是$X$中的开集.这个拓扑就称为商拓扑.

粗略的讲,商拓扑允许粘合一些点,例如,对于$\mathbb{R}$上一个闭区间,如果把两个端点作为一个等价类,其余点每个点自身作为一个等价类,那么这样得到的商拓扑相当于把两个端点粘合在一起,这便得到了一个环(loop).

商空间的泛映射性质.即给定拓扑空间$X$上的一个划分$\sim$,现在定义一个新范畴,它的对象是全部$(f,A)$,其中$A$是一个拓扑空间,$f$是从$X$ 到$A$的连续映射,使得$X$上每个划分在$f$下具备相同的取值,从$(f,A)$到$(g,B)$的态射就是从$A$到$B$的连续函数$\phi$使得如下图表交换.那么所定义的关于等价关系的商空间就是这个新范畴的初对象.

$$\xymatrix{
	X\ar[d]_{f}\ar[dr]^{g}&\\
	A\ar[r]_{\phi}&B
}$$

结合上述泛映射性质,可以看出一个商空间$X/\sim$到某个拓扑空间$Y$的映射$f$连续当且仅当,$f$和商空间的典范映射$p:X\to X/\sim$的复合$f\circ p:X\to Y$是连续的.

现在来考虑上述从空间到商空间的自然映射,来把它抽象成一种特殊的映射.考虑拓扑空间之间的满射$f:X\to Y$,称点$y\in Y$的纤维为$f^{-1}(y)$,于是$X$可以划分为若干纤维的并,将这个划分或者说等价关系记作$\sim_1$,那么$f$可以看作连续映射的复合$X\to X/\sim\to Y$,即商空间的典范映射$p$和连续映射$f':X/\sim\to Y$的复合.其中$f'$定义为$[f^{-1}(y)]\mapsto y$.这里的$f'$是一个连续双射,但是它未必是同胚,例如$f$是一个势为连续统的离散空间$X$到$[0,1]\subset\mathbb{R}$的双射,那么它总是连续的,并且$X/\sim$也是离散空间,所以它不可能和$[0,1]$同胚.

现在来探究这样的满射$f:X\to Y$,使得上述分解成两个映射复合时,后面的$f'$的确是一个同胚,把这种映射称为商映射.一个连续满射$f$是商映射存在如下等价描述:
\begin{enumerate}
	\item $f$可以写作映射的复合$X\to X/\sim\to Y$,其中$\sim$是$X$上某个等价关系,并且$X/\sim\to Y$是同胚.
	\item $f^{-1}(U)$是$X$中开集当且仅当$U$是$Y$中开集.
	\item $f^{-1}(F)$是$X$中闭集当且仅当$F$是$Y$中闭集.
	\item $f$诱导的映射$f':X/\sim_1\to Y$是同胚,这里$\sim_1$是纤维定义的等价关系.
\end{enumerate}
\begin{proof}
	
	1推2,记$f=f'\circ p$,那么$f^{-1}(U)=p^{-1}((f')^{-1}(U))$是$X$中开集当且仅当$(f')^{-1}(U)$是$X/\sim$开集当且仅当$U$是$Y$中开集.2和3的等价性由满射性保证.3推4,首先$f'$已经是连续双射,于是为证它是同胚,只要证明它是闭映射,为此注意到对任意$F\in X/\sim$是闭集,有$f^{-1}(f'(F))=p^{-1}(F)$是$X$中闭集,于是$f'(F)$是闭集.4推1是直接的.
\end{proof}

从第二和三等价描述,看出连续满的开映射和闭映射都是商映射的特殊情况,并且商映射的复合还是商映射,并且商映射是同胚当且仅当是双射.另外,如果连续映射$f:X\to Y$和$g:Y\to Z$满足$f$和$g\circ f$是商映射,那么$g$也是商映射.事实上首先条件保证$g$是满射,另外$U$是$Z$中开集当且仅当$f^{-1}\circ g^{-1}(U)$是$X$中开集,再结合$f$是商映射说明这当且仅当$g^{-1}(U)$是开集,于是$g$是商映射.

最后一个构造是逆向系统和逆向极限.称一个集合$I$是有向集,如果它是一个偏序集,并且对任意的$a,b\in I$,存在$c\in I$使得$a\le c,b\le c$.考虑一族拓扑空间$\{X_i\}_{i\in\Sigma}$,其中指标集$\Sigma$是一个有向集.对每个$\rho\le\sigma$,存在连续映射$\pi_{\rho}^{\sigma}:X_{\sigma}\to X_{\rho}$,满足只要$\gamma\le\rho\le\sigma$,那么有$\pi_{\sigma}^{\sigma}=1_{X_{\sigma}}$和$\pi_{\gamma}^{\rho}\circ\pi_{\rho}^{\sigma}=\pi_{\gamma}^{\sigma}$.这时候称$S=\{X_{\sigma},\pi_{\rho}^{\sigma},\Sigma\}$是一个拓扑空间的逆向系统.特别的,对$\Sigma=\mathbb{N}^+$,赋予标准的全序,称逆向系统为逆向序列.

考虑$\prod_{\sigma\in\Sigma}X_{\sigma}$中这样的元$(x_{\sigma})$,满足$\pi_{\rho}^{\sigma}(x_{\sigma})=x_{\rho}$对任意的$\sigma\le\rho$成立,把这样的元称为逆向系统的线(thread).全体线构成的$\prod X_{\sigma}$的子集赋予积拓扑的子空间拓扑,称为逆向系统$S$的逆向极限,记作$\lim_{\leftarrow}S$.

按照逆向极限上拓扑的定义,得到所有投影映射$p_{\sigma}:\lim_{\leftarrow}S\to X_{\sigma}$都是连续的.

例子.给定一个无穷集合$S$,给定一族以$S$为指标集的拓扑空间$\{X_s\}_{s\in S}$,记$S$的全体有限子集构成的集族为$\Sigma$,那么$\Sigma$上以包含序为一个偏序,并且是一个有向集.对每个$\sigma\in\Sigma$,记$X_{\sigma}=\prod_{s\in\sigma}X_s$,对$\rho\le\sigma$,定义$\pi_{\rho}^{\sigma}:X_{\sigma}\to X_{\rho}$为典范的连续的限制映射,于是$\textbf{S}=\{X_{\sigma},\pi_{\rho}^{\sigma},\Sigma\}$是拓扑空间的一个逆向系统.现在对每个$s\in S$,记$\sigma_s=\{s\}\in\Sigma$,那么存在从$\lim_{\leftarrow}\textbf{S}$到$\prod_{s\in S}X_s$的同胚为把点$(x_{\sigma})$映射为点$(x_{\sigma_s})$.换言之积拓扑可以写作逆向极限的形式.

例子.给定拓扑空间$X$,把$X$的全体子空间构成的集合记作$\Sigma$,$\Sigma$以反向包含序作为一个有向集,对于$M,L\in\Sigma$,$M\subset L$,定义$\pi_L^M:M\to L$为嵌入映射.于是$\textbf{S}=\{M,\pi_L^M,\Sigma\}$是拓扑空间的一个逆向系统,而逆向极限$\lim_{\leftarrow}\textbf{S}$同胚于交子空间$\cap_{\Sigma} M$.

关于范畴角度的注解.在拓扑空间范畴中,任何小图表都存在极限和余极限,也就是说拓扑空间范畴是一个完全范畴和余完全范畴.于是对于有向集作为图表的情况极限和余极限都是存在的,也就是逆向极限和顺向极限都存在.可以验证上述构造的确是逆向极限.另外注意所说的逆向极限实际上在范畴语言中是极限的特例,而顺向极限在范畴语言中是余极限的特例.
\newpage
\section{度量空间}

集合$X$上的一个度量是指从$X\times X$到$\mathbb{R}$的函数,满足:
\begin{enumerate}
	\item 正定性:对全部$x,y\in X$,有$d(x,y)\ge0$,取等当且仅当$x=y$.
	\item 对称性:对全部$x,y\in X$,有$d(x,y)=d(y,x)$.
	\item 次可加性:对全部$x,y,z\in X$,有:$d(x,y)+d(y,z)\ge d(x,z)$.
\end{enumerate}

称两个点在度量下的值是它们之间的距离.对任意正数$\varepsilon$和点$x$,记全部和$x$距离小于$\varepsilon$的点构成以$x$为中心的$\varepsilon$-开球,记作$B_{\varepsilon}(x)$.容易验证全体开球是一个拓扑基,它生成的拓扑称作是度量生成的拓扑.当提及度量空间时,就默认它具备了度量诱导的拓扑.

例如$\mathbb{R}$上的度量$d(x,y)=|x-y|$诱导出$\mathbb{R}$上的标准拓扑.任意点集上$d(x,y)=\delta_{xy}$诱导出离散拓扑.

度量空间$X$上度量是从$X\times X$到$R$的连续函数,并且实际上度量诱导的拓扑就是使得这个函数连续的最粗拓扑.

称一个拓扑空间是可度量化空间,如果它的拓扑可以被一个度量诱导出来,或者说如果空间同胚于一个度量空间.点集拓扑的主要目标是给出一些基本的拓扑不变量,尽管提到过这些不变量对于同胚的判断是几乎无效的,但是它们对于给出可度量化的充要条件已经足够.

现在证实前文提及的度量空间上点列至多收敛到一点,这只需推出度量空间都是hausdorff空间.对任意两个点$x\not=y$,取$d=d(x,y)/2>0$,那么分别以$x,y$为中心以$d$为半径的开球是它们分别的不交的开邻域.

在度量空间中拓扑基就是开球,由此给出连续性的等价描述:
$f$是度量空间$X$到度量空间$Y$的映射,它们分别以$d_{X}$和$d_{Y}$为度量,那么$f$ 连续当且仅当对任意$x\in X$, 对任意$\epsilon>0$,存在一个$\delta>0$使得$d_{X}(x,y)<\delta$时总有$d_ {Y}(f(x),f(y))<\epsilon$.注意到这恰好就是对$R^n$上连续函数的定义.

在度量空间中,定义Cauchy列是这样的序列$\{a_n\}$,它满足对任意$\varepsilon>0$,存在一个正整数$N$,使得$m,n>N$时总有$d(a_m,a_n)<\varepsilon$, 容易看出,收敛的点列都是Cauchy列,但是反过来未必成立,例如$R$上全体有理数作为度量空间的子空间,同样是度量空间,取$\sqrt{2}$ 的有理数逼近$1.4,1.41,1.414,\cdots$,这是一个Cauchy列但是不收敛到有理数.如果一个度量空间中Cauchy列总是收敛的,那么称为\textbf{完备度量空间}.

Cauchy列的收敛性有一个简单准则:如果Cauchy有收敛子列那么它收敛.

完备度量空间的闭子空间同样是完备的.这是因为对于闭子空间中的Cauchy列,它作为原空间中的点列也是Cauchy列,那么按照完备性它收敛,收敛到的点按照闭性也在子空间中,这就说明闭子空间的每个Cauchy都收敛到自身.

按照度量结构同样可以定义一个范畴,它的对象是度量空间,态射是保度量结构的映射,即从$X$到$Y$的映射$f$使得$\forall a,b\in X$有$d_X(a,b)=d_Y(f(a),f(b))$,称为保距映射.按照度量的正定性,看出保距映射总是一个单射,于是也称为保距嵌入.这个范畴的同构是那些双射的保距嵌入,这时可以看出它的逆映射同样是保距嵌入,称为等距同构,也称两个度量空间是保距同构的.保距同构的度量空间作为拓扑空间自然是同胚的.

在前文看到$Q$作为度量空间不是完备的,但是它可以等距嵌入到完备空间$R$.事实上这总是成立的,任一度量空间可以等距嵌入到一个完备度量空间中,使得它是完备度量空间的稠密子空间,这称为完备化,证明和$Q$ 到$R$的考察Cauchy 列的方法一致.任取一个度量空间$(X,d)$, 记$\widetilde{X}$ 表示$X$中全体Cauchy列构成的点集.在$\widetilde{X}$ 中定义一个等价关系,两个Cauchy列$\{x_n\}$和$\{y_n\}$ 相关当且仅当$d(x_n,y_n)\to0$.这是一个等价关系,考虑它的商空间$X'=\widetilde{X}/\sim$, 把Cauchy列$\{x_n\}$ 所在等价类记作$[x_n]$,那么$X'$中定义度量为$d'([x_n],[y_n])=\lim_ {n\to\infty}d(x_n,y_n)$.现在构造从$(X,d)$到$(X',d')$的等距嵌入,把$x\in X$映射为作为恒等点列$\{x\}$ 在商空间$X'$中的对应,记作$[\{x\}]$,那么这的确是个保距嵌入.对$X'$中任意一个点$[x_n]$,取$[\{x_n\}],n\ge1$,这是$X$(作为$X'$中的子空间)中的一列点,并且$\lim_ {n\to\infty}d'([x_m],[\{x_n\}])=\lim_{n\to\infty}\lim_{m\to\infty}d(x_m,x_n)=0$
于是$X$是$X'$的稠密子集.最后利用引理:度量空间$(X,d)$的稠密子空间$Y$ 如果每个Cauchy列都在$X$中收敛,$X$自身是完备度量空间.这是因为,任取$X$中一个Cauchy列$\{x_n\}$,按照稠密性存在一列$Y$中的$\{y_n\}$使得$d(x_i,y_i)<\frac{1}{i}$,那么易得$\{y_n\}$是$Y$中Cauchy列,按照条件它收敛到$x_0\in X$,这又得出$\{x_n\}$收敛到$x_0$.回到原命题的证明,这里已经得出$X$ 是$X'$的稠密子集,现在对于$X$中任意一列Cauchy列$[\{x_n\}],n\ge1$,我们已经看到它收敛到$X'$中的点$[x_n]$.按照引理,这就说明了$(X',d')$是一个完备的度量空间.

这一段来证明完备化在等距同构意义下是唯一的.如果$(X,d)$具有两个完备化$(X_i,d_i),i=1,2$,那么$X$等距同构于$X_1$和$X_2$的分别某个稠密子集,不妨就设$X$ 同时是他们的子空间.对于$X_1$中的点$x_1$,按照稠密性存在$\{a_n\}\subset X$使得$d_1(a_n,x_1)\to0$,由于$\{a_n\}$是$X_2$中的Cauchy列,它收敛到某个点$x_2$,构造这样的映射$\theta:X_1\to X_2$就将每个$x_1$映射为$x_2$.验证它是保距的并且是满射.
\newpage
\section{可数性公理}

局部基.给定拓扑空间$X$,任取一点$x$,一族由点$x$的开邻域构成的集合族称为点$x$处的局部基,如果对$x$的任意一个开邻域$U$,存在这个集合族中的元$V$使得$V\subset U$.于是点$x$的全部开邻域构成的集合必然是点$x$处的一个局部基.另外,如果取点$x$处的一个局部基为$B(x)$,注意这是个集合族,那么集合族$\cup_{x\in X}B(x)$是一个拓扑基.

一个拓扑空间的点$x$的特征定义为它局部基的最小基数,记作$\chi(x,X)$.定义拓扑空间$X$的特征为全体$\chi(x,X)$的上确界,记作$\chi(X)$.称拓扑空间是第一可数的,或者满足第一可数性公理($A_1$),如果$\forall x\in X$都有$\chi(x,X)\le\aleph_0$.称拓扑空间$X$满足第二可数性公理($A_2$),如果有$\chi(X)\le\aleph_0$.注意$\chi(X)\le\aleph_0$这个条件等价于说,拓扑空间存在一个拓扑基是至多可数的.

尽管定义中第一可数和第二可数要求的是局部基或者拓扑基是至多可数的,但是我们习惯称作局部基或者拓扑基是可数的,这不会引起歧义,因为对一个有限的局部基或者拓扑基$\{U_1,U_2,\cdots,U_n\}$,可以取$U_{k}=U_n,k\ge n+1$.那么$\{U_n\}$也是一个基.

如果一点$x$处存在可数局部基,那么它具有标准可数局部基,即可数的局部基$\{U_n\}$满足$U_1\supset U_2\supset\cdots$.这是因为,取任意的可数局部基$A_n,n\ge1$,那么每一个$A_n$包含点$x$,断言$B_n=\cap_ {1\le k\le n}A_k,n\ge1$构成一个递减的局部基.它仍为开集是由拓扑公理保证的,现在对于$x$的任意开邻域$U$,按照$A_n,n\ge1$是开邻域知道存在一个$A_t$使得$x\in A_t\subset B$, 那么就有:$x\in B_t\subset B$,这说明了$B_n,n\ge1$就是标准的可数局部基.

第二可数性公理强于第一可数性公理.如果存在拓扑基是可数的,那么对任意点$x$, 取这个拓扑基中全部和$x$相交的基元素,它们至多可数的并且将构成一个局部基.

这一段给出一个非$A_1$的空间.考虑一个不可数的点集$X$,赋予可数补拓扑,这意味着开集是空集和全集和那些补集是可数集的点集.现在断言空间$X$的每个点处不存在可数的开邻域基.倘若有一个点$x$处存在可数的开邻域基$\mathscr{V}=\{V_n,n\ge1\}$,不妨设其中不含空集,现在对于任意一个异于$x$的点$y$, 注意到$\{y\}^c$ 是一个包含点$x$的开集,按照开邻域基的定义,存在某个$V_i\subset\{y\}^c$,这就是说对每个$y\not=x$,存在一个$\mathscr{V}$ 中的$V_i$ 使得$y\in V_i^c\subset\cup_{i=1}^{\infty}V_i^c$,那么有:$X-\{x\}\subset\cup_{i=1}^{\infty}V_i^c$,但是左边是不可数集,右边是可数集,这矛盾.

我们之前在拓扑空间中定义了序列收敛定义.对于$A_1$空间,序列收敛满足两个我们在分析中比较熟知的结论,即序列引理和Heine归结原理.即如下两个命题,它们均是必要性需要$A_1$条件,而充分性不需要任何条件.
\begin{enumerate}
	\item 序列引理:$X$的子集$A$中有序列收敛到点$x\in X$ 当且仅当$x\in\overline{A}$,另外,$X$的子集$A$中有不取$x$的序列收敛到点$x\in X$ 当且仅当$x$是$A$的聚点.
	\item Heine归结原理:$f$为拓扑空间$X$到$Y$的映射,那么$f$连续当且仅当对任意$X$中收敛序列:$x_n\to x_0\in X$ 有序列收敛$f(x_n)\to f(x_0)$
\end{enumerate}

第二可数的空间的每个拓扑基都存在一个至多可数子族构成了拓扑基.任取可数基$\{B_n,n\ge1\}$,任取拓扑基$\{C_i,i\in I\}$,那么对任意一对正整数$m,n$,如果存在$C_i$中的开集使得$B_n\subset C_i\subset B_m$,就取一个记作$C_{n,m}$,那么这样取出的$C_{n,m}$是至多可数的,并且对每一个点$x$,任取一个$B_m$包含这个点,那么按照$C_i,i\in I$ 是拓扑基,可以取某个$C_i$包含点$x$并且在$B_m$之内,再结合$B_n,n\ge1$是拓扑基又可以取一个$B_n$包含点$x$并且在$C_i$内,于是可以取一个$C_i$使得$B_n\subset C_i\subset B_m$,于是有某个$B_{n,m}$包含点$x$,于是全体$C_{n,m}$覆盖整个空间,并且每个$B_n$是若干$C_{n,m}$的交,下面,如果存在两个$C_{n,m},C_{s,t}$ 相交,任取一个交点$x$, 那么按照拓扑基的定义存在一个$B_u$使得$x\in B_u\subset C_{n,m}\cap C_{s,t}$,那么按照上面的过程知道存在一个$C{p,q}$包含点$x$并且在$B_n$中,这满足拓扑基的第二个要求,于是全体$C_{n,m}$构成可数的拓扑基.

第一第二可数性公理的传递性:
\begin{enumerate}
	\item 第一/第二可数性空间的子空间总是第一/第二可数的.
	\item 第一/第二可数性空间的可数积总是第一/第二可数的.
	\item 第一/第二可数性空间的开的满的连续像是第一/第二可数的.特别的第一第二可数性都是拓扑不变性.但是一般来讲连续像未必保这两个性质.
\end{enumerate}

第二可数性空间$X$上满足如下两个性质:
\begin{enumerate}
	\item $X$具有可数稠密子集.这只要取一个可数基$\{U_n\}$,然后在每个$U_n$中取定一个点$x_n$,那么$\{x_n,n\ge1\}$就是可数的稠密子集.
	\item $X$的任意开覆盖总有至多可数子覆盖.取定开覆盖$\{V_i\}_{i\in I}$,取定可数拓扑基$\{U_n,n\ge1\}$,对每个$U_n$,取$V_n$为覆盖了它的$\{V_i\}$中的开集,如果不存在这样的开集则取$V_n=\varnothing$.我们断言$S=\{V_n\mid V_n\not=\varnothing\}$是子覆盖.任取$x\in X$,那么存在某个$V_i$覆盖了$x$,于是存在基元素$U_m$满足$x\in U_m\subset V_i$.于是此时$V_m$非空,在$S$中,于是$x\in V_m$,于是$S$是至多可数的子覆盖.
\end{enumerate}

满足有可数稠密子集的空间称为可分空间;满足任意开覆盖有至多可数子覆盖的空间称为林德洛夫空间.这两个新性质通常也称作可数性公理.上一个性质说明了第二可数空间总是可分空间和林德洛夫空间.

反过来,度量空间中第二可数和这两个新的可数性公理是等价的.
\begin{proof}
	
倘若度量空间是林德洛夫的,现在固定正整数$n$,考虑$B_{\frac{1}{n}}(x)$,让$x$跑遍整个空间,这将构成一个覆盖,可取可数子覆盖$S_n$,记$S=\cup_{n\ge1}S_n$,我们断言开集族$S$是一个拓扑基.任取一个开集$U$ 任取其中的点$x$,度量结构使得我们找到正数$d(x,U^c)=d$,于是取$e$使得$3e<d$ 将保证上述取的以$e$半径的覆盖整个空间的开球族中覆盖点$x$的那个开球位于$U$之中.

现在设度量空间是可分的,取一个可数的稠密子集$x_n,n\ge1$,取这可数个点每个点的标准可数局部基,得到了可数个开集,我们断言这构成一个拓扑基.对任意开集$U$和$U$中一个点$x$,记$d(x,U^c)=h$,取$x$的一个半径小于1的开球$B_d(x)$包含在$U$中,按照这个可数集是稠密的,这个开球里面会包含某个$x_m$,记$d(x_m,x)=e<h/2$,取大于$e$的最小$\frac{1}{n}$,按照$e<d<1$ 知这可以取到,于是$x\in B_{\frac{1}{n}}\subset U$.
\end{proof}

度量空间总是$A_1$空间,事实上对任意点$x$,$B_{\frac{1}{n}}(x),n\ge1$构成一个标准可数局部基.度量空间可以是第二可数的,例如$\mathbb{R}$,全部有理点的有理长度半径构成一个可数拓扑基.但是也存在非第二可数的度量空间.另外注意紧空间必然是林德洛夫的.于是紧度量空间必然是第二可数的.

这两个新的可数性公理的传递性:
\begin{enumerate}
	\item 可分空间的开子空间是可分的,林德洛夫空间的闭子空间是林德洛夫的.但是一般来讲,可分/林德洛夫空间的子空间通常不是可分/林德洛夫的.
	\item 可分空间的可数积是可分的.但是林德洛夫空间一般甚至不保有限积.
	\item 可分空间/林德洛夫空间的连续满射像是可分/林德洛夫的.这说明它们都是拓扑不变性.
\end{enumerate}

本节最后我们整理下涉及到这四个可数性公理的反例:
\begin{enumerate}
	\item 第一可数,可分,林德洛夫,非第二可数.
	
	实下限空间$\mathbb{R}_l$.实直线$\mathbb{R}$上全部$[a,b)$作为拓扑基生成的拓扑称为下限拓扑.
	
	第一可数性:对每个实数$r$,考虑$[r,r+\frac{1} {n}),n\ge1$,这构成了$r$的一个标准开邻域基,于是下限拓扑是第一可数的.
	
	不满足第二可数性:倘若它存在一个可数基$B_n,n\ge1$,对每个实数$x$, 可以取一个基元素$B_x$ 使得$x\in B_x\subset[x,x+1)$,此时$x=\inf B_x$,这说明对于$x\not=y$有$B_x\not=B_y$,于是基元素不可能只有可数个.
	
	林德洛夫:为此只需证明对于空间的任意一个基元素覆盖,存在可数的子覆盖.设$\mathscr{A}=\{[a_i,b_i)\}_{i\in I}$覆盖整个$R$,取$U=\cup_{i\in I}(a_i,b_i)$,我们断言$R-U$是一个至多可数集.这是因为,任取一个$x\in R-U$,那么$x$必然不被某个$(a_i,b_i)$覆盖,于是存在一个指标$\delta$使得:$x=a_{\delta}$, 那么可以取$(a_{\delta},b_{\delta})$中的一个有理数$q_x$.我们断言从$R-U$到$Q$的映射$x\mapsto q_x$是单射,事实上可以证明如果$x,y\in R-U$并且$x<y$,那么$q_x<q_y$,因为若否,有$x<y<q_y<q_x$, 这导致$y\in (a_{\delta},b_{\delta})$,这和$y\in R-U$矛盾.已经说明了$R-U$是至多可数的,那么就可以取$\mathscr{A}$中可数个集合覆盖全部这可数个点,接下来注意到$U$ 是存在可数字覆盖的,这只需考虑$R$上标准拓扑即可.综上选出了$\mathscr{A}$的可数字覆盖.
	
	可分性:只需注意到全部有理数是一个可数稠密子集.
    \item 第一可数,非可分,非林德洛夫.这个只要取一个不可数集上的离散拓扑.
    \item 林德洛夫,非可分,非第一可数.考虑$X$为不可数集上的余可数拓扑,即开集定义为空集全集和补集为至多可数集.
    
    它是林德洛夫的,因为对任意的$X$的开覆盖$\{U_i\}$,取定一个开集$U_j$,那么$X-U_j$是可数集,记作$\{x_n\}_{n\ge1}$,对每个$x_n$找到$\{U_i\}$中的$U_n$使得$x_n\in U_n$,于是$\{U_j,U_1,U_2,\cdots\}$是一个可数子覆盖.
    
    它不是可分的.假设$X$存在至多可数的稠密子集$S_0$,那么$X-S_0$是开集,但是它不含稠密子集中的点.
    
    它不是第一可数的.任取点$x\in X$,假设$x$有可数局部基$U_n$,那么$A=\cup_{n\ge1}U_n$是至多可数集,于是$B=(X-A)\cup\{x\}$是$x$的开邻域,但是不存在$U_n$满足$x\in U_n\subset B$.
    \item 第一可数,可分,非林德洛夫.两个实下限空间的积称为Sorgenfrey平面.
    
    由于实下限空间是第一可数和可分的,所以Sorgenfrey平面也是第一可数和可分的.现在证明它不是林德洛夫的,注意到全体$[a,b)\times [c,d)$构成了这个空间的基.为了证明原空间不是林德洛夫的,只需寻找一个闭子空间证明它不是林德洛夫的.现在取子空间$A=\{(x,-x),x\in R\}$, 它是一个闭集.并且注意到全体$[a,a+1)\times[-a,-a+1)\cap A,a\in R$均为开集,这告诉实际上$A$ 按照子空间拓扑是离散拓扑,但是它的势不可数,于是必然不会是林德洛夫的.
    \item 第一可数,林德洛夫,非可分.
    
    赋序方体是$[0,1]\times[0,1]$上赋予全序$(x_1,y_1)<(x_2,y_2)$当且仅当要么$x_1<x_2$,要么$x_1=x_2$且$y_1<y_2$.
    
    林德洛夫:作为线性连续统,由于它有上下确界,说明它是紧致的,于是也是林德洛夫的.
    
    第一可数:对于点$(x,y),y\not=0,1$,那么它包含于$x\times(0,1)$中,这同胚于$(0,1)$,于是这里的点总是有可数局部基的.对于方体上下两侧边界点,例如考虑$(x,1)$,可验证$((x,1-1/n),(x+1/n,0))$构成了可数局部基.同理对于$(x,0)$型的点也存在可数局部基,于是赋序方体是第一可数的.
    
    非可分:这只需注意到它存在不可数个两两不交的开集族$\{x\times(0,1),x\in[0,1]\}$.
    \item 这里我们再给出说明可分和林德洛夫不能互相推出的两个有趣的反例.
    
    取一个不可数集$X$,取不在$X$中的一个点$p$,记$Y=\{p\}\cup X$,定义$Y$上的拓扑为,全体$X$的子集,以及子集$\{p\}\cup Z$,这里$Z$在$X$中的补集是可数集.它是林德洛夫空间,因为对开覆盖$\{V_i\}$,可取覆盖了点$p$的一个开集为$V_0$,那么$X-V_0$是至多可数集,它可以被$\{V_i\}$中至多可数个开集覆盖,这就找到了一个可数子覆盖.它不是可分的,因为如果存在可数稠密子集$S$,那么$X-S$非空,任取$x\in X-S$,则$\{x\}$是开集,但是它不包含$S$中的点.
    
    取$X$是包含$x$轴的$\mathbb{R}^2$的上半平面.定义拓扑基为全体包含在开的上半平面$S_0=\{(x,y)\in\mathbb{R}^2\mid y>0\}$的开圆盘,以及全体和$x$轴$L$相切的$S$中的开圆盘.那么这个空间是可分的,因为全体$S$的有理点构成了可数稠密子集.但是它不是林德洛夫的,因为林德洛夫空间的闭子集应该也是林德洛夫的,但是$S$的闭子集$L$是一个不可数的离散空间,它不是林德洛夫的.
    \item 最后给出传递性的反例.Sorgenfrey平面不是林德洛夫的就说明了两个林德洛夫空间的积空间不一定是林德洛夫的.另外取Sorgenfrey平面的子集$L=\{(x,-x)\mid x\in\mathbb{R}\}$,它是不可数的离散空间,这说明可分空间的子空间一般不是可分的.赋序方体是林德洛夫空间,但子空间$[0,1]\times(0,1)$可被不可数的两两不交的开集族$x\times(0,1)$覆盖,它没有可数子覆盖,于是这个子空间不是林德洛夫的.
\end{enumerate}
\newpage
\section{分离性公理}

$T_0$空间.$T_0$空间也称为Kolmogorov空间,是指给定空间中任意两个点,至少存在其中一个点,它有开邻域不含另一个点.空间是$T_0$的等价于任意两个不同单点子集的闭包是不同的.称两个点是拓扑不可区分的,如果这两个点分别的全体开邻域构成的集合是相同的.那么$T_0$空间也等价于说任意两个点都不是拓扑不可区分的.拓扑不可区分是空间上的等价关系,空间在这个等价关系下的商空间总是一个$T_0$空间,这称为空间的Kolmogorov商.

$T_1$空间.$T_0$空间也称为Tychonoff空间,是指全体单点集是闭集.$T_1$空间等价于讲每个点的所有开邻域的交只有自身.那么$T_1$空间必然是$T_0$空间;所有有限子集都是闭集,所有余有限子集都是开集;在$T_1$空间上子集$A$的聚点还可以描述为,该点的每个去心邻域都包含$A$中无穷个点.

$T_2$空间.$T_2$空间也称为Hausdorff空间,是指任意两个不同的点都存在一对不交的开邻域.这个条件在所有分离公理中是最常见的一个,它保证了空间上序列,网,滤子等的极限的唯一性.$T_2$条件可以说明任何单点的闭包不会包含另一个点,于是$T_2$空间都是$T_1$的.另外空间$X$是$T_2$空间当且仅当,$X\times X$的对角线$\{(x,x)\mid x\in X\}$是闭集.

正则性和正规性.空间称为正则的,如果对任意闭集$C$和任意不在$C$中的点$p$,存在各自的开邻域不交.空间称为正规的,如果对任意两个不交的闭集$C_1,C_2$,存在各自的开邻域不交.称正则的$T_2$空间为$T_3$空间,称正规的$T_2$空间为$T_4$空间.

注意正则和正规两个概念没有关系,互相均不能推出.但是如果添加$T_2$作为$T_3$和$T_4$空间,那么容易看出$T_4$空间必然是$T_3$空间,$T_3$空间必然是$T_2$空间.另外,$T_3$空间也等价于正则的$T_0$空间,$T_4$空间也等价于正规的$T_1$空间.换句话说,正则条件下$T_0$和$T_1$和$T_2$等价,正规条件下$T_1$和$T_2$等价.
\begin{proof}
	
	后一个等价是直接的,因为$T_1$说明单点集都是闭集.现在我们说明前一个等价性.假设$X$是正则的$T_0$空间,任取两个不同的点$x,y$,按照$T_0$性,不妨设$x$存在开邻域$U$不包含$y$,于是$U$的补集$C$是包含$y$不包含$x$的闭集.再按照正则性,可取$x$的开邻域$V_1$和$C$的开邻域$V_2$满足$V_1$和$V_2$不交,此时$V_2$也是$y$的开邻域,这就说明$X$是$T_2$的.
\end{proof}

按照开集闭集是互为补集的,正则性和正规性就有如下等价描述:
\begin{enumerate}
	\item 正则性等价于对任意点$x$和它的一个开邻域$U$,存在一个$x$的开邻域$V$ 使得$x\in V\subset\overline{V}\subset U$.
	\item 正规性等价于对任意闭集$A$和一个包含它的开集$U$,存在一个开集$V$ 使得$A\subset V\subset\overline{V}\subset U$.
\end{enumerate}

林德洛夫空间上$T_3$和$T_4$等价.
\begin{proof}

对一个满足$T_3$的林德洛夫空间$X$中任意两个不交的闭集$A,B$.对$A$中每个点$x$,有$B^c$是它的开邻域,按照正则性,存在一个$x$的开邻域$V_x$并且闭包在$B^c$中,对每个$x\in A$取到这样的$V_x$,它们闭包都和$B$ 无交并且覆盖整个$A$,按照林德洛夫空间对闭子空间具有遗传性,这个开覆盖存在可数字覆盖,记作$A_n,n\ge1$, 同理可以取覆盖了$B$的可数的开集族$W_n,n\ge1$, 使得其中每个开集的闭包在$A^c$ 中,现在得到了$A,B$分别的可数覆盖,但是它们之间未必是无交的,为此,取:
$$V_n'=V_n-\cup_{i=1}^n\overline{W_i};W_n'=W_n-\cup_{i=1}^n\overline{V_i}$$

那么$V_n',W_n'$都是开集并且$V=\cup_{n}V_n';W=\cup_nW_n'$分别覆盖了$A$和$B$, 按照构造方式,知道$V$和$W$ 无交的开集,这得证.
\end{proof}

度量空间是$T_4$的.按照$\cap_{n\ge1}B_{\frac{1}{n}}(x_0)=\{x_0\}$说明度量空间是$T_1$的.接下来说明正规性.任取不交的两个闭集$A,B$,对任意$a\in A$和$b\in B$,定义 $d(a,B)=\varepsilon_a;d(A,b)=\varepsilon_b$,取$U=\cup_{a\in A}B_{\varepsilon_a/2}(a)$,$V=\cup_{b\in B}B_{\varepsilon_b/2}(b)$,那么$U,V$分别是$A,B$的开邻域,它们是不交的,因为假设$z\in B_{\varepsilon_a/2}(a)\cap B_{\varepsilon_b/2}(b)$,那么三角不等式得到$d(a,b)<\frac{\varepsilon_a+\varepsilon_b}{2}\le\max\{\varepsilon_a,\varepsilon_b\}$.这和$\varepsilon_a$或者$\varepsilon_b$的定义矛盾.

正规性的重要等价描述:Urysohn引理.$X$是正规空间当且仅当对任意无交的闭集$A,B$,存在$X$到$[0,1]$(作为$R$上赋予标准拓扑的子空间拓扑)的连续函数$f$使得$f(x)=0,\forall x\in A$,$f(x)=1,\forall x\in B$
\begin{proof}
	
	充分性是显然的.对于必要性,给定任意不交的闭集$A,B$.我们期望对$[0,1]$中每个有理数$p$定义$X$中的一个开集$U_p$,满足如果$[0,1]$中有理数$p,q$满足$p<q$,那么有$\overline{U_p}\subset U_q$.先记$[0,1]$上的有理数排列为$\{r_n\}$,不妨设$r_1=1$,$r_2=0$,取$U_1=X-B$,那么$U_1$是包含了$A$的开集,按照正规性,可取$A$的开邻域$U_0$满足$A\subset U_0\subset\overline{U_0}\subset U_1$.假设对前$n$个$\{r_n\}$的点$p$已经构造了$U_p$,并且满足对其中的$p<q$有$\overline{U_p}\subset U_q$.现在设$r=r_{n+1}$,那么在实数的序关系下可设$\{r_1,\cdots,r_n\}$中$r$的最大的前继元是$p$,最小的后继元是$q$,那么按照构造有$\overline{U_p}\subset U_q$.现在按照正规性,就可构造开集$U_r$满足$\overline{U_p}\subset U_r\subset\overline{U_r}\subset U_q$.于是对任意的$p\in\{r_1,\cdots,r_n\}$,从$p<r$得到$\overline{U_p}\subset U_r$,从$r<p$得到$\overline{U_r}\subset U_p$.于是对任意的$p,q\in\{r_1,r_2,\cdots,r_{n+1}\}$,从$p<q$得到$\overline{U_p}\subset U_q$.于是归纳构造下去,就对$[0,1]$中每个有理数$p$构造了$X$中的开集$U_p$,使得只有$p,q\in[0,1]$且$p<q$,那么有$\overline{U_p}\subset U_q$.
	
	第二步我们把这个以$[0,1]$上有理数集为指标集的开集族延拓为以$\mathbb{Q}$为指标集的开集族.即定义$p<0$时$U_p=\varnothing$,$p>1$时$U_p=X$.于是对任意有理数$p<q$,总有$\overline{U_p}\subset U_q$.
	
	第三步构造映射.给定点$x\in X$,记$\mathbb{Q}(x)=\{p\in\mathbb{Q}\mid x\in U_p\}$.那么这个集合总不会取小于0的有理数,会取大于1的任何有理数.于是这个集合下有界.定义$f(x)=\inf\mathbb{Q}(x)$.
	
	第四步证明$f(x)$是满足条件的连续映射.首先如果$x\in A$,那么对所有$\ge0$的有理数$p$,就有$x\in U_p$,于是$\inf\mathbb{Q}(x)=0$,即$f(x)=0$.如果$x\in B$,那么没有$\le1$的有理数$p$满足$x\in U_p$,于是$f(x)=1$.最后我们只要证明$f$的连续性.
	
	为此,先来证明两个基本事实.若$x\in\overline{U_r}$,则$f(x)\le r$,若$x\not\in U_r$,则$f(x)\ge r$.因为如果$x\in\overline{U_r}$,那么对$s>r$,就有$x\in U_s$,于是$\mathbb{Q}(x)$包含了所有大于$r$的有理数,于是$f(x)=\inf\mathbb{Q}(x)\le r$.同理可证第二个.
	
	现在给$X$的点$x_0$,以及包含$f(x_0)$的开区间$(c,d)$.我们期望找到$x_0$的一个开邻域$U$满足$f(U)\subset(c,d)$.取有理数$p<q$使得$c<p<f(x_0)<q<d$,考虑开集$U=U_q-\overline{U_p}$.那么按照上述两个基本事实,有$x_0\in U$.接下来任取$x\in U$,则$f(x)\le q$和$f(x)\ge p$,于是有$f(x)\in[p,q]\subset(c,d)$.
\end{proof}

通常把Urysohn引理中的新条件称为可以用一个连续函数分离不交闭集$A,B$.于是正规性等价于讲对任意两个不交闭集,总可以用连续函数将他们分离.一个自然的猜测是正则性是否等价于总可以把不交的点和闭集用连续函数分离,遗憾的是这不成立,后者要更强,它称为完全正则性,由于它的强度介于$T_3$和$T_4$之间,于是被记作$T_{3.5}$公理.

这里来给出Urysohn引理的一个简单应用,将证明如果一个$T_4$空间(这里有用到$T_1$)不是全不连通的,那么空间的势是不可数的.任取一个多于一个点的连通分支$C$,取其中两个不同点$a,b$,它们作为单点子集是不交闭集,按照Urysohn引理,这说明存在一个连续映射$f:X\to[0,1]$使得$f(a)=0,f(b)=1$,$C$连通告诉$f(C)$连通,于是$f(C)= [0,1]$, 它的满射像是不可数的,那么$C$本身必然是不可数的.

正规性另一个等价描述,Tietze扩张定理.空间$X$正规当且仅当满足下面的任一条件:
\begin{enumerate}
	\item 对$X$中任意闭子集$M$,任一连续的$M\to [a,b]$(作为$\mathbb{R}$的子空间)可以连续延拓成整个空间到$[a,b]$的函数.
	\item 对$X$中任意闭子集$M$,任一连续的$M\to\mathbb{R}$(标准拓扑)可以连续延拓成整个空间到$\mathbb{R}$的函数.
\end{enumerate}

\begin{proof}
	
	1推正规性.取$X$中两个无交闭集$A,B$,那么$A\cup B$是$X$中的闭集,定义函数$f:A\cup B\to[a,b]$为$f(A)=a$,$f(B)=b$.那么$f$是一个连续函数,它可以延拓为连续映射$g:X\to[a,b]$.于是从Urysohn引理得到$X$是正规空间.同理得到2推正规性.
	
	接下来证明正规性推1和2.
	
	第一步,对每个连续函数$f:A\to[-r,r]$,这里$A$是$X$的闭集,我们构造全空间上的连续函数$g:X\to\mathbb{R}$,使得$\forall x\in X$有$|g(x)|\le\frac{r}{3}$,并且$\forall a\in A$有$|g(a)-f(a)|\le\frac{2r}{3}$.为此,先把闭区间$[-r,r]$三等分为$I_1=[-r,-\frac{r}{3}]$,$I_2=[-\frac{r}{3},\frac{r}{3}]$,$I_3=[\frac{r}{3},r]$.记$B=f^{-1}(I_1)$和$C=f^{-1}(I_3)$,它们是$A$的闭子集,由于$A$是闭集,于是$B,C$是$X$的不交闭子集.按照Urysohn引理,存在连续函数$g:X\to[-\frac{r}{3},\frac{r}{3}]$满足$g(B)=-\frac{r}{3}$,$g(C)=\frac{r}{3}$.于是$|g(x)|\le\frac{r}{3},\forall x\in X$,并且如果$a\in A$,若$a\in B$则$f(a),g(a)$同时落在$I_1$中;若$a\in C$则$f(a),g(a)$同时落在$I_3$;若$a\in A-B\cup C$,则$f(a),g(a)$同时落在$I_2$.无论哪种情况,都有$|f(a)-g(a)|\le\frac{2r}{3}$.
	
	第二步,证明正规性推出命题1.不妨设$[a,b]=[-1,1]$,设$f:A\to[-1,1]$是连续函数,那么$f$满足第一步中取$r=1$,于是存在全空间$X$上的连续函数$g_1$满足$|g_1(x)|\le\frac{1}{3},x\in X$,$|f(a)-g_1(a)|\le\frac{2}{3},a\in A$.假设已经构造了$X$上的函数$g_1,\cdots,g_n$,满足$\left|f-g_1-g_2-\cdots-g_n\right|\le\left(\frac{2}{3}\right)^n,a\in A$.那么考虑函数$f-g_1-\cdots-g_n$,在第一步中取$r=\left(\frac{2}{3}\right)^n$,就得到了$X$上的函数$|g_{n+1}(x)|\le\frac{1}{3}\left(\frac{2}{3}\right)^n,x\in X$,并且$\left|f(a)-g_1(a)-\cdots-g_{n+1}(a)\right|\le\left(\frac{2}{3}\right)^{n+1},a\in A$.这就归纳构造了$X$上的函数列$\{g_n(x)\}$.最后构造$g(x)=\sum_{n\ge1}g_n(x)$,验证$g$收敛到$[-1,1]$,连续性,并且在$A$上和$f(x)$一致.于是它就是$f$的在全空间上的延拓.
	
	第三步,证明命题1推出命题2.先取一个$\mathbb{R}$到$(-1,1)$的同胚$\tau$,现在对任意连续的$f:M\to R$, 有$g=\tau\circ f$是$M$到$(-1,1)$的连续函数,由$2$知有它在整个空间上的连续延拓$\overline{g}:X\to[-1,1]$,但是此时新函数可能会取到$\{-1,1\}$.于是我们不能用$\tau$将它同胚回$R$上.由于$N=\overline{g}^{-1}(\{-1,1\})$是$X$中和$M$不交的闭集,由Urysohn引理知存在连续函数$h:X\to[0,1]$使得:$f(x)=0,x\in N;f(x)=1,x\in M$,从而有$f$的连续延拓:
	$$\overline{f}(x)=\tau^{-1}\left(\overline{g}(x)h(x)\right)$$
\end{proof}

Urysohn可度量化定理.一个拓扑空间是可分的可度量化空间当且仅当它满足$A_2$和$T_3$.按照前文证明的林德洛夫空间下$T_3$和$T_4$等价以及$A_2$蕴含林德洛夫,于是这也等价于空间满足$A_2$和$T_4$.
\begin{proof}

首先前文证明了度量空间中可分和$A_2$等价,并且度量空间实际上是$T_4$的,这得到了必要性.为了证明充分性,先构造一个可分度量空间$\mathbb{H}$,然后我们证明满足$A_2,T_3$的空间总可以嵌入到这个空间,可度量空间的子空间可度量,这就得到了充分性.

$\mathbb{H}$作为集合是全体平方收敛实数列构成的集合:
$$(a_1,a_2,\cdots)\in H\Leftrightarrow a_i\in R,\sum_{n=1}^{+\infty}a_n^2<+\infty$$

在$\mathbb{H}$上定义度量为,$a=(a_1,a_2,\cdots),b=(b_1,b_2,\cdots)\in H$,$d(a,b)=\sqrt{\sum_{n=1}^{+\infty}\left(a_n-b_n\right)^2}$

验证该函数取值有限并且是度量,称一个序列终端为0,如果存在一个正整数$N$使得$n>N$时$a_n=0$,那么全体终端为0的有理数序列是这个空间的可数稠密子集,于是$H$是可分的度量空间.

现在给定一个$A_2,T_3$的空间$X$,我们知道它也是$T_4$空间.取它的一个不含空集的可数基$\mathscr{B}$,那么$\mathscr{B}\times\mathscr{B}$是可数的$X\times X$的子集族,取$\mathscr{A}=\{(U,V)\in\mathscr{B}\times\mathscr{B},\overline{U}\subset V\}$,这也是可数的子集族.于是可以把它排列为:$\{(U_i,V_i),i\ge1\}$.由Urysohn引理,对每个$i\ge1$有$X\to[0,1]$的函数$f_i$使得$f_i(\overline{U_i})=0,f_i(X-V_i)=1$.现在定义从$X$到$\mathbb{H}$的映射:
$$f(x)=\left(f_1(x),\frac{1}{2}f_2(x),\frac{1}{3}f_3(x),\cdots\right)$$

由于$0\le f_i\le1$,得出$\sum_ {i=1}^{+\infty}\left(\frac{1}{i}f_i(x)\right)^2<+\infty$.这说明$f(x)\in\mathbb{H}$.现在只需验证$f$是一个嵌入,即它是单射并且和它的像空间同胚.

这一段证明$f$是单射.取$x,y\in X,x\not=y$,那么按照$T_1$知道存在$V\in\mathscr{B}$使得$x\in V,y\not\in V$,再由$X$正规,取$x$的一个开邻域$U\in\mathscr{B}$使得$\overline{U}\subset V$,那么$(U,V)\in\mathscr{A}$,不妨设它们的指标是$j$,有$f_j(x)=0,f_j(y)=1$,于是$f(x)\not=f(y)$.

这一段证明$f$连续.对任$x\in X,\varepsilon>0$,取$N\in N^+$使得$\sum_ {i\ge N+1}\frac{1}{i^2}<\frac{\varepsilon^2}{2}$.按照$f_i,1\le i\le N$的连续性,取$x$的一个开邻域$W$使得$\forall y\in W,1\le i\le N$有$|f_i(x)-f_i(y)|<\frac{\varepsilon}{\sqrt{2N}}$.那么当$y\in W$有:
\begin{align*}
d(f(x),f(y))&=\sqrt{\sum_{i=1}^N\frac{(f_i(x)-f_i(y))^2}{i^2}+\sum_{i>N}\frac{(f_i(x)-f_i(y))^2}{i^2}}\\
&<\sqrt{\left(\frac{\varepsilon}{\sqrt{2N}}\right)^2N+\frac{\varepsilon^2}{2}}\\
&=\varepsilon
\end{align*}

最后证明$X$经$f$同胚于像$f(X)$.这只需证明$f$是到$f(X)$的开映射,即对$X$ 中开集$W$总有$f(W)$是$f(X)$的开集.取$x\in W$,$y=f(x)$.那么可以取$(U_i,V_i)\in\mathscr{A}$使得$x\in U_i\subset W$.于是$f_i(x)=0,f(X-W)=1$.那么对$z\in X-W$,有$d(y,f(z))=d(f(x),f(z))\ge\frac{1}{i}$.这说明$f(X-W)$和$H$ 中以$y$ 为中心以$\frac{1}{i}$半径的闭圆是不交的.从$f$单射告诉我们$f(X-W)=f(X)-f(W)$,那么得到$f(X)\cap B_{\frac{1}{i}}(y)\subset f(W)$.于是$y$是$f(W)$的一个内点,按照$x$的任意性就得到$f(W)$是开集.
\end{proof}

这个证明过程可以说明一个空间是完全正则空间当且仅当它同胚于$[0,1]^J$的子空间.

关于分离公理的传递性.
\begin{enumerate}
	\item $T_0,T_1,T_2,T_3,T_{3,5}$以及正则空间都对子空间有传递性.$T_4$和正规性对于闭子空间是传递的,但是对一般子空间来讲未必传递.
	\begin{proof}
		
		$T_0,T_1,T_2$对子空间的传递性是直接的.$T_3$的情况只要运用之前提到的正则性的等价描述.并且这个等价描述并不适用于正规性的传递性证明.最后来说明完全正则性对子空间的传递性.对完全正则空间$X$,取子空间$Y$,取$x_0\in Y$,取$Y$中的不含$x_0$的闭集$A$,那么$A=\overline{A}\cap Y$.于是$x_0\not\in\overline{A}$.于是存在连续函数$f:X\to[0,1]$,使得$f(x_0)=1,f(\overline{A})=\{0\}$.把$f$限制在$Y$上就说明了$Y$是完全正则的.
	\end{proof}
	\item $T_0,T_1,T_2,T_3,T_{3,5}$对积空间具有传递性.存在这样的两个$T_4$空间,它们的积空间不是$T_4$的.
	\begin{proof}
		
		$T_0,T_1,T_2$对积空间的传递性同样是直接的.$T_3$的情况同样只要运用之前提到的正则性的等价描述.并且这个等价描述同样不适用于正规性的传递性证明.最后来说明完全正则性对积空间的传递性.考虑一族完全正则空间的积空间$X=\prod_iX_i$.取点$b=(b_i)$和不含这个点的闭子集$A$.取一个基元素$\prod_iU_i$包含点$b$,那么只有有限个指标$i$不满足$U_i=X_i$,记作$j_1,\cdots,j_n$.对每个$j_k$取连续函数$f_k:X_{j_k}\to[0,1]$满足$f_k(b_{j_k})=1$和$f_k(X_{j_k}-U_{j_k})=\{0\}$.取$g_k=f_k\circ\pi_{j_k}:X\to[0,1]$,那么$g=g_1g_2\cdots g_n:X\to[0,1]$满足$g(b)=1$和$g(X-\prod_i U_i)=\{0\}$.
	\end{proof}
\end{enumerate}

由于正规空间的子空间未必正规,我们定义子空间都是正规空间的正规空间为完全正规空间或者遗传正规空间.称满足$T_1$的完全正规空间为$T_5$空间.$T_5$空间就是所有子空间都是$T_4$的空间.

最后我们整理一些涉及到分离性公理的反例.
\begin{enumerate}
	\item $T_0$非$T_1$的空间.Sierpinski空间,它是二元点集$\{a,b\}$上取拓扑为$\{\varnothing,\{a\},\{a,b\}\}$.它是最简单的非离散非平凡的拓扑空间.
	\item $T_1$非$T_2$的空间.无限集上的余有限拓扑.
	\item $T_2$空间的商未必是$T_2$的.例如将空间$\mathbb{R}\times\{0\}\cup\mathbb{R}\times\{1\}$做粘合$(r,0)\sim(r,1),r\not=0$,这得到带两个原点的实直线,它不是$T_2$的.
	\item $T_2$非$T_3$的空间.取$\mathbb{R}$的可数子集$K=\{\frac{1}{n}\mid n\ge1\}$.记$\mathbb{R}$上全体$(a,b)-K$作为拓扑基生成的空间为$\mathbb{R}_K$.它自然是$T_2$的.我们来说明它不是正则的.$K$是不包含点$0$的闭集.假设存在不交的开集$U,V$分别包含$0$和$K$.取一个基元素$(a,b)-K$满足$0\in (a,b)-K\subset U$.取足够大的正整数$n$使得$\frac{1}{n}\in(a,b)$,于是可取到基元素$(c,d)-K$满足$\frac{1}{n}\in(c,d)-K\subset V$.最后取$z\in(\max\{c,,\frac{1}{n+1}\},\frac{1}{n})$,那么$z$同时落在$U,V$中,和交为空集矛盾.
	\item 实下限拓扑$\mathbb{R}_l$是正规空间.单点集是闭集是直接的.给定两个不交闭集$A,B$.对每个$a\in A$可取开集$[a,x_a)$不和$B$相交,对每个$b\in B$可取开集$[b,x_b)$不和$A$相交.那么我们断言$U=\cup_{a\in A}[a,x_a)$和$V=\cup_{b\in B}[b,x_b)$是不交的开集.因为如果出现$[a,x_a)$和$[b,x_b)$有交,这必然导致$[a,x_a)$包含$b$或者$[b,x_b)$包含$a$,矛盾.
	\item Sorgenfrey平面不是正规空间.这个事实可以说明两件事,首先由于$\mathbb{R}_l$是正规空间,所以它也是正则空间,正则空间的积空间也是正则的,于是Sorgenfrey平面是正则空间,于是这个事实说明了正则空间未必是正规的;另外,这个事实还告诉我们两个正规空间的积未必是正规的.
	
	假设Sorgenfrey平面$X$是正规空间,取它的闭离散子空间$L=\{(x,-x)\mid x\in\mathbb{R}\}$.于是$L$的任意子集都是$X$中的闭集.于是$A\in L$和$L-A$是一对无交闭集,于是可以找到开集$U_A$和$V_A$不交,并且分别包含$A$和$L-A$.
	
	记$X$中全体有理点构成的集合为$D$,它是$X$的稠密子集,定义映射$\theta$把$L$的每个子集映射为$D$的一个子集:$\theta(A)=D\cap U_A$,如果$\varnothing\subsetneqq A\subsetneqq L$;$\theta(\varnothing)=\varnothing$;$\theta(L)=D$.
	
	我们断言$\theta$作为$L$的幂集到$D$的幂集的映射是单射.为此只需证对$L$的非空真子集$A,B$有$\theta(A)\not=\theta(B)$.不妨设$x\in A,x\not\in B$,那么$x\in L-B$,于是$x\in U_A\cap V_B$,而$U_A\cap V_B$是开集,所以包含$D$中的点,这个点属于$U_A$但不属于$U_B$.于是$\theta$是单射.
	
	但是$P(D)$具有基数$c$,而$P(L)$的基数是$2^c$,理应是$c<2^c$,这就矛盾.
	
	\item 给定不可数集$J$,那么$\mathbb{R}^J$作为积空间不是正规的.这个事实可以说明三件事,首先说明正则空间未必是正规空间;如果把$\mathbb{R}^J$看作$[0,1]^J$的子空间,后者是紧致(这需要后文的Tychonoff定理)Hausdorff空间,于是后者正规,这说明正规空间的子空间未必正规;最后说明正规空间的积空间未必证明.
\end{enumerate}
\newpage
\section{连通性}

称一个拓扑空间是连通空间,如果它不能表示为两个不交非空开集的并.按照开闭集的对偶性,也等价于说不能表示为两个不交非空闭集的并,另外也等价于讲空间的既开又闭的子集只有全集和空集,也等价于讲空间不会是更小子空间的拓扑和.最后一个等价描述说明了连通性描述了拓扑空间在拓扑和意义下的不可分解性.

连通性有时还会用分离概念描述.即空间是连通的当且仅当不能分解为两个分离的非空子集的并.事实上倘若可以表示,记作$X=A\cup B$,那么此时有$X=\overline{A}\cup B$且$\overline{A}\cap B=\varnothing$,于是$A=\overline{A}$,于是$A$是闭集,同理$B$是闭集,这就和连通性矛盾.反过来按照不交开集总是分离的,于是这个条件推出空间连通.
\begin{enumerate}
	\item 设$Y$是$X$的子集,如果$A,B$是$Y$的分离子集,那么它们也是$X$的分离子集.事实上$A,B$是$Y$的分离子集说明了$\overline{A}\cap Y\cap B=\varnothing$,由于$B\subset Y$,这等价于$\overline{A}\cap B=\varnothing$.同理$\overline{B}\cap A=\varnothing$.于是$A,B$作为$X$的子集也是分离的.
	\item 如果$Y$是空间$X$的一个连通子集,并且$X$中有分离子集$A$和$B$,满足$Y\subset A\cup B$,那么必然有$Y$只包含于$A,B$中的一个.事实上我们只需证明$A_1=Y\cap A$和$B_1=Y\cap B$是$Y$的分离子集.而$A_1$在$Y$中的闭包为$\overline{A}\cap Y$,$B_1$在$Y$中的闭包为$\overline{B}\cap Y$,于是它们在$Y$中分离.
\end{enumerate}

为了说明连通性是拓扑不变性,可以利用一种从连续映射角度的等价描述:空间$X$连通,当且仅当任意的$X\to D$的连续映射都是常值映射,这里$D$是任意的二元离散空间.另外我们在下文可以证明连通性实际上满足比拓扑不变性更强的性质:不需要同胚映射,连续满射就能保证总把连通空间映射为连通空间.

我们接下来给出判断连通性的一些结论.
\begin{enumerate}
	\item $A$是$X$的连通子集,有$A\subset B\subset\overline{A}$,那么$B$是连通的.特别的,连通子空间的闭包是连通的.事实上假设$B$不是连通子集,那么$B=B_1\cap B_2$,其中$B_1,B_2$是$B$的非空分离子集,于是它们也是$X$中的分离子集,于是$A$必然包含在$B_1,B_2$中的一个,不妨设$A\subset B_1$,于是$\overline{A}\subset\overline{B_1}$,于是$B_2\subset\overline{A}\subset\overline{B_1}$矛盾.
	\item 如果空间$X$的一族连通子空间的交非空,那么这些子空间的并是连通的.记空间$X$的一族连通子空间为$\{Y_i\}_{i\in I}$,记$Y=\cup_{i\in I}Y_i$,假设$Y$不是连通子集,那么有$Y$的非空分离子集$Z_1,Z_2$使得$Y=Z_1\cap Z_2$,于是每个$Y_i$必然落在$Z_1,Z_2$中的一个,倘若它们不是落在同一个$Z_i$内,由于$Z_1$和$Z_2$不交,会导致和$\{Y_i\}_{i\in I}$的交非空矛盾.于是可不妨设$Y\subset Z_1$,这导致$Z_2=\varnothing$矛盾.
	\item 连通空间的连续满射像是连通的,特别的连通空间的所有商空间都是连通的.事实上设$X$是连通空间,取连续满射$f:X\to Y$,假设$Y$不连通,则存在不交非空开集$Y_1,Y_2$使得$Y=Y_1\cup Y_2$,于是$X=f^{-1}(Y_1)\cup f^{-1}(Y_2)$.这是两个不交开集,满射说明它们都非空,这就和$X$的连通性矛盾.
	\item 有限个连通空间的积空间是连通的.按照归纳法只需证明两个连通空间的积空间是连通的.给定连通空间$X,Y$,先取积空间$X\times Y$的一个基本点$(a,b)$,考虑子空间$T_x=(X\times\{b\})\cup(\{x\}\times Y)$,这是两个具有公共点$(x,b)$的连通子空间的并,按照第二条知每个$x\in X$,有$T_x$是连通子空间.下面考虑并$\cup_{x\in X}T_x$,事实上它就是$X\times Y$,它们是有公共点$(a,b)$的一族连通子空间的并,按照第二条就得到这仍然是一个连通空间.
	\item 任意一族连通空间的积空间是连通的.设$\{X_i\}_{i\in I}$是一族连通空间.取定积空间中的一个点$(a_i)_{i\in I}$.记$J$为$I$的全部有限子集构成的集合.对$J$中的每个元$J_0$,记$C_{J_0}=\prod_{i\in I}A_i$,这里如果$i\not\in J_0$则取$A_i=\{a_i\}$,如果$i\in J_0$则取$A_s=X_s$.于是按照上一条的结论,每个$C_{J_0}$都是连通空间.而$\{C_{J_0}\}_{J_0\in J}$是一族交非空的连通空间族,于是它们的并$C$是连通空间.最后验证$C$是$\prod_{i\in I}X_i$的稠密子集,于是按照第一条说明了积空间是连通的.
\end{enumerate}

连通性是分析中介值定理的本质.$f$是连通空间$X$到序拓扑空间$Y$的连续函数,对任意$a,b\in X$如果有$Y$中的$r$介于$f(a),f(b)$之间那么$r$在$f(X)$中.事实上假设$r$不在像中,那么$f^{-1}(\{s\in Y\mid s<r\})$和$f^{-1}(\{s\in Y\mid s>r\})$是$X$上不交非空开集,并且并是整个$X$,这和连通性矛盾.

实直线的所有区间(两侧可开可闭)和所有射线(非无穷的一侧侧可开可闭)和全集都是连通子集.事实上这完全刻画的是实直线的所有凸子集,于是我们只需证明实直线的凸子集$Y$都是连通子集.假设凸子集$Y$不连通,那么$Y$可以表示为不交非空的两个在$Y$中开的子集$A,B$的并.取$a\in A$和$b\in B$,按照凸性得到$[a,b]\subset Y$.于是$[a,b]$是非空且不交的在$[a,b]$中开子集$A_0=A\cap[a,b]$和$B_0=B\cap[a,b]$的并.记$[a,b]=[a_0,b_0]$,假设已经构造了闭区间$[a_n,b_n]$,取中点$c_n$,如果它属$A_0$中则记$[c_n,b_n]=[a_{n+1},b_{n+1}]$;如果它属$B_0$中则记$[a_n,c_n]=[a_{n+1},b_{n+1}]$,由此归纳构造闭区间套$[a_n,b_n],n\ge0$.满足$(b_{n+1}-a_{n+1})=\frac{b_n-a_n}{2}$.于是$b_n-a_n=\frac{b_0-a_0}{2^n}$趋于0,按照实直线上的闭区间套定理,存在唯一的点$c_0\in\cap_{n\ge0}[a_n,b_n]$.那么$c_0$必然落在$A_0$和$B_0$之一,不妨设$c_0\in A_0$,由于$A_0$是$[a,b]$中开集,于是存在$c_0$的在$\mathbb{R}$中的开邻域$U$,使得$U\cap[a,b]\subset A_0$.按照$b_n-a_n$趋于0,可取足够大的正整数$N$使得$[a_N,b_N]\subset U$,这导致$b_N\in A_0$,和构造矛盾.

连通分支.给定拓扑空间$X$,定义一个等价关系为,$x\sim y$当且仅当$x$和$y$同时落在$X$的一个连通子空间中.先来说明这的确是一个等价关系.首先,单点子集必然是连通子集说明了自反性.对称性是显然的.最后来说明传递性,假设$x,y$落在同一个连通子集$A$中,$y$和$z$落在同一个连通子集$B$中,那么$A$和$B$是交非空(至少包含点$y$)的连通子集,于是它们的并也是连通子集,而$x,z\in A\cup B$,于是$x\sim z$.

在这个等价关系下,等价类称为空间$X$的连通分支.每个连通分支都是$X$的连通子集.另外按照定义,连通分支恰好就是极大的连通子集,即子集$Y$是连通分支当且仅当对任意包含$Y$的连通子集$Z$,有$Y=Z$.按照连通子集的闭包还是连通子集,说明每个连通分支都是闭集.

但是一般来讲连通分支未必是开集.例如$\mathbb{Q}$作为$\mathbb{R}$的子空间,它的连通分支都是单点,都不是开集.这种连通分支都是单点集的空间称为全不连通空间.连通空间可以理解为连通分支唯一的空间,那么全不连通空间则是另一个极端,它是连通分支极大的空间.离散空间必然是全不连通空间,$\mathbb{Q}$的例子给出了一个非离散空间的全不连通空间.这样的例子还有Cantor集.

给定空间$X$,称$X$上点$x_1$到$x_2$的道路是一个连续映射$f:[a,b]\to X$,使得$f(a)=x_1,f(b)=x_2$.这里$[a,b]$满足$a<b$是实直线的一个有界闭子区间.按照$[a,b],a<b$总同胚于$[0,1]$,通常这个定义里也会直接把$[a,b]$取做$[0,1]$.如果空间$X$上任意两个点都存在道路相连,就称$X$是道路连通空间.

\begin{enumerate}
	\item 道路连通是一种特殊的连通性.假设$X$道路连通,假设有$X=A\cup B$其中$A,B$是非空的分离子集,取$a\in A$和$b\in B$,按照道路连通性,可取连续映射$f:[0,1]\to X$使得$f(0)=a$,$f(1)=b$.我们知道$[0,1]$连通,于是$f([0,1])$连通,于是它落在$A,B$之一,这导致$a,b$落在$A,B$中同一个,这矛盾.
	\item 反过来连通空间未必是道路连通空间,反例见后文拓扑学家的正弦曲线.
	\item 和连通情况不同,给道路连通子集添加聚点未必会保道路连通性,见后文拓扑学家的正弦曲线.
	\item 如果空间的一族道路连通子空间的交非空,那么它们的并自然是道路连通的.
	\item 道路连通空间的满射像也是道路连通的,特别的道路连通性是拓扑不变性.事实上如果$f:X\to Y$是道路连通空间$X$到空间$Y$的满射像,任取$Y$中的点$f(x_1)$和$f(x_2)$,按照道路连通性可取连接$x_1$和$x_2$的连续映射$p:[0,1]\to X$,于是$f\circ p$是连接$f(x_1)$和$f(x_2)$的道路.
	\item 一族道路连通空间的积空间是道路连通的,可以直接把分量道路连通性得到的道路作为分量拼凑为到积空间的道路.
\end{enumerate}

道路连通分支.在空间$X$上定义等价关系为$x\sim y$当且仅当存在$x$到$y$的道路.先来验证它是等价关系,首先自反性是直接的,任取$x\in X$,可直接取$[0,1]\to X$的连续映射为取值为$x$的常值映射;对称性,若存在$x_1\in X$到$x_2\in X$的连续映射$f:[0,1]\to X$,那么$g(x)=f(1-x)$是从$x_2$到$x_1$的连续映射;传递性,如果存在$f_1:[0,1]\to X$为$x_1$到$x_2$的道路,$f_2:[0,1]\to X$为$x_2$到$x_3$的道路,取$f(x):[0,1]\to X$为,当$x\in[0,\frac{1}{2}]$时取$f_1(2x)$,当$x\in[\frac{1}{2},1]$时取$f_2(2x-1)$,按照粘合引理知$f(x)$连续,它就是从$x_1$到$x_3$的道路.

称这个等价关系的等价类为道路连通分支.类似于连通分支,道路连通分支自身是一个道理连通子集,并且它就是极大的道路连通子集.由于道路连通是蕴含连通的,说明每个道路连通分支必然包含于某个连通分支中,并且两种分支都是空间的划分,于是连通分支是若干道路连通分支的无交并.

这里指出,由道路连通性或者连通性可以证明$R$和$R^m,m\ge2$不同胚.因为倘若同胚,那么两个空间均扣去一个点仍然同胚,但是前者不连通/不道路连通,后者连通/道路连通.这矛盾.按照这个思路可以把连通和道路连通推广到高维从而证明当$m\not=n$的时候有$R^m$和$R^n$不同胚,这便是代数拓扑中同伦群和同调群的内容.从拓扑空间到它道路连通分支/连通分支构成的集合是一个函子.它们相当于0维的同伦/同调"群"(0维情况并没有群结构).

局部连通性.称空间$X$是局部连通空间,如果对任意点$x\in X$和它的任意开邻域$U$,存在$x$的连通的开邻域$V$满足$V\subset U$.这个条件等价于讲$X$有由连通开集构成的拓扑基.另外局部连通性还有一个重要等价描述是,空间$X$的任一开集的任一连通分支都是开集.
\begin{proof}
	
	一方面如果$X$是局部连通空间,任取$X$的开集$U$,任取$U$的连通分支$C$,任取$x\in C$,按照定义有连通开子集$C_x$使得$x\in C_x\subset C$,于是$C$中每个点都是内点,于是$C$是开集.另一方面如果任一开集的任一连通分支都是开集,考虑全体连通开子集构成的集族,验证它是拓扑基即可.
\end{proof}
\begin{enumerate}
	\item 按照连通性是拓扑不变性,可说明局部连通性也是拓扑不变性.但是它未必在连续满射下不变,不过局部连通空间的商空间总是局部连通的.这只要注意到商映射$p:X\to Y$保证了如果$C$是$Y$的开子集$U$的连通分支,那么$p^{-1}(C)$是$p^{-1}(U)$的某些连通分支的并.
	\item 按照积空间的拓扑基可以取为分量拓扑基的积,以及连通空间的积是连通的,说明局部连通空间的积是局部连通的.
	\item 局部连通空间的开子集的连通分支都是开集,特别的全集的连通分支都是开集,于是局部连通空间的连通分支都是既开又闭的,于是此时连通分支分解就是空间的拓扑和分解.
	\item 局部连通空间的开子集都是局部连通空间.
\end{enumerate}

类似可定义局部道路连通空间.称空间$X$是局部道路连通空间,如果每个点$x\in X$和每个开邻域$U$,存在$x$的道路连通开邻域$V$使得$V\subset U$.这同样等价于讲$X$有道路连通开集构成的拓扑基.同样等价于讲任一开集的任一道路连通分支都是开集.
\begin{enumerate}
	\item 按照道路连通推出连通,可得局部道路连通空间必然是局部连通空间.
	\item 局部道路连通性是拓扑不变性.事实上在连续开映射下会保局部道路连通性.
	\item 一族局部道路连通空间的积也是局部道路连通的.
	\item 局部道路连通空间的道路连通分支都是开集,另外按照连通分支都是道路连通分支的无交并,可说明局部道路连通条件下,空间的连通分支和道路连通分支是一致的,因为任取一个连通分支$C$,如果它包含不止一个道路连通分支,这导致$C$表示为非空不交开子集的并,和连通性矛盾,特别的局部道路连通条件下连通和道路连通等价.
	\item 局部道路连通空间的开子集都是局部道路连通空间.
\end{enumerate}

尽管一般来讲一个性质理应满足它所对应的局部版本的性质,但是连通和道路连通均不会推出它们各自对应的局部性质.本节最后我们就来整理一些涉及到连通性的反例.
\begin{enumerate}
	\item 连通,非道路连通,非局部道路连通,非局部连通.另外这个例子也说明了道路连通子集的闭包未必是道路连通的.
	
	拓扑学家的正弦曲线.考虑$\mathbb{R}^2$的子集$S=\{(x,\sin\frac{1}{x})\mid 0<x\le1\}$.它的闭包为$\overline{S}=S\cup\{(0,y)\mid -1\le y\le1\}$.
	
	这个闭包不是道路连通空间:假设$\overline{S}$是道路连通的,那么存在从$(0,0)$到$(1,\sin1)$的道路$f:[0,1]\to\mathbb{R}^2$,$f(t)=(x(t),y(t))$,这里$y(t)=\sin\frac{1}{x(t)}$.记$\sup\{t\in[0,1]\mid x(t)=0\}=t_0$,按照$y(t)$在$t=0$处的连续性,有足够小的$\delta>0$,使得$0<t-t_0<\delta$时有$y(t)<\frac{1}{2}$.再由连续性可取足够大的正整数$N$使得$x_1=\frac{1}{2N\pi+\frac{\pi}{2}}$落在$x(t_0,t_0+d)$中,于是由介值性存在$t_1\in(t_0,t_0+\delta)$使得$x(t_1)=x_1$,此时$y(t_1)=\sin\frac{1}{x_1}=1$,这和$y(t)<\frac{1}{2}$矛盾.
	
	这个闭包不是局部连通的:考虑任一点$(0,y),y\not=0$,它足够小的开邻域把$\overline{S}$划分为若干不交的小曲线段,这里每个小曲线段都是这个小开邻域的连通分支,因而闭包不会是局部连通的.
	\item 连通,道路连通,非局部道路连通,非局部连通.
	
	考虑$\mathbb{R}^2$的子集,将点$(0,0)$与$(1,0)$相连,对每个$1/n,n\ge1$,将$(1/n,0)$与$(1/n,1)$相连.
	
	这个空间道路连通,因而连通,但任取它的点$(0,a),a>0$,这个点足够小的开邻域上总有若干两两不交的垂直线段,每个线段都是这个开邻域的连通分支,因而空间非局部道路连通,非局部连通.
	\item 连通,局部连通,非道路连通,非局部道路连通.
	
	赋序方体.考虑$\mathbb{R}^2$的子集$S=\{(a,b)\mid 0\le a,b\le1\}$.在$S$上赋予全序为,$(x,y)<(a,b)$当且仅当,要么$x<a$,要么$x=a$且$y<b$.
	
	由于$S$可作为线性连续统,因而它连通,但是它不道路连通,因为倘若道路连通,则可取$(0,0)$到$(1,1)$的道路,即连续映射$f:[a,b]\to S$.那么介值定理说明$f([a,b])$必然包含$S$的每个点,于是$U_x=f^{-1}(x\times[0,1])$是非空的$[a,b]$的开子集,并且对不同的$x\in(0,1)$两两不交,这导致$[a,b]$中有不可数个两两不交的开集,每个都可以取一个有理数,这和有理数集可数矛盾.
	\item 道路连通,连通,局部连通,非局部道路连通.
	
	这里我们给出两个有趣的例子.第一个例子是锥体构造,事实上将一些性质不好的空间取锥体总会得到一些很特别的例子.记空间$Y$是$\mathbb{R}$上赋予余可数拓扑,取$Y$的锥体,即$X=Y\times[0,1]/\sim$,这里$\sim$定义为$(x_1,t_1)\sim(x_2,t_2)$当且仅当$t_1=t_2=1$.
	
	作为锥体自然是道路连通的,因而是连通的.现在说明局部连通性.首先$Y$作为不可数集,它的任意两个开集的交非空,因而$Y$是连通的,而由于开集总是不可数集,说明$Y$的每个开集都是连通的,这就说明$Y$是局部连通的.接下来$Y$到$X$的构造是先取积再取商,这两个操作都是保局部连通性的,因而$X$是局部连通的.最后说明$X$不是局部道路连通的.这可先证明$Y$的紧致连通子集总是单点集,导致所有$Y$上的道路都是常值映射,于是没有开集是道路连通的,所以$Y$不是局部道路连通的.接下来取定点$(x,t)\in Y\times[0,1)$,任何到$Y\times[0,1)$的道路$f$都可以复合投影映射$Y\times[0,1)$而得到$Y$上的道路$p\circ f$,但是这个复合是常值的,也就是说每个$Y\times[0,1)$上的道路的横坐标固定,于是不好存在从$(x_1,t_1)$到$(x_2,t_2)$的道路,这里$x_1\not=x_2$.于是$X$不是局部道路连通的.
	
	第二个例子是长线,即取第一个不可数序数$\omega_1$,在$\omega_1\times[0,1)$上赋予二元序.它可以理解为$\omega_1$个半开区间$[0,1)$首位相连得到的一条直线.这个空间很神奇,它和实直线的势相同,但是拓扑角度讲它比实直线长很多:长线不是第二可数的,但实直线是.另外极为反常的,如果我们把$\omega_1$取为更大的序数,将不会保证长线是局部欧式空间,尽管定义上讲它依然是很多个半开区间$[0,1)$的衔接.
	
	长线本身是道路连通的,因为任意取两个可数序数$\beta_1,\beta_2$,那么$(\beta_1,t_1)$到$(\beta_2,t_2)$只涉及了可数个$[0,1)$的首位衔接,于是此时衔接得到的线段就和$\mathbb{R}$的闭区间同胚的.这就得到了连接的道路,但是倘若给长线添加$(\omega_1,1)$这个点,此时从$(0,0)$到$(\omega_1,1)$就会包含不可数个区间,因而必然不会存在道路将它们相连.称添加了这个点的长线为闭长线或者紧长线.现在我们构造例子为,在闭长线上,把首位点以一条道路相连.这使得原本不道路连通的闭长线变成道路连通的,于是它也是连通的.局部连通性也是直接的,但是它将不会是局部道路连通的,因为每个可数序数$\beta$,从$(\beta,t_1)$到$(\omega_1,t_2)$总不存在道路.
	\item 局部连通,非道路连通,非连通,非局部道路连通.这个只要取两个相同的第三个例子做无交并(拓扑和)即可.
	\item 局部道路连通,局部连通,非连通,非道路连通.只要取$\mathbb{R}$的子空间$[0,1]\cup[2,3]$.
	\item 最后两个简单的例子:同时满足所有四个连通性的例子有$\mathbb{R}$;不满足全部四个连通性的例子有全不连通空间,例如$\mathbb{Q}$.
\end{enumerate}

最后介绍不可约性,它是一种比较强的连通性.一个拓扑空间称为不可约空间,如果它非空集,并且不能表示为两个真闭子集的并.不可约空间的一些性质:
\begin{enumerate}
	\item 非空空间是不可约的,当且仅当任意两个非空开集的交非空,当且仅当任意非空开集都是稠密集,当且仅当任意真闭子集不包含内点.
	\item 不可约空间的开子集是不可约的.
	\begin{proof}
		
		任取非空开子集$U$,如果有$X$中的闭集$E,F$均不包含$U$,且$U\subset E\cup F$,那么$U^c$和$E\cup F$是闭子集并且覆盖了$X$,按照$X$的不可约性得到至少有一个是$X$自身,按照$U$非空说明只能是$E\cup F=X$,但是这导致$E,F$中至少一个是$X$,这和$E,F$均不覆盖$U$矛盾.
	\end{proof}
	\item 不可约子空间的闭包是不可约的.
	\begin{proof}
		
		设$Y$是$X$的一个不可约子空间,如果存在$X$中两个闭集$E_1,E_2$均不覆盖$\overline{Y}$,那么它们同样均不覆盖$Y$.于是如果$E_1\cup E_2$覆盖$\overline{Y}$,会得到$E_1\cup E_2$覆盖$Y$,这和$Y$的不可约性矛盾.
	\end{proof}
	\item 不可约性可以视为一种比较强的连通性:不可约空间总是连通的和局部连通的.其中局部连通性只需注意的第二条说明所有开邻域都是不可约的.
	\item 如果一个Hausdorff空间至少包含两个点,那么它是可约的.事实上任取两点$x,y\in X$,存在两个点分别的不交的开邻域$U,V$,那么得到$X=(U^c)\cup (V^c)$,这导致空间可约.
	\item 不可约空间的连续像是不可约的.事实上如果$X$是不可约空间,$f(X)$是连续像,假设它可约,那么存在严格包含于$f(X)$中的闭子集$E,F$使得$f(X)=E\cup F$.那么得到$X=f^{-1}(E)\cup f^{-1}(F)$,导致右侧至少一个是整个$X$,不妨设$X=f^{-1}(E)$,导致$E=f(X)$,这矛盾.
	\item 上两条说明,从不可约空间到Hausdorff空间的连续映射只能是常值映射.
	\item 空间的不可约分支指的是不可约子空间在包含序下的极大元,按照第三条说明不可约分支总是闭子集.另外和连通分支不同的是,不同的不可约分支可能是相交的.另外按照不可约空间总是连通的,说明不可约分支总落在某个连通分支中.最后Hausdorff空间上的不可约分支恰好就是任意单点集.
\end{enumerate}
\newpage
\section{紧致性}

称空间$X$是紧致的,或者称为紧空间,如果它的每个开覆盖都有有限的子覆盖.开集闭集是对偶概念,紧致性也可以对偶的利用闭集描述:称一个子集族具有有限交性质,如果它的每个有限子族中的集合的交都是非空的.紧致性即,每一个具有有限交性质的闭集族,它的所有成员的交是非空的.

这里强调两件比较好用的事实.当提及一个紧致子集的时候,原空间对这个子集具有开覆盖有有限子覆盖的性质和它作为子空间拓扑是紧致的是等价的.具体的讲,空间$X$ 的子集$A$按照子空间拓扑是紧致的当且仅当$A$被$X$中的任一开覆盖必有有限的子覆盖;另外有限点集总是紧集.

紧致性和分离性的一些联系:
\begin{enumerate}
	\item 紧致空间上$T_2,T_3,T_4$是等价的.
	\begin{proof}
		
		设$X$是紧致的$T_2$空间,先来证明$T_3$,对紧致空间中任意一个闭集$A$和一个不在$A$ 中的点$x$,按照空间是$T_2$的,对每个$a\in A$可以取$a$的开邻域$U_a$和$x$的开邻域$V_a$不交,那么按照紧集的闭子集紧,从$U_a,a\in A$中可以取出有限个$U_{a_i},1\le i\le r$覆盖整个$A$,考虑$U=\cup_{1\le i\le r}U_{a_i}$,和$V=\cap_{1\le i\le r}V_{a}$,它们分别是$A$和$x$的开邻域并且不交,这就得到了正则性.$T_1$是由$T_2$直接推出的.类似的可从$T_3$推出$T_4$.
	\end{proof}
	\item 紧致空间的闭子集紧致.
	\begin{proof}
	
	取紧致空间$X$的闭子集$A$,取$A$的一个开覆盖$\{U_i,i\in I\}$,那么$\{U_i,i\in I\}\cup\{X-A\}$是全空间的开覆盖,于是可取有限子覆盖$\{U_1,\cdots,U_n,X-A\}$,由于$X-A$和$A$不交,于是$\{U_1,\cdots,U_n\}$必然是$A$的有限覆盖,于是$A$是紧致的.
	\end{proof}
    \item $T_2$空间的任一紧致子集是闭集.
    \begin{proof}
    
    取Hausdorff空间$X$的紧子集$Y$,需要证明$X-Y$是开集,为此任取$x\in X-Y$,对每个$y\in Y$,可取不交的$x,y$各自的开邻域$U_y,V_y$.那么$\{V_y,y\in Y\}$构成了$Y$的一个开覆盖,按照条件它有有限子覆盖$\{V_1,\cdots,V_n\}$,于是此时$U=U_1\cap U_2\cap\cdots\cap U_n$和$V=V_1\cup V_2\cup\cdots\cup V_n$是$x$和$Y$分别的不交的开邻域.于是$U\in X-Y$,于是$X-Y$是开集.
    \end{proof}
    \item 上两条说明,紧致$T_2$空间中闭集等价于紧致子集.
    \item 紧子集自身具备很好的分离性质:$T_2$空间中的两个不交紧子集总是可以被不交开集所分离.
    \begin{proof}
    	
    	在第三条的证明中,我们看到对Hausdorff空间$X$,给定一个点$x$和不含它的紧集$K$,存在不交开集将它们分离.现在任取$X$的两个不交紧集$A,B$,对每个$a\in A$,存在$a$和$B$分别的不交的开邻域$U_a,V_a$.那么$\{U_a,a\in A\}$构成了$A$的开覆盖,可取有限子覆盖$\{U_1,\cdots,U_n\}$,最后取$U=\cup U_i$和$V=\cap V_i$,那么它们分别是$A$和$B$的开邻域,并且它们不交.
    \end{proof}
\end{enumerate}

紧空间的连续像是紧空间.另外这一事实可以说明紧集到Hausdorff空间的连续双射总是同胚.
\begin{proof}
	
	设$f:X\to Y$是连续映射,这里$X$是紧空间,任取$f(X)$在$Y$中的一个开覆盖$\{U_i,i\in I\}$,那么$\{f^{-1}(U_i),i\in I\}$构成了$X$的开覆盖,取有限子覆盖$\{f^{-1}(U_1),\cdots,f^{-1}(U_n)\}$,于是得到$\{U_1,\cdots,U_n\}$是$f(X)$的有限覆盖.
	
	第二个命题,由于$f:X\to Y$已经是连续双射,为说明它是同胚只需说明它是闭映射,为此任取$X$的闭集$A$,它是紧空间的闭子集,于是是紧集,按照紧集的连续像紧,于是$f(A)$是$Y$的紧子集,按照Hausdorff空间的紧子集是闭集,就得到了$f$是闭映射.
\end{proof}

紧致性的传递性我们已经给出了保闭子集(但不保一般子空间),保连续像,最后紧性是保任意积的.为此我们先来证明有限积的情况,这需要先证明管状引理:$Y$紧致空间,若$N$为$X\times Y$中包含截片$\{x_0\}\times Y$ 的开集,那么$N$必包含含有
$\{x_0\}\times Y$的一个"管子"$W\times Y$,其中$W$为$x_0$在$X$中的一个开邻域.
\begin{proof}
	
	由于$\{x_0\}\times Y$同胚于$Y$,于是它是紧集,于是可以用有限个包含于$N$的基元素$\{U_i\times V_i,i=1,2,\cdots,n\}$覆盖.那么只要取$W=\cap_{1\le i\le n}U_i$即满足$\{x_0\}\times Y\subset U\times Y\subset N$.
\end{proof}

有限个紧空间的积是紧致的.
\begin{proof}
	
	只需证明两个紧空间的积是紧空间,然后归纳即可.设$X,Y$是紧空间,取$X\times Y$的开覆盖$\{A_i,i\in I\}$.给定$x_0\in X$,那么截片$\{x_0\}\times Y$是紧子集,于是可取有限个$\{A_i\}$中的元将其覆盖,不妨设为$\{A_1,\cdots,A_m\}$.那么开集$N=\cup_{1\le i\le m}A_i$是$\{x_0\}\times Y$的开邻域,按照管状引理,$N$包含了"管子"$W\times Y$,这里$W$是$x_0$在$X$中的开邻域.
	
	于是对每个$x\in X$,可取它在$X$中的开邻域$W_x$,使得$W_x\times Y$被$\{A_i\}$中有限个开集覆盖.$\{W_x,x\in X\}$构成了$X$的开覆盖,按照紧性可取有限子覆盖$\{W_1,W_2,\cdots,W_k\}$.于是有限个管子$W_i\times Y,1\le i\le k$覆盖了整个积空间$X\times Y$,这里每个管子都被有限个$\{A_i\}$中的开集覆盖,于是整个积空间被有限个$\{A_i\}$中的开集覆盖.于是积空间是紧集.
\end{proof}

事实上紧空间的任意积都是紧致的,这称为Tychonoff定理:给定一族紧空间$\{X_{\alpha},\alpha\in I$,那么在积拓扑下$X=\prod_{\alpha\in I}X_{\alpha}$是紧致空间.
\begin{proof}
	
我们的思路是运用紧性的对偶描述,有限交性质.考虑二元积的情况.给定紧空间$X_1,X_2$,考虑$X_1\times X_2$上满足有限交性质的闭集族$\{A_i,i\in I\}$.那么$\{\pi_1(A_i)\mid i\in I\}$是$X_1$中的满足有限交性质的子集族,但它未必是闭集族,不过子集族的有限交性质可以推出闭包构成的子集族$\{\overline{\pi_1(A_1)},i\in I\}$满足有限交性质,按照$X_1$的紧性,有$\cap_{i\in I}\overline{\pi_1(A_i)}$非空,取一个元$x_1$,同理可取$\cap_{i\in I}\overline{\pi_2(A_2)}$的元$x_2$.一个合理的现象是$(x_1,x_2)\in\cap_{i\in I}A_i$.这就完成了证明.

但是遗憾的是这不是总成立的,本质上讲我们对$x_1$和$x_2$做了过少的限制,会导致一些不好的点被这样选中.为此,我们的思路是扩大给定的满足有限交性质的闭集族,使得上述操作取到的点符合要求.

给定空间$X$上的满足有限交性质的子集族$\mathscr{A}$,验证Zorn引理的条件可得到包含$\mathscr{A}$的$X$的子集族$\mathscr{D}$,并且它在包含序下满足极大性.另外$\textbf{D}$作为极大元还满足如下性质:
\begin{enumerate}
	\item $\textbf{D}$中有限个元素的交也在$\textbf{D}$中.
	\item 和$\textbf{D}$中任一元的交非空的$X$的子集必然在$\textbf{D}$中.
\end{enumerate}

现在来证明原定理.记$X=\prod_{\alpha\in I}X_{\alpha}$,取$X$中一族满足有限交性质的子集族$\mathscr{A}$,只需证明$\cap_{A\in\mathscr{A}}\overline{A}$非空,就说明$X$是紧空间.为此找到包含$\mathscr{A}$的极大的满足有限交性质的子集族$\mathscr{D}$,于是现在只需说明$\cap_{D\in\mathscr{D}}\overline{D}$是非空的.

对任一$\alpha\in I$,有$\{\overline{\pi_{\alpha}(D)},D\in\textbf{A}\}$是$X_{\alpha}$上的具有有限交性质的闭集族,取$x_{\alpha}
\in\cap_{D\in\textbf{A}}\overline{\pi_{\alpha}(D)}$.记$\textbf{x}=(x_{\alpha})$,下证$\textbf{x}\in\cap_{D\in\textbf{A}}\overline{D}$.对任意$D\in\textbf{A}$
证明每个子基元素$\pi_{\beta}^{-1}\left(u_{\beta}\right)$和$D$有交,从而任一子基元素在$\textbf{B}$,从而含$\textbf{x}$的任一基元素在$\textbf{B}$中它们和每一个$D\in\textbf{A}$有交,于是$\textbf{x}\in\overline{D},\forall D\in\textbf{A}$.
\end{proof}

现在介绍几个新的紧致性.称空间是可数紧致的,如果每个可数覆盖都存在有限子覆盖.称空间是序列紧致的,如果任一序列存在收敛子列.称空间是点集紧致,如果任一无穷点集存在聚点.

各种紧致性之间的联系:
\begin{enumerate}
	\item 可数紧致等价于空间中任意非空递减闭集列交非空.
	\item 紧致蕴含可数紧致蕴含点集紧致.
	\begin{proof}
		
		紧致蕴含可数紧致是直接的.假设$X$是可数紧致空间,取$X$的无穷点集$A$,假设$A$没有聚点,那么聚点集是空集,$A$包含了自身聚点集,于是$A$是闭集.并且对任意的$a\in A$,可取开邻域$U_a$满足$U_a\cap A=\{a\}$.按照$A$是无穷集,可抽$A$的可数子集$\{a_n,n\ge1\}$,这里$a_n$两两不同,记$V_m=\cup_{a\in A,a\not=a_n,n\ge1}U_a\cup U_{a_m}$,那么$\{V_m,m\ge1\}$和$X-A$是$X$的开覆盖,抽有限子覆盖,导致$\{a_n,n\ge1\}$是一个有限集,矛盾.
	\end{proof}
	\item 可数紧致的林德洛夫空间是紧致的.
	\item 序列紧致蕴含可数紧致蕴含点集紧致.
	\begin{proof}
		
		只需证明前一个蕴含.设$X$是序列紧致空间,取$X$中的一列非空递减的闭集列$\{F_n,nn\ge1\}$,在每个$F_n$中取一个点$x_n$,考虑序列$\{x_n,n\ge1\}$,按照条件有子列$\{x_{n_i},i\ge1\}$收敛,设收敛于$y$.按照$F_n$都是闭集,得到$y\in\cap_{n\ge1}F_n$,于是$X$是可数紧致的.
	\end{proof}
	\item 可数紧致的$A_1$空间是序列紧致的.
	\begin{proof}
		
		设$X$是满足$A_1$的可数紧致空间,取$X$的一个序列$\{x_n,n\ge1\}$.对每个$n\ge1$,记$E_n=\{x_i,i\ge n\}$,记$E_n$闭包为$F_n$,那么$\{F_n,n\ge1\}$是一列非空递减的闭集,于是可取$x\in\cap_{n\ge1}F_n$.
		
		按照空间是第一可数的,可取$x$处的规范可数局部基$\{U_i,i\ge1\}$,即满足$U_1\supset U_2\supset\cdots$.对每个$i,j\ge1$,总有$U_j\cap E_i$非空.取最小的指标$j$使得$x_j\in U_1\cap E_1$为$N_1$,如果已经构造$N_{i-1}$,设$N_i$是最小的指标$j$使得$x_j\in U_i\cap E_{N_{i-1}+1}$.那么此时$\{N_1,N_2,\cdots\}$是严格递增的正整数序列,并且$x_{N_i}\in U_i,\forall i\ge1$.
		
		我们断言$\{x_{N_i}\}$是收敛子列,它收敛于$x$.任取$x$的开邻域$U$,那么存在正整数$k$使得$U_{N_k}\subset U$,于是当$i>k$的时候有$x_{N_i}\in U_i\subset U_k\subset U$.
	\end{proof}
	\item 点集紧致的$T_1$空间是可数紧致的.
	\begin{proof}
		
	设$X$是满足$T_1$的点集紧致空集,假设$X$不是可数紧致空间,则存在递减的非空闭集列$\{F_n,n\ge1\}$使得交为空集.在每个$F_n$中去一个点$x_n$,考虑集合$A=\{x_1,x_2,\cdots\}$.如果$A$是有限集,那么必然存在严格递增的正整数子列$\{n_1,n_2,\cdots\}$使得$x=x_{n_1}=x_{n_2}=\cdots$.于是$x\in\cap_{n\ge1}F_n$,这和交为空集矛盾.现在设$A$是无穷集,那么它有聚点$y$.按照$X$是$T_1$空间,$y$的每个开邻域都有$F_n$中的点,于是$y\in F_n,\forall n\ge1$,于是$y\in\cap_{n\ge1}F_n$,矛盾.
	\end{proof}
\end{enumerate}

总结一下,各紧致性的关系为:
$$\xymatrix{\text{紧致}\ar@<0.3ex>[r]&\text{可数紧致}\ar@<0.3ex>[d]^{A_1}\ar@<0.3ex>[l]^{Lindelof}\ar@<0.3ex>[r]&\text{点集紧致}\ar@<0.3ex>[l]^{T_1}\\
&\text{序列紧致}\ar@<0.3ex>[u]&
}$$

我们接下来讨论度量空间中的紧致性.在度量空间中,上述四种紧致性两两等价.
\begin{proof}
	
	结合我们已证的内容,接下来只需证明点集紧致推出序列紧致,序列紧致推出紧致.先来证点集紧致推出序列紧致.设$X$是点集紧致的度量空间,取点列$\{x_n,n\ge1\}$,记这个点列构成的集合为$A$.如果$A$是有限集,那么存在一个点$x$满足有无穷个$n$使得$x_n=x$.此时这样的$x_n$就构成了一个收敛子列.假设$A$是无穷集,那么$A$有聚点$x$,现在构造收敛到$x$的子列:先取$n_1$使得$x_{n_1}\in B_1(x)$.假设已经构成了$x_{n_{i-1}}$,由于$B_{1/i}(x)$和$A$中无穷个点相交,于是可其指标$n_i>n_{i-1}$,使得$x_{n_i}\in B_{1/i}(x)$.那么序列$\{x_{n_i}\}$收敛于$x$.
	
	现在来证明序列紧致推出紧致.首先我们证明如果度量空间$X$满足序列紧致性,那么$X$满足Lebesgue引理:给定$X$的开覆盖$\mathscr{A}$,存在正数$\varepsilon$,称为关于这个开覆盖的Lebesgue数,使得对任意直径小于$\varepsilon$的$X$的子集$A$,它必然被开覆盖中的某个开集所覆盖.假设对开覆盖$\mathscr{A}$,不存在Lebesgue数,则对每个正整数$n$,存在一个子集$C_n$,它的直径小于$\frac{1}{n}$,但是不被任何一个$\mathscr{A}$中的开集覆盖.取$x_n\in C_n$,按照条件知序列$\{x_n\}$存在收敛子列$\{x_{n_i}\}$,记收敛为$a$.那么$a$被$\mathscr{A}$中的某个开集$U$所覆盖.那么可以取一个足够小的$\varepsilon>0$,使得$B_{\varepsilon}(a)\subset A$.取足够大的$i$使得$1/n_i<\varepsilon/2$,于是此时$C_{n_i}$落在$x_{n_i}$为中心半径为$\varepsilon/2$的开圆内.如果把$i$取足够大使得$d(x_{n_i},a)<\varepsilon/2$,那么$C_{n_i}$会落在$a$为中心半径为$\varepsilon$的开圆内,这导致$C_{n_i}\subset A$,和假设矛盾.
	
	第二步,证明空间是完全有界的,即对每个$\varepsilon>0$,存在有限个半径为$\varepsilon$的开圆覆盖整个$X$.假设这不成立,设存在$\varepsilon_0>0$,使得$X$不能被有限个$\varepsilon_0$半径的开圆覆盖.先任取点$x_1\in X$,假设已经构造了$\{x_1,x_2,\cdots,x_n\}$.那么$\cup_{1\le i\le n}B_{\varepsilon_0}(x_i)$不会是整个空间$X$,取一个点$x_{n+1}\in X$不在这个并中.据此我们构造了一个点列$\{x_n,n\ge1\}$,满足只要$m\not=n$就有$d(x_m,x_n)\ge\varepsilon_0$.那么$\{x_n,n\ge1\}$没有收敛子列.
	
	最后来证明$X$是紧致空间.任取一个开覆盖$\mathscr{A}$,取Lebesgue数$\delta>0$.取$\varepsilon=\delta/3$.按照第二步,可取有限的半径$\varepsilon$的开圆覆盖整个空间,其中每个开圆的直径为$2\delta/3$,于是对每个开圆存在$\mathscr{A}$中的开集覆盖了它,这就导致$\mathscr{A}$有有限子覆盖.
\end{proof}

上述证明还说明了紧致度量空间上有Lebesgue数引理:给定$X$的开覆盖$\mathscr{A}$,存在正数$\varepsilon$,称为关于这个开覆盖的Lebesgue数,使得对任意直径小于$\varepsilon$的$X$的子集$A$,它必然被开覆盖中的某个开集所覆盖.

紧致性是分析中最值定理和Cantor定理的本质.
\begin{enumerate}
	\item 紧致空间到序拓扑空间的连续映射会取到最值.
	\begin{proof}
		
		设$f:X\to Y$是命题中的连续映射,那么$f(X)$是$Y$的紧子集.只需证明$f(X)$有最大元和最小元.以最大元为例,假设$f(X)$没有最大元,那么$\{(-\infty,a)\mid a\in f(X)\}$是$f(X)$的开覆盖,可取有限子覆盖,于是存在一个$(-\infty,a)$覆盖了整个$f(X)$,但是这和$a\in f(X)$矛盾.
	\end{proof}
	\item  $Cantor$定理:从紧致度量空间到度量空间的连续函数是一致连续的.这里称度量空间之间的映射$f:X\to Y$一致连续是指,对任意的$\varepsilon>0$,存在$\delta>0$,使得只要$d_X(x_1,x_2)<\delta$,就有$d_Y(f(x_1),f(x_2))<\varepsilon$.那么一致连续蕴含连续.
	\begin{proof}
		
		给定正数$\varepsilon>0$,取$Y$的开覆盖$\{B_{\varepsilon/2}(y)\mid y\in Y\}$,记这些开球的原像构成的$X$的开覆盖为$\mathscr{A}$,取$\mathscr{A}$的Lebesgue数$\delta$,那么如果$x_1,x_2\in X$满足$d_X(x_1,x_2)<\delta$,那么$X$的子集$\{x_1,x_2\}$的直径小于$\delta$,于是$Y$的子集$\{f(x_1),f(x_2)\}$落在某个开球$B_{\varepsilon/2}(y)$中,于是$d_Y(f(x_1),f(x_2))<\varepsilon$.
	\end{proof}
\end{enumerate}

度量空间的紧子集的结构.度量空间的紧子集必然是有界闭集,但反之不真.
\begin{proof}
	
	$T_2$空间的紧子集自然是闭集.至于有界性,注意到对固定的$x\in X$,有$\{B_n(x),n\ge1\}$是紧子集的开覆盖,于是有有限子覆盖,取其中最大的$n$,于是得到紧子集是有界的.
\end{proof}

现在来给出紧致性的实例.设$X$是具有上确界性质的序拓扑空间,则任一闭区间是紧致的,特别的$\mathbb{R}$上任一闭区间紧致.
\begin{proof}
	
	给定$X$中的元$a<b$,考虑闭区间$[a,b]$,设有开覆盖$\mathscr{A}$.需要证明存在有限子覆盖.第一步,我们证明如果$x\in[a,b]$并且$x\not=b$,那么存在$y>x,y\in[a,b]$使得$[x,y]$被$\mathscr{A}$中至多两个开集覆盖.倘若$x$有后继元$y$,那么$[x,y]=\{x,y\}$,于是至多两个$\mathscr{A}$中的开集覆盖$[x,y]$.倘若$x$没有后继元,选取$\mathscr{A}$中的开集$U$覆盖点$x$,由于$x\not=b$且$U$是开集,于是$U$包含了一个半开区间$[x,c)$,其中$c\in[a,b]$,取$y\in(x,c)$,那么$[x,y]$被这个开集$U$覆盖.
	
	第二步,设$C$为全体$(a,b]$中的点$y$构成的集合,使得$[a,y]$被$\mathscr{A}$中有限个开集覆盖,在第一步中取$y=a$,我们看到$C$非空.记$c=\sup C$,那么$c\in(a,b]$.我们来证明$c\in C$.即$[a,c]$可以被$\mathscr{A}$中有限个开集覆盖.取$\mathscr{A}$中的开集$U$覆盖点$c$,于是存在半开区间$(d,c]\subset[a,b]\cap U$.如果$c$不在$C$中,那么必然有$(d,c)\cap C$非空,否则$d$是一个更小的$C$的上界.取$(d,c)\cap C$中的元$z$,那么$[a,z]$被$\mathscr{A}$中有限个开集覆盖,再算上覆盖了$[z,c]$的开集$U$,这就说明$c\in C$.
	
	第三步,证明$c=b$.假设$c<b$,对$x=c$运用第一步,可取$b\ge y>c$,使得$[c,y]$可被有限个$\mathscr{A}$中的开集覆盖,于是$[a,y]=[a,c]\cup[c,y]$可以被$\mathscr{A}$中有限个开集覆盖,这和$\inf C=c$矛盾.
\end{proof}

另外我们可以证明欧式空间$\mathbb{R}^n$上的紧集恰好就是有界闭集.为此只需证明闭矩体是紧集,然后利用有界闭集总是闭矩体的闭子集,和紧集的闭子集紧.这个定理被称为有限覆盖定理,它是互相等价的七个实数系基本定理之一.证明需要从实数的构造入手.

为了描述一般度量空间的紧致性,我们还需要一个新概念.称度量空间$(X,d)$是完全有界的,如果对任意的$\varepsilon>0$.事实上这个概念在证明序列紧致度量空间紧致时出现过.一个度量空间$(X,d)$是紧致的当且仅当它完备并且完全有界.
\begin{proof}
	
	设$X$是紧致的度量空间,那么$X$的任意柯西序列有收敛子列,于是$X$是完备的.对每个正实数$\varepsilon>0$,考虑开覆盖$\{B_{\varepsilon}(x),x\in X\}$,它有有限子覆盖说明$X$是完全有界的.
	
	现在设$X$是完备且完全有界的度量空间.我们只要说明$X$是序列紧致的,就得到$X$的紧致性.取序列$\{x_n,n\ge1\}$,先将$X$用有限个半径为1的开球覆盖,于是至少存在一个开球$B_1$覆盖了无穷个$\{x_n\}$中的点,记这些指标为$\mathbb{Z}$的子集$J_1$.现在假设构造了$\mathbb{Z}$的子集$J_k$,取$J_k$的子集$J_{k+1}$为存在半径为$\frac{1}{k+1}$的开球$B_{k+1}$覆盖了无穷个$J_k$中指标的$\{x_n,n\ge1\}$的点.
	
	取$n_1\in J_1$,如果构造了$n_k$,取$n_{k+1}\in J_{k+1}$使得$n_{k+1}>n_k$.那么当$i,j>k$的时候,点$x_{n_i}$和$x_{n_j}$同时落在$B_k$中,这说明$(x_{n_i})$是柯西列,于是有收敛子列.
\end{proof}

局部紧致性.存在很多不同的对局部紧致性的定义:
\begin{enumerate}
	\item 每个点存在一个紧致的邻域.
	\item 每个点存在闭包紧致的局部基.
	\item 每个点存在紧致邻域构成的局部基.
\end{enumerate}

这三个定义的一些联系和区别:
\begin{enumerate}
	\item 定义2和3总能推出定义1,反之一般不成立.另外定义2和3之间一般不能互相推出.紧性可以推出定义1和2,但是一般不能推出定义3.这里给出一个紧致空间不满足定义3的例子.考虑$\mathbb{Q}\cup\{e\}$,其中$\mathbb{Q}$的拓扑定义为常规拓扑,而自然对数$e$的开邻域定义为余有限拓扑,即满足$U-\mathbb{Q}$是有限集的包含$e$的子集$U$.那么这个空间是紧集,但是任意有理数不存在包含于开集$\mathbb{Q}$的紧邻域.
	\item 在定义1和Hausdorff条件下,空间满足这样一个强于$T_3$,弱于$T_4$的条件:对任意紧集$K$和一个开邻域$D$,存在闭包紧的开集$E$,满足$K\subset E\subset\overline{E}\subset D$.
	\begin{proof}
		
		先来证明$K$是单点集的情况.取点$x$的紧致邻域$N$,那么$N$是紧致Hausdorff空间,于是它是正规的,考虑$N$的闭子集$A=\{x\}$和$B-N-D$,那么存在不交的$N$中的开集$U,V$分别包含了$A$和$B$.于是存在$X$中的开集$U_0$和$V_0$满足$U=U_0\cap N$和$V=V_0\cap N$.取$E=\mathrm{Int}(U)$,那么$E\subset U\subset N$,于是有$\overline{E}\subset\overline{N}=N$,于是$E\cap V_0\subset U\cap V_0=N\cap U_0\cap V_0=\varnothing$.于是$E\subset X-V_0$.按照$X-V_0$是闭集,于是有$\overline{E}\subset X-V_0$.最后注意到$\overline{E}\subset N\cap(X-V_0)\subset N\cap(X-(X-D))=N\cap D\subset D$.
		
		对于一般的紧集$K$,对每个点$x\in K$可取上一段中的开邻域$E(x)$满足$\overline{E(x)}$是紧集,并且有$x\in E(x)\subset\overline{E(x)}\subset D$.于是得到$K\subset\cup_{x\in K}E(x)$,按照紧性可取有限个点$x_1,x_2,\cdots,x_n\in K$满足$K\subset\cup_{1\le i\le n}E(x_i)$,记后者是$E$,这是$K$的开邻域,并且满足$K\subset E\subset\overline{E}\subset\cup_{1\le i\le n}\overline{E(x_i)}\subset D$.
	\end{proof}
	\item 
	 在Hausdorff条件下三个定义等价.于是当提及"局部紧致$T_2$空间"时定义是明确的.另外上一段告诉我们局部紧致$T_2$空间是$T_3$的.
	 \begin{proof}
	 	
	 	3推2和2推1都是直接的,于是我们只需说明1推3.设$X$是满足定义1的Hausdorff空间.我们断言全体$x$的紧致邻域构成了局部基.任取$x$的开邻域$U$,取$x$的一个紧致邻域$D$,那么$U\cap\mathrm{Int}(D)$是$x$的开邻域,由于$X$是$T_3$的,于是存在$x$的开邻域$V$使得$\overline{V}\subset U\cap\mathrm{Int}(D)$.这里闭集$\overline{V}$作为紧致空间$D$的闭子集,是紧致的,这就找到了一个落在$D$中的$x$的紧致邻域.
	 \end{proof}
\end{enumerate}

局部紧致Hausdorff空间上的Urysohn引理:设$X$是局部紧致的Hausdorff空间,设$K,F$是$X$的子集,前者是紧集,后者是闭集,那么存在连续映射$f:X\to[0,1]$满足$f(K)=\{1\}$和$f(F)=\{0\}$.于是局部紧致的Hausdorff空间实际上是$T_{3.5}$的.
\begin{proof}
	
	$X-F$是包含$K$的开集,于是可取$K$的开邻域$E$满足$\overline{E}$是紧集并且$K\subset E\subset\overline{E}\subset X-F$.再考虑$K$的开邻域$E$,可取开邻域$G$满足$\overline{G}$是紧集并且$K\subset G\subset\overline{G}\subset E$.限制考虑紧空间$\overline{E}$,它是hausdorff空间,于是它是正规空间,按照Urysohn引理,得到连续映射$\overline{E}\to[0,1]$满足$g(K)=\{0\}$和$g(\overline{E}-G)=\{0\}$.现在构造全空间$X$上的连续映射$f:X\to[0,1]$为在$\overline{E}$上的限制是$g$,在$X-\overline{E}$上恒取0.那么$f$在开集$E$上和$g$一致,于是是$E$上的连续映射,另外$f$在$A=X-\overline{G}$上恒取0,于是也是开集$A$上的连续映射,按照$X=A\cup E$,由粘合引理就说明$f$是整个$X$上的连续映射,并且满足$f(K)=\{1\}$和$f(F)=\{0\}$.
\end{proof}

在点集拓扑中存在两种性质良好的空间,度量空间和紧致Hausdorff空间.一个自然的想法是把空间嵌入到性质良好的空间中.对于前者,嵌入到度量空间的空间也就是可度量化空间,于是处理这个嵌入问题等价于寻找可度量化定理.我们已经给出过可分度量空间的等价描述,在下一节会给出更一般的可度量化定理;对于后者,能嵌入到紧致Hausdorff空间未必是紧致的,一个自然的问题是什么样的空间可以嵌入到紧致Hausdorff空间.

称空间$X$的紧致化,是指一个紧致Hausdorff空间$Y$,它以$X$为子空间,并且满足$\overline{X}=Y$.空间$X$的两个紧致化$Y_1,Y_2$是等价的,如果存在同胚$h:Y_1\to Y_2$使得在$X$上的限制为$h(x)\equiv x$.

最简单的紧致化是单点紧致化,即空间添加一个新的点,并且约定这个点的邻域构成的空间是紧致Hausdorff空间.可以单点紧致化恰好就是局部紧致的等价描述:$X$是局部紧致的Hausdorff空间当且仅当存在紧致Hausdorff空间$Y$使得它以$X$为子空间,$Y-X$是单点集.另外局部紧致空间的任意两个单点紧致化总是等价的紧致化.对于本身已经紧致的Hausdorff空间,它的单点紧致化是添加了一个孤立点(即单点开集).如果在单点紧致化的定义中要求$X$在$Y$中的闭包是$Y$,那么会排除$X$自身已经紧致的情况.
\begin{proof}
	
	第一步,先证明单点紧致化的唯一性.设$Y$和$Y'$都是局部紧致的Hausdorff空间$X$的单点紧致化,定义$h:Y\to Y'$为在$X$上取$h(x)=x$的唯一映射,即把唯一的$Y-X$中的点$p$映射为唯一的$Y'-X$中的点$p'$.设$U$为$Y$中的开集,需要验证$h(U)$是$Y'$中的开集,这样由对称性就得到$h$是同胚.先设$U$不含点$p$,那么$h(U)=U$,由于$X$在$Y'$中是开集,于是$h(U)$是$Y'$中的开集.现在设$U$包含点$p$,那么$C=Y-U$是$Y$中闭集,于是它是$Y$的紧子集,也是$X$的紧子集,于是$f(C)=C$也是$Y'$的紧子集,于是它是$Y'$的闭集,于是$Y'-C=h(U)$是$Y'$的开集.
	
	第二步,设$X$是局部紧致的Hausdorff空间,我们来构造单点紧致化$Y$.取一个不在$X$中的点$\infty$,记$Y=X\cup\{\infty\}$.构造$Y$的拓扑为,$X$的部分拓扑不变,点$\infty$的开邻域定义为,全体$Y-C$形式的集合,这里$C$是$X$的紧子集.称$X$自身的开集为第一型的开集,称$Y-C$形式的开集为第二型的开集.全集和空集自然是$Y$的开集.验证有限交:$U_1\cap U_2$是第一型的开集;$(Y-C_1)\cap(Y-C_2)=Y-(C_1\cup C_2)$是第二型的开集;$U_1\cap(Y-C_1)=U_1\cap(X-C_1)$是第一型的开集.再验证任意并:$\cup_i U_i$是第一型的开集;$\cup(Y-C_i)=Y-(\cap_iC_i)$是第二型的开集;$U\cup(Y-C)=Y-(C-U)$,这里$C-U$是$C$的闭子集,于是是紧集,于是这是第二型的开集.综上$Y$上定义了拓扑.
	
	这一段验证$X$可作为$Y$的子空间,为此需要说明$Y$中开集和$X$的交是$X$中开集,以及对每个$X$中开集都可以写作$Y$中开集和$X$的交.如果$U$是$Y$中第一型开集,那么$U\cap X=U$是$X$中开集;如果$Y-C$是$Y$中第二型开集,那么$(Y-C)\cap X=X-C$是$X$中的开集.反过来$X$中每个开集都是$Y$中的第一型开集.于是$X$是$Y$的子空间.
	
	这一段说明$Y$是紧致的,为此任取$Y$的开覆盖$\mathscr{A}$.那么$\mathscr{A}$必然至少包含一个第二型的开集$Y-C$,现在取$\mathscr{A}$中全部和$Y-C$不同的开集,将它们和$X$相交,这得到$X$的开覆盖,特别的它是$C$的开覆盖,由$C$的紧性只这可取有限子覆盖,取和$X$相交之前对应的开集,算上$Y-C$,就得到了$\mathscr{A}$的有限子覆盖.
	
	这一段说明$Y$是Hausdorff空间.给定$Y$中两个不同的点$x,y$,如果它们同属$X$,那么按照$X$是Hausdorff的,以及$Y$的第一型开集就是$X$的开集,于是存在不交开集将两个点分离.现在设其中一个点是$\infty$,不妨设$y=\infty$,按照$X$是局部紧致的,可取点$x$的紧致邻域$C$,那么$C$包含了$x$的一个开邻域$U$,于是$Y$中的开集$U$和$Y-C$是不交的并且分别包含点$x$和点$\infty$.
	
	第三步,证明逆命题.设$Y$是紧致的Hausdroff空间,$X$是$Y$的子空间,满足$Y-X$是单点集.那么$X$是Hausdorff空间.最后说明$X$是局部紧致空间,任取$x\in X$,取不交的开集$U,V$分别包含点$x$和点$Y-X$,那么$C=Y-V$是$Y$的闭子集,于是它是$Y$的紧子集,而$C\subset X$,于是$C$也是$X$的紧子集,并且包含了$X$的开邻域$U$.即$X$是局部紧致的.
\end{proof}

局部紧致的Hausdorff空间$X$的开子空间和闭子空间都是局部紧致的Hausdorff空间.
\begin{proof}
	
	设$A$是$X$的闭子集,任取$x\in A$,那么有$x$在$X$中的紧邻域$C$,记$x$的开邻域$U$包含于$C$.那么$C\cap A$是$C$的闭子集,于是是紧集,它包含了$x$在$A$中的开邻域$U\cap A$,于是$x$在$A$中有紧致邻域.
	
	设$A$是$X$的开子集,任取$x\in A$,那么存在$x$的开邻域$V$满足$\overline{V}$是紧集,并且$\overline{V}\subset A$.那么$\overline{V}$是$x$在$A$中的紧致邻域.
\end{proof}

上述定理说明紧致Hausdorff空间的开子集是局部紧致的,另一方面单点紧张化说明局部紧致的Hausdorff空间同胚于一个紧致Hausdorff空间的开子集.于是得到:$X$是局部紧致的Hausdorff空间当且仅当它同胚于一个紧致Hausdorff空间的开子集.

诺特性是比较强的一种紧致性.一个拓扑空间称为诺特的,如果每个闭集的降链总会终止,即如果有闭集列$Y_0\supset Y_1\supset\cdots$,那么存在一个正整数$N$使得$n\ge N$时候有$Y_n=Y_N$.诺特空间的一些性质:
\begin{enumerate}
	\item 首先是链条件的若干等价描述.空间是诺特的当且仅当全体闭集满足降链条件;当且仅当全体闭集满足极小条件,即任意闭集构成的非空集合存在包含序下的极小元;当且仅当全体开集满足升链条件;当且仅当全体开集满足极大条件.
	\item 诺特条件具有传递性,即诺特空间的子空间仍然是诺特的.
	\item 诺特条件可以视为一种比较强的紧致性:诺特空间总是紧致的,并且事实上空间是诺特的当且仅当每个子空间都是紧致的.事实上任取子集$Y$的开覆盖$\mathscr{U}$,先取一个非空开集$U_1$,如果取定了$U_1,U_2,\cdots,U_n$,如果它们不是有限子覆盖,那么可取$\mathscr{U}$中的一个开集$U_{n+1}\not\subset\cup_{1\le i\le n}U_i=V_n$,考虑开集升链$\{V_n\}$,按照升链条件就说明它在有限步后终止,于是存在有限子覆盖.
	\item 如果$X$是诺特空间,$Y\subset X$是闭子集,那么$Y$可以表示为有限个不可约闭子集的并$Y=Y_1\cup Y_2\cup\cdots\cup Y_r$.如果约定$i\not=j$时有$Y_i\not\subset Y_j$,那么这样的分解是唯一的.
	\begin{proof}
		
		先来证明分解的存在性.假设$S$是由$X$的不能做这样分解的闭子集构成的集合,需要证明的就是$S$是空集.如果$S$非空,可取一个闭子集$Y_0\in S$.那么$Y_0$必然不是不可分空间,否则它自身已经构成了这样一个分解.于是存在$Y_0$的两个真闭子集$Y_0'$和$Y_0''$,它们的并就是$Y_0$.于是$Y_0'$和$Y_0''$中至少存在一个属于$S$中,否则它们可以表示为自身真闭子集的并,取两个表示的并就得到$Y_0\not\in S$矛盾.记属于$S$中的一个为$Y_1$,对$Y_1$做相同的事情,这就得到一个闭集降链$Y_0\supsetneqq Y_1\supsetneqq$,这和诺特性矛盾.说明$S$是空集.
		
		接下来说明唯一性.假设存在$Y=\cup_{1\le i\le r}Y_i=\cup_{1\le j\le s}Y_i'$.并且对任意$i\not=i'$和$j\not=j'$,有$Y_i\not\subset Y_{i'}$和$Y_j'\not\subset Y_{j'}'$.于是有$Y_1'\subset\cup_i Y_i$.我们断言$Y_1'$恰好落在某个$Y_i$中,若否,设某个$Y_t$不包含在$E=Y_{j\not=1}Y_j'$中,注意$E$和$Y_1'$都是闭集,导致$E\cap Y_t$和$Y_1'\cap Y_t$是$Y_t$的真闭子集,它们的并是整个$Y_t$,这就和$Y_t$的不可约性矛盾.不妨设$Y_1'$落在$Y_t$中,同理得到$Y_t$落在某个$Y_i'$中,于是$Y_1'\subset Y_i'$,按照条件只能有$i=1$,于是$Y_1'=Y_t$.不妨重新排号使得$t=1$,于是$Y_1=Y_1'$.最后为了继续对$\min\{r,s\}$归纳,只需注意到$\cup_{2\le i\le r}Y_i=\overline{Y-Y_1}=\overline{Y-Y_1'}=\cup_{2\le j\le s}Y_j'$是闭集.这就完成证明.
	\end{proof}
	\item 一个诺特空间是Hausdorff空间当且仅当它是赋予了离散拓扑的有限点集.
	\begin{proof}
		
		只需说明必要性,设$X$是条件中的空间,按照不可约子空间分解,可记$X=\cup_iY_i$,其中每个$Y_i$是不可约的闭子集,并且任一不包含另一个.我们证明过至少包含两个点的Hausdorff空间必然是可约的,这说明每个$Y_i$是单点集.于是$X$就是有限点集上的离散拓扑.
	\end{proof}
\end{enumerate}
\newpage
\section{可度量化定理}

空间$X$的一个子集族$\mathscr{A}$称为局部有限的,如果对每个$X$中的点$x$,存在一个开邻域只和$\mathscr{A}$中有限个集合相交.先给出局部有限性的一些基本结论:
\begin{enumerate}
	\item 满足局部有限性的子集族$\mathscr{A}$的任意子族都是局部有限的.
	\item 设$\mathscr{A}$满足局部有限性,那么$\mathscr{B}=\{\overline{A},A\in\mathscr{A}\}$也满足局部有限性.
	\begin{proof}
		
		任取点$x$,设开邻域$U$满足至多和$\mathscr{A}$中有限个子集相交.那么$U$只会和这些子集的闭包可能相交,于是$U$至多和$\mathscr{B}$中有限个子集相交.
	\end{proof}
	\item 若$\mathscr{A}$满足局部有限性,那么有$\overline{\cup_{A\in\mathscr{A}}A}=\cup_{A\in\mathscr{A}}\overline{A}$.
	\begin{proof}
		
		我们知道一般空间中右侧包含于左侧.记$Y=\cup_{A\in\mathscr{A}}A$,需要证明$\overline{Y}\subset\cup\overline{A}$.设$x\in\overline{Y}$,取$x$的开邻域$U$至多和$\mathscr{A}$中有限个子集有交,不妨设为$A_1,\cdots,A_k$.我们断言$x$落在某个$\overline{A_i}$中,否则有$U-\cup_{1\le i\le k}\overline{A_i}$是$x$的开邻域,它不和任意$\mathscr{A}$中有交,这导致$x$不会在$\overline{Y}$中.
	\end{proof}
\end{enumerate}

称子集族是$\sigma-$局部有限性的,如果它是可数个局部有限族的并.称子集族的加细是一个子集族,使得新的子集族中每个元素,都存在原本子集族中的子集包含它,特别的如果加细中的子集都是开集或闭集,就称为开加细或闭加细.

设$X$是度量空间,如果$\mathscr{A}$是$X$的开覆盖,那么存在$\mathscr{A}$的开加细覆盖$\mathscr{B}$是$\sigma-$局部有限的.
\begin{proof}
	
	赋予$\mathscr{A}$良序$<$,记$X$上的度量为$d$.对每个正整数$n$,对每个开覆盖中的开集$U$,记$S_n(U)=\{x\mid B_{1/n}(x)\subset U\}$,几何上看就是把开集$U$缩小$1/n$的边界,这是闭集.记$T_n(U)=S_n(U)-\cup_{V<U,V\in\mathscr{A}}V$,这是闭集.我们断言对$V\not=W\in\mathscr{A}$,只要$x\in T_n(V)$和$y\in T_n(W)$,就有$d(x,y)\ge\frac{1}{n}$.不妨设$V<W$,因为$x\in T_n(V)$,于是$x\in S_n(V)$,于是$B_{1/n}(x)$落在$V$中.但是按照$V<W$,就有$y$不在$V$中,于是$y$不在$B_{1/n}(x)$中.
	
	现在记$E_n(U)=\cup_{x\in T_n(U)}B_{1/3n}(x)$.那么$E_n(U)$是开集.记$\mathscr{C}_n=\{E_n(U)\mid U\in\mathscr{A}\}$.我们断言$\mathscr{C}_n$是局部有限的.因为对每个点$x\in X$,$B_{1/6n}(x)$至多和一个$E_n(U)$相交.最后记$\mathscr{C}=\cup_{n\ge1}\mathscr{C}_n$,那么这是一个$\sigma$局部有限的$\mathscr{A}$的开加细.最后我们说明它覆盖全空间.任取$x\in X$,取$U\in\mathscr{A}$为最小的包含$x$的$\mathscr{A}$中的开集,那么可取足够大的正整数$n$使得$B_{1/n}(x)\subset U$.于是有$x\in S_n(U)$,于是$x\in E_n(U)$.
\end{proof}

设$X$是正则空间,有$\sigma-$局部有限的拓扑基$\mathscr{B}$,那么$X$是正规的,并且每个$X$中的闭集都是一个$G_{\delta}$集.这里$G_{\delta}$集指的是可数个开集的交.
\begin{proof}
	
	第一步,任取开集$W$,我们证明存在可数个开集$\{U_n,n\ge1\}$满足$W=\cup U_n=\cup\overline{U_n}$.可记$\mathscr{B}=\cup_{n\ge1}\mathscr{B}_n$,其中每个$\mathscr{B}_n$是局部有限的.记$\mathscr{C}_n=\{B\in\mathscr{B}_n\mid\overline{B}\subset W\}$.那么作为局部有限集族的子族,$\mathscr{C}_n$是局部有限的.再记$U_n=\cup_{B\in\mathscr{C}_n}B$,那么$U_n$是开集,并且有$\overline{U_n}=\cup_{B\in\mathscr{C}_n}\overline{B}$.于是$\overline{U_n}\subset W$.于是$\cup_{n\ge1}U_n\subset\cup_{n\ge1}\overline{U_n}\subset W$.现在断言这里三个式子是相等的.任取$x\in W$,那么存在基元素$B$满足$x\in B\subset\overline{B}\subset W$.可设$B\in\mathscr{B}_n$,那么按照定义有$B\in\mathscr{C}_n$,于是有$x\in U_n$,于是$W\subset\cup U_n$.
	
	第二步,证明每个闭集都是$G_{\delta}$集.给定闭集$C$,记$W=X-C$,那么第一步说明有开集列$\{U_n\}$满足$W=\cup\overline{U_n}$,于是$C=\cap(X-\overline{U_n})$.于是$C$是$G_{\delta}$集.
	
	第三步,证明$X$是正规空间.任取不交的闭集$C,D$.第一步说明$X-D$可表示为$\cup U_n=\cup\overline{U_n}$.那么$\{U_n\}$是$C$的开覆盖.并且每个$\overline{U_n}$和$D$不交.同理,存在$D$的可数开覆盖$\{V_n\}$每个开集的闭包和$C$不交.接下来的证明类似于$A_2$的$T_3$空间是正规的.记$U_n'=U_n-\cup_{1\le i\le n}\overline{V_i}$和$V_n'=V_n-\cup_{1\le i\le n}\overline{U_i}$.那么$U'=\cup U_n'$和$V'=\cup V_n'$是不交的开集,分别包含$C$和$D$.
\end{proof}

引理.若$X$是正规空间,$A$是一个$G_{\delta}$的闭集.那么存在连续映射$f:X\to[0,1]$满足$f(A)=\{0\}$且$x\not\in A$时$f(x)>0$.
\begin{proof}
	
	记$A=\cap U_n$,其中每个$U_n$都是开集.对每个正整数$n$,选取连续映射$f_n:X\to[0,1]$使得$f(A)=\{0\}$且$f(x)=1,x\in X-U_n$.定义$f(x)=\sum\frac{f_n(x)}{2^n}$,这个级数一致收敛,于是连续,它满足要求.
\end{proof}

Nagata-Smirnov度量化定理.一个空间$X$是可度量化空间当且仅当它是正则的并且存在$\sigma-$局部有限的拓扑基.
\begin{proof}
	
	第一步,证明充分性.设$X$是正则空间,并且有$\sigma$局部有限的拓扑基$\mathscr{B}$.那么我们已经证明了$X$是正规的,并且每个闭集都是$G_{\delta}$集.我们通过把$X$嵌入到$(\mathbb{R}^J,\rho)$来说明它可度量化.记$\mathscr{B}=\cup\mathscr{B}_n$,其中每个$\mathscr{B}_n$是局部有限的集族.对每个正整数$n$和每个$B\in\mathscr{B}_n$,定义$f_{n,B}:X\to[0,\frac{1}{n}]$为引理中满足$f_{n,B}(B)>0$和$f_{n,B}(X-B)=0$的连续映射.连续映射族$\{f_{n,B}\}$满足分离$X$中的点:给定点$x_0$和开邻域$U$,存在基元素$B$满足$x_0\in B\subset U$.设$B\in\mathscr{B}_n$,那么有$f_{n,B}(x_0)>0$,且$f_{n,B}(X-B)=0$.
	
	现在设$J$为$\mathbb{Z}_+\times\mathscr{B}$的子集,是全体$(n,B)$,满足$B\in\mathscr{B}_n$.定义映射$F:X\to[0,1]^J$为,$F(x)=(f_{n,B}(x))_{(n,B)\in J}$.如果$[0,1]^J$赋予积拓扑,按照嵌入定理,有$F$是连续映射.现在赋予$[0,1]^J$一致度量$\rho$,需要证明此时嵌入$F$是连续映射.由于一致度量诱导的拓扑比积拓扑细,于是$F$把$X$的开集映射为$F(X)$中的开集.任取$x_0\in X$和正数$\varepsilon$.需要找到$x_0$的开邻域$W$满足$x\in W$时有$\rho(F(x),F(x_0))<\varepsilon$.取$x_0$的开邻域$U_n$使得它只和$\mathscr{B}_n$中有限个子集有交.这就是说当$B$跑遍$\mathscr{B}_n$的时候,只有有限个$f_{n,B}(U_n)$不恒取0.又因为每个$f_{n,B}$连续,于是可取$x_0$的开邻域$V_n\subset U_n$使得$f_{n,B}(V_n)$上度量的变化至多为$\frac{\varepsilon}{2}$.现在取足够大的$N$使得$\frac{1}{N}\le\frac{\varepsilon}{2}$.取$W=V_1\cap V_2\cap\cdots\cap V_N$.我们断言$W$是所求的开邻域.任取$x\in W$,若$n\le N$,那么有$|f_{n,B}(x)-f_{n,B}(x_0)|\le\frac{\varepsilon}{2}$.若$n>N$,那么有$|f_{n,B}(x)-f_{n,B}(x_0)|\le\frac{1}{n}<\frac{\varepsilon}{2}$.综上得到$\rho(F(x),F(x_0))\le\frac{\varepsilon}{2}<\varepsilon$.
	
	第二步,证明必要性.假设$X$是可度量化空间,那么$X$是正则的.现在需要说明$X$有$\sigma-$局部有限的拓扑基.对每个正整数$m$,记$\mathscr{A}_m$为全部半径为$\frac{1}{m}$的开球构成的集族.那么我们知道存在$X$的开覆盖$\mathscr{B}_m$加细了$\mathscr{A}_m$并且是局部有限的.$\mathscr{B}_m$中每个开集的直径至多是$\frac{2}{m}$.记$\mathscr{B}=\cup_{n\ge1}\mathscr{B}_m$,我们来说明$\mathscr{B}$是拓扑基.任取$x\in X$和$\varepsilon>0$,需要找$\mathscr{B}$中的元$B$满足$x\in B\subset B_{\varepsilon}(x)$.为此只要取足够大的正整数$m$使得$\frac{1}{m}<\frac{\varepsilon}{2}$.再取覆盖了$x$的$\mathscr{B}_m$中的开集$B$,它就满足要求.
\end{proof}

为了描述第二个可度量化定理,先来介绍仿紧致性.首先,紧致性具有如下等价描述:空间是紧致的,当且仅当任一开覆盖有有限的开加细覆盖.一方面如果空间紧致,那么任意开覆盖取有限子覆盖,就是一个有限的开加细覆盖.另一方面如果空间满足这个条件,任取开覆盖,取有限的开加细覆盖,那么对这有限个开集中的每一个,存在原开覆盖中的某个开集包含它,这就取到有限子覆盖.

称空间是仿紧致的,如果空间上任一开覆盖有局部有限的开加细覆盖.我们先来说明一个熟悉的例子,$X=\mathbb{R}^n$是仿紧致空间.
\begin{proof}
	
   任取$X$的开覆盖$\mathscr{A}$.取$B_m=B_m(0)$.对每个$m$,取$\mathscr{A}$中有限个开集覆盖$\overline{B_m}$.把这些开集每个都和$X-\overline{B_{m-1}}$相交,由此得到的开集族记作$\mathscr{C}_m$.于是$\mathscr{C}=\cup\mathscr{C}_m$是$\mathscr{A}$的开加细.它是局部有限的,因为对每个开集$B_m$至多和$\mathscr{C}$中有限个开集相交.最后说明$\mathscr{C}$覆盖了整个空间.因为对每个$x\in X$,可取$m$为最小的正整数,使得$x\in\overline{B_m}$,那么$x$会被$\mathscr{C}_m$中某个开集覆盖.
\end{proof}

若$X$是正则空间,那么如下结论是等价的:
\begin{enumerate}
	\item 空间的每个开覆盖都有$\sigma$局部有限的开加细覆盖.
	\item 空间的每个开覆盖都有局部有限的加细覆盖(未必开).
	\item 空间的每个开覆盖都有局部有限的闭加细覆盖.
	\item 空间的每个开覆盖都有局部有限的开加细覆盖.
\end{enumerate}
\begin{proof}
	
	4推1是直接的.先来证明1推2.设$\mathscr{A}$是$X$的开覆盖,取$\sigma$局部有限的开加细覆盖$\mathscr{B}=\cup\mathscr{B}_n$.这里每个$\mathscr{B}_n$是局部有限的.记$V_i=\cup_{U\in\mathscr{B}_i}U$.再对每个$\mathscr{B}_n$中的$U$定义$S_n(U)=U-\cup_{i<n}V_i$.记$\mathscr{C}_n=\{S_n(U)\mid U\in\mathscr{B}_n\}$.记$\mathscr{C}=\cup\mathscr{C}_n$.我们断言它是局部有限的加细覆盖.任取$x\in X$,记$N$是最小的正整数使得$x$被$\mathscr{B}_N$中的元覆盖,设$U$是这样的元.由于$x$不被$\mathscr{B}_i,i<N$中的开集覆盖,于是$x\in S_N(U)$,于是$\mathscr{C}$覆盖了$x$.另外按照$\mathscr{B}_n$总是局部有限的,于是可取$n=1,2,\cdots,N$时的开邻域$W_n$使得只和$\mathscr{B}_n$中有限个元相交.如果$W_n$和$S_n(V)$有交,那么它必然和$V$有交,按照$S_n(V)\subset V$,于是$W_n$只和有限个$\mathscr{C}_n$中的集合相交.又因为$U$在$\mathscr{B}_N$中,于是$U$不和满足$n>N$的$\mathscr{C}_n$中的集合相交,于是$W_1\cap\cdots\cap W_N\cap U$是$x$的只和$\mathscr{C}$中有限个集合相交的开邻域.
	
	这一段说明2推3.任取开覆盖$\mathscr{A}$,设$\mathscr{B}$是全体这样的开集$U$构成的集族,使得$\overline{U}$落在$\mathscr{A}$中某个开集中.那么按照正则性,$\mathscr{B}$覆盖整个空间.可取$\mathscr{B}$的局部有限的加细覆盖$\mathscr{C}$.最后取$\mathscr{D}=\{\overline{C}\mid C\in\mathscr{C}\}$,这就是一个局部有限的闭加细覆盖.
	
	这一段说明3推4.任取开覆盖$\mathscr{A}$,取局部有限的加细覆盖$\mathscr{B}$(这里没有用到是闭加细覆盖).对每个$x\in X$,可取开邻域只和$\mathscr{B}$中有限个集合相交.于是全体只和$\mathscr{B}$中有限个集合相交的开集构成了$X$的一个开覆盖.可取这个开覆盖的局部有限的闭加细覆盖$\mathscr{C}$.那么$\mathscr{C}$中的每个闭集至多只和$\mathscr{B}$中有限个集合相交.对每个$B\in\mathscr{B}$,记$\mathscr{C}(B)$为集族$\{C\in\mathscr{C}\mid C\subset X-B\}$.再定义$E(B)=X-\cup_{C\in\mathscr{C}(B)}C$.由于$\mathscr{C}$是局部有限的,于是$E(B)$是一个开集,并且有$B\subset E(B)$.对每个$B\in\mathscr{B}$,取$\mathscr{A}$中的开集$F(B)$包含了$B$,最后定义$\mathscr{D}=\{D(B)=E(B)\cap F(B)\mid B\in\mathscr{B}\}$.于是$\mathscr{D}$是$\mathscr{A}$的开加细.并且由于$\mathscr{B}$覆盖全空间,$B\subset D(B)$,于是$\mathscr{D}$覆盖全空间.最后证明$\mathscr{D}$是局部有限的.任取点$x\in X$,取开邻域$W$只和$\mathscr{C}$中有限个集合相交,设相交的集合为$C_1,C_2,\cdots,C_k$.由于$\mathscr{C}$是全空间的开覆盖,于是$W$被$C_1,C_2,\cdots,C_k$覆盖.于是只需证明每个$\mathscr{C}$中的$C$只和有限个$\mathscr{D}$中的开集相交.如果$C$和$D(B)$相交,那么$C$和$E(B)$相交,按照定义它不含于$X-B$,于是$C$和$B$相交.而$C$只和有限个$\mathscr{B}$中的集合相交,这就完成了证明.
\end{proof}

仿紧致性的性质:
\begin{enumerate}
	\item 仿紧致的Hausdorff空间是正规的.
	\begin{proof}
		
		先来证明正则性.任取点$a\in X$和不交的闭集$B$.对每个$b\in B$,可取$b$的开邻域$U_b$的闭包不包含点$a$.那么全体$U_b$和开集$X-B$构成了全空间的开覆盖,于是可取局部有限的开加细覆盖$\mathscr{C}$.取子族为那些和$B$相交的集合构成的集族$\mathscr{D}$.那么$\mathscr{D}$覆盖了$B$.并且对每个$D\in\mathscr{D}$,有$\overline{D}$和$a$不交.由于$D$和$B$有交,说明$D$落在某个$U_b$中.记$V=\cup_{D\in\mathscr{D}}D$.那么$V$是包含$B$的开集,又因为$\mathscr{D}$是局部有限的,于是$\overline{V}=\cup_{D\in\mathscr{D}}\overline{D}$.于是$\overline{V}$和$a$不交,正则性得证.
		
		从正则性到正规性的证明是类似的,只要把$a$替换为闭集$A$,把Hausdorff条件替换为正则性.
	\end{proof}
	\item 仿紧致空间的闭子空间是仿紧致的.
	\begin{proof}
		
		设$Y$是仿紧致空间$X$的闭子集,取$Y$中的开覆盖$\mathscr{A}$,对每个$A\in\mathscr{A}$,有$X$中开集$A'$满足$A'\cap Y=A$.那么全部$A'$和$X-Y$构成了$X$的开覆盖.于是可取局部有限的开加细覆盖$\mathscr{B}$.最后取$C=\{B\cap Y\mid B\in\mathscr{B}\}$,它就是局部有限的$\mathscr{A}$的开加细覆盖.
	\end{proof}
	\item 度量空间总是仿紧致的.
	\begin{proof}
		
		设$X$是度量空间,那么我们已经知道每个开覆盖都有$\sigma$局部有限的开加细覆盖.这在正则条件下等价于每个开覆盖都有局部有限的开加细覆盖,于是空间是仿紧致的.
	\end{proof}
	\item 正则的林德洛夫空间是仿紧致空间.这个结论可以提供仿紧致空间的二元积不仿紧致的例子,即Sorgenfrey平面.
	\begin{proof}
		
		若$X$是正则的林德洛夫空间,任一开覆盖$\mathscr{A}$有可数子覆盖.这个可数子覆盖自动是$\sigma$局部有限集,于是每个开覆盖都有局部有限的开加细覆盖,也就是仿紧致空间.
	\end{proof}
\end{enumerate}

称空间是局部可度量化空间,如果每个点有开邻域作为开子空间是可度量化空间.Smirnov可度量化定理:空间是可度量化空间当且仅当它是局部可度量化空间,并且是仿紧致Hausdorff空间.
\begin{proof}
	
	如果空间是可度量化的,那么它自然是局部可度量化的,另外我们证明过度量空间是仿紧致的.这就完成了必要性的证明.反过来,设$X$是局部可度量化的仿紧致Hausdorff空间,那么$X$是正则的,于是按照上一个可度量化定理,只需验证$X$有$\sigma$局部有限的拓扑基.
	
	用可度量化的开集覆盖全空间,于是可取局部有限的开加细覆盖$\mathscr{C}$.那么对每个$\mathscr{C}$中的开集$C$都是可度量化空间,于是可取$d_C:C\times C\to\mathbb{R}$为度量.对每个$x\in C$,取$B_C(x,\varepsilon)$为全体满足$d_C(x,y)<\varepsilon$的$C$中的元构成的集合.于是$B_C(x,\varepsilon)$是$X$中的开集.
	
	现在对每个正整数$m$,取$\mathscr{A}_m=\{B_C(x,\frac{1}{m})\mid x\in C,C\in\mathscr{C}\}$.这是全空间的开覆盖.于是可取局部有限的开加细覆盖$\mathscr{D}_m$.记$\mathscr{D}=\cup\mathscr{D}_m$,于是$\mathscr{D}$是$\sigma$局部有限的开集族.
	
	最后我们证明$\mathscr{D}$是拓扑基.任取$x\in X$和开邻域$U$,那么$x$只属于有限个$\mathscr{C}$中的开集,不妨设为$C_1,\cdots,C_k$.那么$U\cap C_i$是$x$的在$C_i$中的开邻域,于是存在$\varepsilon_i>0$使得$B_{C_i}(x,\varepsilon_i)\subset U\cap C_i$.选取足够大的$m$使得$\frac{2}{m}<\min\{\varepsilon_1,\cdots,\varepsilon_k\}$.由于$\mathscr{D}_m$覆盖全空间,于是必然存在一个元$D$覆盖点$x$.由于$\mathscr{D}_m$加细了$\mathscr{A}_m$,于是存在$\mathscr{A}_m$中的元$B_C(y,\frac{1}{m})$包含$D$.从$x\in D\subset B_C(y,\frac{1}{m})$说明$C$覆盖$x$,于是$C$必然等于某个$C_t$,由于$B_C(y,\frac{1}{m})$局部度量下直径为$\frac{2}{m}<\varepsilon_t$,于是$x\in D\subset B_{C_i}(x,\varepsilon_i)\subset U$.完成证明.
\end{proof}