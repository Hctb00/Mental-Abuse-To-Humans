\chapter{域论}
\section{代数扩张}
\subsection{基本概念}

首先,域上每个非0元是单位,这意味着域上没有非平凡的理想,只有0理想和单位理想.这说明从域到一个环(特别的,到另一个域)的映射的核只能是这两个理想中的一个,前者对应单射,后者对应恒取0元的映射,此时只能是映射到0环.把0环的情况从域中划掉,于是,域之间的非平凡映射只能是单射.即域范畴Fid中的态射都是单射.那么每一个域同态给出了一个子域/扩张的关系,即如果$f:k\to K$是一个域同态,那么$k$是$K$的子域,也会称$K$是$k$的扩张.研究域之间的态射就等价于研究域扩张.

关于扩张的第一个不变量是域的特征.我们知道整环上特征只能是0或者素数,特征是最小的正整数$n$使得$n$个单位元相加为0,于是域的扩张是保特征的.特征不同的域之间不存在态射,这使得可以把域范畴划分为$\mathrm{Fld}_0,\mathrm{Fld}_p$,其中$p$是任意一个素数.这些子范畴都存在初对象,$\mathrm{Fld}_0$的初对象是$\mathbb{Q}$,而$\mathrm{Fld}_p$的初对象是$p$阶域$\mathbb{F}_p$.按照域范畴中特有的性质"态射都是单射",得出它们是一个域所包含的最小的域,基于这个性质,把这些作为初对象的域称为素子域.

域范畴上不满足Cantor-Schroeder-Bernstein定理.换句话说如果存在$E$到$F$的扩张和$F$到$E$的扩张,可能出现$E,F$互相不同构的情况.这等价于构造域扩张$K\subseteq E\subseteq F$使得$K$和$F$同构但是$E$不和它们同构.取$\mathbb{C}$在$\mathbb{Q}$上的可数的代数无关集$\{t_0,t_1,\cdots\}$,取$\mathbb{Q}(t_1,\cdots)$在$\mathbb{C}$中的代数闭包为$K$,取$E=K(t_0)$,取$F$为$\mathbb{Q}(t_0,t_1,\cdots)$在$\mathbb{C}$中的代数闭包.那么有$K\subseteq E\subseteq F$.这里$K$和$F$同构是直接的,因为$\mathbb{Q}(t_1,\cdots)$和$\mathbb{Q}(t_0,\cdots)$是同构的,于是代数闭包也是同构的.现在需要验证不存在$K$到$E$的同构,假设有这样的同构$\psi$.记$t=\psi^{-1}(t_0)$,那么按照$K$是代数闭的,得到$P(x)=x^2-t$在$K$中有根,把根记作$\omega$,记$\omega_0=\psi(\omega)\in E$,于是从$\omega^2=t$作用$\psi$得到$\omega_0^2=t_0$.但是由于$\omega_0\in E$,说明存在两个$K$系数的非零多项式$P,Q$,满足$\omega_0=\frac{P(t_0)}{Q(t_0)}$,导致$P(t_0)^2=t_0Q(t_0)^2$,考虑两边$t_0$的次数就矛盾.

域扩张具有线性空间结构.如果$k\subseteq K$是一个域扩张,则$K$能够作为$k$上的线性空间,并且它和$K$自身的乘法结构吻合,于是$K$是一个$k$交换代数.按照线性空间理论,可以引出一个概念:称扩张的维数就是把$K$视为$k$ 线性空间的维数,记作$[K:k]$.于是如果有域扩张$k\subseteq K\subseteq F$,则扩张的维数满足有$[F:k]=[F:K][K:k]$.如果扩张维数有限,就称有限扩张,否则称为无限扩张.

关于有限型代数和有限生成代数.称$R$是域$k$上有限生成代数,如果$R$作为$k$线性空间是有限维的,称$R$是域$k$上有限型代数,如果$R$作为$k$代数是有限生成的,这等价于讲$R$是$k$上有限未定元的多项式环的商.那么有限生成代数必然是有限型代数.但是反过来一般不成立,例如$k[x]$是$k$ 上有限型代数但是不是有限维线性空间.但是,对于域扩张这是等价的,即对于域扩张$k\subseteq F$,有$F$作为$k$的有限维线性空间等价于$F$作为$k$的有限型交换代数.这称为Zariski引理,它可以视为诺特正规化引理的特例.由它可以推出希尔伯特零点定理.

这里再强调下作为基域的有限生成代数和下面将要给出的作为基域的有限生成扩张是两个概念.后者是指$k(a_1,a_2,\cdots,a_n)$的形式,而有限生成$k$代数指多项式环的商$k[x_1,x_2,\cdots,x_n]/I$.它也可以表示为$k[a_1,a_2,\cdots,a_n]$,其中$a_i=x_i+I$.当要求有限生成代数是域的时候,蕴含了每个$k[a_1,a_2,\cdots,a_n]$中的多项式的乘法逆依然是一个多项式.这个蕴含的条件使得实际上所有$a_i$都是代数元,导致扩张是有限扩张.即Zariski引理.

现在给定域扩张$k\subseteq F$,如果$X$是$F$的一个子集,那么定义$X$在$k$上生成的环$k[X]$为,$F$的全体包含$k$和$X$的子环中的最小元,这等价于全体$F$的包含$k$和$X$的子环的交.同理定义$X$在$k$上生成的域$k(X)$为全体$F$的包含$k$和$X$的子域的最小元,这等价于全体$F$的包含$k$和$X$的子域的交.按照$k[X]$和$k(X)$第二个等价描述,知它们总是存在的.如果$X$是有限集$\{a_1,a_2,\cdots,a_n\}$,直接写作$k[X]=k[a_1,a_2,\cdots,a_n],k(X)=k(a_1,a_2,\cdots,a_n)$.特别的,如果$k$的域扩张$K$可以表示为$K=k(a_1,\cdots,a_n)$,其中$a_i\in K$,则称$K$是$k$的有限生成扩张.更特别的,如果$k$的域扩张$K$可以表示为$k(a)$,其中$a\in K$,则称$K$为$k$的单扩张.我们将在后文中看到有限(维)扩张必然是有限生成扩张.

如果$F$是$k$的域扩张,对于$a_1,\cdots,a_n\in K$,必然有如下元素表示.这个证明只要先验证它们的确是环或者域,然后验证每个包含$k$和$X$的环或者域必然包含这些元.
$$k[a_1,a_2,\cdots,a_n]=\{f(a_1,\cdots,a_n)\mid f\in k[x_1,\cdots,x_n]\}$$
$$k(a_1,a_2,\cdots,a_n)=\{\frac{f(a_1,\cdots,a_n)}{g(a_1,\cdots,a_n)}\mid f,g\in k[x_1,\cdots,x_n];g(a_1,\cdots,a_n)\not=0\}$$

倘若$X$是$F$的无限子集,记$X$的全体有限子集构成的集合为$S$,那么可验证$k[X]=\cup_{X_0\in S}k[X_0]$和$k(X)=\cup_{X_0\in S}k(X_0)$.这说明尽管$X$是无穷集合,但是每涉及到$k[X]$和$k(X)$单个元的时候,它们只包含了$X$中有限个元.此时有:
$$k[X]=\{f(a_1,\cdots,a_n)\mid n\in\mathbb{N}^+;n\le|X|;f\in k[x_1,\cdots,x_n];a_i\in X\}$$
$$k(X)=\{\frac{f(a_1,\cdots,a_n)}{g(a_1,\cdots,a_n)}\mid n\in\mathbb{N}^+;n\le|X|;f,g\in k[x_1,\cdots,x_n];g(a_1,\cdots,a_n)\not=0;a_i\in X\}$$

代数元和超越元.现在考虑某个域扩张$k\subseteq F$,其中$F$中的元$\alpha$在$k$上生成的单扩张$k(\alpha)$,存在从多项式环$k[x]$到$k(\alpha)$的环同态$\phi$,即把$f(x)\in k[x]$映射为$f(\alpha)$.这个映射的核是一个主理想$(p(x))$,此时需要分两种情况,要么它是零理想,要么是非零理想.
\begin{enumerate}
	\item 如果它是零理想,那么$\phi$是单射,于是$1,\alpha,\alpha^2,\cdots$是$k$线性无关的,这说明单扩张是无限维扩张,此时称$\alpha$为$k$上的超越元.此时$k(\alpha)$同构于有理函数域$k(x)$.并且此时$k[x]$是$k(x)$的真子环.
	\item 如果主理想$(p(x))$不是零理想,注意这里的核就是满足$f(\alpha)=0$的全体多项式构成的理想.按照$k[x]$是欧氏整环,得到这个核可以表示为$(p(x))$,其中$p(x)$是理想中首一的次数最小的非零多项式,这样的多项式是唯一的,否则做差会得到一个次数更小的非零多项式.另外这里的$p(x)$是不可约多项式,否则做分解会和次数最小性相矛盾.于是此时$\frac{k[x]}{(p(x))}$是一个域,其中$x+p(x)$就是的$\alpha$,但是按照定义$k(\alpha)$是包含$\alpha$的最小域,这说明映射是满射,即$\frac{k[x]}{(p(x))}=k(\alpha)$.并且此时单扩张的维数恰好就是多项式$p(x)$的次数$r$,$\{1,\alpha,\cdots,\alpha^{r-1}\}$就是一组基.此时称$\alpha$是$k$上的代数元,并且称$p(x)$是$\alpha$的极小多项式.注意此时$k[\alpha]=k(\alpha)$,因为后者中的分式总可以用Bezout定理转化为一个整式,或者还可以,按照$k[\alpha]\subseteq k(\alpha)$但是二者作为$k$上线性空间维数相同,于是它们是相同的集合.
\end{enumerate}

整理一下,对于域扩张$k\subseteq F$,取$F$中的元$\alpha$.如果$k\subseteq k(\alpha)$是有限扩张,就称$\alpha$是代数元.如果$k\subseteq k(\alpha)$是无限扩张,称$\alpha$是超越元.如果扩张$k\subseteq F$满足$F$中每个元都是代数元,则称扩张是代数扩张,如果哪怕$F$中存在一个超越元,则称扩张是超越扩张.

给定扩张$k\subseteq F$,取代数元$\alpha\in F$,那么这个元诱导了$k$线性空间$F$上的线性变换,即左乘$\alpha$.由于$\alpha$的极小多项式是不可约的,并且零化这个线性变换,于是它实际上就是这个线性变换的极小多项式.我们会在迹和范数一节回到这一内容.

另外,代数元的讨论中可以说明,给定域$k$上的不可约多项式$f(x)$,那么可以构造$k$的有限扩张$F=k[x]/(f(x))$,使得$f(x)$在$F$中至少包含一个根.这一事实会在分裂域中用到.

对于两个有限单扩张,如果相应代数元的极小多项式存在对应,这个对应可以延拓到单扩张之间.具体的讲,给定两个有限单扩张$k_i\subseteq F_i=k_i(\alpha_i),i=1,2$,记$p_i(x)\in k_i[x]$是$\alpha_i$的极小多项式,$i=1,2$,那么如果存在一个$k_1$到$k_2$的同态$l$满足$l(f_1(x))=f_2(x)$,那么它可以唯一延拓为$F_1\to F_2$的同态使得$l(\alpha_1)=\alpha_2$.这只需注意到从$l(f_1(x))=f_2(x)$得到一个商环同态$\frac{k_1[x]}{(f_1)}\to\frac{k_2[x]}{(f_2)}$.

特别的,如果取$k_1=k_2,f_1=f_2$, 说明对同一个$k[x]$中不可约多项式的,位于同一个扩域$F$中的两个不同根$\alpha_i,i=1,2$,总唯一存在$k(\alpha_i),i=1,2$之间的同构,它限制在域$k$上是恒等映射,并且把$\alpha_1$ 映射为$\alpha_2$.

例子.设$k$是域,考虑它的有理分式域$k(t)$,取$K$中的一个分式$u=\frac{f(t)}{g(t)}\not\in k$,其中$(f,g)=1$,我们断言扩张$k(u)\subseteq k(t)$的次数为$\max\{\deg f,\deg g\}$.
\begin{proof}
	
	按照$k(t)=k(u)(t)$,只需寻找$t$在$k(u)$上的极小多项式.考虑多项式$p(x)=ug(x)-f(x)\in F[x]$.那么$t$是它的根,于是$t$是$k(u)$上的代数元.现在考虑$p(x)$的次数,假设$g$最高次项是$ax^n$,$f$最高次项是$bx^m$,如果$m=n$,按照$u$不是$k$中元说明$ua-b\not=0$.于是无论$m=n$还是$m\not=n$都有$p(x)$的次数是$\max\{\deg f,\deg g\}$.最后我们只要说明$p(x)$是$k(u)$上的不可约多项式.
	
	首先$u$是$k$上的超越元,否则有$[k(t):k]=[k(u)(t):k(u)][k(u):k]<\infty$.于是$k[u]\cong k[x]$.把$p$视为$u$的多项式,此时$p\in k[x][u]\subseteq k(x)[u]$是一次多项式,按照$(f,g)=1$说明$p$是本原多项式,于是$p$在$k[x]$上不可约.这就说明$p$在$k[u]$上不可约,于是$p$在商域$k(u)$上不可约.
\end{proof}

扩张的Galois群.给定扩张$k\subseteq K$,称$K$的全体限制在$k$上不变的域自同构为扩张的自同构,全体这样的自同构构成群,称为扩张的Galois群,记作$Aut_k(K)$.上一段就是在说,扩张的Galois群作用在同一个不可约多项式上全体根集合是可迁.即,扩张的自同构限制在一个不可约多项式的全体出现在某个固定扩域中的根上是双射,并且对任取的两个根$a,b$,总存在一个唯一的自同构把$a$映射为$b$.

于是有限单扩张的Galois群阶数的信息蕴含在根的个数中:有限维单扩张$k\subseteq k(\alpha)$的Galois群的阶数恰好就是$\alpha$极小多项式的那些位于$k(\alpha)$中的不同根的个数.于是$|\mathrm{Aut}_k(k(\alpha))|\le [k(\alpha):K]$.特别的取等当且仅当极小多项式在$k(\alpha)$上可以表示为若干个不同的一次多项式的乘积.注意尽管有悖于我们的直觉,一个不可约多项式在扩域中的分解是可能存在重根的,这是下一节中的可分概念,因此在上面定理中需要强调"不同根","不同的一次多项式的乘积".

有限扩张总是代数扩张.如果$k\subseteq F$是一个有限扩张,任取$F$中的元$\alpha$,按照维数公式$[F:k]= [F:k(\alpha)][k(\alpha):k]$,于是有限扩张总是代数扩张.

事实上有限扩张总是一个有限生成的代数扩张.取有限扩张$k\subseteq F$,那么$[F:k]$是有限的.现在任取一个$a_1\in F-k$,它是代数元,并且$[F:k]=[F:k(a_1)][k(a_1):k]$,由于$a_1$是不在$k$上的代数元,得到$[k(a_1):k]>1$,于是$[F:k(a_1)]$严格小于$[F:k]$.再考虑集合$F-k(a_1)$,如果空已经得证.如果非空,任取一个$a_2$,再考虑$[F:k(a_1)]=[F:k(a_1,a_2)][k(a_1,a_2):k(a_1)]$,同样的必然有$[k(a_1,a_2):k(a_1)]>1$,于是又可以把情况转化为在$[F:k(a_1,a_2)]$.这里$[F:k],[F:k(a_1)],\cdots$是严格减少的非负整数列,经过有限次后,必然取1,于是有$F=k(a_1,\cdots,a_n)$,得证.

对于单扩张,如果它是代数扩张,那么扩张维数就是扩张元的极小多项式的次数,这说明对于单扩张,有限扩张$\Leftrightarrow$代数扩张,事实上这也可以推广到有限生成扩张中,如果$k\subseteq F=k(a_1,a_2,\cdots,a_n)$是域扩张,那么下面三个条件等价:
\begin{enumerate}
	\item $k\subseteq F$是有限扩张.
	\item $k\subseteq F$是代数扩张.
	\item 每个$a_i$都是$k$上代数元.
\end{enumerate}
\begin{proof}

首先有限扩张总是代数扩张.下面代数扩张按照定义每个$a_i$都是代数元.最后只需证明3推1.首先,对于一个单扩张$k\subseteq k(\alpha)$,每个$a_i$是$k$上代数元也说明了每个$a_i$是$k(a_1,\cdots a_{i-1})$的代数元.这就得到扩张链:
$$k\subseteq k(a_1)\subseteq k(a_1)(a_2)\subset\cdots\subseteq k(a_1)(a_2)\cdots(a_n)$$

每一步扩张都是一个单代数扩张,于是是有限扩张,按照维数公式,得到$k\subseteq k(a_1,\cdots,a_n)$是有限扩张.完成证明.
\end{proof}

这里总结一下代数元和代数扩张的一些性质.
\begin{enumerate}
	\item 给定域扩张$k\subseteq F$,如果$a\in F$是$k$的代数元,那么$a$也是扩张的任意中间域的代数元.这是因为基域中零化$a$的多项式也是中间域上零化$a$的多项式.于是这说明给定代数扩张$k\subseteq F$,那么$F$也是任意中间域的代数扩张.
	\item 两个代数扩张的复合仍然是代数扩张.考虑扩张$k\subseteq K\subseteq F$,如果$k\subseteq K$和$K\subseteq F$都是代数扩张,我们断言$k\subseteq F$也是代数扩张.任取$a\in F$,需要证明$a$是$k$中代数元,首先$a$是$K$中代数元,于是存在一组$b_0,b_1,\cdots,b_n\in K$使得$b_na^n+\cdots+b_0=0$,考虑$E=k(b_0,b_1,\cdots,b_n)$,则$E$是$k$的代数扩张.而$a$是$E$上的代数元,于是$k\subseteq E$和$E\subseteq E(a)$都是有限扩张,按照维数公式,$k\subseteq E(a)$是一个有限扩张,于是$k\subseteq k(a)$也是有限扩张,这说明$a$是$k$上代数元.
	\item 上两条可以总结为,给定扩张链$k\subseteq K\subseteq F$,那么$k\subseteq F$是代数扩张当且仅当$k\subseteq K$和$K\subseteq F$均为代数扩张.
	\item 如果$X\subseteq F$是一组可能无穷个的$k$上的代数元构成的集合,那么$k(X)$同样是代数扩张.这是因为对于每一个$k(X)$中的具体元,它的表达式中只出现了$X$中有限个元,于是这归结为有限的情况.但是注意这时扩张未必是有限维的.
	\item 给定扩张$k\subseteq F$,若$X\subseteq F$是$k$上代数元构成的集合,则$k(X)=k[X]$.我们曾经证明了如果$a$是$k$上代数元,那么有$k(a)=k[a]$.由此可以归纳说明如果$X$是代数元构成的有限集合,则$k(X)=k[X]$,因为$k(a,b)=k(a)(b)=k(a)[b]=k[a][b]=k[a,b]$.再按照如果$X$是无限集合,则$k(X)=\cup k(X_0)$,$k[X]=\cup k[X_0]$,其中$X_0$跑遍$X$的有限子集,以及$k[X_0]=k(X_0)$,就得到$k(X)=k[X]$.
\end{enumerate}

本段给出一个例子,$\mathbb{Q}\subseteq \mathbb{Q}(\sqrt{2},\sqrt{3})$,首先$\sqrt{2},\sqrt{3}$都是代数元,于是这个扩张是有限扩张也是代数扩张,现在注意到,一方面$\mathbb{Q}(\sqrt{2}+\sqrt{3})\subseteq \mathbb{Q}(\sqrt{2},\sqrt{3})$,另一方面,$\sqrt{3}-\sqrt{2}=\frac{1}{\sqrt{2}+\sqrt{3}}\in \mathbb{Q}(\sqrt{2}+\sqrt{3})$,于是$\sqrt{3},\sqrt{2}\in \mathbb{Q}(\sqrt{2}+\sqrt{3})$,于是$\mathbb{Q}(\sqrt{2},\sqrt{3})=\mathbb{Q}(\sqrt{2}+\sqrt{3})$.现在求它的极小多项式,记$t=\sqrt{2}+\sqrt{3}$,那么$t^2=5+2\sqrt{6}$,那么$(t^2-5)^2=24$,那么$t^4-10t^2+1=0$,考察素数$p=2$,按照爱森斯坦判别法,$f(t)=t^4-10t^2+1$是$\mathbb{Z}[t]$上一个不可约多项式,于是它也是它商域$\mathbb{Q}$上的不可约多项式,于是这就是扩张元的极小多项式.按照简单的计算看到另外三个根是$\sqrt{2}-\sqrt{3},\sqrt{3}-\sqrt{2},-\sqrt{2}-\sqrt{3}$.

这里给出上述例子的两个注解.第一,这个有限扩张是单扩张并不是巧合,实际上一个可分的有限扩张总是单扩张,这被称为本原元定理,将在后文中给出.第二,$f(t)$的这四个根的形式很好的反映了Galois的信息.回顾一个单扩张的扩张元的极小多项式的全部位于扩域的根恰好一一对应于扩张的自同构.如果把$\mathbb{Q}(\sqrt{2}+\sqrt{3})$视为$\mathbb{Q}(\sqrt{2})(\sqrt{3})$,则可得到在$\mathbb{Q}(\sqrt{2})$上恒等,把$\sqrt{3}$映射为$-\sqrt{3}$的扩张自同构,而它把$\sqrt{2}+\sqrt{3}$映射为$\sqrt{2}-\sqrt{3}$同样是极小多项式的根.类似可以得到另外两个根.

域扩张的代数闭包.如果$a,b$是$k$上两个代数元,那么$k(a,b)$是$k$的代数扩张,于是$a\pm b$和$ab$,$a/b,b\not=0$都是$k$上代数元.但是如果我们考虑分别存在以$a,b$为根的$k$上不可约多项式这个定义,很难由此入手构造出$a\pm b,ab,a/b$它们所满足的$k$上的不可约多项式.另外,由此可以直接得出,给定域扩张$k\subseteq F$,那么$F$上全体在$k$上是代数元的元构成了$F$的一个包含$k$的子域$K$(即满足$k\subseteq K\subseteq F$,这称为$K$是扩张$k\subseteq F$的中间域).称它为域扩张的代数闭包.一个域的极大的代数扩张是什么样的?回答这个问题需要代数闭域的概念.

代数闭域.称一个域$K$是代数闭域,如果它满足如下等价条件中的任何一个:
\begin{enumerate}
	\item $K$上的不可约多项式只有全体1次多项式.
	\item $K$上每个非常数多项式都可以分解为若干一次多项式的乘积.
	\item $K$上每一个非常数多项式都有根.
	\item $K$没有非平凡的代数扩张.
\end{enumerate}

对这个定义做一些注解.第一,一个代数闭域必然含有无限个元,因为倘若只有有限个元$a_1,a_2,\cdots,a_n,n\ge2$,那么考虑$f(x)=\prod_{1\le i\le n}(x-a_i)+1$,这是一个没有根的非1次非常数多项式.第二,复数域是一个代数闭域,这也被称为代数学基本定理.第三,尽管有限域不是代数闭域,是存在特征$p>0$的代数闭域的,事实上即将证明每一个域都包含于某个代数闭域.

称一个域$k$的代数闭包(而不特指具体扩张的代数闭包)是指一个代数扩张$k\subseteq F$使得$F$是代数闭域.现在的主要任务就是证明,每一个域都存在代数闭包,并且在同构意义下唯一.我们先来证明存在性,在给出证明前需要指出,在这个证明中用到了Zorn引理(具体的讲是极大理想的存在性用到了Zorn引理).不过不像在线性代数理论中任何线性空间存在基是和Zorn引理等价的,这里代数闭包的存在性等价于所谓的一阶逻辑的紧性定理,它要比选择公理弱.
\begin{proof}

任给域$k$,首先证明存在一个域$K$,使得$k$中每个非常数多项式在$K$中都有根.为此我们取$k$上全体不可约多项式构成的集合$I$(取为全体非常数多项式也是以的),首先这是一个集合,因为它是可数个$k$的直和的子集.现在取一个和$I$存在双射的集合$A=\{t_f,f\in I\}$,把$A$作为不定元集,考虑多项式环$k[A]$.取$I$是由全体$f(t_f),f\in I$生成的$k[A]$中的理想,首先它不会是一个单位理想,这是因为若否,那么存在一组$k$中的元$k_i$和一个有限和$1=\sum k_i f_i(t_{f_i})$.但是可以取一个代数扩张$F$使得这有限个$f_i$都存在根,在域$F$中把这些根带入就得到了矛盾.那么$I$是一个真理想,它必然包含于一个极大理想$m$中,考虑$K=\frac{k[A]}{m}$,这是一个域,并且对于$k$中任意不可约多项式$f$,有$K$中的$t_f+m$是它的根.

现在,把上述$K$记作$K_1$,按照上述构造取$K_{i+1}$使得$K_i$中每个不可约多项式在$K_{i+1}$中都有根.由此得到了一个链:
$$k\subseteq K_1\subseteq K_2\subset\cdots$$

取$L=\cup_{i\ge1} K_i$,这是一个域,因为任取$a,b\in L$,可找到一个$n$使得$a,b\in K_n$,导致$a\pm b,ab,ab^{-1}\in K_n\subseteq L$.并且$L$是代数闭的,因为任取$L$的一个不可约多项式$f(x)$,可找到一个$n$使得$f(x)\in K_n[x]$,但是按照构造$K_{n+1}$中包含了$f(x)$的根,这就说明$L$是代数闭域.

但是至此还没有结束,因为$L$未必是$k$的代数扩张.取$k$在$L$中的代数闭包记作$\overline{k}$,那么$k\subset\overline{k}$是代数扩张,最后只需说明$\overline{k}$是代数闭的.任取$\overline{k}$上的不可约多项式$f(x)$,那么它在$L$中有根$\alpha$,于是$\alpha$同样是$k$上的代数元,导致$\alpha\in\overline{k}$,于是$\overline{k}$是代数闭域.
\end{proof}

无论代数闭包的唯一性还是下一节开篇中分裂域的唯一性,都需要稍微复杂的集合论手段,为此先整理出如下同构扩张定理,它将被多次用到:对于域扩张$k\subseteq L$,其中$L$是代数闭域,如果$k\subseteq F$是任意一个代数扩张,那么存在从$F$到$L$的嵌入.即存在$F$到$L$的同态使得在$k$上的限制是恒等.在给出证明前先指出,这个嵌入并不是唯一的,在可分扩张中将会把这个嵌入的个数定义为代数扩张的可分维数.
\begin{proof}
	
这个证明依赖于Zorn引理.记全体对$(E,i_E)$构成的集合是$S$,其中$E$是代数扩张$k\subseteq F$的中间域,$i_E$是从$E$到$L$的嵌入,并且限制在$k$是恒等映射.在$S$上赋予一个偏序,$(E,i_E)\le (N,i_N)$当且仅当$k\subseteq E\subseteq N\subseteq F$并且$i_N$在$E$上限制是$i_E$.证明升链总具有上界,按照Zorn 引理我们得到$S$中的极大元$(G,i_G)$,最后如果$G$并不是$F$,取$a\in F,a\not\in G$,取$H=i_G(G)\subseteq L$,记$k$上代数元$a$的极小多项式是$f$, 那么相应的$i_G(f)$是$H$中的不可约多项式,取它在$L$中的一个根$b$,那么存在从$G(a)$到$H(b)$的$k$同构延拓了$i_G$,这和$G$是$S$中极大元矛盾.
\end{proof}

现在证明代数闭包在同构意义下唯一.事实上如果$k$存在两个代数闭包$\overline{k}_i,i=1,2$.那么按照引理存在从$\overline{k}_1\to\overline{k}_2$的扩张,注意$\overline{k}_2$中的元总是$\overline{k}_1$上的代数元,按照代数闭域的定义,$\overline{k}_1$不会有非平凡的代数扩张,导致$\overline{k}_1=\overline{k}_2$.

关于两个域的乘法.设两个域$L_1,L_2$均是扩张$k\subseteq F$的中间域,定义$L_1L_2=\{\sum_ia_ib_i\mid a_i\in L_1,b_i\in L_2\}$.那么这同样是扩张$k\subseteq F$的中间域,并且:
\begin{enumerate}
	\item 有限生成扩张.如果$L_1=k(a_1,a_2,\cdots,a_n)$,$L_2=k(b_1,b_2,\cdots,b_m)$,那么$L_1L_2$是$k$的有限生成扩张,并且有$L_1L_2=k(a_1,a_2,\cdots,a_n,b_1,b_2,\cdots,b_m)$.
	\item 有限扩张.上一条可说明$k\subseteq L_1$和$k\subseteq L_2$都是有限扩张,当且仅当$k\subseteq L_1L_2$是有限扩张.
	\item 代数扩张.$k\subseteq L_1$和$k\subseteq L_2$都是代数扩张,当且仅当$k\subseteq L_1L_2$是代数扩张.
	\item 维数公式.$[L_1L_2:k]\le[L_1:k][L_2:k]$,如果$[L_1:k]$和$[L_2:k]$互素那么不等式取等号.
\end{enumerate}
\newpage
\subsection{分裂域与正规扩张}

在单代数扩张中我们证明过如果扩张元的极小多项式在扩域中是若干一次多项式的乘积,那么扩张的Galois群和极小多项式的不同根是一一对应的.这说明保留极小多项式分裂这个性质理应会提供Galois群和极小多项式根之间的联系,这就归结为分裂域的概念.

给定扩张$k\subseteq F$,称$k$中的一个多项式在$F$中分裂,如果这个多项式要么是常数,要么在$F[x]$中是若干一次多项式的乘积.给定一族$k[x]$中的非常数多项式$\{f_i,i\in I\}$,给定$k$的一个代数闭包$M$,那么这族多项式中每个多项式都在$M$中分裂,记全体根构成的$M$的子集为$X$,称$k$在$M$中的关于这族多项式的分裂域就是$k$和$X$在$M$中生成的域$k(X)$.我们现在主要想证明的是,尽管选取不同的代数闭包会得到不同的分裂域,但是分裂域在同构意义下是唯一的.

分裂域的唯一性.唯一性可以由下述引理直接推出:如果有域同构$\sigma:k_1\to k_2$.设$S_1=\{f_i,i\in I\}$是$k_1$上的一族多项式,那么$S_2=\{\sigma f_i,i\in I\}$是$k_2$上的一族多项式.如果$F_i$是$k_i$在某个代数闭包$\overline{k}_i$中的关于多项式族$S_i$的分裂域,其中$i=1,2$,那么$F_1$和$F_2$是同构的.
\begin{proof}
	
	设$S_i$在$\overline{k}_i$中的根集合为$X_i$,其中$i=1,2$.那么有$F_i=k_i(X_i)$,其中$i=1,2$.$F_1$可视为$k_2$的代数扩张,按照同构扩张定理,得到$F_1$到$\overline{k}_2$的域嵌入$\sigma'$,它在$k_1$上的限制为$\sigma$.注意$\sigma'$把$X_1$中的元映射为$X_2$中的元,并且固定$i\in I$时,$\sigma'$限制在$f_i$的根集到$\sigma f_i$的根集上是单射,但是这两个根集是元素个数相同的有限集合,说明限制到$f_i$根集的映射是双射,这说明$\sigma'$限制在$X_1\to X_2$时是双射,因此有$\sigma'(F_1)=F_2$,这就得到了$F_1$到$F_2$的同构.
\end{proof}

有限个多项式和无限个多项式.如果多项式族是有限个多项式$\{f_1,f_2,\cdots,f_n\}$构成的集合,那么此时它在代数闭包中的根集等同于单个多项式$f=f_1f_2\cdots f_n$在代数闭包中的根集,于是此时归结为关于单个多项式的分裂域.这说明关于多项式族的分裂域可分成两种情况:要么是关于单个多项式的分裂域,要么是关于无限个多项式的分裂域.

分裂域总是代数扩张.给定一族有限或者无限的$k$上多项式族$\{f_i,i\in I\}$,按照定义分裂域是$k(X)$,这里$X$是全体$f_i$在某个固定的代数闭包$M$中的全部根,于是$X$是由$k$上代数元构成的集合,于是$k(X)$是$k$的代数扩张.

关于单一多项式的分裂域的维数上界.我们来归纳证明,如果这个多项式的次数是$n$,那么分裂域的维数不超过$n!$.
\begin{proof}

首先当$n=1$的时候没什么需要证的,倘若对小于$n$次的多项式均成立,对于$n$次多项式$f$,不妨设$f$不在$k[x]$中分裂(否则扩张维数是1),则可取$f$的不可约因子$g$,于是可取$k$的有限单扩张$F=k(u)$使得$g$在$F$中有根$u$.记$F$中有$f=(x-u)h$,其中$h\in F[x]$是次数$n-1$的多项式,按照归纳假设$F$关于$h$的分裂域$K$的维数$[K:F]\le(n-1)!$.注意$K$同样是$k$关于$f$的分裂域,这说明$[K:k]=[K:F][k(u):k]\le(n-1)!\cdot n=n!$,完成归纳.
\end{proof}

分裂域存在一个不容易看出来的惊人性质.尽管在定义中仅仅约定了只有给定的多项式是分裂的,事实上对于基域的每一个不可约多项式,倘若它在分裂域中存在根,那么这个不可约多项式在分裂域中是分裂的!我们赋予这个性质一个名字,称一个代数扩张$k\subseteq F$是正规扩张,如果对于$k$的每一个不可约多项式,只要它在$F$中存在一个根,那么它在$F$中是分裂的.上述性质可以归结为:
\begin{enumerate}
	\item 一个域扩张$k\subseteq F$是有限正规扩张,当且仅当,$F$是$k$上单一多项式的分裂域.
	\item 一个域扩张$k\subseteq F$是任意正规扩张,当且仅当,$F$是$k$上一族多项式的分裂域.
\end{enumerate}
\begin{proof}

先来证明必要性.如果$k\subseteq F$是正规扩张,取全体$F-K$中的元的极小多项式,按照正规性,关于这个多项式族的分裂域恰好就是$F$.如果$k\subseteq F$是有限的正规扩张,那么它是有限生成扩张$F=k(a_1,a_2,\cdots,a_n)$,其中每一个$a_i$都是$k$的代数元,取$a_i$的极小多项式是$f_i$,取$f=\prod f_i$,那么按照正规性得到$k$关于$f$ 的分裂域恰好就是$F$.

困难的是另一侧.先来证明单一多项式的情况.取$F$是域$k$关于多项式$f$的分裂域,那么首先$F$是$k$的有限扩张,下面取$k$的代数闭包是$\overline{k}$,按照同构扩张定义,可视$F$为$\overline{k}$的包含$k$的子域,任取$k$上一个不可约因式$p$,设它在$F$中有一个根$\alpha$,设$\beta\in\overline{k}$是$p(x)$的另一个根.现在只需证明$\beta\in F$.

首先,有同构$l:k(\alpha)\to k(\beta)$,它在$k$上是恒等映射,并把$\alpha$映射为$\beta$.另外$F$可以看作$k(\alpha)$上关于$f$的分裂域,$F(\beta)$可以看作$k(\beta)$关于$f$的分裂域,于是根据分裂域的唯一性,$l$可以延拓为$F$到$F(\beta)$的同构$l'$,并且$l'$限制在$k$上是恒等.这说明$l'$可以作为$k$线性空间的同构,特别的,$[F:k]=[F(\beta):k]$,最后考虑扩张$k\subseteq F\subseteq F(\beta)\subset\overline{k}$,这得到$[F(\beta):F]=1$,于是$\beta\in F$,完成单一多项式情况的证明.
$$\xymatrix{
	&k(\alpha)\ar[r]\ar[dd]^l&F\ar[r]\ar[dd]^{l'}&\overline{k}\\
	k\ar[ur]\ar[dr]&&&\\
	&k(\beta)\ar[r]&F(\beta)\ar[r]&\overline{k}
}$$

现在考虑无限个多项式的情况.如果$F$是$k$上一族多项式$f_i$的分裂域.记$X$为全部$f_i$的全部根,我们证明过$k(X)=\cup_{Y\in P_0(X)}k(Y)$.其中$P_0(X)$表示$X$的全部有限子集.对每个$Y=\{a_1,\cdots,a_r\}\in P_0(X)$,记$a_i$对应于多项式族中的$g_i$,把$g_1,\cdots,g_r$的全部根添加到$Y$中,记作集合$Y'$,那么有$F=k(X)=\cup_ {Y\in P_0(X)}k(Y')$.现在任取$a\in F$,不妨设某个$Y\in P(X)$使得$a\in k(Y')$,但是按照构造$k(Y')$是有限个多项式的分裂域,于是按照我们在有限情况下的证明,$a$的极小多项式在$k(Y')$中分裂,于是自然在$F$中分裂,这就完成了证明.
\end{proof}

有限扩张$k\subseteq F$是正规扩张的另一个等价描述:取$k$的代数闭包$M$,对任意$k$同态(即限制在$k$上是恒等映射)$\sigma:F\to M$,总有$\sigma(F)=F$.我们曾给出过对于代数扩张$k\subseteq F$,存在不同的从$F$到$\overline{k}$中的嵌入.于是当代数扩张是正规扩张的时候,这些到代数闭包的嵌入实际上都是$k\subseteq F$的扩张自同构.另外我们在下文会给出到代数闭包的嵌入个数是扩张的可分维数,于是对于正规扩张,它的自同构群的阶数就是扩张的可分维数.
\begin{proof}

必要性.任取$F$中的元$a$,记$a$在$k$上的极小多项式为$p(x)$,那么按照正规条件得到$p(x)$在$F$中分裂,但是对每个从$F$到$M$的$k$同态$\sigma$,必然有$\sigma$限制在$p(x)$的根集合上是一个双射.这就保证了$\sigma(F)=F$.

充分性.任取$a\in F$,取$a$在$k$上极小多项式$p(x)$,需要证明$p$在$F$上分裂.$p(x)$在$M$中分裂,任取一个根$b$,那么存在$k$同态$F\to M$使得把$a$映射到$b$,按照$\sigma(F)=F$,看到$b\in F$,于是$p(x)$分裂.
\end{proof}

正规扩张不满足传递性.$k\subseteq F$和$F\subseteq K$是正规扩张不能保证$k\subseteq K$是正规的.例如特征零域上二次扩张必然正规,于是$\mathbb{Q}\subset\mathbb{Q}(\sqrt{2})$和$\mathbb{Q}(\sqrt{2})\subset\mathbb{Q}(\sqrt[4]{2})$均是正规扩张,但是$\mathbb{Q}\subset\mathbb{Q}(\sqrt[4]{2})$不会是正规扩张,它的正规闭包理应包含四次单位根,但是这个扩域根本不含非实数.
\newpage
\subsection{可分扩张与纯不可分扩张}

我们指出过一个不可约多项式在扩域中的分解未必没有重根,这本质是可分性的问题.它有悖于我们熟悉的数域情况是因为,数域情况都是特征0的域,而这时候不可约多项式是的确不会出现重根的,现在我们详细讨论可分性.

一个域$k$上的多项式称为可分的,如果它的分裂域中这个多项式没有重根.否则称为不可分的多项式.这个名字是很形象的,没有重根意味着多项式的根可以被完全分离开,因而称为可分的,如果存在重根意味着若干个相同根总是成团的,意味着根不可分离.

这一段给出一些简单观察.首先分裂域已经是指包含根集的最小的代数扩张,这说明$k$上的不可约多项式$p$可分当且仅当$f$在它的任意代数扩张中没有重根,不可分等价于讲至少存在一个代数扩张中该不可约多项式存在重根;如果一个不确定可不可约的多项式,它在分裂域里已经没有重根了,那么它必然是可分的;如果多项式整除一个可分多项式,那么这个多项式是可分的;有限个可分多项式的乘积是可分的;一个多项式在域$k$上可分,那么它在$k$的任意域扩张中都是可分的.

形式导数.对于一个$k$上多项式$f(x)=a_nx^n+a_{n-1}x^{n-1}+\cdots+a_0$,记它的形式导数是$f'(x)=na_nx^{n-1}+\cdots+a_1$,注意到这里对于导数的定义是形式上的,也就是说并不涉及极限问题,如同定义环上幂级数环一样特别称它是"形式的"幂级数.尽管如此,导数运算仍然有$(fg)'=f'g+g'f$和线性.借助形式导数可给出可分性的描述.域$k$上的非常数的多项式$f$可分当且仅当它和它的导数是互素的.
\begin{proof}
	
	一方面,如果$f(x)$放在某个代数扩张中有重根,则可在这个扩张中做分解$f(x)=(x-a)^2g(x)$,做形式导数,得到$f'(x)=(x-a)^2g'(x)+2(x-a)g(x)$,于是$f(x)$和$f'(x)$在扩域中有公因式$(x-a)$.但是我们知道扩域不改变两个多项式的互素性,说明$f(x)$和$f'(x)$在$k$上不互素.
	
	另一方面,如果$f(x)$和$f'(x)$在$k$上有非1的公因式$g(x)$,可取$k$的单扩张$F$使得$g(x)$在$F$中有根$a$,于是在$F$中唯一分解,得到$f(x)=(x-a)^kh(x)$,其中$h(a)\not=0$.计算形式导数,得到$k\ge2$,于是$f$在扩域$F$中有重根.
\end{proof}

给定一个非常数的不可约多项式$f(x)\in F[x]$:
\begin{enumerate}
	\item 若$F$的特征为0,那么$f$总是可分的.如果$F$的特征为$p>0$,那么$f$可分当且仅当$f'(x)\not=0$,而这成立当且仅当$f(x)\not\in F[x^p]$.
	\begin{proof}
		
		由于$f(x)$是不可约多项式,于是$f'(x)$和$f(x)$互素当且仅当$f'(x)=0$,而特征0的时候$f'(x)$总不为0,于是此时$f(x)$总是可分的;如果特征是$p>0$,则$f'(x)=0$等价于$f(x)x\in F[x^p]$.
	\end{proof}
	\item 若$F$特征$p>0$,取$m$是最大的非负整数,使得$f(x)\in F[x^{p^m}]$,此时记有$f(x)=g(x^{p^m})$,那么此时$g(x)\in F[x]$是$F$上的不可约多项式,否则就会推出$f(x)$是不可约的多项式.
\end{enumerate}

Frobenius映射.特征$p>0$的域$k$上的映射$x\mapsto x^p$称为Frobenius映射.按照特征$p$,这是一个域同态,于是它必然是单射,但是它未必是满射.称一个域是完美域,如果它是特征0的,或者是特征$p$的并且Frobenius映射是同构.那么对于特征$p$的域$k$,它是完美域等价于上述映射是满射,等价于$k=k^p$.有限集合自身上的单射必然是双射,于是有限域总是完美域.

完美域存在两个更便于理解的等价描述,这一段给出第一个:一个域是完美域等价于它的每个不可约多项式都是可分的.
\begin{proof}

一方面,如果域是完美域,如果$f$是一个不可约的不可分多项式,那么它是$x^p$的多项式,记作$f(x)=\sum_{i=0}^n a_ix^{pi}$, 按照Frobenius映射是同构,取$b_i\in k$使得$b_i^p=a_i$,那么$f(x)=\left(\sum_{i=0}^n b_ix^i\right)^p$,这和不可约矛盾.

另一方面,如果域不是完美域.那么存在$a\in K$在Frobenius映射下没有原像,考虑多项式$f(x)=x^p-a$,我们断言它是不可约的不可分多项式,这就完成证明.不可分部分是直接的,因为$f'(x)=0$.现在只需说明它是不可约多项式,这一事实我们实际上有证明过,如果$p$是素数,那么域$k$上的多项式$x^p-a$不可约等价于在$k$上没有根,这里没有根是由$a$的选取保证的.不过我们这里再证一次这个特殊的有特征要求的情况的证明:

记$k$关于$f$的分裂域是$F$,在$F$中取$f$的一个根为$b$,那么在$F$中有分解$f(x)=x^p-b^p=(x-b)^p$,现在假设$f$是可约的,那么存在一个$(x-b)^i\in k[x]$,我们断言不存在$1\le i\le p-1$使得$b^i\in k$,这就得到矛盾,说明$f$不可约.若否,则有整数$s,t$使得$si+tp=1$, 那么$(b^{i})^s=b\in k$,这和$x^p-a$在$k$中没有根矛盾.于是$f$是不可约的不可分多项式.
\end{proof}

这里做一个注解,在上述证明中,我们假设$k$不是完美域,构造了一个不可分的不可约多项式$f(x)=x^p-a$,并且构造了它的分裂域$F=k(b)$.这实际上提供了一个纯不可分扩张的例子.

可分扩张.给定域$k$的代数元$a$,它的极小多项式是$k$上的不可约多项式,如果这个多项式可分/不可分,就称代数元$a$是$k$上的可分元/不可分元.如果一个代数扩张的每个元都是可分元,就称扩张是可分扩张,否则称为不可分扩张.在超越扩张内容里我们会定义超越可分扩张,直到超越扩张那一节之前,当我们提及可分扩张时都是指本段所定义的代数可分扩张.按照可分扩张的定义,能够给出完美域的第二个等价描述:一个域是完美域当且仅当它的每个代数扩张都是可分扩张.

总结下域$k$是完美域的三个等价描述:
\begin{enumerate}
	\item 域$k$要么是特征0的域,要么是特征$p$的域并且满足Frobenius映射$x\mapsto x^p$是同构.
	\item 域$k$的全部不可约多项式都是可分多项式.
	\item 域$k$的全部代数扩张都是可分扩张.
\end{enumerate}

代数扩张的可分维数.这个维数存在两种等价描述,这里先给出第一种.对代数扩张$k\subseteq F$,在前文中给出过同构扩张定理,即$F$总可以嵌入到$k$的代数闭包中,并且我们有强调这个嵌入不是唯一的.这个定理在代数闭包的唯一性,分裂域的唯一性均有用到,现在第三次使用它.这个代数扩张的可分维数就定义为不同的到代数闭包中的嵌入的个数,记作$[F:k]_s$

这个定义看似和可分性无关的,其实不然.对一个单的有限扩张$k\subseteq k(a)$,它的可分维数恰好就是$a$的极小多项式在$k$的代数闭包中的不同根的个数.于是$[k(a):k]_s\le[k(a):k]$,特别的取等当且仅当$a$是一个可分元.
\begin{proof}

取$a$的极小多项式$p$,对每一个从$k(a)$到$k$的代数闭包$\overline{k}$的$k$嵌入$l$,把它映射为$f$的根$l(a)$,那么由于嵌入$l$已经在$k$上恒等,这个映射被$l(a)$的值唯一确定,于是把嵌入映射为$p$的根是一个单射.下面,如果$b$是$p$的另一个根,则存在$k(a)$到$k(b)$的$k$同构,那么它可以得到一个从$k(a)$到$\overline{k}$ 的嵌入使得$l(a)=b$.于是这个从全体嵌入到$p$的不同根的映射还是一个满射.特别的,嵌入个数恰好是$p$的次数等价于说$p$是可分的,于是$a$是可分元.
\end{proof}

可分维数满足维数公式.即如果有代数扩张链$k\subseteq L\subseteq F$,那么$[F:k]_s=[F:L]_s[E:L]_s$.
\begin{proof}

取$k$的代数闭包为$M$,那么$M$同样是$L$和$F$的代数闭包,取从$L$到$M$的全体不同的$k$域同态$\sigma_1,\cdots,\sigma_r$,于是$r=[L:k]_s$.再取全体不同的从$F$到$M$的$L$域同态$\tau_1,\cdots,\tau_s$,于是$s=[F:L]_s$.按照分裂域的唯一性,可把所有$\sigma_i$和$\tau_j$都唯一的延拓为$M$上的域同态,我们采取相同的记号.现在断言每个$F$到$M$的$k$域同态都具有形式$\sigma_i\circ\tau_j\mid_F$,如果$\rho:F\to M$是$k$域同态,把它唯一延拓为$M$上的域自同构并采取相同记号,那么$\rho\mid_L$是某个$\sigma_i$,那么$\sigma_i^{-1}\circ\rho$是$M$上的$F$自同构,于是$\sigma_i^{-1}\circ\rho\mid_F$是某个$\tau_j$,于是$\rho=\sigma_i\circ\tau_j$.另外$\sigma_i\circ\tau_j$两两不同,这就说明$[F:k]_s=[F:L]_s[L:k]_s$.
\end{proof}

结合维数公式可以把单扩张和可分的联系推广到有限生成代数扩张中,即,如果$k\subseteq F$是有限扩张,那么$[F:k]_s\le [F:k]$,并且下面三个条件是互相等价的:
\begin{enumerate}
	\item $F=k(a_1,a_2,\cdots,a_n)$其中每个$a_i$都是可分元.
	\item $k\subseteq F$是可分扩张.
	\item $[F:k]=[F:k]_s$.
\end{enumerate}
\begin{proof}
	
	先来证明维数不等式.由于$k\subseteq F$是有限扩张,说明可以表示为$F=k(a_1,a_2,\cdots,a_n)$,其中$a_i$都是$k$上代数元.那么就有维数不等式:
	\begin{align*}
	[F:k]_s&=[k(a_1,a_2,\cdots,a_{n-1})(a_n):k(a_1,a_2,\cdots,a_{n-1})]_s\cdots[k(a_1):k]_s\\
	&\le [k(a_1,a_2,\cdots,a_{n-1})(a_n):k(a_1,a_2,\cdots,a_{n-1})]\cdots[k(a_1):k]\\
	&=[F:k]
	\end{align*}
	
	现在证明三个条件的等价性.1推3,如果每个$a_i$都是可分元,按照单扩张的情况,说明总有维数公式$[k(a_1,\cdots,a_{i-1})(a_i):k(a_1,\cdots,a_{i-1})]_s=[k(a_1,\cdots,a_{i-1})(a_i):k(a_1,\cdots,a_{i-1})]$,于是上述维数不等式取等,即$[F:k]=[F:k]_s$.3推2,任取$F$中的元$a$,那么有扩张链$k\subseteq k(a)\subseteq F$,于是有维数等式:
	$$[F:k(a)]_s[k(a):k]_s=[F:k]_s=[F:k]=[F:k(a)][k(a):k]$$
	
	但是最右侧两个做乘积的维数都分别不超过对应的最左侧的两个做乘积的维数,而乘积相等,说明有$[k(a):k]_s=[k(a):k]$,也即$a$是可分元.2推1是可分扩张定义保证的.
\end{proof}

总结一下可分元和可分扩张的一些性质:
\begin{enumerate}
	\item 给定代数扩张$k\subseteq F$,如果$a\in F$是$k$的可分元,那么$a$也是扩张的任意中间域的可分元.这是因为可分多项式的因式总是可分的.于是这也说明给定可分扩张$k\subseteq F$,那么$F$也是任意中间域的可分扩张.
	\item 两个可分扩张的复合仍然是可分扩张.对于有限扩张的情况,只需按照上述第三个等价条件和维数公式.对于无限维的扩张,取可分扩张链$k\subseteq K\subseteq F$,取$a\in F$,需要证明$a$是$k$的可分元,$a$在$K$中的极小多项式记作$f(x)=b_nx^n+\cdots+b_0$,那么这个多项式在代数闭包里分解为不同一次因式的乘积,现在考虑$k\subseteq E=k(b_0,b_1,\cdots,b_n)\subseteq E(a)$,那么这就归结为有限的可分扩张的情况,从而得到$a$是$k$的可分元.
	\item 上两条可以总结为,给定代数扩张链$k\subseteq K\subseteq F$,那么$k\subseteq F$是可分扩张当且仅当$k\subseteq K$和$K\subseteq F$均为可分扩张.
	\item 类似代数扩张的情况,如果$X\subseteq F$是域$k$上的一族可分元,则$k(X)$是$k$的可分扩张.因为取$X$有限子集构成的集合$S$,那么$k(X)=\cup_{Y\in S}k(Y)$.
	\item 我们已经证明了有限扩张的可分维数总满足$[F:k]_s\le[F:k]$,在给出可分维数的第二个等价定义后我们会看到事实上总有$[F:k]_s$整除$[F:k]$.
\end{enumerate}

可分闭包.上述等价命题说明,如果$a_1,a_2$是$k$的可分元,那么$a_1a_2,a_1+a_2,a_1/a_2,a_2\not=0$都是$k$的可分元.于是对于一个代数扩张$k\subseteq F$,取$F$中全体$k$上可分元就构成了一个中间域,称为这个代数扩张的可分闭包,记作$F_{sep}$.如果$F$是$k$的代数闭包,此时的$F_{sep}$就直接称为可分闭包(而不提及所指的代数扩张).

纯不可分元和纯不可分扩张.对任意域$k$上的代数元$a$,它的极小多项式记作$f(x)$,那么存在一个最小的非负整除$r$使得$f$是关于$x^{p^r}$的多项式但不是关于$x^{p^{r+1}}$的多项式,那么设$a$的维数是$n$,就有$n=p^rm$.注意!这里并不是说$p\not\mid m$,仅仅是说$f$只能最多表示为$x^{p^r}$的多项式.称$m$是$a$和不可约多项式$f$的可分次数,$p^r$是$a$和不可约多项式$f$的不可分次数,$r$是$a$和不可约多项式$f$的不可分指数.那么多项式可分当且仅当它的不可分次数是1,当且仅当它的不可分指数的0.对于另一个极端,即可分次数是1,就称多项式是纯不可分的,称$a$是纯不可分元.如果一个代数扩张中每个元都是纯不可分元,就称扩张是纯不可分扩张.
既是可分元又是纯不可分元的元只能是基域的元素.因为纯不可分元$a$的极小多项式具有形式$x^{p^r}-c$,可分意味着形式导数不为0,这只能有$r=0$.于是如果代数扩张$k\subseteq F$是纯不可分扩张,只能有扩张的可分闭包为$F_{sep}=k$.

纯不可分元的等价描述.特征$p$上域$k$的代数元$a$是纯不可分元等价于说存在一个$p^r$使得$a^{p^r}\in k$,另外如果$a$是纯不可分元,那么满足$a^m\in k$的最小的正整数$m$恰好是一个素数幂$p^r$,它就是$a$的不可分次数.
\begin{proof}

一方面如果$a$是纯不可分元,我们知道可分次数是1,也即$a$的极小多项式可以表示为$g(x^{p^r})$,其中$g$是$k$上一次多项式不妨记作$x-c$,那么$a$的极小多项式就是$x^{p^r}-c$.另一方面如果$k$上代数元$a$满足$a^{p^r}=c\in k$,那么在某个代数扩张中会有$x^{p^r}-c=(x-a)^{p^r}$,于是$a$的极小多项式是某个$(x-a)^m$.倘若$m$和$p$互素,那么存在整数$x,y$满足$xm+yp^r=1$,于是从$a^m\in k$得到$a=a^{xm+yp^r}\in k$,此时$a$的可分次数和不可分次数都是1,于是$a$自然是纯不可分元.倘若$m$不和$p$互素,我们断言$m$必然是$p$次幂,记$m=p^sm_0$,其中$p\not\mid m_0$,则$s\le r$,于是存在整数$x,y$满足$xm_0+yp^{r-s}=1$,于是$a^{p^s}=a^{p^sxm_0+yp^r}\in k$,按照$m$的最小性说明$m_0=1$.于是此时极小多项式为$(x-a)^{p^s}=x^{p^s}-a^{p^s}$.特别的$p^s$就是最小的满足$a^m\in k$的次数$m$.
\end{proof}

可分次数描述了不可约多项式的根的"抱团"数,不可分次数描述了每个"团"的大小.如果代数元$a$的极小多项式$f$的不可分次数是$p^r$,那么得到一个不可约多项式$\widetilde{f}$使得$f(x)=\widetilde{f}(x^{p^r})$.$\widetilde{f}$的不可分次数是1,取基域的代数闭包$F$,在$F$中有$\widetilde{f}(x)=\prod(x-b_i)$,其中$b_i$两两不同.若$a$是$f$的一个根,那么$a^{p^r}$是$\widetilde{f}$的一个根,于是$\widetilde{f}(x)=(x-a^{p^r})\widetilde{g}(x)$.于是有$f(x)=(x-a)^{p^r}\widetilde{g}(x^{p^r})$.并且$\widetilde{g}(x^{p^r})$不再以$a$为根.于是,一个代数元的极小多项式的根会分裂为可分次数个团,其中每一个团由不可分次数个相同根构成的.

代数扩张分解为可分扩张和纯不可分扩张的复合.已经说明了对于一个代数扩张$k\subseteq F$,那么全部$F$中关于$k$的可分元构成了一个中间域$F_{sep}$,这是一个极大的可分扩张.现在考虑扩张$F_{sep}\subseteq F$.对$a\in F$,如果$a$不是$k$的可分元,那么它在$k$中纯不可分次数是$p^r>1$,那么$a^{p^r}$将是$k$的一个可分元,即$a^{p^r}\in F_{sep}$,于是$a$是$F_{sep}$的纯不可分元.

现在给出代数扩张的可分维数的第二个等价定义.约定代数扩张的可分维数就是扩张的可分闭包的扩张维数,即$[F_{sep}:k]$.我们来证明两个定义是吻合的.按照维数公式有$[F:k]_s=[F:F_{sep}]_s[F_{sep}:k]$,我们已经知道这里$F_{sep}\subseteq F$是纯不可分扩张,所以只需证明有限纯不可分扩张的可分维数总是1.
\begin{proof}

设$k\subseteq F$是纯不可分的单扩张,可记$F=k(\alpha)$,按照定义$\alpha$是$k$上纯不可分元,它的极小多项式是$x^{p^r}-a$,它在分裂域中分解为$(x-\alpha)^{p^r}$,按照有限单扩张的可分维数恰好就是扩张元的极小多项式在扩张中的不同根个数,立马得出$[k(\alpha):k]_s=1$.

再设$k\subseteq F$是有限的纯不可分扩张,可记$F=k(a_1,a_2,\cdots,a_n)$,其中每个$a_i$都是纯不可分的,并且每个$a_{i+1}$都是$k(a_1,a_2,\cdots,a_i)$的纯不可分元,于是考虑扩张链:
$$k\subseteq k(a_1)\subseteq k(a_1)(a_2)\subseteq \cdots\subseteq k(a_1)\cdots(a_n)$$

按照维数公式,有$[F:k]_s=1$,即对于纯不可分扩张$k\subseteq F$,总有$[F:k]=[F:k]_i$.
\end{proof}

模仿有限扩张是可分扩张的等价描述,可以得到如下定理:如果$k\subseteq F$是有限扩张,那么下面三个条件等价:
\begin{enumerate}
	\item $F=k(a_1,a_2,\cdots,a_n)$,其中每个$a_i$都是纯不可分元.
	\item $k\subseteq F$是纯不可分扩张.
	\item $[F:k]_s=1$.
\end{enumerate}
\begin{proof}

2推3就是上面刚刚所证明的,而3推2是直接的,因为可分维数1意味着基域在扩张中的可分闭包就是自身,于是该扩张是纯不可分扩张.2推1是直接的,最后证明1推2.

不妨设域$k$的特征$p>0$,否则全部$a_i$都是$k$中元,此时$F=k$自然是纯不可分扩张.按照每个$a_i$都是$k$上纯不可分元,于是存在一个足够大的正整数$m$使得$a_i^{p^m}\in k$对每个$1\le i\le n$均成立.我们知道由于$a_i$都是代数元,于是有$F=k[a_1,a_2,\cdots,a_n]$.于是$F$中的每个元$a$都可以表示为$a_i$的多元多项式,并且系数都是$k$中的元,这按照特征$p$知$a^{p^m}$是关于$a_i^{p^m}$的多元多项式,并且系数都是$k$中元,于是$a^{p^m}\in k$,即$a$是纯不可分元.
\end{proof}

总结一下纯不可分元和纯不可分扩张的一些性质:
\begin{enumerate}
	\item 给定代数扩张$k\subseteq F$,如果$a\in F$是$k$的纯不可分元,那么$a$也是扩张的任意中间域$E$的纯不可分元,这是因为从纯不可分元的等价描述,得到$a^{p^r}\in k\subseteq E$.于是这也说明给定纯不可分扩张$k\subseteq F$,那么$F$也是任意中间域的纯不可分扩张.
	\item 两个纯不可分扩张的复合仍为纯不可分扩张.如果$k\subseteq K\subseteq F$是两个纯不可分扩张的复合,那么任取$a\in F$,有素数幂$p^r$使得$a^{p^r}\in K$,但是它又是$k$上的纯不可分元,于是又有素数幂$p^s$使得$a^{p^{r+s}}\in k$,这就说明$a$是$k$上的纯不可分元.
	\item 上述两条可以总结为,给定代数扩张$k\subseteq E\subseteq F$,那么$k\subseteq F$是纯不可分扩张当且仅当$k\subseteq E$和$E\subseteq F$都是纯不可分扩张.
	\item 类似代数扩张和可分扩张的情况,如果$F=k(X)$,这里$X$是一族可以无限个的$k$上的纯不可分元构成的集合,那么$F$是$k$的纯不可分扩张.
	\item 有限纯不可分扩张的维数是$p$的次幂.我们知道有限纯不可分扩张可以写作纯不可分单扩张的有限长的扩张链,于是只需对单扩张的情况加以证明.但是纯不可分元的次数自然是素数幂,这就得证.
	\item 第四条允许我们定义代数扩张的纯不可分闭包,于是一个代数扩张还可以分解为先纯不可分扩张再可分扩张的复合.这个事实可以说明如果特征$p>0$域的有限扩张是不可分扩张,那么$p$必然整除扩张维数.
\end{enumerate}

我们接下来介绍可分扩张的一个有趣结论,即有限可分扩张必然是单扩张,这称为本原元定理.在给出这个定理的证明之前,先给出代数扩张是单扩张的一个等价描述:一个代数扩张是单扩张当且仅当它只存在有限个不同中间域.
\begin{proof}

必要性,我们会证明单代数扩张的中间域被扩张元极小多项式的因式唯一决定.如果有单代数扩张$F=k(\alpha)$,设$q(x)$是$\alpha$在$k$上的极小多项式,那么对于扩张的任意一个中间域$E$,有$F$是$E$的单代数扩张,并且极小多项式$q_E(x)$是$q(x)$的一个因子.现在断言中间域实际上被这个因子唯一确定.这就说明了只有有限个中间域.为此我们证明$E$实际上被$q_E(x)$的系数在$k$上生成.取$E'$是$E$的子域,由$q_E(x)$的系数在$k$上生成,于是$q_E(x)\in E'(x)$,既然$q_E(x)$在$E$上不可约,得到$q_E(x)$在$E'$上不可约,注意到$E'\subseteq E\subseteq F$,得到$q_E(x)$的次数是$[F:E']$同样是$[F:E]$,于是得到$[E:E']=1$,这说明$E=E'$.

充分性,如果代数扩张$k\subseteq F$只有有限个中间域,那么首先这个扩张必然是有限生成的,否则我们可以构造一个严格递增的升链$k\subseteq k(a_1)\subseteq k(a_1,a_2)\subset\cdots$,于是这是一个有限的代数扩张.如果$k$是有限域,那么$F$同样是有限域,我们将在下一节证明这是一个单扩张.如果$k$是无限域,现在证明每个有限生成的代数扩张$F=k(a_1,\cdots,a_r)$是单扩张.运用归纳法,只需证明$r=2$的情况,即取$F=k(\alpha,\beta)$,对每个$c\in k$,得到一个中间域$k(c\alpha+\beta)$,但是只有有限个中间域,于是存在$k$中的$c\not=d$,使得$k(d\alpha+\beta)=k(c\alpha+\beta)$,于是$\alpha=\frac{(d\alpha+\beta)-(c\alpha+\beta)}{d-c}\in k(c\alpha+\beta)$,同样有$\beta\in k(c\alpha+\beta)$,这说明$k(\alpha,\beta)=k(c\alpha+\beta)$.得证.
\end{proof}

现在来证明本原元定理,即有限维可分扩张必然是单扩张.
\begin{proof}

按照归纳法,只需证明$F=k(\alpha,\beta)$是$k$的单扩张,其中$\alpha,\beta$是$k$上的可分元.如果$k$是有限域,那么$F$同样是有限域,我们会在有限域一节证明此时扩张是单扩张.现在设$k$是无限域,取$k$的一个代数闭包$\overline{k}$,那么$F$到$\overline{k}$的全体固定$k$的嵌入个数就是可分维数$[F:k]_s=[F:k]$,不妨设这个维数大于1,否则$F=k$,此时没什么需要证的.记全体这样的嵌入构成的集合为$I$,如果$l,l'$是$I$中两个不同的元,那么多项式$l(\alpha)x+l(\beta)$和$l'(\alpha)x+l'(\beta)$是不同的多项式,考虑多项式:
$$f(x)=\prod_{l\not=l'}\left((l(\alpha)x+l(\beta))-(l'(\alpha)x+l'(\beta))\right)$$

这个多项式非0,由于$k$是无限域,可以取$c\in k$使得$f(c)\not=0$.这等价于讲,如果记$\gamma=c\alpha+\beta\in F$,则当$l$取遍$I$中元的时候,有$l(\gamma)$是两两不同的元.于是说明$[k(\gamma):k]\ge[F:k]_s$.但是又有$[k(\gamma):k]\le[F:k]$,结合$[F:k]_s=[F:k]$就得到$F=k(\gamma)$,完成证明.
\end{proof}

在上述两个命题中都把有限域的有限扩张是单扩张留在有限域一节证明.这里指出证明思路,我们会证明有限域的乘法群是一个循环群,于是只要取这个循环群的生成元作为扩张元,立刻得出有限域之间的扩张总是单扩张.

一个有限可分扩张的$k\subseteq F$必然满足$|\mathrm{Aut}_kF|\le[F:k]$,取等号当且仅当它是正规扩张.我们会在后文给出这个结论实际上对任意有限扩张均成立.
\begin{proof}

根据本原元定理,$k\subseteq F$是单代数扩张,于是单代数扩张的Galois群的阶数必然不超过扩张维数.并且二者取等当且仅当扩张元的极小多项式在扩域中有次数个根,这等价于说扩域是关于扩张元极小多项式的分裂域,而在有限扩张条件下,正规扩张和关于单个多项式的分裂域是等价的,这就得到了取等当且仅当扩张是一个正规扩张.
\end{proof}

正规闭包.给定代数扩张$k\subseteq F$,它的正规闭包$N$是指$k$关于全体$a\in F$的极小多项式构成的不可约多项式族的分裂域,那么此时有$k\subseteq F\subseteq N$.记号同上,我们来给出正规闭包的一些性质:
\begin{enumerate}
	\item $N$是$k$的包含域$F$的极小正规扩张,即如果有$k$的正规扩张$M$满足$F\subseteq M\subseteq N$,那么有$M=N$.
    \item 如果$F=k(a_1,a_2,\cdots,a_n)$,那么$N$就是$k$关于$a_i$的极小多项式这$n$个不可约多项式的分裂域.特别的,如果$k\subseteq F$是有限扩张,那么$k\subseteq N$是有限扩张.
    \item 正规闭包的唯一性.如果$N'$是一个包含$F$的$k$的正规扩张,那么$N\subseteq N'$.这导致,如果$N'$是不包含能成为$k$的包含$F$的正规扩张的真子域的$k$的正规扩张,那么$N$和$N'$是同构的.
\end{enumerate}
\newpage
\subsection{有限域}

有限域是指阶数有限的域,也被称为Galois域.在域开篇说明了域范畴可以按特征分为$\mathrm{Fld}_0$和$\mathrm{Fld}_p$,不同特征的域之间不存在态射,并且划分出来的子范畴上均存在初对象,特征0时初对象是$\mathbb{Q}$,特征$p$时初对象是$p$阶域$\mathbb{F}_p$,于是每个特征$p$的域都可以作为$\mathbb{F}_p$的扩域,特别的,有限域都是$\mathbb{F}_p$的线性空间,于是有限域都是素数幂阶的.

有限域的存在性.对任意素数幂$q=p^d$,我们断言$\mathbb{F}_p$关于$f(x)=x^q-x$的分裂域$F$是一个$q$阶域.事实上$f'(x)=-1$,于是$f$是一个可分多项式,从而它的根的集合$E\subseteq F$恰好由$q$个元素构成,下面说明$E$本身是一个域,从而就有$E=F$.如果$a,b\in E$,那么$(a-b)^q=a^q-b^q=a-b$,另外当$b\not=0$时有$(ab^{-1})^q=a^qb^{-q}=ab^{-1}$.于是$a-b$和$ab^{-1}$都是$E$中的元.

有限域的乘法群是循环群.回顾在群论中给出过循环准则,即一个有限群是循环群当且仅当它阶数的每一个因子$d$,$x^d=1$在群中至多有$d$个解.按照域上多项式的根不超过多项式的次数,说明域的乘法有限子群必然是循环群,特别的有限域的乘法群必然是循环群.

有限域的唯一性.给定一个素数幂$q$,所有的$q$阶域都是同构的.因为乘法群是一个$q-1$阶群,于是每个元素满足$x^{q-1}=1$,加上0元,相当于满足$x^{q}-x=0$.这说明$q$阶域的全部元恰好就是$\mathbb{F}_p$的某个代数闭包中$x^q-x$的全部根,于是$q$阶域恰好就是$\mathbb{F}_p$关于$x^q-x$的分裂域,按照分裂域的唯一性得证.

有限域扩张的描述.存在有限域扩张$\mathbb{F}_{p^s}\subset\mathbb{F}_{p^t}$当且仅当$s\mid t$.另外如果$s\mid t$,那么这样的域扩张恰好存在一个,换句话说此时$\mathbb{F}_{p^t}$恰好存在一个子域同构于$\mathbb{F}_{p^s}$.最后,有限域之间的扩张总是单扩张.
\begin{proof}

如果有扩张$\mathbb{F}_{p^s}\subseteq \mathbb{F}_{p^t}$,考虑扩张链$\mathbb{F}_p\subset\mathbb{F}_{p^s}\subset\mathbb{F}_{p^t}$,其中$\mathbb{F}_p\subset\mathbb{F}_{p^s}$的扩张维数是$s$,而$\mathbb{F}_p\subset\mathbb{F}_{p^t}$的扩张维数是$t$,从维数公式说明必然有$s\mid t$.

反过来,如果满足$s\mid t$,我们说明必然有扩张$F_1=\mathbb{F}_{p^s}\subseteq F_2=\mathbb{F}_{p^t}$.注意到$F_2$是$\mathbb{F}_p$关于$x^{p^t}-x$的分裂域,$F_1$是$\mathbb{F}_p$关于$x^{p^s}-x$的分裂域,这两个多项式后者整除前者,于是我们可以把第二个断言归结为如下定理:如果域$k$关于多项式$f$的分裂域是$F$,如果$f$在$k[x]$中有因式$g$,$k$关于$g$的分裂域是$G$,那么$F$恰好存在一个子域是$K$同构于$G$的.

一方面,取$g$在$F$中的全部根为$\{a_1,a_2,\cdots,a_n\}$,那么$F$的子域$k(a_1,a_2,\cdots,a_n)$是同构于$G$的子域,这就说明存在性.另一方面,如果$F$存在另一个子域同构于$G$,则它可以记作$k(b_1,b_2,\cdots,b_n)$,其中$b_i$对应于$a_i$是一个域同构,导致全体$b_i$也是$g$的全部根的一个排列,但是$g$在$F$中的根集是唯一确定的,于是这样的子域是唯一的.

最后一个断言是因为,有限域的乘法群是循环群,于是只要取扩张元为扩域的这个循环群的生成元,就说明总是单扩张.
\end{proof}

有限域上存在任意次不可约多项式.对每一个有限域$\mathbb{F}_{p^d}$,对每一个正整数$n$,上一个定理说明总有扩张$\mathbb{F}_{p^d}\subset\mathbb{F}_{p^{dn}}$,并且这是单扩张,那么扩张元的极小多项式是有限域$F_{p^d}$上的一个$n$次不可约多项式,于是有限域上存在任意次数的不可约多项式.

有限域扩张的Galois群.记素数幂$p^d=q$,考虑域扩张$k=\mathbb{F}_{q}\subseteq F=\mathbb{F}_{q^n}$,它的扩张的Galois群是由$x\mapsto x^q$生成的扩张维数阶的循环群.特别的,如果$q=p$,那么扩张的Galois群是由Frobenius映射$x\mapsto x^p$生成的循环群.
\begin{proof}

首先域$k$中的元满足$x=x^q$,这说明$F$上的映射$x\mapsto x^q$是一个扩张的自同构.记这个映射在扩张的Galois群中的阶数为$m$,我们需要证明的是$m=n$.首先阶数$m$意味着在域$F$中恒有$x^{q^m}=x$.按照$F$的乘法群是循环群,说明$F$中有元$a$满足$a^r=1$的最小正整数$r$为$q^n-1$,这说明$m\ge n$.又因为有限域扩张是可分扩张(有限域是完美域),并且扩张是正规扩张,我们证明过这种情况下Galois群的阶数就是扩张维数$n$,但是$m$不超过Galois群的阶数,因而有$m\le n$,于是得到$m=n$.
\end{proof}
\newpage
\section{Galois理论}
\subsection{基本定理}

我们先重新整理下扩张自同构的性质.给定$k$的两个扩域$K,L$,如果域同态$f:K\to L$满足$f$在$k$上的限制是恒等映射,就称$f$是一个$k$同态.如果把$K$和$L$均视为$k$线性空间,则$f$可以作为它们之间的线性映射.并且按照域同态总是单射,说明这样的线性映射总是单的.另外如果$[K:k]=[L:k]<\infty$,按照线性空间维数性质说明此时$K=L$.特别的,如果$K=L$,此时$k$同态$f$自动是一个双射,此时称$f$为$K$自同构.全体$K$上的$k$自同构构成一个群,称为扩张$k\subseteq K$的Galois群,记作$\mathrm{Aut}_kK$.

如果$F=k(X)$,那么$k\subseteq F$的自同构可以被它在集合$X$下的限制映射完全决定.即如果$\sigma,\tau\in\mathrm{Aut}_kF$满足$\sigma\mid_X=\tau\mid_X$,那么$\sigma=\tau$.这只要按照$F$中的元可以表示为$k$系数的以$X$中元为未定元的分式即得证.

另一个有用的事实是扩张自同构限制在基域的多项式$f$在扩域中的根集上是置换映射.我们在前文讨论过这一事实.据此可以说明一个有限扩张的Galois群是有限群.事实上有限扩张$k\subseteq F$总可以表示为$F=k(a_1,a_2,\cdots,a_n)$.每一个扩张自同构被它在$a_i$这$n$个元上的取值所决定,但是每个元都是代数元,它的极小多项式在扩域中的根集有限,这就说明扩张自同构至多有有限个.

Galois群的一些具体例子.
\begin{enumerate}
	\item $\mathbb{R}\subset\mathbb{C}$的Galois群是$\{\mathrm{id},\sigma\}$,其中$\sigma$是复共轭映射.首先它们的确都是扩张的自同构,为了说明扩张不存在其他的自同构,只要注意到$\mathbb{C}=\mathbb{R}(i)$,于是扩张的自同构被它在$i$下的像完全决定.但是$i$的极小多项式是$x^2+1$,恰好有两个根$\pm i$,于是该扩张至多有两个自同构.
	\item $\mathbb{Q}\subset\mathbb{Q}(\sqrt[3]{2})$的Galois群是$\{\mathrm{id}\}$.$\sqrt[3]{2}$在$\mathbb{Q}$上的极小多项式是$x^3-2$,如果记$\omega=e^{2\pi i/3}$,那么这个多项式的全部根是$\{\sqrt[3]{2},\omega\sqrt[3]{2},\omega^2\sqrt[3]{2}\}$.但是它们当中只有$\sqrt[3]{2}$一个元落在扩域中,于是扩张自同构只有一个.
	\item 上一条这个扩张的正规闭包是$\mathbb{Q}(\sqrt[3]{2},\omega)$,其中$\omega$是三次本原根,此时扩张维数是6(这一事实需要验证$\sqrt[3]{2}$不能表示为$1,\omega,\omega^2$的$\mathbb{Q}$线性组合,这只要在等式两侧取实部).那么扩张的自同构群恰好有六个元:自同构被扩张元的像所决定,而$\sqrt[3]{2}$有三种选择,$\omega$有两种选择,于是它们的复合一共得到六个自同构.
	\item 记$\mathbb{F}_2$上的有理分式域为$\mathbb{F}_2(t)$,扩张$\mathbb{F}_2(t^2)\subset\mathbb{F}_2(t)$的Galois群是$\{\mathrm{id}\}$.
\end{enumerate}

关于$\mathbb{Q},\mathbb{R},\mathbb{C}$上的域自同构.
\begin{enumerate}
	\item $\mathrm{Aut}(\mathbb{Q})=\{\mathrm{id}\}$.设自同构为$f$,从$f(1)=1$得到$f(n)=n,n\in\mathbb{Z}$,再按照分式化的泛映射性质,得到唯一的延拓$f(p/q)=p/q$.
	\item $\mathrm{Aut}(\mathbb{R})=\{\mathrm{id}\}$.设自同构为$f$,上一条已经说明$f$限制在$\mathbb{Q}$上是恒等的.关键是证明$f$是单调函数,于是按照有理数两侧单调逼近无理数,就得到$f$实际上是恒等.单调是因为对正实数$a=b^2$,有$f(a)=f^2(b)\ge0$,于是$a\ge b$时候$f(a)=f(b)+f(a-b)\ge f(b)$.
	\item $|\mathrm{Aut}(\mathbb{C})|=2^c$.首先注意到$c^c=(2^{\aleph_0})^c=2^{\aleph_0\cdot c}=2^c$.另外$\mathbb{C}$上的映射基数是$c^c$,所以只需说明自同构群的基数$\ge c^c$.取$\mathbb{Q}\subset\mathbb{C}$的代数闭包$K$,那么$|K|=\aleph_0$.于是$K\subset\mathbb{C}$是纯超越扩张,并且超越维数是$c$.超越基上的不同同构诱导不同的扩张自同构,而基数$a$的集合上的双射的基数是$a^a$,于是这就得到$c^c$个不同的$\mathbb{C}$上的自同构.
\end{enumerate}

Galois对应.给定扩张$k\subseteq F$,它的Galois群记作$G$.所谓Galois对应是指从扩张的中间域到$G$的子群之间的对应.一方面,取定中间域$E$,那么$\mathrm{Aut}_EF$是$G$的子群.另一方面,取定$G$的子群$H$,记$\mathscr{F}(H)=\{a\in F\mid\forall\sigma\in H,\sigma(a)=a\}$,于是它是扩张的一个中间域.我们接下来的目标就是探究何时这个对应是一个双射.

首先给出这个对应的一些基本性质.
\begin{enumerate}
	\item 如果$k\subseteq L_1\subseteq L_2\subseteq F$,那么$\mathrm{Aut}_{L_1}F\subset\mathrm{Aut}_{L_2}F$;如果$S_1\subseteq S_2\subseteq G$,那么$\mathscr{F}(S_2)\subset\mathscr{F}(S_1)$.这说明我们期望得到的双射是从中间域集到$G$子群集的反序双射.
	\item 对应的二次复合会把点集变大:如果$L$是中间域,那么有$L\subset\mathscr{F}(\mathrm{Aut}_LF)$;如果$S$是$G$的子群(子集),那么有$S\subset\mathrm{Aut}_{\mathscr{F}(S)}F$.
	\item 对应的一次复合与三次复合相同:如果$L$是中间域,那么$\mathrm{Aut}_LF=\mathrm{Aut}_{\mathscr{F}(\mathrm{Aut}_kL)}F$;如果$S$是$G$的子群(子集),那么$\mathscr{F}(S)=\mathscr{F}(\mathrm{Aut}_{\mathscr{F}(S)}F)$.
	\item 第三条告诉我们,Galois对应如果限制在全体具有形式$\mathscr{F}(S)$,其中$S$是子群(子集)的中间域和全体具有形式$\mathrm{Aut}_LF$,其中$L$是中间域,上是反序的一一对应.
\end{enumerate}

我们接下来给出更加精细的对有限扩张Galois群阶数的信息.为此先要介绍乘性特征和戴德金引理.群$G$在域$K$上的一个乘性特征是指从$G$到去掉0元的乘法群$K^*$的群同态.关于乘性特征有戴德金引理:一个群上的不同乘性特征总是在域上线性无关的.即如果$G$在域$K$上的不同乘性特征$\{\tau_i\}$,如果存在$K$中的系数$\{c_i\}$满足$\sum_i c_i\tau_i(g)=0,\forall g\in G$,那么有系数$c_i\equiv0$.
\begin{proof}
	
	假设结论不成立,设反例中最小的是$k$个两两不同的乘性特征$\{\tau_i,1\le i\le k\}$,于是存在全不为0的系数$\{c_i,1\le i\le k\}$满足$\sum_ic_i\tau_i(g)=0,\forall g\in G$.这里$\{c_i\}$全不为0是因为不然会和$k$的最小性矛盾.
	
	按照$\tau_1\not=\tau_2$,可以找到一个$h\in G$使得$\tau_1(h)\not=\tau_2(h)$.按照$\sum_i(c_i\tau_1(h))\tau_i(g)=0,\forall g\in G$和$\sum_i(c_i\tau_i(h))\tau_i(g)=0,\forall g\in G$.得到$\sum_ic_i(\tau_1(h)-\tau_i(h))\tau_i(g)=0,\forall g\in G$.这里的系数不全为0,导致和$k$的最小性矛盾.
\end{proof}

戴德金引理还可以描述成线性空间版本.设全体从群$G$到域$K$的映射构成的集合为$V$,那么$V$按照加法和数乘成为$K$上的线性空间.那么戴德金引理说的就是$G$到$K$的乘性特征集是$V$的线性无关集.

给定有限扩张$k\subseteq F$,那么有$|\mathrm{Aut}_kF|\le[F:k]$.
\begin{proof}
	
	我们有证明过有限扩张的Galois群是有限群.记$\mathrm{Aut}_kF=\{r_1,\cdots,r_n\}$,假设有$[F:k]=m<n$,取一组$F$在$k$上的基为$a_1,\cdots,a_m$,取$F$上矩阵:
	$$A=\left(\begin{array}{cccc}
	r_1(a_1)&r_1(a_2)&\cdots&r_1(a_m)\\
	r_2(a_1)&r_2(a_2)&\cdots&r_2(a_m)\\
	\vdots&\vdots&\ddots&\vdots\\
	r_n(a_1)&r_n(a_2)&\cdots&r_n(a_m)
	\end{array}\right)$$
	
	那么按照$r(A)\le m<n$,知道它的行向量组是线性相关的,于是存在$c_i\in F$满足$\sum_ic_ir_i(a_j)=0,\forall j$.对$F$中每个非0元$g$,存在$b_i\in k$满足$g=\sum_jb_ja_j$,于是有:
	$$\sum_ic_ir_i(g)=\sum_ic_ir_i(\sum_jb_ja_j)=\sum_jb_j\left(\sum_ic_ir_i(a_j)\right)=0$$
	
	但是注意到每个$r_i$都是群$F^*$在$F$上的不同乘性特征,按照戴德金引理这些特征标在$F$上应该是线性无关的,这就矛盾.
\end{proof}

我们继续回到Galois对应何时成立这个问题.我们接下来要证明的引理是,如果把扩张$k\subseteq F$限制为有限扩张,对于每个Galois群的子群$G$,有$|G|=[F:\mathscr{F}(G)]$和$G=\mathrm{Aut}_{\mathscr{F}(G)}F$.这一事实还可以说明此时从中间域到扩张Galois群子群的映射是满射.为此我们证明如下更一般的结果:

阿廷引理.设$G$是域$F$的自同构构成的一个有限子群,设$k=\mathscr{F}(G)$,那么$F$可视为$k$的域扩张,此时有$|G|=[F:k]$和$G=\mathrm{Aut}_kF$.
\begin{proof}
		
按照上一定理,由于$G\subset\mathrm{Aut}_kF$,说明有$|G|\le[F:k]$.现在假设$|G|<[F:k]$,记前者为$n$,取$F$中的$n+1$个元$\{a_1,a_2,\cdots,a_{n+1}\}$在$k$上线性无关.记$G=\{r_1,r_2,\cdots,r_n\}$.考虑$F$上的矩阵:
$$A=\left(\begin{array}{cccc}
r_1(a_1)&r_1(a_2)&\cdots&r_1(a_{n+1})\\
r_2(a_1)&r_2(a_2)&\cdots&r_2(a_{n+1})\\
\vdots&\vdots&\ddots&\vdots\\
r_n(a_1)&r_n(a_2)&\cdots&r_n(a_{n+1})
\end{array}\right)$$
		
那么$A$的列向量组是线性相关的,设列向量组秩为$k$,不妨重排$a_i$使得它们由前$k$个列向量构成,于是存在$F$中的一组不全为0的系数$\{c_i,1\le i\le k\}$满足$\sum c_ir_j(a_i)=0,\forall j$成立.按照$k$的最小性说明$c_i,1\le i\le k$实际上全不为0,于是我们可以不妨设$c_1=1$,否则以$\frac{c_i}{c_1}$替代$c_i$.如果每个$c_i$都落在$k$中,那么对每个$j$有$r_j(\sum_ic_ia_i)=0$,得到$\sum_ic_ia_i=0$,这会和$\{a_i\}$的线性无关性矛盾.
		
现在任取$G$中的元$\sigma$,按照左乘$\sigma$是$G$上的一个置换,说明有$\sum_i\sigma(c_i)r_j(a_i)=0,\forall j$成立.按照$c_1=1$说明了$\sum_{2\le i\le k}(c_i-\sigma(c_i))r_j(a_0)=0$对任意$j$成立.按照$k$的最小性,说明每个$c_i-\sigma(c_i)=0$.既然这个等式对每个$\sigma\in G$成立,说明每个$c_i\in\mathscr{F}(G)=k$.但是我们已经看到这一事实可以推出矛盾,综上有$|G|=[F:k]$.此时按照$|G|=[F:k]\ge|\mathrm{Aut}_kF|$就得到$G=\mathrm{Aut}_kF$.
\end{proof}

我们现在给出Galois扩张的两个等价描述.我们一共会给出三种等价描述.一个扩张$k\subseteq F$称为Galois扩张如果满足$k=\mathscr{F}(\mathrm{Aut}_kF)$,它等价于这个有限扩张满足$|\mathrm{Aut}_kF|=[F:k]$.
\begin{proof}
	
	一方面如果$k=\mathscr{F}(\mathrm{Aut}_kF)$,按照上一个结果得到$[F:k]=|\mathrm{Aut}_kF|$.另一方面如果阶数和次数的等式成立,取$L=\mathscr{F}(\mathrm{Aut}_kF)$,按照上一个结果得到$\mathrm{Aut}_kF=\mathrm{Aut}_LF$,于是$|\mathrm{Aut}_kF|=[F:L]\le[F:k]$,但是按照条件等式,就迫使$[F:L]=[F:k]$,于是$L=k$.
\end{proof}

对于一个有限单扩张$k\subseteq k(a)$,判断它是Galois扩张是比较容易的.我们已经讨论过此时扩张Galois群的阶数就是$a$的极小多项式在扩域$k(a)$中的不同根的个数.于是它是Galois扩张当且仅当极小多项式在扩域中有次数个不同根.我们知道这等价于可分正规扩张.

这里给出第三个等价描述,扩张$k\subseteq F$是Galois扩张当且仅当它是可分正规扩张.注意这里必要性实际上没有用到有限扩张这个条件.
\begin{proof}
	
	一方面如果扩张是Galois扩张,任取$a\in F$,设$\mathrm{Aut}_kF$作用在$a$上得到的像集合是$\{a_i\}$,考虑多项式$f(x)=\prod_i(x-a_i)\in F[x]$.对$\mathrm{Aut}_kF$中每个元$\tau$,它限制在$\{a_i\}$上是置换,这说明$\tau f(x)=f(x)$,由于$\tau\in\mathrm{Aut}_kF$的任意性,说明$f(x)$的系数都落在$\mathscr{F}(\mathrm{Aut}_kF)=k$中,即$f(x)\in k[x]$.现在$a$的极小多项式整除$f(x)$,于是这个极小多项式是可分的,并且它的全部次数个根都落在扩域$F$中.让$a$取遍$F$中的元,这就说明扩张是一个可分正规扩张.
	
	另一方面假设扩张是可分正规扩张,这等价于它是一族可分多项式的分裂域.先证明有限扩张情况,为此对$n=[F:k]$归纳.$n=1$没有什么需要证的,假设$n>1$,并且对次数小于$n$的有限扩张均能推出扩张是Galois扩张.设$F$是$k$的关于可分多项式$f(x)$的分裂域.可取$f$的某个根$a\in F$使得$a\not\in k$.记$L=k(a)$,那么$[L:k]>1$于是$[F:L]<n$.$F$同样是$L$关于可分多项式$f(x)$的分裂域,于是按照归纳假设有$L\subseteq F$是Galois扩张.记扩张Galois群为$H$,它是$G=\mathrm{Aut}_kF$的子群.记$a$在$k$上极小多项式在$F$中的全部不同根为$\{a_1,a_2,\cdots,a_r\}$,其中$r=[L:k]$.那么存在$G$中的元$r_i$使得$r_i(a)=a_i,1\le i\le r$.我们断言陪集$r_iH$两两不同,因为从$r_i^{-1}r_j\in H$会得到$r_i(a)=r_j(a)$,得到$i=j$.于是得到如下不等式,这就得到$|G|=[F:k]$,于是扩张是Galois扩张.
	$$|G|=[G:H]\cdot|H|\ge r\cdot|H|=[L:k]\cdot[F:L]=[F:k]$$
	
	现在说明扩张次数未必有限时扩张仍然为Galois扩张.此时$F$是$k$关于一族可分多项式的集合$S$的分裂域.记$S$中所有多项式的所有根构成的集合是$X$,那么$F=k(X)$.任取$a\in\mathscr{F}(G)$,我们希望证明$a\in F$.为此可取有限个$X$中的元$\{a_1,a_2,\cdots,a_n\}$使得$a\in F(a_1,a_2,\cdots,a_n)$.记$L$是$k$关于每个$a_i$的极小多项式构成的可分多项式集的分裂域,那么$L$是扩张的中间域,按照有限扩张的情况,得到$k\subseteq L$是Galois扩张,并且$a\in L$.按照同构扩张定理,$\mathrm{Aut}_kL$中每个元可以延拓为$G$中的元,于是$\mathrm{Aut}_kL$可视为$G$中的元在$L$上的限制.于是$a\in\mathscr{F}(\mathrm{Aut}_kL)=k$,于是$k\subseteq F$是Galois扩张.
\end{proof}

至此我们给出了Galois扩张的三个等价描述,现在我们证明Galois理论基本定理,即对于有限Galois扩张,有Galois对应是双射.
\begin{proof}
	
	任取扩张的中间域$L$,那么$F$同样是$L$上的可分正规扩张.于是有$L=\mathscr{F}(\mathrm{Aut}_LF)$.结合有限扩张下对扩张Galois群的子群$G$恒有$G=\mathrm{Aut}_{\mathscr{F}(G)}F$,这就说明Galois对应的两个方向的映射互为逆映射,即Galois对应是双射.
\end{proof}

有限Galois扩张还有如下一些基本性质,有的教材里它们会归为基本定理的内容.
\begin{enumerate}
	\item 在有限Galois扩张$k\subseteq F$下,如果中间域$L$对应子群$H$,那么有$[F:L]=|H|$和$[L:k]=[G:H]$.事实上按照$L\subseteq F$是Galois扩张得到$|H|=[F:L]$,按照$|G|=|H|\cdot[G:H]$和$[F:k]=[F:L]\cdot[L:k]$就得到$[L:k]=[G:H]$.
	\item 给定有限Galois扩张$k\subseteq F$,取中间域$E_1,E_2$和Galois对应下的子群$G_1,G_2$,有在Galois对应下$E_1\cap E_2$对应于$<G_1,G_2>$,$E_1E_2$对应于$G_1\cap G_2$.
	\item 给定有限Galois扩张$k\subseteq F$,取中间域$L$,那么$L\subseteq F$是Galois扩张,但是一般来讲$k\subseteq L$不是Galois扩张,它是Galois扩张当且仅当$L$对应的子群$H=\mathrm{Aut}_LF$是$G=\mathrm{Aut}_kF$的正规子群,并且此时$k\subseteq L$的Galois群$\mathrm{Aut}_kL$就是商群$G/H$.
	\begin{proof}
		
		假设$H$是$G$的正规子群,记$L=\mathscr{F}(H)$.任取$a\in L$,任取$a$的极小多项式在$F$中的任意一个根$b$.按照同构扩张定理,存在$G$中的元$\sigma$使得$\sigma(a)=b$.任取$\tau\in H$,那么有$\tau(b)=\sigma(\sigma^{-1}\tau\sigma(a))$.但是正规性说明$\sigma^{-1}\tau\sigma\in H$,于是$a$在它下的取值是$a$,于是$\tau(b)=\sigma(a)=b$.于是$b\in\mathscr{F}(H)=L$.于是$k\subseteq L$是正规扩张,它是可分扩张因为$k\subseteq F$是可分扩张,这就说明了$k\subseteq L$是Galois扩张.
		
		反过来,假设$L$是$k$的Galois扩张.定义映射$\theta:G\to\mathrm{Aut}_kL$为把$G$中的映射$\sigma$限制到$L$上.按照$k\subseteq L$是正规扩张说明这个限制映射的确是$\mathrm{Aut}_kL$上的映射.验证它是群同态,同构扩张定理说明它是满同态,它的核$\ker\theta$就是$\mathrm{Aut}_LF=H$,于是按照群同构定理得到$\mathrm{Aut}_kL\cong G/H$.
	\end{proof}
    \item 如果$k\subseteq F$是Galois扩张,$k\subseteq K$是任意有限扩张,那么$K\subseteq KF$是有限Galois扩张,并且满足$\mathrm{Aut}_KKF\cong\mathrm{Aut}_{F\cap K}F$.
    \begin{proof}
    	
    	因为$k\subseteq F$是有限Galois扩张,于是$F$可视为$k$关于一个可分多项式$f(x)$的分裂域,$f(x)$同样是$K[x]$上的多项式,于是$KF$可视为$K$关于$f(x)$的分裂域,于是$K\subseteq KF$是Galois扩张.
    	
    	取$\sigma\in\mathrm{Aut}_KKF$,它可以限制为$F\to\sigma(F)$的域同态,按照$k\subseteq F$是正规扩张,说明$\sigma(F)=F$.于是做限制得到了一个群同态$\rho:\mathrm{Aut}_KKF\to\mathrm{Aut}_kF$.
    	
    	现在验证它是单射,对$\sigma\in\mathrm{Aut}_KKF$,它已经在$K$上恒等,倘若它在$F$上也恒等,说明在$KF$上也恒等,于是$\rho$是单射.因此$\mathrm{Aut}_KKF$可视为$\mathrm{Aut}_kF=G$的子群.
    	
    	对每个$\sigma\in\mathrm{Aut}_KKF$,它已经在$K$上恒等,于是限制映射$\rho(\sigma)$在$F\cap K$上恒等,说明了$F\cap K\subset\mathscr{F}(G)$.反过来,假设$a\in\mathscr{F}(G)$,那么$K(a)$落在$\mathrm{Aut}_KKF$的固定子域中.单射按照$K\subseteq KF$是Galois扩张,说明这个固定子域就是$K$自身.即$a\in K$,综上得到$\mathscr{F}(G)=F\cap K$,这就说明了$G=\mathrm{Aut}_{F\cap K}F$.
    \end{proof}
\end{enumerate}

例子.假设$k\subseteq F$是有限可分扩张,设正规闭包为$N$,记$G=\mathrm{Aut}_kN$,$H=\mathrm{Aut}_FN$.假设$k\subseteq F$不是Galois扩张,那么按照上述定理有$H$不会是$G$的正规子群.此时我们可以说明$H$唯一包含的$G$的正规子群是平凡群.这是因为倘若$H'$是$H$包含的$G$的正规子群,按照基本定理,它对应于中间域$k\subseteq F\subseteq L\subseteq N$的$L$,并且$k\subseteq L$会是一个Galois扩张,特别的它是一个正规扩张,但是按照正规闭包的性质,此时只能有$L=N$,即$|H'|=[N:L]=1$,于是$H'$是平凡群.另外群论告诉我们包含于$H$的$G$的最大的正规子群是$\cap_{g\in G}gHg^{-1}$,于是上述讨论说明它是平凡群.

一元有理函数域.给定域$k$,称$k[t]$的商域$k(t)$是$k$的一元有理函数域.
\begin{enumerate}
	\item 设$u=f(t)/g(t)\in k(t)$,其中$(f,g)=1$,那么$k(u)\subseteq k(t)$的扩张维数是$\max\{\deg f(x),\deg g(x)\}$.
	\begin{proof}
		
		记$K=k(u)$,$F=k(t)$.考虑$K$上多项式$p(x)=ug(x)-f(x)$,那么$t$是$p(x)$的一个根,于是$t$是$K$的代数元,于是扩张$K\subseteq F$是有限扩张.
		
		现在注意到$p(x)$的次数恰好就是$\max\{\deg f(x),\deg g(x)\}$.于是只需说明$p(x)$是$K$上不可约多项式,就说明扩张$K\subseteq F=K(t)$的维数恰好是$f,g$次数的最大值.
		
		首先$u$不会是$k$上的代数元,否则$[F:k]=[F:K][K:k]$是有限数.于是$k[u]\cong k[t]$.把$p$视为关于$u$的多项式,那么$p\in k[x][u]\subseteq k(x)[u]$是一次的,于是$p$在$k(x)$上不可约.再按照条件$(f,g)=1$,说明$p$视为$k[x]$上关于$u$的多项式是本原多项式,这就说明$p$是$k[x]$上的不可约多项式,最后按照$k[x][u]\cong k[u][x]$,就说明$p$是$k[u]$上的不可约多项式,于是$p$在$K=k(u)$上不可约.
	\end{proof}
    \item 任取$k\subseteq k(t)$的扩张自同构$\phi$,那么$\phi$被它在$t$的取值完全决定,如果记$u=\phi(t)\in k(t)$,那么上一条说明$[k(t):k(u)]=1$当且仅当$u$具有形式$\frac{ax+b}{cx+d}$,其中互素条件等价于要求$ad-bc\not=0$.于是存在$k$上可逆二元矩阵$\left(\begin{array}{cc} a&b\\c&d\end{array}\right)$到$\mathrm{Aut}_kk(t)$的满射,它的核是全体数量矩阵构成的正规子群,于是$\mathrm{Aut}_kk(t)$同构于$k$上二阶的投射一般线性群$\mathrm{PGL}_2(k)$.
\end{enumerate}
\newpage
\subsection{迹和范数}

有限扩张上做线性代数.取域扩张$k\subseteq F$,设$[F:k]=n$,任取$a\in F$,定义$F$上的映射$L_a:b\mapsto ab$,这是$k$线性空间$F$上的线性变换,当$a\not=0$时它是单射,于是可以在选定一组基时表示为基域$k$上的可逆矩阵.按照线性变换的迹和范数(行列式)不依赖基的选取,于是可定义有限扩张$k\subseteq F$上的范数映射和迹映射,它们取值都在基域:
$$N_{F/k}(a)=\det(L_a);\mathrm{T}_{F/k}(a)=Tr(L_a)$$

例如$\mathbb{R}\subset\mathbb{C}$,取定基$\{1,i\}$,那么复数$a+bi$在这组基下的矩阵表示为$(a+bi)(1,i)=(1,i)\left(\begin{array}{cc}
a&-b\\ b&a\end{array}\right)$,于是$N_{\mathbb{C}/\mathbb{R}}(a+bi)=a^2+b^2$,$\mathrm{T}_{\mathbb{C}/\mathbb{R}}(a+bi)=2a$.再比如对任意域$k$,取二次扩张$F=k(\sqrt{d})$,其中$d\in k-k^2$,那么$N_{F/k}(a+b\sqrt{d})=-d$,$\mathrm{T}_{F/k}(a+b\sqrt{d})=0$.

迹和范数的一些基本性质.设$k\subseteq F$是$n$次有限扩张.在取定一组基的前提下,$F$中的元可视为$k$上的$n$次方阵,如果$a,b\in F$分别对应的矩阵表示是$A,B$,那么$k_1a+k_2b,k_i\in k$对应的矩阵表示是$k_1A+k_2B$;$ab$对应的矩阵表示是$AB$(此时有$AB=BA$).这两个性质说明三个结果:迹映射是$k$线性空间$F$上的线性泛函;范数映射是$F\to k$的乘性映射;$k$中的元$a$在任一组$k\subseteq F$的基下的矩阵表示的对角矩阵$\mathrm{diag}\{a,a,\cdots,a\}$,于是它的迹是$na$,范数是$a^n$.

给定$n$次有限扩张$k\subseteq F$,任取$a\in F$,记$a$作为$F$上线性映射的极小多项式为$p(x)$,记$a$作为$k$上代数元的极小多项式为$q(x)$.那么$q(x)$是基域上的不可约多项式,并且按照上一段的基本性质,说明$p(x)\mid q(x)$.这迫使二者相同.另外按照线性代数理论,线性变换的极小多项式和特征多项式具有相同的不可约因式,差异仅仅是次数不同.而在这里的情况极小多项式本身是一个不可约多项式,于是如果记$a$的极小多项式的次数为$m$,那么有$m\mid n$并且$a$作为线性变换的特征多项式恰好就是$p(x)^{n/m}$.再结合线性代数理论中特征多项式的系数和迹与范数之间的联系,就得到:如果$a$的极小多项式是$p(x)=x^m+a_{m-1}x^{m-1}+\dots+a_0$,那么有$N_{F/k}(a)=(-1)^na_0^{n/m}$和$\mathrm{T}_{F/k}(a)=-\frac{n}{m}a_{m-1}$.这说明只要知道有限扩张扩域中元的极小多项式,那么可直接得到它的迹和范数.

我们接下来探究对于$n$次有限扩张$k\subseteq F$,当$a\in F$视为线性变换的时候,它的特征值具体都是什么.我们曾说明过给定有限扩张$k\subseteq F$,那么可分维数$[F:k]_s$就是$F$到$k$的某个固定的代数闭包的嵌入个数.记全部嵌入为$\{\sigma_1,\sigma_2,\cdots,\sigma_r\}$,对每个$a\in F$,我们断言此时$a$作为$k$线性变换的特征多项式恰好就是$g(x)=\left(\prod_j(x-\sigma_j(a))\right)^{[F:k]_i}$.
\begin{proof}
	
	证明中会用到我们在Galois扩张基本性质里给出过的如下结论:给定有限Galois扩张$k\subseteq F$,那么对任意的$k$的扩域$L$,有$L\subseteq FL$必然是有限Galois扩张,并且$Aut_L(FL)\cong Aut_{F\cap L}F$,并且$[FL:L]= [F:F\cap L]$.
	
	记$k$的一个代数闭包为$M$,那么可分维数$r=[F:k]_s$恰好就是从$F$到$M$的全体不同的$k$线性映射的个数.把这些线性映射记作$\sigma_1,\cdots,\sigma_r$.取$g(x)=\left(\prod_j(x-\sigma_j(a))\right)^{[F:k]_i}\in M[x]$.取$k$在$F$中的可分闭包为$S$,那么可分维数$r$同样是$[S:k]$.而$g$的次数为$r[F:k]_i=n$.
	
	我们断言$g(x)\in k[x]$,并且$g$和$a$的极小多项式$p$具有相同的根.如果这成立,那么按照$p$是基域的不可约多项式说明有$p$整除$g$,具有相同根还说明$p$是$g$的唯一的不可约因式,这得到$g(x)=p(x)^{n/m}$.我们曾说明过后者就是$a$作为$k$线性变换的特征多项式,这就会完成证明.
	
	先说明$g,p$具有相同根,一方面由于$a$是$p$的根,于是每个$\sigma_j(a)$都是$p$的根,反过来对$p$的任一个根$b$,按照同构扩张定理存在某个$\sigma_j$把$a$映射为$b$,这就说明二者具有相同的根.	
	
	最后只要证明$g(x)\in k[x]$.取有限扩张$k\subseteq S$的正规闭包$N$,那么$k\subseteq N$必然是有限Galois扩张.按照证明开头我们提到的定理,有$F\subseteq FN$是Galois扩张,并且$[FN:F]=[F:F\cap N]$整除$[N:S]$,于是$[FN:N]$整除$[F:S]=[F:k]_i$.按照$S\subseteq F$是纯不可分扩张,说明$N\subseteq FN$是纯不可分扩张,于是对任意$c\in FN$有$c^{[F:k]_i}\in N$.因为$S\subseteq FN$是一个Galois扩张和纯不可分扩张的复合,于是他是一个正规扩张,于是$\sigma_j(F)\subseteq FN$.再结合$(FN)^{[F:k]_i}\subseteq N$,于是$g(x)\in N[x]$,这导致$g$的系数都是$k$上可分元.现在对于任意$\mathrm{Aut}_NM$中的元$\gamma$,有$\gamma\circ\sigma_i,1\le i\le r$在$K$上的限制恰好就是全体$\sigma_i,1\le i\le r$.这导致$g(x)$在$\gamma$下不变,于是$g$的系数落在$\mathrm{Aut}_kM$的固定域中,这个固定域是$k$在$M$中的纯不可分闭包,于是$g$的系数是$k$上还是纯不可分元.$g$的系数同时是$k$上可分元和纯不可分元,这就得到$g(x)\in k[x]$.
\end{proof}

上述特征多项式的表示可以提供多个结论:
\begin{enumerate}
	\item 对有限扩张$k\subseteq F$,记全部从$F$到$k$的某个固定代数闭包中的不同嵌入为$\{\sigma_1,\sigma_2,\cdots,\sigma_r\}$,将$a\in F$视为$k$线性变换.记特征多项式$g(x)=x^n+r_ {n-1}x^{n-1}+\cdots+r_0$,则有$N_{F/k}(a)=(-1)^nr_0$,$\mathrm{T}_{F/k}(a)=-r_{n-1}$.而分解$g(x)$上述的表示得到$r_0=(\prod_j(-\sigma_j(a)))^{[F:k]_i}$和$r_{n-1}=-[F:k]_i\sum_j\sigma_j(a)$,这就提供迹和范数的等价定义:
	$$N_{F/k}(a)=\left(\sigma_1(a)\sigma_2(a)\cdots\sigma_r(a)\right)^{[F:k]_i};\mathrm{T}_{F/k}(a)=[F:k]_i\left(\sigma_1(a)+\sigma_2(a)+\cdots+\sigma_r(a)\right)$$
	\item 如果有限扩张$k\subseteq F$是Galois扩张,那么这些不同嵌入恰好就是Galois群$G$中的全部元,并且此时有$[F:k]_i=1$,于是此时迹和范数表示为:
	$$N_{F/k}(a)=\prod_{\sigma\in G}\sigma(a);\mathrm{T}_{F/k}(a)=\sum_{\sigma\in G}\sigma(a)$$
	\item 在扩张链上迹和范数满足复合,即给定域扩张$k\subseteq L\subseteq F$,并且$[F:k]<\infty$,那么对任意$a\in F$有:
	$$N_{F/k}=N_{L/k}\circ N_{F/L}(a),\mathrm{T}_{F/k}=\mathrm{T}_{L/k}\circ \mathrm{T}_{F/L}(a)$$
	\begin{proof}
		
	我们在扩张链的可分维数的公式里已经证明了如下事实:取$k$的代数闭包为$M$,那么$M$同样是$L$和$F$的代数闭包,取从$L$到$M$的全体不同的$k$域同态$\sigma_1,\cdots,\sigma_r$,再取全体不同的从$F$到$M$的$L$域同态$\tau_1,\cdots,\tau_s$.按照分裂域的唯一性,可把所有$\sigma_i$和$\tau_j$都唯一的延拓为$M$上的域同态,采取相同的记号.那么每个$F$到$M$的$k$域同态都具有形式$\sigma_i\circ\tau_j\mid_F$.
		
	于是按照范数和迹的第二个等价定义,得到:
	$$N_{F/k}(a)=\left(\prod_{i,j}\sigma_i\tau_j(a)\right)^{[F:k]_i}=\left(\prod_i\sigma_i\left(\prod_j\tau_j(a)\right)^{[F:L]_i}\right)^{[F:k]_i}=N_{L/k}(N_{F/L}(a))$$
	$$\mathrm{T}_{F/k}(a)=[F:k]_i\sum_{i,j}\sigma_i\tau_j(a)=[L:k]_i\left(\sum_i\sigma_i\left([F:L]_i\sum_j\tau_j(a)\right)\right)=\mathrm{T}_{L/k}(\mathrm{T}_{F/L}(a))$$
	\end{proof}
\end{enumerate}

判别式.给定特征零的域上的有限扩张$k\subseteq F$,设维数是$n$,设$\{\sigma_1,\sigma_2,\cdots,\sigma_n\}$是全部$F\to\overline{k}$的$k$嵌入.任取$a_1,a_2,\cdots,a_n\in F$,定义它们的判别式为$d_{F/k}(a_1,a_2,\cdots,a_n)=|(\sigma_i(a_j))|^2$.
\begin{enumerate}
	\item 按照定义判别式和这组元的顺序无关.
	\item 按照矩阵和行列式的关系,得到$d_{F/k}(a_1,a_2,\cdots,a_n)=|AA^T|=|(\mathrm{T}_{F/k}(a_ia_j))|$,其中$A=(\sigma_i(a_j))$.特别的,这说明$n$个元的判别式总是基域中的元.
	\item 判别式$d_{F/k}(a_1,a_2,\cdots,a_n)\not=0$当且仅当$\{a_1,a_2,\cdots,a_n\}$是$F$的一组$k$基.
	\begin{proof}
		
		必要性,如果$a_1,a_2,\cdots,a_n$是线性相关的,那么方阵$(\sigma_i(a_j))$的行向量组是线性相关的,于是判别式为零.充分性,如果判别式为零,那么$(\sigma_i(a_j))$的行向量组$\{R_1,R_2,\cdots,R_n\}$是线性相关的,于是存在不全为零的$k_1,k_2,\cdots,k_n\in k$使得$\sum_ik_iR_i$是零行向量.按照$\{a_1,a_2,\cdots,a_n\}$是一组$k$基,说明$a=\sum_ik_ia_i$非零元.按照$\sum_ik_iR_i=0$得到$\mathrm{T}_{F/k}(aa_j)=0$对每个$j$成立.于是对任意$F$中的元$b$都有$\mathrm{T}_{F/k}(ab)=0$,特别的取$b=a^{-1}$得到$\mathrm{T}_{F/k}(1_k)=n1_k=0$,这矛盾.
	\end{proof}
    \item 如果$(a_1,a_2,\cdots,a_n)$和$(b_1,b_2,\cdots,b_n)$是两组$F$上的$k$基,设$(a_1,a_2,\cdots,a_n)=(b_1,b_2,\cdots,b_n)A$,这里$A$是一个$k$上的可逆方阵,那么$d_{F/k}(a_1,a_2,\cdots,a_n)=|A|^2d_{F/k}(b_1,b_2,\cdots,b_n)$.
    \item 按照本原元定理,记$F=k(\alpha)$,那么$\alpha$的极小多项式的次数是$n$,记全部$n$个不同的共轭元是$a_1=\alpha,a_2,\cdots,a_n$,此时有$d_{F/k}(1,\alpha,\cdots,\alpha^{n-1})=\prod_{1\le i<j\le n}(a_i-a_j)^2=(-1)^{\frac{n(n-1)}{2}}\mathrm{N}_{F/k}(f'(\alpha))$.
    \item 对$\alpha\in F$,定义$\alpha$的判别式$d(\alpha)=d_{F/k}(1,\alpha,\cdots,\alpha^{n-1})$.那么$d(\alpha)\not=0$当且仅当$1,\alpha,\cdots,\alpha^{n-1}$是一组$k$基,当且仅当$F=k(\alpha)$.
\end{enumerate}

给定有限域扩张$k\subseteq F$,迹映射$\mathrm{T}_{F/k}$诱导了一个二次型$(-,-):F\times F\to k$为$(x,y)=\mathrm{T}_{F/k}(xy)$.它称为扩张的迹二次型.一个二次型称为非退化的,如果对每个非零的$x\in F$,都存在某个$y\in F$使得$(x,y)\not=0$.

可分性,迹映射和迹二次型的联系.一个有限扩张$k\subseteq F$的如下条件互相等价:
\begin{enumerate}
	\item $k\subseteq F$是可分扩张.
	\item 迹映射$\mathrm{T}_{F/k}(-)$不是零映射.
	\item 迹二次型$(-,-)$是非退化的.
\end{enumerate}
\begin{proof}

2推1,如果$k\subseteq F$不是可分的,那么可设$k$的特征是素数$p$,取$k$在$F$中的可分闭包为$S$,那么$S\not=F$,并且$S\subseteq K$是一个纯不可分扩张,可设$[K:S]=p^t,t\ge1$.那么对于$a\in F$,有$\mathrm{T}_{F/k}(a)=\mathrm{T}_{S/k}(\mathrm{T}_{F/S}(a))$.设$\sigma_1,\cdots,\sigma_r$是不同的从$K$到$F$代数闭包的$S$线性映射,那么有如下等式.但是按照特征是$p$,得到$\mathrm{T}_{K/S}(a)=0$,于是$\mathrm{T}_{K/F}$总是一个零映射.
$$\mathrm{T}_{K/S}(a)=p^t(\sigma_1(a)+\cdots+\sigma_r(a))$$

1推2,假设$F$在$k$上是有限可分扩张.记$k\subseteq F$的正规闭包是$N$.倘若证明了$\mathrm{T}_{N/k}$是零映射,按照扩张链的迹映射的关系式,就得到$\mathrm{T}_{F/k}$是零映射.注意此时$k\subseteq N$是Galois扩张,记Galois群为$G=\{\sigma_1,\sigma_2,\cdots,\sigma_n\}$.任取$a\in N$,就有$\mathrm{T}_{N/k}(a)=\sum_i\sigma_i(a)$.但是每个$\sigma_i$是$N$的乘法群在域$k$上的乘性特征,我们曾证明过戴德金引理,即不同乘性特征总是线性无关的,这说明迹映射不能是零映射.完成证明.

最后如果迹映射是零映射自然得到迹二次型是退化的.反过来如果迹映射不是零映射,可取$a\in F$使得$\mathrm{T}_{F/k}(a)\not=0$,于是对每个非零的$b\in F$,就有$(b,a/b)=\mathrm{T}_{F/k}(a)\not=0$.
\end{proof}

迹二次型诱导的对偶基.设$k\subseteq F$是$n$次可分扩张,于是上一定理说明迹二次型$(-,-)$是非退化的.任取$k$线性空间$F$的一组基$\{a_1,a_2,\cdots,a_n\}$,它的对偶基定义为一组基$\{b_1,b_2,\cdots,b_n\}$,使得$(a_i,b_j)=\delta_{ij}$.
\begin{proof}
	
	这个二次型提供了一个$k$线性映射$\phi:F\to F^*$,即把$a\in F$映射为$\lambda_a:b\mapsto(b,a)$.于是这里非退化条件保证了$\phi$是一个单态射.按照有限维条件,得到$\dim F=\dim F^*$,于是$\phi$是一个线性同构.取$F^*$中$\{a_1,a_2,\cdots,a_n\}$对偶基的原像,满足结论.
\end{proof}
\newpage
\subsection{分圆扩张}

分圆扩张本质上是在一般域上添加本原根.分圆扩张的特点是提供了一般域上构造Galois群交换的Galois扩张,这样的扩张称为阿贝尔扩张.构造阿贝尔扩张并没有很多一般性构造方法,其他的构造诸如库莫尔扩张等都需要对基域添加一些特定的条件.

对每一个正整数$n$,我们知道复数域上$x^n=1$恰好有$n$个根,它们称为单位根,具有解析表示$\zeta_{n}^k,0\le k\le n-1$其中$\zeta_{n}=e^{\frac{2\pi i}{n}}$.分圆这个名字的来源便是这些单位根恰好把复平面的单位圆周等分为$n$份.这$n$个复数在乘法下乘法构成一个$n$阶循环群.称一个本原$n$阶单位根是指能够生成这个循环群的那些单位根,那么实际上它恰好是$(m,n)=1$的那些$\zeta_n^m$,一共有$\phi(n)$个,其中$\phi$是欧拉函数.

而在一般域$k$上,需要先固定一个代数闭包$\overline{k}$充当上面的复数域,称元$a\in\overline{k}$是$k$上$n$次单位根,如果存在某个正整数$n$满足$a^n=1$.如果$n$是最小的满足这个等式的正整数,则称$a$是$k$上的$n$次本原根.于是$a$作为本原根的次数$m$恰好就是它在乘法群$\overline{k}^*$上的阶数,于是总有$m\mid n$.反过来如果$m\mid n$,那么$a$总是一个$n$次单位根.注意当我们提及单位根或者本原根时,已经约定了它们属于某个预先约定的基域的代数闭包

但是和复数域的情况不同,一般域上涉及到特征的问题.如果$a$是特征$p>0$的域$k$上的$n$次单位根,这说明$a^n-1=0$.但是倘若$p\mid n$,可记$n=pm$,此时便有$0=a^n-1=(a^m-1)^p$,说明了$a^m-1=0$.这个事实说明两件事:一个是如果$a$是$k$上的$n$次本原根,那么必然有$p$不整除$n$;另一个是$k$关于$x^n-1$和$x^{pn}-1$的分裂域是相同的.

全体$n$次单位根构成了$\overline{k}^*$的子群,按照域的乘法群的有限子群总是循环群,于是这个群是循环群.当$p$不整除$n$时,这个循环群的生成元就是$n$次本原根.它的全部生成元的个数是$\phi(n)$,这里$\phi$是欧拉函数,表示的是小于$n$的正整数中和$n$互素的数的个数.

如果$k$的特征不整除正整数$n$,就把$k$关于$x^n-1$的分裂域$F$称为$n$次分圆扩张.那么$k\subseteq F$是Galois扩张,它可以表示为单扩张$F=k(a)$,其中$a$是$k$的任一个$n$次本原根.此时Galois群$G$同构于$\mathbb{Z}/n$乘法群$U(n)$的一个子群,因而$G$交换并且阶数整除$\phi(n)$.
\begin{proof}
	
	按照特征$p$不整除$n$,于是$x^n-1$的导师不为0,并且和$x^n-1$互素,这说明$x^n-1$是一个可分多项式.于是$k\subseteq F$是有限的可分正规扩张,于是它是Galois扩张.
	
	任取一个$n$次本原根$a$,则全部$n$次单位根都是它的某个次幂,于是$F$可以表示为单扩张$k(a)$.
	
	我们知道有限单扩张的自同构被它在扩张元上的取值唯一决定,而域同态是单射,说明它只能把$n$次本原根映射到$n$次本原根,如果$\sigma\in G$把$a$映射为$a^t$,那么约定$\sigma\mapsto [t]\in\mathbb{Z}/n$,这得到一个$G$到$(\mathbb{Z}/n)^*$的群同态$\theta$.这个群同态的核由满足$\sigma(a)=a$的$\sigma$构成,于是$\ker\theta=\{\mathrm{id}\}$.于是$G$是$\mathbb{Z}/n$的子群.
\end{proof}

这里我们举例说明$n$次分圆扩张的维数未必是$\phi(n)$,有时候是严格它的因数.设$k=\mathbb{F}_2$,记$a$是$k$上的7次本原根,于是$a$是多项式$x^7-1=(x-1)(x^3+x+1)(x^3+x^2+1)$的根.但是$a\not=1$,说明$a$的极小多项式是上述两个三次多项式之一.这说明$k$上的7次分圆扩张的维数不是$\phi(7)=6$,而是因数3.另一个例子是特征0的例子,考虑$\mathbb{R}$上的五次单位根$a$,那么$\mathbb{R}\subset\mathbb{R}(a)\subset\mathbb{C}$,这导致五次分圆扩张的维数$[\mathbb{R}(a):\mathbb{R}]$只能是2,它严格小于$\phi(5)=4$.

接下来我们讨论$\mathbb{Q}$上的分圆扩张,我们会看到此时扩张维数恰好就是$\phi(n)$,并且Galois群就是$(\mathbb{Z}/n)^*$.为此我们需要探究一种特殊的多项式.称$\Phi_n(x)=\prod_{1\le k\le n,(k,n)=1}(x-\zeta_n^k)$为$n$阶分圆多项式,它的次数就是欧拉函数值$\phi(n)$,即小于$n$的全部和$n$互素的正整数个数.

在$n$是素数的情况下有$\Phi_n(x)=\frac{x^n-1}{x-1}=x^{n-1}+\cdots+x+1$.这是一个整系数多项式,并且按照艾森斯坦判别法(需要先做代换x=y+1),它在$\mathbb{Q}$和$\mathbb{Z}$上不可约.这并不是巧合,我们接下来首要任务就是证明每一个分圆多项式都是首一整系数多项式,并且在$\mathbb{Q}$上(等价于$\mathbb{Z}$上)不可约多项式.

首先我们证明一个技术性引理.有恒等式$x^n-1=\prod_{1\le d\mid n}\Phi_d(x)$.这个公式的意义是它提供一种对分圆多项式阶数做归纳的做法.事实上,把等式两侧均分解为一次因式乘积,那么右侧的每个一次因式$(x-\zeta)$,其中$\zeta$必然是一个$n$次单位根,反过来,对于每一个$\zeta=\zeta_{n}^i$,记它在$n$次单位根群中的阶数是$d$,那么它对应于一个本原$d$次单位根,于是它只落在$\phi_d(x)$的展开式中,综上两侧的一次因式一一对应,这得到等式成立.

另外,考虑两边次数会得到一个等式$n=\sum_{1\le d\mid n}\phi(d)$.它的证明还可以直接考虑有限循环群$\mathbb{Z}/n$,注意到$\phi(m)$就是$\mathbb{Z}/m$中所有和$m$互素的元的个数,它们就是$\mathbb{Z}/m$中全部生成元.对每个正整数$d\mid n$,$\mathbb{Z}/n$存在唯一的$d$阶子群,这个子群的全部生成元个数是$\phi(d)$,注意到$\mathbb{Z}/n$中同一个元不能成为两个不同子群的生成元,因为阶数会不同,这立马说明$n=\sum_{1\le d\mid n}\phi(d)$.

现在证明分圆多项式都是整系数多项式.
\begin{proof}
	
	首先分圆多项式都是首一的多项式.$n=1$的时候分圆多项式就是$x-1$,没有什么需要证的.现在假设对于每个小于$n$的正整数,分圆多项式都是首一整系数多项式.于是$f(x)=\prod_{1\le d\mid n,d<n}\Phi_d(x)$是一个首一的整系数多项式.在$\mathbb{Z}[x]$中做带余除法:$x^n-1=f(x)q(x)+r(x)$.其中$q(x)$和$r(x)$都是整系数多项式.我们断言$r(x)=0$,$q(x)$就是$\phi_n(x)$.因为按照$x^n-1=\prod_{1\le d\mid n}\phi_d(x)$,于是$x^n-1=f(x)\phi_n(x)$.那么得到$f(x)(\phi_n(x)-q(x))=r(x)$.比较次数说明右侧只能是0,于是有$\phi_n(x)=q(x)$是整系数多项式.
\end{proof}

现在证明分圆多项式都是$\mathbb{Z}[x]$中的不可约多项式.
\begin{proof}
	
	若否,设某个$n$使得$\Phi_n(x)$是可约的多项式,设$\Phi_n(x)=f(x)g(x)$,其中$f(x)$是一个不可约多项式,现在取$f(x)$的一个根$\zeta_n^{m}$,又取$g(x)$的一个根$\zeta_n^s$,那么按照定义$m,s$都和$n$互素,我们断言可以取一个和$n$互素的素数$p$使得$\zeta_n^{mp}=\zeta_n^{s}$. 
	
	这等价于讲在mod $n$下找到一个素数$p$满足$pm=s$,而按照$s$和$n$互素,记$s$的逆元是$d$,则相当于在$n+d,2n+d,\cdots$中有素数,这是算术级数的Dirichlet定理.
	
	$f(x)$是$\zeta_n^m$的极小多项式,并且$g(\zeta_n^{mp})=0$,于是$\zeta_n^{m}$是$g(x^p)$的根,于是有$f(x)\mid g(x^p)$.记$g(x^p)=f(x)h(x)$,其中$h(x)\in\mathbb{Z}[x]$,现在把等式两侧mod $p$,得到在$\mathbb{F}_p[x]$中有:
	$$\left(\overline{g}(x)\right)^p=\overline{g}(x^p)=\overline{f}(x)\overline{h}(x)$$
	
	于是$\overline{f}(x)$和$\overline{g}(x)$必然有非平凡的公共因子$\overline{d}(x)$,于是$\overline{d}^2(x)\mid\overline{f}(x)\overline{g}(x)$.于是$\Phi_n(x)$ 在mod $p$下必然存在重因式,于是$x^n-1$作为$\mathbb{F}_p(x)$中的多项式必然存在重根.但是$x^n-1$的导数是$nx^{n-1}$,它们在$\mathbb{F}_p(x)$下互素,说明$x^n-1$是$\mathbb{F}_p(x)$的可分多项式,这矛盾.
\end{proof}

至此证明了分圆多项式总是整系数的并且是不可约的,于是单代数扩张$\mathbb{Q}\subseteq \mathbb{Q}(\zeta_n)$的极小多项式恰好就是$n$次分圆多项式,它的扩张维数是欧拉函数值$\phi(n)$.把$\mathbb{Q}(\zeta_n)$称为$n$次分圆域,扩张称为分圆扩张.另外分圆扩张$\mathbb{Q}\subset\mathbb{\zeta_n}$也等价于基域关于$x^n-1$的分裂域.

分圆扩张的Galois群.将证明它就是$\mathbb{Z}/n$的乘法群$U(n)$.
\begin{proof}
	
	首先阶数是可知的,回顾在前文中证明了一个有限的可分正规扩张的Galois群的阶数恰好就是扩张维数.于是分圆扩张$\mathbb{Q}\subseteq \mathbb{Q}(\zeta_n)$的Galois群的阶数恰好就是欧拉函数值$\phi(n)$.于是下面只需构造从$U(n)$到$\mathrm{Aut}_{\mathbb{Q}}(\mathbb{Q}(\zeta_n))$的单同态.
	
	对每个$m\in U(n)$,定义它诱导的扩张的自同构$f_m$是,在$\mathbb{Q}$上恒等,并且把$\zeta_n$映射为$\zeta_n^m$.验证它不依赖于$m$在mod $n$下的选取,并且是一个单同态.
\end{proof}
\newpage
\subsection{小于五次多项式的分裂域}

给定特征非2的域$F$,取多项式$f(x)\in F[x]$,取$F$关于$f$的某个分裂域上的$f$的全部根$\alpha_1,\cdots,\alpha_n$,记$\Delta=\prod_ {i<j}(\alpha_i-\alpha_j)\in K$,那么判别式$\mathrm{disc}(f)$定义为$\Delta^2$.给定特征非2的域$F$的一个代数扩张$K$,那么$K$中的元$\alpha$的判别式$\mathrm{disc}(\alpha)$约定为$\alpha$在$F$上极小多项式的判别式.

那么看到$\Delta$的定义依赖于根的顺序,会导致相差一个负号,而$\mathrm{Disc}$的定义不依赖于根的顺序,另外,如果多项式存在重根,那么判别式必然为0.如果考虑的是$F$上一个可分的不可约多项式,设分裂域$K$,那么可以把Galois群$Aut_FK$看作$S_n$的子群.

给定一个特征非2的域$F$,取不可约的可分多项式$f(x)\in F[x]$,设分裂域为$K$,取记号$\Delta$和$\mathrm{Disc}$同上,那么Galois群中的一个元$\sigma$是奇置换当且仅当$\sigma(\Delta)=-\Delta$,是偶置换当且仅当$\sigma(\Delta)=\Delta$.

Galois群落在交错群$A_n$上当且仅当有$\mathrm{Disc}(f)\in F^2$.并且在Galois对应下,中间域$F(\Delta)$对应于$G$的子群$G\cap A_n$.

为了计算判别式,考虑域$K$上$n$个元$\alpha_1,\cdots,\alpha_n$的Vandermonde矩阵:
$$V(\alpha_1,\cdots,\alpha_n)=\left(\begin{array}{ccccc}
1&\alpha_1&\alpha_1^2&\cdots&\alpha_1^{n-1}\\
1&\alpha_2&\alpha_2^2&\cdots&\alpha_2^{n-1}\\
\ddots&\ddots&\ddots&\vdots&\ddots\\
1&\alpha_n&\alpha_n^2&\cdots&\alpha_n^{n-1}
\end{array}\right)$$

那么判别式恰好就是$\det V(\alpha_1,\cdots,\alpha_n)^2$.按照矩阵技巧,记$n$个根的Vandermonde矩阵为$A$,那么$|A|^2$也同样是$A^TA$的行列式,而这个矩阵$A^TA$就是:
$$A^TA=\left(\begin{array}{cccc}
t_0&t_1&\cdots&t_{n-1}\\
t_1&t_2&\cdots&t_{n}\\
\ddots&\ddots&\vdots&\ddots\\
t_{n-1}&t_n&\cdots&t_{2n-2}
\end{array}\right),t_i=\sum_j\alpha_j^i$$

那么为了计算判别式,需要求出根的幂方和,一个多项式的系数就是根的初等对称多项式,固定几个元的初等对称多项式和幂方和是可以互相求出的,这就是牛顿公式,即,如果$f(x)=x^n+a_{n-1}x^{n-1}+\cdots+a_1x+a_0$,记根是$\alpha_i,1\le i\le n$.记$t_i$同上,那么:
$$t_m+a_{n-1}t_{m-1}+\cdots+a_{n-m+1}t_1+ma_{n-m}t_0=0,m\le n$$
$$t_m+a_{n-1}t_{m-1}+\cdots+a_0t_{m-n}=0,m>n$$

关于判别式还存在一个范数表示:给定$F$的域扩张$L=F(\alpha)$,记$f(x)$为$\alpha$在$F$上的极小多项式,那么$\mathrm{Disc}(f)=(-1)^{\frac{n(n-1)}{2}}N_{L/F}(f'(\alpha))$.证明.设$K$是$F$关于$f$的分裂域,记$f(x)=\prod(x-\alpha_i)\in K[x]$,那么有$f'(\alpha_j)=\prod_{i\not=j}(\alpha_j-\alpha_i)$.如果$\sigma_i,1\le i\le n$是全部$L$到$K$的$F$同态,使得$\sigma_i(\alpha)=\alpha_i$,那么有:
$N_{L/F}(f'(\alpha))=\prod_j(f'(\alpha))=\prod_jf'(\alpha)
=\prod_j\prod_{i\not=j}(\alpha_j-\alpha_i)=(-1)^{\frac{n(n-1)}{2}}\mathrm{Disc}(f)$,于是得证.

现在推广判别式这个概念.给定域$F$的$n$次可分扩张$K$,设从$K$到$F$某个代数闭包的全部不同的$F$同态为$\sigma_1,\cdots,\sigma_n$,如果任取$K$中的$n$个元$\alpha_1,\alpha_2,\cdots,\alpha_n$,约定$(\alpha_1,\cdots,\alpha_n)$的判别式为$\mathrm{disc}(\alpha_1,\cdots,\alpha_n)=\det (\sigma_i(\alpha_j))^2$.任取$K$的一组$F$基$\beta_1,\cdots,\beta_n$,那么约定$\mathrm{disc}(K/F)=\mathrm{disc}(\beta_1,\cdots,\beta_n)$.

那么按照这个定义扩张的判别式依赖于基的选取,先给出另一个等价描述:若$K$是$F$的$n$次可分扩张,取$\alpha_1,\cdots,\alpha_n\in K$,那么有$\mathrm{disc}(\alpha_1,\cdots,\alpha_n)=\det(\mathrm{Tr} _{K/F}(\alpha_i\alpha_j))$.这导致$\mathrm{disc}(\alpha_1,\cdots,\alpha_n)\in F$.证明.按照初始定义,$\mathrm{disc}(\alpha_1,\cdots,\alpha_n)$是一个矩阵$A$行列式的平方,那么可以看作矩阵$A^TA$的行列式,这这个矩阵的$(i,j)$项是$\sum_k\sigma_k(\alpha_i)\sigma_k(\alpha_j)=\mathrm{Tr} _{K/F}(\alpha_i\alpha_j)$.得证.

判别式可以用来判断线性无关性.给定$F$的$n$次可分扩张$K$,取$\alpha_1,\cdots,\alpha_n\in K$,那么$\mathrm{disc}(\alpha_1,\cdots,\alpha_n)=0$当且仅当$\alpha_1,\cdots,\alpha_n$是$F$上线性相关的.证明.假设$\alpha_i$在$F$上线性相关,那么有$\alpha_i=\sum_ {k\not=i}a_k\alpha_k$,那么有$\mathrm{Tr} _{K/F}(\alpha_i\alpha_j)=\sum_ka_k\mathrm{Tr}_{K/F}(\alpha_k\alpha_j)$.于是迹矩阵的列向量组线性相关,于是行列式0.反过来,假设$\det\mathrm{Tr} _{K/F}(\alpha_i\alpha_j)=0$.那么矩阵的行向量组$\{R_i\}$是线性相关的.那么存在不全为0的$a_i\in F$使得$\sum a_iR_i=0$,这意味着$\forall j$有$\sum_ia_i\mathrm{Tr}_{K/F}(\alpha_i\alpha_j)=0$,那么取$x=\sum a_i\alpha_i$,得到$\mathrm{Tr}_{K/F}(x\alpha_j)=0,\forall j$.倘若$\alpha_i$是线性无关的,那么它是一组基,导致有$\mathrm{Tr} _{K/F}(xy)=0,\forall y\in K$,这导致迹函数恒为0,和戴德金无关引理矛盾.于是$\{\alpha_i,1\le i\le n\}$是线性相关的.

现在来看取定不同基对判别式的影响.取两组$K$的$F$基$\{\alpha_1,\cdots,\alpha_n\}$和$\{\beta_1,\cdots,\beta_n\}$,取前一组到后一组的过渡矩阵$A=(a_{ij})$,也就是说$\beta_j=\sum_ia_ {ij}\alpha_i$.那么有$\sigma_i(\beta_j)=\sum_ka_ {kj}\sigma_i(\alpha_k)$,于是$(\sigma_i(\beta_j))=A^T(\sigma_i(\alpha_j))$,于是得到:$\mathrm{disc}(\beta_1,\cdots,\beta_n)=\det A^2\mathrm{disc}(\alpha_1,\cdots,\alpha_n)$.最后按照$\mathrm{disc}(K/F)$总是非0的,于是看到在$F^*/F^{*2}$意义下扩张的判别式是唯一的.

现在来讨论次数不超过4的不可约可分多项式对应的Galois群.总假定多项式是可分的,注意到如果域特征非2和3,那么这一点是必然可保证的.现在如果给域$F$上的一个可分的不可约多项式$f$,设$F$关于$f$的分裂域是$K$,那么在$K$上有$f(x)=\prod(x-\alpha_i)$,如果$f$的次数是$n$,那么$n$整除了$[K:F]=|Aut_FK|$.其中Galois群$G$同构于$S_n$的一个可迁子群,也就是说对任意的$1\le i,j\le n$,有$\sigma\in G$使得$\sigma x_i=x_j$.这里把$G$也称为$f$的Galois群.

来讨论最简单的二次多项式.设$f(x)=x^2+bx+c\in F[x]$是可分并且不可约的,那么$f$的Galois群必然是$S_2$,并且当$F$特征非2时候,可以计算根为$\frac{-b\pm\sqrt{b^2-4c}}{2}$.

现在考虑三次不可约的可分多项式,那么按照之前的讨论,这时候Galois群是$S_3$的子群,并且阶数被3整除,那么只可能是$A_3$和$S_3$.至于具体是哪个,有如下定理:

若$f(x)\in F[x]$是三次的不可约的可分多项式,设$F$关于$f$的分裂域是$K$,设$D$是多项式$f$的判别式,那么$Aut_FK\sim S_3$当且仅当$D\not\in F^2$,而$Aut_FK\sim A_3$当且仅当$D\in F^2$.事实上在前文证明了$G\subseteq A_n$当且仅当$\mathrm{disc}(f)\in F^2$,于是得证.

现在来给出三次方程的一种解法.考虑$g(x)=x^3+ax^2+bx+c$,那么首先,如果取代换$y=x-\frac{a}{3}$,可以使得二次项消去,这说明只需考虑方程$f(x)=x^3+px+q$.Cardano对接下来的处理方法是,设$x=u+v$,然后要求$u,v$满足方程组.来把$x=u+v$带入,得到$u^3+v^3+q=(3uv+p)(u+v)=0$,要求$u^3+v^3+q=0,3uv+p=0$.那么第二式告诉$v=-p/3u$,带入第一式得到$4u^6+qu^3-p^3/27=0$,由此解出$u^3=-q/2\pm\sqrt{\Gamma}$,其中$\Gamma=q^2/4+p^3/27$,事实上这就是$-D/108$,其中$D$为$f$的判别式.按照$u,v$对称性,取$A=-q/2+\sqrt{\Gamma},B=-q/2-\sqrt{\Gamma}$,得到$A=u^3,B=v^3$,于是$u=w\sqrt[3]{A},w=0,1,2$, $v=w\sqrt[3]{B},B=0,1,2$,注意到$\sqrt[3]{AB}=-p/3$,于是$f$的解为:
$$x_1=\sqrt[3]{A}+\sqrt[3]{B},x_2=w\sqrt[3]{A}+w^2\sqrt[3]{B},x_3=w^2\sqrt[3]{A}+w\sqrt[3]{B}$$

现在讨论四次多项式.取域$F$上的四次不可约可分多项式$f(x)=x^4+ax^3+bx^2+cx+d$,设$f$在某个分裂域上分裂为四个根$\alpha_i,1\le i\le4$.核心思路是考虑一个和$f$相关的三次多项式.取$\beta_1=\alpha_1\alpha_2+\alpha_3\alpha_4$, $\beta_2=\alpha_1\alpha_4+\alpha_2\alpha_4$,$\beta_3=\alpha_1\alpha_4+\alpha_2\alpha_3$,然后取$r(x)=(x-\beta_1)(x-\beta_2)(x-\beta_3)$,那么有:
$$r(x)=x^3-bx^2+(ac-4d)x+4bd-a^2d-c^2\in F[x]$$

称$r(x)$为四次多项式$f$相伴的三次多项式.容易验证$f$和$r$具有相同的判别式,取$K=F(\alpha_1,\alpha_2,\alpha_3,\alpha_4)$为$F$关于$f$的分裂域,取$L=F(\beta_1,\beta_2,\beta_3)$为$F$关于$r$的分裂域.于是$F\subseteq L$和$F\subseteq K$都是Galois扩张.取$V=\{e,(12) (34),(13)(24),(14)(23)\}$为$S_4$的四阶子群,那么$V\subseteq A_4$,并且$V$在$S_4$中正规.并且每个$\beta_i$都被$V$中元素固定,于是$G\cap V$的固定域包含了$L$,反过来$G-G\cap V$中每个元都会改变$\beta_i$中的至少一个,于是看到$L$恰好就是$G\cap V$的固定域.知道$G$是$S_4$的可迁子群,并且阶数是4的倍数.按照群论知识,$S_4$的24阶子群和12阶可迁子群是$S_4$和$A_4$,而4阶可迁子群是$V$和由$(1234)$生成的循环群$C_4$.被$(1234)$和$(24)$生成的子群是可迁8阶子群$D_4$,即8阶的二面体群.记$m=[L:F]$,按照三次多项式的情况,这是可计算的,那么:
\begin{enumerate}
	\item $G\sim S_4$当且仅当$r(x)$是$F$上不可约多项式并且判别式$D\not\in F^2$,当且仅当$m=6$.
	\item $G\sim A_4$当且仅当$r(x)$是$F$上不可约多项式并且判别式$D\in F^2$,当且仅当$m=3$.
	\item $G\sim V$当且仅当$r(x)$在$F$上分裂,当且仅当$m=1$.
	\item $G\sim C_4$当且仅当$r(x)$在$F$上恰好存在一个根$t$,并且$h(x)= (x^2-tx+d)(x^2+ax+b-t)$在$L$上分裂,当且仅当$m=2$并且$f(x)$在$L$上可约.
	\item $G\sim D_4$当且仅当$r(x)$在$F$上恰好存在一个根$t$,并且$h(x)$不在$L$上分裂,当且仅当$m=2$并且$f$在$L$上不可约.
\end{enumerate}

首先$[K:L]$必然$\le4$.另外按照三次多项式的情况,$r$不可约当且仅当$m=6$或3.另外$r$恰有1个根当且仅当$m=2$,最后注意到如果$\sigma$是一个4轮换,那么$\sigma^2\in V$.现在给出证明.首先如果$r(x)$在$F$上不可约,那么$m$是6或者3,于是3整除$|G|$,这使得$G$必然是$S_4$或$A_4$.此时总有$V\subseteq G$,于是$L=K^V$,即$L$是$V$的固定域,于是$[K:L]=4$,所以$G=S_4$当且仅当$m=6$,$G=A_4$当且仅当$m=3$.另外按照三次多项式的情况,看到$m=6$等价于$D\not\in F^2$,而$m=3$等价于$D\in F^2$,这就证明了前两条.$r(x)$在$F$上分裂等价于$L=F$,等价于$m=1$,并且如果这成立,那么$L$同时对应于$G$和$G\cap V$,导致$G\subseteq V$,但是$|G|$被4整除,于是只能有$G=V$.反过来,如果$G=V$,那么$L$对应$G\cap V=G$,于是$L=F$,于是$r(x)$在$F$上分裂.这就证明了第三条.现在证明最后一种情况.如果$r$恰好在$F$上有一个根$t$,那么必然有$m=2$,于是$|G:G\cap V|=2$,于是$G\not\subseteq V$,那么只有两种可能:$C_4$和$D_4$.反过来如果$G$同构于$D_4$或者$C_4$,那么$[G:G\cap V]=2$,那么$r(x)$在$F$上恰有一个根.现在如果$f$在$L$上不可约,等价于$[K:L]=4$,等价于$[K:F]=8$,等价于$G\sim D_4$.于是$G\sim C_4$等价于$f$在$L$上可约.现在不妨设$t=\alpha_1\alpha_2+\alpha_3\alpha_4$,那么$h(x)$有分解:$h(x)= (x-\alpha_1\alpha_2)(x-\alpha_3\alpha_4)(x-(\alpha_1+\alpha_2))(x-(\alpha_3+\alpha_4))$.如果$h$在$L$上分裂,那么$\alpha_1\alpha_2$和$\alpha_1+\alpha_2$都在$L$中,所以$\alpha_1$满足二次多项式$x^2- (\alpha_1+\alpha_2)x+\alpha_1\alpha_2$,于是$[K:L]\le2$,那么$[K:F]\le4$,这导致$G\sim C_4$.反过来如果$G\sim C_4$,取$G$的生成元$\sigma$,那么$\sigma^2\in G\cap V$,那么为了固定$t=\alpha_1\alpha_2+\alpha_3\alpha_4$,必然有$\sigma^2=(12)(34)$,但是这样$\sigma^2$必然固定$\alpha_1+\alpha_2,\alpha_3+\alpha_4,\alpha_1\alpha2,\alpha3,\alpha4$,于是它们落在$L$中,于是$h$在$L$上分裂,这就证明了最后两条.

四次多项式的解法.取一个特征非2的域$k$,取$K=k(t_1,t_2,t_3,t_4)$为$k$上的四元有理函数域.取$f(x)=\prod_ {i=1,2,3,4}(x-t_i)=x^4+ax^3+bx^2+cx+d\in k(s_1,s_2,s_3,s_4)[x]$.其中$s_i$是$t_j$的初等对称多项式.即$s_1=-a,s_2=b,s_3=-c,s_4=d$.那么弱$F=k(s_1,s_2,s_3,s_4)$,就有$K=F(t_1,t_2,t_3,t_4)$是$F$关于$f$的分裂域,并且有$Aut_FK=S_4$.取$\beta_1=t_1t_2+t_3t_4$, $\beta_2=t_1t_3+t_2t_4$,$\beta_3=t_1t_4+t_2t_3$.于是相伴的三次多项式为:$r(x)=\prod(x-\beta_i)=x^3-bx^2+(ac-4d)x+4bd-a^2d-c^2$.取$L=F(\beta_1,\beta_2,\beta_3)$,他是子群$V=\{1,\sigma_1=(12) (34),\sigma_2=(13)(24),\sigma_3=(14)(23)\}$的固定域,取$u= (t_1+t_2)-(t_3+t_4)$,那么有$\sigma_1(u)=u,\sigma_i(u)=-u,i=2,3$.于是得到$u^2\in L$.记$M=L(u)$,那么$M$对应于子群$\{1,\sigma_1\}$.再记$v=t_1-t_2$,那么$\sigma_1(v)=-v$,于是$v^2\in M$.现在$M(v)$只被恒等固定,于是有$K=M(v)$.计算可得$u^2=a^2-4b+4\beta_1$和$v^2=\frac{1} {4}(u-a)^2-2\left(\beta_1+\frac{1}{u}(2c-a\beta_1)\right)$.最后注意到$t_1=\frac{1} {2}(t_1+t_2+t_1-t_2)=\frac{1}{2}\left(v+\frac{1}{2}(u-a)\right)$.另外有$\sigma_i(t_1)=t_i,i=2,3,4$.另外注意到$\sigma_2(v)=\sigma_3(v)=v'=t_3-t_4$于是得到:
$$t_1=\frac{1}{2}\left(v+\frac{1}{2}(u-a)\right)$$
$$t_2=\frac{1}{2}\left(-v+\frac{1}{2}(u-a)\right)$$
$$t_3=\frac{1}{2}\left(v'+\frac{1}{2}(-u-a)\right)$$
$$t_4=\frac{1}{2}\left(-v'+\frac{1}{2}(-u-a)\right)$$
\newpage
\subsection{循环扩张}

循环扩张.称一个Galois扩张是循环扩张,如果它的Galois群是一个循环群.我们遇到过一些循环扩张:有限域之间的扩张总是循环扩张;对于素数$p$,$p$次循环扩张$\mathbb{Q}\subset\mathbb{Q}_p$的Galois群是$\mathbb{Z}/(p-1)$;再比如最简单的,取特征非2的域$k$,取$a\in k^*-(k^*)^2$,那么$k\subseteq k(\sqrt{a})$是循环扩张.

给出一般域的循环扩张是十分复杂的,例如$\mathbb{Q}$的循环扩张就已经没有简单的描述了.本节将会给出两种特殊情况的描述:第一个是Kummer扩张的特殊情况,即对于一个包含$n$次本原根的域的循环扩张;第二个是所谓的Artin-Schreier理论,即处理特征$p$域的$p$次循环扩张.

这里以二次扩张为例解释这两种情况.首先如果特征非2,按照二次多项式求根公式,此时扩张总可以写作$k(\sqrt{d})$.当特征为2的时候,并不能一定消去一次项,因此此时会出现两种情况,第一种是上述情况,第二种则是某个$x^2-x+a$的分裂域.能写作$k(\sqrt{d})$可以归结为Kummer扩张的特例,如果要求域包含二次本原根,那么这保证了两件事:首先域不会特征为2,因为此时从$x^2=1$得到$(x-1)^2=0$.另外特征非2的域上总存在二次本原根$-1$.于是这个要求保证了扩张总会写作$k(\sqrt{d})$.对于第二种情况,我们会看到特征$p$的$p$次循环扩张总可以表示为某个$x^p-x+a$的分裂域.

我们先来处理第一个情况,先来解释这个特殊的条件.
\begin{enumerate}
	\item 首先域$k$具有$n$次本原根等价于讲它恰好具有$n$个两两不同的$n$次单位根.
	\item 二次扩张看起来很容易的一个理由是如果扩域包含某个$\sqrt{d}$,那么它自动包含$-\sqrt{d}$,换句话说$x^2-d$在扩域分裂.如果约定域$k$包含了一个$n$次本原根$\omega$,那么会产生同样的效果:倘若某个扩张$F$里存在基域中某个元的$n$次平方根$a$,换句话说满足$a^n=b\in k$,那么$x^n-b=\prod_{0\le i\le n-1}(x-\omega^ia)$.换句话说只要基域上形如$x^n-b$的多项式在$F$中有根,那么它是分裂的.另外按照$\omega^ia,0\le i\le n-1$两两不同,说明这是一个可分多项式.
	\item 于是如果基域$k$包含$n$次本原根,取扩域$F=k(\sqrt[n]{b}),b\in k$,那么这是一个可分正规扩张,于是是Galois扩张.事实上我们还可以说明这是一个循环扩张,任取Galois群$G$中的一个$k$自同构$\sigma$,既然$a=\sqrt[n]{b}$的极小多项式整除$x^n-b$,说明比如有$\sigma(a)=\omega^ia$,这里的$i$在$\mod n$意义下唯一.于是可构造映射$G\to\mathbb{Z}/n\mathbb{Z}$为把每个$\sigma\in G$映射为对应的$i\mod n$.容易说明这是群同态,并且按照单扩张的自同构被扩张元的取值唯一确定,说明这个映射是满的,这就得到了一个单同态$G\to\mathbb{Z}/n\mathbb{Z}$.于是$G$是循环的.
	\item 但是上述单同态未必总是满的,于是还剩下一个值得讨论的问题是扩张的次数$m$,以及$a$的极小多项式是什么.取定$G$的一个生成元$\tau$,那么多项式$f(x)=\prod_{0\le i\le m-1}(x-\tau^{(i)}(a))$在$G$中任意一个元下固定,这说明$f(x)$是基域中的多项式,特别的观察常数项得到$a^m\in k$.这说明$a$的极小多项式是$x^m-a^m$,换句话说是$x^m-d,d\in k$的形式.
	\item 最后我们断言$m$恰好就是$b(k^*)^n$在$k^*/(k^*)^n$中的阶数.一方面如果记$\tau(a)=\omega^ta$,从$\tau^m=\mathrm{Id}$说明$\omega^{tm}=1$,并且$m$恰好就是满足这个等式的最小的正整数,于是$n\mid tm$,于是$m$被这个阶数$m'$整除.另一方面,从$b^{m'}\in (k^*)^n$得到$b^{m'}=c^n$,其中$c\in k^*$,这导致$a^{m'n}=c^n$,于是$a^{m'}=c\omega^i$,导致$a^{m'}\in k$,导致$\tau(a^{m'})=a=\omega^{tm'}a$,这导致$m\mid m'$,综上得到$m=m'$.
\end{enumerate}

总结一下上述讨论,我们得到如下结论:给定域$k$,设它具有一个$n$次本原根,取$K=F(\sqrt[n]{b}),b\in k$,那么$k\subseteq G$是一个循环Galois扩张.并且$m=[K:F]$就是$bk^ {*n}$在群$k^*/k^{*n}$中的阶数,并且$\sqrt[n]{b}$的极小多项式有形式$x^m-d,d\in F$.

上述定理的逆命题可归结如下:给定域$k$,约定它包含某个$n$次本原根$\omega$,设$k\subseteq F$是$n$阶循环扩张,那么扩张可以表示为$F=k(\sqrt[n]{b}),b\in k$的形式.处理这个定理的标准工具是希尔伯特90定理,不过眼下我们仅需一个弱化版本:我们先承认如果设Galois群$G$的一个生成元为$\sigma$,那么可以找到$F$中的一个元$a$使得$\sigma(a)=\omega a$.那么得到$\sigma^i(a)=\omega^ia$.于是$\mathrm{Aut}_{k(a)}F=\langle\mathrm{Id}\rangle$,从Galois理论基本定理得到$F=k(a)$.最后$a^n$被$G$中每个元固定,于是导致$a^n=b\in k$,完成证明.

这里补充下希尔伯特90定理的弱化版本及证明.如果域$k$包含某个$n$次本原根$\omega$,并且$k\subseteq F$是$n$次循环扩张,设Galois群$G$的生成元为$\sigma$,那么存在$a\in F$使得$\sigma(a)/a=\omega$.
\begin{proof}
	
	$\sigma$可视为$k$线性空间$F$上的线性同构,于是验证最后这个等式等价于证明$\omega$是$\sigma$的特征值.而按照$\sigma$是$G$的生成元,说明了$\sigma^k=\mathrm{Id}$的最小正整数$k$就是$n$,这说明$\mathrm{Id},\sigma,\sigma^2,\cdots,\sigma^{n-1}$是两两不同的乘性特征,戴德金引理说明它们是基域上线性无关的,于是如果存在次数$m<n$的基域上的多项式$g$使得$g(\sigma)=0$,会和线性无关性矛盾,这就导致$x^n-1$就是线性变换$\sigma$的极小多项式,从而$\omega$是$\sigma$的特征值.
\end{proof}

最后一个值得考虑的问题是探究这种扩张的中间域.如果$k\subseteq F$是$n$次循环扩张,并且$k$包含了某个$n$次本原根,现在我们已经知道此时有$F=k(\sqrt[n]{b}),b\in k$.记Galois群$G$的生成元为$\sigma$,那么按照Galois基本定理,每个中间域对应于子群$\langle\sigma^t\rangle$,其中$t$是$n$的某个因数.并且所对应的中间域恰好就是被$\sigma^t$所固定的中间域.记$n=tm$,记$a=\sqrt[n]{b}$,那么$\sigma^t(a^m)=a^m$,于是$a^m$被$\sigma^t$固定.另一方面$b^tk^{*n}$在$k^*/k^{*n}$中的阶数是$m$,于是$k(\sqrt[m]{b})$的次数恰好是$m$,这迫使$\sigma^t$对应的中间域是$k(\sqrt[m]{b})$.综上,我们说明了:

若$k\subseteq F$是一个$n$次循环扩张,并且设$k$具有一个$n$次本院根,则$K=F(\sqrt[n]{b}),b\in k$,那么$k\subseteq F$的每个中间域都具有形式$F(\sqrt[m]{b}),m\mid n$.

现在进入到第二部分,即Artin-Schreier理论.给定一个特征$p$的域$F$,定义$F$上的映射为$\phi:a\mapsto a^p-a$.这是一个加法群同态并且核为$\mathbb{F}_p$.注意到如果$\phi(a)=b$,那么$\phi(a+i)=b,\forall i\in\mathbb{F}_p$.结合根的个数不超过次数,说明$\phi^{-1}(b)=\{a+i\mid i\in\mathbb{F}_p\}$.

设域$k$的特征为$p$,设$k\subseteq F$是$p$次循环Galois扩张,那么$F=k(\alpha)$,其中$\alpha$满足$\alpha^p-\alpha=a\in k$.换句话说$F=k(\phi^{-1}(a))$.这个定理一般要用加法版本的希尔伯特90定理来证明.这里同样给出一个线性代数证明.
\begin{proof}
	
	取Galois群的一个生成元$\sigma$,我们的思路是找到扩张元$\alpha$,使得它满足$\sigma(\alpha)=\alpha+1$,这就导致$a=\alpha^p-\alpha$满足$\sigma(a)=a$,于是$a$被$G$固定,于是$a\in k$.
	
	取$F$上的线性变换$T=\sigma-\mathrm{Id}$,那么它的核恰好就是被$G$中所有元固定的元,于是$\ker T=k$.另外有$T^p=(\sigma-\mathrm{Id})^p=0$,于是有$\mathrm{Im}T^{p-1}\subset\ker T=k$.但是前者是$k$线性空间$F$的子空间,后者恰好是1维子空间,导致$\mathrm{Im}T^{p-1}=k$或者为0.为0是不能成立的,因为这导致$\sigma$满足更小次数的零化多项式$(x-1)^{p-1}$矛盾.于是只能有$\mathrm{Im}T^{p-1}=k$.于是存在某个$c\in F$使得$1=T^{p-1}(c)$,取$\alpha=T^{p-2}(c)$,导致$T(\alpha)=1$,于是$\sigma(\alpha)=\alpha+1$.最后按照$\alpha$不被$\sigma$所固定,说明$\alpha\not\in k$,而$[F:k]$是素数,这只能有$F=k(\alpha)$,完成证明.
\end{proof}

上述定理的逆命题可归结如下:如果$k$的特征是$p$,任取$a\in k-\phi^{-1}(k)$,那么$f(x)=x^p-x-a$是$k$上的不可约多项式,并且$k$关于$f(x)$的分裂域是$p$次循环扩张.
\begin{proof}
	
	首先注意到如果$\alpha$是$f(x)=x^p-x-a$的任意一个根,那么$\alpha,\alpha+1,\cdots,\alpha+p-1$恰好就是它的全部根,并且它们都在$F$中,并且总有$F=k(\alpha)$.
	
	先来说明$f$是不可约多项式.假设有唯一分解$f(x)=g_1(x)g_2(x)\cdots g_r(x)$,其中每个$g_i(x)$都是$k$上的不可约多项式,如果$\beta$是某个$g_i$的根,那么有$F=k(\beta)$,导致$[F:k]=\deg g_i$.于是让$\beta$取遍$g_i$的根,导致全部$g_i$的次数是相同的,记这个次数是$s$,那么有$\deg f=rs$.但是按照$\deg f=p$是素数,并且条件已经要求了$f$不在$k$上分解,于是只能有$r=1$,即$f$是不可约的.
	
	现在任取$f$的根$\alpha$,那么$F=k(\alpha)$,按照同构扩张定理,存在$k$自同构$\sigma$满足$\sigma(\alpha)=\alpha+1$,于是$\sigma$的阶数是$p$,按照有限扩张的自同构群阶数不超过扩张维数,导致扩张的自同构群的阶数恰好是$p$,于是这是循环Galois扩张.
\end{proof}



现在进入本节的核心内容,群上同调和Hilbert90定理.如果$K$是一个域,$G$是$Aut(K)$的一个子群,称一个导数$f$为$G\to K^*$的函数,满足$f(\sigma\tau)=f(\sigma)\sigma(f(\tau))$.

如果$K$是$F$的Galois扩张,记Galois群为$G$,如果$f:G\to K^*$是一个导数,那么存在$a\in K$使得导数的表达式为$f(\tau)=\tau(a)/a,\forall \tau\in G$.事实上按照戴德金无关引理,存在一个$c\in K$使得$\sum_{\sigma\in G}f(\sigma)\sigma(c)\not=0$,记作$b$,于是验证有$f(\gamma)\gamma(b)=b$.于是$a=b^{-1}$满足要求.

Hilbert90定理.给定循环Galois扩张$F\subseteq K$,取Galois群$G$的生成元$\sigma$,那么有$u\in K$使得$N_{K/F}(u)=1$当且仅当存在一个$a\in K$使得$u=\sigma(a)/a$.证明,充分性是直接的,对于必要性,定义$f:G\to K^*$为$f(id)=1,f(\sigma)=u$,并且$f(\sigma^i)=u\sigma(u)\cdots\sigma^{i-1}(u),i<n$.那么这是一个导数,于是存在$a\in K$使得$f(\sigma^i)=\sigma^i(a)/a$,于是$u=f(\sigma)=\sigma(a)/a$.

群上同调.给定群$G$和一个交换群$M$,会定义一族群$H^n(G,M),n\ge0$,称为上同调群.称$M$是一个$G$模,如果$M$是群环$Z[G]$上的一个左模.例如如果$K$是域$F$上的一个Galois扩张,记Galois群为$G$,那么如果定义$\sigma a=\sigma(a)$,$K^*$成为一个$G$模.同样的,加法群$(K,+)$也构成一个$G$模.

现在取$C^n(G,M)$为全体从笛卡儿积$G^n$到$M$的映射,其中$C^0(G,M)$约定为$M$,称$C^n(G,M)$中的元素为$n$链,那么这等价于是从以$G^n$为基的自由交换群到$M$的全体同态构成的交换群,即$C^n(G,M)=\mathrm{Hom}_Z(Z[G^n],M)$.现在我们定义映射$\delta_n:C^n(G,M)\to C^{n+1}(G,M)$为:
$$\delta_n(f)(x_1,\cdots,x_{n+1})=x_1f(x_2,\cdots,x_{n+1})
+\sum_{i=1}^{n}(-1)^if(x_1,\cdots,x_ix_{i+1},\cdots,x_{n+1})
+(-1)^{n+1}f(x_1,\cdots,x_n)$$

特别的,看到$\delta_0:M\to C^1(G,M)$约定为了$m(x)\mapsto xm-m$.验证$\delta_n$是同态并且是边界算子,也就是说满足$\delta_ {n+1}\circ\delta_n=0$.记$Z^n(G,M)=\ker\delta_n$,其中元素称为余$n$圈,取$B^n(G,M)=\mathrm{im}\delta_{n-1}$,称为余$n$边,由此得到的上同调群记作:
$$H^n(G,M)=\frac{Z^n(G,M)}{B^n(G,M)},n\ge0$$

两个余$n$圈称为上同调的,如果它们的差是一个余$n$边.那么首先0阶上同调群具有描述:
$$H^0(G,M)=M^G=\{m\in M\mid \sigma m=m,\forall \sigma\in G\}$$

另外余1圈就是所定义的导数.而余1边就是满足存在一个$m\in M$使得$g(\sigma)=\sigma(m)-m,\forall \sigma\in G$的$g$.那么在上面证明了如果$K$是$F$的一个Galois扩张,那么对于$G$模$K^*$,其中$G$是Galois群,有余1圈总是余1边.即得到了$H^1(G,K^*)=0$.事实上如果把$K$看作$G$模,这个结果同样成立:

考虑Galois扩张$F\subseteq K$,记Galois群为$G$,取$g:G\to K$是余1圈,也就是导数,那么$g$同样是余1边,也就是说,存在一个$a\in K$使得对任意的$\tau\in G$总有$g(\tau)=\tau(a)-a$.证明.按照$F\subseteq K$可分,看到迹映射$T_ {K/F}$非0,于是存在一个$c\in K$使得$\mathrm{T}_{K/F}(c)=\alpha\not=0$.那么$\alpha\in F^*$并且$\mathrm{T}_{K/F}(\alpha^{-1}c)=1$,以$\alpha^{-1}c$代替$c$,也就是说找到了一个元$c\in K$使得$\mathrm{T}_{K/F}(c)=1$.现在按照$T_ {K/F}(x)=\sum_{\sigma\in G}\sigma(x)$.取$b=\sum_{\sigma\in G}g(\sigma)\sigma(c)$.那么可得$g(\tau)=b-\tau(b)$,取$a=-b$,即满足$g(\tau)=\tau(a)-a$.

综上,得到Hilbert90定理的上同调版本:给定Galois扩张$F\subseteq K$,取Galois群$G$,那么有$H^1(G,K^*)=H^1(G,K)=0$.
\newpage
\subsection{kummer扩张}

本节最后介绍Kummer扩张.如果$F$是一个包含$n$次本原根的域,称一个Galois扩张$F\subseteq K$是$n$次Kummer扩张,如果Galois群$G$是交换群,并且存在$n$的一个因数$m$满足$mG=0$

按照有限交换群都是有限个循环群的直和,以及Galois基本定理,结合之前证明的循环扩张的描述,可以得出Kummer扩张的描述:

给定域$F$,设它含有$n$次本原根,设$K$是$F$的有限扩张,那么$K$是$F$的$n$次Kummer扩张当且仅当存在$a_1,\cdots,a_r\in F$使得$K=F(\sqrt[n]{a_1},\sqrt[n]{a_2},\cdots,\sqrt[n]{a_r})$.

证明.设$K=F(\alpha_1,\cdots,\alpha_r)$,那么必然有$\alpha_i^n=a_i\in F$.如果$w\in F$是$n$次本原根,那么$\alpha_i,w\alpha_i,\cdots,w^{n-1}\alpha_i$是$x^n-a_i$在$K$中的全部不同的$n$个根,也就是说$x^n-a_i$在$F$上可分并且在$K$上分裂.于是$K$是$\{x^n-a_i,1\le i\le r\}$在$F$上的分裂域.于是$K\subseteq F$是Galois扩张.取Galois群$G$中的元$\sigma$,那么有$\sigma^n(\alpha_i)=\alpha_i,\forall i$,于是$\sigma^n=id$,于是存在$n$的一个因数$m$有$mG=0$.现在任取$\sigma,\tau\in G$,对任意$i$,有$\sigma(\alpha_i)=w^j\alpha_i$, $\tau(\alpha_i)=w^k\alpha_i$.那么有$\sigma\tau(\alpha_i)=\tau\sigma(\alpha_i)$,于是$\sigma$和$\tau$可交换,于是$G$是交换群.现在来证明充分性.如果$F\subseteq K$是一个Galois扩张并且Galois群$G$是交换群并且有$n$的因数$m$使得$mG=0$.按照有限交换群的结构定理,有$G=C_1\times\cdots\times C_r$.其中每个$C_i$是有限循环群,记$m_i=|C_i|$,取$m_1,\cdots,m_r$的最大公约数$m$,那么$m$整除$n$并且有$mG=0$.现在把直和中的$C_i$划去得到的子群记作$H_i$,那么$G/H_i\sim C_i$,取$H_i$的固定域$L_i$,那么按照$H_i$是正规子群,由Galois基本定理2,得到$L_i$是$F$的Galois扩张,并且Galois群为$G/H_i\sim C_i$是循环群.于是按照循环扩张的结构定理,有$L_i=F(\alpha_i),\alpha_i\in L_i$,并且$\alpha_i^ {m_i}=a_i\in F$.按照$m_i$整除$n$,得到$\alpha_i^n=a_i\in F$,于是按照Galois基本定理1,看到$F(\alpha_1,\cdots,\alpha_r)=L_1\cdots L_r$对应于群$H_1\cap\cdots\cap H_r=<id>$,于是有$K=F(\alpha_1,\cdots,\alpha_r)=F(\sqrt[n]{a_1},\sqrt[n]{a_2},\cdots,\sqrt[n]{a_r})$.

如果$F$包含一个$n$次本原根,那么$F(\sqrt[n]{a_1},\sqrt[n]{a_2},\cdots,\sqrt[n]{a_r})$是一个$n$次kummer扩张,于是这个扩张的次数不会超过$n^r$,但是有的时候并不是一定取$n^r$的,例如,对于两两不同的素数$p_i$,有$K=Q(\sqrt{p_1},\sqrt{p_2},\cdots,\sqrt{p_r})$是$Q$的2-kummer扩张,并且次数是$2^r$,但是比如$Q(\sqrt{2},\sqrt{3},\sqrt{6})$是$Q$上的4次扩张.于是一个自然的问题是给定一个kummer扩张的形式$F(\sqrt[n]{a_1},\sqrt[n]{a_2},\cdots,\sqrt[n]{a_r})$,如何求出它的次数.注意对于$r=1$的情况已经解决,即$[F(\sqrt[n]{a}):F]$就是$aF^*$在$F^*/F^{*n}$的阶数.为了证明一般kummer扩张的结论,需要更多的工具.

给定两个有限交换群$G,H$,给定循环群$C$,称一个函数$B:G\times H\to C$是双线性同态,如果满足$B(g_1g_2,h)=B(g_1,h)B(g_2,h)$以及$B(g,h_1h_2)=B(g,h_1)B(g,h_2)$.一个双线性同态称为非退化的,如果$B(g,h)=e$对任意$g\in G$成立当且仅当$g=e$,并且$B(g,h)$对任意$h\in H $成立当且仅当或$g=e$.

引理.给定双线性同态$B:G\times H\to C$,如果$h\in H$,约定$B_h$是$G\to C$的同态$g\mapsto B(g,h)$.于是$\phi:h\mapsto B_h$是从$H$到$\mathrm{Hom}(G,C)$的同态.记$\mathrm{exp}(G)$为最小的正整数$r$满足$rG=0$.如果$B$是非退化的,那么$\mathrm{exp}(G)$整除$|C|$,并且$\phi$是单射,而且诱导了$G\sim H$.事实上按照非退化有$\ker\phi=<e>$,于是$\phi$是单射,取$m=|C|$,那么$e=B(e,h)=B(g,h)^m=B(g^m,h)$,于是按照非退化条件得到$g^m=e$,于是$\mathrm{exp}(G)\mid |G|$.利用特征群的知识可以证明$H\sim\mathrm{im}\phi=\mathrm{Hom}(G,C)\sim G$.

现在给定一个$n$次kummer扩张$F\subseteq K$,记$\mu(F)$为全体$F$的$n$次根,于是$\mu(F)$是一个循环群.再记$\mathrm{KUM}(K/F)$为全体$K^*$中的元$a$满足$a^n\in F$,那么$\mathrm{KUM}(K/F)$是$K^*$的一个子群,并且它以$F^*$为子群.记$K=F(\sqrt[n]{a_1},\sqrt[n]{a_2},\cdots,\sqrt[n]{a_r})$,那么取$\mathrm{kum}(K/F)=\frac{\mathrm{KUM}(K/F)}{F^*}$.现在定义kummer扩张的双线性同态为:
$$B:Aut_FK\times\mathrm{kum}(K/F)\to\mu(F)$$
$$B(\sigma,\alpha F^*)=\sigma(\alpha)/\alpha$$

现在来证明,上述双线性同态是非退化的,于是有$\mathrm{kum}(K/F)\sim Aut_FK$.如果$\sigma\in G=Aut_FK$满足对任意$aF^*$有$B(\sigma,\alpha F^*)=1$,那么$\sigma(\alpha)=\alpha,\forall \alpha\in \mathrm{KUM}(K/F)$.但是$\mathrm{KUM}(K/F)$中的元生成了域$K$,于是$\sigma=id$.另一方面如果$B(\sigma,\alpha F^*)=1$,对任意的$\sigma\in G$,那么$\sigma(\alpha)=\alpha$,但是按照Galois对应,得到$\alpha\in F^*$,于是$B$非退化.

至此,为了求kummer扩张的次数,可以转而求Galois群的阶数,按照上诉定理又可以等价的转而求$\mathrm{kum}(K/F)$的阶数:

取$n$次kummer扩张$F\subseteq K$,那么存在从$\mathrm{kum}(K/F)$到$F^*/F^ {*n}$的单同态$f$:$\alpha F^*\mapsto \alpha^nF^{*n}$.事实上如果$\alpha F^*\in\ker f$,那么$\alpha^n\in F^{*n}$,于是存在$a\in F$满足$\alpha^n=a^n$.于是$\alpha/a$是$n$次单位根,于是$\alpha/a\in F$.于是$\alpha F^*=F^*$,得证.
\newpage
\section{Galois理论的应用}
\subsection{对称多项式基本定理}

考虑域$k$上的$n$元有理函数域$F=k(x_1,x_2,\cdots,x_n)$,它是$n$元多项式环$R=k[x_1,x_2,\cdots,x_n]$的商域.任取$\sigma\in S_n$,它诱导了$R$上的环同构$f(x_1,x_2,\cdots,x_n)\mapsto f(x_{\sigma(1)},x_{\sigma(2)},\cdots,x_{\sigma(n)})$.按照分式化的泛映射性质,这延拓为$F$的域自同构,于是$S_n$是$\mathrm{Ant}(F)$的子群.取关于$S_n$的固定域$E$,称为$k$上的$n$元对称函数域.即其中的有理函数满足,任意置换未定元不改变这个函数.

记$s_1=\sum_{1\le i\le n}x_i$,$s_2=\sum_{i<j}x_ix_j$,$\cdots$,$s_n=x_1x_2\cdots x_n$.那么$\{s_1,s_2,\cdots,s_n\}\subseteq E$.所谓对称多项式基本定理是指$E$中每个元可以表示为$s_i$的多项式,换句话讲有$E=k(s_1,s_2,\cdots,s_n)$.

记$M=k(s_1,s_2,\cdots,s_n)$,那么$M\subseteq E$.按照阿廷引理,$E\subseteq F$是Galois扩张,于是它的扩张维数即$|S_n|=n!$.我们来证明$M\subseteq F$的维数不超过$n!$,这就导致维数实际上是$n!$,于是得到$M=E$.事实上$F=E(x_1,x_2,\cdots,x_n)$可视为关于多项式$t^n-s_1x^{n-1}+s_2x^{n-2}-\cdots+(-1)^ns_n$的分裂域,分裂域的维数自然不超过多项式次数的阶数,这就完成证明.
\newpage
\subsection{代数学基本定理}

代数学基本定理即$\mathbb{C}$是一个代数闭域.证明.任取一个非常数的多项式$f(x)\in C[x]$.需要证明$f(x)$在$C$上有根.假设$f$在$C$上没有根,那么$g(x)=f(x)\overline{f(x)}\in R[x]$在$C$上同样没有根.取$R$关于$g$的分裂域为$F$,把$F$嵌入到$R$的一个代数闭包中,考虑延拓$R\subseteq F(i)$,这个扩张是有限的,并且可分并且正规的,于是它是Galois扩张,取$G=Aut_R(F(i))$.按照Sylow存在性定理,有非平方的2-Sylow子群$H$,那么$[G:H]$是奇数,于是$H$对应了一个有限的可分扩张$R\subseteq E$,$[E:R]$是奇数.但是$R$中的每个奇数次多项式都存在根,于是得到$R\subseteq E$是平凡的.于是$[G:H]=1$,于是$G$是一个2-群.注意到$C$是扩张$R\subseteq F(i)$的中间域,于是Galois扩张$C\subseteq F(i)$对应了$G$的一个子群,那么$|Aut_C(F(i))|=2^n,n\ge0$.但是知道$2^n$阶群必然存在$2^{n-1}$阶子群,那么这个子群对应了Galois扩张$C\subseteq F(i)$的一个中间域$E$,并且$[E:C]=2$,但是知道$C$上任意一个二次多项式必然存在解.这导致矛盾.
\newpage
\subsection{$e$和$\pi$的超越性}

无理数的存在性要追溯到两千年前,毕达哥拉斯给出了$\sqrt{2}$的无理性.直到1761年,Lambert第一个给出了$\pi$的无理性.随后Lambert于1767年给出了$e$的无理性.在同一时期,人们开始猜测不是所有的数都是代数数,超越数的存在性一直是一个公开问题,直到1844年Liouville给出了一个复数是代数数的准则,并且证明了超越数存在.$e$的超越性直到1893年由Hermite给出,九年后Lindemann运用Hermite的方法证明了$\pi$的超越性.这里给出Lindemann方法的一般结论,并由此同时证明$e$和$\pi$的超越性.

Lindemann-Weierstrauss定理.如果$\alpha_1,\cdots,\alpha_m$是不同的代数数,那么$e^{\alpha_1},\cdots,e^{\alpha_m}$在$Q$上是线性无关的.

先来看这个定理如何证明$\pi$和$e$的超越性.首先如果$e$是代数数,那么存在有理数$r_i$满足$\sum_{i=0}^{n}r_ie^i=0$,但是这导致$e^0,e^1,\cdots,e^n$线性相关,可是$0,1,\cdots,n$是不同的代数数,这就矛盾.对于$\pi$,如果$\pi$是$Q$上代数数,那么$\pi i$必然也是代数数,可是$e^{\pi i}+e^0=0$,这又和上述定理矛盾.

\begin{proof}
	
	证明见gtm167P134.
	
\end{proof}
\newpage
\subsection{尺规作图}

尺规作图问题可以追溯到古希腊时期,其中有些问题遗留了下来,直到域论的发展才得以解决.在这里会用域论解决如下四个经典问题:是否可以三等分任意角?是否可以倍立方?是否可以构造一个圆,然后尺规出一个面积和圆相同的正方形?对哪些$n$可以尺规作出正$n$边形?

尺规作图的要求.给定两个初始点$O,P$,称为初始可构造点,允许构造更多的点或直线或圆周,称为可构造点,直线,圆周,要求为:
\begin{enumerate}
	\item 如果已经构造出两个点$A,B$,那么可以构造出过两点的直线.
	\item 如果已经构造出两个点$A,B$,那么可以构造出以其中一点为圆心过另一点的圆周.
	\item 已构造出的直线或者圆周,如果相交,那么交点可构造.
\end{enumerate}

距离来讲,给定两个可构造点$A,B$,可以构造一个以这两个点为顶点的等边三角形:以$A$圆心做过点$B$的圆周,以$B$圆心做过点$A$的圆周,两个圆周相交于点$C$,那么$ABC$构造等边三角形.下面证明一些基本操作:
\begin{enumerate}
	\item 有可构造直线$l$和一个可构造点$A$,那么可以构造过点$A$的直线和$l$垂直.证明,按照$l$是可构造的,可以取$l$上的一个可构造点$P$,以$A$为圆心做过点$P$的圆周,那么这个圆周如果只和$l$交于一点$B$,则直线$AB$垂直直线$l$.如果圆周和$l$交于点$P$和另一点$Q$,那么$Q$是可构造点.现在以$P$圆心做过点$Q$圆周,再以$Q$圆心做过点$P$圆周,这两个圆周交于两个点,于是必然有一个点$B$和$A$不同,那么直线$AB$和$l$垂直.
	\item 有可构造直线$l$和不在$l$上的一点$A$,那么可构造过点$A$的和$l$平行的直线.证明,按照上一条,过点$A$可构造直线$l'$和$l$垂直,再取过点$A$的和$l'$垂直的直线$l''$,那么$l''$和$l$平行.
\end{enumerate}

按照可构造垂直线,看到对于给定的两个初始可构造点$O,P$,可以构造直线$OP$作为$x$轴,过点$O$做和$OP$垂直的直线作为$y$轴,那么可以约定$OP$长度是单位1,由此得到带度量的直角坐标系.那么做垂直线看到,一个点$(x,y)$是可构造的,等价于点$(x,0)$和$(0,y)$是可构造的,而过原点做圆周,看到$(0,y)$可构造等价于$(y,0)$可构造.由此定义:一个实数$r$称为可构造的,如果点$(r,0)$在上述直角坐标系中是可构造的.全体可构造实数构成的集合为$\mathscr{A}_R\subseteq R$.

另外,可以把上述直角坐标系看作复平面,其中原点是0,点$P$是1,那么称可构造复数是指点$x+yi$,它在上述复平面上可构造,全体可构造复数构成的集合记作$\mathscr{A}_C\subseteq C$.那么看到,$x+yi\in\mathscr{A}_C$当且仅当$x,y\in\mathscr{A}_R$.

可构造集的结构.$\mathscr{A}_R$是$R$的一个子域,$\mathscr{A}_C$是$C$的子域,并且有$\mathscr{A}_C=\mathscr{A}_R(i)$.

为了证明$\mathscr{A}_R$是$R$的子域,只要看到给定两个可构造实数$a,b$,那么$a-b$是可构造实数,这个是显然的.另外要证明如果$a,b$是可构造实数并且$b\not=0$,那么$a/b$是可构造实数.事实上点$A=(0,a)$和$B=(b,0)$都是可构造点,连接点$AB$,过点$(1,0)$做$AB$的平行线$l$,那么$l$和$y$轴交于一点$C$,于是有$C=(0,a/b)$.于是$\mathscr{A}_R$是$R$的子域,易证由此得到$\mathscr{A}_C$是$C$的子域.最后按照二者都是域,自然有$\mathscr{A} _C=\mathscr{A}_R(i)$.

按照$Q$是特征0的域的素子域,得到$Q\subseteq \mathscr{A}_R\subseteq \mathscr{A}_C$.下面进一步描述可构造实数.

如果给定了$R$的子域$K$,如果目前所构造的全部直线,圆周,点的表达式中系数都是$K$中的元,那么:
\begin{enumerate}
	\item 任意过两个点的直线的表达式的系数都是$K$中元,任以一点为圆心过另一点的圆周的表达式的系数都是$K$中元.
	\item 任意两个直线的交要么空要么是$K$中的点(即横纵坐标都是$K$中元).这从域上线性方程组的解直接得出.
	\item 直线和圆周要么不交,要么交点是一个或两个,并且存在$u\in K,u\ge0$,使得交点落在域$K(\sqrt{u})$中.证明,考虑圆周$x^2+y^2+ax+by+c=0$和直线$dx+ey+f=0$.不妨设$d\not=0$,否则$e\not=0$,交换两坐标轴.那么通过除以$d$,不妨设$d=1$,于是直线为$-x=ey+f$.带入圆周得到:
	$$(e^2+1)y^2+(2ef-ae+b)y+(f^2-af+c)=0$$
	
	把方程记作$\alpha y^2+\beta y+\gamma=0$,如果$\alpha=0$,那么$y\in K$,从而$x\in K$.如果$\alpha\not=0$,那么要么无解,要么有$y\in K(\sqrt{\beta^2-4\alpha\gamma})$,其中$\beta^2\ge 4\alpha\gamma$.
	\item 两个圆周要么不交,要么交点是一个或两个,并且存在$u\in K,u\ge0$使得交点落在域$K(\sqrt{u})$中.这可以归结到上一情况.
\end{enumerate}

由此证明了下面定理的必要性:

一个实数$c$是可构造实数,当且仅当存在域扩张链$Q=K_0\subseteq K_1\subset\cdots\subseteq K_r$,使得$c\in K_r$.并且$[K_ {i+1}:K_i]\le2,\forall i$.

现在证明充分性.如果存在条件中的域的链.来对$r$归纳证明$c$是可构造实数.对$r=0$的情况是平凡的.设$r>0$,并且$K_{r-1}$中的元素都是可构造实数.按照$[K_r:K_{r-1}]\le2$,看到存在$a\in K_{r-1}$使得$K_r=K_ {r-1}(\sqrt{a})$.于是只要证明,如果$a\ge0$是可构造的,那么$\sqrt{a}$是可构造的.如果$A(a,0)$可构造,那么$B(-a,0)$可构造.取$BP$的中点$C$,以$C$为圆心做过点$P$的圆周,那么这个圆周交$y$轴于$Q$,那么$CQ^2=\frac{(1+a)^2} {4}$,而$CO^2=\frac{(1-a)^2}{4}$,得到$OQ^2=a$,于是$\sqrt{a}$可构造,得证.

按照这个定理,看到如果$c$是可构造实数,那么$c$是$Q$上代数元,并且$[Q(c):Q]$是2的次幂.

\begin{enumerate}
	\item 不能尺规作图三等分$60$度角.事实上如果可行,那么可构造出$x=\cos20^\circ$,但是$x$满足方程$f(x)=8x^3-6x-1=0$,并且$f$是$Q$上不可约多项式,于是$[Q(x):Q]=3$矛盾.
	\item 不能尺规作图出倍立方,也就是说不能做出长度$\sqrt[3]{2}$.事实上设这个数是$x$,有$x$在$Q$上极小多项式是$x^3-2$,同样得到$[Q(x):Q]=3$矛盾.
	\item 不能画出一个面积和半径1的圆相同的正方形.若可行等价于$x=\sqrt{\pi}$是可构造实数.但是按照Lindemann-Weierstrauss定理,$x$是$Q$上超越元,这矛盾.
\end{enumerate}

关于尺规作图的最后一个问题的解决需要稍微更多的域论知识.给定$C$上$n$次本原根$w$,那么按照分圆域的内容,有$[Q(w):Q]=\phi(n)$,其中$\phi(n)$表示$1$到$n$中和$n$互素的数的个数.断言:

一个正$n$边形可构造当且仅当$\phi(n)$是2的幂次.证明,注意到一个正$n$边形可构造当且仅当角度$\frac{2\pi}{n}$是可构造的.而这又等价于$\cos\frac{2\pi} {n}$是可构造的.取$w=e^{\frac{2\pi i}{n}}$,那么等价于$w$是可构造的.现在$w$是$x^2-2\cos(\frac{2\pi}{n})x+1$的根,于是有$[Q(w):Q(\cos\frac{2\pi}{n})]=2$.于是倘若$w$可构造,必然有$[Q(\cos(\frac{2\pi}{n})):Q]$是2的次幂.于是得到$\phi(n)= [Q(w):Q]$是2的次幂.反过来,如果$\phi(n)$是2的次幂,那么按照分圆扩张的描述,我们看到$Q(w)$是$Q$的Galois扩张,并且Galois群是交换群.如果取$Q(\cos(\frac{2\pi}{n}))\subseteq Q(w)$对应的子群是$H$,那么按照有限p群的内容,看到存在子群链$H_0\subseteq H_1\subset\cdots\subseteq H_r=H$.其中每个$[H_{i+1}:H_i]=2$,记对应的中间域是$L_i$,那么看到$L_i=L_ {i+1}(\sqrt{u_i})$,但是已经证明了可构造实数的根是可构造实数,这就证明了$Q(\cos(\frac{2\pi}{n}))$是可构造的.
\newpage
\subsection{根式解问题}

现在给出Galois理论的最后一个应用,即多项式的根式可解问题.三次和四次多项式的解法在16世纪中旬被人们得出,这和二次多项式的解决距离了整整一千多年.这一结果让人们相信高次方程总有一般的公式.但是1824年Abel证明了不存在五次方程的一般解法,即根不能总是表示为方程系数的算术运算和取根运算.可解性的彻底解决是由Galois给出,它给出了一个方程可解的充要条件.

先来给出根式可解的具体定义.给定一个域扩张$F\subseteq K$,称为根式扩张,如果存在扩张链:
$$F\subseteq F(a_1)\subseteq F(a_1,a_2)\subset\cdots\subseteq F(a_1,\cdots,a_r)$$

满足存在正整数$n_1,\cdots,n_r$使得$a_1^{n_1}\in F$,$a_i^{n_i}\in F(a_1,\cdots,a_{i-1}),\forall i>1$.如果$n_i$都等于一个固定整数$n$,那么称$K$是$F$的$n$次根式扩张,那么每个根式扩张都是$n=n_1\cdots n_r$次的根式扩张.给定多项式$f(x)\in F[x]$,称$f$是根式可解的,如果存在$F$的根式扩张$L$使得$f$在$L$上分裂.

那么举例来讲,任意特征非2域的2-kummer扩张都是2-根式扩张,如果$F$包含$n$次本原根,那么每个$n$次循环扩张$F\subseteq K$都是$n$-根式扩张.

注意.多项式的可解并没有要求一个多项式$f$在$F$上的分裂域自身是一个根式扩张.事实上一个多项式可以自身可解但是它的分裂域不是一个根式扩张.

如果$K$是$F$的$n$次根式扩张,设$N$是这个扩张的正规闭包,那么$N$也是$F$的$n$次根式扩张.证明.记$K=F(\alpha_1,\cdots,\alpha_r)$,满足$\alpha_i^n\in F(\alpha_1,\cdots,\alpha_{i-1})$.对$r$归纳.如果$r=1$,那么$K=F(\alpha)$并且$\alpha^n=a\in F$.那么$N=F(\beta_1,\cdots,\beta_m)$,其中$\beta_i$表示的是$\alpha$在$F$上极小多项式的全部根.这个极小多项式整除$x^n-a$,于是每个$\beta_i$的$n$次幂就是$a$.于是$N$是$F$的一个$n$次根式扩张.现在假设$r>1$,取$N_0$为$F(\alpha_1,\cdots,\alpha_{r-1})\subseteq K$的正规闭包,按照归纳假设,$N_0$是$F$的$n$次根式扩张.$N_0$是$F$关于$\alpha_i,1\le i\le r-1$的这$r-1$个极小多项式的分裂域,可以记$N=N_0(\gamma_1,\cdots,\gamma_m)$,其中$\gamma_i$是$\alpha_i$在$F$上极小多项式的全部根.另外有$\alpha_r^n=b\in F(\alpha_1,\cdots,\alpha_{r-1})\subseteq N_0$.另外存在$\sigma_i\in Aut_FN$,使得$\sigma_i(\alpha_r)=\gamma_i$,那么$\gamma_i^n=\sigma_i(b)$.现在按照$N_0$在$F$上正规得到$\sigma_i(b)\in N_0$.于是每个$\gamma_i$都是$N_0$中某个元的$n$次幂,于是$N$是$N_0$的$n$次根式扩张.于是$N$是$F$的$n$次根式扩张.

若$F$特征0,取$f(x)\in F[x]$,那么$K$如果是$f$在$F$上的分裂域,有$f$可解当且仅当$Aut_FK$是可解群.证明,假设$f$是根式可解的,那么存在一个$n$次根式扩张$F\subseteq M$使得$K\subseteq M$.按照特征0,可以取$w$为$M$的某个有限扩张中的$n$次本原根.那么$M\subseteq M(w)$是一个$n$次根式扩张,于是$F\subseteq M(w)$是一个$n$次根式扩张,取这个扩张的正规闭包$L$,那么按照上述引理有$L$同样是$F$的$n$次根式扩张,于是$L$同样是$F(w)$的$n$次根式扩张,记$L=F(\alpha_1,\cdots,\alpha_r)$,于是得到扩张链:
$$F=F_0\subseteq F_1=F_0(w)\subseteq F_2=F_1(\alpha_1)\subset\cdots\subseteq F_r=L$$

按照$F_i$包含了$n$次本原根,从循环扩张的描述,看到每个$F_i\subseteq F_{i+1}$都是循环的Galois扩张,另外按照$F_0\subseteq F_1$是分圆扩张,于是它的Galois群是交换群.下面注意到$F\subseteq L$是正规的,结合特征0必然可分,得到$F\subseteq L$是Galois扩张.取$G=Aut_FL$,取$H_i=Aut_{F_i}L$,那么得到子群链:
$$<id>=H_r\subseteq H_{r-1}\subset\cdots\subseteq H_0=G$$

按照Galois基本定理2,看到$H_{i+1}$总是在$H_i$上正规的,并且$H_i/H_ {i+1}$都是交换群,于是这得到$G$是可解群.于是按照$Aut_FK\sim G/Aut_KL$,得到$Aut_FK$是可解群.现在反过来,假设$Aut_FK$是可解群,那么存在链:
$$Aut_FK=H_0\supset H_1\supset\cdots\supset H_r=<id>$$

其中每个$H_{i+1}$在$H_i$中正规,并且$H_i/H_{i+1}$都是交换群.取$H_i$的固定域为$K_i$,按照基本定理2,有$K_{i+1}$在$K_i$上Galois,并且有$Aut_ {K_i}K_{i+1}\sim H_i/H_{i+1}$.取$n$为最小的正整数使得$Aut_KF$中每个元的阶数都整除$n$.取某个有限扩张中的$n$次本原根$w$,设$L_i=K_i(w)$,那么得到链:$$F\subseteq L_0\subseteq L_1\subset\cdots\subseteq L_r$$

其中$K\subseteq L_r$.注意到$L_{i+1}=L_iK_{i+1}$,那么按照$K_{i}\subseteq K_{i+1}$是Galois扩张,按照一个定理,得到$L_{i}\subseteq L_{i+1}$是Galois扩张并且$Aut_{L_i}L_{i+1}$是$Aut_ {K_i}K_{i+1}$的子群,于是是交换群,并且存在$n$的一个因数$m$使得$mAut_{L_i}L_{i+1}=0$.这就保证了$L_{i+1}$是$L_i$的$n$次根式扩张,于是得到$L_r$是$F$的根式扩张,并且$K\subseteq L_r$,于是得到$f$是根式可解的.

注意.上述证明可以放在域$F$特征$p$并且不整除$n$的情况.对于整除的情况还会在下文介绍.

给定特征0域上一个次数$\ge5$的一般形式多项式,那么不存在$f$的根式解.事实上考虑$n$次方程的一般形式:
$$f(x)=(x-t_1)(x-t_2)\cdots(x-t_n)=x^n-s_1x^{n-1}+\cdots+(-1)^ns_n\in
k(t_1,\cdots,t_n)[x]$$

其中$s_i$是$t_1,\cdots,t_n$的初等对称多项式.但是对$K=k(t_1,\cdots,t_n)$和$F=k(s_1,\cdots,s_n)$,有$Aut_FK=S_n$,对于$n\ge5$,有$S_n$不可解,这就得证.

最后给出$Q$上的一个简单情况:如果$f(x)\in Q[x]$不可约,并且次数是一个素数$p$,假设$f(x)$恰好只有两个非实数的复根,那么$f(x)$的Galois群必然是$S_p$.事实上,取共轭是分裂域上的一个自同构,并且作用在两个非实数根上为交换二者.这说明Galois群$G$作为$S_p$的子群,包含一个对换.现在按照多项式次数$p$,看到$G$的阶数被$p$整除,于是按照Cauchy定理得到一个$p$阶元,那么这对应于$S_p$中的$p$轮换,但是知道一个$p$轮换和一个对换必然生成整个$S_p$,于是$G=S_p$.

最后解释下特征$p$的情况.给定特征为$p$的域$F$,称$F$上的Artin-Schreier映射为$x\mapsto x^p-x$.为了让根式扩张适用于一般特征的域,来改进根式扩张的定义为:一个特征$p$域$F$的扩张$K$称为根式扩张,如果存在域的链$F=F_0\subseteq F_1\subset\cdots\subseteq F_n=K$,满足$F_ {i+1}=F_i(u_i)$,并且要么$u_i^{n_i}\in F_i$,要么$u_i$在Artin-Schreier映射下落在$F_i$中,即$u_i^p-u_i\in F_i$.在这个改进根式扩张的定义下.一个多项式$f(x)\in F[x]$是根式可解的当且仅当对$F$关于$f$的分裂域$K$,有$Aut_FK$是可解群.
\newpage
\section{无限Galois理论}

这一节讨论无限维的Galois理论.约定一个无限维的代数扩张$F\subseteq K$是Galois扩张,如果它是正规扩张也是可分扩张.那么首先,在一个无限维扩张中,从中间域到自同构子群的映射未必会是一个满射了.会看到正确得到Galois对应的方法是,赋予Galois群一个拓扑,称为Krull拓扑,那么把子群限制为闭子群,就得到到全部中间域的双射!先定义Krull拓扑,并给出若干拓扑性质,随后证明Galois两个基本定理的推广,即:无限维Galois扩张的Galois对应,从中间域到Galois群的闭子群存在保逆序的双射;以及,中间域是基域的Galois扩张当且仅当中间域对应的闭子群是正规子群,此时新的Galois群就是这个正规子群的商.

取$K$为$F$的一个Galois扩张,也就是说是正规可分的代数扩张.记$G$为扩张的自同构群.记$\mathscr{I}$是全体这样的中间域$E$,满足$[E:F]$有限,并且$F\subseteq E$是Galois扩张.记$\mathscr{N}$为$G$的全体子群$N$,满足存在$E\in\mathscr{I}$使得$N=Aut_EK$.即,$\mathscr{I}$表示的是全部在$F$上是有限Galois扩张的中间域,而$\mathscr{N}$是这样有限Galois中间域对应的自同构群.

如果$F\subseteq K$是正规扩张,并且有域扩张$F\subseteq L\subseteq K\subseteq N$,如果$\gamma:L\to N$是一个$F$同态,那么$\gamma(L)\subseteq K$,并且存在$\sigma\in Aut_FK$,使得$\sigma\mid_L=\gamma$.

先来探究一些$\mathscr{I}$和$\mathscr{N}$的简单性质:
\begin{enumerate}
	\item 如果$\alpha_1,\cdots,\alpha_n\in K$,那么存在$E\in\mathscr{I}$包含了全体$\alpha_1,\cdots,\alpha_n$.证明.取$E\subseteq K$为全体$\alpha_i$在$F$上极小多项式的分裂域,注意到每个元$\alpha_i$都是$F$上可分的,这就导致$E$是$F$的Galois扩张,另外这是有限个多项式的分裂域,所以维数有限,于是$E\in\mathscr{I}$.
	\item 若$N\in\mathscr{N}$,也就是说存在$E\in\mathscr{I}$使得$N=Aut_EK$.那么有$E$为$N$的固定域,并且$N$在$G$中正规,而且$G/N\sim Aut_FE$.于是有$|G/N|=[E:F]$是有限阶的.证明,一方面任取$E$中的元,按照$N$的定义看到$N$固定了$E$中元,限制假设存在$K$中的一个元$k$,使得$N$中每个元都固定$k$.如果$k$不在$E$中,记$k$在$E$上的极小多项式是$p(x)$,那么存在一个固定$E$的$K$自同构把$k$映射到另一个不同的$p(x)$的根$k'$.这矛盾.为了证明$N$是$G$的正规子群,只要构造$G\to Aut_FE$为把自同构限制到$E$上,按照$E$在$F$上是正规的,另外固定$F$的$K$自同构只能是在极小多项式的根上置换,这就说明每个$G$中的映射限制到$E$上的确是一个自同构.注意到这个群同态的核是$N=Aut_EK$,于是它是正规子群.另外注意到每个$Aut_FE$上的映射都可以延拓为$G$中的元,于是$G\to Aut_FE$是满射,于是得到结论.
	\item 有$\cap_{N\in\mathscr{N}}N={id}$.证明,取交中的一个元$\tau$,取$a\in K$,按照第一条,存在$E\in\mathscr{I}$包含了$a$,现在记$N=Aut_EK\in\mathscr{N}$.那么按照$\tau\in N$得到$\tau$固定$E$,于是$\tau(a)=a$,于是$\tau=id$.
	\item 若$N_1,N_2\in\mathscr{N}$,那么$N_1\cap N_2\in\mathscr{N}$.证明,取$N_i=Aut_{E_i}K$,其中$E_1,E_2\in\mathscr{I}$.每个$E_i$都是$F$上的有限Galois扩张,于是按照基本定理1,得到$E_1E_2$是$F$的有限Galois扩张,于是$E_1E_2\in\mathscr{I}$,但是$Aut_{E_1E_2}K=N_1\cap N_2$.这是因为,任取$\sigma\in N_1\cap N_2$,当且仅当$\sigma$在$E_1$和$E_2$上的限制都是恒等,这等价于$E_1$和$E_2$都在$\sigma$的固定域中,等价于$E_1E_2$在$\sigma$的固定域中.而这等价于$\sigma\in Aut_{E_1E_2}K$,得证.
\end{enumerate}

现在给Galois群上赋予拓扑,称为Krull拓扑:对$G$的子集$X$,它是开的当且仅当$X$是空集或者可以写作$X=\cup_i\sigma_iN_i$,其中$\sigma_i\in G$,并且$N_i\in\mathscr{N}$.那么注意到空集和全集都是开集,另外开集的任意并也是开集,最后只要说明两个开集的交仍然是一个开集.这只要证明$\tau_1N_1\cap\tau_2 N_2$是开集.取$\sigma\in\tau_1N_1\cap\tau_2N_2$,那么得到这个交就是$\sigma(N_1\cap N_2)$,按照上述引理4得到$N_1\cap N_2\in\mathscr{N}$,于是得证.

来给出Krull拓扑的一些性质.
\begin{enumerate}
	\item $\mathscr{N}$中全体子群的左陪集构成了Krull拓扑的一个拓扑基.在证明Krull拓扑的确是拓扑的时候已经给出证明.另外给定$N\in\mathscr{N}$,那么$[G:N]$有限,导致$G-\sigma N$必然是$N$的有限个左陪集的并,着说明每个$\sigma N$都是既开又闭的.
	\item Krull拓扑是Hausdorff空间.取两个不同的点$\sigma$和$\tau$,按照上述引理3看到$\{\sigma\}=\cap_{N\in\mathscr{N}}\sigma N$.于是存在一个$N\in\mathscr{N}$使得$\tau\not\in\sigma N$.按照$\sigma N$既开又闭,得到它和它的补分别包含$\sigma$和$\tau$,得证.
	\item Krull拓扑是全不连通的,这是说,它的每个连通子集都是单点子集.事实上如果$X$是$G$的一个存在两个不同点的连通子集,取两个点为$\sigma$和$\tau$,那么有$X=(\sigma N\cap X)\cup((G-\sigma N)\cap X)$,于是$X$并不连通,得证.
	\item Galois群的拓扑可以从有限Galois群上的离散拓扑构造出来.考虑$N\in\mathscr{N}$,那么$G/N$是有限群,赋予离散拓扑,考虑乘积$P=\prod_{N\in\mathscr{N}}G/N$,赋予乘积拓扑.那么存在从$G$到$P$的自然群同态$f$为$\sigma\mapsto\{\sigma N\in G/N\}_{N\in\mathscr{N}}$.那么按照上述引理3,得到$f$是单同态,现在来证明$f$是到像$\mathrm{im}f$的同胚.注意到按照积拓扑的定义,$P$中的一组拓扑基为全体有限个$\pi_N^{-1}(\gamma N)$的交.于是为证明$f$连续,只要证明每个$f^{-1}(\pi_N^{-1}(\gamma N))$是$G$中开集,但是这个原像就说$\gamma N$,是开集,于是$f$连续.另外有$f(\gamma N)=\pi_N^{-1}(\gamma N)\cap\mathrm{im}(f)$,于是$f^{-1}$是从$\mathrm{im}f$到$G$的连续映射,于是$f$是从$G$到$\mathrm{im}$的同胚.
	\item 上述$\mathrm{im}f$在$P$中是闭集.证明,按照$G/N$同构于$Aut_FE$,这里$E_N$是$N$的固定域.在这个同构下,陪集$\gamma N$对应于$\gamma\mid_{E_N}$.于是对任意$\rho\in P$,有$\pi_N(\rho)$是$E_N$上的一个自同构.并且对每个$\gamma\in G$有$\pi_N(f(\gamma))=\gamma\mid_{E_N}$.现在取:
	$$C=\{\rho\in P:\forall N,M\in\mathscr{N},\pi_N(\rho)\mid_{E_N\cap E_M}=\pi_M(\rho)\mid_{E_N\cap E_M}\}$$
	
	那么断言$C=\mathrm{im}f$.一方面必然有$\mathrm{im}f\subseteq C$.现在任取$\rho\in C$,定义$\gamma:K\to K$为,对任意的$a\in K$,取$E_N\in \mathscr{I}$使得$a\in E_N$,定义$\gamma(a)=\pi_N(\rho)(a)$.按照$C$的定义看出这是良性的.为了验证这是环同态,只要注意到对任意的$a,b\in K$,存在一个$E_N\in\mathscr{I}$包含了$a,b$,而$\gamma$限制在$E_N$上是$\pi_N(\rho)$是环同态,于是成立.利用$\rho^{-1}$可以验证$\gamma^{-1}$存在,于是$\gamma$是双射.由此得到$\gamma\in G$,于是$f(\gamma)=\rho$.于是$C=\mathrm{im}f$.最后验证$C$在$P$中是闭集,任取$\rho\in P$,$\rho\not\in C$.那么存在$N,M\in\mathscr{N}$使得$\pi_N(\rho)\mid_{E_N\cap E_M}\not=\pi_M(\rho)\mid_{E_N\cap E_M}$.于是$\pi_N^ {-1}(\pi_N(\rho))\cap\pi_M^{-1}(\pi_M(\rho))$是$P$的一个开集,并且和$C$不交,这就说明了$C$是闭集.
	\item Krull拓扑是紧的.按照上一条,注意到离散拓扑是$T_2$也是紧的,于是$P$是$T_2$也是紧的(紧保积拓扑是Tychonoff定理).但是紧致$T_2$空间的闭子集是紧的,于是$\mathrm{im}f$是紧的.
\end{enumerate}

集合$\mathscr{N}$赋予逆向包含序是一个定向集,这是说如果$N_1,N_2\in\mathscr{N}$,那么存在一个$N_3\in\mathscr{N}$使得$N_3\subseteq N_1\cap N_2$.对集合$G/N,N\in\mathscr{N}$,和自然投射$G/N_1\to G/N_2$,如果$N_1\subseteq N_2$,构成了一个群的逆向系统.那么$G$可以作为这个逆向系统的逆向极限.称有限群的逆向极限是profinite群.为了给出无限Galois扩张的基本定理,还需要:

给定$G$的子群$H$,取$H'$为扩张$K^H\subseteq K$对应的自同构群,这里$K^H$为$H$的固定域,那么$H'$是$H$的闭包.
\begin{proof}
	
	易知$H\subseteq H'$,下面来证明$H'$闭,并且$H'\subseteq H$的闭包.为证$H'$闭,任取$\sigma\in G-H'$,那么存在一个$\alpha\in K^H$使得$\sigma(\alpha)\not=\alpha$.按照引理1知存在了一个$E\in \mathscr{I}$使得$\alpha\in E$,那么$N=Aut_EK\in\mathscr{N}$.于是对任意的$\gamma\in N$,有$\gamma(\alpha)=\alpha$.于是$\sigma\gamma(\alpha)\not=\alpha$,于是$\sigma N$是和$H'$不交的开集,于是$H'$是闭集.为证明$H'\subseteq H$的闭包.设$L=K^H$,取$\sigma\in H'$,取$N\in\mathscr{N}$,记$E=K^N\in\mathscr{I}$.记$H_0=\{\rho\mid_E:\rho\in H\}$这是有限群$Aut_FE$的子群.有$K^{H_0}=K^H\cap E=L\cap E$,按照有限Galois扩张的基本定理,有$H_0=Aut_{E\cap L}E$,按照$\sigma\in H'$得到$\sigma\mid_L=id$,于是$\sigma\mid_E\in H_0$.于是存在$\rho\in H$使得$\rho\mid_E=\sigma\mid_E$.于是$\sigma^{-1}\rho\in N$,于是$\rho\in\sigma N\cap H$.这说明每个Krull拓扑的基元素$\sigma N$和$H$相交,于是$\sigma$在$H$的闭包中,得证.
	
\end{proof}

无限Galois扩张基本定理.给定Galois扩张$F\subseteq K$,给Galois群$G$赋予Krull拓扑.
\begin{enumerate}
	\item 映射$L\mapsto Aut_LK$和$H\mapsto K^H$是从扩张的中间域和$G$的闭子群之间的反保序双射.
	\item 若中间域$L$对应闭子群$H$,那么$[G:H]$有限当且仅当$[L:F]$有限,当且仅当$H$是开集.并且当这成立时$[G:H]=[L:F]$.
	\item 若中间域$L$对应闭子群$H$,那么$H$是正规子群当且仅当$L$在$F$上是Galois扩张,并且当着成立时,群的典范同构$Aut_FL\sim G/N$也是同胚.
\end{enumerate}
\begin{proof}
	
	取中间域$L$,那么$K$在$L$上正规并且可分,于是$K$在$L$上是Galois的.于是$Aut_LK$的固定域是$L$.限制取$G$的子群$H$,按照引理看到$H=Aut_ {K^H}K$当且仅当$H$是闭集.于是1得证.
	
	现在任取中间域$L$,取$H=Aut_LK$,假设$[G:H]$有限,那么$G-H$是$H$陪集的有限并,每一个都是闭集,于是$G-H$是闭集,于是$H$是开集.反过来,如果$H$是开集,那么$H$包含了$id$的一个拓扑基元素,于是存在$\mathscr{N}$中的$N$使得$N\subseteq H$.取$N$的固定域为$E$,那么$L\subseteq E$,于是必然有$[L:F]$有限.如果$[L:F]$有限,取$E\in\mathscr{I}$,使得$L\subseteq E$.取$N=Aut_EK$,那么$N\subseteq H$.于是有$[G:H]\le[G:N]$有限.最后注意到$G/N\sim Aut_FE$通过对应$\sigma N\mapsto\sigma\mid_E$.于是$H/N$对应于$\{\rho\mid_E:\rho\in H\}$,它的固定域为$L\cap E=L$.于是按照有限Galois扩张基本定理,得到$[G:H]= [G/H:H/N]=\frac{|G/N|}{|H/N|}=\frac{[E:F]}{[E:L]}=[L:F]$.这就证明了2.
	
	现在证明最后一条.记$H=Aut_LK$.假设$H$是正规子群.对$a\in L$,取在$F$上的极小多项式为$f(x)$,取$f$的任意根$b\in K$.于是存在$\sigma\in G$使得$\sigma(a)=b$.任取$\gamma\in H$,有$\gamma(b)=\sigma^{-1}(\sigma\gamma\sigma^{-1}(a))=\sigma^{-1}(a)=b$.于是$b$落在$H$的固定域,也就是$L$中.于是$f$在$L$上分裂,这说明$L$在$F$上正规.反过来,如果$L$在$F$上是Galois的.于是可以构造$\theta:G\to Aut_FL$为限制映射,它的核是$H$,于是$H$正规子群.另外这个映射是满射,于是按照同态定理得到$v:G/H\sim Aut_FL$.
	
	最后来证明$v:G/H\to Aut_FL$是一个同胚.注意到$Aut_FL$的一个拓扑基为全体$\rho Aut_EL$,其中$E$为$F$上有限Galois扩张的$F\subseteq L$的中间域.现在取$N=Aut_EK\in\mathscr{N}$,那么$\theta^{-1}(Aut_EL)=N$.于是$\theta^ {-1}(\rho Aut_EL)=\gamma N$,其中$\gamma\in G$是任意的在$L$上限制为$\rho$的同构.于是原像是$G$中开集,于是$\theta$是连续的.现在注意到$G$是紧的,而$Aut_FL$是Hausdorff的,于是$\theta$是闭映射.那么诱导的商映射$v$也是连续并且闭的,结合它是双射,看到它是同胚.
	
\end{proof}

对于有限Galois扩张$F\subseteq K$,那么$Aut_FK$是有限群,此时按照定义Krull拓扑是离散拓扑,于是定理结论吻合于有限Galois扩张基本定理.

例子.考虑$Q$上又全部单位根生成的域$K$,即$K$是$Q$的关于多项式集$\{x^n-1,n\ge1\}$的分裂域,这说明$Q\subseteq K$是Galois扩张.断言这里的Galois群是交换群.事实上如果任取有限Galois扩张的中间域$L$,即$L$是中间域并且是$Q$的有限Galois扩张,那么看到$L$必然包含在某个分圆域中,但是分圆域的Galois群是交换群,结合前文证明的Galois群就说全体有限Galois群的直积的一个子群,于是立马得到是交换群.

例子.考虑$F_p$的代数闭包$K$,由于$F_p$是完美域,说明$K$是可分的,由此得到$F_p\subseteq K$是Galois扩张.扩张的Galois群为:
$$G\sim\left\{\{n_r\}\in\prod_rF_{p^r}: if\quad r\mid s,then\quad n_s\equiv n_r(\mod r) \right\}$$

可分闭包.给定域$F$,称$F$是代数可分闭的,如果每个代数可分扩张都是自身.取$F$的一个代数闭包$\overline{F}$,那么$F\subset\overline{F}$是正规扩张.但是它未必是可分扩张,导致$\overline{F}$未必是$F$的Galois扩张.如果取$F_s$为$F$在$\overline{F}$中的可分闭包,即$\overline{F}$中全部可分元构成的中间域,那么:

$F_s$在$F$上是Galois扩张,并且$Aut_FF_s\sim Aut_F\overline{F}$.$F_s$是$F$的极大可分扩张,即$F$的任意可分扩张都包含在$F_S$里,由此得到$F_s$是代数可分闭的.证明,域$F_s$在$F$上是Galois是容易的:任取$a\in F_s$,考虑$a$的极小多项式$p(x)\in F[x]$,那么$p(x)$没有重根,并且$p(x)$在$\overline{F}$上分裂,那么它的全部根都是$F$上的可分元,于是$p(x)$在$F_s$上分裂,这就导致$F\subseteq F_s$是正规扩张.于是它是Galois扩张.现在构造群同态$\theta:Aut_F\overline{F}\to Aut_FF_s$为限制到$F_s$.那么它的核为$Aut_{F_s}\overline{F}$.但是$F_s\subset\overline{F}$是纯不可分扩张,它的自同构群是平凡群,由此得到了自同构群之间的同构.现在假设有域代数扩张$F_s\subseteq L$,并且$F\subseteq L$是可分代数扩张.那么可以把$L$嵌入到$\overline{F}$中,于是必然有$L=F_s$.

把$Aut_FF_s\sim Aut_F\overline{F}$称为域$F$的绝对Galois群.那么从基本定理看到,任意$F$的Galois扩张的Galois群都是绝对Galois群的同态像.
\newpage
\section{超越扩张}
\subsection{超越基}

代数无关性.给定域扩张$k\subseteq F$,若$t_1,\cdots,t_n\in F$满足对任意$f\in k[x_1,\cdots,x_n]$有$f(t_1,\cdots,t_n)\not=0$,就称$\{t_1,\cdots,t_n\}$是$k$上的代数无关集.$F$上的一个无限子集称为代数无关集,如果它的任意有限子集都是代数无关的.不是代数无关集的$K$的子集称为代数相关集.下面给出一些基本性质和例子:
\begin{enumerate}
	\item 给定扩张$k\subseteq F$,那么$a\in F$自身构成代数无关集$\{a\}$当且仅当$a$是$k$上的超越元.在$k=\mathbb{Q}$的情况下就是超越数的概念,例如$\{e\}$,$\{\pi\}$分别都是代数无关集.另外$\{e,e^2\}$是代数相关的.$\{e,\pi\}$是否的代数无关集还是公开问题.
	\item 若$\{t_1,t_2,\cdots,t_n\}\subseteq F$是$K$上的代数无关系,那么必然有$K[t_1,\cdots,t_n]$和$K$上的$n$元多项式环$K[X_1,\cdots,X_n]$是$K$同构的.我们在后文证明这个逆命题.另外如果$r_1,\cdots,r_n$是一组正整数,那么$\{t_1^{r_1},\cdots,t_n^{r_n}\}$也是$K$上的代数无关集.
	\item 设$\{x_1,\cdots,x_n\}\subseteq F$是$K$上的代数无关集,设$A=(a_{ij})$是$K$上的一个矩阵,设$y_j=\sum_ia_{ij}x_i$,那么$\{y_1,\cdots,y_n\}$是$K$上的代数无关集当且仅当$\det A\not=0$.
	\begin{proof}
		
		首先$x_i\mapsto y_i,1\le i\le n$诱导了$F=K[X_1,\cdots,X_n]$到自身的一个环同态$\varphi$.如果$A$是可逆矩阵,逆矩阵$A^{-1}$也诱导了$F$上的一个环同态$\psi$,并且有$\psi$和$\varphi$互为逆映射,于是特别的$\varphi$是单射,也即对任意非零多项式$h(X_1,\cdots,X_n)$,有$h(y_1,\cdots,y_n)\not=0$,也即$\{y_1,\cdots,y_n\}$是$K$上代数无关集.
		
		\qquad
		
		反过来如果$\det A=0$,那么$A$的列向量组$\{C_1,\cdots,C_n\}$是线性相关的,记$\sum_jb_jC_j=0$,其中$b_j\in K$不全为零,那么有$\sum_jb_jy_j=\sum_{i,j}b_ja_{ij}x_i=\sum_ix_i\sum_jb_ja_{ij}=0$.于是$\{y_1,\cdots,y_n\}$是代数相关的.
	\end{proof}
	\item 考虑域扩张链$K\subseteq F\subseteq L$,如果子集$S\subseteq L$在$F$上代数无关,那么$S$也在$K$上代数无关.如果$S\subseteq F$在$K$上代数无关,那么任意子集$S_1\subseteq S$也在$K$上代数无关.
    \item 设$F/K$是域扩张,设$t_1,\cdots,t_n\in F$,那么如下命题互相等价:
    \begin{enumerate}
    	\item 集合$\{t_1,\cdots,t_n\}$是$K$上的代数无关集.
    	\item 对任意$1\le i\le n$,有$t_i$是$K(t_1,\cdots,t_{i-1},t_{i+1},\cdots,t_n)$上的超越元.
    	\item 对任意$1\le i\le n$,有$t_i$是$K(t_1,\cdots,t_{i-1})$上的超越元.
    \end{enumerate}

    特别的,如果$S\subseteq F$在$K$上代数无关,并且$t\in F$在$K(S)$上超越,那么$S\cup\{t\}$在$K$上代数无关.
    \begin{proof}
    	
    	(a)$\Rightarrow$(b):如果存在$K(t_1,\cdots,t_{i-1},t_{i+1},\cdots,t_n)$系数的非零的一元多项式$f[T]$使得$f(t_i)=0$,把$f(t_i)=0$这个等式整式化,就得到一个非零多项式$\widetilde{f}\in K[T_1,\cdots,T_n]$使得$\widetilde{f}(t_1,\cdots,t_n)=0$,这和代数无关性相矛盾.
    	
    	\qquad
    	
    	(b)$\Rightarrow$(c)就是因为代数无关集在更小的域上一定还是代数无关的.(c)$\Rightarrow$(a):假设$\{t_1,\cdots,t_n\}$不是$K$上的代数无关集,可取最小的正整数$m$,使得$\{t_1,\cdots,t_{m-1}\}$是$K$上代数无关集(如果$m=0$约定这是空集),但是$\{t_1,\cdots,t_m\}$不是$K$上代数无关集.于是存在非零多项式$f\in K[T_1,\cdots,T_m]$,使得$f(t_1,\cdots,t_m)=0$.可记$f=\sum_{j\ge0}f_j(T_1,\cdots,T_{m-1})T_m^j$,其中$f_j$是$m-1$元的$K$系数多项式,这里$f_1,f_2,\cdots$中至少存在一项不是零多项式,否则就导致$\{t_1,\cdots,t_{m-1}\}$代数相关和选取矛盾.但是这就导致$t_m$是$K(t_1,\cdots,t_{m-1})$上的代数元,和条件矛盾.
    \end{proof}
\end{enumerate}

纯超越扩张和超越基.域扩张$F/K$称为纯超越扩张,如果$F$是$K$同构于$K$的某个有理函数域.设$F/K$是域扩张,一个子集$S\subseteq F$称为该扩张的超越基,如果$S$在$K$上代数无关,并且$F$在$K(S)$上是代数扩张.
\begin{enumerate}
	\item 我们约定空集也是代数无关集,那么一个扩张是代数扩张当且仅当空集是超越基.
	\item 如果$\{x_1,\cdots,x_n\}$是$F/K$的超越基,设$r_1,\cdots,r_n$是一组正整数,那么$\{x_1^{r_1},\cdots,x_n^{r_n}\}$也是$F/K$的一组超越基.这是因为明显的有$K(x_1^{r_1},\cdots,x_n^{r_n})\subseteq K(x_1,\cdots,x_n)$是代数扩张.
	\item 例子.设$k$是域,设$f\in k[X_1,\cdots,X_n]$是不可约多项式,设$A=k[X_1,\cdots,X_n]/(f)$,这是整环,商域记作$K$.假设$f'_n\not=0$,也即如果记$f=\sum_{i=0}^mg_ix_n^i$,其中$g_i\in k[X_1,\cdots,X_{n-1}]$,那么$m\ge1$.记$t_i=x_i+(f)\in A$,我们断言$\{t_1,\cdots,t_{n-1}\}$是$K/k$的一组超越基.
	\begin{proof}
		
		一方面有$K=k(t_1,\cdots,t_n)$,并且$t_n$在$k(t_1,\cdots,t_{n-1})$上代数.另一方面我们证明$\{t_1,\cdots,t_{n-1}\}$是$k$上的代数无关集.如果$h\in k[X_1,\cdots,X_{n-1}]$使得$h(t_1,\cdots,t_{n-1})=0$,那么$h(X_1,\cdots,X_{n-1})\in (f)$,也即存在$g\in k[X_1,\cdots,X_n]$使得$h=fg$.但是这里$h$不包含不定元$X_n$,而$f$包含,这迫使$h=0$.
	\end{proof}
    \item 超越基的存在性.设$F/K$是域扩张.
    \begin{enumerate}
    	\item $F/K$存在超越基,并且超越基就是$F$的在$K$上的代数无关集在包含序下的极大元.
    	\item 如果$T\subseteq F$使得$K(T)\subseteq F$是代数扩张,那么$T$包含了$F/K$的一组超越基.
    	\item 如果$S\subseteq F$在$K$上代数无关,那么$S$可以扩充成$F/K$的一组超越基.
    	\item 如果$S\subseteq T\subseteq F$满足$S$在$K$上代数无关,并且$K(T)\subseteq F$是代数扩张,那么存在$F/K$的一组超越基$X$满足$S\subseteq X\subseteq T$.
    \end{enumerate}
    \begin{proof}
    	
    	归结为证明(d).记$\mathscr{A}$表示$F$的所有满足$S\subseteq X\subseteq T$的代数无关集$X$,那么$X$是非空的,因为$S\in\mathscr{A}$.如果有$X_i\in\mathscr{A},i\in I$,其中$I$是一个全序集,满足只要$i\le j$就有$X_i\subseteq X_j$.那么有$S\subseteq\cup_iX_i\subseteq T$,并且$\cup_iX_i$是代数无关集.于是Zorn引理说明$\mathscr{A}$中存在极大元,仍然记作$X$,我们断言$K(X)\subseteq K(T)$是代数扩张,因为假设$t\in T$在$K(X)$上超越,那么$X\cup\{t\}$理应仍然是$K$上代数无关集,这和$X$的极大性矛盾,于是复合$K(X)\subseteq K(T)\subseteq F$也是代数扩张,这说明$X$是超越基.
    \end{proof}
    \item 设$F/K$是域扩张,并且$S,T\subseteq F$都是扩张的超越基,那么$|S|=|T|$.
    \begin{proof}
    	
    	先设$S$是有限集,记作$S=\{s_1,\cdots,s_n\}$.按照定义$F$不会是$K(S-\{s_1\})$上的代数扩张,于是必然存在某个$t\in T$使得$t$在$K(S-\{s_1\})$上超越(否则的话$T$中的元都在$K(S-\{s_1\})$上代数,结合$K(T)\subseteq F$是代数扩张,得到$F$在$K(S-\{s_1\})$上代数,矛盾),于是$\{s_2,\cdots,s_n,t\}$是$K$上的代数无关集.另外$s_1$必须在$K(s_2,\cdots,s_n,t)$上代数,因为否则的话$\{s_1,\cdots,s_n,t\}$也是代数无关集,这和$S$的极大性矛盾.记$t_1=t$,于是我们证明了$\{s_2,\cdots,s_n,t_1\}$也是一组超越基.归纳操作下去得到$T$中的$n$个元$t_1,\cdots,t_n$构成扩张的超越基,这迫使$|T|=n$.
    	
    	\qquad
    	
    	下面设$S$和$T$都是无限集合.任取$t\in T$,那么它在$K(S)$上代数,否则$S\cup\{t\}$也是代数无关集和$S$的极大性矛盾.于是存在一个有限集合$S_t\subseteq S$使得$t$在$K(S_t)$上代数.记$S'=\cup_{t\in T}S_t\subseteq S$,那么$T$在$K(S')$上代数.又因为$F$在$K(T)$上代数,得到$F$在$K(S')$上代数.于是$S'$得包含扩张的一组超越基,这迫使$S=S'$.于是有$|S|=|S'|=|\cup_{t\in T}S_t|\le|T\times\mathbb{N}|=|T|$.同理得到$|T|\le|S|$,于是$|S|=|T|$.
    \end{proof}
    \item 于是扩张$F/K$的超越基的势不依赖于超越基的选取,它称为扩张的超越维数,记作$\mathrm{tr.deg}(F/K)$.
    \item 推论.设$t_1,\cdots,t_n\in F$,那么$\{t_1,\cdots,t_n\}$在$K$上代数无关当且仅当$K(t_1,\cdots,t_n)$和$K(X_1,\cdots,X_n)$是$K$同构的.
    \begin{proof}
    	
    	必要性证过了,反过来如果有$K$同构$K(t_1,\cdots,t_n)\cong K(X_1,\cdots,X_n)$,由于$K\subseteq K(X_1,\cdots,X_n)$的超越维数是$n$,得到$K\subseteq K(t_1,\cdots,t_n)$的超越维数是$n$,我们之前解释了此时$\{t_1,\cdots,t_n\}$必须包含一组超越基,但是元素个数必须是$n$,这迫使$\{t_1,\cdots,t_n\}$是代数无关的.
    \end{proof}
    \item 超越维数的传递公式.设$K\subseteq F\subseteq L$是域的扩张链,那么有:
    $$\mathrm{tr.deg}(L/k)=\mathrm{tr.deg}(L/F)+\mathrm{tr.deg}(F/k)$$
    \begin{proof}
    	
    	设$F/K$的一组超越基为$S$,设$L/F$的一组超越基为$T$,那么明显有$T\cap S=\emptyset$,我们断言$S\cup T$是$L/K$的一组超越基,这就得到维数公式.因为$T$在$F$上代数无关,就有$T$在$K(S)$上代数无关,于是$S\cup T$在$K$上代数无关.又因为$F(T)\subseteq L$和$K(S)\subseteq F$是代数扩张,就得到$F(T)$是$K(S)(T)=K(S\cup T)$上代数扩张,于是$K(S\cup T)\subseteq L$是代数扩张.这说明$S\cup T$是$L/K$的超越基.
    \end{proof}
    \item 例子.设$k$是域,$K=k(X_1,\cdots,X_n)$,记$\{X_1,\cdots,X_n\}$的初等对称多项式为$S_1,\cdots,S_n$.我们解释过$k(S_1,S_2,\cdots,S_n)\subseteq k(X_1,X_2,\cdots,X_n)$是维数$n!$的有限扩张,这里$k\subseteq k(X_1,\cdots,X_n)$的超越维数是$n$,于是$\{S_1,\cdots,S_n\}$要包含一组超越基,迫使这个超越基就是$\{S_1,\cdots,S_n\}$本身.
\end{enumerate}

超越扩张版本的同构扩张定理.
\begin{enumerate}
	\item 设有域扩张$k_i\subseteq F_i,i=1,2$,并且有$S_i\subseteq F_i,i=1,2$使得$S_i$在$k_i$上代数无关.任取集合间的单射$\phi:S_1\to S_2$,以及一个域同态$\sigma:k_1\to k_2$,那么$\sigma$唯一的延拓为$\sigma':k_1(S_1)\to k_2(S_2)$,使得$\sigma'(s)=\phi(s),\forall s\in S$.特别的,如果$\phi$是双射,$\sigma$是同构,那么延拓的$\sigma'$是同构.更特别的,如果$F_1,F_2$都是代数闭域,并且$\mathrm{tr.deg}{F_1/k_1}=\mathrm{tr.deg}{F_2/k_2}$,那么每个域同构$k_1\cong k_2$均可延拓为域同构$F_1\cong F_2$.
	\begin{proof}
		
		对每个$n\ge1$,$\sigma$诱导了单的环同态$k_1[x_1,x_2,\cdots,x_n]\to k_2[x_1,x_2,\cdots,x_n]$.现在$k_1(S_1)$中每个元可约表示为$\frac{f(s_1,s_2,\cdots,s_n)}{g(s_1,s_2,\cdots,s_n)},s_i\in S$.构造$\sigma':k_1(S_1)\to k_2(S_2)$如下,验证它满足要求即可.
		$$\frac{f(s_1,s_2,\cdots,s_n)}{g(s_1,s_2,\cdots,s_n)}\mapsto\frac{\sigma f(\phi(s_1),\phi(s_2),\cdots,\phi(s_n))}{\sigma g(\phi(s_1),\phi(s_2),\cdots,\phi(s_n))}\in k_2(S_2)$$
	\end{proof}
    \item 例子.我们来说明$|\mathrm{Aut}_{\mathbb{Q}}(\mathbb{C})|=\infty$.考虑域扩张$\mathbb{Q}\subset\mathbb{C}$,它的超越基必然是无限的:如果有有限的超越基$t_1,t_2,\cdots,t_n$,那么$\mathbb{C}$在$\mathbb{Q}(t_1,t_2,\cdots,t_n)$上代数,这导致$\mathbb{Q}$和$\mathbb{C}$具有相同的基数,这矛盾.现在取定超越基$T$,那么$\mathbb{Q}(T)\subset\mathbb{C}$是代数扩张,对$T$上每一个置换$\sigma$,它诱导了固定$\mathbb{Q}$的$\mathbb{C}$上的自同构.换句话说,$|\mathrm{Aut}_{\mathbb{Q}}(\mathbb{C})|=\infty$.
\end{enumerate}

有理函数域的Luroth定理.若$k(t)$是域$k$上的一元有理函数域,任取扩张$k\subseteq k(t)$的中间域$F$,那么$F$是单扩张,把扩张元记作$u=f(t)/g(t)$,其中$f,g$互素,那么有$[k(t):F]=\max\{\deg f,\deg g\}$.于是特别的$k\subseteq k(t)$的每个子扩张都是纯超越扩张.
\begin{proof}
	
	记$K=k(t)$,首先任取$u\in K-k$.那么存在$f,g\in k[t]$并且$(f,g)=1$,记$F=k(u)$,断言$[K:F]=\max\{\deg f,\deg g\}$.于是$F\subseteq K$是一个有限扩张.为此,考虑多项式$p(x)=ug(x)-f(x)\in F[x]$.那么$p(t)=0$,于是$t$在$F$上是代数元,并且有$K=F(t)$,于是有$F\subseteq K$是代数扩张.于是$[K:F]$有限.记$f(t)=\sum_{i=0}^{n}a_it^i$,记$g(t)=\sum_{i=0}^{m}b_it^i$.那么$\deg p=\max\{\deg f,\deg g\}$,因为如果$\deg f=\deg g$并且最高次项系数为0,会导致$ub_n-a_n=0$,导致$u\in k$和定义矛盾.现在证明$p(x)$在$F$上不可约.首先$u$不会是$k$上代数元,否则导致$k\subseteq K$是有限扩张.于是$u$是$k$上超越元,于是$k[u]\sim k[x]$.按照$p\in k[x][u]\subseteq k(x)[u]$并且$p$是关于$u$的一次多项式,于是$p$在$k(x)$上不可约.再按照$(f,g)=1$,看到$p$是$k[x][u]$上的本原多项式.于是$p$是$k[x]$上的不可约多项式,这导致了$p$是$k[u]$上的作为$x$的多项式的不可约多项式.于是得证.现在回到原问题,任取$F$中一个不在$k$中的元$v$,那么按照之前的论述,看到$[K:k(v)]$有限,于是根据$k(v)\subseteq F$得到$[K:F]$有限.取$t$在$F$上的极小多项式为$f(x)=x^n+l_{n-1}x^{n-1}+\cdots+l_0$.那么$[K:F]=n$.按照$t$在$k$上超越,看到存在某个$l_i\not\in k$.记这个$l_i$为$u$,记$m=[K:k(u)]$.那么有$m\ge n$.只要证明$m\le n$,就得到$k(u)=F$,这会完成证明.记$c_0,\cdots,c_{n-1},d\in k[t]$,使得$l_j=c_j(t)/d(t)$,其中$(d,c_0,\cdots,c_{n-1})=1$.按照$u=c_i(t)/d(t)$,得到$m\le\max\{\deg c_i,\deg d\}$.现在取$f(x,t)=d(t)f(x)=\sum c_j(t)x^t$,其中$c_n(t)=d(t)$.于是看到$f(x,t)$是$k[x,t]$中的多项式,并且作为$x$的多项式是本原的.现在通过除掉$(c_i,d)$,让$u=g(t)/h(t)$,其中$(g,h)=1$,于是$t$是$g(x)-uh(x)\in F[x]$的根,于是有$g(x)-uh(x)=q(x)f(x)$,其中$q(x)\in F[x]$.于是结合$f$本原得到$g(x)h(t)-g(t)h(x)=r(x,t)f(x,t)$.这个等式左边的$t$的次数至多是$m$,但是$f$上$t$的次数至少为$m$,于是得到$r(x,t)=r(x)\in k[x]$,于是$rf$作为$k[t]$上的关于$x$的多项式是本原的.于是作为$k[x]$上的关于$t$的多项式是本原的,但是$r(x)$理应整除右侧作为$t$多项式的全部系数,这导致$r\in k$.于是得到$n=\deg_xf(x,t)=\deg_x(g(x)h(t)-g(t)h(x))=\deg_t(g(x)h(t)-g(t)h(x))=\deg_tf(x,t)\ge m$,得证.
\end{proof}
\newpage
\subsection{线性无交}

线性无交性.
\begin{enumerate}
	\item 设$K\subseteq C$的两个中间域为$F,L$,称$F$和$L$在$K$上线性无交(linearly disjoint),如果它们满足如下等价命题中的任意一个:
	\begin{enumerate}
		\item 如果$x_1,\cdots,x_n\in F$在$K$上线性无关,那么它们也在$L$上线性无关.
		\item 如果$y_1,\cdots,y_m\in L$在$K$上线性无关,那么它们也在$F$上线性无关.
		\item 典范同态$F\otimes_KL\to FL$,$a\otimes b\mapsto ab$是同构(等价于单射,因为这个同态总是满射,也等价于$F\otimes_kL$是域,因为源端为域的非零同态只能是单的).
	\end{enumerate}

    模仿(c),如果$A,B$是域扩张$K\subseteq C$的两个中间环,称$A$和$B$在$K$上线性无交,如果典范同态$A\otimes_KB\to C$,$a\otimes b\mapsto ab$是单射(也等价于同构,因为这个典范同态总是满射).
    \begin{proof}
    	
    	(a)$\Rightarrow$(c):设$x=\sum_{i=1}^na_i\otimes b_i$落在$F\otimes_KL\to FL$的核中.我们可以不妨设$a_1,\cdots,a_n$在$K$上线性无关,否则可以选取$F/K$上的一组基,把$a_1,\cdots,a_n$拆成这组基的线性组合再整理成$\sum_{i=1}^ma_i'\otimes b_i'$,使得$a_1',\cdots,a_m'$在$K$上线性无关.那么从$\sum_{i=1}^na_ib_i=0$,以及$\{a_1,\cdots,a_n\}$在$L$上线性无关,得到$b_1=\cdots=b_n=0$,于是$x=0$.
    	
    	\qquad
    	
    	(c)$\Rightarrow$(a):如果$x_1,\cdots,x_n\in F$在$K$上线性无关,假设有$y_1,\cdots,y_n\in L$使得$\sum_{i=1}^nx_iy_i=0$,那么$\sum_{i=1}^nx_i\otimes y_i$就在$F\otimes_KL\to FL$的核中,于是$\sum_{i=1}^nx_i\otimes y_i=0$.但是这迫使$y_1=\cdots=y_n=0$,于是$x_1,\cdots,x_n$也在$L$上线性无关.
    \end{proof}
    \item 设$K\subseteq C$是域扩张,如果环$A,A',B,B'$满足$K\subseteq A\subseteq A'\subseteq C$和$K\subseteq B\subseteq B'\subseteq C$,并且$A'$和$B'$在$K$上线性无交,那么$A$和$B$也在$K$上线性无交.
    \begin{proof}
    	
    	由于$K$是域,典范同态$A\otimes_KB\to A'\otimes_KB'$是单射,所以如果$A'\otimes_KB'\to C$,$a'\otimes b'\mapsto a'b'$是单射,那么限制在$A\otimes_KB$上也是单射.
    \end{proof}
    \item 如果$F,L$是$K$的有限扩张,那么$\dim_K(F\otimes_KL)=[F:K][L:K]$,由于典范同态$F\otimes_KL\to FL$是满同态,就有$[FL:K]\le[F:K][L:K]$.那么$F,L$在$K$上线性无交当且仅当$[FL:K]=[F:K][L:K]$.特别的如果$[F:K]$和$[L:K]$有限并且互素,那么$[F:K]\mid[FL:K]$和$[L:K]\mid[FL:K]$导致$[F:K][L:K]\mid[FL:K]$,于是$[F:K][L:K]\le[FL:K]$.
    \item 如果$F$和$L$在$K$上线性无交,那么$F\cap L$和自身在$K$上线性无交,但是这迫使$F\cap L=K$,因为否则的话取$a\in F\cap L-K$,那么$1,a\in L$在$K$上线性无关,但是它们在$F\cap L$上线性无关.不过反过来$F\cap L=K$一般不能推出$F$和$L$在$K$上线性无交.
    \item 但是对于Galois扩张,$F\cap L=K$已经够了:设$F/K$是有限Galois扩张,设$L/K$是任意扩张,那么$F$和$L$在$K$上线性无交当且仅当$F\cap L=K$.这件事是因为,条件下有$[FL:L]=[F:F\cap L]$,所以$[FL:K]=[F:K][L:K]$等价于$[F:F\cap L]=[F:K]$,等价于$F\cap L=K$.
    \item 如果记$A,B$在$C$中的商域分别是$F,L$,我们断言$A$和$B$在$K$上线性无交当且仅当$F$和$L$在$K$上线性无交.
    \begin{proof}
    	
    	一方面如果$F$和$L$在$K$上线性无交,那么$A$和$B$在$K$上线性无交.反过来如果$A$和$B$在$K$上线性无交,设$\{x_1,\cdots,x_n\}\subseteq F$在$K$上线性无关,设$y_1,\cdots,y_n\in L$满足$\sum_ix_iy_i=0$.那么存在$s\in A$和$t\in B$使得每个$a_i=sx_i\in A$和每个$b_i=ty_i\in B$.并且$\{a_1,\cdots,a_n\}\subseteq A$也在$K$上线性无关,由于$\sum_ia_ib_i=0$,导致$\sum_ia_i\otimes b_i$在$A\otimes_KB\to C$的核中,于是$\sum_ia_i\otimes b_i=0$,这迫使$b_1=\cdots=b_n=0$,于是$y_1=\cdots=y_n=0$,于是$x_1,\cdots,x_n$在$L$上线性无关,于是$F$和$L$在$K$上线性无交.
    \end{proof}
    \item 例子.代数扩张和纯超越扩张总是线性无交的.
    \begin{proof}
    	
    	设$F/K$是代数扩张,设$L/K$是纯超越扩张,我们要证明$F$和$L$在$K$上线性无交.设$L/K$的一组超越基为$X$,那么$L=K(X)$,那么$K[X]$是商域为$L$的环,于是归结为证明$F$和$K[X]$在$K$上线性无交.那么$F$和$K[X]$的合成域就是$F[X]$,所以我们要证的是典范同态$F\otimes_KK[X]\to F[X]$是同构,而这是因为它有逆映射$ax\mapsto a\otimes x$,其中$a\in F$.
    \end{proof}
    \item 线性无交的传递性.考虑扩张链$K\subseteq E\subseteq F$和$K\subseteq L$,设这里所有域都包含在某个域$C$中.那么如下两个命题互相等价:
    \begin{enumerate}
    	\item $F$和$L$在$K$上线性无交.
    	\item $E$和$L$在$K$上线性无交,并且$F$和$EL$在$E$上线性无交.
    \end{enumerate}
    $$\xymatrix{&FL&&\\F\ar[ur]&&EL\ar[ul]&\\&E\ar[ul]\ar[ur]&&L\ar[ul]\\&&K\ar[ul]\ar[ur]&}$$
    \begin{proof}
    	
    	典范同态$F\otimes_KL\to FL$可视为满同态的复合$F\otimes_KL=F\otimes_E(E\otimes_KL)\to F\otimes_EEL\to FL$.但是两个满同态的复合是同构当且仅当这两个满同态都是同构.
    \end{proof}
\end{enumerate}
\newpage
\subsection{导数和微分模}

导数和$k$导数.
\begin{itemize}
	\item 设$M$是$A$模,一个从$A\to M$的导数是指一个交换群同态$D$,使得$D(ab)=bD(a)+aD(b)$.全体$A\to M$的导数构成的集合记作$\mathrm{Der}(A,M)$,这是一个$A$模,它的加法是$(D+D')(a)=D(a)+D(a')$,它的数乘结构是$(aD)(b)=aD(b)$.
	\item 设$k$是环,设$A$是$k$代数,记结构映射为$f:k\to A$,一个导数$D$称为$k$导数,如果$D\circ f=0$.全部$A\to M$的$k$导数构成的集合记作$\mathrm{Der}_k(A,M)$,这是$A$模$\mathrm{Der}(A,M)$的子模.按照$1\cdot1=1$,得到$D(1)=D(1)+D(1)$,于是$D(1)=0$.这说明如果把$A$视为$\mathbb{Z}$代数,那么$\mathrm{Der}(A,M)=\mathrm{Der}_{\mathbb{Z}}(A,M)$.
	\item 对于特殊情况$M=A$,记$\mathrm{Der}_k(A)$为$\mathrm{Der}_k(A,A)$.对两个导数$D,D'$,记$[D,D']=DD'-D'D$,这使得$\mathrm{Der}_k(A)$成为一个李代数.
\end{itemize}
\begin{enumerate}
	\item 取导数$D$,任取$a\in A$,就有$D(a^n)=na^{n-1}D(a)$.特别的如果$A$是特征$p$的环,就有$D(a^p)=0$.
	\item 有莱布尼兹法则:$a,b\in A$,则$D^n(ab)=\sum_{0\le k\le n}\left(\begin{array}{c}n\\k\end{array}\right)D^k(a)D^{n-k}(b)$.如果环$A$的特征为$p$,就有$D^p(ab)=D^p(a)+D^p(b)$.于是此时$D^p$也是导数.
\end{enumerate}

设$k$是环,$A$是$k$代数,考虑$\textbf{A-Mod}$上的函子,在对象上是$M\mapsto\mathrm{Der}_k(A,M)$,在态射上把$A$模同态$f:M\to N$映射为$A$模同态$\mathrm{Der}_k(A,M)\to\mathrm{Der}_k(A,N)$为$D\mapsto f\circ D$.我们证明这是一个可表函子.
\begin{enumerate}
	\item 设$k$是环,设$B$是$k$代数,设$N$是$B$的理想,满足$N^2=0$,记环$A=B/N$,那么$N$可视为$A$模,此时有如下短正合列,我们称满足这个条件的$B,A,N$为$B$是$k$代数$A$的关于$A$模$N$的扩张.
	$$\xymatrix{0\ar[r]&N\ar[r]^i&B\ar[r]^f&A\ar[r]&0}$$
	
	称这个扩张是平凡的或者是分裂的,如果存在$k$代数同态$\varphi:A\to B$,使得$f\circ\varphi=1_A$.那么如果扩张是平凡的,就有$k$模同构$B\cong A\oplus N$.反过来给定$k$代数$A$,任取$A$的理想$N$,总可以构造一个平凡延拓如下:考虑$k$模的直和$B=A\oplus N$,在其上定义乘法为$(a,n)(a',n')=(aa',an'+a'n)$,这使得$B$构成一个环,它也会记作$A\star N$,满足$N\cong\{0\}\oplus N$是$B$的理想,并且满足$N^2=0$.
	\item 设$k$是环,设$A,B,C$是$k$代数,固定一个$k$同态$f:B\to A$,记$N=\ker f$,约定$N^2=0$.对任意$k$同态$g:C\to A$,如果提升$k$同态$h:C\to A$存在,那么$N$可以经$cn=h(c)n$定义为$C$模,并且这个定义不依赖于提升$h$的选取,因为$\mathrm{im}(h_1-h_2)\subseteq N$,而$N^2=0$.如果$h_1,h_2:C\to B$是$g$的两个提升,我们断言$h_1-h_2\in\mathrm{Der}_k(C,N)$.反过来如果$D\in\mathrm{Der}_k(C,N)$,如果$h$是$g$的一个提升,那么$D+h$也是$g$的一个提升.
	$$\xymatrix{B\ar[rr]^f&&A\\&&C\ar[u]_g\ar@{-->}[ull]^h}$$
	\begin{proof}
		
		$h_1-h_2$首先是交换群同态.另外任取$c_1,c_2\in C$,那么有:
		\begin{align*}
			c_1(h_1-h_2)(c_2)+c_2(h_1-h_2)(c_1)&=h_1(c_1)(h_1(c_2)-h_2(c_2))+h_2(c_2)(h_1(c_1)-h_2(c_1))\\&=h_1(c_1c_2)-h_2(c_1c_2)\\&=(h_1-h_2)(c_1c_2)
		\end{align*}
		
		如果$D\in\mathrm{Der}_k(C,N)$且$h$是$g$的提升,那么有$f\circ(h+D)=f\circ h+f\circ D=f\circ h=g$.
	\end{proof}
	\item 接下来为证明这个函子是可表函子,归结为找一个$A$模$M_0$和一个$\mathrm{d}\in\mathrm{Der}_k(A,M)$,使得对任意$A$模$M$和任意$D\in\mathrm{Der}_k(A,M)$,都存在唯一的$A$模同态$f:M_0\to M$,使得$D=f\circ\mathrm{d}$.考虑同态$\mu:A\otimes_kA\to A$为$x\otimes y\mapsto xy$,记$I=\ker\mu$,$\Omega_{A/k}=I/I^2$,$B=A\otimes_kA/I^2$.那么$\mu$诱导了$\mu':B\to A$,存在$\varphi:A\to B$为$a\mapsto a\otimes a+I^2$,那么$\mu'\circ\varphi=1_A$,所以如下短正合列分裂,所以$B$是$k$代数$A$关于$A$模$\Omega_{A/k}$的平凡延拓.
	$$\xymatrix{0\ar[r]&\Omega_{A/k}\ar[r]&B\ar[r]^{\mu'}&A\ar[r]&0}$$
	
	接下来考虑$\lambda_1:A\to B$为$a\mapsto a\otimes 1+I^2$和$\lambda_2:A\to B$为$a\mapsto 1\otimes a+I^2$,它们都是$1_A:A\to A$的关于$\mu'$的提升.于是$\mathrm{d}=\lambda_2-\lambda_1$是$\mathrm{Der}_k(A,\Omega_{A/k})$中的元.我们断言$M_0=\Omega_{A/k}$和$\mathrm{d}$就满足泛映射性质.
	$$\xymatrix{B\ar[rr]^{\mu'}&&A\\&&A\ar[u]_{1_A}\ar@<0.5ex>[ull]^{\lambda_1}\ar@<-0.5ex>[ull]_{\lambda_2}}$$
	\begin{proof}
		
		任取$A$模$M$和一个导数$D\in\mathrm{Der}_k(A,M)$,定义$\varphi:A\otimes_kA\to A\star M$为$\varphi(x\otimes y)=(xy,xDy)$,那么$\varphi$把$I$映入$M$,因为如果$\sum_ix_i\otimes y_i\in I$,也即$\sum_ix_iy_i=0$,那么有$\varphi(\sum_ix_i\otimes y_i)=(0,\sum_ix_iDy_i)\in M$.并且有$M^2=0$.于是$\varphi$就诱导了$f:I/I^2=\Omega_{A/k}\to M$.那么对任意$a\in A$,就有:
		\begin{align*}
			f(\mathrm{d}a)&=f(1\otimes a-a\otimes1+I^2)\\&=\varphi(1\otimes a)-\varphi(a\otimes1)\\&=Da-aD(1)=Da
		\end{align*}
		
		于是有$f\circ\mathrm{d}=D$.另外$f$是$A$模同态,这里$\Omega_{A/k}$上的$A$模结构是$a(a_1\otimes a_2+I^2)=aa_1\otimes a_2+I^2=a_1\otimes aa_2+I^2$,第二个等式是因为$(a\otimes1-1\otimes a)\in I$.于是如果$\sum_ix_i\otimes y_i\in I$,那么$f(a\sum_ix_i\otimes y_i)=f(\sum ax_i\otimes y_i)=\sum_iax_iDy_i=a\sum_ix_iDy_i=af(\sum_ix_i\otimes y_i)$.
		
		\qquad
		
		接下来我们断言$\Omega_{A/k}$被全体$\{\mathrm{d}a\mid a\in A\}$生成,这就保证了满足$D=f\circ\mathrm{d}$的模同态$f$是唯一的.按照$a\otimes a'=(a\otimes1)(1\otimes a'-a'\otimes1)+aa'\otimes1$,于是如果$\sum_ix_i\otimes y_i\in I$,那么在$\mathrm{mod}I^2$下就有$\sum_ix_i\otimes y_i\equiv\sum x_i\mathrm{d}y_i+\sum_ix_iy_i\otimes1=\sum x_i\mathrm{d}y_i$.
	\end{proof}
	\item 我们称$\Omega_{A/k}$是环$k$上的代数$A$的微分模,对$a\in A$,称$\mathrm{d}a\in\Omega_{A/k}$是$a$的微分.按照可表性,就有自然同构:
	$$\mathrm{Der}_k(A,M)\cong\mathrm{Hom}_A(\Omega_{A/k},M)$$
    \item 另外$\Omega_{A/k}$具有如下表示:它是由集合$\{\mathrm{d}a\mid a\in A\}$生成的自由$A$模,再模去由$\mathrm{d}(ab)-a\mathrm{d}b-b\mathrm{d}a$和$\mathrm{d}(a+b)-\mathrm{d}a-\mathrm{d}b$和$\mathrm{d}c,c\in k$生成的子模.
	\item 设$k$是环,如果$A$作为$k$代数被一个子集$S$生成,那么$\Omega_{A/k}$作为$A$模就被$\{\mathrm{d}s\mid s\in S\}$生成.这是因为任取$f(X)=k[X_1,\cdots,X_n]$,对$a=f(s_1,\cdots,s_n)$,其中$s_i\in S$,就有$\mathrm{d}a=\sum_{i=1}^nf_i(s_1,\cdots,s_n)\mathrm{d}s_i$.特别的,如果$A=k[X_1,\cdots,X_n]$,那么$\Omega_{A/k}=A\mathrm{d}X_1+\cdots+A\mathrm{d}X_n$,并且$\mathrm{d}X_1,\cdots,\mathrm{d}X_n$是线性无关的,因为它们对应于$\mathrm{Der}_k(A)$中的导数$D_1,\cdots,D_n$,满足$D_iX_j=\delta_{ij}$,而这些导数是线性无关的.
	\item 基变换.设$B$和$A'$都是$A$代数,记基变换$B'=B\otimes_AA'$,设典范导数$\mathrm{d}:B\to\Omega_{B/A}$,那么$\mathrm{d}\otimes\mathrm{id}_{A'}:B'=B\otimes_AA'\to\Omega_{B/A}\otimes_AA'$满足微分模的泛性质,于是我们有典范同构$\Omega_{B'/A'}\cong\Omega_{B/A}\otimes_BB'=\Omega_{B/A}\otimes_AA'$.
\end{enumerate}

光滑,非分歧和平展.设$k$是环,设$A$是$k$代数.
\begin{itemize}
	\item 称$k$代数$A$是0-光滑(0-smooth)的,如果对任意$k$代数$C$,对任意$C$的理想$N$使得$N^2=0$,对任意$k$同态$u:A\to C/N$使得下面实线图表交换,那么存在提升$k$同态$v:A\to C$使得图表交换.
	\item 称$k$代数$A$是0-非分歧(0-unramified)的,如果对任意$k$代数$C$,对任意$C$的理想$N$使得$N^2=0$,对任意$k$同态$u:A\to C/N$使得下面实线图表交换,那么至多存在一个(这包含了不存在的情况)提升$k$同态$v:A\to C$使得图表交换.
	\item 称$k$代数$A$是0-平展(0-etale)的,如果它同时是0-光滑和0-非分歧的.换句话讲对任意$k$代数$C$,对任意$C$的理想$N$使得$N^2=0$,对任意$k$同态$u:A\to C/N$使得下面实线图表交换,那么恰好存在一个提升$k$同态$v:A\to C$使得图表交换.
\end{itemize}
$$\xymatrix{A\ar[rr]^u\ar@{-->}[drr]_v&&C/N\\k\ar[u]\ar[rr]&&C\ar[u]}$$
\begin{enumerate}
	\item $k$代数$A$是0-非分歧的当且仅当$\Omega_{A/k}=0$.
	\begin{proof}
		
		充分性是因为,如果存在$C,N$使得上面实线图表交换,并且存在两个不同的提升$k$同构$g_1,g_2$,那么$0\not=g_1-g_2\in\mathrm{Der}_k(A,N)\cong\mathrm{Hom}_A(\Omega_{A/k},N)$,于是$\Omega_{A/k}\not=0$.必要性我们取$I$是$\mu:A\otimes_kA\to A$,$a_1\otimes a_2\mapsto a_1a_2$的核,取$C=A\otimes_kA/I^2$,取$N=I/I^2$是$C$的理想,那么满足$N^2=0$,如果取$u:A\to C/N$为$a\mapsto a\otimes1+I=1\otimes a+I$.那么$A\to C$的两个映射$\lambda_1:a\mapsto a\otimes1+I^2$和$\lambda_2:1\otimes a+I^2$都是$u$的提升,于是0-非分歧条件保证了$\mathrm{d}=\lambda_1-\lambda_2=0$,但是$\Omega_{A/k}$是被$\mathrm{d}a$生成的,就导致$\Omega_{A/k}=0$.
	\end{proof}
	\item 如果$A$是环,$S\subseteq A$是乘性闭子集,那么$S^{-1}A$作为$A$代数(是指结构同态取典范的$A\to S^{-1}A$)是0-平展的.
	\begin{proof}
		
		取$A$代数$C$和$C$理想$N$使得$N^2=0$,我们要证明存在唯一的$k$同态$v$使得如下图表交换:
		$$\xymatrix{S^{-1}A\ar[rr]^u\ar@{-->}[drr]^v&&C/N\\A\ar[rr]_f\ar[u]_i&&C\ar[u]}$$
		
		我们断言$C/N$中的单位在$C$中的提升还是单位,假设$a\in C$使得$\overline{a}\in C/N$是单位,此即存在$b\in C$使得$ab-1=x\in N$,于是$(ab-1)^2=0$,于是$a(ab^2-2b)=1$,也即$a\in C$是单位.回到原题,对任意$s\in S$,有$u(i(s))$是$C/N$中的单位,于是它的提升$f(s)$就是$C$中的单位.直接构造$v:S^{-1}A$为$a/s\mapsto f(a)f(s)^{-1}$,并且它是唯一的使得图表交换的同态.
	\end{proof}
	\item 引理.给定$B$模同态链$\xymatrix{N_1\ar[r]^{\alpha}&N_2\ar[r]^{\beta}&N_3}$,如果对任意$B$模$T$都有正合列:
	$$\xymatrix{\mathrm{Hom}_B(N_3,T)\ar[r]^{\beta^*}&\mathrm{Hom}_B(N_2,T)\ar[r]^{\alpha^*}&\mathrm{Hom}_B(N_1,T)}$$
	
	那么$\xymatrix{N_1\ar[r]^{\alpha}&N_2\ar[r]^{\beta}&N_3}$也是正合列.如果是对$B$模的短正合列:
	$$\xymatrix{0\ar[r]&N_1\ar[r]^{\alpha}&N_2\ar[r]^{\beta}&N_3\ar[r]&0}$$
	
	满足对任意$B$模$T$都有:
	$$\xymatrix{0\ar[r]&\mathrm{Hom}_B(N_3,T)\ar[r]^{\beta^*}&\mathrm{Hom}_B(N_2,T)\ar[r]^{\alpha^*}&\mathrm{Hom}_B(N_1,T)\ar[r]&0}$$
	
	那么$\xymatrix{0\ar[r]&N_1\ar[r]&N_2\ar[r]&N_3\ar[r]&0}$是一个分裂短正合列.这是因为取$T=N_1$,那么存在$B$模同态$\alpha':N_2\to N_1$使得$\alpha'\circ\alpha=1_{N_1}$,于是短正合列分裂.
	\item 第一基本正合列.设有环同态的链$\xymatrix{k\ar[r]^f&A\ar[r]^g&B}$,它诱导了如下$B$模的正合列,其中$\alpha:\mathrm{d}_{A/k}a\otimes b=b\mathrm{d}_{B/k}g(a)$和$\beta:\mathrm{d}_{B/k}b\mapsto\mathrm{d}_{B/A}b$.
	$$\xymatrix{\Omega_{A/k}\otimes_AB\ar[r]^{\alpha}&\Omega_{B/k}\ar[r]^{\beta}&\Omega_{B/A}\ar[r]&0}$$
	
	如果额外的还有$B$在$A$上0-光滑,那么有分裂短正合列:
	$$\xymatrix{0\ar[r]&\Omega_{A/k}\otimes_AB\ar[r]^{\alpha}&\Omega_{B/k}\ar[r]^{\beta}&\Omega_{B/A}\ar[r]&0}$$
	\begin{proof}
		
		我们有:
		\begin{align*}
			\mathrm{Hom}_B(\Omega_{A/k}\otimes_AB,T)&=\mathrm{Hom}_A(\Omega_{A/k},\mathrm{Hom}_B(B,T))\\&=\mathrm{Hom}_A(\Omega_{A/k},T)
		\end{align*}
		
		按照引理,我们要证明的正合列归结为证明对任意$B$模$T$有:
		$$\xymatrix{0\ar[r]&\mathrm{Hom}_B(\Omega_{B/A},T)\ar[r]^{\beta^*}&\mathrm{Hom}_B(\Omega_{B/k},T)\ar[r]^{\alpha^*}&\mathrm{Hom}_A(\Omega_{A/k},T)}$$
		
		按照微分模的可表性,等价于证明对任意$B$模$T$有:
		$$\xymatrix{0\ar[r]&\mathrm{Der}_A(B,T)\ar[r]^{\beta_0}&\mathrm{Der}_k(B,T)\ar[r]^{\alpha_0}&\mathrm{Der}_k(A,T)}$$
		
		我们来描述$\alpha_0$和$\beta_0$:
		\begin{itemize}
			\item $\beta:\Omega_{B/k}\to\Omega_{B/A}$就是$\mathrm{d}_{B/k}b\mapsto\mathrm{d}_{B/A}b$,它诱导的$\beta^*:\mathrm{Hom}_B(\Omega_{B/A},T)\to\mathrm{Hom}_B(\Omega_{B/k},T)$就是$\varphi\mapsto\varphi\circ\beta$,于是它对应的$\beta_0$可以这样描述,任取$D=\varphi\circ\mathrm{d}_{B/A}\in\mathrm{Der}_A(B,T)$,那么$\beta_0(D)=\varphi\circ\beta\circ\mathrm{d}_{B/k}$,换句话讲$\beta_0$就是把$B\to T$的$A$导数视为$B\to T$的$k$导数.本身作为映射是不变的.
			\item 任取$B$模同态$\varphi:\Omega_{B/k}\to T$,那么$\alpha^*(\varphi)$视为$\mathrm{Hom}_B(\Omega_{A/k}\otimes_AB,T)$中的元是$\varphi\circ\alpha:\mathrm{d}_{A/k}a\otimes b\mapsto\varphi(b\mathrm{d}_{B/k}g(a))$.如果把$\alpha^*(\varphi)$视为$\mathrm{Hom}_A(\Omega_{A/k},T)$中的元,则是$\varphi':\mathrm{d}_{A/k}a\mapsto\varphi(\mathrm{d}_{B/k}g(a))$.所以如果$D=\varphi\circ\mathrm{d}_{B/k}\in\mathrm{Der}_k(B,T)$,那么$\alpha_0(D)=\varphi'\circ\mathrm{d}_{A/k}$就是$a\mapsto\varphi(\mathrm{d}_{B/k}g(a))$,也即$\alpha_0(D)=D\circ g$.
		\end{itemize}
		
		下面验证正合性:
		\begin{itemize}
			\item $\ker\beta_0=0$:$\beta_0$就是把$B\to T$的$A$导数视为$B\to T$的$k$导数,这明显是单射.
			\item $\mathrm{im}\beta_0\subseteq\ker\alpha_0$,也即$\alpha_0\circ\beta_0=0$.因为一个$B\to T$的$A$导数$D$按照定义就有$D\circ g=0$.
			\item $\ker\alpha_0\subseteq\mathrm{im}\beta_0$.如果$D\in\mathrm{Der}_k(B,T)$满足$D\circ g=0$,那么明显$D\in\mathrm{Der}_A(B,T)$.
		\end{itemize}
		
		这证明了定理的前半段.下面设$B$在$A$上0-光滑,我们要证明的是$\alpha_0$是满射.任取$D\in\mathrm{Der}_k(A,T)$,取$\varphi:A\to B\star T$是$a\mapsto(g(a),Da)$,我们有如下实线的交换图表,于是按照0-光滑条件,就存在提升$h:B\to B\star T$使得图表交换.
		$$\xymatrix{B\ar[rr]^{1_B}\ar@{-->}[drr]^h&&B\\A\ar[u]^g\ar[rr]_{\varphi}&&B\star T\ar[u]}$$
		
		可记$h(b)=(b,D'b)$,那么从$h(b_1b_2)=h(b_1)h(b_2)$得到$D'$是$B\to T$的导数.并且有$D=D'\circ g$.这得到$\alpha_0$是满射.最后我们证明了$\mathrm{Hom}$函子保这个短正合列,所以这个短正合列是分裂的.
	\end{proof}
	\item 推论.如果$\xymatrix{k\ar[r]^f&A\ar[r]^g&B}$是环同态的链,如果$g$使得$B$是$A$的商(也即$g$是满同态),或者$g$使得$B$是$A$的分式化,那么上一条中的$\alpha$都是满同态,就导致$\Omega_{B/A}=0$.直接验证微分模为零也不麻烦:如果$B$是$A$的商,那么$\mathrm{d}(b)=a\mathrm{d}(1)=0$;如果$B=S^{-1}A$,对任意$b\in B$,有$s\in S$使得$sb\in A$,于是$t\mathrm{d}(b)=\mathrm{d}(tb)=0$,但是$t$在$B$里是可逆的,就得到$\mathrm{d}(b)=0$.
	\item 推论.我们解释过分式化是0-平展的,用在第一基本正合列上,取$B=S^{-1}A$,就得到典范同构$S^{-1}\Omega_{A/k}\cong\Omega_{S^{-1}A/k}$.
	\item 第二基本正合列.设$\xymatrix{k\ar[r]^f&A\ar[r]^g&B}$是环同态的链,设$g$是满同态,如果记$\mathfrak{m}=\ker g$,那么$B=A/\mathfrak{m}$.此时有如下$B$模同态的正合列,其中$\delta$为$x+\mathfrak{m}^2\mapsto\mathrm{d}_{A/k}x\otimes1$.而$\alpha$定义同上.
	\begin{equation}\label{1}
		\xymatrix{\mathfrak{m}/\mathfrak{m}^2\ar[r]^{\delta}&\Omega_{A/k}\otimes_AB\ar[r]^{\alpha}&\Omega_{B/k}\ar[r]&0}
	\end{equation}
	
	如果额外的$B$在$k$上是0-光滑的,那么有分裂的短正合列:
	\begin{equation}\label{2}
		\xymatrix{0\ar[r]&\mathfrak{m}/\mathfrak{m}^2\ar[r]^{\delta}&\Omega_{A/k}\otimes_AB\ar[r]^{\alpha}&\Omega_{B/k}\ar[r]&0}
	\end{equation}
	\begin{proof}
		
		我们已经解释了条件下有$\alpha$是满射.按照引理,为证明$\ref{1}$,只需证明对任意$B$模$T$都有正合列:
		$$\xymatrix{\mathrm{Hom}_B(\Omega_{B/k},T)\ar[r]^{\alpha^*}&\mathrm{Hom}_A(\Omega_{A/k},T)\ar[r]^{\delta^*}&\mathrm{Hom}_B(\mathfrak{m}/\mathfrak{m}^2,T)}$$
		
		按照微分模的可表性,等价于证明有正合列:
		$$\xymatrix{\mathrm{Der}_k(B,T)\ar[r]^{\alpha_0}&\mathrm{Der}_k(A,T)\ar[r]^{\delta_0}&\mathrm{Hom}_B(\mathfrak{m}/\mathfrak{m}^2,T)}$$
		
		描述$\delta_0$:首先$\delta^*$是把$A$模同态$\varphi:\Omega_{A/k}\to T$映射为$\varphi':\mathfrak{m}/\mathfrak{m}^2\to T$,$x+\mathfrak{m}^2\mapsto\varphi(\mathrm{d}_{A/k}x)$.于是如果$D=\varphi\circ\mathrm{d}_{A/k}\in\mathrm{Der}_k(A,T)$,那么$\delta_0(D)$是同态$x+\mathfrak{m}^2\mapsto\varphi(\mathrm{d}_{A/k}x)$,换句话讲$\delta_0(D):\mathfrak{m}/\mathfrak{m}^2\to T$就是$D:A\to T$诱导的商同态.
		
		验证正合性:
		\begin{itemize}
			\item 如果$D\in\mathrm{Der}_k(B,T)$,$\alpha_0(D)=D\circ g$,那么$D\circ g(\mathfrak{m})=0$,于是诱导的$\delta_0(\alpha_0(D))=0$.
			\item 如果$D\in\mathrm{Der}_k(A,T)$满足$\delta_0(D)=0$,此即$D(\mathfrak{m})=0$,于是$D$也可以视为$A/\mathfrak{m}=B\to T$的导数.
		\end{itemize}
		
		接下来设$B$在$k$上0-光滑,我们来证明$\delta_0$是满射.首先我们断言$0\to\mathfrak{m}/\mathfrak{m}^2\to A/\mathfrak{m}^2\to B\to0$是分裂短正合列.考虑如下实线交换图表,0-光滑保证了存在同态$s$使得如下图表交换:
		$$\xymatrix{B\ar[rr]^{1_B}\ar@{-->}[drr]^s&&B=A/\mathfrak{m}\\k\ar[u]\ar[rr]&&A/\mathfrak{m}^2\ar[u]_g}$$
		
		于是$gs=1_B$,于是$g(1_{A/\mathfrak{m}^2}-sg)=0$,但是按照$\ker g$是$A/\mathfrak{m}^2$的平方为零的理想,并且$1_{A/\mathfrak{m}^2}$和$sg$都是$g$的提升,于是$D=1-sg$是$A/\mathfrak{m}^2\to\mathfrak{m}/\mathfrak{m}^2$的导数.
		$$\xymatrix{A/\mathfrak{m}^2\ar[rr]^g&&A/\mathfrak{m}\\&&A/\mathfrak{m}^2\ar[u]_g\ar@<0.5ex>[ull]^{sg}\ar@<-0.5ex>[ull]_1}$$
		
		接下来任取$B$模同态$\psi:\mathfrak{m}/\mathfrak{m}^2\to T$.考虑复合$\xymatrix{A\ar[r]&A/\mathfrak{m}^2\ar[r]^D&\mathfrak{m}/\mathfrak{m}^2\ar[r]^{\psi}&T}$,记作$D'$,这是$\mathrm{Der}_k(A,T)$中的导数.我们断言有$\delta^*(D')=\psi$,因为$\delta^*D'(\overline{x})=D'(x)=\psi(D(\overline{x}))=\psi(\overline{x}-sg(\overline{x}))=\psi(\overline{x})$.
	\end{proof}
    \item 推论.设$A$是环,设$B$是有限型$A$代数,或者这样代数的分式化,那么$\Omega_{B/A}$是有限$B$模.
    \item 推论.设局部环$(A,\mathfrak{m},k)$存在系数域(这是指$A$包含了一个子域$k'$,并且它在$A\to A/\mathfrak{m}=k$下是同构),那么我们有$\delta:\mathfrak{m}/\mathfrak{m}^2\to\Omega_{A/k}\otimes_AA/\mathfrak{m}$是同构.
    \begin{proof}
    	
    	根据第二基本正合列直接得证.单射也可以直接证:归结为证明对偶映射$\delta^*:\mathrm{Hom}_k(\Omega_{A/k}\otimes_AA/\mathfrak{m},k)\to\mathrm{Hom}_k(\mathfrak{m}/\mathfrak{m}^2,k)$是满射.也即$\mathrm{Der}_k(A,k)\to\mathrm{Hom}_k(\mathfrak{m}/\mathfrak{m}^2,k)$是满射.设$\varphi:\mathfrak{m}/\mathfrak{m}^2\to k$是$k$线性变换,任取$x\in A$,它可以唯一的分解为$x=x_1+x_2$,其中$x_1\in k$,$x_2\in\mathfrak{m}$,我们定义$D(x)=\varphi(x_2)$,那么$D:B\to k$是一个$k$导数,并且满足$\delta^*(D)=\varphi$.
    \end{proof}
	\item 例子,取$k$是环,$A=k[X_1,\cdots,X_n]$,$B=k[X_1,\cdots,X_n]/(f_1,\cdots,f_m)$.那么结构同态$A\to B$是满的,按照第二基本正合列,就有$\Omega_{B/k}=\Omega_{A/k}\otimes_AB/\mathrm{im}\delta$.我们解释过$\Omega_{A/k}$是以$\mathrm{d}X_1,\cdots,\mathrm{d}X_n$为基的自由$A$模,于是$\Omega_{A/k}\otimes_AB$是以$\mathrm{d}X_1,\cdots,\mathrm{d}X_n$为基的自由$B$模.而$\mathrm{im}\delta$是被$\mathrm{d}f_i=\sum_j(\partial f_i/\partial X_j)\mathrm{d}X_j,1\le i\le m$生成的$B$子模.例如取$k$是域,取$B=k[X,Y]/(X^2+Y^2)$,如果$k$的特征非2,那么$\Omega_{B/k}$是$F/R$,其中$F$是$\mathrm{d}X$和$\mathrm{d}Y$生成的自由$B$模,$R$是由$X\mathrm{d}X+Y\mathrm{d}Y$生成的$B$子模;如果$k$的特征为2,那么$\Omega_{B/k}$是以$\mathrm{d}X$和$\mathrm{d}Y$为基的自由$B$模.
	\item 设$B_1,B_2$是两个$A$代数,设$R=B_1\otimes_AB_2$,那么有如下典范同构:
	$$\varphi:(\Omega_{B_1/A}\otimes_{B_1}R)\oplus(\Omega_{B_2/A}\otimes_{B_2}R)\cong\Omega_{R/A}$$
	$$(\mathrm{d}b_1\otimes r_1)+(\mathrm{d}b_2\otimes r_2)\mapsto r_1\mathrm{d}(b_1\otimes1)+r_2\mathrm{d}(1\otimes b_2)$$
	\begin{proof}
		
		考虑$A\to B_1\to R$,按照基变换有$\Omega_{R/B_1}\cong\Omega_{B_2/A}\otimes_{B_2}R$,于是第一基本正合列得到:
		$$\xymatrix{\Omega_{B_1/A}\otimes_{B_1}R\ar[r]^{\varphi_1}&\Omega_{R/A}\ar[r]^{\psi_2}&\Omega_{B_2/A}\otimes_{B_2}R\ar[r]&0}$$
		
		同理有如下第一基本正合列:
		$$\xymatrix{\Omega_{B_2/A}\otimes_{B_1}R\ar[r]^{\varphi_2}&\Omega_{R/A}\ar[r]^{\psi_1}&\Omega_{B_1/A}\otimes_{B_1}R\ar[r]&0}$$
		
		验证$\varphi_i$是$\psi_i$的截面.得到$\varphi=\varphi_1\oplus\varphi_2$是同构.
	\end{proof}
	\item 设$L/K$是域的可分代数扩张,那么$L$在$K$上0-平展.另外如果$k$是$K$的子域,那么$\Omega_{L/k}=\Omega_{K/k}\otimes_KL$.
	\begin{proof}
		
		取$K$代数$C$,取$C$的理想$N$使得$N^2=0$,取同态$u:L\to C/N$,我们要证明存在唯一的提升同态$v:L\to C$使得图表交换:
		$$\xymatrix{L\ar[rr]^u\ar@{-->}[drr]^v&&C/N\\K\ar[u]\ar[rr]&&C\ar[u]}$$
		
		只需证明对$L/K$的任意中间域$L'$,使得$L'/K$是有限扩张,存在唯一的提升$v$,唯一性就保证了这些有限生成子扩张上的提升同态可以粘合成唯一的提升$v:L\to C$.换句话讲我们可设$L/K$本身是有限扩张.按照本原元定理,就有$L=K(\alpha)$,记$\alpha$在$K$上的极小多项式为$f(X)$,可分性保证$f'(\alpha)\not=0$.寻找唯一的提升同态等价于确定唯一的一个元$y\in C$作为$v(\alpha)$,它只需要满足$f(y)=0$和$y-u(\alpha)\in N$.先取$y'$是$u(\alpha)\in C/N$在$C$中的一个提升,那么在$C/N$中有$\overline{f(y')}=f(u(\alpha))=u(f(\alpha))=0$,于是$f(y')\in N$.下面任取$n\in N$,按照$N^2=0$以及$f$是多项式,就有$f(y'+n)=f(y')+f'(y')n$.由于$u(f'(\alpha))=f'(u(\alpha))=\overline{f'(y')}$是$C/N$中的单位,导致$f'(y')$也是$C$中的单位(这件事是因为,如果$a\in C$满足$\overline{a}\in C/N$是单位,此即存在$b\in C$使得$ab-1\in N$,但是$N^2=0$,就有$(ab-1)^2=0$,于是$a(ab^2-2b)=1$,于是$a$是$C$中单位).我们取$\eta=-f(y')/f'(y')\in N$,那么$y=y'+\eta$也是$u(\alpha)$的提升,并且$f(y)=0$.这个$y$必然是唯一的,如果有$y_1,y_2$都是$u(\alpha)$的提升,也即$y_1-y_2\in N$,并且$f(y_1)=f(y_2)$,那么有$f(y_1)=f(y_2+(y_1-y_2))=f(y_2)+f'(y_2)(y_1-y_2)$,导致$f'(y_2)(y_1-y_2)=0$,迫使$y_1=y_2$.这证明了$v$是唯一的.
		
		\qquad
		
		最后按照$L$在$K$上0-平展.从0-非分歧得到$\Omega_{L/K}=0$,从0-光滑,按照第一基本正合列就得到$\Omega_{K/k}\otimes_KL=\Omega_{L/k}$.
	\end{proof}
\end{enumerate}

Hochschild公式.
\begin{enumerate}
	\item 设$K$是域,设$0\not=D\in\mathrm{Der}(K)$.
	\begin{enumerate}
		\item 如果$K$的特征是$p>0$,那么映射$1,D,\cdots,D^{p-1}$在$K$上线性无关.
		\item 如果$K$的特征是0,那么对任意$n$有$1,D,\cdots,D^n$在$K$上线性无关.
		\item 如果$K$的特征是$p>0$,如果$c_0+c_1D+\cdots+c_{p-1}D^{p-1}$是$K\to K$的导数,那么只能有$c_0=c_2=\cdots=c_{p-1}=0$.
	\end{enumerate}
	\begin{proof}
		
		对$a\in K$,我们用$a_L$表示$K\to K$的左乘$a$的同态.从$D(ax)=D(a)x+aDx$得到$D\circ a_L=D(a)_L+aD$.于是按照莱布尼兹法则有:
		\begin{equation}\label{Hochschild.1}
			D^i\circ a_L=aD^i+iD(a)D^{i-1}+\left(\begin{array}{c}i\\2\end{array}\right)D^2(a)D^{i-2}+\cdots+D^i(a)_L
		\end{equation}
		
		证明(a)和(b).假设有自然数$i$满足$1,D,\cdots,D^{i-1}$是线性无关的,而$1,D,\cdots,D^i$是线性相关的.在$\mathrm{char}K=p>0$时我们额外要求$i<p$.那么存在$c_j\in K$满足如下等式(这用到了$K$是域)
		\begin{equation}\label{Hochschild.2}
			D^i=c_{i-1}D^{i-1}+\cdots+c_1D+c_0
		\end{equation}
		
		由于$D\not=0$,可取$a\in K$使得$D(a)\not=0$.对$\ref{Hochschild.2}$右复合$a_L$,结合$\ref{Hochschild.1}$,就得到:
		$$\left(ac_{i-1}+iD(a)\right)D^{i-1}+\left(\text{复合次数}\le i-2\text{的项的线性组合}\right)$$
		$$=ac_{i-1}D^{i-1}+\left(\text{复合次数}\le i-2\text{的项的线性组合}\right)$$
		
		线性无关性就迫使$iD(a)D^{i-1}=0$.如果$\mathrm{char}K=p>0$,我们要求了$i<p$,于是$i$是$K$中的非零元.但是这导致$D^{i-1}$本身是线性相关的,这就和$i$的选取相矛盾.
		
		\qquad
		
		证明(c):设$E=c_iD^i+\cdots+c_1D+c_0$是$K$的一个导数,其中$i<p$并且$c_i\not=0$.首先有$0=E(1)=c_0$.假设$i>1$,选取$a\in K$使得$D(a)\not=0$.右复合$a_L$得到$E\circ a_L=c_iD^i\circ a_L+\cdots$.左侧就是$aE+E(a)_L$,右侧带入每个$D^j\circ a_L=aD^j+jD(a)D^{j-1}+\cdots$,得到:
		$$ac_iD^i+ac_{i-1}D^{i-1}+\left(\deg\le i-2\right)=ac_iD^i+\left(ic_iD(a)+ac_{i-1}\right)D^{i-1}+\left(\deg\le i-2\right)$$
		
		但是按照$1,D,\cdots,D^{p-1}$是线性无关的,就有$ic_iD(a)=0$,但是这矛盾.
	\end{proof}
	\item 如果$K$不是域,上一条就未必成立.例如取$k$是特征$p$的域,取$A=k[x]=k[X]/(X^p)$,如果$D$是$k[X]$上的导数,那么$D(aX^p)=X^pD(a)+aD(X^p)=X^pD(a)$,换句话讲$D$总把$(X^p)$映入$X^p$,于是$D$总诱导了$A=k[X]/(X^p)$上的导数.我们取$k[X]$上的导数$X^{p-1}\partial/\partial X$,它诱导的$A$上的导数就是$D(x)=x^{p-1}$.如果$i>1$就有$D(x^i)=ix^{i-1}D(x)=0$,于是$D^2=0$.于是$p>2$时这是(a)的反例.
	\item Hochschild公式.设$A$是特征$p$的环,设$a\in A$和$D\in\mathrm{Der}(A)$,那么有:
	$$(aD)^p=a^pD^p+(aD)^{p-1}(a)D$$
	
	特别的$(aD)^p$总是$D^p$和$D$的线性组合.
	\begin{proof}
		
		记$E=aD=a_L\circ D$,那么$E^2=E\circ a_L\circ D=(aE+E(a))D=a^2D^2+E(a)D$.归纳可得$E^k=a^kD^k+\sum_{i=2}^{k-1}b_{k,i}D^i+E^{k-1}(a)D$.其中$b_{k,i}$是$a,D(a),\cdots,D^{k-i}$的$\mathbb{Z}/(p)$系数多项式,把多项式记作$f_{k,i}$.这个多项式不依赖于$A,a,D$.问题归结为证明$f_{p,i}=0,\forall 1<i<p$.所以我们可以选取特例来计算$f_{p,i}$.设$k$是一个特征$p$的域,设$x_1,x_2,\cdots$是可数个未定元,记$K=k(x_1,x_2,\cdots)$,定义$K$上的$k$导数$D$为$Dx_i=x_{i+1},\forall i\ge1$.取$a=x_1$,取$E=x_1D$.那么$E^p-x_1^pD^p=\sum_{i=2}^{p-1}b_{p,i}D^i+E^{p-1}(a)D$是一个导数,但是按照前面定理就有每个$b_{p,i}=f_{p,i}(x_1,x_2,\cdots,x_{p-i+1})=0,1<i<p$,于是$f_{p,i}=0$.
	\end{proof}
\end{enumerate}
\newpage
\subsection{可分扩张}

可分性.设$k$是域,设$A$是$k$代数,称$A$在$k$上可分,如果对$k$的任意扩张$k'$,都有$A'=A\otimes_kk'$是既约环,此即没有非平凡幂零元.如果$K/k$是域扩张,使得$K$在$k$上可分,我们就称$K/k$是可分扩张.
\begin{enumerate}
	\item 一些基本性质:
	\begin{enumerate}
		\item 如果$A$在$k$上可分,取$k'=k$,说明$A$本身得是既约环.
		\item 一个可分$k$代数的子代数仍然是可分的.
		\item 一个$k$代数是可分的当且仅当它的每个有限生成$k$子代数都是可分的.
		\item 如果$k$代数$A$是可分的,对任意$k$的域扩张$k'$,有$k'$代数$A\otimes_kk'$也是可分的.
	\end{enumerate}
	\item 设$k$代数$A$作为模是有限生成的,选取一组基$\{\omega_1,\cdots,\omega_n\}$,考虑判别式$d=\det\left(\mathrm{tr}_{A/k}(\omega_i\omega_j)\right)$,这是$k$中的元,因为$A$中元素的迹是$k$中的元.再选一组$k$模$A$的基$\{\omega_1',\cdots,\omega_n'\}$,如果记$\omega_i'=\sum_jc_{ij}\omega_j$,其中$c_{ij}\in k$,记$d'=\det\left(\mathrm{tr}_{A/k}(\omega_i'\omega_j')\right)$,那么$d'=\det(c_{ij})^2d$,所以判别式是否为零是不依赖于基的选取的.我们断言$A$在$k$上可分当且仅当判别式$d\not=0$.
	\begin{proof}
		
		先设$d\not=0$.设$k\subseteq k'$是域扩张,设$A'=A\otimes_kk'$,那么$\omega_1,\cdots,\omega_n$也是$A'$作为$k'$模的一组基.所以$d$也是$A'$在$k'$上的判别式.倘若$N=\mathrm{nil}(A')$不是零,那可以选取$N$的一组基$\omega_1',\cdots,\omega_r'$使得$r\ge1$,把它扩充为$A'$在$k'$上的一组基$\omega_1',\cdots,\omega_n'$,那么对$i\le r$,对任意$1\le j\le n$,总有$\omega_i'\omega_j'$是幂零元,导致$\mathrm{tr}_{A/k}(\omega_i'\omega_j')=0$,于是$d'=0$,这和$d=0$矛盾.
		
		\qquad
		
		反过来设$A$在$k$上可分,设$K$是$k$的代数闭包,那么$A\otimes_kK$是既约环.如果取$A$的一组基,取判别式为$d$,那么$d$也是$A\otimes_kK$在$K$上的一个判别式,于是为证明判别式不为零,归结为设$k$本身是代数闭域并且$A$是既约环.由于$A$作为$k$模是有限生成的,说明$A$是阿廷环,它是有限个阿廷局部环的直积,但是既约条件导致这些阿廷局部环都是域,这些域得是$k$的有限扩张,但是$k$已经是代数闭的,所以$A$就是有限个$k$的直积.于是$A=ke_1+\cdots+ke_n$,满足$e_i^2=e_i$和$e_ie_j=0,\forall i\not=j$.那么基$\{e_1,\cdots,e_n\}$的判别式是$1\not=0$.
	\end{proof}
	\item 如果$F/k$是代数扩张,那么$F$在$k$上可分等价于我们平常的代数可分扩张的定义.换句话讲如果$F/k$是代数扩张,那么如下命题互相等价.今后我们提及可分扩张就不再有歧义.
	\begin{enumerate}
		\item 对$k$的任意域扩张$k'$,都有$F\otimes_kk'$是既约环.
		\item $F$中的每个元都是$k$上的可分代数元.
	\end{enumerate}
	\begin{proof}
		
		(a)$\Rightarrow$(b):任取$\alpha\in F$,设它在$k$上的极小多项式为$p(X)$.取$k'=k(\alpha)=k[X]/(p(X))$,那么$F\otimes_kk'=F[X]/(p(X))$是既约环,那么$p(X)$在$F$中做唯一分解后$(X-\alpha)$的次数只能是1次,否则就和既约矛盾.这说明$\alpha$是可分元(一个不可约多项式在代数闭包中的分解应该是可分维数个不同的根,每个根的重数是纯不可分维数,所以$\alpha$的重数是1保证了$p(X)$是可分多项式).
		
		\qquad
		
		(b)$\Rightarrow$(a):我们解释过归结为证$F$的每个有限生成$k$子代数都在$k$上可分.但是每个有限生成$k$子代数肯定包含在$F$的某个有限生成$k$子域中,所以归结为证$F$的每个有限生成$k$子域都在$k$上可分,也即归结为设$F/k$本身是(常义的)有限可分扩张.那么按照本原元定理,就有$\alpha\in F$使得$F=k(\alpha)$,设$\alpha$在$k$上的极小多项式为$p(X)$,那么这是一个可分多项式.对$k$的任意域扩张$k'$,我们有$F\otimes_kk'=k[X]/(p(X))\otimes_kk'=k'[X]/(p(X))$.由于$p(X)$是可分多项式,它在每个扩域$k'$中也是可分的,即做唯一分解后不可约因式的次数都是1,设$p(X)$在$k'$中分解为$p_1(X)\cdots p_r(X)$,其中每个$p_i(X)$都是$k'$中的不可约多项式,并且$p_i(X)$两两不同.于是有如下分解,这是一个既约环,于是$F$在$k$上可分.
		$$k'[X]/(p(X))\cong k'[X]/(p_1(X))\times k'[X]/(p_2(X))\times\cdots\times k'[X]/(p_r(X))$$
	\end{proof}
	\item 一个域扩张$F/K$称为可分生成扩张,或者称具有可分超越基,如果存在扩张的超越基$X$,使得$F$在$K(X)$上是代数可分扩张.此时称$X$是可分超越基.几个简单的例子:
	\begin{itemize}
		\item 平凡的,一个代数扩张是可分生成扩张当且仅当它是代数可分扩张.
		\item 一个特征零的域上扩张的超越基总是可分超越基,所以特征零的域扩张总是可分生成扩张,按照下一条也总是可分扩张.
		\item 设$k$是特征$p>0$的域,那么$\{X\}$和$\{X^p\}$都是$k\subseteq k(X)$的超越基,但是$\{X\}$是可分超越基,$\{X^p\}$则不是,因为$k(X^p)\subseteq k(X)$是纯不可分扩张.
		\item 我们解释过一个域扩张是可分扩张等价于它所有有限生成子扩张是可分的.但是对于可分生成扩张这不成立:设$k$是特征$p$的域,设$X$是未定元,取$K=k(x,x^{1/p},x^{1/p^2},\cdots)$,它的超越维数是1,它不是可分生成扩张,但它每个有限生成子扩张都是可分生成扩张.于是它也是可分扩张但不是可分生成扩张的例子.
	\end{itemize}
	\item 可分生成扩张一定是可分扩张.
	\begin{proof}
		
		设$K/k$是可分生成扩张,记可分超越基为$X$.任取$k$的域扩张$k'$,那么$k(X)\otimes_kk'$是$k[X]\otimes_kk'=k'[X]$的分式域,于是$k(X)\otimes_kk'$是整环,并且商域是$k'(X)$.于是$K\otimes_kk'=K\otimes_{k(X)}(k(X)\otimes_kk')$是$K\otimes_{k(X)}k'(X)$的子环.由于$k(X)\subseteq K$是可分代数扩张,就导致$K\otimes_{k(X)}k'(X)$是既约环,于是它的子环$K\otimes_kk'$也是既约环.
	\end{proof}
	\item 设$k$是特征$p>0$的域,设$K/k$是有限生成域扩张,那么如下命题互相等价:
	\begin{enumerate}
		\item $K/k$是可分扩张.
		\item $K\otimes_kk^{1/p}$是既约的.这里$k^{1/p}$是固定$k$的一个代数闭包$\overline{k}$后取$k^{1/p}=\{x\in\overline{k}\mid x^p\in k\}$.
		\item $K/k$是可分生成扩张.
	\end{enumerate}
	
	如果$K=k(x_1,\cdots,x_n)$是$k$的有限生成的可分扩张,那么可以从$\{x_1,\cdots,x_n\}$中选取一组可分超越基.
	\begin{proof}
		
		(c)$\Rightarrow$(a)证明过了,(a)$\Rightarrow$(b)因为定义.下面证明(b)$\Rightarrow$(c):设$K=k(x_1,\cdots,x_n)$,那么$\{x_1,\cdots,x_n\}$包含了一组超越基,适当重排这些元,设$\{x_1,\cdots,x_r\}$是一组超越基,再设$x_{r+1},\cdots,x_q,q\le n$是$k(x_1,\cdots,x_r)$上的可分代数元,并且$y=x_{q+1}$不是$k(x_1,\cdots,x_r)$上的可分代数元.那么$y$在$k(x_1,\cdots,x_r)$上的极小多项式可以表示为$f(Y^p)$(因为不是可分元导致纯不可分维数至少是$p$),这里$f(Y^p)$的系数是$x_1,\cdots,x_r$的有理函数,去分母得到一个多项式$F(X_1,\cdots,X_r,Y^p)\in k[X_1,\cdots,X_r,Y]$使得$F(x_1,\cdots,x_r,y^p)=0$.倘若$\partial F/\partial X_i=0,\forall 1\le i\le r$,则$F(X_1,\cdots,X_r,Y^p)$是一个系数落在$k^{1/p}$中的多项式$G(X_1,\cdots,X_r,Y)$的$p$次幂.但是这导致:
		\begin{align*}
			k[x_1,\cdots,x_r,y]\otimes_kk^{1/p}&=(k[X_1,\cdots,X_r,Y]/(F(X_1,\cdots,X_r,Y^p)))\otimes_kk^{1/p}\\&=k^{1/p}[X_1,\cdots,X_r,Y]/(G(X_1,\cdots,X_r,Y))^p
		\end{align*}
		
		这就导致$K\otimes_kk^{1/p}$的一个子环不是既约的,这个矛盾说明至少存在某个$1\le i\le r$使得$\partial F/\partial X_i\not=0$,不妨设为$i=1$.这导致$x_1$在$k(x_2,\cdots,x_r,y)$上是可分代数元,于是$\{x_2,\cdots,x_r,y\}$也是一组超越基.而$x_{r+1},\cdots,x_q$都在$k(x_1,\cdots,x_r)$上可分代数,于是$x_{r+1},\cdots,x_q$都是$k(x_2,\cdots,x_r,y)$上的可分代数元.于是交换$x_1$和$y=x_{q+1}$,导致$x_{r+1},\cdots,x_{q+1}$都是$k(x_1,\cdots,x_r)$上的可分代数元.归纳操作下去,得到$x_1,\cdots,x_n$的一个重排,使得$\{x_1,\cdots,x_r\}$是一组超越基,并且$x_{r+1},\cdots,x_n$都是$k(x_1,\cdots,x_r)$上的可分代数元,于是$\{x_1,\cdots,x_r\}$是一组可分超越基.
	\end{proof}
	\item 设$k$是完全域,那么$k$的每个域扩张都是可分扩张,并且$k$代数$A$是可分的当且仅当是既约的.这里完全域的定义是每个代数扩张都是可分代数扩张.
	\begin{proof}
		
		特征零的时候域扩张总是可分扩张.下面设$k$是特征$p>0$的完全域,于是$k=k^{1/p}$,于是上一条说明对任意域扩张$K/k$,它的每个有限生成子域扩张都是可分的,于是它本身是可分扩张.接下来如果$k$代数$A$是可分的,那么当然$A=A\otimes_kk$本身是既约的,反过来如果$k$代数$A$是既约的,不妨设它是有限生成代数,那么$A$是诺特环,它的全商环是有限个域的直积(既约诺特环的全商环是有限个域的直积),记作$K=K_1\times\cdots\times K_r$,因为$k$是完全域,每个$K_i$都在$k$上可分,于是$K$也是可分的(环上代数的张量和有限直积可交换,既约环的有限直积是既约的),于是它的子环$A$也是可分的.
	\end{proof}
	\item 设$k$是特征$p$的域,固定它的一个代数闭包$\overline{k}$,记$k^{1/p^n}=\{x\in\overline{k}\mid x^{p^n}\in k\}$和$k^{1/p^{\infty}}=\cup_{n\ge0}k^{1/p^n}$.它们都是$k$的纯不可分(代数)扩张,并且$k^{1/p^{\infty}}$是包含$k$的最小的完全域.另外如果再取一个域扩张$K/k$,我们接下来要考虑$K$和$k^{1/p^n}$的线性无交性,所以这两个域至少要包含在统一的某个域中,我们这样处理:因为$k\subseteq k^{1/p^{\infty}}$是代数扩张,那么存在$k^{1/p^{\infty}}\to\overline{K}$的$k$嵌入,这个嵌入使得$k^{1/p^{\infty}}$可视为$\overline{K}$的子域,于是此时$K$和$k^{1/p^{\infty}}$都是$\overline{K}$的子域了.
	\item 设$k\subseteq K$是域扩张,那么有:
	\begin{enumerate}
		\item 如果$K/k$是可分扩张,那么$K$和每个$k^{1/p^n}$(其中$n$取正整数或者$\infty$)都在$k$上线性无交.
		\item 如果$K$和某个$k^{1/p^n}$(其中$n$取正整数或者$\infty$)在$k$上线性无交,那么$K/k$是可分扩张.
	\end{enumerate}
	\begin{proof}
		
		(a):设$K/k$是可分扩张,归结为证明$K$和$k^{1/p^{\infty}}$在$k$上线性无交.设$\{x_1,\cdots,x_m\}\subseteq K$是$k$上的线性无关集,设$y_1,\cdots,y_m\in k^{1/p^{\infty}}$使得$\sum_ix_iy_i=0$,取$k_1=k(y_1,\cdots,y_m)$,这是$k$的有限维的纯不可分扩张,于是存在一个足够大的$n$使得$k_1^{p^n}\subseteq k$.取$A=K\otimes_kk_1$,这是既约环,它元素的$p^n$次幂都落在$K$中,所以元素的$p^n$次幂要么是单位要么是零,但是既约导致非零元的$p^n$次幂必须是单位,于是它的非零元都是单位,于是$A$是域,于是典范映射$A=K\otimes_kk_1\cong Kk_1$是同构,于是从$x_1,\cdots,x_n$在$k$上线性无关得到它们在$k_1$上线性无关,于是$y_1=\cdots=y_m=0$.
		
		\qquad
		
		(b):因为从$K$和$k^{1/p^n}$线性无交可推出$K$和$k^{1/p}$线性无交,所以不妨就设$K$和$k^{1/p}$线性无交.此时有$K\otimes_kk^{1/p}$是域,于是如果$K'$是$K/k$的有限生成子域扩张,那么$K'\otimes_kk^{1/p}$仍然是既约环,我们解释过这个条件导致$K'$在$k$上可分,于是$K/k$是可分扩张.
	\end{proof}
	\item 传递性.设$k\subseteq F\subseteq L$是域扩张链,那么$k\subseteq L$是可分扩张可推出$k\subseteq F$是可分扩张,但是推不出$F\subseteq L$是可分扩张(这和可分代数情况不一样).反过来从$k\subseteq F$和$F\subseteq L$是可分扩张可推出$k\subseteq L$是可分扩张.
	\begin{proof}
		
		如果$k\subseteq L$是可分扩张,任取$k$的域扩张$k'$,那么$F\otimes_kk'$作为既约环$L\otimes_kk'$的子环也是既约的,于是$k\subseteq F$是可分扩张.
		
		\qquad
		
		反例考虑$k$是一个特征$p>0$的域,取$k\subseteq k(X^p)\subseteq k(X)$,那么$k\subseteq k(X)$是可分扩张,但是$k(X^p)\subseteq k(X)$不是可分扩张.
		
		\qquad
		
		如果$k\subseteq F$和$F\subseteq L$都是可分扩张,要证明$k\subseteq L$是可分扩张,等价于证明$L$和$k^{1/p}$在$k$上线性无交.因为$k\subseteq F$是可分扩张,导致$F$和$k^{1/p}$在$k$上线性无交.又因为$F\subseteq L$是可分扩张,导致$L$和$F^{1/p}$在$F$上线性无交.由于$Fk^{1/p}\subseteq F^{1/p}$,于是$L$和$Fk^{1/p}$也是线性无交的,按照线性无交的传递性得到$L$和$k^{1/p}$是线性无交的.
		$$\xymatrix{&Lk^{1/p}&&\\L\ar[ur]&&Fk^{1/p}\ar[ul]&\\&F\ar[ul]\ar[ur]&&k^{1/p}\ar[ul]\\&&k\ar[ul]\ar[ur]&}$$
	\end{proof}
\end{enumerate}

微分基,超越基和$p$-基.设$K/k$是域扩张,设$\Omega_{K/k}$是微分模,那么它是$K$上的线性空间.
\begin{itemize}
	\item 如果$\{\mathrm{d}x_i\mid x_i\in K\}$是$K$线性空间$\Omega_{K/k}$的一组基,我们就称$\{x_1,\cdots,x_n\}\subseteq K$是$K/k$的微分基(differential basis).
	\item $K/k$的超越基指的是该扩张的一个代数无关集$X=\{x_i\mid x_i\in K\}$,使得$K/k(X)$是代数扩张,也等价于讲是极大的代数无关集.
	\item 如果$k$的特征是$p>0$,称$x_1,\cdots,x_n\in K$是$p$-无关的,如果它们两两不同,并且有$[K^pk(x_1,\cdots,x_n):K^pk]=p^n$,换句话讲集合$\{x_1^{r_1}\cdots x_n^{r_n}\mid 0\le r_i\le p-1,\forall 1\le i\le n\}$是$K^pk$上线性无关的,这个集合中的元称为$\{x_1,\cdots,x_n\}$的$p$-单项式.一个子集$S\subseteq K$称为$p$-无关的,如果它的每个有限子集是$p$-无关的.称一个子集$X\subseteq K$是$K/k$的$p$-基,如果它本身是$p$-无关的,并且有$K=K^pk(X)$.
\end{itemize}
\begin{enumerate}
	\item 微分基总是存在的,因为$\{dx\mid x\in K\}$在$K$上生成了整个$\Omega_{K/k}$,于是它包含了一组基.
	\item 一组元$X=\{x_i\mid i\in I\}\subseteq K$是$K/k$的微分基当且仅当对任意$y_i\in K,i\in I$,存在唯一的$D\in\mathrm{Der}_k(K)$使得$D(x_i)=y_i,\forall i\in I$.
	\begin{proof}
		
		因为$X=\{x_i\mid i\in I\}\subseteq K$是$K/k$的微分基当且仅当$\{\mathrm{d}x_i\mid i\in I\}$是$\Omega_{K/k}$作为$K$线性空间的一组基,于是它对偶空间中的线性泛函可以被它在每个$x_i$上的取值唯一绝对,并且不需要满足其它条件.但是反过来如果$\{\mathrm{d}x_i\mid i\in I\}$在$\Omega_{K/k}$上是$K$线性相关的,比方说$a_1x_1+\cdots+a_nx_n=0,a_i\in K$不全为零,那么$a_1y_1+\cdots+a_ny_n=0$,这导致$y_i$不能任取的.
	\end{proof}
	\item 设$k$的特征为零.我们断言$\mathrm{d}x_1,\cdots,\mathrm{d}x_n\in\Omega_{K/k}$在$K$上线性无关当且仅当$x_1,\cdots,x_n$在$k$上代数无关.于是特别的,特征零的情况下$K/k$的微分基等同于超越基.
	\begin{proof}
		
		必要性.假设$\{x_1,\cdots,x_n\}$在$k$上不是代数无关的,选取一个次数最小的多项式$0\not=f(X_1,\cdots,X_n)\in k[X_1,\cdots,X_n]$使得$f(x_1,\cdots,x_n)=0$.不妨设$f(X)$的表达式中出现$X_1$项,也即$f_1=\partial f/\partial X_1\not=0$,但是$\deg f_1<\deg f$,按照我们的选取就有$f_1(x_1,\cdots,x_n)\not=0$,于是$0=\mathrm{d}f(x_1,\cdots,x_n)=\sum_if_i(x)\mathrm{d}x_i$,这说明$\mathrm{d}x_1,\cdots,\mathrm{d}x_n$在$K$上线性相关.
		
		\qquad
		
		充分性.假设$\{x_1,\cdots,x_n\}$在$K$上代数无关,那么它包含在$K/k$的某个超越基$X=\{x_i\mid i\in I\}$中.取$D_i\in\mathrm{Der}_kk(X),i\in I$为$D_i(x_j)=\delta_{ij}$.我们断言$D_i$都可以延拓为$\mathrm{Der}_kK$中的导数.一旦这成立,由于$D_i$就是$\mathrm{d}x_i\in\Omega_{K/k}$在对偶空间中的对偶元,从$D_i,i\in I$线性无关就得到$\mathrm{d}x_i$在$K$上线性无关.最后解释下$D_i$为什么可以延拓,由于$K/k(X)$是代数可分扩张,我们解释过有$\Omega_{K/k}=\Omega_{k(X)/k}\otimes_{k(X)}K$,于是有:
		\begin{align*}
			\mathrm{Der}_kK&=\mathrm{Hom}_K(\Omega_{K/k},K)\\&=\mathrm{Hom}_K(\Omega_{k(X)/k}\otimes_{k(X)}K,K)\\&=\mathrm{Hom}_{k(X)}(\Omega_{k(X)/k},\mathrm{Hom}_K(K,K))\\&=\mathrm{Hom}_{k(X)}(\Omega_{k(X)/k},K)\\&=\mathrm{Der}_k(k(X),K)
		\end{align*}
	\end{proof}
    \item 如果$K/k$的特征是$p$,那么$p$-基总是存在的,它就是极大的$p$-无关组.借助Zorn引理我们可以证明每个$p$-无关组都可以扩充成一个$p$-基.
    \begin{proof}
    	
    	Zorn引理可以说明包含一个预先取定的$p$-无关组的所有$p$-无关组中存在极大元.我们来证明极大$p$-无关组就是$p$-基.假设$X$是$K/k$的极大$p$-基,假设$K\not=K^pk(X)$,任取$t\in K-K^pk(X)$,我们只需证明$X\cup\{t\}$是$p$-无关的.假设它是$p$相关的,那么存在$K^pk(X)$中不全为零的元$g_0,g_1,\cdots,g_{p-1}$,使得$g_{p-1}t^{p-1}+\cdots+g_1t+g_0=0$,于是$t$满足$K^pk(X)$上的次数不超过$p-1$的非零多项式.另一方面按照$t^p=\alpha\in K^p$,结合$p$是素数知$T^p-\alpha$是$t$在$K^pk(X)$上的极小多项式(因为特征$p$导致$T^p-\alpha$在$K$中分解为$(T-t)^p$,倘若它在$K^pk(X)$上的极小多项式为$(T-t)^m$,那么$t^m\in K^pk(X)$,所以如果$m<p$就有$m$和$p$互素,导致$t=t^{am+bp}\in K^pk(X)$矛盾,所以$m=p$).这就和$t$还满足一个次数$\le p-1$的非零多项式相矛盾.
    \end{proof}
    \item 设$k$的特征为$p>0$,我们断言微分基等同于$p$-基.我们解释过$X=\{x_i\mid i\in I\}$是$p$-无关的等价于讲$\Gamma_X=\{x_1^{r_1}\cdots x_n^{r_n}\mid x_i\in X,1\le i\le n\text{两两不同} 0\le r_i\le p-1\}$在$K^pk$上线性无关.
    \begin{proof}
    	
    	充分性.如果$X$是$p$-基,那么任意映射$D:X\to K$可以唯一延拓为$\mathrm{Der}_kK$中的元,因为必须有$D(x_1^{r_1}\cdots x_n^{r_n})=\sum_{i=1}^nr_ix_1^{r_1}\cdots x_i^{r_i-1}\cdots x_n^{r_n}D(x_i),\forall r_i\in\mathbb{N}$.这导致$D$的延拓必须是先对$p$-单项式按上述等式定义导数,再$K^pk$线性延拓到整个$K$,所以$D$延拓为$\mathrm{Der}_kK$的导数是唯一的,我们解释过此时$X$就是微分基.
    	
    	\qquad
    	
    	必要性.设$X'$是$K/k$的微分基,我们断言$X'$是$p$-无关的,一旦这成立,那么$X'$就要包含在某个$p$-基$X$中,但是充分就已经说明$X$就要是微分基,这迫使$X'=X$,就说明了$X'$是$p$-基.现在假设有$x_1,\cdots,x_n\in X'$是$p$-相关的,那么可不妨设$x_1\in K^pk(x_2,\cdots,x_n)$,于是可记$x_1=f(x_2,\cdots,x_n)$,其中$f$是系数在$K^pk$中的多项式.那么在$\Omega_{K/k}$中就要有$\mathrm{d}x_1=\sum_{i=2}^nf_i(x)\mathrm{d}x_i$,这就和$\mathrm{d}x_1,\cdots,\mathrm{d}x_n$在$K$上线性无关相矛盾.
    \end{proof}
    \item 例如设$k$的特征$p>0$,设$K=k(a)$是非平凡的纯不可分扩张,那么$\mathrm{Der}_kK$是一维的,因为$\{a\}$是一个$p$-基.
\end{enumerate}

域扩张的微分模.设$k$是域,它的素子域$\Pi$只能是$\mathbb{Q}$或者$\mathbb{F}_p$,取决于$k$的特征,我们把$\Omega_{k/\Pi}$简记为$\Omega_k$.
\begin{enumerate}
	\item 设$K/k$是域扩张,如下命题互相等价:
	\begin{enumerate}
		\item $K/k$是可分扩张.
		\item 对任意子域$k'\subseteq k$,典范映射$\Omega_{k/k'}\otimes_kK\to\Omega_{K/k'}$是单射.
		\item 对任意子域$k'\subseteq k$和$k/k'$的微分基$X$,它可以延拓为$K/k'$的微分基.
		\item 典范映射$\Omega_k\otimes_kK\to\Omega_K$是单射.
		\item 对任意$K$模$M$,对任意$k\to M$的导数都可以延拓为$K\to M$的导数.
	\end{enumerate}
    \begin{proof}
    	
    	(b)和(c)等价是因为$\Omega_{k/k'}\otimes_k\to\Omega_{K/k'}$是单射等价于对偶映射$\mathrm{Der}_{k}(k,K)=\mathrm{Hom}_k(\Omega_{k/k'},K)=\mathrm{Hom}_K(\Omega_{k/k'}\otimes_kK,K)\to\mathrm{Hom}_K(\Omega_{K/k'},K)=\mathrm{Der}_{k'}(K)$是单射.类似的(d)和(e)也是等价的.(b)推(d)是平凡的.于是问题归结为在特征$p>0$的情况下证明(a)$\Rightarrow$(c)和(d)$\Rightarrow$(a).
    	
    	\qquad
    	
    	(a)$\Rightarrow$(c):如果$K/k$是可分的,那么$K$和$k^{1/p}$在$k$上线性无交,于是$K^p$和$k$在$k^p$上线性无交,进而有$K^pk'$和$k$在$k^pk'$上线性无交.如果取$k/k'$的$p$-基为$X$,那么由$p$-单项式构成的集合$\Gamma_X$就在$k^pk'$上线性无关,按照线性无交性,它们就也在$K^pk'$上线性无关,于是$X\subseteq K$也是$k'$上的$p$-无关集,所以它可以扩充为$K/k'$的$p$-基.
    	$$\xymatrix{k&&K^pk'&\\&k^pk'\ar[ul]\ar[ur]&&K^p\ar[ul]\\&&k^p\ar[ur]\ar[ul]&}$$
    	
    	(d)$\Rightarrow$(a):记$k$的素子域为$\Pi$,取$k/\Pi$的$p$-基为$X$,那么$\Gamma_X$是$k/k^p$的一组基.按照$p$-基就是微分基,有$\{\mathrm{d}x\mid x\in X\}$是$\Omega_k$在$k$上的基.按照$\Omega_k\otimes_kK\to\Omega_K$是单射,就有$\{\mathrm{d}x\mid x\in X\}$也是$\Omega_K$的线性无关组,于是$\Gamma_X$也是$K^p$上的线性无关组,这导致典范映射$k\otimes_{k^p}K^p\to K^pk$是单射,于是$k$和$K^p$在$k^p$上线性无交.于是$k^{1/p}$和$K$在$k$上线性无交,于是$K/k$是可分的.
    \end{proof}
    \item 设$k$的特征$p>0$,设$\Pi\subseteq k$是素子域,称$k/\Pi$的一个$p$-基为$k$的绝对$p$-基.如果$k_0\subseteq k$是一个完全域,那么$k_0=k_0^p\subseteq k^p$,于是有$k^pk_0=k^p=k^p\Pi$,于是$k$的绝对$p$-基一定也是$k/k_0$的$p$-基.
    \item 设$k$的特征$p>0$,设$K/k$是域扩张,那么$K/k$是0-平展的当且仅当$k$的每个绝对$p$-基都是$K$的绝对$p$-基.
    \begin{proof}
    	
    	必要性.如果$K/k$是0-平展的,那么从0-非分歧就得到$\Omega_{K/k}=0$,从0-光滑和微分模的第一基本正合列得到$\Omega_K=\Omega_k\otimes_kK$.于是如果$X$是$k$的绝对$p$-基,那么$\{\mathrm{d}x\mid x\in X\}$是$\Omega_k$在$k$上的基,那么这个等式说明$\{\mathrm{d}x\mid x\in X\}$也是$\Omega_K$的基,于是$X$也是$K$的绝对$p$-基.
    	
    	\qquad
    	
    	充分性.设$C$是$k$代数,设$N$是$C$的理想使得$N^2=0$,考虑如下交换图表,我们要找唯一的同态$K\to C$使得图表交换.任取$\alpha\in K$,任取$u(\alpha)$在$C$中的提升$a$,那么$a^p$是不依赖于提升$a$的选取的,因为如果$a_1$是另一个提升,那么存在$x\in N$使得$a_1=a+x$,于是$a_1^p=a^p+x^p$,但是这里$p\ge2$,所以$x^p=0$.构造同态$v_0:K^p\to C$为$\alpha^p\mapsto a^p$,这是定义良性的.至此我们还没有用$K/k$上的条件.按照条件$K/k$是可分扩张,并且$K=K^pk$.于是$K$和$k^{1/p}$在$k$上线性无交,于是$K^p$和$k$在$k^p$上线性无交,于是有$K=K^pk=K^p\otimes_{k^p}k$.于是$v_0:K^p\to C$可以延拓为$v:K\to Cc$使得图表交换.这证明了$K/k$是0-光滑的,最后从$K^pk=K$得到$K/k$的$p$-基是空集,导致$\Omega_{K/k}=0$,也即$K/k$是0-非分歧的.
    	$$\xymatrix{K\ar[rr]^u\ar@{-->}[drr]^v&&C/N\\k\ar[u]^i\ar[rr]_j&&C\ar[u]_g}$$
    \end{proof}
    \item 设$K/k$是特征$p>0$的可分扩张,设$X$是$K/k$的$p$-基,那么$X$在$k$上代数无关.但是这里$X$未必是$K/k$的超越基,比如说取$K=k(X,X^{1/p},X^{1/p^2},\cdots)$,由于$K^pk=K$,此时$p$-基是空集,但是$K/k$是超越扩张.
    \begin{proof}
    	
    	假设$X$不在$k$上代数无关,设$0\not=f(X_1,\cdots,X_n)\in k[X_1,\cdots,X_n]$是次数最小的非零多项式,使得存在$x_1,\cdots,x_n\in X$,满足$f(x_1,\cdots,x_n)=0$,记$\deg f=d$.记$f(X_1,\cdots,X_n)=\sum_{0\le i_1,\cdots,i_n\le p-1}g_{i_1,\cdots,i_n}(X_1^p,\cdots,X_n^p)X_1^{i_1}\cdots X_n^{i_n}$.于是由于$f(x_1,\cdots,x_n)=0$,并且$\{x_1,\cdots,x_n\}$是$p$-无关的,就有$g_{i_1,\cdots,i_n}(x_1^p,\cdots,x_n^p)=0$对任意$0\le i_1,\cdots,i_n\le p-1$成立.但是按照我们对$f$的选取,从$d\ge\deg g_{i_1,\cdots,i_n}(X_1^p,\cdots,X_n^p)+i_1+\cdots+i_n$,说明必须有$i_1,\cdots,i_n$不全为零时$g_{i_1,\cdots,i_n}(X_1^p,\cdots,X_n^p)=0$,并且有$f(X_1,\cdots,X_n)=g_{0,\cdots,0}(X_1^p,\cdots,X_n^p)$.于是存在$h(X)\in k^{1/p}[X_1,\cdots,X_n]$使得$f(X)=h(X)^p$.但是由于$K$和$k^{1/p}$在$k$上线性无交,由于次数$<d$的关于$x_1,\cdots,x_n$的单项式在$k$上线性无关,就有它们也在$k^{1/p}$上线性无关,于是有$h(x_1,\cdots,x_n)\not=0$,这个矛盾说明$X$在$k$上代数无关.
    \end{proof}
    \item 域扩张$K/k$是可分扩张当且仅当$K/k$是0-光滑的.
    \begin{proof}
    	
    	充分性.如果$K/k$是0-光滑的,按照微分模的第一基本正合列,有$\Omega_k\otimes_kK\to\Omega_K$是单射,我们解释过此时有$K/k$是可分扩张.
    	
    	\qquad
    	
    	必要性.如果$K/k$是可分扩张,设$X$是$K/k$的微分基,上一条说明$X$在$k$上代数无关,于是$k(X)$是$k$上的纯超越扩张,那么必然有$k(X)$在$k$上0-光滑.我们断言$K/k(X)$是0-平展的,在特征零的情况下微分基就是超越基,于是$K/k(X)$是可分代数扩张,我们解释过可分代数扩张是0-平展的;在特征$p>0$的情况下我们有短正合列$0\to\Omega_k\otimes_kK\to\Omega_K\to\Omega_{K/k}\to0$,于是把$k$的一个绝对$p$-基并上$X$就得到$K$的一组绝对$p$-基,它当然也是$k(X)$的绝对$p$-基,于是$K/k(X)$是0-平展的.于是$K/k$作为两个0-光滑域扩张$k\subseteq k(X)\subseteq K$的复合,是0-光滑的.
    \end{proof}
\end{enumerate}

非完全模.设$k\to A\to B$是环同态,我们用$\Gamma_{B/A/k}$表示典范映射$\Omega_{A/k}\otimes_AB\to\Omega_{B/k}$的核,称为$A$代数$B$在$k$上的非完全模(imperfection module).
\begin{enumerate}
	\item 设$k\to K\to L\to L'$是域扩张,我们有如下短正合列:
	$$\xymatrix{0\ar[r]&\Gamma_{L/K/k}\otimes_LL'\ar[r]&\Gamma_{L'/K/k}\ar[r]&\Gamma_{L'/L/k}\ar[r]&\Omega_{L/K}\otimes_LL'\ar[r]&\Omega_{L'/K}\ar[r]&\Omega_{L'/L}\ar[r]&0}$$
	\begin{proof}
		
		按照非完全模的定义和微分模的第一基本正合列,我们有正合列:
		$$\xymatrix{0\ar[r]&\Gamma_{L/K/k}\ar[r]&\Omega_{K/k}\otimes_KL\ar[r]&\Omega_{L/k}\ar[r]&\Omega_{L/K}\ar[r]&0}$$
		
		张量上$-\otimes_LL'$,得到正合列:
		$$\xymatrix{0\ar[r]&\Gamma_{L/K/k}\otimes_LL'\ar[r]&\Omega_{K/k}\otimes_KL'\ar[r]&\Omega_{L/k}\otimes_LL'\ar[r]&\Omega_{L/K}\otimes_LL'\ar[r]&0}$$
		
		考虑如下实线图表,上行和下行都是正合列,其中$f_2$定义为$\Omega_{L/k}\otimes_LL'\to\Omega_{L'/k}$,$\mathrm{d}x\otimes y=y\mathrm{d}x$.这使得中间方格交换.那么存在唯一的虚线同态$f_1$和$f_3$使得图表交换.
		$$\xymatrix{0\ar[r]&\Gamma_{L/K/k}\otimes_LL'\ar[r]\ar@{-->}[d]_{f_1}&\Omega_{K/k}\otimes_KL'\ar[r]\ar@{=}[d]&\Omega_{L/k}\otimes_LL'\ar[r]\ar[d]_{f_2}&\Omega_{L/K}\otimes_LL'\ar[r]\ar@{-->}[d]_{f_3}&0\\0\ar[r]&\Gamma_{L'/K/k}\ar[r]&\Omega_{K/k}\otimes_KL'\ar[r]&\Omega_{L'/k}\ar[r]&\Omega_{L'/K}\ar[r]&0}$$
		
		一般的,如果有如下交换图表:
		$$\xymatrix{0\ar[r]&X\ar[r]\ar[d]_{f_1}&A\ar[r]\ar@{=}[d]&B\ar[r]\ar[d]_{f_2}&P\ar[r]\ar[d]_{f_3}&0\\0\ar[r]&Y\ar[r]&A\ar[r]&C\ar[r]&Q\ar[r]&0}$$
		
		它就诱导了如下交换图表,其中上行和下行都是短正合列:
		$$\xymatrix{0\ar[r]&A/X\ar[r]\ar[d]_{f_1'}&B\ar[r]\ar[d]_{f_2}&P\ar[r]\ar[d]_{f_3}&0\\0\ar[r]&A/Y\ar[r]&C\ar[r]&Q\ar[r]&0}$$
		
		按照蛇形引理,以及$f_1$诱导的$f_1'$是满射,我们得到短正合列:
		$$\xymatrix{0\ar[r]&Y/X\ar[r]&\ker f_2\ar[r]&\ker f_3\ar[r]&0}$$
		
		它诱导了正合列:
		$$\xymatrix{0\ar[r]&X\ar[r]&Y\ar[r]&\ker f_2\ar[r]&P\ar[r]&Q\ar[r]&\mathrm{coker}f_3\ar[r]&0}$$
		
		此即要证的正合列.
	\end{proof}
    \item Cartier等式.设$k$是完全域,设$K/k$是域扩张,设$L/K$是有限生成域扩张,那么有如下等式:
    $$\dim_L\Omega_{L/K}=\mathrm{tr.deg}_KL+\dim_L\Gamma_{L/K/k}$$
    \begin{proof}
    	
    	假设有域扩张$k\to K\to L\to L'$,其中$L/K$和$L'/L$都是有限生成域扩张.如果命题对$k\to L\to L'$和$k\to K\to L$均成立,按照引理有:
    	\begin{align*}
    		\dim_{L'}\Omega_{L'/K}-\dim_{L'}\Gamma_{L'/K/k}&=\left(\dim_{L'}\Omega_{L'/L}-\dim_{L'}\Gamma_{L'/L/k}\right)+\left(\dim_L\Omega_{L/K}-\dim_L\Gamma_{L/K/k}\right)\\&=\mathrm{tr.deg}_LL'+\mathrm{tr.deg}_KL=\mathrm{tr.deg}_KL'
    	\end{align*}
    
        于是命题对$k\to K\to L'$也成立.我们把$K\to L$拆成有限个单扩张的复合,那么按照我们刚才的观察,归结为证明$L/K$是单扩张的情况.此时分两种情况:
        \begin{itemize}
        	\item $L=K(\alpha)$,其中$\alpha$是$K$上的超越元或者可分代数元.此时域扩张$K\to L=K(\alpha)$总是可分扩张,于是$L$在$K$上0-光滑,按照微分模的第一基本正合列,得到$\Gamma_{L/K/k}=0$.如果$\alpha$是超越元,那么$\Omega_{L/K}$以$\mathrm{d}\alpha$为基,此时$\dim_L\Omega_{L/K}=\mathrm{tr.deg}_KL=1$;如果$\alpha$是可分代数元,此时$L/K$是0-平展的,有$\Omega_{L/K}=0$,于是$\dim_L\Omega_{L/K}=\mathrm{tr.deg}_KL=0$.无论哪种情况都有等式成立.
        	\item $L=K(\alpha)$,其中$K$的特征为$p>0$,并且$\alpha^p=a\in K$.不妨设$\alpha\not\in K$.考虑环同态链$k\to K[X]\to K[X]/(X^p-a)=L$,这里$K[X]\to L$就是满射,按照微分模的第二基本正合列,就有$\Omega_{L/k}=\left(\Omega_{K[X]/k}\otimes_{K[X]}L\right)/\mathrm{im}\delta$.这里$\delta$把$(X^p-a)$中的元映射为导数,而$\mathrm{d}(x^p-a)=-\mathrm{d}a$,于是$\mathrm{im}\delta=L\mathrm{d}a$.另一方面考虑环同态链$k\to K\to K[X]$,这里$K[X]$在$K$上0-光滑,于是微分模的第一基本正合列说明有$\Omega_{K[X]/k}=\Omega_{K[X]/K}\oplus\left(\Omega_{K/k}\otimes_KK[X]\right)$,于是有:
        	\begin{align*}
        		\Omega_{L/k}&=\left(\Omega_{K[X]/k}\otimes_{K[X]}L\right)/L\mathrm{d}a\\&=\left(\frac{\Omega_{K/k}\otimes_KL}{L\mathrm{d}a}\right)\oplus\left(\Omega_{K[X]/K}\otimes_{K[X]}L\right)
        	\end{align*}
        
            按照定义$\Gamma_{L/K/k}$是典范映射$\Omega_{K/k}\otimes_KL\to\Omega_{L/k}$的核,也即$\Omega_{K/k}\otimes_KL$到上式第一个分量的典范映射,这个映射的核就是$L\mathrm{d}a$,所以是一维的.于是我们证明了$\dim_L\Gamma_{L/K/k}=1$.但是$L/k$的$p$-基就是$\{\alpha\}$,它也是微分基,于是$\dim_L\Omega_{L/K}=1$,且$\mathrm{tr.deg}_KL=0$.于是等式成立.
        \end{itemize}
    \end{proof}
    \item 推论.一般的设$L/K$是有限生成域扩张,取上一条中$k$是$K$包含的素子域(于是它是完全域),那么总有$\dim_L\Omega_{L/K}\ge\mathrm{tr.deg}_KL$.我们断言这个不等式取等当且仅当$L/K$是可分扩张.特别的,有限生成域扩张$L/K$满足$\Omega_{L/K}=0$当且仅当$L/K$是可分代数扩张.
    \begin{proof}
    	
    	首先如果$\Omega_{L/K}=0$,那么$L/K$不能有超越元也不能有纯不可分元(不能有超越元因为命题里的不等式,不能有纯不可分元因为上一条证明中得到了此时$\Omega_{L/K}$是非平凡的).如果$L/K$是可分扩张,那么$L/K$是0-光滑的,按照第一基本正合列得到$\Gamma_{L/K/k}=0$,于是此时有$\dim_L\Omega_{L/K}=\mathrm{tr.deg}_KL$.反过来如果这个等式取等,取$L/K$的微分基是$\{x_1,\cdots,x_n\}$,也即$\mathrm{d}x_1,\cdots,\mathrm{d}x_n$是$\Omega_{L/K}$的$L$-基.那么按照第一基本正合列得到:
    	$$\xymatrix{\Omega_{K(x_1,\cdots,x_n)/K}\otimes_{K(x_1,\cdots,x_n)}L\ar[r]&\Omega_{L/K}\ar[r]&\Omega_{L/K(x_1,\cdots,x_n)}\ar[r]&0}$$
    	
    	但是这里第一个同态是满的,说明$\Omega_{L/K(x_1,\cdots,x_n)}=0$,按照之前的解释就有$L/K(x_1,\cdots,x_n)$是可分代数扩张.但是$L/K$是超越维数$n$的扩张,这迫使$\{x_1,\cdots,x_n\}$本身是一组超越基.
    \end{proof}
    \item 环$R$的理想$I$称为微分理想(differential ideal),如果$A$的每个导数都把$I$映入$I$.一个环如果不存在非平凡的微分理想,就称它是微分单环(differentiably simple ring).关于微分单环我们有如下Harper定理:一个特征$p>0$的诺特环$R$是微分单环当且仅当$R=k[T_1,\cdots,T_n]/(T_1^p,\cdots,T_n^p)$,其中$k$是特征$p$的域.
\end{enumerate}
\newpage
\subsection{高阶导数}

设$\xymatrix{k\ar[r]^f&A\ar[r]^g&B}$是环同态的复合($k$约定是个环),对$m\le\infty$,一个长度为$m$的从$A$到$B$的高阶导数是指一个序列$D=(D_0,D_1,\cdots,D_m)$,其中每个$D_i:A\to B$是$k$模同态.它们满足$\forall 1\le i\le m$和$\forall x,y\in A$有$D_i(xy)=\sum_{r+s=i}D_r(x)D_x(y)$.如果$A=B$,并且$D_0$是$A$上的恒等映射,我们记全体满足这个条件的长度$m$的高阶导数构成的集合是$\mathrm{HS}_k(A,m)$,如果$m=\infty$就记作$\mathrm{HS}_k(A)$.
\begin{enumerate}
	\item 一般的,记$B_m$是一个$k$代数,当$m<\infty$时取$B_m=B[t]/(t^{m+1})$;当$m=\infty$时取$B_{\infty}=B[[t]]$.那么$D=(D_0,D_1,\cdots,D_m)$是一个高阶导数等价于讲映射$E_t:A\to B_m$,$x\mapsto\sum_{i=1}^mD_i(x)t^i$是一个满足$E_t(x)\equiv D_0(x)\mathrm{mod}t$的$k$代数同态.
	\item 如果$D=(D_0,D_1,\cdots,D_m)$是高阶导数,那么$D_1\in\mathrm{Der}_k(A,B)$,并且$i\ge1$时$D_i$都在$k$上为零.
	\item 存在从$\mathrm{HS}_k(A,m)$到$A_m$上的满足$E(a)\equiv a(\mathrm{mod}t),a\in A$的$k$代数同构之间的典范双射.后者在复合下构成一个群,于是这使得$\mathrm{HS}_k(A,m)$上自然具备一个群结构(一般未必是阿贝尔的):对$D=(D_0,D_1,\cdots,D_m)$和$D'=(D_0',D_1',\cdots,D_m')$,有$DD'=(D_0'',D_1'',\cdots,D_m'')$,其中$D_i''=\sum_{p+q=i}D_pD_q'$.而$D^{-1}=(D_0^*,D_1^*,\cdots,D_m^*)$,其中$D_0^*=D_0=1$,$D_1^*=-D_1$,$D_2^*=D_1^2-D_2$,$D_3^*=-D_1^3+D_1D_2+D_2D_1-D_3$等.
	\begin{proof}
		
		对$D=(D_0,D_1,\cdots,D_m)\in\mathrm{HS}_k(A)$,我们之前解释过它对应于一个$k$代数同态$E:A\to A_m$,$x\mapsto\sum_{i=1}^mD_i(x)t^i$.我们断言它可以经$E(\sum_ja_jt^j)=\sum_j\sum_{i=0}^mD_i(a_j)t^{i+j}$延拓为满足$E(a)\equiv a(\mathrm{mod}t),\forall a\in A$的$k$代数同构$A_m\to A_m$.它是单射是因为,如果记$x=a_rt^r+a_{r+1}t^{r+1}+\cdots\in A_m$,其中$a_r\not=0$,那么$E(x)$的最低次项是$D_0(a_r)t^r=a_rt_r\not=0$.它是满射是因为,任取非零元$x=a_rt^r+a_{r+1}t^{r+1}+\cdots\in A_m$,那么$x-E(a_rt^r)=b_{r+1}t^{r+1}+\cdots$,进而有$x-E(a_rt^r+b_{r+1}t^{r+1})=c_{r+2}t^{r+2}+\cdots$,归纳操作下去得到$x$在$E$的像集中.于是我们解释了高阶导数$D$对应的$E$是$k$代数同构.反过来如果$E:A_m\to A_m$是满足$E(a)\equiv a(\mathrm{mod}t),\forall a\in A$的$k$代数同构,它在$A$上的限制就对应了一个高阶导数.这两种对应是互逆的,这就得证.
		
		\qquad
		
		最后设$D=(D_0,D_1,\cdots,D_m)$和$D'=(D_0',D_1',\cdots,D_m')$对应的$A_m\to A_m$的$k$代数同构分别为$E$和$E'$,那么有:
		\begin{align*}
			E(E'(a))&=E(a+D_1'(a)t+D_2'(a)t^2+\cdots)\\&=(a+D_1(a)t+D_2(a)t^2+\cdots)+t(D_1'(a)+D_1(D_1'(a))t+\cdots)\\&=a+(D_1+D_1')(a)t+(D_2+D_1D_1'+D_2')(a)t^2+\cdots
		\end{align*}
	
	    于是$E\circ E'$对应的高阶导数$D''=(D_0'',D_1'',\cdots,D_m'')$就满足$D_i''=\sum_{p+q=i}D_pD_q'$.另外如果要解逆元,就是解$\sum_{p+q=i}D_pD_q^*,i\ge1$.
	\end{proof}
    \item 设$S\subseteq A$和$T\subseteq B$是两个乘性闭子集,满足$g(S)\subseteq T$,那么$g:A\to B$就诱导了$A_S\to B_T$的环同态.任取$A\to B$的一个高阶导数$D=(D_0,D_1,\cdots,D_m)$,它对应于一个$k$代数同态$E:A\to B_m$,把它复合为$A\to B_m\to (B_T)_m$.那么对任意$x\in S$,它在$(B_T)_m$中的像$g(x)_T+D_1(x)_Tt+\cdots$也是一个单位,于是这诱导了$A_S\to(B_T)_m$的$k$代数同态,就对应于$A_S\to B_T$的一个长度也是$m$的高阶导数.它称为$D$在分式化下诱导的高阶导数.
    \item 考虑$A\to B$的一个长度为$m<\infty$的高阶$k$导数$D=(D_0,D_1,\cdots,D_m)$,把它延拓为长度$m+1$的高阶导数也就是把$k$代数同态$E_m:A\to B_m$延拓为$A\to B_{m+1}$的$k$代数同态的问题.这里$(t^{m+1})/(t^{m+2})$是$B_{m+1}=B[t]/(t^{m+2})$的平方为零的理想.于是如果$A$在$k$上是0-光滑的,那么每个$A\to B$的长度有限的高阶导数都可以延拓为长度为无穷大的高阶导数.
    $$\xymatrix{A\ar[rr]\ar@{-->}[drr]&&B[t]/(t^{m+1})\\k\ar[u]\ar[rr]&&B[t]/(t^{m+2})\ar[u]}$$
    \item 如果$A$是特征$p>0$的环,如果$D$是一阶导数,我们知道总有$D(a^p)=0$.但是这对高阶导数未必成立.比方说$D=(D_0,D_1,\cdots,D_m)$是长度$m\ge p$的高阶导数,它对应的$k$同态记作$E:A\to A_m$,那么对任意$a\in A$有$E(a^p)=E(a)^p=a^p+D_1(a)^pt^p+\cdots$,于是有$D_p(a^p)=D_1(a)^p$.一般的有$D_{p^r}(a^{p^r})=D_1(a)^{p^r}$.
\end{enumerate}

迭代导数.一个长度无穷的高阶导数$D=(D_0,D_1,\cdots)\in\mathrm{HS}_k(A)$称为迭代的(iterative),如果对任意$i,j\ge0$都有:
$$D_i\circ D_j=\left(\begin{array}{c}i+j\\i\end{array}\right)D_{i+j}$$
\begin{enumerate}
	\item 设长度无穷的高阶导数$D$对应的$k$代数同态为$E:A\to A[[t]]$,那么$D$是迭代的当且仅当$E$满足如下交换图表,其中$E_t(a)=\sum_iD_i(a)t^i$,$E_u(\sum_ia_it^i)=\sum_iE_u(a_i)t^i$.换句话讲有$E_{t+u}=E_u\circ E_t$.
	$$\xymatrix{A[[t]]\ar[rr]^{E_u}&&A[[t,u]]\\A\ar[u]^{E_t}\ar[rr]^{E_{t+u}}&&A[[t+u]]\ar[u]}$$
	\begin{proof}
		$$E_u(E_t(a))=E_u(\sum_iD_i(a)t^i)=\sum_it^i\sum_ju^jD_jD_i(a)$$
		$$E_{t+u}(a)=\sum_k(t+u)^kD_k(a)=\sum_it^i\sum_ju^j\left(\begin{array}{c}i+j\\i\end{array}\right)D_{i+j}(a)$$
	\end{proof}
    \item 如果$A$包含了域$\mathbb{Q}$,那么长度无穷的高阶导数$D\in\mathrm{HS}_k(A)$是迭代的当且仅当$D_n=D_1^n/n!$.但是如果$A$的特征是$p>0$,一个一阶导数$D_1$要想延拓为迭代高阶导数,就要满足$D_i=D_1^i/i!,i<p$和$D_1^p=0$,这通常未必是一个高阶导数.所以我们不能总指望一个一阶导数可以延拓为高阶迭代导数.
    \begin{proof}
    	
    	一方面如果$D$是迭代导数,对$n$归纳得到$D_n=D_1^n/n!$.反过来对一阶导数$D$,取$(1,D,D^2/2!,\cdots)$是一个迭代导数.
    \end{proof}
    \item 设$\xymatrix{k\ar[r]^f&A\ar[r]^g&B}$是环同态,设$B$在$A$是0-平展,给定长度无穷的$A\to B$的高阶导数$D=(D_0,D_1,\cdots)$,那么存在唯一的高阶导数$D'=(D_0',D_1',\cdots)\in\mathrm{HS}_k(B)$使得$D_i'(g(a))=D_i(a),\forall i$.如果$D^*=(D_0^*,D_1^*,\cdots)\in\mathrm{HS}_k(A)$满足$D_i=g\circ D_i^*,\forall i$,那么如果$D^*$是迭代的,就有$D'$也是迭代的.我们称这里的$D'$是$D$或者$D^*$在$B$上的延拓(尽管$A$未必是$B$的子环).
    \begin{proof}
    	
    	按照定义$D_0'=1_B$,假设$(D_0',D_1',\cdots,D_m')\in\mathrm{HS}_k(B,m)$已经构造,使得$D_i'\circ D_0=D_i,\forall i\le m$.我们定义$h:A\to B_{m+1}$为$a\mapsto\sum_{i=0}^{m+1}D_i(a)t^i$,定义$u:B\to B_m$为$b\mapsto\sum_{i=0}^mD_i'(b)t^i$,于是我们有如下实线交换图表,按照$B$在$A$上0-平展,就存在唯一的虚线提升$A$同态$v$使得图表交换.
    	$$\xymatrix{B\ar[rr]^u\ar@{-->}[drr]&&B_m\\A\ar[u]^g\ar[rr]^h&&B_{m+1}\ar[u]}$$
    	
    	归纳操作下去我们得到满足$D_i'\circ D_0=D_i$的长度无穷的导数$D'$是唯一存在的.接下来设$D^*$是迭代的,设$D^*$和$D'$分别对应的$k$代数同态为$E_t:A\to A[[t]]$和$E_t':B\to B[[t]]$.我们解释过迭代条件就是$E_u\circ E_t=E_{t+u}$.我们要证明的是$E_u'\circ E_t'=E_{t+u}'$.我们来对$m$做归纳,假设$\forall b\in B$有$E_u'(E_t'(b))\equiv E_{t+u}'(b)(\mathrm{mod}(t,u)^{m+1})$.那么有如下实线交换图表,按照$B$在$A$上0-平展得到存在唯一的虚线同态使得图表交换,这说明$E_u'(E_t'(b))\equiv E_{t+u}'(b)(\mathrm{mod}(t,u)^{m+2}),\forall b\in B$成立.于是有$E_u'\circ E_t'=E_{t+u}'$.
    	$$\xymatrix{B\ar[rrrr]^{E_{t+u}'}\ar@{-->}[drrrr]&&&&B[[t,u]]/(t,u)^{m+1}\\A\ar[u]\ar[rr]^{E_{t+u}}&&A[[t,u]]\ar[rr]&&B[[t,u]]/(t,u)^{m+2}\ar[u]}$$
    \end{proof}
    \item 设$A$是特征$p>0$的环,设$x\in A$,设$D\in\mathrm{Der}(A)$满足$D(x)=1$和$D^p=0$.记$A_0=\{a\in A\mid D(a)=0\}$,那么$A$是$A_0$上的以$\{1,x,\cdots,x^{p-1}\}$为基的自由模.
    \begin{proof}
    	
    	假设$\alpha_j\in A_0$使得$\alpha_0+\alpha_1x+\cdots+\alpha_ix^i=0$,其中$i<p$,作用$D^i$得到$i!\alpha_i=0$,于是$\alpha_i=0$,归纳操作下去得到$\{1,x,\cdots,x^{p-1}\}$是线性无关的.另一方面,因为$D^p=0$,说明对任意$a\in A$,我们有$0\le i\le p-1$使得$D^{i+1}a=0$.如果$i=0$,那么$Da=0$,也即$a\in A_0$.我们归纳假设$D^i(a)=0$推出$a\in A_0+A_0x+\cdots+A_0x^{i-1}$,那么如果$D^{i+1}(a)=0$,得到$D^i(a-x^iD^ia/i!)=0$,于是$a-x^iD^ia/i!\in A_0+A_0x+\cdots+A_0x^{i-1}$,于是$a\in A_0+\cdots+A_0x^i$.归纳下去取$i=p-1$得到$A=A_0+A_0x+\cdots+A_0x^{p-1}$.
    \end{proof}
    \item 设$k$是特征$p>0$的域,设$K/k$是可分扩张,设$0\not=D\in\mathrm{Der}_kK$满足$D^p=0$,记$K_0=\{a\in K\mid Da=0\}$,那么存在$x\in K$使得$Dx=1$,并且存在$B_0\subseteq K_0$使得$B=B_0\cup\{x\}$是$K/k$的一组$p$-基.
    \begin{proof}
    	
    	因为$D\not=0$,存在$z\in K$使得$Dz\not=0$,由于$D^pz=0$,所以存在$i$满足$D^iz\not=0$但$D^{i+1}z=0$.记$y=D^iz$,记$x=D^{i-1}z/y$,那么有$Dx=1$.按照上一条,我们有$K=K_0(x)$和$[K:K_0]=p$.如果$x^p\in K_0^pk$,那么$x\in K_0k^{1/p}$,于是可记$x=\sum_{i=1}^n\omega_i\alpha_i$,其中$\omega_i\in K_0$是线性无关的,并且$\alpha_i\in k^{1/p}$.因为$k\subseteq K_0$和$x\not\in K_0$,说明$x,\omega_1,\cdots,\omega_n$在$k$上线性无关.由于$K/k$是可分扩张,导致$x,\omega_1,\cdots,\omega_n$在$k^{1/p}$上线性无关,这和$x=\sum_i\omega_i\alpha_i$矛盾,于是$x^p\not\in K_0^pk$.但是$x^p\in K_0$,于是$K_0/k$的$p$无关组$\{x^p\}$可以扩充为$K_0/k$的一组$p$基$C$.记$B_0=C-\{x^p\}$,如果$y_1,\cdots,y_n$是$B_0$中两两不同的元,那么$[K_0^pk(x^p,y_1,\cdots,y_n):K_0^pk]=p^{n+1}$.但是$K=K_0(x)$,于是$K_0^p(x^p)=K^p$,于是$[K^pk(y_1,\cdots,y_n):K^pk]=p^n$,于是$B_0$是$K/k$的$p$-无关组,于是$B_0$是$K/k$的$p$-无关组,并且有$K=K_0(x)=K_0^pk(x,B_0)$,于是$B_0\cup\{x\}$是$K/k$的一组$p$-基.
    \end{proof}
    \item 设$k$的特征$p>0$,设$K/k$是可分扩张,那么$D\in\mathrm{Der}_k(K)$可以延拓为一个迭代高阶导数当且仅当$D^p=0$.
    \begin{proof}
    	
    	只需证明充分性.不妨设$D\not=0$,按照上一条,记$K_0=\{a\in K\mid Da=0\}$,可取$x\in K$满足$Dx=1$,和一组$B_0\subseteq K_0$,使得$B=\{x\}\cup B_0$是$K/k$的一组$p$基.记$K'=k(B_0)$,那么$D$也是一个$K'$导数(零导数),并且$K$在$K'(x)$上0-平展:一般的,如果$K/k$是特征$p>0$的域扩张,取$p$-基$B$,那么$K$在$k(B)$上0-平展,因为$K=K^pk(B)$,所以提升$v$只要定义在$K^p$上的取值即可,对$a\in K$,取$u(a)$在$C$中的提升$c$,定义$a^p\mapsto c^p$,如果$c+n$是另一个$u(a)$的提升,那么$(c+n)^p=c^p+n^p=c^p$,所以这个映射不依赖$u(a)$提升的选取,并且$a^p$的取值是唯一固定的,所以提升存在且唯一.
    	$$\xymatrix{K\ar[rr]^u\ar@{-->}[drr]&&C/N\\k(B)\ar[u]\ar[rr]&&C\ar[u]}$$
    	
    	如果$x$在$K'=k(B_0)$上代数,那么$x^n+a_1x^{n-1}+\cdots+a_n=0$,其中$a_i$的导数都是零,作用$D^n$得到$1=0$矛盾,于是$x$在$K'$上超越.定义$E_t:K'(x)\to K'(x)[[t]]$为对$\alpha\in K'$有$\alpha\mapsto\alpha$和$x\mapsto x+t$,那么有$E_u(E_t(x))=x+u+t=E_{t+u}(x)$.于是$E_u\circ E_t=E_{t+u}$在整个$K'(x)$上成立.于是$E_t$定义了$K'(x)/K'$的迭代高阶导数$D^*$.又因为$K$在$K'(x)$上0-平展,于是$D^*$可以延拓为$K$在$K'$上的迭代高阶导数$D'$,按照延拓的唯一性,这里$D'_1=D$,所以$D'$就是$D$延拓的迭代高阶导数.
    \end{proof}
\end{enumerate}









