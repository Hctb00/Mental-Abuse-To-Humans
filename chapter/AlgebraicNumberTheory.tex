\chapter{代数数论}
\section{整体域}
\subsection{数域和数环}

我们要讨论的基本结构是这样的:首先给定一个正规整环$A$,记它的商域是$K$,考虑$K$的有限扩张$L$(在某些情况下会添加可分条件),设$A$在$L$中的正规化为$B$,它称为$L$中的整元.整理一下基本性质:
\begin{enumerate}
	\item 如下条件互相等价.
	\begin{enumerate}
		\item $x\in L$是$A$上的整元.
		\item $L$的子环$A[x]$是$A$的有限模.
		\item 存在$L$的子环$L'$包含了$x$,使得$L'$是有限$A$模.
	\end{enumerate}
    \item 给定$K$的域扩张$K\subseteq L\subseteq F$,设$B$在$F$中的正规化为$C$,那么$C$也是$A$在$F$中的正规化.
    \item $x\in L$是$K$上的代数元,设它的极小多项式为$f(x)$,那么$x$是$A$上的整元当且仅当$f(x)\in A[x]$.
    \begin{proof}
    	
    	设$f(x)$在$L$的某个固定的代数闭包中的全部根为$\{x_1,x_2,\cdots,x_n\}$,记$x=x_1$,那么$A$模$A[x_1]$都同构于$A$模$A[x_i]$,于是每个$x_i$都是$A$上的整元,于是$A[x_1,x_2,\cdots,x_n]$是$A$的整扩张.现在$f$的系数落在这个环里,于是$f$的系数是$L$中的在$A$上的整元,于是按照正规性得到$f[x]\in A[x]$.
    \end{proof}
    \item 任取$x\in L$,那么存在$a\in A$使得$ax\in B$.事实上设$x$满足的极小多项式为$x^n+(a_1/b_1)x^{n-1}+\cdots+(a_n/b_n)=0$,其中$a_i,b_i\in A$,取$a=b_1b_2\cdots b_n$,那么有合适的$c_i\in A$使得$(ax)^n+c_1(ax)^{n-1}+\cdots+c_n=0$,于是$ax\in B$.
\end{enumerate}

数域的情况.
\begin{enumerate}
	\item 取正规整环$\mathbb{Z}$,它的商域是有理数域$\mathbb{Q}$,设$\mathbb{Q}\subseteq K$是$n$次域扩张,此时称$K$是一个$n$次数域.把$\mathbb{Z}$在$K$中的正规化记作$\mathscr{O}_K$,它称为数域$K$的代数整数环.
	\item 一个注解.通常我们探究一个域的代数扩张的时候,可以预先取定一个代数闭包,然后把代数扩张全部视为这个固定代数闭包的子域.在数域的情况下,$\mathbb{Q}$的一个代数闭包可取为$\mathbb{C}$的子域$\overline{\mathbb{Q}}$,它由复数域上的全体代数数构成.今后我们讨论数域的时候,不妨总约定它视为复数域的子域.
	\item 取数域$K$,那么$\mathscr{O}_K$是一个正规整环,取$K$的有限扩张$L$,那么$\mathscr{O}_K$在$L$中的正规化恰好就是$\mathscr{O}_L$(整扩张的传递性).
\end{enumerate}

数域上的实嵌入和复嵌入.
\begin{enumerate}
	\item 数域总是特征零的域,于是数域都是完美域(代数扩张都是可分扩张),数域之间的扩张总是有限扩张.于是按照本原元定理,数域之间的扩张总是单扩张.于是可记$L=K(\alpha)$,如果扩张维数是$n$,那么$K(\alpha)$作为$K$上线性空间的一组基可取为$\{1_K,\alpha,\cdots,\alpha^{n-1}\}$.
	\item 嵌入.按照域论的内容,由于数域之间的代数扩张都是可分扩张,说明如果$K\subseteq L$是数域之间的$n$次扩张,那么恰好有$n$个不同的$K$嵌入$L\to\mathbb{C}$.如果设$L=K(\alpha)$,那么$\alpha$满足一个$n$次的$K$上不可约多项式,它在$\mathbb{C}$上有$n$个不同的根,这$n$个不同的$K$嵌入恰好就是把$\alpha$映射为这$n$个不同根得到的.
	\item 特别的,设$K$是一个$n$次数域,可记$K=\mathbb{Q}(\alpha)$,记$\alpha$的极小多项式是一个$n$次的,它在$\mathbb{C}$中的$n$个不同根分别对应了一个$\mathbb{Q}$嵌入$K\to\mathbb{C}$.对于实根,它对应的嵌入的像集落在$\mathbb{R}$中,这样的嵌入称为实嵌入.对于非实根,我们知道它们总是成对出现的,并且它们对应的$\mathbb{Q}$嵌入$K\to\mathbb{Q}$是互相共轭的,这样的嵌入称为复嵌入.于是对于$n$次数域$K$,设它有$r_1$个实嵌入,有$r_2$对复嵌入,那么$n=r_1+2r_2$,并且这里$(r_1,r_2)$是数域$K$的一个不变量.
	\item 如果数域之间的扩张$K\subseteq L$是Galois扩张,按照特征零,这也等价于扩张是正规的,按照正规性的一个等价描述,这时候所有$K$嵌入$L\to\mathbb{C}$的像集是相同的,这说明此时嵌入要么都是实嵌入,要么都是复嵌入.换句话讲此时$(r_1,r_2)$要么是$(n,0)$要么是$(0,n/2)$.
\end{enumerate}

域扩张上的迹和范数.给定有限扩张$K\subseteq L$,任取$x\in L$,那么数乘$x$是$K$线性空间$L$上的线性变换,这个线性变换的迹和范数称为这个元$x$的迹和范数,分别记作$\mathrm{T}_{L/K}(x)$和$\mathrm{N}_{L/K}(x)$.
\begin{enumerate}
	\item 设$K\subseteq L$是$n$次可分扩张,任取$x\in L$,记它在$K$上的极小多项式为$p_x(t)$,设它的次数为$m$,那么它的特征多项式是$p_x(t)^{n/m}$.它的全部特征值也即$x$的全部$m$个共轭元.
	\item 设$K\subseteq L$是$n$次可分扩张,任取$x\in L$,设$L\to\mathbb{C}$的全部$K$嵌入为$\{\sigma_1,\sigma_2,\cdots,\sigma_n\}$.那么$\mathrm{T}_{L/K}(x)=\sum_{1\le i\le n}\sigma_i(x)$和$\mathrm{T}_{L/K}(x)=\prod_{1\le i\le n}\sigma_i(x)$.也可以用特征多项式给出迹和范数的表达式.于是特别的迹和范数都是基域$K$上的元.
	\item 如果有有限的域扩张链$k\subseteq L\subseteq F$,那么对任意$x\in F$有$\mathrm{T}_{L/k}(\mathrm{T}_{F/L}(x))=\mathrm{T}_{F/k}(x)$和$\mathrm{N}_{L/k}(\mathrm{N}_{F/L}(x))=\mathrm{N}_{F/k}(x)$.
	\item 设$K\subseteq L$是$n$次可分扩张,记$L\to\mathbb{C}$的全部$k$嵌入为$\{\sigma_1,\sigma_2,\cdots,\sigma_n\}$,任取$L$中的$n$个元$a_1,a_2,\cdots,a_n$,那么有如下等式成立,另外这个等式非零当且仅当$a_1,a_2,\cdots,a_n$构成了$K$线性空间$L$上的一组基.
	$$d_{L/K}(a_1,a_2,\cdots,a_n)=\det((\sigma_i(a_j))^2)=\det(\mathrm{T}_{L/K}(a_ia_j))=\prod_{1\le i<j\le n}\left(\sigma_i(\alpha)-\sigma_j(\alpha)\right)^2$$
	$$=(-1)^{\frac{n(n-1)}{2}}\mathrm{N}_{L/K}(f'(\alpha))$$
	\item 如果$A$是正规整环,商域记作$K$,取$K$的有限可分扩张$L$,取$A$在$L$中的正规化为$B$,那么$B$中元的迹和范数都是$A$中的元.
	\item 设$K\subseteq L$是有限可分扩张,设$x\in B$,那么$x\in B^*$当且仅当$\mathrm{N}_{L/K}(x)\in A^*$.
	\begin{proof}
		
		这里必要性是范数的乘性,反过来如果存在$a\in A$使得$a\mathrm{N}_{L/K}(x)=1$,记$L\to\overline{L}$的全部$K$嵌入为$\{\sigma_1,\sigma_2,\cdots,\sigma_n\}$,那么有$a\prod_i\sigma_i(x)=1$,可把它整理为$xy=1$,这些$\sigma_i(x)$都是$A$上的整元,于是$y$也是$A$上整元,而$y=x^{-1}\in L$,于是$y\in B$,于是$x\in B^*$.
	\end{proof}
    \item 设扩张$K\subseteq K(\alpha)$是一个$n$次可分扩张,设全部$K(\alpha)\to\overline{K}$的$K$嵌入为$\{\sigma_1,\sigma_2,\cdots,\sigma_n\}$,设$f(x)$是$\alpha$的(首一)极小多项式,那么判别式和极小多项式的判别式是相同的,并且可用结式$R(f,g)$表示:
    $$d(1,\alpha,\alpha^2,\cdots,\alpha^{n-1})=\det((\sigma_i(\alpha)^j)^2)=\prod_{1\le i<j\le n}(\sigma_i(\alpha)-\sigma_j(\alpha))^2=(-1)^{\frac{n(n-1)}{2}}|R(f,f')|$$
\end{enumerate}

整基.设$A$是正规整环,商域为$K$,取$K$的有限扩张$L$,设$A$在$L$中的正规化为$B$,那么一般来讲$B$未必是$A$自由模,如果它是,此时的一组基称为$B$在$A$上的整基,也称为有限扩张$K\subseteq L$的整基.如果$B$上存在整基$\{b_1,b_2,\cdots,b_n\}$,那么$d_{L/K}(b_1,b_2,\cdots,b_n)$称为这组整基的判别式.
\begin{enumerate}
	\item $B$在$A$上的整基自然也是$L$在$K$上的基.
	\item 整基的判别式一定是$A$中的元,因为我们证明过这组整基的判别式可以写作$\det(\mathrm{T}_{L/K}(b_ib_j))$,每个$b_ib_j\in B$,于是迹落在$A$中.
	\item 设$A$是一个PID,于是它也是正规整环,设商域为$K$,设$K\subseteq L$是有限可分扩张,设$A$在$L$中的正规化为$B$,那么每个$B$模$L$的非零有限生成子模都是秩为$[L:K]=n$的自由$A$模.特别的,$B$是秩为$[L:K]$的自由$A$模.于是此时$B$上具有整基.
	\begin{proof}
		
		取$K\subseteq L$的一组基为$\alpha_1,\cdots,\alpha_n$.那么其中每个元都是$K$中代数元,对每个$\alpha_i$,可以找到一个$A$中非0元$\beta_i$使得$\beta_i\alpha_i$满足$A$系数首一多项式,于是它们都在$B$中,记$b=\prod\beta_i$,那么$b\alpha_i$都在$B$中,于是可以不妨设所取的$L$的一组$K$基$\alpha_1,\alpha_2,\cdots,\alpha_n$都是$B$中的元.
		
		现在我们断言,如果记$d=d(\alpha_1,\alpha_2,\cdots,\alpha_n)$是判别式,那么$dB\subseteq A\alpha_1+A\alpha_2+\cdots+A\alpha_n$.任取$\alpha=\sum_ia_i\alpha_i$,其中$a_i\in A$,于是$a_i$是线性方程组$\mathrm{T}_{L/K}(\alpha\alpha_i)=\sum_j\mathrm{T}_{L/K}(\alpha_i\alpha_j)a_j$的解.按照克莱姆法则,就有$a_j=c_j/d$,其中$c_j\in A$,此即$dB\subseteq A\alpha_1+A\alpha_2+\cdots+A\alpha_n$.
		
		现在设$M$是$B$模$L$的有限生成子模.按照$A$是PID,从$dB\subseteq A\alpha_1+A\alpha_2+\cdots+A\alpha_n=B_0$,右侧是秩$n$自由$A$模得到$B\cong dB$是秩不超过$n$的自由$A$模.但是$\alpha_1,\alpha_2,\cdots,\alpha_n\in B$是$A$线性无关的,这得到$B$作为$A$自由模的秩恰好是$n$.
		
		现在设$\mu_1,\mu_2,\cdots,\mu_r$是$B$模$M$的一组生成元集.那我们知道存在$a\in A$使得每个$a\mu_i\in B$.于是$aM\subseteq B$,于是$adM\subseteq dB\subseteq B_0$.于是$M\cong adM$是一个秩不超过$n$的自由$A$模.最后按照$n=\mathrm{rank}(B)\le\mathrm{rank}(M)=\mathrm{rank}(adM)\le\mathrm{rank}(B_0)=n$,就得到$\mathrm{rank}(M)=n$.
	\end{proof}
    \item 设$A$是诺特正规整环(特别的戴德金整环满足这个条件),设$A$的商域$K$,取$K$的有限可分扩张$L$,设$A$在$L$中的正规化为$B$,尽管此时$B$不一定是$A$自由模,但是我们可以证明此时$B$是有限$A$模.
    \begin{proof}
    	
    	首先取$K$线性空间$L$的一组基$\{u_1,u_2,\cdots,u_n\}$,不妨约定$u_i\in B$(否则可取某个$A$中的元$d$使得每个$du_i\in B$).按照扩张是有限可分的,说明迹二次型$(x,y)=\mathrm{T}_{F/k}(xy):F\times F\to K$是非退化的,于是可取这组基$\{u_1,u_2,\cdots,u_n\}$的对偶基$\{w_1,w_2,\cdots,w_n\}\subseteq F$,此即满足$(u_i,w_j)=\delta_{ij}$的一组基.
    	
    	现在我们断言$B\subseteq C=Aw_1+Aw_2+\cdots+Aw_n$,导致$B$是$A$模$C$的子模,按照$C$是诺特环上的有限生成模,说明它是诺特模,于是$B$作为$A$模是有限生成的.
    	
    	任取$x\in B$,可记$x=\sum_jx_jw_j$,其中$x_j\in K$,需要证明每个$x_j\in A$.按照$xv_i\in B$,说明$\mathrm{T}_{F/k}(xv_i)\in A$.但是$\mathrm{T}_{F/k}(xv_i)=\sum_j\mathrm{T}_{F/k}(x_jw_jv_j)=\sum_jx_j\mathrm{T}_{F/k}(v_jw_i)=x_i$,于是每个$x_i\in A$.
    \end{proof}
    \item 定理.设$K\subseteq L$和$K\subseteq L'$是两个有限可分扩张,次数分别是$n$和$n'$,满足$[LL':K]=nn'$.设$\{\omega_1,\omega_2,\cdots,\omega_n\}$和$\{\omega_1',\omega_2',\cdots,\omega_n'\}$分别是$K\subseteq L$和$K\subseteq L'$的整基,并且判别式分别是$d$和$d'$.倘若$(d)+(d')=A$,那么$\{\omega_i\omega_j',1\le i\le n,1\le j\le n'\}$是$K\subseteq LL'$的整基,如果记$A$在$L$和$L'$中的正规化分别是$B$和$B'$,那么$BB'$也就是$A$在$LL'$中的正规化,并且这组整基的判别式是$d^{n'}(d')^n$.
    \begin{proof}
    	
    	按照$[LL':K]=nn'$,于是$\{\omega_i\omega_j',1\le i\le n,1\le j\le n'\}$的确是$K\subseteq LL'$的一组基.任取$LL'$中的$A$整元$\alpha$,可记$\alpha=\sum_{i,j}a_{ij}\omega_i\omega_j'$,其中$a_{ij}\in K$.我们需要证明每个$a_{ij}\in A$.
    	
    	记$\beta_j=\sum_ia_{ij}\omega_i$,记$LL'$到代数闭包的$L'$嵌入群和$LL'$到代数闭包的$L$嵌入群分别是$\{\sigma_1,\sigma_2,\cdots,\sigma_n\}$和$\{\sigma_1',\sigma_2',\cdots,\sigma_{n'}'\}$.于是$LL'$到代数闭包的$k$嵌入群就是$\mathrm{Gal}_{LL'/K}=\{\sigma_s\sigma_t'\mid 1\le s\le n,1\le t\le n'\}$.
    	
    	记$T=(\sigma_t'(\omega_j'))$,$a=(\sigma_1'(\alpha),\sigma_2'(\alpha),\cdots,\sigma_{n'}(\alpha))^t$和$b=(\beta_1,\beta_2,\cdots,\beta_{n'})^t$.于是有$\det T^2=d'$和$a=Tb$.
    	
    	设$T^*$是$T$的伴随矩阵,那么有$(\det T)b=T^*a$,按照$T^*$和$a$中的元都是整元,说明$d'b$中的元都是$L$中的整元,也即每个$d'\beta_j=\sum_id'a_{ij}\omega_i$是$L$中的整元.于是$d'a_{ij}\in A$.同理得到$da_{ij}\in A$,最后按照$d$和$d'$是互素的,就有$xd+x'd'=1_A$,于是$a_{ij}\in A$.
    	
    	最后我们说明这组整基的判别式是$d^{n'}(d')^n$.这个判别式是一个$nn'$方阵$M=(\sigma_s\omega_i\sigma_t'\omega_j')$的行列式,这是$n$阶矩阵$(\sigma_s\omega_i)$和$(\sigma_t'\omega_j')$的张量积,张量积的行列式是$d^{n'}(d')^n$.
    \end{proof}
    \item 注意如下每个条件可以推出上述定理中的$[LL':K]=nn'$:
    \begin{enumerate}
    	\item $K\subseteq L$和$K\subseteq L'$都是有限Galois扩张,并且$L\cap L'=K$.
    	\item $L$和$L'$关于$k$是线性无交的(线性无交条件等价于$[LL':K]=[L:K][L':K]$).
    \end{enumerate}
\end{enumerate}

数域上的整基和判别式.按照$\mathbb{Z}$是PID,说明每个代数整数环$\mathscr{O}_K$都存在整基.由于$\mathbb{Z}$上单位的平方都是1,导致数域上不同的整基具有相同的判别式,它称为这个代数整数环或者这个数域的判别式,记作$d_K$或者$d(\mathscr{O}_K)$.
\begin{enumerate}
	\item 任取$\mathscr{O}_K$中的$n$个元$\{b_1,b_2,\cdots,b_n\}$,那么$d_K(b_1,b_2,\cdots,b_n)$是$d_K$乘以一个平方有理整数,倘若$d_K(b_1,b_2,\cdots,b_n)=d_K$,那么$\{b_1,b_2,\cdots,b_n\}$是一组整基.
	\item Stickelberger准则.数域$L$的判别式满足$D(L)\equiv 0,1(\mathrm{mod} 4)$.
	\begin{proof}
		
		取$\alpha_1,\cdots,\alpha_n$是$L$上的一个整基,那么判别式就是$D(L)=\det(\sigma_i\alpha_j)^2$.取$M$为包含$L$的正规域,那么它是Galois域,现在把每个$\sigma_i$延拓为$M$上的自同构,仍赋予相同记号.现在行列式$\det(\sigma_i\alpha_j)$是$n!$项的和,每一项对应着$\{1,2,\cdots,n\}$的一个排列.取偶排列的和是$P$,奇数排列赋值的和是$N$,那么$D(L)=(P-N)^2=(P+N)^2-4PN$,注意到$\sigma\in\mathrm{Aut}_{\mathbb{Q}}M$的作用相当于重排行列式中行的次序,于是$P,N$要么不变要么交换数值,这导致$P+N$和$PN$不变,于是它们是有理整数,于是$D(L)\equiv 0,1(\mathrm{mod} 4)$.
	\end{proof}
    \item 数域$K$的判别式$d_K$的符号为$(-1)^{r_2}$,这里$r_2$表示$K$到$\mathbb{C}$恰有$r_2$对复嵌入.
    \begin{proof}
    	
    	我们把行列式$\det(\sigma_i\alpha_j)$中互为共轭的$\sigma_i$所在的行交换,按照行列式的性质,得到$\overline{\det(\sigma_i\alpha_j)}=\det(\overline{\sigma_i\alpha_j})=(-1)^{r_2}\det(\sigma_i\alpha_j)$,于是$r_2$为偶数对应$\det(\sigma_i\alpha_j)$是实数,于是$d_K=\det(\sigma_i\alpha_j)^2\ge0$;$r_2$是奇数对应$\det(\sigma_i\alpha_j)$是纯虚数,于是$d_K=|\det(\sigma_i\alpha_j)|^2<0$.
    \end{proof}
    \item 引理.设$b_1,b_2,\cdots,b_n\in\mathscr{O}_K$是$K$上一组$\mathbb{Q}$基,那么它不是一组整基当且仅当存在有理素数$p$使得$p^2\mid d_K(b_1,b_2,\cdots,b_n)$,并且存在不全为零的$x_i\in\{0,1,\cdots,p-1\}$使得$\sum_{1\le i\le n}x_ib_i\in p\mathscr{O}_K$.
    \begin{proof}
    	
    	取一组整基$\{a_1,a_2,\cdots,a_n\}$,可记有有理整数上的矩阵$C$使得$(b_1,b_2,\cdots,b_n)=(a_1,a_2,\cdots,a_n)C$.按照$\{b_1,b_2,\cdots,b_n\}$构成了$K$上的一组基,说明这里的$\det C\not=0$.于是$\{b_i\}$构成一组整基当且仅当这里$\det C=\pm1$.
    	
    	现在假设它不是一组整基,那么可取有理素数$p\mid\det C$,于是$p^2\mid d_K(b_1,b_2,\cdots,b_n)=(\det C)^2d_K$.记$\overline{C}$是$C$在$\mathrm{mod}p$下的像.那么$C$的行列式在$\mathrm{mod}p$下为零,于是在$\mathbb{F}_p$中存在非零解$(\overline{x_1},\overline{x_2},\cdots,\overline{x_n})$.每个$\overline{x_i}$唯一的提升为$\{0,1,2,\cdots,p-1\}$中的元$x_i$.于是得到$\sum_ix_ib_i\in p\mathscr{O}_K$.
    	
    	反过来假设存在这组不全为零的$x_i\in\{0,1,2,\cdots,p-1\}$使得$\sum_ix_ib_i\in p\mathscr{O}_K$,那么$C$在$\mathrm{mod}p$下的像$\overline{C}$存在非零解,于是$\det C$被$p$整除,于是$\{b_1,b_2,\cdots,b_n\}$不是一组整基.
    \end{proof}
    \item 定理.设$\alpha\in\mathscr{O}_K$使得$K=\mathbb{\alpha}$,设$f(T)\in\mathbb{Z}[T]$是$\alpha$的极小多项式.假设对每个满足$p^2\mid d_K(1,\alpha,\alpha^2,\cdots,\alpha^{n-1})$的有理素数$p$,都存在一个整数$i$使得$f(T+i)$是一个关于$p$的爱森斯坦多项式(这是指这个多项式$T^n+a_1T^{n-1}+\cdots+a_n$满足$p\mid a_i$和$p^2\not| a_n$),那么$\mathscr{O}_K=\mathbb{Z}[\alpha]$.
    \begin{proof}
    	
    	按照上一定理,以及$\mathbb{Z}[\alpha]=\mathbb{Z}[\alpha-i]$,只需说明如果$\alpha$的极小多项式$f(T)$是关于有理素数$p$的爱森斯坦多项式,那么对任意不全为零的$x_i\in\{0,1,\cdots,p-1\}$,都有$x=\frac{1}{p}\sum_{0\le i\le n-1}x_i\alpha^i\not\in\mathscr{O}_K$.设$j=\min\{i\mid x_i\not=0\}$,那么有$\mathrm{N}_{K/\mathbb{Q}}(x)=\frac{\mathrm{N}_{K/\mathbb{Q}}(\alpha)^j}{p^n}\mathrm{N}_{K/\mathbb{Q}}(\sum_{j\le i\le n-1}x_i\alpha^{i-j})$.我们只需证明这里$\mathrm{N}_{K/\mathbb{Q}}(\sum_{j\le i\le n-1}x_i\alpha^{i-j})\equiv x_j^n(\mathrm{mod}p)$,按照$p^2\not|\mathrm{N}_{K/\mathbb{Q}}(x)$,这就导致$\mathrm{N}_{K/\mathbb{Q}}(x)\not\in\mathbb{Z}$,从而$\alpha\not\in\mathscr{O}_K$.最后证明这个断言,我们有:
    	
    	$$\mathrm{N}_{K/\mathbb{Q}}(\sum_{j\le i\le n-1}x_i\alpha^{i-j})=\prod_{1\le k\le n}\left(x_j+x_{j+1}\sigma_k(\alpha)^{i-j}+\cdots+x_{n-1}\sigma_k(\alpha)^{n-1}\right)$$
    	
    	右侧展开式中除了$x_j^n$以外的项的和可以表示为$\alpha_1,\alpha_2,\cdots,\alpha_n$的若干对称多项式之和,而按照条件初等对称多项式在$\mathrm{mod}p$下都是零,这就得证.
    \end{proof}
    \item 一个例子.设$K=\mathbb{Q}(\alpha)$,其中$\alpha^3=2$,计算可得$d_K(1,\alpha,\alpha^2)=-3^22^2$.但是$p=2$时$f(T)=T^3-2$是2-爱森斯坦多项式;$p=3$时$f(T-1)=T^3-3T^2+3T-3$是3-爱森斯坦多项式,于是上一定理说明$\mathscr{O}_K=\mathbb{Z}[\alpha]$.
\end{enumerate}

数环的乘积.给定两个数域$K,L$,那么一般只有$\mathscr{O}_K\mathscr{O}_L$,另一侧的包含关系不成立.假设$K\cap L=\mathbb{Q}$,记$d_K$和$d_L$的最大公约数是$d$,那么有$\mathscr{O}_{KL}\subset\frac{1}{d}\mathscr{O}_K\mathscr{O}_L$.特别的,如果$d_K$和$d_L$互素,那么$\mathscr{O}_K\mathscr{O}_L=\mathscr{O}_{KL}$.
\begin{proof}
	
	设$\{a_1,a_2,\cdots,a_n\}$和$\{b_1,b_2,\cdots,b_m\}$分别是$K$和$L$上的整基.任取$x\in\mathscr{O}_{KL}$,那么有$x=\sum_{i,j}\frac{x_{ij}}{r}a_ib_j$,其中$m_{ij}$和$r\in\mathbb{Z}$,并且可约定$(m_{ij})=1$.只需验证$r\mid d$.按照对称性,只需验证$r\mid d_L$.设$(a_i^{\vee})$表示$(a_i)$关于迹二次型的对偶基(存在性因为有限可分扩张下迹二次型是非退化的),那么有:
	$$\mathrm{T}_{KL/L}(a_i^{\vee}x)=\sum_{k,l}\mathrm{T}(a_i^{\vee}a_kb_l)=\sum_l\frac{x_{i,l}}{r}b_l$$
	
	现在按照对偶基的定义和Cramer法则,说明每个$a_i^{\vee}\in\frac{1}{d_K}\mathscr{O}_K$.于是$xa_i^{\vee}\in\frac{1}{d_K}\mathscr{O}_{KL}$.于是有$\mathrm{T}_{KL/L}(xa_i^{\vee})\in\frac{1}{d_K}\mathrm{T}_{KL/L}(\mathscr{O}_{KL})\subset\frac{1}{d_K}\mathscr{O}_L$.这说明每个$d_K\frac{x_{i,l}}{r}\in\mathbb{Z}$,按照全体$x_{i,l}$互素,得到$r\mid d_K$.
\end{proof}

二次数域.按照定义,二次数域是复数域的子域,使得它是$\mathbb{Q}$的二次扩张.
\begin{enumerate}
	\item 按照本原元定理,二次数域可以表示为$\mathbb{Q}(\alpha)$,这里$\alpha$是一个二次不可约有理系数多项式的复根.按照求根公式,容易得出二次数域可以整理为$\mathbb{Q}(\sqrt{m})$的形式,这里$m$是一个不含平方因子的$\not=0,1$的整数.另外容易看出这个二次不可约多项式的另一个根也落在扩域里,于是二次数域总是$\mathbb{Q}$的一个Galois扩张.
	\item 给定二次数域$\mathbb{Q}(\sqrt{m})$,这里$m>1$当且仅当它是它没有复嵌入,有两个实嵌入,此时称为实二次数域;$m<0$当且仅当它有两个互相共轭的复嵌入,没有实嵌入,此时称为虚二次数域.另外这里非平凡的嵌入总可以表示为$a+b\sqrt{m}\mapsto a-b\sqrt{m}$.
	\item 下一件事是求出二次数域上代数整数环.对此我们有如下结论:记二次数域$K=\mathbb{Q}(\sqrt{m})$,其中$m$没有平方因子,那么$O_K$具有如下描述:
	\begin{enumerate}
		\item 当$m\equiv2,3(\mathrm{mod}4)$时候,有$O_K=\{a+b\sqrt{m}\mid a,b\in\mathbb{Z}\}=\mathbb{Z}[\sqrt{m}]$.此时$\{1,\sqrt{m}\}$是一组整基,此时数域的判别式是$\left|\begin{array}{cc}1&\sqrt{m}\\1&-\sqrt{m}\end{array}\right|^2=4|m|$.
		\item 当$m\equiv1(\mathrm{mod}4)$时候,有$O_K=\{a+b\left(\frac{1+\sqrt{m}}{2}\right)\mid a,b\in\mathbb{Z}\}=\mathbb{Z}[\frac{1+\sqrt{m}}{2}]$.此时$\{1,\frac{1+\sqrt{m}}{2}\}$是一组整基,此时数域的判别式是$\left|\begin{array}{cc}1&\frac{1+\sqrt{m}}{2}\\1&\frac{1+\sqrt{m}}{2}\end{array}\right|^2=|m|$
	\end{enumerate}
	\begin{proof}
		
		任取$\alpha=a+b\sqrt{m}\in K$,其中$a,b$都是有理数,设它是一个代数整数,那么按照二次数域必然是Galois扩张,并且非平凡的$\mathbb{Q}$自同构只有$a+\sqrt{m}b\mapsto a-\sqrt{m}b$,看到$a-b\sqrt{m}$也是代数整数.于是$a+b\sqrt{m}$是代数整数当且仅当$\alpha+\overline{\alpha}=2a$和$\alpha\overline{\alpha}=a^2-mb^2$都是有理整数.那么有$m(2b)^2=(2a)^2-4(a^2-mb^2)\in\mathbb{Z}$.倘若$2b$不是有理整数,那么它的既约表示的分母必然包含平方因子,但是$m$不包含平方因子,乘积不会成为有理整数,于是$2b$必然是有理整数.于是可以记$a=u/2,b=v/2$,其中$u,v\in\mathbb{Z}$.那么$a^2-mb^2\in\mathbb{Z}$变为$u^2-mv^2\in4\mathbb{Z}$.此时$a+b\sqrt{m}$是有理整数当且仅当$u^2\equiv mv^2(\mathrm{mod}4)$.
		
		现在分两种情况,如果$m\equiv2,3(\mathrm{mod}4)$,那么$u^2\equiv mv^2(\mathrm{mod}4)$成立当且仅当$u,v$都是偶数,于是此时$a+b\sqrt{m}$是代数整数当且仅当$a,b$是有理整数.
		
		如果$m\equiv1(\mathrm{mod}4)$,那么$u^2-mv^2\equiv0(\mathrm{mod}4)$成立当且仅当$u,v$是同奇偶性的有理整数,于是此时$a+b\sqrt{m}$是代数整数当且仅当它可以表示为$c+d\left(\frac{1+\sqrt{m}}{2}\right)$的形式,其中$c,d$是有理整数.
	\end{proof}
\end{enumerate}

分圆域.按照定义分圆域是指$\mathbb{Q}(\zeta_n)$,其中$\zeta_n$是$\mathbb{Q}$上的$n$次本原根.域论里我们证明过$\mathbb{Q}\subset\mathbb{Q}(\zeta_n)$是一个$\varphi(n)$次的Galois扩张.它是分圆多项式$\Phi_n(x)=\prod_{1\le k\le n,(k,n)=1}(x-\zeta_n^k)$的分裂域.(我们证明过分圆多项式总是整系数不可约多项式).于是分圆域也是一个数域.现在我们来求它的代数整数环,整基和判别式.
\begin{enumerate}
	\item 引理.给定$K$的两个分别$n$次和$n'$次的Galois扩张$L,L'$,设$L\cap L'=K$.记它们分别的整基为$\{\omega_1,\cdots,\omega_n\}$和$\{\omega_1',\cdots,\omega_n'\}$.如果两个判别式$d,d'$在$\mathbb{Z}$中互素,即$(d)+(d')=\mathbb{Z}$.那么$\{\omega_i\omega_j',1\le i,j\le n\}$是$LL'$的一组整基.于是$\mathscr{O}_L\mathscr{O}_{L'}=\mathscr{O}_{LL'}$,另外有这组$LL'$整基的判别式为$d^{n'}(d')^{n}$.
	\item 分圆域$K=\mathbb{Q}(\xi_{n})$的代数整数环就是$\mathbb{Z}[\xi_n]$,它的一组整基是$\{1,\xi_n,\cdots,\xi_n^{\varphi(n)-1}\}$.判别式为$(-1)^{\varphi(n)/2}n^{\varphi(m)}/\prod_{p\mid n}p^{\varphi(n)/(p-1)}$.
	\begin{proof}
		
		先证明素数幂的情况.按照定义$\xi_n$的极小多项式为$\Phi_n(T)=\prod_{a\in(\mathbb{Z}/n\mathbb{Z})^*}(T-\xi_n^a)$.于是有$T^n-1=\prod_{m\mid n}\Phi_m(T)$.按照M\"obius反演公式,得到$\Phi_n(T)=\prod_{m\mid n}(T^m-1)^{\mu(n/m)}$,其中$\mu$是M\"obius函数.于是当$n$是素数幂的时候,就有$\Phi_{p^n}(T)=\frac{T^{p^n}-1}{T^{p^{n-1}}-1}$.
		
		我们来证明$\Phi(T+1)$是一个关于素数$p$的爱森斯坦多项式,这就说明$\xi_n$生成了循环整基.事实上按照$\Phi(T+1)((T+1)^{p^{n-1}}-1)=((T+1)^{p^n}-1)$.在$\mathrm{mod}p$下就有$\overline{\Phi}(T+1)T^{p^{n-1}}=T^{p^n}$,此即$\Phi(T+1)$的全部非首系数都被$p$整除.最后$\Phi(T+1)=\sum_{0\le i\le p-1}(T+1)^{ip^n}$的常数项就是$p$,它不被$p^2$整除,这说明$\Phi(T+1)$是爱森斯坦多项式.
		
		再求判别式.首先考虑$\alpha=\xi_p$的判别式,有:
		\begin{align*}
		d(K)&=(-1)^{\frac{\varphi(p)(\varphi(p)-1)}{2}}\prod\limits_{\substack{i\not=j\\1\le i,j\le p-1}}\left(\xi_p^i-\xi_p^j\right)=(-1)^{\frac{p(p-1)}{2}}\prod_{1\le i\le p-1}\prod_{j\not=i}(1-\xi_p^{j-i})\\&=(-1)^{\frac{p(p-1)}{2}}\left(\prod_{1\le j\le p-1}(1-\xi_p^j)\right)^{p-2}=(-1)^{\frac{p(p-1)}{2}}p^{p-2}
		\end{align*}
		
		再求$\alpha=\xi_{p^n}$的判别式,我们已经给出了它的极小多项式为$f(T)=\frac{T^{p^n}-1}{T^{p^{n-1}}-1}=\sum_{i=0}^{p-1}T^{p^{n-1}i}$,那么有:
		$$f'(T)=\sum_{i=1}^{p-1}p^{n-1}iT^{p^{n-1}i-1}=p^{n-1}T^{p^{n-1}-1}\sum_{i=1}^{p-1}T^{p^{n-1}(i-1)}i$$
		
		带入$T=\alpha$,就得到:$f'(\alpha)=p^{n-1}\alpha^{p^{n-1}-1}\sum_{i=1}^{p-1}\xi_p^{i-1}i$,但是按照我们计算的$\xi_p$的判别式,得到$\mathrm{N}(\sum_{i=1}^{p-1}\xi_p^{i-1}i)=(-1)^{\frac{p(p-1)}{2}}p^{p-2}$.整理得到判别式为$(-1)^{p^{n-1}(p-1)}p^{p^{n-1}(np-n-1)}$.
		
		再证一般情况,为此我们来对$n$的不同素因子个数做归纳.假设$K=\mathbb{Q}(\zeta_n)$,其中$n=p_1^{n_1}p_2^{n_2}$,其中$p_1$和$p_2$是两个不同素数,并且$n_1,n_2\ge1$.取$K_1=\mathbb{Q}(\zeta_{p_1^{n_1}})$和$K_2=\mathbb{Q}(\zeta_{p_2^{n_2}})$.于是有$K_1K_2=\mathbb{Q}(\zeta_{p_1^{n_1}},\zeta_{p_2^{n_2}})=\mathbb{Q}(\zeta_n)=K$.另外按照素数幂的情况,有$d(K_1)=(-1)^{p_1^{n_1}/2}p_1^{(p_1^{n_1}-1)/(p_1-1)}$和$d(K_2)=(-1)^{p_2^{n_1}/2}p_2^{(p_2^{n_2}-1)/(p_2-1)}$,它们是互素的整数,于是引理的条件全部满足,于是$\mathscr{O}_K=\mathscr{O}_{K_1}\mathscr{O}_{K_2}=\mathbb{Z}[\zeta_{p_1^{n_1}},\zeta_{p_2^{n_2}}]=\mathbb{Z}[\zeta_n]$,它的一组整基为$\{1,\zeta_n,\cdots,\zeta_n^{\varphi(n)-1}\}$,它的判别式为:
		$$(d(K_1))^{\varphi(p_2^{n_2})}(d(K_2))^{\varphi(p_1^{n_1})}=(-1)^{\varphi(n)/2}p_1^{\varphi(n)/(p_1-1)}p_2^{\varphi(n)/(p_2-1)}$$
	\end{proof}
    \item 对于$n$是素数幂的情况我们还可以借助之前给出过的一个定理证明分圆域$K=\mathbb{Q}(\zeta_{p^n})$的代数整数环恰好是$\mathscr{O}_K=\mathbb{Z}[\zeta_{p^n}]$:记$\Phi_{p^n}(X)$是分圆多项式,那么$\Phi_{p^n}(X+1)$是$p$-爱森斯坦多项式.
\end{enumerate}

1847年,法国数学家Lame在巴黎学院给出演讲,宣称证明了Fermat最后定理.他的主要思路是,对任意奇素数$p$,倘若$a,b,c$是一种最小的全不为零的整数满足$a^p+b^p=c^p$,将他变形作$a^p=c^p-b^p=\prod_{i=0}^{p-1}(c-\omega^ib),\omega=e^{2\pi i/p}$.Lame根据$c-\zeta_p^ib$两两互素并且乘积是$p$次幂,说明每个$c-\omega^ib$都是$p$次幂,由此得到了一个更小的方程的非零解.Liouville敏锐的质疑上述关于$p$次幂的断言,而Lame本人并不能给出证明.按照现代观点,$\mathbb{Z}[\zeta_p]$并不总是UFD,于是这一证明是错误的.尽管Lame的做法有误,人们开始意识到数的唯一分解在更一般的数域上并不是总成立的.Kummer期望能够引入更多的理想的数解决这一问题.戴德金随后将理想数的概念改进为现在所说的环上的理想.当环满足某些特定条件时,理想总可以唯一分解为素理想的乘积.这些条件后来被提炼为戴德金整环:一维正规诺特整环称为戴德金整环.

代数整数环是戴德金整环和几个注解.(戴德金整环的性质见交换代数).
\begin{enumerate}
	\item 代数整数环总是戴德金整环.
	\begin{proof}
		
		我们来验证$\mathscr{O}_K$是诺特的,是正规的,是一维的.任取$\mathscr{O}_K$的理想$I$,它也是$\mathscr{O}_K$作为加法群的子群,我们证明过代数整数环作为加法群都是有限生成自由交换群,它的子群$I$也是有限生成的,即$I$作为$\mathbb{Z}$模是有限生成的,于是它作为$\mathscr{O}_K$模也是有限生成的,于是$\mathscr{O}_K$是诺特环.
		
		$\mathscr{O}_K$是正规的是因为正规化本身是正规的.最后我们来说明它是一维的,按照它是整环,这只需验证$\mathscr{O}_K$上的非零素理想$p$都是极大理想.现在$\mathbb{Z}\subset\mathscr{O}_K$是整环之间的整扩张,于是$\mathscr{O}_K$的理想$p$是极大理想当且仅当它的回拉$p\cap\mathbb{Z}$是$\mathbb{Z}$中的极大理想.但是这里任取$0\not=y\in p$,它的极小多项式记作$y^n+a_1y^{n-1}+\cdots+a_n=0$,那么这里$0\not=a_n\in p\cap\mathbb{Z}$,于是$p\cap\mathbb{Z}$是非零的素理想,而$\mathbb{Z}$上的非零素理想都是极大理想,完成证明.
	\end{proof}
    \item 代数整数环上的每个非零分式理想都可以唯一的分解为若干素理想次幂的乘积,这里次数可以取负整数.
    \item $\mathscr{O}_K$作为戴德金整环,它的分式理想恰好是$\mathscr{O}_K$模$K$的有限生成子模,按照我们之前所证的一个定理,这里$\mathscr{O}_K$上的非零分式理想作为加法群总是秩为$[K:\mathbb{Q}]$的.
    \item 另外事实上在交换代数中证明过,如果$A$是一维诺特整环,$K$是它商域,设$L$是$K$的有限扩张,设$A$在$L$中的正规化是$B$,那么$B$是戴德金整环.
    \item 戴德金整环是UFD当且仅当是PID,当且仅当类数为1.
\end{enumerate}

数环上理想的绝对范数.给定代数整数环$\mathscr{O}_K$的非零理想$I$,它的范数定义为$|\mathscr{O}_K/I|$,这是一个有限数,记作$\mathrm{N}_K(I)$.
\begin{enumerate}
	\item 首先我们说明理想的范数的确是一个有限数.取$n$次数域$K$,给定代数整数环$\mathscr{O}_K$的非零理想$I$,我们解释过$I$是一个秩为$n$的自由交换群,任取$\mathscr{O}_K$上的整基$\{a_1,a_2,\cdots,a_n\}$,任取$I$上的一组基$\{b_1,b_2,\cdots,b_n\}$,那么有$\mathbb{Z}$上的$n$阶方阵$T$使得$(b_1,b_2,\cdots,b_n)=(a_1,a_2,\cdot,a_n)T$,容易验证这里$|\det T|$不受两组基的选取所影响.按照有限生成自由交换群的子群结构定理,我们可以取定一组整基$\{a_1,a_2,\cdots,a_n\}$,使得存在一组整数$r_1,r_2,\cdots,r_n$满足$I=\mathbb{Z}[r_1a_1]\oplus\mathbb{Z}[r_2a_2]\oplus\cdots\oplus\mathbb{Z}[r_na_n]$,此时过渡矩阵是一个对角矩阵$T=\mathrm{diag}\{r_1,r_2,\cdots,r_n\}$,按照中国剩余定理,得到$|\mathscr{O}_K/I|=|r_1r_2\cdots r_n|=|\det T|$.
	\item 理想的范数是一个乘性映射.即如果$I,J$是$\mathscr{O}_K$的两个非零理想,那么$\mathrm{N}_K(IJ)=\mathrm{N}_K(I)\mathrm{N}_K(J)$.
	\begin{proof}
		
		取理想$I$的素理想乘积分解$I=p_1^{e_1}p_2^{e_2}\cdots p_r^{e_r}$,我们先来解释下如果$A$的理想$I,J$互素,也即$I+J=A$,那么对任意正整数$a,b$有$I^a$和$J^b$是互素的:$A=(I+J)^{a+b}\subseteq I^a+J^b$.于是特别的这里$p_i$和$p_j$是不同的极大理想说明它们互素,导致$p_i^{e_i}$和$p_j^{e_j}$是互素的,于是可以用中国剩余定理.得到$\mathrm{N}_K(I)=\prod_{1\le i\le n}\mathrm{N}_K(p_i)^{e_i}$.于是按照非零理想的素理想乘积分解是乘性的,得到这个公式.
	\end{proof}
    \item 按照第一条的性质,如果$\{a_1,a_2,\cdots,a_n\}$是$I$上的一组$\mathbb{Z}$基,那么有$d_K(a_1,a_2,\cdots,a_n)=\mathrm{N}_K(I)^2d(K)$.
    \item 如果$I=(a)$是一个非零主理想,那么$\mathrm{N}_K(I)=|\mathrm{N}_K(a)|$.即主理想的范数就是对应元素的范数的绝对值,这个性质使得它称为绝对范数.
    \begin{proof}
    	
    	设$\mathscr{O}_K$上的一组整基为$\{b_1,b_2,\cdots,b_n\}$,那么$\{ab_1,ab_2,\cdots,ab_n\}$是$I$上的一组$\mathbb{Z}$基.于是上一条得到$d_K(ab_1,ab_2,\cdots,ab_n)=\mathrm{N}_K(I)^2d(K)$.另一方面按照行列式的性质有$d_K(ab_1,ab_2,\cdots,ab_n)=\det(\sigma_i(ab_j))^2=\left(\prod_{1\le i\le n}\sigma_i(a)\right)^2\det(\sigma_i(b_j))^2=(\mathrm{N}_K(a))^2d(K)$.于是$\mathrm{N}_K(I)=|\mathrm{N}_K(a)|$.
    \end{proof}
\end{enumerate}
\newpage
\subsection{素理想的分解}

素理想提升的基本量.设$A\subseteq B$是戴德金整环的扩张,换句话讲如果记$K=\mathrm{Frac}(A)$,那么$B$是$A$在$K$的某个有限扩张$L$中的正规化.
\begin{enumerate}
	\item 对$p\in\mathrm{Spec}A$,断言$pB\not=B$.
	\begin{proof}
		
		不妨设$p$是极大理想,按照$A\subseteq B$是整扩张,说明满足提升条件,并且极大理想的提升仍然是极大理想,所以存在$B$中极大理想$q$使得$q\cap A=p$.于是有$pB=(q\cap A)B\subseteq q$不是单位理想.
	\end{proof}
	\item 我们之前证明过$pB$唯一分解中出现的素理想$q_i$恰好是全部的$p$在$B$中的提升素理想(此即满足$q\cap A=p$的$B$中的素理想$q$).
	\item 上一条说明$A$中的全部非零素理想对应了$B$中全部非零素理想的一个划分,每个等价类由同一个$A$中非零素理想的提升素理想构成,这是从$A$中非零素理想到全部等价类的双射.特别的,按照$\mathbb{Z}$上非零素理想是无穷的,说明每个代数整数环上非零素理想是无穷的.
	\item 如果$B$中两个理想$I$和$J$互素,那么$A\cap I$和$A\cap J$在$A$中互素.事实上$B$中这两个理想互素当且仅当$I+J=B$,也即$I$的素理想分解和$J$的素理想分解没有公共的素理想.按照上一条,这说明$A\cap I$和$A\cap J$在$A$中的素理想分解没有公共素理想,于是它们互素.
	\item 设$q\subseteq B$是非零素理想$p\subseteq A$的提升素理想,我们称域扩张$A/p\subseteq B/q$为$p$在$q$处的剩余域扩张.剩余域扩张总是一个有限扩张:
	\begin{enumerate}
		\item 首先对于数域的情况.我们解释过对非零理想$I$,总有$\mathscr{O}_K/I$是有限环,于是在这种情况下,剩余域扩张$\mathscr{O}_K/p\subset\mathscr{O}_L/q$是两个有限域的扩张,于是必然是有限可分扩张.
		\item 如果$A$是戴德金整环,$K$是商域,$K\subseteq L$是有限可分扩张,$B$是$A$在$L$中的正规化.任取非零素理想$p\subseteq A$,任取提升素理想$q\subseteq B$.尽管此时剩余域$A/p$和$B/q$未必是有限域,但是此时$A/p\subseteq B/q$仍然是有限扩张:我们证明过在条件下$B$是有限$A$模(事实上,我们在前文的证明中仅要求$A$是诺特正规整环),于是$A/p\subseteq B/q$是有限扩张.
		\item 上一条中事实上$K\subseteq L$仅要求是有限扩张,就能得到每个剩余域扩张$A/p\subseteq B/q$是有限扩张.这可借助交换代数中的Krull-Akizuki定理:设$A$是一维诺特整环,商域$K$,设$K\subseteq L$是有限扩张,设环$B$满足$A\subseteq B\subseteq L$,那么$B$是维数至多1的诺特环,并且如果$J$是$B$的非零理想,那么$B/J$是有限长度$A$模.于是如果取$B$是我们这里的正规化,取$J$是非零素理想$p\subseteq A$的提升素理想,那么$B/q$是有限长度$A$模得到$B/q$是有限长度$A/p$模,但是域上有限长度模等价于有限维线性空间,于是$A/p\subseteq B/q$是有限扩张.
	\end{enumerate}
	\item 基本量的定义.设$q\subseteq B$是非零素理想$p\subseteq A$的提升素理想,称$q$在$pB$的唯一分解中的次数为$q$的分歧指数,记作$e(q/p)$;称扩张维数$[B/q:A/p]$为$q$的惯性次数或者剩余类次数,记作$f(q/p)$;称$p$在$B$中唯一分解中不同的素理想的个数$g$,它称为分裂次数.
	\begin{itemize}
		\item 如果只存在一个素理想$q$提升了极大理想$p\in\mathrm{Spec}A$,并且$f(q/p)=1$,则称$L/K$在$p$是完全分歧的.
		\item 如果$e(q/p)=1$,并且$B/q$是$A/p$的可分扩张,就称$L/K$在$q$是非分歧的.如果$L/K$在$p\in\mathrm{Spec}A$的每个提升素理想都是非分歧的,就称$L/K$在$p$是非分歧的.
	\end{itemize}
\end{enumerate}

基本量的性质.设$A$是戴德金整环,$K$是商域,$K\subseteq L$是$n$次可分扩张,$A$在$L$中的正规化为$B$.
\begin{enumerate}
	\item 任取非零素理想$p\subseteq A$,设唯一分解$pB=q_1^{e_1}q_2^{e_2}\cdots q_g^{e_g}$.那么分歧指数,剩余域次数和扩张维数满足等式$\sum_{1\le i\le g}e_if_i=n$.这个等式解释了惯性次数的含义:惯性次数越小,那么这个素理想更趋向于分裂为更多的素理想乘积.
	\begin{proof}
		
		按照中国剩余定理,有$B/pB\cong\oplus_{1\le i\le g}B/q_i^{e_i}$.记$\kappa=A/p$,我们断言$\dim_{\kappa}(B/pB)=n$和$\dim_{\kappa}(B/q_i^{e_i})=e_if_i$.先证明第二个等式,有$\kappa$模的升链$(0)\subseteq q_i^{e_i-1}/q_i^{e_i}\subset\cdots\subseteq q_i/q_i^{e_i}\subseteq B/q_i^{e_i}$.相邻两项的商模都$\kappa$线性同构于$B/q_i$,于是得到$\dim_{\kappa}(B/q_i^{e_i})=\sum_{1\le i\le e_i}\dim_{\kappa}(B/q_i)=e_if_i$.这就证明了第二个等式,下面仅需验证$\dim_{\kappa}(B/pB)=n$.
		
		我们解释过$B$是有限$A$模,于是$B/pB$是有限$A/p$模,这里$A/p$是域,于是$B/pB$是$A/p$有限维线性空间.设$\{\omega_1,\omega_2,\cdots,\omega_m\}\subseteq B$使得它在$B/pB$下的像$\{\overline{\omega_1},\overline{\omega_2},\cdots,\overline{\omega_m}\}$是一组$A/p$基.我们只需验证$\{\omega_1,\omega_2,\cdots,\omega_m\}$是$K$线性空间$L$的一组基.
		
		$\{\omega_1,\omega_2,\cdots,\omega_m\}$是线性无关的:如果有不全为零的$a_1,a_2,\cdots,a_m\in A$使得$\sum_ia_i\omega_i=0$,考虑$A$的理想$I=(a_1,a_2,\cdots,a_m)$.那么按照素理想乘积的唯一分解有真包含关系$I^{-1}p\subsetneqq I^{-1}$,于是可取$a\in I^{-1}$但是$a\not\in I^{-1}p$.于是每个$aa_i\in A$,但是它们不全落在$p$中,于是$\sum_iaa_i\omega_i\equiv0(\mathrm{mod}p)$,导致$\{\overline{\omega_1},\overline{\omega_2},\cdots,\overline{\omega_m}\}$并不是$B/pB$上的一组$A/p$基,矛盾.
		
		$\{\omega_1,\omega_2,\cdots,\omega_m\}$在$K$上生成整个$L$:考虑$A$模$M=A\omega_1+A\omega_2+\cdots+A\omega_m$,记$N=B/M$.按照$B=M+pB$,得到$pN=N$.按照我们解释过的$B$是有限$A$模,得到$N$是有限$A$模,按照NAK引理,存在$a\in p$使得$(1+a)N=0$,于是对每个$b\in B$,有$(1+a)b=m\in M$,于是$b=m/(1+a)\subseteq K\omega_1+K\omega_2+\cdots+K\omega_m$.
	\end{proof}
	\item 推论.如果记$[L:K]=n$,任取$A$的极大理想$p$,那么$B$中提升$p$的素理想的个数至少有1个至多有$n$个.特别的,如果$A$只有有限个素理想(我们证明过对于戴德金整环这等价于它是主理想整环),那么$B$也只有有限个素理想(等价于是主理想整环).
	\item 传递公式.如果戴德金整环$A$的商域记作$K$,如果$K\subseteq L\subseteq F$是有限扩张,$A$在$F$中的正规化记作$C$,设$A$的非零素理想$p$在$B$中有提升素理想$q$,设$q$在$C$中有提升素理想$Q$,那么有传递公式$e(Q/p)=e(Q/q)e(q/p)$和$f(Q/p)=f(Q/q)f(q/p)$.
	\begin{proof}
		
		设$pB=q_1^{e_1}q_2^{e_2}\cdots q_s^{e_s}$和$q=q_1$,再设$qC=Q_1^{r_1}Q_2^{r_2}\cdots Q_t^{r_t}$和$Q=Q_1$.于是有$qC=Q_1^{r_1e_1}Q_2^{r_2e_1}\cdots Q_t^{r_te_1}\prod_i\mathscr{P}_i$,其中$\mathscr{P_i}$是$Q_2^{r_2}\cdots Q_t^{r_t}$的素理想分解,于是这些$\mathscr{P}_i$不包含$Q_1$,于是$e(Q/p)=e(Q/q)e(q/p)$.
		
		另一个等式是直接的:$f(Q/p)=[C/Q:A/p]=[C/Q:B/q]\cdot[B/q:A/p]=f(Q/q)f(q/p)$.
	\end{proof}
	\item 设$q$是$B$的极大理想,记$p=A\cap q$,明显有$v_q(x)=e(q/p)v_p(x),x\in K$.这可以理解为$v_q$是$v_p$的指数为$e(q/p)$的在$L$上的延拓.反过来如果$w$是$L$上的离散赋值,并且是$v_p$的指数为$e$的延拓,我们断言存在$p$的提升素理想$q$,使得$w=v_q$,$e=e(q/p)$.
	\begin{proof}
		
		设$L$由离散赋值$w$定义的赋值环为$W$,设$W$的极大理想是$Q$.我们证明过赋值环都是正规整环,并且商域就是$L$.从$w=ev_p$得到$A\subseteq W$,按照$W$是正规的得到$B\subseteq W$.记$q=Q\cap B$,那么$q\cap A=p$,所以$q$是$p$的提升素理想.下面断言$B_q\subseteq W$,只需证明如果$s\not\in q$,那么$1/s\in W$.假设$1/s\not\in W$,按照$W$是赋值环,有$s\not\in W$,于是$s$不会是$W$中的单位,否则$1/s\in W$,于是$s\in Q$,于是$s\in Q\cap B=q$,这矛盾.于是$B_q\subseteq W$,但是二者都是赋值环,它们之间的包含映射是一个局部映射,满足这个关系的两个赋值环称为控制,赋值环的一个等价描述是控制关系下的极大元,这迫使$B_q=W$.于是有$w=v_q$,$e=e(q/p)$.
	\end{proof}
\end{enumerate}

范数同态和包含同态.设$A$是戴德金整环,用$I_A$表示$A$的理想群,它是由$A$的全体非零素理想生成的自由阿贝尔群.设$A\subseteq B$是戴德金整环的扩张,定义包含同态$i:I_A\to I_B$为$p\mapsto pB=\prod_{q\mid p}q^{e(q/p)}$;定义范数同态$\mathrm{N}:I_B\to I_A$为$q\mapsto p^{f(q/p)}$,其中$p=q\cap A$.
\begin{enumerate}
	\item 按照基本量的关系式,对$\alpha\in I_A$明显有$\mathrm{N}(i(\alpha))=\alpha^n$.
	\item 设$M$是长度为$m<\infty$的$A$模,它的合成链记作$0=M_0\subseteq M_1\subseteq\cdots\subseteq M_m=M$,那么每个$M_i/M_{i-1}$同构于某个$A$单模,即$A/p_i$其中$p_i$是$A$的非零素理想.按照Jordan-H\"older定理,这里的全体$p_i$在记重数意义下是被$M$唯一决定的,记$\chi_A(M)=\prod_ip_i$.例如如果$\alpha\subseteq\beta$是$A$的两个分式理想,那么$\chi_A(M)=ab^{-1}\subseteq A$.
	\item 把全体有限长度$A$模构成的类记作$\mathscr{C}_A$,那么映射$\chi_A$是$\mathscr{C}_A\to I_A$的映射,如果有有限长度$A$模的短正合列$0\to M'\to M\to M''\to0$,那么有$\chi_A(M)=\chi_A(M')\chi_A(M'')$.我们称$\chi_A$定义了$\mathscr{C}_A$关于$I_A$的Grothendieck群.反过来对任意乘性映射$f:\mathscr{C}_A\to G$,其中$G$是乘法群,都可以唯一的表示为$g\circ\chi_A$,其中$g:I_A\to G$是群同态,这只需定义$g(p)=f(A/p)$即可.
	\item 范数同态在Grothendieck群下的描述.如果$M$是$B$单模,那么$M$作为$B$模同构于某个$B/q$,其中$q$是极大理想,它作为$A$模是$A$单模$A/p$的有限直和,所以有限长度$B$模作为$A$模也是有限长度的.我们断言有如下交换图表,换句话讲对有限长度$B$模$M$有$\chi_A(M)=\mathrm{N}(\chi_B(M))$.
	$$\xymatrix{\mathscr{C}_B\ar[rr]^{\chi_B}\ar@{^{(}->}[d]&&I_B\ar[d]^{\mathrm{N}}\\\mathscr{C}_A\ar[rr]_{\chi_A}&&I_A}$$
	\begin{proof}
		
		按照正合列的线性归结为证明$M=B/q$的情况,其中$q$是极大理想.但是此时$\mathrm{N}(\chi_B(B/q))=\mathrm{N}(q)=p^{f(q/p)}$.另一方面$B/q$作为$A$模同构于$(A/p)^{f(q/p)}$,于是$\chi_A(B/q)=p^{f(q/p)}$.
	\end{proof}
	\item 包含同态在Grothendieck群下的描述.如果$M$是有限长度$A$模,那么$M_B=M\otimes_AB$作为$B$模也是有限长度的.我们断言$M\mapsto M_B$作为$\mathscr{C}_A\to\mathscr{C}_B$的函子是正合的,归结为在局部化下这个函子是正合的,但是$A$在极大理想的局部化是主理想整环(甚至是DVR),此时$B$是自由$A$模,所以是正合的.我们断言有如下交换图表,换句话讲对有限长度$A$模$M$总有$\chi_B(M_B)=i(\chi_A(M))$.
	$$\xymatrix{\mathscr{C}_A\ar[d]_{-\otimes_AB}\ar[rr]^{\chi_A}&&I_A\ar[d]^{i}\\\mathscr{C}_B\ar[rr]_{\chi_B}&&I_B}$$
	\begin{proof}
		
		按照正合列的线性归结为证明$M=A/p$的情况,其中$p$是极大理想.此时$M_B=B/pB$,所以$\chi_B(M_B)=pB$,而按照定义$i(\chi_A(M))=i(p)=pB$.
	\end{proof}
	\item 我们断言这里的范数同态在主理想上的限制就是域扩张的范数同态.换句话讲对任意$x\in L$有$\mathrm{N}(xB)=\mathrm{N}_{L/K}(x)A$.
	\begin{proof}
		
		不妨设$x$在$A$上整,否则可以给$x$适当乘以一个$A$中的元,这不影响要证的等式.不妨设$A$是主理想整环,否则对局部化证明这个等式也能得到一般情况下这个等式成立.此时环$B$是一个有限秩自由$A$模,设秩为$n$.记$B$上左乘$x$的$A$线性映射为$u_x$,那么有$\mathrm{N}_{L/K}(x)=\det(u_x)$,并且$\mathrm{N}(xB)=\chi_A(B/xB)=\chi_A(\mathrm{coker}u_x)$.下面我们证明更一般的结论:如果$A$是PID,如果$u:A^n\to A^n$是线性映射,满足$\det(u)\not=0$,那么$\det(u)A=\chi_A(\mathrm{coker}(u))$.
		
		\qquad
		
		对$u$左右乘以一个可逆线性映射不影响要证的等式的两侧,按照线性映射和矩阵的理论,可以设$u$在$A^n\to A^n$的两组基的选取下是对角矩阵.接下来对$n$归纳即可.
	\end{proof}
\end{enumerate}

完全分歧和非分歧的最简单例子,即单扩张的情况.设$(A,m,k)$是局部环,设$f\in A[x]$是次数为正整数$n$的首一多项式,记$B_f=A[x]/(f)$,这是一个秩$n$的自由$A$模,$\{1,x,\cdots,x^{n-1}\}$是一组基.
\begin{enumerate}
	\item 首先描述$B_f$的极大理想.记$\overline{B_f}=B_f/mB_f=A[x]/(m,f)=k[x]/(\overline{f})$,其中$\overline{f}$是$f$在$k=A/m$中的像.记$\overline{f}$在$k[x]$中的唯一分解是$\overline{f}=\prod_i\varphi_i^{e_i}$.对每个$i$选取$\varphi_i$在$A[x]$中的提升为$g_i$.记$m_i=(m,g_i)$,那么$m_i$是$B_f$的两两不同的极大理想,并且它们是$B_f$的全部极大理想,并且有$B_f/m_i\cong k_i=k[x]/(\varphi_i)$.
	\begin{proof}
		
		有$B_f/m_i=\overline{B_f}/(\overline{\varphi_i})=k[x]/(\varphi_i)=k_i$是域,所以$m_i$是$B_f$的剩余类域为$k_i$的极大理想.任取$B_f$的极大理想$n$,为了说明$n$是某个$m_i$,只需证明$m\subseteq n$,因为这导致$\overline{n}$是$\overline{B_f}\cong k[x]/(\overline{f})$的极大理想,于是$n$是某个$(\varphi_i)$在$B_f\to\overline{B_f}$下的原像,所以$n=(g_i,m)=m_i$.最后来说明$m\subseteq n$,如果这不成立,那么有$n+mB_f=B_f$,按照$B_f$是有限$A$模,NAK引理导致$n=B_f$,这矛盾.
	\end{proof}
	\item 非分歧的例子.设$A$是DVR,设$\overline{f}$是不可约多项式,那么$B_f$是以$mB_f$为极大理想的DVR,剩余类域是$k[x]/(\overline{f})$.记$K=\mathrm{Frac}(A)$,那么$L=K[x]/(f)$是域,并且$B_f$恰好是$A$在$L$中的整闭包.如果额外的还有$\overline{f}$是可分多项式,那么此时$L/K$是非分歧扩张.
	\begin{proof}
		
		这里$\overline{f}$是不可约多项式,所以按照上一条结论,有$B_f$是以$mB_f$为极大理想的局部环,它的剩余类域是$k[X]/(\overline{f})$.按照$A$是DVR,设有$\pi$生成了$m$,那么$\pi$在$B_f$中的像生成了$mB_f$,所以$B_f$是DVR.
		
		\qquad
		
		$f$本来是$A$上的首一多项式,所以它是本原的,所以它在$K$上也是不可约多项式,所以$K[x]/(f)$是域.DVR都是正规整环,所以$B_f$是正规的,并且$B_f$的商域是$L$,但是$B_f$是$A$上添加了整元,所以$B_f$恰好是$A$在$L$中的整闭包.最后如果额外的$\overline{f}$是可分多项式,$L/K$是非分歧扩张就是定义.
	\end{proof}
	\item 反过来如果$A$是DVR,商域$K$,设$L/K$是次数$n$的扩张,设$B$是$A$在$L$中的整闭包.假设$B$是DVR,并且$B$的剩余类域$\overline{L}$是$A$的剩余类域$k$的$n$次单扩张,任取这个生成元$\overline{x}$在$B$中的提升$x$,设$x$在$K$上的特征多项式(自然首一)为$f$,那么$X\mapsto x$的同态$A[X]\to B$诱导了同构$B_f\cong B$.
	\begin{proof}
		
		特征多项式$f$是$x$在$K$上极小多项式的次幂,$x$的极小多项式的系数都是$A$上整元,并且都在$K$中,按照DVR是正规的,说明$x$的极小多项式是$A$上的多项式,于是$f$是$A$上的多项式.从$f(x)=0$得到$A[X]\to B$可分解为$A[X]\to B_f=A[X]/(f)\to B$.从$\overline{f}(\overline{x})=0$和$\deg\overline{f}=n$得到$\overline{f}$是$\overline{x}$在$k$上的极小多项式,所以不可约的,所以上一条中的条件满足,所以结论成立.
	\end{proof}
	\item 完全分歧的例子.设$A$是DVR,设$f$是爱森斯坦多项式,换句话讲如果记$f=X^n+a_1X^{n-1}+\cdots+a_n$,那么$a_i\in m$和$a_n\not\in m^2$.那么$B_f$也是DVR,它的极大理想被$X$在$B_f$中的像$x$生成,剩余类域是$k$.如果额外的$f$是$K[X]$的不可约多项式,记域$L=K[X]/(f)$,那么$B_f$是$A$在$L$中的整闭包.
	\begin{proof}
		
		我们有$\overline{f}=X^n$,于是第一条说明$B_f$是以$(m,x)$为极大理想的局部环.按照$a_n\in m$,$a_n^2\not\in m$,说明$a_n$生成了$m$,另外从$x^n+a_1x^{n-1}+\cdots+a_{n-1}x=-a_n$得到$a_n\in(x)$,于是$(m,x)=(x)$.因为$a_n$不是幂零元,得到$x$也不是,所以$B_f$是DVR.
	\end{proof}
	\item 反过来如果$A$是DVR,商域记作$K$,设$L/K$是次数$n$的有限扩张,设$A$在$L$中的整闭包是$B$.假设$B$也是DVR,并且延拓了$A$的离散赋值具有分歧指数$n$.记$B$的极大理想被$x$生成,记$x$在$K$上的特征多项式为$f$.那么$f$是爱森斯坦多项式,并且$X\mapsto x$的同态$A[X]\to B$诱导了同构$B_f\cong B$.
	\begin{proof}
		
		这里$f$同样是$x$在$K$上极小多项式的次幂,这个极小多项式的系数同样既在$A$上整又是$K$中的元,所以$f$同样是$A$上的多项式.记$f=X^n+a_1X^{n-1}+\cdots+a_n$,那么有$a_0x^n+\cdots+a_n=0$.记$B$上的离散赋值是$w$,那么$w(x)=1$,并且$w(A)\subseteq n\mathbb{Z}$因为如果$A$上离散赋值记作$v$那么$w(a)=nv(a),\forall a\in A$.设$r=\inf\{w(a_ix^{n-i}),0\le i\le n\}$,按照离散赋值的强三角不等式的性质,必然存在$0\le i<j\le n$使得$r=w(a_ix^{n-i})=w(a_jx^{n-j})$.于是有$j-i=w(a_j/a_i)\in n\mathbb{Z}$,但是$i,j$在$0$和$n$之间,这迫使$i=0$和$j=n$.所以$r=n$,$w(a_n)=n$,并且$w(a_i)\ge n-i>0$,于是$f$是爱森斯坦多项式.剩余的结论来自上一条.
	\end{proof}
\end{enumerate}

Galois扩张的情况.依旧设$A$是戴德金整环,$K$是它的商域,设$L/K$是有限Galois扩张,Galois群记作$G(L/K)$,记$A$在$L$中的整闭包是$B$.
\begin{enumerate}
	\item $G(L/K)$中的元限制在$B$上都是自同构.
	\begin{proof}
		
		如果$b\in B$,那么$\sigma(b)$和$b$具有相同的极小多项式,于是$\sigma(b)\in B$,于是$\sigma$限制在$B$上是一个自同态.按照这些$\sigma$构成群,于是它们都是$B$上的自同构.
	\end{proof}
	\item 现在设$p\in\mathrm{Spec}(A)$,它在$B$中的全部提升素理想构成的集合记作$S_p$,那么有自然的$G$在$S_p$上的作用.换句话讲每个$\sigma$是$S_p$上的双射.换句话讲如果$q\in\mathrm{Spec}(B)$是$p$的提升素理想,那么$\sigma(q)$也是$p$的提升素理想,这是因为$\sigma(q)\cap A=\sigma(q\cap A)=\sigma(p)=p$.我们断言$G$在$S_p$上的作用是一个可迁作用.换句话讲,任取$p$的两个提升素理想$q$和$q'$,那么存在某个$\sigma\in G$使得$\sigma(q)=q'$.整理一下,我们定义了$G$在集合$\mathrm{Spec}(B)$上的群作用$(\sigma,q)\mapsto\sigma(q)$.这个作用的每个轨道恰好就是某个素理想$p\mathrm{Spec}(A)$的全部提升素理想.
	\begin{proof}
		
		假设对任意$\sigma\in G$都有$\sigma(q)\not=q'$.按照中国剩余定理,可取$x\in A$使得$x\equiv0(\mathrm{mod}q')$和$x\equiv1(\mathrm{mod}\sigma(q)),\forall\sigma\in G$.那么范数$\mathrm{N}_{L/K}(x)=\prod_{\sigma\in G}\sigma(x)\in q'\cap A=p$.
		
		但是按照$x\not\in\sigma(q),\forall\sigma\in G$,也即$\sigma(x)\not\in q,\forall\sigma\in G$.这就和$\prod_{\sigma\in G}\sigma(x)\in q\cap A=p$矛盾.
	\end{proof}
	\item Galois扩张下素理想分解的基本量关系式.在Galois扩张下,一个素理想$p\in\mathrm{Spec}(A)$的全部提升素理想的惯性次数都是相同的$f_1=f_2=\cdots=f_g=f$,全部分歧指数都是相同的$e_1=e_2=\cdots=e_g=e$.此时基本量满足关系式$n=efg$.
	\begin{proof}
		
		设$pB=q_1^{e_1}q_2^{e_2}\cdots q_g^{e_g}$.对任意$i,j$,存在$\sigma\in G$使得$\sigma(q_i)=q_j$,于是$B/q_i\cong B/\sigma(q_i)=B/q_j$,于是$f_j=f_i$.再按照$q^v\mid pB$当且仅当$\sigma(q^v)\mid\sigma(pB)$当且仅当$\sigma(q^v)\mid pB$,于是$e_i=e_j$.
	\end{proof}
\end{enumerate}

分解群和分解域.设$q\in\mathrm{Spec}(B)$,它在上述群作用下的稳定子群$G_q=\{\sigma\in G\mid\sigma(q)=q\}$称为$q$在$K$上的分解群.这是Galois群的$ef$阶子群,它确定的中间域$Z_q=\{x\in L\mid\sigma x=x,\forall\sigma\in G_q\}$称为$q$在$K$上的分解域.我们依旧用$e,f,g$表示基本量.
\begin{enumerate}
	\item 我们之前解释过Galois扩张下$G$在$S_p$下的作用是可迁的,于是如果$q$和$q'$都是$p$的提升素理想,那么$G_q$和$G_{q'}$是共轭的.另外轨道长度是$p$的提升素理想的个数$g$,按照轨道公式就有分解群的元素个数$|G_q|=ef$.于是按照Galois基本定理,有$[L:Z_q]=ef$,$[Z_q:K]=g$,$G(L/Z_q)=G_q$.
	\item 设$q_Z=q\cap Z_q$是分解域$Z_q$中的素理想,那么有提升素理想链$p\subseteq q_Z\subseteq q$,此时$q_Z$在$q$上的分歧指数和惯性次数都是1,$q$是$q_Z$唯一的提升素理想,它的分歧指数是$e$,惯性次数是$f$.
	\begin{proof}
		
		先说明$q$是唯一的$q_Z$的提升素理想.$Z_q\subseteq L$也是Galois扩张,它的Galois群就是$G_q$.于是$q_Z$的全部提升素理想就是$\sigma(q),\sigma\in G_q$,于是$q$是唯一的$q_Z$的提升素理想.
		
		\qquad
		
		再设$q$在$Z_q$上的分歧指数是$e'$,惯性次数是$f'$.$q_Z$在$p$上的分歧指数是$e''$,惯性次数是$f''$.那么传递公式$e=e'e''$和$f=f'f''$.按照$Z_q\subseteq L$也是Galois扩张,说明$[L:Z_q]=e'f'$.另一方面$|G_q|=[L:Z_q]=ef$,而$e'\le e$和$f'\le f$,从$ef=e'f'$得到$e=e'$和$f=f'$,于是$e''=f''=1$.
	\end{proof} $$\xymatrix{G\ar[d]&&G_q\ar[ll]_g\ar[d]&&(1)\ar[d]\ar[ll]_{ef}\\K\ar[rr]^g\ar[u]&&Z_q\ar[rr]^{ef}\ar[u]&&L\ar[u]\\p\ar[rr]^{e(q_Z/p)=1}\ar[u]&&q_Z\ar[rr]^{e(q/q_Z)=e}\ar[u]&&q\ar[u]\\\kappa(p)\ar[rr]^{\sim}_{f(q_Z/p)=1}\ar[u]&&\kappa(q_Z)\ar[rr]_{f(q/q_Z)=f}\ar[u]&&\kappa(q)\ar[u]}$$
\end{enumerate}

惯性群和惯性域.在Galois扩张的前提下,设$q\in\mathrm{Spec}(B)$是$p\in\mathrm{Spec}(A)$的提升素理想,任取$\sigma\in G_q$,它诱导了$B/q\to B/q$的同构,并且固定了$A/p$,于是得到了一个同态$G_q\to G(\kappa(q)/\kappa(p))$.这个同态的核称为$q$的惯性群,记作$I_q$,它在Galois扩张$L/K$下的中间域称为$q$的惯性域,记作$T_q$.
\begin{enumerate}
	\item 在Galois扩张的前提下,剩余域扩张$\kappa(p)\subset\kappa(q)$是正规扩张,并且之前给出的$G_q\to G(\kappa(q)/\kappa(p))$是满同态.另外按照我们证明的$q_Z$在$K$上的惯性次数是1,说明$\kappa(p)=\kappa(q_Z)$,于是$\kappa(q_Z)\subset\kappa(q)$也是正规扩张.
	\begin{proof}
		
		任取$\overline{\theta}\in\kappa(q)$,任取一个提升$\theta\in B$,设$f(X)$和$\overline{g}(X)$分别是$\theta$在$K$上和$\overline{\theta}$在$\kappa(p)$上的极小多项式.于是$\overline{g}(X)\mid\overline{f}(X)$.按照$K\subseteq L$是正规扩张,说明$f(X)$在$B$上分解为一次因式的乘积,于是$\overline{g}(X)$在$\kappa(p)$上分解为一次因式的乘积.也即$\kappa(p)\subset\kappa(q)$是正规扩张.
		
		\qquad
		
		现在设$\overline{\theta}$是扩张$\kappa(p)\subset\kappa(q)$的可分闭包的本原元.那么有$G(\kappa(q)/\kappa(p))=G(\kappa(p)(\overline{\theta})/\kappa(p))$.任取其中的一个元$\overline{\sigma}$,那么$\overline{\sigma}(\overline{\theta})$是极小多项式$\overline{g}(X)$的根,于是也是$\overline{f}(X)$的根,于是存在$f(X)$的根$\theta'$使得$\overline{\theta'}=\overline{\sigma}(\overline{\theta})$.这里$\theta'$是$\theta$的共轭元,于是可取$\sigma\in G(L/K)$使得$\theta'=\sigma(\theta)$.于是$\sigma$映射为$\overline{\sigma}$,这说明满射.
	\end{proof}
	\item 我们已经解释过$\kappa(p)\subset\kappa(q)$是正规扩张,如果它还是可分扩张,记$q\cap T_q=q_T$,那么有:$|I_q|=[F:T_q]=e$,$[G_q:I_q]=[T_q:Z_q]=f$;$q$在$q_T$上的分歧指数是$e$,惯性次数是1;$q_T$在$q_Z$上的分歧指数是1,惯性次数是$f$.也即有如下对应关系:
	$$\xymatrix{G\ar[d]&&G_q\ar[ll]_g\ar[d]&&I_q\ar[ll]_f\ar[d]&&(1)\ar[d]\ar[ll]_e\\K\ar[rr]^g\ar[u]&&Z_q\ar[rr]^f\ar[u]&&T_q\ar[rr]^e\ar[u]&&L\ar[u]\\p\ar[u]\ar[rr]^{e(q_Z/p)=1}&&q_Z\ar[u]\ar[rr]^{e(q_T/q_Z)=1}&&q_T\ar[rr]^{e(q/q_T)=e}\ar[u]&&q\ar[u]\\\kappa(p)\ar[u]\ar[rr]^{\sim}_{f(q_Z/p)=1}&&\kappa(q_Z)\ar[u]\ar[rr]^{\text{Galois扩张}}_{f(q_T/q_Z)=f}&&\kappa(q_T)\ar[u]\ar[rr]^{\sim}_{f(q/q_T)=1}&&\kappa(q)\ar[u]}$$
	\begin{proof}
		
		我们在之前已经证明了这个图表信息的一部分.先来证明$\kappa(q)=\kappa(q_I)$.任取$\overline{\alpha}\in\kappa(q)$,其中$\alpha\in B$,设$\alpha$在$\mathscr{O}_{T_q}$中的极小多项式是$f(X)$.可记$f(X)=\prod_{\sigma\in I_q}(X-\sigma(\alpha))$.按照$I_q$的定义,在$\mathrm{mod}q$下有$\sigma(\alpha)\equiv\alpha,\forall\sigma\in I_q$.即在$\mathrm{mod}q$下恒有$\sigma(\overline{\alpha})=\overline{\alpha}$.记$f(X)$在$\mathrm{mod}q_I$下为$\overline{f}(X)$,那么它可以表示为$\overline{f}(X)=(X-\overline{\alpha})^{|I_q|}$.现在$\kappa(p)=\kappa(q_Z)\subset\kappa(q)$是一个Galois扩张,于是$\kappa(q_T)\subset\kappa(q)$也是Galois扩张.按照极小多项式是纯不可分的,说明每个自同构都把$\overline{\alpha}$映射为$\overline{\alpha}$,于是$\overline{\alpha}\in\kappa(q_I)$.这说明$\kappa(q)=\kappa(q_I)$.
		
		于是我们得到了$f(q/q_T)=1$和$f(q_T/q_Z)=f$.再按照$G_q/I_q\cong G(\kappa(q)/\kappa(p))$,按照扩张是Galois扩张,得到$|G_q/I_q|=[\kappa(q):\kappa(p)]=f$,于是$[T_q:Z_q]=f$,这就得到$|I_q|=e$和$[L:T_q]=e$.
		
		最后我们解释过$q$是$q_Z$的唯一提升素理想,这说明$q$也是$q_T$唯一的提升素理想,于是$q_T$在$L$中的分裂次数是1,于是按照基本量的关系式,得到$e(q/q_T)=e(q/q_T)f(q/q_T)=e$,从而得到$e(q_T/q_Z)=1$.
	\end{proof}
	\item 剩余域扩张是Galois扩张的前提下我们有:
	\begin{itemize}
		\item $G_q={e}\Leftrightarrow Z_q=L\Leftrightarrow p\text{是完全分裂的}$.
		\item $G_q=G\Leftrightarrow Z_q=K\Leftrightarrow p\text{是非分裂的}$.
		\item $I_q={e}\Leftrightarrow T_q=L\Leftrightarrow p\text{是非分歧的}$.
	\end{itemize}
	\item 如果$\kappa(p)\subset\kappa(q)$未必是可分扩张,我们有如下结论(但是例如$\kappa(p)$是完美域或者$I_q$的阶数和$\char\kappa(p)$互素都会导致它实际上是可分扩张):记$M$是$\kappa(p)$在$\kappa(q)$中的可分闭包,记可分维数$f_0=[M:\kappa(p)]=[\kappa(q):\kappa(p)]_s$,记纯不可分维数$p^s=[\kappa(q):M]=[\kappa(q):\kappa(p)]_i$,于是$f=f_0p^s$.我们断言:
	\begin{enumerate}
		\item $[L:T_q]=ep^s$,$[T_q:Z_q]=f_0$,$[Z_q:K]=g$.
		\item 记$q,q_T,q_Z,p$对应的离散赋值分别是$w,w_T,w_Z,v$,那么$w$是$w_T$的分歧指数$e$的延拓,$w_T$和$w_Z$都是$v$的分歧指数1的延拓.
		\item $\kappa(q_T)=M$,$\kappa(q_Z)=\kappa(p)$.特别的有$[\kappa(q):\kappa(q_T)]=p^s$,$[\kappa(q_T):\kappa(q_Z)]=f_0$,$[\kappa(q_Z):\kappa(p)]=1$.
	\end{enumerate}
	\begin{proof}
		
		我们证明过$G_q$的阶数是$ef$,另外$G_q/I_q$的阶数是$G(\kappa(q)/\kappa(p))$的阶数,此为可分维数$f_0$,于是$|I_q|=ep^s$,这得到(a).
		
		\qquad
		
		考虑扩张$\kappa(q)/\kappa(q_T)$,它的Galois群同构于$I_q/I_q$平凡,说明这个扩张是纯不可分的.于是特别的有$M$是$\kappa(q_T)$的纯不可分扩张.任取$x\in M$,按照$M$的定义它在$\kappa(p)$上可分,所以$x\in\kappa(q_T)$,这说明$M\subseteq\kappa(q_T)$.于是$[\kappa(q):\kappa(q_T)]\le[\kappa(q):M]=p^s$,也即$f(L/T_q)\le p^s$.但是传递公式有$e(L/T_q)\le e(L/K)=e$,从$[L:T_q]=ep^s$只能得到$\kappa(q)=\kappa(q_T)$和$e(L/T_q)=e$.这得到(b)和(c).
	\end{proof}
	\item 设$L/K$有中间域$E$,使得$E/K$也是Galois扩张.那么依旧可以定义分解群$D(L/E)$,$D(E/K)$和惯性群$T(L/E),T(E/K)$.我们断言有$D(L/E)=D(L/K)\cap G(L/E)$和$T(L/E)=T(L/K)\cap G(L/E)$.另外有如下交换图表,其中$\overline{L},\overline{K},\overline{E}$都是剩余类域,并且它的行和列都是短正合列:
	$$\xymatrix{&1\ar[d]&1\ar[d]&1\ar[d]&\\1\ar[r]&T(L/E)\ar[r]\ar[d]&T(L/K)\ar[r]\ar[d]&T(E/K)\ar[r]\ar[d]&1\\1\ar[r]&D(L/E)\ar[r]\ar[d]&D(L/K)\ar[r]\ar[d]&D(E/K)\ar[r]\ar[d]&1\\1\ar[r]&G(\overline{L}/\overline{E})\ar[r]\ar[d]&G(\overline{L}/\overline{K})\ar[r]\ar[d]&G(\overline{E}/\overline{K})\ar[r]\ar[d]&1\\&1&1&1&}$$
	\begin{proof}
		
		前两个等式是直接的.图表交换性也是直接的,并且列的正合性我们已经证明过了,第三行的正合性就是Galois基本定理.归结为证明前两行的正合性,以第二行为例,任取$s\in D(E/K)$,那么存在$t\in G(L/K)$使得它限制在$E$上就是$s$.于是$q$和$t(q)$在$E$上的限制相同,于是可以找到$t'\in G(L/E)$使得$t't(q)=q$,于是$t't\in D(L/K)$,并且限制在$E$上是$s$.这说明$D(L/K)\to D(E/K)$是满射.
	\end{proof}
\end{enumerate}




Kummer定理.这个定理给出了多项式分解和素理想分解之间的一种对应.
\begin{enumerate}
	\item 定理内容.如果域扩张$K\subseteq L$是有限可分扩张,我们可以选取本原元$\theta$落在$B$中,设它的极小多项式为$f(X)\in A[X]$.记$C=A[\theta]$,记$J=\{x\in B\mid xB\subseteq C\}$,这是$B$的最大的落在子环$C$中的理想.如果非零素理想$p\subseteq A$满足$pB$和$J$互素,设在$A/p$中有不可约多项式分解$\overline{f}(X)=\prod_{1\le i\le r}\overline{f_i}(X)^{e_1},f_i\in A[X]$.那么$q_i=pB+f_i(\theta)B,1\le i\le r$是$B$中的全体不同的$p$的提升素理想.那么$q_i$的剩余类次数为$\deg\overline{f_i}(X)$,它的分歧次数为$e_i$.此时有素理想分解$pB=\prod_{1\le i\le r}q_i^{e_i}$.
	\begin{proof}
		
		记$C=A[\theta]$.这个证明的核心是$\overline{A}[X]\to B/pB$,$\overline{g}(X)\mapsto g(\theta)$可诱导出同构$\overline{A}[X]/(\overline{f}(X))\cong B/pB$.它的核为$(\overline{f}(X))$等价于要求$pB\cap C=pC$.它是满射等价于要求$pB+C=B$.一旦诱导出了同构,结论就直接得出了.条件$pB+J=B$唯一的用处是推出这两个等价条件:
		\begin{enumerate}
			\item $pB\cap C=pC$:一方面$pC\subseteq pB\cap C$,另一方面,按照$pB$和$J$是$B$中互素的两个理想,我们解释过这推出$p=pB\cap A$和$J\cap A$是$A$中两个互素的理想,于是$pB\cap C=(pB\cap C)A=(pB\cap C)(p+J\cap A)\subseteq pC$.
			\item $C+pB=B$:仅需注意到$B=C+pB\subseteq J+pB=B$.
		\end{enumerate}
		
		现在在$A/p$中把$\overline{f}(X)$分解为不可约多项式的乘积,按照中国剩余定理,得到$\overline{A}[X]/(\overline{f}(X))\cong\oplus_{1\le i\le r}\overline{A}[X]/(\overline{f_i}(X))^{e_i}$.这说明$R=\overline{A}[X]/(\overline{f}(X))$的全部素理想为$(\overline{f_i})$,每一个是由$\overline{f_i}(X)$在$\mathrm{mod}\overline{f}(X)$下生成的主理想,另外这里$[R/(\overline{f_i}):\overline{A}]$恰好就是$\deg\overline{f_i}$.
		
		于是在上述同构下,每个$(\overline{f_i})$对应$B/pB$的一个素理想$(\overline{f_i}(\theta))/pB$,它又对应于$B$的一个素理想$q_i=pB+f_i(\theta)B$.自然有$e_i$就是它的分歧指数,并且$pB=q_1^{e_1}q_2^{e_2}\cdots q_r^{e_r}$.最后$\overline{B}/\overline{q_i}=B/q_i$,于是$\deg\overline{f_i}=[R/(\overline{f_i}):\overline{A}]=[B/q_i:A/p]$是剩余类次数.
	\end{proof}
    \item 特别的对于数域的情况,如果数域的扩张$K\subseteq L$存在幂元整基,也即有$B=A[\theta]$,那么上述定理的条件是满足的.
\end{enumerate}

素理想分歧性的定义.
\begin{enumerate}
	\item $A$的素理想$p$称为完全分裂的,如果分裂次数恰好就是扩张维数$g=n$.这也等价于它的每个提升素理想的惯性次数$f_i$和分歧指数$e_i$都是1.
	\item $A$的素理想$p$称为非分裂的,如果分裂次数是1.
	\item 一个提升素理想称为非分歧的,如果它的分歧指数是1,并且剩余域扩张是可分的(数域扩张时候这个条件总成立).否则称这个提升素理想是分歧的.
	\item 一个提升素理想称为完全分歧的,如果它是分歧的,并且惯性次数是1.
	\item $A$的素理想$p$称为非分歧的,如果它的每个提升素理想都是非分歧的.也就是说,每个提升素理想的分歧指数都是1,并且每个提升素理想处的剩余域扩张都是可分的.否则称为分歧的素理想.
	\item $A$的素理想$p$称为惯性的,如果分裂次数是1,并且唯一的提升素理想的分歧指数是1,此时唯一的提升素理想的惯性次数就是扩张维数.
\end{enumerate}

关于分歧素理想.
\begin{enumerate}
	\item 如果$K\subseteq L$是可分扩张,那么$K$中只有有限多个非零素理想在$L$中分歧.
	\begin{proof}
		
		设$\theta\in B$是一个本原元,取它的极小多项式$f(X)\in A[X]$.如果$A$的非零素理想$p$和$J$互素,这里$J$是$B$在$A[\theta]$中的最大理想,那么此时Kummer定理成立.此时$p$是非分歧的,等价于它的每个提升素理想的分歧指数为1,并且对应的剩余域扩张是可分的.我们断言这等价于$f(X)$的判别式$d=d_{L/K}(1,\theta,\cdots,\theta^{n-1})\in A$和$p$互素:分歧指数都为1等价于$\overline{f}(X)$没有重因式,等价于$\overline{f}(X)$的判别式不为零,等价于$d$在$\mathrm{mod}p$下不为零,等价于$d$和$p$互素.而此时Kummer定理告诉我们剩余域扩张就是$\overline{A}\subset\overline{A}[X]/(\overline{f_i}(X))$.按照$\overline{f}(X)$没有重因式说明$\overline{f_i}(X)$也没有,于是此时扩张是可分的.
		
		于是$K$中的分歧素理想只能出现在和$J$或和$d$不互素的情况,这只设计到有限个素理想,于是分歧素理想只有有限个.
	\end{proof}
    \item 有理素数$p$在数域$K$上分歧当且仅当$p\mid d(K)$.
    \begin{proof}
    	
    	设$K$是数域,$p$是有理素数,设它有分解$p\mathscr{O}_K=q_1^{e_1}q_2^{e_2}\cdots q_g^{e_g}$.记$B=\mathscr{O}_k/p\mathscr{O}_K$,记$B_i=\mathscr{O}_k/q_i^{e_i}$,那么$B=\oplus_{1\le i\le g}B_i$.这里$B_i$是$e_if_i$维$\mathbb{F}_p$线性空间,设它的一组基为$\eta_1,\eta_2,\cdots,\eta_{e_if_i}$.我们断言$d_{B_i/\mathbb{F}_p}(\eta_1,\eta_2,\cdots,\eta_{e_if_i})=0$当且仅当$e_i\ge2$.为了方便起见我们略去全部角标.我们用$\overline{x}$表示一个元在$\mathrm{mod}q^e$下的像.
    	
    	假设$e\ge2$,任取$\pi\in q-q^2$,那么$\overline{\pi}\not=0$,$\overline{\pi}^e=0$.可取$B_i$的一组$\mathbb{F}_p$基$\{\xi_1,\xi_2,\cdots,\xi_{ef}\}$使得$\xi_1=\overline{\pi}$.那么$(\overline{\pi}\xi_j)^e=0$.记矩阵$A_j$满足$(\overline{\pi}\xi_j)(\xi_1,\xi_2,\cdots,\xi_{ef})=(\xi_1,\xi_2,\cdots,\xi_{ef})A_j$.于是$A_j^e=0$,于是$A_j$是幂零矩阵,于是$0=\mathrm{T}(A_j)=\mathrm{T}(\xi_i\xi_j)$.于是$d_{B_i/\mathbb{F}_p}(\xi_1,\xi_2,\cdots,\xi_{ef})=\det((\mathrm{T}(\xi_i\xi_j)))=0$,于是$d_{B_i/\mathbb{F}_p}(\eta_1,\eta_2,\cdots,\eta_{ef})=0$.
    	
    	假设$e=1$,那么$B=\mathscr{O}_k/q\mathscr{O}_K$是域,并且是$\mathbb{F}_p$的$f$次扩张,于是$B$是$p^f$元域,于是$\mathbb{F}_p\subseteq B$是单扩张,可记$B=\mathbb{F}_p(\overline{\lambda})$.这里$\lambda$的极小多项式恰好就是特征多项式,可取$B$上一组$\mathbb{F}_p$基为$\{1,\overline{\lambda},\overline{\lambda}^2,\cdots,\overline{\lambda}^{f-1}\}$.并且每个$\overline{\lambda}^j$的全部共轭元恰好就是$\overline{\lambda}^{jp^i},0\le i\le p-1$.此时判别式$d_{B/\mathbb{F}_p}(1,\overline{\lambda},\cdots,\overline{\lambda}^{f-1})$是一个范德蒙行列式的平方,它是非零的.
    	
    	现在我们证明原命题.设$\{\omega_1,\cdots,\omega_n\}$是一组$B$上的整基,设$\overline{\omega_i}$是它在$\mathrm{mod}p$下的像,那么$p\mid d(K)$当且仅当$d_{B/\mathbb{F}_p}(\overline{\omega_1},\overline{\omega_2},\cdots,\overline{\omega_n})=0$.现在注意取每个$B_i$上的一组整基可以凑出$B=\oplus_{1\le i\le g}B_i$上的一组整基,此时$B$上的判别式就是每个$B_i$上对应的判别式的乘积.于是$d_{B/\mathbb{F}_p}(\overline{\omega_1},\overline{\omega_2},\cdots,\overline{\omega_n})=0$当且仅当存在某个$i$使得$d_{B_i/\mathbb{F}_p}=0$,这等价于$q_i$是分歧的,于是等价于$p$在$K$中分歧.
    \end{proof}
    \item 上一条结论可推广到一般戴德金整环的扩张,在后面我们会定义这种一般扩张下的判别式.【】
\end{enumerate}

二次数域上有理素数的分歧性.按照基本关系式$n=\sum_ie_if_i$,一个有理素数$p$在一个二次数域上的分解要么分解为同一个素理想的二次方$p=q^2$,要么分解为两个不同素理想的乘积$p=q_1q_2$.为了完全分类这两种情况,我们先引入勒让德符号.
\begin{enumerate}
	\item 勒让德符号.如果$p$是有理素数,$a$是和$p$互素的有理整数,如果$x^2\equiv a(\mathrm{mod}p)$在$\mathbb{Z}/p$下有解,就称$a$是$p$的二次剩余,并且记勒让德符号$\left(\frac{a}{p}\right)=1$,如果它无解,就称$a$是$p$的非二次剩余,并记勒让德符号$\left(\frac{a}{p}\right)=-1$.有时候会约定当$a$和$p$不互素的时候$\left(\frac{a}{p}\right)=0$.
	\item 固定一个有理素数$p$,我们知道有限域$\mathbb{F}_p$的乘法群$\mathbb{F}_p^*$是一个$p-1$阶的循环群,于是$(\mathbb{F}_p^*)^2$是它的指数为2的子群.勒让德符号$\left(\frac{\bullet}{p}\right)$事实上就是典范映射$\mathbb{Z}\to\mathbb{F}_p^*\to\mathbb{F}_p^*/(\mathbb{F}_p^*)^2\cong\mathbb{Z}/2$.这里最后的$\mathbb{Z}/2$视为$\{-1,1\}$上的乘法群.特别的,这说明勒让德符号是乘性的:
	$$\left(\frac{ab}{p}\right)=\left(\frac{a}{p}\right)\left(\frac{b}{p}\right)$$
	\item 欧拉准则.设$p$是有理素数,$a$是有理整数,那么总有$\left(\frac{a}{p}\right)\equiv a^{\frac{p-1}{2}}(\mathrm{mod}p)$.
	\begin{proof}
		
		如果$a$是平方剩余,那么有$a=x^2(\mathrm{mod}p)$,于是$a^{\frac{p-1}{2}}\equiv x^{p-1}\equiv1(\mathrm{mod}p)$;如果$a$是非平方剩余,如果记$\mathbb{F}_p^*$的一个生成元为$t$,那么有$a=t^{2s-1}$,于是只需说明$t^{\frac{p-1}{2}}\equiv-1(\mathrm{mod}p)$.但是$t^{\frac{p-1}{2}}$是满足$x^2\equiv1$的不为1的元,这只能有$x\equiv-1$.
	\end{proof}
    \item $\left(\frac{-1}{p}\right)=(-1)^{\frac{p-1}{2}}$.
    \item $\left(\frac{2}{p}\right)=(-1)^{\frac{p^2-1}{8}}$.
    \begin{proof}
    	
    	在环$\mathbb{Z}[i]$中我们有$(1+i)^p=(1+i)i^{\frac{p-1}{2}}2^{\frac{p-1}{2}}$.于是$\mathrm{mod}p$下就有:
    	$$\left(\frac{2}{p}\right)(1+i)i^{\frac{p-1}{2}}\equiv1+i(-1)^{\frac{p-1}{2}}(\mathrm{mod}p)$$
    	
    	如果$\frac{p-1}{2}$是奇数$2s-1$,得到$\left(\frac{2}{p}\right)=(-1)^s$,如果$\frac{p-1}{2}$是偶数$2s$,得到$\left(\frac{2}{p}\right)=(-1)^s$.这两种情况可以整理为上述结论.
    \end{proof}
    \item 高斯互反律.设$l$和$p$是两个不同的奇素数,那么有等式:
    $$\left(\frac{l}{p}\right)\left(\frac{p}{l}\right)=(-1)^{\frac{l-1}{2}\cdot\frac{p-1}{2}}$$
    \begin{proof}
    	
    	设$\zeta$是一个$l$次本原根,考虑数环$\mathbb{Z}[\zeta]$.定义高斯和$\tau=\sum_{a\in(\mathbb{Z}/l\mathbb{Z})^*}\left(\frac{a}{l}\right)\zeta^a$.我们先来证明$\tau^2=\left(\frac{-1}{l}\right)l$.
    	
    	任取$a,b\in(\mathbb{Z}/l\mathbb{Z})^*$,记$c=ab^{-1}$,那么有:
    	\begin{align*}
    	\left(\frac{-1}{l}\right)\tau^2&=\sum_{a,b}\left(\frac{-ab}{l}\right)\zeta^{a+b}=\sum_{a,b}\left(\frac{ab^{-1}}{l}\right)\zeta^{a-b}=\sum_{b,c}\left(\frac{c}{l}\right)\zeta^{bc-b}\\&=\sum_{c\not=1}\left(\frac{c}{l}\right)\sum_b\zeta^{b(c-1)}+\sum_b\left(\frac{1}{l}\right)\\&=-\sum_{c\not=1}\left(\frac{c}{l}\right)+l-1=l
    	\end{align*}
    	
    	接下来一方面有:
    	$$\tau^p=\tau(\tau^2)^{\frac{p-1}{2}}\equiv\tau(-1)^{\frac{l-1}{2}\cdot\frac{p-1}{2}}\left(\frac{l}{p}\right),(\mathrm{mod}p)$$
    	
    	另一方面有:
    	$$\tau^p\equiv\sum_a\left(\frac{a}{l}\right)\zeta^{ap}\equiv\left(\frac{p}{l}\right)\sum_a\left(\frac{ap}{l}\right)\zeta^{ap}\equiv\left(\frac{p}{l}\right)\tau,(\mathrm{mod}p)$$
    	
    	这两个等式得到原命题.
    \end{proof}
    \item 现在我们完全解决有理素数在二次域上的分解问题.按照基本量的关系式可知,扩张为2的情况下只会出现三种情况:要么$p$在$K$上分歧,此时$p\mathscr{O}_K=q^2$;要么$p$在$K$上完全分裂,此时$p\mathscr{O}_K=q_1q_2$;要么$p$在$K$上惯性,此时$p\mathscr{O}_K=q$:
    $$\left\{\begin{array}{ccccc}p\text{在}K\text{上分歧}&p\mathscr{O}_K=q^2&;&p\mid d(K)&\\p\text{在}K\text{上完全分裂}&p\mathscr{O}_K=q_1q_2&;&p\not| d(K)&\left(\frac{d}{p}\right)=1\\p\text{在}K\text{上惯性}&p\mathscr{O}_K=q&;&p\not| d(K)&\left(\frac{d}{p}\right)=-1\end{array}\right.$$
    \begin{proof}
    	
    	这个证明完全就是Kummer定理的应用.先设$d\equiv2,3(\mathrm{mod}4)$,此时$\mathscr{O}_K=\mathbb{Z}[\sqrt{d}]$,此时$d(K)=4d$,并且$\sqrt{d}$的极小多项式为$x^2-d$.
    	
    	设$p=2$,这时候$p$总是整除判别式的,于是此时有理素理2在$\mathbb{Q}(\sqrt{d})$中总是分歧的.
    	\begin{enumerate}
    		\item 当$d\equiv2(\mathrm{mod}4)$时,$x^2-d\equiv x^2(\mathrm{mod}2)$,有$2\mathscr{O}_K=q^2$,其中$q=(2,\sqrt{d})$.
    		\item 当$d\equiv3(\mathrm{mod}4)$时,此时$x^2-d\equiv(x+1)^2$,此时$2\mathscr{O}_K=q^2$,其中$q=(2,\sqrt{d}+1)$.
    	\end{enumerate}
    	
    	设$p\ge3$.
    	\begin{enumerate}
    		\item 如果$p\mid d(K)=4d$,那么$p$在$\mathbb{Q}(\sqrt{d})$中分歧,此时$x^2-d\equiv x^2(\mathrm{mod}p)$,有$p\mathscr{O}_K=q^2$,$q=(p,\sqrt{d})$.
    		\item 如果$p\not| d(K)$.假设$\left(\frac{d}{p}\right)=1$,那么可记在$\mathrm{mod}p$下有$a^2\equiv d$,此时$x^2-d\equiv(x-a)(x+a)$,于是$p\mathscr{O}_K=q_1q_2$,其中$q_1=(p,\sqrt{d}-a)$和$q_2=(p,\sqrt{d}+a)$.
    		\item 如果$p\not| d(K)$.假设$\left(\frac{d}{p}\right)=-1$,那么在$\mathrm{mod}p$下$x^2-d$已经是不可约多项式,于是$p\mathscr{O}_K=q$.
    	\end{enumerate}
    	
    	现在设$d\equiv1(\mathrm{mod}4)$,此时$\mathscr{O}_K=\mathbb{Z}[\frac{1+\sqrt{d}}{2}]$,此时$d(K)=d$,并且$\frac{1+\sqrt{d}}{2}$的极小多项式是$x^2-x+\frac{1-d}{4}$.后面证明是类似的.
    \end{proof}
\end{enumerate}


\newpage
\subsection{类群和单位群}

设$K$是一个数域,那么$\mathscr{O}_K$中的全部分式理想在乘法下构成一个交换群,记作$J_k$.全体主分式理想构成它的子群,记作$P_K$.考虑群同态$K^*\to J_K$为$x\mapsto(x)$.粗略的讲这个同态刻画数域$K$上与元素分解唯一性的差异程度.这个映射的核是单位群$\mathscr{O}_K^*$,它的余核$J_K/P_K$称为数域的理想类群,记作$\mathrm{Cl}_K$.
$$\xymatrix{1\ar[rr]&&\mathscr{O}_K^*\ar[rr]&&K^*\ar[rr]&&J_K\ar[rr]&&\mathrm{Cl}_K\ar[rr]&&1}$$

本节我们用Minkowiski理论证明关于单位群和类群的两个定理.对于单位群,我们证明Dirichlet单位定理:$\mathscr{O}_K^*\cong\mu_K\times\mathbb{Z}^{r_1+r_2-1}$;对于理想类群,我们证明它总是一个有限群.

格(lattice).设$V$是$\mathbb{R}$上的$n$维线性空间,$V$的有限生成自由子群$\Gamma$称为一个格.换句话讲它是$V$的子集$\Gamma=\mathbb{Z}v_1+\mathbb{Z}v_2+\cdots+\mathbb{Z}v_m$,其中$\{v_1,\cdots,v_m\}$是$V$的一个线性无关组,它称为这个格的基.称集合$\Phi=\{x_1v_1+\cdots+x_mv_m\mid x_i\in\mathbb{R},0\le x<1\}$为这个格的基本网孔.如果这个格的基恰好就是$V$的基,就称格是完备的.
\begin{enumerate}
	\item $n$维$\mathbb{R}$线性空间$V$的加法子群$\Gamma$是格当且仅当它是一个离散子群,也即$V$的范数诱导的拓扑上($\mathbb{R}$上有限维赋范空间上范数诱导的拓扑总是一致的),这个子群是一个离散空间.
	\begin{proof}
		
		必要性.设这个格的一组基是$\{v_1,v_2,\cdots,v_m\}$.把它延拓维$V$上的一组基$\{v_1,v_2,\cdots,v_n\}$.任取$\gamma=a_1v_1+a_2v_2+\cdots+a_mv_m\in\Gamma$.那么$\{x_1v_1+x_2v_2+\cdots+x_nv_n\mid x_i\in\mathbb{R},|a_i-x_i|<1,\forall 1\le i\le m\}$是$\gamma$的开邻域并且不包含$\Gamma$中的任意其他点.
		
		充分性.任取$V$的一个离散子群$\Gamma$.先证明$\Gamma$是闭集:任取0的开邻域$U$,那么可以取包含于$U$中的足够小的开邻域$U'$,使得$U'$中任意两个元的差在$U$中.现在如果可取$x\in\overline{\Gamma}-\Gamma$,那么在$x$的邻域$x+U'$中可以找到两个不同的$\Gamma$中的元$\gamma_1,\gamma_2$.但是这导致$0\not=\gamma_1-\gamma_2\in U$,这导致0不是一个孤立点,矛盾.
		
		设$V_0$为加法子群$\Gamma$生成的$\mathbb{R}$线性子空间.设它的维数为$m$,于是可以取$\Gamma$包含的长度恰为$m$的$\mathbb{R}$极大线性无关组$\{u_1,u_2,\cdots,u_m\}$.记$\Gamma_0=\mathbb{Z}u_1+\mathbb{Z}u_2+\cdots+\mathbb{Z}u_m\subset\Gamma$,这是线性空间$V_0$的完备格.我们断言指数$[\Gamma:\Gamma_0]$是有限的.
		
		取$\Gamma_0$在$\Gamma$中的一组陪集表示为$\gamma_i\in\Gamma,1\le i\le m$.按照$\Gamma_0$在$V_0$中完备,说明它的基本网孔$\Phi_0$关于完备格的平移覆盖了整个$V_0$.于是有$\gamma_i=\mu_i+\gamma_i'$,其中$\mu_i\in\Phi_0$,$\gamma_i'\in\Gamma_0$.注意到$\mu_i$两两不同,并且$\mu_i\in\Gamma\cap\overline{\Phi_0}$,这里$\overline{\Phi_0}$是有界闭集,于是它是紧集,但是离散的紧集是有限点集,于是$\mu_i$只有有限多个.记$q=[\Gamma:\Gamma_0]$.那么有$q\Gamma\subset\Gamma_0$.于是得到:
		
		$$\Gamma\subset\frac{1} {q}\Gamma_0=\mathbb{Z}\left(\frac{1}{q}u_1\right)+\mathbb{Z}\left(\frac{1}{q}u_2\right)+\cdots+\mathbb{Z}\left(\frac{1}{q}u_m\right)$$
		
		于是$\Gamma$是一个有限秩自由交换群的子群,于是存在一个$r\le m$和$r$个$\mathbb{Z}$线性无关元$\{v_1,\cdots,v_r\}$满足$\Gamma=\mathbb{Z}v_1+\cdots+\mathbb{Z}v_r$.但是注意到$v_1,\cdots,v_r$生成了$m$维$\mathbb{R}$线性空间$V_0$,导致$r=m$并且$\{v_1,\cdots,v_r\}$是$\mathbb{R}$线性无关的.这就说明了$\Gamma$是格.
	\end{proof}
    \item $\mathbb{R}$线性空间$V$上的格$\Gamma$是完备的,当且仅当存在一个有界子集$M\subseteq V$,满足$M$关于格$\Gamma$的全部平移覆盖了整个空间$V$.格的完备性等价于说全部基本网孔平移$\Phi+\gamma,\gamma\in\Gamma$覆盖了整个$V$.
    \begin{proof}
    	
    	一方面如果$\Gamma$是完备的,可以直接取基本网孔为所求的一个有界子集.现在假设$M$是有界子集,并且关于$\Gamma$的平移覆盖了整个$V$.取$\Gamma$生成的线性子空间$V_0$,这是一个闭子集.现在任取$v\in V$.那么$v$落在某个$M+\gamma$中,其中$\gamma\in\Gamma$.于是可以写对任意的正整数$n$,有$nv=a_n+\gamma_n$,其中$a_n\in M$,$\gamma_n\in\Gamma\subseteq V_0$.按照$M$有界,得到$\frac{a_n}{n}$收敛于0,于是得到$v=\lim_{n\to\infty}\frac{\gamma_n}{n}\in V_0$.于是$V=V_0$,得证.
    \end{proof}
\end{enumerate}

现在设$V$是一个有限维欧式空间,此即对$n$维$\mathbb{R}$线性空间上赋予一个对称正定双线性型$\langle-,-\rangle:V\times V\to\mathbb{R}$.这样$V$上具有体积概念,或者准确的说Haar测度:给定$n$格向量$\{v_1,v_2,\cdots,v_n\}$,定义$\Phi(v_1,v_2,\cdots,v_n)=\{x_1v_1+x_2v_2+\cdots+x_nv_n\mid x_i\in\mathbb{R},0\le x_i<1\}$,定义对于标准正交基$\{e_i\}$,有$\mathrm{Vol}(\Phi(e_i))=1$;对一般的$n$个向量$\{v_i\}$,定义$\mathrm{Vol}(\Phi(v_i))=|\det A|$,其中$A$是这个向量组在任一标准正交基下的矩阵表示.另外注意到$(\langle v_i,v_j\rangle)=AA^t$,说明$\mathrm{Vol}(\Phi(v_i))=\sqrt{|\det(\langle v_i,v_j\rangle)|}$.

对欧式空间$V$上的完备格$\Gamma$,定义它的体积是它的一组$\mathbb{Z}$基对应的基本网孔的体积.按照不同$\mathbb{Z}$基之间的过渡矩阵是$\mathbb{Z}$上的可逆矩阵,这样的矩阵的行列式的平方是1,说明这个体积定义不依赖于$\mathbb{Z}$基的选取.

\textbf{Minkowski格点定理}.内积空间$V$的子集$X$称为中心对称的,如果对任意$x\in X$,都有$-x\in X$;它称为凸的,如果对任意$x,y\in X$,都有$\{ty+(1-t)x\mid 0\le t\le1\}\subseteq X$.定理断言,如果$\Gamma$是$n$维欧式空间$V$上的完备格,假设$X\subseteq V$是中心对称和凸的子集,如果$\mathrm{Vol}(X)>2^n\mathrm{Vol}(\Gamma_0)$,那么$X$至少包含一个非零的格点$\gamma\in\Gamma$.
\begin{proof}
	
	只要证明存在两个不同的格点$\gamma_1,\gamma_2\in\Gamma$.满足$\left(\frac{1} {2}X+\gamma_1\right)\cap\left(\frac{1}{2}X+\gamma_2\right)\not=\emptyset$.这就保证存在了$x_1,x_2\in X$使得$\frac{1} {2}x_1+\gamma_1=\frac{1}{2}x_2+\gamma_2$.于是得到$\gamma_1-\gamma_2=\frac{1}{2}x_2-\frac{1}{2}x_1$.按照中心对称和凸,得到$\gamma_1-\gamma_2\in X$完成证明.
	
	现在假设集合族$\frac{1}{2}X+\gamma,\gamma\in\Gamma$是两两无交的.那么$\Phi\cap\left(\frac{1}{2}X+\gamma\right)$也是两两无交的.于是导致:
	$$\mathrm{Vol}(\Phi)\ge\sum_{\gamma\in\Gamma}\mathrm{Vol}\left(\Phi\cap(\frac{1}{2}X+\gamma)\right)$$
	
	但是如果把$\Phi\cap\left(\frac{1}{2}X+\gamma\right)$平移$-\gamma$,体积应该是相同的,也就是说$\left(\Phi-\gamma\right)\cap\frac{1}{2}X$具有相同体积,但是$\Phi-\gamma,\gamma\in\Gamma$覆盖了整个$V$,于是自然也覆盖了$\frac{1} {2}X$.于是得到矛盾:
	$$\mathrm{Vol}(\Phi)\ge\sum_{\gamma\in\Gamma}\mathrm{Vol}\left((\Phi-\gamma)\cap\frac{1}{2}X\right)=\mathrm{Vol}(\frac{1}{2}X)=\frac{1}{2^n}\mathrm{Vol}(X)$$
\end{proof}

数域上的Minkowiski空间.
\begin{enumerate}
	\item 设$K$是一个$n$次数域,有如下$\mathbb{C}$线性空间同构,这个空间记作$K_{\mathbb{C}}$.
	$$K\otimes_{\mathbb{Q}}\mathbb{C}\cong\prod\limits_{\sigma\in\mathrm{Hom}_{\mathbb{Q}}(K,\mathbb{C})}\mathbb{C}$$
	\begin{proof}
		
		$k\otimes z\mapsto(\sigma(k)z)_{\sigma}$是一个$\mathbb{C}$线性空间映射.它是单射因为,倘若$\forall\sigma$有$\sigma(k)z=0$,如果$z=0$,那么$k\otimes z=0$;如果$z\not=0$,从$\sigma(k)=0$说明$k=0$.现在这两个空间作为$\mathbb{C}$上线性空间的维数都是$n$,这说明它们之间的满线性映射实际上是同构.
	\end{proof}
    \item 共轭映射和$K_{\mathbb{R}}$.我们定义$K_{\mathbb{C}}$上的共轭映射$F$为,在张量积表示下为$k\otimes z\mapsto k\otimes\overline{z}$.这个映射在第二个表示下为$(z_{\sigma})_{\sigma}\mapsto(\overline{z_{\overline{\sigma}}})_{\sigma}$.我们定义$K_{\mathbb{C}}$的在$F$下不变的元构成的子群是$K_{\mathbb{R}}$.它在第二个表达式下是全体这样的元构成的子群,它的$\sigma$分量和$\overline{\sigma}$分量是互为共轭的.
    \item 定义内积.给定两个元$x,y\in K_{\mathbb{C}}$,定义$\langle x,y\rangle=\sum_{\sigma}x_{\sigma}\overline{y_{\sigma}}$.这是$K_{\mathbb{C}}$上的Hermitian内积,即满足:
    \begin{enumerate}
    	\item 第一个分量是线性的:
    	$$\langle z_1x_1+z_2x_2,y\rangle=z_1\langle x_1,y\rangle+z_2\langle x_2,y\rangle$$
    	\item 第二个分量是共轭线性的:
    	$$\langle x,z_1y_1+z_2y_2\rangle=\overline{z_1}\langle x,y_1\rangle+\overline{z_2}\langle x,y_2\rangle$$
    	\item 共轭对称性:
    	$$\langle x,y\rangle=\overline{\langle y,x\rangle}$$
    	\item 正定性:
    	$$\langle x,x\rangle>0,x\not=0$$
    \end{enumerate}
    \item 这个Hermition内积限制在$K_{\mathbb{R}}$上是实线性空间上的内积.因为如果$x,y\in K_{\mathbb{R}}$有$\langle x,y\rangle=\overline{\langle x,y\rangle}\in\mathbb{R}$.称赋予这个内积的$n$维欧式空间$K_{\mathbb{R}}$为数域$K$的Minkowiski空间.
    \item 我们来构造一个典范的映射$f:K_{\mathbb{R}}\to\prod_{\sigma}\mathbb{R}$,使得Minkowiski空间具有更加标准的形式.设数域$K$上有$r_1$个实嵌入,$r_2$对复嵌入,我们用$\rho$表示实嵌入,用$\tau$表示复嵌入.任取$(z_{\sigma})_{\sigma}\in K_{\mathbb{R}}$,定义$f$把它映射为$(x_{\sigma})_{\sigma}$,使得:
    $$x_{\sigma}=\left\{\begin{array}{cc}z_{\sigma}&\sigma=\rho\\\mathrm{Re}(z_{\sigma})&\sigma=\tau\\\mathrm{Im}(z_{\sigma})&\sigma=\overline{\tau}\end{array}\right.$$
    
    这里我们是对每一对复嵌入$\tau$和$\overline{\tau}$指定了$x$在这个分量的取值是实部还是虚部.这个$f$是一个实线性同构,但它不是一个内积空间的同构.这个同构诱导的$\prod_{\sigma}\mathbb{R}$上的内积为$\langle x,y\rangle=\sum_{\sigma}\alpha_{\sigma}x_{\sigma}y_{\sigma}$,这里$\sigma$是实嵌入时$\alpha_{\sigma}=1$,是复嵌入时$\alpha_{\sigma}=2$,这称为Minkowiski度量.特别的,这说明Minkowiski空间上的体积$\mathrm{Vol}$与勒贝格测度定义的体积$\mathrm{Vol}_L$满足$\mathrm{Vol}(X)=2^{r_2}\mathrm{Vol}_L(X)$,其中$r_2$是$K$上复嵌入的对数.
    \begin{align*}
        \langle f^{-1}(x),f^{-1}(y)\rangle&=\sum_{\rho}f^{-1}(x)_{\rho}\overline{f^{-1}(y)_{\rho}}+\sum_{\tau}f^{-1}(x)_{\tau}\overline{f^{-1}(y)_{\tau}}\\&=\sum_{\rho}x_{\rho}y_{\rho}+\sum_{\tau,\overline{\tau}}\left((x_{\tau}+ix_{\overline{\tau}})(y_{\tau}-iy_{\overline{\tau}})+(x_{\tau}-ix_{\overline{\tau}})(y_{\tau}+iy_{\overline{\tau}})\right)\\&=\sum_{\sigma}\alpha_{\sigma}x_{\sigma}y_{\sigma} 
    \end{align*}
    \item 给定数域$K$,我们有典范嵌入$j:K\to K_{\mathbb{R}}$为$x\mapsto(\sigma(x))_{\sigma}$.如果我们定义$\mathrm{T}:K_{\mathbb{C}}\to\mathbb{C}$为分量之和,那么自然有$\mathrm{T}(jx)=\mathrm{T}_{K/\mathbb{Q}}(x)$.
    \item 记$K$是$n$次数域,我们解释过$O_K$的任一非0理想$I$都是秩$n$的自由交换群.于是$j(I)$是Minkowiski空间$K_{\mathbb{R}}$的完备格.它的基本网孔的体积为$\sqrt{|d_K|}[O_K:I]$.
    \begin{proof}
    	
    	取$I$的一组$\mathbb{Z}$基$\{\alpha_1,\alpha_2,\cdots,\alpha_n\}$.那么有$jI=\mathbb{Z}j\alpha_1+\cdots+\mathbb{Z}j\alpha_n$.设$\{\sigma_1,\sigma_2,\cdots,\sigma_n\}$为全部从$K$到$\mathbb{C}$的$\mathbb{Q}$嵌入.取矩阵$A=(\sigma_i\alpha_j)$,那么有$|\det A|^2=d(\alpha_1,\alpha_2,\cdots,\alpha_n)=[\mathscr{O}_K:I]^2d_K$.另一方面$\mathrm{Vol}(\Phi)=\sqrt{|\det(\langle j\alpha_i,j\alpha_j\rangle)|}=\sqrt{det(A\overline{A}^t)}=|\det A|$.
    \end{proof}
\end{enumerate}

类数有限定理.
\begin{enumerate}
	\item Minkowiski上界.如果正数$M$满足,对任意$\mathscr{O}_K$的非零理想$I$,总存在非零元$a\in I$使得$\mathrm{N}_K(a)\le M\mathrm{N}_K(I)$,就称$M$是一个Minkowiski上界.这个上界的存在性就可以证明数域上的类群$\mathrm{Cl}_K$总是有限的:
	\begin{proof}
		
		任给$\mathscr{O}_K$中的素理想$q$,设$q\cap\mathbb{Z}=(p)$,记$[\mathscr{O}_K/q:\mathbb{Z}/p]=f$,那么有$\mathrm{N}_K(q)=p^f$.于是我们证明了$\mathscr{O}_K$中的素理想的绝对范数总是有理素数幂.另外有理素数$p$的提升素理想只有有限个,于是对任意的正数$M$,只存在有限个素理想的绝对范数被$M$控制.按照$\mathscr{O}_K$是戴德金整环,每个非零理想都可以写作若干素理想的乘积,结合理想绝对范数的乘性,说明$\mathscr{O}_K$中只有有限个理想的绝对范数被$M$控制.
		
		现在取$M$是Minkowiski上界.为证明类群是有限的,只需说明对每个理想类$[I]$,都存在一个代表元$I_1$满足$\mathrm{N}_K(I)\le M$.为此,先任取一个代表元$I$,这是分式理想,按照定义可取非零元$\gamma\in\mathscr{O}_K$使得$J=\gamma I^{-1}$是一个整理想.按照Minkowiski上界的定义,可取非零元$b\in J$使得$\mathrm{N}_K(b)\le M\mathrm{N}_K(J)$.这说明$\mathrm{N}_K((b)J^{-1})=\mathrm{N}_K(b)\mathrm{N}_K(J)^{-1}\le M$.这里$(b)J^{-1}=(b\gamma^{-1})I$和$I$在同一个理想类中,得证.
	\end{proof}
    \item 引理.设$I$是$\mathscr{O}_K$的非零理想,如果$\{c_{\sigma},\sigma\in\mathrm{Hom}_{\mathbb{Q}}(K,\mathbb{C})\}$是一组正实数,满足$c_{\sigma}=c_{\overline{\sigma}}$和$\prod_{\sigma}c_{\sigma}>A[\mathscr{O}_K:I]$,其中$A=\left(\frac{2}{\pi}\right)^{r_2}\sqrt{|d_K|}$(这里$r_2$是复嵌入的对数).那么存在非零元$a\in I$使得$|\sigma(a)|<c_{\sigma},\forall\sigma\in\mathrm{Hom}_{\mathbb{Q}}(K,\mathbb{C})$.
    \begin{proof}
    	
    	取$X=\{(z_{\sigma})\in K_{\mathbb{R}}\mid |z_{\sigma}|<c_{\sigma}\}$.这是中心对称的,并且凸的点集.之前给出过$K_{\mathbb{R}}$到$\prod_{\sigma}\mathbb{R}$的同构$f$,有$f(X)=\{(x_{\sigma})\in\prod_{\sigma}\mathbb{R}\mid |x_{\rho}|<c_{\rho},x_{\tau}^2+x_{\overline{\tau}}^2<c_{\tau}^2\}$,这里$\rho$表示实嵌入,$\tau$表示复嵌入.按照这里$\prod_{\sigma}\mathbb{R}$上的体积形式是$2^{r_2}$倍的勒贝格体积,得到:
    	$$\mathrm{Vol}(X)=2^{r_2}\prod_{\rho}(2c_{\rho})\prod_{\tau}(\pi c_{\tau}^2)=2^{r_1+r_2}\pi^{r_2}\prod_{\sigma}c_{\sigma}$$
    	
    	于是按照条件,有$\mathrm{Vol}(X)>2^n\sqrt{|d_K|}[\mathscr{O}_K:I]=2^n\mathrm{Vol}(\Gamma)$.于是按照Minkowiski格点定理,得到存在非零元$a\in I$,使得$j(a)\in X$,也即$|\sigma(a)|<c_{\sigma}$.
    \end{proof}
	\item $M=\left(\frac{2}{\pi}\right)^{r_2}\sqrt{|d_K|}$是一个Minkowiski上界.即对$\mathscr{O}_K$的每个非零理想$I$,存在非0元$a\in I$满足$|\mathrm{N}_{K/\mathbb{Q}}(a)|\le\left(\frac{2}{\pi}\right)^{r_2}\sqrt{|d_K|}\mathrm{N}_K(I)$.
	\begin{proof}
		
		对任意$\varepsilon>0$,可取正实数组$\{c_{\sigma},\sigma\in\mathrm{Hom}_{\mathbb{Q}}(K,\mathbb{C})\}$使得使得$c_{\sigma}=c_{\overline{\sigma}}$,并且满足:
		$$\prod_{\sigma}c_{\sigma}=\left(\frac{2}{\pi}\right)^{r_2}\sqrt{|d_K|}\mathrm{N}_K(I)+\varepsilon$$
		
		于是按照上一个定理,可以找到非零元$a\in I$,满足$|\sigma(a)|<c_{\sigma},\forall\sigma$.于是:
		$$|\mathrm{N}_{K/\mathbb{Q}}(a)|=\prod|\sigma(a)|<\left(\frac{2}{\pi}\right)^{r_2}\sqrt{|d_K|}\mathrm{N}_K(I)+\varepsilon$$
		
		现在按照左侧这个范数是整数,于是可以取足够小的$\varepsilon$使得:
		$$|\mathrm{N} _{K/\mathbb{Q}}(a)|=\prod|\sigma(a)|\le\left(\frac{2}{\pi}\right)^{r_2}\sqrt{|d_K|}\mathrm{N}_K(I)$$
	\end{proof}
    \item 上一条中的上界有时候不够强,这一条我们证明$M=\frac{n!}{n^n}\left(\frac{4}{\pi}\right)^{r_2}\sqrt{|d_K|}$也是一个Minkowiski上界,其中$n$是扩张的次数.
    \begin{proof}
    	
    	我们只要把之前的引理中考虑的Minkowiski空间的子集改为子集$X=\{(z_{\sigma})\in K_{\mathbb{R}}\mid\sum|z_{\sigma}|<t\}$,此时有$\mathrm{Vol}(X)=2^{r_1}\pi^{r_2}\frac{t^n}{n!}$,类似可证明这个命题.
    \end{proof}
\end{enumerate}

例子.数域$K=\mathbb{Q}(\sqrt{-5})$的类数为2.它的Minkowiski上界为$\frac{2!}{2^2}\left(\frac{4}{\pi}\right)$约为$2.8$.于是绝对范数不超过2的理想生成了整个理想类群.绝对范数1的情况即对应单位理想,绝对范数2的情况,也即考虑有理素数2在$K$中的分解,我们证明过一般情况,有$2\mathscr{O}_K=(2,1+\sqrt{-5})^2$,于是$(2,1+\sqrt{-5})$是仅有的绝对范数为2的理想,它在类群中的阶是2,于是数域$K$的类数为2.

关于二次数域的类数问题.这个问题最早可以追溯到Gauss于1801年先后提出的三个问题:当d趋于正无穷时,虚二次数域$\mathbb{Q}(\sqrt{-d})$的类数趋于无穷,这里$d$是不含平方因子的正整数;寻找全体类数较低的虚二次数域;存在无穷多个实二次数域的类数为1.德国数学家Hans Heilbronn于1934年证明了对于每一个类数,仅存在有限个虚二次数域,从而证明了Gauss的第一个猜想.随后直到1954年德国数学家Kurt Heegner借助模形式和模方程正确证明了除了$d=-1,-2,-3,-7,-11,-19,-43,-67,-163$,不再有其他虚二次数域的类数为1.遗憾的是至少持续十五年内没有人相信Heegner的证明是正确的.直到后期Harold Stark和Bryan Birch的工作证实了Heegner的证明是正确的.与此同时Alan Baker发现了现在超越数论中的Baker定理,给出了虚二次数域类数1分类问题的另一角度证明,并且随后不久证明了$n=2$的情况.对于一般类数的分类问题直到Dorian Goldfeld才得到进展,他发现类数问题可以联系于椭圆曲线上的L-函数.该问题最新的进展是2004年给出了直到$n=100$的全部分类.对于实二次数域了解的很少,Gauss提出的第三个猜想至今仍为公开问题,事实上甚至不知道是否存在无穷个类数1的任意次的数域.

接下来讨论单位群.首先$j:K\to K_{\mathbb{C}}$可限制为同态$K^*\to K_{\mathbb{C}}^*$,这里$K_{\mathbb{C}}^*$表示分量均不取零的子集.再取映射$l:K_{\mathbb{C}}^*\to\prod_{\sigma}\mathbb{R}$为$(z_{\sigma})_{\sigma}\mapsto(\ln|z_{\sigma}|)_{\sigma}$.得到如下交换图,这里$\mathrm{N}(\cdot)$和$\mathrm{T}(\cdot)$分别是对分量取积与对分量取和的映射.这样得到如下交换图:
$$\xymatrix{K^*\ar[rr]^j\ar[d]_{\mathrm{N}_{K/\mathbb{Q}}}&&K^*_{\mathbb{C}}\ar[rr]^l\ar[d]_{\mathrm{N}}&&\prod_{\sigma}\mathbb{R}\ar[d]^{\mathrm{T}}\\\mathbb{Q}^*\ar[rr]&&\mathbb{C}^*\ar[rr]^{\ln|\cdot|}&&\mathbb{R}}$$

现在我们取这个图表在复共轭映射$F$下不变的图表:
$$\xymatrix{\mathscr{O}_K\ar[d]\ar[rr]^{j\mid\mathscr{O}_K}&&S\ar[d]\ar[rr]^{l\mid S}&&H\ar[d]\\K^*\ar[rr]^j\ar[d]_{\mathrm{N}_{K/\mathbb{Q}}}&&K^*_{\mathbb{R}}\ar[rr]^l\ar[d]_{\mathrm{N}}&&[\prod_{\sigma}\mathbb{R}]^+\ar[d]^{\mathrm{T}}\\\mathbb{Q}^*\ar[rr]&&\mathbb{R}^*\ar[rr]^{\ln|\cdot|}&&\mathbb{R}}$$
$$S=\{j(y)\mid\mathrm{N}_{K/\mathbb{Q}}(y)=\pm1\}$$
$$H=\{(x_{\sigma})\in[\prod_{\sigma}\mathbb{R}]^+\mid\mathrm{T}((x_{\sigma}))=0\}$$
$$[\prod_{\sigma}\mathbb{R}]^+=\{(x_{\sigma})\in\prod_{\sigma}\mathbb{R}\mid x_{\sigma}=x_{\overline{\sigma}}\}$$

于是这里$[\prod_{\sigma}\mathbb{R}]^+$可以典范的视为$\mathbb{R}^{r_1+r_2}$.在这个典范同构下,$\mathrm{T}$同样是取分量之和的映射,而$l(x)=(\ln|x_{\rho_1}|,\ln|x_{\rho_2}|,\cdots,\ln|x_{\rho_{r_1}}|,2\ln|x_{\tau_1}|,2\ln|x_{\tau_2}|,\cdots,2\ln|x_{\tau_{r_2}}|)$.另外$H$同构于$\mathbb{R}^{r_1+r_2-1}$.另外注意这里$\mathrm{N}$是$K_{\mathbb{R}}^*$作为$K_{\mathbb{C}}^*$子集的坐标表示下的分量相乘,而不是典范同构后的$\prod_{\sigma}\mathbb{R}$这个坐标下的分量相乘.

记$\lambda$是复合$\mathscr{O}_K^*\to S\to H$.我们断言这个映射的核是单位根群$\mu(K)$.
\begin{proof}
	
	一方面如果$x\in\mu(K)$,那么$j(x)$中每个分量都是模长为1的复数,这在$l$下的取值是零元,所以$\mu(K)\subset\ker\lambda$.另一方面任取$x\in\ker\lambda$,此即$|\sigma(x)|=1$对任意嵌入$\sigma$成立.于是$j(\ker\lambda)$落在Minkowiski空间$K_{\mathbb{R}}$的有界子集上,这里每个$j(x)$都是格$\mathscr{O}_K$中的点,离散子集和有界子集的交必然是有界集合.于是$\ker\lambda$是$\mathscr{O}_K^*$的有限子集.导致其中的点一定是单位根,于是$\ker\lambda\subset\mu(K)$.
\end{proof}

记$\Gamma=\lambda(\mathscr{O}_K^*)$,那么$\Gamma$是$H$中的完备格.于是特别的它同构于$\mathbb{Z}^{r_1+r_2-1}$.
\begin{proof}
	
	考虑集合$A=\{(x_{\sigma})\in\prod_{\sigma}\mathbb{R}\mid |x_{\sigma}|\le c\}$.那么有$l^{-1}(A)=\{(z_{\sigma})\in\prod_{\sigma}\mathbb{C}^*\mid e^{-c}\le|z_{\sigma}|\le e^c\}$是一个有界集合.于是$j(\mathscr{O}_K)\cap l^{-1}(A)$是一个有限集合.这说明$A$只包含了$j(\mathscr{O}_K^*)$中的有限个点.于是有$\lambda(\mathscr{O}_K^*)$在$H$中是格.
	
	为证明它是一个完备格,等价于找一个有界集合$M\subseteq H$,使得$H=\cup_{\gamma\in\Gamma}(M+\gamma)$.我们可以用满同态$l:S\to H$把问题转化为在$S$上的构造.也即找$S$中的一个有界集合$T$,使得$S=\cup_{\varepsilon\in\mathscr{O}_K^*}Tj(\varepsilon)$.
	
	但是这里我们还要说明下如果$T$是$S$中的有界子集,那么$M=l(T)$是$H$中的有界子集.这是因为,按照$T$是有界的,说明$T$中每个分量是有界的,于是$M$中点的每个分量$\ln|\cdot|$是上有界的.但是$M$中点的分量之和恒为1,于是每个分量上有界推出下有界.于是$M$是$H$中的有界子集.
	
	下面来构造$T$.首先取$X=\{(z_{\sigma})\in K_{\mathbb{R}}\mid |z_{\sigma}|<c_{\sigma}\}$,其中$c_{\sigma}$都是正实数.对于$y\in S$,平移后的集合就是$Xy=\{(z_{\sigma})\in K_{\mathbb{R}}\mid |z_{\sigma}|<c_{\sigma}|y_{\sigma}|\}$.并且按照$\prod_{\sigma}|y_{\sigma}|=1$,说明$\prod_{\sigma}c_{\sigma}=\prod_{\sigma}c_{\sigma}|y_{\sigma}|$.现在取正实数$c_{\sigma}$使得$c=\prod_{\sigma}c_{\sigma}>\left(\frac{2}{\pi}\right)^{r_2}\sqrt{|d_K|}$.于是按照之前的定理,存在非零元$a\in\mathscr{O}_K$使得$j(a)\in Xy^{-1}$.于是$y\in X(j(a))^{-1}$.于是有$0<|\mathrm{N}_{K/\mathbb{Q}}(a)|<c$.
	
	绝对范数有界的理想只有有限个,特别的这样的主理想也只有有限个.同一个主理想的不同代表元差一个单位,这说明固定$c$的时候,存在有限个元$\alpha_1,\alpha_2,\cdots,\alpha_N\in\mathscr{O}_K$,使得从$0<|\mathrm{N}_{K/\mathbb{Q}}(a)|<c$可推出$a\in\alpha_i\mathscr{O}_K^*$.于是有$y\in Xj(\alpha_i)^{-1}j(\varepsilon)$,其中$\varepsilon\in\mathscr{O}_K^*$.我们就取$T=S\cap\left(\cup_{1\le i\le N}Xj(\alpha_i)^{-1}\right)$.此时就有$S=\cup Tj(\varepsilon)$.
\end{proof}

数域$K$上的单位根群构成一个乘法群$\mu(K)$是一个有限循环群.
\begin{proof}
	
	设$w\in\mu(K)$,设它的极小多项式为$f(x)\in\mathbb{Q}[x]$.设$w^n=1$,于是$f(x)\mid x^n-1$,于是$f(x)$在$\mathbb{C}$中的每个元都满足$x^n=1$,于是特别的每个元的模长都是1,于是按照韦达定理,多项式$f(x)$的系数被$m$所控制.现在$m\le[K:\mathbb{Q}]$.说明能够成为单位根的极小多项式的多项式的个数被$[K:\mathbb{Q}]$所控制,换句话讲固定$K$的时候是有限的,于是单位根的个数是有限的.最后域的有限乘法子群总是循环的,于是这里$\mu(K)$是循环群.
\end{proof}

Dirichlet单位定理.设$K$是数域,那么单位群$\mathscr{O}_K^*\cong\mu(K)\oplus\mathbb{Z}^{r_1+r_2-1}$.它的无挠部分的一组基称为基本单位.于是数域上的每个单位可以唯一的表示为$\varepsilon=\zeta\varepsilon_1^{v_1}\varepsilon_2^{v_2}\cdots\varepsilon_t^{e_t}$,其中$t=r_1+r_2-1$,$v_i$是整数,$\zeta$是一个单位根,$\{\varepsilon_1,\varepsilon_2,\cdots,\varepsilon_t\}$是一个基本单位组.
$$\xymatrix{1\ar[r]&\mu(K)\ar[r]&\mathscr{O}_K^*\ar[r]&\Gamma\ar[r]&0}\Rightarrow\mathscr{O}_K^*\cong\mu(K)\oplus\Gamma$$
\newpage
\section{局部域}
\subsection{赋值和完备化}

Ostrowski定理完全刻画了数域$K$上的绝对值.
\begin{enumerate}
	\item 数域$K$上的阿基米德绝对值$|\bullet|$可以表示为$|x|=|\sigma(x)|^c_{\infty}$,这里$|\bullet|_{\infty}$表示复数域上的经典绝对值,$\sigma$表示$K\to\mathbb{C}$的一个$\mathbb{Q}$嵌入,$c$是满足$0<c\le1$的实数.
	\begin{proof}
		
		首先这里的$|\sigma(x)|^c_{\infty}$的确是一个阿基米德绝对值,其中阿基米德性是因为$|n|=|\sigma(n)|^c_{\infty}=n^c$是无界的.下面任取$K$上的阿基米德绝对值$|\bullet|$.按照$K$是$\mathscr{O}_K$的商域,我们只需验证存在$c>0$和$\mathbb{Q}$嵌入$K\to\mathbb{C}$使得$\forall x\in\mathscr{O}_K$有$|x|=|\sigma(x)|_{\infty}^c$.我们先断言存在$1\le c>0$使得对每个$m\in\mathbb{Z}$有$|m|=|m|_{\infty}^c$.
		
		\qquad
		
		首先总有$|\pm1|=1$,于是不妨设$m$是$>1$的整数.固定一个$>1$的整数$m_0$,把$m^N$写成$m_0$进制,也即记$m^N=\sum_{0\le n\le N\ln(m)/ln(m_0)}a_nm_0^n$,其中$0\le a_n<m_0$.记$\{|a|\mid0\le a<m_0\}$的一个上界为$A$.按照三角不等式,如果$|m_0|\le1$导致$|m|^N\le A(1+Nln(m)/ln(m_0))$.这个不等式两侧做$N$次根再令$N\to\infty$就导致$|m|\le1$.但是这里$m$是任意的,导致$|m|\le1$总成立,这和阿基米德性矛盾,于是必然有$m>1$时总有$|m|>1$.按照三角不等式,有$|m|^N\le A(1+Nln(m)/ln(m_0))|m_0|^{Nln(m)/ln(m_0)}$.两边取$N$次根再令$N\to\infty$,得到$|m|\le|m_0|^{ln(m)/ln(m_0)}$,此即$|m|^{1/ln(m)}\le|m_0|^{1/ln(m_0)}$.但是我们可以交换$m$和$m_0$的选取得到相反的不等式,于是对任意$>1$的整数$m$有定值$C=|m|^{1/ln(m)}$,换句话讲存在一个定值$c>0$使得$|m|=m^c$对任意$m>1$成立.其中$c=\ln C$,容易从$|m|\le m$得到$c\le1$,这证明了我们的断言.
		
		\qquad
		
		对非零元$\alpha\in\mathscr{O}_K$,我们对$K\to\mathbb{C}$的全部$\mathbb{Q}$嵌入做一个排序,使得在古典绝对值$|\bullet|_{\infty}$下总有$|\sigma_i(\alpha)|\ge|\sigma_{i+1}(\alpha)|$.注意这个排序是依赖$\alpha$的.下面记$\prod_{\sigma}(X-\sigma(\alpha^N))=X^n+a_{n-1}X^{n-1}+\cdots+a_0$和$P_m=\prod_{1\le i\le m}|\sigma_i(\alpha^N)|$.那么这里$\pm a_{n-m}$是$\{\sigma(\alpha^N)\}$的初等对称多项式,它是$\left(\begin{array}{c}n\\m\end{array}\right)$个由$\sigma(\alpha^N)$中$m$项做乘积的和,于是有$|a_{n-m}|\le\left(\begin{array}{c}n\\m\end{array}\right)P_m\le 2^nP_m$.另外如果有$|\sigma_{m+1}(\alpha)|<|\sigma_m(\alpha)|$,那么我们可以取$N$足够大使得$|a_{n-m}|>P_m/2$.最后这里$\alpha$是代数整数,所以实际上$a_{n-m}$都是整数,于是按照上一段有$|a_{n-m}|=|a_{n-m}|_{\infty}^c$.
		
		\qquad
		
		我们先断言存在指标$k$使得$|\alpha|=|\sigma_k(\alpha)|_{\infty}^c$.假设这不成立,那么$|\alpha|>|\sigma_1(\alpha)|_{\infty}^c$和$|\alpha|<|\sigma_n(\alpha)|^c$和$\exists k,|\sigma_{k+1}(\alpha)|_{\infty}^c<|\alpha|<|\sigma_k(\alpha)|^c$中至少有一个成立.以最后一种情况为例我们来推矛盾,其余情况是类似的.首先在$\prod_{\sigma}(X-\sigma(\alpha^N))=X^n+a_{n-1}X^{n-1}+\cdots+a_0$中带入$X=\alpha^N$,那么有$\sum_{0\le j\le n}a_j\alpha^{Nj}=0$.但是有:
		$$\left|\frac{a_{n-j}\alpha^{N(n-j)}}{a_{n-k}\alpha^{N(n-k)}}\right|\le\left(\frac{2^nP_j}{P_k/2}\right)^c|\alpha|^{N(k-j)}=\left(\frac{2^{n+1}P_j}{P_k}|\sigma_k(\alpha)|_{\infty}^{N(k-j)}\right)^c\left(\frac{|\alpha|}{|\sigma_k(\alpha)|^c_{\infty}}\right)^{N(k-j)}$$
		
		于是如果$k>j$,就有:
		$$\frac{P_j}{P_k}=\frac{1}{\prod_{j<i\le k}|\sigma_i(\alpha^N)|_{\infty}}\le\frac{1}{|\sigma_k(\alpha)|_{\infty}^{k-j}}$$
		
		导致:
		$$\left|\frac{a_{n-j}\alpha^{N(n-j)}}{a_{n-k}\alpha^{N(n-k)}}\right|\le2^{n+1}(c)\left(\frac{|\alpha|}{|\sigma_k(\alpha)|^c_{\infty}}\right)^{N(k-j)}$$
		
		这里$|\alpha|<|\sigma_k(\alpha)|_{\infty}^c$,于是让$N\to\infty$导致右侧是零,于是左侧的极限也是零.如果$j>k$有相同的结论.在$0=\sum_{0\le j\le N}a_{n-j}\alpha^{N(n-k)}$两侧除去$a_{n-k}\alpha^{N(n-k)}$,令$N\to\infty$导致$1=0$矛盾.这个矛盾就得到存在指标$k$使得$|\alpha|=|\sigma_k(\alpha)|_{\infty}^c$.但是我们还要说明$\sigma_k$是不依赖$\alpha$的.
		
		\qquad
		
		取$\mathscr{O}_K$中的两个非零元$\alpha,\beta$,先记$k=k(\alpha)$.我们有$|\sigma_{k(\alpha\beta^N}(\alpha\beta^N)|^c_{\infty}=|\alpha\beta^N|=|\alpha||\beta|^N=|\alpha||\sigma_{k(\beta)}(\beta)|^{cN}$.于是有:
		$$\frac{|\sigma_{k(\alpha\beta^N)}(\beta)|}{|\sigma_{k(\beta)}(\beta)|}=\left(\frac{|\alpha|}{|\sigma_{k(\alpha\beta^N)}(\alpha)|^c}\right)^{1/cN}$$
		
		但是注意$|\sigma_{k(\alpha\beta^N)}(\alpha)|_{\infty}^c$在$N$变动时只取有限个值,于是上述等式右侧在$N\to\infty$的极限就是1.另外由于$|\sigma_i(x)|_{\infty}/|\sigma_j(x)|_{\infty}$在$x$固定时只取有限个值,于是当$N$充分大时上述等式左侧就固定为1.于是有$N$足够大时$|\sigma_{k(\beta)}(\beta)|_{\infty}=|\sigma_{k(\alpha\beta^N)}(\beta)|_{\infty}$.这允许我们就取$k(\alpha\beta^N)=k(\beta)$.类似的当$N$足够大时考虑上述等式右侧有$|\alpha|=|\sigma_{k(\alpha\beta^N)}|$.于是同样的可以取$k(\alpha)=k(\alpha\beta^N)$.这得到$k(\alpha)=k(\beta)$,完成证明.
	\end{proof}
	
	
	
	
	
	
	\item 数域$K$上的无穷素位恰好有$r_1+r_2$个,其中$r_1$是$K$上实嵌入的个数,$r_2$是$K$上复嵌入的对数.当提及素位时,我们指的是一个具体的赋值代表元:每个实嵌入$\sigma$对应的代表元取为$c=1$,也即$|\sigma(x)|_{\infty}$;每个复嵌入取代表圆$c=2$,也即$|\sigma(x)|^2_{\infty}=\sigma(x)\overline{\sigma}(x)$.据此也把无穷素位分类为实素位和复素位.
	\begin{proof}
		
		倘若两个嵌入$\sigma_1$和$\sigma_2$是等价的,也即存在实数$c>0$,对任意的$x\in K$,恒有$|\sigma_1(x)|=|\sigma_2(x)|^c$,取$x=2$得到$c=1$.注意到$\lambda=\sigma_2\sigma_1^{-1}$必然是从$\sigma_1(K)$到$\sigma_2(K)$的同构.记$\sigma_i(K)$在$\mathbb{C}$中的完备化为$K_i,i=1,2$.如果$y\in K_1$,就存在一列$K$中的点列$(x_n)$使得$\sigma_1(x_n)$是关于$\mathbb{C}$中常义赋值的柯西列,并且$y=\lim_{n\to\infty}\sigma_1(x_n)$.那么$\lambda$可以延拓到$K_1$上,即定义$\lambda(y)=\lim_{n\to\infty}\sigma_2(x_n)$:
		
		事实上,按照$|\sigma_2(x_n)-\sigma_2(x_m)|=|\sigma_1(x_n)-\sigma_1(x_m)|$,得到$\sigma_2(x_n)$同样是$\mathbb{C}$中的柯西列,于是它收敛.现在要证明$\lambda(y)$的定义不依赖于柯西列$\sigma_1(x_n)$的选取.如果存在另一个柯西列$\sigma_1(x_n')$同样收敛到$y$,那么记数列$y_n$为$x_1,x_1',x_2,x_2',\cdots$,于是容易验证$\sigma_1(y_n)$是柯西列,于是收敛到子列的极限$y$.于是同样得到$\sigma_2(y_n)$是柯西列,于是它也收敛到子列收敛到的极限$\lim_{n\to\infty}\sigma_2(x_n)$,而$\lim_{n\to\infty}\sigma_2(x_n')$是另一个子列,它们自然收敛到相同的极限,于是$\lambda(y)$的定义不依赖于柯西列的选取.按照$\lambda\circ\sigma_1=\sigma_2$,得到这样定义的$\lambda:K_1\to K_2$的确是之前的$\lambda$的延拓.对偶的构造$\lambda^{-1}=\sigma_1\sigma_2^{-1}$的延拓$K_2\to K_1$.得到$\lambda$是从$K_1\to K_2$的同构.
		
		$\lambda$在0处是连续的:对任意$\sigma_1(x_n)\to0$,按照$|\sigma_1(x)|=|\sigma_2(x)|$,得到$\sigma_2(x_n)\to0$.于是,按照$\lambda$的线性,它在整个$\sigma_1(K)$上连续.于是按照构造的延拓到完备化,必然也是连续的.$\sigma_1(K)$包含了有理数域,导致$K_1$包含了实数域.最后证明一个$\mathbb{R}\subset\mathbb{C}$的中间域$F$,到$\mathbb{C}$的连续域映射$f$必然是恒等或者共轭.事实上按照$f(1)=1$,归纳得到$f(n)=n,\forall n\in\mathbb{Z}$.进而得到$f(q)=q,\forall q\in\mathbb{Q}$.按照连续性,得到$f(r)=r,\forall r\in\mathbb{R}$.最后如果$F$包含虚数,那么它包含$i$,于是$f(i)^2=-1$,于是$f(i)=\pm i$,其中$f(i)=i$对应恒等映射,$f(i)=-i$对应共轭.于是$\lambda$要么是恒等的要么是共轭,导致$\sigma_1$和$\sigma_2$要么相同,要么是互相共轭的复嵌入,完成证明.
	\end{proof}
	\item $p$-adic赋值.给定数域$K$上的一个素理想$p$,对每个非零元$x\in K$,记$(x)$的唯一素理想分解中$p$的次数是$v_p(x)$,这也就是唯一的整数$n$使得$x\in p^n,x\not\in p^{n+1}$.那么$v_p$是$K$上的一个加性赋值.数域$K$上的每个非阿基米德赋值都可以表示为$|x|_p=C^{-v_p(x)},C>1$.特别的,这个赋值总是离散赋值.
	\begin{proof}
		
		任取非0元$t\in\mathscr{O}_K$,把$x$的整方程记作$x^m+a_{m-1}x^{m-1}+\cdots+a_0=0$.如果$|t|>1$,得到$|a_{m-1}t^{m-1}+\cdots+a_0|\le|t|^{m-1}$,但是$|t^m|>|t|^{m-1}$矛盾.于是对任意的非0元$t\in\mathscr{O}_K$有$|t|\le1$.取$\mathscr{O}_K$中赋值小于1的元和0的并构成的集合$p$,那么$p$是$\mathscr{O}_K$的一个非0理想(不是零理想因为我们不考虑平凡赋值).按照$|\alpha_1\alpha_2|<1$则必然有某个$|\alpha_i|<1$得到$p$是素理想.现在任取$\alpha\in K^*$,任取$\pi\in p-p^2$,那么$\alpha/\pi^{v_P(\alpha)}\mathscr{O}_K=I/J$.这里$I,J$是和$p$互素的整理想.于是$J\not\subseteq p$,于是可以找到$\beta_2\in J$满足$\beta_2\not\in p$.于是$|\beta_2|=1$.现在取$\beta_1=\beta_2\alpha/\pi^{v_P(\alpha)}\in I$.那么$\beta_1\not\in P$,并且$v_p(\beta_1)=v_p(\beta_2)$.并且$|\beta_1|=1$.于是按照赋值的乘性,得到$|\alpha|=C^{-v_P(\alpha)},C=|\pi|^{-1}>1$.
	\end{proof}
	\item 数域$K$上的有限素位恰好一一对应于$K$上的全部非零素理想,当提及素位时我们指的是一个具体的赋值代表元,每个有限素位取代表元为$\mathrm{N}_K(p)^{-v_p(x)}$,这里$\mathrm{N}_K(p)=|\mathscr{O}_K/p|$是绝对范数,这称为$K$上的$p$-adic赋值,记作$|\bullet|_p$.
	\item 例如取有理素数$p$,那么$\mathbb{Q}$上的$p$-adic赋值是这样定义的:对任意素数$p$,每个有理数$x$可唯一的表示为$p^ns/t,s,t,n\in\mathbb{Z},(s,t)=1,p\not| s,t$,定义加性赋值$v_p:x\mapsto n$,它对应的非阿基米德赋值取为$|x|=p^{-v(x)}$.它的赋值环是$A=\{p^ns/t\in\mathbb{Q}\mid n\ge0,(s,t)=1,p\not\mid s,t\}$,它的极大理想是$\mathfrak{m}=\{p^ns/t\in\mathbb{Q}\mid n\ge1\}$.剩余域$A/\mathfrak{m}=\mathbb{F}_p$.
\end{enumerate}

上述取代表元的目的是有如下乘积公式.
\begin{enumerate}
	\item 设$K$是一个数域,任取$x\in K^*$,至多有限个素位$|\bullet|_v$使得$|x|_v\not=1$.事实上首先无穷素位个数有限,另外对应有限素位有$|x|_v\not=1$当且仅当$v_p(x)\not=0$,但是这只能出现在$p\mid x\mathscr{O}_K$的时候,而这只涉及到有限个素理想,从而有限个有限素位.
	\item 于是对每个$x\in K^*$,无穷乘积$\prod_v|x|_v$,其中$v$取遍全部素位,只涉及到有限个元的乘积.乘积公式是指对每个$x\in K^*$,这个乘积总是1.
	\begin{proof}
		
		一方面,对有限素位取乘积,得到:
		$$\prod_v|x|_v=\prod_p\mathrm{N}(p)^{-v_p(x)}=1/\mathrm{N}(x\mathscr{O}_K)=|\mathrm{N}_{K/\mathbb{Q}}(x)|^{-1}$$
		
		另一方面,对无穷素位乘积,得到:
		$$\prod_v|x|_v=\prod_{\text{实嵌入}}|\sigma(x)|\prod_{\text{复嵌入}}|\sigma(x)\overline{\sigma}(x)|=\prod_{\sigma}|\sigma(x)|=|\mathrm{N}_{K/\mathbb{Q}}(x)|$$
	\end{proof}
\end{enumerate}

赋值的完备化.给定域$K$,取其上一个赋值$|\bullet|$,这诱导了一个度量$d(x,y)=|x-y|$使得$K$成为一个度量空间.度量空间上的一个柯西列$(x_n)$是指对任意$\varepsilon>0$,存在一个$N$使得$n,m>N$时有$d(x_n,x_m)<\varepsilon$.那么收敛列是柯西列.如果一个度量空间上柯西列总是收敛的,就称度量空间是完备的.
\begin{enumerate}
	\item 记$K$是赋值域,$d$是赋值诱导的度量,即$d(x,y)=|x-y|$.
	\begin{enumerate}
		\item 存在完备的域$(\overline{K},\overline{d})$和一致连续的包含映射同时也是同态$i:K\to\overline{K}$,满足$i(K)$在$\overline{K}$中稠密,并且$\overline{d}$是$d$的延拓.任意两个这样的完备域是同构同胚的.
		\item 这里$\widehat{K}$是$(K,d)$上全部柯西列的等价类构成的域,这里$\widehat{d}$是这样定义的,任取$x\in\widehat{K}$,它是$K$中某个柯西列$(x_n)$的极限,于是$|x_n|_{\infty}$是$\mathbb{R}$中的柯西列,它的极限就定义为延拓赋值$|x|_1$.
		\item 如果$(L,e)$是任意一个完备域,$f:K\to L$是一致连续映射,那么存在唯一的一致连续映射$\overline{f}:\overline{K}\to L$延拓了$f$,也就是说$\overline{f}\circ i=f$.如果这里$f$只要求是连续的,结论未必成立.
	\end{enumerate}
	\begin{proof}
		
		假设已经证明了完备化的存在性,记作$(\overline{K},\overline{d})$,按照$i:K\to\overline{K}$是嵌入,不妨把$K$和$i(K)$等同.现在如果$(L,e)$是一个完备域,设$f:K\to L$是连续映射,那么只存在一种方式把$f$延拓到$\overline{K}$上:因为$K$在$\overline{K}$上稠密,对任意的$x\in\overline{K}$,存在$K$中的点列$(x_n)$收敛到$x$,于是必然有$f(x)$是$f(x_n)$的极限.现在因为$f$是一致连续的,这说明对任意的$\varepsilon>0$,存在$\eta>0$,使得只要$d(x,y)<\eta$,就有$e(f(x),f(y))<\varepsilon$,这导致$f(x_n)$也是一个柯西列,于是$f(x)=\lim_{n\to\infty}f(x_n)$极限存在,并且这不依赖于柯西列$x_n$的选取.最后按照$e(f(x),f(y))\le e(f(x),f(x_n))+e(f(x_n),f(y_m))+e(f(y_m),f(y))$,得到$f$是一致连续的.
		
		另外如果存在$(\overline{K}',\overline{d}')$也满足这个性质,那么一致连续的嵌入$i:K\to\overline{K}$唯一延拓为$\overline{i}:\overline{K}'\to\overline{K}$,同样的,一致连续的嵌入$i':K\to\overline{K}'$唯一延拓为$\overline{i}':\overline{i}':\overline{K}\to\overline{K}'$.并且按照泛映射性质,它们的复合就是$\overline{K}'$和$\overline{K}$上的恒等映射,这就导致二者是同构同胚的.
		
		现在来证明完备化的存在性.记集合$S$表示$K$上全体柯西列构成的集合,记$S_0$为全体$K$中收敛到0的柯西列构成的集合.那么按照逐项相加和相乘,看到$S$是含幺交换环,而$S_0$是一个理想.来证明$S_0$是一个极大理想.如果存在真包含$S_0$的理想$I$,那么可以取$I$中一个柯西列$(x_n)$不收敛于0,于是,存在一个$\varepsilon_0>0$,使得对任意的$N$,存在一个正整数$n(N)>N$,使得$d(0,x_{n(N)})\ge\varepsilon_0$.按照$(x_n)$是柯西列,看到存在$N_0$使得$n,m>N_0$时候有$d(x_n,x_m)<\frac{\varepsilon_0}{2}$.这导致$m>N_0$时候有$d(x_m,0)\ge d(x_{n(N_0)},0)-d(x_m,x_{n(N_0)})\ge\frac{\varepsilon_0}{2}$,也就是说,当$m>N_0$时候$d(x_m,0)$有非0下界.取$S_0$中的元$(1-x_1,\cdots,1-x_{N_0},0,0,\cdots)$.那么它和$(x_n)$的和$(y_n)=(1,1,...,1,x_{N_0+1},x_{N_0+2},\cdots)$位于$I$中.并且当$n,m>N_0$的时候有$d(1/x_n,1/x_m)=\frac{|x_m-x_n|}{|x_m||x_n|}\le\frac{4}{\varepsilon_0^2}d(x_n,x_m)$,这导致$(1/y_n)$是柯西列,于是看到$(1)=(y_n)(1/y_n)\in I$,导致$I=S$,于是$S_0$是极大理想,于是$S/S_0$是域,记作$\overline{K}$.
		
		现在取$\overline{K}$中的一个元$x$,取它在$S$中的表示为$(x_n)$,按照$|x_n-x_m|\ge||x_n|-|x_m||_{\infty}$,这导致$(|x_n|)$是$\mathbb{R}$中的柯西列,按照$\mathbb{R}$的完备性,定义$|x|_1=\lim_{n\to\infty}|x_n|$.如果$(x_n')$是另一个$x$在$S$中的表示,那么我们看到$||x_n|-|x_n'||_{\infty}$趋于0,这导致这个定义不依赖于$x$在$S$中表示的选取.现在取$i:K\to\overline{K}$为把每个$a\in K$映射为柯西列$(a,a,\cdots)$的陪集.那么这是一个域嵌入,并且看到$|\bullet|_1$是$|\bullet|$的延拓.于是$i$自然是一致连续的.现在来证明$i(K)$在$\overline{K}$中稠密.任取$x\in\overline{K}$的在$S$中的表示$(x_n)$,取$i(K)$中的元为$y_n=(x_n,x_n,\cdots)+S_0$,那么断言$y_n$在$\overline{K}$中收敛到$x$,事实上对任意的$\varepsilon>0$,按照$(x_n)$是柯西列,知道存在$N$使得$n,m>N$的时候有$|x_n-x_m|<\frac{\varepsilon}{2}$,于是当$n>N$的时候,有$|x-y_n|_1=\lim_{m\to\infty}|x_m-y_n|\le\frac{\varepsilon}{2}<\varepsilon$.这就说明了$y_n$收敛到$x$,从而$i(K)$在$\overline{K}$中稠密.
		
		最后只要说明$(\overline{K},\overline{d})$是完备域.取$\overline{K}$中的柯西列$(x_n)$,按照稠密,对每个$i$可以找到$y_n\in i(K)$使得$|x_n-y_n|_1<\frac{1}{n}$,于是看到$|y_n-y_m|\le|y_n-x_n|_1+|x_n-x_m|_1+|x_m-y_m|\le|x_n-x_m|+\frac{1}{n}+\frac{1}{m}$,这导致$(y_n)$是$i(K)$中的柯西列,记这个柯西列属于$\overline{K}$中的元$y$,那么就有$\lim_{n\to\infty}|x_n-y|_1=\lim_{n\to\infty}\lim_{m\to\infty}|x_n-y_m|_1\le\lim_{m,n\to\infty}|x_n-x_m|_1+\lim_{m\to\infty}|x_m-y_m|_1$.注意到最后一项是小于$\frac{1}{m}$的,于是极限是0,前一项按照柯西列的定义极限也是0,于是看到$(x_n)$收敛到$y\in\overline{K}$,导致$\overline{K}$是完备的.
	\end{proof}
    \item 非阿基米德赋值的完备化仍然是非阿基米德的,阿基米德绝对值的完备化仍然是阿基米德赋值,离散赋值域的完备化仍然是离散赋值域.前两件事是容易的,我们只证明最后这件事.
    \begin{proof}
    	
    	只需证明如下命题:设$|x|=C^{-v(x)}$是域$K$上的离散赋值,其中$v(x)$取整数,对任意$x\in\widehat{K}^*$,存在整数$k$使得$|x|=C^k$.事实上设$\{x_n\}\subseteq K$收敛于$x$.按照$x\not=0$,不妨约定$x_n$中不含零元,那么$k_n=\ln|x_n|_p/\ln\mathrm{N}(p)$是$\mathbb{R}$上的柯西列.但是这是取值整数的柯西列,于是它收敛到一个整数$k$.
    \end{proof}
    \item 对域$K$上的一个非阿基米德赋值$|\bullet|$,对一个数列$(a_n)\subseteq K$,只要满足$|a_{n+1}-a_n|\to0$,那么它就是柯西列.事实上对任意的$\varepsilon>0$,可以找到$N$使得$n>N$时候有$|a_{n+1}-a_n|<\varepsilon$,那么对$m,n>N$,就得到$|a_m-a_n|\le\max\{|a_{m}-a_{m-1}|,\cdots,|a_{n+1}-a_n|\}<\varepsilon$.于是特别的,对于完备的非阿基米德赋值,一个序列$\{a_n\}$收敛当且仅当$a_n\to0$.
    \item 设$|\bullet|$是域$K$上的非阿基米德赋值,它的完备化记作$(\widehat{K},|\bullet|)$,设$K$的赋值环为$A$,设$\widehat{K}$的赋值环为$B$,那么$B$是$A$在$\widehat{K}$中的闭包.任取$\pi\in A$满足$0<|\pi|<1$,有:
    $$B=\lim\limits_{\substack{\leftarrow\\n\in\mathbb{Z}_{\ge0}}}A/\pi^nA=\{(x_n)_n\in\prod_{n\ge0}A/\pi^nA\mid x_{n+1}\equiv x_n(\mathrm{mod}\pi^n)\}$$
    $$K=B[\pi^{-1}]$$
    \begin{itemize}
    	\item 具体的讲,如果取定$A/\pi A$元素的提升代表元集$S$,那么$B$中的元可以表示为级数$\sum_{n\ge0}s_n\pi^n$,其中$s_n\in S$.而$\widehat{K}$中的元素可以表示为$\sum_{n\ge -N}s_n\pi^n$,其中$s_n\in S$.
    	\item 另外如果$(K,v)$是离散赋值域,这里的$\pi$通常取为素元.
    	\item 如果$(K,v)$本身是完备的,记$(A,m)$是赋值环,结论就是$A\cong\lim\limits_{\substack{\leftarrow\\n}}A/m^n$.
    \end{itemize}
    \begin{proof}
    	
    	按照$K$在$\widehat{K}$中稠密,得到$A$在$B$中稠密.于是对任意$x\in B$,可取$x_n\in A$使得$|x_n-x|<|\pi|^n$,那么$x_n$收敛到$x$.记这个逆向极限为$C$.对每个正整数$n$,存在$N$使得$m\ge N$时有$|x_m-x|<|\pi|^m$.于是这些$x_m,m\ge N$在$\mathrm{mod}\pi^n$下是相同的:强三角不等式得到$|x_{m+1}-x_m|<|\pi|^n$,于是$c=\pi^{-n}(x_{m+1}-x_m)\in A$,于是$x_{m+1}-x_n\in\pi^nA$.我们定义这唯一的在$A/\pi^nA$中的像是$\psi_n(x)$,可验证$(\psi_n(x))$在$C$中,并且不依赖于$\{x_n\}$的选取.这得到同态$\psi:B\to C$,$x\mapsto(\psi_n(x))$.
    	
    	\qquad
    	
    	反过来任取$(x_n)_n\in C$,取$x_n'$为$x_n\in A$的任一提升,那么$|x_n'-x_m'|\le|\pi^m|,n\ge m$.按照$|\pi|<1$得到$\{x_n'\}$是$A$中柯西列,记极限为$\varphi(x)$,那么$\varphi:C\to B$是同态.验证$\psi$和$\varphi$互为逆映射,得证.
    	
    	\qquad
    	
    	最后$K=B[\pi^{-1}]$是因为如果$x\in\widehat{K}$,可取一个足够大的$N$使得$v(x)+Nv(\pi)\ge0$,于是$\pi^Nx\in B$,于是有$x=\sum_{n\ge-N}s_n\pi^n$.
    \end{proof}
    \item 设$(K,v)$是赋值域,设完备化为$\widehat{K}$,两个剩余域是同构的$\kappa(v)\cong\kappa(\widehat{v})$.更有$A/m^r\cong B/n^r$,其中$(A,m)$和$(B,n)$分别是$K$和$\widehat{K}$的赋值环.
    \begin{proof}
    	
    	直接构造$\theta:\mathscr{O}_K\to\mathscr{O}_{\widehat{K}}/\widehat{m}$为$x\mapsto x+\widehat{m}$.这是满射因为$\mathscr{O}_K$在$\mathscr{O}_{\widehat{K}}$中稠密(于是对每个$y\in\mathscr{O}_{\widehat{K}}$就有$x\in\mathscr{O}_K$使得$|y-x|<1$,此即$y-x\in\widehat{m}$).接下来$\ker\theta=\widehat{m}\cap\mathscr{O}_K=m$.
    \end{proof}
\end{enumerate}






数域的完备化.设$v$是数域$K$上的素位,记它的完备化为$K_v$或者在不引起歧义的情况下记作$\widehat{K}$.记
\begin{enumerate}
	\item 无穷素位的情况.如果$v$是实素位,完备化$K_v$就是$(\mathbb{R},|\bullet|_{\infty})$;如果$v$是复素位,完备化$K_v$就是$(\mathbb{C},|\bullet|_{\infty})$.
	\begin{proof}
		
		对于实嵌入,可以把$K$等同于$\sigma(K)\subset\mathbb{R}$,此时这个实数域的子域包含了$\mathbb{Q}$,这是$\mathbb{R}$的稠密子集,结合$\mathbb{R}$本身关于$|\bullet|_{\infty}$完备,导致$\sigma(K)$对于$|\bullet|_{\infty}$的完备化必然是$\mathbb{R}$.
		
		对于复嵌入,同样把$K$等同于$K'=\sigma(K)$,断言$K'$在$\mathbb{C}$中稠密,结合$\mathbb{C}$是完备的,这就得到了$K'$关于$|\bullet|_{\infty}$的完备化就是$\mathbb{C}$.取$K'-\mathbb{R}$中的一个元$x$,断言$\mathbb{Q}(x)$已经在$\mathbb{C}$中稠密:作为实线性空间,$\mathbb{C}$具有基$\{1,x\}$,于是对$\mathbb{C}$中的任意开集$U$,任取一点$y$,有表示$y=k_1+k_2x$.那么存在足够小的分别包含$k_1,k_2$的实数域中的开区间$(a_1,a_2),(b_1,b_2)$,使得$U'=\{a+bx\mid a\in(a_1,a_2),b\in(b_1,b_2)\}\subseteq U$.但是按照$\mathbb{Q}$在$\mathbb{R}$中稠密,于是可以找到有理数$q_1\in(a_1,a_2)$,$q_2\in(b_1,b_2)$.于是得到$q_1+q_2x\in U'\subseteq U$,于是$\mathbb{Q}(x)$在$\mathbb{C}$中稠密,完成证明.
	\end{proof}
	\item 有限素位的情况.设$v$是对应于素理想$p$的有限素位,它的完备化记作$K_p$.特别的,对于$\mathbb{Q}$上的有理素数$p$,对应的$p$-adic赋值记作$v$,它的赋值环是$A=\{a/b\mid a,b\in\mathbb{Z},b\not=0,p\not| b\}$.它的唯一极大理想为$pA$,剩余域为$A/pA\cong\mathbb{Z}/p\mathbb{Z}$,并且有$A/p^nA\cong\mathbb{Z}/p^n\mathbb{Z}$.那么$\mathbb{Q}$在该赋值下的完备化为$\mathbb{Q}_p=\mathbb{Z}_p[1/p]$,其中$\mathbb{Z}_p=\lim\limits_{\leftarrow}\mathbb{Z}/p^n\mathbb{Z}$是完备化的赋值环.
\end{enumerate}

高阶单位群.设$(K,v)$是离散赋值域,赋值环记作$(A,m)$,考虑$U^{(n)}=1+m^n=\{x\in K^*\mid v(1-x)\ge n\}$,它在乘法下构成一个群,称为$n$阶单位群.
\begin{enumerate}
	\item 对$n\ge1$有$A^*/U^{(n)}\cong(A/m^n)^*$和$U^{(n)}/U^{(n+1)}\cong A/m$.
	\begin{proof}
		
		第一个同构因为典范映射$A^*\to(A/m^n)^*$的核就是$U^{(n)}$.第二个同构先取素元$\pi$,构造映射$U^{(n)}\to A/m$为$1+\pi^na\mapsto a+m$,它的核是$U^{(n+1)}$.
	\end{proof}
    \item 有典范同胚同构:
    $$A^*\cong\lim\limits_{\substack{\leftarrow\\n}}A^*/U^{(n)}$$
\end{enumerate}

完备离散赋值域上的Hensel引理.即完备离散赋值域上多项式的分解可以放在剩余域中分解.
\begin{enumerate}
	\item 设$(K,v)$是一个完备离散赋值域,多项式$f(x)=a_0+a_1x+\cdots+a_nx^n\in\mathscr{O}_K[x]$称为本原的,如果$|f|=\max\{|a_0|,|a_1|,\cdots,|a_n|\}=1$.设$f(x)$是本原多项式,设它在$\mathrm{mod}m$下分解为两个互素的多项式的乘积$f(x)\equiv\overline{g}(x)\overline{h}(x)$,那么$f(x)$在$\mathscr{O}_K$中有分解$f(x)=g(x)h(x)$,满足$g$和$h$分解是$\overline{g}$和$\overline{h}$的提升,并且其中有一个次数相同$\deg g=\deg\overline{g}$.如果$f$的首系数是单位,那么有对应次数都是相同的:$\deg f=\deg\overline{f}$,$\deg g=\deg\overline{g}$和$\deg h=\deg\overline{h}$.
	\item 例如多项式$x^{p-1}-1\in\mathbb{Z}_p[x]$在剩余域$\mathbb{F}_p$中分裂,得到它在$\mathbb{Z}_p$中分裂.
	\item 推论.设$(K,v)$是完备的非阿基米德赋值域.对不可约多项式$f(x)=a_0+a_1x+\cdots+a_nx^n\in K[x]$,其中$a_0a_n\not=0$,有$|f|=\max\{|a_0|,|a_n|\}$.特别的,从$a_n=1$和$a_0\in\mathscr{O}_K$得到$|a_i|\le|a_n|$,于是$a_i\in\mathscr{O}_K$,于是$f\in\mathscr{O}_K[x]$.
	\begin{proof}
		
		适当对这个多项式乘以一个元,可设$|f|=1$,于是$f\in\mathscr{O}_K[x]$.设指标$r$是最小的自然数使得$|a_r|=1$,于是$f(x)\equiv x^r(a_r+a_{r+1}x+\cdots+a_nx^{n-r})\mathrm{mod}m$.如果$\max\{|a_0|,|a_n|\}<1$,那么$0<r<n$,按照Hensel引理,这导致$f(x)$是可约的,矛盾.
	\end{proof}
\end{enumerate}

Hensel引理的一些补充(Henselian域和牛顿多边形).设$(K,v)$是一个非阿基米德赋值域,设赋值环$A$,如果它满足Hensel引理就称$K$是Henselian域.换句话讲,如果$f\in A[x]$是一个本原多项式,也即它全部系数的赋值的最大值是1,如果$f$在剩余域中可分解为$f(x)\equiv\overline{g}(x)\overline{h}(x)(\mathrm{mod}m)$,其中$\overline{g}$和$\overline{h}$是互素的,那么在$A$中存在分解$f(x)=g(x)h(x)$,使得$g\equiv\overline{g},h\equiv\overline{h}(\mathrm{mod}m)$,并且有一个对应的次数相同$\deg g=\deg\overline{g}$.那么此时称$K$是Hensel域.也会称$A$是Henselian赋值环或者$v$是Henselian赋值.
\begin{enumerate}
	\item Hensel域的赋值扩张.设$(K,|\bullet|)$是一个Hensel域,给定任意代数扩张$K\subseteq L$,那么$|\bullet|$在等价意义下唯一延拓到$L$上,即$|x|=\sqrt[n]{|\mathrm{N}_{L/K}(x)|}$,这里$n$是$x$所在有限代数子扩张的扩张次数(这个绝对值定义不依赖于这样有限子代数扩张的选取).特别的,扩张赋值的赋值环恰好就是$K$的赋值环在$L$中的整闭包.这个证明完全和之前完备的情况一样证明,另外这个赋值扩张性质完全刻画了Hensel域,为了证明这件事先介绍Newton多边形.
	\item 设$(K,v)$是一个非阿基米德赋值域,取多项式$f(x)=a_0+a_1x+\cdots+a_nx^n\in K[x]$,约定$a_0a_n\not=0$.对每个单项式$a_ix^i$,设它对应的点为$(i,v(a_i))$,我们划去$a_i=0,v(a_i)=\infty$的情况,剩下的点集在$\mathbb{R}^2$中构成的下凸包(即$y$轴负无穷方向的凸包)称为$f(x)$的牛顿多边形.这个多边形从$(0,v(a_0))$起点开始由一些斜率逐渐变大的折线构成(除去$(0,v(a_0))$和$(n,v(a_n))$的连线).
	\item 设$(K,v)$是非阿基米德赋值域,设$f(x)=a_0+a_1x+\cdots+a_nx^n\in K[x]$,设$a_0a_n\not=0$,设$K$关于$f$的分裂域为$L$,设$v$延拓到$L$上为赋值$w$.如果$(r,v(a_r))$到$(s,v(a_s))$构成了出现牛顿多边形中斜率为$-m$的折线段,那么$f(x)$恰好有$s-r$个根$\alpha_1,\alpha_2,\cdots,\alpha_{s-r}$,使得它们的赋值$w(\alpha_1)=w(\alpha_2)=\cdots=w(\alpha_{s-r})=m$.特别的,如果牛顿多边形只由一条直线段构成,那么它的所有根的赋值都是相同的.
	\begin{proof}
		
		对多项式数乘一个非零元相当于把牛顿多边形向上或者向下平移,不改变每个折线的斜率,于是不妨设$a_n=1$.现在把$f$的所有根(计重数)排序为$\alpha_1,\alpha_2,\cdots,\alpha_n$,使得有$w(\alpha_j)=m_i,s_i+1\le j\le s_{i+1}$,其中$s_0=0,s_t=n$,并且$m_1<m_2<\cdots<m_{t+1}$.现在按照韦达定理,每个根都可以视为系数的初等对称多项式,于是有$v(a_n)=v(1)=0$,$v(a_{n-1})\ge\min\{w(\alpha_i)\}=m_1$,$v(a_{n-2})\ge\min\{w(\alpha_i\alpha_j)\}=2m_1$,直到$v(a_{n-s_1})=s_1m_1$.这是第一条折线段,它的斜率是$-m_1$,继续操作下去即可.
	\end{proof}
	\item 一般情况下,如果折线段有$r$个,斜率分别是$-m_r<-m_{r-1}<\cdots<-m_1$,那么$f(x)=a_n\prod_{j=1}^rf_j(x)$,其中$f_j(x)=\prod\limits_{w(\alpha_i)=m_j}(x-\alpha_i)$.这里我们断言如果$v$可以唯一延拓到$L$上,那么这里的这些$f_j(x)$都是$K$上的多项式.
	\begin{proof}
		
		不妨设$a_n=1$.倘若$f$本身是不可约多项式,那么它的不同根之间可经扩张的某个自同构映射,因为如果$\sigma$是扩张的自同构,$w$是一个延拓赋值,那么$w(\sigma(\bullet))$也是一个延拓赋值,这导致所有根在$w$下的取值是相同的,于是$f=f_1$.
		
		现在对$n$归纳,当$n=1$时没什么需要证的,设$n\ge2$,设$p(x)$是$\alpha_1$的极小多项式,那么$g(x)=f(x)/p(x)\in K[x]$.按照$p(x)$的所有根的赋值相同,说明$p(x)\mid f_1(x)$.再记$g_1(x)=f_1(x)/p(x)$,那么$g(x)=g_1(x)\prod_{j=2}^rf_j(x)$,并且$\deg g<\deg f$,于是归纳法说明$f_j(x),g_1(x)\in K[x]$.完成归纳.
	\end{proof}
	\item 另外上一条的这个证明还说明了如果$f$本身是不可约的,那么它的牛顿多边形只由一条直线段构成.于是此时其它的点$(i,v(a_i))$要么在这条直线段上要么在这条直线段以上,这说明:如果$f(x)=a_0+a_1x+\cdots+a_nx^n\in K[x]$是不可约多项式,$a_n\not=0$,如果$|\bullet|$是$K$上的非阿基米德赋值,并且该赋值可唯一延拓到$f$的分裂域上,那么有$|f|=\max\{|a_0|,|a_n|\}$.
	\item 一个非阿基米德赋值域$(K,|\bullet|)$是Henselian域当且仅当它的赋值可以唯一延拓至任意代数扩张上.
	\begin{proof}
		
		证明充分性.设赋值环为$A$,先取本原的不可约多项式$f(x)=a_0+a_1x+\cdots+a_nx^n\in A[x]$,不妨设$a_0a_n\not=0$,我们解释过唯一延拓这个条件使得$|f|=\max\{|a_0|,|a_n|\}$.断言要么$\overline{f}(x)$是常数,要么$|a_n|=1$.因为如果$|a_n|<1$,有$|a_0|=1$和$|a_n|<1$,我们解释过赋值可以唯一延拓到任意代数扩张这个条件下,不可约多项式的牛顿折线是一条直线段,这迫使它是$(0,v(a_0)=0)$连至$(n,v(a_n)>0)$的直线段,这导致$\overline{f}(x)=\overline{a_0}$,此时不需要验证Hensel引理.于是可不妨设$|a_n|=1$,于是有$\deg\overline{f}=\deg f$.下面我们断言存在剩余域中的不可约多项式$\overline{\varphi}(x)\in\kappa(x)$和常数$\overline{a}\in\kappa$,使得$\overline{f}(x)=\overline{a}\overline{\varphi}(x)^m$.
		
		\qquad
		
		设$f(x)$在$K$上的分裂域为$L$,按照条件$K$上赋值唯一延拓至$L$上,延拓赋值依旧记作$|\bullet|$,设$L$的赋值环为$(B,n)$.按照延拓是唯一的,任取$\sigma\in G=G(L/K)$,对任意$\alpha\in L$都有$|\sigma\alpha|=|\alpha|$.于是$\sigma(B)=B$和$\sigma(n)=n$.如果$\alpha$是$f(x)$的零点,我们断言$\sigma(\alpha)\in B$对任意$\sigma\in G$成立:假设$\alpha\not\in B$,就导致$\prod_{\sigma}|\sigma(\alpha)|^{\mu}>1$,导致$f(x)$的常数项$a_0\not\in A$矛盾.于是$\alpha$和每个共轭元$\sigma(\alpha)\in B$.于是每个$\sigma\in G$诱导了$B/n$上的$\kappa$自同构$\overline{\sigma}$,于是$\overline{f}(x)$的全部根$\overline{\sigma}(\overline{\alpha})$两两在$\kappa$上共轭,所以$\overline{f}$在差一个系数意义下是单个不可约多项式的次幂,否则理应存在两个根不是共轭的,于是有不可约多项式$\overline{\varphi}(x)\in\kappa(x)$和常数$\overline{a}\in\kappa$,使得$\overline{f}(x)=\overline{a}\overline{\varphi}(x)^m$.
		
		\qquad
		
		下面任取本原多项式$f(x)\in A[x]$,做它在$K[x]$上的不可约分解$f(x)=f_1(x)\cdots f_r(x)$.从$1=|f|=\prod|f_i|$,对每个$f_i$适当乘以一个系数,可以使得分解不变$f(x)=\prod_if_i(x)$,但是有每个$|f_i|=1$.于是可设每个$f_i\in A[x]$是本原的不可约多项式.于是要么$\overline{f_i}$是常数,要么它差一个倍数是某个不可约多项式次幂.倘若有$\overline{f}=\overline{g}\overline{h}$分解为互素的两个多项式的乘积,那么那些非常数的$\overline{f_i}$只能恰好在$\overline{g}$和$\overline{h}$中一个的唯一分解中.考虑出现在$\overline{g}$中的那些不是常数的$\overline{f_i}$,把对应的$f_i$乘起来记作$g$,这保证了$\deg g=\deg\overline{g}$,把其余的$\overline{f_i}$(包括那些是常数的)对应的$f_i$乘起来记作$h$,适当对$g,h$调整一个倍数,可以使得$g\equiv\overline{g}$和$h\equiv\overline{h}$,并且有$f=gh$.完成证明.
	\end{proof}
	\item 在Henselian域定义中的本原多项式$f(x)$可以改为首一多项式.换句话讲,一个非阿基米德赋值域$(K,v)$是Henselian域当且仅当对每个首一多项式$f(x)\in A[x]$(这自然是本原的),如果在剩余域中有分解$\overline{f}=\overline{g}\overline{h}$,使得$\overline{g}$和$\overline{h}$是互素的,那么存在$\overline{g}$和$\overline{h}$的提升多项式$g,h\in A[x]$,使得$f=gh$,并且$\deg g=\deg\overline{g}$和$\deg h=\deg\overline{h}$.
	\begin{proof}
		
		上一条的证明实际上说明了如果非阿基米德赋值域$(K,v)$的每个不可约多项式的牛顿折线是一条直线段,那么它是Henselian域.我们来证明这个条件.设$f(x)=a_0+a_1x+\cdots+x^n\in K[x]$是首一不可约多项式,设$f$在$K$上的分裂域为$L$,那么$v$可以延拓到$L$上,记作$w$(即便$K$不是完备的,赋值肯定可以延拓到它的任意代数扩张$L$上,因为可以首先延拓到$\widehat{K}$,完备域的赋值肯定可以延拓到代数扩张上,所以可以延拓到$\overline{\widehat{K}}$,再限制到$L$上即可,见下一节).
		
		\qquad
		
		假设$f$的牛顿折线由多于一条直线段构成,设最右侧的直线段是$(m,e)$与$(n,0)$的连线.那么首先$e=0$的情况是排除的:如果$e=0$,凸性导致$v(a_i)\ge0$,于是$f(x)\in A[x]$,凸性还导致$v(a_i)>0,i<m$,于是在$\mathrm{mod}m$下有$f(x)\equiv(x^{n-m}+\cdots+a_m)x^m$,但是$f$是不可约的,这就和$(K,v)$满足首一多项式的Hensel引理相矛盾.
		
		\qquad
		
		下面设$e<0$,因为$f$的牛顿折线最右侧的直线段的斜率是$-e/(n-m)$,于是在$f(x)$的零点中加性赋值最小的元恰有$n-m$个,它的赋值就是$e/(n-m)$.取一个这样的加性赋值最小的零点$\alpha\in L$.考虑$a_m^{-1}\alpha^{n-m}\in K(\alpha)$对于扩张$K(\alpha)/K$的特征多项式记作$g(x)$.如果记$f(x)=\prod_{1\le i\le n}(x-\alpha_i)$,那么有$g(x)=\prod_{1\le i\le n}(x-\alpha_i^ra_m^{-1})$.按照$g$的零点的赋值不全相同,导致$g$的牛顿折线也不是单个直线段.$g$的最右端直线段的斜率是$-w(a_m^{-1}\alpha^{n-m})=v(a_m)-(n-m)w(\alpha)=0$.因为$g$是$a_m^{-1}\alpha^{n-m}$的极小多项式的次幂,所以$g$是一个不可约多项式的次幂,并且它牛顿折线最右端的直线段是平行$x$轴的,套用上一段的证明依旧得到在$\mathrm{mod}m$下有$f(x)\equiv(x^{n-m}+\cdots+a_m)x^m$,但是按照$(K,v)$满足首一多项式的Hensel引理,有$f(x)$的唯一分解中至少由两个不可约因式次幂构成,这依旧是矛盾的.
	\end{proof}
\end{enumerate}
\newpage
\subsection{赋值的扩张}

完备赋值域的扩张.
\begin{enumerate}
	\item 引理.设$(K,|\bullet|)$是完备的非阿基米德赋值域,赋值环记作$A$,设$K\subseteq L$是有限扩张,设$A$在$L$中的整闭包是$B$,那么有$B=\{x\in L\mid\mathrm{N}_{L/K}(x)\in A\}$.
	\begin{proof}
		
		一方面任取$x\in B$,按照定义有首一$A$系数多项式$p(X)$使得$p(x)=0$.因为$K$是$A$的商域,可验证$x$在$K$上的首一极小多项式也是$A$系数的.于是$x$在$K$上的首一特征多项式(作为它极小多项式的次幂),也是$A$系数的,于是常数项,也就是$\mathrm{N}_{L/K}(x)\in A$.
		
		\qquad
		
		另一方面.任取$0\not=x\in L$,使得$\mathrm{N}_{L/K}(x)\in A$.记$x$在$K$上的首一极小多项式为$p(X)=X^n+a_1X^{n-1}+\cdots+a_n$,那么$\mathrm{N}_{L/K}(x)$和$a_n$的某个次幂差一个符号,记作$\pm a_n^m$,导致$|a_n|\le1$,于是$a_n\in A$.但是按照Hensel引理的一个推论,这导致$p(X)\in A[X]$,于是$x\in B$.
	\end{proof}
	\item 设$(K,|\bullet|)$是完备赋值域,设$K\subseteq L$是代数扩张,那么赋值在等价意义下可以唯一的延拓到$L$上(或者视为素位的延拓).如果扩张是有限扩张,对每个$x\in L$,有$|x|=\sqrt[n]{|\mathrm{N}_{L/K}(x)|}$,这里$n$是扩张次数.另外写作加性赋值的话是$w(x)=\frac{1}{n}v(\mathrm{N}_{L/K}(x))$.
	\begin{proof}
		
		假设赋值是阿基米德的,那么Ostrowski定理说明$K=\mathbb{R}$或$\mathbb{C}$.此时非平凡的代数扩张只有有限扩张$\mathbb{R}\subset\mathbb{C}$,此时有$\mathrm{N}_{\mathbb{C}/\mathbb{R}}(z)=z\overline{z}=|z|^2$满足结论.下设赋值是非阿基米德的,按照代数扩张可视为若干有限代数扩张的并,于是我们归结为证明扩张是有限扩张的情况.
		
		赋值扩张的存在性.直接构造$|x|=\sqrt[n]{|\mathrm{N}_{L/K}(x)|}$,这里赋值定义的前两条是直接验证的,接下来需要验证强三角不等式$|x+y|\le\max\{|x|,|y|\}$.等价于讲如果$|x|\le1$,那么$|x+1|\le1$.但是如果设$A$是$(K,v)$上的赋值环,设$A$在$L$中的整闭包是$B$,那么引理说明$B=\{x\in L\mid\mathrm{N}_{L/K}(x)\in A\}$.于是$|x|\le1$等价于$x\in B$,就得到$x+1\in B$,于是$|x+1|\le1$.
		
		赋值扩张的唯一性.假设在有限扩张$L$上存在第二个扩张赋值$|\bullet|'$,设它对应的赋值环为$B'$,设$B$和$B'$分别的唯一极大理想是$m$和$m'$.我们断言$m=m'$,于是这两个赋值是等价的(我们证明过赋值等价的一个等价描述是$|x|_1<1$当且仅当$|x|_2<1$).假设可取$x_0\in m-m'$,设$x_0$在$A$上的极小多项式为$f(x)=x^d+a_1x^{d-1}+\cdots+a_d$.按照$x_0\not\in m'$得到$x_0^{-1}\in m'$,于是$1=-a_1x_0^{-1}-\cdots-a_d(x_0)^{-d}\in m'$矛盾.
	\end{proof}
    \item 设$(K,|\bullet|)$是完备赋值域,设$V$是$K$上的$n$维赋范空间,那么$V$上的拓扑必须是积拓扑.具体的讲对每组基$\{v_1,v_2,\cdots,v_n\}$,极大范数$\Vert x_1v_1+x_2v_2+\cdots+x_nv_n\Vert=\max\{|x_1|,|x_2|,\cdots,|x_n|\}$总是和$V$上的范数等价.
    \begin{proof}
    	
    	需要证明的是存在正实数$a,b$使得$a\Vert x\Vert\le|x|\le b\Vert x\Vert,\forall x\in V$.这里$b$可以直接取为$\sum_i|v_i|$.我们用归纳法证明$a$的存在性.对$n=1$只需取$a=|v_1|$.假设对$n-1$维线性空间命题成立,设$V_i=\oplus_{j\not=i}Kv_j$,那么$V=V_i\oplus Kv_i$.这里$|\bullet|$限制在$V_i$上是完备的,于是它是$V$的闭子集.于是平移得到$V_i+v_i$是闭子集.于是$\cup_{1\le i\le n}(V_i+v_i)$是$V$的闭子集.按照0元不在这个并里,得到0的一个开邻域和这个并无交,也即存在$a>0$使得$|x_i+v_i|\ge a$,$\forall x_i\in V_i$和$1\le i\le n$.现在取$x=\sum_ix_iv_i\not=0$,设$|x_r|=\max\{|x_i|\}$,那么有$|x_r^{-1}x|=\left|\sum_i\frac{x_i}{x_r}v_i\right|\ge a$,于是$|x|\ge a|x_r|=a\Vert x\Vert$.
    \end{proof}
    \item 上一条说明,如果$(K,v)$是完备赋值域,如果$L/K$是有限扩张,那么在范数等价意义下至多存在$L$上一个赋值延拓了$v$,并且如果它存在,那么$L$总是关于该赋值完备的.
    \item 推论.
    \begin{enumerate}
    	\item 如果$(K,|\bullet|)$是完备赋值域,如果$K\subseteq L$是代数扩张,设$K$的赋值环是$A$,设$A$在$L$中的整闭包是$B$,那么$B$恰好就是$L$的赋值环.
    	\begin{proof}
    		
    		因为$x$在$L$的赋值环中等价于$v_L(x)\ge0$,等价于$v_K(\mathrm{N}_{L/K}(x))\ge0$,等价于$\mathrm{N}_{L/K}(x)\in A$,按照引理这等价于$x\in B$.
    	\end{proof}
        \item 如果$K$是完备离散赋值域,它延拓到有限扩张$L$上也是完备离散赋值域(无限维代数扩张不对).特别的,这说明如果$(K,v)$是完备离散赋值域,赋值环记作$A$,设$L/K$是有限扩张,设$A$在$L$中的整闭包是$B$,那么$B$也是DVR.另外$B$是自由$A$模,因为$A$是PID,而$B$是有限无挠$A$模.
    \end{enumerate}
\end{enumerate}

krasner引理和推论.
\begin{enumerate}
	\item 引理内容.设$(K,|\bullet|)$是完备的非阿基米德域,设$L/K$是有限扩张.
	\begin{enumerate}
		\item 如果$x,y\in L$在$K$上共轭,那么$|x|=|y|$.
		\item $L$中的一个元和它共轭元的距离一定不超过它和基域中任意一个元的距离.具体的讲,如果$x,y\in L$在$K$上共轭,那么对任意$a\in K$有$|x-y|\le|a-x|$.
		\item 设$x\in L$是$K$上可分元,如果一个元$a\in L$和$x$的距离严格小于$x$和$x$的在某个固定代数闭包中其余共轭元之间的距离,那么有$x\in K(a)$.
	\end{enumerate}
    \begin{proof}
    	
    	第一件事是因为$x$和$y$的范数相同,按照延拓赋值的表达式,它们的绝对值相同.第二件事假设不成立,那么存在$a\in K$使得$|x-y|>|a-x|$,按照强三角不等式的性质,就有$|a-y|=\max\{|a-x|,|x-y|\}=|x-y|>|a-x|$.但是$a-x$和$a-y$本是共轭的,按照第一条本该赋值相同,这矛盾.第三件事设$z=a-x$,设$F=K(a)$,那么$z$也是$F$上可分元,如果可以证明$z$在$F$上的共轭元只有自身,就导致$z\in F$,从而$x\in F=K(a)$.但是$z$在$F$上的共轭元必然具有形式$z'=a-y$,其中$y$是$x$的共轭元.按照条件有$|z|<|z-z'|$,按照强三角不等式得到$|z'|=|z-z'|$,进而有$|z|=|z-z'|$,此即$|a-x|=|x-y|$,这迫使$x=y$,于是$z=z'$,这得证.
    \end{proof}
    \item 设$(K,|\bullet|)$是完备的非阿基米德域,设$L=K(a)$是$n$次可分单扩张,设$a$在$K$上的极小多项式为$P(x)$,那么存在$\varepsilon>0$,使得对任意满足$|P-Q|<\varepsilon$的首一多项式$Q\in K[x]$,都存在$b\in L$使得$L=K(b)=K(a)$.
    \begin{proof}
    	
    	记满足$|P-Q|<\varepsilon$的首一多项式$Q$在$K$的某个代数闭包中分解为$Q=\prod(x-b_i)$,那么有$\prod(a-b_i)=Q(a)=Q(a)-P(a)$.于是如果记$M=\max_{0\le i\le n}|a|^i$,得到$\prod|a-b_i|=|Q(a)-P(a)|\le|Q-P|M$.那么至少存在一个指标$i$使得$|a-b_i|\le|Q-P|^{1/n}M^{1/n}$.那么如果$\varepsilon$足够小,按照Krasner引理,得到当$|P-Q|<\varepsilon$时有$K(a)\subseteq K(b_i)$.但是$b_i$是一个$n$次多项式的根,所以$n=[L:K]\le[K(b_i):K]\le n$,所以这个不等式都取等号,所以$K(b_i)=K(a)$.
    \end{proof}
\end{enumerate}

赋值延拓的一般结论.
\begin{enumerate}
	\item 设$v$是域$K$上的赋值.记$K$关于$v$的完备化为$K_v$,那么赋值$v$可以典范的延拓到$K_v$上.那么完备离散赋值域$(K,v)$上的赋值又可以唯一的延拓到代数闭包$\overline{K_v}$上,把这个延拓赋值记作$\overline{v}$.设$K\subseteq L$是有限代数扩张,取一个$K$嵌入$\tau:L\to\overline{K_v}$.把$\overline{v}$限制到$\tau L$上,得到$v$在$L$上的延拓赋值$\overline{v}\circ\tau$.换句话讲对每个$K$嵌入$\tau:L\to\overline{K_v}$,都有$v$在$L$上的延拓赋值$|x|=|\tau x|_{\overline{v}}$.这个嵌入可以唯一延拓为$\tau:L_w\to\overline{K_v}$($\tau$是这样定义的,对$L_w$的每个元$x$,设它是$L$在赋值$w$下的一个柯西列$\{x_n\}$的极限,那么$\tau(x)$是$\overline{K_v}$的在$\overline{v}$赋值下柯西列$\{\tau(x_n)\}$的极限).这里我们要证明这刻画了全部延拓.
	\item 对于有限扩张$K\subseteq L$,有$L_w=LK_v$因为首先有$LK_v\subseteq L_w$,并且$K_v$的赋值延拓到$LK_v$上仍然是完备的(因为扩张有限),于是$LK_v$是延拓了$L$上赋值的完备域,这导致必然有$LK_v=L_w$.特别的,如果$L/K$是有限可分扩张,那么$L_w/K_v$也是有限可分扩张.
	$$\xymatrix{L\ar@{-}[rr]&&L_w\\K\ar@{-}[u]\ar@{-}[rr]&&K_v\ar@{-}[u]}$$
	\item 我们已经解释了给定$K$嵌入$\tau:L\to\overline{K_v}$,对应了一个$v$的延拓赋值$w$.假设$\sigma\in G(\overline{K_v}/K_v)$,那么复合$\xymatrix{L\ar[r]^{\tau}&\overline{K_v}\ar[r]^{\sigma}&\overline{K_v}}$也是一个$K$嵌入,它称为在$K_v$上和$\tau$共轭的.
	\item 延拓定理.设$K\subseteq L$是有限代数扩张,设$v$是$K$上的赋值,那么有如下结论.另外如果$K\subseteq L$是无限扩张,用记号$L_w$表示$L/K$的全部有限子扩张的完备化的并,它称为$L$关于赋值$w$的局部化.如果这里记号$L_w$理解为局部化,那么仍然有如下两个结论.
	\begin{itemize}
		\item 每个延拓$v$的$L$上的赋值一定是被某个$K$嵌入$L\to\overline{K_v}$诱导的.
		\begin{proof}
			
			任取$v$在$L$上的延拓赋值$w$,设$L_w$是$w$在$L$上的局部化.于是$w$是$K_v$到$L_w$的唯一延拓.任取$K_v$嵌入$\tau:L_w\to\overline{K_v}$,那么赋值$\overline{v}\circ\tau$只能等价于$w$.于是$\tau$限制为$L\to\overline{K_v}$是诱导了$w$的$K$嵌入.
		\end{proof}
		\item 两个延拓$v$的$L$上的赋值等价当且仅当它们对应的$K$嵌入是互相共轭的.
		\begin{proof}
			
			充分性,任取互相共轭的两个$K$嵌入$\tau$和$\sigma\circ\tau$,其中$\sigma\in G(\overline{K_v}/K_v)$.按照$K_v$上的赋值$v$唯一的延拓到$\overline{K_v}$为$\overline{v}$,得到$\overline{v}=\overline{v}\circ\sigma$.于是$\overline{v}\circ\tau=\overline{v}\circ(\sigma\circ\tau)$,这说明这两个互相共轭的$K$嵌入诱导了等价的赋值.
			
			必要性,设$\tau,\tau':L\to\overline{K_v}$是两个$K$嵌入,满足$\overline{v}\circ\tau=\overline{v}\circ\tau'$.记$\sigma=\tau'\circ\tau^{-1}:\tau(L)\to\tau'(L)$是一个$K$同构.于是$\sigma$可延拓为一个$K_v$同构$\sigma:\tau(L)K_v\to\tau'(L)K_v$:事实上按照$K$在$K_v$上稠密得到$\tau(L)$在$\tau(L)K_v$中稠密,于是$\tau(L)K_v$中每个元$x$可以表示为一个柯西列$\{\tau(x_n)\}$的极限,这里$\{x_n\}$是$K\subseteq L$的某个有限子扩张中的序列.按照$\overline{v}\circ\tau=\overline{v}\circ\tau'$得到序列$\{\tau'(x_n)=\sigma\tau(x_n)\}$收敛于$\tau'(L)K_v$中的一个元.于是$\sigma$延拓为$G(\overline{K_v}/K_v)$中的元,于是$\tau'=\sigma\circ\tau$,这说明$\tau$和$\tau'$是共轭的.
		\end{proof}
	\end{itemize}
	\item 单扩张情况的延拓定理.设$v$为$K$上的一个赋值,设$L=K(\alpha)$是单扩张,设$\alpha$在$K$上的不可约多项式为$f(X)$,设$f(X)$在完备化$K_v$中分解为$f(X)=\prod_{1\le i\le r}f_i(X)^{m_i},m_i\ge1$.其中$f_i(X)$是$K_v$中两两不同的不可约多项式.于是如果$f$是可分多项式,那么这里$m_i\equiv1$.于是这里$K$嵌入$L\to\overline{K_v}$一一对应于$f(X)$在$\overline{K_v}$中的根$\tau(\alpha)=\beta$,$f(\beta)=0$.两个嵌入互相共轭当且仅当它们对应的根$\tau(\alpha)$和$\tau'(\alpha)$在$K_v$中共轭,也即这两个根所属于相同的某个不可约分支$f_i(X)$的根.于是结合上一条结论,说明$v$在$L$上的所有延拓在等价意义下恰好一一对应于每个不可约分支$f_i,1\le i\le r$,这里$f_i$是$f$在$K_v$中唯一分解后所出现的全部不可约因式.
	\item 设$K\subseteq L$是有限扩张,用记号$w\mid v$表示$w$是$v$在$L$上的一个延拓.包含映射$L\to L_w$和延拓映射$K_v\to L_w$诱导了同态$L\otimes_KK_v\to L_w$为$a\otimes b\mapsto ab$.这里$L\otimes_KK_v$实际还是一个$K$代数,于是上述同态是$K$代数同态.于是我们得到了一个$K$代数同态$\varphi:L\otimes_KK_v\to\prod_{w\mid v}L_w$,我们断言如果$K\subseteq L$是有限可分扩张,那么这里$\varphi$是同构.
	\begin{proof}
		
		设$\alpha$是$K\subseteq L$的本原元,于是可记$L=K(\alpha)$,记$f(X)\in K[X]$是它在$K$上的极小多项式.对每个延拓赋值$w\mid v$,记它对应的$f(X)$的不可约因子是$f_w(X)\in K_v[X]$.另外按照可分性,得到$f(X)=\prod_{w\mid v}f_w(X)$.把这些$L_w$都视为$\overline{K_v}$的子域,记$\alpha$在$L\to L_w$下的像为$\alpha_w$,那么$L_w=K_v(\alpha_w)$,并且$\alpha_w$在$K_v$上的不可约多项式为$f_w(X)$.据此得到如下交换图表.按照中国剩余定理,第一行的同态是同构.另外两个垂直映射都是同构(例如左边是$L\otimes_KK_v\cong K[x]/(f)\otimes_KK_v\cong K_v[X]/(f)$),于是第二行的典范映射是一个同构.
		$$\xymatrix{K_v[X]/(f)\ar[rr]\ar[d]&&\prod_{w\mid v}K_v[X]/(f_w)\ar[d]\\L\otimes_KK_v\ar[rr]&&\prod_{w\mid v}L_w}$$
	\end{proof}
	\item 设$K\subseteq L$是有限可分扩张,那么$[L:K]=\sum_{w\mid v}[L_w:K_v]$,并且$\mathrm{N}_{L/K}(\alpha)=\prod_{w\mid v}\mathrm{N}_{L_w/K_v}(\alpha)$和$\mathrm{T}_{L/K}(\alpha)=\sum_{w\mid v}\mathrm{T}_{L_w/K_v}(\alpha)$.
	\begin{proof}
		
		按照$[L:K]=\dim_K(L)=\dim_{K_v}(L\otimes_KK_v)$,于是按照上一条证明的$K$代数同构$L\otimes_KK_v\cong\prod_{w\mid v}L_w$得到这里第一个等式.后面两个等式只要注意到数乘$\alpha$这个线性变换在$K$线性空间$L$上和在$K_v$线性空间$L\otimes_KK_v$是相同的,所以按照上一条的同构得到$p_{L/K}(x)=\prod_{w\mid v}p_{L_w/K_v}(x)$,这里$p_{L/K}$表示$\alpha$在$L/K$上的特征多项式,$p_{L_w/K_v}$同理.这个多项式等式立刻得到后两个等式.
	\end{proof}
	\item 数域的情况.设$K$是数域,设$K_p$是它关于极大理想$p$的完备化,那么$K_p$的有限扩张必然具有形式$L_q$,其中$L/K$是有限扩张,$q\in\mathrm{Spec}\mathscr{O}_L$是$p$的提升素理想.
	\begin{proof}
		
		设$F/K_p$是$n$次扩张,按照本原元定理,有$F=K_p(x)$,不妨设$x$是整元,也即$v(x)\ge0$,其中$v$是$K_p$上的$p$-adic赋值在$F$上的延拓.设$x$的极小多项式为$P(X)=X^n+a_1X^{n-1}+\cdots+a_n$,其中$a_i\in\mathscr{O}_{K_p}$.因为$\mathscr{O}_K$在$\mathscr{O}_{K_p}$中稠密,可选取$\mathscr{O}_K[X]$中的$n$次首一多项式$Q(X)$,使得$|P-Q|$足够小.按照Kransner引理的推论,当$|P-Q|$足够小的时候有$Q$的一个根$y$使得$K_p(x)=K_p(y)$.记$L=K(y)$,那么$[L:K]=n$,于是$L\otimes_KK_p=K_p(y)=K_p(x)=F$.这迫使$L$中存在唯一的素理想$q$提升了$p$,因为如果多于一个的话,我们有$L\otimes_KK_p$是一些非平凡域的直积,导致它甚至不能是整环,从而矛盾.于是有$F=L_q$.
	\end{proof}
    \item Galois扩张的情况.回顾一下无限Galois理论:对无限Galois扩张,此时扩张自同构群上具有所谓Krull拓扑,任取有限Galois子扩张$K\subseteq M$,任取陪集$\sigma G(M/K)$定义为开集,此时Galois群是一个紧Hausdorff拓扑群,此时扩张的中间域一一对应于Galois群的闭子群.考虑一个维数可以是无穷的Galois扩张$K\subseteq L$,设$v$是$K$上的阿基米德或者非阿基米德赋值,设$G$是扩张的Galois群,设$v$在$L$上的所有延拓赋值记作$S$,那么$G$在$S$上的作用是可迁的.也即如果$w$和$w'$是两个$v$在$L$上的延拓赋值,那么存在$\sigma\in G$使得$w\circ\sigma=w'$,这等价于讲$v$在$L$上的延拓赋值在等价意义下唯一.
    \begin{proof}
    	
    	先设$K\subseteq L$是有限Galois扩张,如果延拓赋值$w$和$w'$不是共轭的,那么集合$\{w\circ\sigma\mid\sigma\in G\}$和$\{w'\circ\sigma\mid\sigma\in G\}$是无交的$S$的子集.按照逼近定理,可取$x\in L$使得$|\sigma(x)|_w<1$和$|\sigma(x)|_{w'}>1$对每个$\sigma\in G$成立.这导致$|\alpha|_v=\prod_{\sigma}|\sigma(x)|_w<1$和$|\alpha|_v=\prod_{\sigma}|\sigma(x)|_{w'}>1$矛盾.
    	
    	再设$K\subseteq L$是无限维Galois扩张,那么它是全部有限子扩张$K\subseteq M$的合成.考虑$X_M=\{\sigma\in G: w\circ\sigma\mid_M=w'\mid_M\}$,按照有限情况已经得证,说明对每个有限子扩张$M$,这个集合总是非空的.再说明这个集合是闭的:任取$\sigma\in G-X_M$,那么$\sigma G(L/M)$是包含了$\sigma$的开子集(这是Krull拓扑的定义).于是问题归结为证明$\cap_MX_M$非空,其中$M$取遍全部有限子扩张.如果这个交是空集,按照Galois群的紧性,说明存在有限个有限子扩张$M_i$使得$\cap_{1\le i\le r}X_{M_i}$是空集.但是对于有限扩张我们解释了这个交是非空的,这就矛盾.
    \end{proof}
\end{enumerate}

基本量.设$L$是非阿基米德赋值域$(K,v)$的有限扩张,设$w$是$v$在$L$上的延拓.记$e=e(w\mid v)=[w(L^*):v(K^*)]$是扩张的分歧指数,记$f=f(w\mid v)=[\kappa(L):\kappa(K)]$是惯性次数,这里$\kappa(K)$和$\kappa(L)$是$K$和$L$的相应赋值环$A$和$B$的剩余域.
\begin{enumerate}
	\item 如果$(K,v)$是离散赋值域,我们解释过延拓$(L,w)$也是离散赋值域.记$A$和$B$的极大理想分别是$m$和$n$,我们断言分歧指数恰好满足$mB=n^e$.于是分歧指数的定义吻合于戴德金整环的情况.
	\begin{proof}
		
		记$A$和$B$的素元分别为$\pi$和$\Pi$.那么有$e=[w(\Pi)\mathbb{Z}:v(\pi)\mathbb{Z}]$,于是$ew(\Pi)=v(\pi)$,也即有$\varepsilon\in B^*$使得$\pi=\varepsilon\Pi^e$,也即$mB=n^e$.
	\end{proof}
    \item 基本量对于有限扩张链依旧满足传递公式.即如果有非阿基米德赋值域之间的有限扩张$K\subseteq L\subseteq F$,那么$e(F/K)=e(F/L)e(L/K)$和$f(F/K)=f(F/L)f(L/K)$.
    \item 设$v$是$K$上的离散赋值,任取有限可分扩张$K\subseteq L$,任取$v$在$L$上的延拓赋值$w$,记这个延拓赋值的分歧指数为$e_w=[w(L^*):v(K^*)]$,记惯性次数为$f_w=[\lambda_w:\kappa]$,其中$\lambda_w$是$L_w$上的剩余域.那么我们证明过完备的情况下(并且此时$L_w/K_v$是可分的)有$[L_w:K_v]=e_wf_w$.结合我们证明过的$[L:K]=\sum_{w\mid v}[L_w:K_v]$,得到如下结论:如果$v$是$K$上的离散赋值,$K\subseteq L$是有限可分扩张,那么$[L:K]=\sum_{w\mid v}e_wf_w$.这个结论吻合于我们在整体域情况证明的基本量所满足的恒等式$\sum_{1\le i\le r}e_if_i=[L:K]$,这时候$K$的赋值环$A$是DVR,而$L$的赋值环$B$是戴德金整环.
    \item 一般情况下,如果$(K,v)$未必是离散赋值,有限扩张$L/K$未必是可分的,那么有$[L:K]\ge ef$.
\end{enumerate}

非分歧扩张(unramified extensions).设$K$是完备离散赋值域,设$L/K$是有限扩张,它称为非分歧扩张,如果剩余域扩张是可分的,并且$[L:K]=[\kappa(L):\kappa(K)]$.非分歧扩张记作nr(non ramifi\'ee).下面给出一些性质,这里$K$始终约定为完备离散赋值域.
\begin{enumerate}
	\item 扩张$L/K$未必是可分的,理应未必有基本量恒等式,但是这里是成立的,因为我们解释过有$[L:K]\ge ef$,就有$[L:K]\ge ef=e[L:K]$,于是这个不等式取等,并且有$e=1$,此即赋值群是相同的(没有出现"分歧").换句话讲定义中的$[L:K]=[\kappa(L):\kappa(K)]$等价于讲基本量恒等式成立并且$e=1$.
	\item 设代数扩张$K\subset\overline{K}$存在两个子扩张$K\subseteq L$和$K\subseteq K'$,记$L'=LK'$,如果$K\subseteq L$是非分歧扩张,那么$K'\subseteq L'$是非分歧扩张.
	$$\xymatrix{&L'=LK'&\\L\ar[ur]&&K'\ar[ul]_{\text{nr}}\\&K\ar[ur]\ar[ul]^{\text{nr}}&}$$
	\begin{proof}
		
		分别记$K,K',L,L'$的赋值环,极大理想,剩余域为$\{A,m,\kappa\},\{A',m',\kappa'\},\{B,n,\lambda\}$,$\{B',n',\lambda'\}$.此时剩余域扩张$\kappa\subset\lambda$也是有限扩张.按照条件剩余域扩张还是可分的,于是本原元定理说明$\lambda=\kappa(\overline{\alpha})$,其中$\alpha\in B$,$\overline{\alpha}$是它的一个提升.设$\alpha$在$A$上的极小多项式为$f(x)\in A[x]$.设$\overline{f}(x)$是$f(x)$在$\kappa[x]$中的像.由于$\overline{f}$是首一的,它的提升的次数不改变.那么有:
		$$[\lambda:\kappa]\le\deg\overline{f}=\deg f=[K(\alpha):K]\le[L:K]=[\lambda:K]$$
		
		于是这一串不等式都取等号,特别的有$L=K(\alpha)$,并且$\overline{f}(x)$就是$\overline{\alpha}$在$\kappa$上的极小多项式.于是有$L'=K'(\alpha)$.再取$\alpha$在$K'$上的极小多项式$g(x)\in A'[x]$,设$\overline{g}(x)$是$g(x)$在$\kappa'[x]$中的像.那么$\overline{g}(x)$是$\overline{f}(x)$的因子,于是$\overline{g}(x)$是可分的.另外$\overline{g}(x)$一定是不可约的,否则按照Hensel引理,得到$g(x)$是可约的.于是有:
		$$[\lambda':\kappa']\le[L':K']=\deg g=\deg\overline{g}=[\kappa'(\overline{\alpha}):\kappa']\le[\lambda':\kappa']$$
		
		于是这一串不等式也都取等号,于是$[L':K']=[\lambda':\kappa']$,于是$K'\subseteq L'$是非分歧的.
	\end{proof}
	\item 非分歧扩张的传递性.
	\begin{itemize}
		\item 非分歧扩张的子扩张都是非分歧的.事实上如果$K\subseteq L'$是非分歧扩张,取子扩张$K\subseteq L$,按照上一条结论,得到$L\subseteq L'$是非分歧的,但是从$[L':K]=f(L'/K)$和$[L':L]=f(L'/L)$,传递公式得到$[L:K]=f(L/K)$,于是$K\subseteq L$是非分歧的.另外剩余域扩张$\kappa(K)\subset\kappa(L)$是可分的,因为$\kappa(K)\subset\kappa(L')$和$\kappa(L)\subset\kappa(L')$都是可分的
		\item 两个非分歧扩张的复合是非分歧的.首先两个剩余域扩张是可分的,它们的复合还是可分扩张.下面依旧按照传递公式得到如果$[F:L]=f(F/L)$和$[L:K]=f(L/K)$,那么$[F:K]=f(F/K)$.
		\item 上两条结论说明,给定扩张$K\subseteq L\subseteq F$,那么$K\subseteq F$是非分歧的当且仅当$K\subseteq L$和$L\subseteq F$都是非分歧扩张.
	\end{itemize}
	\item 非分歧扩张的合成仍然是非分歧的:如果$K\subseteq L_1$和$K\subseteq L_2$都是非分歧的,那么$K\subseteq L_1L_2$仍然是非分歧的.
	\item 关于无限维扩张.完备离散赋值域的一个无限维代数扩张称为非分歧的,如果它可以表示为一族有限非分歧扩张的并.我们证明过有限非分歧扩张的子扩张都非分歧,于是这个定义等价于讲这个无限维扩张的每个有限子扩张都是非分歧的.
	\item 给定代数扩张$K\subseteq L$,它的所有非分歧子扩张的合成称为这个代数扩张的极大非分歧子扩张.对于到代数闭包的扩张$K\subset\overline{K}$,它的极大非分歧子扩张就称为$K$的极大非分歧扩张,记作$K^{\mathrm{nr}}$.如果完备离散赋值域的有限扩张$L/K$的极大非分歧子扩张是$K$,则称$L/K$是完全分歧扩张.
\end{enumerate}

非分歧扩张和剩余域可分扩张之间的对应.设$K$是完备离散赋值域,设$\kappa$是剩余域.
\begin{enumerate}
	\item 设$\kappa\subseteq\kappa'$是有限可分扩张,那么存在同构意义下唯一的有限非分歧扩张$K\subseteq K'$,使得$K'/K$的剩余域扩张同构于$\kappa'/\kappa$.并且$\kappa'/\kappa$是Galois扩张当且仅当$K'/K$是Galois扩张.按照无限扩张是全部有限子扩张的并,这说明完备离散赋值域的非分歧扩张恰好一一对应于剩余域的可分扩张,并且次数对应相同.
	\begin{proof}
		
		设有限可分扩张$\kappa'/\kappa$的本原元是$\xi$,设它在$\kappa$上的极小多项式为$\varphi$,记次数为$n$.记$\varphi$在$A[X]$中的一个$n$次的提升为$f$,有$A'=A[X]/(f)$也是DVR,唯一极大理想是$(\mathfrak{m},f(X))$,其中$\mathfrak{m}$是$A$的唯一极大理想,它在$A$上非分歧,剩余域同构于$\kappa'$,那么它的商域$K'$满足命题.另外$f$在$K$上分裂当且仅当$\varphi$在$\kappa$上分裂,所以$K'/K$是Galois的当且仅当$\kappa'/\kappa$是Galois的.
		
		\qquad
		
		下面证明同构意义下的唯一性,假设在固定的代数闭包中存在两个满足这个条件的有限非分歧扩张$L_1$和$L_2$,那么$L_1L_2/K$也是有限非分歧扩张,并且$L_1L_2$的剩余域仍然是$\kappa'$,于是$[L_1L_2:K]=[\kappa':\kappa]=[L_1:K]$,这迫使$L_1=L_2$.
	\end{proof}
    \item 设$K\subseteq L$是代数扩张,设它对应的剩余域扩张为$\kappa\subset\lambda$,设极大非分歧子扩张为$K\subseteq T$,那么$T$所对应的剩余域恰好就是剩余域扩张的可分闭包$\lambda_s$.
    \begin{proof}
    	
    	设$\lambda_0$是$T$的剩余域.任取$\overline{\alpha}\in\lambda$在$\kappa$上可分,需要证明$\overline{\alpha}\in\lambda_0$.设$\overline{\alpha}$在$\kappa$上的极小多项式为$\overline{f}(x)\in\kappa[x]$,设它一个在$K$上的提升为$f(x)\in A[x]$.按照Hensel引理,$f(x)$不可约,并且它在$L$中存在一个根$\alpha$恰好是$\overline{\alpha}$的提升.这导致$[K(\alpha):K]=[\kappa(\overline{\alpha}):\kappa]$.于是$K\subseteq K(\alpha)$是非分歧的,于是$\alpha\in T$,于是$\overline{\alpha}\in\lambda_0$.
    \end{proof}
    \item 设$K'/K$是有限非分歧扩张,对应剩余域扩张记作$\kappa'/\kappa$,设$K''/K'$是任意有限扩张,对应剩余域扩张记作$\kappa''/\kappa'$.那么把$K$代数同态限制在剩余域上得到了如下典范双射:
    $$\mathrm{Hom}_K(K',K'')\cong\mathrm{Hom}_{\kappa}(\kappa',\kappa'')$$
    \begin{proof}
    	
    	设$A'$和$A''$是$A$在$K'$和$K''$的整闭包,那么$K$代数同态做限制就得到典范双射$\mathrm{Hom}_K(K',K'')\cong\mathrm{Hom}_A(A',A'')$.我们来证明$A$代数同态诱导在剩余域上得到的同态$\theta:\mathrm{Hom}_A(A',A'')\to\mathrm{Hom}_{\kappa}(\kappa',\kappa'')$是双射.
    	
    	\qquad
    	
    	设$[K':K]=n$,因为$A$是DVR,所以存在$x\in A'$使得$\{1,x,\cdots,x^{n-1}\}$是$A'$在$A$上的一组基.设$x$的特征多项式为$f$,那么$\overline{f}\in\kappa[X]$是$\overline{x}$在$\kappa$上的特征多项式.所以$\mathrm{Hom}_A(A',A'')$中的元一一对应于满足$f(a'')=0$的元$a''\in A''$.类似的$\mathrm{Hom}_{\kappa}(\kappa',\kappa'')$中的元一一对应于满足$\overline{f}(\xi'')=0$的元$\xi''\in\kappa''$.所以问题归结为$\overline{f}$在$\kappa''$中的根唯一的提升为$f$在$A''$中的根.但是这是因为$K$是完备离散赋值域,所以有Hensel引理,并且$\overline{f}$是可分多项式.
    \end{proof}
    \item 如果取$K$的极大非分歧扩张,它对应的剩余域就是$K$剩余域$\kappa$的可分闭包$\kappa_s$.如果$m$不整除$\kappa$的特征,那么$m$次单位根都在可分闭包$\kappa_s$中,按照Hensel引理,这说明$K^{\mathrm{nr}}$也包含了所有$m$次单位根,这里$m$不整除剩余域的特征.如果$\kappa$是有限域,此时可分闭包恰好由这样的单位根生成,于是此时极大非分歧扩张恰好被这样的单位根生成.
\end{enumerate}

温分歧扩张(tamely ramified extensions).设完备离散赋值域$K$的剩余域$\kappa$具有正特征$p$,设有限代数扩张$K\subseteq L$诱导的剩余域扩张是可分的,设$T$是扩张的极大非分歧扩张,如果该代数扩张的分歧指数$[L:T]$和$p$互素,就称该有限扩张是温分歧扩张.
\begin{enumerate}
	\item 非分歧扩张的分歧指数是1,于是非分歧扩张都是温分歧扩张.
	\item 设$K\subseteq L$是有限扩张,设极大非分歧子扩张为$T$,那么$K\subseteq L$是温分歧扩张当且仅当扩张$T\subseteq L$可表示为$L=T(\sqrt[m_1]{a_1},\sqrt[m_2]{a_2},\cdots,\sqrt[m_r]{a_r})$,其中每个$m_i$都和$p$互素.并且此时有基本恒等式$ef=[L:K]$(尽管扩张未必是可分的,可分条件下基本恒等式总成立).
	\begin{proof}
		
		为证必要性不妨设$T=K$,此时剩余域扩张是可分的,所以有$1=[T:K]=[\kappa(L):\kappa(K)]$,设$K\subseteq L$是温分歧扩张,此即$(e,p)=1$.先证明如果$e=1$,那么有$L=K$.否则可取$\alpha\in L-K$,设它全部共轭元为$\alpha=\alpha_1,\alpha_2,\cdots,\alpha_m$,设$a=\mathrm{T}(\alpha)=\sum_{i=1}^m\alpha_i\in K$.那么$\beta=\alpha-\frac{1}{m}\in L-K$满足$\mathrm{T}(\beta)=0$(这里$m$可逆是因为$([L:T],p)=1$,而$m\mid [L:T]$).再按照$v(K^*)=w(L^*)$,于是可取$b\in K^*$使得$v(b)=w(\beta)$.于是有单位$\varepsilon=\beta/b\in L-K$满足$\mathrm{T}(\varepsilon)=0$.下面断言$\varepsilon$的全部共轭元$\varepsilon_i$在剩余域$\kappa(L)=\kappa(K)$中是相同的:任取$\varepsilon$的共轭元$\varepsilon'$,从典范映射$A/m\to B/n$是同构说明存在$a\in A$使得$a+n=\varepsilon+n$,但是存在$\kappa(L)$上的自同构把$\varepsilon$打为$\varepsilon'$,并且这个自同构把$n$映为$n$,所以有$\varepsilon+n=a+n=\varepsilon'+n$,于是$\varepsilon$和$\varepsilon'$在剩余域上相同.于是从$0=\sum_{i=1}^m\overline{\varepsilon_i}$得到$m\overline{\varepsilon}=0$,但是这里$(m,p)=1$和$\overline{\varepsilon}$是单位,这矛盾.
		
		\qquad
		
		现在取$\omega_1,\omega_2,\cdots\omega_r\in w(L^*)$构成了有限交换群$w(L^*)/v(K^*)$的一组生成元集,设$\omega_i$在商群中的阶数为$m_i$,于是$m_i\mid e$,于是$(m_i,p)=1$.可设$\gamma_i\in L^*$使得$w(\gamma_i)=w_i$,那么有$w(\gamma_i^{m_i})=v(c_i)$,其中$c_i\in K$,于是存在单位$\varepsilon_i\in L$使得$\gamma_i^{m_i}=c_i\varepsilon_i$.按照赋值群相同,可记$\varepsilon_i=b_iu_i$,其中$b_i\in K$和$u_i\in L^*$.按照Hensel引理,$x^{m_i}-u_i=0$在$L$上有根,可设为$\beta_i$,那么有$(\gamma_i\beta_i^{-1})^{m_i}=\varepsilon_ib_i$.记$\alpha_i=\gamma_i\beta_i^{-1}$,有$w(\alpha_i)=\omega_i$,其中$\alpha_i^{m_i}=a_i\in K^*$.于是$K(\sqrt[m_i]{a_i})\subseteq L$.但是这个扩张的赋值群相同,剩余域相同,按照上一段得到二者相同.
		
		\qquad
		
		下面证明基本量恒等式,依旧设$L/K$是完全分歧的,所以等价于证明$[L:K]=e$.但是一般情况下有$[L:K]\ge ef$,所以归结为证明$[L:K]\le e$.记$L_1=K(\sqrt[m_1]{a_1})$,那么按照$\omega_1\in w(L_1^*)$有$e(L_1/K)=[w(L_1^*):v(K^*)]\ge m_1\ge[L_1:K]$.下面$w(L^*)/w(L_1^*)$被$\omega_2,\cdots,\omega_r$生成,对$r$做归纳就得到$e=e(L/L_{r-1})\cdots e(L_1/K)\ge[L:L_{r-1}]\cdots[L_1:K]=[L:K]$.
		
		\qquad
		
		先证充分性.温分歧的复合自然是温分歧的,所以归结为证明$r=1$的情况.再不妨设$K=T$.记$L=K(\sqrt[m]{a})$,其中$(m,p)=1$.我们断言不妨设$\kappa(K)$是可分闭的(即是代数闭包的可分闭包).记$K_1$是$K$在某个代数闭包中的极大非分歧扩张,那么它的剩余域$\kappa(K_1)$就是$\kappa(K)$的可分闭包.记$L_1=K_1(\sqrt[m]{a})$,一旦我们证明$L_1/K_1$是温分歧的,那么$\kappa(L_1)/\kappa(K_1)$是可分扩张,于是它是平凡扩张,于是$\kappa(L)/\kappa(K)$是可分扩张,并且$p\not|[L_1:K_1]=[L:K]$(因为$L\cap K_1=K$).
		
		\qquad
		
		最后证明充分性.设$\alpha=\sqrt[m]{a}$,不妨记$[L:K]=[K(\sqrt[m]{a}):K]=m$,否则设$d$是$m$最大的因子,使得存在$a_0\in K^*$满足$a=a_0^d$,那么用$m_0=m/d$代替$m$就导致$L=K(\sqrt[m_0]{a_0})$满足$[L:K]=m_0$.再记$w(\alpha)$在$w(L^*)/v(K^*)$中的阶数是$n$.从$mw(\alpha)=v(a)\in v(K^*)$得到$m=dn$.记$b\in K^*$使得$nw(\alpha)=v(b)$,那么有$dv(b)=mw(\alpha)=v(a)$.于是存在$K$中单位$\varepsilon$使得$\alpha^m=a=\varepsilon b^d$.按照$d\mid m$得到$(d,p)=1$,于是$X^d-\varepsilon=0$在可分闭域$\kappa(K)$上分裂为不同一次因式乘积.按照Hensel引理,$X^d-\varepsilon=0$就在$K$上分解为不同一次因式乘积.于是可以消掉$\varepsilon$,即存在$b_0\in K$使得$\alpha^m=b_0^d=a$.但是$[L:K]=m$导致$X^m-a$是不可约多项式,所以如果$X^d-a$有解$b_0\in K$就导致$X^m-a$是可约的,这个矛盾导致$d=1$,于是$m=n$,于是$e\ge m=n=[L:K]\ge ef\ge e$,于是$f=1$,并且$p\not| n=e$,于是$L/K$是温分歧的.
	\end{proof}
    \item 既然温分歧扩张满足基本量恒等式,导致$e=[L:T]$.
    \item 设$K\subseteq L$和$K\subseteq K'$是$K\subset\overline{K}$的两个中间域,设$L'=LK'$,如果$K\subseteq L$是温分歧扩张,那么$K'\subseteq L'$也是温分歧扩张.
    \begin{proof}
    	
    	设$K\subseteq L$和$K'\subseteq L'$的极大非分歧子扩张分别为$T$和$T'$,我们解释过$K\subseteq TK'$是非分歧的,于是$TK'\subseteq T'$.按照$K\subseteq L$是温分歧扩张,得到$L=T(\sqrt[m_i]{a_i})$,其中$(m_i,p)=1$,那么$L'=LK'=LT'=T'(\sqrt[m_i]{a_i})$,这说明$K\subseteq L'$是温分歧的.
    	$$\xymatrix{L\ar[rr]\ar[d]&&L'\ar[d]\\T\ar[rr]\ar[d]&&T'\ar[d]\\K\ar[rr]&&K'}$$
    \end{proof}
    \item 温分歧扩张的传递性.给定扩张$K\subseteq L\subseteq F$,那么$K\subseteq F$是温分歧的当且仅当$K\subseteq L$和$L\subseteq F$都是温分歧扩张.
    \item 上两条结论说明,温分歧扩张的合成仍然是温分歧的:如果$K\subseteq L_1$和$K\subseteq L_2$是温分歧的,那么$K\subseteq L_1L_2$仍然是温分歧扩张.
	\item 关于无限维扩张.如果代数扩张$K\subseteq L$是无限维的,称它是温分歧扩张,如果它可以表示为一族有限温分歧扩张的并.按照有限温分歧扩张的子扩张都是温分歧的,说明这个定义等价于讲这个无限维扩张的每个有限维子扩张都是温分歧的.
	\item 给定代数扩张$K\subseteq L$,它的全部温分歧扩张的合成$V$称为这个代数扩张的极大温分歧扩张.记$w(L^*)^{(p)}$表示$w(L^*)$由这样的元$\omega$构成的子群,它满足存在某个和$p$互素的整数$m$使得$m\omega\in v(K^*)$.于是商群$w(L^*)^{(p)}/v(K^*)$由$w(L^*)/v(K^*)$的和$p$互素的元构成的子群.如果记$V$是极大温分歧扩张,那么我们断言赋值群满足$w(V^*)=w(L^*)^{(p)}$,并且极大温分歧扩张的剩余域就是$K\subseteq L$对应的剩余域扩张$\kappa\subset\lambda$的可分闭包$\lambda_s$.
	\begin{proof}
		
		不妨把$K$替换为极大非分歧子扩张,此时$\kappa=\lambda_s$.按照$p$不整除$e(V/K)$,此即赋值商群$w(V^*)/v(K^*)$的阶数,于是有$w(V^*)\subseteq w(L^*)^{(p)}$.反过来对每个$\omega\in w(L^*)^{(p)}$,它的根$\alpha=\sqrt[m]{a}\in L$,其中$a\in K,(m,p)=1$,$w(\alpha)=\omega$,于是$\alpha\in V$,这导致$\omega\in w(V^*)$.
	\end{proof}
\end{enumerate}

整理一下,设$K$是完备离散赋值域,取代数扩张$L$,那么扩张$K\subseteq L$可分解为如下三个扩张.如果$[L:K]=n$,如果$K$的特征为$p$,如果分歧指数$e$可分解为$e=e'p^a$,其中$e'$和$p$互素,那么这里$[V:T]=e'$.如果这个扩张的极大非分歧子扩张是平凡的,也即这里$T=K$,就称扩张是完全分歧扩张(totally ramified extensions).如果这个扩张不是温分歧扩张,也即这里$V\not=L$,就称它是野分歧扩张(wildly ramified extensions).
$$\xymatrix{K\ar@/^3pc/[rrrr]^{\text{极大温分歧}}\ar@/^1pc/[rr]^{\text{极大非分歧}}\ar[rr]\ar[d]&&T\ar@/^3pc/[rrrr]^{\text{完全分歧}}\ar[rr]\ar[d]&&V\ar[rr]\ar[d]&&L\ar[d]\\\kappa\ar[u]\ar[rr]\ar[d]&&\lambda_s\ar[u]\ar[rr]^{=}\ar[d]&&\lambda_s\ar[u]\ar[d]\ar[rr]&&\lambda\ar[u]\ar[d]\\v(K^*)\ar[u]\ar[rr]^{=}&&w(T^*)\ar[u]\ar[rr]&&w(L^*)^{(p)}\ar[u]\ar[rr]&&w(L^*)\ar[u]}$$

数域完备化的情况.我们接下来考虑的完备离散赋值域$K$是数域的完备化,在这种情况下$K$的特征是零,剩余域是有限域,它的代数扩张总是可分的.
\begin{enumerate}
	\item 对每个正整数$n$,在同构意义下存在$\mathbb{Q}_p$的唯一的$n$次非分歧扩张,它就是$f(X)=X^q-X$在$\mathbb{Q}_p$的分裂域,这里$q=p^n$.
	\begin{proof}
		
		首先$\mathbb{Q}_p$的剩余域是$\mathbb{F}_p$,对每个正整数$n$它总存在同构意义下唯一的$n$次扩张,也即$\kappa=\mathbb{F}_q$.按照完备离散赋值域的有限非分歧扩张和剩余域的有限可分扩张的一一对应关系,$\mathbb{Q}_p$在同构意义下唯一的存在$n$次非分歧扩张,记作$K$.下面证明$K$就是$f(X)$在$\mathbb{Q}_p$上的分裂域$L$:首先按照Hensel引理有$f(X)$在$K$上分裂,于是$L\subseteq K$,但是$L$的剩余域也是$\kappa$,这两个非分歧扩张$L,K$具有相同剩余域,所以二者相同.
	\end{proof}
    \item 考虑任意的$p$-adic域$K$(此即某个数域关于某个有限素位的完备化),记它剩余域的阶数是$q$,对每个正整数$n$,在同构意义下存在$K$唯一的$n$次非分歧扩张$L$,它就是$X^Q-X$在$K$上的分裂域,其中$Q=q^n$.扩张$L/K$是循环Galois扩张,它的Galois群的生成元诱导了剩余域扩张的Frobenius自同构$x\mapsto x^q$.
    \item 推论.设$K$是$p$-adic域,它的极大非分歧扩张$K^{\mathrm{ur}}$是通过添加所有$m$次单位根得到的,这里$m$跑遍和$p$互素的正整数,$p$是$K$剩余域的特征.
    \begin{proof}
    	
    	按照上一条,$K^{\mathrm{ur}}$是通过添加全部$q^n-1$次单位根,$n\ge1$得到的.但是对任意和$p$互素的$m$,有$(m,q)=1$,所以存在$n$使得$q^n\equiv1(\mathrm{mod}m)$.于是添加全部$m$次单位根,其中$(m,p)=1$,得到相同的结果.
    \end{proof}
    \item 设$K$是完备离散赋值域,设它的剩余域是有限域$\kappa=\mathbb{F}_q$,$q=p^r$,考虑单扩张$L=K(\zeta)$,其中$\zeta$是一个$n$次本原单位根.扩张$K\subseteq L$对应的赋值环扩张,剩余域扩张分别为$A\subseteq B$和$\kappa\subset\lambda$,设$(n,p)=1$,那么:
    \begin{itemize}
    	\item 扩张$K\subseteq L$是$f$次非分歧扩张,这里$f$也是最小的自然数使得$q^f\equiv1(\mathrm{mod}n)$.
    	\begin{proof}
    		
    		设$\varphi(X)$是$\zeta$在$K$上的极小多项式,因为$\varphi(X)$是$X^n-1$的因式,所以它是首一的$A[X]$中的多项式,这满足Hensel引理的条件.另外$(n,p)=1$导致$X^n-1$是可分的,于是$\overline{\varphi}(X)$也是可分多项式,所以它的不可约因子都是不同的,于是如果$\overline{\varphi}(X)$在$\kappa$中可约,会导致$\varphi$也可约,于是$\overline{\varphi}(X)$是$\overline{\zeta}$的极小多项式,这两个极小多项式次数相同,于是$[L:K]=[\lambda:\kappa]$是惯性次数,于是$K\subseteq L$是非分歧扩张.
    		
    		\qquad
    		
    		按照$(n,p)=1$,有多项式$X^n-1$在$B$中分裂为不同一次因式的乘积,并且它在$\lambda$中也分裂为不同一次因式的乘积.于是$\lambda=\mathbb{F}_{q^f}$包含了$n$次单位根,并且由$n$次本原单位根生成,于是$f$是最小的自然数使得$n$次本原单位根落在$\mathbb{F}^*_{q^f}$,也即最小的自然数$f$使得$n\mid q^f-1$.
    	\end{proof}
    	\item 扩张$K\subseteq L$的自同构群同构于$G(\lambda/\kappa)$,并且它被$\varphi:\zeta\mapsto\zeta^q$生成.这由上一条直接得到,因为上一条证明了扩张$K\subseteq L$的生成元$\zeta$的极小多项式也是$\mathbb{F}_q\subset\mathbb{F}_{q^f}$的极小多项式,于是Galois群由$\zeta\mapsto\zeta^q$生成.
    	\item $B=A[\zeta]$.
    	\begin{proof}
    		
    		按照$K\subseteq L$是非分歧扩张,那么$mB=n$,其中$m$和$n$分别是$A$和$B$的唯一极大理想.另外按照$\{1,\zeta,\zeta^2,\cdots,\zeta^{f-1}\}$是$\kappa\subset\lambda$的一组基,于是$B=A[\zeta]+mB$,按照NAK引理,得到$B=A[\zeta]$.
    	\end{proof}
    \end{itemize}
    \item 设$L/K$是数域完备化之间的有限扩张,它是完全分歧的当且仅当$L=K(\alpha)$,其中$\alpha$是基域上某个爱森斯坦多项式的根.
    \begin{proof}
    	
    	把绝对值和延拓绝对值统一记作$|\bullet|$.先设$\alpha$是一个爱森斯坦多项式$f(X)=X^n+a_1X^{n-1}+\cdots+a_n\in\mathbb{Z}_p[X]$的根.我们先断言有$|\alpha|^n=|a_n|$:首先如果$|\alpha|\ge1$,那么有$|\alpha|^n=|a_1\alpha^{n-1}+\cdots+a_n|\le\max\{|a_i\alpha^{n-i}|\}<|\alpha^{n-1}|$矛盾.于是只能有$|\alpha|<1$.于是$|a_i\alpha^{n-i}|<|a_i|\le|a_n|$,得到$|\alpha|^n=|a_1\alpha^{n-1}+\cdots+a_n|=|a_n|$完成断言的证明.接下来有$w(a_n)=ev(a_n)=e=nw(\alpha)\ge n$,这迫使$e=n$,也即$K\subseteq K(\alpha)$是完全分歧的.
    	
    	\qquad
    	
    	反过来,设$L/K$是$n$次完全分歧扩张,记$L$的素元为$\Pi$,先断言$\Pi^j,0\le j<n$是$K$线性无关的,不妨设$K$上的离散赋值是规范的,如果存在线性组合$\sum_{0\le j<n}a_j\Pi^j=0$,那么这个和式中至少有两项赋值是相同的,否则从$v(x)\not=v(y)$得到$v(x+y)=\min\{v(x),v(y)\}$从而得到矛盾.设有$w(a_i\Pi^i)=w(a_j\Pi^j)$,在$\mathrm{mod}\mathbb{Z}$下就有$iw(\Pi)\equiv jw(\Pi)$,但是$1\le i\not=j<n$,这就矛盾.于是断言成立.接下来按照扩张次数是$n$,有$\Pi^n$可以表示为$\Pi^j,0\le j<n$的线性组合,于是存在$a_i\in K$,$f(X)=X^n+a_1X^{n-1}+\cdots+a_n$,使得$f(\Pi)=0$.
    	
    	\qquad
    	
    	我们断言$f(X)$是爱森斯坦多项式.按照$f(\Pi)=0$,单项式中赋值最小的项至少有两项,但是$a_i\Pi^{n-i}$的赋值都不同,所以仅有的赋值最小且相同的两项是$w(\Pi^n)=w(a_n)$.按照扩张是完全分歧的,有$e=n$,于是有$1=ew(\Pi)=w(\Pi^n)=w(a_n)=v(a_n)$,并且$w(\Pi^n)=w(a_n)$赋值最小说明$w(a_i\Pi^{n-i})=nv(a_i)+(n-i)/n\ge 1,i\ge1$.进而有$v(a_i)\ge1$,于是$f(X)$是爱森斯坦多项式.
    \end{proof}
    \item 例如如果$\zeta$是$\mathbb{Q}_p$的$p^m$次本原单位根,那么$\mathbb{Q}_p\subseteq\mathbb{Q}_p(\zeta)$是$\varphi(p^m)=p^{m-1}(p-1)$次完全分歧扩张,$\zeta-1$是素元.但是如果$K$是$p$-adic域,可能添加某个$p^m$次本原单位根得到的是非分歧扩张,完全分歧的结论仅对$\mathbb{Q}_p$成立.另外这里的Galois群同构于$(\mathbb{Z}/n\mathbb{Z})^*$,即$a$映射为$\xi\mapsto\xi^a$的映射.另外$\zeta-1$是$\mathbb{Q}_p(\zeta)$的uniformizer,并且有$\mathscr{O}_{\mathbb{Q}_p(\zeta)}=\mathbb{Z}_p[\zeta]$.
    \begin{proof}
    	
    	因为$\xi=\zeta^{p^{m-1}}$是$p$次本原单位根,于是$\xi$满足$X^{p-1}+X^{p-2}+\cdots+1=0$.于是$\zeta$是多项式$\varphi(X)=X^{(p-1)p^{m-1}}+X^{(p-2)p^{m-1}}+\cdots+1$的零点.我们断言$\varphi(X+1)$是爱森斯坦多项式,这样按照上一条就得到$\mathbb{Q}_p(\zeta-1)=\mathbb{Q}_p(\zeta)$是$\mathbb{Q}_p$的完全分歧扩张.另外爱森斯坦多项式是不可约的,于是扩张次数是$p^{m-1}(p-1)$.
    	
    	\qquad
    	
    	首先$\varphi(1)=p$,自然有$v(p)=1$.另外$\varphi(X)=(X^{p^m}-1)/(X^{p^{m-1}}-1)$,这在$\mathrm{mod}p$下$\equiv(X-1)^{p^{m-1}(p-1)}$,于是$\varphi(X+1)$展开式中非首项的系数的加性赋值都$\ge1$,于是这是爱森斯坦多项式.按照$\mathrm{N}_{\mathbb{Q}_p(\zeta)/\mathbb{Q}_p}(1-\zeta)=\varphi(1)=p$,得到$w(1-\zeta)=1/p^{m-1}(p-1)=1/e$,于是$1-\zeta$是素元.
    \end{proof}
    \item 推论.如果$\zeta_n$是$\mathbb{Q}_p$的$n$次本原单位根,记$n=n'p^m$,其中$p\not| n'$,那么极大非分歧扩张是$T=\mathbb{Q}_p(\zeta_{n'})$,极大温分歧扩张是$T(\zeta_p)$:
    $$\mathbb{Q}_p\subseteq T=\mathbb{Q}_p(\zeta_{n'})\subseteq V=T(\zeta_p)\subset\mathbb{Q}_p(\zeta_n)$$
\end{enumerate}

$p$进复数域.考虑完备离散赋值域$\mathbb{Q}_p$,我们已经解释了其上赋值可等价的唯一延拓到它的代数闭包$\overline{\mathbb{Q}_p}$上(这就不再是离散赋值了).对于阿基米德的情况,这个代数闭包$\overline{\mathbb{Q}_{\infty}}=\mathbb{C}$自动完备的.但是对于$p$进的情况$\overline{\mathbb{Q}_p}$并不是完备的,它的完备化记作$\mathbb{C}_p$,我们会证明这个域是完备和代数闭的,它充当$p$进数域中的复数域.
\begin{enumerate}
	\item $\overline{\mathbb{Q}_p}$不是完备的.
	\begin{proof}
		
		反证,假设$\overline{\mathbb{Q}_p}$是完备的.对每个正整数$m$记$\zeta_m$是$\overline{\mathbb{Q}_p}$中的$m$次本原单位根,记函数$f:\mathbb{N}^+\to\overline{\mathbb{Q}_p}$,当$p\not|n$时$f(n)=\zeta_n$,当$p\mid n$时$f(n)=1$.那么从$f(n)^n=1$得到$|f(n)|=1$.于是有收敛级数$\sum_{n\ge1}f(n)p^n=\alpha\in\overline{\mathbb{Q}_p}$.扩张$K=\mathbb{Q}_p(\alpha)/\mathbb{Q}_p$是有限扩张,我们归纳证明每个$f(n)\in K$,但是添加和$p$互素的单位根得到无限扩张,这和$K/\mathbb{Q}_p$是有限扩张相矛盾.
		
		\qquad
		
		$n=1$是平凡的,下面设$m>1$,假设对每个$n<m$有$f(n)\in K$,我们来证明$f(m)\in K$.还可以不妨设$p\not|m$.记$\alpha_m=p^{-m}\left(\alpha=\sum_{n=1}^{m-1}f(n)p^n\right)=\sum_{n\ge m}f(n)p^{n-m}$,那么$\alpha_m\in K$,并且有$\alpha\equiv\zeta_m(\mathrm{mod}p)$,于是在$\mathrm{mod}p$下有$X^m-1$有根,按照Hensel引理就有$\zeta\in K$满足$\zeta^m=1$,并且$\zeta\equiv\alpha_m\equiv\zeta_m(\mathrm{mod}p)$.我们断言这只能有$\zeta=\zeta_m$,这个断言导致$\zeta_m\in K$从而完成归纳.
		
		\qquad
		
		为了证明断言只需证明这样一件事:如果$\zeta_1^m=\zeta_2^m=1$,$\zeta_1\not=\zeta_2$,那么在$\mathrm{mod}p$下有$\zeta_1-\zeta_2$是可逆元.为此注意到$(X^m-1)/(X-\zeta_1)=\prod_{\zeta^m=1,\zeta\not=\zeta_1}(X-\zeta)$,带入$X=\zeta_1$就得到$m\zeta_1^{m-1}=\prod_{\zeta^m=1,\zeta\not=\zeta_1}(\zeta_1-\zeta)$.按照$p\not|m$,及$\zeta_1$是单位,就得到右侧每一个乘项都是单位,于是$\zeta_1-\zeta_2$是单位.
	\end{proof}
	\item $\mathbb{C}_p$是代数闭域.
	\begin{proof}
		
		设$f(x)=x^n+a_{n-1}x^{n-1}+\cdots+a_0\in\mathbb{C}_p[x]$是非常数多项式,不妨设这个多项式落在赋值环中,否则可乘以一个加法赋值足够大的非零元$t^n$使得它的首系数仍然是1,并且其余系数的加法赋值足够大落在赋值环中.并且这个新多项式存在根得到原本的多项式存在根.
		
		记$C'$为$\mathbb{C}_p$关于$f$的分裂域.记$r=\max\limits_{i\not=j}\{v(a_i-a_j)\}$,这里$\{a_i\}$是$f$的全部不同根.按照$\mathbb{C}_p$的完备性,可取$b_i\in\overline{\mathbb{Q}_p}$使得$v(b_i-a_i)>rn$.取$g=\sum_ib_ix^i\in\overline{\mathbb{Q}_p}$,按照$\overline{\mathbb{Q}_p}$是代数闭域,可取$g$的一个根$\beta$.取$f$的根$\alpha$是所有根中离$\beta$最近的根,也即$|\beta-\alpha|\le|\beta-\alpha'|$对任意$f$的根$\alpha'$成立.我们来证明$\alpha\in\mathbb{C}_p[\beta]$,于是有$\alpha\in\mathbb{C}_p$,于是$\mathbb{C}_p$是代数闭域.
		
		按照$f(\beta)-g(\beta)=f(\beta)=\prod_i(\beta-\alpha_i)$,于是$v(f(\beta)-g(\beta))=\sum_iv(\beta-\alpha_i)\le nv(\beta-\alpha)$.但是另一方面$f(\beta)-g(\beta)$的系数满足$v(b_i-a_i)>rn$,导致$v(f(\beta)-g(\beta))\ge rn$.于是整理下不等式得到$v(\beta-\alpha)>r$.也即$v(\beta-\alpha)\ge v(\alpha_i-\alpha_j),i\not=j$.于是$|\beta-\alpha|<|\alpha_i-\alpha|,\forall\alpha_i\not=\alpha$.
	\end{proof}
	\item 球完备性.如果一个完备度量空间中的递降闭球列的交总是非空的,就称它是球完备的.注意对于完备度量空间,如果存在递降的闭球列,如果这些闭球的半径趋于零,在每个闭球中取一个点就会构成柯西列,它的极限一定落在交中.但是如果闭球的半径不趋于零,可能存在这样的例子使得交是空集:在自然数集$\mathbb{N}$上定义度量为$d(m,n)=1+\frac{1}{\min\{m,n\}},m\not=n$.这样以$n$为中心半径为$1+1/n$的闭球是$A_n=\{n+1,n+2,\cdots\}$,这些$\{A_n\}$的交是空集.
	\item $\mathbb{C}$和$\mathbb{Q}_p$都是球完备的.
	\item $\mathbb{C}_p$不是球完备的.
	\begin{proof}
		
		首先假设一个递降的闭球列$\{B(x_n,d_n)\}$的交非空,任取交中的一个元$x_0$,按照非阿基米德度量的性质,有$B(x_n,d_n)=B(x_0,d_n)$,于是$\cap_nB(x_0,d_n)=B(x_0,d_0)$,其中$d_0=\lim_{n\to\infty}d_n$.换句话讲,一个递降的闭球列的交如果非空,那么交一定是一个闭球.
		
		给定闭球$B$,构造它包含的两个不交的闭球$B_{0}$和$B_1$.再分别构造$B_0$和$B_1$所包含的不交的两个闭球$B_{01},B_{00}$和$B_{10},B_{11}$.依次构造下去,对$\{0,1\}^{\mathbb{N}}$中的每个元,都对应了一个所构造的递降的闭球列.
		
		现在假设$\mathbb{C}_p$是球完备的.那么$\{0,1\}^{\mathbb{N}}$中每个元所对应的递降的闭球列的交都是闭球,并且按照构造可知不同闭球列的交是互相不交的.于是我们得到了$\mathbb{C}_p$中不可数个两两不交的闭球.但是按照$\overline{\mathbb{Q}}$是$\mathbb{C}_p$的可数稠密子集.导致两两不交的闭球至多可数,这矛盾.
	\end{proof}
\end{enumerate}
\newpage
\subsection{完备离散赋值域的结构}

完备的阿基米德赋值域$K$只有$\mathbb{R}$和$\mathbb{C}$.
\begin{proof}
	
	【】首先由于赋值是阿基米德的,此时域的特征只能是0.于是有$\mathbb{Q}\subseteq K$.按照完备性,可设$\mathbb{R}\subseteq K$,并且$K$上的赋值$|\bullet|$延拓了$|\bullet|_{\mathbb{R}}$.任取$\xi\in K$,定义连续映射$f:\mathbb{C}\to\mathbb{R}$为$z\mapsto|(\xi-z)(\xi-\overline{z})|^2$.按照阿基米德性质,从$z\to\infty$得到$f(z)\to\infty$,于是可取到$|f|$的下确界$m\ge0$.如果$m>0$,那么$S=\{z\in\mathbb{C}\mid f(z)=m\}$是有界闭子集,于是可取$z_0\in S$使得$|z_0|\ge|z|,\forall z\in S$.再设$g(x)=x^2-(z_0+\overline{z_0})x+z_0\overline{z_0}+\varepsilon$,其中$\varepsilon\in(0,m)$,记它的两个根为$z_1$和$\overline{z_1}$,那么有$|z_1|^2=|z_0|^2+\varepsilon$,于是$|z_1|>|z_0|$.
	
	现在考虑多项式$G(x)=|g(x)-\varepsilon|^n-(-\varepsilon)^n\in\mathbb{R}[x]$,设它的根为$\alpha_1=z_1,\alpha_2,\cdots,\alpha_{2n}\in\mathbb{C}$.可记$G(x)^2=\prod_{1\le i\le 2n}(x^2-(\alpha_i+\overline{\alpha_i})x+\alpha_i\overline{\alpha_i})$,就有$|G(\xi)|^2=\prod_{1\le i\le 2n}f(\alpha_i)\ge m^{2n}\frac{f(z_1)}{m}$.
	
	另一方面$|G(\xi)|\le f(z_0)^n+\varepsilon^n=m^n+\varepsilon^n$.于是$f(z_1)/m\le(1+(\varepsilon/m)^n)^2$.让$n\to\infty$得到$f(z_1)\le m$,但是这导致$|z_1|>|z_0|$矛盾,于是$m=0$,于是$K$中的元都是$\mathbb{R}$上的二次代数元.
\end{proof}

Trichm\"uller提升,$p$环和严格$p$环.
\begin{enumerate}
	\item 设$K$是完备离散赋值域,设剩余域$\kappa$的特征是$p>0$.存在唯一一个乘性的提升$r:\kappa\to\mathscr{O}_K$,这里乘性是指$r(ab)=r(a)r(b)$,提升指的是$r(a)$在典范映射$\mathscr{O}_K\to\kappa$下的像就是$a$.这里$r(a)$称为元素$a$的Trichm\"uller提升,记作$[a]$.
	\begin{proof}
		
		按照$\kappa$是完全域,其上映射$x\mapsto x^p$是满射.于是任取$a\in\kappa$,对每个自然数$n$,可取唯一的$a_n\in\kappa$,满足$a_{n+1}^p=a_n$和$a_n^{p^n}=a$.这里唯一性是因为从$a_n^{p^n}=b_n^{p^n}$得到$(a_n-b_n)^{p^n}=0$,于是$a_n=b_n$.下面任取$a_n$的提升为$\widehat{a_n}\in\mathscr{O}_K$,那么$\widehat{a_{n+1}}^p\equiv\widehat{a_n}(\mathrm{mod}m)$,于是$\widehat{a_{n+1}}^p=\widehat{a_n}+\pi c$,其中$\pi$是一个素元.两边取$p^n$次方,得到$\widehat{a_{n+1}}^{p^{n+1}}=\widehat{a_n}^{p^n}+\pi^{n+1}c'$.于是$\widehat{a_{n+1}}^{p^{n+1}}\equiv\widehat{a_n}^{p^n}(\mathrm{mod}m^{n+1})$,这说明$\{\widehat{a_n}^{p^n}\}$是$K$中的一个柯西列,它的极限定义为$r(a)$.如果选取$a_n$不同的提升$\widehat{a_n}'$,在$\mathrm{mod}m^{n+1}$下依旧有$\widehat{a_n}^{p^n}\equiv(\widehat{a_n}')^{p^n}$,所以极限$r(a)$不依赖每个$a_n$提升的选取.乘性是因为如果取$b\in\kappa$的序列$\{b_n\}$使得$b_n^{p^n}=b$,那么$\{a_nb_n\}$是使得$(a_nb_n)^{p^n}=ab$的序列,按照$r(\bullet)$不依赖提升的选取,得到$r(ab)=r(a)r(b)$.最后说明唯一性,假设还有乘性提升映射$t$,记$t(a_n)=\widehat{a_n}$,那么$\{\widehat{a_n}^{p^n}\}$的极限依旧是$r(a)$.于是有:$r(a)=\lim_{n\to\infty}\widehat{a_n}^{p^n}=\lim_{n\to\infty}t(a_n)^{p^n}=\lim_{n\to\infty}t(a)=t(a)$.
	\end{proof}
    \item 更一般的,如果$A$关于滤过$\alpha_1\supseteq\alpha_2\supseteq\cdots\supseteq$的拓扑是完备和Hausdorff的(滤过要求$\alpha_n\alpha_m\subseteq\alpha_{m+n}$),如果$\kappa=A/\alpha_1$是特征$p$的完全环(完全环是指其上映射$x\mapsto x^p$是满射的环),就称$A$是$p$环,此时称$\kappa$是$A$的剩余环.如果额外的,滤过是$p$-adic的,即$\alpha_n=p^nA$,并且如果$p$不是$A$的零因子,就称$A$是严格$p$环.对于$p$环$A$我们总有:
    \begin{enumerate}
    	\item 存在唯一的提升映射$f:\kappa\to A$满足和$p$次幂可交换:$f(\lambda^p)=f(\lambda)^p,\forall\lambda\in\kappa$.这里提升映射依旧指的是$f(\lambda)$在典范映射$A\to\kappa$下的像恰好是$\lambda$.
    	\item 一个元$a\in A$落在$f(\kappa)$中当且仅当$a$在$A$中总存在$p^n,n\ge0$次根.
    	\item 这个映射$f$是乘性的,即对$\lambda,\mu\in\kappa$有$f(\lambda\mu)=f(\lambda)f(\mu)$.
    \end{enumerate}
    \item 如果$A$是$p$环,记唯一的提升映射为$f$,如果$\alpha_0,\alpha_1,\cdots$是$\kappa=A/\alpha_1$的一个序列,那么$\sum_{i\ge0}f(\alpha_i)p^i$收敛到某个$a\in A$.如果$A$还是严格$p$环,那么$A$中的每个元$a$可以唯一的表示成这种级数和,称对应的序列$\{\alpha_i\}$是$a$的坐标.
    \item 严格$p$环的一个例子.设$X_{\alpha}$是一族未定元,考虑$S=\mathbb{Z}[X_{\alpha}^{p^{-\infty}}]=\cup_{n\ge0}\mathbb{Z}[X_{\alpha}^{p^{-n}}]$.那么完备化$\widehat{S}$是一个严格$p$环.它的剩余环是$\widehat{S}/p\widehat{S}=\mathbb{F}_p[X_{\alpha}^{p^{-\infty}}]$,是一个特征$p$的完全域.另外$X_{\alpha}$存在每个$p^n,n\ge1$次根,所以我们可以把$X_{\alpha}\in\widehat{S}$等同于它在剩余环中的像.
\end{enumerate}

设$K$是完备离散赋值域,设$\kappa$是它的剩余域,如果$K$的特征是$p>0$,那么$\kappa$的特征只能是$p$.于是$(\mathrm{char}K,\mathrm{char}\kappa)$只能存在三种情况$(0,0),(p,p),(0,p)$.前两种情况称为等特征的,后一种情况称为混合特征.我们先处理等特征的情况:
\begin{enumerate}
	\item 如果完备离散赋值域$K$是等特征$p$的,那么$\widehat{a_n}^{p^n}+\widehat{b_n}^{p^n}=\widehat{a_n+b_n}^{p^n}$,取极限得到$[a+b]=[a]+[b]$,于是Trichm\"uller提升$[\bullet]:\kappa\to K$实际上是一个域同态.我们之前解释过对于完备离散赋值域$K$,如果取剩余域$\kappa$在赋值环$A$中的一组提升元集合$S$,那么$A$中元可以唯一的表示为$S$系数的关于素元$\pi$的形式幂级数,并且$K=A[\pi^{-1}]$.现在如果剩余域是完全域,我们解释了提升元集合$S$可以取为$K$的一个(同构于$\kappa$的)子域,于是此时有$A\cong\kappa[[\pi]]$和$K\cong\kappa((x))=\kappa[[x]][x^{-1}]$.换句话讲我们解决了剩余域是完全域,特征对是$(p,p)$情况下完备离散赋值域的结构.另外如果剩余域不是完全域,特征对是$(p,p)$的情况依旧有这个结论,即依旧可以取提升元素集合构成了一个域,但是这样的子域未必是唯一的.
	\item 如果$K$是等特征零的完备离散赋值域,我们断言仍有$\mathscr{O}_K\cong\kappa[[\pi]]$和$K\cong\kappa((\pi))$,按照上一条的解释,为此只要证明存在提升同态$\kappa\subseteq\mathscr{O}_K$即可.
	\begin{proof}
		
		先考虑$\kappa$是特征零的情况,有典范映射$\mathbb{Z}\to\mathscr{O}_K\to\kappa$是单射.剩余域中非零元的提升都是单位,于是$\mathbb{Z}-\{0\}$的每个元在$\mathscr{O}_K$中是单位,于是$\mathbb{Q}\subset\mathscr{O}_K$.于是按照Zorn引理,可取$\mathscr{O}_K$中包含的极大子域$S$.设$\overline{S}$是$S$在$\kappa$中的像,那么它们自然是同构的(因为$S$每个元都是$\mathscr{O}_K$的单位),我们断言$\overline{S}$就是整个剩余域$\kappa$,这就得到了$\kappa\subset\mathscr{O}_K$.
		
		\qquad
		
		先说明$\kappa$是$\overline{S}$的代数扩张.否则可取$a\in\mathscr{O}_K$使得$\overline{a}$是$\overline{S}$上的超越元.导致$a$是$S$上的超越元.于是$S[a]\cong\overline{S}[\overline{a}]$,这说明$S[a]\cap m=\emptyset$,不然不会有这个同构.于是有$S(a)\subset\mathscr{O}_K$,这就和$S$的极大性矛盾.
		
		\qquad
		
		于是任取$a\in\kappa$,可取极小多项式$\overline{f}\in\overline{S}[x]$.按照$\kappa$特征零,说明$\alpha$是$\overline{f}$的单根.按照Hensel引理,说明$f(x)$在$\mathscr{O}_K$中有单根$x$是$a$的提升.于是$S(x)$是包含于$\mathscr{O}_K$的域.极大性得到$x\in S$,于是$\kappa=\overline{S}$.
	\end{proof}
\end{enumerate}

对于混合特征的情况,Trichm\"uller提升不再满足$r(a+b)\not=r(a)+r(b)$.处理这时候的结构要先引入Witt向量.设$K$是特征零的完备离散赋值域,剩余域$\kappa$的特征为$p$.设$\underline{X}$是一个未定元序列$(X_0,X_1,\cdots)$,类似定义$\underline{Y}=(Y_0,Y_1,\cdots)$.对每个自然数$n$,定义Witt多项式序列为$W_n(\underline{X})=X_0^{p^n}+pX_1^{p^{n-1}}+\cdots+p^nX_n\in\mathbb{Z}[\underline{X}_n]=\mathbb{Z}[X_0,\cdots,X_n]$.
\begin{enumerate}
	\item 引理.对任意的二元多项式$\Phi\in\mathbb{Z}[X,Y]$,总存在唯一的多项式序列$\Phi_n\in\mathbb{Z}[\underline{X}_n,\underline{Y}_n],n\ge0$,使得:
	$$\Phi(W_n(\underline{X}),W_n(\underline{Y}))=W_n(\Phi_0(\underline{X}_0,\underline{Y}_0),\Phi_1(\underline{X}_1,\underline{Y}_1),\cdots,\Phi_n(\underline{X}_n,\underline{Y}_n))$$
	\begin{proof}
		
		我们先在$\mathbb{Z}[1/p][\underline{X},\underline{Y}]$中解出这个多项式序列.首先必然有$\Phi_0(\underline{X}_0,\underline{Y}_0)=\Phi(X_0,Y_0)$.进而归纳构造:
		$$\Phi_n(\underline{X}_n,\underline{Y}_n)=\frac{1}{p^n}\left(\Phi\left(W_n(\underline{X}),W_n(\underline{Y})\right)-\sum_{i=0}^{n-1}p^i\Phi_i(\underline{X}_i,\underline{Y}_i)^{p^{n-i}}\right)$$
		
		这个多项式落在$\mathbb{Z}[1/p][\underline{X}_n,\underline{Y}_n]$中,并且是唯一存在的.于是归结为证明它的系数都在$\mathbb{Z}$中.为此我们来对$n$归纳,$n=0$没什么需要证的,假设$\Phi_i,0\le i\le n-1$都是$\mathbb{Z}$系数的,我们来证明$\Phi_n$也是$\mathbb{Z}$系数的,按照上述表达式这等价于证明:
		$$\Phi(W_n(\underline{X}),W_n(\underline{Y}))\equiv\Phi_0(\underline{X}_0,\underline{Y}_0)^{p^n}+p\Phi_1(\underline{X}_1,\underline{Y}_1)^{p^{n-1}}+\cdots+p^{n-1}\Phi_{n-1}(\underline{X}_{n-1},\underline{Y}_{n-1})^p\left(\mathrm{mod}p^n\right)$$
		
		但是在$\mathrm{mod}p^n$下有:
		\begin{align*}
			\mathrm{LHS}&\equiv\Phi(W_{n-1}(\underline{X}),W_{n-1}(\underline{Y}))\\&\equiv\Phi_0(\underline{X^p}_0,\underline{Y^p}_0)^{p^{n-1}}+p\Phi_1(\underline{X^p}_1,\underline{Y^p}_1)^{p^{n-2}}+\cdots+p^{n-1}\Phi_{n-1}(\underline{X^p}_{n-1},\underline{Y^p}_{n-1})
		\end{align*}
		
		按照归纳假设每个$\Phi_i(\underline{X}_i,\underline{Y}_i)\in\mathbb{Z}[\underline{X}_i,\underline{Y}_i],i\le n-1$,得到在$\mathrm{mod}p$下有$\Phi_i(\underline{X^p}_i,\underline{Y^p}_i)\equiv(\Phi_i(\underline{X}_i,\underline{Y}_i))^p$,于是在$\mathrm{mod}p^n$下就有$p^i\Phi_i(\underline{X^p}_i,\underline{Y^p}_i)^{p^{n-1-i}}\equiv p^i\Phi_i(\underline{X}_i,\underline{Y}_i)^{p^{n-i}}$.带入上述等式得证.
	\end{proof}
	\item 设$n\ge1$,设$A$是任意交换环,记$W_n(A)=A^n$.我们来借助上述引理定义$W_n(A)$上一个新的环结构.设二元多项式$\Phi=X+Y$和$\Phi=XY$在上述引理中对应的多项式序列分别为$S_i$和$T_i\in\mathbb{Z}[X_0,X_1,\cdots,X_i;Y_0,Y_1,\cdots,Y_i]$.任取$a=(a_0,a_1,\cdots,a_{n-1})$和$b=(b_0,b_1,\cdots,b_{n-1})\in W_n(A)$,记$s_i=S_i(a_0,a_1,\cdots,a_i;b_0,b_1,\cdots,b_i)$和$p_i=P_i(a_0,a_1,\cdots,a_i;b_0,b_1,\cdots,b_i)$,再记$\underline{S}=(s_0,s_1,\cdots)$和$\underline{T}=(t_0,t_1,\cdots)$,那么有:
	$$W_n(\underline{X})+W_n(\underline{Y})=W_n(\underline{S})$$
	$$W_n(\underline{X})W_n(\underline{Y})=W_n(\underline{T})$$
	
	我们断言如下定义的加法和乘法使得$W_n(A)$构成一个环.它称为$A$的长度$n$的Witt环,其中元素称为长度$n$的Witt向量.
	$$a+b=(s_0,s_1,\cdots,s_{n-1}),ab=(t_0,t_1,\cdots,t_{n-1})$$
	
	考虑满的环同态$W_{n+1}(A)\to W_n(A)$为$(a_0,a_1,\cdots,a_n)\mapsto(a_0,a_1,\cdots,a_{n-1})$.这些环同态构成了一个逆向系统,它的逆向极限$W(A)=\lim\limits_{\substack{\leftarrow\\n}}W_n(A)$称为$A$的Witt环,其中元素可以表示为$(a_0,a_1,\cdots)$,它称为$A$的(无限长度的)Witt向量.
	\begin{proof}
		
		构造映射$\rho:W_n(A)\to A^n$为$(a_0,a_1,\cdots,a_{n-1})\mapsto(w_0,w_1,\cdots,w_{n-1})$,其中$w_i=W_i(a)=a_0^{p^{n-i}}+pa_1^{p^{n-i-1}}+\cdots+p^ia_i$.那么有$w_i(a+b)=w_i(a)+w_i(b)$和$w_i(ab)=w_i(a)w_i(b)$.并且这个$\rho$是单射.
		
		\qquad
		
		对于Witt多项式,归纳可得每个$X_n\in\mathbb{Z}[p^{-1}][W_0,W_1,\cdots,W_n]$.所以如果$p$在$A$中可逆,那么上述映射$\rho$就是一个(满射,从而是)双射.此时我们定义的加法和乘法是$A^n$上加法和乘法经这个双射传递给$W_n(A)$的,所以此时$W_n(A)$的确是一个环.(但是$p$不可逆的时候这个$\rho$当然未必总是同构,我们只是借助这个特殊情况来证明对每个环$A$都有$W_n(A)$的上述加法和乘法使它成为一个环).
		
		\qquad
		
		假设$A$没有$p$挠元,即如果$pa=0,a\in A$,那么$a=0$.考虑嵌入$A\subseteq A[1/p]$,那么有$W_n(A)\subseteq W_n(A[1/p])$.上一段说明$W_n(A[1/p])$的确是环.现在任取$a,b\in W_n(A)$,在Witt多项式的引理中取二元多项式$X-Y$就得到$a-b\in W_n(A)$,于是$W_n(A)$是$W_n(A[1/p])$的子环.
		
		\qquad
		
		对于一般的交换环$A$,它可以表示为$R/I$,其中$R$没有$p$挠元,那么上一段说明$W_n(R)$的确是环,记$W_n(I)=\{(a_0,a_1,\cdots,a_{n-1})\mid a_i\in I\}$,它是$W_n(R)$的理想,并且有$W_n(R/I)=W_n(R)/W_n(I)$.这说明$W_n(A)$是环.
	\end{proof}
    \item 对$x,a_n\in\kappa$,$W(\kappa)$上的乘法有:
    $$(x,0,\cdots)(a_0,a_1,\cdots)=(xa_0,x^pa_1,x^{p^2}a_2,\cdots)$$
    \item $W_n$和$W$具有函子性,如果$h:A\to B$是环同态,那么它诱导了Witt环的环同态:
    $$W_n(h):W_n(A)\to W_n(B)$$
    $$(a_0,a_1,\cdots,a_{n-1})\mapsto(h(a_0),h(a_1),\cdots,h(a_{n-1}))$$
    \item $W_n$实际上被$\mathbb{Z}$上的一个仿射群概形表示,它是$\mathrm{Spec}B$,其中$B=\mathbb{Z}[X_0,X_1,\cdots,X_{n-1}]$.它的乘法由如下同态诱导:
    $$m^*:B\to B\otimes_{\mathbb{Z}}B\cong\mathbb{Z}[X_0,X_1,\cdots,X_{n-1};Y_0,Y_1,\cdots,Y_{n-1}]$$
    $$X_i\mapsto S_i(X_0,X_1,\cdots,X_i;Y_0,Y_1,\cdots,Y_i)$$
    \item 对于Witt环$W(\kappa)$,如果$\kappa$是完全域,有提升映射$(a_0,a_1,\cdots)\mapsto a_0$是乘性的,所以它就是Witt环上的Teichm\"uller提升.
    \item 如果$\kappa$是完全域,那么$W(\kappa)$是完备的,它的元素$(a_0,a_1,\cdots)$可以表示为级数$\sum_{n\ge0}f(a_n^{p^{-n}})p^n$,其中$f:\kappa\to W(\kappa)$是唯一的乘性提升映射.
\end{enumerate}

$p$环和严格$p$环的补充.
\begin{enumerate}
	\item 记$S=\mathbb{Z}[X_i^{p^{-\infty}},Y_i^{p^{-\infty}}]$,记$X_i,Y_i$是$\widehat{S}$的未定元,任取元素$x=\sum_{i\ge0}X_ip^i$和$y=\sum_{i\ge0}Y_ip^i$.如果$\ast$表示$+,\times,-$中任意一个运算,那么$x\ast y$也在$\widehat{S}$中,并且有$Q_i^{\ast}\in\mathbb{F}_p[X_i^{p^{-\infty}},Y_i^{p^{-\infty}}]$,使得$x\ast y=\sum_{i\ge0}f(Q_i^{\ast})p^i$.
	\item 设$A$是剩余环为$\kappa$的$p$环,设$f:\kappa\to A$是唯一的乘性提升映射.如果$\{\alpha_i\}$和$\{\beta_i\}$是$\kappa$中的两个序列,如果记$\gamma_i=Q_i^{\ast}(\alpha_0,\alpha_1,\cdots;\beta_0,\beta_1,\cdots)$,那么有:
	$$\sum_{i\ge0}f(\alpha_i)p^i\ast\sum_{i\ge0}f(\beta_i)p^i=\sum_{i\ge0}f(\gamma_i)p^i$$
	\begin{proof}
		
		考虑同态$\theta:S=\mathbb{Z}[X_i^{p^{-\infty}},Y_i^{p^{-\infty}}]$为$X_i\mapsto f(\alpha_i)$和$Y_i\mapsto f(\beta_i)$.它连续的延拓为$\widehat{S}\to A$,依旧记作$\theta$,它把$x=\sum_iX_ip^i$映射为$\alpha=\sum_if(\alpha_i)p^i$,把$y=\sum_iY_ip^i$映射为$\beta=\sum_if(\beta_i)p^i$.并且$\theta$在剩余环上诱导的映射为$\overline{\theta}:\mathbb{F}_p[X_i^{p^{-\infty}},Y_i^{p^{-\infty}}]$.于是有:
		\begin{align*}
			\sum_if(\alpha_i)p^i\ast\sum_if(\beta_i)p^i&=\theta(x)\ast\theta(y)=\theta(x\ast y)\\&=\sum_i\theta(f(Q_i^{\ast}))p^i=\sum_if(\overline{\theta}(Q_i^{\ast}))p^i
		\end{align*}
	
	    这里最后一个等式也即对于$p$环$A_1,A_2$和环同态$\theta:A_1\to A_2$有如下图表交换,其中$f_1$和$f_2$是我们解释过的$p$环上唯一的乘性提升映射.这个图表交换是因为,任取$x\in\kappa_1$,设$x$在$\kappa_1$中的$p^n$次根是$\overline{a_n}$,任取它在$A_1$中的提升$a_n$,那么$a_n^{p^n}$的极限是$f_1(x)$.但是$\overline{\theta}(\overline{a_n})$也是$\overline{\theta}(x)$的唯一$p^n$次根,$\theta(a_n)$是它的一个提升,所以$\theta(a_n^{p^n})$的极限,也就是$\theta(f_1(x))$,也即图表交换.
	    $$\xymatrix{\kappa_1\ar[rr]^{f_1}\ar[d]_{\overline{\theta}}&&A_1\ar[d]^{\theta}\\\kappa_2\ar[rr]_{f_2}&&A_2}$$
	\end{proof}
    \item 设$A$和$A'$是两个$p$环,剩余环分别记作$\kappa$和$\kappa'$,并且设$A$是严格$p$环.那么对每个同态$h:\kappa\to\kappa'$,存在唯一的同态$g:A\to A'$使得如下图表交换.特别的,这说明具有同构剩余环的两个严格$p$群是典范同构的.
    $$\xymatrix{\kappa\ar[d]^{f_A}\ar@/_1pc/[dd]_{\mathrm{id}}\ar[rr]^h&&\kappa'\ar[d]_{f_{A'}}\ar@/^1pc/[dd]^{\mathrm{id}}\\A\ar@{-->}[rr]^g\ar[d]&&A'\ar[d]\\\kappa\ar[rr]^{h}&&\kappa'}$$
    \begin{proof}
    	
    	因为$A$是严格$p$环,所以它的元素具有唯一表示$a=\sum_{n\ge0}f_A(\alpha_n)p^n$,那么要想使得图表交换,就要有:
    	$$g(a)=\sum_{n\ge0}g(f_A(\alpha_n))p^n=\sum_{n\ge0}f_{A'}(h(\alpha_n))p^n$$
    	
    	这就说明了唯一性.至于存在性,我们就按照这个等式定义$g$,只需说明它是环同态,以加法为例,设$b\in A$的唯一表示是$\sum_{n\ge0}f_A(\beta_n)p^n$,那么有:
    	\begin{align*}
    		g(a)+g(b)&=\sum_{n\ge0}g(f_A(\alpha_n))p^n+\sum_{n\ge0}g(f_A(\beta_n))p^n\\&=\sum_{n\ge0}f_{A'}(h(\alpha_n))p^n+\sum_{n\ge0}f_{A'}(h(\beta_n))p^n\\&=\sum_{n\ge0}f_{A'}\left(Q_n^+(h(\alpha_0),h(\alpha_1),\cdots;h(\beta_0),h(\beta_1),\cdots)\right)p^n\\&=\sum_{n\ge0}f_{A'}\left(h(Q_n^+(\alpha_0,\alpha_1,\cdots;\beta_0,\beta_1,\cdots))\right)p^n\\&=\sum_{n\ge0}g(f_A(Q_n^+(\alpha_0,\alpha_1,\cdots;\beta_0,\beta_1,\cdots)))\\&=g(a+b)
    	\end{align*}
    \end{proof}
    \item 引理.设$\varphi:\kappa\to\kappa'$是两个特征$p$完全环之间的满同态,如果存在严格$p$环$A$使得剩余环同构于$\kappa$,那么存在严格$p$环$A'$使得剩余环同构于$\kappa'$.
    \begin{proof}
    	
    	我们要构造的$A'$是$A$的商环.任取$A$中的两个元$a=\sum_{n\ge0}f_A(\alpha_n)p^n$和$b=\sum_{n\ge0}f_A(\beta_n)p^n$.如果对每个$n\ge0$有$\varphi(\alpha_n)=\varphi(\beta_n)$,就记$a\equiv b$.那么如果$a\equiv a'$和$b\equiv b'$,对$\ast=+,-,\times$都有$a\ast b\equiv a'\ast b'$.于是$A$关于这个等价关系的全体等价类构成一个环记作$A'$,或者等价的讲,全体和0等价的元构成了理想,把$A'$取为关于这个理想的商环.作为完备环的商环,$A'$仍然是完备的.对$A'$中任意的元$x$,任取它的代表元$a\in A$,那么$\xi_n=\varphi(\alpha_n)$满足$x=\sum_{n\ge0}f(\xi_n)p^n$,或者称$(\xi_0,\xi_1,\cdots)$是$x$的坐标,并且这不依赖代表元的选取.数乘$p$使得坐标为$(\xi_0,\xi_1,\cdots)$的元变为$(0,\xi_0,\xi_1,\cdots)$的元,所以$p$不是零因子,并且有$\cap_np^nA'=\{0\}$,这说明$A'$上的$p$-adic拓扑是Hausdorff的.最后考虑同态$A'\to\kappa$为把元素$x'$映射为它坐标$(\xi_0,\xi_1,\cdots)$中的$\xi_0$,这诱导了同构$A'/pA'\to\kappa'$,这说明$A'$是剩余环为$\kappa'$的严格$p$环.
    \end{proof}
    \item 对每个特征$p$的完全环$\kappa$,在同构意义下存在唯一的严格$p$环$H$使得剩余环就是$\kappa$,事实上$H$就是Witt环$W(\kappa)$.
    \begin{proof}
    	
    	唯一性我们解释过了.对于存在性,每个特征$p$的完全域都可以表示为环$\mathbb{F}_p[X_i^{p^{-\infty}}]$的商环.所以按照上述引理,问题归结为证明$\kappa=\mathbb{F}_p[X_i^{p^{-\infty}}]$的情况下的存在性.但是此时有$\widehat{\mathbb{Z}[X_i^{p^{-\infty}}]}$是剩余环是$\kappa$的严格$p$环.
    \end{proof}
    \item 例如由于$\mathbb{Z}_p$是剩余域为$\mathbb{F}_p$的严格$p$环,上一条说明$W(\mathbb{F}_p)=\mathbb{Z}_p$.也有$W_n(\mathbb{F}_p)=\mathbb{Z}/p^n\mathbb{Z}$.
    \item 如果把Witt环的函子$W$限制在完全环上,则它是一个完全忠实函子.换句话讲对完全环$\kappa$和$\kappa'$,有:
    $$\mathrm{Hom}_{\textbf{Rings}}(\kappa,\kappa')\cong\mathrm{Hom}_{\textbf{Rings}}(W(\kappa),W(\kappa'))$$
\end{enumerate}

下面处理混合特征的情况.设$(K,v)$是混合特征$(0,p)$的完备离散赋值域,赋值环记作$A$,称$e=v(p)$是$K$或者$A$的绝对分歧指数(absolute ramification index).如果$e=1$,也即$p$是$A$的素元,则称$K$或者$A$是绝对非分歧的(absolutely unramified).
\begin{enumerate}
	\item 如果$A$是严格$p$环,并且剩余环$A/pA$是域,当且仅当$A$是完备离散赋值环,并且是绝对非分歧的.
	\item 
	\begin{enumerate}
		\item 设$\kappa$是一个特征$p$的完全域,那么$W(K)$是同构意义下唯一的剩余域为$\kappa$的绝对非分歧特征零完备离散赋值域.
		\item 如果特征零的完备离散赋值环$A$的剩余域$\kappa$是特征$p>0$的完全域,设$e$是$A$的绝对分歧指数,那么存在唯一的同态$\psi:W(\kappa)\to A$使得如下图表交换,其中$A$是秩$e$的自由$W(\kappa)$模.
		$$\xymatrix{W(\kappa)\ar[rr]^{\psi}\ar[dr]&&A\ar[dl]\\&\kappa&}$$
		
		这里$A=W(\kappa)[\pi]$,其中$\pi$是$\kappa$上一个次数为$e$的爱森斯坦多项式的根.另外我们解释过添加爱森斯坦多项式的根是完全分歧扩张.
	\end{enumerate}
    \begin{proof}
    	
    	第一件事我们已经证明过了.第二件事,首先因为$A$是严格$p$环,$\psi$的存在性和唯一性已经解决了.它就是$(a_0,a_1,\cdots)\mapsto\sum_{n\ge0}p^n[a_n^{p^{-n}}]$,其中$[\bullet]$是Teucgn\"uller提升.因为$A$的特征是零,有$\psi$是单射.取$A$的素元$\pi$,那么每个$a\in A$可以唯一的表示为$a=\sum_{n\ge0}f(\alpha_n)\pi^n$,其中$\alpha_n\in\kappa$是坐标.把$\pi^e$替换为$\varepsilon p$,其中$\varepsilon$是单位.那么$a$可以唯一的表示为:
    	$$a=\sum_{i=0}^{\infty}\sum_{j=0}^{e-1}f(\alpha_{ij})\pi^jp^i,\alpha_{ij}\in\kappa$$
    	
    	于是$\{1,\pi,\cdots,\pi^{e-1}\}$是$A$作为$W(\kappa)$模的一组基.
    \end{proof}
\end{enumerate}

剩余域是非完全域的情况,Cohen环.
\begin{enumerate}
	\item 引理1.如果$(A,\pi A,\kappa)$是完备离散赋值环,如果$\kappa\subseteq K$是域扩张,那么存在离散赋值环$(B,\pi B,K)$包含了$A$.
	\item 引理2.如果$(A,m_A,\kappa_A)$是完备局部环,设$(R,m_R,\kappa_R)$是特征零的绝对非分歧的离散赋值环,这里绝对非分歧是指$m_R=pR$,那么对任意同态$h:\kappa_R\to\kappa_A$,都存在局部同态$g:R\to A$诱导了$h$.这个引理实际上推广了我们之前给出的$\kappa$是完全域的情况.但是和完全域的情况不同的是,满足结论的局部同态$g$未必是唯一的.例如取$\kappa=\mathbb{F}_p(x)$,取$A=\mathbb{Z}_p(x)$,那么对每个$\alpha\in p\mathbb{Z}_p$,同态$A\to A$,$x\mapsto x+\alpha$总诱导了$\kappa$上的恒等映射.
	\item 考虑$A=\mathbb{Z}_p$,任取特征$p>0$的域$\kappa$,引理1说明存在特征零的绝对非分歧的离散赋值环$R$,使得剩余域为$\kappa$.再按照引理2,这样的$R$在同构意义下是唯一的.它称为剩余域是$\kappa$的Cohen环,记作$\mathcal{C}(\kappa)$.下面主要刻画它的结构.
	\item $p$基.设$\kappa$是特征$p>0$的域,它的一个子集$B$称为$p$基,如果满足如下两个条件:
	\begin{enumerate}
		\item 对任意$r$个不同的元素$b_1,\cdots,b_r\in B$,有$[\kappa^p(b_1,\cdots,b_r):\kappa^p]=p^r$.
		\item $\kappa=\kappa^p(B)$.(也等价于对每个$n>0$有$\kappa=\kappa^{p^n}(B)$)
	\end{enumerate}

    如果$\kappa$是完全域,明显的$p$基是空集.如果$\kappa$是非完全域,那么至少存在一个非空集合满足第一条,按照Zorn引理满足第一条的子集存在极大元,那么这个极大元必然满足第二条.于是我们证明了$p$基总存在.
    \item $\mathcal{C}_{n+1}(\kappa)$的定义.首先如果$B$是特征$p>0$的域$\kappa$的一组$p$基,那么对每个$n>0$有$\kappa=\kappa^{p^n}(B)$,也有$B^{p^{-n}}=\{b^{p^{-n}}\mid b\in B\}$是$\kappa^{p^{-n}}$的$p$基(这两件事都可以从$a^p+b^p=(a+b)^p$得到).记$I_n=\{0,1,\cdots,p^n-1\}^B$,即全体以$B$为指标集的$\{0,1,\cdots,p^n-1\}$上的序列.记$T_n=\{\mathfrak{b}^{\alpha}=\prod_{b\in B}b^{\alpha_b}\mid\alpha=(\alpha_b)\in I_n\}$.那么按照$\kappa=\kappa^{p^n}(B)$,说明$T_n$是$\kappa$作为$\kappa^{p^n}$线性空间的有一组基.更一般的,$T_n^{p^m}$是$\kappa^{p^m}$作为$\kappa^{p^{m+n}}$线性空间的一组基.考虑$W_{n+1}(\kappa)$的由$W_{n+1}(\kappa^{p^n})$和全体$[b],b\in B$生成的子环,其中$[\bullet]$是Teichm\"uller提升,也即$[x]=(x,0,\cdots,0)\in W_{n+1}(\kappa)$,这个子环记作$\mathcal{C}_{n+1}(\kappa)$.
    \item $\mathcal{C}_{n+1}(\kappa)$是局部环,它的极大理想是被$p$生成的主理想,它的剩余域是$\kappa$.对每个$r\le n$,数乘$p^r$诱导了同构$\mathcal{C}_{n+1}(\kappa)/p\mathcal{C}_{n+1}(\kappa)\cong p^r\mathcal{C}_{n+1}(\kappa)/p^{r+1}\mathcal{C}_{n+1}(\kappa)$,并且有$p^{n+1}\mathcal{C}_{n+1}(\kappa)=0$.
    \begin{proof}
    	
    	记平移同态$V:W_{n+1}(\kappa)\to W_{n+1}(\kappa)$为$(x_0,x_1,\cdots,x_n)\mapsto(0,x_0,x_1,\cdots,x_{n-1})$.那么$W_{n+1}(\kappa)$中的元素就可以表示为:
    	$$x=(x_0,x_1,\cdots,x_n)=[x_0]+V([x_1])+\cdots+V^n([x_n])$$
    	
    	按照$[a][b]=[ab]$和上述等式,说明$\mathcal{C}_{n+1}(\kappa)$是$W_{n+1}(\kappa)$的被$S=\{[\mathfrak{b}^{\alpha}]V^r(x)\mid\mathfrak{b}^{\alpha}\in T_n,x\in\kappa^{p^n},0\le r\le n\}$.另外我们解释过有$[x](a_0,a_1,\cdots)=(xa_0,x^pa_1,x^{p^2}a_2,\cdots)$,就有$[y]V^r(x)=V^r([y^{p^r}]x)$.于是$S=\{V^r([(\mathfrak{b}^{\alpha})^{p^r}]x)\mid\mathfrak{b}^{\alpha}\in T_{n-r},x\in\kappa^{p^n},0\le r\le n\}$.
    	
    	\qquad
    	
    	记Frobenius映射$\varphi:\kappa\to\kappa$为$x\mapsto x^p$.它诱导了$W(\kappa)\to W(\kappa)$的同态$(a_0,a_1,\cdots)\mapsto(a_0^p,a_1^p,\cdots)$,依旧称为Frobenius同态,依旧记作$\varphi$.那么无论$\kappa$是完全域还是非完全域,都有$V\circ\varphi=\varphi\circ V=p$,也即数乘$p$的映射,这是因为:
    	$$\varphi\circ V(x_0,x_1,\cdots)=\varphi(0,x_0,x_1,\cdots)=(0,x_0^p,x_1^p,\cdots)=\sum_{n\ge0}f(x_n^{p^{-n}})p^{n+1}=p(x_0,x_1,\cdots)$$
    	
    	这说明有$V^r(\varphi^r([x]))=p^r[x]$.记$\mathscr{U}_r=\mathcal{C}_{n+1}(\kappa)\cap V^r(W_{n+1}(\kappa))$.它是$\mathcal{C}_{n+1}(\kappa)$的理想.它是由$\{V^m([(\mathfrak{b}^{\alpha})^{p^m}]x)\mid\mathfrak{b}^{\alpha}\in T_{n-m},x\in\kappa^{p^n},m\ge r\}$生成的加法子群.于是有同构$\mathcal{C}_{n+1}(\kappa)/\mathscr{U}_1\cong\kappa$.并且数乘$p^r$诱导了同构$\mathcal{C}_{n+1}(\kappa)/\mathscr{U}_1\cong\mathscr{U}_r/\mathscr{U}_{r+1},r\le n$.于是$\mathscr{U}_n$就是被$p^n$生成的主理想,归纳得到每个$\mathscr{U}_r=p^r\mathcal{C}_{n+1}(\kappa)$.最后我们说明$\mathscr{U}_1$是唯一极大理想:任取$x\in\mathcal{C}_{n+1}(\kappa)-\mathscr{U}_1$,设$\overline{x}^{-1}\in\mathcal{C}_{n+1}(\kappa)/\mathscr{U}_1$的一个提升为$y$,那么有$xy=1-z$,其中$z\in\mathscr{U}_1$.按照$p^n$零化了$\mathcal{C}_{n+1}(\kappa)$,于是$z^{n+1}=1$,于是有$xy(1+z+\cdots+z^n)=1$,这说明$x$是可逆的,于是$\mathscr{U}_1=p\mathcal{C}_{n+1}(\kappa)$是唯一的极大理想.
    \end{proof}
    \item 考虑典范投影映射$\mathrm{pr}:W_{n+1}(\kappa)\to W_n(\kappa)$,它诱导了满同态$\pi:\mathcal{C}_{n+1}(\kappa)\to\mathcal{C}_n(\kappa)$.这些同态构成逆向系统,我们断言$\lim\limits_{\leftarrow}\mathcal{C}_n(\kappa)$就是Cohen环$\mathcal{C}(\kappa)$.
    \begin{proof}
    	
    	只需证明第一句断言.首先$\mathcal{C}_{n+1}(\kappa)$是$W_{n+1}(\kappa)$的子环,并且它在$\mathrm{pr}$下的像是由$W_n(\kappa^{p^n})$和$[b],b\in B$生成的子环.但是$\mathcal{C}_n(\kappa)$是被$W_n(\kappa^{p^{n-1}})$和$[b],b\in B$生成的子环,所以这个限制映射$\pi$是定义良好的.下面要说明它是满射.
    	
    	\qquad
    	
    	对每个$n\ge1$,考虑$W_n(\kappa)$上的滤过$W_n(\kappa)\supseteq V(W_n(\kappa))\supseteq\cdots\supseteq V^n(W_n(\kappa))=0$,它诱导了$\mathcal{C}_n(\kappa)$上的滤过$\mathcal{C}_n(\kappa)\supseteq p\mathcal{C}_n(\kappa)\supseteq\cdots\supseteq p^n\mathcal{C}_n(\kappa)=0$(因为我们证明了$\mathcal{C}_n(\kappa)\cap V^r(W_n(\kappa))=p^r\mathcal{C}_n(\kappa)$).为了证明$\pi$是满射,只需证明它诱导的分次环之间的同态$\mathrm{gr}\pi$是满射.但是对$r<n$考虑如下交换图表,其中$j$就是$\kappa^{p^n}\subseteq\kappa$的包含映射,所以这里$\mathrm{gr}\pi$是满射:
    	$$\xymatrix{p^r\mathcal{C}_{n+1}(\kappa)/p^{r+1}\mathcal{C}_{n+1}(\kappa)\cong\kappa^{p^r}\ar[rr]^{\mathrm{gr}\pi}\ar[d]_j&&p^r\mathcal{C}_n(\kappa)/p^{r+1}\mathcal{C}_n(\kappa)\cong\kappa^{p^r}\ar[d]_j\\V^rW_{n+1}(\kappa)/V^{r+1}W_{n+1}(\kappa)\cong\kappa\ar[rr]^{\mathrm{gr.pr}=\mathrm{id}}&&V^rW_n(\kappa)/V^{r+1}W_n(\kappa)\cong\kappa}$$
    	
    	而对于$r=n$有终端$p^n\mathcal{C}_n(\kappa)=0$,于是$\mathrm{gr}\pi$是满射.
    \end{proof}
    \item 按照构造,$\mathcal{C}(\kappa)$是$W(\kappa)$的子环.如果记$\kappa_0=\cap_{n\ge0}\kappa^{p^n}$,它是$\kappa$包含的最大的完全域,那么有$\mathcal{C}(\kappa)$包含了$W(\kappa_0)$.另外因为$\mathcal{C}(\kappa)$包含了全部$[b],b\in B$,导致对每个正整数$n$,对每个$\alpha\in I_n$,有$[\mathfrak{b}^{\alpha}],[\mathfrak{b}^{-\alpha}]\in\mathcal{C}(\kappa)$.
\end{enumerate}
\newpage
\subsection{共轭差积和判别式}

格(Lattice).我们总设$A$是戴德金整环,$K$是商域,$V$是有限维$K$线性空间,$V$上的一个格是指$A$模$V$的一个有限生成子模$X$,使得它在$K$上张成整个$V$.
\begin{enumerate}
	\item 如果$A$是PID(戴德金整环情况下这也等价于是DVR),那么$X$是$V$上的格等价于讲$X$是秩为$\dim_KV$的自由模.按照戴德金整环在非零素理想上的局部化是DVR,我们总可以对$A$做局部化归结为这一情况.
	\item 如果$X_2\subseteq X_1$是$V$上两个格,我们断言$X_1/X_2$是有限长度模.一般的对于有限长度$A$模$M$,可取合成链$0=M_0\subsetneqq M_1\subsetneqq\cdots\subsetneqq M_n=M$,使得每个$M_{i}/M_{i-1}$是单模,也即$M_i/M_{i-1}\cong A/\mathfrak{p}_i,1\le i\le n$,其中$\mathfrak{p}_i$是$A$的极大理想.我们定义过$\chi_A(M)=\prod_i\mathfrak{p}_i$.
	\begin{proof}
		
		任取$x\in X_1$,它可以表示为$X_2$中有限个元的$K$线性组合,因为$K=\mathrm{Frac}(A)$,所以存在$c\in A$使得$cx\in X_2$.现在设$X_1$被有限个元$\{a_1,a_2,\cdots,a_r\}$在$A$上生成,那么存在$c_i\in A$使得$c_ia_i\in X_2$,所以如果取$c=\prod_ic_i$,那么$cX_1\subseteq X_2$.所以$X_1/X_2$是$A/cA$上的有限生成模.但是按照$A$是戴德金整环,有$A/cA$是零维诺特环,也即阿廷的,但是阿廷环上的有限生成模等价于有限长度的,于是$X_1/X_2$是有限长度$A/cA$模,于是它也是有限长度$A$模.
	\end{proof}
    \item 如果$X_1,X_2$是$V$上的两个格,对任意的格$X_3\subseteq X_1\cap X_2$,有$A$的分式理想$\chi(X_1/X_3)\chi(X_2/X_3)^{-1}$是不变的.我们把这个不依赖$X_3$的选取的分式理想记作$\chi(X_1,X_2)$.
    \begin{proof}
    	
    	记$A$模$X_4=X_1\cap X_2$,那么有短正合列:
    	$$\xymatrix{0\ar[r]&X_4/X_3\ar[r]&X_1/X_3\ar[r]&X_1/X_4\ar[r]&0}$$
    	
    	所以按照$\chi$对短正合列的加性,就有$\chi(X_1/X_3)=\chi(X_4/X_3)\chi(X_1/X_4)$.类似的有$\chi(X_2/X_3)=\chi(X_4/X_3)\chi(X_2/X_4)$.这就说明$\chi(X_1/X_3)\chi(X_2/X_3)^{-1}=\chi(X_1/X_4)\chi(X_2/X_4)^{-1}$.也即和$X_3\subseteq X_1\cap X_2$的选取无关.
    \end{proof}
    \item 对于$\chi(\bullet,\bullet)$有如下等式:
    \begin{enumerate}
    	\item $\chi(X_1,X_2)\chi(X_2,X_3)\chi(X_3,X_1)=1$.
    	\item $\chi(X_1,X_2)\chi(X_2,X_1)=1$.
    	\item 如果$X_2\subseteq X_1$,那么$\chi(X_1,X_2)=\chi(X_1/X_2)$.
    \end{enumerate}
    \begin{proof}
    	
    	第一件事只要取$X=X_1\cap X_2\cap X_3$,那么$\chi(X_1,X_2)=\chi(X_1/X)\chi(X_2/X)^{-1}$等,相乘就是1.第二件事也是取$X=X_1\cap X_2$即可.第三件事平凡.
    \end{proof}
    \item 如果$X$是$V$上的格,如果$u$是$V$上的$K$自同态,那么$\chi(X,uX)$是$\det(u)$生成的主理想.
    \begin{proof}
    	
    	通过局部化,我们不妨设$A$是DVR,设$X=A^n$,适当对$u$乘以一个$A$中的元,可以使得$u$是$A^n\to A^n$的模同态,那么$\mathrm{coker}u=X/uX$,但是我们证明过此时如果$\det(u)\not=0$,那么$\det(u)A=\chi_A(\mathrm{coker}u)$.
    \end{proof}
    \item $\chi(X,X')$的等价描述.设$\dim_KV=n$,记$W=\bigwedge^nV$,它是$K$上的一维线性空间.如果$X$是$V$的格,则$X_W=\bigwedge^nX$是$W$的格.但是$W$是$K$上一维线性空间,所以$X$可视为$K$上的分式理想(即$A$模$K$的有限生成子模).如果$X'$也是$V$的格,那么存在唯一的分式理想$\mathfrak{a}$使得$X'=\mathfrak{a}X$,把这里$X$和$X'$替换为对应的分式理想,得到$\mathfrak{a}=\chi(X,X')$.
\end{enumerate}

格上双线性型的判别式.设$\mathrm{T}(\bullet,\bullet)$是$n$维$K$线性空间$V$上的非退化双线性型(非退化指的是如果$x\in V$使得对任意$y\in V$有$\mathrm{T}(x,y)=0$,则$x=0$),那么$\mathrm{T}$可以按照如下公式延拓为$W=\bigwedge^nV$上的非退化双线性型.它诱导了同构$W\otimes_KW\cong K$.那么$A$子模$X_W\otimes_AX_W$在$\mathrm{T}$下的像就是$K$的一个分式理想,它称为格$X$关于非退化双线性型$\mathrm{T}$的判别式,记作$d_{X,\mathrm{T}}$或者不引起歧义时记作$d_X$.
\begin{enumerate}
	\item 如果$X$是自由$A$模,记$S=\{e_1,\cdots,e_n\}$是一组基,那么$d_{X,\mathrm{T}}$是被$\det(\mathrm{T}(e_i,e_j))$生成的主理想.
	\begin{proof}
		
		因为此时$X_w$是以$\{e=e_1\wedge\cdots\wedge e_n\}$为基的自由模,并且有$\mathrm{T}(e,e)=\det(\mathrm{T}(e_i,e_j))$,所以$d_{X,\mathrm{T}}$,作为$X_W\otimes_AX_W$在$\mathrm{T}$下的像,是$\det(\mathrm{T}(e_i,e_j))$生成的主理想.
	\end{proof}
    \item 设$X$是$V$上的格,设$X_{\mathrm{T}}^*=\{y\in V\mid\mathrm{T}(x,y)\in A,\forall x\in V\}$称为格的对偶,那么这是$V$上的格,并且有$d_{X,\mathrm{T}}=\chi(X,X_{\mathrm{T}}^*)$.
    \begin{proof}
    	
    	局部化后可归结为$A$是DVR的情况,此时$X$是自由$A$模.$\{e_1,\cdots,e_n\}$存在对偶基$\{e_1^*,\cdots,e_n^*\}$,此即一组$e_i^*\in V$使得$\mathrm{T}(e_i,e_j^*)=\delta_{ij}$.考虑$V$上的自同态$u$把$e_i\mapsto e_i^*$,记$e_i=\sum_jx_{ij}e_j^*$,也即$x_{ij}=\mathrm{T}(e_j,e_i)$.那么我们解释过$\chi(X_{\mathrm{T}}^*,X)=(\det(u))$.但是这里$\det(u)=\det(x_{ij})=d_{X,\mathrm{T}}$.
    \end{proof}
    \item 设$X,X'$是$V$的两个格,那么有$d_{X',\mathrm{T}}=d_{X,\mathrm{T}}\cdot\chi(X,X')^2$.特别的,如果格$X'\subseteq X$,那么有$d_{X',\mathrm{T}}=d_{X,\mathrm{T}}\cdot\mathfrak{a}^2$,其中$\mathfrak{a}$是整理想($A$的常义理想,区分于分式理想).并且有$\mathfrak{a}=A$当且仅当$X'=X$.
    \begin{proof}
    	
    	设$\mathfrak{a}=\chi(X,X')$,我们解释过$X_W'=\mathfrak{a}X_W$,于是$d_{X',\mathrm{T}}=\mathrm{T}(X_W'\otimes_AX_W')=\mathfrak{a}^2\mathrm{T}(X_W\otimes_AX_W)=\mathfrak{a}^2d_{X,\mathrm{T}}$.
    \end{proof}
\end{enumerate}

共轭差积(different)和判别式.设$A$是戴德金整环,记商域是$K$,取有限可分扩张$K\subseteq L$,设$A$在$L$中的整闭包为$B$,那么$B$是$L$的格,取$L$上的双线性型$\mathrm{T}(\bullet,\bullet)$是迹映射$\mathrm{Tr}(xy)$,可分条件保证了它是非退化的.
\begin{itemize}
	\item 格$B$关于$\mathrm{T}$的判别式称为扩张$A\subseteq B$的判别式,记作$d_{B/A}$,也会记作$d_{L/K}$.换句话讲它是全体$\det\mathrm{T}(x_iy_j),x_i,y_j\in B$生成的$A$模$L$的子模.按照$A$是戴德金整环可以验证它也是全体$d_{B/A}(b_1,\cdots,b_n)=\det\mathrm{T}(b_ib_j),b_i\in B$生成的子模,但是这些$d_{B/A}(b_1,\cdots,b_n)\in A$,所以这里$d_{B/A}$是$A$的整理想.
	\item 格$B$的对偶记作$\mathfrak{C}_{B/A}=B^*=\{x\in L\mid\mathrm{T}(xB)\subseteq A\}$,这是戴德金整环$B$的一个分式理想,有时称为余共轭差积或者逆共轭差积,它(作为分式理想)的逆$\mathfrak{D}_{B/A}=\mathfrak{C}_{B/A}^{-1}$称为扩张$A\subseteq B$的共轭差积,也会记作$\mathfrak{D}_{L/K}$.按照$B\subset\mathfrak{C}_{B/A}$,得到$\mathfrak{D}_{B/A}\subseteq B$,于是共轭差积实际上是$B$的一个整理想.
\end{itemize}

共轭差积的性质:
\begin{enumerate}
	\item 设$S\subseteq A$是乘性闭子集,那么有$\mathfrak{D}_{S^{-1}B/S^{-1}A}=S^{-1}\left(\mathfrak{D}_{B/A}\right)$.
	\begin{proof}
		
		容易验证对$A$的整理想$I$有$(S^{-1}I)^{-1}=S^{-1}I^{-1}$(作为分式理想的逆).所以归结为证明余共轭差积满足这个公式,也即证明$(S^{-1}B)^*=S^{-1}B^*$.一方面如果$x_1=s_1^{-1}y$,$x_2=s_2^{-1}b$,其中$s_1,s_2\in S$,$y\in B^*$和$b\in B$,那么$\mathrm{T}(x_1x_2)=s_1^{-1}s_2^{-1}\mathrm{T}(yb)\in S^{-1}A$,这说明$S^{-1}B^*\subseteq(S^{-1}B)^*$.另一方面设$\{b_i\}$是$B$作为$A$模的有限生成元集,设$x\in(S^{-1}B)^*$,那么每个$\mathrm{T}(xb_i)\in S^{-1}A$,记$\mathrm{T}(xb_i)=s^{-1}a_i$,其中$a_i\in A$,于是$sx\in B^*$,这得到$(S^{-1}B)^*\subseteq S^{-1}B^*$.
	\end{proof}
	\item 传递性.设有戴德金整环的扩张$A\subseteq B\subseteq C$,那么$\mathfrak{D}_{C/A}=\mathfrak{D}_{C/B}\mathfrak{D}_{B/A}$.
	\begin{proof}
		
		取逆等价于证明$\mathfrak{C}_{C/A}=\mathfrak{C}_{C/B}\mathfrak{C}_{B/A}$.一方面$\mathrm{T}_{C/A}(\mathfrak{C}_{C/B}\mathfrak{C}_{B/A}C)=\mathrm{T}_{B/A}\left(\mathrm{T}_{C/B}\right)$.这里$\mathrm{T}_{C/B}(\mathfrak{C}_{C/B}C)\subseteq B$,而$\mathrm{T}_{B/A}(\mathfrak{C}_{B/A}B)\subseteq A$,于是得到$\mathfrak{C}_{C/B}\mathfrak{C}_{B/A}\subset\mathfrak{C}_{C/A}$.
		
		另一方面按照$BC=C$,得到$\mathrm{T}_{B/A}(\mathrm{T}_{C/B}(\mathfrak{C}_{C/A}C)B)=\mathrm{T}_{C/A}(\mathfrak{C}_{C/A}C)\subseteq A$,于是这个理想$\mathrm{T}_{C/B}(\mathfrak{C}_{C/A}C)\subset\mathfrak{C}_{B/A}$.于是$\mathrm{T}_{C/B}(\mathfrak{D}_{B/A}\mathfrak{C}_{C/A}C)\subseteq B$.于是得到$\mathfrak{D}_{B/A}\mathfrak{C}_{C/A}\subset\mathfrak{C}_{C/B}$,得到$\mathfrak{C}_{C/A}\subset\mathfrak{C}_{C/B}\mathfrak{C}_{B/A}$.
	\end{proof}
	\item 设$\mathfrak{p}$是$A$的素理想,设$\mathfrak{P}\subseteq B$是它的一个提升素理想,设$A$和$B$在素理想$\mathfrak{p}$和$\mathfrak{P}$处完备化(即adic完备化,或者赋值语言下是关于相应离散赋值的完备化)分别为$\widehat{A}$和$\widehat{B}$,那么$\mathfrak{D}_{B/A}\widehat{B}=\mathfrak{D}_{\widehat{B}/\widehat{A}}$.
	\begin{proof}
		
		不妨设$A$是DVR,否则可按照我们证明的共轭差积与分式化可交换,对$A$和$B$做关于$S=A-\mathfrak{p}$的分式化.我们来证明$\mathfrak{C}_{B/A}$在$\mathfrak{C}_{\widehat{B}/\widehat{A}}$中稠密,这就得到$\mathfrak{C}_{B/A}\widehat{B}=\mathfrak{C}_{\widehat{B}/\widehat{A}}$,进而得到$\mathfrak{D}_{B/A}\widehat{B}=\mathfrak{D}_{\widehat{B}/\widehat{A}}$.
		
		\qquad
		
		此时有$\mathrm{T}_{L/K}=\sum_{\mathfrak{P}\mid\mathfrak{p}}\mathrm{T}_{L_{\mathfrak{P}}/K_{\mathfrak{p}}}$.取$y\in\widehat{B}$,按照赋值的逼近定理,可取$\eta\in L$使得$|\eta-y|_{\mathfrak{P}}\to0$和$|\eta|_{\mathfrak{P}'}\to0$,$\mathfrak{P}'\not=\mathfrak{P}$.于是如果$x\in\mathfrak{C}_{B/A}$得到$\mathrm{T}(x\eta)=\mathrm{T}_{L_{\mathfrak{P}}/K_{\mathfrak{p}}}(x\eta)+\sum_{\mathfrak{P}'\not=\mathfrak{P}}\mathrm{T}_{L_{\mathfrak{P}'}/K_{\mathfrak{p}}}(x\eta)\in\widehat{A}$.现在令$\eta$取极限,得到这个式子趋于$T_{L_{\mathfrak{P}}/K_{\mathfrak{p}}}(xy)$,按照完备性,这个极限掉在$\widehat{A}$中,于是$\mathfrak{C}_{B/A}\subset\mathfrak{C}_{\widehat{B}/\widehat{A}}$.
		
		\qquad
		
		反过来如果$x\in\mathfrak{C}_{\widehat{B}/\widehat{A}}$,可取$\xi\in L$在赋值$v_{\mathfrak{P}}$下足够接近$x$,在其它的赋值$v_{\mathfrak{P}'},\mathfrak{P}'\not=\mathfrak{P}$下接近0.我们断言有$\xi\in\mathfrak{C}_{B/A}$,任取$y\in B$,那么有:
		$$\mathrm{T}_{L/K}(\xi y)=\mathrm{T}_{L_{\mathfrak{P}}/K_{\mathfrak{p}}}((\xi-x)y)+\mathrm{T}_{L_{\mathfrak{P}}/K_{\mathfrak{p}}}(xy)+\sum_{\mathfrak{P}'\not=\mathfrak{P}}\mathrm{T}_{L_{\mathfrak{P}'}/K_{\mathfrak{p}}}(\xi y)$$
		
		这里由于$(\xi-x)y\in L$在$v_{\mathfrak{P}}$赋值下足够小(绝对值意义下足够小,加性赋值意义下足够大),$\xi y\in L$在$v_{\mathfrak{P}'}$的赋值下足够小,就导致这里第一项和最后和式每一项都在$\widehat{A}$中.第二项在$\widehat{A}$是因为$x\in\mathfrak{C}_{\widehat{B}/\widehat{A}}$.另外这些取迹的元素都是$L$中的元,所以取迹就掉在$K$里,最后按照$\widehat{A}\cap K=A$,就得到这三项都落在$A$中,于是$\xi\in\mathfrak{C}_{B/A}$.得证.
	\end{proof}
    \item 任取$\alpha\in B$,记它在$A$上的极小多项式为$f(x)$,如果$L=K(\alpha)$,记$\delta_{B/A}(\alpha)=f'(\alpha)$,否则记$\delta_{B/A}(\alpha)=0$.那么如果$B=A[a]$,就有$\mathfrak{D}_{B/A}=(\delta(a))$.这里$f'(\alpha)=\prod_{\alpha_i\not=\alpha}(\alpha-\alpha_i)$是$\alpha$和自己共轭元的差的积,所以我们称different为共轭差积.
    \begin{proof}
    	
    	设$f(X)=a_0+a_1X+\cdots+a_nX^n$,设$f(X)=(X-\alpha)g(X)$,其中$g(X)=b_0+b_1X+\cdots+b_{n-1}X^{n-1}$.按照$\{1,\alpha,\cdots,\alpha^{n-1}\}$是$B$在$A$上的一组基,我们断言它关于迹二次型$\mathrm{T}(xy)$的对偶基是$\{\frac{b_0}{f'(\alpha)},\frac{b_1}{f'(\alpha)},\cdots,\frac{b_{n-1}}{f'(\alpha)}\}$.事实上按照插值公式得到$\sum_{1\le i\le n}\frac{f(X)}{X-\alpha_i}\frac{\alpha_i^r}{f'(\alpha_i)}=X^r$,其中$\{\alpha_1,\alpha_2,\cdots,\alpha_n\}$是$f(X)$的$n$个根.这个等式可写作$\mathrm{T}(\frac{f(X)}{X-\alpha}\frac{\alpha^r}{f'(\alpha)})=X^r$.于是考虑等式两边的次数,得到$\mathrm{T}(\alpha^i\frac{b_j}{f'(\alpha)})=\delta_{ij}$.
    	
    	于是按照对偶基可得$\mathfrak{C}_{B/A}=\frac{1}{f'(\alpha)}\left(Ab_0+Ab_1+\cdots+Ab_{n-1}\right)$.再考虑$b_i$和$a_j$的关系式:$b_{n-1}=1$,$b_{n-2}-\alpha b_{n-1}=a_{n-1}$等,得到$Ab_0+Ab_1+\cdots+Ab_{n-1}=A[\alpha]=B$.于是$\mathfrak{C}_{B/A}=f'(\alpha)^{-1}B$,于是$\mathfrak{D}_{B/A}=(f'(\alpha))$.
    \end{proof}
    \item 共轭差积$\mathfrak{D}_{L/K}$恰好是由全体元素的共轭差积$\delta_{L/K}(\alpha),\alpha\in B$生成的理想.
    \begin{proof}
    	
    	回顾一下导子(conductor)的概念,设$\alpha\in B$使得$L=K(\alpha)$,导子$\mathfrak{f}_{B[\alpha]}$定义为$\{x\in L\mid xB\subseteq A[\alpha]\}$.记$b=f'(\alpha)$,记$x\in L$,那么$x\in\mathfrak{f}$等价于讲$xB\subseteq A[\alpha]$,等价于讲$b^{-1}xB\subseteq b^{-1}A[\alpha]$.按照上一条的结论,这等价于讲$\mathrm{T}(b^{-1}xB)\subseteq A$,等价于讲$x\in b\mathfrak{C}_{B/A}$.于是我们证明了$(f'(\alpha))=\mathfrak{f}\mathfrak{D}_{B/A}$.于是特别的有$f'(\alpha)\in\mathfrak{D}_{B/A}$.
    	
    	于是我们证明了所有元素的共轭差积落在扩张的共轭差积中,并且后者甚至整除每个元素的共轭差积.按照戴德金整环上理想的唯一素理想分解,为了说明前者生成的理想恰好是后者,我们只需证明对每个$B$的素理想$q$,都存在某个$\alpha\in B$满足$L=K(\alpha)$,使得$v_q(\mathfrak{D}_{L/K})=v_q(f'(\alpha))$.
    	
    	现在设完备化扩张$K_p\subseteq L_q$对应的赋值环是$A'\subseteq B'$.我们证明过$v_q(\mathfrak{D}_{B/A})=v_q(\mathfrak{D}_{B'/A'})$.我们还证明过此时$B'$在$A'$上被单个素元生成,即$B'=A'[a]$.于是$v_q(\mathfrak{D}_{B'/A'})=v_q(\delta(a))$.在之前的证明中可以看出如果$|b-a|$充分小,依旧有$B'=A'[b]$.于是归结为证明$b\in L$使得$|b-a|$充分小,并且$L=K(b)$,那么$v_q(\delta_{B'/A'}(b))=v_q(\delta_{B/A}(b))$.此时把$\alpha$取为$b$即可.
    	
    	取嵌入$\sigma_i:L\to\overline{K_p},1\le i\le r$使得它们对应全部$r$个不同的素位$q_1,q_2,\cdots,q_r$.现在取$\alpha$使得$|\tau(a)-\alpha|=1$对任意$\tau\in G(\overline{K}_p/K_p)$成立.这是可以做到的,因为如果某个$\tau(a)$赋值恰好是1,那直接取$\alpha=0$即可;如果所有$\tau(a)$赋值都不取1,那么它的赋值大于1,于是取$\alpha=1$得到赋值恰好是1,按照不同的$\tau$导致它们是互相共轭的,于是它们的赋值相同都是1.
    	
    	按照中国剩余定理,可取$b$使得$|b-a|$和$|\sigma_i(b)=\alpha|$都是充分小的.另外这里我们可以不妨设$L=K(b)$.因为若否,记$L=K(\gamma)$,这个单扩张$K\subseteq K(b)$只有有限个中间域,于是存在不同的两个正整数$m$和$n$使得$K(b+\Pi^n\gamma)=K(b+\Pi^m\gamma)$,导致这个中间域包含了$\gamma$,于是$K(b+\Pi^n\gamma)=L$,于是把$b$代替为$b+\Pi^n\gamma$即可.
    	
    	最后$\delta_{L_q/K_p}(b)=\prod_{1\not=\tau\in\mathrm{Hom}_{K_p}(L_q,\overline{K}_p)}(b-\tau(b))$,并且$\delta_{B/A}(b)=\prod_{1\not=\sigma\in\mathrm{Hom}_K(L,\overline{K})}(b-\sigma(b))$.这里$K\to L\to\overline{K}$的嵌入可以提升为嵌入链$K_p\to L_q\to\overline{K}_p$,于是这里的$\sigma$都可以分解为$\tau_{ij}\circ\sigma_i$,其中$\sigma$是$L\to\overline{K}_p$提升得到的$L_{q_i}\to\overline{K}_p$,而$\tau_{ij}$是$p$扩张到$q$的分解群中的元.于是$\delta_{B/A}(b)$可表示为$\prod_{\tau\not=1}(b-\tau b)\prod_{2\le i\le r}\prod_j(b-\tau_{ij}\sigma_i(b))$.于是只需验证后面这个乘积式子为1.我们有$|b-\tau_{ij}\sigma_i(b)|=|\tau_{ij}^{-1}(b)-\sigma_i(b)|=|\tau_{ij}^{-1}b-\alpha+\alpha-\sigma_i(b)|$.其中$\tau_{ij}^{-1}(b)-\alpha$赋值是1,而$\alpha-\sigma_i(b)$赋值充分小,于是整个的赋值为1.
    \end{proof}
    \item 设$A$是戴德金整环,$K$是商域,设$K\subseteq L$是有限可分扩张,设$A$在$L$中的整闭包为$B$.设扩张的共轭差积为$\mathfrak{D}_{L/K}$,设$q$是$B$的素理想,那么$q$在$A$上分歧当且仅当$q\mid\mathfrak{D}_{L/K}$.设$s=v_q(\mathfrak{D}_{L/K})$,设$e=e(q/p)$.如果$q$是温分歧的,那么$s=e-1$;如果$q$是野分歧的,那么$e\le s\le e-1+v_q(e)$.
    \begin{proof}
    	
    	不妨设$A$是完备离散赋值环,否则直接做$p$处的局部化和完备化.我们证明过此时$B=A[a]$,并且此时有$\mathfrak{D}_{B/A}=(f'(a))$,其中$f$是$a$在$A$上的极小多项式.如果$K\subseteq L$非分歧,记$\overline{a}=a(\mathrm{mod}q)$和$\overline{f}\equiv f(\mathrm{mod}p)$.按照扩张可分,得到$f'(\alpha)\in A^*$.此时$s=0=e-1$.
    	
    	设$K\subseteq L$是完全分歧的,此时$e=[L:K]$,可记$f(x)=c_ex^e+c_{e-1}x^{e-1}+\cdots+c_0$,其中$c_e=1$.按照素元$a$是这个多项式的根,得到$c_1,c_2,\cdots,c_{e-1}$的赋值都$\ge1$,于是这个多项式是爱森斯坦多项式.此时$f'(x)=ex^{e-1}+(e-1)c_{e-1}x^{e-2}+\cdots+c_1$.每个$v_q(ic_ia^{i-1})\equiv i-1(\mathrm{mod}e)$,导致$v_q(f'(a))=\min\{v_q(ic_i)+i-1\}$.如果$K\subseteq L$是温分歧的,也即$v_p(e)=0$,那么这里最小项恰好是第一项,也即$s=e-1$.如果$K\subseteq L$是野分歧,那么$e\le v_q(f'(a))\le v_q(e)+e-1$.
    \end{proof}
    \item 和微分模的联系.给定环扩张$A\subseteq B$,定义微分模$\Omega^1_{B/A}$是$db,b\in B$在$B$上生成的自由模,再模去$d(xy)-xdy-ydx$和$dA$.那么$\mathfrak{D}_{B/A}$是$B$模$\Omega_{B/A}^1$的零化子,也即$\mathfrak{D}_{B/A}=\{x\in B\mid xdy=0,\forall y\in B\}$.
    \begin{proof}
    	
    	不妨设$A$是完备离散赋值环,设剩余域扩张$\kappa\subset\lambda$是可分的,那么存在$\overline{x}\in\lambda$使得$\lambda=\kappa(\overline{x})$.取它在$\kappa$中的极小多项式$\overline{f}(X)\in\kappa(X)$,取一个提升$f(x)\in A[X]$,任取$\overline{x}$的提升$x$,那么$f(x)\in q$,我们期望选取的提升$x$使得$v_L(f(x))=1$.因为如果$v_L(f(x))\ge2$,按照可分性得到$\overline{f}'(\overline{x})=0$,导致$f'(x)\in B^*$,此时$x+\Pi_L$也是$\overline{x}$的提升,并且$v_L(f(x+\Pi_L))=1$.于是此时选取的提升$x$使得$f(x)$是$B$中的素元.我们之前证明过此时$\{x^if(x)^j,0\le i\le f-1,0\le j\le e-1\}$生成了整个$B$,于是$B=A[x]$.此时有$\Omega^1_{B/A}=Bdx$.设$g(X)$零化$x$,那么$g'(x)dx=dg(x)=0$,于是零化子恰好是$(g'(x))$.
    \end{proof}
\end{enumerate}

判别式的性质:
\begin{enumerate}
	\item 数域的情况.设$K$是数域,那么它的代数整数环$\mathscr{O}_K$是$\mathbb{Z}$自由交换群,它的一组基$\{\alpha_1,\alpha_2,\cdots,\alpha_n\}$称为这个代数整数环的整基,整基可定义判别式$d_K(\alpha_1,\alpha_2,\cdots,\alpha_n)=\left|\det(\sigma_i\alpha_j)\right|^2$.不同整基之间的过渡矩阵是$\mathrm{GL}_n(\mathbb{Z})$中的元,它的行列式是$\pm1$,平方就是1,于是数域的判别式不依赖于整基的选取.现在考虑一般的数域扩张$K\subseteq L$,此时$\mathscr{O}_L$未必未必是$\mathscr{O}_K$的自由模,即便它是自由模,它定义出来的判别式在不同基的选取下将会相差一个$(\mathscr{O}_K^*)^2$中的元.考虑全体满足$\mathscr{O}_K\alpha_1+\mathscr{O}_K\alpha_2+\cdots+\mathscr{O}_K\alpha_n\subset\mathscr{O}_L$的$n$元组$\{\alpha_1,\alpha_2,\cdots,\alpha_n\}$,考虑$d_{L/K}(\alpha_1,\alpha_2,\cdots,\alpha_n)\in\mathscr{O}_K$.全体这样的元生成的$\mathscr{O}_K$的理想称为是扩张$K\subseteq L$的判别式$d_{L/K}$.
	\item 和共轭差积的关系.我们有$d_{B/A}=\chi_A(B^*/B)=\mathrm{N}_{L/K}(\mathfrak{D}_{B/A})$.
	\begin{proof}
		
		不妨设$A$是离散赋值环,特别的它是一个PID,按照具有有限个素理想的戴德金整环是DVR,说明$B$也是一个PID.我们之前解释过$A$是PID的时候有$B$是有限生成自由$A$模,于是存在一组$A$基$\{\alpha_1,\alpha_2,\cdots,\alpha_n\}$使得判别式$d_{B/A}$是$(d_{B/A}(\alpha_1,\alpha_2,\cdots,\alpha_n))$.此时$\mathfrak{D}^{-1}_{B/A}$被对偶基生成$A\alpha_1^{v}+A\alpha_2^v+\cdots+A\alpha_n^v$.这是一个主分式理想,可记作$(\beta)$.于是$\{\beta\alpha_1,\beta\alpha_2,\cdots,\beta\alpha_n\}$也是一组$A$基.判别式满足$d(\beta\alpha_1,\beta\alpha_2,\cdots,\beta\alpha_n)=\mathrm{N}_{L/K}(\beta)^2d(\alpha_1,\alpha_2,\cdots,\alpha_n)$.
		
		但是这里$(\mathrm{N}_{L/K}(\beta))=\mathrm{N}_{L/K}(\mathfrak{C}_{B/A})=\mathrm{N}_{L/K}(\mathfrak{D}_{L/K})^{-1}$.另外$d(\alpha_1,\cdots,\alpha_n)=\det(\sigma_i\alpha_j)^2$和$d(\alpha_1^v,\cdots,\alpha_n^v)=\det(\sigma_i\alpha_j^v)^2$结合$\mathrm{T}(\alpha_i\alpha_j^v)=\delta_{ij}$说明$d(\alpha_1,\cdots,\alpha_n)d(\alpha_1^v,\cdots,\alpha_n^v)=1$,于是有:
		\begin{align*}
			d_{L/K}^{-1}&=d(\alpha_1,\cdots,\alpha_n)^{-1}=d(\alpha_1^v,\cdots,\alpha_n^v)=(d(\beta\alpha_i))\\&=\mathrm{N}_{L/K}(\mathfrak{D}_{L/K})^{-2}d_{L/K}
		\end{align*}
		
		另外如果借助格的语言证明更简单:我们在格上双线性型的判别式中证明过有$d_{B/A}=\chi_A(B^*/B)$.也证明过$B^*\chi_B(B^*/B)=B^*\chi_B(B^*,B)=B$于是有$\chi_B(B^*/B)=\mathfrak{D}_{B/A}$.最后只要注意的,如果$B$模$M$是有限长度的,那么有$\chi_A(M)=\mathrm{N}(\chi_B(M))$,取$M=B^*/B$得到结论.
	\end{proof}
	\item 如果$S\subseteq A$是一个乘性闭子集,那么$d_{S^{-1}B/S^{-1}A}=S^{-1}d_{B/A}$.
    \item 如果有戴德金整环的扩张链$A\subseteq B\subseteq C$,对应商域扩张链是$K\subseteq L\subseteq M$,那么$d_{C/A}=d_{B/A}^{[M:L]}\mathrm{N}_{L/K}(d_{C/B})$.
    \begin{proof}
    	\begin{align*}
    	d_{C/A}&=\mathrm{N}_{M/K}(\mathfrak{D}_{C/A})\\&=\mathrm{N}_{M/K}(\mathfrak{D}_{C/B}\mathfrak{D}_{B/A})\\&=\mathrm{N}_{L/K}(\mathrm{N}_{M/L}(\mathfrak{D}_{C/B}))\mathrm{N}_{L/K}(\mathfrak{D}_{B/A}^{[M:L]})\\&=\mathrm{N}_{L/K}(d_{C/B})d_{B/A}^{[M:L]}
    	\end{align*}
    \end{proof}
    \item 回顾一下给定扩张$A\subseteq B$,那么素理想$p\in A$称为非分歧的如果它的每个提升素理想都是非分歧的.于是$p\not|\mathrm{N}_{L/K}(\mathfrak{D}_{B/A})=d_{B/A}$当且仅当$p$是非分歧的.特别的,判别式为单位理想1当且仅当扩张的所有素理想都是非分歧的.
    \item 设$K\subseteq L$的判别式为$d$,设$q$跑遍$B$的非零素理想,记$d_q$表示扩张$K_p\subseteq L_q$的判别式,其中$p=q\cap A$,那么乘积公式和上面判别式是共轭差积的范数得到:$d=\prod_qd_q$.
\end{enumerate}

判别式在数域上的一些应用.
\begin{enumerate}
	\item 定理.设$K$是数域,全部素位构成的集合记作$\Sigma_K$,取它的有限子集$S\subset\Sigma_K$,固定正整数$n$,那么只存在有限个$n$次扩张$K\subseteq L$使得扩张在$S$以外非分歧.
	\begin{proof}
		
		按照扩张在$S$以外非分歧,说明判别式满足$d_{L/K}=\prod_{p\in S}p^{n_p}$.我们固定了扩张维数,于是这里$n_p$都是有界的,即判别式整除了某个固定元$a\in K$.于是此时判别式只有有限种选择,于是归结于证明额外的固定判别式时扩张只有有限个.另外我们还可以不妨设$K=\mathbb{Q}$,因为如果$K\subseteq L$是判别式为$d$的次数$n$的扩张,那么$\mathbb{Q}\subseteq L$是判别式为$d_{K/\mathbb{Q}}^n\mathrm{N}_{K/\mathbb{Q}}(d)$的次数为$n[K:\mathbb{Q}]$的扩张.最后我们不妨约定$L$只有复素位,因为否则的化可以考虑二次扩张$L\subseteq L(\sqrt{-1})$,此时$d_{L(i)/\mathbb{Q}}=d_{L/\mathbb{Q}}^2\mathrm{N}_{L/\mathbb{Q}}(d_{L(i)/L})$.整理一下,我们只需证判别式$d_{K/\mathbb{Q}}$固定,满足$i\in K$,并且$[K:\mathbb{Q}]$固定为$n$的数域$K$只有有限个.
		
		设全部的复嵌入为$\tau_0,\tau_1,\cdots,\tau_{n-1}:K\to\mathbb{C}$.考虑Minkowski空间$K_{\mathbb{R}}$.其上有凸集$X=\{(z_{\tau})\in K_{\mathbb{R}}\mid |\mathrm{Im}(z_{\tau_0})|<c\sqrt{d},|\Re(z_{\tau_0})|<1,|z_{\tau}|<1,\tau\not=\tau_0,\overline{\tau_0}\}$.取$c$充分大使得$\mathrm{Vol}(X)>2^n\sqrt{d}=2^n\mathrm{Vol}(\mathscr{O}_K)$.按照Minkowski格点定理,此时存在一个$0\not=\alpha\in O_K$使得$j\alpha\in X$.这个$\alpha$是整元,于是$|\mathrm{N}_{K/\mathbb{Q}}|\ge1$,而这个范数的绝对值是$|\prod_{\tau}\tau(\alpha)|$,按照我们的构造,导致$|\tau_0(\alpha)|>1$,并且$\tau_0(\alpha)$的虚部的绝对值也是$>1$的.特别的这个虚部不为零,于是$\tau_0(\alpha)\not=\overline{\tau_0}(\alpha)$.其它的$\tau$作用下的绝对值都小于1,于是$\tau_0(\alpha)\not=\tau(\alpha),\tau\not=\tau_0,\overline{\tau_0}$.我们断言此时$K=\mathbb{Q}(\alpha)$,因为如果$\mathbb{Q}(\alpha)\subsetneqq K$,那么$\tau_0$限制到$\mathbb{Q}(\alpha)$上这个同态将会存在一个异于$\tau_0$的提升,按照我们的构造这不可能成立.
		
		最后说明这样的$\alpha$只有有限多个即可:按照$c$和$d$和$n$都是固定的,于是$\alpha$的共轭元有界,于是$\alpha$满足的极小多项式的系数有界,但是这个极小多项式是整系数的,导致极小多项式只有有限多个,于是根$\alpha$是有限的.
	\end{proof}
    \item $K$是$n$次数域,那么有$\sqrt{|d_K|}\ge\frac{n^n}{n!}\left(\frac{\pi}{4}\right)^{n/2}$.
    \begin{proof}
    	
    	取对称凸集$X_t=\{(z_{\tau})\in K_{\mathbb{R}}\mid\sum|z_{\tau}|<t\}$,那么$\mathrm{Vol}(X_t)=2^{r_1}\pi^{r_2}\frac{t^n}{n!}$.当$\mathrm{Vol}(X_t)=2^n\sqrt{|d_K|}+\varepsilon$时按照Minkowski格点定理,存在$0\not=\alpha\in\mathscr{O}_K$使得$j\alpha\in X_t$.然后有:
    	\begin{align*}
    	1&\le|\mathrm{N}_{K/\mathbb{Q}}(\alpha)|=\prod|\tau(\alpha)|\le\left(\frac{\sum|\tau(\alpha)|}{n}\right)^n\\&=\left(\frac{t}{n}\right)^n=\frac{n!}{2^{r_1}\pi^{r_2}n^n}2^n\sqrt{|d_K|}+c\varepsilon
    	\end{align*}
    	
    	得到$\sqrt{|d_K|}\ge\left(\frac{\pi}{4}\right)^{r_2}\frac{n^n}{n!}-c'\varepsilon$.这里$|d_K|$是整数,于是选取足够小的$\varepsilon$会导致$\sqrt{|d_K|}\ge\left(\frac{\pi}{4}\right)^{r_2}\frac{n^n}{n!}$,再用下$r_2\le n/2$.
    \end{proof}
    \item 上一条中的下界是一个单调增数列,导致$a_n>a_2=\frac{\pi}{2}>1,n\ge2$,于是不为$\mathbb{Q}$的数域$K$的判别式总不是$\pm1$.特别的,这说明数域的扩张$\mathbb{Q}\subseteq K$总存在分歧素理想,也即$\mathbb{Q}$不存在非平凡的非分歧扩张.
\end{enumerate}
\newpage
\subsection{分歧群}

分解群,惯性群和分歧群.设$K\subseteq L$是Galois扩张,设$v$是$K$上的赋值,设$w$是$v$在$L$上的一个延拓赋值.设$(K,v)$和$(L,w)$的赋值环分别为$A$和$B$,设它们的唯一极大理想分解为$m$和$n$.
\begin{itemize}
	\item $w$的固定子群$G_w=G_w(L/K)=\{\sigma\in G(L/K)\mid w\circ\sigma=w\}$称为$w\mid v$的分解群.
	\item $w\mid v$的惯性群定义为$I_w=I_w(L/K)=\{\sigma\in G_w\mid\sigma(x)\equiv x(\mathrm{mod}n),\forall x\in B\}$.
	\item $w\mid v$的分歧群定义为$R_w=R_w(L/K)=\{\sigma\in G_w\mid\frac{\sigma(x)}{x}\equiv1(\mathrm{mod}n),\forall x\in L^*\}$.
\end{itemize}
\begin{enumerate}
	\item $R_w\subseteq I_w\subseteq G_w$都是$G$的闭子群.例如我们证明$G_w$是闭的:任取$\sigma\in\overline{G_w}$,那么对$\sigma$的每个开邻域$\sigma G(M/K)$,其中$M$是有限Galois子扩张,那么存在$\sigma_M\in G_w$,于是$\sigma_M\mid_M=\sigma\mid_M$,并且$w\circ\sigma_M=w$得到$w\circ\sigma\mid_M=w\mid_M$,按照$L$是这样全体有限Galois子扩张的合成,得到$w\circ\sigma=w$,于是$\sigma\in G_w$,于是$G_w$是闭子集.
	\item 考虑两个Galois扩张$K\subseteq L$和$K'\subseteq L'$,设它们满足如下交换图表,这里$\tau$是包含映射,那么它诱导了群同态$\tau^*:G(L'/K')\to G(L/K)$为$\sigma'\mapsto\tau^{-1}\sigma'\tau$.这个映射有意义是因为,按照$K\subseteq L$是正规扩张,得到$\tau(K)\subset\tau(L)$是正规扩张,于是$\sigma'\circ\tau(L)\subset\tau(L)$.
	$$\xymatrix{L\ar[rr]^{\tau}&&L'\\K\ar[u]\ar[rr]^{\tau}&&K'\ar[u]}$$
	\item 记号同上一条,设$w'$是$L'$的一个赋值,设$v'$是$w'$在$K'$上的限制,设$w=w'\circ\tau$,设$v$是$w$在$K$上限制,那么$\tau^*:G(L'/K')\to G(L/K)$诱导了如下三个同态:$G_{w'}(L'/K')\to G_w(L/K)$,$I_{w'}(L'/K')\to I_w(L/K)$和$R_{w'}(L'/K')\to R_w(L/K)$.对于最后两个情况,约定$v$是非阿基米德的.
	\item 特别的,上一条中取$L=L'$和$K=K'$,取$\tau\in G(L/K)$,就得到$G_{w\circ\tau}=\tau^{-1}G_w\tau$,$I_{w\circ\tau}=\tau^{-1}I_w\tau$,$R_{w\circ\tau}=\tau^{-1}R_w\tau$.
	\item 特别的,任取扩张$K\subseteq L$的中间域$M$,考虑如下交换图表,那么$\tau^*$是包含映射$G(L/M)\to G(L/K)$,于是有$G_w(L/M)=G_w(L/K)\cap G(L/M)$,$I_w(L/M)=I_w(L/K)\cap G(L/M)$,$R_w(L/M)=R_w(L/K)\cap G(L/M)$.
	$$\xymatrix{L\ar[rr]^{=}&&L\\K\ar[u]\ar[rr]^{\tau}&&M\ar[u]}$$
	\item 特别的,取$K\subseteq L$上的延拓赋值$w\mid v$,如果域扩张是无穷的,我们解释过此时$L_w$是$w$在$L$上的局部化,即全部有限子扩张的完备化的合成,此时$\tau^*$是限制映射$G(L_w/K_v)\to G(L/K)$为$\sigma\mapsto\sigma\mid_L$.此时有$G_w(L/K)\cong G(L_w/K_w)$,$I_w(L/K)\cong I(L_w/K_w)$和$R_w(L/K)\cong R(L_w/K_w)$.事实上这些事情可以从$G_w(L/K)$恰好由$G(L/K)$中的在赋值$w$下连续的自同构$\sigma$构成.
	\begin{proof}
		
		首先我们断言如果$\sigma\in G$,那么$\sigma\in G_w(L/K)$当且仅当$\sigma$在$w$赋值下是连续映射:必要性因为$|\sigma(x)|_w=|x|_w$自然得到$\sigma$连续,充分性要说明如果$\sigma$满足$w\circ\sigma$不等价于$w$,那么$\sigma$不是连续的.如果它是连续的,那么$|x|_w<1$推出$x^n\to0$,连续性导致$\sigma(x)^n\to0$,导致$|\sigma(x)|_w<1$,于是$w$和$\sigma w$是等价的.
		
		现在$L\subseteq L_w$是稠密的,于是赋值$w$上的连续映射$\sigma$唯一的提升为$L_w$上的连续映射.反过来$G(L_w/K_v)$中的映射限制到$L$上落在$G_w(L/K)$中,于是得到$G_w(L/K)\cong G(L_w/K_v)$.
	\end{proof}
	\item 记$v$在$L$中的全部延拓赋值为$W_v$,考虑$S$为子群$G_w$在$G$中的全部右陪集,存在双射$S\to W_v$为$G_w\sigma\mapsto w\sigma$,于是$v$在$L$上延拓赋值的个数恰好是$[G:G_w]$.
\end{enumerate}

分解域.设$K\subseteq L$是Galois扩张,设$G$是Galois群,称分解群对应的固定中间域为分解域,即$Z_w=Z_w(L/K)=\{x\in L\mid\sigma(x)=x,\forall\sigma\in G_w\}$.
\begin{enumerate}
	\item $w$在$Z_w$上的限制记作$w_Z$,那么$w_Z$在$L$上的唯一延拓赋值是$w$本身,粗略的讲这说明赋值延拓只能分解到分解域,再往上不能进一步分解.事实上任取$w_Z$在$L$上的延拓赋值$w'$,那么$w'$和$w$是共轭的,按照它们都是$w_Z$的延拓,说明存在$\sigma\in G(L/Z_w)=G_w$使得$w'=w\circ\sigma\sim w$,于是它们是等价的.
	\item $Z_w=L\cap K_v$,换句话讲$Z_w$上对$w_Z$做完备化仍然是$K_v$.事实上按照$G(L_w/K_v)=G_w(L/K)$,于是Galois基本定理得到$Z_w=L\cap K_v\subseteq L_w$.
	\item 如果$v$是非阿基米德赋值,那么$w_Z$和$v$的剩余域相同,但是这个扩张又非分歧,于是换句话讲这个扩张是一个完全分歧扩张.事实上按照$K\subseteq Z_w\subseteq K_v$,从$K$和$K_v$的剩余域相同得到$Z_w$的剩余域和它们相同.
\end{enumerate}

惯性域.惯性群对应的固定中间域称为惯性域,即$T_w=T_w(L/K)=\{x\in L\mid\sigma(x)=x,\forall\sigma\in I_w\}$.
\begin{enumerate}
	\item 设$K\subseteq L$是Galois扩张,设延拓赋值$w\mid v$是非阿基米德赋值,设$K$和$L$的赋值环分别是$A$和$B$,任取$\sigma\in G_w$,有$\sigma(n)=n$,于是$\sigma$诱导了剩余域上的同构$\overline{\sigma}:B/n\to B/n$,并且它固定$A/m$,于是$\overline{\sigma}\in G(\kappa(L)/\kappa(K))$.于是我们得到了一个群同态$G_w\to G(\kappa(L)/\kappa(K))$,我们断言这是一个满射,并且核即$I_w$,换句话讲有如下正合列.另外剩余域扩张$\kappa(K)\subset\kappa(L)$是正规扩张.
	$$\xymatrix{1\ar[r]&I_w\ar[r]&G_w\ar[r]&G(\kappa(L)/\kappa(K))\ar[r]&1}$$
	\begin{proof}
		
		记$K$和$L$的剩余域分别是$\kappa$和$\lambda$.如果$\overline{\sigma}$是$G(\lambda/\kappa)$中的恒等同态,于是在$\mathrm{mod}n$下$x\equiv\sigma(x)$,此即$\sigma\in I_w$,于是这个同态的核恰好是$I_w$.再说明这个剩余域扩张是正规的.任取$\overline{\theta}\in\lambda$,其中$\theta\in B$,设$\overline{\theta}$在$\kappa$中的极小多项式为$\overline{g}$,其中$g\in B[x]$.记$\theta$在$K$中极小多项式为$f$,按照$K\subseteq L$是Galois扩张,于是它是正规扩张,于是$f$在$L$中分解为不同一次因式乘积,于是$\overline{f}$在$\lambda$中分解为(未必不同)一次因式乘积,并且$\overline{g}\mid\overline{f}$,于是$\overline{g}$在$\lambda$中分解为一次因式的乘积,于是剩余域扩张是正规的.
		
		现在设$K=Z_w$不影响条件,此时$K$是Hensel域于是满足Hensel引理,设$\kappa$在$\lambda$中的可分闭包为$\kappa^s$,设$\kappa^s=\kappa(\overline{\theta})$,取$\overline{\sigma}\in G(\lambda/\kappa)=G(\kappa(\overline{\theta})/\kappa)$,那么$\overline{\sigma}(\overline{\theta})$也是$\overline{g}$的根.于是按照Hensel引理存在$f$的根$\theta'$使得$\theta'\equiv\overline{\sigma}(\overline{\theta})(\mathrm{mod}n)$.按照$\theta'$是$\theta$的共轭元.于是存在$\sigma\in G=G_w$(因为我们设了$K=Z_w$)使得$\theta'=\sigma(\theta)$.这说明这个映射是满射.
	\end{proof}
	\item 上一条中的短正合列得到$G(T_w/Z_w)\cong G(\kappa(L)/\kappa(K))$.
	\item $Z_w\subseteq T_w$就是$Z_w\subseteq L$的极大非分歧子扩张.
	\begin{proof}
		
		不妨设$K=Z_w$,设$K\subseteq T$是$K\subseteq L$的极大非分歧子扩张,于是有典范的同态$G(T/K)\to G(\lambda_s/\kappa)$,这里$\lambda_s$是$\kappa\subset\lambda$的可分闭包.这个同态是满射的因为我们之前证明过的短正合列中,把$K\subseteq L$改成$K\subseteq T$即可.它是单射的因为按照扩张非分歧得到$e=1$,得到这两个群的元素阶数相同,于是此时满射必然是单射.于是这个典范同态是同构,得到$G(L/T)=I_w=G(L/T_w)$,于是有$T=T_w$.
	\end{proof}
\end{enumerate}

分歧域.分歧群的固定中间域称为分歧域,即$V_w=V_w(L/K)=\{x\in L\mid\sigma(x)=x,\forall\sigma\in R_w\}$.
\begin{enumerate}
	\item 构造$\chi(L/K)=\mathrm{Hom}(w(L^*)/v(K^*),\lambda^*)$,构造$I_w\to\chi(L/K)$为把$\sigma\in I_w$映射为$w(x)\mapsto\frac{\sigma(x)}{x}(\mathrm{mod}n)$.先解释这个定义是良性的,$w(x)\mapsto\sigma(x)/x(\mathrm{mod}n)$首先是$w(L^*)\to\kappa(L)^*$的同态,因为$w(x)+w(y)=w(xy)\mapsto \sigma(xy)/xy+n=(\sigma(x)/x+n)(\sigma(y)/y+n)$.另外$v(K^*)$包含在这个映射的核中,因为$x\in v(K^*)$导致$\sigma(x)=x$.于是它诱导了$w(L^*)/v(K^*)\to\kappa(L)^*$的同态.再说明我们构造的$I_w\to\chi(L/K)$是同态.任取两个$\sigma,\tau\in I_w$,那么$\sigma\tau$把$w(x)$映射为$\frac{\sigma\tau(x)}{x}$,于是需要证明这个像等于$\sigma$和$\tau$分别的像的乘积$\frac{\sigma(x)\tau(x)}{x^2}$.但是把$x$替换为共轭元不影响它的赋值,结合$v(\tau(x)/x)=0$,于是$x$替换为$\tau(x)$不改变它的像,于是$\frac{\sigma(x)}{x}\frac{\tau(x)}{x}=\frac{\sigma\tau(x)}{\tau(x)}\frac{\tau(x)}{x}=\frac{\sigma\tau(x)}{x}$.可证明这个群的阶数和$p$互素.
	\item 按照$\lambda$是特征$p$的,说明$\lambda^*$的阶和$p$互素,得到$\chi(L/K)=\mathrm{Hom}(w(L^*)/v(K^*),\lambda^*)=\mathrm{Hom}(w(L^*)^{(p)}/v(K^*),\lambda^*)$.
	\item 这一条我们证明上面构造的同态诱导了如下短正合列,换句话讲这个同态是满的,它的核明显是$R_w$,我们断言$R_w$恰好是$I_w$唯一的Sylow-$p$子群.
	$$\xymatrix{1\ar[r]&R_w\ar[r]&I_w\ar[r]&\chi(L/K)\ar[r]&1}$$
	\begin{proof}
		
		先不妨设$K\subseteq L$是完备域的扩张,因为我们解释过$I_w(L/K)=I(L_w/K_v)$和$R_w(L/K)=R(L_w/K_v)$.先证明$R_w$是$I_w$的$p$群,若否可取$\sigma\in R_w$的阶是一个素数$l\not=p$.设$K'$是$\sigma$的固定中间域,设$\kappa'$是$K'$的中间域.按照$V_w$是固定整个$R_w$的中间域,说明$K'$落在$V_w$和$T_w$之间.按照$T$的剩余域是$\kappa\subset\lambda$的可分闭包,于是$\kappa'\subset\lambda$是纯不可分扩张.于是它的扩张次数是$p$的某个次幂.任取$\overline{\theta}\in\lambda$,其中$\theta\in L$是它的一个提升.设$\overline{g}$在$\kappa'$中的极小多项式为$\overline{g}$,再记$\theta$在$K'$中的极小多项式为$f\in K'[x]$,于是$\overline{g}\mid\overline{f}$,按照$f$是不可约多项式,$\overline{f}$只能是$\overline{g}$的次幂(否则多余的不可约因子按照Hensel引理导致$f$是可约的).记$\overline{f}=\overline{g}^m$.但是按照$[L:K']=l$说明$\deg f$要么是1要么是$l$.而$\overline{g}$的次数是$p$的次幂,这导致$\deg\overline{g}=1$,于是$\kappa'=\lambda$.于是$K'\subseteq L$是温分歧扩张,于是有$L=K'(\alpha)$,其中$\alpha=\sqrt[e]{a}$,于是有$\sigma(\alpha)=\zeta\alpha$,其中$\zeta$是$l$次本原单位根.但是按照$\sigma\in R_w$,得到$\zeta=\sigma(\alpha)/\alpha\equiv1(\mathrm{mod}n)$,导致$\zeta=1$,这和$\zeta$是$l$次本原单位根矛盾.综上我们证明了$R_w$是$I_w$的$p$子群.
		
		但是我们解释过$\chi(L/K)$的阶数和$p$互素,而$I_w/R_w$是$\chi(L/K)$的子群,于是$I_w/R_w$的阶数和$p$互素,于是$R_w\subseteq I_w$是Sylow-$p$子群.
		
		最后我们证明这里的同态是满射.为此只需证明$[V_w:T_w]$恰好是$\chi(L/K)$的元素个数.按照下一条,得到$T_w\subseteq V_w$的惯性次数是1.于是$[V_w:T_w]=[w(V_w^*):w(T_w^*)]=[w(L^*)^{(p)}:v(K^*)]$.于是问题归结为证明$\chi(L/K)$的元素个数恰好是$|w(L^*)^{(p)}|/|v(K^*)|$.按照这里的群都是有限群,问题归结为证明如果$w(L^*)^{(p)}/v(K^*)$中有$m$阶元,那么$\lambda^*$中有$m$阶元.因为这样按照有限交换群结构定理,有限直和在第一个位置和Hom函子的交换性,结合$\mathrm{Hom}(\mathbb{Z}/n,M)\cong M[n]$,得到$\chi(L/K)$元素阶数和$|w(L^*)^{(p)}|/|v(K^*)|$的元素阶数相同.而仅剩的要说明的这件事在温分歧扩张结构定理中证明过.
	\end{proof}
	\item $Z_w\subseteq V_w$恰好是$Z_w\subseteq L$的极大温分歧扩张.
	\begin{proof}
		
		同样不妨设$K\subseteq L$是完备域的扩张.任取次数与$p$互素的扩张$T\subseteq L$,它剩下的那一段扩张的Galois群必然包含了Sylow-$p$子群$R_w$,于是$T\subseteq V_w$.于是极大温分歧子扩张$V$包含在$V_w$中.并且$[V_w:V]$依旧与$p$互素.这里剩余域都是特征$p$的,和扩张维数互素,于是剩余域扩张$V\subseteq V_w$是可分的,于是$V\subseteq V_w$是温分歧扩张,但是$V$本身已经是极大温分歧扩张,迫使$V=V_w$.
	\end{proof}
\end{enumerate}

整理一下:
$$\xymatrix{\text{域扩张:}&K\ar[rr]_{\text{完全分裂}}^{e=1}&&Z_w\ar[rr]_{\text{极大非分歧}}^{e=1}&&T_w\ar[rr]_{\text{极大温分歧}}^{e=e'}&&V_w\ar[rr]_{\text{野分歧}}^{e=p^a}&&L\\\text{剩余域扩张:}&\kappa\ar[rr]^{f=1}&&\kappa\ar[rr]_{\text{可分闭包}}&&\lambda^s\ar[rr]^{f=1}&&\lambda^s\ar[rr]_{\text{纯不可分}}&&\lambda}$$

高阶分歧群.设$(K,v_K)$是完备离散赋值域,其中$v_K$是规范的(即赋值群是$\mathbb{Z}$),设$L/K$是有限Galois扩张,设$G$是Galois群,设$K$的赋值环是$(A,\mathfrak{m})$,设$A$在$L$中整闭包是$(B,\mathfrak{n})$,总设剩余域扩张$\kappa\subseteq\lambda$是可分的,设扩张的分歧指数$e_{L/K}$,惯性次数$f_{L/K}$,那么有$ef=[L:K]$.记$v_L$是$L$上的规范离散赋值,换句话讲如果$v$延拓在$L$上的延拓赋值记作$w$,记$v_L=ew$.
\begin{enumerate}
	\item 引理.设$K\subseteq L$是Galois扩张,设$v$是$K$上的规范离散赋值,设$w$为$v$在$L$上的延拓赋值,设$K$和$L$的赋值环分别是$A$和$B$,设剩余域扩张$\kappa\subset\lambda$是可分的,那么存在$B$中的素元$a$使得$B=A[a]$.
	\begin{proof}
		
		取剩余类域扩张$\kappa\subset\lambda$的本原元$\overline{x}$的一个代表元$x\in B$,取$B$的素元$\Pi$,那么$\{x^i\Pi^j\mid 0\le i\le f-1,0\le j\le e-1\}$是$A$模$B$的一组基.
		
		\qquad
		
		如果有线性组合$\sum_{i,j}a_{ij}x^i\Pi^j=0$,其中$a_{ij}\in A$,如果$s_j=\sum_ia_{ij}x^i\not=0$,我们断言$w(s_j)\in\mathbb{Z}$(这里设$K$上赋值是规范的,$w$是在$L$上的延拓赋值).考虑这个线性组合非零系数赋值的最小值,让$s_j$这个等式除以这个赋值最小的系数$a$,新的线性组合中非零系数至少有一项赋值是1,所以$\mod\mathfrak{n}$下($\mathfrak{n}$是$B$的极大理想)不是零,所以它是$B$的单位,所以$w(s_i)=w(a)\in\mathbb{Z}$.
		
		\qquad
		
		接下来考虑$0=\sum_js_j\Pi^j$.这其中非零项至少有两个赋值是相同的,因为如果$w(x)\not=w(y)$理应会有$w(x+y)=\min\{w(x),w(y)\}$.设$w(s_j\Pi^j)=w(s_i\Pi^i)$,在$\mod\mathbb{Z}$下就有$i w(\Pi)=j w(\Pi)$,导致$i=j$就矛盾.至此我们说明了$\{x^i\Pi^j\mid 0\le i\le f-1,0\le j\le e-1\}$在$A$上线性无关.
		
		\qquad
		
		下面说明它生成了整个$B$.考虑这个集合在$A$上生成的$B$的子模$M$,再记$N$是$\{x^i\}$在$A$中生成的模,于是有$M=N+\Pi N+\cdots+\Pi^{e-1}N$.并且有$B=N+\Pi B$,反复迭代得到如下等式,其中$\mathfrak{m}$是$A$的极大理想,于是按照NAK引理得到$B=M$.
		$$B=N+\Pi(N+\Pi B)=N+\Pi N+\cdots+\Pi^{e-1}N+\Pi^eB=M+\mathfrak{m}B$$
		
		\qquad
		
		最后我们证明取本原元$\overline{x}$的极小多项式的一个提升$f$,那么存在本原元的提升$x$,使得$f(x)$是$B$的素元.一旦这件事得证,那么$x^i\Pi^j=x^if(x)^j$就是$x$的多项式,导致$B=A[x]$.先任取本原元的提升$x$,那么$f(x)$在$\mod\mathfrak{n}$下的像是零,导致$w(f(x))\ge1$.假设$w(f(x))\ge2$,任取$B$的素元$\Pi$,有$f(x+\Pi)=f(x)+f'(x)\Pi+\cdots$,没写出来的项是赋值$\ge2$的项,这里$f'(x)$是$B$中的单位,因为剩余类域扩张是可分的导致$\overline{f}(\overline{x})\not=0$,导致它是单位,于是$f'(x)\Pi$的赋值是1,所以从$w(x)\not=w(y)$时$w(x+y)=\min\{w(x),w(y)\}$得到$w(f(x+\Pi))=1$,而$x+\Pi$也是$\overline{x}$的提升.
	\end{proof}
	\item 设$\sigma\in G$,设$s\ge-1$是任意实数,那么如下三个条件互相等价,对固定的$s$,全体满足这个等价条件的$\sigma$构成了$G$的正规子群,记作$G_s$,称为$s$阶分歧群,这是一个递降的正规子群链,有$G_{-1}=G$,当$s$足够大时有$G_s=\{e\}$,于是分歧群列是$G$的有限长度的滤过.另外按照$L$上赋值群就是$\mathbb{Z}$,如果$n\ge-1$是整数,如果$n<s\le n+1$,那么$G_s=G_{n+1}$.
	\begin{enumerate}
		\item $\sigma$诱导了$B/\mathfrak{n}^{s+1}$上的恒等映射.
		\item $v_L(\sigma(a)-a)\ge s+1,\forall a\in B$.
		\item 设$x\in B$是满足$B=A[x]$的素元,有$v_L(\sigma(x)-x)\ge s+1$.
	\end{enumerate}
    \begin{proof}
    	
    	(a)和(b)的等价性平凡.(a)和(c)的等价性是因为,由于$B=A[x]$,导致$\sigma$诱导的$B/\mathfrak{n}^{s+1}\to B/\mathfrak{n}^{s+1}$是恒等映射当且仅当它把$x+\mathfrak{n}^{s+1}$映射为自身,也即$v_L(\sigma(x)-x)\ge s+1$.
    	
    	\qquad
    	
    	验证$G_s$是正规子群.任取$\sigma\in G_s$的共轭元$\tau^{-1}\sigma\tau$,得到$v(\tau^{-1}\sigma\tau(x)-x)=v(\tau^{-1}(\sigma\tau(x)-\tau(x)))=v(\sigma(\tau(x))-\tau(x))$(共轭元当然具有相同赋值),当$x$跑遍$B$时$\tau(x)$同样跑遍$B$,于是这个赋值总$\ge s+1$.另外用(a)可能更直接点,如果$\sigma$诱导在$B/\mathfrak{n}^{s+1}$上是恒等的,那么$\tau^{-1}\sigma\tau$明显诱导在$B/\mathfrak{n}^{s+1}$上也是恒等的.
    	
    	\qquad
    	
    	验证当$s$足够大时有$G_s=\{e\}$.因为这里$L/K$是有限扩张,只要取$N=\max\{v_L(\sigma(x)-x)\}$,其中$\sigma\not=\mathrm{id}$,那么当$s\ge N$时就有每个$\sigma\not=\mathrm{id}$都不在$G_s$里.也即$G_s=\{e\}$.
    \end{proof}
    \item 函数$i_G:G-\{\mathrm{id}\}\to\mathbb{Z}$定义为,取定一个素元$x\in B$使得$B=A[x]$,记$i_G(\sigma)=v_L(\sigma(x)-x)$.可约定$i_G(\mathrm{id})=+\infty$.我们断言这个函数的定义不依赖于单生成元$x\in B$的选取.另外$i_G$有如下基本性质:
    \begin{enumerate}
    	\item $i_G(\sigma)\ge s+1\Leftrightarrow\sigma\in G_s$.
    	\item $i_G(\tau\sigma\tau^{-1})=i_G(\sigma)$.
    	\item $i_G(\sigma\tau)\ge\min\{i_G(\sigma),i_G(\tau)\}$.
    \end{enumerate}
    \begin{proof}
    	
    	前两件事已经证过了.特别的(a)说明$i_G(\sigma)=\min\{v_L(\sigma(a)-a)\mid a\in B\}$,这说明$i_G(\sigma)$的定义不依赖于满足$B=A[x]$的素元$x\in B$的选取.(c)是因为$i_G(\sigma\tau)=v_L(\sigma\tau(x)-x)=v_L(\sigma(\tau(x))-\tau(x)+\tau(x)-x)\ge\min\{v_L(\sigma(y)-y),v_L(\tau(x)-x)\}\ge\min\{i_G(\sigma),i_G(\tau)\}$.
    \end{proof}
    \item 设$H\subseteq G$是子群,它确定的中间域为$K'$,那么$L/K'$是Galois群为$H$的Galois扩张.所以依旧可以考虑函数$i_H$,以及讨论分歧群$H_s$.如果$H\subseteq G$是正规子群,那么$K'/K$是Galois群为$G/H$的Galois扩张,所以依旧可以考虑$i_{G/H}$和分歧群$(G/H)_s$.
    \begin{enumerate}
    	\item 设$H$是子群,对每个$\sigma\in H$,有$i_H(\sigma)=i_G(\sigma)$,并且总有$H_i=G_i\cap H$.
    	\begin{proof}
    		
    		任取$\sigma\in H$,有$i_H(\sigma)=\min\{v_L(\sigma(b)-b)\mid b\in B\}=i_G(\sigma)$.另外$\sigma\in H_i$当且仅当$i_H(\sigma)\ge s+1$,当且仅当$i_G(\sigma)\ge s+1$且$\sigma\in H$,当且仅当$\sigma\in H\cap G_i$.
    	\end{proof}
        \item 设$H$是正规子群,对每个$\sigma'\in G/H$,有如下等式:
        $$i_{G/H}(\sigma')=\frac{1}{e_{L/K'}}\sum\limits_{\sigma\mid K'=\sigma'}i_G(\sigma)$$
        \begin{proof}
        	
        	当$\sigma'=1$时等式两边都是无穷.再设$\sigma'\not=1$,考虑$w\mid L'$,这个赋值的赋值环记作$B'$.于是可取$b\in B'$使得$B'=A[\Pi']$.按照定义有$e'i_{L'/K}(\sigma)=v_L(\sigma'(\Pi')-\Pi')$.取$\sigma'$的一个提升$\sigma\in G(L/K)$,即有$\sigma\mid L'=\sigma'$.那么$\sigma'$的所有提升可表示为$\sigma\tau$,其中$\tau\in H=G(L/L')$.记$\alpha=\sigma(\Pi')-\Pi'$和$\beta=\prod_{\tau\in H}(\sigma\tau(\Pi)-\Pi)$,其中$\Pi$是满足$B=A[\Pi]$的素元.那么$v_L(\alpha)=i_{L'/K}(\sigma')$,$v_L(\beta)=\frac{1}{e'}\sum_{\sigma\mid L'=\sigma'}i_{L/K}(\sigma)$.于是我们要证明的是$v_L(\alpha)=v_L(\beta)$.
        	
        	设$f(x)\in B'[x]$是$\Pi\in B$在$B'$中的极小多项式,于是$f(x)=\prod_{\tau\in H}(x-\tau(\Pi))$.记$\sigma f(x)=\prod_{\tau\in H}(x-\sigma\tau(\Pi))$.那么$\sigma f-f$的系数一定都是$\sigma f(\Pi')-\Pi'$的倍式,于是$\alpha\mid\sigma f(X)-f(X)$,于是带入$\Pi$得到$u\mid\sigma f(\Pi)-f(\Pi)=\prod_{\tau\in H}(\Pi-\sigma\tau(\Pi))=\pm v$.于是$u\mid v$.
        	
        	反过来存在$g\in A[x]$使得$\Pi'=g(\Pi)$.于是$g(x)-\Pi'=f(x)h(x)$,其中$h(x)\in B'[x]$.作用$\sigma$得到$g(x)-\sigma(\Pi')=\sigma f(x)\sigma h(x)$.带入$x=\Pi$得到$\Pi'=\sigma(\Pi')=\sigma f(\Pi)\sigma h(\Pi)=\pm v\sigma h(\Pi)$.于是$v\mid u$.综上$u$和$v$相伴,于是它们赋值相同.
        \end{proof}
        \item 如果$H=G_s$,其中$s\ge0$是整数,那么$(G/H)_t=G_t/H,t\le s$和$(G/H)_t=\{e\},t\ge s$.
        \begin{proof}
        	
        	首先$\{G_t/H,t\le s\}$构成了$G/H$的有限滤过,任取$1\not=\sigma'\in G/H$,存在唯一的指标$t<s$,使得$\sigma'\in G_t/H$但$\sigma'\not\in G_{t+1}/H$.设$\sigma\in G$是$\sigma'$的提升,那么有$\sigma\in G_t$但$\sigma\not\in G_{t+1}$.因为$H\subseteq G_0$,说明$L/K'$是完全分歧的,所以$e_{L/K'}=|H|$.带入上一条得到$i_{G/H}(\sigma')=|H|(t+1)/|H|=t+1$.于是我们证明了$i_{G/H}(\sigma')=i_G(\sigma)$,这说明$(G/H)_t$和$G_t/H$是相同的滤过,这得到结论.
        \end{proof}
    \end{enumerate}
    \item 和共轭差积的关系.
    \begin{enumerate}
    	\item 设$\mathfrak{D}_{L/K}$是扩张$L/K$的共轭差积,那么有如下等式.我们证明过当$n$足够大时$|G_n|=1$,所以右侧这个无穷和有意义.
    	$$v_L(\mathfrak{D}_{L/K})=\sum_{\sigma\not=1}i_G(\sigma)=\sum_{n\ge0}(|G_n|-1)$$
    	\begin{proof}
    		
    		可取$x\in B$使得$B=A[x]$,记$x$的极小多项式为$f$,那么$\mathfrak{D}_{L/K}$是$f'(x)$生成的主理想,于是有:
    		$$v_L(\mathfrak{D}_{L/K})=v_L(f'(x))=\sum_{\sigma\not=1}v_L(\sigma(x)-x)$$
    		
    		如果记$r_n=|G_n|-1$,其中$n$是整数,那么$G$中使得$v_L(\sigma(x)-x)=n$的$\sigma\not=1$的元素个数是$r_{n-1}-r_n$,于是有:
    		$$\sum_{\sigma\not=1}v_L(\sigma(x)-x)=\sum_{n\ge0}n(r_{n-1}-r_n)=\sum_{n\ge0}r_n=\sum_{n\ge0}(|G_n|-1)$$
    	\end{proof}
        \item 设子群$H\subseteq G$确定的中间域是$K'$,取$K'$上的规范离散赋值$v_{K'}$,那么有:
        $$v_{K'}(\mathfrak{D}_{K'/K})=\frac{1}{e_{L/K'}}\sum_{\sigma\not\in H}i_G(\sigma)$$
        \begin{proof}
        	
        	按照上一条,我们有$v_L(\mathfrak{D}_{L/K})=\sum_{\sigma\not=1}i_G(\sigma)$和$v_L(\mathfrak{D}_{L/K'})=\sum_{1\not=\sigma\in H}i_G(\sigma)$.于是有:
        	\begin{align*}
        		v_{K'}(\mathfrak{D}_{K'/K})&=\frac{1}{e_{L/K'}}v_L(\mathfrak{D}_{K'/K})\\&=\frac{1}{e_{L/K'}}v_L(\mathfrak{D}_{L/K}\mathfrak{D}_{L/K'}^{-1})\\&=\frac{1}{e_{L/K'}}\left(\sum_{\sigma\not=1}i_G(\sigma)-\sum_{1\not=\sigma\in H}i_G(\sigma)\right)\\&=\frac{1}{e_{L/K'}}\sum_{\sigma\not\in H}i_G(\sigma)
        	\end{align*}
        \end{proof}
    \end{enumerate}
    \item 我们解释下整体情况如何约化到这里完备离散赋值域的情况.设$A$是戴德金整环,商域记作$E$,设$F/E$是有限Galois扩张,Galois群记作$G$,设$\mathfrak{P}\in\mathrm{Spec}B$,那么$\mathfrak{p}=\mathfrak{P}\cap A\in\mathrm{Spec}A$.那么分解群$D_{\mathfrak{P}}=\{\sigma\in G\mid\sigma(\mathfrak{P})=\mathfrak{P}\}$是完备化的扩张$F_{\mathfrak{P}}/E_{\mathfrak{p}}$的Galois群.如果这个完备化扩张的剩余域扩张是可分的,就可以定义$G$关于$\mathfrak{P}$的高阶分歧群$G_i(\mathfrak{P})$.于是对$\sigma\in G$有$\sigma\in G_s(\mathfrak{P})$当且仅当$\sigma(x)\equiv x\mathrm{mod}\mathfrak{P}^{s+1},\forall x\in B$.
    \item 我们有$G_0=I_w$是惯性群,$G_1=R_w$是分歧群.
    \begin{proof}
    	
    	注意到$R_w=\{\sigma\in G_w\mid\sigma(x)/x\equiv1(\mathrm{mod}n),\forall x\in L^*\}$中的$x\in L^*$可替换为$x\in B-\{0\}$,因为否则的话可以把$L^*$中的元乘以一个赋值足够大的$A$中的元使得结果落在$B-\{0\}$中.于是类似的对$\sigma\in R_w$.如果$v(x)\ge1$,从$\sigma(x)/x\equiv1(\mathrm{mod}n)$得到$\sigma(x)-x\in\Pi^2B$,这里$\Pi$满足$B=A[\Pi]$,于是$v_L(\sigma(x)-x)\ge2$;如果$v(x)=0$,那么$x=a_0+a_1\Pi+\cdots$,其中$a_i\in A$,于是这些系数被$\sigma$固定,于是$\sigma(x)-x$只有赋值$\ge2$的项,于是$v_L(\sigma(x)-x)\ge2$.反过来,如果$\sigma$满足$v_L(\sigma(x)-x)\ge2$,如果$v(x)\le1$,就有$v(\sigma(x)/x-1)\ge1$;对一般的$x\in B-\{0\}$,可记$x=\Pi^ny$,其中$v(y)=0$,于是$\sigma(x)/x=\left(\sigma(\Pi)/\Pi\right)^n(\sigma(y)/y)$,这里$\sigma(\Pi)/\Pi$落在$1+\Pi B$中,于是$\sigma(x)/x$落在$1+\Pi B$中,于是$v(\sigma(x)/x-1)\ge1$.
    \end{proof}
\end{enumerate}

关于商$G_s/G_{s+1},s\ge0$.
\begin{enumerate}
	\item 记$U_L^{(s)}$是第$s$个高阶单位群(乘法群),也即$U_L^{(s)}=1+\Pi^sB$,这里$\Pi$是$B$的一个素元.当$s\ge0$时我们有典范的同态$\theta_s:G_s/G_{s+1}\to U_L^{(s)}/U_L^{(s+1)}$为$\sigma\mapsto\frac{\sigma(\Pi)}{\Pi}$.我们断言这个映射不依赖于素元$\Pi$的选取,并证明它总是单射.另外我们可以具体表示出这个映射:如果$\sigma\in G_0$,那么存在$u\in B^*$使得$\sigma(\Pi)=u\Pi$,那么$\theta_0(\sigma)=\overline{u}\in\lambda$;如果$\sigma\in G_s,s\ge1$,那么$\sigma(\Pi)=\Pi(1+a)$,其中$a\in\mathfrak{n}^s$,那么$\theta_s(\sigma)$定义为$a$在$\mathfrak{n}^s/\mathfrak{n}^{s+1}$中的像.
	\begin{proof}
		
		先解释这个映射不依赖于$\Pi$的选取.条件下可记$B=A[a]$,其中$a$是$B$中的素元.于是有$\Pi=a_1a+a_2a^2+\cdots$.从$\sigma\in G_s$得到$v(\sigma(a)-a)\ge s+1$,于是有$v(\sigma(a^2)/a^2-1)\ge s$,于是$v(\sigma(a^2)-a^2)\ge s+2$.于是有$\sigma(\Pi)/\Pi=1+\frac{\sigma(a)-a}{\Pi}+\cdots$,这里省略的项的赋值都是$\ge s+1$的,于是在$\mathrm{mod}U^{(s+1)}$下就有$\sigma(\Pi)/\Pi\equiv1+\frac{\sigma(a)-a}{a}=\sigma(a)/a$.即这个映射不依赖于$\Pi$的选取.最后单射是因为,如果$\sigma(\Pi)/\Pi\in U^{(s+1)}$,等价于$\sigma(\Pi)=\Pi+\varepsilon\Pi^{s+2}$,等价于$\sigma\in G_{s+1}$.
	\end{proof}
	\item 如果$\mathrm{char}\lambda=p\not=0$,那么$G_0/G_1$是阶数与$p$互素的有限循环群.并且每个$G_s/G_{s+1},s\ge1$都是循环$p$阶群的直和,并且$G_s,s\ge1$都是$p$群,进而$G_1$是$G_0$的极大(也即唯一的Sylow)$p$群.另外我们断言$G_0$是一个阶数和$p$互素的循环群与一个阶数是$p$次幂的正规子群的半直积.
	\begin{proof}
		
		按照$U_L^{(0)}/U_L^{(1)}\cong\lambda^*$以及$\theta_0$是单射,说明$G_0/G_1$是域$\lambda$的乘法群的有限子群,所以它必须是循环群,另外按照$\lambda^*$的阶数和$p$互素,导致$G_0/G_1$的阶数和$p$互素.另外$G_s/G_{s+1},s\ge1$同构于$\lambda$加法群的子群,所以它是$\mathbb{F}_p$上的线性空间,所以它是$p$阶循环群的直和.最后$G_s$的阶数是$G_s/G_{s+1},G_{s+1}/G_{s+2},\cdots$的阶数的乘积,并且当$t$足够大时$G_t=\{e\}$,导致$G_s$的阶数是$p$的次幂,从而是$p$群.
		
		\qquad
		
		最后$G_1$是$p$群并且是$G_0$的正规子群,而$G_0/G_1$是循环群,所以要说明存在$G_0$的子群$H$和$G_0/G_1$同构.这是一个群论问题.设$\sigma\in G_0$使得它在$G_0/G_1$中的像是循环群的生成元,设$|G_0/G_1|=e_0$,设$|G_1|=p^n$.那么$(e_0,p)=1$,所以存在正整数$N$满足$p^N\equiv1\mathrm{mod}e_0$.适当把$N$替换为自身的次幂,可要求$N\ge n$.取$\tau=\sigma^{p^N}$,那么有$\tau^{e_0}=\sigma^{|G_0|b}=1$.另一方面因为$p^N\equiv1\mathrm{mod}e_0$,导致$\tau$在$G_0/G_1$中的像和$\sigma$的像相同.所以被$\tau$生成的$G_0$的循环子群$H$就典范的和$G_0/G_1$同构.
	\end{proof}
	\item 如果$\mathrm{char}\lambda=0$,那么$G_1=\{e\}$,并且$G_0$是有限循环群.
	\begin{proof}
		
		如果$s\ge1$,那么$U_L^{(s)}/U_L^{(s+1)}\cong\lambda$,这是一个特征零的加法群,它没有非平凡有限子群.而按照$\theta_s$是单射,迫使$G_s/G_{s+1}$是平凡群,于是$G_1=G_2=\cdots$,但是分歧群在指标足够大时平凡,就导致$G_1=\{e\}$.类似的$G_0/G_1$是循环群,就导致$G_0$是循环群.
	\end{proof}
	\item 群$G_0$是可解的.如果$\kappa$是有限群,那么$G$也是可解的.
	\begin{proof}
		
		第一件事是因为循环群,$p$群总是可解的,并且可解群的延拓还是可解的.第二件事是因为此时$G/G_0=\mathrm{Gal}(\lambda/\kappa)$是循环群.
	\end{proof}
	\item 推论.设$k$是特征零的代数闭域,设$K=k((T))$,那么$K$的代数闭包$\overline{K}$是$K_n=k((T^{1/n}))$的并,其中$n\ge1$.特别的,这说明$\mathrm{Gal}(\overline{K}/K)\cong\widehat{\mathbb{Z}}$.
	\begin{proof}
		
		先取$\overline{K}/K$的有限Galois子扩张$L$,设Galois群为$G$.有$K$的剩余域就是$k$,按照剩余域是代数闭的,说明$G=G_0$.按照$k$是特征零的,有$G=G_0$是有限循环群.取$n$是【】
	\end{proof}
\end{enumerate}

下分歧群,Herbrand定理.
\begin{enumerate}
	\item 定义$\varphi(s)=\varphi_{L/K}(s)=\int_0^s\frac{\mathrm{d}x}{[G_0:G_x]}$,其中$-1\le s\le0$时约定$[G_0:G_s]=[G_s:G_0]^{-1}$,也就是1.于是当$s=-1$时$t=-1$.这里被积函数是一个分段线性连续函数.于是如果$s>0$满足$m\le s<m+1$,就有$\eta_{L/K}(s)=\frac{1}{g_0}\left(g_1+g_2+\cdots+g_m+(s-m)g_{m+1}\right)$,其中$g_i$表示分歧群$G_i$的元素个数.$\varphi$是$[-1,+\infty)\to[-1,+\infty)$的连续,分段线性,严格递增,上凸的函数.他的逆映射记作$\psi$,这是$[-1,+\infty)\to[-1,+\infty)$的连续,分段线性,严格递增,下凸的函数.另外如果$t$是整数,那么$s=\psi(t)$也是整数.
	\begin{proof}
		
		只证明最后这件事.设$m\le s<m+1$,那么有$g_0t=g_1+\cdots+g_m+(s-m)g_{m+1}$.这里每个$g_1,\cdots,g_{m+1}$如果非零则都整除$g_0$,所以如果$t$是整数就导致$s$是整数.
	\end{proof}
	\item 我们定义上指标的分歧群(或称为下分歧群)为$G^t=G_{\psi(t)}$,或者等价的定义为$G^{\varphi(s)}=G_s$.那么有$G^{-1}=G=G_{-1}$,$G^0=G_0$,当$t$足够大时有$G^t=\{e\}$.另外有$\psi(t)=\int_0^t[G^0:G^x]\mathrm{d}x$.
	\item 引理.$\varphi_{L/K}(s)=\frac{1}{g_0}\sum_{\sigma\in G}\min\{i_{L/K}(\sigma),s+1\}-1$.
	\begin{proof}
		
		右侧同样是一个分段线性连续函数,记作$\theta(s)$,于是要证明两个分段连续线性函数相同,只需说明它们过一个相同的点,并且除折点外导函数处处相同.首先$\theta(-1)=\varphi_{L/K}(-1)=1$.接下来$\varphi'_{L/K}(s)=\frac{g_{m+1}}{g_0},m<s\le m+1$.而当$m<s\le m+1$时$m+1<s+1\le m+2$,此时$i_{L/K}(\sigma)\ge m+2$,于是$\sigma\in G_{m+1}$,于是$\theta'(s)=\frac{g_{m+1}}{g_0}$.
	\end{proof}
	\item Herbrand定理,即$G_sH/H=G_t(L'/K)=(G/H)_t$,其中$t=\varphi_{L/L'}(s)$.另外按照群同态定理,$G_s(L/K)H/H=G_s(L/K)/H\cap G_s(L/K)=G_s(L/K)/H_s(L/K)$.
	\begin{proof}
		
		先约定记号,记$G=G(L/K)$,$H=G(L/L')$,$G'=G(L'/K)$.对任意$\sigma'\in G'$,记$\sigma\in G$是它的提升,使得$i_{L/K}(\sigma)$取最大的.我们断言$i_{L'/K}(\sigma')-1=\varphi_{L/L'}(i_{L/K}(\sigma)-1)$.一旦这件事得证,任取$\sigma\in G_s$,等价于$i_{L/K}(\sigma)-1\ge s$.于是断言的这件事说明它又等价于$i_{L'/K}(\sigma')-1\ge t$,因为$\varphi$是单调增函数.这说明它等价于$\sigma'\in G_t$.这个等价就说明$G_t$的原像在$G_s$里,而$G_s$的像在$G_t$中.
		
		\qquad
		
		最后证明我们的断言.记$m=i_{L/K}(\sigma)$,设$\tau\in H_{m-1}=G_{m-1}(L/L')=G_{m-1}(L/K)\cap H$,于是$i_{L/K}(\tau)\ge m$.另外按照$\sigma\tau(\Pi)-\Pi=\left(\sigma\tau(\Pi)-\sigma(\Pi)\right)+\left(\sigma(\Pi)-\Pi\right)$,前一个括号的赋值$\ge m$(因为是一个赋值$\ge m$的元的共轭元),后一个赋值就是$m$,于是它们的和的赋值$\ge m$.但是我们选取的$\sigma$是$\sigma'$的提升在$i$函数下取最大值的,于是这只能得到$i_{L/K}(\sigma\tau)=m$.再设$\tau\not\in H_{m-1}$,那么$i_{L/K}(\tau)<m$,于是有$i_{L/K}(\sigma\tau)=i_{L/K}(\tau)<m$.整理得$i_{L/K}(\sigma\tau)=\min\{m,i_{L/K}(\tau)\}$,其中$\tau\in H$.于是得到$i_{L'/K}(\sigma')=\frac{1}{e'}\sum_{\tau\in H}\min\{m,i_{L/K}(\tau)\}$.
		
		现在$\varphi_{L/L'}(i_{L/K}(\sigma)-1)=\frac{1}{g_0(L/L')}\sum_{\tau\in H}\min\{i_{L/K}(\tau),m\}-1$.这里$g_0(L/L')=e'$,于是结合上段最后等式得到$i_{L'/K}(\sigma')-1=\varphi_{L/L'}(i_{L/K}(\sigma)-1)$.
	\end{proof}
	\item 推论.设$L'$是$L/K$的中间域,那么有传递公式:
	$$\varphi_{L/K}=\varphi_{L'/K}\circ\varphi_{L/L'},\psi_{L/K}=\psi_{L/L'}\circ\psi_{L'/K}$$
	\begin{proof}
		
		有$e=e_{L'/K}e'$,按照Herbrand定理有$G_s/H_s=(G/H)_t$,其中$t=\varphi_{L/L'}(s)$.考虑商群的元素个数,我们有$\frac{1}{e}|G_s|=\frac{1}{e_{L'/K}}|(G/H)_t|\frac{1}{e'}|H_s|$.这个等式左侧恰好是$\varphi_{L/K}(s)$的导数,右侧恰好是复合函数$\varphi_{L'/K}\circ\varphi_{L/L'}$的导数,而这两个原函数经过原点,这就得到它们是相同的函数(因为都是分段连续线性函数).第二个等式取逆映射即可.
	\end{proof}
	\item 按照上指标的记号Herbrand定理即$G^tH/H=(G/H)^t$,换句话讲上指标的分歧群和商群相适应,下指标的分歧群和子群相适应.
	\begin{proof}
		
		记$u=\psi_{L'/K}(t)$和$s=\psi_{L/L'}(u)=\psi_{L/K}(t)$(用到了传递性),那么有$(G/H)^t=(G/H)_u$,用Herbrand定理得到$(G/H)_u=G_sH/H=G^tH/H$.
	\end{proof}
    \item 设$G$是某个Galois扩张的Galois群,我们已经定义了上指标的分歧群$G^t$,称一个$\ge-1$的实数$t$是跳跃点,如果对任意$\varepsilon>0$有$G^t\not=G^{t+\varepsilon}$.那么一般来讲即便对于有限Galois扩张,上指标的跳跃点未必是整数.但是我们有Hasse-Arf定理:如果$G$是阿贝尔Galois群,那么$\{G^t\}$的跳跃点都是整数.
\end{enumerate}

例子.记$K=\mathbb{Q}_p$,用$K_n$表示$K$添加$n=p^m$阶本原单位根得到的扩域.我们给出过此时$K_n/K$的性质,它是$\varphi(n)=p^{m-1}(p-1)$阶完全分歧扩张,它的Galois群$G(n)\cong(\mathbb{Z}/n\mathbb{Z})^*$.我们来描述它的分歧群.记$G(n)^v=\{a\in(\mathbb{Z}/n\mathbb{Z})^*\mid a\equiv1(\mathrm{mod}p^v)\}$,那么它的中间域是$K_{p^v}$,所以$G(n)^v=\mathrm{Gal}(K_n/K_{p^v})$.我们断言$G_u$是:
$$\left\{\begin{array}{cc}G_u=G&-1\le u<1\\G_u=G(n)^1=\mathrm{Gal}(K_n/K_{p^1})&1\le u<p\\G_u=G(n)^2=\mathrm{Gal}(K_n/K_{p^2})&p\le u<p^2\\\cdots&\cdots\\G_u=G(n)^m=\{e\}&p^{m-1}\le u\end{array}\right.$$
\begin{proof}
	
	取$1\not=a\in G(n)$,设$v$是最大的整数使得$a\equiv1(\mathrm{mod}p^v)$,那么有$a\in G(n)^v$和$a\not\in G(n)^{v+1}$.另外有$i_G(a)=v_{K_n}(a(\xi)-\xi)=v_{K_n}(\xi^a-\xi)=v_{K_n}(\xi^{a-1}-1)$.其中$\xi$是$p^n$次本原单位根.这里$\xi^{a-1}$是$p^{m-v}$次本原单位根,所以$\xi^{a-1}-1$是$K_{p^{m-v}}$的uniformizer,所以有$i_G(a)=[K_n:K_{p^{m-v}}]=p^v$.所以$a\in G_u$当且仅当$p^v\ge u+1$,这得到$G(n)^v=G_v$.
\end{proof}
\newpage
\subsection{范数}

设$(K,v_K)$是规范完备离散赋值域,赋值环记作$(A,\mathfrak{m})$.设$L/K$是有限Galois扩张,设$A$在$L$中的整闭包是$(B,\mathscr{n})$,记$A$和$B$的单位群分别是$U_K$和$U_L$.那么域扩张的范数映射$\mathrm{N}$可限制为$L^*\to K^*$的映射,也可限制为$U_L\to U_K$的映射.按照$v_K(\mathrm{N}(x))=nw(x)=fv_L(x)$,其中$n=[L:K]$,$f$是惯性次数.于是有如下短正合列之间的交换图表:
$$\xymatrix{0\ar[r]&U_L\ar[r]\ar[d]_{\mathrm{N}}&L^*\ar[r]\ar[d]_{\mathrm{N}}&\mathbb{Z}\ar[d]_f\ar[r]&0\\0\ar[r]&U_K\ar[r]&K^*\ar[r]&\mathbb{Z}\ar[r]&0}$$

设$L/K$是非分歧扩张.
\begin{enumerate}
	\item 范数映射$\mathrm{N}$把$U_L^{(n)}$映入$U_K^{(n)},\forall n$.
	\begin{proof}
		
		设$x=1+y$,其中$y\in\mathfrak{n}^n$,对任意$\sigma\in G=\mathrm{Gal}(L/K)$,有$\sigma(x)=1+\sigma(y)$,其中$\sigma(y)\in\mathfrak{n}^n$.于是在$\mathrm{mod}\mathfrak{n}^{2n}$下有$\mathrm{N}(x)=\prod_{\sigma\in G}(1+s(y))\equiv1+\sum_{\sigma\in G}$.最后按照$L/K$是非分歧的,就有$\mathfrak{n}^n\cap K=\mathfrak{m}^n$.于是$\mathrm{N}(x)\equiv1(\mathrm{mod}\mathfrak{m}^n)$,也即$\mathrm{N}(x)\in U_K^n$.
	\end{proof}
    \item 于是对于非分歧扩张,范数诱导了商映射$\mathrm{N}_n:U_L^n/U_L^{n+1}\to U_K^n/U_K^{n+1}$.我们之前解释过$n=0$时高阶单位群的商就是剩余域的乘法群,$n\ge1$时高阶单位群的商就是剩余域的加法群.容易验证有如下交换图表:
    $$\xymatrix{U_L/U_L^1\ar[rr]^{\cong}\ar[d]_{\mathrm{N}_0}&&(\kappa(L)^*,\times)\ar[d]^{\mathrm{N}_{\kappa(L)/\kappa(K)}}\\U_K/U_K^1\ar[rr]^{\cong}&&(\kappa(K)^*,\times)}\qquad\xymatrix{U_L^n/U_L^{n+1}\ar[rr]^{\cong}\ar[d]_{\mathrm{N}_n}&&(\kappa(L),+)\ar[d]^{\mathrm{Tr}_{\kappa(L)/\kappa(K)}}\\U_K^n/U_K^{n+1}\ar[rr]^{\cong}&&(\kappa(K),+)}$$
    
    换句话讲,我们有如下短正合列的交换图表:
    $$\xymatrix{0\ar[r]&U_L^1\ar[r]\ar[d]_{\mathrm{N}}&U_L\ar[r]\ar[d]_{\mathrm{N}}&(\kappa(L)^*,\times)\ar[r]\ar[d]_{\mathrm{N}}&0\\0\ar[r]&U_K^1\ar[r]&U_K\ar[r]&(\kappa(K)^*,\times)\ar[r]&0}\quad\xymatrix{0\ar[r]&U_L^{n+1}\ar[r]\ar[d]_{\mathrm{N}}&U_L^n\ar[r]\ar[d]_{\mathrm{N}}&(\kappa(L),+)\ar[r]\ar[d]_{\mathrm{Tr}}&0\\0\ar[r]&U_K^{n+1}\ar[r]&U_K^n\ar[r]&(\kappa(K),+)\ar[r]&0}$$
    \begin{proof}
    	
    	以第二个交换图为例.设$\pi$是$K$的素元,按照扩张是非分歧的,有$\pi$也是$L$的素元.设$G=\mathrm{Gal}(L/K)=\{\sigma_1,\cdots,\sigma_s\}$.任取$1+\pi^na\in U^n_L/U^{n+1}_L$,那么它同构到$\kappa(L)^*$中为$a$,再取迹得到$\mathrm{Tr}(a)$.另一方面$\mathrm{N}(1+\pi^na)=\prod_{\sigma}(1+\pi^n\sigma(a))=1+\pi^n\sum_{\sigma}\sigma(a)+\cdots$,这里余项落在$\mathfrak{n}^{2n}$中,所以在$\mathrm{mod}U_L^{n+1}$下有$\mathrm{N}(1+\pi^na)=1+\pi^n\sum\sigma(a)$,它同构到$\kappa(K)^*$中为$\sum\sigma(a)$.按照$\mathrm{Gal}(L/K)\to\mathrm{Gal}(\kappa(L)/\kappa(K))$是同构,就得到这里$\sum\sigma(a)=\mathrm{Tr}(a)$,得证.
    \end{proof}
    \item 对上述两个交换图表用蛇形引理,得到:
    \begin{enumerate}
    	\item $\mathrm{N}(U_L^n)=U_K^n,n\ge1$.
    	\item $U_K/\mathrm{N}(U_L)\cong\kappa(K)^*/\mathrm{N}(\kappa(L)^*)$.
    	\item $K^*/\mathrm{N}L^*\cong\mathbb{Z}/f\mathbb{Z}\times\kappa(K)^*/\mathrm{N}(\kappa(L)^*)$,其中$f=[L:K]=[\kappa(L):\kappa(K)]$.
    \end{enumerate}
    \begin{proof}
    	
    	我们来解释最后这件事.取定$K$的一个素元$\pi$,那么$K^*$在$G$作用下同构于$\mathbb{Z}\times U_K$,类似的$L^*$在$G$作用下同构于$\mathbb{Z}\times U_L$.于是有:
    	$$K^*/\mathrm{N}(L^*)=\mathbb{Z}\times U_K/\mathrm{N}(\mathbb{Z}\times U_L)=\mathbb{Z}\times U_K/f\mathbb{Z}\times\mathrm{N}U_L\cong\mathbb{Z}/f\mathbb{Z}\times U_K/\mathrm{N}U_L\cong\mathbb{Z}/f\mathbb{Z}\times\kappa(K)^*/\mathrm{N}(\kappa(L)^*)$$
    \end{proof}
    \item 推论.如下三个条件互相等价,并且如果$\kappa(K)$是有限域,这个条件是满足的.
    \begin{enumerate}
    	\item $[K^*:\mathrm{N}L^*]=f$.
    	\item $U_K=\mathrm{N}U_L$.
    	\item $\kappa(K)^*=\mathrm{N}(\kappa(L)^*)$.
    \end{enumerate}
\end{enumerate}

设$L/K$是素数$l$阶完全分歧的循环扩张,按照完全分歧的定义此时剩余域是相同的.
\begin{enumerate}
	\item 基本性质.
	\begin{enumerate}
		\item 设$\sigma$是循环群$G=\mathrm{Gal}(L/K)$的生成元,记$t=i(\sigma)-1$,那么$G$的分歧群是$G=G_0=\cdots=G_t$和$\{e\}=G_{t+1}=G_{t+2}=\cdots$.
		\item 我们解释过$G_0/G_1$是与$p=\mathrm{char}\kappa$互素的有限循环群,所以$t=0$当且仅当$G_0/G_1$是$l$阶循环群,当且仅当$l\not=p$.反过来$l=p$当且仅当$t>0$.
		\item 函数$\varphi$和$\psi$是:
		$$\varphi(x)=\left\{\begin{array}{cc}x&-1\le x\le t\\t+\frac{x-t}{l}&x\ge t\end{array}\right.\qquad\psi(x)=\left\{\begin{array}{cc}x&-1\le x\le t\\t+l(x-t)&x\ge t\end{array}\right.$$
		\item $L/K$的共轭差积是$\mathfrak{n}^m$,其中$m=(t+1)(l-1)$.
		\item 对任意$n\ge0$有$\mathrm{Tr}(\mathscr{n}^n)=\mathfrak{m}^r$,其中$r=[(m+n)/l]$,而$m$和上一条一样是$(t+1)(l-1)$.
		\begin{proof}
			
			我们之前解释过$\mathrm{Tr}(\mathscr{n}^n)\subseteq\mathfrak{m}^r$当且仅当$\mathfrak{n}^n\subseteq\mathfrak{m}^r\mathfrak{D}^{-1}=\mathfrak{m}^{lr-m}$,也即当且仅当$r\le(m+n)/l$.
		\end{proof}
	    \item 如果$x\in\mathfrak{n}^n$,有$\mathrm{N}(1+x)\equiv1+\mathrm{Tr}(x)+\mathrm{N}(x)\mathrm{mod}\mathrm{Tr}(\mathfrak{n}^{2n})$.
	    \begin{proof}
	    	
	    	记$G=\mathrm{Gal}(L/K)$,对$u=\sum_ik_i\sigma_i\in\mathbb{Z}[G],x\in L$,记$x^u=\prod_i\sigma_i(x)^{k_i}$.那么$\mathrm{N}(1+x)=\prod_{\sigma\in G}(1+\sigma(x))=\sum x^u$,其中$u$取遍$\mathbb{Z}[G]$的系数为0或1的元,换句话讲$u$可以表示为$\sigma_{i_1}+\cdots+\sigma_{i_k}$,其中$\sigma_{i_j}$是两两不同的$G$中的元.记$n(u)=k$表示$u$的长度.那么长度0,1,l的$u$对$x^u$求和分别为$1$,$\mathrm{Tr}(x)$和$\mathrm{N}(x)$.所以问题归结为对$2\le n(u)\le l-1$的$x^u$求和落在$\mathrm{Tr}(\mathfrak{n}^{2n})$中.但是对这样的$u$有$\sigma u\not=u$,其中$\sigma$是$G$的生成元.于是$\{s^iu,0\le i\le l-1\}$作用在$x$上求和得到$\mathrm{Tr}(x^u)$,因为$n(u)\ge2$,导致$x^u\in\mathfrak{n}^{2n}$.得证.
	    \end{proof}
	\end{enumerate}
	\item 对每个自然数$n\ge0$,有$\mathrm{N}(U_L^{\psi(n)})\subseteq U_K^n$和$\mathrm{N}(U_L^{\psi(n)+1})\subseteq U_K^{n+1}$.于是范数诱导了商同态$\mathrm{N}_n:U_L^{\psi(n)}/U_L^{\psi(n)+1}\to U_K^n/U_K^{n+1},n\ge0$.它们满足:
	\begin{enumerate}
		\item 对$n=0$,$\mathrm{N}_0$是$(\kappa^*,\times)$上的映射$\xi\mapsto\xi^l$.和之前的记号一样,$t$表示最大的正整数,使得它作为指标时分歧群是非平凡的.如果$t\not=0$,那么$\mathrm{N}_0$是单射.如果$t=0$,那么$\mathrm{N}_0$的核是$l$阶循环群,并且这个核就是$G$在映射$\theta_0:G\to U_L/U_L^1$,$\sigma\mapsto(\sigma(\pi)/\pi)$(我们之前解释过这个映射不依赖于素元$\pi$的选取)的像.
		\item 对$1\le n\le t-1$,映射$\mathrm{N}_n$是$(\kappa,+)$上的映射$\xi\mapsto\alpha_n\xi^p$,其中$\alpha_n\in\kappa^*$.这是一个单射.另外我们解释过$t>0$当且仅当$p=l$.
		\item 如果$n=t\ge1$,映射$\mathrm{N}_n$是$(\kappa,+)$上的映射$\xi\mapsto\alpha\xi^p+\beta\xi$,其中$\alpha,\beta\in\kappa^*$.它的核是一个$p=l$阶循环群,是$\theta_t(G)$,其中$\theta_t:G\to U_L^n/U_L^{n+1}$是$\sigma\mapsto\sigma(\pi)/\pi$,我们解释过这个映射不依赖素元$\pi$的选取.
		\item 对$n>t$,映射$\mathrm{N}_n$是$(\kappa,+)$上的映射$\xi\mapsto\beta_n\xi$,其中$\beta_n\in\kappa^*$,此时映射是同构.
	\end{enumerate}
	\begin{proof}
		
		如果$n=0$.此时明显有$\mathrm{N}(U_L)\subseteq U_K$和$\mathrm{N}(U_L^1)\subseteq U_K^1$.并且$\mathrm{N}_0(\xi)=\xi^l$.如果$t\not=0$,我们解释过$l=p$是剩余域的特征,于是$\mathrm{N}_0$是单射.如果$t=0$,我们解释过$l\not=p$,那么此时核的阶数$\le l$.但是$\theta_0(G)$中的元在$\kappa^*$中可以表示为$\sigma(\pi)/\pi$,它的范数是1,所以$\theta_0(G)=\theta_0(G_0/G_1)$包含在$\ker\mathrm{N}_0$中,前者的阶数恰好是$l$,所以$\theta_0(G)$恰好是$\ker\mathrm{N}_0$.
		
		\qquad
		
		如果$1\le n\le t-1$.这种情况下$p=l$,并且有$\psi(n)=n$.按照完全分歧有$v_K\circ\mathrm{N}=ew=v_L$,于是有$\mathrm{N}(\mathfrak{n}^n)\subseteq\mathfrak{m}^n$.另外我们解释过$\mathrm{Tr}(\mathfrak{n}^n)\subseteq\mathfrak{m}^r$,其中$r=\left[\frac{(t+1)(l-1)+n}{l}\right]=\left[n+2-\frac{2}{l}\right]\ge n+1$.类似的有$\mathrm{Tr}(\mathfrak{n}^{2n})\subseteq\mathfrak{m}^{n+1}$.于是有$\mathrm{N}(1+x)\equiv1+\mathrm{N}(x)\mathrm{mod}\mathfrak{m}^{n+1}$,其中$x\in\mathfrak{n}^n$.综上$\mathrm{N}$把$U_L^n$映入$U_K^n$,把$U_L^{n+1}$映入$U_K^{n+1}$.接下来如果设$\pi$是$L$上的素元,设$x=u\pi^n$,其中$u$是$K$中单位,为了确定$\mathrm{N}$,我们只要看$\mathrm{N}(1+u\pi^n)$即可.有$\mathrm{N}(x)=u^p\mathrm{N}(\pi)^n$.按照$v_L=v_K\circ\mathrm{N}$,就有$\mathrm{N}(\pi)^n=\alpha_n\pi_K^n$,其中$\pi_K$是$K$中一个素元.那么在$\mathrm{mod}\mathfrak{m}^{n+1}$下有$\mathrm{N}(1+u\pi^n)\equiv1+a_nu^p\pi_K^n$,换句话讲把$\mathrm{N}$视为$(\kappa,+)$上的映射是$u\mapsto\alpha_nu^p$.
		
		\qquad
		
		如果$n=t\ge1$.此时也有$p=l$,并且$\psi(t)=t$.类似上一段的证明有$\mathrm{mod}\mathfrak{m}^{n+1}$下有$\mathrm{N}(1+x)\equiv1+\mathrm{Tr}(x)+\mathrm{N}(x)$,其中$x\in\mathfrak{n}^n$.由于$\mathrm{Tr}(\mathfrak{n}^n)\subseteq\mathfrak{m}^n$和$\mathrm{Tr}(\mathfrak{n}^{n+1})\subseteq\mathfrak{m}^{n+1}$.于是$\mathrm{N}$把$U_L^n$映入$U_K^n$,把$U_L^{n+1}$映入$U_K^{n+1}$.设$\pi$是$L$上素元,设$u$是$K$上单位,为了确定$\mathrm{N}$我们只需计算$\mathrm{N}(1+u\pi^t)$.记$\pi_K$为$K$中素元,记$\mathrm{Tr}(\pi^t)=b\pi_K^t$和$\mathrm{N}(\pi)^t=a\pi_K^t$,其中$a,b$是$A$中元,并且这里$a$是$A$的单位.那么在$\mathrm{mod}\mathfrak{m}^{t+1}$下就有$\mathrm{N}(1+u\pi^t)\equiv1+(\beta u+\alpha u^p)\pi_L^t$,其中$\alpha,\beta$是$a,b$在$\kappa=A/\mathfrak{m}$中的像.于是$\mathrm{N}_t$是把$\xi\mapsto\beta\xi+\alpha\xi^p$.因为$a$是$A$中单位导致$\alpha\not=0$,我们断言$\beta$也$\not=0$,因为如果$\beta=0$,导致$\mathrm{N}_t$是单射,但是$\ker\mathrm{N}_t$至少包含$\theta_t(G)$,这是一个$p$阶循环群,矛盾.于是$\beta\not=0$,此时$\mathrm{N}_t$的核的阶数$\le p$,所以它必须恰好是$\theta_t(G)$.
		
		\qquad
		
		如果$n>t$.我们有$\psi(n)=t+l(n-t)$.计算共轭差积得到$\mathrm{Tr}(\mathfrak{n}^{\psi(n)})=\mathfrak{m}^n$.另外有$\mathrm{N}(\mathfrak{n}^{\psi(n)})\subseteq\mathfrak{m}^{\psi(n)}\subseteq\mathfrak{m}^{n+1}$中.于是在$\mathrm{mod}\mathfrak{m}^{n+1}$下就有$\mathrm{N}(1+x)\equiv1+\mathrm{Tr}(x)$.于是$\mathrm{N}_n$把$U_L^{\psi(n)}$映入$U_K^n$,把$U_L^{\psi(n)+1}$映入$U_K^{n+1}$.于是类似之前的情况有$\beta_n\in\kappa$使得$\mathrm{N}_n(\xi)=\beta_n\xi$.倘若$\beta_n=0$,按照上述同余式就有$\mathrm{Tr}(\mathfrak{n}^{\psi(n)})\subseteq\mathfrak{m}^{n+1}$.但是这和$\mathrm{Tr}(\mathfrak{n}^{\psi(n)})=\mathfrak{m}^n$矛盾.于是$\beta_n\not=0$,于是$\mathrm{N}_n$是双射.
	\end{proof}
    \item 引理.设$A$和$A'$是阿贝尔群,分别有递降的子群链$\{A_n\}$和$\{A_n'\}$,设$A_0=A$和$A_0'=A'$.设这两个滤过诱导的拓扑使得$A$和$A'$是完备Hausdorff空间,这个条件等价于讲典范映射$A\to\lim\limits{\leftarrow}A/A_n$和$A'\to\lim\limits_{\leftarrow}A'/A_n'$都是同构.现在设$u:A\to A'$是群同态,满足$u(A_n)\subseteq A_n'$.如果$u$诱导的同态$u_n:A_n/A_{n+1}\to A_n'/A_{n+1}'$都是单射或者都是满射,那么$u$是单射或者是满射.
	\item 推论.
	\begin{enumerate}
		\item 如果$n\not=t$,就有$\mathrm{N}_n$是单射.在$n=t$时有如下正合列:
		$$\xymatrix{0\ar[r]&G\ar[r]^{\theta_t}&U_L^t/U_L^{t+1}\ar[r]^{\mathrm{N}_t}&U_K^t/U_K^{t+1}}$$
		\item $\mathrm{N}_n$在$n>t$时是满射,如果$\kappa$是完全域,那么$n<t$时也是满射,如果$\kappa$是代数闭域,那么对所有$n$都是满射.
		\item 对$n>t$有$\mathrm{N}(U_L^{\psi(n)})=U_K^n$.对$n\ge t$有$\mathrm{N}(U_L^{\psi(n)+1})=U_K^{n+1}$.如果$\kappa$是代数闭域,那么这个命题对所有$n$都成立.
		\begin{proof}
			
			取$U_L^{\psi(n)}$的递降滤过$\{U_L^{\psi(m)},m\ge n\}$,取$U_K^n$的递降滤过$\{U_K^m,m\ge n\}$.考虑典范映射的如下复合:
			$$\xymatrix{U_L^{\psi(m)}/U_L^{\psi(m+1)}\ar[r]&U_L^{\psi(m)}/U_L^{\psi(m)+1}\ar[r]^{\mathrm{N}_m}&U_L^m/U_L^{m+1}}$$
			
			如果$m\ge n>t$,我们解释了$\mathrm{N}_m$是满射,于是上述复合映射是满射,按照引理就有$\mathrm{N}:U_L^{\psi(n)}\to U_K^n$是满射.类似的如果$\kappa$是代数闭域,那么对每个$n$有上述复合映射是满的,于是此时$\mathrm{N}:U_L^{\psi(n)}\to U_K^n$是满射.最后按照$U_K^{n+1}=\mathrm{N}(U_L^{\psi(n+1)})\subseteq\mathrm{N}(U_L^{\psi(n)+1})\subseteq U_K^{n+1}$,得到$\mathrm{N}(U_L^{\psi(n)+1})=U_K^{n+1}$.
		\end{proof}
	    \item 对非负实数$v$,记号$U_L^v$表示$U_L^n$,其中$n$表示$\ge v$的最小整数,这吻合于整数的定义.那么对实数$v>t$有$\mathrm{N}(U_L^{\psi(v)})=U_K^v$.如果$\kappa$是代数闭域,那么对所有正实数$v$成立.
	    \begin{proof}
	    	
	    	设$n<v\le n+1$,其中$n$是整数,那么$\psi(n)<\psi(v)\le\psi(n+1)$.设$m$是$\ge\psi(v)$的最小整数,那么有$\psi(n)+1\le m\le\psi(n+1)$(因为$\psi(n)$都是整数),于是上一条得到在相应的条件下有$\mathrm{N}(U_L^{\psi(v)})=\mathrm{N}(U_L^m)=U_K^{n+1}=U_K^v$.
	    \end{proof}
        \item 如果$t=0$,那么$\mathrm{coker}\mathrm{N}_t=\kappa^*/(\kappa^*)^l$.如果$t\not=0$,那么$\mathrm{coker}\mathrm{N}_t=\kappa/\mathscr{P}(\kappa)$,其中$\mathscr{P}:\kappa\to\kappa$为$\xi\mapsto\xi^p-\xi$.
        \begin{proof}
        	
        	设$t\not=0$,只需证明$\mathrm{coker}\mathrm{N}_t=\mathrm{coker}\mathscr{P}$.我们给出过$\mathrm{N}_t(\xi)=\alpha\xi^p+\beta\xi$,其中$\alpha,\beta\not=0$.任取一个非零元$\eta\in\ker\mathrm{N}_t$,那么$\mathrm{N}_t(\xi)=\alpha\eta^p((\xi/\eta)^p-\xi/\eta)=\gamma\mathscr{P}(\xi/\eta)$.所以有$\mathrm{Im}\mathrm{N}_t=\gamma\mathrm{Im}\mathscr{P}$.所以两个余核同构.
        \end{proof}
        \item 如果$\kappa$是完全域,对$n\le t$,有$\mathrm{N}$诱导的$U_L/U_L^n\to U_K/U_K^n$是同构.
        \item 我们有如下三个典范同构:
        \begin{itemize}
        	\item $U_K^t/\mathrm{N}(U_L^t)\to\mathrm{coker}\mathrm{N}_t$是同构.考虑如下短正合列的交换图表:
        	$$\xymatrix{0\ar[r]&U_L^{t+1}\ar[r]\ar[d]_{\mathrm{N}}&U_L^t\ar[r]\ar[d]_{\mathrm{N}}&U_L^t/U_L^{t+1}\ar[r]\ar[d]_{\mathrm{N}_t}&0\\0\ar[r]&U_K^{t+1}\ar[r]&U_K^t\ar[r]&U_K^t/U_K^{t+1}\ar[r]&0}$$
        	
        	因为最左侧的垂直映射是满射,蛇形引理就得到中间和右侧的垂直映射的余核是典范同构的,也即$U^t_K/\mathrm{N}(U^t_L)\to\mathrm{coker}\mathrm{N}_t$是同构.
        	\item $U_K^t/\mathrm{N}(U_L^t)\to U_K/\mathrm{N}(U_L)$是同构.考虑如下短正合列的交换图表:
        	$$\xymatrix{0\ar[r]&U_L^t\ar[r]\ar[d]_{\mathrm{N}}&U_L\ar[r]\ar[d]_{\mathrm{N}}&U_L/U_L^t\ar[r]\ar[d]_{\alpha}&0\\0\ar[r]&U_K^t\ar[r]&U_K\ar[r]&U_K/U_K^t\ar[r]&0}$$
        	
        	我们解释了这里$\alpha$是同构,蛇形引理就得到左侧和中间的垂直映射的余核是典范同构的,也即$U_K^t/\mathrm{N}(U_L^t)\to U_K/\mathrm{N}(U_L)$是同构.
        	\item $U_K/\mathrm{N}(U_L)\to K^*/\mathrm{N}(L^*)$是同构.考虑如下短正合列的交换图表:
        	$$\xymatrix{0\ar[r]&U_L\ar[r]\ar[d]_{\mathrm{N}}&L^*\ar[r]\ar[d]_{\mathrm{N}}&\mathbb{Z}\ar[r]\ar[d]_{f=1}&0\\0\ar[r]&U_K\ar[r]&K^*\ar[r]&\mathbb{Z}\ar[r]&0}$$
        	
        	这里$f=1$所以最右侧垂直映射是同构,蛇形引理就得到中间和左侧的垂直映射的余核是典范同构的,也即$U_K/\mathrm{N}(U_L)\to K^*/\mathrm{N}(L^*)$是同构.
        \end{itemize}
    
        特别的如果$\kappa$是有限域,如果$t=0$,也即$l$和$p=\mathrm{char}\kappa$互素,我们解释过$\mathrm{coker}\mathrm{N}_t=\kappa^*/(\kappa^*)^l$,这是一个$l$阶循环群;如果$t>0$,也即$l=p$,我们解释过$\mathrm{coker}\mathrm{N}_t=\kappa/\mathscr{P}(\kappa)$,它也是$l$阶循环群.于是在$\kappa$是有限域的情况下上述四个群都是$l$阶循环群.
	\end{enumerate}
\end{enumerate}

完全分歧扩张下的剩余域扩张.设$L/K$是有限完全分歧扩张,设剩余域$\kappa=\kappa(K)=\kappa(L)$是完全域.
\begin{enumerate}
	\item 设$\kappa'/\kappa$是有限扩张(所以是可分的,因为$\kappa$是完全域),那么存在有限非分歧扩张$K'/K$,使得对应的剩余域扩张就是$\kappa'/\kappa$.记$L'=K'L$,我们会证明$L'$典范同构于$K'\otimes_KL$(也即扩张$L/K$和$K'/K$是线性无交的),所以扩张$L'/K'$类似于代数几何中纤维积变换,它的只依赖$L/K$的性质可以视为某种几何不变性.
	\begin{enumerate}
		\item 扩张$L/K$和$K'/K$是线性无交的.
		\begin{proof}
			
			我们有$e(L'/K)\ge e(L/K)=[L:K]$和$f(L'/K)\ge f(K'/K)=[K':K]$.于是$[L':K]\ge[L:K][K':K]$.而另一侧的不等式是总成立的.
		\end{proof}
	    \item $L'/L$也是非分歧扩张,对应的剩余域扩张也是$\kappa'/\kappa$.所以取正向极限得到极大非分歧扩张满足$L_{\mathrm{nr}}=L\otimes_KK_{\mathrm{nr}}$.
	    \item 如果$L/K$是Galois扩张,那么$L'/K'$也是Galois扩张.因为如果记$L=K[X]/(f(X))$,其中$f(X)$是可分多项式,并且在$L$中分裂,那么$L'=K'\otimes_KL=K'\otimes_KK[X]/(f(X))\cong K'[X]/f(X)K'[X]$.因为线性无交性保证它已经是域,所以$f(X)$是$K'[X]$上的不可约多项式,它是可分的,它在$L$上已经分裂了,所以在$L'$中分裂.
	    \item 如果$L/K$是Galois扩张,那么$\mathrm{Gal}(L/K)$典范同构于$\mathrm{Gal}(L'/K')$,并且相应的分歧群和$\theta_i$都是相同的.
	    \item 如果$G$还是素数$l$阶循环群,取定$K$的素元$\pi_K$和$L$的素元$\pi_L$,它们也分别是$K'$和$L'$的素元,记$K$和$L$的剩余域是$\kappa$,记$K'$和$L'$的剩余域是$\kappa'$.对每个$n\ge0$有如下范数映射:
	    $$\mathrm{N}_n:U_L^{\psi(n)}/U_L^{\psi(n)+1}\to U_K^n/U_K^{n+1}$$
	    $$\mathrm{N}_n':U_{L'}^{\psi(n)}/U_{L'}^{\psi(n)+1}\to U_{K'}^n/U_{K'}^{n+1}$$
	    
	    如果$n=0$这两个映射分别是$(\kappa^*,\times)\to(\kappa^*,\times)$和$((\kappa')^*,\times)\to((\kappa')^*,\times)$,如果$n\ge1$这两个映射分别是$(\kappa,+)\to(\kappa,+)$和$(\kappa',+)\to(\kappa',+)$.我们之前给出过$\mathrm{N}_n$和$\mathrm{N}_n'$的表达式,它要么是1次要么是$p$次的,这里两个映射的系数实际上是相同的.
	\end{enumerate}
	\item 设$L/K$是素数$l$次循环扩张,设$x\in K^*$,那么存在次数$\le l$的扩张$\kappa'/\kappa$,如果记对应的非分歧扩张是$K'/K$,记$L'=K'L$,那么有$x\in\mathrm{N}(L')$.
	\begin{proof}
		
		按照我们之前证明的典范同构$U_K^t/\mathrm{N}(U_L^t)\cong K^*/\mathrm{N}(L^*)$.可不妨设$x\in U_K^t$,因为它们只差一个范数的倍数.设$x$在$U_K^t/U_K^{t+1}$中的像是$\xi$,那么$\mathrm{N}_t(x)-\xi$是$\kappa$上的$l$次多项式,所以存在一个次数$\le l$的扩张$\kappa'$使得这个多项式至少存在一个解$\eta\in\kappa'$.于是$x$在$\mathrm{coker}\mathrm{N}_t'$中是零,按照我们之前证的典范同构$\mathrm{coker}\mathrm{N}_t'\cong U_{K'}^t/\mathrm{N}(U_{L'}^t)$,就说明$x\in\mathrm{N}(L')$.
	\end{proof}
	\item 依旧设$L/K$是素数$l$次循环扩张,我们有$\mathrm{N}(L_{\mathrm{nr}}^*)=K_{\mathrm{nr}}^*$.
	\begin{proof}
		
		如果$x\in K_{\mathrm{nr}}^*$,那么存在有限扩张$\kappa_0/\kappa$,使得$x$落在这个扩张对应的$K_0$中.把上一条的结论用在$L_0/K_0$和$x$上,就说明$x\in\mathrm{N}(L_{\mathrm{nr}}^*)$.
	\end{proof}
    \item 引理.设$L$有递增的子域族$\{L_i\mid i\in I\}$使得$\cup_iL_i=L$,设$M/L$是次数$n$的扩张,
    \begin{enumerate}
    	\item 存在指标$i\in I$,和$n$次扩张$M_i/L_i$,使得$M_i$和$L$在$L_i$上线性无交,并且有$M_iL=M$.
    	\item 如果$M_i$和$M_j$是两个满足这个条件的域,那么存在指标$k\ge i,j$,使得$M_iL_k=M_jL_k$.
    	\item 如果$M/L$是可分的或者Galois的,那么$M_i/L_i$可以选取可分的或者Galois的并且满足上面的断言.
    \end{enumerate}
    \begin{proof}
    	
    	设$M/L$的一组基为$\{m_{\alpha},\alpha=1,2,\cdots,n\}$,我们记$m_{\alpha}m_{\beta}=\sum_{\gamma}c_{\alpha\beta}^{\gamma}m_{\gamma}$,其中$c_{\alpha\beta}^{\gamma}\in L$.选取指标$i$足够大使得每个$c_{\alpha\beta}^{\gamma}\in L_i$,那么$M_i=L_i(m_1,\cdots,m_n)$就满足$M_i\otimes_{L_i}L=M$.于是此时$M_i$和$L$在$L_i$上线性无交.如果设$M/L$是可分的,不妨设特征$p>0$,【】
    \end{proof}
    \item 设$K$是完备离散赋值域,剩余域$\kappa$是完全域,设$K_{\mathrm{nr}}$是$K$的极大非分歧扩张,设$E$是$K_{\mathrm{nr}}$的$n$次扩张,那么存在$K_{\mathrm{nr}}/K$的有限子扩张$K'$,以及一个$n$次扩张$E'/K'$,使得$E'$和$K_{\mathrm{nr}}$在$K'$上线性无交,并且$E=E'K_{\mathrm{nr}}$.扩张$E'/K'$是完全分歧的,并且$E$典范同构于$E_{\mathrm{nr}}'$.另外如果$E/K_{\mathrm{nr}}$是可分的或者Galois的,那么$E'/K'$是可分的或者Galois的.
    $$\xymatrix{&E&\\E'\ar[ur]&&K_{\mathrm{nr}}\ar[ul]_n\\&K'\ar[ul]_n\ar[ur]&\\&K\ar[u]&}$$
    \begin{proof}
    	
    	$K_{\mathrm{nr}}$的剩余域就是$\kappa$的代数闭包$\overline{\kappa}$.设$\{\kappa_i,i\in I\}$是$\overline{\kappa}/\kappa$的递增的有限子扩张,使得并是整个扩张,设$\{K_i,i\in I\}$是对应的$K_{\mathrm{nr}}/K$的有限子扩张,所以它也是递增的并且并是整个扩张.所以按照上述引理存在有限子扩张$K'/K$和$n$次扩张$E'/K'$,使得$E'$和$K_{\mathrm{nr}}$在$K'$上线性无交,并且$E=E'K_{\mathrm{nr}}$.如果$E'$的剩余域不等同于$\kappa(K')$,则$E'/K'$包含了非平凡的非分歧扩张,就和线性无交性矛盾.于是$E$恰好就是$E'$的极大非分歧扩张.
    \end{proof}
    \item 设$K$是完备离散赋值域,剩余域$\kappa$是完全域,设$K_{\mathrm{nr}}$是$K$的极大非分歧扩张,设$K_{\mathrm{nr}}\subseteq E\subseteq F$是有限扩张链,使得$F/E$是可分的,那么$\mathrm{N}(F^*)=E^*$.(事实上这个命题把$F/E$可分条件去掉也成立)
    \begin{proof}
    	
    	不妨设$E\subseteq F$是有限Galois的,否则取$E\subseteq F\subseteq M$使得$M/E$是有限Galois扩张,一旦证明了Galois的情况.那么有$E^*=\mathrm{N}_{M/E}(M^*)=\mathrm{N}_{F/E}(\mathrm{N}_{M/F}(M^*))=\mathrm{N}_{F/E}(F^*)$.现在按照上一条结论,存在$K_{\mathrm{nr}}/K$的有限子扩张$K'/K$,以及$E/K'$和$F/K'$的子扩张$K'\subseteq E'\subseteq F'$,使得$E'$和$K_{\mathrm{nr}}=K_{\mathrm{nr}}'$在$K'$上线性无交,$F'$和$K_{\mathrm{nr}}=K_{\mathrm{nr}}'$在$K'$上线性无交,并且$F'/E'$是Galois的.因为$F',E',K'$的剩余域相同,所以$F'/E'$还是完全分歧的,它的Galois群是$p$群,所以扩张可以分解为$p$阶循环扩张的复合.但是对于素数阶循环扩张,我们已经证明过有$\mathrm{N}((F_{\mathrm{nr}}')^*)=(E_{\mathrm{nr}})^*$.
    \end{proof}
\end{enumerate}

乘性多项式和加性多项式.设$k$是域,设$p$是它的指数特征,即如果$\mathrm{char}k=0$,则$p=1$,如果$\mathrm{char}k>0$则$p=\mathrm{char}k$.
\begin{enumerate}
	\item 一个多项式$P(X)\in k[X]$称为乘性的,如果$P(XY)=P(X)P(Y)$且$P(1)=1$.那么乘性多项式必然具有形式$X^h$.记$\deg P=h$,记$h=h_0p^r$,其中$(h_0,p)=1$.称$h_0$是$P$的可分次数,记$\deg_s(P)=h_0$.
	\item 设$P$是乘性多项式,乘法群同态$P:k^*\to k^*$的核就是$k$的$\deg_s(P)$次单位群,它的阶数整除$\deg_s(P)$.如果$P$和$Q$是两个乘性多项式,那么$P\circ Q$也是乘性多项式,并且满足$\deg(P\circ Q)=\deg(P)\deg(Q)$和$\deg_s(P\circ Q)=\deg_s(P)\deg_s(Q)$.
	\item 一个多项式$P(X)\in k[X]$称为加性的,如果$P(X+Y)=P(X)+P(Y)$.那么如果$\mathrm{char}k=0$,加性多项式具有形式$P(X)=aX,a\in k$;如果$\mathrm{char}k=p>0$,加性多项式是单项式$X^{p^h}$的$k$线性组合.如果$P\not=0$,并且$k$是完全域,那么加性多项式$P$可以唯一的表示为$(P')^{p^h}=X^{p^h}\circ P'$,其中$P'=a_0+\cdots+a_kX^{p^k},a_0a_k\not=0$.有$\deg(P)=p^{h+k}$,称$p^k$是$P$的可分次数,记作$\deg_s(P)$.
	\item 设$P$是加性多项式,加法群同态$P:k\to k$的核也就是$P'$的核,它是$k$的加性子群,阶数整除$\deg_s(P)$.另外因为$P'$是可分的,所以如果$P'$在$k$中分裂,那么这个核的阶数恰好是$\deg_s(P)$.另外如果$P,Q$是两个加性多项式,那么$P\circ Q$也是加性多项式,并且满足$\deg(P\circ Q)=\deg(P)\deg(Q)$和$\deg_s(P\circ Q)=\deg_s(P)\deg_s(Q)$.
\end{enumerate}

设$L/K$是完全分歧的有限Galois扩张.
\begin{enumerate}
	\item 对每个整数$n\ge0$,有$\mathrm{N}(U_L^{\psi(n)})\subseteq U_K^n$和$\mathrm{N}(U_L^{\psi(n)}+1)\subseteq U_K^{n+1}$.我们之前证明了素数阶循环的完全分歧Galois扩张满足这些式子.类似素数阶循环的情况,范数映射诱导了$\mathrm{N}_0:(\kappa^*,\times)\to(\kappa^*,\times)$和$\mathrm{N}_n:(\kappa,+)\to(\kappa,+),n\ge1$.
	\begin{proof}
		
		我们对$|G|$归纳.如果$G=\{e\}$,那么$\varphi(n)=\psi(n)=n$,此时这两个包含关系是平凡的.下面设$G$非平凡,按照完全分歧有$G=G_0$,而$G_0$是可解群,所以$G$存在一个商群是素数阶循环群,也即存在$L/K$的子扩张$K'/K$的次数是素数$l$,按照归纳假设,以及我们证明的素数阶循环扩张的情况,以及范数和$\psi$的传递公式,就有:
		$$\mathrm{N}_{L/K}(U_L^{\psi_{L/K}(n)})=\mathrm{N}_{L/K'}\left(\mathrm{N}_{L/K'}(U_L^{\psi_{L/K'}(\psi_{K'/K}(n))})\right)\subseteq\mathrm{N}_{L/K'}\left(U_{K'}^{\psi_{K'/K}(n)}\right)\subseteq U_K^n$$
		
		类似的归纳法可证明第二个包含关系.
	\end{proof}
	\item 我们断言$n=0$时$\mathrm{N}_0$被一个非常值加性多项式$P_0$诱导,$n\ge1$时$\mathrm{N}_n$被一个非常值乘性多项式$P_n$诱导.次数满足$\deg(P_n)=|G_{\psi(n)}|$和$\deg_s(P_n)=[G_{\psi(n)}:G_{\psi(n)+1}]=\psi'_+(n)/\psi'_-(n)$,这里$\psi'_+$和$\psi'_-$是右导数和左导数,最后这个导数的商也记作$\psi'_{+/-}(n)$.另外我们有如下正合列,其中$\theta$是把$\sigma\in G_{\psi(n)}$映射成$\sigma(\pi)/\pi$,我们解释过这个映射不依赖素元$\pi$的选取.
	$$\xymatrix{0\ar[r]&G_{\psi(n)}/G_{\psi(n)+1}\ar[r]^{\theta}&U_L^{\psi(n)}/U_L^{\psi(n)+1}\ar[r]^{\mathrm{N}_n}&U_K^n/U_K^{n+1}}$$
	\begin{proof}
		
		对$|G|$归纳.同样如果$G=\{e\}$,那么每个$\mathrm{N}_n(\xi)=\xi$,所以$P_n$的次数和可分次数都是1,容易验证正合列成立.下面设$G$非平凡,那么$G=G_0$是可解的,所以存在$L/K$的循环子扩张$K'/K$,记$n'=\psi_{K'/K}(n)$和$n''=\psi_{L/K'}(n')$.记$\mathrm{N}'=\mathrm{N}_{K'/K}$和$\mathrm{N}''=\mathrm{N}_{L/K'}$.那么$\mathrm{N}_n$恰好是如下典范映射的复合:
		$$\xymatrix{U_L^{n''}/U_L^{n''+1}\ar[r]^{\mathrm{N}''}&U_{K'}^{n'}/U_{K'}^{n'+1}\ar[r]^{\mathrm{N}'}&U_K^n/U_K^{n+1}}$$
		
		按照归纳假设,如果$n=0$则$\mathrm{N}'$和$\mathrm{N}''$被加性多项式$P'$和$P''$诱导;如果$n>0$则$\mathrm{N}'$和$\mathrm{N}''$被乘性多项式$P'$和$P''$诱导.于是$\mathrm{N}_n$被加性或者乘性多项式$P_n=P'\circ P''$诱导.结合归纳假设和素数阶情况就得到次数的结论.最后这个正合列,我们解释过$\theta$是单射,另外$\mathrm{N}_n$的核的阶数整除$\deg_s(P_n)$,但是按照$\mathrm{N}(\sigma(\pi)/\pi)=1$,说明这个核包含了$\mathrm{im}\theta$,而$\mathrm{im}\theta$的阶数恰好是$[G_{\psi(n)}:G_{\psi(n)+1}]=\deg_s(P_n)$,于是必须有$\mathrm{im}\theta=\ker(\mathrm{N}_n)$.
	\end{proof}
    \item 推论.
    \begin{enumerate}
    	\item $\mathrm{N}_n$是单射当且仅当$G_{\psi(n)}=G_{\psi(n)+1}$.
    	\item 如下三种情况的任一种都导致$\mathrm{N}_n$是满射:
    	\begin{itemize}
    		\item $\kappa$是代数闭域.因为此时$P_n(X)=\xi$总有解.
    		\item $\kappa$是完全域,并且$G_{\psi(n)}=G_{\psi(n)+1}$.因为此时$P_n(X)=cX^{p^r}$,完全域说明它是满射.
    		\item $G_{\psi(n)}=\{e\}$.因为此时$P_n=cX$.
    	\end{itemize}
        \item 如果$G_{\psi(n)}=\{e\}$,那么$\mathrm{N}(U_L^{\psi(n)})=U_K^n$;如果$G_{\psi(n+1)}=\{e\}$,那么$\mathrm{N}(U_L^{\psi(n)+1})=U_K^{n+1}$.如果$\kappa$是代数闭域,那么$\mathrm{N}(U_L^{\psi(n)})=U_K^n$和$\mathrm{N}(U_L^{\psi(n)+1})=U_K^{n+1}$对任意$n\ge0$总成立.
        \item 设$v\ge0$是实数,我们定义过$U^v=U^n$,其中$n$是$\ge v$的最小整数.如果$G_{\psi(v)}=\{e\}$或者$\kappa$是代数闭域,那么有$\mathrm{N}(U_L^{\psi(v)})=U_K^v$.
    \end{enumerate}
    \item 如果$\kappa$是完全域,如果$x\in K^*$,那么存在次数$\le[L:K]$的扩张$\kappa'/\kappa$,使得如果记剩余域扩张对应的非分歧扩张是$K'/K$,记$L'=LK'$,那么$x$是$L'/K'$的范数.
\end{enumerate}

Hasse-Arf定理的证明.我们之前给出过这个定理,如果$K$是完备离散赋值域,$L/K$是有限阿贝尔扩张,记Galois群为$G$,记剩余域扩张$\lambda/\kappa$是可分的,定理断言$G$的上指标分歧群$\{G^v,v\ge-1\}$的跳跃点都是整数.这里跳跃点指的是实数$v$使得对任意$\varepsilon>0$有$G^{v+\varepsilon}\not=G^v$.如果用下指标分歧群,结论就是说如果整数$\mu$满足$G_{\mu}\not=G_{\mu+1}$,那么$\varphi_{L/K}(\mu)$也是整数.
\begin{enumerate}
	\item 首先不妨设$L/K$是完全分歧的,因为否则可用$L/K$的极大非分歧扩张替代$K$,此时惯性群$G_0$作为Galois群,而不改变$G_s,s\ge0$.设$v$是上指标分歧群的一个跳跃点,记$G'=G^v$,设$G''$是下一个分歧群,也即对足够小的$\varepsilon>0$有$G''=G^{v+\varepsilon}$.按照扩张是阿贝尔的,有$G/G''$是有限个循环群的直积,所以肯定能取它的一个商群$G/H$是循环群,并且要求$G'H\not=H$,那么$H$对应于一个扩张$L'/K$,按照Herbrand定理有$(G/H)^v=G'H/H$和$(G/H)^{v+\varepsilon}=\{e\},\forall\varepsilon>0$.所以一旦我们证明了对于$L/K$是循环完全分歧阿贝尔扩张的情况下跳跃点$v$是整数,就得到一般情况下跳跃点也是整数.最后我们可以约定剩余域不是素数阶域,因为我们之前给出过完全分歧扩张关于剩余域扩张的延拓,并且解释了分歧群都是一致的.
	\item 引理1.记$r=|G|$和$r'=|G_{\mu}|$,设$k=r/r'$.如果$\sigma$是循环群$G$的一个生成元,那么$\tau=\sigma^k$是$G_{\mu}$的一个生成元.用记号$x^{\sigma}$表示$\sigma(x)$.设$V=\{x\in L^*\mid\mathrm{N}(x)=1\}$.那么Hilbert90定理说明$V=\{y^{\sigma-1}\mid y\in L^*\}$.设$W=\{y^{\sigma-1}\mid y\in U_L\}$,它是$V$的子群.我们断言$V/W$是循环群.
	\begin{proof}
		
		$y\mapsto y^{\sigma-1}$是$L^*\to V$的满同态,它诱导了$L^*/U_L\cong\mathbb{Z}\to V/W$的满同态,所以$V/W$是循环群.
	\end{proof}
    \item 引理2.对非负整数$m$,记$V_m=V\cap U_L^m$和$W_m=W\cap U_L^m$.典范映射$V_m/W_m\to V/W$是单射,所以$V_m/W_m$都可以视为$V/W$的子群,它们构成了$V/W$的递降滤过.我们断言当$m$足够大时有$V_m=W_m$.
    \begin{proof}
    	
    	取$t\in L$使得$\mathrm{Tr}(t)=1$,记$m_0=-w(t)$,其中$w$是$L$上的规范离散赋值.设$x\in V_m$,其中$m>m_0$,我们来证明$x\in W_m$.取$y=\sum_{i=0}^{r-1}x^{1+\sigma+\cdots+\sigma^{i-1}}t^{\sigma^i}$.按照$1=\mathrm{Tr}(t)=\sum_{i=0}^{r-1}t^{\sigma^i}$,得到$y-1=\sum_{i=0}^{r-1}(x^{1+\sigma+\cdots+\sigma^{i-1}}-1)t^{\sigma^i}$.按照$x\in U_L^m$得到$x^{1+\sigma+\cdots+\sigma^{i-1}}-1$的赋值$\ge m$,按照$m>m_0$就得到$w(y-1)>0$.于是$y\in U_L$,按照$\mathrm{N}(x)=1$,得到$y^{1-\sigma}=x$,这就说明$x\in W_m$.
    \end{proof}
    \item 引理3.如果$\varphi(m)$是整数并且$G_m=G_{m+1}$,那么有$V_m=V_{m+1}$.
    \begin{proof}
    	
    	记$n=\varphi(m)$,那么$m=\psi(n)$.设$x\in V_m$,设$\overline{x}$是$x$在$U_L^m/U_L^{m+1}$中的像.我们有$x$在$\mathrm{N}_n:U_L^m/U_L^{m+1}\to U_K^n/U_K^{n+1}$的核中,因为$x=z^{\sigma-1}$,得到$\mathrm{N}(x)=x^{1+\sigma+\cdots+\sigma^{r-1}}=z^{\sigma^r-1}=1$.但是我们之前给出过$\mathrm{N}_n$的核就是$G_m/G_{m+1}$,按照条件它是平凡的,所以$\overline{x}=0$,也即$x\in V_{m+1}$.
    \end{proof}
    \item 引理4.设$m$是正整数,如果$W_m$在$U_L^m/U_L^{m+1}$中的像是非平凡的,那么它的像是整个$U_L^m/U_L^{m+1}$.
    \begin{proof}
    	
    	设$x\in W_m-U_L^{m+1}$,记$x=y^{\sigma-1}$,其中$y\in U_L$,按照$U_L/U_L^1\cong\kappa^*$,适当对$y$乘以一个$U_K$中的元不改变$x$,但是可以使得$y\in U_L^1$.于是记$y=1+z$,其中$w(z)\ge1$.下面对任意$a\in\mathscr{O}_K$,记$y_a=1+az$和$x_a=y_a^{\sigma-1}$,那么有$x_a-1=\frac{\sigma(y_a)-y_a}{y_a}=\frac{a(\sigma(y)-y)}{y_a}=a\frac{y}{y_a}(x-1)$.这里$y/y_a\in U_L^1$,所以$x_a\in W_m$,并且$x_a$在$U_L^m/U_L^{m+1}$的像是$\overline{a}\overline{x}$,其中$\overline{a}$是$a$在剩余域中的像,但是$U_L^m/U_L^{m+1}$中每个元都可以表示为$\overline{a}\overline{x},a\in\mathscr{O}_K$,这说明满射.
    \end{proof}
    \item 引理5.设整数$n$使得$G_{\psi(n+1)}=\{e\}$,设整数$m$使得$n<\varphi(m)<n+1$,那么$V_m$和$W_m$在$U_L^m/U_L^{m+1}$中的像都是整个$U_L^m/U_L^{m+1}$.
    \begin{proof}
    	
    	设$x\in U_L^m$,它的像记作$\overline{x}\in U_L^m/U_L^{m+1}$.我们有$\psi(n)<m<\psi(n+1)$.我们给出过完全分区循环扩张的$\psi$的表达式,有$\psi(n)$总是整数,所以$m\ge\psi(n)+1$.于是$\mathrm{N}(x)\in U_K^{n+1}$.我们之前证明过$G_{\psi(n+1)}=\{e\}$推出$\mathrm{N}(U_L^{\psi(n+1)})=U_K^{n+1}$.于是可取$y\in U_L^{\psi(n+1)}\subseteq U_L^{m+1}$使得$\mathrm{N}(y)=\mathrm{N}(x)$.记$x'=xy^{-1}$,那么$x'$也是$\overline{x}$的提升,并且$x'$范数1导致它在$V$中.换句话讲我们证明了$V_m$在$U_L^m/U_L^{m+1}$中的像是整个$U_L^m/U_L^{m+1}$,也即$U_L^m=V_mU_L^{m+1}$.再考虑$W_m$,它在$U_L^m/U_L^{m+1}$中的像是子群$H_m=W_mU_L^{m+1}/U_L^m$,那么$U_L^m/U_L^{m+1}$关于$H_m$的商群$V_mU_L^{m+1}/W_mU_L^{m+1}$是循环群$V_m/W_m$的子群商,这里$V_m/W_m$是循环群因为$V/W$是循环群.但是$U_L^m/U_L^{m+1}$同构于$(\kappa,+)$,我们约定过它不是循环群,所以这里$H_m\not=0$,于是引理4说明$H_m$必须是整个$U_L^m/U_L^{m+1}$.
    \end{proof}
    \item 引理6.设$m$是整数,设$n$是最小的整数使得$n+1\ge\varphi(m)$,设$G_{\psi(n+1)}=\{e\}$,那么$V_m=W_m$.
    \begin{proof}
    	
    	我们先证明$V_m=V_{m+1}W_m$.如果$\varphi(m)$是整数,那么$\varphi(m)=n+1$,于是$\psi(n+1)=m$,那么引理3说明$V_m=V_{m+1}W_m$.如果$\varphi(m)$不是整数,设整数$n$满足$n<\varphi(m)<n+1$,引理5说明$V_m$和$W_m$在$U_L^m/U_L^{m+1}$中的像都是整个$U_L^m/U_L^{m+1}$,于是$U_L^{m+1}V_m=U_L^{m+1}W_m$,于是$V_m\subseteq W_mU_L^{m+1}$,于是$V_m\subseteq W_m(U_L^{m+1}\cap V)\subseteq W_mV_{m+1}$.于是有$V_m=V_{m+1}W_m$.
    	
    	\qquad
    	
    	接下来条件$n+1\ge\varphi(m)$可推出$\psi(n+2)\ge\psi(n+1)+1\ge m+1$,并且$G_{\psi(n+2)}=\{e\}$,所以得到$V_{m+1}=V_{m+2}W_{m+1}$,于是$V_m=V_{m+2}W_m$,继续归纳下去有$V_m=V_{m+k}W_m,\forall k$成立.但是引理2得到$k$足够大时$V_{m+k}=W_{m+k}\subseteq W_m$,所以得到$V_m\subseteq W_m$,也即$V_m=W_m$.
    \end{proof}
    \item 证明Hasse-Arf定理.假设$\varphi(\mu)$不是整数,设$v$是最小的整数满足$v+1\ge\varphi(\mu)$,于是有$\psi(v+1)>\mu$.按照$\mu$的定义有$G_{\psi(v+1)}=\{e\}$.按照引理6有$V_{\mu}=W_{\mu}$.记$\tau=\sigma^k$是$G_{\mu}$的生成元,记$x=\pi^{\tau-1}$,其中$\pi$是$L$的素元,那么$x\in V$,另外有$x\in U_L^{\mu}$,因为按照$\tau\in G_{\mu}$有$w(x-1)=w(\tau(\pi)/\pi-1)=w(\tau(\pi)-\pi)-1\ge\mu$.于是$x\in V\cap U_L^{\mu}=V_{\mu}=W_{\mu}$,于是存在$y\in U_L$使得$\pi^{\tau-1}=y^{\sigma-1}$.我们记$z=y^{-1}\pi^{1+\sigma+\cdots+\sigma^{k-1}}$,于是$z^{\sigma-1}=1$,也即$\sigma(z)=z$,按照扩张是循环的,$\sigma$是生成元,就有$z\in K^*$.最后记$L$上规范离散赋值是$w$,按照$L/K$是完全分歧的,就有$r=[L:K]\mid w(z)$,但是$w(y)=0$和$w(\pi)=1$,得到$r\mid k=r/r'$,这矛盾.
\end{enumerate}








\newpage
\subsection{Artin表示}












\newpage
\section{类域论}
\subsection{抽象类域论}

首先我们介绍抽象Galois理论.每个射影有限群都是某个Galois扩张(可能无限维)的Galois群,这是一个不平凡的结论.于是我们可以纯粹从群论角度描述Galois理论.
\begin{enumerate}
	\item 设$G$是射影有限群.它的闭子群记作$G_K$,其中形式指标$K$称为它对应的域.闭子群$G$的指标$k$称为基域.闭子群$\{1\}$的指标记作$\overline{k}$.
	\item 两个形式的域记号$K,L$记作$K\subseteq L$,如果有$G_L\subseteq G_K$,这称为域扩张.它称为有限扩张,如果$G_L$是$G_K$的开子集,等价于$[L:K]=[G_K:G_L]$是有限的,这个有限数称为这个有限扩张的维数.
	\item 两个形式的域记号$K\subseteq L$称为正规扩张(或者Galois扩张,因为我们实际遇到的情况都是可分的),如果$G_L$是$G_K$的正规子群.此时扩张$K\subseteq L$的Galois群记作$G(L/K)=G_K/G_L$.于是如果$K\subseteq L\subseteq N$都是Galois扩张,那么有$G(L/K)=G(N/K)/G(N/L)$.此时可定义$G(N/K)$中的元$\sigma$在$L$上的限制维$\sigma+G(N/L)\in G(L/K)$.
	\item 如果群$G(L/K)$是循环的,交换的,可解的等,就称域扩张$K\subseteq L$是循环的,交换的,可解的等.
	\item 给定有限个形式的域$\{E_i\}$,它们的交定义为$G_{\cap E_i}=\prod G_{E_i}\le G$.给定一族形式的域$\{E_i\}$,它们的并定义为$G_{\cup E_i}=\cap G_{E_i}\le G$.
\end{enumerate}

设$G$是射影有限群,取$G$模$A$,赋予$A$离散拓扑.我们通常把$A$上的运算记作乘法,这样群作用就写成指数形式$a^g=g(a)$.
\begin{enumerate}
	\item 任取$a\in A$,它在$G\times A\to A$的原像是开集,这个原像包含了点$(1,a)$,于是存在开邻域$G_K\times\{a\}$包含于这个原像.换句话讲$G_K$在$\{a\}$上作用不动,于是有$a\in A_K=A^{G_K}$.于是我们证明了$A=\cup_{[K:k]<\infty}A_K$.
	\item 给定有限域扩张$K\subseteq L$,此时$A_K\subseteq A_L$,定义范数映射$\mathrm{N}_{L/K}:A_L\to A_K$为$a\mapsto\prod_{\sigma}a^{\sigma}$,这里$\sigma$取遍$G_K/G_L$的陪集代表元.
	\item 如果$K\subseteq L$是Galois扩张,那么$A_L$可视为$G(L/K)$模,此时有$A_L^{G(L/K)}=A_K$.
	\item 考虑Tate上同调,有如下表达式,其中$I_{G(L/K)}A_L$表示$\mathrm{N}_{L/K}A_L$中的被$a^{\sigma-1}$生成的子群.
	$$H^0(G(L/K),A_L)=A_K/\mathrm{N}_{L/K}A_L;H^{-1}(G(L/K),A_L)=A_L^{\mathrm{N}=1}/I_{G(L/K)}A_L$$
	\item 我们要处理的情况中群上同调要满足如下公理:如果$K\subseteq L$是有限循环扩张,那么$H^{-1}(G(L/K),A_L)=1$.对于具体的Galois理论取$A_L=L^*$是成立的,此即Hilbert90定理.另外具体Galois理论中诺特证明了如下结论:对任意有限Galois扩张$K\subseteq L$,总有$H^1(G(L/K),L^*)=1$.按照循环情况下Tate上同调是2周期的,得到$H^{-1}(G(L/K),L^*)=1$.
	
\end{enumerate}

抽象分歧理论.
\begin{enumerate}
	\item 设$G$是射影有限群,我们解释过$\widehat{\mathbb{Z}}$是一个射影有限群,选取一个连续满同态$d:G\to\widehat{\mathbb{Z}}$.它的核$I$是一个闭正规子群,对应了一个域$\widetilde{k}$,于是$G(\widetilde{k}/k)\cong\widehat{\mathbb{Z}}$这里这里$\widetilde{K}$称为$K$的(抽象)极大非分歧扩张.
	\item 对每个由$G$的闭子群定义的域$K$,把$d$限制在$G_K\to\widehat{\mathbb{Z}}$的核定义为$K$对应的惯性群,记作$I_K$.于是有$I_K=G_K\cap I=G_K\cap G_{\widetilde{k}}=G_{K\widetilde{k}}$.换句话讲$I_K$的固定域$\widetilde{K}=K\widetilde{k}$.称$K\subset\widetilde{K}$为$K$的极大非分歧扩张.
	\item 记$f_K=[\widehat{\mathbb{Z}}:d(G_K)]$称为惯性次数,记$e_K=[I:I_k]$称为分歧指数.此事$d$限制在$G_K$上未必是满射,但是如果$f_K$有限,那么$d_K=\frac{1}{f_K}d:G_K\to\widehat{\mathbb{Z}}$是满射,它的核自然是$G_K\cap I=I_K$.于是此时有同构$G(\widetilde{K}/K)\cong\widehat{\mathbb{Z}}$.整理一下得到如下域扩张图表:
	$$\xymatrix{&&\overline{k}\ar@/^2pc/[dd]^{G_K}\\\widetilde{k}\ar[rr]^{e_K\text{阶}}\ar[urr]^{I}&&\widetilde{K}\ar[u]_{I_K}\\\widetilde{k}\cap K\ar[u]_{f_K\widehat{\mathbb{Z}}}\ar[rr]^{e_K\text{阶}}&&K\ar[u]_{f_K\widehat{\mathbb{Z}}}\\k\ar[u]_{f_k\text{阶}}\ar@/^2pc/[uu]^{\widehat{\mathbb{Z}}}&&}$$
	\item 对于一般的抽象域扩张$K\subseteq L$,定义它的惯性次数$f_{L/K}=[d(G_K):d(G_L)]$,定义它的分歧指数$e_{L/K}=[I_K:I_L]$.
	\item 对于抽象域扩张链$K\subseteq L\subseteq M$,惯性次数核分歧指数依旧有关系式:
	$$f_{M/K}=f_{L/K}f_{M/L},e_{M/K}=e_{L/K}e_{M/L}$$
	\item 对于抽象域扩张$K\subseteq L$,类似局部域的分歧理论一样有基本关系式$[L:K]=f_{L/K}e_{L/K}$.
	\begin{proof}
		
		考虑如下两个短正合列之间的交换图表:
		$$\xymatrix{1\ar[r]&I_L\ar[r]\ar[d]&G_L\ar[r]\ar[d]&d(G_L)\ar[r]\ar[d]&1\\1\ar[r]&I_K\ar[r]&G_K\ar[r]&d(G_K)\ar[r]&1}$$
		
		这里$d(G_L)\to d(G_K)$是单射,假设扩张$K\subseteq L$是Galois扩张,那么按照蛇形引理得到如下正合列:
		$$\xymatrix{1\ar[r]&I_K/I_L\ar[r]&G(L/K)\ar[r]&d(G_K)/d(G_L)\ar[r]&1}$$
		
		考虑元素阶数得到基本关系式.对于未必Galois的扩张$K\subseteq L$,取它的Galois闭包$K\subseteq L\subseteq M$,此即$G_L$在$G$中包含的极大的$G$的正规子群$\cap_{g\in G} g^{-1}G_Lg$,它对应的闭子群$G_M$对应的$M$是这里的Galois闭包.按照$K\subseteq M$和$L\subseteq M$都是Galois扩张,对应的基本关系式相除得到所求基本关系式.
	\end{proof}
    \item 于是给定射影有限群$G$和一个满连续同态$G\to\widehat{\mathbb{Z}}$,我们形式化的定义了一套分歧理论.继续形式化的定义相关概念:如果$f_{L/K}=1$,称扩张是完全分歧,换句话讲有$L\cap\widetilde{K}=K$;如果$L\subset\widetilde{K}$,并且$e_{L/K}=1$,称扩张$K\subseteq L$是非分歧的.
    \item 有限非分歧扩张都是循环扩张,因为极大非分歧扩张的Galois群按照我们的构造是$\widehat{\mathbb{Z}}$,于是有限非分歧扩张的Galois群就是$\widehat{Z}$的有限商群,而这必然是有限循环群.
\end{enumerate}

Frobenius自同构.我们解释过$\widehat{\mathbb{Z}}$是射影循环群,它可以表示为1生成的循环群的闭包,在同构$G(\widetilde{K}/K)\cong\widehat{\mathbb{Z}}$下右侧中的1对应的扩张自同构$\varphi_K$称为$K$的Frobenius自同构.
\begin{enumerate}
	\item 例如有限域$k=\mathbb{F}_q$的绝对Galois群就是$\widehat{\mathbb{Z}}$,在有限域中我们定义的映射$x\mapsto x^q$就是一个Frobenius自同构.
	\item 如果$K\subseteq L$是有限非分歧的,并且$f_K$有限,那么有满同态$G(\widetilde{K}/K)\to G(L/K)$,称Frobenius映射$\varphi_K$的像为这个扩张$K\subseteq L$的Frobenius自同构.
	\item 取射影有限群$G$,取惯性次数有限的$f_K$.考虑Galois扩张$K\subseteq L$,我们有如下图表,其中扩张$\widetilde{L}\subset\overline{k}$对应的自同构群为$G_{\widetilde{L}}=I_L\subseteq I_K$.于是满同态$d_K:G_K\to\widehat{\mathbb{Z}}$诱导了$G(\widetilde{L}/K)=G_K/I_L\to\widehat{\mathbb{Z}}$的同态,这个同态依旧记作$d_K$.我们记$\mathrm{Frob}(\widetilde{L}/K)=\{\sigma\in G(\widetilde{L}/K)\mid d_K(\sigma)\in\mathbb{N}^+\}$.这是一个半群.
	$$\xymatrix{&&\overline{k}\\&\widetilde{K}\ar[r]&\widetilde{L}\ar[u]\\\widetilde{k}\ar@/^1pc/[uurr]\ar[r]&\widetilde{K}\cap L\ar[u]\ar[r]&L\ar[u]\ar@/_1pc/[uu]\\\widetilde{k}\cap K\ar[u]\ar[r]&K\ar[u]\ar[ur]&\\k\ar@/^2pc/[uu]\ar[u]\ar[ur]&&}$$
	\item 对有限Galois扩张$K\subseteq L$,映射$\mathrm{Frob}(\widetilde{L}/K)\to G(L/K)$为把$\sigma$限制在$L$上.我们断言这是一个满射.
	\begin{proof}
		
		任取$\sigma\in G(L/K)$,取$\varphi_K$在$G(\widetilde{L}/K)$中的提升$\varphi$,即$\varphi\mid\widetilde{K}=\varphi_K$.另外$\varphi\mid L\cap\widetilde{K}=\varphi_{L\cap\widetilde{K}/K}$.这里$K\subset\widetilde{K}\cap L$的Galois群是$\widehat{\mathbb{Z}}$的有限商群,这总是有限循环群,它被Frobenius自同构生成.于是$\sigma$在$\widetilde{K}\cap L$上的限制是这个Frobenius自同构的某个次幂,记$\sigma\mid L\cap\widetilde{K}=\varphi_{L\cap\widetilde{K}/K}^n$.那么$\sigma\varphi^{-n}\mid\widetilde{K}\cap L=1$.于是$\sigma\varphi^{-n}\in G(\widetilde{L}/\widetilde{K}\cap L)$.再设$\sigma\varphi^{-n}$在$L$上的限制为$\tau$,这是$G(L/\widetilde{K}\cap L)\cong G(\widetilde{L}/\widetilde{K})$中的元.于是取$\widetilde{\sigma}=\tau\varphi^n\in G(\widetilde{L}/K)$满足$\widetilde{\sigma}\mid L=\sigma$.而这里$\tau$在$\widetilde{K}$上作用平凡,于是$\widetilde{\sigma}\mid\widetilde{K}=\varphi_K^n$,这导致$d_K(\widetilde{\sigma})=n$,于是$\widetilde{\sigma}\in\mathrm{Frob}(\widetilde{L}/K)$.
	\end{proof}
    \item 任取$\widetilde{\sigma}\in\mathrm{Frob}(\widetilde{L}/K)$,设$\widetilde{\sigma}$的固定域为$\Sigma$,那么我们有如下交换图表,我们断言有如下四件事,其中第四件事得到$\mathrm{Frob}(\widetilde{L}/K)$中的元总是一个Frobenius自同构.
    \begin{itemize}
    	\item $f_{\Sigma/K}=d_K(\widetilde{\sigma})$.
    	\item $[\Sigma:K]<\infty$.
    	\item $\widetilde{\Sigma}=\widetilde{L}$.
    	\item $\widetilde{\sigma}=\varphi_{\Sigma}$.
    \end{itemize}
    $$\xymatrix{\widetilde{\Sigma}&&\widetilde{K}\ar[ll]\ar@{=}[rr]&&\widetilde{K}\ar[rr]&&\widetilde{L}\\\Sigma\ar[u]&&\widetilde{K}\cap\Sigma\ar[ll]\ar[u]&&\widetilde{K}\cap L\ar[u]\ar[rr]&&L\ar[u]\\&&&K\ar[ul]\ar[ulll]\ar[ur]&&&}$$
    \begin{proof}
    	
    	第一件事.因为$\Sigma$是固定$\widetilde{\sigma}$的中间域,得到$\Sigma\cap\widetilde{K}$是固定$\widetilde{\sigma}\mid\widetilde{K}$的固定域.于是$f_{\Sigma/K}=[\Sigma\cap\widetilde{K}:K]$,这个指数是$\widetilde{\sigma}\mid\widetilde{K}$限制在$K\subset\widetilde{K}\cap\Sigma$上作为循环群的一个元的以$\varphi_K$为生成元的次数,也即$d_K(\widetilde{\sigma})$.此时它是有限的.
    	
    	第二件事.按照$\widetilde{K}\subset\Sigma\widetilde{K}=\widetilde{\Sigma}\subset\widetilde{L}$.于是有$e_{\Sigma/K}=[I_K:I_{\Sigma}]=|G(\widetilde{\Sigma}/\widetilde{K})|\le |G(\widetilde{L}/\widetilde{K})|$是有限的.于是$[\Sigma:K]=f_{\Sigma/K}e_{\Sigma/K}$是有限的.
    	
    	第三件事.记$\Gamma=G(\widetilde{L}/\Sigma)$,那么$\Gamma=\overline{(\widetilde{\sigma})}$,这是一个射影循环群,此时必然有$[\Gamma:\Gamma^n]\le n,n\ge1$.单射$\Gamma=G(\widetilde{L}/\Sigma)\to G(\widetilde{\Sigma}/\Sigma)=\widehat{\mathbb{Z}}$是满射,导致诱导的$\Gamma/\Gamma^n\to\widehat{\mathbb{Z}}/n\widehat{\mathbb{Z}}=\mathbb{Z}/n$是满射,但是这只能导致是同构,于是$\Gamma\to\widehat{\mathbb{Z}}$是同构,导致$\widetilde{L}=\widetilde{\Sigma}$.
    	
    	第四件事.按照$f_{\Sigma/K}d_{\Sigma}(\widetilde{\sigma})=d_K(\widetilde{\sigma})=f_{\Sigma/K}$,导致$d_{\Sigma}(\widetilde{\sigma})=1$,于是有$\widetilde{\sigma}=\varphi_{\Sigma}$.
    \end{proof}
\end{enumerate}

Kummer理论.设$A$是$G$模,设$\mathscr{P}:A\to A$是满的$G$模同态,设它核是一个有限循环群$\mu_{\mathscr{P}}$,设核的阶数为$n$,称为$\mathscr{P}$的指数.这里我们主要关心的是$\mathscr{P}$为取$n$次幂的映射,此时$\mu_{\mathscr{P}}=\mu_n=\{\xi\in A\mid\xi^n=1\}$是$A$的$n$阶单位群.固定域$K$使得$\mu_{\mathscr{P}}\subseteq A_K$.对每个子集$B\subseteq A$,记$K(B)$表示$G_K$的闭子群$H=\{\sigma\in G_K\mid b^{\sigma}=b,\forall b\in B\}$的固定域.如果$B$是$G_K$不变的,那么$K\subseteq K(B)$总是Galois扩张.关于$\mathscr{P}$的一个Kummer扩张是指一个形如$K\subseteq K(\mathscr{P}^{-1}(\Delta))$的扩张,其中$\Delta\subseteq A_K$.
\begin{enumerate}
	\item Kummer扩张总是指数$n$的阿贝尔扩张,这里指数$n$是指对应Galois群的元素的阶都整除$n$.
	\begin{proof}
		
		任取$a\in\Delta$,有单同态$G(K(\mathscr{P}^{-1}(a))/K)\to\mu_{\mathscr{P}}$为$\sigma\mapsto\alpha^{\sigma-1}$,其中$\alpha\in\mathscr{P}^{-1}(a)$.因为$\mu_{\mathscr{P}}\subseteq A_K$说明这个映射不依赖于原像$\alpha$的选取.
		
		有$L=K(\mathscr{P}^{-1}(\Delta))=\prod_{a\in\Delta}K(\mathscr{P}^{-1}(a))$,于是复合映射$G(L/K)\to\prod_{a\in\Delta}G(K(\mathscr{P}^{-1}(a))/K)\to\mu_{\mathscr{P}}^{\Delta}$是单射.
	\end{proof}
    \item 反过来指数$n$的阿贝尔扩张总是Kummer扩张:设$K\subseteq L$是指数$n$的阿贝尔扩张,记$\Delta=A_L^{\mathscr{P}}\cap A_K$,那么$L=K(\mathscr{P}^{-1}(\Delta))$.特别的如果$K\subseteq L$是循环扩张,那么有$L=K(\alpha)$,其中$\alpha^{\mathscr{P}}=a\in A_K$.
    \begin{proof}
    	
    	首先有$\mathscr{P}^{-1}(\Delta)\subseteq A_L$,因为如果$x\in A$满足$x^{\mathscr{P}}=a\in\Delta$,那么$a\in A_K$,并且存在$\alpha\in A_L$使得$a=\alpha^{\mathscr{P}}$,于是有$x=\xi\alpha\in A_L$,其中$\xi\in\mu_{\mathscr{P}}\subseteq A_K$.进而有$K(\mathscr{P}^{-1}(\Delta))\subseteq L$.
    	
    	反过来按照有限交换群结构定理,说明$K\subseteq L$是循环子扩张的复合.于是任取$K\subseteq L$的循环子扩张$K\subseteq M$,只需验证$M\subseteq K(\mathscr{P}^{-1}(\Delta))$.设$G(M/K)$的生成元为$\sigma$,设$\mu_{\mathscr{P}}$的生成元为$\zeta$.记$d=[M:K]$,记$d'=n/d$和$\xi=\zeta^{d'}$.按照$\mathrm{N}_{M/K}(\xi)=\xi^d=1$,上同调公理说明存在$\alpha\in A_M$使得$\xi=\alpha^{\sigma-1}$.于是$K\subseteq K(\alpha)\subseteq M$,这里$K(\alpha)$表示固定$\alpha$的$G_K$的闭子群对应的子群.但是固定$\alpha$的自同构应该满足$\alpha^{\sigma^i}=\xi^i\alpha$,于是等价于$d\mid i$,这只能有$i=d$,于是$K(\alpha)=M$.最后有$\alpha\in\mathscr{P}^{-1}(\Delta)$,因为$(\alpha^{\mathscr{P}})^{\sigma-1}=\xi^{\mathscr{P}}=1$.
    \end{proof}
    \item 存在从满足$A_K^{\mathscr{P}}\subset\Delta\subseteq A_K$的子群$\Delta$到$K$的指数$n$的阿贝尔扩张$L$的一一对应为$\Delta\mapsto L=K(\mathscr{P}^{-1}(\Delta))$,逆对应为$L\mapsto A_L^{\mathscr{P}}\cap A_K$.另外如果$\Delta$和$L$互相对应,此时有典范同构$\Delta/A_K^{\mathscr{P}}\cong\mathrm{Hom}(G(L/K),\mu_{\mathscr{P}})$为$a\mathrm{mod}A_K^{\mathscr{P}}\mapsto\chi_a$,这里$\chi_a$定义为$\sigma\in G(L/K)\mapsto\alpha^{\sigma-1}$,其中$\alpha\in\mathscr{P}^{-1}(a)$.
    \begin{proof}
    	
    	设$K\subseteq L$是指数$n$的阿贝尔扩张,我们之前已经证明了有$L=K(\mathscr{P}^{-1}(\Delta))$和$\Delta=A_L^{\mathscr{P}}\cap A_K$.先证明后面的同构.考虑同态$\Delta\to\mathrm{Hom}(G(L/K),\mu_{\mathscr{P}})$为$a\mapsto\chi_a$,其中$\chi_a$把$\sigma$映射为$\alpha^{\sigma-1}$,其中$\alpha\in\mathscr{P}^{-1}(a)$.那么$\chi_a=1$当且仅当$\alpha^{\sigma-1}$对任意$\sigma\in G(L/K)$成立,当且仅当$\alpha\in A_K$,当且仅当$a=\alpha^{\mathscr{P}}\in A_K^{\mathscr{P}}$.这说明了诱导商映射的单射性.
    	
    	为证明满射,任取$\chi\in\mathrm{Hom}(G(L/K),\mu_{\mathscr{P}})$.那么$\ker\chi$是$G(L/K)$的闭子群,设它对应的中间域为$M$,于是有单同态$\overline{\chi}:G(M/K)\to\mu_{\mathscr{P}}$,这说明$K\subseteq M$是一个循环扩张,记生成元$\sigma$.于是有$\mathrm{N}_{M/K}(\overline{\chi})=\overline{\chi}(\sigma)^{[M:K]}=1$,按照上同调公理得到$\overline{\chi}(\sigma)=\alpha^{\sigma-1}$,其中$\alpha\in A_M$.于是有$(\alpha^{\mathscr{P}})^{\sigma-1}=\overline{\chi}(\sigma)^{\mathscr{P}}=1$,于是有$a=\alpha^{\mathscr{P}}\in A_L^{\mathscr{P}}\cap A_K=\Delta$.任取$\tau\in G(L/K)$,就有$\chi(\tau)=\overline{\chi}(\tau\mid M)=\alpha^{\tau-1}=\chi_a(\tau)$.这说明满射性.
    	
    	最后来说明这个一一对应.我们之前已经证明过从满足$A_K^{\mathscr{P}}\subset\Delta\subseteq A_K$的子群$\Delta$到$K$的指数$n$的阿贝尔扩张$L$的映射$\Delta\mapsto L=K^{\mathscr{P}^{-1}(\Delta)}$是满射.下面仅需验证这是一个单射,换句话讲如果$\Delta$满足介于$A_K^{\mathscr{P}}$和$A_K$之间,设$L=K(\mathscr{P}^{-1}(\Delta))$,记$A_L^{\mathscr{P}}\cap A_K=\Delta'$,需要证明$\Delta=\Delta'$.
    	
    	首先有$\Delta\subset\Delta'$,因为一方面$\Delta\subseteq A_K$,另一方面$\Delta\subseteq A_L^{\mathscr{P}}$等价于讲子群有包含关系$\{\sigma\in G\mid b^{\sigma}=b,b\in\mathscr{P}^{-1}(\Delta)\}\supset\{\sigma\in G\mid b^{\sigma}=b,b\in\mathscr{P}^{-1}(A_L^{\mathscr{P}})\}$,但是后者是平凡群,包含关系一定成立.另外有$\Delta'/A_K^{\mathscr{P}}\cong\mathrm{Hom}(G(L/K),\mu_{\mathscr{P}})$.按照$\Delta$是$\Delta'$的子群,取$H=\{\sigma\in G(L/K)\mid\chi_aa(\sigma)=1,\forall a\in\Delta\}$,就有$\Delta/A_K^{\mathscr{P}}\cong\mathrm{Hom}(G(L/K)/H,\mu_{\mathscr{P}})$.但是$\chi_a(\sigma)=\alpha^{\sigma-1}=1$,说明$H$是固定$\mathscr{P}^{-1}(\Delta)$中的元,导致$L=K(\mathscr{P}^{-1}(\Delta))$,于是$H=1$,于是这两个$\mathrm{Hom}$集合是相同的,导致$\Delta=\Delta'$.
    \end{proof}
\end{enumerate}

抽象Henselian赋值.设$G$是射影有限群,选取一个满连续同态$d:G\to\widehat{\mathbb{Z}}$,设$A$是$G$模,记$A_k=A^{G}$,$A_k$上的一个Henselian赋值是指一个同态$v:A_k\to\widehat{\mathbb{Z}}$,满足:
\begin{itemize}
	\item $\mathbb{Z}\subseteq Z=v(A_k)$,满足$\forall n\ge1$有$Z/nZ\cong\mathbb{Z}/n\mathbb{Z}$.
	\item 对任意有限扩张$k\subseteq K$,有$v(\mathrm{N}_{K/k}A_k)=f_kZ$.
\end{itemize}
\begin{enumerate}
	\item 设$k\subseteq K$是有限扩张,那么$A_k$上的赋值可以延拓到$A_K$上:$v_K=\frac{1}{f_K}v\circ\mathrm{N}_{K/k}:A_K\to Z$是满同态,满足如下条件:
	\begin{itemize}
		\item $\forall\sigma\in G$,有$v_K=v_{K^{\sigma}\circ\sigma}$.
		\begin{proof}
			
			如果$\tau$跑遍$G_k/G_K$的陪集代表元,那么$\sigma^{-1}\tau\sigma$跑遍$G_k/\sigma^{-1}G_K\sigma=G_k/G_{K^{\sigma}}$的代表元,于是对任意$a\in A_K$就有:(这里$f_K=f_{K^{\sigma}}$因为$d(G_K)=d(G_{K^{\sigma}})$)
			$$v_{K^{\sigma}}(a^{\sigma})=\frac{1}{f_{K^{\sigma}}}v(\prod_{\tau}a^{\sigma\sigma^{-1}\tau\sigma})=\frac{1}{f_K}v(\mathrm{N}_{K/k}(a))=v_K(a)$$
		\end{proof}
	    \item 对每个有限扩张$K\subseteq L$,有如下交换图表:
	    $$\xymatrix{A_L\ar[rr]^{v_L}\ar[d]_{\mathrm{N}_{L/K}}&&\widehat{\mathbb{Z}}\ar[d]^{f_{L/K}}\\A_K\ar[rr]^{v_K}&&\widehat{\mathbb{Z}}}$$
	    \begin{proof}
	    	
	    	任取$a\in A_L$,有:
	    	\begin{align*}
	    	f_{L/K}v_L(a)&=f_{L/K}\frac{1}{f_L}v(\mathrm{N}_{L/k}(a))=\frac{1}{f_K}v(\mathrm{N}_{K/k}(\mathrm{N}_{L/K}(a)))\\&=v_K(\mathrm{N}_{L/K}(a))
	    	\end{align*}
	    \end{proof}
	\end{itemize}
    \item 素元定义.$A_K$的素元$\pi_K$是指满足$v_K(\pi_K)=1$的元.再记单位群$U_K=\{a\in A_K\mid v_K(a)=0\}$.
    \begin{itemize}
    	\item 如果扩张$K\subseteq L$是非分歧的,此时$f_{L/K}=[L:K]$,按时上面交换图表,说明$A_K$中的素元$\pi_K$同样也是$A_L$中的素元.
    	\item 如果扩张$K\subseteq L$是完全分歧的,也即$f_{L/K}=1$,此时$\pi_L$的范数$\mathrm{N}_{L/K}(\pi_L)$就是$A_K$的素元$\pi_K$.
    \end{itemize}
\end{enumerate}

互反映射.给定有限非分歧扩张,我们约定有上同调公理$H^i(G(L/K),U_L)=1,i=-1,0$.互反映射是指一个典范映射(后面我们会证明所构造的实际上是同构)$r_{L/K}:G(L/K)^{\textbf{ab}}\to A_K/\mathrm{N}_{L/K}A_L$.我们解释过总存在满射$\mathrm{Frob}(\widetilde{L}/K)\to G(L/K)$,于是可以先构造$\mathrm{Frob}(\widetilde{L}/K)\to A_K/\mathrm{N}_{\widetilde{L}/K}A_{\widetilde{L}}$.
\begin{enumerate}
	\item 首先对于无限扩张$K\subseteq L$,我们约定$\mathrm{N}_{L/K}A_L=\cap_M\mathrm{N}_{M/K}A_M$,其中$K\subseteq M$跑遍$K\subseteq L$的有限子扩张.这个定义吻合于有限情况.
	\item 我们先来构造映射$r_{\widetilde{L}/K}:\mathrm{Frob}(\widetilde{L}/K)\to A_K/\mathrm{N}_{\widetilde{L}/K}A_{\widetilde{L}}$,其中$\mathrm{Frob}(\widetilde{L}/K)=\{\sigma\in G(\widetilde{L}/K)\mid d_K(\sigma)\in\mathbb{N}\}$.设$\sigma\in\mathrm{Frob}(\widetilde{L}/K)$的固定域是$\Sigma$,定义$r_{\widetilde{L}/K}$把$\sigma$映射为$\mathrm{N}_{\Sigma/K}(\pi_{\Sigma})(\mathrm{mod}\mathrm{N}_{\widetilde{L}/K}A_{\widetilde{L}})$.
	\item 首先要说明这个映射和$\Sigma$上素元的选取无关.
	\begin{proof}
		
		假设再取一个素元,那么它和原本的素元$\pi_{\Sigma}$只差了一个$U_{\Sigma}$中的元$u$.于是只需验证$\mathrm{N}_{\Sigma/K}(u)\in\mathrm{N}_{\widetilde{L}/K}A_{\widetilde{L}}$.按照$K\subset\widetilde{L}$可能是无限维的,按照定义这等价于验证对$K\subset\widetilde{L}$的每个有限子扩张$K\subseteq M$都有$\mathrm{N}_{\Sigma/K}(u)\in\mathrm{N}_{M/K}A_M$.
		
		我们解释过$K\subset\Sigma$是有限扩张,于是不妨设$\Sigma\subseteq M$.按照上同调公理,可取$\varepsilon\in U_M$使得$u=\mathrm{N}_{M/\Sigma}(\varepsilon)$.于是有$\mathrm{N}_{\Sigma/K}(u)=\mathrm{N}_{\Sigma/K}(\mathrm{N}_{M/\Sigma}(\varepsilon))=\mathrm{N}_{M/K}(\varepsilon)\in\mathrm{N}_{M/K}A_M$.
	\end{proof}
    \item 我们证明构造的$r_{\widetilde{L}/K}$是乘性映射.
    \begin{proof}
    	
    	取$\sigma\in G(\widetilde{L}/K)$,定义$\sigma-1:A_{\widetilde{L}}\to A_{\widetilde{L}}$为$a\mapsto a^{\sigma-1}=a^{\sigma}/a$(这里交换群$A$的运算记作乘法).再构造$\sigma_n:A_{\widetilde{L}}\to A_{\widetilde{L}}$为$a\mapsto\prod_{0\le i\le n-1}a^{\sigma^i}$.于是有$(\sigma-1)\circ\sigma_n=\sigma^n-1$.再记$\mathrm{N}=\mathrm{N}_{\widetilde{L}/\widetilde{K}}$.
    	
    	现在取$\sigma_1,\sigma_2\in\mathrm{Frob}(\widetilde{L}/K)$,记$\sigma_3=\sigma_1\sigma_2$,设$\sigma_i$的固定域是$\Sigma_i$,取它的素元$\pi_i$,其中$i=1,2,3$.于是问题归结为证明$\pi_1$的范数和$\sigma_2,\sigma_3$的范数的乘积是同余的:
    	$$\mathrm{N}_{\Sigma_1/K}(\pi_1)\mathrm{N}_{\Sigma_2/K}(\pi_2)\equiv\mathrm{N}_{\Sigma_3/K}(\pi_3)(\mathrm{mod}\mathrm{N}_{\widetilde{L}/K}A_{\widetilde{L}})$$
    	
    	选取$\varphi\in G(\widetilde{L}/K)$使得$d_K(\varphi)=1$.取$\tau_i=\sigma_i^{-1}\varphi^{n_i}\in G(\widetilde{L}/K)$,其中$d_K(\sigma_i)=n_i,i=1,2,3$.那么$d_K(\tau_i)=0$,导致$\tau_i$实际上落在$G(\widetilde{L}/\widetilde{K})$中.按照$\sigma_3=\sigma_1\sigma_2$,说明$n_1+n_2=n_3$.现在$\tau_3=(\sigma_2^{-1}\varphi^{n_2})(\varphi^{-n_2}\sigma_1\varphi^{n_2})^{-1}\varphi^{n_1}$.这里左侧起第一个括号就是$\tau_2$,第二个括号记作$\sigma_4$,那么$n_4=d_K(\sigma_4)=n_1$,记$\sigma_4$的固定域是$\Sigma_4$,于是$\Sigma_4=\Sigma_1^{\varphi^{n_2}}$,选取素元$\pi_4=\pi_1^{\varphi^{n_2}}\in A_{\Sigma_4}$.再记$\tau_4=\sigma_4^{-1}\varphi^{n_4}$,于是有$\tau_3=\tau_2\tau_4$.并且有$\mathrm{N}_{\Sigma_4/K}(\pi_4)=\mathrm{N}_{\Sigma_1/K}(\pi_1)$因为它们互相共轭,于是问题归结为证明:
    	$$\mathrm{N}_{\Sigma_4/K}(\pi_4)\mathrm{N}_{\Sigma_2/K}(\pi_2)\equiv\mathrm{N}_{\Sigma_3/K}(\pi_3)(\mathrm{mod}\mathrm{N}_{\widetilde{L}/K}A_{\widetilde{L}})$$
    	
    	引理1.$\varphi\in G(\widetilde{L}/K)$满足$d_K(\varphi)=1$,$\sigma\in G(\widetilde{L}/K)$满足$d_K(\sigma)=n$.记$\sigma$的固定域是$\Sigma$,任取$a\in A_{\Sigma}$,那么有$\mathrm{N}_{\Sigma/K}(a)=(N\circ\varphi_n)(a)=(\varphi_n\circ N)(a)$.
    	
    	引理1的证明.取$K\subset\Sigma$的极大非分歧扩张$\Sigma^0=\Sigma\cap\widetilde{K}$,我们解释过这里$K\subset\Sigma^0$的扩张次数就是惯性次数$n$.于是Galois群$G(\Sigma^0/K)$是$n$阶循环群,它的生成元可以取$\varphi_{\Sigma^0/K}$,也即$\varphi$在$\Sigma^0$的限制.于是$\mathrm{N}_{\Sigma^0/K}=\varphi_n\mid A_{\Sigma^0}$.另外我们解释过有$\Sigma\widetilde{K}=\widetilde{\Sigma}=\widetilde{L}$,于是$\mathrm{N}_{\Sigma/\Sigma^0}$就是$\mathrm{N}$在$A_{\Sigma}$上的限制.于是任取$a\in A_{\Sigma}$就有$\mathrm{N}_{\Sigma/K}(a)=\mathrm{N}_{\Sigma^0/K}(\mathrm{N}_{\Sigma/\Sigma^0}(a))=\mathrm{N}(a)^{\varphi_n}=\mathrm{N}(a^{\varphi_n})$.最后这个等式因为$\varphi_n$在$G(\widetilde{L}/\widetilde{K})$的正规化子中.
    	
    	回到原证明,按照引理1有$\mathrm{N}_{\Sigma_i/K}(\pi_i)=\mathrm{N}(\pi_i^{\varphi_{n_i}}),i=1,2,3,4$.于是问题归结为记$u=\pi_3^{\varphi_{n_3}}\pi_4^{-\varphi_{n_4}}\pi_2^{-\varphi_{n_2}}$有$\mathrm{N}(u)\in\mathrm{N}_{\widetilde{L}/K}A_{\widetilde{L}}$
    	
    	引理2.设$x\in H_0(G(\widetilde{L}/\widetilde{K}),U_{\widetilde{L}})=U_{\widetilde{L}}/I_{G(\widetilde{L}/\widetilde{K})}U_{\widetilde{L}}$被一个元$\varphi\in G(\widetilde{L}/K)$固定,满足$d_K(\varphi)=1$,那么有$\mathrm{N}(x)\in\mathrm{N}_{\widetilde{L}/K}U_{\widetilde{L}}$.
    	
    	引理2的证明.设$x$在$U_{\widetilde{L}}$中的提升为$u$,那么$x$被$\varphi$固定等价于讲$\varphi(u)/u\in I_{G(\widetilde{L}/\widetilde{K})}U_{\widetilde{L}}$,这等价于讲存在$u_i\in U_{\widetilde{L}}$和$\tau_i\in G(\widetilde{L}/\widetilde{K})$使得$u^{\varphi-1}=\prod_{1\le i\le r}u_i^{\tau_i-1}$.为了证明$\mathrm{N}(x)\in\mathrm{N}_{\widetilde{L}/K}U_{\widetilde{L}}$,需要证对$K\subset\widetilde{L}$的每个有限Galois子扩张$K\subseteq M$都有$\mathrm{N}(u)\in\mathrm{N}_{M/K}U_M$.按照定义无限扩张的范数是有限子扩张范数的交,于是我们适当扩大$M$不会损失一般性.于是不妨设$u,u_i\in U_M$,并且$L\subseteq M$.记$n=[M:K]$,记$\sigma=\varphi^n$,记$\Sigma$是被$\sigma$固定的包含$M$的域.记$\Sigma_n$是$\Sigma$的被$\sigma^n=\varphi^n_{\Sigma}$固定的域,它是$\Sigma$的一个$n$次非分歧扩张.按照上同调公理,可选取$\widetilde{u},\widetilde{u}_i\in U_{\Sigma_n}$使得$u=\mathrm{N}_{\Sigma_n/\Sigma}(\widetilde{u})=\widetilde{u}^{\sigma_n}$和$u_i=\mathrm{N}_{\Sigma_n/\Sigma}(\widetilde{u}_i)=\widetilde{u}_i^{\sigma_n}$,其中$\sigma_n=1+\sigma+\cdots+\sigma^{n-1}$.另外有$\widetilde{u}^{\varphi-1}=\widetilde{x}\prod_i\widetilde{u}_i^{\tau_i-1}$,其中$\widetilde{x}\in U_{\Sigma_n}$,按照上同调公理,有$\widetilde{x}=\widetilde{y}^{\sigma-1}$,其中$\widetilde{y}\in U_{\Sigma_n}$.于是有$\widetilde{u}^{\varphi-1}=(\widetilde{y}^{\varphi_n})^{\varphi-1}\prod_i\widetilde{u}_i^{\tau_i-1}$.作用范数$\mathrm{N}$,按照$\widetilde{u}_i^{\tau_i}$和$\widetilde{u}_i$的范数相同,得到$\mathrm{N}(\widetilde{u})^{\varphi-1}=\mathrm{N}(\widetilde{y}^{\varphi_n})^{\varphi-1}$.于是有$\mathrm{N}(\widetilde{u})=\mathrm{N}(\widetilde{y}^{\varphi_n})z$,其中$z\in U_{\widetilde{K}}$满足$z^{\varphi-1}=1$.于是$z\in U_K$.最后取$y=\widetilde{y}^{\sigma_n}=\mathrm{N}_{\Sigma_n/\Sigma}(\widetilde{y})\in U_{\Sigma}$,于是有:
    	\begin{align*}
    	\mathrm{N}(u)&=\mathrm{N}(\widetilde{u})^{\sigma_n}=\mathrm{N}(\widetilde{y}^{\varphi_n})^{\sigma_n}z^{\sigma_n}\\&=\mathrm{N}(y^{\varphi_n})z^n=\mathrm{N}_{\Sigma/K}(y)\mathrm{N}_{M/K}(z)\in\mathrm{N}_{M/K}U_M
    	\end{align*}
    	
    	回到原证明,按照引理2,为证明$\mathrm{N}(x)\in\mathrm{N}_{\widetilde{L}/K}A_{\widetilde{L}}$归结为证明$x^{\varphi-1}$落在$I_{G(\widetilde{L}/\widetilde{K})}U_{\widetilde{L}}$.按照$\varphi_{n_i}\circ(\varphi-1)=\varphi^{n_i}-1$,得到$\pi_i^{\varphi^{n_i}-1}=\pi_i^{\tau_i-1}$(因为$\pi_i$被$\sigma_i$固定).于是有$u^{\varphi-1}=\pi_3^{\tau_3-1}\pi_4^{1-\tau_4}\pi_2^{1-\tau_2}$.按照$\tau_3=\tau_2\tau_4$,得到$(\tau_3-1)+(1-\tau_4)+(1-\tau_2)=(1-\tau_2)(1-\tau_4)$.再记$\pi_3=u_3\pi_4$和$\pi_2=u_2^{-1}\pi_4$和$\pi_4^{\tau_2}=u_4\pi_4$,其中$u_i\in U_{\widetilde{L}}$.于是有$u^{\varphi-1}=u_2^{\tau_2-1}u_3^{\tau_3-1}u_4^{\tau_4-1}$落在$I_{G(\widetilde{L}/\widetilde{K})U_{\widetilde{L}}}$中.
    \end{proof}
    \item 现在我们构造互反同态.设$K\subseteq L$是有限Galois扩张,构造映射$r_{L/K}:G(L/K)\to A_K/\mathrm{N}_{L/K}A_L$为,任取$\sigma\in G(L/K)$,我们解释过$\mathrm{Frob}(\widetilde{L}/K)\to G(L/K)$是满射,于是可取$\sigma$的一个Frobenius提升$\widetilde{\sigma}$,它在$r_{\widetilde{L}/K}$下映射到$A_K/\mathrm{N}_{\widetilde{L}/K}A_{\widetilde{L}}$,按照$\mathrm{N}_{\widetilde{L}/K}A_{\widetilde{L}}\subset\mathrm{N}_{L/K}A_L$,又有典范映射$A_K/\mathrm{N}_{\widetilde{L}/K}A_{\widetilde{L}}\to A_K/\mathrm{N}_{L/K}A_L$,这得到一个映射$r_{L/K}:G(L/K)\to A_K/\mathrm{N}_{L/K}A_L$.
    \item 我们需要证明这个映射不依赖于$\sigma$的提升$\widetilde{\sigma}$的选取.
    \begin{proof}
    	
    	选取$\sigma$的两个Frobenius提升$\widetilde{\sigma}$和$\widetilde{\sigma}'$,设它们对应的固定域分别为$\Sigma$和$\Sigma'$.如果$d_K(\widetilde{\sigma})=d_K(\widetilde{\sigma}')$,那么有$\widetilde{\sigma}$和$\widetilde{\sigma}'$在$\widetilde{K}$上的限制相同.而它们在$L$上的限制都是$\sigma$,于是它们在$\widetilde{K}L=\widetilde{L}$上是相同的,也即$\widetilde{\sigma}=\widetilde{\sigma}'$,此时没什么需要证的.
    	
    	现在不妨设$d_K(\widetilde{\sigma})<d_K(\widetilde{\sigma}')$.那么存在$\widetilde{\tau}=\widetilde{\sigma}^{-1}\widetilde{\sigma}'\in\mathrm{Frob}(\widetilde{L}/K)$,那么$\widetilde{\tau}$在$L$上的限制是恒等映射.于是$\widetilde{\tau}$的固定域$\Sigma''$包含了域$L$.于是$r_{\widetilde{L}/K}(\widetilde{\tau})\equiv\mathrm{N}_{\Sigma''/K}(\pi_{\Sigma''})$,这在$\mathrm{mod}\mathrm{N}_{L/K}A_L$下是同余1的.
    \end{proof}
    \item 还需要证明$r_{L/K}$是群同态.这是因为我们证明过$r_{\widetilde{L}/K}$是乘性映射,另外如果$\widetilde{\sigma_i}$是$\sigma_i$的Frobenius提升,$i=1,2$,那么$\widetilde{\sigma_1}\widetilde{\sigma_2}$是$\sigma_1\sigma_2$的Frobenius提升.
\end{enumerate}

互反同态的性质.
\begin{enumerate}
	\item 如果$K\subseteq L$是有限非分歧扩张,那么它的Galois群$G(L/K)$是被$\varphi_{L/K}$生成的有限循环群.此时$\varphi_{L/K}$的一个Frobenius提升可取$\varphi_K$,但是$\varphi_K$的固定域$\Sigma$就是$K$本身,导致$\mathrm{N}_{\Sigma/K}(\pi_K)=\pi_K$,于是$r_{L/K}(\varphi_{L/K})$就是$\pi_K(\mathrm{mod}\mathrm{N}_{L/K}A_L)$.我们断言此时互反同态是一个同构.
	\begin{proof}
		
		首先赋值映射$v_K$诱导了同态$A_K/\mathrm{N}_{L/K}A_L\to Z/nZ\cong\mathbb{Z}/n\mathbb{Z}$,这是满射因为$v_K(A_K)=Z$,这是单射因为,如果$a\in A_K$使得$v_K(a)=nd$,那么有$a=u\pi_K^{dn}$,其中$u$是单位,按照上同调公理得到某个$\varepsilon\in U_L$使得$u=\mathrm{N}_{L/K}(\varepsilon)$.于是有$a=\mathrm{N}_{L/K}(\varepsilon\pi_K^d)\in\mathrm{N}_{L/K}A_L$.
		
		于是$A_K/\mathrm{N}_{L/K}A_L$是以$\pi_K$为生成元的$n$阶循环群,但是这里互反同态把循环群的生成元$\varphi_{L/K}$映射为循环群的生成元$\pi_K$,而它们又是相同阶的循环群,于是这是同构.
	\end{proof}
    \item 互反映射的函子性.如果$K\subseteq L$和$K'\subseteq L'$是有限Galois扩张,满足$K\subseteq K'$和$L\subseteq L'$,任取$\sigma\in G$,那么有如下交换图表:
    $$\xymatrix{G(L'/K')\ar[rr]^{r_{L'/K'}}\ar[d]^{\sigma'\mapsto\sigma'\mid L}&&A_{K'}/\mathrm{N}_{L'/K'}A_{L'}\ar[d]^{\mathrm{N}_{K'/K}}\\G(L/K)\ar[rr]^{r_{L/K}}&&A_K/\mathrm{N}_{L/K}A_L}\quad\xymatrix{G(L/K)\ar[rr]^{r_{L/K}}\ar[d]^{\tau\mapsto\tau^{\sigma}}&&A_{K}/\mathrm{N}_{L/K}A_{L}\ar[d]^{\sigma}\\G(L^{\sigma}/K^{\sigma})\ar[rr]^{r_{L^{\sigma}/K^{\sigma}}}&&A_{K^{\sigma}}/\mathrm{N}_{L^{\sigma}/K^{\sigma}}A_{L^{\sigma}}}$$
    \item 转移映射(transfer).设$G$是群,取有限指数的子群$H$,我们有转移同态$\mathrm{Ver}:G^{\mathrm{ab}}\to H^{\mathrm{ab}}$定义如下:取双陪集分解$G=\cup_{\tau}(\sigma)\tau H$,其中$\tau$跑遍一组陪集代表,记$f(\tau)$是最小的正整数使得$\sigma_{\tau}=\tau^{-1}\sigma^{f(\tau)}\tau\in H$,定义$\mathrm{Ver}(\sigma(\mathrm{mod}G'))=\prod_{\tau}\sigma_{\tau}(\mathrm{mod}H')$.这是一个良定群同态.转移映射提供了如下函子性:设$K\subseteq L$是有限Galois扩张,设$K'$是扩张的一个中间域,那么有如下交换图表:
    $$\xymatrix{G(L/K')^{\mathrm{ab}}\ar[rr]^{r_{L/K'}}&&A_{K'}/\mathrm{N}_{L/K'}A_{L}\\G(L/K)^{\mathrm{ab}}\ar[rr]^{r_{L/K}}\ar[u]^{\mathrm{Ver}}&&A_K/\mathrm{N}_{L/K}A_L\ar[u]}$$
\end{enumerate}

一个抽象类域论是指如下信息:一个射影有限群$G$,一个连续满同态$d:G\to\widehat{\mathbb{Z}}$,一个连续$G$模$A$,一个Henselian赋值$v:A\to\widehat{\mathbb{Z}}$,满足如下条件:
\begin{itemize}
	\item 上同调公理:对任意循环扩张$K\subseteq L$,有:
	$$H^{-1}(G(L/K),A_L)=1;\#H^0(G(L/K),A_L)=[L:K]$$
	\item 互反映射.对有限Galois扩张$K\subseteq L$,总有自然的互反映射:
	$$r_{L/K}:G(L/K)^{\mathrm{ab}}\tilde{\rightarrow} A_K/\mathrm{N}_{L/K}A_L$$
\end{itemize}
\begin{enumerate}
	\item 上同调公理能推出对于有限非分歧扩张$K\subseteq L$,总有$H^i(G(L/K),U_L)=1$,其中$i=-1,0$.
	\begin{proof}
		
		首先我们解释过有限非分歧扩张总是循环扩张.按照$H^{-1}(G(L/K),A_L)=1$,说明$U_L$中每个满足$\mathrm{N}_{L/K}(u)=1$的元$u$,都具有形式$u=a^{\sigma-1}$,其中$a\in A_L$.扩张是非分歧的,我们解释过此时$A_K$的素元$\pi_K$也是$A_L$的素元,于是$a$可以表示为$\varepsilon\pi_K^m$,其中$\varepsilon\in U_L$,但是$\pi_K\in A_K$,导致$\sigma$在$\pi_K$上作用恒等,于是$u=a^{\sigma-1}=\varepsilon^{\sigma-1}$,这说明$H^{-1}(G(L/K),U_L)=1$.
		
		考虑映射$v_K:A_K\to Z$,记$n=[L:K]=f_{L/K}$,我们之前解释过$v_K(\mathrm{N}_{L/K}A_L)=f_{L/K}Z=nZ$,于是$v_K$诱导了同态$A_K/\mathrm{N}_{L/K}A_L\to Z/nZ=\mathbb{Z}/n\mathbb{Z}$.$\pi_K$在这个诱导同态下的像是1,导致这是满射,左边这个商就是$H^0(G(L/K),A_L)$,按照公理它的阶数是$n$.于是这个诱导同态必须是同构.
		
		最后验证$H^0(G(L/K),U_L)=1$,也即$U_K/\mathrm{N}_{L/K}U_L=1$,因为如果$u\in U_K$,即$v_K(u)=0$,那么存在$a\in A_L$使得$\mathrm{N}_{L/K}(a)=u$,但是有$0=v_K(u)=v_K(\mathrm{N}_{L/K}(a))=nv_L(a)$,导致$n_L(a)=0$,导致$a\in U_L$,此即$U_K\subset\mathrm{N}_{L/K}U_L$.
	\end{proof}
    \item 对有限Galois扩张$K\subseteq L$,互反映射$r_{L/K}$诱导的映射$r_{L/K}:G(L/K)^{\mathrm{ab}}\to A_K/\mathrm{N}_{L/K}A_L$总是一个同构.
    \begin{proof}
    	
    	设$K\subseteq M$是$K\subseteq L$的Galois子扩张.互反映射提供了如下交换图表:
    	$$\xymatrix{1\ar[rr]&&G(L/M)\ar[rr]\ar[d]^{r_{L/M}}&&G(L/K)\ar[rr]\ar[d]^{r_{L/K}}&&G(M/K)\ar[rr]\ar[d]^{r_{M/K}}&&1\\&&A_M/\mathrm{N}_{L/M}A_L\ar[rr]^{\mathrm{N}_{M/K}}&&A_K/\mathrm{N}_{L/K}A_L\ar[rr]&&A_K/\mathrm{N}_{M/K}A_M\ar[rr]&&1}$$
    	
    	第一步,我们说明约化为$G(L/K)$是阿贝尔情况.因为一旦这得证,选取$M$是$K\subseteq L$的极大阿贝尔扩张$L^{\mathrm{ab}}$.此时$G(L/M)=G(L/K)'$是换位子群,这是一个正规子群,于是有$G(L/K)^{\mathrm{ab}}=G(M/K)$.那么我们假定的此时$r_{M/K}$是同构.现在任取$G(L/K)$中的元$x$落在$\ker r_{L/K}$中,那么这个元映射到$A_K/\mathrm{N}_{M/K}A_M$中是零,按照$r_{M/K}$是同构,得到$x$映射到$G(M/K)$是零,导致$x$落在$G(L/M)$中,于是我们证明了$\ker r_{L/K}\subseteq G(L/M)$.但是反过来换位子群总是落在$\ker r_{L/K}$中(因为$r_{L/K}$终端是一个交换群),于是$r_{L/K}$诱导的$G(L/K)^{\mathrm{ab}}\to A_K/\mathrm{N}_{L/K}A_L$是单射.
    	
    	再证明满射.如果$G(L/K)$是可解群,此时要么$L=M$,此即$G(L/K)$本身交换;要么$[L:M]<[L:K]$.前一种情况已经约化到阿贝尔扩张情况,后一种情况我们对$[L:K]$归纳假设,从$r_{M/K}$和$r_{L/M}$是满射得到$r_{L/K}$是满射.如果$G(L/K)$未必是可解群,取$M$是它某个Sylow-$p$子群的固定域,此时$K\subseteq M$未必是Galois扩张.但是我们依旧有上面图表左侧小方格是交换的.按照归纳假设这里$r_{L/M}$是满射.于是$\mathrm{N}_{M/K}$的像是$G(L/M)$在$r_{L/K}$下的像.一旦我们证明$\mathrm{N}_{M/K}$的像是$A_K/\mathrm{N}_{L/K}A_L$的Sylow-$p$子群$S_p$,结合交换群上所有Sylow-$p$子群生成整个群,就得到$r_{L/K}$是满射.现在包含映射$A_K\subseteq A_M$诱导了如下同态$i:A_K/\mathrm{N}_{L/K}A_L\to A_M/\mathrm{N}_{L/M}A_L$,它满足$\mathrm{N}_{M/K}\circ i=[M:K]$.按照$([M:K],p)=1$,得到$S_p$上数乘$[M:K]$的映射是满射,于是$S_p$落在$\mathrm{N}_{M/K}$的像中.(另外这里有一个小细节,我们需要$[L:M]<[L:K]$才能归纳,如果$G(L/K)$本身是$p$群,此时不能用这种方法证明满射,但是$p$群都是可解群,这归结到证满射的第一种情况中).
    	
    	第二步,我们说明约化为$G(L/K)$是循环群的情况.一旦这个结论得证,任取循环子扩张$K\subseteq M$,从图表交换性,以及假设的$r_{M/K}$是同构,得到$\ker r_{L/K}\subset\ker(G(L/K)\to\prod_MG(M/K))$,其中$M$取遍循环子扩张.但是按照$G(L/K)$是阿贝尔的,导致右侧这个核是平凡的,于是$r_{L/K}$是单射.满射我们已经在上一种情况里证明过了,因为交换群都是可解群.
    	
    	第三步,我们说明约化为循环完全分歧扩张.一旦这成立,选取它的极大非分歧扩张$M=L\cap\widetilde{K}$,那么$K\subseteq M$非分歧,$M\subseteq L$完全分歧,但是我们证明过有限非分歧情况下$r_{M/K}$是同构(注意有限非分歧扩张总是循环扩张),假设了完全分歧情况$r_{L/M}$是同构,此时考虑元素个数得知上面交换图表中下行实际上是一个短正合列,于是按照短五引理有$r_{L/K}$是同构.
    	
    	第四步,证明有限循环完全分歧扩张是成立的.完全分歧就是$f_{L/K}=1$,记$G(L/K)$的生成元是$\sigma$.完全分歧使得$K$在$L$中的极大非分歧扩张只有自己,也即$L\cap\widetilde{L}=K$,于是$G(L/K)\cong G(\widetilde{L}/\widetilde{K})$.我们把$\sigma$视为$G(\widetilde{L}/\widetilde{K})$的生成元.记$\widetilde{\sigma}=\sigma\varphi_L$,这是$\mathrm{Frob}(\widetilde{L}/K)$中的元,因为$d_K(\widetilde{\sigma})=d_K(\varphi_L)=f_{L/K}=1$.设$\widetilde{\sigma}$的固定域是$\Sigma$,那么$K\subset\Sigma$也是完全分歧扩张,于是$\Sigma\cap\widetilde{K}=K$.选取一个包含$\Sigma L$的有限Galois扩张$K\subseteq M$,选取$K\subseteq M$的极大非分歧扩张$M^0=M\cap\widetilde{K}$.记$\mathrm{N}=\mathrm{N}_{M/M^0}$,那么有$\mathrm{N}\mid A_{\Sigma}=\mathrm{N}_{\Sigma/K}$和$\mathrm{N}\mid A_L=\mathrm{N}_{L/K}$.
    	
    	为证明$r_{L/K}$的单射,需要证明如果$r_{L/K}(\sigma^k)=1$,其中$0\le k<n=[L:K]$,那么$k=0$.选取$A_{\Sigma}$和$A_L$的素元分别为$\pi_{\Sigma}$和$\pi_L$.按照$\Sigma\subseteq M$和$L\subseteq M$是非分歧的,说明$\pi_{\Sigma}$和$\pi_L$都是$A_M$中的素元.于是$\pi_{\Sigma}^k=u\pi_L^k$,其中$u\in U_M$.于是有$1\equiv r_{L/K}(\sigma^k)\equiv\mathrm{N}(\pi_{\Sigma}^k)\equiv\mathrm{N}(u)(\mathrm{mod}\mathrm{N}_{L/K}A_L)$.于是存在$v\in U_L$使得$\mathrm{N}(u)=\mathrm{N}(v)$.于是$\mathrm{N}(uv^{-1})=1$,按照上同调公理得到$uv^{-1}=a^{1-\sigma}$,其中$a\in A_M$.
    	
    	按照$\pi_L^kv\in L$,说明$(\pi_L^kv)^{\sigma-1}=(\pi_L^kv)^{\widetilde{\sigma}-1}$.进而有$=(\pi_{\Sigma}^ku^{-1}v)^{\widetilde{\sigma}-1}=(a^{\sigma-1})^{\widetilde{\sigma}-1}=(a^{\widetilde{\sigma}-1})^{\sigma-1}$.于是$x=\pi_L^kva^{1-\widetilde{\sigma}}$掉在$\sigma-1$的核中,也即$A_{M^0}$中.按照$k=v_m(x)=nv_{M^0}(x)$得到$k=0$,这证明了单射.但是两个同阶有限群之间的单射一定是同构,这得到$r_{L/K}$是同构.
    \end{proof}
\end{enumerate}

范剩余符号(norm residue symbol).我们证明了对于有限Galois扩张$K\subseteq L$,有$r_{L/K}:G(L/K)^{\mathrm{ab}}\to A_K/\mathrm{N}_{L/K}A_L$是同构,它的逆映射复合典范商映射$A_K\to A_K/\mathrm{N}_{L/K}A_L$得到的满同态$(-,L/K):A_K\to G(L/K)^{\mathrm{ab}}$称为范剩余符号.
\begin{enumerate}
	\item 范剩余符号具有如下函子性:设$K\subseteq L$核$K'\subseteq L'$是有限Galois扩张,使得$K\subseteq K'$核$L\subseteq L'$,设$\sigma\in G$,设$M$是$K\subseteq L$的中间域,那么有如下交换图表:
	$$\xymatrix{A_{K'}\ar[rr]^{(-,L'/K')}\ar[d]_{\mathrm{N}_{K'/K}}&&G(L'/K')^{\mathrm{ab}}\ar[d]\\A_K\ar[rr]^{(-,L/K)}&&G(L/K)^{\mathrm{ab}}}\quad\xymatrix{A_K\ar[rr]^{(-,L/K)}\ar[d]_{\sigma}&&G(L/K)^{\mathrm{ab}}\ar[d]^{\sigma^*}\\A_{K^{\sigma}}\ar[rr]^{(-,L^{\sigma}/K^{\sigma})}&&G(L^{\sigma}/K^{\sigma})^{\mathrm{ab}}}$$
	$$\xymatrix{A_{K'}\ar[rr]^{(-,L/K')}&&G(L/K')^{\mathrm{ab}}\\A_K\ar[u]\ar[rr]^{(-,L/K)}&&G(L/K)^{\mathrm{ab}}\ar[u]_{\mathrm{Ver}}}$$
	\item 范剩余符号可以推广到无限Galois扩张上.我们知道对于无限Galois扩张$K\subseteq L$,有$G(L/K)=\lim\limits_{\leftarrow}G(L_i/K)$,这里$L_i$取遍有限Galois子扩张.这里阿贝尔化和逆向极限可交换,就得到$G(L/K)^{\mathrm{ab}}=\lim\limits_{\leftarrow}G(L_i/K)^{\mathrm{ab}}$.于是对$a\in A_K$,$(a,L_i/K)$确定了一个元$(a,L/K)\in G(L/K)^{\mathrm{ab}}$.
	\item 对于极大非分歧扩张$K\subset\widetilde{K}$,我们有$(a,\widetilde{K}/K)=\varphi_K^{v_K(a)}$,于是有$d_K\circ(-,\widetilde{K}/K)=v_K$.
	\begin{proof}
		
		设$K\subseteq L$是$K\subset\widetilde{K}$的次数为$f$的子扩张.按照$Z/fZ\cong\mathbb{Z}/f\mathbb{Z}$,于是$v_K(a)$可以表示为$n+fz$,其中$n\in\mathbb{Z}$和$z\in Z$.换句话讲$a=u\pi_K^nb^f$,其中$u\in U_K$,$b\in A_K$.于是有如下等式,这说明$(a,\widetilde{K}/K)=\varphi_K^{v_K(a)}$.
		$$(a,\widetilde{K}/K)\mid L=(a,L/K)=(\pi_K,L/K)^n(b,L/K)^f=(\pi_K,L/K)^n=\varphi_{L/K}^n=\varphi_K^{v_K(a)}\mid L$$
	\end{proof}
\end{enumerate}

域论的一个核心目标是求域的所有代数扩张,并且扩张理应被基域$K$内部的结构所决定.类域论解决了域$K$的全部阿贝尔扩张.它给出了这样的扩张和$A_K$的特定子群之间的一一对应.
\begin{enumerate}
	\item 给定域$K$,赋予$A_K$上的范数拓扑为,以全部陪集$a\mathrm{N}_{L/K}A_L$为拓扑基,其中$L$跑遍$K$的所有有限Galois扩张,而$a$跑遍$A_K$中的元.
	\item $A_K$中的开子群恰好就是有限指数的闭子群.
	\begin{proof}
		
		假设$H$是一个开子群,它的所有陪集都是开子集,于是所有不为$H$的陪集的并是一个开子集,于是$H$是闭子集.按照$H$包含了某个$\mathrm{N}_{L/K}A_L$,导致它必然是有限指数的.反过来假设$H$是有限指数的闭子群,它的所有陪集都是闭子集,于是这有限个全部不为$H$的陪集的并是闭子集,于是$H$是开子集.
	\end{proof}
    \item 赋值映射$v_K:A_K\to\widehat{Z}$是连续的.
    \begin{proof}
    	
    	$f\widehat{Z},f\ge1$构成了$0\in\widehat{\mathbb{Z}}$的一组开邻域基,于是如果$K\subseteq L$是$f$次非分歧扩张,那么有$v_K(\mathrm{N}_{L/K}A_L)=fv_L(A_L)\subseteq f\widehat{\mathbb{Z}}$.结合拓扑是线性的说明$v_K$处处是连续的.
    \end{proof}
    \item 如果$K\subseteq L$是有限扩张,那么$\mathrm{N}_{L/K}:A_L\to A_K$是连续的.
    \begin{proof}
    	
    	首先$\mathrm{N}_{M/K}A_M$是$1\in A_K$的开邻域基,其中$M$取遍有限Galois扩张.那么有$K\subseteq ML$是有限Galois扩张,此时$\mathrm{N}_{L/K}(\mathrm{N}_{ML/L}A_{ML})=\mathrm{N}_{ML/L}A_{ML}\subset\mathrm{N}_{M/K}A_M$.
    \end{proof}
    \item $A_K$是Hausdorff的当且仅当$A_K^0=\cap_L\mathrm{N}_{L/K}A_L$是平凡的.
    \item 定理.映射$L\mapsto\mathscr{N}_L=\mathrm{N}_{L/K}A_L$是从$K$的全体有限阿贝尔扩张$L$到$A_K$的开子群的一一对应.这里$\mathscr{N}_L$称为有限阿贝尔扩张$L$的类域.它满足:
    \begin{itemize}
    	\item 这个对应是反序的,即$L_!\subseteq L_2$当且仅当$\mathscr{N}_{L_2}\subset\mathscr{N}_{L_1}$.
    	\item 域的复合对应于开子群的交:$\mathscr{N}_{L_1L_2}=\mathscr{N}_{L_1}\cap\mathscr{N}_{L_2}$.
    	\item 域的交对应于开子群的乘积:$\mathscr{N}_{L_1\cap L_2}=\mathscr{N}_{L_1}\mathscr{N}_{L_2}$.
    \end{itemize}
    \begin{proof}
    	
    	先证明$\mathscr{N}_{L_1L_2}=\mathscr{N}_{L_1}\cap\mathscr{N}_{L_2}$.一方面按照范数的传递性,得到$\mathscr{N}_{L_1L_2}\subset\mathscr{N}_{L_1}\cap\mathscr{N}_{L_2}$.另一方面取$a\in\mathscr{N}_{L_1}\cap\mathscr{N}_{L_2}$,那么$(a,L_1L_2/K)\in G(L_1L_2/K)$投影到$G(L_i/K)$上都是平凡的,于是$(a,L_i/K)=1,i=1,2$.于是$(a,L_1L_2/K)=1$,于是$a\in\mathscr{N}_{L_1L_2}$.
    	
    	再证明这个对应是反序的.因为$\mathscr{N}_{L_1}\supset\mathscr{N}_{L_2}$等价于讲$\mathscr{L_1}\cap\mathscr{N}_{L_1L_2}=\mathscr{N}_{L_2}$,等价于$[L_1L_2:K]=[L_2:K]$,这等价于讲$L_1\subseteq L_2$.另外这个证明还说明这个对应是单射.
    	
    	再说明这个对应是满射.选取$A_K$的开子群$\mathscr{N}$,那么它包含了一个开子群$\mathscr{N}_L=\mathrm{N}_{L/K}A_L$,其中$K\subseteq L$是有限Galois扩张.另外我们证明的$G(L/K)^{\mathrm{ab}}\cong A_K/\mathrm{N}_{L/K}A_L$说明$\mathscr{N}_L=\mathscr{N}_{L^{\mathrm{ab}}}$,于是可不妨设$K\subseteq L$是有限阿贝尔扩张.于是可设$(\mathscr{N},L/K)=G(L/L')$,其中$L'$是$K\subseteq L$的某个中间域.这里$\mathscr{N}$是$G(L/L')$在$(-,L'/K):A_K\to G(L'/K)$的原像,于是$\mathscr{N}=\mathscr{N}_{L'}$.这说明满射性.
    	
    	最后证明$\mathscr{N}_{L_1\cap L_2}=\mathscr{N}_{L_1}\mathscr{N}_{L_2}$.首先从$L_1\cap L_2\subseteq L_i$得到$\mathscr{N}_{L_1}\mathscr{N}_{L_2}\subset\mathscr{N}_{L_1\cap L_2}$.另一方面$\mathscr{N}_{L_1}\mathscr{N}_{L_2}$是开子群,满射性得到存在某个$L$使得$\mathscr{N}_{L_1}\mathscr{N}_{L_2}=\mathscr{N}_L$,其中$L$是$K$的某个有限阿贝尔扩张,从$\mathscr{N}_{L_i}\subset\mathscr{N}_L$得到$L\subseteq L_1\cap L_2$,这就得到$\mathscr{N}_{L_1}\mathscr{N}_{L_2}=\mathscr{N}_L\supset\mathscr{N}_{L_1\cap L_2}$.
    \end{proof}
\end{enumerate}
\newpage
\subsection{局部类域论}

局部类域论是把抽象类域论运用在局部域上.所谓局部域是指一个剩余域有限的完备离散赋值域,这等价于讲是$\mathbb{Q}_p$或者$\mathbb{F}_p[[t]]$的有限扩张.固定一个局部域$k$,射影有限群$G$取为绝对Galois群$G(\overline{k}/k)$,把$G$模$A$取为乘法群$\overline{k}^*$.给定有限扩张$k\subseteq K$,那么有$A_K=K^*$.首先要验证上同调公理.
\begin{enumerate}
	\item 对局部域的循环扩张$K\subseteq L$,总有:
	$$H^{-1}(G(L/K),L^*)=1;\#H^0(G(L/K),L^*)=[L:K]$$
	\begin{proof}
		
		$H^{-1}(G(L/K),L^*)=1$是Hilbert90定理.于是为证$\#H^0(G(L/K),L^*)=[L:K]$只需证Herbrand商$h(G,L^*)=[L:K]$.记$G=G(L/K)$,考虑$G$模的如下正合列,其中$\mathbb{Z}$视为平凡$G$模.
		$$\xymatrix{1\ar[r]&U_L\ar[r]&L^*\ar[r]^{v_L}&\mathbb{Z}\ar[r]&0}$$
		
		按照Herbrand商的乘性,得到$h(G,L^*)=h(G,\mathbb{Z})h(G,U_L)$,而$\mathbb{Z}$是平凡$G$模导致$h(G,\mathbb{Z})=|G|=[L:K]$,于是问题归结为证明$h(G,U_L)=1$.为此选取$K\subseteq L$的正规基$\{\alpha^{\sigma}\mid\sigma\in G\}$,其中$\alpha\in\mathscr{O}_L$.考虑$G$模$M=\sum_{\sigma\in G}\mathscr{O}_K\alpha^{\sigma}\subset\mathscr{O}_L$,这是$\mathscr{O}_L$的开子集【?】.有$V^n=1+\pi_K^nM,n\ge1$构成了$U_L$中乘法幺元的开邻域基,于是可取$N$使得$\pi_K^N\mathscr{O}_L\subseteq M$.此时$n\ge N$时$V^n$是$U_L$的有限指数子群:因为有$(\pi_K^nM)(\pi_K^nM)=\pi_K^{2n}MM\subset\pi_K^{2n}\mathscr{O}_L\subset\pi_K^{2n-N}M\subset\pi_K^nM$.于是$V^nV^n\subseteq V^n$.另外对$\mu\in M$,有$(1-\pi_K^n\mu)^{-1}=1+\pi_K^n(\sum_{i\ge1}\mu^i\pi_K^{n(i-1)})$,于是$V^n$是子群.进而它是$G$子模.
		
		我们有$G$模同构$V^n/V^{n+1}\cong M/\pi_KM$为$1+\pi_K^n\alpha\mapsto\alpha(\mathrm{mod}\pi_KM)$.而后者又有同构$M/\pi_KM=\oplus_{\sigma\in G}(\mathscr{O}_K/\pi_K)\alpha^{\sigma}$,此即诱导模$\mathrm{Ind}_G(\mathscr{O}_K/\pi_K)$.
		
		如果$H$是$G$的子群,如果$B$是$H$模,那么诱导模满足$H^i(G,\mathrm{Ind}_G^HB)=H^i(H,A)$.但是这里的情况$H$是平凡群,于是这里$H^i(H,A)=1$,这就说明$H^i(G,V^n/V^{n+1})=1$,其中$n\ge N$.我们断言这导致$H^0(G,V^n)=1,n\ge N$.因为按照上同调平凡,对于$a\in(V^n)^G$就有$a_1\in(V^{n+1})^G$使得$a=(\mathrm{N}_Gb_0)a_1$,其中$b_0\in V^n$.继续选取$a_2$和$b_1$使得$a_1=(\mathrm{N}b_1)a_2$,使得$b_1\in V^{n+1}$和$a_2\in(V^{n+2})^G$.归纳构造下去得到$a_i=(\mathrm{N}_Gb_i)a_{i+1}$,使得$b_i\in V^{n+i}$和$a_i\in(V^{n+i})^G$.于是有$a=\mathrm{N}_Gb$,其中$b=\prod_{i\ge0}b_i$是收敛的.这说明$H^0(G,V^n)=1$.同理得到$H^{-1}(G,V^n)=1$,于是有$h(G,V^n)=1$.最后按照$U_L/V^n$是有限群,得到$h(G,U_L/V^n)=1$,于是$h(G,U_L)=h(G,U_L/V^n)h(G,V^n)=1$.
	\end{proof}
    \item 设$K\subseteq L$是局部域之间的有限非分歧扩张,在抽象类域论中我们证明过此时$H^i(G(L/K),U_L)=1,i=0,-1$.这里我们断言$H^i(G(L/K),U_L^{(n)})=1,n\ge1,i=0,-1$,于是特别的零阶同调群平凡得到$\mathrm{N}_{L/K}U_L=U_K$和$\mathrm{N}_{L/K}U_L^{(n)}=U_K^{(n)}$.
    \begin{proof}
    	
    	设$L$的剩余域为$\lambda$,首先有$H^i(G,\lambda)=H^i(G,\lambda^*)=1$.这是因为局部域的剩余域都是有限的,导致零阶上同调群平凡和Herbrand商为1.
    	
    	或者因为剩余域扩张是有限的,也可以这样证明:设$K$的剩余域为$\kappa$,记元素个数是$q$,记$f=[\lambda:\kappa]$,记$\kappa\subset\lambda$的Frobenius自同构为$\varphi$,那么$\lambda$中范数平凡的元$x$满足$\sum_{0\le i\le f-1}x^{\phi^i}=\sum_{0\le i\le f-1}x^{q^i}=0$,这至多有$q^{f-1}$个元.另外$\varphi-1:\lambda\to\lambda$的核就是$\kappa$,导致$(\varphi-1)\lambda$的元素个数为$q^{f-1}$,于是$H^{-1}(G,\lambda)=\lambda^{N=1}/(\phi-1)\lambda$元素个数至多为1,这只能是平凡的.
    	
    	现在考虑$G$模的短正合列$1\to U_L^{(1)}\to U_L\to\lambda^*\to1$,从$H^i(G,\lambda^*)=1$得到$H^i(G,U_L^{(1)})=H^i(G,U_L)=1$.再考虑短正合列$1\to U_L^{(n+1)}\to U_L^{(n)}\to\lambda\to1$,其中$U_L^{(n)}\to\lambda$是这样定义的:设$\pi$为$K$的素元,那么它也是$L$的素元,把$1+a\pi^n$映射为$a(\mathrm{mod}(\pi))$,这是一个$G$模同态.于是从$H^i(G,\lambda)=0$得到$H^i(G,U_L^{(n+1)})=H^i(G,U_L^{(n)})=1$.
    \end{proof}
\end{enumerate}

局部类域论.
\begin{enumerate}
	\item 我们在局部域理论中解释过对于完备离散赋值域$k$,设剩余域为$\kappa$,那么$k$的极大非分歧扩张$\widetilde{k}$的剩余域是$\kappa$的可分闭包,并且Galois群相同.如果$k$是局部域,此时$\kappa$是有限域,它的极大可分闭包就是代数闭包$\overline{\kappa}$,于是有$G(\widetilde{k}/k)\cong G(\overline{\kappa}/\kappa)\cong\widehat{\mathbb{Z}}$.我们取定一个同构为,把$\widehat{\mathbb{Z}}$中的元1对应为$G(\overline{k}/k)$的Frobenius自同构$x\mapsto x^q$,其中$q$是$\kappa$的阶数,再对应为$G(\widetilde{k}/k)$的Frobenius自同构$\phi_k$.记绝对Galois群$G=G(\overline{k}/k)$,那么就有连续满同态$d:G\to\widehat{\mathbb{Z}}$.此时抽象分歧理论和局部域理论中的分歧是一致的.
	\item 把$G$模$A$取为$\overline{k}^*$,那么对有限扩张$k\subseteq K$有$A_K=K^*$.这里我们断言局部域理论中的规范赋值映射$v_k:k^*\to\mathbb{Z}$的确是一个抽象Henselian赋值:一方面$v_k(k^*)=\mathbb{Z}$,另一方面对于有限扩张$k\subseteq K$,有$\frac{1}{e_K}v_K$是$v_k$在$K^*$上的延拓,此时有$\frac{1}{e_K}v_K(K^*)=\frac{1}{[K:k]}v_k(\mathrm{N}_{K/k}K^*)=\frac{1}{e_Kf_K}v_k(\mathrm{N}_{K/k}K^*)$,于是得到$v_k(\mathrm{N}_{K/k}K^*)=f_Kv_K(K^*)=f_K\mathbb{Z}$.
	\item 至此局部域$k$上的信息$G=G(\overline{k}/k)$,$d:G\to\widehat{\mathbb{Z}}$,$v_k:k^*\to\mathbb{Z}$是一个抽象类域论,这称为局部类域论.此时有局部互反律:如果$K\subseteq L$是局部域之间的有限Galois扩张,那么互反映射是一个同构,这里互反映射定义为,对$\sigma\in G(L/K)$,选取一个提升$\widetilde{\sigma}\in G(\widetilde{L}/K)$,使得$d_K(\widetilde{\sigma})\in\mathbb{N}$,换句话讲有$\widetilde{\sigma}\mid\widetilde{K}=\phi_K^n$,设$\Sigma$为$\widetilde{\sigma}$的固定域,设$\pi_{\Sigma}$是$\Sigma$的素元,那么有$r_{L/K}(\sigma)=\mathrm{N}_{\sigma/K}(\pi_{\Sigma})(\mathrm{mod}\mathrm{N}_{L/K}L^*)$.
	$$r_{L/K}:G(L/K)^{\mathrm{ab}}\cong K^*/\mathrm{N}_{L/K}L^*$$
	\item 上述互反映射的逆映射复合上典范商映射$K^*\to K^*/\mathrm{N}_{L/K}L^*$得到的映射$(-,L/K):K^*\to G(L/K)^{\mathrm{ab}}$称为局部范剩余映射.这是一个满同态,它的核是$\mathrm{N}_{L/K}L^*$.
	\item 局部类域论给出了局部域上有限阿贝尔扩张的描述,此即类域存在定理:设$K$是局部域,存在从$K$的有限阿贝尔扩张到$K^*$的有限指数开子群的反序一一对应$L\mapsto\mathscr{N}_L=\mathrm{N}_{L/K}L^*$.并且这个对应满足$\mathscr{N}_{L_1L_2}=\mathscr{N}_{L_1}\cap\mathscr{N}_{L_2}$和$\mathscr{N}_{L_1\cap L_2}=\mathscr{N}_{L_1}\mathscr{N}_{L_2}$.
	\begin{proof}
		
		在抽象类域论中我们证明了相同的结论.在那里是对$G$模$A_K$赋予了范拓扑.这里我们仅需验证$K^*$在范拓扑下的开子群恰好就是在赋值拓扑下的有限指数开子群.先假设$\mathscr{N}$是$K^*$在范拓扑下的开子群.按照范拓扑定义它包含了形如$\mathrm{N}_{L/K}L^*$的子群,但这在$K^*$中是有限指数子群,于是$\mathscr{N}$是有限指数的.再说明它在赋值拓扑下是开的:首先上同调公理有$\mathrm{N}_{L/K}U_L=U_K$,这是赋值拓扑下的开子集,它包含在子群$\mathscr{N}$中,于是$\mathscr{N}$写成这个开子群的陪集的并,是开集的并,于是它是赋值拓扑下的开子群.下面需要证明赋值拓扑下的有限指数开子群$\mathscr{N}$是范拓扑下的开子群,等价于讲它包含一个范数群$\mathrm{N}_{L/K}L^*$(因为按照范拓扑定义这样的范数群是开的,包含开子群的子群肯定是开的,因为它是陪集的并),记指数为$n$,我们分特征情况:
		
		如果$\mathrm{char}(K)\not| n$,那么有$(K^*)^n\subset\mathscr{N}$.归结为证明$(K^*)^n$包含一个范群.不妨设$K^*$包含了一个$n$次单位根$\mu_n$,否则可以设$K_1=K(\mu_n)$,倘若我们找到了$(K_1^*)^n$所包含的范群$\mathrm{N}_{L_1/K_1}L_1^*$,选取$L$是包含$L_1$的$K$的有限Galois扩张,那么就有$\mathrm{N}_{L/K}L^*=\mathrm{N}_{K_1/K}(\mathrm{N}_{L/K_1}L^*)\subset\mathrm{N}_{K_1/K}(\mathrm{N}_{L_1/K_1}L_1^*)\subset\mathrm{N}_{K_1/K}((K_1^*)^n)\subset(K^*)^n$.
		
		于是不妨设$K$包含了一个$n$次单位根$\mu$.取$L=K(\sqrt[n]{K^*})$是$K$的极大的指数$n$的阿贝尔扩张,在Kummer理论中我们证明了$\mathrm{Hom}(G(L/K),\mu_n)\cong K^*/(K^*)^n$.在局部域理论中我们解释过这里$K^*/(K^*)^n$是有限的,于是这里$G(L/K)$必然是有限的.于是$K\subseteq L$是指数$n$的有限阿贝尔扩张.最后按照局部互反律有$K^*/\mathrm{N}_{L/K}L^*$同构于$G(L/K)$,于是它也是指数$n$的,于是$(K^*)^n\subset\mathrm{N}_{L/K}L^*$.但是$K^*/(K^*)^n$和$K^*/\mathrm{N}_{L/K}L^*$是阶数都是$|G(L/K)|$的群(因为此时有$\mathrm{Hom}(G,\mu_n)=\mathrm{Hom}(G,\mathbb{Q}/\mathbb{Z})\cong G$,其中$G$是有限阿贝尔群),这导致$(K^*)^n=\mathrm{N}_{L/K}L^*$.
		
		如果$\mathrm{char}(K)=p\mid n$可用Lubin-Tate理论,见后文.
	\end{proof}
    \item 推论.如果$K$包含$n$次本原单位根,$K$的特征不整除$n$,那么指数$n$的极大阿贝尔扩张$K\subseteq K(\sqrt[n]{K^*})$是有限扩张,并且有$\mathrm{N}_{L/K}L^*=(K^*)^n$和$G(L/K)\cong K^*/(K^*)^n$.
\end{enumerate}

导子(conductor).我们知道全体$U_K^{(n)},n\ge0$是$K^*$幺元1的开邻域基,如果$K\subseteq L$是有限阿贝尔扩张,记$U_K^{(0)}=U_K$,$U_K^{(n)},n\ge1$是高阶单位群,设$n$是最小的自然数使得$U_K^{(n)}\subset\mathrm{N}_{L/K}L^*$,就称$\mathfrak{f}=p_K^n$为扩张$K\subseteq L$的导子.导子可以完全刻画非分歧性:一个有限阿贝尔扩张$K\subseteq L$是非分歧的当且仅当扩张的导子$\mathfrak{f}=1$.
\begin{proof}
	
	一方面如果扩张是非分歧的,我们在证明上同调公理时得到推论$U_K=\mathrm{N}_{L/K}U_L$,于是有$\mathfrak{f}=1$.另一方面如果$\mathfrak{f}=1$,得到$U_K\subset\mathrm{N}_{L/K}U_L$.记$n=[K^*:\mathrm{N}_{L/K}L^*]$,那么$\pi_K^n\in\mathrm{N}_{L/K}L^*$.如果$K\subseteq M$是$n$次非分歧扩张,那么$\mathrm{N}_{M/K}M^*=(\pi_K^n)\times U_K\subset\mathrm{N}_{L/K}L^*$,于是有$L\subseteq M$,也即$K\subseteq L$是非分歧的.
\end{proof}

Kronecker-Weber理论.
\begin{enumerate}
    \item 设$K$是局部域,那么乘法群有结构$K^*\cong\pi^{\mathbb{Z}}\times U_K$.任取$K^*$的有限指数开子群$\mathscr{N}$,那么它必然包含形如$\pi^{f\mathbb{Z}}\times U_K^{(n)}$的子群.这样的子群也是有限指数开子群,所以这种子群的类域是特别重要的,后文会给出这种类域的刻画.
    \item 如果把基域取为$K=\mathbb{Q}_p$,设$\mu_{p^n}$是基域上的$p^n$阶本原单位根,那么分圆扩张$\mathbb{Q}_p\subset\mathbb{Q}_p(\mu_{p^n})$对应的范数群就是$(p)\times U_{\mathbb{Q}_p}^{(n)}$.
    \begin{proof}
    	
    	记$L=\mathbb{Q}_p(\mu_{p^n})$,那么扩张$K\subseteq L$是次数为$p^{n-1}(p-1)$的完全分歧扩张:生成元的极小多项式为$\varphi(X)=(X^{p^n}-1)/(X^{p^{n-1}}-1)=X^{p^{n-1}(p-1)}+X^{p^{n-1}(p-2)}+\cdots+1=0$.那么$\mathrm{N}_{L/K}(1-\zeta)=\prod_{\sigma}(1-\sigma\zeta)=\varphi(1)=p$.设$v_L$是$K$上$p$-adic完备离散赋值在$L$上的延拓,那么有$v_L(1-\zeta)=\frac{v_p(p)}{[L:K]}=\frac{1}{p^{n-1}(p-1)}$,于是$p\mathscr{O}_L=(1-\zeta)^{p^{n-1}(p-1)}$,于是$K\subseteq L$是完全分歧的.
    	
    	我们要说明的是$\mathrm{N}_{L/K}L^*=(p)\times U_K^{(n)}$,一旦证明右侧包含于左侧,按照$\mathbb{Q}_p^*=p^{\mathbb{Z}}\times\mathbb{Z}/(p-1)\mathbb{Z}\times U^{(1)}$和$\#U^{(n)}/U^{(1)}=p^{n-1}$,得到要证明相同的这两个子群的元素个数相同,这就得到它们是相同的.但是上一段解释了$(p)\subset\mathrm{N}_{L/K}L^*$,于是唯一需要验证的是$U_K^{(n)}\subset\mathrm{N}_{L/K}L^*$.
    	
    	我们有指数同构$\exp:p_K^v\to U_K^{(v)}$,其中$p\ge3$时只需取$v=1$,$p=2$时可以取$v=2$.对任意$s\ge1$有如下交换图.于是当$p\ge3$时有$U_K^{(n)}=(U_K^{(1)})^{p^{n-1}(p-1)}$,当$p=2$时有$U_K^{(n)}=(U_K^{(2)})^{2^{n-2}}$.按照$p^{n-1}(p-1)$就是$\mathrm{N}_{L/K}L^*$在$K^*$中的指数,说明$U_K^{(n)}=(U_K^{(1)})^{p^{n-1}(p-1)}$落在$\mathrm{N}_{L/K}L^*$中.但是$p=2$的情况还要继续处理:有$U_K^{(2)}=U_K^{(3)}\cup 5U_K^{(3)}$(因为$U_K^{(2)}$是模4余1,$U_K^{(3)}=(U_K^{2})^2$是模8余1,于是还需要并上一个模8余5).于是有$U_K^{(n)}=(U_K^{(2)})^{2^{n-1}}\cup 5^{2^{n-2}}(U_K^{(2)})^{2^{n-1}}$.这个并的前者必然落在$\mathrm{N}_{L/K}L^*$中(因为$2^{n-1}$是扩张次数),而$5^{2^{n-2}}=\mathrm{N}_{L/K}(2+i)\in\mathrm{N}_{L/K}L^*$.这就说明了$U^{(n)}\subset\mathrm{N}_{L/K}L^*$.
    	$$\xymatrix{p_K^v\ar[rr]^{\cong}\ar[d]^{\cong}_{(p-1)p^{s-1}}&&U_K^{(v)}\ar[d]^{\exp p^{s-1}(p-1)}_{\cong}\\p_K^{v+s-1}\ar[rr]^{\cong}&&U_K^{(v+s-1)}}$$
    \end{proof}
    \item 局部Kronecker-Weber定理.$\mathbb{Q}_p$的每个有限阿贝尔扩张都包含在某个分圆扩张中,换句话讲,$\mathbb{Q}_p$的极大阿贝尔扩张$\mathbb{Q}_p^{\mathrm{ab}}$通过添加全部单位根得到.
    \begin{proof}
    	
    	我们先解释$(p^f)\times U_{\mathbb{Q}_p}$对应的类域就是$\mathbb{Q}_p(\mu_{p^f-1})$.这实际上是一个$f$次非分歧扩张.【?】
    	
    	接下来任取有限阿贝尔扩张$L$,那么可选取$f$和不被$p$整除的$n$使得$(p^f)\times U_{\mathbb{Q}_p}^{(n)}\subset\mathrm{N}_{L/K}L^*$.而这里$(p^f)\times U_{\mathbb{Q}_p}^((n))=((p^f)\times U_{\mathbb{Q}_p})\cap((p)\times U_{\mathbb{Q}_p}^{(n)})$.按照阿贝尔扩张和范群的对应,这两个相交的群分别对应于扩张$\mathbb{Q}_p(\mu_{p^n})$和$\mathbb{Q}_p(\mu_{p^f-1})$,于是交对应的扩张是这两个扩张的合成,也即$\mathbb{Q}_p(\mu_{p^n(p^f-1)})$,并且$L$是包含于这个分圆扩张.
    \end{proof}
    \item 经典Kronecker-Weber定理.$\mathbb{Q}$的每个有限阿贝尔扩张$L$都包含在某个分圆扩张$\mathbb{Q}(\zeta)$中,换句话讲$\mathbb{Q}$的极大阿贝尔扩张$\mathbb{Q}^{\mathrm{ab}}$是通过添加全部单位根得到的.
    \begin{proof}
    	
    	记$S$表示所有在$L$中分歧的素数,这是一个有限集合.对每个$p\in S$,记$L_p$表示$L$在$p$的某个提升的完备化.那么$\mathbb{Q}_p\subseteq L_p$也是有限阿贝尔扩张,于是按照局部版本的结论,有$L_p\subset\mathbb{Q}_p(\mu_{n_p})$.我们记$n=\prod_{p\in S}p^{e_p}$,其中$e_p=v_p(n_p)$,换句话讲它是$n_p$唯一分解中$p$的次数.仅需验证$L\subset\mathbb{Q}(\mu_n)$,为此仅需验证$M=L(\mu_n)$满足$M=\mathbb{Q}(\mu_n)$.
    	
    	$\mathbb{Q}\subseteq M$是一个有限阿贝尔扩张,并且对$S$以外的素数$p$在$\mathbb{Q}\subseteq M$上都是不分歧的.取$p$是$\mathbb{Q}\subseteq M$的分歧素数,那么$p\in S$,选取$p$在$M$中适当提升素理想,使得完备化$M_p$在$L$上的限制就是$L_p$.于是有$M_p=L_p(\mu_n)=\mathbb{Q}_p(\mu_{p^{e_p}n'})=\mathbb{Q}_p(\mu_{p^{e_p}})\mathbb{Q}_p(\mu_{n'})$,其中$n'$和$p$互素,$\mathbb{Q}_p\subset\mathbb{Q}_p(\mu_{n'})$就是$\mathbb{Q}_p\subseteq M_p$的极大非分歧扩张.于是$\mathbb{Q}_p\subseteq M_p$的惯性群$I_p=G(\mathbb{Q}_p(\mu_{p^{e_p}})/\mathbb{Q}_p)$,于是$I_p$有阶数$\varphi(p^{e_p})$.这些$I_p$都可以视为$G(M/\mathbb{Q})$的子群,这些$I_p,p\in S$生成了$G(M/\mathbb{Q})$的子群$I$,并且$I$的固定域在每个有限素位非分歧.但是我们解释过$\mathbb{Q}$的极大非分歧扩张就是$\mathbb{Q}$自己,于是$I=G(M/\mathbb{Q})$.
    	
    	最后我们有$\#I\le\prod_{p\in S}\#I_p=\prod_{p\in S}\varphi(p^{e_p})=\varphi(n)=[\mathbb{Q}(\mu_n):\mathbb{Q}]$,于是$M=\mathbb{Q}(\mu_n)$.
    \end{proof}
\end{enumerate}

为了处理一般局部域上的分圆扩张,需要引入形式群的概念.环$A$上的一个形式群是指一个形式幂级数$F(X,Y)\in A[[X,Y]]$,满足如下三个条件.
\begin{itemize}
	\item $F(X,Y)\equiv X+Y(\mathrm{mod}\deg2)$.换句话讲$F$的不超过1次的部分必须是$X+Y$.
	\item $F(X,Y)=F(Y,X)$.
	\item $F(X,F(Y,Z))=F(F(X,Y),Z)$.
\end{itemize}
\begin{enumerate}
	\item 例如$G_a(X,Y)=X+Y$定义的就是常规的加法群结构,$G_a$称为形式加法群.再比如定义$G_m(X,Y)=X+Y+XY$,那么有$(x+_{G_m}y)+1=(x+1)(y+1)$,它称为形式乘法群.
	\item 再比如如果$f(X)=a_1X+a_2X^2+\cdots\in A[[X]]$,其中$a_1$是乘法逆元,那么存在形式幂级数$f^{-1}(X)=a_1^{-1}X+\cdots$是复合意义下$f$的逆元.对于这样的$f$,定义$F(X,Y)=f^{-1}(f(X)+f(Y))$是一个形式群.这里$f$称为$F$的对数.
	\item 形式群之间的同态.形式群是一个二元形式幂级数,两个形式群之间的同态$f:F\to G$定义为一个一元形式幂级数$f(X)=a_1X+a_2X^2+\cdots\in XA[[X]]$,使得$f(F(X,Y))=G(f(X),f(Y))$.换句话讲有$f(X+_Fy)=f(X)+_Gf(Y)$.这个同态称为同构,如果$a_1=f'(0)$是$A$中单位(因为形式幂级数环上在复合运算下是可逆元当且仅当它的常数项为0,一次项系数是基环中的单位,此时可以反解出复合逆),换句话讲存在同态$g=f^{-1}:G\to F$使得$f(g(X))=g(f(X))=X$.
	\item 例如如果$A$是一个$\mathbb{Q}$代数,换句话讲$n$都是可逆的,那么存在从$G_m$到$G_a$的形式群同构为$\mathrm{Ln}(X)=\ln(1+X)=\sum_{n\ge1}(-1)^{n+1}\frac{X^n}{n}$,那么有$\mathrm{Ln}(X+_{G_m}Y)=\ln((1+X)(1+Y))=\ln(1+X)+\ln(1+Y)=\mathrm{Ln}(X)+_{G_a}\mathrm{Ln}(Y)$.
	\item 设$F$是$A$上一个形式群,$F$上全部自同态构成一个环$\mathrm{End}_A(F)$,加法定义为$(f+_Fg)(X)=F(f(X),g(X))$,乘法定义为$(f\circ g)(X)=f(g(X))$.
\end{enumerate}

Lubin-Tate级数.设$K$是局部域,设$\pi$是它素元,一个级数$e(X)\in\mathscr{O}_K[[X]]$称为Lubin-Tate级数,如果满足$e(X)\equiv\pi X(\mathrm{mod}\deg2)$和$e(X)\equiv X^q(\mathrm{mod}\pi)$.全体$\pi$的Lubin-Tate级数构成的集合记作$S_{\pi}$.
\begin{enumerate}
	\item 例如取$e(X)=uX^q+\pi(a_{q-1}X^{q-1}+\cdots+a_2X^2)+\pi X$,其中$u\equiv1(\mathrm{mod}\pi)$,在$S_{\pi}$中,它称为Lubin-Tate多项式.
	\item 如果$\pi,\pi'$是$K'=\widehat{K^{\mathrm{ur}}}$的素元,如果$e(X)\in S_{\pi}$和$e'(X)\in S_{\pi'}$是Lubin-Tate级数.取$G(F/K)$中的Frobenius映射,设$L(X_1,X_2,\cdots,X_n)$是$\mathscr{O}_{K'}$系数的线性齐次多项式,满足$\pi L(X_1,X_2,\cdots,X_n)=\pi'L^{\phi}(X_1,X_2,\cdots,X_n)$,这里$L^{\phi}$表示$L$的系数都作用了映射$\phi$.那么存在唯一的$\mathscr{O}_{K'}$系数的形式幂级数$F(X_1,X_2,\cdots,X_n)$满足$F=L(\mathrm{mod}\deg2)$,并且$e(F(X_1,X_2,\cdots,X_n))=F^{\phi}(e'(X_1),e'(X_2),\cdots,e'(X_n))$.
\end{enumerate}

形式$A$模和Lubin-Tate形式模.
\begin{enumerate}
	\item 设$F$是$A$上的形式群,选定一个环同态$A\to\mathrm{End}_A(F)$,作用记作$a\mapsto[a]_F(X)$,要求$[a]_F(X)\equiv aX(\mathrm{mod}\deg2)$.两个形式$A$模之间的同态$f:F\to G$是指形式群之间的同态$f$,额外满足$f([a]_F(X))=[a]_G(f(X)),\forall a\in A$.
	\item 设$K$是局部域,$\pi$是素元,$q$是剩余域的大小,如果$\mathscr{O}_K$形式模$F$满足$[\pi]_F(X)\equiv X^q(\mathrm{mod}\pi)$,就称$F$是一个Lubin-Tate模.
	\item 取$\pi$的两个Lubin-Tate级数$e(X),e'(X)$,那么存在唯一的形式幂级数$F_e(X,Y)$和$[a]_{e,e'}(X)$,满足如下四件事,另外如果$e=e'$,就把$[a]_{e,e'}(X)$简单记作$[a]_e(X)$.
	\begin{itemize}
		\item $F_e(X,Y)\equiv X+Y(\mathrm{mod}\deg2)$.
		\item $e(F_e(X,Y))=F_e(e(X),e(Y))$.
		\item $[a]_{e,e'}(X)\equiv aX(\mathrm{mod}\deg2)$.
		\item $e([a]_{e,e'}(X))=[a]_{e,e'}(e'(X))$.
	\end{itemize}
    \item 给定$\pi$的Lubin-Tate级数$e(X)$,那么上一条中的$F_e(X,Y)$是唯一的Lubin-Tate形式模,使得$e\in\mathrm{End}_{\mathscr{O}_K}(F)$,并且$[\pi]_F(X)=e(X)$.
    \item 如果$e,e'$是两个Lubin-Tate级数,设它们对应的形式模分别是$F,F'$,那么$[a]_{F,F'}(X)$是$F\to F'$的同态,并且如果$a$是单位时它是一个同构.特别的,同一个素元的不同Lubin-Tate形式模是同构的.
    \item 例子.取$K=\mathbb{Q}_p$,取Lubin-Tate级数$e(X)=(1+X)^p-1$,此时$\mathbb{Z}_p\to\mathrm{End}_{\mathbb{Z}_p}(G_m)$为$a\mapsto (1+X)^a-1$(的形式幂级数展开).这使得$G_m$是一个形式$\mathbb{Z}_p$模.
\end{enumerate}

运用在局部域上.
\begin{enumerate}
	\item 设$K$是局部域,取它的代数闭包$\overline{K}$,取一个$\mathscr{O}_K$形式模$F$,对$x,y\in\overline{m}$是$\mathscr{O}_{\overline{K}}$极大理想中的元,那么$F(x.y)$总是收敛的,于是$x+_Fy=F(x,y)$和$ax=[a]_F(x)$定义了$\overline{m}$上的一个新的加法群结构以及一个$\mathscr{O}_K$模结构.我们把这个$\mathscr{O}_K$模结构记作$\overline{m}_F$.
	\item 如果$f:F\to G$是形式$\mathscr{O}_K$模之间的模同态,那么$f:\overline{m}_F\to\overline{m}_G$,$x\mapsto f(x)$是一个$\mathscr{O}_K$模同态.
	\item 如果$F$是$\mathscr{O}_K$的一个Lubin-Tate形式模,设素元为$\pi$,定义$\pi^n$除点(division points)集合$F(n)=\{\lambda\in\overline{m}_F\mid\pi^n\lambda=0\}$,此即$\ker([\pi^n]_F)$.例如当我们把$G_m$视为一个形式$\mathbb{Z}_p$模时,有$\ker[p^n]_{G_m}=\{\zeta-1\mid\zeta\in\mu_{p^n}\}$.
	\item $F(n)$是一个$\mathscr{O}_K$模,它被$\pi^n$零化,于是它也是一个$\mathscr{O}_K/\pi^n\mathscr{O}_K$模.我们断言这是一个秩1的自由模.
	\begin{proof}
		
		我们解释过同一个素元的不同Lubin-Tate形式模都是同构的.这样的同构$f:F\to G$诱导了$\mathscr{O}_K$模同构$\overline{m}_F\to\overline{m}_G$,又诱导了$\mathscr{O}_K$模同构$f:F(n)\to G(n)$.于是问题归结为设$F=F_e$,其中$e(X)=X^q+\pi X=[\pi]_F(X)$的情况.此时$F(n)$是$e(X)$的$n$次迭代的零点集,这个迭代是一个$q^n$次可分多项式,从而零点集恰好由$q^n$个点构成.选取$\lambda_n\in F(n)-F(n-1)$,于是$\mathscr{O}_K\to F(n)$的映射$a\mapsto a\lambda_n$是一个核为$\pi^n\mathscr{O}_K$的同态,于是它诱导了单同态$\mathscr{O}_K/\pi^n\mathscr{O}_K$,但是这两侧的模都是$q^n$个元素的,所以这个单同态是一个同构.
	\end{proof}
	\item 映射$a\mapsto[a]_F$诱导了典范同构$\mathscr{O}_K.\pi^n\mathscr{O}_K\cong\mathrm{End}_{\mathscr{O}_K}(F(n))$和典范同构$U_K/U_K^{(n)}\cong\mathrm{Aut}_{\mathscr{O}_K}(F(n))$.
	\begin{proof}
		
		前一个同构因为$\mathscr{O}_K/\pi^n\mathscr{O}_K\cong F(n)$和$\mathrm{End}_{\mathscr{O}_K}(\mathscr{O}_K/\pi^n\mathscr{O}_K)=\mathscr{O}_K/\pi^n\mathscr{O}_K$.对这个环同构取单位群得到第二个同构.
	\end{proof}
\end{enumerate}

Lubin-Tate扩张.称$L_n=K(F(n))$为$K$的$\pi^n$除点域.于是得到域扩张链$K\subseteq L_1\subseteq L_2\subset\cdots\subseteq L_{\pi}=\cup_{n\ge1}L_n$.这些域扩张也称为Lubin-Tate扩张.
\begin{enumerate}
	\item 局部域$K$的Lubin-Tate扩张只依赖于素元$\pi$,和它的Lubin-Tate模$F$的选取无关.事实上如果再取一个Lubin-Tate模$G$,我们解释过存在同构$f:F\to G$,它是$\mathscr{O}_K$上的一个形式幂级数,并且诱导了同构$f:F(n)\to G(n)$,于是有$G(n)=f(F(n))\subseteq K(F(n))$(因为完备性),于是$K(G(n))\subseteq K(f(n))$,同理有另一侧包含关系,于是$K(F(n))=K(G(n))$.
	\item 于是选取$\pi$的一个Lubin-Tate多项式$e(X)$,那么$K\subseteq L_n$就是$e(X)$的$n$次迭代这个可分多项式的分裂域,于是这是Galois扩张.
	\item 例如选取$K=\mathbb{Q}_p$,选取Lubin-Tate多项式$e(X)=(1+X)^p-1$,那么$n$次迭代$e^{(n)}(X)=(1+X)^{p^n}-1$,它的零点集就是$\zeta-1$,其中$\zeta$取遍$p^n$阶单位根,此时Lubin-Tate扩张$L_n=\mathbb{Q}_p(\mu_{p^n})$,即$p^n$次的分圆扩张.
	\item Lubin-Tate扩张$K\subseteq L_n$是次数为$q^{n-1}(q-1)$的完全分歧阿贝尔扩张,它的Galois群为$G(L_n/K)\cong\mathrm{Aut}_{\mathscr{O}_K}(F(n))\cong U_K/U_K^{(n)}$.这个同构把$\sigma\in G(L_n/K)$映射为唯一的等价类$u\mathrm{mod}U_K^{(n)}$,使得对每个$\lambda\in F(n)$有$\lambda^{\sigma}=[u]_F(\lambda)$.另外如果$e(X)$是$\pi$的一个Lubin-Tate多项式,设它对应的Lubin-Tate模为$F$,任取$\lambda_n\in F(n)-F(n-1)$,那么$L_n=K(\lambda_n)$,它的极小多项式为$\varphi_n(X)=\frac{e^{(n)}(X)}{e^{(n-1)}(X)}=X^{q^{n-1}(q-1)}+\cdots+\pi\in\mathscr{O}_K(X)$,并且有$\mathrm{N}_{L_n/K}(-\lambda_n)=\pi$.
	\begin{proof}
		
		取$e(X)=X^q+\pi(a_{q-1}X^{q-1}+\cdots+a_2X^2)+\pi X$是一个Lubin-Tate多项式,那么有:
		$$\varphi_n(Xx)=\frac{e^{(n)}(X)}{e^{(n-1)}(X)}=e^{(n-1)}(X)^{q-1}+\pi\left(a_{q-1}e^{(n-1)}(X)^{q-2}+\cdots+a_2e^{(n-1)}(X)\right)+\pi$$
		
		是一个次数为$q^{n-1}(q-1)$的爱森斯坦多项式.选取$e$对应的一个Lubin-Tate模$F$,那么$\varphi_n$的根恰好对应于$F(n)-F(n-1)$的元,任取这样的元$\lambda_n$,我们期望说明$L_n=K(\lambda_n)$.任取$\sigma\in G(L_n/K)$,它诱导了$F(n)$上的一个自同构,于是得到同态$G(L_n/K)\to\mathrm{Aut}_{\mathscr{O}_K}(F(n))$,这是单射因为$F(n)$就是生成$L_n$的集合,它是满射因为考虑元素个数有$\#G(L_n/K)\ge[K(\lambda_n):K]=q^{n-1}(q-1)=\#U_k/U_K^{(n)}$,于是有$L_n=K(\lambda_n)$.最后按照$\varphi_n$是爱森斯坦多项式,于是$\lambda_n$的赋值是$1/q^{n-1}(q-1)$,于是这个扩张是完全分歧的.
	\end{proof}
	\item 设$L_n$是局部域$K$的Lubin-Tate扩张,记$a=u\pi^{v_K(a)}\in K^*$,其中$u\in U_K$,那么有$(a,L_n/K)(\lambda)=[u^{-1}]_F(\lambda)$对任意$\lambda\in F(n)$成立.
	\begin{proof}
		
		首先我们解释过$\pi$是一个范数($\mathrm{N}_{L_n/K}(-\lambda_n)=\pi$),于是$\pi$在范剩余符号下是平凡的,于是问题归结为设$a=u\in U_K$.设$\sigma$满足$\lambda^{\sigma}=[u^{-1}]_F(\lambda),\forall\lambda\in F(n)$,选取$\widetilde{\sigma}\in G(\widetilde{L_n}/K)$使得$\widetilde{\sigma}\mid L_n=\sigma$,并且有$d(\widetilde{\sigma})=1$(因为$K\subseteq L_n$是完全分歧的).记$\check{K}$为$\widetilde{K}$的完备化(这未必是一个代数扩张),记$\check{L_n}=L_n\check{K}$,把$\widetilde{\sigma}$视为$\check{L_n}$上的自同构,设$\widetilde{\sigma}$的固定域为$\Sigma$.那么$f_{\Sigma/K}=d_K(\widetilde{\sigma})=1$,于是$K\subset\Sigma$是一个完全分歧扩张,于是我们有如下两个扩张四边形,于是有$[\Sigma:K]=[\widetilde{\Sigma}:\widetilde{K}]=[\widetilde{L_n}:\widetilde{K}]=[L_n:K]=q^{n-1}(q-1)$.
		$$\xymatrix{\widetilde{\Sigma}\ar@{=}[rrrr]\ar@{-}[drr]\ar@{-}[dd]&&&&\widetilde{L_n}\ar@{-}[dll]\ar@{-}[dd]\\&&\widetilde{K}\ar@{-}[dd]&&\\\Sigma\ar@{-}[drr]&&&&L_n\ar@{-}[dll]\\&&K&&}$$
		
		记$\pi=u\pi‘$,设$\pi$和$\pi'$的Lubin-Tate级数分别为$e,e'$,取对应的Lubin-Tate模分别为$F,F'$,我们解释过存在低于1次项部分是$\varepsilon X$的形式幂级数$\theta(X)\in\mathscr{O}_{\check{K}}[[X]]$,其中$\varepsilon\in U_{\check{K}}$,满足$\theta^{\varphi}=\theta\circ[u]_{F'}$和$\theta^{\varphi}\circ e\circ\theta$.这里$\varphi=\varphi_K$.考虑$\lambda_n\in F'(n)-F'(n-1)$,它是$L_n$的素元.我们断言$\pi_{\Sigma}=\theta(\lambda_n)$是$\Sigma$上的素元:因为有$\pi_{\Sigma}^{\widetilde{\sigma}}=\theta^{\varphi}(\lambda_n^{\sigma})=\theta^{\varphi}([u^{-1}]_{F'}(\lambda_n))=\theta(\lambda_n)=\pi_{\Sigma}$,于是$\pi_{\Sigma}\in\Sigma$,它是素元因为$\lambda_n$是素元.
		
		现在$e^{(i)}(\theta(\lambda_n))=\theta^{\varphi^i}(\lambda_n^{\sigma})=\theta^{\varphi}((e')^{(i)}(\lambda_n))$,在$i=n$时为零,在$i=n-1$时非零.于是$\pi_{\Sigma}\in F(n)-F(n-1)$.于是有$\Sigma=K(\pi_{\Sigma})=K(F(n))$.并且$\mathrm{N}_{\Sigma/K}(-\pi_{\Sigma})=\pi=u\pi'$.但是$\pi'=\mathrm{N}_{L_n/K}(-\lambda_n)\in\mathrm{N}_{L_n/K}L_n^*$,于是有$r_{L_n/K}(\sigma)=\mathrm{N}_{\Sigma/K}(-\pi_{\Sigma})=\pi\equiv u(\mathrm{mod}\mathrm{N}_{L_n/K}L_n^*)$.于是有$(u,L_n/K)=\sigma$.
	\end{proof}
    \item Lubin-Tate扩张$K\subseteq L_n$的范群就是$(\pi)\times U_K^{(n)}\subseteq K^*$.
    \begin{proof}
    	
    	记$a=u\pi^{v_K(a)}\in K^*$,那么有:
    	\begin{align*}
    	a\in\mathrm{N}_{L_n/K}L_n^*&\Leftrightarrow(a,L_n/K)=1\Leftrightarrow[u^{-1}]_F(\lambda)=\lambda,\forall\lambda\in F(n)\\&\Leftrightarrow[u^{-1}]_F=\mathrm{id}_{F(n)}\Leftrightarrow u^{-1}\in U_K^{(n)}\Leftrightarrow a\in(\pi)\times U_K^{(n)}
    	\end{align*}
    \end{proof}
\end{enumerate}

借助Lubin-Tate扩张可以描述一般局部域$K$上的极大阿贝尔扩张:$K$的极大阿贝尔扩张$K^{\mathrm{ab}}=\widetilde{K}L_{\pi}$,其中$\widetilde{K}$是极大非分歧扩张,$L_{\pi}$是全部Lubin-Tate扩张的并$L_{\pi}=\cup_{n\ge1}L_n$.
\begin{proof}
	
	设$K\subseteq L$是一个有限阿贝尔扩张.那么范群$\mathrm{N}_{L/K}L^*$包含了某个$\pi^f$.按照$\mathrm{N}_{L/K}L^*$是$K^*$的开子群,并且$U_K^{(n)},n\ge1$构成1的开邻域基,于是有某个$n$使得$(\pi^f)\times U_K^{(n)}\subset\mathrm{N}_{L/K}L^*$.而这个群就是$\left((\pi)\times U_K^{(n)}\right)\cap\left((\pi^f)\times U_K\right)$.这个交前者的类域是$L_n$,后者的类域是一个扩张次数为$f$的非分歧扩张.这说明$L$包含在$\widetilde{K}L_{\pi}$中.反过来$\widetilde{K}$和$L_{\pi}$本身都是阿贝尔扩张,这就得证.
\end{proof}

尽管范剩余符号要求局部域,这里我们补充定义$(a,\mathbb{C}/\mathbb{R})=\mathrm{sgn}(a)\in G(\mathbb{C}/\mathbb{R})$.

希尔伯特符号.设$K$是包含$n$次单位根群的局部域,或者$K=\mathbb{R},\mathbb{C}$(此时$n$只能取2或者1),设$L=K(\sqrt[n]{K^*})$是极大的指数$n$的阿贝尔扩张,这里$n$是一个和$\mathrm{char}K$互素的.按照局部类域论,有$G(L/K)\cong K^*/K^{*n}$.按照Kummer理论有同构$\mathrm{Hom}(G(L/K),\mu_n)\cong K^*/K^{*n}$.这得到一个非退化的双线性型$G(L,K)\times\mathrm{Hom}(G(L/K),\mu_n)\to\mu_n$为$(\sigma,\chi)\mapsto\chi(\sigma)$.于是按照同构得到一个非退化的双线性型$(-,-)_p:K^*/K^{*n}\times K^*/K^{*n}\to\mu_n$.这里记号里强调极大理想$p$是因为一般要考虑整体域上关于某个素理想的希尔伯特符号.
\begin{enumerate}
	\item 对$a,b\in K^*$,有:$$(a,K(\sqrt[n]{b})/K)\sqrt[n]{b}=(a,b)_p\sqrt[n]{b}$$
	\begin{proof}
		
		$a\in K^*$在同构$K^*/K^{*n}\cong G(L/K)$下的像是$\sigma=(a,L/K)$,而$b\in K^*$在同构$K^*/K^{*n}\cong\mathrm{Hom}(G(L/K),\mu_n)$下的像是$\chi_b$,这满足$\chi_b(\tau)=\tau(\sqrt[n]{b})/\sqrt[n]{b}$.于是按照定义有$(a,b)_p=\chi_b(a)=\sigma(\sqrt[n]{b})/\sqrt[n]{b}$,整理得到公式.
	\end{proof}
    \item 一些基本公式.
    \begin{itemize}
    	\item 乘性,即$(aa',b)=(a,b)(a',b)$,$(a,bb')=(a,b)(a,b')$.
    	\item $(a,b)=1$当且仅当$a$是扩张$K\subseteq K(\sqrt[n]{b})$的范数.
    	\item 有$(a,1-a)=1$,$(a,-a)=1$.
    	\item $(a,b)=(b,a)^{-1}$.
    	\item 如果$(a,b)=1$对任意$b\in K^*$成立,那么有$a\in K^{*n}$.
    	\item 另外对于$\mathbb{R}\subset\mathbb{C}$的情况下有$(a,b)=(-1)^{\frac{(\mathrm{sgn}(a)-1)(\mathrm{sgn}(b)-1)}{4}}$.并且此时也满足上述性质.
    \end{itemize}
    \begin{proof}
    	
    	首先$(-,b)$的乘性直接来自范剩余符号是乘性的.第二件事是因为我们上面给出的公式,从$(a,b)=1$得到$(a,K(\sqrt[n]{b})/K)=1$,于是$a\in\mathrm{N}_{K(\sqrt[n]{b})/K}K(\sqrt[n]{b})^*$.第三件事:先设$b\in K^*$和$x\in K$使得$x^n-b\not=0$,那么有$x^n-b=\prod_{0\le i\le n-1}(x-\zeta^i\beta)$,其中$\beta^n=b$,$\zeta$是$n$次本原单位根.设$d$是最大的$n$的因数使得$y^d=b$在$K$中有解,设$n=dm$,那么扩张$K\subseteq K(\beta)$是$m$次循环扩张,于是$x-\zeta^i\beta$的共轭元是形如$x-\zeta^j\beta$,其中$j\equiv i(\mathrm{mod}d)$的元,于是有$x^n-b=\prod_{0\le i\le d-1}\mathrm{N}_{K(\beta)/K}(x-\zeta^i\beta)$,于是$x^n-b$是扩张$K\subseteq K(\sqrt[n]{b})$的范数.于是有$(x^n-b,b)=1$.选取$x=1,b=1-a$和$x=0,-a$得到第三件事.第四件事只要注意到:
    	$$(a,b)(b,a)=(a,-a)(a,b)(b,a)(b,-b)=(a,-ab)(b,-ab)=(ab,-ab)=1$$
    \end{proof}
\end{enumerate}

温希尔伯特符号(tame Hilbert symbol).设$K$是局部域,并且它的特征不整除扩张次数$n$,此时把希尔伯特符号称为温希尔伯特符号.
\begin{enumerate}
	\item 在条件下有$n\mid q-1$,其中$q$是$K$的剩余域$\kappa$的阶数.因为此时有$\kappa^*$包含了$\mu_n$,但是$\kappa^*$是$q-1$阶循环群,所以$n\mid q-1$.
	\item 引理.设$p\not| n$,$x\in K^*$,那么扩张$K\subseteq K(\sqrt[n]{x})$是非分歧的当且仅当$x\in U_KK^{*n}$.
	\begin{proof}
		
		充分性.设$x=uy^n$,其中$u\in U_K$,$y\in K^*$.那么$K(\sqrt[n]{x})=K(\sqrt[n]{u})$.设$X^n-u(\mathrm{mod}p)$在$K$的剩余域$\kappa$上的分裂域为$\kappa'$.这是一个可分扩张,于是存在非分歧扩张$K\subseteq K'$使得剩余域扩张就是$\kappa\subset\kappa'$.按照Hensel引理有$X^n-u$在$K'$上分裂,于是$K(\sqrt[n]{u})\subseteq K'$是非分歧的.这说明$K\subseteq K(\sqrt[n]{x})$是非分歧的.
		
		必要性.设$L=K(\sqrt[n]{x})$在$K$上非分歧,设$x=u\pi^r$,其中$u\in U_k$,$\pi$是$K$的素元.那么有$v_L(\sqrt[n]{u\pi^r})=\frac{1}{n}v_L(\pi^r)=\frac{r}{n}\in\mathbb{Z}$,于是$n\mid r$,导致$\pi^r\in K^{*n}$,于是$x\in U_KK^{*n}$.
	\end{proof}
    \item 按照$U_K=\mu_{q-1}\times U_K^{(1)}$,即每个$u\in U_K$都可以表示为$u=\omega(u)u'$,其中$\omega(u)\in\mu_{q-1}$,$u'\in U_K^{(1)}$,并且$u\equiv\omega(u)(\mathrm{mod}p)$.在这个记号下我们有:如果$p\not| n$,并且$a,b\in K^*$,那么有$(a,b)_p=\omega\left((-1)^{\alpha\beta}\frac{b^{\alpha}}{a^{\beta}}\right)^{(q-1)/n}$,其中$\alpha=v_K(a),\beta=v_K(b)$.
    \begin{proof}
    	
    	这个公式的两侧都是关于$a$和$b$的(乘性)双线性型的.于是不妨设$a=\pi$是素元,$b=-\pi u$,其中$u$是单位.把要证的这个公式右侧记作$\langle a,b\rangle$.再按照$\langle\pi.-\pi\rangle=(\pi,=\pi)_p=1$.于是问题归结为设$a=\pi$是素元,$b=u$是单位.设$y=\sqrt[n]{u}$和$K'=K(y)$,那么$\langle\pi,u\rangle=\omega(u)^{(q-1)/n}$和$(\pi,K'/K)y=(\pi,u)_py$.上面引理已经解释了这里$K\subseteq K'$是非分歧的.另外我们解释过这里$(\pi,K'/K)$就是Frobenius自同构$\phi=\phi_{K'/K}$,于是有:
    	$$(\pi,u)_p=\phi(y)/y\equiv y^{q-1}\equiv u^{(q-1)/n}\equiv\omega(u)^{(q-1)/n}\equiv\langle\pi,u\rangle(\mathrm{mod}p)$$
    	
    	这是两个$\mu_n$中的元素在$\mathrm{mod}p$下相同,于是它们相同.
    \end{proof}
    \item 上一条说明$(\pi,u)_p=\omega(u)^{(q-1)/n}$不依赖于素元$\pi$的选取.我们定义此时勒让德符号$\left(\frac{u}{p}\right)=(\pi,u)_p,u\in U_K$.它推广了二次剩余符号:如果$(n,p)=1$,并且$u\in U_K$,那么$\left(\frac{u}{p}\right)=1$当且仅当$u$在剩余域$\kappa$中存在$n$次根.
    \begin{proof}
    	
    	设$\zeta$是$q-1$次本原单位根,设$m=(q-1)/n$,那么$\zeta^n$是$m$次本原单位根.并且有:
    	\begin{align*}
    	\left(\frac{u}{p}\right)=\omega(u)^m=1&\Leftrightarrow\omega(u)\in\mu_m\Leftrightarrow\omega(u)=(\zeta^n)^i\\&\Leftrightarrow u\equiv\omega(u)=(\zeta^i)^n(\mathrm{mod}p)
    	\end{align*}
    \end{proof}
\end{enumerate}

类域论观点下的高阶分歧群.
\begin{enumerate}
	\item 考虑有限阿贝尔扩张的范剩余符号$(-,L/K):K^*\to G(L/K)$.这两侧都有一个典范滤过.左侧的是高阶单位群$K^*\supset U_K=U_K^{(0)}\supset U^{(1)}\supset\cdots$.右侧的是上指标的高阶分歧群,它的定义如下:首先定义下指标的高阶分歧群为$G_i(L/K)=\{\sigma\in G(L/K)\mid v_L(\sigma(a)-a)\ge i+1,\forall a\in\mathscr{O}_L\}$.取单调增函数$\eta_{L/K}(s)=\int_0^s\frac{\mathrm{d}x}{[G_0:G_x]}$,反函数记作$\psi$,定义上指标高阶分歧群为$G^i(L/K)=G_{\psi_{L/K}(i)}(L/K)$.
	\item 取局部域$K$的素元$\pi$,取它的Lubin-Tate扩张$L_n$,那么对$q^{k-1}\le i\le q^k-1$,有$G_i(L_n/K)=G(L_n/L_k)$.
	\begin{proof}
		
		我们解释过范剩余符号诱导了同构$U_K/U_K^{(k)}\cong G(L_k/K)$.于是有$G(L_n/L_k)=(U_K^{(k)},L_n/K)$.于是问题归结为证明$G_i(L_n/K)=(U_K^{(k)},L_n/K)$,其中$q^{k-1}\le i\le q^k-1$.
		
		取$\sigma\in G_1(L_n/K)$,设$\sigma=(u^{-1},L_n/K)$,于是有$u\in U_K^{(1)}$,因为同构$(-,L_n/K):U_K/U_K^{(n)}\cong G(L_n/K)$把Sylow-$p$子群$U_K^{(1)}/U_K^{(n)}$映射为$G(L_n/K)$的Sylow-$p$子群$G_1(L_n/K)$.于是我们证明了$i=1$的情况.
		
		现在设$u=1+\varepsilon\pi^m$,其中$\varepsilon\in U_K$,设$\lambda\in F(n)-F(n-1)$,我们解释过$\lambda$是$L_n$的一个素元,并且有$\lambda^{\sigma}=[u]_F(\lambda)=F(\lambda,[\varepsilon\pi^m]_f(\lambda))$.倘若$m\ge n$,那么$\sigma=1$,此时$v_{L_n}(\lambda^{\sigma}-\lambda)=\infty$;倘若$m<n$,那么$\lambda_{n-m}=[\pi^m]_F(\lambda)$是$L_{n-m}$的素元,于是$[\varepsilon\pi^m]_F(\lambda)=[\varepsilon]_F(\lambda_{n-m})$也是$L_{n-m}$的素元.按照$L_{n-m}\subseteq L_n$是扩张次数为$q^m$的完全分歧扩张,可记$[\varepsilon\pi^m]_F(\lambda)=\varepsilon_0\lambda^{q^m}$,其中$\varepsilon_0\in U_{L_n}$.按照$F(X,0)=X$和$F(0,Y)=Y$,得到$F(X,Y)=X+Y+XYG(X,Y)$,其中$G(X,Y)\in\mathscr{O}_K[[X,Y]]$.于是$i_{L_n/K}(\sigma)=v_{L_n}(\lambda^{\sigma}-\lambda)$在$m<n$时取$q^m$,在$m\ge n$时取无穷.我们之前解释过$G_i(L_n/K)=\{\sigma\in G(L_n/K)\mid i_{L_n/K}(\sigma)\ge i+1\}$.设$q^{k-1}\le i\le q^k-1$,如果$u\in U_K^{(k)}$,那么$m\ge k$,也即$i_{L_n/K}(\sigma)\ge q^k\ge i+1$,于是$\sigma\in G_i(L_n/K)$,这说明$(U_K^{(k)},L_n/K)\subseteq G_i(L_n/K)$.反过来如果$\sigma\in G_i(L_n/K)$并且$\sigma\not=1$,那么$i_{L_n/K}(\sigma)=q^m>i\ge q^{k-1}$,于是$m\ge k$,于是$u\in U_K^{(k)}$,这证明了另一侧的包含关系.
	\end{proof}
	\item 如果$K\subseteq L$是有限阿贝尔扩张,那么$U_K^{(n)}$在范剩余符号$(-,L/K):K^*\to G(L/K)$下的像就是上指标分歧群$G^n(L/K),\forall n\ge0$.
	\item (Hasse-Arf)上指标分歧群$G^t(L/K)$是对任意实数$t\ge-1$定义的,它的跳跃点是指那些对任意$\varepsilon>0$都有$G^t(L/K)\not= G^{t+\varepsilon}(L/K)$的点$t$.我们断言对于有限阿贝尔扩张$K\subseteq L$,它高阶分歧群的跳跃点恰好是有理整数.
\end{enumerate}
\newpage
\subsection{整体类域论}

阿代尔(ad\'ele)和依代尔(id\'ele).给定数域$K$.
\begin{itemize}
	\item $K$的阿代尔环$\mathbb{A}_K$是指限制直积$\prod_v(K_v,\mathscr{O}_v)$,其中$v$跑遍$K$的全部素位,这里限制直积是指它是直积$\prod_vK_v$的子环,满足对几乎所有素位$v$该分量落在$\mathscr{O}_v$中.如果$v$是无穷素位,定义$\mathscr{O}_v=K_v$.
	\item $K$的依代尔群$I_K$是指阿代尔环的乘法群.也即限制直积$\prod_v(K_v^*,U_v)$,其中当$v$是有限素位时$U_v=\mathscr{O}_v^*$;当$v$是复无穷素位时$U_v=\mathbb{C}^*$;当$v$是实无穷素位时$U_v=\mathbb{R}^{>0}$.设$S$是$K$上素位的有限子集,记$I_K^S=\prod_{p\in S}K_p^*\times\prod_{p\not\in S}U_p$.这是依代尔群的子群.那么当$S$跑遍素位的有限子集时有$I_K=\cup_SI_K^S$.
	\item 我们有典范的对角嵌入$K^*\to I_K$,于是$K^*$可视为$I_K$的子群,其中的元称为主依代尔.商群$C_K=I_K/K^*$称为$K$的依代尔类群.
\end{itemize}
\begin{enumerate}
	\item 对素位的有限子集$S$,记$K^S=K^*\cap I_K^S$,它由这样的$a\in K^*$构成,满足当$v\not\in S$并且是有限素位时分量$a_v$都是$\mathscr{O}_v$的单位,对于实无穷素位$v$该分量为正实数.$K^S$中的元称为$S$单位.下面设$S$是包含$K$的全部无穷素位的有限个素位集合,那么映射$\lambda:K^S\to\prod_{p\in S}\mathbb{R}$,$a\mapsto\left(\ln|a|_p\right)_{p\in S}$的核是$\mu(K)$,即单位根群,像是$s-1$维迹零空间$H=\{(x_p)\in\prod_{p\in S}\mathbb{R}\mid\sum_{p\in S}x_p=0\}$中的一个完备格,这里$s=|S|$.在这个命题中取$S$恰好是全体无穷素位构成的有限集合,即为Dirichlet单位定理.
	\begin{proof}
		
		设无穷素位构成的有限集合是$S_{\infty}$,如果$S_f=S-S_{\infty}$是空集,此即Dirichlet单位定理,我们已经证明过了,于是不妨设$S_f$是非空有限集合.设$J(S_f)$是分式理想群$J_K$的由$p\in S_f$生成的子群.设$i$把$K^S$中的每个元$a$映射为主分式理想$(a)$,那么$(a)$落在$J(S_f)$中.于是有如下交换图表:
		$$\xymatrix{1\ar[r]&\mathscr{O}_K^*\ar[d]^{\lambda'}\ar[r]&K^S\ar[d]^{\lambda}\ar[r]^i&J(S_f)\ar[d]^{\lambda''}\\0\ar[r]&\prod_{p\in S_{\infty}}\mathbb{R}\ar[r]&\prod_{p\in S}\mathbb{R}\ar[r]^i&\prod_{p\in S_f}\mathbb{R}}$$
		
		这个图表中的两行都是正合的,这里$\lambda''$把$\prod_{p\in S_f}p^{v_p}$映射为$-\prod_{p\in S_f}v_p\ln\mathscr{R}(p)$,其中$|a|_p=\mathscr{R}(p)^{-v_p(a)}$.于是$\lambda''$把$J(S_f)$同构的映射为被向量$e_p=(0,\cdots,0,\ln\mathscr{R}(p),0,\cdots,0),p\in S_f$生成的完备格.于是蛇形引理得到$\ker\lambda=\ker\lambda'=\mu(K)$(最后一个等式即Dirichlet单位定理的部分结论).另外蛇形引理还得到正合列$0\to\mathrm{Im}(\lambda')\to\mathrm{Im}(\lambda)\to\mathrm{Im}(\lambda'')$.
		
		按照$\mathrm{Im}(\lambda')$和$\mathrm{Im}(\lambda'')$都是格,我们可以得到$\mathrm{Im}(\lambda)$是格(我们解释过格等价于是离散子群):任取$x\in\mathrm{Im}(\lambda)$,按照$\mathrm{Im}(\lambda'')$是离散子群,说明存在$i(x)$的开邻域$U$只包含$\mathrm{Im}(\lambda'')$中的点$i(x)$.于是$i^{-1}(U)$只包含了陪集$x+\mathrm{Im}(\lambda')$,但是$\mathrm{Im}\lambda'$是离散的,这得到开集只包含点$x$,于是$\mathrm{Im}(\lambda)$是离散子群,于是它是格.
		
		对$p\in S_f$,设$K$的类数是$h$,那么$p^h\in i(K^S)$,这说明$J(S_f)^h\subseteq i(K^S)\subseteq J(S_f)$.这里$J(S_f)^h$和$J(S_f)$的秩都是$|S_f|$,导致$i(K^S)$也是.在上面正合列中$i$的像的秩是$|S_f|$,它的核$\mathrm{Im}(\lambda')$的秩是$|S_{\infty}|-1$(这也是Dirichlet单位定理的部分结论),于是$\mathrm{Im}(\lambda)$是秩为$|S_f|+|S_{\infty}|-1=s-1$的格.最后按照我们证明过的$\prod_{p\in S}|a|_p=\prod_p|a|_p=1,a\in K^S$,得到$\mathrm{Im}(\lambda)$落在$s-1$维迹零空间$H$中.
	\end{proof}
    \item 依代尔类群和理想类群.我们之前定义过$K$的分式理想群$J_K$模去主分式理想子群得到的商群$\mathrm{Cl}_K$称为理想类群.定义$I_K\to J_K$为,把依代尔类群中的元$\alpha$,映射为分式理想$\prod_{\mathfrak{p}\not\mid\infty}\mathfrak{p}^{v_{\mathfrak{p}}(\alpha_\mathfrak{p})}$,这是良性的因为按照定义只有有限个$\alpha_{\mathfrak{p}}$不是单位,也即只有有限个$v_{\mathfrak{p}}(\alpha_{\mathfrak{p}})$不为零.这个同态是满射,它的核是$I_K^{S_{\infty}}=\prod_{\mathfrak{p}\mid\infty}K_{\mathfrak{p}}^*\times\prod_{\mathfrak{p}\not\mid\infty}U_{\mathfrak{p}}$.另外这个同态把$K^*$中的元映射为主分式理想,于是它诱导了满的群同态$C_K\to\mathrm{Cl}_K$,并且核是$I_K^{S_{\infty}}K^*/K^*$.于是理想类群可视为依代尔群的商:
    $$\mathrm{Cl}_K\cong I_K/I_K^{S_{\infty}}K^*$$
    \item 依代尔类群和充足理想类群【】.定义$I_K\to J(\overline{\mathscr{O}_K})$为把依代尔$\alpha$映射为充足理想$\prod_pp^{v_p(\alpha_p)}$.这是一个满同态,它的核是$I_K^0=\{(\alpha_p)\in I_K\mid|\alpha_p|_p=1,\forall p\}$,这个映射把主依代尔映射为主充足理想,于是它诱导了满同态$C_K\to\mathrm{Pic}(\overline{\mathscr{O}_K})$,它的核是$I_K^0K^*/K^*$,于是充足理想类群可视为依代尔群的商:
    $$\mathrm{Pic}(\overline{\mathscr{O}_K})\cong I_K/I^0_KK^*$$
    \item 对足够大的由某些素位构成的有限集合$S$,有$I_K=I_K^SK^*$.
    \begin{proof}
    	
    	记数域$K$的类数为$h$,记分式理想$\mathfrak{a}_1,\cdots,\mathfrak{a}_h$构成整个$J_K/P_K$.把这些分式理想做唯一分解,这只涉及到有限个素理想$\mathfrak{p}_1,\cdots,\mathfrak{p}_n$.设$S$包含这有限个素理想以及全部无穷素位,我们断言$I_K=I_K^SK^*$.
    	
    	\qquad
    	
    	为此任取$\alpha\in I_K$,那么按照$I_K/I_K^{S_{\infty}}\cong J_K$,就有$(\alpha)=\prod_{\mathfrak{p}}\mathfrak{p}^{v_{\mathfrak{p}}(\alpha_{\mathfrak{p}})}$和某个$\alpha_r$在相同的理想类中,于是存在$a\in K^*$使得$(\alpha)=a_r(a)$.取$\alpha'=\alpha a^{-1}\in I_K$,那么有$(\alpha')=a_r\in I_K^S$,于是$\alpha\in I_K^SK^*$.
    \end{proof}
    \item 限制直积拓扑.
    \begin{enumerate}[(1)]
    	\item 限制直积拓扑.设$\{X_i\mid i\in I\}$是一族拓扑空间,对每个$X_i$取一个开子空间$Y_i$,定义限制直积$X=\prod_i(X_i,Y_i)$为$\prod_iX_i$的子集,其中的元$(x_i)$满足对几乎所有分量$i$有$x_i\in Y_i$.定义限制直积上的拓扑被拓扑基$\{\prod_iU_i\}$定义,其中$U_i$总是$X_i$的开子集,并且对几乎所有指标$i$有$U_i=Y_i$.
    	\item 对指标的有限子集$S$,记$X^S=\left(\prod_{i\in S}X_i\right)\times\left(\prod_{i\not\in S}Y_i\right)$.那么$X^S$上赋予乘积拓扑和赋予$X$的子空间拓扑是一致的.
    	\item 如果$Y_i'\subseteq X_i$是取定的另一族开子空间,使得$Y_i=Y_i'$对几乎全部指标成立,那么$\prod_i(X_i,Y_i)$和$\prod_i(X_i,Y_i')$是相同的拓扑空间.
    	\item 如果每个$X_i$是局部紧空间,并且每个$Y_i$是紧空间,那么限制直积空间$X$是局部紧空间.
    	\begin{proof}
    		
    		这件事是因为开子集$X^S\subseteq X$已经是局部紧的了.
    	\end{proof}
    \end{enumerate}
    \item 按照$U_v$是$K_v^*$的开子集,我们赋予$I_K=\prod_v(K_v^*,U_v)$限制直积拓扑,这是局部紧度量拓扑群.我们断言$K^*\subseteq I_K$是离散子群.进而它是闭子群(因为Hausdorff拓扑群的局部紧子群总是闭子群).
    \begin{proof}
    	
    	我们断言如下$e\in I_K$的开邻域$\mathscr{U}$只包含了$e$这一个主依代尔:
    	$$\mathscr{U}=\{\alpha\in I_K\mid|\alpha_{\mathfrak{p}}|_{\mathfrak{p}}=1,\mathfrak{p}\not\mid\infty;|\alpha_{\mathfrak{p}}-1|_{\mathfrak{p}}<1,p\mid\infty\}$$
    	因为如果$x\in\mathscr{U}$是不同于$e$的主依代尔,那么有矛盾:
    	\begin{align*}
    		1&=\prod_{\mathfrak{p}}|x-1|_{\mathfrak{p}}\\&=\prod_{\mathfrak{p}\mid\infty}|x-1|_{\mathfrak{p}}\cdot\prod_{p\not\mid\infty}|x-1|_{\mathfrak{p}}\\&<\prod_{\mathfrak{p}\not\mid\infty}|x-1|_{\mathfrak{p}}\\&\le\prod_{\mathfrak{p}\not\mid\infty}\max\{|x|_{\mathfrak{p}},1\}=1
    	\end{align*}
    \end{proof}
\end{enumerate}



\newpage
\section{L函数}
\subsection{Riemann's Zeta函数}

定义Riemann Zeta函数为复变函数:
$$\zeta(s)=\sum_{n\ge1}\frac{1}{n^s}$$
\begin{enumerate}
	\item 对任意正实数$\delta>0$,有$\zeta(s)$在$\mathrm{Re}(s)\ge1+\delta$上绝对且一致收敛(换句话讲它在开平面$\mathrm{Re}(s)>1$上内闭绝对且一致收敛).于是$\zeta(s)$是半平面$\mathrm{Re}(s)>1$上的解析函数.此时有如下欧拉等式,其中$p$跑遍全部素数:
	$$\zeta(s)=\prod_p\frac{1}{1-p^{-s}}$$
	\begin{proof}
		
		绝对且一致收敛是因为$|1/n^s|\le|1/n^{1+\delta}|$.我们来证明欧拉等式.记等式右侧为$E(s)$,它的收敛性是指如下级数的收敛性:
		$$\ln E(s)=\sum_p\sum_{n\ge1}\frac{1}{np^{ns}}$$
		
		这个级数也在$\mathrm{Re}(s)\ge1+\delta$上绝对且一致收敛,因为有如下放缩:
		$$\left|\sum_p\sum_{n\ge1}\frac{1}{np^{ns}}\right|\le\sum_p\sum_{n\ge1}\frac{1}{p^{n(1+\delta)}}=\sum_p\frac{1}{p^{1+\delta}-1}\le2\sum_p\frac{1}{p^{1+\delta}}$$
		
		对正整数$N$记不超过它的全部素数为$p_1,\cdots,p_r$,那么有:
		$$\prod_{p\le N}\frac{1}{1-p^{-s}}=\sum_{v_1,\cdots,v_r\ge0}\frac{1}{(p_1^{v_1}\cdots p_r^{v_r})^s}=\sum'_n\frac{1}{n^s}$$
		
		这里$\sum'$表示$n$跑遍那些只被$\le N$的素数整除的正整数.进而有:
		$$\prod_{p\le N}\frac{1}{1-p^{-s}}=\sum_{n\le N}\frac{1}{n^s}+\sum'_{n>N}\frac{1}{n^s}$$
		
		按照如下放缩就得到$E(s)=\zeta(s)$:
		$$\left|\prod_{p\le N}\frac{1}{1-p^{-s}}-\zeta(s)\right|\le\left|\sum_{n>N,p_i\not\mid n}\frac{1}{n^s}\right|\le\sum_{n>N}\frac{1}{n^{1+\delta}}$$
	\end{proof}
	\item 回顾Gamma函数.此为如下复变函数:
	$$\Gamma(s)=\int_0^{\infty}e^{-y}y^{s-1}\mathrm{d}y$$
	\begin{enumerate}[(1)]
		\item $\Gamma(s)$在开平面$\mathrm{Re}(s)>0$上内闭绝对且一致收敛.它可以亚纯延拓到整个$\mathbb{C}$上.此时它没有零点,全部极点是$\{0,-1,-2,\cdots\}$.
		\item 它满足如下等式.
		\begin{itemize}
			\item $\Gamma(s+1)=s\Gamma(s)$(亚纯延拓就是用这个等式).
			\item $\Gamma(s)\Gamma(1-s)=\frac{\pi}{\sin\pi s}$.
			\item $\Gamma(s)\Gamma(s+\frac{1}{2})=2^{1-2s}\sqrt{\pi}\Gamma(2s)$.
			\item $\Gamma(1/2)=\sqrt{\pi}$,$\Gamma(n+1)=n!,\forall n\in\mathbb{N}$.
		\end{itemize}
	\end{enumerate}
    \item 回顾Theta函数.此为如下复变函数:
    $$\theta(z)=\sum_{n\in\mathbb{Z}}e^{\pi in^2z}$$
    \begin{enumerate}[(1)]
    	\item 它在上半平面$\mathrm{Im}(z)>0$内闭绝对且一致收敛.
    	\item 满足$\theta(-1/z)=\sqrt{z/i}\theta(z)$.
    \end{enumerate}
	\item 完备Zeta函数.先在$\Gamma(s)$表达式中把$y$替换为$\pi n^2y$,得到:
	$$\pi^{-s}\Gamma(s)\frac{1}{n^{2s}}=\int_0^{\infty}e^{-\pi n^2y}y^s\frac{\mathrm{d}y}{y}$$
	
	对$n\ge1$求和,得到:
	$$\pi^{-s}\Gamma(s)\zeta(2s)=\int_0^{\infty}\sum_{n\ge1}e^{-\pi n^2y}y^s\frac{\mathrm{d}y}{y}$$
	
	这里积分和求和可交换是因为有:
	\begin{align*}
		\sum_{n\ge1}\int_0^{\infty}|e^{-\pi n^2y}y^s|\frac{\mathrm{d}y}{y}&=\sum_{n\ge1}\int_0^{\infty}e^{-\pi n^2y}y^{\mathrm{Re}(s)}\frac{\mathrm{d}y}{y}\\&=\pi^{-\mathrm{Re}(s)}\Gamma(\mathrm{Re}(s))\zeta(2\mathrm{Re}(s))<\infty
	\end{align*}

    借助Theta函数这里$\sum_{n\ge1}e^{-\pi n^2y}=\frac{1}{2}(\theta(iy)-1)$.定义完备Zeta函数(completed zeta function)为:
    $$Z(s)=\pi^{-s/2}\Gamma(s/2)\zeta(s)=\frac{1}{2}\int_0^{\infty}\left(\theta(iy)-1\right)y^{s/2}\frac{\mathrm{d}y}{y}$$
    \item 设$f:\mathbb{R}^+\to\mathbb{C}$是正实数集上的复值连续函数.定义Mellin变换为如下积分:
    $$L(f,s)=\int_0^{\infty}\left(f(y)-f(\infty)\right)y^s\frac{\mathrm{d}y}{y}$$
    
    前提是$f(\infty)=\lim_{y\to+\infty}f(y)$和这个积分存在.如下定理称为Mellin原理:设$f,g:\mathbb{R}^+\to\mathbb{C}$是两个连续函数,满足如下等式,其中$c,\alpha$都是正数.
    $$f(y)=a_0+O(e^{-cy^{\alpha}}),g(y)=b_0+O(e^{-cy^{\alpha}}),y\to+\infty$$
    
    如果存在正实数$k$和非零复数$C$满足$f(1/y)=Cy^kg(y)$,那么有如下结论:
    \begin{enumerate}[(1)]
    	\item $L(f,s)$和$L(g,s)$在右半平面$\mathrm{Re}(s)>k$内闭绝对且一致收敛,并且它们可以解析延拓到$\mathbb{C}\backslash\{0,k\}$.
    	\item 有留数:
    	$$\mathrm{Res}_{s=0}L(f,s)=-a_0,\mathrm{Res}_{s=k}L(f,s)=Cb_0$$
    	$$\mathrm{Res}_{s=0}L(g,s)=-b_0,\mathrm{Res}_{s=k}L(g,s)=C^{-1}a_0$$
    	
    	于是如果$a_0=0$或者$b_0=0$时$L(f,s),L(g,s)$没有相应极点;如果$a_0\not=0$或者$b_0\not=0$时$L(f,s),L(g,s)$有相应的一阶极点.
    	\item 它们满足函数方程$L(f,s)=CL(g,k-s)$.
    \end{enumerate}
    \begin{proof}
    	
    	设$\sigma=\mathrm{Re}(s)$,在$s$跑遍$\mathrm{Re}(s)>k$的一个紧子集时,当$y\ge1$时$e^{-cy^{\alpha}}y^{\alpha}$上有界.结合$f(y)=a_0+O(e^{-cy^{\alpha}})$,就得到当$y\ge1$和$s$跑遍该紧子集时有:
    	$$\left|(f(y)-a_0)y^{s-1}\right|\le Be^{-cy^{\alpha}}y^{\sigma+1}y^{-2}\le\frac{B'}{y^2}$$
    	
    	右侧这个式子在$[1,+\infty)$上收敛,于是积分$\int_1^{\infty}\left(f(y)-a_0\right)y^{s-1}\mathrm{d}y$在$s$跑遍一个紧子集上绝对且一致收敛.类似的$\int_1^{\infty}\left(g(y)-b_0\right)y^{s-1}\mathrm{d}y$满足相同的事情.至于$[0,1]$区间上的那部分积分,做代换$y\mapsto 1/y$,结合$f(1/y)=Cy^kg(y)$,得到:
    	\begin{align*}
    		\int_0^1\left(f(y)-a_0\right)y^{s-1}\mathrm{d}y&=-\frac{a_0}{s}+\int_1^{\infty}f(1/y)y^{-s}\frac{\mathrm{d}y}{y}\\&=-\frac{a_0}{s}-\frac{Cb_0}{k-s}+C\int_1^{\infty}\left(g(y)-b_0\right)y^{k-s-1}\mathrm{d}y
    	\end{align*}
    	
    	其中积分函数依旧在$\mathrm{Re}(s)>k$上内闭绝对且一致收敛.于是我们有:
    	$$L(f,s)=-\frac{a_0}{s}+\frac{Cb_0}{s-k}+F(s)$$
    	
    	其中:
    	$$F(s)=\int_1^{\infty}\left((f(y)-a_0)y^s+C(g(y)-b_0)y^{k-s}\right)\frac{\mathrm{d}y}{y}$$
    	
    	把$f,g$替换,按照$g(1/y)=C^{-1}y^kf(y)$得到:
    	$$L(g,s)=-\frac{b_0}{s}+\frac{C^{-1}a_0}{s-k}+G(s)$$
    	
    	其中:
    	$$G(s)=\int_1^{\infty}\left((g(y)-b_0)y^s+C^{-1}(f(y)-a_0)y^{k-s}\right)\frac{\mathrm{d}y}{y}$$
    	
    	这里$F(s)$和$G(s)$在整个复平面上绝对收敛,且内闭一致收敛.于是上面等式就把$L(f,s)$和$L(g,s)$解析延拓到$\mathbb{C}\backslash\{0,k\}$上,并且有$L(f,s)=CL(g,k-s)$.
    \end{proof}
    \item 完备Zeta函数$Z(s)$可以解析延拓到$\mathbb{C}\backslash\{0,1\}$上,并且在$s=0$和$s=1$处是一阶极点,留数分别是$-1$和$1$.并且它满足$Z(s)=Z(1-s)$.
    \begin{proof}
    	
    	$Z(2s)$是关于$f(y)=\frac{1}{2}\theta(iy)$的Mellin变换:
    	$$Z(2s)=\frac{1}{2}\int_0^{\infty}\left(\theta(iy)-1\right)y^s\frac{\mathrm{d}y}{y}=L(f,s)$$
    	
    	按照:
    	$$\theta(iy)=1+2e^{-\pi y}\left(1+\sum_{n\ge2}e^{-\pi(n^2-1)y}\right)$$
    	
    	就有$f(y)=\frac{1}{2}+O(e^{-\pi y})$.按照$\theta(-1/z)=\sqrt{z/i}\theta(z)$,就有:
    	$$f(1/y)=\frac{1}{2}\theta(-1/iy)=\frac{1}{2}y^{1/2}\theta(iy)=y^{1/2}f(y)$$
    	
    	于是按照Mellin原理,就有$Z(2s)=L(f,s)$解析延拓到$\mathbb{C}\backslash\{0,1/2\}$,其中$s=0,1/2$处是两个一阶极点,留数分别为$-1/2$和$1/2$,并且满足$L(f,s)=L(f,1/2-s)$.进而$Z(s)=L(f,s/2)$可以解析延拓到$\mathbb{C}\backslash\{0,1\}$,其中$s=0,1$是两个一阶极点,留数分别为$-1$和$1$,并且满足:
    	$$Z(s)=L(f,\frac{s}{2})=L(f,\frac{1}{2}-\frac{s}{2})=Z(1-s)$$
    \end{proof}
    \item Zeta函数$\zeta(s)$可以解析延拓到$\mathbb{C}\backslash\{1\}$,并且在$s=1$处是一阶极点,留数为1,并且它满足如下函数方程:
    $$\zeta(1-s)=2(2\pi)^{-s}\Gamma(s)\cos\left(\frac{\pi s}{2}\right)\zeta(s)$$
    \begin{proof}
    	
    	按照$Z(s)=\pi^{-s/2}\Gamma(s/2)\zeta(s)$,从$Z(s)$和$\Gamma(s/2)$都在$s=0$处是一阶极点,于是$\zeta(s)$在$s=0$处不是极点,类似的考虑得到$\zeta(s)$在$s=1$处是一阶极点.
    \end{proof}
    \item 关于零点.按照欧拉等式,延拓的解析函数$\zeta(s)$在$\mathrm{Re}(s)>1$上没有零点.按照$Z(s)=Z(1-s)$,有如下等式:
    $$\pi^{-s/2}\Gamma(s/2)\zeta(s)=\pi^{(s-1)/2}\Gamma((1-s)/2)\zeta(1-s)$$
    
    于是$\zeta(s)$在$\mathrm{Re}(s)<0$上的零点就是$\Gamma(s/2)$的极点,也即$s=-2,-4,-6,\cdots$.它们称为$\zeta(s)$的平凡零点.那些在$0\le\mathrm{Re}(s)\le1$的零点称为$\zeta(s)$的非平凡零点.黎曼猜想就是说非平凡零点一定落在$\mathrm{Re}(s)=1/2$上.
\end{enumerate}
\subsection{Dirichlet L-级数}
\begin{enumerate}
	\item 设$m$是自然数,一个$m$-Dirichlet特征标是指复值函数$\chi:\mathbb{Z}\to\mathbb{C}$,满足对任意整数$a,b$有:
	\begin{itemize}
		\item $\chi(ab)=\chi(a)\chi(b)$.
		\item $|\chi(a)|=\left\{\begin{array}{cc}0&(a,m)>1\\1&(a,m)=1\end{array}\right.$
		\item $\chi(a+m)=\chi(a)$.
	\end{itemize}
	\begin{enumerate}[(1)]
		\item 一个$m$-Dirichlet特征标也等价于一个同态$\chi:(\mathbb{Z}/m\mathbb{Z})^*\to S^1=\{z\in\mathbb{C}\mid|z|=1\}$.
		\item 如果存在正整数$m'\mid m$,使得特征标$\chi$可以分解为$(\mathbb{Z}/m\mathbb{Z})^*\to(\mathbb{Z}/m'\mathbb{Z})^*\to S^1$,那么$\chi$就也可以视为一个$m'$-Dirichlet特征标.全体满足这个性质的$m$的正因数中存在一个最小元$m_0$(它也就是这些正因数的最大公约数),$m_0$称为特征标$\chi$的导子(conductor).
		\item 如果$m$-Dirichlet特征标$\chi$的导子是$m$自身,就称$\chi$是素特征标(primitive).
		\item 每个$m$-Dirichlet特征标都对应于唯一的一个$m_0$-Dirichlet素特征标,其中$m_0$是它的导子.
		\item 取$m$-Dirichlet特征标$\chi$为当$(n,m)=1$时$\chi(n)=1$,否则$\chi(n)=0$.这是一个素特征标,称为平凡$m$-Dirichlet特征标.称$1$-Dirichlet平凡特征标为主特征标.
	\end{enumerate}
    \item 设$\chi$是Dirichlet特征标,它对应的Dirichlet L-级数定义为复变函数:
    $$L(\chi,s)=\sum_{n\ge1}\frac{\chi(n)}{n^s}$$
    
    取$\chi$是主特征标就得到Riemann Zeta函数.上一节的内容都可以用在Dirichlet L-级数上:$L(\chi,s)$在半平面$\mathrm{Re}(s)>1$上内闭绝对且一致收敛,并且仍然有欧拉恒等式:
    $$L(\chi,s)=\prod_p\frac{1}{1-\chi(p)p^{-s}}$$
    
    我们接下来把Dirichlet L-级数延拓到整个$\mathbb{C}$上.这件事实际上对更大的一类 L-级数都成立,即Hecke L-级数,后文会介绍.
    \item Hecke特征标.我们这里介绍的Hecke特征标只是$\mathbb{Q}$上的,一般数域上见后文.设$\chi$是$m$-Dirichlet特征标,它的符号$p\in\{0,1\}$指的是$\chi(-1)=(-1)^p\chi(1)$的$p$.考虑$\mathbb{Z}$的所有和$m$互素的主理想$(n)$构成的半群,定义其上的函数$\chi((n))=\mathrm{sgn}(\chi)^p\chi(n)$.称为该Dirichlet特征标对应的Hecke特征标.
    \item 完备L-级数.考虑如下Gamma函数:
    $$\Gamma(\chi,s)=\Gamma\left(\frac{s+p}{2}\right)=\int_0^{\infty}e^{-y}y^{(s+p)/2}\frac{\mathrm{d}y}{y}$$
    
    做换元$y\mapsto\pi n^2y/m$:
    $$\left(\frac{m}{\pi}\right)^{(s+p)/2}\Gamma(\chi,s)\frac{1}{n^s}=\int_0^{\infty}n^pe^{-\pi n^2y/m}y^{(s+p)/2}\frac{\mathrm{d}y}{y}$$
    
    数乘$\chi(n)$,再对$n\ge1$求和,得到:
    $$\left(\frac{m}{\pi}\right)^{(s+p)/2}\Gamma(\chi,s)L(\chi,s)=\int_0^{\infty}\sum_{n\ge1}\chi(n)n^pe^{-\pi n^2y/m}y^{(s+p)/2}\frac{\mathrm{d}y}{y}$$
    
    其中积分和求和可交换是因为有:
    $$\sum_{n\ge1}\int_0^{\infty}\left|\chi(n)n^pe^{-\pi n^2y/m}y^{(s+p)/2}\right|\frac{\mathrm{d}y}{y}\le\left(\frac{m}{\pi}\right)^{(\mathrm{Re}(s)+p)/2}\Gamma\left(\frac{\mathrm{Re}(s)+p}{2}\right)\zeta(\mathrm{Re}(s))<\infty$$
    
    被积分的函数记作:
    $$g(y)=\sum_{n\ge1}\chi(n)n^pe^{-\pi n^2y/m}$$
    
    定义$\chi$诱导的Theta级数为:
    $$\theta(\chi,z)=\sum_{n\in\mathbb{Z}}\chi(n)n^pe^{\pi in^2z/m}$$
    
    其中约定$0^0=1$.按照$\chi(n)n^p=\chi(-n)(-n)^p$,就有:
    $$\theta(\chi,z)=\chi(0)+2\sum_{n\ge1}\chi(n)n^pe^{\pi in^2z/m}$$
    
    其中当$\chi$是平凡特征标的时候$\chi(0)=1$,否则有$\chi(0)=0$.进而有$2g(y)=\theta(\chi,iy)-\chi(0)$.称如下部分是$L(\chi,s)$的欧拉因子:
    $$L_{\infty}(\chi,s)=\left(\frac{m}{\pi}\right)^{(s+p)/2}\Gamma(\chi,s)$$
    
    综上我们定义$L(\chi,s)$的完备L-级数为:
    $$\Lambda(\chi,s)=L_{\infty}(\chi,s)L(\chi,s)=\frac{c(\chi)}{2}\int_0^{\infty}\left(\theta(\chi,iy)-\chi(0)\right)y^{(s+p)/2}\frac{\mathrm{d}y}{y}$$
    
    其中$c(\chi)=\left(\frac{m}{\pi}\right)^{p/2}$.这是半平面$\mathrm{Re}(s)>1$上的内闭绝对且一致收敛的.
    \item 高斯和.设$\chi$是$m$-Dirichlet特征标,设$n$是整数,定义高斯和为如下复数:
    $$\tau(\chi,n)=\sum_{v=0}^{m-1}\chi(v)e^{2\pi ivn/m}$$
    
    简记$\tau(\chi,1)$为$\tau(\chi)$.我们断言对$m$-Dirichlet素特征标$\chi$,记$\overline{\chi}$是它的共轭特征标,那么总有$\tau(\chi,n)=\overline{\chi}(n)\tau(\chi)$和$|\tau(\chi)|=\sqrt{m}$.
    \begin{proof}
    	
    	先证第一个等式.如果$(n,m)=1$,那么$\chi(vn)=\chi(v)\chi(n)$直接得到这个等式.如果$(n,m)=d>1$,归结为证明$\tau(\chi,n)=0$:按照$\chi$是素的,可以取$\mathrm{mod}(m/d)$下的$a\equiv1$使得$m\not\mid a-1$和$\chi(a)\not=1$,此时有$\chi(a)\tau(\chi,n)=\tau(\chi,n)$,从而$\tau(\chi,n)=0$.至于第二件事,我们有:
    	\begin{align*}
    		\left|\tau(\chi)\right|^2&=\tau(\chi)\overline{\tau(\chi)}\\&=\tau\sum_{v=0}^{m-1}\overline{\chi}(v)e^{-2\pi iv/m}\\&=\sum_{v=0}^{m-1}\tau(\chi,v)e^{-2\pi iv/m}\\&=\sum_{v=0}^{m-1}\sum_{\mu=0}^{m-1}\chi(\mu)e^{2\pi iv\mu/m}e^{-2\pi iv/m}\\&=\sum_{\mu=0}^{m-1}\chi(\mu)\sum_{v=0}^{m-1}e^{2\pi iv(\mu-1)/m}
    	\end{align*}
    
        在$\mu=1$时对$v$求和是$m$,在$\mu\not=0$时这个求和是零是因为它们恰好是$X^m-1$的全部根.于是得到$|\tau(\chi)|^2=m$.
    \end{proof}
    \item 设$a,b,\mu$是实数,其中$\mu>0$,定义级数:
    $$\theta_{\mu}(a,b,z)=\sum_{g\in\mu\mathbb{Z}}e^{\pi i(a+g)^2z+2\pi ibg}$$
    
    它在半平面$\mathrm{Im}(z)>0$上内闭绝对且一致收敛.任取上半平面中的元$z$,下一节会证明它满足如下等式,并且关于变量$a,b$在局部上是一致收敛的.
    $$\theta_{\mu}(a,b,-1/z)=e^{-2\pi iab}\frac{\sqrt{z/i}}{\mu}\theta_{1/\mu}(-b,a,z)$$
    
    记如下函数:
    $$\theta_{\mu}^p(a,b,z)=\sum_{g\in\mu\mathbb{Z}}(a+g)^pe^{\pi i(a+g)^2+2\pi ibg}$$
    
    对$a$求$p=0,1$阶导数,得到如下公式:
    $$\frac{\mathrm{d}^p}{\mathrm{d}a^p}\theta_{\mu}(a,b,z)=(2\pi iz)^p\theta_{\mu}^p(a,b,z)$$
    $$\frac{\mathrm{d}^p}{\mathrm{d}a^p}e^{-2\pi iab}\theta_{1/\mu}(-b,a,z)=(2\pi i)^pe^{-2\pi iab}\theta_{1/\mu}^p(-b,a,z)$$
    
    于是对上述$\theta_{\mu}$和$\theta_{1/\mu}$的等式对$a$求$p$阶导数,得到如下变换公式:
    $$\theta_{\mu}^p(a,b,-1/z)=\left(i^pe^{2\pi iab}\mu\right)^{-1}(z/i)^{p+1/2}\theta_{1/\mu}^p(-b,a,z)$$
    \item 设$\chi$是$m$-Dirichlet素特征标,那么我们有如下公式:
    $$\theta(\chi,-1/z)=\frac{\tau(\chi)}{i^p\sqrt{m}}(z/i)^{p+1/2}\theta(\overline{\chi},z)$$
    \begin{proof}
    	
    	先把$\theta$的求和做$\mathrm{mod}m$下的分解:
    	\begin{align*}
    		\theta(\chi,z)&=\sum_{n\in\mathbb{Z}}\chi(n)n^pe^{\pi i n^2z/m}\\&=\sum_{a=0}^{m-1}\chi(a)\sum_{g\in m\mathbb{Z}}(a+g)^pe^{\pi i(a+g)^2z/m}\\&=\sum_{a=0}^{m-1}\chi(a)\theta_m^p(a,0,z/m)
    	\end{align*}
    	
    	按照上一条最后的变换公式,得到:
    	$$\theta_m^p(a,0,-1/mz)=\frac{1}{i^pm}(mz/i)^{p+1/2}\theta_{1/m}^p(0,a,mz)$$
    	
    	把这个等式乘以$\chi(a)$,再对$0\le a\le m-1$求和,得到:
    	\begin{align*}
    		\theta(\chi,-1/z)&=\frac{1}{i^pm}(mz/i)^{p+1/2}\sum_{a=0}^{m-1}\chi(a)\theta_{1/m}^p(0,a,mz)\\&=\frac{1}{i^pm^{p+1}}(mz/i)^{p+1/2}\sum_{n\in\mathbb{Z}}\left(\sum_{a=0}^{m-1}\chi(a)e^{2\pi ian/m}\right)n^pe^{\pi in^2z/m}\\&=\frac{1}{i^p\sqrt{m}}(z/i)^{p+1/2}\tau(\chi)\sum_{n\in\mathbb{Z}}\overline{\chi}(n)n^pe^{\pi in^2z/m}\\&=\frac{\tau(\chi)}{i^p\sqrt{m}}(z/i)^{p+1/2}\theta(\overline{\chi},z)
    	\end{align*}
    \end{proof}
    \item 设$\chi$是非平凡$m$-Dirichlet素特征标,那么完备L-级数$\Lambda(\chi,s)$可以解析延拓到整个复平面$\mathbb{C}$,并且满足:
    $$\Lambda(\chi,s)=W(\chi)\Lambda(\overline{\chi},1-s)$$
    
    其中$W(\chi)=\frac{\tau(\chi)}{i^p\sqrt{m}}$,它的绝对值是1.
    \begin{proof}
    	
    	记欧拉因子$c(\chi)=\left(\frac{\pi}{m}\right)^{p/2}$,取:
    	$$f(y)=\frac{c(\chi)}{2}\theta(\chi,iy)$$
    	$$g(y)=\frac{c(\chi)}{2}\theta(\overline{\chi},iy)$$
    	
    	按照$\chi$非平凡,就有$\chi(0)=0$,于是:
    	$$\theta(\chi,iy)=2\sum_{n\ge1}\chi(n)n^pe^{-\pi n^2y/m}$$
    	
    	于是有$f(y)=O(e^{-\pi y/m})$和$g(y)=O(e^{-\pi y/m})$.我们之前给了:
    	$$\Lambda(\chi,s)=\frac{c(\chi)}{2}\int_0^{\infty}\theta(\chi,iy)y^{(s+p)/2}\frac{\mathrm{d}y}{y}$$
    	
    	于是我们有Mellin变换:
    	$$\Lambda(\chi,s)=L(f,s'),\Lambda(\overline{\chi},s)=L(g,s'),s'=\frac{s+p}{2}$$
    	
    	上一条给出了它们满足:
    	$$f(1/y)=W(\chi)y^{p+1/2}g(y),W(\chi)=\frac{\tau(\chi)}{i^p\sqrt{m}}$$
    	
    	于是$\Lambda(\chi,s)$解析延拓到整个$\mathbb{C}$上,没有极点,并且满足:
    	$$\Lambda(\chi,s)=W(\chi)\Lambda(\overline{\chi},1-s)$$
    \end{proof}
\end{enumerate}
\subsection{高阶Theta函数}
\begin{enumerate}
	\item 抽象Minkowiski空间.设$X$是有限$n$元集合,附带一个共轭$\tau\mapsto\overline{\tau}$(此为复合为恒等映射的一个双射,也称为involution).设$\{\rho_i\}$是那些被共轭固定的元,那么我们可以把$X$记作:
	$$X=\{\rho_1,\cdots,\rho_{r_1},\sigma_1,\overline{\sigma_1},\cdots,\sigma_{r_2},\overline{\sigma_{r_2}}\}$$
	
	考虑$n$维$\mathbb{C}$代数$\textbf{C}=\prod_{\tau\in X}\mathbb{C}$.定义其上的共轭为$\overline{z}=\overline{(z_{\tau})_{\tau}}=(\overline{z_{\overline{\tau}}})_{\tau}$.换句话讲共轭作用在$z$上把那些$\rho_i$分量取共轭,把$\sigma_i$和$\overline{\sigma_i}$分量取共轭,并且把它们交换位置.那些在共轭下不变的元素构成的子集记作$\textbf{R}$,此即满足$z_{\tau}=\overline{z_{\overline{\tau}}}$的$z\in\textbf{C}$.换句话讲这样的元素满足$\rho_i$分量是实数,而$\sigma_i$和$\overline{\sigma_i}$分量互为共轭.按照如下典范同构,就有$\textbf{R}$是一个$n$维实代数.进而有$\textbf{C}=\textbf{R}\otimes_{\mathbb{R}}\mathbb{C}$.
	$$\textbf{R}\cong\prod_{\sigma\in X}\mathbb{R}$$
	$$(z_{\tau})_{\tau}\mapsto(x_{\tau})_{\tau}$$
	$$x_{\tau}=\left\{\begin{array}{cc}z_{\tau}&\tau=\rho_i\\\mathrm{Re}(z_{\tau})&\tau=\sigma_i\\\mathrm{Im}(z_{\tau})&\tau=\overline{\sigma_i}\end{array}\right.$$
	
	特别的,如果$K$是$n$次数域,取$X=\mathrm{Hom}_{\mathbb{Q}}(K,\mathbb{C})$,那么$\textbf{C}=K_{\mathbb{C}}$和$\textbf{R}=K_{\mathbb{R}}$是我们之前定义的Minkowiski空间.
	\item 设$z\in\textbf{C}$或者$\textbf{C}^*$,$\textbf{C}$上数乘$z$这个线性变换的迹或者范数(行列式)就是如下加法群同态或者乘法群同态:
	$$\mathrm{Tr}:\textbf{C}\to\mathbb{C},z\mapsto\sum_{\tau}z_{\tau}$$
	$$\mathrm{N}:\textbf{C}^*\to\mathbb{C}^*,z\mapsto\prod_{\tau}z_{\tau}$$
	
	定义$\textbf{C}$上的Hermite内积为:
	$$\langle x,y\rangle=\sum_{\tau}x_{\tau}\overline{y_{\tau}}$$
	
	如果记${^*z}={^*(z_{\tau})_{\tau}}=(\overline{z_{\tau}})_{\tau}$,这个同态的二次复合是恒等,那么有:
	$$\langle x,y\rangle=\mathrm{Tr}(x({^*y}))$$
	
	进而${^*z}$是$z$在这个内积下的伴随元,也即对任意$x,y\in\textbf{C}$有:
	$$\langle zx,y\rangle=\langle x,{^*z}y\rangle$$
	
	如果再记$z^*=(z_{\tau})_{\tau}^*=(z_{\overline{\tau}})_{\tau}$,这个同态的二次复合也是恒等,并且有${^*z}^*=({^*z})^*={^*(z^*)}=\overline{z}$.
	
	最后这个Hermite内积限制在$\textbf{R}$上是一个实内积.
	\item 半平面.记$\textbf{R}$中那些满足$x=x^*$的元$x$构成的子集为$\textbf{R}_{\pm}$.换句话讲这样的元满足$x_{\overline{\tau}}=x_{\tau}\in\mathbb{R}$.此时$x$的分量都是实数,用记号$x>0$表示它的分量都$>0$.定义乘法群$\mathbb{R}_+^*=\{x\in\mathbb{R}_{\pm}\mid x>0\}$.我们定义两个乘法群同态,它们完全模仿了复平面的情况:
	$$|\bullet|:\textbf{R}^*\to\textbf{R}_+^*\qquad x=(x_{\tau})_{\tau}\mapsto|x|=(|x_{\tau}|)_{\tau}$$
	$$\ln:\textbf{R}_+^*\cong \mathbb{R}_{\pm}\qquad x=(x_{\tau})_{\tau}\mapsto\ln x=(\ln x_{\tau})_{\tau}$$
	
	定义$X$诱导的上半平面为:
	$$\textbf{H}=\textbf{R}_{\pm}+i\textbf{R}^*_+$$
	
	如果对$z\in\textbf{C}$记$\mathrm{Re}(z)=\frac{1}{2}(z+\overline{z})$和$\mathrm{Im}(z)=\frac{1}{2i}(z-\overline{z})$,那么上半平面可以记作:
	$$\textbf{H}=\{z\in\textbf{C}\mid z=z^*,\mathrm{Im}(z)>0\}$$
	
	如果$z\in\textbf{H}$,按照$z\overline{z}\mathrm{Im}(-1/z)=-\mathrm{Im}(z^{-1}z\overline{z})=\mathrm{Im}(z)>0$,以及$z\overline{z}\in\textbf{R}_+^*$,就有$-1/z\in\textbf{H}$.
	
	对$z=(z_{\tau})_{\tau},p=(p_{\tau})_{\tau}\in\textbf{C}$,记$z^p=(z_{\tau}^{p_{\tau}})_{\tau}=(e^{p_{\tau}\ln z_{\tau}})_{\tau}$,这里$\ln$取去掉负半实轴的分支.
	
	综上我们有完全模仿复平面的设定:
	$$\textbf{H}\subseteq\textbf{C}\supseteq\textbf{R}\supseteq\textbf{R}_{\pm}\supseteq\textbf{R}_+^*$$
	\item Schwartz函数.一个Schwartz函数$f:\textbf{R}\to\mathbb{C}$是指一个光滑函数(任意阶可导),并且对任意自然数$m$和$n$有$\lim\limits_{|x|\to\infty}x^mf^{(n)}(x)=0$.它的傅里叶变换定义为:
	$$\widehat{f}(y)=\int_{\textbf{R}}f(x)e^{-2\pi i\langle x,y\rangle}\mathrm{d}x$$
	
	这里$\mathrm{d}x$是$\textbf{R}$上的规范Haar测度(规范指的是在标准正交基生成的单位超正方体的体积为1,在某个等距下它就是$\mathbb{R}^n$上的勒贝格测度).Schwartz函数的傅里叶变换一定是绝对且一致收敛的,并且仍然是一个Schwartz函数.
	\item 
	\begin{enumerate}[(1)]
		\item 最基本的Schwartz函数是$h(x)=e^{-\pi\langle x,x\rangle}$,它是自身的傅里叶变换.
		\item 设$f$是任意Schwartz函数,设$A$是$\textbf{R}$上的线性变换,设$f_A(x)=f(Ax)$,那么它有如下傅里叶变换,其中${^tA}$是$A$的伴随变换:
		$$\widehat{f_A}(y)=\frac{1}{|\det A|}\widehat{f}({^tA}^{-1}y)$$
	\end{enumerate}
    \begin{proof}
    	
    	(1):按照$h(x)=\prod_{1\le i\le n}e^{-\pi x_i^2}$和$\widehat{h}(x)=\prod_{1\le i\le n}\widehat{(e^{-\pi x_i^2})}$,归结为设$n=1$.此时对$\widehat{h}(y)$求导有:
    	$$\frac{\mathrm{d}}{\mathrm{d}y}\widehat{h}(y)=-2\pi i\int_{\mathbb{R}}xh(x)e^{-2\pi ixy}\mathrm{d}x=-2\pi y\widehat{h}(y)$$	
    	于是有$\widehat{h}(y)=Ce^{-\pi y^2}$.带入$y=0$结合$\int_{\mathbb{R}}e^{-\pi x^2}\mathrm{d}x=1$得到$C=1$.(2)是因为:
    	\begin{align*}
    		\widehat{f_A}(y)&=\int_{\textbf{R}}f(Ax)e^{-2\pi i\langle x,y\rangle}\mathrm{d}x\\&=\int_{\textbf{R}}f(x)e^{-2\pi i\langle A^{-1}x,y\rangle}|\det A|^{-1}\mathrm{d}x\\&=\frac{1}{|\det A|}\int_{\textbf{R}}f(x)e^{-2\pi i\langle x,{^tA}^{-1}y\rangle}\mathrm{d}x\\&=\frac{1}{|\det A|}\widehat{f}({^tA}^{-1}y)
    	\end{align*}
    \end{proof}
    \item 泊松求和公式(Poisson summation formula).设$\Gamma$为$\textbf{R}$的完备格,它基本网孔的体积记作$\mathrm{vol}(\Gamma)$.记$\Gamma$的对偶格:
    $$\Gamma'=\{g'\in\textbf{R}\mid\langle g,g;\rangle\in\mathbb{Z},\forall g\in\Gamma\}$$
    那么对任意Schwartz函数$f$有:
    $$\sum_{g\in\Gamma}f(g)=\frac{1}{\mathrm{vol}(\Gamma)}\sum_{g'\in\Gamma'}\widehat{f}(g')$$
    \begin{proof}
    	
    	把$\textbf{R}$等距同构到$\mathbb{R}^n$,此时规范Haar测度就是Lebesgue测度.另外此时就存在可逆线性变换$A$满足$\Gamma=A\mathbb{Z}^n$,那么$\Gamma$的基本网孔的体积$\mathrm{vol}(\Gamma)=|\det A|$.我们有$g'\in\Gamma'$等价于对任意$x\in\mathbb{Z}^n$有${^tx}{^tA}g'\in\mathbb{Z}$,等价于${^tA}g'\in\mathbb{Z}^n$,等价于$g'\in{^tA}^{-1}\mathbb{Z}^n$,于是有$\Gamma'=A^*\mathbb{Z}^n$,其中$A^*={^tA}^{-1}$.于是问题变为:
    	$$\sum_{x\in\mathbb{Z}^n}f_A(x)=\sum_{x\in\mathbb{Z}^n}\widehat{f_A}(x)$$
    	
    	不妨用$f$替换$f_A$,记$g(x)=\sum_{k\in\mathbb{Z}^n}f(x+k)$,按照$f$是Schwartz函数,当$x$跑遍一个紧集时对几乎所有的$k\in\mathbb{Z}^n$有$|f(x+k)|\cdot|k|^{n+1}\le C$,于是从$\sum_{k\not=0}\frac{1}{|k|^{n+1}}$收敛得到$g$是绝对且局部一致收敛的.并且这件事对$g$的任意导数成立,于是$g$是光滑函数.它是周期的,对任意$\textbf{n}\in\mathbb{Z}^n$有$g(x+\textbf{n})=g(x)$.于是它有傅里叶展开$g(x)=\sum_{\textbf{n}\in\mathbb{Z}^n}a_{\textbf{n}}e^{2\pi i{^t\textbf{n}}x}$,它的系数是熟知的:
    	\begin{align*}
    		a_{\textbf{n}}&=\int_{[0,1]^n}g(x)e^{-2\pi i{^t\textbf{n}}x}\mathrm{d}x\\&=\sum_{k\in\mathbb{Z}^n}\int_{[0,1]^n}f(x+k)e^{-2\pi i{^t\textbf{n}}x}\mathrm{d}x\\&=\widehat{f}(\textbf{n})
    	\end{align*}
        于是有:
        $$\sum_{\textbf{n}\in\mathbb{Z}^n}f(\textbf{n})=g(0)=\sum_{\textbf{n}\in\mathbb{Z}^n}a_{\textbf{n}}=\sum_{\textbf{n}\in\mathbb{Z}^n}\widehat{f}(\textbf{n})$$ 	
    \end{proof}
    \item 设$p=(p_{\tau})_{\tau}$由非负整数构成,满足对那些共轭下不变的指标$\rho$有$p_{\rho}\in\{0,1\}$,对其余的指标有$p_{\tau}p_{\overline{\tau}}=0$,这样的元$p$称为容许的(admissible).设$a,b\in\textbf{R}$,考虑如下函数:
    $$f(x)=f_p(a,b,x)=\mathrm{N}\left((x+a)^p\right)e^{-\pi\langle a+x,a+x\rangle+2\pi i\langle b,x\rangle}$$
    我们断言$f(x)$是$\textbf{R}$上的Schwartz函数,并且它的傅里叶变换为:
    $$\widehat{f}(y)=\left[i^{\mathrm{Tr}(p)}e^{2\pi i\langle a,b\rangle}\right]^{-1}f_p(-b,a,y)$$
    \begin{proof}
    	
    	按照$|f_p(a,b,x)|=|P(x)|e^{-\pi\langle a+x,a+x\rangle}$,其中$P(x)$是一个多项式,得到$f_p(a,b,x)$是Schwartz函数.如果$p=0$,就有$f(x)=f_0(a,b,x)=h(a+x)e^{2\pi i\langle b,x\rangle}$,其中$h(x)=e^{-\pi\langle x,x\rangle}$是自身的傅里叶变换,进而有:
    	\begin{align*}
    		\widehat{f}(y)&=\int_{\textbf{R}}h(a+x)e^{2\pi i\langle b,x\rangle}e^{-2\pi i\langle x,y\rangle}\mathrm{d}x\\&=\int_{\textbf{R}}h(x)e^{-\2pi i\langle y-b,x-a\rangle}\mathrm{d}x\\&=e^{2\pi i\langle y-b,a\rangle}\widehat{h}(y-b)\\&=e^{-2\pi i\langle a,b\rangle}e^{-\pi\langle y-b,y-b\rangle+2\pi i\langle y,a\rangle}\\&=e^{-2\pi i\langle a,b\rangle}f_0(-b,a,y)
    	\end{align*}
        我们证明了$p=0$的情况.对于任意的容许$p$,对上述等式的$a$求$p$次偏导数就能得到一般情况,但是当$\tau\not=\overline{\tau}$时这里的函数对$a$未必是解析的.为此,我们设$X=\{\rho_i,\sigma,\overline{\sigma}\}$,这个排序使得$p_{\overline{\sigma}}=0$.那么有$\langle a+x,a+x\rangle=\sum_{\rho}(a_{\rho}+x_{\rho})^2+2\sum_{\sigma}(a_{\sigma}+x_{\sigma})(a_{\overline{\sigma}}+x_{\overline{\sigma}})$.对$p=0$的情况的等式中的所有$a_{\rho}$求导$p_{\rho}$次,设$a_{\overline{\sigma}}=\xi_{\overline{\sigma}}+i\eta_{\overline{\sigma}}$,再对结果做$p_{\sigma}$次偏导数$\frac{\partial}{\partial a_{\overline{\sigma}}}=\frac{1}{2}\left(\frac{\partial}{\partial\xi_{\overline{\sigma}}}-i\frac{\partial}{\partial\eta_{\overline{\sigma}}}\right)$:
        \begin{itemize}
        	\item $p=0$的等式左侧是:
        	$$\widehat{f_0}(a,b,y)=\int_{\textbf{R}}e^{-\pi\langle a+x,a+x\rangle+2\pi i\langle b,x\rangle}e^{-2\pi i\langle x,y\rangle}\mathrm{d}x$$
        	按照上述求偏导就得到:
        	\begin{align*}
        		&\quad\int_{\textbf{R}}\prod_{\rho}\left(-2\pi(a_{\rho}+x_{\rho})\right)^{p_{\rho}}\prod_{\sigma}\left(-2\pi(a_{\sigma}+x_{\sigma})\right)^{p_{\sigma}}e^{-\pi\langle a+x,a+x\rangle+2\pi i\langle b,x\rangle-2\pi i\langle x,y\rangle}\mathrm{d}x\\&=\mathrm{N}\left((-2\pi)^p\right)\int_{\textbf{R}}\mathrm{N}\left((a+x)^p\right)e^{-\pi\langle a+x,a+x\rangle+2\pi i\langle b,x\rangle}e^{-2\pi i\langle x,y\rangle}\mathrm{d}x\\&=\mathrm{N}\left((-2\pi)^p\right)\widehat{f}_p(a,b,y)
        	\end{align*}
            \item $p=0$的等式右侧是:
            $$e^{-2\pi i\langle a,b\rangle-\pi\langle-b+y,-b+y\rangle+2\pi i\langle a,y\rangle}=e^{2\pi i\langle a,-b+y\rangle-\pi\langle-b+y,-b+y\rangle}$$
            这里:
            $$\langle a,-b+y\rangle=\sum_{\rho}a_{\rho}(-b_{\rho}+y_{\rho})+\sum_{\sigma}\left(a_{\sigma}(-b_{\overline{\sigma}}+y_{\overline{\sigma}})+a_{\overline{\sigma}}(-b_{\sigma}+y_{\sigma})\right)$$
            于是上述求导就得到:
            \begin{align*}
            	&\quad\mathrm{N}\left((2\pi i)^p\right)\mathrm{N}\left((-b+y)^p\right)e^{-2\pi i\langle a,b\rangle}f_0(-b,a,y)\\&=\mathrm{N}\left((2\pi i)^p\right)e^{-2\pi i\langle a,b\rangle}f_p(-b,a,y)
            \end{align*}
        \end{itemize}
    \end{proof}
    \item 广义Theta函数.对$\textbf{R}$的任意完备格$\Gamma$,对任意$a,b\in\textbf{R}$和容许的$p\in\textbf{R}$,定义广义Theta函数$\textbf{R}\times\textbf{R}\times\textbf{H}\to\mathbb{C}$为:
    $$\theta_{\Gamma}^p(a,b,z)=\sum_{g\in\Gamma}\mathrm{N}\left((a+g)^p\right)e^{\pi\langle(a+g)z,a+g\rangle+2\pi i\langle b,g\rangle}$$
    特别的如果$a=b=p=0$就简单记作:
    $$\theta_{\Gamma}(z)=\sum_{g\in\Gamma}e^{\pi i\langle gz,g\rangle}$$
    我们断言对固定的$\Gamma$和$p$,广义Theta函数是$\textbf{R}\times\textbf{R}\times\textbf{H}\to\mathbb{C}$上的解析函数,并且在任意紧子集上绝对且一致收敛.
    \begin{proof}
    	
    	对正实数$\delta$,对任意$z\in\textbf{H}$,我们有:
    	$$|\mathrm{N}\left((a+g)^p\right)e^{\pi\langle(a+g)z,a+g\rangle+2\pi i\langle b,g\rangle}|\le|\mathrm{N}\left((a+g)^p\right)|e^{-\pi\delta\langle a+g,a+g\rangle}$$
    	把右侧记作$f_g(a)$,其中$g\in\Gamma$和$a\in\textbf{R}$.对任意紧子集$\textbf{K}\subseteq\textbf{R}$,记$|f_g|_{\textbf{K}}=\sup_{x\in\textbf{K}}|f_g(x)|$.问题归结为证明:
    	$$\sum_{g\in\Gamma}|f_g|_{\textbf{K}}<\infty$$
    	取$g_1,\cdots,g_n$是$\Gamma$的一组$\mathbb{Z}$基.任取$g=\sum_{1\le i\le n}m_ig_i\in\Gamma$,记$\mu_g=\max_i|m_i|$.再记$\Vert x\Vert=\sqrt{\langle x,x\rangle}$.如果$\Vert g\Vert\ge 4\sup_{x\in\textbf{K}}\Vert x\Vert$,再记$\varepsilon=\inf_{\sum y_i^2=1}\langle g_i,g_j\rangle y_iy_j$是矩阵$\left(\langle g_i,g_j\rangle\right)$的最小特征值.那么对任意$a\in\textbf{K}$就有:
    	\begin{align*}
    		\langle a+g,a+g\rangle&\ge\left(\Vert a\Vert-\Vert g\Vert\right)^2\\&\ge\Vert g\Vert^2-2\Vert a\Vert\cdot\Vert g\Vert\\&\ge\frac{1}{2}\Vert g\Vert^2\\&\ge\frac{1}{2}\varepsilon\sum_{i=1}^nm_i^2\\&\ge\frac{1}{2}\varepsilon\mu_g^2
    	\end{align*}
    	$\mathrm{N}\left((a+\sum_im_ig_i)^p\right)$关于$m_i$是次数为$q=\mathrm{Tr}(p)$的多项式,它的系数是$a$的连续函数,如果取$\mu_g$足够大,就可以保证对任意$a\in\textbf{K}$有$|\mathrm{N}\left((a+g)^p\right)|\le\mu_g^{q+1}$.那些$\Vert g\Vert<4\sup_{x\in\textbf{K}}\Vert x\Vert$的$g$只有有限个,除去这有限个点的子集记作$\Gamma'\subseteq\Gamma$.那么有:
    	$$\sum_{g\in\Gamma'}|f_g|_{\textbf{K}}\le\sum_{\mu\ge0}P(\mu)\mu^{q+1}e^{-\pi\delta\varepsilon\mu^2/2}$$
    	
    	这里$P(\mu)=\#\{\textbf{m}\in\mathbb{Z}^n\mid\max_i|m_i|=\mu\}=(2\mu+1)^n-(2\mu-1)^n$.于是得到收敛性.
    \end{proof}
    \item Theta变换公式,这里$\Gamma'$是完备格$\Gamma$的对偶格:
    $$\theta_{\Gamma}^p(a,b,-1/z)=\left[i^{\mathrm{Tr}(p)}e^{2\pi i\langle a,b\rangle}\mathrm{vol}(\Gamma)\right]^{-1}\mathrm{N}\left((z/i)^{p+1/2}\right)\theta^p_{\Gamma'}(-b,a,z)$$
    特别的有:
    $$\theta_{\Gamma}(-1/z)=\frac{\sqrt{\mathrm{N}(z/i)}}{\mathrm{vol}(\Gamma)}\theta_{\Gamma'}(z)$$
    \begin{proof}
    	
    	我们已经证明了广义Theta函数的解析性.不妨设$z$的实部为零【?】,于是可以做代换$z=iy$,其中$y\in\textbf{R}_+^*$,再记$t=y^{-1/2}$,于是有$z=i/t^2$和$-1/z=it^2$.按照$t={^*t}=t^*$,有$\langle \xi t,\eta\rangle=\langle\xi,{^*t}\eta\rangle=\langle\xi,t\eta\rangle$,于是有:
    	$$\theta_{\Gamma}^p(a,b,-1/z)=\mathrm{N}(t^{-p})\sum_{g\in\Gamma}\mathrm{N}\left((ta+tg)^p\right)e^{-\pi\langle ta+tg,ta+tg\rangle+2\pi i\langle t^{-1}b,tg\rangle}$$
    	如果记$\alpha=ta$和$\beta=t^{-1}b$,那么有:
    	$$f_p(\alpha,\beta,x)=\mathrm{N}\left((\alpha+x)^p\right)e^{-\pi\langle\alpha+x,\alpha+x\rangle+2\pi i\langle\beta,x\rangle}$$
    	记$\varphi_t(\alpha,\beta,x)=f_p(\alpha,\beta,tx)$,得到:
    	$$\theta_{\Gamma}^p(a,b,-1/z)=\mathrm{N}(t^{-p})\sum_{g\in\Gamma}\varphi_t(\alpha,\beta,g)$$
    	类似的有:
    	$$\theta_{\Gamma'}^p(-b,a,z)=\mathrm{N}(t^p)\sum_{g'\in\Gamma'}\varphi_{1/t}(-\beta,\alpha,g')$$
    	对$f(x)=\varphi_t(\alpha,\beta,x)=f_p(\alpha,\beta,tx)$用泊松求和公式:
    	$$\sum_{g\in\Gamma}f(g)=\frac{1}{\mathrm{vol}(\Gamma)}\sum_{g'\in\Gamma'}\widehat{f}(g')$$
    	下面计算它的傅里叶变换.设$h(x)=f_p(\alpha,\beta,x)$,那么$f(x)=h(tx)=h_t(x)$.变换$A:x\mapsto tx$是自伴随的,并且行列式为$\mathrm{N}(t)$,于是我们之前证明了$\widehat{f}(y)=\frac{1}{\mathrm{N}(t)}\widehat{h}(t^{-1}y)$.于是我们之前给出了傅里叶变换为:
    	\begin{align*}
    		\widehat{f}(y)&=\left[\mathrm{N}(i^p)\mathrm{N}(t)e^{2\pi i\langle a,b\rangle}\right]^{-1}f_p(-\beta,\alpha,t^{-1}y)\\&=\left[\mathrm{N}(i^p)\mathrm{N}(t)e^{2\pi i\langle a,b\rangle}\right]^{-1}\varphi_{1/t}(-\beta,\alpha,y)
    	\end{align*}
    	于是上述泊松求和公式就给出了:
    	$$\theta_{\Gamma}^p(a,b,-1/z)=\left[\mathrm{N}(i^pt^{2p+1})e^{2\pi i\langle a,b\rangle}\mathrm{vol}(\Gamma)\right]^{-1}\theta_{\Gamma'}^p(-b,a,z)$$
    	按照$t=(z/i)^{-1/2}$就得到我们要的等式.
    \end{proof}
\end{enumerate}
\subsection{高阶Gamma函数}
\begin{enumerate}
	\item 设$\mathfrak{p}=\{\tau,\overline{\tau}\}$是$X$的一个轨道,如果它只由一个元构成,就称它是实的,否则由两个元构成就称为复的.于是有分解:
	$$\textbf{R}_+^*=\prod_{\mathfrak{p}}\textbf{R}_{+\mathfrak{p}}^*$$
	其中对实轨道$\mathfrak{p}$有$\textbf{R}_{+\mathfrak{p}}^*=\mathbb{R}_+^*$,对复轨道$\mathfrak{p}$有$\textbf{R}_{+\mathfrak{p}}^*=\{(y,y)\mid y\in\mathbb{R}_+^*\}$.定义同构$\textbf{R}_{+\mathfrak{p}}^*\cong\mathbb{R}_+^*$为$y\mapsto y$或者$(y,y)\mapsto y^2$.于是我们得到同构:
	$$\varphi:\textbf{R}_+^*\cong\prod_{\mathfrak{p}}\mathbb{R}_+^*$$
	$\mathbb{R}_+^*$上的规范Haar测度记作$\mathrm{d}t/t$,把$\prod_{\mathfrak{p}}\mathbb{R}_+^*$上的乘积Haar测度在上述同构下对应的$\textbf{R}_+^*$上的Haar测度记作$\mathrm{d}y/y$,它称为标准测度.它在$\ln:\textbf{R}_+^*\cong\textbf{R}_{\pm}$复合上$\textbf{R}_{\pm}=\prod_{\mathfrak{p}}\textbf{R}_{\pm\mathfrak{p}}\cong\prod_{\mathfrak{p}}\mathbb{R}$,$x_{\mathfrak{p}}\to x_{\mathfrak{p}}$和$(x_{\mathfrak{p}},x_{\mathfrak{p}})\mapsto2x_{\mathfrak{p}}$,就是$\prod_{\mathfrak{p}}\mathbb{R}$上的Lebesgue测度.
	\item 对$\textbf{s}=(s_{\tau})\in\textbf{C}$满足$\mathrm{Re}(s_{\tau})>0$,定义Gamma函数为:
	$$\Gamma_X(\textbf{s})=\int_{\textbf{R}_+^*}\mathrm{N}(e^{-y}y^{\textbf{s}})\frac{\mathrm{d}y}{y}$$
	我们有:
	$$\Gamma_X(\textbf{s})=\prod_{\mathfrak{p}}\Gamma_{\mathfrak{p}}\Gamma_{\mathfrak{p}}(\textbf{s}_{\mathfrak{p}})$$
	
	其中如果$\mathfrak{p}=\{\rho\}$是实轨道,那么$\textbf{s}_{\mathfrak{p}}=s_{\rho}$;如果$\mathfrak{p}=\{\sigma,\overline{\sigma}\}$是复轨道,那么$\textbf{s}_{\mathfrak{p}}=(s_{\tau},s_{\overline{\tau}})$.这里乘积的分量定义为:
	$$\Gamma_{\mathfrak{p}}(\textbf{s}_{\mathfrak{p}})=\left\{\begin{array}{cc}\Gamma(\textbf{s}_{\mathfrak{p}})&\mathfrak{p}\text{是实数轨道}\\2^{1-\mathrm{Tr}(\textbf{s}_{\mathfrak{p}})}\Gamma(\mathrm{Tr}(\textbf{s}_{\mathfrak{p}}))&\mathfrak{p}\text{是复数轨道}\end{array}\right.$$
	
	特别的,这得到$\Gamma_X(\textbf{s})$在$\mathrm{Re}(s_{\tau})>0$上是解析的,并且可以亚纯延拓到整个$\textbf{C}$上.
	\begin{proof}
		
		做乘积分解,问题归结为设$X$上的$G$作用是可迁的.如果$X$是单元集合,那么明显有$\Gamma_X(\textbf{s})=\Gamma(s)$.设$X=\{\tau,\overline{\tau}\}$是二元集合.考虑映射:
		$$\psi:\mathbb{R}_+^*\to\textbf{R}_+^*$$
		$$t\mapsto\left(\sqrt{t},\sqrt{t}\right)$$
		于是有:
		\begin{align*}
			\int_{\textbf{R}_+^*}\mathrm{N}\left(e^{-y}y^{\textbf{s}}\right)\frac{\mathrm{d}y}{y}\\&=\int_{\mathbb{R}_+^*}\mathrm{N}\left(e^{-(\sqrt{t},\sqrt{t})}(\sqrt{t},\sqrt{t})^{s_{\tau},s_{\overline{\tau}}}\right)\frac{\mathrm{d}t}{t}\\&=\int_0^{\infty}e^{-2\sqrt{t}}\sqrt{t}^{\mathrm{Tr}(\textbf{s})}\frac{\mathrm{d}t}{t}
		\end{align*}
		做代换$t\mapsto(t/2)^2$得到:
		$$\int_{\textbf{R}_+^*}\mathrm{N}(e^{-y}y^{\textbf{s}})\frac{\mathrm{d}y}{y}=2^{1-\mathrm{Tr}(\textbf{s})}\Gamma(\mathrm{Tr}(\textbf{s}))$$
	\end{proof}
    \item 定义$X$上的L函数为:
    $$L_X(\textbf{s})=\mathrm{N}(\pi^{-s/2})\Gamma_X(\textbf{s}/2)$$
    那么明显有:
    $$L_X(\textbf{s})=\prod_{\mathfrak{p}}L_{\mathfrak{p}}(\textbf{s}_{\mathfrak{p}})$$
    其中如果$\mathfrak{p}=\{\rho\}$则$\textbf{s}_{\mathfrak{p}}=s_{\rho}$;否则$\mathfrak{p}=\{\sigma,\overline{\sigma}\}$有$\textbf{s}_{\mathfrak{p}}=(s_{\sigma},s_{\overline{\sigma}})$.并且:
    $$L_{\mathfrak{p}}(\textbf{s}_{\mathfrak{p}})=\left\{\begin{array}{cc}\pi^{-\textbf{s}_{\mathfrak{p}}}\Gamma(\textbf{s}_{\mathfrak{p}}/2)&\mathfrak{p}\text{是实数轨道}\\2(2\pi)^{-\mathrm{Tr}(\textbf{s}_{\mathfrak{p}})/2}\Gamma(\mathrm{Tr}(\textbf{s}_{\mathfrak{p}})/2)&\mathfrak{p}\text{是复数轨道}\end{array}\right.$$
    对$s\in\mathbb{C}$,记$\Gamma_X(s)=\Gamma_X(s\textbf{1})$,那么如果记$X$上实轨道和复轨道个数分别为$r_1$和$r_2$,记$|X|=n$.那么有:
    $$\Gamma_X(s)=2^{(1-2s)r_2}\Gamma(s)^{r_1}\Gamma(2s)^{r_2}$$
    $$L_X(s)=L_X(s\textbf{1})=\pi^{-ns/2}\Gamma_X(s/2)$$
    特别的:
    $$\begin{array}{cc}L_{\mathbb{R}(s)}=L_X(s)=\pi^{-s/2}\Gamma(s/2)&X=\{\rho\}\\L_{\mathbb{C}}(s)=L_X(s)=2(2\pi)^{-s}\Gamma(s)&X=\{\sigma,\overline{\sigma}\}\end{array}$$
    进而有:
    $$L_X(s)=L_{\mathbb{R}}(s)^{r_1}L_{\mathbb{C}}(s)^{r_2}$$
    我们有如下性质:
    \begin{enumerate}[(1)]
    	\item $L_{\mathbb{R}}(1)=1$,$L_{\mathbb{C}}(1)=1/\pi$.
    	\item $L_{\mathbb{R}}(s+2)=\frac{s}{2\pi}L_{\mathbb{R}}(s)$,$L_{\mathbb{C}}(s+1)=\frac{s}{2\pi}L_{\mathbb{C}}(s)$.
    	\item $L_{\mathbb{R}}(1-s)L_{\mathbb{R}}(1+s)=\frac{1}{\cos\pi s/2}$,$L_{\mathbb{C}}(s)L_{\mathbb{C}}(1-s)=\frac{2}{\sin\pi s}$.
    	\item $L_{\mathbb{R}}(s)L_{\mathbb{R}}(s+1)=L_{\mathbb{C}}(s)$.
    \end{enumerate}
    \item $L_X(s)=A(s)L_X(1-s)$,其中:
    $$A(s)=(\cos\pi s/2)^{r_1+r_2}(\sin\pi s/2)^{r_2}L_{\mathbb{C}}(s)^n$$
    \begin{proof}
    	
    	一方面有:
    	$$\frac{L_{\mathbb{R}}(s)}{L_{\mathbb{R}}(1-s)}=\frac{L_{\mathbb{R}}(s)L_{\mathbb{R}}(1+s)}{L_{\mathbb{R}}(1-s)L_{\mathbb{R}}(1+s)}=(\cos\pi s/2)L_{\mathbb{C}}(s)$$
    	另一方面有:
    	$$\frac{L_{\mathbb{C}}(s)}{L_{\mathbb{C}}(1-s)}=\frac{L_{\mathbb{C}}(s)^2}{L_{\mathbb{C}}(1-s)L_{\mathbb{C}}(s)}=\frac{1}{2}(\sin\pi s)L_{\mathbb{C}}(s)^2=(\cos\pi s/2)(\sin\pi s/2)L_{\mathbb{C}}(s)^2$$
    	于是得到:
    	\begin{align*}
    		\frac{L_X(s)}{L_X(1-s)}&=\frac{L_{\mathbb{R}}(s)^{r_1}L_{\mathbb{C}}(s)^{r_2}}{L_{\mathbb{R}}(1-s)^{r_1}L_{\mathbb{C}}(1-s)^{r_2}}\\&=(\cos\pi s/2)^{r_1+r_2}(\sin\pi s/2)^{r_2}L_{\mathbb{C}}(s)^n
    	\end{align*}
    \end{proof}
\end{enumerate}
\subsection{Dedekind's Zeta函数}
\begin{enumerate}
	\item 设$K$是$n$次数域,其上的Dedekind zeta函数定义为:
	$$\zeta_K(s)=\sum_{\mathfrak{a}}\frac{1}{\mathfrak{R}(\mathfrak{a})^s}$$
	
	其中$\mathfrak{a}$跑遍$K$的整理想,$\mathfrak{R}(\mathfrak{a})$是理想的绝对范数,也即$|\mathscr{O}_K/\mathfrak{a}|$.它在半平面$\mathrm{Re}(s)>1$上内闭绝对且一致收敛.我们的主要目标依旧是对它做解析延拓.
	\item 对理想类群$\mathrm{Cl}(K)=J/P$中的等价类$\mathscr{R}$,定义$\zeta(\mathscr{R},s)=\sum_{\mathfrak{a}}\frac{1}{\mathfrak{R}(\mathfrak{a})^s}$,其中$\mathfrak{a}$跑遍$\mathscr{R}$中的整理想.于是有:
	$$\zeta_K(s)=\sum_{\mathscr{R}\in\mathrm{Cl}(K)}\zeta(\mathscr{R},s)$$
	\item 对分式理想$\mathfrak{a}$,单位群$\mathscr{O}^*=\mathscr{O}_K^*$就作用在$\mathfrak{a}^*$上,把轨道集合记作$\mathfrak{a}^*/\mathscr{O}^*$.我们断言如果$\mathfrak{a}$是整理想,那么有如下双射:
	$$\mathfrak{a}^*/\mathscr{O}^*\cong\{\mathscr{R}\text{中的整理想}\mathfrak{b}\}$$
	$$\overline{a}\mapsto\mathfrak{b}=a\mathfrak{a}^{-1}$$
	\begin{proof}
		
		设$a\in\mathfrak{a}^*$,那么$a\mathfrak{a}^{-1}$是$\mathscr{R}$中的整理想,并且如果$a\mathfrak{a}^{-1}=b\mathfrak{a}^{-1}$,那么有$(a)=(b)$,进而$ab^{-1}\in\mathscr{O}^*$,这证明了单射.反过来任取整理想$\mathfrak{b}\in\mathscr{R}$,按照定义就有$\mathfrak{b}=a\mathfrak{a}^{-1}$,进而有$a\in\mathfrak{a}\mathfrak{b}\subseteq\mathfrak{a}$.
	\end{proof}
	\item 取$X=\mathrm{Hom}_{\mathbb{Q}}(K,\mathbb{C})$,那么数域$K$可以典范嵌入$K_{\mathbb{R}}=\textbf{R}$中,即把$a\in K$映为$(\tau(a))_{\tau}$.此时对$a\in K^*$就有$\mathscr{R}((a))=|\mathrm{N}_{K/\mathbb{Q}}(a)|=|\mathrm{N}(a)|$.最后一项中把$a$视为$(\tau(a))_{\tau}$.上一条就导致:设$\mathscr{R}$的一个代表元是整理想$\mathfrak{a}$的逆,那么有
	$$\zeta(\mathscr{R},s)=\mathfrak{R}(\mathfrak{a})^s\sum_{\overline{a}\in\mathfrak{a}^*/\mathscr{O}^*}\frac{1}{|\mathrm{N}(\overline{a})|^s}$$
	\item Theta函数.我们知道整理想$\mathfrak{a}$构成了$\textbf{R}$的完备格,它的基本网孔的体积为$\mathrm{vol}(\mathfrak{a})=\sqrt{d_{\mathfrak{a}}}$,这里$d_{\mathfrak{a}}=\mathfrak{R}(\mathfrak{a})^2|d_K|$,其中$d_K$是$K$的判别式.对每个级数$\zeta(\mathscr{R},s)$赋予Theta函数:
	$$\theta(\mathfrak{a},z)=\theta_{\mathfrak{a}}(z/d_{\mathfrak{a}}^{1/n})=\sum_{a\in\mathfrak{a}}e^{\pi i\langle az/d_{\mathfrak{a}}^{1/n},a\rangle}$$
	\item 完备函数$Z(\mathscr{R},s)$.考虑Gamma函数:
	$$\Gamma_K(s)=\Gamma_X(s)=\int_{\textbf{R}_+^*}\mathrm{N}(e^{-y}y^s)\frac{\mathrm{d}y}{y}$$
	其中$s\in\mathbb{C}$的实部为正.做代换$y\mapsto\pi|a|^2y/d_{\mathfrak{a}}^{1/n}$,其中$|a|=(|a_{\tau}|)_{\tau}$,得到:
	$$|d_K|^s\pi^{-ns}\Gamma_K(s)\frac{\mathfrak{R}(\mathfrak{a})^{2s}}{|\mathrm{N}(a)|^{2s}}=\int_{\textbf{R}_+^*}e^{-\pi\langle ay/d_{\mathfrak{a}}^{1/n},a\rangle}\mathrm{N}(y)^s\frac{\mathrm{d}y}{y}$$
	对$a$跑遍$\mathfrak{a}^*/\mathscr{O}^*$的一个代表元集合求和,得到:
	$$|d_K|^s\pi^{-ns}\Gamma_K(s)\zeta(\mathscr{R},2s)=\int_{\textbf{R}_+^*}g(y)\mathrm{N}(y)^s\frac{\mathrm{d}y}{y}$$
	其中$g(y)=\sum_{a\in\mathfrak{a}^*/\mathscr{O}^*}e^{-\pi\langle ay/d_{\mathfrak{a}}^{1/n},a\rangle}$.取欧拉因子:
	$$Z_{\infty}(s)=|d_K|^{s/2}\pi^{-ns/2}\Gamma_K(s/2)=|d_K|^{s/2}L_X(s)$$
	定义完备函数:
	$$Z(\mathscr{R},s)=Z_{\infty}(s)\zeta(\mathscr{R},s)$$
	模仿Riemann Zeta函数的情况,我们依旧需要把这个完备函数表示为Theta函数的积分.但是这里$g(y)$是对$\mathfrak{a}^*/\mathscr{O}^*$一组代表元求和,而Theta函数是对$a\in\mathfrak{a}$求和.这个困难可以按照如下方式修正.
	\item $Z(\mathscr{R},s)$的积分形式.
	\begin{enumerate}[(1)]
		\item 有典范分解$\textbf{R}_+^*=\textbf{S}\times\mathbb{R}_+^*$为把$y\in\textbf{R}_+^*$映为$(y/\mathrm{N}(y)^{1/n},\mathrm{N}(y)^{1/n})$.
		\item 记乘法群$\textbf{S}$上的规范Haar测度为$\mathrm{d}^*x$,那么有$\frac{\mathrm{d}y}{y}=\mathrm{d}^*x\times\frac{\mathrm{d}t}{t}$.
		\item $\ln:\textbf{R}_+^*\to\textbf{R}_{\pm}$把$\textbf{S}$映为迹零空间$H=\{x\in\textbf{R}_{\pm}\mid\mathrm{Tr}(x)=0\}$.$|\mathscr{O}^*|$包含在范数1超平面$\textbf{S}=\{x\in\textbf{R}_+^*\mid\mathrm{N}(x)=1\}$中.按照Dirichlet单位根定理,$|\mathscr{O}^*|$就对应于$H$中的一个完备格$G$,任取完备格$2G$的基本网孔的原像记作$F$.
	\end{enumerate}
	记$w=|\mu(K)|$是数域$K$的单位根群的阶数,我们有Mellin变换$Z(\mathscr{R},2s)=L(f,s)$,其中:
	$$f(t)=f_F(\mathfrak{a},t)=\frac{1}{w}\int_F\theta(\mathfrak{a},ixt^{1/n})\mathrm{d}^*x$$
	\begin{proof}
		
		考虑分解$\textbf{R}_+^*=\textbf{S}\times\mathbb{R}_+^*$,那么有:
		$$Z(\mathscr{R},2s)=\int_0^{\infty}\int_{\textbf{S}}\sum_{a\in\mathscr{R}}e^{-\pi\langle axt',a\rangle}\mathrm{d}^*xt^s\frac{\mathrm{d}t}{t}$$
		其中$t'=(t/d_{\mathfrak{a}})^{1/n}$.有无交并:
		$$\textbf{S}=\coprod_{\eta\in|\mathscr{O}^*|}\eta^2F$$
		这里$\textbf{S}$上的变换$x\mapsto\eta^2x$不改变Haar测度,把$F$映为$\eta^2F$,于是:
		\begin{align*}
			\int_{\textbf{S}}\sum_{a\in\mathscr{R}}e^{-\pi\langle axt',a\rangle}\mathrm{d}^*x&=\sum_{\eta\in|\mathscr{O}^*|}\int_{\eta^2F}\sum_{a\in\mathscr{R}}e^{-\pi\langle axt',a\rangle}\mathrm{d}^*x\\&=\frac{1}{w}\int_F\sum_{\varepsilon\in\mathscr{O}^*}\sum_{a\in\mathscr{R}}e^{-\pi\langle a\varepsilon xt',a\varepsilon\rangle}\mathrm{d}^*x\\&=\frac{1}{w}\int_F\left(\theta(\mathfrak{a},ixt^{1/n})-1\right)\mathrm{d}^*x\\&=f(t)-f(\infty)
		\end{align*}
	    其中第二个等号是因为$\mu(K)$是$\mathscr{O}^*\to|\mathscr{O}^*|$的核,于是$\sum_{|\mathscr{O}^*|}=\frac{1}{w}\sum_{\mathscr{O}^*}$.第三个等号是因为$a\varepsilon$恰好对应$\mathfrak{a}^*$中的元.综上得到:
	    $$Z(\mathscr{R},2s)=\int_0^{\infty}\left(f(t)-f(\infty)\right)t^s\frac{\mathrm{d}t}{t}=L(f,s)$$
	\end{proof}
	\item 上一条中的$F$关于Haar测度$\mathrm{d}^*x$的体积是$2^{r-1}R$,其中$r=r_1+r_2$是无穷素位的个数,$R$是数域$K$的regulator.
	\begin{proof}
		
		在下述同构下,$\textbf{R}^*_+$上的典范测度$\mathrm{d}y/y$对应为$\mathrm{d}^*x\times\mathrm{d}t/t$:
		$$\alpha:\textbf{S}\times\mathbb{R}_+^*,(x,t)\mapsto xt^{1/n}$$
		按照$I=\{t\in\mathbb{R}_+^*\mid1\le t\le e\}$在$\mathrm{d}t/t$下的测度为1,有$\mathrm{vol}(F)$是$F\times I$在$\mathrm{d}^*x\times\mathrm{d}t/t$下的体积.也即$\alpha(F\times I)$在$\mathrm{d}y/y$下的体积.考虑如下复合映射$\psi:$
		$$\xymatrix{\textbf{R}_+^*\ar[r]^{\ln}&\textbf{R}_{\pm}\ar[r]^{\varphi}&\prod_{\mathfrak{p}=\infty}\ar@{=}[r]&\mathbb{R}^r}$$
		这个同构把测度$\mathrm{dy}/y$对应为欧氏空间中的勒贝格测度,于是有:
		$$\mathrm{vol}(F)=\mathrm{vol}_{\mathbb{R}^r}\left(\psi\alpha(F\times I)\right)$$
		下面计算$\psi\alpha(F\times I)$,记$\textbf{1}=(1,\cdots,1)\in\textbf{S}$,于是:
		$$\psi\alpha\left((\textbf{1},t)\right)=\frac{1}{n}\textbf{e}\ln t$$
		其中$\textbf{e}=(e_1,\cdots,e_r)\in\mathbb{R}^r$,如果指标$i$对应的无穷素位$\mathfrak{p}$是实的或者复的,就分别对应$e_i=1$或者2.那么按照$F$的定义就有:
		$$\psi\alpha\left(F\times\{1\}\right)=2\Phi$$
		这里$\Phi$是$G$在迹零空间$H=\{(x_i)\in\mathbb{R}^r\mid\sum_ix_i=0\}$中的基本网孔,于是:
		$$\psi\alpha\left(F\times I\right)=2\Phi+[0,\frac{1}{n}]\textbf{e}$$
		如果记$\Phi$被$\textbf{e}_1,\cdots,\textbf{e}_{r-1}$张成,那么这个像是被$2\textbf{e}_1,\cdots,2\textbf{e}_{r-1},\frac{1}{n}\textbf{e}$生成的平行多面体.于是它的体积是$\frac{1}{n}2^{r-1}$乘以如下行列式:
		$$\left|\begin{array}{cccc}\textbf{e}_{11}&\cdots&\textbf{e}_{r-1,1}&e_1\\\vdots&\ddots&\vdots&\vdots\\\textbf{e}_{1r}&\cdots&\textbf{e}_{r-1,r}&e_r\end{array}\right|$$
		把前$r-1$列加到最后一列中,那么最后一列除了最后一项以外都为零,最后一项是$n$,左上角的主子式是$K$的regulator.得证.
	\end{proof}
	\item 设$\mathfrak{a}$是整理想,记$\textbf{R}$的完备格$\Gamma=\mathfrak{a}$的对偶格记作$\Gamma'$,设${^*x}$是我们之前定义的${^*x}=(\overline{x_{\tau}})_{\tau}$,设$\mathfrak{o}$是$K/\mathbb{Q}$的共轭差积.那么有:
	$${^*\Gamma}'=(\mathfrak{a}\mathfrak{o})^{-1}$$
	\begin{proof}
		
		按照$\langle x,y\rangle=\mathrm{Tr}(({^*x})y)$,有:
		$${^*\Gamma}'=\{^*g\in\textbf{R}\mid\langle g,a\rangle\in\mathbb{Z},\forall a\in\mathfrak{a}\}=\{x\in\textbf{R}\mid\mathrm{Tr}(x\mathfrak{a})\subseteq\mathbb{Z}\}$$
		按照$\mathrm{Tr}(x\mathfrak{a})\subseteq\mathbb{Z}$,我们先证明$x\in K$:任取$\mathfrak{a}$的一组$\mathbb{Z}$基$a_1,\cdots,a_n$,记$x=\sum_ix_ia_i$,其中$x_i\in\mathbb{R}$,那么:
		$$\mathrm{Tr}(xa_j)=\sum_ix_i\mathrm{Tr}(a_ia_j)=n_j\in\mathbb{Z}$$
		是$\{x_i\}$的线性方程组,其中系数$\mathrm{Tr}(a_ia_j)\in\mathbb{Q}$.于是这些$x_i\in\mathbb{Q}$,进而$x\in K$.进而有:
		$${^*\Gamma}'=\{x\in K\mid\mathrm{Tr}(x\mathfrak{a})\subseteq\mathbb{Z}\}$$
		按照共轭差积的定义,就有$x\in{^*\Gamma}'$当且仅当$\mathrm{Tr}_{K/\mathbb{Q}}(xa\mathscr{O})\subseteq\mathbb{Z}$对任意$a\in\mathfrak{a}$成立,当且仅当$x\mathfrak{a}\subseteq\mathscr{o}^{-1}$,当且仅当$x\in(\mathfrak{a}\mathfrak{o})^{-1}$.
	\end{proof}
	\item 函数$f_F(\mathfrak{a},t)$满足如下变换公式:
	$$f_F(\mathfrak{a},1/t)=t^{1/2}f_{F^{-1}}\left((\mathfrak{a}\mathfrak{o})^{-1},t\right)$$
	满足(回顾一下这里$w=|\mu(K)|$是单位根群的元素个数,$R$是数域$K$的regulator,$\mathfrak{o}$是共轭差积):
	$$f_F(\mathfrak{a},t)=\frac{2^{r-1}}{w}R+O(e^{-ct^{1/n}}),t\to\infty,c>0$$
	\begin{proof}
		
		我们证明过Theta函数的变换公式:
		$$\theta_{\Gamma}(-1/z)=\frac{\sqrt{\mathrm{N}(z/i)}}{\mathrm{vol}(\Gamma)}\theta_{\Gamma'}(z)$$
		取整理想$\Gamma=\mathfrak{a}$,我们证明过它的基本网孔的体积是:
		$$\mathrm{vol}(\Gamma)=\mathfrak{R}(\mathfrak{a})|d_K|^{1/2}$$
		我们还计算过对偶格满足:
		$${^*\Gamma}'=(\mathfrak{a}\mathfrak{o})^{-1}$$
		按照$\langle({^*g})z,{^*g}\rangle=\langle gz,g\rangle$,就有:
		$$\theta_{\Gamma'}(z)=\theta_{{^*\Gamma}'}(z)$$
		另外有:
		$$d_{(\mathfrak{a}\mathfrak{o})^{-1}}=\mathfrak{R}(\mathfrak{a})^{-2}\mathfrak{R}(\mathfrak{o})^{-2}|d_K|=1/(\mathfrak{R}(\mathfrak{a})^2|d_K|)=1/d_{\mathfrak{a}}$$
		按照$x\mapsto x^{-1}$固定$\textbf{S}$上的规范测度(也即$x\mapsto-x$固定$\mathbb{R}^n$上的勒贝格测度),并且把$F$的基本网孔映为$F^{-1}$的基本网孔.这里$\ln{F^{-1}}$仍然是完备格$2\ln|\mathscr{O}^*|$的基本网孔.对$x\in\textbf{S}$有$\mathrm{N}\left(x(td_{\mathfrak{a}})^{1/n}\right)=td_{\mathfrak{a}}$.于是:
		\begin{align*}
			f_F(\mathfrak{a},1/t)=\frac{1}{w}\int_F\theta_{\mathfrak{a}}\left(ix/(td_{\mathfrak{a}})^{1/n}\right)\mathrm{d}^*x\\&=\frac{1}{w}\int_{F^{-1}}\theta_{\mathfrak{a}}\left(-1/ix(td_{\mathfrak{a}})^{1/n}\right)\mathrm{d}^*x\\&=\frac{1}{w}\frac{(td_{\mathfrak{a}})^{1/2}}{\mathrm{vol}(\mathfrak{a})}\int_{F^{-1}}\theta_{(\mathfrak{a}\mathfrak{o})^{-1}}\left(ixt^{1/n}d_{(\mathfrak{a}\mathfrak{o})^{-1/n}}\right)\mathrm{d}^*x\\&=t^{1/2}f_{F^{-1}}\left((\mathfrak{a}\mathfrak{o})^{-1},t\right)
		\end{align*}
		最后有:
		$$f_F(\mathfrak{a},t)=\frac{1}{w}\int_F\mathrm{d}^*x+\frac{1}{w}\left(\theta(\mathfrak{a},ixt^{1/n})-1\right)\mathrm{d}^*x=\frac{\mathrm{vol}(F)}{w}+r(t)$$
		这里$\theta(\mathfrak{a},ixt^{1/n})-1=\sum_{0\not=a\in\mathfrak{a}}e^{-\pi\langle ax,a\rangle {t'}^{1/n}}$,其中$t'=t/d_{\mathfrak{a}}$.按照$F$包含在紧集$\overline{F}$中,于是可设对任意$\tau$有$x_{\tau}\ge\delta>0$,进而有$\langle ax,a\rangle=\sum_{\tau}|\tau a|^2x_{\tau}\ge\delta\langle a,a\rangle$.于是:
		$$r(t)\le\frac{\mathrm{vol}(F)}{w}\left(\theta_{\mathfrak{a}}(i\delta{t'}^{1/n})-1\right)$$
		设$m=\min\{\langle a,a\rangle\mid a\in\mathfrak{a},a\not=0\}$和$M=\#\{a\in\mathfrak{a}\mid\langle a,a\rangle=m\}$,那么有:
		$$\theta_{\mathfrak{a}}(i\delta{t'}^{1/n})-1=e^{-\pi\delta m{t'}^{1/n}}\left(M+\sum_{\langle a,a\rangle>m}e^{-\pi\delta(\langle a,a\rangle-m){t'}^{1/n}}\right)=O(e^{-ct^{1/n}}),c=\pi\delta m/d_{\mathfrak{a}}^{1/n}$$
	\end{proof}
    \item 完备函数:
    $$Z(\mathscr{R},s)=Z_{\infty}(s)\zeta(\mathscr{R},s)$$
    定义在半平面$\mathrm{Re}(s)>1$上,其中:
    $$Z_{\infty}(s)=|d_K|^{s/2}\pi^{-ns/2}\Gamma_K(s/2)$$
    这个完备函数可以解析延拓到$\mathbb{C}-\{0,1\}$上,并且满足函数方程:
    $$Z(\mathscr{R},s)=Z(\mathscr{R}',1-s)$$
    其中$\mathscr{R}'$是唯一满足$\mathscr{R}\mathscr{R}'=[\mathfrak{o}]$的等价类,这里$\mathfrak{o}$是共轭差积.这个解析延拓在$s=0$和$s=1$处是一阶极点,留数分别为$-2^rR/w$和$2^rR/w$.
    \begin{proof}
    	
    	设$f(t)=f_F(\mathfrak{a},t)$和$g(t)=f_{F^{-1}}((\mathfrak{a}\mathfrak{o})^{-1},t)$,那么有:
    	$$f(1/t)=t^{1/2}g(t)$$
    	$$f(t)=a_0+O(e^{-ct^{1/n}}),g(t)=a_0+O(e^{-ct^{1/n}}),a_0=\frac{2^{r-1}R}{w}$$
    	按照Mellin原理,$L(f,s)$和$L(g,s)$可以解析延拓到$\mathbb{C}-\{0,\frac{1}{2}\}$上,并且满足:
    	$$L(f,s)=L(g,\frac{1}{2}-s)$$
    	其中$L(f,s)$在$s=0$和$1/2$处是一阶极点,留数分别为$-a_0$和$a_0$.进而有:
    	$$Z(\mathscr{R},s)=L(f,\frac{s}{2})$$
    	解析延拓到$\mathbb{C}-\{0,1\}$上,它在$s=0$和1处是一阶极点,留数分别为$-2a_0$和$2a_0$.并且满足:
    	$$Z(\mathscr{R},s)=L(f,\frac{s}{2})=L(g,\frac{1-s}{2})=Z(\mathscr{R}',1-s)$$
    \end{proof}
    \item 完备Zeta函数.定义数域$K$上的完备Zeta函数为:
    $$Z_K(s)=Z_{\infty}(s)\zeta_K(s)=\sum_{\mathscr{R}}Z(\mathscr{R},s)$$
    其中:$$Z_{\infty}(s)=|d_K|^{s/2}\pi^{-ns/2}\Gamma_K(s/2)$$
    那么$Z_K(s)$可以解析延拓到$\mathbb{C}-\{0,1\}$上,并且满足函数方程:
    $$Z_K(s)=Z_K(1-s)$$
    并且它在$s=0$和$s=1$处是一阶极点,留数分别为$-2^rhR/w$和$2^rhR/w$,其中$h$是$K$的类数.
    \item 推广.任取理想类群上的特征标$\chi:\mathrm{Cl}(K)=J/P\to S^1$,定义Zeta函数为:
    $$Z(\chi,s)=Z_{\infty}(s)\zeta(\chi,s)$$
    其中:$$\zeta(\chi,s)=\sum_{\mathfrak{a}}\frac{\chi(\mathfrak{a})}{\mathfrak{R}(\mathfrak{a})^s}$$
    这里求和跑遍$K$的整理想,记号$\chi(\mathfrak{a})$是$\mathfrak{s}$所在理想类的取值.那么有:
    $$Z(\chi,s)=\sum_{\mathscr{R}}\chi(\mathscr{R})Z(\mathscr{R},s)$$
    按照$\mathscr{R}'=\mathscr{R}^{-1}[\mathfrak{o}]$,就有:
    $$Z(\chi,s)=\chi(\mathfrak{o})Z(\overline{\chi},1-s)$$
    如果$\chi\not=1$,那么$Z(\chi,s)$在整个$\mathbb{C}$上解析.
    \item $\zeta_K(s)$的解析延拓.
    \begin{enumerate}[(1)]
    	\item Dedekind Zeta函数$\zeta_K(s)$解析延拓到$\mathbb{C}-\{1\}$上.
    	\item 在$s=1$处是一阶极点,它的留数是:
    	$$\kappa=\frac{2^{r_1}(2\pi)^{r_2}}{w|d_K|^{1/2}}hR=hR/e^g$$
    	其中$h$是$K$的类数,$g=\ln\frac{w|d_K|^{1/2}}{2^{r_1}(2\pi)^{r_2}}$是数域的类数.这个留数公式称为解析类数公式.
    	\item $\zeta_K(s)$满足如下函数方程:
    	$$\zeta_K(1-s)=A(s)\zeta_K(s)$$
    	其中:
    	$$A(s)=|d_K|^{s-1/2}\left(\cos\frac{\pi s}{2}\right)^{r_1+r_2}\left(\sin\frac{\pi s}{2}\right)^{r_2}L_{\mathbb{C}}(s)^n$$
    \end{enumerate}
    \item 设$K=\mathbb{Q}(\mu_m)$是$m$次单位根数域,那么有:
    $$\zeta_K(s)=G(s)\prod_{\chi}L(\chi,s)$$
    其中$\chi$跑遍$m$-Dirichlet特征标,并且:
    $$G(s)=\prod_{\mathfrak{p}\mid m}\frac{1}{1-\mathfrak{R}(\mathfrak{p})^{-s}}$$
    \begin{proof}
    	
    	固定素数$p$,设$p=(\mathfrak{p}_1\cdots\mathfrak{p}_r)^e$是素数$p$在$K$中的分解.设惯性次数为$f$,也即$\mathfrak{R}(\mathfrak{p}_i)=p^f$.那么$\zeta_K(s)$就包含乘积因子:
    	$$\prod_{\mathfrak{p}\mid p}(1-\mathfrak{R}(\mathfrak{p})^{-s})^{-1}=(1-p^{-fs})^{-r}$$
    	而$\prod_{\chi}L(\chi,s)$有乘积因子:
    	$$\prod_{\chi}(1-\chi(p)p^{-s})^{-1}$$
    	当$p\mid m$时这个因子就是1,下面设$p\not\mid m$,此时$p^f\equiv1(\mathrm{mod}m)$,并且$e=1$,进而有$r=\varphi(m)/f$是$p$在$G=(\mathbb{Z}/m\mathbb{Z})^*$中生成的子群$G_p$的指数.定义$\widehat{G}$是$G$上特征标构成的群,那么$\chi\mapsto\chi(p)$是同构$\widehat{G_p}\cong\mu_f$,有如下短正合列:
    	$$\xymatrix{1\ar[r]&\widehat{G/G_p}\ar[r]&\widehat{G}\ar[r]&\mu_f\ar[r]&1}$$
    	进而在$\widehat{G}$中可以找到$r$个像为$\chi(p)$的特征标,进而有(对$p\not\mid m$):
    	\begin{align*}
    		\prod_{\chi}(1-\chi(p)p^{-s})^{-1}=\prod_{\zeta\in\mu_f}(1-\zeta p^{-s})^{-r}\\&=(1-p^{-fs})^{-r}\\&=\prod_{\mathfrak{p}\mid p}(1-\mathfrak{R}(\mathfrak{p})^{-s})^{-1}
    	\end{align*}
    	让$p$跑遍全部素数,乘积得到原等式.
    \end{proof}
    \item 推论.对任意Dirichlet特征标$\chi$,总有$L(\chi,1)\not=0$.
    \begin{proof}
    	
    	对$m$-平凡Dirichlet特征标$\chi^0$,我们有:
    	$$L(\chi^0,s)=\zeta(s)\prod_{p\mid m}(1-p^{-s})$$
    	进而有:
    	$$\zeta_K(s)=G(s)\zeta(s)\prod_{p\mid m}(1-p^{-s})\prod_{\chi\not=\chi^0}L(\chi,s)$$
    	按照$\zeta_K(s)$和$\zeta(s)$都在$s=1$处有一阶极点,于是$\prod_{\chi\not=\chi^0}L(\chi,s)$不以$s=1$为零点.
    \end{proof}
    \item Dirichlet素数定理.设$(a,m)=1$,那么算术序列:
    $$a,a\pm m,a\pm 2m,a\pm 3m,\cdots$$
    包含了无穷个素数.
    \begin{proof}
    	
    	设$\chi$是$m$-Dirichlet特征标,那么对$\mathrm{Re}(s)>1$有:
    	$$\ln\mathscr{L}(\chi,s)=-\sum_p\ln(1-\chi(p)p^{-s})=\sum_p\sum_{m\ge1}\frac{\chi(p^m)}{mp^{ms}}=\sum_p\frac{\chi(p)}{p^s}+g_{\chi}(s)$$
    	其中$g_{\chi}(s)$在$\mathrm{Re}(s)>1/2$上解析.对这个等式数乘$\chi(a^{-1})$,再对全部$m$-Dirichlet特征标$\chi$求和,得到:
    	\begin{align*}
    		\sum_{\chi}\chi(a^{-1})\ln L(\chi,s)&=\sum_{\chi}\sum_p\frac{\chi(a^{-1}p)}{p^s}+g(s)\\&=\sum_{b=1}^m\sum_{\chi}\chi(a^{-1}b)\sum_{\chi}\chi(a^{-1}b)\sum_{p\equiv b(\mathrm{mod}m)}\frac{1}{p^s}+g(s)\\&=\sum_{p\equiv a(\mathrm{mod}m)}\frac{\varphi(m)}{p^s}+g(s)
    	\end{align*}
    	这里最后一个等号是因为:
    	$$\sum_{\chi}\chi(a^{-1}b)=\left\{\begin{array}{cc}0&a\not=b\\\varphi(m)=\#(\mathbb{Z}/m\mathbb{Z})^*&a=b\end{array}\right.$$
    	当$s$从实轴正方向趋近1时,对$\chi\not=\chi^0$,那么$\ln L(\chi,s)$是有界的,而$\ln L(\chi^0,s)=\sum_{p\mid m}\ln(1-p^{-s})+\ln\zeta(s)$取于无穷,因为$\zeta(s)$在$s=1$处是极点.于是上述等式的左侧取于无穷,右侧的$g(s)$在$s=1$处是解析的,于是:
    	$$\lim\limits_{s\to1}\sum_{p\equiv a(\mathrm{mod}m)}\frac{\varphi(m)}{p^s}=\infty$$
    	这迫使$\{a,a\pm m,a\pm 2m,a\pm 3m,\cdots\}$中有无穷个素数.
    \end{proof}
\end{enumerate}
\subsection{Hecke特征标}
























\newpage
\section{Riemann-Roch定理}
\subsection{???}

设$\mathscr{O}_K$是代数整数环.
\begin{itemize}
	\item 它的除子指的是如下形式和,其中$n_v\in\mathbb{Z}$和$\lambda_{\sigma}\in\mathbb{R}$,并且对几乎所有的指标$v$有$n_v=0$.这个除子的次数记作$\sum n_v\ln N(v)+\sum\lambda_{\sigma}$.定义如下除子的全体除子构成的群记作$\mathrm{Div}(\mathscr{O}_K)$.
	$$\sum_{v\in\mathrm{Spec.max}(\mathscr{O}_K)}n_v[v]+\sum_{\sigma\mid\infty}\lambda_{\sigma}[\sigma]$$
	\item 设$f\in K^*$,定义$f$的主除子如下.其中$\varepsilon_{\sigma}$是这样取的,如果$\sigma$是实素位,就取$\varepsilon_{\sigma}=1$,如果$\sigma$是复素位,就取$\varepsilon_{\sigma}=2$.全体主除子构成了$\mathrm{Div}(\mathscr{O}_K)$的子群,记作$\mathrm{Pr}(\mathscr{O}_K)$.
	$$(f)=\sum_v\mathrm{ord}_v(f)[v]-\sum_{\sigma\mid\infty}\varepsilon_{\sigma}\ln|f|_{\sigma}[\sigma]$$
	\item 按照乘积公式,主除子的次数是零,于是次数同态诱导了满同态$\mathrm{Div}(\mathscr{O}_K)/\mathrm{Pr}(\mathscr{O}_K)\to\mathbb{R}$.
\end{itemize}











首先我们补充一些事情,设$K$是数域,设$p$是一个有限素位,那么我们定义过剩余域$\kappa(p)$是$K$的赋值环商去极大理想,我们定义过该素位上标准赋值的指数表示$|a|_p=\mathscr{R}(p)^{-v_p(a)}$,其中$\mathscr{R}(p)$是绝对范数$\mathrm{N}(p)$.对于无限素位$p$(通常记作$p\mid\infty$),类似定义如下内容:
\begin{itemize}
	\item 定义$K$关于$p$的剩余域就是完备化$K_p$本身.定义标准赋值的加法形式为$v_p(a)=-\ln|\tau(a)|$,其中$\tau$是无穷素位$p$对应的嵌入$K\to\mathbb{C}$.
	\item 此时$p$的惯性次数理应定义为$f_p=[K_p:\mathbb{R}]$,此时取$\mathscr{R}(p)=e^{f_p}$,那么有$|a|_p=\mathscr{R}(p)^{-v_p(a)}$,这吻合于我们定义的$|a|_p=|\tau(a)|$和$|a|_p=|\tau(a)|^2$,前者对应实嵌入,后者对应复嵌入.这个约定使得我们称$e$为无穷素数.
	\item 另外对于无穷素位我们定义赋值环和赋值环的单位群$U_p$为:如果$v$是实素位,定义$\mathscr{O}_v=\mathbb{R}_+$和$U_v=\mathbb{R}_+^*$;如果$v$是复素位定义$\mathscr{O}_v=\mathbb{C}$和$U_v=\mathbb{C}^*$.
\end{itemize}

充足理想.设$K$是数域,群$J(\overline{\mathscr{O}})=J_K\times\prod_{p\mid\infty}\mathbb{R}_+^*$中的元素称为充足理想,它相当于添加无穷素位信息的分式理想.
\begin{enumerate}
	\item 关于记号的一点补充.当我们写下记号$\prod_{p\mid\infty}p^{v_p}$时总是指$\prod_{p\mid\infty}\mathbb{R}_+^*$中的元,而不是$\mathbb{R}$中实际数字的乘积,另外这里$p$分量下的$p^{v_p}$的底数视为我们约定的无穷素数$e$.
	\item 一个充足理想可以表示为$I=\prod_{p\not|\infty}p^{v_p}\times\prod_{p\mid\infty}p^{v_p}$,也即$\prod_{p}p^{v_p}$,其中$p$跑遍$K$上的素位,并且当$p$是有限素位时$v_p$就取整数,当$p$是无穷素位时$v_p$取实数.另外我们记$I_f=\prod_{p\not|\infty}p^{v_p}\times\prod_{p\mid\infty}1$和$I_{\infty}=(1)\times\prod_{p\mid\infty}p^{v_p}$.那么充足理想总具有分解$I=I_f\cdot I_{\infty}$,这里乘法是分量相乘.
	\item 对$a\in K^*$,定义它的主充足理想为$\prod_{p}p^{v_p(a)}=(a)\times\prod_{p\mid\infty}$,这构成了充足理想群$J(\overline{\mathscr{O}})$的子群,记作$P(\overline{\mathscr{O}})$.定义充足理想类群或者称充足Picard群为商群$J(\overline{\mathscr{O}})/P(\overline{\mathscr{O}})$,它记作$\mathrm{Pic}(\overline{\mathscr{O}})$.
	\item 绝对范数.我们之前的绝对范数只定义在分式理想上,对于充足理想$I=\prod_pp^{v_p}$可补充定义$\mathscr{R}(I)=\prod_p\mathscr{R}(p)^{v_p}$.于是对于主充足理想按照乘积公式有$\mathscr{R}([a])=\prod_p|a|_p^{-1}=1$,于是绝对范数诱导了$\mathrm{Pic}(\overline{\mathscr{O}})\to\mathbb{R}_+^*$的同态.
\end{enumerate}

设$K\subseteq L$是数域的扩张,它们的充足理想群之间定义两个同态:
\begin{itemize}
	\item $i_{L/K}:J(\overline{\mathscr{O}_K})\to J(\overline{\mathscr{O}_L})$为$\prod_pp^{v_p}\mapsto\prod_p\prod_{q\mid p}q^{e_{q/p}v_p}$.
	\item $\mathrm{N}_{L/K}:J(\overline{\mathscr{O}_L})\to J(\overline{\mathscr{O}_K})$为$\prod_qq^{v_q}\mapsto\prod_p\prod_{q\mid p}p^{f_{q/p}v_q}$.
\end{itemize}
\begin{enumerate}
	\item 给定数域的扩张链$K\subseteq L\subseteq M$,那么有传递公式:
	$$\mathrm{N}_{M/K}=\mathrm{N}_{L/K}\circ\mathrm{N}_{M/L},i_{M/K}=i_{M/L}\circ i_{L/K}$$
	\item 对$I\in J(\overline{\mathscr{O}_K})$有$\mathrm{N}_{L/K}(i_{L/K}(I))=I^{[L:K]}$.
	\item 对$J\in J(\overline{\mathscr{O}_L})$有$\mathscr{R}(\mathrm{N}_{L/K}(J))=\mathscr{R}(J)$.
	\item 设$K\subseteq L$是Galois扩张,记Galois群为$G$,对$\mathscr{O}_L$的每个素理想$q$有$\mathrm{N}_{L/K}(q)\mathscr{O}_L=\prod_{\sigma\in G}\sigma(q)$.
	\item 对$K$的每个充足主理想$[a]$有$i_{L/K}([a])=[a]$.特别的,$i_{L/K}$诱导了Picard群之间的同态.
	\item 对$L$的每个充足主理想$[a]$有$\mathrm{N}_{L/K}([a])=[\mathrm{N}_{L/K}(a)]$.特别的,$\mathrm{N}_{L/K}$诱导了Picard群之间的同态.
	\item 回顾对充足理想$J$有$J_f$表示它有限部分对应的分式理想,我们有$\mathrm{N}_{L/K}(J_f)=\mathrm{N}_{L/K}(J)_f$是由范数$\mathrm{N}_{L/K}(a),a\in J_f$生成的$K$的分式理想.
	\item 对数域的有限扩张$K\subseteq L$,有如下图表交换:
	$$\xymatrix{\mathrm{Pic}(\overline{\mathscr{O}}_L)\ar[rr]^{\mathscr{R}}\ar@<0.5ex>[d]^{\mathrm{N}_{L/K}}&&\mathbb{R}_+^*\ar@<0.5ex>[d]^{\mathrm{id}}\\\mathrm{Pic}(\overline{\mathscr{O}}_K)\ar[rr]^{\mathscr{R}}\ar@<0.5ex>[u]^{i_{L/K}}&&\mathbb{R}_+^*\ar@<0.5ex>[u]^{[L:K]}}$$
\end{enumerate}

分式理想和充足分式理想可以等价的描述为除子.
\begin{itemize}
	\item 给定数域$K$,除子群$\mathrm{Div}(\mathscr{O}_K)$定义为全部形式和$D=\sum_{p\not|\infty}v_pp$,其中$v_p\in\mathbb{Z}$,并且对几乎所有的有限素位$p$有$v_p=0$.主除子群定义为它的由全体$\mathrm{Div}(a)=\sum_{p\not|\infty}v_p(a)p,a\in K^*$构成的子群$P(\mathscr{O}_K)$.除子类群定义为它们的商$\mathrm{CH}^1(\mathscr{O}_K)=\mathrm{Div}(\mathscr{O}_K)/\mathrm{P}(\mathscr{O}_K)$.
	\item 类似的定义充足除子群是全体形式和$D=\sum_pv_pp$,其中$p$跑遍全部素位,对于有限素位$p$约定$v_p\in\mathbb{Z}$,对于无限素位$p$约定$v_p\in\mathbb{R}$,另外对几乎所有的$p$有$v_p=0$.于是充足除子群和除子群之间满足$\mathrm{Div}(\overline{\mathscr{O}_K})\cong\mathrm{Div}(\mathscr{O}_K)\times\oplus_{p\mid\infty}\mathbb{R}p$.赋予拓扑为,乘号左侧上赋予离散拓扑,右侧赋予$\mathbb{R}$上标准的乘积拓扑.
	\item 定义典范映射$\mathrm{div}:K^*\to\mathrm{Div}(\overline{\mathscr{O}_K})$,$\mathrm{div}(a)=\sum_pv_p(a)p$的像为主充足理想,这构成一个子群.它们的商称为充足除子类群,记作$\mathrm{CH}^1(\overline{\mathscr{O}_K})$.
\end{itemize}
\begin{enumerate}
	\item 除子映射$\mathrm{div}:K^*\to\mathrm{Div}(\overline{\mathscr{O}_K})$和映射$\mathrm{Div}(\overline{\mathscr{O}_K})\to\prod_{p\mid\infty}\mathbb{R}$,$\sum_pv_pp\mapsto(v_pf_p)_{p\mid\infty}$的复合在相差一个符号的意义下恰好就是Minkowski理论中的映射$\lambda:K^*\to\prod_{p\mid\infty}$,$\lambda(a)=(\cdots,\ln|a|_p,\cdots)$.于是它把单位群$\mathscr{O}^*$映射为迹零空间$H=\{(x_p)\in\prod_{p\mid\infty}\mathbb{R}\mid\sum_{p\mid\infty}x_p=0\}$的完备格.
	\item 除子映射$\mathrm{div}:K^*\to\mathrm{Div}(\overline{\mathscr{O}_K})$的核是$K$中的单位根群$\mu(K)$,它的像,即充足主除子群是充足除子群的离散子群.
	\begin{proof}
		
		
	\end{proof}
\end{enumerate}

\newpage
\section{Galois上同调}
\subsubsection{基本内容}

Galois上同调是指Galois群的上同调.
\begin{enumerate}
	\item 设$K/k$是有限Galois扩张,记Galois群为$G$.那么$G$自然的作用在加法群$K$和乘法群$K^*$上.
	\begin{enumerate}[(1)]
		\item 对任意整数$n$有$\widehat{\mathrm{H}}^n(G,K)=0$.
		\begin{proof}
			
			按照正规基定理,$G$模$K$是余诱导模,于是它的Tate上同调均平凡.
		\end{proof}
		\item $\mathrm{H}^1(G,K^*)=0$.
		\begin{proof}
			
			我们要证明的是$Z^1(G,K^*)\subset B^1(G,K^*)$.任取$c:G\to K^*\in Z^1(G,K^*)$,任取$g\in G$,把$c(g)$记作$c_g$.在乘法记号下,$c$在$Z^1(G,K^*)$中等价于要求$c_{gh}=g(c_h)c_g$对任意$g,h\in G$成立.
			
			现在任取$e\in K^*$,构造$b=\sum_{g\in G}c_gg(e)$,按照Dedekind引理,存在$e$使得$b\not=0$,对于这样的$e$就有:
			\begin{align*}
				h(b)&=\sum_{g\in G}h(c_g)hg(e)=\sum_{g\in G}c_{hg}c_h^{-1}hg(e)=c_h^{-1}\sum_{g\in G}c_{hg}hg(e)\\&=c_h^{-1}b
			\end{align*}
			
			于是$c_h=bh(b)^{-1}$,此即$c\in B^1(G,K^*)$,得证.
		\end{proof}
	    \item Hilbert90定理.如果$G$是循环群,生成元记作$\sigma$,那么$u\in K^*$满足范数$N(u)=1$当且仅当存在$v\in K^*$使得$u=\sigma(v)v^{-1}$.
	    \begin{proof}
	    	
	    	对于有限循环群我们证明过$H^1(G,K^*)=\ker N/\mathrm{Im}D$.上一条解释了$H^1(G,K^*)=0$,于是$\ker N=\mathrm{Im}D$.另外注意这里数乘$N$恰好把$K^*$中的元映射为它的范数,于是$N(u)=1$当且仅当$u\in\ker N=\mathrm{Im}D$,于是存在$v\in K^*$使得$D(v)=u$,此即$\sigma(v)/v=u$.
	    \end{proof}
        \item $G$自然的作用在$\mathrm{GL}(n,K)$上,我们有非交换群上同调$\mathrm{H}^1(G,\mathrm{GL}(n,K))=1$.
        \begin{proof}
        	
        	依旧任取$\left(c:G\to\mathrm{GL}(n,K)\right)\in Z^1(G,\mathrm{GL}(n,K))$.对任意矩阵$M\in\mathrm{GL}(n,K)$,取$b=\sum_{g\in G}c_gg(M)$.如果$K$是无限域,此时有自同构的代数无关性,于是存在$M\in\mathrm{GL}(n,K)$使得$b\not=0$,此时依旧可以验证$h(b)=c_h^{-1}b,\forall h\in G$成立.但是如果$K$是有限域这个代数无关性不成立.
        	
        	\qquad
        	
        	设$K$是有限域,取向量$x\in K^n$,取$b(x)=\sum_{g\in G}c_g(g(x))$.我们断言$b(K^n)$生成了整个$K^n$:如果$K^n$上的线性变换$u$在$b(K^n)$上恒为零,任取$y\in K$和$x\in K^n$,就有$0=u(b(yx))=\sum_gc_gu(g(y)g(x))=\sum_gg(y)u(c_g(g(x)))$.按照Dedekind引理,就有每个$u(c_g(g(x)))=0$,但是$c_g$是可逆矩阵,就有$u=0$完成断言的证明.
        	
        	\qquad
        	
        	最后设$x_1,\cdots,x_n\in K^n$使得$y_i=b(x_i)$构成$K^n$的一组基.取矩阵$M$满足$Me_i=x_i$,其中$e_i$是$K^n$上的标准基.那么有$b(e_i)=\sum_{g\in G}c_gg(M)e_i=\sum_{g\in G}c_gg(Me_i)=\sum_{g\in G}c_gg(x_i)=y_i$.于是此时$b$是可逆矩阵.
        \end{proof}
        \item $\mathrm{H}^1(G,\mathrm{SL}(n,K))=1$.
        \begin{proof}
        	
        	考虑如下短正合列:
        	$$\xymatrix{1\ar[r]&\mathrm{SL}(n,K)\ar[r]&\mathrm{GL}(n,K)\ar[r]^{\det}&K^*\ar[r]&1}$$
        	它诱导了如下长正合列:
        	$$\xymatrix{\mathrm{H}^0(G,\mathrm{GL}(n,K))\ar[r]&\mathrm{H}^0(G,K^*)\ar[r]&\mathrm{H}^1(G,\mathrm{SL}(n,K))\ar[r]&1}$$
        	但是第一个同态就是$\mathrm{GL}(n,k)\to k^*$,这是满射,于是$\mathrm{H}^1(G,\mathrm{SL}(n,K))=1$.
        \end{proof}
    \end{enumerate}
    \item 设$K/k$是未必有限的Galois扩张.设$A$是离散$G$模,那么我们有:
    $$\mathrm{H}^q(G,A)=\varinjlim\mathrm{H}^q(G/H,A^H)$$
    其中$H$跑遍$G$的开正规子群.
    \begin{enumerate}[(1)]
    	\item 通过对有限Galois扩张取正向极限,得到:
    	$$\widehat{\mathrm{H}}^n(G,K)=0,\forall n\in\mathbb{Z}$$
    	$$\mathrm{H}^1(G,K^*)=0$$
    	\item Artin-Schreier理论:设$k$是特征$p>0$的域,设$K$是它的可分代数闭包,设$G=\mathrm{Gal}(K/k)$,定义$\mathscr{P}:K\to K$为$x\mapsto x^p-x$.这是一个$G$满同态,它的核是$\mathbb{Z}/p\mathbb{Z}$(带平凡$G$作用).于是我们有如下$G$模的短正合列:
    	$$\xymatrix{0\ar[r]&\mathbb{Z}/p\mathbb{Z}\ar[r]&K\ar[r]^{\mathscr{P}}&K\ar[r]&0}$$
    	取长正合列,得到:
    	$$\xymatrix{\mathrm{H}^0(G,K)\ar[r]^{\mathscr{P}}&\mathrm{H}^0(G,K)\ar[r]&\mathrm{H}^1(G,\mathbb{Z}/p\mathbb{Z})\ar[r]&\mathrm{H}^1(G,K)\ar[r]&0}$$
    	于是$\mathrm{H}^1(G,\mathbb{Z}/p\mathbb{Z})=\mathrm{Hom}_{\mathrm{cont}}(G,\mathbb{Z}/p\mathbb{Z})=k/\mathscr{P}(k)$.这个同构把$a\in k$对应为连续同态$\varphi_a:G\to\mathbb{Z}/p\mathbb{Z}$,$g\mapsto g(x)-x$,其中$x$是$x^p-x=a$的根.
    	\item Kummer理论:设整数$n$和$\mathrm{char}k$互素,设$k^*$包含$n$次单位根群$\mu_n$.仍然记$K$是$k$的可分代数闭包.取$u:K^*\to K^*$为$x\mapsto x^n$,我们有离散$G$模的短正合列(其中$\mu_n$依旧是赋予平凡$G$作用):
    	$$\xymatrix{0\ar[r]&\mu_n\ar[r]&K^*\ar[r]^u&K^*\ar[r]&0}$$
    	取长正合列,得到典范同构$k^*/{k^*}^n\cong\mathrm{H}^1(G,\mu_n)=\mathrm{Hom}_{\mathrm{cont}}(G,\mu_n)$.它把$a\in k^*$映为连续同态$\varphi_a:s\mapsto s(x)x^{-1}$,其中$x$是$x^n=a$的一个根.
    \end{enumerate}
\end{enumerate}
\subsubsection{张量和一阶上同调}

设$k$是域.
\begin{enumerate}
	\item 张量.设$V$是$k$上线性空间,一个$(p,q)$型张量指的是$\left(V^{\otimes q}\right)\otimes\left({V^*}^{\otimes p}\right)$中的元.两个张量$(V,x)$和$(V',x')$称为$k$同构的,如果存在$k$线性同构$f:V\to V'$,使得延拓到张量代数上满足$f(x)=x'$.
	\item 设$K/k$是有限Galois扩张,Galois群记作$G$,记$V_K=V\otimes_kK$.那么$V$上的张量$x$就可以基变换为$V_K$上的张量$x_K=x\otimes1_K$,没有歧义的前提下仍然记作$x$.称两个张量$(V,x)$和$(V',x')$是$K$同构的,如果$(V_K,x_K)$和$(V'_K,x_K')$是$K$同构的.记$S$表示全体$K$同构于$(V,x)$的张量构成的类,它在$k$同构下的全体等价类构成的集合记作$\mathrm{E}_{(V,x)}(K/k)$.我们的目标是把这个集合描述为一阶上同调.
	\item 设$(V_K,x_K)$的全部$K$自同构构成的群为$A_K$.那么$G$作用在$A_K$上:首先$G$作用在$V_K$上,也即$g(x\otimes\lambda)=x\otimes g(\lambda)$.进而任取$(V_K,x_K)$的自同构$f$,定义$g(f)=g\circ f\circ g^{-1}$即可.我们断言有典范双射:
	$$\theta:\mathrm{E}_{(V,x)}(K/k)\cong\mathrm{H}^1(G,A_K)$$
	\begin{proof}
		
		任取$(V',x')\in\mathrm{E}_{(V,x)}(K/k)$,设$f:V_K\to {V'}_K$是$K$同构,使得$f(x_K)=x'_K$.取$p:G\to A_K$为$p_g=f^{-1}\circ g(f)=f^{-1}\circ g\circ f\circ g^{-1}$.那么$p$是1-余圈.假设$f_1:V_K\to {V'}_K$是另一个$K$同构,满足$f_1(x_K)=x'_K$.那么只要取$a=f_1^{-1}\circ f\in A_K$,就有$p(g)=a^{-1}p_1(g)g(a)$.换句话讲选取不同的$f$诱导的1-余圈相差一个1-余边界,所以$\theta(V',x')=p$是定义良性的.
		
		\qquad
		
		$\theta$是单射:设$(V_1',x_1')$和$(V_2',x_2')$对应了相同的1-余圈所在的等价类,设它们对应的$K$自同构为$f_1:V_K\to V_{1,K}',f_2:V_K\to V_{1,K}'$.在适当给$f_1$复合上某个$a\in A_K$,我们可以不妨设这两个张量干脆对应了相同的1-余圈$p$.此时有$f_1^{-1}\circ g(f_1)=f_2^{-1}\circ g(f_2)$,进而有$f=f_2f_1^{-1}$满足$g(f)=f,\forall g\in G$,于是$f$是$(V_1',x_1')$和$(V_2',x_2')$的$k$同构,这说明$\theta$是单射.
		
		\qquad
		
		$\theta$是满射:任取1-余圈$p:G\to A_K$,按照$A_K\subseteq\mathrm{GL}(V_K)$以及$\mathrm{H}^1(G,\mathrm{GL}(n,K))=0$,就存在$V_K$上的自同构$f$使得$p_s=f^{-1}\circ g(f),\forall g\in G$.把$f$延拓到$V_K$上的张量代数,并记$x_K'=f(x_K)$.按照$g(x'_K)=g(f)(g(x_K))=g(f)(x_K)=f\circ p_g(x_K)=f(x_K)=x'_K$.于是$x'_K$形如$x'\otimes1_K$,于是$(V,x')$是$\mathrm{E}_{(V,x)}(K/k)$中的元,并且经$\theta$映为$p$.
	\end{proof}
    \item 推论.设张量$x$是非退化二次型,此时$\mathrm{E}_{(V,x)}(K/k)$是全体$K$同构于$x$的二次型的等价类.$A_K$是域$K$上二次型$x$的正交群$\textbf{O}_K(x)=\{y\in\mathrm{GL}(n,K)\mid y^txy=x\}$.于是有:$\mathrm{H}^1(G,\textbf{O}_K(x))$和$K$同构于$x$的$k$二次型的等价类一一对应.
    \item 推论.设张量$x$是非退化交错形式,此时$A_K=\textbf{Sp}(n,K)$是辛群,我们知道具有相同秩的非退化交错二次型总是等价的,于是得到$\mathrm{H}^1(G,\textbf{Sp}(n,K))=1$.
\end{enumerate}
\subsubsection{Brauer群}

\begin{enumerate}
	\item 回顾中心单$k$代数.
	\begin{enumerate}[(1)]
		\item 定义.设$k$是域,设$A$是有限维$k$代数,如下命题互相等价,此时称$A$是中心单$k$代数.
		\begin{enumerate}[(a)]
			\item $A$是中心单$k$代数.此即它是单环并且中心是$k$.
			\item 设$\overline{k}$是$k$的代数闭包,那么$A\otimes_k\overline{k}$同构于某个矩阵代数$\mathrm{M}(n,\overline{k})$.
			\item 存在有限Galois扩张$K/k$,使得$A\otimes_kK$同构于某个矩阵代数$\mathrm{M}(n,K)$.
			\item $A$同构于某个$\mathrm{M}(n,D)=D\otimes_k\mathrm{M}(n,k)$,其中$D$是某个中心为$k$的可除代数.
		\end{enumerate}
	    \item 等价.两个中心单$k$代数$A_1,A_2$称为等价的,如果它们表示为$\mathrm{M}(n,D_i)$时,它们的这个可除代数$D_i$是$k$同构的.这也等价于讲存在正整数$m,n$满足有$k$同构$A_1\otimes_k\mathrm{M}(n,k)\cong A_2\otimes_k\mathrm{M}(m,k)$.此时$k$作为自身代数所在的等价类就是$\{\mathrm{M}(n,k)\mid n\ge1\}$.
	    \item Brauer群.固定$k$时全体中心单$k$代数的等价类在张量积下构成阿贝尔群,称为$k$的Brauer群,记作$\mathrm{Br}(k)$.其中零元是$k$本身所在的等价类,一个中心单$k$代数$A$的逆元就是它的反代数$A^{\mathrm{op}}$,因为我们有$k$代数同构$A\otimes_kA^{\mathrm{op}}\to\mathrm{End}_k(A)$,把$a\otimes b$映为自同态$\varphi_{a,b}:x\mapsto axb$.这是单射因为$A\otimes_kA^{\mathrm{op}}$是单环,进而由于维数相同得到它是线性同构.
	    \item 函子性.如果有域扩张$k\subseteq K$,那么有典范群同态$A_k\to A_K$为把$A$所在的等价类映为$A\otimes_kK$所在的等价类.它的核记作$\mathrm{Br}(K/k)$.那么有$\mathrm{Br}(k)=\cup_L\mathrm{Br}(L/k)$,其中$L$跑遍$k$的某个固定代数闭包$\overline{k}$中的有限Galois扩张.
	\end{enumerate}
	\item 设$K/k$是未必有限的Galois扩张,Galois群记作$G$,记$\mathrm{H}^q(K/k)=\mathrm{H}^q(G,K^*)$.记$\mathrm{H}^q(k)=\mathrm{H}^q(\overline{k}/k)$.设$L/k$是有限Galois扩张,那么有:
	$$\mathrm{H}^2(L/k)=\ker(\mathrm{H}^2(k)\to\mathrm{H}^2(L))$$
	特别的,按照上同调和极限交换,就有$\mathrm{H}^2(k)=\varinjlim_L\mathrm{H}^2(L/k)=\cup_L\mathrm{H}^2(L/k)$,其中$L$跑遍$k$在$\overline{k}$中的有限Galois扩张.
	\begin{proof}
		
		按照$\mathrm{H}^1(G_L,\overline{k}^*)=0$,我们有inf-res正合列:
		$$\xymatrix{0\ar[r]&\mathrm{H}^2(\mathrm{Gal}(L/k),L^*)\ar[r]^{\quad\mathrm{inf}}&\mathrm{H}^2(G_k,\overline{k}^*)\ar[r]^{\mathrm{res}\qquad}&\mathrm{H}^2(G_L,\overline{k}^*)^{\mathrm{Gal}(L/k)}\ar@{=}[r]&\mathrm{H}^2(G_L,\overline{k}^*)}$$
	\end{proof}
	\item 对任意域$k$有自然同构$\mathrm{Br}(k)\cong\mathrm{H}^2(k)$.
	\begin{enumerate}[(1)]
		\item 这件事归结为证明对任意有限Galois扩张$K/k$,有函子性的同构$\mathrm{Br}(K/k)\cong\mathrm{H}^2(K/k)$.这里函子性就是指对任意$k$嵌入$K\subseteq K'$有交换图表:
		$$\xymatrix{\mathrm{Br}(K/k)\ar[rr]\ar[d]&&\mathrm{H}^2(K/k)\ar[d]\\\mathrm{Br}(K'/k)\ar[rr]&&\mathrm{H}^2(K'/k)}$$
		\item 记$\mathrm{Br}(n,K/k)$为全体满足$A_K=A\otimes_kK$同构于某个$\mathrm{M}(n,K)$的中心单$k$代数的等价类构成的集合.那么有$\mathrm{Br}(K/k)=\cup_n\mathrm{Br}(n,K/k)$.我们断言有典范同构:
		$$\theta:\mathrm{Br}(n,K/k)\cong\mathrm{H}^1(G,\mathrm{PGL}(n,K))$$
		\begin{proof}
			
			$\mathrm{Br}(n,K/k)$中的元对应于$K$上的张量$(V,x)$,其中$V$是$n^2$维线性空间,$x$是一个$(1,2)$型张量,并且$(V,x)$是$K$同构于$\mathrm{M}(n,K)$,其中$x$对应于矩阵环$\mathrm{M}(n,K)$上的乘法,而$\mathrm{M}(n,K)$的自同构群是$\mathrm{M}(n,K)$本身,从而$\mathrm{M}(n,K)$上的$K$自同构群就是$\mathrm{M}(n,K)/K^*=\mathrm{PGL}(n,K)$.于是按照张量等价类和一阶上同调的描述,就有该同构.
		\end{proof}
	    \item 考虑短正合列$1\to K^*\to\mathrm{GL}(n,K)\to\mathrm{PGL}(n,K)\to1$,它诱导了上同调群的边界同态:
	    $$\Delta_n:\mathrm{H}^1(G,\mathrm{PGL}(n,K))\to\mathrm{H}^2(G,K^*)$$
	    复合上之前的$\theta$得到同态:
	    $$\delta_n:\mathrm{Br}(n,K/k)\to\mathrm{H}^2(G,K^*)=\mathrm{H}^2(K/k)$$
	    如果$A\in\mathrm{Br}(n,K/k)$和$B\in\mathrm{Br}(m,K/k)$,那么有:
	    $$\delta_{mn}(A\otimes B)=\delta_n(A)+\delta_m(B)$$
	    进而定义了群同态:
	    $$\delta:\mathrm{Br}(K/k)\to\mathrm{H}^2(K/k)$$
	    我们断言这是函子性的群同构.
	    \begin{proof}
	    	\begin{itemize}
	    		\item 单射:我们有非阿贝尔的群上同调长正合列:
	    		$$\xymatrix{1\ar@{=}[r]&\mathrm{H}^1(G,\mathrm{GL}(n,K))\ar[r]&\mathrm{H}^1(G,\mathrm{PGL}(n,K))\ar[r]^{\Delta_n}&\mathrm{H}^2(G,K^*)}$$
	    		于是$\delta_n(A)=0$等价于$\theta(A)=0$,等价于$A$是$k$上的矩阵代数.
	    		\item 满射:我们断言如果$[K:k]=n$,那么就有$\delta_n:\mathrm{Br}(n,K/k)\to\mathrm{H}^2(K/k)$是满射.按照$\delta=\Delta_n\circ\theta$,并且$\theta$是双射,归结为证明$\Delta_n$是满射.换句话讲,对任意取值在$K^*$的2-余圈$a_{s,t}$,都可以表示为$a_{s,t}=p_ss(p_t)p_{st}^{-1}$,其中$p_s\in\mathrm{GL}(n,K)$.为此取$V$是域$K$上的线性空间,以$\{e_s\mid s\in G\}$为一组基.取$p_s\in\mathrm{Hom}_K(V,V)$为把$e_t$映为$a_{s,t}e_{st}$,于是有:
	    		\begin{align*}
	    			p_ss(p_t)(p_t)(e_u)&=a_{s,tu}s(a_{t,u})e_{stu}\\&=a_{s,t}a_{st,u}e_{stu}\\&=a_{s,t}p_{st}(e_u)
	    		\end{align*}
	    	\end{itemize}
	    \end{proof}
	\end{enumerate}
    \item 设$k$是域,如下命题互相等价:
    \begin{enumerate}[(a)]
    	\item $\mathrm{Br}(k)=0$.
    	\item 对任意有限可分扩张$K/k$和任意有限Galois扩张$L/K$,都有$\mathrm{Gal}(L/K)$模$L^*$是上同调平凡模.
    	\item 对任意有限可分扩张$K/k$和任意有限Galois扩张$L/K$,都有范数映射$\mathrm{N}_{L/K}:L^*\to K^*$是满射(此即$\widehat{\mathrm{H}}^0(\mathrm{Gal}(L/K),L^*)=0$).
    \end{enumerate}
    \begin{proof}
    	
    	如果(b)成立,那么明显有(a)和(c)成立.如果(a)或者(c)成立,对任意子群$H\le\mathrm{Gal}(L/K)$,结合Hilbert90定理有$\widetilde{\mathrm{H}}^1(H,L^*)=0$,(a)可以推出$\widetilde{\mathrm{H}}^2(H,L^*)=0$.于是此时(b)成立就是我们之前给出的有限群$G$关于模$A$的上同调如果对每个素数$p$都有$G_p$模$A$的某两个相邻的上同调群平凡,则$A$是上同调平凡$G$模.
    \end{proof}
    \item 例子.
    \begin{enumerate}
    	\item 下文会证明$C_1$域的Brauer群平凡,于是特别的有限域,代数闭域上的1维超越扩张域,剩余域是完全域的完备离散赋值域的极大非分歧扩张的Brauer群都平凡.
    	\item $\mathbb{Q}$的包含全部单位根的代数扩张的Brauer群平凡.Artin猜测这样的域都是$C_1$域.
    	\item 最简单的Brauer群非平凡的例子是$\mathrm{Br}(\mathbb{R})=\mathbb{Z}/2\mathbb{Z}$.
    	\item 剩余域是有限余的完备离散赋值域的Brauer群同构于$\mathbb{Q}/\mathbb{Z}$.
    	\item 设$K$是代数数域,它的全部完备化记作$\{K_i\}$,对那些非阿基米德赋值,$\mathrm{Br}(K_i)=\mathbb{Q}/\mathbb{Z}$;对实无穷素位$\mathrm{Br}(K_i)=\mathbb{Z}/2\mathbb{Z}$;对复无穷素位$\mathrm{Br}(K_i)=0$.可以证明$\mathrm{Br}(K)$可以嵌入到$\oplus_i\mathrm{Br}(K_i)$中,并且有如下短正合列,其中$\sigma$为$(x_i)\mapsto\sum x_i$:
    	$$\xymatrix{0\ar[r]&\mathrm{Br}(K)\ar[r]&\oplus_i\mathrm{Br}(K_i)\ar[r]^{\sigma}&\mathbb{Q}/\mathbb{Z}\ar[r]&0}$$
    	\item $\mathbb{F}_q(t)$的Brauer群非平凡.
    \end{enumerate}
\end{enumerate}
\subsubsection{域的上同调维数}

设$k$是域,它的$p$-上同调维数和上同调维数定义为它的绝对Galois群的$p$-上同调维数和上同调维数,分别记作$\mathrm{cd}_p(k)$和$\mathrm{cd}(k)$.
\begin{enumerate}
	\item 设$k$是域,设$p$是和$\mathrm{char}(k)$不同的素数,那么如下命题互相等价:
	\begin{enumerate}[(a)]
		\item $\mathrm{cd}_p(k)\le1$.
		\item 对任意可分代数扩张$K/k$有$\mathrm{Br}(K)\{p\}=0$.
		\item 对任意有限可分代数扩张$K/k$有$\mathrm{Br}(K)[p]=0$(这是$p$扭部分).
		\item 对任意可分代数扩张$K/k$和任意Galois群为$\mathbb{Z}/p\mathbb{Z}$的Galois扩张$L/K$,有范数映射$\mathrm{N}_{L/k}:L^*\to K^*$是满射.
	\end{enumerate}
    \begin{proof}
    	
    	(a)推(b):取$k$的一个可分代数闭包$k^{\mathrm{sep}}$,那么$G_K=\mathrm{Gal}(k^{\mathrm{sep}}/K)$是$G_k$的闭子群,于是有$\mathrm{cd}_p(K)\le\mathrm{cd}_p(k)\le1$.于是有$\mathrm{Br}(K)[p^i]=\mathrm{H}^2(G_K,\mu_{p^i})=0$.(b)推(c)平凡.
    	
    	\qquad
    	
    	(c)推(d):按照循环群的上同调结论,有$\mathrm{Br}(L/K)=\mathrm{H}^2(\mathbb{Z}/p\mathbb{Z},L^*)=K^*/\mathrm{N}_{L/K}(L^*)$.按照$\mathbb{Z}/p\mathbb{Z}$被$p$零化,得到$\mathrm{Br}(L/K)\subseteq\mathrm{Br}(K)[p]=0$.
    	
    	\qquad
    	
    	(d)推(a):取$G_k$的一个Sylow-$p$子群$G_p$,按照$\mathrm{cd}_p(G)=\mathrm{cd}_p(G_p)$,问题归结为证明$\mathrm{cd}_p(G_p)\le1$.这也等价于证明$\mathrm{H}^2(G_p,\mathbb{Z}/p\mathbb{Z})=0$.按照$k(\mu_p)/k$的扩张次数$p-1$和$p$互素,就有$G_p$的固定域$k_p$包含了$\mu_p$,于是有$\mathrm{H}^2(G_p,\mathbb{Z}/p\mathbb{Z})=\mathrm{H}^2(k_p,\mu_p)=\mathrm{Br}(k_p)[p]$.任取有限可分扩张$K_p/k_p$,Galois群记作$P$,我们知道$K_p$跑遍有限可分扩张时$\mathrm{Br}(K_p/k_p)$的并是$\mathrm{Br}(k_p)$.于是问题归结为证明$\mathrm{Br}(K_p/k_p)[p]=0$.这里$P$是有限$p$群,我们知道$p$群的极大真子群是指数为$p$的正规子群,于是我们有如下链:
    	$$P=P_0\supseteq P_1\supseteq\cdots\supseteq P_n=\{1\}$$
    	其中$P_{i+1}$都是$P_i$的正规子群,并且有$P_i/P_{i+1}\cong\mathbb{Z}/p\mathbb{Z}$.它对应了扩张链:
    	$$k_p=K_0\subseteq K_1\subseteq\cdots\subseteq K_n=K_p$$
    	其中$\mathrm{Gal}(K_i/K_0)=P/P_i$.我们对$i$归纳证明$\mathrm{Br}(K_i/K_0)[p]=0$.$i=0$是平凡的.假设$\mathrm{Br}(K_{i-1}/K_0)[p]=0$.取inf-res正合列:
    	$$\xymatrix{0\ar[r]&\mathrm{H}^2(P/P_{i-1},K_{i-1}^*)\ar[r]&\mathrm{H}^2(P/P_i,K_i^*)\ar[r]&\mathrm{H}^2(P_{i-1}/P_i,K_i^*)}$$
    	限制在$p$扭部分,得到正合列:
    	$$\xymatrix{0\ar[r]&\mathrm{Br}(K_{i-1}/K_0)[p]\ar[r]&\mathrm{Br}(K_i/K_0)[p]\ar[r]&\mathrm{Br}(K_i/K_{i-1})[p]}$$
    	按照条件(d)有$\mathrm{Br}(K_i/K_{i-1})[p]=0$,按照归纳假设有$\mathrm{Br}(K_{i-1}/K_0)[p]=0$,于是$\mathrm{Br}(K_i/K_0)[p]=0$,完成归纳.
    \end{proof}
    \item 设$k$是特征$p>0$的域,那么$\mathrm{cd}_p(k)\le1$.
    \begin{proof}
    	
    	取绝对Galois群$G$的一个Sylow-$p$子群$G_p$,这是闭子群,它对应于某个中间域$K$的绝对Galois群,我们解释过有$\mathrm{cd}_p(G)=\mathrm{cd}_p(G_p)$,于是问题归结为设$G$本身是射影$p$群.此时问题归结为证明$\mathrm{H}^2(G,\mathbb{Z}/p\mathbb{Z})=0$.但是按照Artin-Schreier短正合列,这是成立的.
    \end{proof}
\end{enumerate}
\subsubsection{$C_1$域}

拟代数闭域.一个域$k$称为拟代数闭域(quasi-algebraically closed)或者$C_1$域,如果对任意$k$上的$n$元$d$次齐次多项式$f$,满足$0<\deg f<n$,都有非平凡零点.
\begin{enumerate}
	\item 代数闭域一定是拟代数闭域.事实上代数闭域上的二元正次齐次多项式有非零解.
	\item 有限域是$C_1$域.我们甚至可以证明$\mathbb{F}_q$($q=p^r$)上的$n$元正次多项式,满足$d=\deg f<n$,则零点个数被$p$整除.
	\begin{proof}
		
		对多项式$f$记$N(f)$表示它在$\mathbb{F}_q^n$上的零点个数.并且记:
		$$\Sigma(g)=\sum_{(a_1,\cdots,a_n)\in\mathbb{F}_q^n}(g(a_1,\cdots,a_n))^{q-1}$$
		按照$a\in\mathbb{F}_q^*$总有$a^{q-1}=1$,说明$\Sigma(g)$落在$\mathbb{F}_p$中,并且有:
		$$q^n-\Sigma(g)=\sum_{(a_1,\cdots,a_n)\in\mathbb{F}_q^n}\left(1-(g(a_1,\cdots,a_n))^{q-1}\right)\equiv N(g)(\mathrm{mod}p)$$
		于是问题归结为证明对上述$f$有$\Sigma(f)$在$\mathbb{F}_p$中为零.把$f(x_1,\cdots,x_n)^{q-1}$展开,问题归结为证明$\Sigma(x_1^{r_1}\cdots x_n^{r_n})$在$\mathbb{F}_p$中为零,其中$\sum_ir_i<(q-1)n$.如果某个$r_i=0$,此时$\Sigma(x_1^{r_1}\cdots x_n^{r_n})$明显为零.于是不妨设全部$r_i>0$,那么此时存在某个$0<r_i<q-1$,不妨设为$r_1$.固定$(a_2,\cdots,a_n)\in\mathbb{F}_q^{n-1}$,取$\mathbb{F}_q^*$的一个生成元为$\omega$,那么有:
		$$\sum_{a\in\mathbb{F}_q}a^{r_1}a_2^{r_2}\cdots a_n^{r_n}=(a_2^{r_2}\cdots a_n^{r_n})\frac{\omega^{r_1(q-1)}-1}{\omega^{r_1}-1}$$
		这在$\mathbb{F}_p$中为零,再对$(a_2,\cdots,a_n)\in\mathbb{F}_q^{n-1}$求和得证.
	\end{proof}
    \item 设$k$是$C_1$域,那么对$k$的任意代数扩张$K$,都有$K$是$C_1$域.
    \begin{proof}
    	
    	设$F$是$K$上的次数为$d$的$n$元齐次多项式,满足$0<d<n$.按照$F$的系数都是$k$上的带深渊,我们可以不妨设$K$是$k$上有限扩张,扩张次数记作$m$.取$K/k$的一组基$\{v_1,\cdots,v_m\}$,引入新未定元$x_{ij}$,定义$x_i=x_{i1}v_1+\cdots+x_{im}v_m$,那么$f(x)=\mathrm{N}_{K/k}(F(x))$是$k$上的$mn$元$md$次齐次多项式.进而按照$k$是$C_1$域知存在$a_{ij}\in k$使得$a_i=\sum_ja_{ij}v_j$满足$\mathrm{N}_{K/k}(f(a_1,\cdots,a_n))=0$,于是$f(a_1,\cdots,a_n)=0$.这里从$a_{ij}$不全为零以及$\{v_1,\cdots,v_m\}$是一组基得到$\{a_i\}$不全为零.
    \end{proof}
    \item 设$k$是代数闭域,设$K/k$的超越维数为1,那么$K$是$C_1$域.
    \begin{proof}
    	
    	按照上一条,问题归结为设$K=k(t)$.取正次齐次多项式$f\in k(t)[x_1,\cdots,x_n]$,满足次数$d=\deg f<n$.不妨设$f$的系数落在$k[t]$中,只需构造一个非零的$k[t]^n$中的解.取一个正整数$N$,取$x_i=\sum_{j=0}^Na_{ij}t^j$,其中$a_{ij}\in k$未定.考虑:
    	$$0=f(x_1,\cdots,x_n)=\sum_{l=0}^{dN+r}f_l(a_{10},\cdots,a_{nN})t^l$$
    	其中$r$是$f$的系数作为$k[t]$中元的次数的最大值,而$f_l$是$f$展开后关于$a_{ij}$的齐次多项式.按照$d<n$,可以让$N$足够大使得$dN+r+1<n(N+1)$,于是这些$f_l=0$定义了$\mathbb{P}^{n(N+1)-1}_k$的非空闭子集.按照$k$是代数闭域,这个非空闭子集有$k$有理点,这是满足$f_l=0$的一组$\{a_{ij}\}\subseteq k$.
    \end{proof}
    \item 设$k$是完全域,那么$k((t))_{\mathrm{nr}}$是$C_1$域.特别的对代数闭域$k$就有$k((t))$是$C_1$域.【central simple algebras and Galois cohomology P144】
    \item 设$k$是$C_1$域,那么$\mathrm{cd}(k)\le1$,并且总有$\mathrm{Br}(k)=0$.
    \begin{proof}
    	
    	如果$p=\mathrm{char}(k)$,我们解释过此时总有$\mathrm{cd}_p(k)\le1$.下面设$p\not=\mathrm{char}(k)$,我们解释过此时$\mathrm{cd}_p(k)\le1$等价于验证对任意可分代数扩张$K/k$有$\mathrm{Br}(k)\{p\}=0$.于是问题归结为证明对$C_1$域$k$有$\mathrm{Br}(k)=0$.为此任取$k$的有限可分扩张$K$以及$K$的Galois扩张$L$,我们之前解释过只需验证$\mathrm{N}_{L/K}:L^*\to K^*$是满射.但是按照$k$是$C_1$域,得到它的代数扩张$K$和$L$都是$C_1$域.任取$a\in L^*$,考虑方程$\mathrm{N}_{L/K}(x)=ax_0^d$,如果记$[L:K]=d$,那么这是$L$上的$d$次$d+1$元齐次多项式,所以它有非平凡零点,也即存在$x\in L$和$x_0\in K^*$使得$\mathrm{N}_{L/K}(x/x_0)=a$.
    \end{proof}
\end{enumerate}













