\chapter{表示论基础}
\section{结合代数上的表示论}

表示.给定域$k$上的结合代数$A$,$A$的一个左表示是指一个$k$线性空间$V$以及一个$k$代数同态$\rho:A\to\mathrm{End}(V)$.右表示则是指把上述代数同态替换为$A^{op}\to\mathrm{End}(V)$的代数同态.换句话说满足反向的关系式$\rho(ab)=\rho(b)\rho(a)$和$\rho(1)=1$.通常把$\rho(a)v$简单记作$av$,右表示则记作$va$.如果$V$是有限维线性空间,我们称$V$是一个$A$的$\dim V$维的表示,而且在约定一组基的前提下,$A$中每个元可视为$\dim V$阶的方阵.下面给出一些基本例子:
\begin{enumerate}
	\item 取$V=A$,把$\rho:A\to\mathrm{End}(A)$定义为左乘,即$\rho(a)b=ab$.这个表示称为代数的(左)正则表示.右正则表示则定义为$\rho(a)b=ba$.
	\item 取$A=k$,那么$\rho(c)v=c\rho(1)v=cv$,即此时$A$的表示恰好就是$V$自身的线性空间结构.
	\item 取$A=k[x_1,x_2,\cdots,x_n]$,那么$A$在$V$上的一个表示恰好就是$V$自身的线性空间结构具备了$n$个预先指定的线性映射$\rho(x_1),\rho(x_2),\cdots,\rho(x_n)$.
\end{enumerate}

下面给出表示的一些构造.
\begin{enumerate}
	\item 子表示.给定表示$\rho:A\to\mathrm{End}(V)$,它的子表示是指一个子空间$W\subset V$,满足在每个线性映射$\rho(a),a\in A$下不变,即$\rho(a)(W)\subset W,\forall a\in A$.此时$W$上的表示约定为$\rho$在$W$上的限制.零空间和全空间总是子表示,它们称为平凡的子表示.如果一个线性空间$V$的子表示只有平凡的,就称它是不可约表示,或者单表示.
	\item 表示的同态.给定代数$A$的两个表示$\rho_i:A\to\mathrm{End}(V_i),i=1,2$.称一个线性同态$\gamma:V_1\to V_2$是两个表示的同态,如果这个线性映射和$A$的作用是可交换的,换句话讲满足如下交换图.特别的,如果$\gamma$是线性同构,就称它是表示之间的同构,也称两个表示是同构的.我们把$A$的全体$V_1\to V_2$的同态构成的集合记作$\mathrm{Hom}_A(V_1,V_2)$.
	$$\xymatrix{
		V_1\ar[r]^{\rho_1(a)}\ar[d]_{\gamma}&V_1\ar[d]^{\gamma}\\
		V_2\ar[r]^{\rho_2(a)}&V_2},\forall a\in A$$
	\item 表示的直和.给定代数$A$的两个表示$V_1,V_2$,那么线性空间$V_1\oplus V_2$上自然的存在一个$A$表示结构,即$a(v_1\oplus v_2)=(av_1)\oplus(av_2)$.这称为表示的直和.代数$A$的一个非零表示(即线性空间不为零)称为不可分解的,如果它不同构于两个非零表示的直和.
\end{enumerate}

一些注解.
\begin{enumerate}
	\item 在表示是有限维的前提下,所有内容可以用矩阵语言描述.子表示等价于讲,存在$V$上$r$个线性无关向量(它实际上是子表示$W$的一组基),它们在扩充为一组基后每个$\rho(a)$的矩阵表示是左上角的分块为$r\times r$的矩阵的分块上三角矩阵;表示的同态等价于讲取定$V_1$和$V_2$各自的一组基,记$a\in A$在表示$\rho_i$下的矩阵为$M_i(a)$,那么存在一个$\dim V_1=\dim V_2$阶的可逆矩阵$T$,使得有同时成立的相似关系$M_2(a)=TM_1(a)T^{-1}$.
	\item 表示之间的同态如果是双射,那么它的逆映射自动是同态,即双射同态自动是同构.
	\item 不可约表示总是不可分解的,反过来一般不成立.
	\item 给定代数$A$上表示的同态$\phi:V_1\to V_2$,那么$\ker\phi$是$V_1$的子表示;$\mathrm{im}\phi$是$V_2$的子表示.
\end{enumerate}

舒尔引理:描述不可约表示之间的同态.
\begin{enumerate}
	\item 设$A$是域$F$上的结合代数,设$V_1,V_2$是$A$的两个表示,给定非零同态$\phi:V_1\to V_2$,那么如果$V_1$不可约,则$\phi$的核只能是$V_1$的零子空间,于是$\phi$是单射;如果$V_2$不可约,则$\phi$的像只能是$V_2$的零空间,于是$\phi$是满射.综上得到不可约表示之间的同态只能是零映射和同构.
	\item 如果域$F$是代数闭域,$A$是$F$代数,设$V$是有限维的$A$上的不可约表示,那么$V\to V$的同态$\phi$只能是数乘映射,即$\phi=\lambda\mathrm{id}_V$,其中$\lambda\in k$.
	\begin{proof}
		
		按照$V$是有限维的,以及$F$是代数闭域,于是可取$\phi$作为线性变换的特征值$\lambda\in k$.现在$\phi-\lambda\mathrm{id}$同样是$V$上的表示,而且作为线性变换它不会是同构映射(因为有特征值0).于是按照上一条它只能是零映射,于是$\phi=\lambda\mathrm{id}$.
	\end{proof}
\end{enumerate}

如果$A$是代数闭域上的一个交换代数,那么每个有限维不可约表示$V$都是1维的.这个结论的矩阵角度描述是,给定代数闭域上的一个两两可交换的矩阵族,那么它们有公共特征向量.这里我们从舒尔角度给出证明,首先对任意$a\in A$,我们断言$\rho(a)$实际上可视为表示的同态,这只要注意到$\rho(a)(\rho(b)v)=\rho(ab)v=\rho(ba)v=\rho(b)(\rho(a)v)$.于是按照舒尔引理,说明每个$\rho(a)$都是数量算子,这导致$V$的每个子空间都是子表示,按照$V$不可约,只能有$V$是一维的.

不可约表示和不可分解表示的一些例子:
\begin{enumerate}
	\item 设$A=k$,此时$A$的表示恰好就是线性空间,于是$A$的不可约表示和不可分解表示只有一维的情况$V=k$.
	\item 设$A=k[x]$,其中$k$是代数闭域,那么$A$是交换代数,于是$A$的所有不可约表示都是1维的.另外$A$的表示被线性空间和赋予的一个线性变换$\rho(x)$所唯一确定,这里不可约表示是一维的,于是线性变换$\rho(x)$被它在$1_k$下的像唯一确定.于是此时$A$的全部不可约为$\{V_{\lambda}=k,\lambda\in k\}$,其中在$V_{\lambda}$上$\rho(x)=\lambda\mathrm{Id}$.这些表示是两两不同构的.
	\item 设$A=k[x]$,其中$k$是代数闭域,按照Jordan标准型理论,$A$的不可分解表示恰好就是全体$\{V_{\lambda,n}=k^n,\lambda\in k\}$.其中$V_{\lambda,n}$上的$\rho(x)$约定为把标准基中的$e_i,i>1$映射为$\lambda e_i+e_{i-1}$,把$e_1$映射为$\lambda e_1$.
	\item 考虑群代数$A=k[G]$,其中$G$是一个群.那么$A$的一个表示$V$,等价于讲$V$是$k$线性空间,并且指定了一个$k$代数同态$k[G]\to\mathrm{End}(V)$.这个$k$代数同态给出的信息等价于约定一个群同态$G\to\mathrm{Ant}(V)=\mathrm{GL}(V)$,再线性延拓到整个$k[G]$上.我们定义群$G$在$k$线性空间$V$上的表示是群同态$G\to\mathrm{GL}(V)$.于是群表示和群代数表示是某种意义上等价的.
	\item 如果$V$是非零的有限维表示,那么它必然存在不可约的子表示.但是对于无限维的情况,可考虑$A=k[x]$在自身的正则表示,它的每个元生成的子表示都同构于$V=k[x]$自身.于是倘若有不可约的子表示$W$,任取$p(x)\in W$,那么$W=p(x)V$,导致$p^2(x)V$是一个非平凡的$W$的子表示,这和不可约性矛盾.
\end{enumerate}

半单表示.给定代数$A$,称$V$是$A$的半单表示或者完全可约表示,如果它同构于$A$的一族不可约(单)表示的直和.












给定群$G$和一个域$F$,$G$在域$F$上的一个表示是指一个群同态$\rho:G\to\mathrm{GL}(n,F)$.这里称$n$为表示的次数.



任取一组基,那么$\mathrm{GL}(V)$中的一个元可以表示为$n\times n$的$F$上可逆矩阵.


给定域$F$上的一个线性空间$V$,记$\mathrm{GL}(V)$表示$V$上全体线性同构构成的群.当$V$是有限维$n$维的时候,任取一组基,那么$\mathrm{GL}(V)$中的一个元可以表示为$n\times n$的$F$上可逆矩阵.于是在取定一组基的条件下,$\mathrm{GL}(V)$同构于$F$上$n$阶可逆方阵群.

取一个有限群$G$,那么称$G$在$V$上的一个线性表示为从$G$到$\mathrm{GL}(V)$的群同态$\rho$,如果这个同态是单的,就称表示是忠实的.当约定好一个线性表示的时候,我们称$V$是$G$的一个表示空间.若$V$是$n$维的线性空间,就称其上的表示是$n$维的.记$G$的一个表示空间为$(V,\rho)$.

给定群$G$的两个表示$\rho:G\to\mathrm{GL}(V)$和$\rho':G\to\mathrm{GL}(V')$,称一个线性同态$\gamma:V\to V'$是两个表示的同态,如果满足交换图:
$$\xymatrix{
V\ar[r]^{\rho(s)}\ar[d]_{\gamma}&V\ar[d]^{\gamma}\\
V'\ar[r]^{\rho'(s)}&V'
},\forall s\in G$$

特别的,如果$\gamma$是同构,就称它是表示之间的同构.用矩阵形式,就是,取定$V$和$V'$各自的一组基,那么设$s\in G$对应的$V$和$V'$上的自同构为$R_s$和$R_s'$,那么存在一个$n$阶可逆矩阵$T$使得对任意$s\in G$有$R_s'=TR_sT^{-1}$.

记$Hom_F(V_1,V_2)$为两个线性空间之间的线性同态构成的交换群.记$Hom_G(V_1,V_2)$为两个表示空间之间的表示同态构成的交换群.并且后者是前者的子群.现在定义$Hom_F(V_1,V_2)$上的$G$表示为,$\forall g\in G$,$\forall f\in Hom_F(V_1,V_2)$.定义$(gf)(v)=gfg^{-1}(v),\forall v\in V_1$.那么有$Hom_G(V_1,V_2)$是一个子表示.

如果取$\rho(s)=1_V,\forall s\in G$,这个表示称为单位表示,或平凡表示.

一个一次的复表示就是从群$G$到乘法群$C^*$的群同态.因为群$G$是有限群,那么每个元的阶数有限,于是它们必然映射到$C^*$中的单位根.特别的,总有$|\rho(s)|=1$.

记$G$的阶数是$g$,设线性空间$V$的维数是$g$,取一组基记作$(e_t)_{t\in G}$,对每个$s\in G$,约定$\rho_s$为把$e_t$映射到$e_{st}$的线性同构.这个忠实的表示称为$G$的正则表示.注意到$e_1$在$G$中所有元下的像恰好构成一组基.反过来如果存在一个$V$上的表示$\rho:G\to\mathrm{GL}(V)$,存在一个$V$中的元$w$使得$\rho_s(w)$构成了$V$上的一组基,那么这个表示和正则表示是同构的.

更一般的,考虑一个有限集$X$,考虑$G$作用在$X$上,取$V$是以$(e_x)_{x\in X}$为基的线性空间,让$s\in G$诱导的线性同构为把$e_x$映射到$e_{sx}$.这称为$G$的由$X$诱导的置换表示.

子表示.给定一个群表示$\rho:G\to\mathrm{GL}(V)$.取$V$的子空间$W$,称$W$是$G$不变子空间,如果对任意$s\in W$和任意$x\in W$,有$\rho_s(x)\in W$.记$\rho_s$在$W$上的限制为$\rho_s^W$,那么这诱导了$W$上的表示$\rho^W:G\to\mathrm{GL}(W)$.这时称$W$是$V$的子表示.

商表示.给定$G$的表示空间$(V,\rho)$的子表示$(W,\rho^W)$,考虑商空间$V/W$,对任意$s\in G$,$x+W\in V/W$,定义$\rho'(x+W)=\rho(x)+W$.那么$(U/W,\rho')$称为关于子表示的商表示.

表示的张量积.给定有限群$G$的两个表示$\rho^1:G\to\mathrm{V_1}$, $\rho^2:G\to\mathrm{V_2}$.那么约定表示的张量积为$\rho:G\to\mathrm{V_1\otimes V_2}$,满足$\rho_s=\rho_s^1\otimes\rho_s^2$.

共轭表示.取一个$G$表示空间$(V,\rho)$,取$V$的共轭空间$V^*=Hom_F(V,F)$.考虑$\rho^*:G\to GL(V^*)$为$\rho^*(g)(f)(v)=f(g^{-1}v)$.

称一个线性表示$\rho:G\to\mathrm{GL}(V)$是不可约表示,如果不存在非0真子空间是这个表示的不变子空间.如果一个$G$表示空间$V$能够分解为若干不可约表示的直和,我们就称这个表示空间是完全可约表示.

注意.完全可约表示的不可约分解未必是唯一的,可以考虑一个有限群$G$在某个$>1$维线性空间的平凡表示,那么这个表示可以分解为任意种直线的直和分解.但是会看到,表示的不可约分解在同构意义下是唯一的.

Schur引理.给定群$G$的两个不可约表示$\rho^1:G\to\mathrm{GL}(V_1)$和$\rho^2:G\to\mathrm{GL}(V_2)$,如果$f:V_1\to V_2$是表示的同态,也就是说满足对任意$s\in G$有:
$$\xymatrix{
V_1\ar[r]^{\rho^1_s}\ar[d]_{f}&V_1\ar[d]^f\\
V_2\ar[r]^{\rho^2_s}&V_2
}$$

那么两个$G$的表示之间的同态要么是同构,要么是0同态.另外,一个表示到自身的同态$Hom_G(V,V)$,包含了$\lambda 1_V$这种情况,当域$F$是代数闭域的时候,$Hom_G(V,V)$恰好就是全体$\lambda 1_V$.

证明.取两个表示之间的同态$f:V_1\to V_2$,取$W_1=\ker f$是子空间,那么有$f\circ\rho_s^1x=\rho_s^2\circ f(x)=0$,于是$\rho_s^1(x)\in W_1$.于是$W_1$在$G$下不变.但是由于$V_1$是不可约的,于是要么$W_1=V$要么$W_1=0$,前者对应了0同态.同理讨论$\mathrm{im}f=W_2$,它要么是0要么是$V_2$,前者对应0映射,如果两种情况都满足后者,那么这是同构.现在取一个表示$\rho:G\to\mathrm{GL}(V)$,如果域是代数闭域,那么可取表示的同态$f$的一个特征值$\lambda$.取$f'=f-\lambda 1_V$,那么$\ker f'$非0,并且$f'$也是一个表示的同态.于是按照前一种情况,只能有$f'=0$,于是$f=\lambda 1_V$,证毕.

如果任取两个线性空间之间的同态$h:V_1\to V_2$.那么取$h'=\frac{1} {g}\sum_{t\in G}(\rho_t^2)^{-1}\circ h\circ\rho_t^1$,那么$h'$是两个表示空间之间的同态.特别的,一个线性空间$V$上的同态$h$,取相同的$h'$,那么按照Schur引理,有$h'$必然是某个$\lambda 1_V$.注意到$Tr(h)=Tr(h')=n\lambda$,得到$\lambda=\frac{Tr(h)}{n}$.

Maschke定理.设$G$是有限群,若$F$特征0,或者$F$的特征不整除$G$的阶数,那么$G$的每一个$F$表示都是完全可约的.

证明,给定一个线性表示$\rho:G\to\mathrm{GL}(V)$,给定表示的不变子空间$W\subset V$,那么存在$V$的另一个不变子空间$W'$,满足$V=W\oplus W'$.证明,先取任意的$W$的补$W''$,取从$V$到$W$的投射为$p$,记$p':V\to W$为$p'=\frac{1}{|G|}\sum_{t\in G}\rho_t\circ p\circ\rho_t^{-1}$,按照特征的要求,这个定义是良性的.那么$p'$也是从$V$到$W$的投影,于是记$W'=\ker p'$有$V=W\oplus W'$.最后注意到$\forall s\in G$有$\rho_s\circ p'=p'\circ \rho_s$,于是对$x\in W'$,有$0=\rho_s\circ p'(x)=p'\circ\rho_s(x)$,于是$\rho_s(x)\in W'$,这说明$W'$在$G$下不变,得证.

Maschke定理的逆同样成立.事实上如果$G$是有限群,那么只要$G$的$F$正则表示是完全可约的,就有$F$的特征要么0,要么特征不整除$|G|$.假设$F$的特征整除了$|G| $,取$G=\{g_1=1,g_2,\cdots,g_n\}$,设$n=pq$,其中$p$是$F$的特征,取$z=\sum_{g\in G}e_g$.考虑$W=Fz$,那么$W$是$V$的$G$不变子空间,于是按照完全可约性得到存在$W$的一个补$W'$也是$G$不变子空间.于是有$e_1=az+w,a\in F,w\in W'$,于是$w=e_1-az$,有$g_iw=g_ie_1-g_iaz=e_{g_i}-az$.于是$\sum g_iw=\sum g_i-anz=z\in W\cap W'=\{0\}$,这导致这组基是线性相关的,矛盾.

在Maschke定理条件下,每个表示空间都可以分解为直和$V=n_1V_1\oplus\cdots\oplus n_sV_s$.其中$V_i$两两不同构.这一段证明这样的分解在同构意义下唯一.如果还存在分解$V=m_1W_1\oplus\cdots\oplus m_sW_s$.按照Schur引理得到$Hom_G(V,W_i)\sim m_iHom_G(W_i,W_i)\not=0$,但是还有$Hom_G(V,W_i)=\oplus_j n_jHom_G(V_j,W_i)$,于是必然存在某个$V_j=W_i$,另外按照$V_k$两两不同构,得到这样的$j$是唯一存在的,于是$V_i$就是$W_i$的一个重排.最后证明对应的有$m_j=n_i$,于是分解是唯一的.

正则表示的不可约分解.在Maschke条件下,$G$的任意不可约$F$表示都是$G$的正则表示的直和项,特别的,$G$的全部不同构的不可约表示是有限的.把正则表示$\rho$分解为不可约表示的直和$\rho=\oplus_in_i\rho_i$,现在记正则表示空间为$R$,任取$G$的不可约表示$(V,\rho')$,考虑$Hom_G(R,V)\to V$为$f\mapsto f(e_1)$,这得到了$Hom_G(R,V)$到$V$的同构,特别的$Hom_G(R,V)$非0.现在有$Hom_G(R,V)\sim\oplus_i n_iHom_G(V_i,V)$.于是按照Schur引理,存在一个$V_i=V$.

按照上述证明,看到正则表示分解中$\rho_i$的系数就是$\frac{\dim_FV_i} {\dim_F Hom_G(V_i,V_i)}$.再取等式两边的次数,得到这样一个关系式:取$t_i=\dim_FV_i$,$d_i=\dim_F Hom_G(V_i,V_i)$,那么有$|G|=\sum_i\frac{t_i^2}{d_i}$.其中右边每一项对应于不同构的不可约表示.特别的,当$F$是特征0代数闭域的时候全部$d_i=1$.

特征标.给定一个线性表示$\rho:G\to\mathrm{GL}(V)$,定义表示的特征标为$G\to F$的函数$\chi_{\rho}(s)=\mathrm{Tr}(\rho_s)$.
\begin{enumerate}
  \item 同构的线性表示具有相同的特征标.
  \item 设表示的维数是$n$,那么$\chi(1)=n1_F$.
  \item $\chi(tst^{-1})=\chi(s)$.即特征标是群$G$上的类函数,也就是说在同一个共轭类上取相同的值.
  \item 表示的直和对应特征标的和,表示的张量积对应特征标的积,表示的商对应特征标的减法.
  \item 对于复表示.$\chi(s^{-1})=\chi(s)^*$,这里$*$表示的是共轭.事实上因为群$G$的阶是有限的,于是每个矩阵$\phi_s$都满足某个$x^m=1$.于是$\phi_s$的特征值都是模长为1的复数,由此得证.
  \item 若$\rho^*$是$\rho$的共轭表示,那么$\chi_ {\rho^*}(g)=\chi_{\rho}(g^{-1})$.
  \item 正则表示的特征标.按照正则表示的定义,看到正则表示$(V,\rho)$的特征标$\chi$满足$\chi(1)=|G|1_F$,并且$\chi(s)=0,\forall s\not=1$.
\end{enumerate}

特征标的正交性.给定两个$G$上的$F$值函数$\phi$和$\psi$,记$(\phi,\psi)=\frac{1}{|G|}\sum_{t\in G}\phi(t)\psi(t^{-1})$.那么$(\phi,\psi)$是分别关于$\phi$和$\psi$是双线性的,并且是对称的,即$(\phi,\psi)=(\psi,\phi)$.如果$(\phi,\psi)=0$,那么就称$\phi$和$\psi$是正交的.

第一正交关系.设域$F$要么特征0,要么特征不整除$|G|$.
\begin{enumerate}
  \item 给定两个不同构的不可约表示的特征标$\chi$和$\chi'$,那么有$(\chi,\chi')=0$.
  \item 如果$\chi$是域$F$上线性空间$V$的不可约表示,设域$F$上线性空间$Hom_G(V,V)$的维数是$n$,那么有$(\chi,\chi)=n1_F$.
  \item 特别的,当$V$是特征0代数闭域上的线性空间的时候,有$(\chi,\chi)=1_F$.
\end{enumerate}

这个定理存在多种证明.可约运用复杂的矩阵记号,也可以运用半单代数理论.这里给出一个回避它们的做法.

引理.给定一个$G$的表示空间$(V,\rho)$,记特征标是$\chi$,记子空间$W=\{v\in V\mid gv=v,\forall g\in G\}$,称为$V$在$G$下的不变子空间.那么这是$G$的一个子表示,记特征标为$\chi_W$.断言有:$$\frac{1}{|G|}\sum_{g\in G}\chi(g)=\frac{1}{|G|}\sum_{g\in G}\chi_W(g)$$
\begin{proof}

考虑$V$上的线性变换$z=\frac{1}{|G|}\sum_{g\in G}\rho_g$,按照特征的要求这个定义是良性的.按照$\rho_hz=z\rho_h=z,\forall h\in G$,得到$z^2=z$.于是$z$是幂等线性变换,于是特征值是0和1.于是得到特征子空间分解$V=V_0\oplus V_1$.其中$V_i$是特征值$i$的特征子空间.那么看到$W\subset V_1$.反过来对$x\in V_1$,有$gv=gz(v)=z(v)=v$,于是$v\in W$,于是得到$W=V_1$.于是有:$$\frac{1}{|G|}\sum_{g\in G}\chi(g)=\mathrm{tr}(z)=\mathrm{tr}(z\mid_{V_1})=\frac{1}{|G|}\sum_{g\in G}\chi_W(g)$$

\end{proof}

现在给出第一正交关系的证明.如果两个不可约表示不同构,那么按照Schur引理得到$Hom_G(V_1,V_2)=0$.于是这归结到第一种情况.任取$G$的两个表示空间$U,V$,设对应的特征标为$\chi_U,\chi_V$.考虑表示空间$Hom_F(U,V)$,首先证明它在$G$下的不变子空间是$Hom_G(U,V)$.于是有$(\chi_V,\chi_U)=\frac{1} {|G|}\sum_{g\in G}\chi_{U^*}(g)\chi_V(g)=\frac{1}{|g|}\sum_{g\in G}\chi_{U^*\otimes V}(g)$.现在注意到有表示的同构$Hom_F(U,V)\sim U^*\otimes V$.于是按照引理得到$(\chi_U,\chi_V)=\frac{1} {|G|}\sum_{g\in G}\chi_{Hom_G(U,V)}(g)$.最后,当$U=V$的时候,设$Hom_G(V,V)$作为$F$上线性空间的维数是$n$,那么它恰好是$n$个单位表示的直和.于是得到最后一式为$n1_F$.

特征0的域上两个表示同构当且仅当它们具有相同的特征标.只要证明充分性,
给定$G$的表示空间$V$,设特征标为$\phi$.设$V$分解为不可约表示的直和$V=\oplus_{1\le i\le k}n_iW_i$,其中$W_i$是两两不同构的不可约子表示,设$W_i$的特征标是$\phi_i$,那么得到$\phi=\sum_{1\le i\le k}n_i\phi_i$,那么对一个固定的不可约子表示$W$,取特征标$\chi$,将其正交于上述等式,得到每个$(\phi_i\mid\chi)$是$n_id_i=n_i\dim_F Hom_G(W_i,W_i)$.现在任给另一个不可约表示分解,按照正交性看到特征标的分解是唯一的,得证.

给定满足Maschke条件下的表示$(V,\rho)$,设不可约分解为$V=\oplus_in_iV_i$,其中$V_i$两两不同构,记$n_i=\dim_FV_i$,记$d_i=\dim_FHom_G(V_i,V_i)$,记$V$的特征标为$\chi$,那么有$(\chi,\chi)=\sum_in_i^2d_i$.特别的当域是特征0代数闭域的时候,有$(\chi,\chi)=\sum_in_i^2$,于是对于复表示,一个表示是不可约的当且仅当特征标$\chi$满足$(\chi,\chi)=1$.

类函数.群$G$到域$F$上的类函数是指,在同一个共轭类上取值恒定的函数.设全体$G\to F$的类函数构成的集合为$H$.下面主要证明复数域上两个定理,一个是$G$上全部不同构的不可约复表示构成了$H$的一个正交基.另一个是$G$的全部不同构的不可约复表示的个数就是$G$的共轭类的个数.事实上这里可以把复数域改为特征0或者特征不整除$|G|$的域$F$,并且$F$满足对每个不可约表示空间$V$,有$Hom_G(V,V)$作为$F$线性空间是1维的,那么复数域是它的一个特例.

引理.给定$G$上类函数$f$,给定$G$的表示空间$(V,\rho)$,取$V$上的线性映射维$f'=\sum_{t\in G}f(t)\rho_t$.如果$V$是$n$次不可约的,并且特征标记作$\chi$,那么$f'=\lambda 1_V$.证明,如果$f'$和每个$\rho_s$都可交换,那么就有$f'=\lambda 1_V$,并且其中$\lambda=\frac{g}{n}(f,\chi^*)$,于是只要验证:
$$\rho_s^{-1}f'\rho_s=\sum_{t\in G}f(t)\rho_s^{-1}\rho_t\rho_s=
\sum_{t\in G}f(t)\rho_{s^{-1}ts}=f'$$

$G$上全体不同构的不可约特征标构成了$H$上的一组正交基.正交性由第一正交关系已经得到.现在只要证明它生成了$H$.而这只要证明$H$中的和$\chi_i^*$都正交的类函数是0.取$H$中这样的类函数$f$,按照引理取$f'$.那么$f'=0$,由此得到$f=0$.

看到$H$作为线性空间的维数就说$G$的共轭类个数,结合上一段看到维数还是不同构的不可约表示的个数,于是二者相同.

\section{复表示的例子}

一个有限群的不可约复表示都是一维的当且仅当它是交换群.充分性,设$(V,\rho)$是有限交换群$G$的不可约复表示.那么对任意$g,h\in G$有$\rho_g$和$\rho_h$可交换的可逆线性映射,那么按照线性代数,存在公共的特征向量$v\in V$使得$\rho_g(v)=\lambda_gv,\forall g\in C$,那么取$V$的子空间$W=Cv$,于是$W$是$V$的$G$不变子空间,按照不可约性得到$V=W$,于是$V$的维数是1.必要性,设$G$的共轭类个数是$s$,如果不可约表示都是1维的,那么每个$\chi_i(1)=1$,于是得到$|G|=\sum_{i=1}^{s}\chi_i(1)^2=s$,于是$G$是交换群.

循环群$C_n$.循环群是交换群,于是每个不可约表示都是一维的.设循环群$C_n$生成元是$r$,那么$r^n=1$,设特征标是$\chi$,记$\chi(1)$是复数$w$,那么有$w^n=1$,于是找到了$n$个一维不可约表示为:
$$\chi_h(r^k)=e^{\frac{2\pi ihk}{n}},h=0,1,2,\cdots,n-1$$

二面体群$D_n$.

\section{代数的表示}

给定有限群$G$,给定交换环$K$,记$K[G]$为$G$在$K$上生成的群代数.