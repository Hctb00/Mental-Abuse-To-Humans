\chapter{线性代数和矩阵论}
\section{矩阵和线性变换}
\subsection{矩阵}

把域上的模称为线性空间,把线性空间之间的模同态称为线性变换.当提及线性空间或线性变换时总是指域上的情况.

我们证明过交换环满足维数不变性的,即交换环$R$上取定一个自由模$F$,那么它的任意基都有相同的势,称为自由模的秩,在线性代数里把自由模的秩称为它的维数,记作$\dim F$.任取势为这个秩的集合$A$,就有$F\cong R^{\oplus A}$.这个事实既有好处也有坏处.好处是,自由模之间的同态构成的$R$模$\mathrm{Hom}_R(F_1,F_2)$,有很好的处理方法:如果取定$F_1,F_2$分别的基$A_1,A_2$,那么有$\mathrm{Hom}_R(F_1,F_2)\cong\mathrm{Hom}_R(R^{\oplus A_1},R^{\oplus A_2})$.坏处是,这种同构依赖于$F_1,F_2$上基的选取.

考虑交换环$R$上的有限生成自由模$N,M$,如果已经分别取定了一组基,那么$\mathrm{Hom}_R(N,M)$可以同构为$\mathrm{Hom}_R(R^n,R^m)$.这时候同态集可以被一个二维数表刻画,便是矩阵.

称元素在交换环$R$中的$m\times n$数表是一个$m\times n$型矩阵,记作$(r_ {ij})$,其中第$i$行第$j$列的元记作$r_{ij}$.会用大写字母诸如$A,B,C,P,Q$来表示一个矩阵.全体$m\times n$的$R$上矩阵记作$M_{m,n}(R)$,如果所在的环在上下文中是没有歧义的,就简单记作$M_{m,n}$.如果两个矩阵的行数和列数相同,就称它们为同型的矩阵.

这里来说明模同态如何对应于矩阵.取定同态$f\in\mathrm{Hom}_R(R^n,R^m)$,记$R^n$中的标准基为$\varepsilon_1,\cdots,\varepsilon_n$,记$R^m$中的标准基为$\eta_1,\cdots,\eta_m$.那么$f$被在$\{\varepsilon_1,\cdots,\varepsilon_n\}$上的取值所唯一决定.现在有唯一分解:
$$f(\varepsilon_1)=\sum_{i=1}^mr_{i1}\eta_i=r_{11}\eta_1+r_{21}\eta_2+\cdots+r_{m1}\eta_m$$
$$f(\varepsilon_2)=\sum_{i=1}^mr_{i2}\eta_i=r_{12}\eta_1+r_{22}\eta_2+\cdots+r_{m2}\eta_m$$
$$\cdots$$
$$f(\varepsilon_n)=\sum_{i=1}^mr_{in}\eta_i=r_{1n}\eta_1+r_{2n}\eta_2+\cdots+r_{mn}\eta_m$$

于是,全体$r_{ij}$唯一决定了$f$,也就是说$f$被一个$m\times n$的矩阵$(r_{ij})$所决定.反过来,现在任意给定一个$m\times n$的矩阵$(r_{ij})$,按照上述若干等式,可以确定一个同态$f:R^n\to R^m$.于是$\mathrm{Hom}_R(R^n,R^m)$和$M_{m,n}$是一一对应的.

接下来通过同态之间的关系给矩阵赋予相应的运算.首先$\mathrm{Hom}_R(R^n,R^m)$是一个$R$模,先来给矩阵集赋予$R$模结构.

首先是加法,如果$f,g:R^n\to R^m$是同态,并且分别对应的$m\times n$矩阵为$A=(a_{ij}),B=(b_{ij})$.那么考虑$(f+g)(\varepsilon_i)$的分解,就得到$f+g$理应对应的矩阵是$m\times n$的矩阵$(c_{ij})$,其中$c_{ij}=a_{ij}+b_{ij}$.这个运算约定为$M_{m,n}$的加法.即两个矩阵相加就是对应位置的项做和构成的新的矩阵.$M_{m,n}$在这个运算下构成交换群,其中0元就是每项都是0的矩阵,称为0矩阵,一个矩阵的加法逆元就是每一项都取负元构成的矩阵.

现在看模结构,给定$f:R^n\to R^m$和$r\in R$,设$f$对应矩阵$(a_{ij})$,有$(rf)(x)=f(rx)$,据此,得到$rf$对应的矩阵为$(ra_{ij})$.交换群$M_{m,n}$上的$R$模结构就是$r(a_{ij})=(ra_{ij})$.于是$\mathrm{Hom}_R(R^n,R^m)$和$M_{m,n}$甚至是$R$模同构的.

同态之间的运算不止有加法这一种,还存在复合运算.考虑同态$f:R^n\to R^m,g:R^p\to R^n$,设对应的矩阵分别为$m\times n$的矩阵$A=(a_{ij})$和$n\times p$的矩阵$B=(b_{ij})$.我们期望映射的复合$f\circ g:R^p\to R^m$会提供矩阵上的一种乘法运算.即有交换图:
$$\xymatrix{
	M_{m,p}(R)\times M_{p,n}(R)\ar[rr]\ar[d]_{\sim}&&M_{m,n}(R)\ar[d]_{\sim}\\
	Hom_R(R^p,R^m)\times Hom_R(R^n,R^p)\ar[rr]&&Hom_R(R^n,R^m)
}$$

记$R^p$中的标准基是$\alpha_1,\cdots,\alpha_p$,记$R^n$中的标准基为$\beta_1,\cdots,\beta_n$,记$R^m$中的标准基为$\gamma_1,\cdots,\gamma_m$.现在计算:
$$f\circ g(\alpha_i) =f\left(\sum_{j=1}^nb_{ji}\beta_j\right)
=\sum_{j=1}^nb_{ji}f(\beta_j)
=\sum_{j=1}^nb_{ji}\sum_{k=1}^ma_{kj}\gamma_k
=\sum_{k=1}^m\left(\sum_{j=1}^na_{kj}b_{ji}\right)\gamma_k$$

于是,$f\circ g$对应的矩阵为$(c_{ij})$,其中$c_{ij}=\sum_{k=1}^na_{ik}b_{kj}$.矩阵的乘法就定义为$AB=(c_{ij})$.注意,按照矩阵乘法的定义,只有当$A$的列数和$B$的行数相同的时候,才能有乘积$AB$,并且新矩阵的行数是$A$的行数,新矩阵的列数是$B$的列数.

矩阵乘法是满足结合律的,即如果$AB$可乘$BC$可乘,那么$(AB)C=A(BC)$.

特别的,如果$m=n$,此时把$M_{n,n}$简单记作$M_n$,其中的矩阵称为$n$阶方阵,$M_n$中任意两个矩阵可以相乘.这个乘法满足分配律,其中乘法单位元是$(\delta_{ij})$,即$i=j$时元素取1,$i\not=j$时元素取0.乘法单位元称为$n$阶单位矩阵,通常记作$E_n,I_n$或者在阶数不引起歧义的前提下略去角标.于是$M_n(R)$是一个环,称为$R$上的$n$阶矩阵环,正如映射的复合一般来讲未必交换,这个环一般来讲也是不交换的环.另外,可以验证数乘和矩阵乘法是可交换的$r(MN)=(rM)N=M(rN)$,于是$M_n$实际上是一个$R$代数.

如同所有环一样,矩阵环上存在乘法可逆元,我们知道对于一般的未必交换的环上,存在左逆和存在右逆之间未必有联系,只有当同时存在左逆和右逆的情况下,所有的左逆右逆都是唯一一个元.但是矩阵环上的情况比较特别,我们会在介绍足够多内容后证明,一个矩阵的左逆也必然是右逆,反过来也一样,所以只要矩阵存在单侧逆,那么它必然是唯一的双侧逆,这样的矩阵称为可逆矩阵.在眼下这个阶段,先把可逆矩阵定义为具有双侧逆的矩阵$A$,即存在$B$满足$AB=BA=I_n$.从同态角度看,可逆矩阵对应于模同构.

把$n\times 1$型的矩阵称为列向量,把$1\times m$型的矩阵称为行向量.那么一个有限生成自由模$R^n$中的元可以理解为行向量或者列向量.一般来讲把它理解做列向量.此时$R^n$的标准基就定义为:
$$e_1=\left(\begin{array}{c}
1\\
0\\
\vdots\\
0\end{array}\right);e_2=\left(\begin{array}{c}
0\\
1\\
\vdots\\
0\end{array}\right);\cdots;e_n=\left(\begin{array}{c}
0\\
0\\
\vdots\\
1\end{array}\right)$$

于是每个列向量可以有基表示:
$$v=\left(\begin{array}{c}
v_1\\
v_2\\
\vdots\\
v_n\end{array}\right)=\sum_{j=1}^nv_je_j$$

在这种表示下,模同态的作用完全被矩阵运算代替.一个同态$f:R^n\to R^m$作用在一个向量$v\in R^n$上,等价于$f$对应的$m\times n$矩阵$A$做乘于列向量$v\in R^n$上:
$$Av=\left(\begin{array}{cccc}
a_{11}&a_{12}&\cdots&a_{1n}\\
a_{21}&a_{22}&\cdots&a_{2n}\\
\vdots&\vdots&\ddots&\vdots\\
a_{m1}&a_{m2}&\cdots&a_{mn}
\end{array}\right)\left(\begin{array}{c}
v_1\\
v_2\\
\vdots\\
v_n\end{array}\right)=\left(\begin{array}{c}
a_{11}v_1+a_{12}v_2+\cdots+a_{1n}v_n\\
a_{21}v_1+a_{22}v_2+\cdots+a_{2n}v_n\\
\vdots\\
a_{m1}v_1+a_{m2}v_2+\cdots+a_{mn}v_n\end{array}\right)\in R^m$$

转置矩阵.一个$m\times n$型矩阵$A=(a_{ij})$的转置矩阵是$n\times m$型矩阵$(b_{ij})$,其中$b_{ij}=a_{ji}$,记作$A^T$.转置变换具有如下基本性质:$(A+B)^T=A^T+B^T$,$(rA)^T=rA^T$,$(A_1A_2\cdots A_s)^T=A_s^TA_{s-1}^T\cdots A_1^T$,$(A^T)^T=A$.注意,当$R$非交换时,$(AB)^T=B^TA^T$并不总成立.

矩阵转置对应的是映射取“对偶”.给定交换环$R$上有限生成模$E$,它的对偶空间是$E^*=\mathrm{Hom}_R(E,R)$,即全体$E\to R$的模同态构成的$R$模,也把$E^*$中的元称为$E$上的线性函数.那么对偶空间$E^*$和$E$的维数是相同的,事实上$E$的任意一组基$\{\varepsilon_1,\varepsilon_2,\cdots,\varepsilon_n\}$,那么取$f_i\in E^*$满足$f_i(\varepsilon_j)=\delta_{ij}$,这表示$i=j$时取$1_R$,$i\not=j$时取$0_R$.则$\{f_1,\cdots,f_n\}$构成了$E^*$上的基,称为$\{\varepsilon_1,\varepsilon_2,\cdots,\varepsilon_n\}$的对偶基.注意这样得到的同构$E\simeq E^*$也是依赖于基的选取的.

现在取交换环$R$上有限生成模之间的同态$\phi:E\to F$,分别取$E,F$的基$\{\varepsilon_1,\varepsilon_2,\cdots,\varepsilon_n\}$,$\{\eta_1,\cdots,\eta_m\}$,记$f$在这组基下的矩阵为$m\times n$的矩阵$A=(a_{ij})$,记这两组基分别的对偶基为$\{f_1,\cdots,f_n\}$和$\{g_1,\cdots,g_m\}$.那么$\phi$诱导了一个$F^*\to E^*$的模同态$\phi^*$,满足:$\phi^*(f)=f\circ\phi$.那么$\phi^*$在对偶基下的矩阵表示即$A^T$:
$$\phi^*(f_i)(\sum_{k=1}^{n}r_k\varepsilon_k)
=\sum_{k=1}^{n}r_kf_i\left(\phi(\varepsilon_k)\right)
=\sum_{k=1}^{n}r_kf_i\left(\sum_{j=1}^{m}a_{jk}\eta_j\right)
=\sum_{k=1}^{n}r_ka_{ik}=\sum_{j=1}^{n}a_{ij}f_j\left(\sum_{k=1}^{n}r_k\varepsilon_k\right)$$

一个$n$阶方阵$A=(a_{ij})$的主对角线,是指$n$个元$\{a_ {11},a_{22},\cdots,a_{nn}\}$构成的列,这些元也称为对角元.如果方阵$A$满足$a_{ij}=0,i>j$,即主对角线下方的元都为0,就称$A$是上三角矩阵.如果满足$a_{ij}=0,i<j$,即主对角线上方的元都为0,就称为下三角矩阵.如果矩阵的非对角元都为0,就称矩阵是对角矩阵.把对角矩阵记作$\mathrm{diag}\{a_ {11},a_{22},\cdots,a_{nn}\}$.对角矩阵$rI=\mathrm{diag}\{r,r,\cdots,r\}$称为数量矩阵.
\newpage
\subsection{Smith标准型}

如之前所说,尽管利用矩阵处理有限生成自由模之间的同态十分方便,但是对有限生成自由模$M,N$,从$\mathrm{Hom}_R(M,N)$到$M_{n,m}$的同构不是唯一的,它依赖于事先取定的基.因此,为了寻找这两个集合之间更自然的,不依赖于基的联系,不得不舍弃寻找它们之间模同构这一思路,转而寻找某种双射.本节就是讨论如何寻找一个从$\mathrm{Hom}_R(M,N)$到$M_{n,m}$的合适的对应.我们会根据映射和矩阵之间的关系,分别给两个集合赋予等价关系,全部等价类之间存在双射,并且这种等价关系可以被Smith标准型所刻画.

首先,基的不同选取会使得同一个同态的矩阵表示不同,先来看改变基后矩阵会如何变化.给定同一个有限生成自由模$F$上的两个基,对应于两个同构:
$$\xymatrix{
	R^{\oplus A}\ar[rr]^{\phi}&&F
}\qquad\qquad
\xymatrix{
	R^{\oplus B}\ar[rr]^{\psi}&&F
}
$$

那么得到:
$$\xymatrix{
	R^{\oplus A}\ar[rr]^{\psi^{-1}\circ\phi}&&R^{\oplus B}
}
$$

模同构$\psi^{-1}\circ\phi$就对应于一个$\dim F$阶数的可逆方阵,它称为从基$\phi$ 到基$\psi$的基变换矩阵,记作$N_A^B$,于是对于一个有限生成自由模之间的模同态$f$,有:
$$\xymatrix{
	R^{\oplus B}\ar[dr]^{\phi}\ar[dd]_{v_A^B}&&&R^{\oplus D}\ar[dl]_{\rho}\ar[dd]^{\mu_A^B}\\
	&F\ar[r]^{f}&G&\\
	R^{\oplus A}\ar[ur]_{\psi}&&&R^{\oplus C}\ar[ul]^{\sigma}
}$$

于是,如果有限生成模之间的同态$f:F\to G$,取$F$的基$A$和$G$的基$C$,对应的矩阵为$M$,现在又取了$F$的基$B$和$G$的基$D$,那么如果从$B$到$A$的基变换矩阵为$Q$,从$D$到$C$的基变换矩阵为$P$,那么在新的两组基下同态$f$的矩阵表示为$PMQ$,其中$P,Q$是可逆矩阵.

反过来,如果$f:F\to G$对分别$F,G$的基$A,C$下的矩阵为$M$,如果任取分别以$A$的行数为阶数和以$A$的列数为阶数的可逆方阵$P,Q$,由于同构会把一组基映射为另一组基,于是$PMQ$是$f$在$F,G$分别的基$B,D$下的矩阵,这里$B=\{(v_A^B)^{-1}a\mid a\in A\}$,$D=\{(\mu_C^D)^{-1}c\mid c\in C\}$.

换一个记号可以把这个内容表示的更加清晰.给定模同态$f:F\to G$,取$F$上的一组基为$A=\{\alpha_1,\cdots,\alpha_n\}$,取$G$上的一组基$C=\{\gamma_1,\cdots,\gamma_m\}$.那么在这两组基下,$f$的矩阵表示记作$M$,转化成等式写法,就有:
$$f(\alpha_1,\alpha_2,\cdots,\alpha_n)=(\gamma_1,\gamma_2,\cdots,\gamma_m)M$$

如果再取$F$上的一组基$B=\{\beta_1,\cdots,\beta_n\}$和$G$上的一组基$D=\{\delta_1,\cdots,\delta_m\}$.在这两组基下矩阵表示记作$N$,转化成等式写法,就有:
$$f(\beta_1,\beta_2,\cdots,\beta_n)=(\delta_1,\delta_2,\cdots,\delta_m)M$$

设基$A$在基$B$下的矩阵表示为$P$,设基$C$在基$D$下的矩阵表示为$Q$,于是有:
$$(\alpha_1,\alpha_2,\cdots,\alpha_n)=(\beta_1,\beta_2,\cdots,\beta_n)P$$
$$(\gamma_1,\gamma_2,\cdots,\gamma_m)=(\delta_1,\delta_2,\cdots,\delta_m)Q$$

于是得到:
$$f(\alpha_1,\alpha_2,\cdots,\alpha_n)=f(\beta_1,\beta_2,\cdots,\beta_n)P=(\gamma_1,\gamma_2,\cdots,\gamma_m)MP
=(\delta_1,\delta_2,\cdots,\delta_m)QMP$$

于是得到了$N=QMP$,即换基后得到了相抵的矩阵表示.

总结一下至此所得到的结果.若有限生成自由模之间的模同态$f$在一对基下的表示是矩阵$A$,那么换基后的矩阵表示是$MAN$,其中$M$和$N$是两个可逆矩阵,它们是相应的基变换矩阵.反过来,如果$f$在某组基下的矩阵是$A$,那么任取两个可逆矩阵$M,N$,则$MAN$是$f$在某个基下的矩阵.据此抽象出针对矩阵的概念:称两个同型矩阵相抵当且仅当它们是同一个模同态在不同基下的表示,这等价于说存在两个可逆矩阵两侧乘以其中一个矩阵得到另一个.相抵是一个等价关系.在这个相抵关系下,每个映射唯一的对应于一个矩阵的相抵关系等价类.

这个对应是单的吗?不是,同一个矩阵可以被不同的映射取定基意义下所对应.设存在两个线性映射$f,g:E\to F$对应矩阵$A$,设$E$的基$X_1=\{u_1,\cdots,u_n\}$和$F$的基$Y_1=\{p_1,\cdots,p_m\}$使得$f$对应了矩阵$A$,设$E$的基$X_2=\{v_1,\cdots,v_n\}$和$F$的基$Y_2=\{q_1,\cdots,q_m\}$使得$g$对应了矩阵$A$,设$E$上的自同构$h$为把$v_i$映射为$u_i$,$F$上的自同构$k$为把$q_i$映射为$p_i$,那么有:
$$f(k_1u_1+\cdots+k_nu_n)=A(k_1,\cdots,k_n)^T(p_1,\cdots,p_m)$$
$$f\circ h(k_1v_1+\cdots+k_nv_n)=k\left(A(k_1,\cdots,k_n)^T(q_1,\cdots,q_m)\right)
=f\circ h(k_1v_1+\cdots+k_nv_n)=k\circ g(k_1v_1+\cdots+k_nv_n)$$

于是$k^{-1}\circ f\circ h=g$.即如果同一个矩阵被两个映射$f,g$所对应,那么存在两个相应空间上的模同构$h,k$,满足$k^{-1}\circ f\circ h=g$.反过来,如果有两个映射$f,g:E\to F$满足存在相应空间上的同构使得$k^{-1}\circ f\circ h=g$,取定$E,F$分别的一组基$X_1,Y_1$,使得$f$在这组基下的矩阵是$A$,$k^{-1}$和$h$在这组基下的矩阵分别是可逆的$P,Q$,于是$g$在这组基下的矩阵是$PAQ$,它和$A$是相抵的,于是存在一组基下$g$的矩阵表示是$A$.

据此抽象出针对映射的概念:两个同基空间和像空间的同态$f,g$称为相抵的,如果它们存在矩阵表示相同,这等价于说存在相应空间上的同构$h,k$满足$h\circ f\circ k=g$.映射的相抵关系是一个等价关系.

现在得到了矩阵上的相抵等价关系和映射上的相抵等价关系,一个自然的问题是这两组等价类之间是同构的吗?是的!给定映射$f$,设所在的等价类为$[f]$,任取$f$的矩阵表示$A$,设$A$所在的等价类是$[A]$,那么$[f]\to [A]$是一个双射.首先证明映射的良性,如果$f$还有矩阵表示$A'$,那么$A,A'$相抵,自然有$[A]=[A']$.满射是因为每个矩阵都存在某个映射和某组基使得矩阵表示是它,单射是因为,如果有$[f]$和$[g]$都对应$[A]$,那么$f$和$g$都有矩阵表示$A$,这导致$[f]=[g]$.

接下来的思路是,对于有限生成自由模之间的同态$f$,选取特殊的基,使得在这对基下的矩阵能具有一种统一类型的,尽可能简洁方便的形式.换句话说,从矩阵的相抵这个等价关系的每个等价类中,选取一个特殊的尽可能简洁方便的矩阵.将要介绍的选取出来的这种矩阵称为Smith标准型.一个矩阵在相抵等价关系下所在等价类的那个选取出的矩阵称为它的Smith标准型.为此先来介绍三种特殊的可逆矩阵.

三种初等行/列变换:
\begin{enumerate}
	\item 将矩阵的两行或者两列交换.
	\item 将某行/列的倍数(即乘以环中某元素)加至另一行/列.
	\item 将某行/列乘以一个环中的单位
\end{enumerate}

三种初等矩阵:
\begin{enumerate}
	\item 将单位矩阵的两行或者两列交换.
	\item 将单位矩阵的某行/列的倍数(即乘以环中某元素)加至另一行/列.
	\item 将单位矩阵某行/列乘以一个环中的单位
\end{enumerate}

容易验证,对矩阵做初等行变换相当于左乘相应的初等矩阵,对矩阵做初等列变换相当于右乘相应的初等矩阵.从初等变换的逆操作存在,并且都是初等变换,看出初等矩阵都是可逆矩阵,并且逆矩阵也是初等矩阵.

把$M_n(R)$的单位群称为$R$上$n$阶一般线性群,记作$\mathrm{GL}_n(R)$.那么,一般的交换环上初等矩阵未必生成一般线性群.但是在欧氏整环的情况下这是成立的,为此先来证明域的情况是成立的:

域上的一个$n$阶可逆方阵总可以写作若干初等矩阵的乘积.等价的说,每个可逆矩阵都可以经过初等变换变为单位矩阵.

\begin{proof}
	
	对矩阵的阶数归纳.$n=1$没什么可证的.假设对$n-1$阶矩阵成立.取$n$阶矩阵$A=(a_{ij})$.那么这个矩阵的第一列不能全为0,否则这个矩阵作用到$R^n$的标准基的第一个向量$e_1$上会取0,导致矩阵对应的$R^n\to R^n$的同态把一个非0元映射为了0,于是同态不是双射.于是第一列存在某个元是非0的,对这行乘以第一个元的逆元,让这个非0元变为1,再把所在行和第一行交换,于是,把元矩阵初等变换为如下矩阵:
	$$\left(\begin{array}{cccc}
	1&a_{12}&\cdots&a_{1n}\\
	a_{21}&a_{22}&\cdots&a_{2n}\\
	\vdots&\vdots&\ddots&\vdots\\
	a_{n1}&a_{n2}&\cdots&a_{nn}
	\end{array}\right)$$
	
	现在,把第一行的$-a_{i1}$倍加至第$i$行,$1\le i\le n$,于是得到如下形式矩阵:
	$$\left(\begin{array}{cccc}
	1&a_{12}&\cdots&a_{1n}\\
	0&a_{22}&\cdots&a_{2n}\\
	\vdots&\vdots&\ddots&\vdots\\
	0&a_{n2}&\cdots&a_{nn}
	\end{array}\right)$$
	
	再把第一列的$-a_{1i}$倍加至第$i$列,$1\le i\le n$,得到如下形式矩阵:
	$$A_1=\left(\begin{array}{cccc}
	1&0&\cdots&0\\
	0&a_{22}&\cdots&a_{2n}\\
	\vdots&\vdots&\ddots&\vdots\\
	0&a_{n2}&\cdots&a_{nn}
	\end{array}\right)=\left(\begin{array}{cc}
	1&0\\
	0&A'\end{array}\right)$$
	
	按照$A$是可逆矩阵,把它的逆矩阵$B$分块的记作:
	$$B=\left(\begin{array}{cc}
	b&\alpha\\
	\beta&B'\end{array}\right)$$
	
	按照矩阵乘法,从$AB=BA=I_n$,得到$A'B'=B'A'=I_{n-1}$.于是$A'$是一个$n-1$阶可逆矩阵.但是现在对上述$A_1$中的$A'$做初等变换,总是相当于对整个$A_1$做初等变换,这个可以一一验证.由此按照归纳假设,存在初等变换把$A_1$中右下角的$A'$变作$I_{n-1}$,这就完成了归纳.
\end{proof}

这个结论提供了一种求可逆矩阵的逆矩阵的算法.如果把可逆矩阵写作了初等矩阵的乘积$A=P_1P_2\cdots P_r$,每个初等矩阵的逆矩阵是容易写出的,于是逆矩阵就是$B=P_r^{-1}\cdots P_2^{-1}P_1^{-1}$.在计算具体矩阵的时候,可以先在原矩阵右侧补一个单位矩阵,然后两个矩阵同时做初等行变换,目标是把左侧原矩阵最终变为单位矩阵,此时右侧矩阵就会是所求的逆矩阵.
$$\left(\left.\begin{array}{cccc}
a_{11}&a_{12}&\cdots&a_{1n}\\
a_{21}&a_{22}&\cdots&a_{2n}\\
\vdots&\vdots&\ddots&\vdots\\
a_{n1}&a_{n2}&\cdots&a_{nn}
\end{array}\right|\begin{array}{cccc}
1&0&\cdots&0\\
0&1&\cdots&0\\
\vdots&\vdots&\ddots&\vdots\\
0&0&\cdots&1
\end{array}\right)$$

这个结论可以证明域上矩阵的相抵标准型定理.即域$F$上每个$m\times n$的矩阵都相抵于唯一的一个如下形式的矩阵.于是两个矩阵相抵当且仅当相抵于同一个下述形式的矩阵,矩阵的相抵关系就被这个数字$0\le r\le\min\{m,n\}$所完全描述.
$$\left(\begin{array}{cc}
I_r&0\\
0&0
\end{array}\right),r=1,2,\cdots,\min\{m,n\}$$

先来说明每个矩阵都可以相抵于这样的矩阵.回忆上一个定理的证明中,归纳下去的理由是右下角的矩阵$A'$可逆,但是实际上只要右下角的矩阵$A'$中含有一个元不是0,那么就可以用初等变换把它变作1,再把它通过对换行或列变到$A'$左上角的位置,接下来再用初等变换把变化后的1所在位置的行和列都消除为0,继续操作下去,只有当右下角的矩阵$A'=0$的时候操作停止,也就得到了上述标准型.

相抵的矩阵是同一个线性变换在不同基下的表示,如果有$s,t\le\min\{m,n\},s\not=0$使得$\left(\begin{array}{cc}
I_s&0\\
0&0
\end{array}\right)$相抵于$\left(\begin{array}{cc}
I_r&0\\
0&0
\end{array}\right)$,取一个同态$f:R^n\to R^m$使得它在两对基下的矩阵分别是这两个矩阵,那么有$\mathrm{im}f$是$R^m$的$r$维和$s$维子空间,这矛盾了.

于是一个矩阵的Smith标准型中的$r$,实际上就是这个矩阵对应的(任一)线性变换的像空间的维数.把这个数字称为相应矩阵或者相应映射的秩.秩完全刻画了映射的相抵关系和矩阵的相抵关系,或者说秩就是这两个关系的全系不变量.

关于秩还有一个重要的等式关系.为了描述它先回归到映射角度.给定域上一个有限维线性空间$V$,取子空间$W$,那么有商空间$V/W$,我们断言有维数关系$\dim V=\dim W+\dim V/W$.
\begin{proof}
	
	取$W$的一组基$X$,把它扩充为$V$上的基$Y$,设添加的向量为$\{v_1,\cdots,v_s\}$,只要证明$\{v_i+W\mid 1\le i\le s\}$就是$V/W$的一组基,原命题就得证.
	
	取有限维线性空间之间的一个同态$f:V\to V'$,按照定义它的秩$r$就是$\mathrm{im}f$的维数,按照刚刚所证的结论,结合$V/\ker f\cong\mathrm{im}f$,就有关系式$\dim\ker f+\dim\mathrm{im}f=\dim V$.
\end{proof}

现在把这个关系式转化为矩阵语言.如果$f$对应某个$m\times n$矩阵$A$,那么$\ker f$同构于$R^n$的这样一个子空间$S=\{x\in R^n\mid Ax=0\}$,这个子空间称为矩阵$A$的解空间,因为它是齐次线性方程组$Ax=0$的全部解构成的子空间.于是上述等式变为:一个矩阵的秩加上解空间维数,等于它的列数.

事实上取上一段证明中的基,就会使得$f$在这组基下的矩阵表示为smith标准型,这也提供了域上Smith标准型的映射角度证明.给定线性变换$f:V\to V'$,记它的秩为$r$,取$\ker f$的一组基$\{v_{r+1},\cdots,v_n\}$,把它扩充为$V$上的基$X=\{v_1,\cdots,v_n\}$,我们断言$\{f(v_1),\cdots,f(v_r)\}$是$\mathrm{im}f$的一组基,这只要证明它是线性无关的,倘若线性相关,那么存在不全为0的系数$k_1,\cdots,k_r$使得$f(k_1v_1+\cdots+k_rv_r)=0$,这导致$k_1v_1+\cdots+k_rv_r\in\ker f$,但是这就得到了$v_1,\cdots,v_n$的线性组合为0,其中系数不全为0,这就矛盾.现在把$\mathrm{im}f$的这组基扩充为$V'$的基$Y=\{t_1=f(v_1),\cdots,t_r=f(v_r),t_{r+1},\cdots,t_m\}$,那么在$X$和$Y$下,线性映射$f$的矩阵表示就是Smith标准型.

给定域$F$上一个$n$维线性空间$V$,在取定一组基的前提下,$V$中的元可以表示为$n\times1$或者$1\times n$的向量.于是在取定一组基的前提下会把$V$中的元称为向量.若干个向量构成的集合称为一个向量组.一个向量组生成的$V$的子空间的维数称为这个向量组的秩.一个向量组$S$的线性无关子集如果满足任意给这个子集添加$S$中的另一个向量都会变为线性相关组,就称这个子集是$S$的极大线性无关组.那么$S$的极大线性无关组的向量个数必然是$S$的秩.事实上如果记$S$生成的$V$的子空间为$V'$,按照$S$中每个向量都可以被极大线性无关组线性表出(否则添加这个向量不会变为线性相关组),就有极大线性无关组生成了$V'$,再结合它是线性无关的,就得出极大线性无关组的向量个数必然是$S$的秩.

行秩与列秩.设$A$是$m\times n$的矩阵,设有模同态$f$在某组基$\{\varepsilon_1,\cdots,\varepsilon_n\}$下对应了这个矩阵,那么$\mathrm{im}f$到列向量组生成的子空间之间存在同构.于是二者维数相同,把列向量组生成的子空间的维数称为矩阵的列秩,于是按照定义矩阵的秩也就是矩阵的列秩.对偶的定义行向量组在$R^n$中生成的子空间的维数为矩阵的行秩.我们断言域上矩阵的行秩和列秩相同.
\begin{proof}
	
	给定$m\times n$型的矩阵$A$,设$A$的列秩为$r$,于是可以取列空间的一组基$\{c_1,\cdots,c_r\}$,那么可以把这$r$个向量凑成一个$m\times r$型矩阵$C$,按照$A$的每个列向量都可以线性表示为$C$的列向量的线性组合,得到了一个$r\times n$型矩阵$R$满足$A=CR$.于是$A$的每个行向量都是$R$的行向量组的线性组合,于是$A$的行空间可以被$r$个元素生成,于是$A$的行秩不超过它的列秩.现在考虑$A$的转置矩阵$A^t$,重复上述操作得到$A$转置矩阵的行秩不超过列秩,但是转置矩阵的行空间是原矩阵的列空间,转置矩阵的列空间是原矩阵的行空间,这就得证.
\end{proof}

这里给出行秩和列秩相等的一个映射角度的证明.我们知道矩阵的转置会把行空间和列空间调换.而矩阵转置对应于映射取对偶.所以等价于证明,一个有限维线性空间之间的线性变换,和它的对偶变换具有相同的秩.
\begin{proof}
	
	设$f:E\to F$是有限维线性空间之间的线性变换.设$f$的秩是$r$,那么可以取$E$上的一组基$X=\{u_1,\cdots,u_n\}$使得$\{u_{r+1},\cdots,u_n\}$是$\ker f$的一组基,而$Y_1=\{f(u_1),f(u_2),\cdots,f(u_r)\}$是$\mathrm{im}f$的一组基.现在把$Y_1$延拓为$F$上的一组基$Y=\{t_1=f(u_1),\cdots,t_r=f(u_r),t_{r+1},\cdots,t_m\}$.现在取$X,Y$的对偶基为$x^*$和$Y^*$.那么有:
	$$f^*(t_i^*)(u_j)=t_i^*(f(u_i))=\left\{\begin{array}{c}
	t_i^*(t_j)=\delta_{ij},j=1,2,\cdots,r\\
	0,j=r+1,r+2,\cdots,n
	\end{array}\right.$$
	
	于是对$i=1,2,\cdots,r$有$f^*(t_i^*)=u_i^*$,对$i=r+1,\cdots,n$有$f^*(t_i^*)=0$,这就说明$\mathrm{im}f^*$的维数也是$r$.
\end{proof}

我们已经给出了域上矩阵相抵下的标准型.如果把域的条件放宽,有PID上矩阵的Smith标准型:对$A\in M_{m,n}(R)$,其中$R$是一个PID,它必然相抵于如下矩阵,并且理想列$(d_1),(d_2),\cdots,(d_r)$被等价类唯一决定.【】
$$\left(\begin{array}{cc}
D&0\\
0&0
\end{array}\right),D=\mathrm{diag}\{d_1,d_2,\cdots,d_r\},d_1\mid d_2\mid\cdots\mid d_r$$

这个证明会放后【】,眼下这个阶段来证明当$R$是欧氏整环的情况.
\begin{proof}
	
	记赋值为$v$,倘若矩阵是一个0矩阵,那么已经不需要做任何事,现在假设矩阵存在非0元,下面对阶数归纳,这对于1阶矩阵自然是成立的,设阶数大于1,对每一个$A$的相抵矩阵,它的非0元中存在赋值最小数,于是可以取$A$的一个相抵矩阵,它的这个最小数是所有相抵矩阵中最小的,不妨经初等变换令这个元的位置在(1,1),倘若第一行和第一列的元素中存在不能被$a$整除,那么由第二初等变换和带余除法,将会得到一个赋值更小元,这和的选取矛盾,于是只要经初等变换,便可把第一行与第一列的全部其他元变为0,记此时的矩阵为$B$,它和$A$相抵,我们断言$a$整除$B$的每个元,倘若其他位置存在一个非0元不被$a$整除,不妨设为$(i,j)$,其中$i$和$j$均不为1,那么只需把第$i$行加至第一行,那么$(1,j)$的位置是这个不被$a$整除的元,同样经过初等变换和带余除法,得到了一个赋值更小元,并且新的矩阵和$A$相抵,这和$A$的选取矛盾,综上$B=diag\{a,C\}$,其中$C$的每个元被$a$整除,按照归纳假设,$C$相抵于对角矩阵$diag\{c_1,c_2,\cdots,c_{r-1},0\}$,对$C$进行初等变换不改变元素被$a$整除的性质,于是$a\mid c_1$,这便得到想要的形式.
\end{proof}

按照欧氏整环上Smith标准型的存在性,可以说明欧氏整环上初等矩阵生成了一般线性群.取一个可逆方阵$A$,按照上述Smith标准型的证明过程,$A$可以左右乘以若干初等矩阵变成一种对角矩阵$\mathrm{diag}\{a_1,\cdots,a_n\}$,对角矩阵是可逆矩阵当且仅当对角元都是可逆元,于是可以再经过初等变换把这个对角矩阵变为单位矩阵,于是就得到$A$表示为若干初等矩阵的乘积.
\newpage
\subsection{行列式}

从多重线性映射的角度来引出行列式的概念.给定$R$模$B_1,B_2,\cdots,B_n,C$,称$f::B_1\times B_2\times\cdots\times B_n\to C$为$n$重线性映射,如果对每个分量都是线性的,即满足下面等式.特别的,当$B_1=B_2=\cdots=B_n=B,C=R$时,称$f$是$B$上的$n$重线性函数.
$$f(x_1,\cdots,px_{i}+qx'_{i},\cdots,x_n)=pf(x_1,\cdots,x_{i},\cdots,x_n)
+qf(x_1,\cdots,x'_{i},\cdots,x_n),\forall 1\le i\le n$$

现在定义三种特殊的多重线性映射:
\begin{enumerate}
	\item 称$f$是对称的,如果对任意$\sigma\in S_n$有:
	$$f(x_{\sigma(1)},x_{\sigma(2)},\cdots,x_{\sigma(n)})=
	f(x_1,x_2,\cdots,x_n)$$
	\item 称$f$是斜对称的,如果对任意$\sigma\in S_n$有:
	$$f(x_{\sigma(1)},x_{\sigma(2)},\cdots,x_{\sigma(n)})=(\mathrm{sgn}(\sigma))
	f(x_1,x_2,\cdots,x_n)$$
	\item 称$f$是交错的,如果对任意$x_i=x_j,i\not=j$有:
	$$f(x_1,x_2,\cdots,x_n)=0$$
\end{enumerate}

斜对称和交错的关系.交错的多重线性映射必然是斜对称的,给定交错的多重线性映射$f$,为了验证它是斜对称的,按照$S_n$中的置换总可以表示为对换的乘积,不妨设只有两个分量,那么此时,对任一的$a,b\in B$,有$0=f(b_1+b_2,b_1+b_2)=f(b_1,b_2)+f(b_2,b_1)$,于是看到$f(b_2,b_1)=(\mathrm{sgn})f(b_1,b_2)$.但是反过来,斜对称并不总是交错的,但是对于2是单位的环$R$上,这的确是成立的,因为$f(a,b)=-f(b,a)$,就有$2f(x,x)=0$,结合2是单位就得到$f(x,x)=0$.

给定正整数$n$,取定交换环$R$的一个元$r$,那么$R^n$上存在唯一一个交错$n$重$R$线性映射,满足对$R^n$上标准基$\varepsilon_1,\cdots,\varepsilon_n$有$f(\varepsilon_1,\cdots,\varepsilon_n)=r$.

证明是直接的,对$R^n$中$n$个向量$X_i=(a_{i1},a_{i2},\cdots,a_{in})$,记住$f$是交错的,有:
\begin{align*}
f(X_1,X_2,\cdots,X_n)&=f(\sum_{i_1}a_{1i_1}\varepsilon_{i_1},\sum_{i_2}a_{2i_2}\varepsilon_{i_2},
\cdots,\sum_{i_n}a_{ni_n}\varepsilon_{i_n})\\
&=\sum_{i_1,i_2,\cdots,i_n}a_{1i_1}a_{2i_2}\cdots a_{ni_n}f(\varepsilon_{i_1},\varepsilon_{i_2},\cdots,\varepsilon_{i_n})\\
&=\sum_{\sigma\in S_n}a_{1\sigma(1)}a_{2\sigma(2)}\cdots a_{n\sigma(n)}f(\varepsilon_{\sigma(1)},\varepsilon_{\sigma(2)},\cdots,\varepsilon_{\sigma(n)})\\
&=\sum_{\sigma\in S_n}(\mathrm{sgn}\sigma)a_{1\sigma(1)}a_{2\sigma(2)}\cdots a_{n\sigma(n)}
\end{align*}

特别的,取$r=1$对应的唯一的交错多重线性映射$f$,现在给定一个$n$阶方阵$A$,它的$n$个(有序的)列向量在$f$下的取值,称为方阵$A$的\textbf{行列式},记作$|A|$或者$\det A$.那么首先,利用行列式的表达式,不难算出,一个方阵的行向量组在$f$下的取值也是行列式,由此得出行列式保转置,即$\det A=\det A^T$.

注意到$n$个$R^n$ 中元素可以拼做一个$n\times n$的矩阵, 往往就将行列式看为$R$ 上$n\times n$ 矩阵到$R$ 的映射,这个值称为矩阵(方阵)的行列式,对矩阵$A$,记行列式为$|A|$.

按照行列式的定义,直接看出,一个上三角或下三角或者对角矩阵,它的行列式就是全体对角元的乘积.单位矩阵的行列式自然就是1.

行列式是乘性函数,即$\det AB=\det A\det B$.证明,固定方阵$B$,设它的行向量组为$\{Y_1,\cdots,Y_n\}$,现在任取方阵$X$,设列向量组为$\{X_1,\cdots,X_n\}$,那么$XB$的第$i$行为$(X_iY_1,X_iY_2,\cdots,X_iY_n)$.于是,可以验证$X\mapsto\det XB$是一个交错$n$重线性函数,于是按照定理有$|XB|=f(X)=r|X|$为了确定$r$的值,带入$X=I$,就得到$r=|B|$,完成证明.

矩阵初等变换对行列式的影响.这些内容都可以从定义得到.交换两行或两列会让行列式变号;把某一行/列乘以单位或者任意的$R$中的元,会让行列式变为原来的这个元的倍数;最后把一行/列的某个倍数加至另一行/列,行列式不改变.事实上这个内容就是在说初等矩阵的行列式,即,第一类初等矩阵行列式-1,第二类初等矩阵的行列式就是所乘的那个单位,第三类是1.它的意义是提供一种求行列式的做法,对于欧氏整环的情况,按照方阵的Smith标准型,可以把一个方阵表示为$MDN$,其中$D$是一个对角矩阵,行列式是对角元的乘积,那么原矩阵的行列式是$\det M\det N\det D$,于是可以把$M,N$分解为初等矩阵的乘积,就可以求出原矩阵的行列式.注意如果$D$的对角元存在0,那么此时行列式必然是0.

这里来把秩的概念推广到一般交换环上.首先,关于域上矩阵的秩还有一个等价定义.矩阵的秩就是它不为0的子式的最高阶数.如果$m\times n$阶矩阵$A$的秩是$r$,一方面,那么存在$r$个行向量是线性无关组,取这$r$行构成的$r\times n$阶子矩阵$A_1$,那么$A_1$的秩是$r$,于是又可以取$A_1$的$r$个列向量线性无关,这样得到的$r\times r$子矩阵是满秩的,于是行列式非0.另一方面,取$r<k\le\min\{m,n\}$,任取$A$的一个$k$阶子方阵$A_1$.它对应了$A$的$k$个行向量,按照行秩是$r$,看到这$k$个行向量是线性相关的,导致$A_1$的行向量组是线性相关的,于是$A_1$不满秩,就导致行列式为0.

把一个交换环上的矩阵的秩就定义为非0子式的最大阶数.按照初等变换对行列式的影响,我们看到初等变换不会改变矩阵是否为0.另外对矩阵做初等变换也就是对子式做初等变换,这就说明了初等变换不会改变矩阵的秩.另外对于整环上的对角矩阵,它的秩就是对角元上非0元的个数.对角元存在0就是说矩阵的秩严格小于阶数.不过非整环上这自然未必成立.我们把矩阵的秩恰好等于行数或列数的矩阵称为行满秩或列满秩矩阵,对于方阵的情况二个概念一致,此时称为满秩矩阵.于是,矩阵满秩等价于要求行列式非0.

【分解秩和行列式秩】

另外,由于欧氏整环上可逆矩阵被初等矩阵生成,初等变换不改变矩阵的秩,看到欧氏整环上相抵关系是保矩阵的秩的.

下面来给出矩阵可逆和行列式的联系.将证明交换环上矩阵可逆当且仅当它的行列式是环上的可逆元.这个命题的必要性是直接的,困难的在于充分性.首先对于PID的情况(或域的情况),根据前文提及的Smith标准型,这时候矩阵可以分解为一个对角矩阵左右分别乘以若干初等矩阵,已经知道行列式是乘性的,于是它们行列式的乘积是可逆元.那么原本矩阵行列式是可逆元等价于说Smith标准型的行列式是可逆元,而这个对角矩阵的行列式是对角元的乘积,这说明每个对角元均为可逆元,这说明对角矩阵本身是若干个初等矩阵的乘积,从而将任意一个行列式为可逆元的矩阵分解为若干初等矩阵的乘积,这说明它是可逆矩阵.对于一般交换环上问题的困难性在于不再有smith标准型.为此需要引入更多的概念.

对于$n$阶方阵$A$,把划去$A$的第$i$行和第$j$列得到的$n-1$阶方阵的行列式称为$(i,j)$的子式(minor),把这个子式再乘以$(-1)^{i+j}$称为$(i,j)$的代数余子式(cofactor).关于代数余子式有如下公式:
$$\sum_{k=1}^na_{ik}A_{jk}=\delta_{ij}|A|;\sum_{k=1}^na_{ki}A_{kj}=\delta_{ij}|A|$$

这个证明可以利用行列式的表达式,或者利用行列式的映射定义,去证明$A=(a_{ij})\mapsto\sum_{k=1}^na_{ik}A_{jk}$是交错的多重线性映射.这个结论也成为行列式按一行/一列展开.按照这个公式,倘若取矩阵$A^a$表示$(i,j)$元是$A_{ji}$的矩阵,那么有:$AA^a=A^aA=|A|I_n$.把这里的$A^a$称为$A$的相伴矩阵,它是由代数余子式作为项转置构成的方阵.

利用相伴余子式可以完成上述必要性对一般交换环的证明.即倘若$|A|$可逆元,那么直接构造出了逆$\frac{A'}{|A|}$.另外,利用这个结果,还看到交换环上的矩阵环上,一个元有单侧逆那么必然有双侧逆.因为如果$AB=I_n$,取行列式就得到$A$和$B$的行列式是单位,于是它们都有双侧逆.最后,这直接提供了一般交换环上求逆矩阵的一种方法.

称全体环$R$上$n$阶可逆矩阵以乘法构成的群为$R$上$n$阶一般线性群$\mathrm{GL}_n(R)$.那么行列式是$\mathrm{GL}_n(R)$到$R$上单位群的群同态.这个同态的核称为特殊线性群,记作$\mathrm{SL}_n(R)$.它就是行列式为1的矩阵构成的群.

这里总结一下满秩,可逆,行列式取值的联系.在交换环上,可逆等价于行列式取单位,它们推出满秩;在域上,可逆等价行列式非0等价满秩.

关于矩阵的行列式还有如下两个公式.第一个Laplace展开,相当于推广了之前的按一行一列展开,这里是按$r$行或$r$列展开.第二个Binet-Cauchy公式,也可以推出行列式乘性的性质.

\begin{enumerate}
	\item Laplace展开:$1\le t_1<t_2<\cdots<t_r\le n$,$t_1\cdots,t_n$是$1,2,\cdots,n$的排列,并且
	$1\le t_{r+1}<\cdots<t_n\le n$,那么:
	$$|A|=\sum_{1\le k_1<k_2<\cdots<k_r\le n}(-1)^{\sum_{i=1}^r(k_i+t_i)}A
	\left(
	\begin{array}{cccc}
	t_1&t_2&\cdots&t_r\\
	k_1&k_2&\cdots&k_r
	\end{array}\right)A\left(\begin{array}{cccc}
	t_{r+1}&t_{r+2}&\cdots&t_n\\
	k_{r+1}&k_{r+2}&\cdots&k_n
	\end{array}\right)$$
	\begin{enumerate}
		\item
		$$\left|\begin{array}{cccccc}
		a_{11}&\cdots&a_{1r}&0&\cdots&0\\
		\vdots&\ddots&\vdots&\vdots&\ddots&\vdots\\
		a_{r1}&\cdots&a_{rr}&0&\cdots&0\\
		a_{r+1,1}&\cdots&a_{r+1,r}&a_{r+1,r+1}&\cdots&a_{r+1,n}\\
		\vdots&\ddots&\vdots&\vdots&\ddots&\vdots\\
		a_{n,1}&\cdots&a_{n,r}&a_{n,r+1}&\cdots&a_{n,n}
		\end{array}\right|=
		\left|\begin{array}{ccc}
		a_{11}&\cdots&a_{1r}\\
		\vdots&\ddots&\vdots\\
		a_{r1}&\cdots&a_{rr}
		\end{array}\right|\left|\begin{array}{ccc}
		a_{r+1,r+1}&\cdots&a_{r+1,n}\\
		\vdots&\ddots&\vdots\\
		a_{n,r+1}&\cdots&a_{n,n}
		\end{array}\right|
		$$
		\item $A=(a_{ij})$,$\xi_i=\sum_{j=1}^na_{ij}\varepsilon_j,1\le i\le n$,$1\le k_1<k_2<\cdots<k_r\le n$,那么:
		$$\sum_{\left(
			\begin{array}{cccc}
			k_1&k_2&\cdots&k_r\\
			j_1&j_2&\cdots&j_r
			\end{array}\right)}a_{1,i_1}a_{2,i_2}\cdots a_{r,i_r}\det(\varepsilon_{j_1},\varepsilon_{j_2},\cdots,\varepsilon_{j_p},\xi_{p+1},\xi_{p+2},\cdots,\xi_n)
		$$
		$$=(-1)^{\sum_{i=1}^r(i+k_i)}A\left(
		\begin{array}{cccc}
		1&2&\cdots&r\\
		k_1&k_2&\cdots&k_r
		\end{array}\right)A\left(
		\begin{array}{cccc}
		r+1&r+2&\cdots&n\\
		k_{r+1}&k_{r+2}&\cdots&k_n
		\end{array}\right)$$
	\end{enumerate}
	\item Binet-Cauchy公式:$A$是$p\times q$矩阵,$B$是$q\times p$矩阵,那么:
	$$\det(AB)=\left\{
	\begin{array}{lcl}
	0 & & q<p \\
	\det(A)\det(B) &  &q=p \\
	\sum_{1\le j_1<j_2<\cdots<j_p\le q}A\left(\begin{array}{cccc}
	1&2&\cdots&p\\
	j_1&j_2&\cdots&j_p
	\end{array}\right)B\left(\begin{array}{cccc}
	j_1&j_2&\cdots&j_p\\
	1&2&\cdots&p
	\end{array}\right) &  &q>p
	\end{array}\right.$$
\end{enumerate}

本节最后来证明欧氏整环上Smith标准型的唯一性.首先,考虑欧氏整环上矩阵的Smith标准型,那么,原矩阵的秩就是标准型中非0对角元的个数.这就说明了smith标准型中非0对角元的个数是固定的.

给定欧氏整环上$m\times n$的矩阵$A$,称$A$的全部$k$阶子式生成的理想,也就是它们的最大公因数生成的主理想(欧氏整环都是PID),为$A$的$k$阶行列式因子,把这个主理想记作$(D_k)$,这里$D_k$在相差一个单位的意义下唯一.按照初等变换只会把同一个子式乘以一个单位,看出矩阵的初等变换不会改变各阶行列式因子.

可以证明行列式因子完全决定了相抵关系:欧氏整环上的两个同型矩阵相抵当且仅当它们具有相同的各阶行列式因子.按照欧氏整环上可逆矩阵由初等矩阵生成,看到必要性已经得证了.为说明充分性,只要注意到,如果矩阵$A$相抵于Smith型矩阵$\mathrm{diag}\{d_1,d_2,\cdots,d_r,\textbf{0}\}$.那么对行列式因子中的$D_i$,有等式$(d_1)=(D_1)$,$(d_2)=(\frac{D_2}{D_1})$,$\cdots$,$(D_r)=(\frac{D_{r}}{D_{r-1}})$.也就是说,如果两个矩阵和行列式因子相同,那么它们相抵于一个相同的smith型矩阵.

于是,看到同一个矩阵的Smith标准型,在$(d_1),(d_2),\cdots,(d_r)$固定的意义下唯一.把每个$d_i$称为矩阵$A$的不变因子,它在相差一个单位的意义下唯一.称$\{d_1,d_2,\cdots,d_r\}$是矩阵$A$的不变因子组.

按照欧氏整环是唯一分解整环,可以把每个不变因子$d_i$分解为若干不可约元的次幂的乘积$d_i=\prod_{j=1}^{s_j}p_{j}^{r_{ij}}$,把全体$p_j^{r_{ij}}$称为矩阵$A$的初等因子.全体初等因子构成初等因子组.

注意初等因子组和不变因子组严格意义上讲不是集合,因为它们可能会包含重复的同一个元.

初等因子组和秩可以推出不变因子组.如果给出一组不变因子组,把不同的不可约因式分别按照次数排列,让它们都排列为秩数个,不够的拿0次幂补齐,让全部次数最高的一次因式次幂对齐,第二高的对齐,把对齐的不同一次因式次幂相称,每一个就对应一个不变因子,由此得出不变因子组.于是在域上初等因子组和秩也是Smith标准型的全系不变量.

准对角矩阵的初等因子组可以通过取每个对角块的初等因子组得到.为此只要证明,对$A=\mathrm{diag}\{A_1,A_2\}$,那么$A$的初等因子组可以由$A_1,A_2$的初等因子组合并得到.
\begin{proof}
	
	设$A_1,A_2$的不变因子组分别是$\{d_1,d_2,\cdots,d_r\}$和$\{d_{r+1},d_{r+2},\cdots,d_{r+s}\}$.也就是说存在可逆矩阵$P_1,P_2,Q_1,Q_2$满足:
	$$P_1A_1Q_1=\mathrm{diag}\{d_1,d_2,\cdots,d_r,\textbf{0}\}$$
	$$P_2A_2Q_2=\mathrm{diag}\{d_{r+1},d_{r+2},\cdots,d_{r+s},\textbf{0}\}$$
	
	记$P=\mathrm{diag}\{P_1,P_2\}$,$Q=\mathrm{diag}\{Q_1,Q_2\}$,那么有:
	$$PAQ=B=\mathrm{diag}\{D_1,\textbf{0},D_2,\textbf{0}\};D_1=\mathrm{diag}\{d_1,d_2,\cdots,d_r\},D_2=\mathrm{diag}\{d_{r+1},d_{r+2},\cdots,d_{r+s}\}$$
	
	于是$A$和$B$相抵,具有相同的初等因子组.现在记:
	$$d_1=p_1^{e_{11}}p_2^{e_{12}}\cdots p_k^{e_{1k}}$$
	$$d_2=p_1^{e_{21}}p_2^{e_{22}}\cdots p_k^{e_{2k}}$$
	$$\cdots$$
	$$d_{r+s}=p_1^{e_{r+s,1}}p_2^{e_{r+s,2}}\cdots p_k^{e_{r+s,k}}$$
	
	其中$e_{ij}$是非负整数,并且按照不变因子组的整除关系,有$0\le e_{1j}\le e_{2j}\le\cdots\le e_{rj}$,$0\le e_{r+1,j}\le e_{r+2,j}\le\cdots\le e_{r+s,j}$,其中$j=1,2,\cdots,k$.那么$A_1$的初等因子组为$p_j^{e_{ij}},e_{ij}>0,1\le i\le r,1\le j\le j$.$A_2$的初等因子组为$p_j^{e_{r+i,j}},e_{r+i,j}>0,1\le i\le sm1\le j\le k$.现在把$e_{1j},e_{2j},\cdots,e_{r+s,j}$从小到大重排为$e_{1j}',e_{2j}',\cdots,e_{r+s,j}'$.计算$B$的行列式因子,进而求出$B$的初等因子组,可以得到结论.
	
\end{proof}
\newpage
\subsection{线性方程组}

本节给出线性代数内容的第一个应用,会用所介绍的内容来处理线性方程组.关于线性方程组,比较有意义的是如下四个问题:如何判断方程组是否有解;有解的话解是否唯一;如果解唯一,这个解的表达式是什么;如果解不唯一,那么全体解构成的集合的结构是什么样的.

首先来把线性方程组问题转化为矩阵问题,取定一个交换环$R$,现在给出$R$上一个线性方程组,它有$m$个方程,未定元的个数是$n$,所有系数$a_{ij}$以及$b_i$都在$R$中.
$$\left\{\begin{array}{c}
a_{11}x_1+a_{12}x_2+\cdots+a_{1n}x_n=b_1\\
a_{21}x_1+a_{22}x_2+\cdots+a_{2n}x_n=b_2\\
\cdots\\
a_{m1}x_1+a_{m2}x_2+\cdots+a_{mn}x_n=b_m
\end{array}\right.$$

现在取$A=\left(\begin{array}{ccc}
a_{11}&\cdots&a_{1n}\\
\vdots&\ddots&\vdots\\
a_{m1}&\cdots&a_{mn}\end{array}\right),b=\left(\begin{array}{c}
b_1\\
\vdots\\
b_m\end{array}\right),x=\left(\begin{array}{c}
x_1\\
\vdots\\
x_n\end{array}\right)$,那么线性方程组可以转化为矩阵形式:
$$Ax=b$$

按照线性方程组的矩阵表示,立刻可以看出,矩阵有解的充要条件是,向量$b$落在系数矩阵$A$的列空间中.这便回答了何时有解的问题.

现在对于一般型的系数矩阵$A$,如果环是欧氏整环或者PID,来利用Smith标准型来回答如何判断线性方程组有解时,解是否唯一的问题.按照Smith标准型,有$MAN=diag\{d_1,\cdots,d_r,0\}$,于是可以把方程改写作:
$$MAN(N^{-1}x)=Mb;Mb=c,N^{-1}x=y\Rightarrow \left(\begin{array}{cccc}
d_1&\cdots&0&\textbf{0}\\
\vdots&\ddots&\vdots&\textbf{0}\\
0&\cdots&d_r&\textbf{0}\\
\textbf{0}&\cdots&\textbf{0}&\textbf{0}
\end{array}\right)
\left(\begin{array}{c}
y_1\\
y_2\\
\vdots\\
y_n
\end{array}\right)
=
\left(\begin{array}{c}
c_1\\
c_2\\
\vdots\\
c_m
\end{array}\right)$$

于是,方程有解当且仅当$d_i\mid c_i,1\le i\le r$,$c_{r+i}=0,1\le i\le m-r$,方程有唯一解当且仅当在这基础上还满足系数矩阵的秩等于列数,即列满秩,如果系数矩阵的秩小于列数,看到$y_{r+1},\cdots,y_n$可以取任意$R$中的数,于是此时的解的个数是不唯一的,如果$R$是无限环(特别的,无限域),那么方程组必然有无穷个解.

现在来讨论域上有唯一解时,唯一解的表达式.首先,对一个线性方程组$Ax=b$,对系数矩阵做行变换是不改变解的,因为行变换等价于左乘相应型的可逆矩阵.于是,如果有唯一解,我们可以通过行变换,把列满秩的$m\times n$的矩阵$A$,转化为上面是$n\times n$的可逆矩阵,下面是$(m-n)\times n$的0矩阵,(注意使用了$m\ge n$,事实上按照行秩=列秩,所以列满秩矩阵的行数必然不小于列数).那么线性方程组就可以舍弃下面$m-n$个“0”方程,于是转化为系数矩阵是可逆方阵的情况.

现在不妨设$A$是可逆方阵,那么线性方程组必然可以变为$x=A^{-1}b$,于是此时线性方程组有解并且解是唯一的.关于逆矩阵的求法介绍过两种,第一种是域的情况下,把可逆矩阵$A$表示为初等矩阵乘积,再挨个把初等矩阵取逆再反向相乘.第二种是一般交换环上借助代数余子式和相伴矩阵给出逆的直接构造.利用代数余子式可以得到所谓的Cramer法则,回答了解唯一情况下解的表达式问题:

给定交换环上$n$个未定元的$n$个线性方程构成的线性方程组$Ax=b$,如果$A$可逆,那么从$x=A^{-1}b=|A|^{-1}(A^ab)$得到唯一的解的表达式:
$$x_j=|A|^{-1}\left(\sum_{i=1}^{n}b_iA_{ij}\right)$$

关于域$k$上线性方程组的解的结构.给定线性方程组$Ax=b$,设$A$是$m\times n$型的,如果$b=0$,就称这是一个齐次线性方程组,如果$b$非0,就称为非齐次的.齐次线性方程组总是有解的,因为$x=0$就是解,那么全体解向量构成了一个$k^n$的子空间.称为矩阵$A$的解空间,解空间的基称为$A$上的基础解系.考虑矩阵的Smith标准型,设矩阵$A$的秩是$r$,看到此时$N^{-1}\varepsilon_{r+1},\cdots,N^{-1}\varepsilon_{n}$就是一组基础解系,于是,矩阵的解空间维数就是列数减去秩.(事实上这个结果已经在前面证明了,在那里我们先证明了同态的秩的一个关于秩和$\ker$维数的等式,然后同构的得到矩阵的情况).这就得到了齐次线性方程组解的结构.

现在考虑非齐次线性方程组$Ax=b$,称$Ax=0$是它对应的齐次方程组.如果非齐次方程组有解,它的解的结构是:特解+通解.即,取$Ax=b$的特解$x_0$,设$Ax=0$解空间$W$,那么$Ax=b$的全部解就是陪集$x_0+W$.至此回答了本节开篇中的全部问题.
\newpage
\subsection{相似关系}

在前文中,讨论了交换环上两个有限生成自由模$F,G$,同态模$\mathrm{Hom}_R(F,G)$和矩阵代数$M_{m,n}$之间的联系.最后利用Smith标准型完全描述了二者之间的联系.现在,来讨论一个固定的有限生成自由模$F$的自同态环$\mathrm{End}_R(F)$和矩阵环之间的联系.给定$F$上一个自同态$\alpha$,有时也把它称为线性算子.现在选取的基不再是从定义域$F$和值域$F$分别取一组基,现在要求这两组基选为固定一个.即,不希望两个$F$之间的同构关系依赖于基的选取.

那么在这种新的选取基的方式下,仍然有,每个自同态在选定基的情况下对应一个方阵,反过来一个方阵必然对应于某个自同态和某组基.和之前相抵的情况一样,需要先来讨论基变换下矩阵的变换.

$$\xymatrix{
	R^{\oplus n}\ar[dr]^{\phi}\ar[dd]_{\pi}&&&R^{\oplus n}\ar[dl]_{\phi}\ar[dd]^{\pi}\\
	&V\ar[r]^{f}&V&\\
	R^{\oplus n}\ar[ur]_{\psi}&&&R^{\oplus n}\ar[ul]^{\psi}
}$$

给定$V$上自同态$f$,取两组基$X=\{\varepsilon_1,\varepsilon_2,\cdots,\varepsilon_n\}$和$Y=\{\eta_1,\eta_2,\cdots,\eta_n\}$,设$f$在这两组基下的矩阵表示分别是$A,B$,于是有:
$$f(\varepsilon_1,\varepsilon_2,\cdots,\varepsilon_n)=(\varepsilon_1,\varepsilon_2,\cdots,\varepsilon_n)A$$
$$f(\eta_1,\eta_2,\cdots,\eta_n)=(\eta_1,\eta_2,\cdots,\eta_n)B$$


现在设基$Y$在基$X$下的矩阵表示是$P$,也就是说:
$$(\eta_1,\eta_2,\cdots,\eta_n)=(\varepsilon_1,\varepsilon_2,\cdots,\varepsilon_n)P$$

于是,就得到:
$$f(\eta_1,\eta_2,\cdots,\eta_n)=f(\varepsilon_1,\varepsilon_2,\cdots,\varepsilon_n)P
=(\varepsilon_1,\varepsilon_2,\cdots,\varepsilon_n)AP=(\eta_1,\eta_2,\cdots,\eta_n)P^{-1}AP$$

于是得到$B=P^{-1}AP$,即两个矩阵表示是相似的.

和相抵的情况一样,同一个矩阵并不一定是同一个线性算子的矩阵表示,倘若两个线性算子$f,g:V\to V$,存在$V$上的同构$h$满秩$h\circ f\circ h^{-1}=g$,称算子$f,g$是共轭的,共轭关系也是一个等价关系.那么容易看出$f,g$的矩阵表示同样是相似的.事实上,从群论角度看,这是$Aut_R(V)$到$End_R(V)$的共轭作用,这个作用的轨道类,和同阶方阵在共轭关系下的等价类,是一一对应的.于是,遇到了和处理$\mathrm{Hom}(E,F)$一样的情况:只有选取基才可以描述自同构,但是基的选取非常之多,为了理解哪些自同态在选取基的意义下是“本质上相同的”,就要处理方阵的相似关系.或着说,就要找出相似关系下的矩阵标准型.

在相抵关系下最终找到的标准型是一种对角矩阵,对角矩阵的确是一种形式非常简单的矩阵.但是在相似关系下,会看到矩阵并不总会相似于对角矩阵,于是,退一步,期望寻找类似于对角矩阵的形式作为标准型,即分块的对角矩阵.一个$n$阶方阵称为准对角矩阵或者分块对角矩阵,如果存在若干个正整数$n_1,n_2,\cdots,n_r$,它们的和是$n$,存在$n_i$阶方阵$A_i$,使得矩阵具有形式:
$$\left(\begin{array}{cccc}
A_1&0&\cdots&0\\
0&A_2&\cdots&0\\
\vdots&\vdots&\ddots&\vdots\\
0&0&\cdots&A_r\end{array}\right)$$

把上面准对角矩阵记作$\mathrm{diag}\{A_1,A_2,\cdots,A_r\}$.充当标准型的准对角矩阵,理应要求每一个小分块$A_i$不能成为若干更小的分块构成的准对角线.(剧透部分:会看到在代数闭域上,Jordan块就是要找的这种不能再细分的方阵,并且每个矩阵都可以相似为若干Jordan块构成的准对角矩阵.这种标准型称为Jordan标准型.)

映射和矩阵是同一个东西的两种描述,在前文中已经看到了很多命题都可以从这两个角度分别证明.于是一个自然的问题是,上述矩阵角度的准对角型,从映射角度是如何解释的?

给定有限生成自由模$V$上的一个自同态$f$,称$V$的子空间$W$是$f$的不变子空间,如果对每个$x\in W$,有$f(x)\in W$.那么不变子空间的一个用处是,映射$f$总可以限制到自身任一不变子空间上作为子空间上的自同态.现在如果取定$W$上的一组基$\{v_1,v_2,\cdots,v_r\}$,把它扩充为$V$上的基$\{v_1,v_2,\cdots,v_n\}$.在这组基下,按照不变子空间的定义,对$1\le i\le r$的每个$i$,有$f(v_i)$的线性表示中,$v_{r+1},\cdots,v_n$的系数都是0,于是得到了如下矩阵表示:
$$\left(\begin{array}{cc}
A&B\\
0&C\end{array}\right)$$

其中矩阵$A$就是$f$限制在$W$上的自同构在基$\{v_1,v_2,\cdots,v_r\}$下的矩阵表示.而$C$就是$f$限制在商空间$V/W$上的自同构在基$\{v_{r+1}+W,\cdots,v_n+W\}$下的矩阵表示.于是,看到自同态具有一个$r$维不变子空间,那么存在一组基下的矩阵表示是相应的准上三角矩阵.反过来,如果$f$在一组基$\{w_1,\cdots,w_n\}$下的矩阵是上述准上三角型矩阵,设$A$的阶数是$r$,那么我们看到$\{w_1,\cdots,w_r\}$生成了$V$的一个$f$不变子空间.综上,看到不变子空间对应于矩阵的准上三角型.

现在,如果矩阵相似于一个准对角矩阵设对角块$A_1,A_2,\cdots,A_s$的阶数分别是$n_1,n_2,\cdots,n_s$,那么设对应的自同态和基是$f$和$\{v_1,\cdots,v_n\}$,那么看到$W_1=\{v_1,\cdots,v_{n_1}\}$,$W_2=\{v_{n_1+1},\cdots,v_{n_1+n_2}\}$,类似定义到$W_s$.那么按照对角块上侧的项都是0,看到$W_1,\cdots,W_s$是$f$不变子空间,于是,这就对应了$V$的关于$f$的不变子空间的直和分解$V=W_1\oplus W_2\oplus\cdots\oplus W_s$.反过来,如果$V$存在关于自同态$f$的不变子空间的直和分解$V=W_1\oplus W_2\oplus\cdots\oplus W_s$,分别对$W_i$取基,按顺序排列构成$V$上的基,那么$f$在这组基下的矩阵表示就是一个相应的准对角矩阵.

综上,证明了:一个$n$阶矩阵相似于准对角矩阵,对角块的阶数分别是$n_1,n_2,\cdots,n_s$,$n_1+n_2+\cdots+n_s=n$,当且仅当,存在$V$上自同态$f$,和$f$的$s$个不变子空间$W_1,W_2,\cdots,W_s$,其中$W_i$的维数是$n_i$,使得有直和分解$V=W_1\oplus W_2\oplus\cdots\oplus W_s$.简单的说,矩阵的相似准对角化,等价于空间关于映射的不变子空间的直和分解.

于是,从映射角度处理相似标准型的思路是,将空间$V$分解为自同态的不变子空间的直和.并且,最好的这种分解理应要求空间分解中的不变子空间不可再分解.(剧透部分:会看到在代数闭域上,循环子空间就是要找的这种不可再分解的不变子空间,固定一个自同态时,空间总可以分解为若干循环子空间的直和.每个循环子空间对应于Jordan块.)

接下来要做的就是,从矩阵角度和映射角度,分别按照上述思路,探究代数闭域上方阵的相似标准型.为此,先来介绍一种重要的不变子空间:特征子空间.
\newpage
\section{相似标准型理论}
\subsection{特征值和特征向量}

给定线性空间$V$上的线性变换$f$,$f$在$V$向量上的最简单的作用是等同于数乘作用,也就是说存在一个域$F$中的元$\lambda$和向量$x\in V$,使得$f(x)=\lambda x$,注意$x=0$的情况是平凡的,这时候可以取任意实数$\lambda$,所以把这一情况划去.如果存在非0$x\in V$和域$F$中的元$\lambda$使得$f(x)=\lambda x$,就称$(\lambda,x)$是一个关于$f$的特征对,其中$\lambda$称为$f$的特征值,$x$称为关于特征值$\lambda$的特征向量.

首先,特征向量的意义在于它对应于一维不变子空间.一方面,如果$f$存在一个特征向量$x\in V$,那么记对应的特征值$\lambda$,会看到$(x)$是$V$的关于$f$的不变子空间,事实上$f(x)=\lambda x\in (x)$.另一方面,如果$f$存在一个一维的不变子空间$W$,任取$W$中的非0向量$x$,那么按照定义,$f(x)\in (x)$,于是存在一个$\lambda\in F$使得$f(x)=\lambda x$.

为了求线性映射的特征向量,来把上述定义转化为矩阵语言.如果线性空间$V$上的线性变换$f$存在一个特征对$(\lambda,v)$,任取一组基,设$f$在基下的矩阵表示是$A$,设$v$在基下的向量表示是$x$,于是$f(v)=\lambda v$在这组基下的矩阵表示是$Ax=\lambda x$.给方阵也定义特征值和特征向量:

给定域$F$上方阵$A$,如果存在非0向量$x$和域$F$中的元$\lambda$使得$Ax=\lambda x$,就称$(\lambda,x)$是一个特征对,其中$\lambda$称为$A$的一个特征值,$x$称为关于特征值$\lambda$的特征向量.

按照定义,考虑矩阵$A$的特征对$(\lambda,x_0)$,那么也就是说$\lambda x_0-Ax_0=0$,也就是说$x_0$是线性方程组$(\lambda E-A)x=0$的非0解.也就是说,一个数$\lambda$是$A$的特征值当且仅当$(\lambda E-A)x=0$有非0解,于是按照域上线性方程组的解法,$\lambda$是特征值当且仅当$\mid\lambda E-A\mid=0$.特别的,看到域上矩阵可逆当且仅当0不是特征值.

把$\lambda E-A$称为$A$的特征矩阵,它是$F[\lambda]$上的矩阵,这个矩阵的行列式是关于$\lambda$的多项式$f(\lambda)=\mid\lambda E-A\mid$,称为$A$的特征多项式.按照行列式定义拆开特征矩阵的行列式,看到最高次项是$\lambda^n$,于是特征多项式是矩阵的阶数次的一个首一多项式.特征值恰好就是特征多项式的全部根.而关于特征值$a$的全部特征向量就是$(aE-A)x=0$的全部非0解.
$$\left(\begin{array}{cccc}
\lambda-a_{11}&-a_{12}&\cdots&-a_{1n}\\
-a_{21}&\lambda-a_{22}&\cdots&-a_{2n}\\
\vdots&\vdots&\ddots&\vdots\\
-a_{n1}&-a_{n2}&\cdots&\lambda-a_{nn}\end{array}\right)$$

相似不改变矩阵的特征多项式.事实上$|\lambda E-A|=|P^{-1}||\lambda E-A||P|=|\lambda E-P^{-1}AP|$.于是,线性变换在不同基下的矩阵表示的特征多项式是相同的,它就称为线性变换的特征多项式,由此看出线性变换在不同基下矩阵表示的特征值也必然是相同的.但是注意特征向量的表示是会变的.

注意特征多项式相同并不是相似的充要条件,例如$\left(\begin{array}{cc}
0&1\\
0&0
\end{array}\right)$和二阶0矩阵.

特征值的存在性.首先对于一个代数闭域上的有限维线性空间,知道特征多项式肯定有根,这说明矩阵或者线性变换必然有特征值.但是指出,对于一个无限维线性空间,即便是代数闭域上,线性变换也未必有特征值.考虑$V=\{(a_1,a_2,\cdots),a_i\in C\}$,线性变换$A:(a_1,a_2,\cdots)\mapsto (0,a_1,a_2,\cdots)$,那么每个满足$Ax=\lambda x$的x必然有$x=0$.

如果矩阵$A$是上三角或者下三角矩阵,设对角元是$\{a_1,a_2,\cdots,a_n\}$,那么特征矩阵$\lambda E-A$也是一个上三角或者下三角矩阵,对角元是$\{\lambda-a_1,\lambda-a_2,\cdots,\lambda-a_n\}$,于是行列式可以直接算出,就是对角元的乘积,于是上三角或者下三角矩阵的特征多项式是$\prod_{k=1}^{n}(\lambda-a_k)$,其中$a_k$是全部对角元,于是上三角或者下三角矩阵的特征值就是对角元.

代数重数.特征值是特征多项式的根,把特征值在特征多项式中的重数称为它的代数重数.尽管域$F$上的特征值并不是总存在的,可以把矩阵放在$F$的代数闭包$\overline{F}$中考虑,这时候特征多项式是可裂的,于是在计重数意义下它存在恰好阶数个特征值,这时全部特征值在计重数意义下构成的集合称为矩阵或者对应线性变换的谱.

考虑特征多项式$f(t)=|tE-A|=t^n-a_1t^ {n-1}+a_2t^{n-2}-...+(-1)^na_n$,那么按照行列式定义,看到$a_i$是$A$的全部$i$阶主子式的和,特别的,全部一阶主子式就是全部对角元,它们的和就是矩阵对角元的和,这称为矩阵的迹,记作$\mathrm{tr}(A)$,而$n$阶主子式只有1个,就是矩阵的行列式$a_n=\det A$.另外按照韦达定理,迹就是全部(计重数)特征值的核,行列式就是全部(计重数)特征值的积.行列式的性质已经熟知,关于矩阵的迹,有等式$\mathrm{tr}(AB)=\mathrm{tr}(BA)$.

相似不改变迹和行列式.这可以从相似不改变特征多项式直接得出,也可以从迹和行列式自身的性质得出,例如$\mathrm{tr}(P^{-1}AP)=\mathrm{tr}((P^{-1}A)P)=\mathrm{tr}(P(P^{-1}A))=\mathrm{tr}(A)$.于是可以对线性映射定义迹和行列式.

称一个矩阵$A$的零化多项式是指多项式$p$满足$p(A)=0$,那么如果$p$是$A$的零化多项式,有对任意可逆矩阵$P$有$p(P^{-1}AP)=0$.称线性空间$V$上的线性变换$f$的零化多项式是一个多项式$p$,使得$p(f)$是0映射.那么共轭的线性映射也具有相同的零化多项式.另外,一个线性变换以$p$为零化多项式,当且仅当对任意基下的矩阵表示,$p$是这个矩阵的零化多项式.

矩阵的零化多项式必然存在.这是因为矩阵作为域$F$上线性空间是$n^2$维的,于是$E,A,\cdots,A^{n^2}$线性相关.另外,实际上特征多项式总是一个零化多项式.这是Cayley-Hamilton定理.
\begin{proof}

记$tE-A$的伴随矩阵为$B(t)$,那么有$(tE-A)B(t)=\det(tE-A)E$,注意这个等式说明$B$中元素最多是关于$t$的$n-1$次多项式,可记$B=\sum_{k=0}^{n-1}t^kB_k$,其中$B_k$是域$F$上的矩阵,再记特征多项式$\det(tE-A)=f(t)=t^n+t^{n-1}c_{n-1}+\cdots+c_0$,那么有:
\begin{align*}
f(t)E&=(tE_n-A)B(t)\\
&=(tE_n-A)\sum_{k=0}^{n-1}t^kB_k\\
&=t^nB_{n-1}+\sum_{k=1}^{n-1}t^k(B_{k-1}-AB_k)-AB_0
\end{align*}

比较等式两边$t^k$的系数,得到:
$$B_0-AB_1=c_1E_n$$
$$B_1-AB_2=c_2E_n$$
$$\cdots$$
$$B_{n-1}-AB_n=c_nE_n$$
$$B_{n-1}=E_n$$

这些等式说明:
$$f(A)=\sum_{k=0}^nc_kA^k=A^nB_{n-1}+\sum_{k=1}^{n-1}(A^kB_{k-1}-A^{k+1}B_k)-AB_0=0$$
\end{proof}

Cayley-Hamilton定理的上述证明还说明了矩阵$A$的伴随矩阵总可以表示为$A$的多项式.在证明中我们定义的$B(t)=\sum_{k=0}^{n-1}t^kB_k$是$(tE-A)$的伴随矩阵,于是$B(0)=B_0$就是$-A$的伴随矩阵,于是$A$的伴随矩阵是$(-1)^{n-1}B_0$.而反复代换证明中的那列等式就得到$B_0=c_1E_n+c_2A+\cdots+c_nA^{n-1}$.换句话说,如果$A$的特征多项式是$f(t)=t^n+c_{n-1}t^{n-1}+\cdots+t_0$,那么伴随矩阵$\mathrm{adj}(A)$恰好就是多项式$(-1)^{n-1}\frac{f(t)-t_0}{t}$带入$t=A$.

给定方阵或者线性变换,考虑$F[x]$中的全部零化多项式构成的理想$I$,那么,方阵对应的$I$和相应线性变换对应的$I$是一致的.按照$F[x]$是PID,理想$I$是主理想,可以记作$(d(x))$,这里$d(x)$可以取为首一的,那么这个$d(x)$就是全部零化多项式中,唯一的首一并且次数最小的那个,并且每个零化多项式都被它整除,称它为方阵或者线性变换的最小多项式.注意方阵的最小多项式和相应线性变换的最小多项式是一致的.特别的,看到最小多项式整除特征多项式.

最小多项式和特征多项式具有相同的根.事实上如果最小多项式$p(x)$不以某个特征值$\lambda$为根,那么$p(x)$和$(x-\lambda)$在域上互素,于是存在多项式$s(x),t(x)$满足$s(x)p(x)+t(x)(x-\lambda)=1$.带入矩阵$A$,得到$t(A)(A-\lambda E)=E$.这导致$A-\lambda E$可逆,于是和$\lambda$是特征值矛盾.

特征子空间.给定线性空间$V$上的线性变换$f$,如果它存在特征值$\lambda$,称$\lambda$的特征子空间是$V_{\lambda}=\{x\in V\mid f(x)=\lambda x\}$.那么特征子空间总是一个不变子空间.把特征子空间的维数称为对应特征值的几何重数.对偶的,对域$F$上$n$阶方阵$A$,如果它存在特征值$\lambda$,就把$F^n$的子空间$V_{\lambda}=\{x\in F^n\mid Ax=\lambda x\}$为$\lambda$的特征子空间,这个维数称为几何重数.那么线性变换的一个特征值$\lambda$的特征子空间,和它任意一个矩阵表示的特征值$\lambda$的特征子空间是同构的,于是二者几何重数相同.

已经定义了线性映射和矩阵的代数重数和几何重数,它们之间的数量关系是几何重数总是不超过代数重数的.为了证明这个命题,先来看线性无关特征向量和矩阵的联系.事实上特征向量的线性无关程度,对应于矩阵可对角化的程度.首先说明,特征子空间的和是直和:
\begin{proof}

设$A$的全部不同特征值是$\lambda_1,\lambda_2,\cdots,\lambda_r$,设$\lambda_i$的特征子空间记作$V_i$,那么为了证明$V_1+V_2+\cdots+V_r=\oplus_{k=1}^rV_k$.只要证明,如果有$x_i\in V_i$使得$x_1+x_2+\cdots+x_r=0$,那么全部$x_i\equiv0$.为此,注意到$A(x_1+x_2+\cdots+x_r)=0$得到$\lambda_1x_1+\lambda_2x_2+\cdots+\lambda_rx_r=0$,反复左乘$A$,得到:
$$\left\{\begin{array}{c}
x_1+x_2+\cdots+x_r=0\\
\lambda_1x_1+\lambda_2x_2+\cdots+\lambda_rx_r=0\\
\cdots\\
\lambda_1^{r-1}x_1+\lambda_2^{r-1}x_2+\cdots+\lambda_r^{r-1}x_r=0
\end{array}\right.$$

把全部$x_i$取第$j$个固定的分量,得到方程组:
$$\left\{\begin{array}{c}
x_{j1}+x_{j2}+\cdots+x_{jr}=0\\
\lambda_1x_{j1}+\lambda_2x_{j2}+\cdots+\lambda_rx_{jr}=0\\
\cdots\\
\lambda_1^{r-1}x_{j1}+\lambda_2^{r-1}x_{j2}+\cdots+\lambda_r^{r-1}x_{jr}=0
\end{array}\right.$$

其中系数矩阵的行列式是范德蒙行列式,按照$\lambda_i$两两不同得到这个行列式非0,于是,解出全部$x_{ji}$都是0,于是全部向量$x_i$都是0.这就说明了特征子空间的和是直和.
\end{proof}

于是,极大的由特征向量构成的线性无关组,可以先对每个特征子空间取基,再把它们并起来,这个线性无关组的个数刻画了特征向量的线性无关程度.现在把这个线性无关组记作$\{v_1,v_2,\cdots,v_k\}$,设$v_i$对应的特征值是$a_i$,那么,把这个线性无关组扩充成$F^n$上一组基$\{v_1,v_2,\cdots,v_n\}$,把它们作为列向量拼凑为一个可逆矩阵$P$,那么$P^{-1}AP$具有如下形式:
$$\left(\begin{array}{cc}
D&M\\
0&N
\end{array}\right),D=\mathrm{diag}\{a_1,a_2,\cdots,a_k\},M\in M_{k,n-k},N\in M_{n-k}$$

反过来,如果矩阵$A$相似于上述形式,那么取$P$的前$k$个列向量,它们就是线性无关的$A$的特征多项式.另外,矩阵相似于对角矩阵,当且仅当它存在阶数个线性无关的特征向量.于是,看出,特征向量的线性无关称度,描述的是矩阵的相似对角化程度.

现在利用这个内容来说明代数重数大于等于几何重数.
\begin{proof}

给定矩阵$A$的一个特征值$\lambda_0$,设几何重数为$r$,于是存在$r$个线性无关的关于$\lambda_0$的特征向量$\{v_1,v_2,\cdots,v_r\}$,把这个线性无关组扩充为$F^n$上的一组基$\{v_1,v_2,\cdots,v_n\}$,那么按照上述讨论,在这组基下,矩阵相似于:

$$\left(\begin{array}{cc}
D&M\\
0&N
\end{array}\right),D=\mathrm{diag}\{\lambda_0,\lambda_0,\cdots,\lambda_0\},M\in M_{k,n-k},N\in M_{n-k}$$

现在这个新的矩阵的特征多项式和$A$的特征多项式相同,而新的矩阵的特征多项式,按照定义是$\mid\lambda E-D\mid\bullet\mid\lambda E-N\mid$.其中$\mid\lambda E-D\mid=(\lambda-\lambda_0)^r$,于是,必然得到$\lambda_0$在特征多项式的重数是大于等于$r$的,完成证明.
\end{proof}

对于特征值$\lambda$,比较特殊的情况便是它的几何重数等于代数重数,此时称特征值$\lambda$是半单的.那么如果矩阵$A$的特征多项式可裂,并且每个特征值都是半单的,则它就有阶数个线性无关的特征向量,于是此时矩阵是可对角化的,反过来按照几何重数不超过代数重数,说明如果矩阵可对角化,那么它的特征多项式可裂并且每个特征值都是半单的.即我们证明了:一个矩阵可对角化当且仅当它的特征多项式可裂并且每个特征值都是半单的.

特征值$\lambda$半单有如下等价描述:
\begin{enumerate}
	\item $\lambda$的几何重数和代数重数相等.
	\item $\lambda$的根子空间和不变子空间相同.
	\item $\ker(A-\lambda E)=\ker(A-\lambda E)^2$.
	\item $\mathrm{im}(A-\lambda E)=\mathrm{im}(A-\lambda E)^2$.
	\item $\ker(A-\lambda E)\cap\mathrm{im}(A-\lambda E)=\{0\}$.
\end{enumerate}

关于最后这个等价描述我们做一个注解,首先给定矩阵$A$的一个不变子空间,通常来讲$A$不变的补空间是不唯一的.但是对于$A$的属于特征值$\lambda$的特征子空间,它的不变补空间如果存在则只能是$\mathrm{im}(A-\lambda E)$.这个情况出现当且仅当$\lambda$是半单的.

我们接下来说明矩阵可对角化当且仅当它的最小多项式是不同的一次因式的乘积.我们会给出两个角度的证明.
\begin{proof}

如果从Jordan标准型角度入手,对角矩阵也是Jordan型矩阵,于是一个矩阵可对角化等价于说,它的Jordan标准型中,只有一阶的Jordan块.而Jordan标准型中一个特征值出现的Jordan块的最高次数就是最小多项式中这个特征值对应的一次项的次数(也就是几何重数),说明矩阵可对角化等价于说最小多项式由不同一次因式的乘积构成.

不过这个事实还可以避开Jordan标准型来说明.首先如果矩阵可对角化,那么可以验证由不同特征值的一次因式乘积构成的多项式就是零化多项式,按照特征多项式和最小多项式有相同的根,说明这个零化多项式次数不能更小了,于是它就是最小多项式.现在反过来,需要说明如果记方阵$A$的不同特征值为$\{a_1,a_2,\cdots,a_d\}$,如果$f(x)=(x-a_1)(x-a_2)\cdots(x-a_d)$是$A$的极小多项式,就有$A$可对角化.为此,注意到$(A-a_1E)(A-a_2E)\cdots(A-a_dE)=0$.设$A-a_iE$的秩是$r_i$,那么$a_i$的几何重数就是$n-r_i$,于是$A$的由特征向量构成的极大的线性无关组的向量个数是$s=dn-\sum_{i=1}^{d}r_i$.现在利用一个关于秩的引理,如果$AB=0$那么$r(A)+r(B)\le n$,归纳得到$\sum_{i=1}^{d}r_i\le(d-1)n$,于是得到$s\ge n$,但是$s$必然不能超过阶数$n$,于是$s=n$,于是$A$可对角化.
\end{proof}

可对角化的另一空间角度的等价描述.若域$k$上的矩阵$A$的特征多项式可裂,那么$A$可对角化当且仅当对$A$的每一个$k^n$的不变子空间$W$,存在$A$不变的补空间$U$,即有$k^n=W\oplus U$.
\begin{proof}
	
	假设$A$可对角化,任取一个$A$不变子空间$W$,则限制变换$A|W$也满足$A$的极小多项式,于是限制变换$A|W$的极小多项式是不同一次因式的乘积,这说明$A|W$同样是可对角化的变换.于是可取$W$中$\dim W$个线性无关的特征向量构成的向量组$S$.我们断言$S$可以扩充为由$A$的特征向量构成的一组全空间上的基.事实上对每个特征值$\lambda$,考虑$\ker(A-\lambda E)$,如果$S$中存在若干特征向量包含于核,则把它扩充为核空间的一组基,把添加的特征向量加进$S$中;如果$S$与这个核空间的交是空集,那么把这个核空间的任一组基添加到$S$中.按照$A$的每个特征值的几何重数等于代数重数,这就说明这样操作后$S$会扩充为由$n$个特征向量构成的线性无关组.于是这些新添加的特征向量生成的子空间是$A$不变子空间,并且它是$W$的补空间.
	
	反过来假设$A$的不变子空间总有不变补空间.为证明$A$可对角化,只需证明对$A$的每个特征值$\lambda$,记$B=A-\lambda E$,有$\ker B\cap\mathrm{im}B=\{0\}$.为此先任取$\ker B$的不变补空间$W$,则对任意$x\in k^n$有唯一分解$x=y+z$,其中$y\in\ker B,z\in W$.那么得到$Bx=Bz\in W$.于是得到$\mathrm{im}B\subset W$,但是按照$\dim W=n-\dim\ker B=\dim\mathrm{im}B$,说明$W=\mathrm{im}B$.于是得到$\ker B\cap\mathrm{im}B=\{0\}$.完成证明
\end{proof}

总结下矩阵$A$可对角化的等价描述:
\begin{enumerate}
	\item $A$存在阶数个线性无关的特征向量.
	\item $A$的特征多项式可裂,并且每个特征值都是半单的.
	\item $A$的极小多项式是不同一次因式的乘积.
	\item $A$的特征多项式可裂,并且每个$A$不变子空间都有$A$不变的补空间.
\end{enumerate}

最后一个等价描述允许我们合理的猜测对于具备正交性的空间(实复线性空间)上,把不变补空间改为不变正交补空间就会得到特殊的相似对角化.我们会在后文给出复矩阵酉对角化当且仅当是正规矩阵,而正规矩阵恰好有这样的等价描述:矩阵是复正规矩阵当且仅当对每个不变子空间,它的正交补也是不变子空间;实的情况是实矩阵正交相似对角化当且仅当它是特征多项式可裂的实正规矩阵,而实正规矩阵等价于对每个不变子空间,它的正交补也是不变子空间.

上三角化.称域上一个矩阵是可上三角化的,如果它在这个域上相似于上三角矩阵.上三角化的描述相对复杂,矩阵$A$相似于如下矩阵,当且仅当空间上存在$k$个线性无关的向量$\alpha_i,1\le i\le k$,使得对每个$1\le j\le k$,有$\alpha_1,\cdots,\alpha_j$生成了$A$的一个不变子空间,特别的,$A$相似于上三角矩阵当且仅当上述描述的$k$可以取$n$.
$$\left(\begin{array}{cc}
D&M\\
0&N
\end{array}\right),D\text{是一个上三角矩阵},M\in M_{k,n-k},N\in M_{n-k}$$

可上三角化具有一个更简单的等价描述.矩阵可上三角化当且仅当特征多项式在所在的域上分裂.如果一个矩阵可以上三角化,那么它必然在所在域上具有阶数个(计重数)特征值,也就是特征多项式在域上能写成一次多项式的乘积.反过来,来对阶数归纳.如果矩阵是一阶的,那么没什么需要证明的.现在假设对$n-1$阶矩阵是成立的.现在取$n$阶的特征多项式分裂的矩阵$A$,按照特征多项式是分裂的,于是它至少存在一个特征值$\lambda_0$,于是可以取对应的特征向量$v_1$,把$v_1$扩充为一组$F^n$上的基$\{v_1,v_2,\cdots,v_n\}$.把这个向量组作为列向量组,构成一个矩阵$P$,于是我们得到了:
$$P^{-1}AP=\left(\begin{array}{cc}
\lambda_0&\beta\\
\textbf{0}&A_1\end{array}\right)$$

现在,知道$A$的特征多项式和$P^{-1}AP$是一样的,于是得到$(\lambda-\lambda_0)\mid\lambda E_{n-1}-A_1\mid$是可裂的多项式,这就保证了$A_1$的特征多项式也是可裂的.于是,按照归纳假设,存在$n-1$阶可逆矩阵$Q_1$使得$Q_1^{-1}A_1Q_1$是上三角矩阵,现在取$Q$是分块对角的可逆矩阵$\mathrm{diag}\{1,Q_1\}$,那么得到:
$$Q^{-1}P^{-1}APQ=\left(\begin{array}{cc}
1&\textbf{0}\\
\textbf{0}&Q_1^{-1}\end{array}\right)\left(\begin{array}{cc}
\lambda_0&\beta\\
\textbf{0}&A_1\end{array}\right)\left(\begin{array}{cc}
\lambda_0&\textbf{0}\\
\textbf{0}&Q_1\end{array}\right)=\left(\begin{array}{cc}
\lambda_0&\beta Q_1\\
\textbf{0}&Q_1^{-1}A_1Q_1\end{array}\right)$$

这是一个上三角矩阵,和$A$相似,完成归纳!

给定域$F$上的多项式$f$,给定矩阵$A$,利用上三角化可以求出$f(A)$的全部特征值.我们把矩阵$A$放在$F$的代数闭域中,此时特征多项式可裂,于是$A$在代数闭域中可以相似上三角化,那么,如果设对角元依次是$\{a_1,a_2,\cdots,a_n\}$,直接看出$f(A)$是对角元为$\{f(a_1),f(a_2),\cdots,f(a_n)\}$的上三角矩阵,于是$f(A)$在原来域中的全部(计重数)特征值就是$f(a_i)$划去不在域$F$中的元剩下的那些数.

按照可上三角化的上述等价描述,特征多项式可裂.看到分块上三角矩阵是可上三角化的等价于每个对角分块都是可上三角化的,因为特征多项式可裂当且仅当全部对角块的特征值可裂.这个结论对可对角化也是成立的,即一个分块对角矩阵是可对角化的当且仅当每个对角块都是可对角化的.这个命题可以利用可对角化当且仅当极小多项式是不同一次因式乘积直接得出,这里我们给出另一个角度的证明.
\begin{proof}

充分性是直接的.现在证明必要性,设矩阵$B=\mathrm{diag}\{B_1,B_2,\cdots,B_d\}$是可对角化的.需要证明每个$B_i$都是可对角化的.当$d=1$时没什么可证的.现在只要证明$d=2$时候是成立的,就可以依次说明每个$B_i$都是可对角化的.设$B_1$的阶数是$n$,设$B_2$的阶数是$m$.按照条件,存在可逆矩阵$S$使得$S^{-1}BS=D=\mathrm{diag}\{\lambda_1,\cdots,\lambda_{m+n}\}$.把等式写作$BS=SD$,再把$S$分解为列向量组$S=(s_1,s_2,\cdots,s_{m+n})$.再把每个列向量$s_i$分解为$\left(\begin{array}{c}
\xi_i\\
\eta_i\end{array}\right)$,其中$\xi_i\in F^n$,$\eta_i\in F^m$.于是从$BS=SD$得到:$B_1\xi_i=\lambda_i\xi_i$和$B_2\eta_i=\lambda_i\eta_i$,其中$i=1,2,\cdots,n+m$.现在注意到$\{\xi_1,\xi_2,\cdots,\xi_{m+n}\}$凑成一个秩为$n$的矩阵,说明它包含了$n$个$B_1$的特征向量构成的线性无关组,于是$B_1$是可对角化的.同理从$\{\eta_1,\cdots,\eta_{m+n}\}$包含了$m$个$B_2$的特征向量构成的线性无关组,得到$B_2$可对角化,完成证明.
\end{proof}
\newpage
\subsection{矩阵角度}

定义对角元为$\lambda$的$k$阶Jordan块$J_k(\lambda)$为如下的矩阵.那么一阶Jordan块$J_1(\lambda)$就是一个$1\times1$的矩阵$(\lambda)$.称一个Jordan型矩阵是指对角块是Jordan块的准对角型矩阵.
$$J_k(\lambda)=\left(\begin{array}{ccccc}
\lambda&1&&&\\
&\lambda&1&&\\
&&\ddots&\ddots&\\
&&&\lambda&1\\
&&&&\lambda\end{array}\right)$$

通过如下四步骤证明代数闭域上,Jordan型矩阵在不计Jordan块次序的意义下是矩阵相似关系的标准型.
\begin{enumerate}
	\item 矩阵相似于上三角形.
	\item 上三角形矩阵相似于分块对角矩阵,其中每个对角块是上三角矩阵,对角元是相同的特征值,不同对角块的对角元不同.
	\item 每个对角元相同的上三角矩阵相似于Jordan型矩阵.
	\item 给定两个Jordan型矩阵,如果对角块不存在一种排列使得两个矩阵相同,那么它们不相似.
\end{enumerate}

第一步已经证明过了,事实上证明了一个矩阵相似上三角化当且仅当它的特征多项式是可裂的,在代数闭域中这自然是成立的.

现在开始处理第二步.首先来说明,一个上三角矩阵,如果重新约定对角元的一种排列,那么必然存在和这个矩阵相似的一个上三角矩阵,它的对角元的排列是所预先设定的.事实上,按照排列总被对换生成,只要证明如果上三角矩阵$A$的对角元排列为$\{a_1,a_2,\cdots,a_n\}$,那么任取取$1\le i<j\le n$,那么存在和$A$相似的上三角矩阵,对角元的排列是$\{a_1,\cdots,a_j,\cdots,a_i,\cdots,a_n\}$.为此,只要取把$i,j$行交换的初等矩阵$Q$,注意到$Q^{-1}=Q$,此时就有$Q^{-1}AQ$是所求形式.

按照刚刚所证的,任取上三角矩阵$A$,设它的全部不同的特征值是$\lambda_1,\cdots,\lambda_s$,其中$\lambda_i$的代数重数记作$n_i$,这也等价于说$\lambda_i$在对角线上出现了$n_i$次.那么矩阵$A$相似于如下准上三角矩阵,其中$T_{ii}$是对角元为$\lambda_i$的$n_i$阶方阵,并且$\lambda_i$两两不同:
$$A_0=\left(\begin{array}{cccc}
T_{11}&T_{12}&\cdots&T_{1s}\\
0&T_{22}&\cdots&T_{2s}\\
\vdots&\vdots&\ddots&\vdots\\
0&0&\cdots&T_{ss}\end{array}\right)$$

断言实际上$A_0$相似于准对角矩阵$D=\mathrm{diag}\{T_{11},T_{22},\cdots,T_{ss}\}$.为此需要Sylvester定理:给定域$F$上的两个特征多项式可裂的矩阵$A,B$,它们的阶数分别是$n,m$,那么对任意$C\in M_{n,m}$,$AX-XB=C$有唯一解$X\in M_{n,m}$,当且仅当$A,B$没有公共特征值.

\begin{proof}
	
	给定矩阵$A,B,C$,把$X$的项看作$mn$个未知数,那么矩阵方程$AX-XB=C$是一个线性方程组,于是它有唯一解当且仅当对应的齐次线性方程组$AX-XB=0$只有零解.现在设$A,B$的特征多项式分别为$P_A(x)$和$P_B(x)$,假设$A,B$没有公共特征值.按照Cayley-Hamilton定理,有$p_B(B)=0$,按照$AX=XB$得到$p_B(A)X=Xp_B(B)=0$.于是为了证明$X$只能为0,只要说明$p_B(A)$可逆.为此,我们记$p_B(x)=(x-t_1)(x-t_2)\cdots(x-t_n)$,那么如果存在某个$A-t_iE$不可逆,就得到$t_i$是$A$的特征值,这说明存在公共特征值和条件矛盾.于是每个$A-t_iE$可逆,于是得到$p_B(A)$可逆.
	
	现在反过来,如果$AX=XB$有非0解,于是得到$p_B(A)X=0$有非0解,于是$p_B(A)$是不可逆矩阵,于是某个$A-t_iE$不可逆,于是$t_i$就是公共特征值.
	
\end{proof}

现在来证明$A_0$相似于准对角矩阵$D$.把矩阵$A_0$记作$\left(\begin{array}{cc}
T_{11}&Y\\
0&S_2\end{array}\right)$,其中$S_2=(T_{ij}),2\le i,j\le s$,$T_{11}$唯一的特征值是$\lambda_1$,于是$T_{11}$和$S_2$没有公共特征值.按照Sylvester定理,矩阵方程$T_{11}X-XS=-Y$有唯一解$X$,取矩阵$M=\left(\begin{array}{cc}
I_{n_1}&X\\
0&I\end{array}\right)$,那么$M^{-1}=\left(\begin{array}{cc}
I_{n_1}&-X\\
0&I\end{array}\right)$,于是$M^{-1}A_0M=\left(\begin{array}{cc}
T_{11}&0\\
0&S_2\end{array}\right)$.接下来只要对$s$归纳,就得到$A_0$和$D$相似.这就完成了第二步.

第三步,要证明上述$D$中的每个对角元固定为$\lambda_i$的上三角对角块$T_{ii}$,都会相似于若干$J_r(\lambda_i)$作为对角块构成的准对角矩阵.为此,只要证明对角元为0的上三角矩阵满足就可以了,因为每个$T_{ii}=\lambda_i E+R_i$,其中$R_i$是对角元为0的上三角矩阵,那么,只要证明了$R_i$相似于若干$J_r(0)$构成的准对角矩阵,那么立刻看到在相同的过渡矩阵下,$T_{ii}$相似于对应的若干$J_r(\lambda_i)$构成的准对角矩阵.称对角元为0的Jordan块是幂零Jordan块,它们有如下简单结论,其中第二条也是它叫幂零Jordan块的原因:
$$J_k^T(0)J_k(0)=diag\{0,I_{k-1}\};J_k(0)^p=0,p\ge k$$

设$A\in M_n$为对角元为0的上三角矩阵,那么存在可逆矩阵$S\in M_n$,使得存在整数$n_1,n_2,\cdots,n_m$满足$n_1\ge n_2\ge\cdots\ge n_m\ge1$,并且$n_1+n_2+\cdots+n_m=n$.满足:
$$S^{-1}AS=\mathrm{diag}\{J_{n_1}(0),J_{n_2}(0),\cdots,J_{n_m}(0)\}$$

\begin{proof}
	
	对阶数$n$归纳.如果$n=1$这时候没什么好证的,现在假设阶数小于$n$的情况都已经得证,来看$n$的情况.设$A$具有如下形式:
	$$\left(\begin{array}{cc}
	0&a^T\\
	0&A_1
	\end{array}\right)$$
	
	其中$A_1\in M_{n-1}$是对角元为0的上三角矩阵,那么按照归纳假设,存在一个可逆矩阵$S_1$,存在整数$k_1\ge k_2\ge\cdots\ge k_s\ge1$满足$k_1+k_2+\cdots+k_s=n-1$,使得$S_1^ {-1}A_1S_1=J_0=diag\{J_{k_1}(0),J_{k_2}(0),\cdots,J_{k_s}(0)\}$
	是Jordan矩阵.那么有:
	$$\left(\begin{array}{cc}
	1&0\\
	0&S_1^{-1}
	\end{array}\right)A\left(\begin{array}{cc}
	1&0\\
	0&S_1
	\end{array}\right)=\left(\begin{array}{cc}
	0&a^TS_1\\
	0&J
	\end{array}\right)$$
	
	现在把$a^TS_1$划分为$(a_1^T,a_2^T)$,分别是$F^{k_1}$和$F^{n-k_1-1}$的元,于是上面矩阵化为:
	$$\left(\begin{array}{ccc}
	0&a_1^T&a_2^T\\
	0&J_{k_1}&0\\
	0&0&J
	\end{array}\right)$$
	
	现在考虑相似:
	$$\left(\begin{array}{ccc}
	0&a_1^T&a_2^T\\
	0&J_{k_1}&0\\
	0&0&J
	\end{array}\right)\left(\begin{array}{ccc}
	1&-a_1^TJ_{k_1}^T&0\\
	0&I&0\\
	0&0&I
	\end{array}\right)\left(\begin{array}{ccc}
	0&a_1^T&a_2^T\\
	0&J_{k_1}&0\\
	0&0&J
	\end{array}\right)=\left(\begin{array}{ccc}
	0&a_1^Te_1e_1^T&a_2^T\\
	0&J_{k_1}&0\\
	0&0&J
	\end{array}\right)$$
	
	其中运用了$(I-J_k^TJ_k)x=x^Te_1e_1^T$.现在分两种情况,$a_1^Te_1=0$和$a_1^Te_1\not=0$.
	
	如果$a_1^Te_1\not=0$,那么有相似:
	$$\left(\begin{array}{ccc}
	1/a_1^Te_1&0&0\\
	0&I&0\\
	0&0&(1/a_1^Te_1)I
	\end{array}\right)\left(\begin{array}{ccc}
	0&a_1^Te_1e_1^T&a_2^T\\
	0&J_{k_1}&0\\
	0&0&J
	\end{array}\right)\left(\begin{array}{ccc}
	a_1^Te_1&0&0\\
	0&I&0\\
	0&0&(a_1^Te_1)I
	\end{array}\right)=\left(\begin{array}{ccc}
	0&a_1^Te_1e_1^T&a_2^T\\
	0&J_{k_1}&0\\
	0&0&J
	\end{array}\right)$$
	$$=\left(\begin{array}{cc}
	J'&e_1a_2^T\\
	0&J
	\end{array}\right)$$
	
	其中$J=\left(\begin{array}{cc}
	0&e_1^T\\
	0&J_{k_1}
	\end{array}\right)=J_{k_1+1}(0)$.下面按照$J'e_{i+1}=e_i$,$i=1,2,\cdots,k_1$,得到:
	$$\left(\begin{array}{cc}
	I&e_2a_2^T\\
	0&I
	\end{array}\right)\left(\begin{array}{cc}
	J'&e_1a_2^T\\
	0&J
	\end{array}\right)\left(\begin{array}{cc}
	I&-e_2a_2^T\\
	0&I
	\end{array}\right)=\left(\begin{array}{cc}
	J'&e_2a_2^TJ\\
	0&J
	\end{array}\right)$$
	
	下面反复利用如下相似:
	$$\left(\begin{array}{cc}
	I&e_{i+1}a_2^TJ^{i-1}\\
	0&I
	\end{array}\right)\left(\begin{array}{cc}
	J'&e_1a_2^TJ^{i-1}\\
	0&J
	\end{array}\right)\left(\begin{array}{cc}
	I&-e_{i+1}a_2^TJ^{i-1}\\
	0&I
	\end{array}\right)=\left(\begin{array}{cc}
	J'&e_2a_2^TJ^i\\
	0&J
	\end{array}\right)$$
	
	按照$J^{k_1}=0$,于是至多$k_1$步,得到$A$相似于$\mathrm{diag}\{J',J\}$.
	
	如果$a_1^Te_1=0$.已经把矩阵相似于:
	$$\left(\begin{array}{ccc}
	0&0&a_2^T\\
	0&J_{k_1}&0\\
	0&0&J
	\end{array}\right)$$
	
	它置换相似于:
	$$A_1=\left(\begin{array}{ccc}
	J_{k_1}&0&0\\
	0&0&a_2^T\\
	0&0&J
	\end{array}\right)$$
	
	那么按照归纳假设,存在$n-k$阶可逆矩阵$S_2$使得$S_2^{-1}\left(\begin{array}{cc}
	0&a_2^T\\
	0&J
	\end{array}\right)S_2$相似于Jordan型矩阵.于是得到$A_1$相似于Jordan型矩阵.综上得到对角元为0的上三角矩阵相似于Jordan型矩阵,但是对角块的排列未必是所预先设定的,为此只要做置换相似即可.综上完成证明.
	
\end{proof}

于是,得到了第三步的一般情况:给定域$F$上的特征多项式可裂的$n$阶矩阵$A$,那么存在可逆矩阵$S\in M_n$,和$q$个正整数$n_1,n_2,\cdots,n_q$使得$n_1+n_2+\cdots+n_q=n$,和$\lambda_1,\cdots,\lambda_q\in F$,满足如下等式:
$$S^{-1}AS=\left(\begin{array}{ccc}
J_{n_1}(\lambda_1)&&\\
&\ddots&\\
&&J_{n_q}(\lambda_q)\end{array}\right)$$

最后来处理第四步,即矩阵的相似关系被Jordan块不计顺序的意义下唯一决定.为此,来说明对矩阵$A$的每个特征值$\lambda$,它的$s$阶Jordan块的个数是被矩阵$A$决定的.

对$n$阶矩阵$A$,设$r_k(A,\lambda)=\mathrm{rank}(A-\lambda E)^k$,并且约定$r_0(A,\lambda)=n$.那么首先,如果$\lambda$不是$A$的特征值,总有$r_k(A,\lambda)=n,\forall k\ge0$.现在设$\lambda$是$A$的特征值.如果$A$相似于Jordan型矩阵$J$,就有:
$$J-\lambda E=\mathrm{diag}\{J_{m_1}(0),\cdots,J_{m_p}(0), J'-\lambda E\}$$

其中$J'-\lambda E$是一个可逆矩阵.于是$(A-\lambda E)^k$相似于:
$$(J-\lambda E)^k=\mathrm{diag}\{J_{m_1}^k(0),\cdots,J_{m_p}^k(0),(J-\lambda E)^k\}$$

按照秩的定义,看到:
$$r_k(A,\lambda)=\mathrm{rank}J_{m_1}^k(0)+\cdots+\mathrm{rank}J_{m_p}(0)+\mathrm{rank}(J'-\lambda E)^k$$

按照Jordan块的幂零性质,看到:
$$\mathrm{rank}J_l^ {k-1}(0)-\mathrm{rank}J_l^k(0)=\left\{\begin{array}{c}
1,k\le l\\
0,k>l\end{array}\right.$$

其中$J_l^0(0)$约定为单位矩阵,于是秩是$l$.现在定义定义$w_k(A,\lambda)=r_{k-1}(A,\lambda)-r_k(A,\lambda)$.那么首先,如果$\lambda$不是$A$的特征值,就有$w_k(A,\lambda)=0,k=1,2,\cdots,n$.当$\lambda$是$A$的特征值的时候, $w_k(A,\lambda)$表示的是属于特征值$\lambda$的那些阶数至少为$k$的Jordan块的个数.称矩阵$A$的关于$\lambda$的Weyr特征为$w(A,\lambda)=(w_1(A,\lambda),\cdots,w_q(A,\lambda)),\cdots)$.对于特征值$\lambda$,也把Weyr特征简单写作前有限个非0数,即$w(A,\lambda)= (w_1(A,\lambda),\cdots,w_q(A,\lambda)),w_{q+1}(A,\lambda)=0$.

可以给出Weyr特征的若干观察.首先Weyr特征列是不增的.并且$w_1(A,\lambda)=n-\mathrm{rank}(A-\lambda E)$就是属于$\lambda$的全部Jordan块的个数.另外属于特征值$\lambda$的Jordan块中阶数最大的$q$,恰好就是使得$\mathrm{rank}(A-\lambda E)^k=\mathrm{rank}(A-\lambda)^{k+1}$的最小正整数$q$,它也就是关于特征值$\lambda$的Weyr特征的长度.另外关于特征值$\lambda$的$k$阶Jordan块的个数就是:
$$w_k(A,\lambda)-w_{k+1}(A,\lambda)=\mathrm{rank}(A-\lambda E)^{k+1}+\mathrm{rank}(A-\lambda E)^{k-1}-2\mathrm{rank}(A-\lambda E)^{k}$$

于是每个阶数的Jordan块的个数被矩阵$A$自身决定,至此完成了第四步的证明.并且,两个矩阵相似当且仅当它们关于每一个数的Weyr特征是相同的,也可以说两个矩阵相似当且仅当它们具有相同的特征值并且每个特征值的Weyr特征列相同.

矩阵$A$关于特征值$\lambda$的Segre特征是指把所属全部Jordan块的阶数按照从大到小排列,记作$s_k(A,\lambda)$.举例来讲,如果$A$的关于$\lambda$的全部Jordan块的阶数分别是$1,2,3,2,3,2$,那么Segre特征是$\{3,3,2,2,2,1\}$
$$s_1(A,\lambda)\ge s_2(A,\lambda)\ge\cdots\ge s_{w_1(A,\lambda)}(A,\lambda)>0$$

有时也约定$s_k(A,\lambda)=0$,当$k>w_1(A,\lambda)$,$\lambda$是$A$的一个特征值.Weyr特征和Segre特征可以互相推出.于是可以说两个代数闭域上的矩阵相似当且仅当它们具有相同的特征值并且对于每一个特征值的Segre特征列相同.

最小多项式.知道相似不改变矩阵的最小多项式,而一个准对角矩阵的最小多项式必然是对角块的最小多项式的最小公倍式.但是知道那些对角元为固定的特征值$\lambda$的对角块构成的准对角矩阵,它的最小多项式就是$(x-\lambda)^r$,其中$r$是属于特征值$\lambda$的Jordan块的阶数的最大值.于是,看到最小多项式的唯一分解中,$(x-\lambda)$的次数,就是属于这个特征值的Jordan块的阶数最大元.

现在来讨论实矩阵的Jordan标准型.首先给定一个实矩阵,它的特征多项式是一个实多项式,于是它的非实数特征值一定共轭的成对出现.另外,注意到$rank(A-\lambda E)^k=rank\overline{(A-\lambda E)^k}=rank(A-\overline{\lambda} E)^k$,于是共轭的两个复特征值的Segre特征相同,也就是说它们对应的Jordan块是一致的.

于是考虑实矩阵$A$的Jordan标准型,如果它存在非实数特征值$\lambda$,那么必然可以写作若干如下矩阵拼凑而成的准对角型:diag$\{J_k(\lambda),J_k(\overline{\lambda})\}$.而这个矩阵可以置换相似于如下形式:
$$\left(\begin{array}{ccccc}
D(\lambda)&E_2&&&\\
&D(\lambda)&E_2\\
&&\ddots&\ddots&\\
&&&\ddots&E_2\\
&&&&D(\lambda)
\end{array}\right);D(\lambda)=\mathrm{diag}\{\lambda,\overline{\lambda}\}$$

注意到$D(\lambda)$相似于$C(a,b)=\left(\begin{array}{cc}
a&b\\
-b&a
\end{array}\right)$,于是上述矩阵可以相似于:

$$C_k(a,b)=\left(\begin{array}{ccccc}
C(a,b)&E_2&&&\\
&C(a,b)&E_2\\
&&\ddots&\ddots&\\
&&&\ddots&E_2\\
&&&&C(a,b)
\end{array}\right)\in M_{2k}(R)$$

结合无限域扩域不改变相似关系(见有理标准型一节),得到实矩阵的Jordan标准型结论:给定实矩阵$A$,它相似于如下准对角型矩阵,其中$\lambda_k=a_k+ib_k$是$A$的非实数特征值,$\mu_i$是$A$的实特征值,每个$C_{n_k}(a_k,b_k)$对应于一对共轭Jordan块$J_ {n_k}(\lambda_k),J_{n_k}(\overline{\lambda_k})$.
$$\mathrm{diag}\{C_{n_1}(a_1,b_1),\cdots,C_{n_p}(a_p,b_p),J_{m_1}(\mu_1),\cdots,J_{m_q}(\mu_q)\}$$

借助实Jordan标准型结论,可以得到一个复矩阵$A$相似于某个实矩阵的充要条件:它的特征多项式是实的,并且一对共轭复特征值的Weyr特征相同.
\newpage
\subsection{空间角度}

在前文中已经解释了,从空间理论角度探究相似标准型,需要把全空间分解为关于一个线性变换的不变子空间的直和.

首先容易想到取特征子空间是远远不够的,这是因为特征子空间的维数是几何重数,它是可以严格小于代数重数的,这就导致直和的维数可能小于整个全空间的维数.将会用根子空间来代替循环子空间,对每个特征值$\lambda$,有子空间升链:
$$\ker(f-\lambda\mathrm{Id})\subset\ker(f-\lambda \mathrm{Id})^2\subset\ker(f-\lambda\mathrm{Id})^3\subset\cdots$$

定义特征值$\lambda$的根子空间是这个升链的并:
$$W_{\lambda}=\cup_{k\ge1}\ker(f-\lambda\mathrm{Id})^k=\{x:\exists k\in N^+,s.t. (f-\lambda\mathrm{Id})^k(x)=0\}$$

关于根子空间的观察.同一个特征值的特征子空间包含于根子空间;根子空间是线性变换的不变子空间;两个根子空间的交只有0,事实上如果$x\in W_{\lambda}\cap W_{\mu}$,那么存在正整数$m,n$使得$(f-\lambda\mathrm{Id})^mx=(f-\mu\mathrm{Id})^nx=0$,按照$((x-\lambda)^m,(x-\mu)^n)=1$,于是存在多项式$s(x),t(x)$满足$s(x)(x-\lambda)^m+t(x)(x-\mu)^n=1$,于是带入$f$,作用到$x$上,得到$0=x$.

根子空间的维数恰好是代数重数.为了证明这个定理,先来证明准素分解定理:给定线性空间$V$上的线性变换$f$的一个零化多项式$p(x)$,如果$p(x)$在域上有分解$p(x)=\prod_{k=1}^sp_k(x)$,其中$p_k(x)$是域上的两两互素多项式,记$V_k=\ker p_k(f)$,那么有直和分解$V=\oplus_{k=1}^sV_k$.

\begin{proof}
	
	首先证明空间$V$是$V_k$的和,即每个$V$中向量可以表示为每个$V_k$选取一个向量之和.按照$p_k(x)$是两两不同的不可约多项式,于是看到$t_k(x)=\prod_{j\not=k}p_j(x)$这$s$个多项式是互素的,于是,存在多项式$q_1,q_2,\cdots,q_s$满足$\sum_{k=1}^{s}q_k(x)t_k(x)=1$,带入线性映射$f$,得到$\sum_{k=1}^{s}q_k(f)t_k(f)=\mathrm{Id}$.
	
	记线性映射$h_k=q_k(f)t_k(f)$.那么有$\sum_{k=1}^{s}h_k=\mathrm{Id}$,并且$i\not=j$时候有$h_ih_j=0$.结合这两个性质看到$h_k^2=h_k$.
	
	现在来证明$\mathrm{im}h_k=\ker p_k(f)$.一方面如果$x\in\mathrm{im}h_k$,就有$x=h_k(y)$,那么$p_k(f)(x)=p_kq_kt_k(f)(y)=0$,于是$x\in\ker p_k(f)$.另一方面,如果$x\in V_k=\ker p_k(f)$,当$j\not=k$时候,有$p_k(x)$整除$h_j(x)$,于是得到$h_j(x)=q_j(f)t_j(f)(x)=0$,于是$x=\sum_{i=1}^{s}h_i(x)=h_k(x)$,于是$x\in\mathrm{im}h_k$.
	
	最后,首先对每个$x\in V$有分解$x=\sum_{i=1}^{s}h_i(x)$,其中$h_i(x)\in\mathrm{im}h_i=\ker p_i(f)=V_i$.现在假设存在分解$x=\sum_{i=1}^{s}x_i$,其中$x_i\in V_i$,那么每个$x_i$可以写作某个$h_i(y_i)$,于是有:$h_k(x)=\sum_{i}^{s}h_k(x_i)=\sum_{i=1}^{s}h_kh_i(y_i)=h_kh_k(y_i)=h_k(y_i)=x_i$,于是每个$V$中向量可以唯一的表示为$V_k$选取一个向量的和的形式,于是得到直和分解.
	
\end{proof}

给定线性映射$f$,那么总会得到一个升链$\ker f\subset\ker f^2\subset\cdots$.倘若存在一个正整数$n$满足$\ker f^n=\ker f^{n+1}$,断言对任意正整数$p$都有$\ker f^n=\ker f^{n+p}$.为此只要证明对任意正整数$k$有$\ker f^{n+k}=\ker f^{n+k+1}$,为此只要归纳的证明$\ker f^{n+1}=\ker f^{n+2}$.事实上,如果$x\in\ker f^{n+2}$,那么$f^{n+2}(x)=0$,那么$f(x)\in\ker f^{n+1}=\ker f^n$,于是$f^{n+1}(x)=0$,于是得证.

现在对线性变换$f$考虑两个零化多项式:特征多项式和极小多项式,分别记作$p_f(x)=\prod_{k=1}^{s}(x-x_k)^{r_k}$和$p(x)=\prod_{k=1}^{s}(x-x_k)^{u_k}$.那么$r_k\ge u_k$.按照上述定理,看到$V=\oplus_{k=1}^s\ker(f-x_k\mathrm{Id})^{r_k}$和$V=\oplus_ {k=1}^s\ker(f-x_k\mathrm{Id})^{u_k}$.于是从$\ker(f-x_k\mathrm{Id})^{u_k}\subset\ker(f-x_k\mathrm{Id})^{r_k}$,看出二者相等,于是看到$\ker(f-x_k\mathrm{Id})^ {u_k}=\ker(f-x_k\mathrm{Id})^{u_k+p},\forall p\in\mathbb{N^+}$.于是看到根子空间具有表达式$W_{\lambda}=\ker(f-x_k\mathrm{Id})^ {u_k}$,这里$u_k$是最小多项式中,特征值$\lambda$的次数.

现在来证明根子空间的维数是代数重数.记根子空间分解$V=\oplus_ {k=1}^sV_k$,那么已经知道$V_k=\ker(f-x_k\mathrm{Id})^{u_k}$,其中$u_k$是特征值$x_k$在极小多项式中的次数.现在在每个$V_k$中取一组基,按照$V_k$是不变子空间,得到了矩阵表示是一个准对角矩阵$\mathrm{diag}\{A_1,A_2,\cdots,A_s\}$,其中$A_i$是$f$限制在$V_i$下的线性映射的矩阵表示,那么$A_i$的阶数就是$V_i$的维数.那么$f$限制在$V_i$下的映射,记作$f\mid V_i$,以$(x-x_i)^{u_i}$为零化多项式,这导致$f\mid V_i$的最小多项式是某个$(x-x_i)^{v_i}$,其中$v_i\le u_i $.但是按照准对角矩阵的最小多项式是全部对角块的最小多项式的最小公倍式,于是看到$u_i=v_i$,即$f\mid V_i$最小多项式是$(x-x_i)^{u_i}$.于是说明$A_i$的特征多项式是$(x-x_i)^{n_i}$,这里$n_i$是$A_i$的阶数,于是$A$的特征多项式是它们的乘积,也就是$\prod_{k=1}^{s}(x-x_i)^{n_i}$,于是看到根子空间的维数是代数重数.

得到了根子空间分解:全体根子空间是一个关于线性变换的不变子空间的直和分解,并且按照定义,把线性映射限制到关于特征值$\lambda$的根子空间上,此时的线性映射只有一个特征值$\lambda$.根子空间分解称为第一空间分解定理.

这对应于矩阵角度推导Jordan标准型步骤的第二步,即任意矩阵$A$,设不同特征值个数$s$,那么它相似于准对角矩阵$\mathrm{diag}\{T_{11},T_{22},\cdots,T_{ss}\}$,其中每个$T_{ii}$有且只有一个特征值,并且两两不同,并且$T_{ii}$的阶数就是这个特征值的代数重数.

接下来为了进一步得到Jordan标准型,需要把线性变换$f$限制在一个根子空间上这个线性变换继续进行分解.那么不妨考虑特征值只有一个的线性变换.需要把它分解为不能再分解得更小的不变子空间的直和,这种不可继续细分的不变子空间就是循环子空间:

设$\alpha$是线性空间$V$上的一个非0元素,给定$V$上一个线性变换$f$,称所有包含$\alpha$的$f$的不变子空间的交为由$\alpha$生成的关于$f$的循环子空间.那么这个子空间恰好就是由$\alpha,f\alpha,f^2\alpha,\cdots$生成的$V$的子空间.怎么描述它的维数?如果多项式$p$满足$p(f)\alpha=0$,称$p$是$\alpha$关于$f$的零化多项式,那么全体这样的零化多项式构成$F[x]$的一个理想,于是存在一个次数最小的首一非0零化多项式,称为$\alpha$关于$f$的极小多项式,并且每个非0的零化多项式都是极小多项式的倍式.

现在设$\alpha$关于$f$的极小多项式的次数是$r$,那么断言$\{f^{r-1}\alpha,f^{r-2}\alpha,\cdots,\alpha\}$是$\alpha$关于$f$的循环子空间的基.事实上,一方面这个向量组如果是线性相关的,那么可以得到一个关于$\alpha$的次数小于$r$的非0的零化多项式,这和极小多项式的定义矛盾;另一方面,如果记极小多项式是$x^r+a_1x^{r-1}+\cdots+a_r$,那么看到$f^r\alpha=-a_1f^{r-1}\alpha-\cdots-a_r$,对次数归纳下去,就看出对每个正整数$n$,$f^n\alpha$都能被这个向量组生成,于是这个向量组生成了整个循环子空间.完成证明.

另外,$f$限制在$\alpha$生成的循环子空间上的最小多项式,也就是$\alpha$关于$f$的最小多项式,于是看到$f\mid_{(\alpha)}$的最小多项式的次数和特征多项式的次数相同,于是必然有这个限制映射的最小多项式和特征多项式是相同的.

现在取$f$是特征值只有1个的代数闭域上线性映射,不妨设这个特征值是$\lambda$,那么此时,考虑$f$在基$\{(f-\lambda\mathrm{Id})^{r-1}\alpha,(f-\lambda\mathrm{Id})^{r-2}\alpha,\cdots,\alpha\}$下的矩阵表示,就是:
$$f((f-\lambda\mathrm{Id})^ {r-1}\alpha,(f-\lambda\mathrm{Id})^{r-2}\alpha,\cdots,\alpha)=((f-\lambda\mathrm{Id})^ {r-1}\alpha,(f-\lambda\mathrm{Id})^{r-2}\alpha,\cdots,\alpha)\left(\begin{array}{ccccc}
\lambda&1&\cdots&0&0\\
0&\lambda&\cdots&0&0\\
\vdots&\vdots&\vdots&\vdots&\vdots\\
0&0&\cdots&\lambda&1\\
0&0&\cdots&0&\lambda\end{array}\right)$$

即,循环子空间对应于Jordan块!于是循环子空间分解就对应于矩阵角度处理Jordan标准型的第三步.已经指出了只要把变换在每个根子空间上的限制分解为循环子空间直和即可,注意变换限制在根子空间上是一个特征值恰好只有一个的变换,如果适当加上一个数量变换会使得它的特征值全部为0,于是导致$x^s$是特征多项式,于是导致映射变为幂零变换,于是只需考虑幂零变换.先来证明,如果$f$的幂零指数是$m$,即最小的满足$f^k=0$的正整数$k=n$,则:
\begin{enumerate}
	\item 存在非0向量$\alpha_1\in V$使得它生成的循环子空间$C_1$的维数是$m$,$f$限制在$C_1$上是$m$次幂零变换.
	\item 存在另一个不变子空间$V_1$使得$V=C_1\oplus V_1$,并且$f$限制在$V_1$上是指数不超过$m$的幂零变换.
\end{enumerate}

\begin{proof}
	
	按照定义,有$f^m=0,f^{m-1}\not=0$,于是存在非0向量$\alpha_1\in V$满足$f^{m-1}(\alpha_1)\not=0$.于是$\alpha_1$关于$f$的一个零化多项式就是$x^m$,于是极小多项式可以记作某个$x^l$,其中$1\le l\le m$.假设$l$严格小于$m$,就得到$f^{m-1}(\alpha_1)=0$矛盾.于是看到$\alpha_1$关于$f$的最小多项式就是$x^m$.记$\alpha_1$生成的关于$f$的循环子空间是$C_1$,那么它的维数就是最小多项式的次数,也就是$m$.并且$f$限制在$C_1$上是$m$次幂零线性变换.这就证明了第一条.
	
	为证明第二条,对幂零指数$m$归纳.当$m=1$时,$f$是零变换,于是$\alpha_1$生成的循环子空间$C_1$是一维的,现在把$\alpha_1$扩充为$V$上一组基$\{\alpha_1,\cdots,\alpha_n\}$,记$\{\alpha_2,\cdots,\alpha_n\}$生成的子空间为$V_1$,那么有$V=C_1\oplus V_1$.并且显然$V_1$是$f$的不变子空间,而且限制在$V_1$上是幂零指数1的幂零变换.这就证明了$m=1$是成立的.
	
	现在假设第二条对$m-1$已经成立.现在考虑$m$次幂零变换$f$.记$U=\mathrm{im}f$,记$\beta_1=f(\alpha_1)\in U$,按照$U$是$f$不变子空间,于是$f$可以限制到$U$上,断言这个限制映射是一个$m-1$次的幂零变换.事实上一方面任取$x\in U$,也就是说存在$y\in V$使得$x=f(y)$,那么$f^{m-1}(x)=f^m(y)=0$.另一方面倘若存在比$m-1$小的正整数$l$满足$\forall x\in U$有$f^l(x)=0$,等价于说对任意$y\in V$有$f^{l+1}(y)=0$.但是$l+1$严格小于$m$,和幂零指数是$m$矛盾.于是按照归纳假设,$f\mid_U$在$U$上存在以向量$\beta_1$关于这个限制映射生成了循环子空间$C_1'$,它的维数是$m-1$,并且$f\mid_U$限制在$C_1'$上是$m-1$次幂零变换.另外,还存在$U$上的关于$f\mid_U$的不变子空间$U_1$满足$U=C_1'\oplus U_1$并且$f\mid_U$在$U_1$上的限制是幂零指数不超过$m-1$的幂零变换.
	
	现在记$V_1'=\{x\in V\mid f(x)\in U_1\}$,来证明如下命题:
	\begin{enumerate}
		\item $C_1\cap U_1=\{0\}$.
		
		事实上,如果取$x\in C_1\cap U_1$.从$x\in C_1$得到有线性表示$x=b_0\alpha_1+b_1f(\alpha_1)+\cdots+b_{m-1}f^{m-1}(\alpha_1)$.于是得到$f(x)=b_0\beta_1+b_1f(\beta_1)+\cdots+b_{m-2}f^{m-2}(\beta_1)\in C_1'$.从$x\in U_1$,按照不变子空间定义得到$f(x)\in U_1$,于是$f(x)\in C_1'\cap U_1=\{0\}$.于是$f(x)=0$,从$f(x)$在循环子空间$C_1'$中的表示,得到$b_0=b_1=\cdots=b_{m-2}=0$,于是得到$x=b_{m-1}f^{m-1}(\alpha_1)=f^{m-2}(\beta_1)\in C_1'\cap U=\{0\}$.于是得到$x=0$.
		
		\item $U_1\subset V_1'$.
		\item $V=C_1+V_1'$.
		
		任取$x\in V$,那么$f(x)\in U$,于是存在分解$f(x)=\xi+\eta$,其中$\xi\in C_1'$,$\eta\in U_1$.于是有分解$\xi=b_0\beta_1+b_1f(\beta_1)+\cdots+b_{m-2}f^{m-2}(\beta_1)=f(b_0\alpha_1+b_1f(\alpha_1)+\cdots+b_{m-2}f^{m-2}(\alpha_1))$.于是得到$f(x-b_0\alpha_1-\cdots-b_{m-2}f^{m-2}(\alpha_1))=\eta\in U_1$.于是得到$x-b_0\alpha_1-\cdots-b_{m-2}f^{m-2}(\alpha_1)=\zeta\in V_1'$,于是得到$\alpha=(b_0\alpha_1+\cdot+b_{m-2}f^{m-2}(\alpha_1))+\zeta$,其中前一项是$C_1$中的,就得到$V=C_1+V_1'$.
	\end{enumerate}
	
	但是最后得到的$V=C_1+V_1'$并不是直和.从$U_1\cap(C_1\cap V_1')=\{0\}$看出有直和$U_1\oplus(C_1\cap V_1')\subset V_1'$.通过扩充基,找到一个$V_1'$的子空间$W$满足$V_1'=W\oplus U_1\oplus(C_1\cap V_1')$.记$V_1=W\oplus U_1$.来证明$V_1$就是所求.
	
	验证$V_1$是$f$不变子空间:对$x\in V_1\subset V_1'$,有$f(x)\in U_1\subset W\oplus U_1=V_1$;验证$f\mid_{V_1}$是幂零变换,幂零指数$m_1\le m$:$f^m(x)=0,\forall x\in V_1$;验证$C_1\cap V_1=\{0\}$:取$x\in C_1\cap V_1$,于是$x\in V_1'$,于是$x\in V_1\cap(C_1\cap V_1')=\{0\}$;验证$V=C_1+V_1$:任取$x\in V$,按照$V=C_1+V_1'$,得到$x=\eta+\zeta$,其中$\eta\in C_1$,$\zeta\in V_1'$,从$V_1'=V_1\oplus(C_1\cap V_1')$,得到$\zeta=\zeta_1+\zeta_2$,其中$\zeta_1\in V_1$,$\zeta_2\in C_1\cap V_1'$,于是$x=(\eta+\zeta_2)+\zeta_1$,完成证明.
	
\end{proof}

归纳的继续对命题中的$V_1$继续操作下去,看出:对线性空间$V$上的$m$次幂零线性变换$f$,$V$可以分解为若干关于$f$的循环子空间的直和$V=C_1\oplus C_2\oplus\cdots\oplus C_k$,其中$f$限制在$C_i$上记作$m_i$次幂零变换,这个$m_i$同样是$C_i$的维数,满足$m=m_1\ge m_2\ge\cdots\ge m_k$.由此得到\textbf{第二空间分解定理}:

设$W_{\lambda_j}$是$V$上线性变换$f$的属于特征值$\lambda_j$的根子空间,那么有如下直和分解,其中每个$C_{ji}$都是关于线性变换$f\mid_{W_{\lambda_j}}$的循环子空间,同时也是关于幂零线性变换$(f-\lambda_j\mathrm{Id})\mid_{W_{\lambda_j}}$的循环子空间,并且它的幂零指数恰好就是$C_{ji}$的维数,其中$i=1,2,\cdots,k_i,j=1,2,\cdots,t$.
$$W_{\lambda_j}=C_{j1}\oplus C_{j2}\oplus\cdots\oplus C_{jk_j}$$

另外,如果取生成$C_{ji}$的向量是$\alpha_{ji}$,分别取$C_{ji}$的基$\{(f-\lambda_j\mathrm{Id})^{m_{ji}-1}(\alpha_i),(f-\lambda_j\mathrm{Id})^{m_{ji}-2}(\alpha_i),\cdots,\alpha_i\}$,把每个$C_{ji}$的这种基拼在一起构成$V$上的一组基,那么线性映射$f$的矩阵表示是由若干Jordan块构成的Jordan型矩阵.综上,按照根子空间分解和循环子空间分解,得到了代数闭域上矩阵总相似于一个Jordan型矩阵.

至此可以说明,特征值唯一的循环子空间就是不能再非平凡的分解为不变子空间直和的不变子空间.如果$W$是线性变换$f$一个特征值唯一的循环子空间,于是它的极小多项式和特征多项式相同,记作$(x-r)^m$.现在假设有直和分解$W=W_1\oplus W_2$,其中$W_i$都是$f$不变子空间.那么$(x-r)^m$是$f\mid_{W_1}$和$f\mid_{W_2}$的零化多项式,于是可以设$f\mid_{W_i}$的最小多项式是$(x-r)^{s_i}$,$i=1,2$,$s_1,s_2\le m$,不妨设$s_1\ge s_2$,于是对每个$x=x_1+x_2\in W_1\oplus W_2$,有$(f-r\mathrm{Id})^{s_1}(x_1+x_2)=0$,于是$(x-r)^{s_1}$是$f\mid_{W}$的零化多项式,于是得到$s_1=m$.现在设$f\mid_{W_i}$的特征多项式是$(x-r)^{t_i}$,$i=1,2$,那么$t_i\ge s_i$,于是$f\mid_W$的特征多项式是乘积$(x-r)^{t_1+t_2}$,于是$t_1+t_2=m$,导致$t_2=0$,于是$W_2=0$.另一侧的命题的证明已经包含在第二空间分解中.

下面为了说明同一个矩阵的Jordan标准型在不计Jordan块顺序的意义下是唯一的,需要证明幂零变换的循环子空间分解在不计直和顺序的意义下是唯一的,严格说就是:给定$V$上幂零线性变换$f$,如果$V$存在两种方式分解为$f$的循环子空间的直和,即下图,其中$C_j$由向量$\alpha_j$生成的,设$f\mid_{C_j}$是$m_j$次幂零的,$j=1,2,\cdots,k$,并且$m=m_1\ge m_2\ge\cdots\ge m_k$,再记$C_i'$是由$\alpha_i'$生成的,并且$f\mid_{C_i'}$是$m_i'$次的幂零变换$i=1,2,\cdots,l$,那么有$k=l$,并且在适当调整$C_i'$顺序后会得到$m_i'=m_i,i=1,2,\cdots,k$.

$$V=C_1\oplus C_2\oplus\cdots\oplus C_k=C_1'\oplus C_2'\oplus\cdots\oplus C_l'$$

\begin{proof}
	
	先记$m'=\{m_1',m_2',\cdots,m_l'\}$,那么$f$在$\oplus_{i=1}^lC_i'$上的幂零指数就是$m'$,这导致$m'=m$,记$m_1'=m$,否则可以重新排序.现在设上述两种分解为$V=C_1\oplus V_1=C_1'\oplus V_1'$,那么$f$限制在$V_1,V_1'$上都是幂零变换,它们的幂零指数分别记作$r,r'$.那么得到$f^r(V)=f^r(C_1)=f^r(C_1')\oplus f^r(V_1')$,容易从基看出$\dim f^r(C_1)=m-r=\dim f^r(C_1')$,于是得到$\dim f^r(V_1')=0$,导致$r'\le r$,类似得到$r\le r'$,于是$r=r'$.于是可以重复上述操作,证明存在某个$m_i'=m_2=r$,继续下去,看到分解是在不计直和项次序的意义下是唯一的.
	
\end{proof}

对每个循环子空间$C_{ji}$,设它的唯一特征值是$\lambda_j$,那么$f$在其上限制的最小多项式是$(x-\lambda_j)^{m_{ji}}$,$C_{ji}$对应于Jordan标准型中的Jordan块$J_{m_{ji}}(\lambda_j)$.把$C_{ji}$对应的最小多项式$(x-\lambda_j)^{m_{ji}}$称为线性变换的初等因子,把全部$C_{ji}$对应的初等因子构成的集合称为初等因子组.把Jordan块$J_{m_{ji}}(\lambda_j)$称为初等因子对应的Jordan块.于是初等因子也是对应Jordan块的最小多项式.由此可以得到如下事实:
\begin{enumerate}
	\item 两个矩阵相似当且仅当它们具有相同的初等因子组.
	\item 线性变换的特征多项式就是全部初等因子的乘积.
	\item 线性变换的最小多项式,就是对每个特征值,取关于它的初等因子中次数最大的那个,再把选取出来的若干初等因子相乘.
\end{enumerate}
\newpage
\subsection{有理标准型}

本节从模论角度给出一般域上方阵相似关系的标准型,称为有理标准型.会看到上一节对代数闭域的处理会是这里的特例.

给定域$F$上的线性空间$V$,给定$V$上的线性变换$f$,来把$V$重新作为一个$F[t]$模,即定义$\forall x\in V$,有$xt=f(t)$.于是这是一个PID上的有限生成模.反过来,如果把域$F$上的线性空间$V$赋予了$F[t]$模结构,那么看到$t$作用在$V$上是一个线性映射.于是,指定线性空间$V$上一个线性变换,等同于说赋予$V$上一个$F[t]$模结构.

在这个观点下,如果指定了线性空间上一个线性变换$f$,那么可以讨论多项式在向量上的作用,即定义$p(t)x,\forall p\in F[t],x\in V$为$p(f)(x)$.另外,在这个观点下,线性变换$f$的不变子空间,也就是$V$作为$F[t]$模的子模.

$R[t]$模结构包含了关于相似的全部信息:如果给定线性空间$V$上两个线性变换$\alpha,\beta$,那么它们共轭当且仅当,诱导的$V$的$F[t]$模结构是同构的.一方面,记$\alpha,\beta$分别诱导的$F[t]$模为$V_{\alpha},V_{\beta}$.假设$\alpha$和$\beta$是共轭的,也就是存在$V$上的$F$线性同构$\pi$,使得$\beta=\pi\circ\alpha\circ\pi^{-1}$.这个等式告诉$\pi$可以看作$V_{\alpha}\to V_{\beta}$的$F[t]$模同态,事实上有$\pi(tx)=\pi\circ\alpha(x)=\beta\circ\pi(x)=t\pi(x)$.类似的,$\pi^{-1}$也可以看作$V_{\beta}\to V_{\alpha}$的$F[t]$模同态,于是它们是$F[t]$模同构.反过来,如果两个模同构,把同构映射限制到$V$上,就得到两个线性映射是共轭的.

于是,存在从$V$上线性映射的相似关系的等价类,到$V$作为$F[t]$模结构的等价类之间的一一对应.相似问题转化为$F[t]$上有限生成模的结构问题.即如何把这个模分解为更小的子模的直和.如果得到了一种直和分解,把它看作$F$上模的直和分解,在每个直和项取基,就得到矩阵相似于一种准对角型.关于模的分解,有PID上有限生成模的结构定理,即此时模总是若干循环子模的直和.注意到按照特征多项式必然是零化多项式,$V$作为$F[t]$模总是挠模,于是直和分解中不会出现无挠模的部分,于是有$F[t]$模同构:
$$V\simeq\oplus_{i=1}^m\frac{F[t]}{(a_i(x))}\simeq\oplus_{i=1}^r\oplus_{j=1}^{s_i}\frac{F[t]}{(p_j^{r_{ij}}(x))}$$

循环子模总是具有形式$V=\frac{F[t]}{(f(t))}$.它的零化子就是分母$(f(t))$.这里可以把$f$取为首一形式$f(t)=t^n+r_{n-1}t^{n-1}+\cdots+r_0$.它作为$F$上的模,自然具有基$\{1,t,\cdots,t^{n-1}\}$,在这组基下,$f$的矩阵表示就是:
$$\left(\begin{array}{ccccc}
0&&&&-r_0\\
1&0&&&-r_1\\
&1&\ddots&&\vdots\\
&&\ddots&0&-r_{n-2}\\
0&&&1&-r_{n-1}\end{array}\right)$$

这称为关于首一多项式$f(t)$的有理块.于是,看到一般域上的矩阵必然可以相似于由有理块作为对角块构成的准对角矩阵,其中,友矩阵就是把线性变换理解为$V$上赋予$F[t]$模结构的结构定理中的初等因子,对应的有理块.并且按照空间分解的唯一性,得到这种相似的唯一性.一般域上矩阵相似到的,由有理块构成的矩阵称为矩阵的有理标准型.它在不计有理块顺序的意义下完全刻画了一般域上方阵的相似关系.

特别的,在代数闭域的情况下,$F[t]$上的不可约元就是一次多项式,于是,上一节所证明的第二空间分解定理,当时的理解是把空间分解为关于线性变换$f$的循环子空间的直和,这实际上就是把线性空间$V$分解为$F[t]$上循环子模$\frac{F[t]}{(x-\lambda_j)^{r_{ij}}}$的直和.也就是说实际上证了一遍$F[t]$上有限生成模的结构定理.对每个循环子模$\frac{F[t]}{(x-\lambda_j)^{r_{ij}}}$,选取一组基下它的矩阵表示就是Jordan块,于是从本节的内容也重新得到了代数闭域上Jordan标准型的存在性和唯一性定理.

那么现在对于一般域上的方阵$A$快速写出它的有理标准型,或者对于代数闭域上快速写出Jordan标准型,都依赖于如何求出方阵对应的初等因子组或者不变因子组.这里介绍一种不变因子组的快速求法.为此,注意到在PID有限生成模结构定理中,的不变因子组,实际上是说,对PID上有限生成模$M$,存在从$R^n$到$M$的满同态$f$,这个满同态的核是$R^n$的子模,存在$R^n$的一组基$\{x_1,x_2,\cdots,x_n\}$和$R$中依次整除的元$\{a_1\mid a_2\mid\cdots\mid a_m\},m\le n$,使得$\ker f$以$\{a_1x_1,a_2x_2,\cdots,a_mx_m\}$为基,这里的$a_1,a_2,\cdots,a_m$就是$M$的不变因子组.

现在取$R^m$的一组基$\{x_1,x_2,\cdots,x_m\}$,取$R^m$的子模$N$,假设$N$被$n$个元$y_1,y_2,\cdots,y_n$生成,注意没有要求这是$N$的一组基.于是对每个$y_j$有分解$y_j=\sum_{i=1}^{m}a_{ij}x_i$.于是可以把系数凑成一个$m\times n$的矩阵$A=(a_{ij})$,称为关系矩阵.

核心命题是,如果对矩阵$A$做了初等变换,得到新的矩阵$A'$,那么存在$R^m$中的基和$N$中的生成元集使得它们的关系矩阵是$A'$.于是,如果把$R$约定为欧氏整环,那么按照欧氏整环上可逆矩阵都被初等矩阵生成,矩阵$A$可以经过初等变换变为Smith标准型$\mathrm{diag}\{a_1,a_2,\cdots,a_l,\textbf{0}\}$.于是,此时的$\{a_1,a_2,\cdots,a_l\}$就是要找的,可以使得存在$R^m$上一组基$\{x_1,x_2,\cdots,x_m\}$使得$\{a_1x_1,\cdots,a_lx_l\}$是$N$的基的一组$R$中的元.

现在给定域$F$上的一个线性空间$V$,指定一个线性变换$f$,取$V$上一组基$\{v_1,v_2,\cdots,v_n\}$,设$f$在这组基下的矩阵表示是$A=(a_{ij})$.现在把$V$看作一个$F[t]$模,于是按照PID上有限生成模的结构定理,得到有$F[t]$模同构:
$$V\simeq\frac{F[x]}{(a_1(x))}\oplus \frac{F[x]}{(a_2(x))}\oplus\cdots\oplus\frac{F[x]}{(a_m(x))}$$

现在设$F[x]^n$的一组基为$\{e_1,e_2,\cdots,e_n\}$,取$F[t]$模同态$\phi:F[t]^n\to V$为把$e_i\mapsto v_i$.于是按照$v_i$在$V$作为$F$模时生成整个$V$,自然在$V$作为$F[t]$模时也生成了整个$V$,于是$\phi$是一个满同态,它的核$\ker f$是$F[t]^n$的子模.

于是按照刚刚所说的思路,为了求这些$a_i(x)$,等价于找$F[t]^n$的一组基$\{e_1',e_2',\cdots,e_n'\}$,使得存在$\{a_1,a_2,\cdots,a_m\},a_i\in F[t]$满足$\{a_1e_1',\cdots,a_me_m'\}$是$\ker\phi$的一组基.为此只要找$\ker f$的一组生成元集.为此注意到在$\phi$下把$f(v_j)$映射为$\sum_{i=1}^{n}a_{ij}e_i$,也就是说$y_j=te_j-\sum_{i=1}^{n}a_{ij}e_i\in\ker\phi$,验证$y_j,j=1,2,\cdots,n$生成了整个$\ker f$.那么关于基$\{e_1,e_2,\cdots,e_n\}$和$\ker\phi$上的生成元集$\{y_1,y_2,\cdots,y_n\}$的关系矩阵就是$tE-A$.按照所指出的核心命题,这个矩阵作为$F[t]$上矩阵的Smith标准型,的非常数对角元,就是全部不变因子.

综上,说明了对一个线性变换$f$,任取它的表示矩阵$A$,那么考虑$F[t]$上的矩阵$tE-A$,这个矩阵的Smith标准型的对角元,就是全体不变因子.这就提供了一种求矩阵不变因子组或者初等因子组的方法.

另外,两个域$F$上矩阵$A,B$相似当且仅当它们具有相同的不变因子组,当且仅当特征矩阵$tE-A$和$tE-B$具有相同的Smith标准型上的对角元,当且仅当$tE-A$和$tE-B$在$F[t]$上是相抵的.但是所给的这个证明实际上涉及到了模论,如果利用纯矩阵技巧也是可以做出的.

在给出证明之前回顾一些内容.首先指出,有时候把$F[t]$上的矩阵称为域$F$上的$t$矩阵.那么按照$F[t]$是欧氏整环,知道$t$矩阵可逆当且仅当行列式是域$F$中的非0元;$t$矩阵满秩当且仅当行列式非0;可逆的$t$矩阵可以被初等矩阵生成.

域$F$上两个矩阵$A,B$相似当且仅当特征矩阵作为$F[t]$上的矩阵是相抵的.

\begin{proof}
	
	必要性是显然的,至于充分性,需要含幺环上的带余除法:给定$D$为矩阵,$T(\lambda)$是一个$n$阶$\lambda$方阵,那么存在唯一的$n$阶$\lambda$矩阵$T_{1}(\lambda)$和$T_2(\lambda)$以及方阵$W_1,W_2$使得:
	$$T(\lambda)=(\lambda E_n+D)T_1(\lambda)+W_1=T_2(\lambda)(\lambda E_n+D)+W_2$$
	
	按照条件,存在可逆的$\lambda$矩阵$P(\lambda),Q(\lambda)$使得:
	$$P(\lambda)(\lambda E_n-A)=(\lambda E_n-B)Q(\lambda)$$
	
	于是按照带余除法,得到:
	$$P(\lambda)=(\lambda E_n-B)P_1(\lambda)+P;Q(\lambda)=Q_1(\lambda)(\lambda E_n-A)+Q$$
	
	于是将第一式右乘$(\lambda E_n-A)$减去第二式左乘$(\lambda E_n-B)$得到:
	$$\lambda(P-Q)+BQ-PA=(\lambda E_n-B)(Q_1(\lambda)-P_1(\lambda))(\lambda E_n-A)$$
	
	考虑两边次数,立马看到$P_1(\lambda)=Q_1(\lambda)$,于是有$P=Q,BQ=PA$.于是为了证明$A,B$相似,只要证明$P$可逆即可,设:
	$$P(\lambda)^{-1}=(\lambda E_n-A)C(\lambda)+P_1$$
	
	于是利用$P(\lambda)P(\lambda)^{-1}=E_n$得到:
	$$E_n=(\lambda E_n-B)(Q(\lambda)C(\lambda)+P_1(\lambda)P_1)+PP_1$$
	
	考虑$\lambda$的次数就得到$Q(\lambda)C(\lambda)+P_1(\lambda)P_1=0$,于是$PP_1=E_n$,于是$P$可逆.
	
\end{proof}

注意这个证明过程提供给求两个相似矩阵的过渡矩阵的操作性方法.如果$A$和$B$相似(特别的$B$是相似标准型),那么$\lambda E-A$和$\lambda E-B$相抵,于是存在可逆的$\lambda$矩阵$P(\lambda)$和$Q(\lambda)$满足$P(\lambda)(\lambda E-A)=(\lambda E-B)Q(\lambda)$.做带余除法$P(\lambda)=(\lambda E-B)P_1(\lambda)+P$,那么$P$是一个数项可逆矩阵,并且满足$PAP^{-1}=B$.

另外上述讨论中还证明了域上特征矩阵的初等因子组,和从有限生成模结构定理那里定义的矩阵初等因子组是吻合的.这里给出一个代数闭域的情况下,不依赖模论的证明.为此,注意到已经证明了准对角矩阵的初等因子组就是每个对角块的初等因子组的合并.于是,只需要证明,对每个Jordan块$J=J_r(\lambda)$,它的初等因子组就是$\{1,1,\cdots,(x-\lambda)^r\}$.为此只要求出$J$的不变因子组.事实上,按照$J$的$n-1$个1,看到$J$存在一个$r-1$阶子式行列式1,这就迫使从1阶到$r-1$阶行列式因子全部是1,于是前$r-1$个不变因子都是1,第$r$个不变因子是$(x-\lambda)^r$.

在代数闭域的情况下指出过每个特征值取次数最大的初等因子相乘,就得到矩阵的最小多项式,于是看出最大的不变因子恰好就是矩阵的最小多项式.

而这个命题实际上在一般域上也是成立的,为此注意到对$F[t]$上的一个模$M$,如果把$M$当作$F$模,那么知道$t$就相当于$F$模$M$上的一个线性变换,而这个线性变换的极小多项式在$F[t]$中生成的理想,实际上就是$M$作为$F[t]$模的零化子$\mathrm{Ann}(M)=\{r\in F[t]\mid rM=0\}$.现在对于循环模$F[t]/(f(t))$,它的零化子就是模去的理想$(f(t))$,于是,如果取$F$上线性空间$V$上的线性变换$f$,赋予相应的$F[t]$模结构,按照PID上有限生成模结构定理得到$V\simeq\frac{F[t]}{(a_1(x))}\oplus\cdots\oplus\frac{F[t]}{(a_m(x))}$.那么$f$的极小多项式生成的理想也就是右侧作为$F[t]$模的零化子,再结合$\mathrm{Ann}(M_1\oplus M_2)=(\mathrm{Ann}(M_1),\mathrm{Ann}(M_2))$,和$a_1\mid a_2\mid\cdots\mid a_m$,就看到$V$的零化子就是最大的不变因子在$F[t]$中生成的理想,这就证明了在一般域上,最高阶不变因子总是最小多项式.

这里总结一下扩域所不改变的矩阵的性质.给定域扩张$k\subset F$,设$A$是$k$上的方阵.
\begin{enumerate}
	\item 扩域不影响多项式的(首一)最大公因式.首先最核心的内容是,如果$f,g$是域$k$上的两个多项式,求它们的首一最大公因式需要做带余除法,而每一步操作得到的新的多项式仍然是域$k$上的:假设$f=gq+r$是在域$F$上的带余除法,其中$\deg r<\deg g$.可取$F$上的基由幺元1扩充而成$\{1,a_i\}$,那么多项式可分解为$f=g(q_0+\sum a_iq_i)+r_0+\sum a_ir_i$,按照分量1的系数为零得到$f=gq_0+r_0$.再结合固定域上带余除法是唯一的,就说明实际上这个带余除法是$k$上的.进而$f,g$在域$F$上的首一最大公因式是$k$上的多项式.
	\item 扩域不改变矩阵的秩.矩阵的秩的一个等价定义是最大非零子式的阶数,既然扩域不改变子式的值,自然不改变秩.
	\item 扩域不改变相似关系.如果两个域都是无限域,证明是比较容易的:给定两个域$k$上矩阵$A,B$,需要说明的是如果它们在$F$上相似,那么在$k$上相似.它们在$F$上相似等价于说矩阵方程$AX=XB$在$F$上存在可逆解.这个矩阵方程实际上是个$n^2$阶齐次线性方程组,所以它在$F$上有非零解说明系数矩阵不满秩.任取一组在$k$上的基础解系$T_1,T_2,\cdots,T_s$,我们断言它实际上也是$F$上的线性无关组,否则可以取一组基得到矛盾.于是上一条说明它也是$AX=XB$在$F$上的基础解系.最后对$s$归纳,结合域是无限域来说明$f(t_1,t_2,\cdots,t_s)=|t_1T_1+t_2T_2+\cdots+t_sT_s|$值域中有非零元,它所对应的矩阵就是$k$上的非零矩阵,于是$A,B$在$k$上相似.
	\item 扩域不改变子式的值,于是不改变首一的行列式因子,于是不改变首一的不变因子组.
	\item 扩域不改变相似关系的一般情况.给定两个$k$上的方阵$A,B$,它们在$F$上相似当且仅当不变因子组相同,于是上一条说明它们在$k$上的不变因子组相同,从而它们在$k$上相似.
\end{enumerate}
\newpage
\section{内积空间理论}
\subsection{双线性函数}

设$V$是域$F$上的$n$维线性空间,$f:V\times V\to F$称为双线性函数,如果它满足,对任意的$\alpha,\beta\in V$和任意的$\lambda_i,\mu_i\in F$有:
$$f(\lambda_1\alpha_1+\lambda_2\alpha_2,\beta)=\lambda_1f(\alpha_1,\beta)+\lambda_2f(\alpha_2,\beta)$$
$$f(\alpha,\mu_1\beta_1+\mu_2\beta_2)=\mu_1f(\alpha,\beta_1)+\mu_2f(\alpha,\beta_2)$$

例如如果$V$上两个线性函数$f_1,f_2$,那么$f_1(x)f_2(y)$是双线性函数,再比如,把$F^n$作为域$F$上的线性空间,取$M_n(F)$中的矩阵$A$,那么$f(X,Y)=X^TAY$是$F^n$上的双线性函数.双线性函数$f$必然满足$f(0,x)=f(y,0)=0$.另外总有:
$$f(\sum_{k=1}^{n}r_kx_k,\sum_{k=1}^{m}s_ky_k)=\sum_{i=1}^{n}\sum_{j=1}^{m}r_is_jf(x_i,y_j)$$

双线性函数在基下的矩阵表示.给定域$F$上有限维线性空间$V$上的一组基$\{e_1,e_2,\cdots,e_n\}$,给定$V$上的双线性函数$f$,那么称双线性函数$f$关于基$\{e_1,e_2,\cdots,e_n\}$下的矩阵为:
$$A=(f(e_i,e_j))=\left(\begin{array}{cccc}
f(e_1,e_1)&f(e_1,e_2)&\cdots&f(e_1,e_n)\\
f(e_2,e_1)&f(e_2,e_2)&\cdots&f(e_2,e_n)\\
\vdots&\vdots&\ddots&\vdots\\
f(e_n,e_1)&f(e_n,e_2)&\cdots&f(e_n,e_n)\end{array}\right)$$

那么如果$\alpha,\beta$在基下的表示分别为$x=(x_1,x_2,\cdots,x_n)$和$y=(y_1,y_2,\cdots,y_n)$,那么有$f(\alpha,\beta)=\sum_{i,j=1}^{n}x_iy_jf(e_i,e_j)$,也就是说:$$f(\alpha,\beta)=xAy^T$$

矩阵表示提供给的第一个观察是关于双线性函数集合的结构:把域$F$上$n$维线性空间$V$上的全部双线性函数构成的集合是$L(V,F)$,那么这个集合自然的构成一个$F$线性空间,即加法为$(f+g)(\alpha,\beta)=f(\alpha,\beta)+g(\alpha,\beta)$,模结构为$(af)(\alpha,\beta)=af(\alpha,\beta)$.于是,如果取定$V$上一组基,我们看到从双线性函数到这组基下的矩阵表示是从$L(V,F)$到$M_n(F)$的线性同构.特别的,看到$L(V,F)$是$n^2$维线性空间.

现在讨论基变换下双线性函数矩阵表示的变化.如果给定线性空间$V$上两组基$\{\xi_1,\xi_2,\cdots,\xi_n\}$和$\{\eta_1,\eta_2,\cdots,\eta_n\}$,如果记两个基下双线性函数$f$的矩阵表示是$A,B$,记从前一个基到后一个基的过渡矩阵为$P=(p_{ij})$,也就是说可逆矩阵$P$满足$(\eta_1,\eta_2,\cdots,\eta_n)=(\xi_1,\xi_2,\cdots,\xi_n)P$,那么有$B=P^TAP$.另一方面,如果$f$在基$\{\xi_1,\xi_2,\cdots,\xi_n\}$的矩阵表示为$A$,如果任取一个可逆矩阵$P$,那么$P^TAP$就是$f$在基$(\xi_1,\xi_2,\cdots,\xi_n)P$下的矩阵表示.

称两个矩阵$A,B$是相合的,如果存在可逆矩阵$P$满足$B=P^TAP$,相合是$M_n(F)$上的等价关系.于是双线性函数在不同基下的矩阵表示是相合关系的.按照相合不改变域上矩阵的秩,于是一个双线性函数在基下的矩阵表示的秩是固定的,它就称为是双线性函数的秩.如果满秩,就称$f$是非退化的,否则称为退化的.

如果存在$V$上两个向量$\alpha,\beta$满足$f(\alpha,\beta)=0$,就称$\alpha$关于$f$左正交于$\beta$,也称$\beta$关于$f$右正交于$\alpha$.给定$V$的子集$S$,记$\perp_L(S)=\{\alpha\in V\mid f(\alpha,\beta)=0,\forall\beta\in S\}$,记$\perp_R(S)=\{\beta\in V\mid f(\alpha,\beta)=0,\forall\alpha\in S\}$,于是它们都是$V$的子空间,如果$S\subset T$,有:
\begin{enumerate}
	\item $\perp_L(S)\subset\perp_L(T)$,$\perp_R(S)\subset\perp_R(T)$.
	\item $S\subset\perp_L\left(\perp_R(S)\right)$,$S\subset\perp_R\left(\perp_L(S)\right)$.
\end{enumerate}

特别的,称$\perp_L(V)$和$\perp_R(V)$分别为$V$的关于双线性函数$f$的左正交补和右正交补.它们的维数和$f$的秩有关系:
$$\dim\perp_L(V)=\dim\perp_R(V)=\dim V-\mathrm{rank}f$$
\begin{proof}
	
	记$f$在一组基$\{e_1,e_2,\cdots,e_n\}$下的矩阵表示是$A$,记向量$\alpha,\beta$在基下的表示为向量$x=(x_1,x_2,\cdots,x_n)$和$y=(y_1,y_2,\cdots,y_n)$.那么知道$f(\alpha,\beta)=xAy^T$.现在设$\alpha\in\perp_L(V)$,那么对任意的$\beta\in V$有$f(\alpha,\beta)=0$,也就是说对任意的向量$y$有$xAy^T=0$,于是得到$xA=0$,于是$\alpha\in\perp_L(V)$等价于$\alpha$的向量表示$x$满足$xA=0$.反过来如果给出$xA=0$解空间的向量$x$,在基下对应的向量就落在$\perp_L(V)$中,于是$xA=0$解空间和$\perp_L(V)$同构,最后按照$xA=0$解空间维数等于$A^Tx^T=0$解空间维数,等于$n-r(A^T)$,等于$n-r(A)$,就完成证明.对于右正交补的证明是相同的.
	
\end{proof}

特别的,看到双线性函数非退化当且仅当,空间关于$V$的左正交补或者右正交补是零子空间.

现在从映射角度定义两侧的正交补.首先给定$V$上双线性函数$f$,如果取定向量$\alpha\in V$,那么$f(\alpha,\beta)$是以$\beta$为变量的$V$上线性函数,于是得到了从$V$到对偶空间$V^*$的一个线性映射$\mathscr{A}:\alpha\mapsto f(\alpha,-)$.那么这个线性映射的核就是$V$的关于$f$的左正交补.

据此可以给出非退化的另一个等价描述.双线性函数$f$是非退化的当且仅当左正交补是零子空间,于是当且仅当上述线性映射$\mathscr{A}$是单射,但是知道有限维线性空间上对偶空间的维数和原空间相同,于是线性映射是单的当且仅当是同构,于是$\mathscr{A}$是单射当且仅当它是同构,于是当且仅当,对$V$上任意一个线性函数$g$,存在$\alpha\in V$满足$g=f(\alpha,-)$.这是Riesz表示定理的一种形式.

对于一般的双线性函数$f$,正交性往往不具备对称性,也就是说如果$\alpha$左正交于$\beta$,未必有$\beta$左正交于$\alpha$.如果要求正交性是对称的将会提供极大的便利.为此探究正交性何时满足对称性.

一方面,如果$f(\alpha,\beta)=0$当且仅当$f(\beta,\alpha)=0$,那么如果任取三个向量$\alpha,\beta,\gamma$,取$\xi=f(\alpha,\beta)\gamma-f(\alpha,\gamma)\beta$,那么有$f(\alpha,\xi)=0$,对偶的,就有$f(\xi,\alpha)=0$,于是得到$f(\alpha,\beta)f(\gamma,\alpha)=f(\alpha,\gamma)f(\beta,\alpha)$.取$\alpha=\beta$,得到$f(\alpha,\alpha)(f(\alpha,\gamma)-f(\gamma,\alpha))=0$.现在断言,这个等式恒成立,当且仅当,要么恒有$f(\alpha,\gamma)=f(\gamma,\alpha)$,要么恒有$f(\alpha,\alpha)=0$.事实上,如果有向量$\eta,\zeta\in V$满足$f(\eta,\zeta)\not=f(\zeta,\eta)$,并且存在$\varepsilon\in V$使得$f(\varepsilon,\varepsilon)\not=0$,那么从上述等式得到$f(\eta,\eta)=f(\zeta,\zeta)=0$并且$f(\varepsilon,\eta)=f(\eta,\varepsilon)$和$f(\varepsilon,\zeta)=f(\zeta,\varepsilon)$.那么按照上述等式,得到$f(\varepsilon,\eta)=f(\eta,\varepsilon)=f(\varepsilon,\zeta)=f(\zeta,\varepsilon)=0$,导致$f(\varepsilon+\zeta,\varepsilon+\zeta)=0$,但是展开它就得到$f(\varepsilon,\varepsilon)=0$矛盾.于是证明了,如果正交关系具有对称性,要么恒有$f(\alpha,\beta)=f(\beta,\alpha)$,这个条件称$f$是对称的;要么恒有$f(\alpha,\alpha)=0$,这个条件称$f$是交错的,知道在2是单位的环上交错双线性函数等价于斜对称双线性函数,也就是$f(\alpha,\beta)=-f(\beta,\alpha)$.

但是反过来对称双线性函数和斜对称双线性函数的正交关系都是对称的,于是证明了正交关系具有对称性当且仅当,双线性函数是对称的或者斜对称的.

对于对称或者斜对称的双线性函数,一个子集$S$的左右正交补是相同的,它称为子集的正交补,记作$S^{\perp}$.

$n$维线性空间$V$上全体对称双线性函数构成的集合$S(V,F)$是一个$F$线性空间,它在一组基下的矩阵表示是对称矩阵,这得到从$S(V,F)$到对称矩阵空间上的同构,于是看到$S(V,F)$的维数是$\frac{n^2+n}{2}$.类似的,$V$上全体斜对称双线性函数构成的$F$线性空间$A(V,F)$和斜对称矩阵空间是同构的,于是它的维数是$\frac{n^2-n}{2}$.按照双线性函数总可以唯一的表示为对称双线性函数和斜对称双线性函数的和,这对应于矩阵总可以唯一的表示为对称矩阵和斜对称矩阵的和,看到双线性函数空间有直和分解$L(V,F)\simeq S(V,F)\oplus A(V,F)$.

接下来介绍对称和斜对称双线性函数的标准型,即选取一种特殊的基,使得矩阵表示是一种简单形式的矩阵.这个问题也等价于说寻找对称矩阵或者斜对称矩阵在相合关系下的标准型.

给定域$F$上线性空间$V$上的对称双线性函数$f$,那么存在一组基下的矩阵表示为对角矩阵$D=\mathrm{diag}\{a_1,a_2,\cdots,a_r,\textbf{0}_{n-r}\}$.其中$a_i$是$F$上不为0的$r$个元,并且这里$r$是$f$的秩.即,在这组基下对称双线性函数具有表达式$f(\alpha,\beta)=a_1x_1y_1+a_2x_2y_2+\cdots+a_rx_ry_r$.这个命题对应的矩阵表述是,对$F$上的$n$阶对称矩阵$S$,存在可逆方阵$P$使得$P^TSP=\mathrm{diag}\{a_1,a_2,\cdots,a_r,\textbf{0}\}$,其中$a_i$是$F$上全不为0的$r=r(A)$个元.即一般域上对称矩阵总相合于对角矩阵.
\begin{proof}
	
	来对空间维数$n$归纳.当$n=1$时没什么需要证的.假设$n\ge2$,并且命题对$n-1$成立,来证明对$n$成立.首先不妨设双线性函数不是零函数,那么存在两个向量$\alpha_0,\beta_0$使得$f(\alpha_0,\beta_0)\not=0$,于是得到$f(\alpha_0+\beta_0,\alpha_0+\beta_0)-f(\alpha_0-\beta_0,\alpha_0-\beta_0)=4f(\alpha_0,\beta_0)\not=0$,于是$V$中必然存在一个向量$\xi_1$使得$a_1=f(\xi_1,\xi_1)\not=0$.把$V$中向量$\xi_1$生成的子空间记作$W$,如果证明了有$V=W\oplus W^{\perp}$.于是$W^{\perp}$的维数是$n-1$,现在把$f$限制到$W^{\perp}$上,按照归纳假设,存在基$\{\xi_2,\cdots,\xi_n\}$下的矩阵表示是$\mathrm{diag}\{a_2,a_3,\cdots,a_s,\textbf{0}_{n-s}\}$.于是看到$f$在基$\{\xi_1,\xi_2,\cdots,\xi_n\}$下的矩阵表示是$\mathrm{diag}\{a_1,a_2,\cdots,a_s,\textbf{0}_{n-s}\}$.最后按照秩的定义看到$s=r$,完成证明.
	
	最后来证明直和分解.首先取$\alpha\in W\cap W^{\perp}$,那么有$\alpha=a\xi_1$,于是得到$f(\alpha,\alpha)=a^2f(\xi_1,\xi_1)=0$,于是得到$a=0$,于是$W\cap W^{\perp}=\{0\}$.再来看$V=W+W^{\perp}$,为此任取$\alpha\in V$,考虑一个待定的向量$\beta=\alpha-\lambda\xi_1$,如果能让$\beta\in W^{\perp}$,就可以得到$V=W+W^{\perp}$.于是需要让$f(\xi_1,\beta)=0$,等价于$\lambda=\frac{f(\xi_1,\alpha)}{f(\xi_1,\xi_1)}$,按照$f(\xi_1,\xi_1)\not=0$,就完成证明.
	
\end{proof}

这里再给出一种矩阵角度的证明.即域$F$上$n$阶对称方阵总可以合同为对称矩阵.
\begin{proof}
	
	来对阶数$n$归纳.如果$n=1$没什么可证的,现在假设$n\ge2$,假设$n-1$时命题成立.现在取$n$阶对称矩阵$S$,不妨设它不是0矩阵,否则没什么需要证的.现在假设$S$存在一个对角元$a_{jj}\not=0$,那么可以取$P$是把第一行和第$j$行对换的第二类初等矩阵,于是有$S_1=P^TSP$的$(11)$元是$a_{jj}$非0.注意到$S_1$仍然是一个对称矩阵,于是这种情况下$S$合同于一个$(11)$元非0的对称矩阵.如果$S$的全不对角元都是0,那么按照$S\not=0$,知道存在一个$a_{jk}\not=0$,$j\not=k$.于是可以取初等矩阵$P$使得$S_1=P^TSP$的$(12)$元是$a=a_{jk}$非0,也就是说$S_1$的左上角的二阶子矩阵是$S_0=\left(\begin{array}{cc}
	0&a\\
	a&0\end{array}\right)$,取$P_0=\left(\begin{array}{cc}
	1&-1\\
	1&1\end{array}\right)$,于是$S_1$合同于:
	$$S_2=\left(\begin{array}{cc}
	P_0^T&0\\
	0&E_{n-2}\end{array}\right)S_1\left(\begin{array}{cc}
	P_0&0\\
	0&E_{n-2}\end{array}\right)$$
	
	那么$S_2$是和$S$合同的对称矩阵,并且$S_2$的$(11)$元是$2a$非0.于是,不妨设$S$的$(11)$元$b_1$非0,记$S=\left(\begin{array}{cc}
	b_1&\beta^T\\
	\beta&S_2\end{array}\right)$,那么它合同于:
	$$S'=\left(\begin{array}{cc}
	1&0\\
	-b_1^{-1}\beta&E_{n-1}\end{array}\right)\left(\begin{array}{cc}
	b_1&\beta^T\\
	\beta&S_2\end{array}\right)\left(\begin{array}{cc}
	1&-b_1^{-1}\beta^T\\
	0&E_{n-1}\end{array}\right)=\left(\begin{array}{cc}
	b_1&0\\
	0&S_2-b_1^{-1}\beta\beta^T\end{array}\right)$$
	
	现在注意到$S_2-b_1^{-1}\beta\beta^T$是$n-1$阶对称矩阵,于是按照归纳假设有$n-1$阶的可逆矩阵$P_1$使得$P_1^T(S_2-b_1^{-1}\beta\beta^T)P_1$是对角矩阵.于是取$P=\mathrm{diag}\{1,P_1\}$,就得到$P^TS'P$是对角矩阵,完成证明.
	
\end{proof}

如果把域$F$取为复数域$\mathbb{C}$,那么复对称矩阵总可以合同于$S=\mathrm{diag}\{a_1,a_2,\cdots,a_r,\textbf{0}_{n-r}\}$.其中$r$是矩阵的秩,并且$a_i$都是非0复数.如果取复数$b_i$是$a_i$的平方根,那么取$P=\{b_1^{-1},b_2^{-1},\cdots,b_r^{-1},\textbf{0}_{n-r}\}$.就有$P^TSP=\mathrm{diag}\{E_r,\textbf{0}_{n-r}\}$.于是看到复矩阵总可以合同于对角矩阵,其中对角元由0和1构成,1的个数就是矩阵的秩.

实数域上的情况比较不同.如果已经把实对称矩阵合同于对角矩阵$S=\mathrm{diag}\{a_1,a_2,\cdots,a_r,\textbf{0}_{n-r}\}$,那么如果取$P=\mathrm{diag}\{\sqrt{|a_1|}^{-1},\sqrt{|a_2|}^{-1},\cdots,\sqrt{|a_r|}^{-1},\textbf{0}_{n-r}\}$,就得到$P^TSP$是由若干1和-1和0构成的对角矩阵,其中1和-1的个数就是矩阵的秩.于是我们证明了,实对称矩阵必相合于对角矩阵$\mathrm{diag}\{E_s,-E_t,0_{n-r}\}$.

那么一个自然的问题是,这里的$s,t$是否是被实对称矩阵唯一决定的.即它们是否可以当作实对称矩阵合同关系下的不变量,如果是,那么它们完全刻画了这种合同关系.会从矩阵和空间两个角度证明这个事实.先考虑矩阵角度,为此需要Witt引理:

如果$S$是可逆的$n$阶实对称矩阵,$S_1,S_2$是两个$m$阶可逆实对称矩阵,那么如果$m+n$阶可逆矩阵$\mathrm{diag}\{S,S_1\}$和$\mathrm{diag}\{S,S_2\}$是合同的,那么$S_1$和$S_2$是合同的.
\begin{proof}
	
	设$n$阶可逆方阵$P$使得$S=PDP^{T}$,其中$D=\mathrm{diag}\{a_1,a_2,\cdots,a_n\}$是一个对角矩阵.于是得到:
	$$\mathrm{diag}\{S,S_i\}=\mathrm{diag}\{P,E_m\}\mathrm{diag}\{D,S_i\}\mathrm{diag}\{P,E_m\}^T$$
	
	于是问题转化为,如果$\mathrm{diag}\{a_1,a_2,\cdots,a_n,S_1\}$和$\mathrm{diag}\{a_1,a_2,\cdots,a_n,S_2\}$合同,证明$S_1$和$S_2$合同.断言一旦证明了这个命题对$n=1$成立,那么对任意的$n$都成立.事实上,运用$n=1$成立得到$\mathrm{diag}\{a_2,\cdot,a_n,S_1\}$和$\mathrm{diag}\{a_2,\cdots,a_n,S_2\}$合同,归纳下去,就得到$S_1$和$S_2$合同.
	
	现在假设$\mathrm{diag}\{t,S_1\}$和$\mathrm{diag}\{t,S_2\}$合同,其中按照$S$可逆,需要要求$t$是非0元.于是存在可逆矩阵$P=\left(\begin{array}{cc}
	a&\beta^T\\
	\alpha&P_1\end{array}\right)$,满足:
	$$\left(\begin{array}{cc}
	t&0\\
	0&S_2\end{array}\right)=\left(\begin{array}{cc}
	a&\beta^T\\
	\alpha&P_1\end{array}\right)\left(\begin{array}{cc}
	t&0\\
	0&S_1\end{array}\right)\left(\begin{array}{cc}
	a&\alpha^T\\
	\beta&P_1^T\end{array}\right)=\left(\begin{array}{cc}
	a^2t+\beta^TS_1\beta&at\alpha^T+\beta^TS_1P_1^T\\
	at\alpha+P_1S_1\beta&t\alpha\alpha^T+P_1S_1P_1^T\end{array}\right)$$
	
	于是得到等式关系:
	$$\left\{\begin{array}{c}
	a^2t+\beta^TS_1\beta=t\\
	at\alpha+P_1S_1\beta=0\\
	t\alpha\alpha^T+P_1S_1P_1^T=S_2\end{array}\right.$$
	
	如果这里$a=1$,那么可以把$-P$代替为$P$,于是总可以假设$a\not=1$,那么经过计算看到对$Q=P_1+(1-a)^{-1}\alpha\beta^T$,有$QS_1Q^T=S_2$.两侧取行列式,就得到$Q$是可逆的,完成证明.
	
\end{proof}

利用Witt引理,来证明实对称矩阵的惯性定理:实对称矩阵合同于对角矩阵$\mathrm{diag}\{E_p,-E_q,0_{n-r}\}$,那么这里的$p$和$q$是唯一确定的,称为实对称矩阵的正惯性指数和负惯性指数,把它们做差的绝对值称为符号差.
\begin{proof}
	
	假设两个$n$阶实对称矩阵$D_1=\mathrm{diag}\{E_p,-E_{r-p},0_ {n-r}\}$和$D_2=\mathrm{diag}\{E_q,-E_{s-q},0_{n-s}\}$是合同的.那么首先按照合同不改变秩,得到$r=s$.现在假设有可逆矩阵$P$使得$PD_1P^T=D_2$,把$P$分块为$\left(\begin{array}{cc}
	P_{11}&P_{12}\\
	P_{21}&P_{22}\end{array}\right)$,其中$P_{11}$是$r$阶矩阵,$P_{22}$是$n-r$阶矩阵,于是:
	$$D_2=\left(\begin{array}{cc}
	P_{11}&P_{12}\\
	P_{21}&P_{22}\end{array}\right)D_1\left(\begin{array}{cc}
	P_{11}&P_{12}\\
	P_{21}&P_{22}\end{array}\right)^T$$
	
	这就得到:
	$$\left(\begin{array}{cc}
	E_q&0\\
	0&-E_{r-q}\end{array}\right)=P_{11}\left(\begin{array}{cc}
	E_p&0\\
	0&-E_{r-p}\end{array}\right)P_{11}^T$$
	
	取行列式看到$P_{11}$是可逆的.于是$\left(\begin{array}{cc}
	E_q&0\\
	0&-E_{r-q}\end{array}\right)$和$\left(\begin{array}{cc}
	E_p&0\\
	0&-E_{r-p}\end{array}\right)$是合同的.假设$p\not=q$,不妨设$p>q$,按照Witt引理,得到$-E_{r-q}$和$\mathrm{diag}\{E_{p-q},-E_{r-p}\}$是合同的.也就是说存在可逆矩阵$Q$使得$\mathrm{diag}\{-E_{p-q},E_{r-p}\}=QQ^T$.但是如果记$Q=(q_{ij})$,得到$QQ^T$的$(11)$元是$q_{11}^2+q_{12}^2+\cdots+q_{1n}^2$,按照$Q$可逆得到这个元是正实数,它不能为$-1$,这就矛盾.
	
\end{proof}

现在来给出空间角度的证明.首先指出,对于实数域上的双线性函数,还存在一种依赖于实数序关系的性质,即正定性.给定实数域上线性空间$V$上的双线性函数$f$,称$f$是半正定的,如果对任意的$V$中向量$x$总有$f(x,x)\ge0$,如果取等号当且仅当$x=0$,就称$f$是正定的.对偶的,称$f$是半负定的,如果总有$f(x,x)\le0$,称负定的如果只有$x=0$的时候不等式取等号.

那么正负定条件会直接体现在合同标准型上,如果对称双线性函数是半正定的,它的矩阵表示合同到的对角矩阵,对角线上不能出现负数,否则就取到负实数.正定的时候,矩阵表示合同到的对角矩阵,只能是对角线上$n$个正数,否则就会取到非负数.据此,看到一个对称双线性函数半正定当且仅当它的矩阵表示合同于$\mathrm{diag}\{E_r,0_{n-r}\}$,其中$r$是秩.而对称双线性函数正定,当且仅当矩阵合同于$E_n$.对偶的有关于负定的命题.

惯性定理的空间版本.给定$n$维实线性空间$V$上的对称双线性函数$f$,那么$V$可以分解为直和$V=V^+\oplus V^-\oplus V^{\perp}$.其中$V^{\perp}$是$V$关于$f$的正交补,$f$限制在$V^+$和$V^-$上分别是正定和负定的.另外,倘若存在另一种直和分解$V=V_1^+\oplus V_1^-\oplus V^{\perp}$,使得$f$在$V_1^+$和$V_1^-$上的限制分别是正定和负定的,那么有$\dim V^+=\dim V_1^+$,$\dim V^-=\dim V_1^-$.
\begin{proof}
	
	存在性已经证明了.现在证明维数唯一性.设$\alpha\in V_1^+\cap(V^-\oplus V^{\perp})$,那么从$\alpha\in V_1^+ $有$f(\alpha,\alpha)\ge0$,从$\alpha\in V^-\oplus V^{\perp}$,有$\alpha=\beta+\gamma$,其中$\beta\in V^-$,$\gamma\in V^{\perp}$,于是得到$f(\alpha,\alpha)=f(\beta+\gamma,\beta+\gamma)=f(\beta,\beta)\le0$,于是得到$f(\alpha,\alpha)=0$,得到$\alpha=0$.于是有直和$V_1^+\oplus V^-\oplus V^{\perp}$.于是得到维数不等式$\dim V_1^+\le\dim V^+$,类似得到相反的不等式,于是维数相同.
	
\end{proof}

斜对称的情况.给定域$F$上$n$维线性空间$V$,取$V$上一个斜对称双线性函数$f$,那么存在一组基下的矩阵表示是准对角型矩阵$\mathrm{diag}\{A,\cdots,A,\textbf{0}_{n-2s}\}$,其中$A$是二阶方阵$\left(\begin{array}{cc}
0&1\\
-1&0\end{array}\right)$,并且$2s$是$f$的秩.即存在一组基下的表达式是$f(\alpha,\beta)=(x_1y_2-x_2y_1)+\cdots+(x_{2s-1}y_{2s}-x_{2s}y_{2s-1})$.这对应的矩阵结论是,对域$F$上的$n$阶斜对称方阵$K$,存在可逆矩阵$P$使得:
$$P^TKP=\mathrm{diag}\left\{\left(\begin{array}{cc}
0&1\\
-1&0\end{array}\right),\cdots,\left(\begin{array}{cc}
0&1\\
-1&0\end{array}\right),\textbf{0}_{n-2s}\right\}$$

\begin{proof}
	
	来对维数$n$归纳,当$n=1$时没什么需要证的.如果$n\ge2$,并且证明了维数小于$n$的情况下命题都成立,来证明$n$维的情况.首先可以不妨设$f$不是零映射.于是存在向量$\eta,\zeta\in V$使得$f(\eta,\zeta)=b\not=0$,记$\xi_1=\eta$,$\xi_2=b^{-1}\zeta$,那么有$f(\xi_1,\xi_1)=f(\xi_2,\xi_2)=0$,并且$f(\xi_1,\xi_2)=1$.断言$\{\xi_1,\xi_2\}$线性无关,否则有$a_1\xi_1+a_2\xi_2=0$,导致$f(\xi_1,a_1\xi_1+a_2\xi_2)=a_2=0$和$f(\xi_2,a_1\xi_1+a_2\xi_2)=-a_1=0$,这矛盾.现在记$\{\xi_1,\xi_2\}$生成的子空间是$W$,那么如果证明了$V=W\oplus W^{\perp}$,就可以把$f$限制到$W^{\perp}$上.按照归纳假设就可以完成证明.
	
	最后来证明直和$V=W\oplus W^{\perp}$.如果有$\alpha\in W\cap W^{\perp}$,设$\alpha=a_1\xi_1+a_2\xi_2$,证明$a_1=a_2=0$,得到$W\cap W^{\perp}=\{0\}$.接下来验证$V=W+W^{\perp}$.为此对任意$\alpha\in V$,取$\beta=\alpha-\lambda_1\xi_1-\lambda_2\xi_2$,按照正交要求求出$\lambda_i$就完成证明.
	
\end{proof}

关于斜对称矩阵的秩是偶数,还可以用初等变换来证明.事实上为此只要证明奇数阶斜对称矩阵的行列式必然为0.把奇数阶斜对称矩阵每一行乘以$-1$,那么整个矩阵的行列式应该变为相反数,但是这个新的矩阵是原矩阵的转置,它们的行列式应该相同,就得到行列式为0.

定义一种特殊的复数域上的双线性函数.给定复数域上线性空间$V$,称$f:V\times V\to\mathbb{C}$是共轭双线性函数,如果它满足:
$$f(\lambda_1\alpha_1+\lambda_2\alpha_2,\beta)=\lambda_1f(\alpha_1,\beta)+\lambda_2f(\alpha_2,\beta)$$
$$f(\alpha,\mu_1\beta_1+\mu_2\beta_2)=\overline{\mu_1}f(\alpha,\beta_1)+\overline{\mu_2}f(\alpha,\beta_2)$$

类似之前的情况,可以证明在取定一组基下共轭双线性函数对应于复矩阵,于是共轭双线性函数空间是一个$n^2$维的$\mathbb{C}$线性空间.另外共轭双线性函数在一组基下的矩阵如果是$A$,设基变换为另一组基的过渡矩阵是复矩阵$P$,那么在新的基下共轭双线性函数的矩阵表示是$P^*AP$,这里$P^*$表示矩阵$P$的共轭转置.把两个复矩阵$A,B$满足存在可逆复矩阵$P$使得$P^*AP=B$称为复相合关系,那么它是一个等价关系.处理共轭双线性函数在不同基下矩阵表示的标准型问题,就是找复相合关系下的标准型.另外按照复相合不改变秩,把共轭双线性函数的矩阵表示的秩称为共轭双线性函数的秩.如果这个矩阵满秩就称$f$是非退化的,否则称为退化的.

给定共轭双线性函数$f$,如果两个向量$\alpha,\beta$满足$f(\alpha,\beta)=0$,就称$\alpha$关于$f$左正交于$\beta$,也称$\beta$右正交于$\alpha$.对$V$的子集$S$,类似定义左右正交补,有类似的包含关系,并且,$V$的左右正交补的维数相同,它就是$n$减去$f$的秩.于是$f$非退化等价于左右正交补任一个是零子空间.

同样可以定义$V\to V^*$的线性映射为,把$\alpha$映射为$f(-,\alpha)$.于是$f$非退化当且仅当对任意$V$上线性函数$g$,有$\alpha\in V$使得$g=f(-,\alpha)$.

共轭双线性函数$f$的正交性具有对称性,即$f(\alpha,\beta)=0$等价于$f(\beta,\alpha)$,当且仅当,$f$满足$f(\alpha,\beta)=\overline{f(\beta,\alpha)}$或者$f(\alpha,\beta)=-\overline{f(\beta,\alpha)}$.这两种情况分别称为Hermite共轭双线性函数和斜Hermite共轭双线性函数.它们对应的矩阵表示就是分别满足$A=A^*$和$A=-A^*$的复矩阵,它们分别称为Hermite矩阵和斜Herimite矩阵.另外对于Hermite和斜Hermite共轭双线性函数,一个子集的左右正交补是相同的,就称为子集的正交补,同样记作$S^{\perp}$.

Hermite矩阵在复数域上的角色相当于实数域上的对称矩阵.Hermite矩阵的复相合标准型是和实对称矩阵相合标准型类似的.它同样有矩阵角度和空间角度两种证明,不再赘述.给定$n$阶Hermite矩阵$H$,那么存在可逆复矩阵$P$使得$P^*HP=\mathrm{diag}\{E_p,-E_q,0_{n-r}\}$.这里$p+q$是矩阵的秩,并且$p,q$是唯一确定的,称为Hermite矩阵的正惯性指数和复惯性指数,它们差的绝对值称为Hermite矩阵的符号差.

其中它的空间角度描述同样依赖于正定负定的概念.称一个共轭双线性函数$f$是半正定的,如果对任意的向量$x$有$f(x,x)$是一个非负实数,如果还要求取等号当且仅当$x$是零向量,那么就称$f$是正定的.
\newpage
\subsection{内积空间}

为了赋予线性空间度量,讨论诸如向量长度,向量夹角的概念,需要在线性空间上引入一种特殊的双线性函数,称为内积.本节把域$F$固定取做实数域$\mathbb{R}$或复数域$\mathbb{C}$.

给定复数域$\mathbb{C}$上的一个线性空间$V$,称$V$上的一个内积$f$,是指正定,共轭对称的共轭双线性函数$f(\bullet,\bullet):V\times V\to\mathbb{C}$的映射,即满足:
\begin{enumerate}
	\item 共轭对称性:$$f(x,y)=\overline{f(y,x)},\forall x,y\in V$$
	\item 共轭双线性:$$f(ax+by,z)=af(x,z)+bf(y,z),x,y,z\in V,a,b\in\mathbb{C}$$
	$$f(x,ay+bz)=\overline{a}f(x,y)+\overline{b}f(x,z),x,y,z\in V,a,b\in\mathbb{C}$$
	\item 正定性:$\forall x\in V,f(x,x)\ge0$,取等号当且仅当$x$是0向量.
\end{enumerate}

给定实数域$\mathbb{R}$上的一个线性空间$V$,称$V$上的一个内积$f$,是指正定对称的双线性函数$f(\bullet,\bullet):V\times V\to\mathbb{R}$的映射,即满足:
\begin{enumerate}
	\item 对称性:$$f(x,y)=f(y,x),\forall x,y\in V$$
	\item 双线性:$$f(ax+by,z)=af(x,z)+bf(y,z),x,y,z\in V,a,b\in\mathbb{R}$$
	$$f(x,ay+bz)=af(x,y)+bf(x,z),x,y,z\in V,a,b\in\mathbb{R}$$
	\item 正定性:$\forall x\in V,f(x,x)\ge0$,取等号当且仅当$x$是0向量.
\end{enumerate}

如果对实数域或者复数域上的有限维线性空间赋予内积,就称线性空间是内积空间.其中实内积空间也称为欧氏空间,复内积空间也称为酉空间.最简单的内积空间的例子是$\mathbb{R}^n$和$\mathbb{C}^n$,其中内积定义为$x=(x_i),y=(y_i)$有$(x,y)=\sum_{i=1}^{n}x_iy_i$.

内积满足Cauhcy不等式.即对任意向量$x,y$,有$|(x,y)|^2\le(x,x)(y,y)$.分实复情况分别证明这个命题.首先对于实情况,取实参数$t$,那么$x+ty$是一个向量,按照正定性它满足$(x+ty,x+ty)\ge0$,这等价于说$(x,x)+2t(x,y)+t^2(y,y)\ge0$对任意实数$t$成立,于是知道它的判别式不能大于0,于是有$(x,y)^2\le(x,x)(y,y)$.对于复情况,同样引入参数$t=a+ib$,其中$a,b$是实数,那么从$(x+ty,x+ty)\ge0$得到$(x,x)+t\overline{x,y}+\overline{t}(x,y)+|t|^2(y,y)\ge0$.当$y=0$时Cauchy不等式自然成立,现在假设$y\not=0$,那么配方有如下式,为了使得对任意$x,y$成立,那必须就有$|(x,y)|^2\le(x,x)(y,y)$.
$$(y,y)\left(a+\frac{\Re(x,y)}{(y,y)}\right)^2+(y,y)\left(b+\frac{\Im(x,y)}{(y,y)}\right)^2+(x,x)-\frac{|(x,y)|^2}{(y,y)}\ge0$$

内积可以诱导出一种向量的长度的定义,称为向量范数,线性空间上范数的标准定义是,一个从$V$到非负实数的映射$|\bullet|$,满足:
\begin{enumerate}
	\item 正定性:$|x|\ge0,\forall x\in V$,取等当且仅当$x$是零向量.
	\item 数乘性:$\forall x\in V,d\in\mathbb{F}$,有$|dx|=d|x|$.
	\item 三角不等式:$|x+y|\le|x|+|y|,\forall x,y\in V$.
\end{enumerate}

那么对于内积$(-,-)$,约定它诱导的范数是$|x|=\sqrt{(x,x)}$.注意到其中三角不等式是来自柯西不等式的.把长度为1的向量称为单位向量.于是每个非零向量,都可以通过除以它的范数得到一个单位向量.

和双线性函数的情况相同,称内积空间上两个向量$x,y$正交如果$(x,y)=0$.

证明过无论是实数域上的正定双线性函数还是复数域上的正定共轭双线性函数,都存在一组基下的矩阵表示是对角矩阵,其中对角元是正实数,甚至可以要求矩阵表示是单位矩阵.这等价于说,内积空间上存在基$\{e_1,e_2,\cdots,e_n\}$满足$e_i$是两两正交的,把这样的基称为内积空间的正交基.如果取矩阵表示是单位矩阵的基,那么这等价于说存在基$\{e_1,e_2,\cdots,e_n\}$满足每个向量都是单位长度,并且它们两两正交,这时候就称基为标准正交基.注意标准正交基的存在性已经在对称双线性函数矩阵表示标准型那里证明过了.

正交和线性无关存在联系,事实上一方面,如果实或复内积空间上有若干两两正交的非零向量组$\{a_1,a_2,\cdots,a_m\}$,断言这是一个线性无关组.如果存在一组域上的系数$\{\lambda_1,\lambda_2,\cdots,\lambda_m\}$满足$\sum_{i=1}^{m}\lambda_ia_i=0$,那么有$0=(0,a_j)=(\sum_{i=1}^{m}\lambda_ia_i,a_j)=\lambda_j$,于是所有$\lambda_j=0$.

但是反过来,线性无关组自然未必是两两正交的,不过可以通过一种称为Schmidt正交化的手段,把一组线性无关组转化为正交组,并且保证它们长成的子空间不变:给定实或复内积空间$V$上的一组基$\{\alpha_1,\alpha_2,\cdots,\alpha_n\}$,那么存在两两正交的非零向量$\{\beta_1,\beta_2,\cdots,\beta_n\}$,满足对任意$1\le k\le n$,有$\{\beta_1,\beta_2,\cdots,\beta_k\}$和$\{\alpha_1,\alpha_2,\cdots,\alpha_k\}$生成了相同的子空间.
\begin{proof}
	
	首先取$\beta_1=\alpha_1$,那么已经有$\{\beta_1\}$是线性无关组并且长成的子空间和$\{\alpha_1\}$相同.现在假设已经构造了两两正交的非零向量$\{\beta_1,\beta_2,\cdots,\beta_{k-1}\}$使得它们是$\{\alpha_1,\alpha_2,\cdots,\alpha_{k-1}\}$生成的子空间的一组基.现在要找向量$\beta_k$满足$\{\beta_1,\cdots,\beta_k\}$生成了和$\{\alpha_1,\cdots,\alpha_k\}$相同的子空间.为此待定一个向量:
	$$\beta_k=\alpha_k+\lambda_1\beta_1+\lambda_2\beta_2+\cdots+\lambda_{k-1}\beta_{k-1}$$
	
	那么不论系数取什么,已经有$\beta_k$非零并且两个向量组生成了相同的子空间.所以我们只需要要求$\beta_k$和每个$\beta_1,\beta_2,\cdots,\beta_{k-1}$正交,为此注意到$(\beta_k,\beta_j)=(\alpha_k,\beta_j)+\sum_{i=1}^{k-1}\lambda_i(\beta_i,\beta_j)=(\alpha_k,\beta_j)+\lambda_j(\beta_j,\beta_j)$,我们需要它为0,于是有$\lambda_j=-\frac{(\alpha_k,\beta_j)}{(\beta_j,\beta_j)}$,于是得到$\beta_k$表达式,从而完成归纳构造.
	$$\beta_k=\alpha_k-\frac{(\alpha_k,\beta_1)}{\beta_1,\beta_1}\beta_1-\frac{(\alpha_k,\beta_2)}{\beta_2,\beta_2}\beta_2-\cdots-\frac{(\alpha_k,\beta_{k-1})}{\beta_{k-1},\beta_{k-1}}\beta_{k-1}$$
	
\end{proof}

这实际上提供了内积空间上正交基存在的另一种证明,另外如果把每个向量$\beta_i$再除去它的范数,会把全部$\beta_i$不改变两两正交关系的前提下化作单位矩阵,于是这提供了内积空间上标准正交基存在的另一证明.但是Schmidt正交化过程实际上告诉更多:给定一组非零并且两两正交的向量组,它可以延拓为一组正交基.

另外,内积空间上的维数实际上和线性空间上的维数定义是不同的,在内积空间上定义维数是标准正交基的势,尽管在有限维线性空间上这两个维数定义相同,但是对于无限维线性空间上往往不同.

关于Schmidt正交化过程,它对应的矩阵解释是QR分解.
\begin{enumerate}
	\item 实情况.给定一个$m\times n$的实矩阵$A$,那么$A$总可以表示为一个$m$阶正交方阵$O$和一个$m\times n$的对角元非负的上三角方阵$T$的乘积,这里上三角方阵的定义是,记$r=\min\{m,n\}$,那么$T=(t_{ij})$满足$i>j$时$t_{ij}=0$.当$A$是可逆方阵的时候,这种表示$A=OT$是唯一的.
	\item 复情况.给定一个$m\times n$的复矩阵$A$,那么$A$总可以表示为一个$m$阶酉矩阵$U$和一个$m\times n$的对角元非负的上三角矩阵$T$的乘积.并且如果$A$是可逆复方阵那么这种表示是唯一的.
\end{enumerate}
\begin{proof}
	
	来对实的情况证明这个命题,注意复情况是类似的.首先来证明存在性,对阶数归纳,设$A$的第一列是$\alpha_1$,如果$\alpha_1\not=0$,那么$\beta_1=\frac{\alpha_1}{|\alpha_1|}$可以扩充为$\mathbb{R}^m$上的一组标准正交基$\{\beta_1,\beta_2,\cdots,\beta_m\}$,那么取$P=(\beta_1,\beta_2,\cdots,\beta_m)$,得到$P^TP=E$,于是$P$是正交矩阵.并且有$A=P\left(\begin{array}{cc}
	|\alpha_1|&*\\
	0&A_1\end{array}\right)$,而当$\alpha_1=0$的时候直接取$P=E_m$就有$A=P\left(\begin{array}{cc}
	|\alpha_1|&*\\
	0&A_1\end{array}\right)$.于是无论如何有正交矩阵$P$满足这个等式.接下来按照归纳假设有$A_1=O_1T_1$,其中$O_1$是$m-1$阶正交矩阵,$T_1$是对角元全部非负的上三角矩阵,于是取$O=P\mathrm{diag}\{1,O_1\}$,取相应的$T=\left(\begin{array}{cc}
	|\alpha_1|&*\\
	0&T_1\end{array}\right)$就得到$A=OT$满足要求.
	
	现在假设$A$可逆,有两种分解$A=OT=O_1T_1$,其中$T$和$T_1$只能是对角线全部为正的上三角矩阵,因为$T=O^{-1}A$可逆,不可能有0特征值.现在把等式写作$O_1^TO=T_1T^{-1}=C$,那么$C$是正交矩阵,还是上三角矩阵,这只能说明$C$是对角元为正负1的对角矩阵,结合$T_1,T^{-1}$对角元都是正数,得到$C=E$,于是完成了唯一性证明.
	
\end{proof}

线性空间上讨论基,在内积空间上赋予了新的结构,于是提及内积空间上的基,理应考虑标准正交基.在线性空间上,两组基之间的过渡矩阵是一种特殊的矩阵,即可逆矩阵,当考虑内积空间上的标准正交基之间的过渡矩阵时,它应该是一种比可逆矩阵更特殊的矩阵,事实上实数域上它对应于正交矩阵,即满足转置是逆的矩阵$PP^T=E$;在复数域上它对应于酉矩阵,即满足共轭转置是逆的矩阵$UU^*=E$.

\begin{proof}
	
	这个证明是直接的,如果取两组基和过渡矩阵为:
	$$(\eta_1,\eta_2,\cdots,\eta_n)=(\xi_1,\xi_2,\cdots,\xi_n)M$$
	
	其中记$M=(m_{ij})$.对实数域的情况,有$\eta_i=\sum_{j=1}^{n}m_{ji}\xi_j$,于是按照正交性,得到$(\eta_i,\eta_j)=\delta_{ij}$等价于$\sum_{k=1}^{n}m_{ki}m_{kj}=\delta_{ij}$,即$MM^T=E$.复数域情况是类似的.
	
\end{proof}

反过来,如果取定内积空间上的一组标准正交基$\{\xi_1,\xi_2,\cdots,\xi_n\}$,对实复的情况分别取$P$为正交矩阵和酉矩阵,那么$(\xi_1,\xi_2,\cdots,\xi_n)P$同样是标准正交基.即,能过渡实复内积空间上标准正交基的矩阵恰好就是正交/酉矩阵.

正交/酉矩阵的另一个等价描述是,方阵的行向量组恰好构成$\mathbb{R}^n$或$\mathbb{C}^n$的一组标准正交基,也等价于列向量组构成一组标准正交基.为此,只要注意到$\mathbb{R}^n$上的内积定义为$x^Ty$,$\mathbb{C}^n$的内积定义为$x^*y$.这只要注意到,在实情况下$PP^T=E$等价于说$P$的行向量组是一组标准正交基,$P^TP=E$等价于$P$的列向量组是一组标准正交基.对复情况是类似的.

可以写出全部二阶实正交矩阵,它们是如下形式:
$$\left(\begin{array}{cc}
\cos\theta&-\sin\theta\\
\sin\theta&\cos\theta\end{array}\right);\left(\begin{array}{cc}
\sin\theta&\cos\theta\\
\cos\theta&-\sin\theta\end{array}\right),\theta\in[0,2\pi)$$

正交矩阵和酉矩阵在内积空间中扮演非常重要的角色,来从矩阵和映射两种角度详细讨论.先来看正交矩阵和酉矩阵的一些基本性质.首先正交/酉矩阵的逆矩阵都是正交/酉矩阵;两个正交/酉矩阵的乘积仍然是正交/酉矩阵;正交方阵的行列式是$\pm1$,酉方阵的行列式是模长为1的复数.按照这些基本性质,看到全体$n$阶正交矩阵和全体$n$阶酉矩阵构成的集合其实都是群,分别记作$O_n(\mathbb{R})$和$U_n(\mathbb{C})$,称为$n$阶正交群或者$n$阶酉群.它是一般线性群的子群.如果把这个群看作$R^{n^2}$/$C^{n^2}$的子集,赋予子空间拓扑,那么这是一个紧集.来说明这个结论.注意到欧式空间的子集是紧集当且仅当是有界闭集.首先按照等价定义,正交矩阵/酉矩阵的行向量组每个向量模长1,这便给出了一个界.最后说明这是闭集,这只需要说明保极限点,如果一列正交/酉矩阵满足$U_n\to U$,那么在$U_n^TU_n=E$或者$U_n^*U_n=E$取极限就看出$U$也是正交/酉矩阵.

现在定义相关的映射.给定实/复内积空间中的线性变换$f$,称它是正交变换/酉变换,如果满足对任意的向量$x$总有$(f(x),f(x))=(x,x)$,或者等价的说,$f$是保范数不变的线性映射$|f(x)|=|x|$.

那么首先,保范数是和保内积等价的,即线性变换$f$保范数不变$|f(x)|=|x|$,当且仅当它保内积不变,即$(f(x),f(y))=(x,y),\forall x,y\in V$.只要说明必要性,对于实情况.注意到对任意的向量$x,y$和实数$t$,总有$(f(x+ty),f(x+ty))=(x+ty,x+ty)$.展开就看到$(f(x),f(y))=(x,y)$.

实/复内积空间上的线性变换是正交/酉变换当且仅当它的任一矩阵表示(在标准正交基下)是正交/酉矩阵.
\begin{proof}
	
	只来证明实情况.充分性,给定线性变换$f$在一组标准正交基下的矩阵表示是$A$,那么任取向量$\alpha$在这组基下的表示是$x=(x_1,x_2,\cdots,x_n)^T$,设$f(\alpha)=\beta$在这组基下的表示是$y=A(x_1,x_2,\cdots,x_n)^T$.于是内积$(\beta,\beta)=y^TEy=y^Ty=(Ax)^T(Ax)=x^Tx=(\alpha,\alpha)$,于是$f$是保范数的.必要性.如果有对任意$x$和$y=Ax$有$x^Tx=y^Ty$,等价于对任意的$x$有$x^TA^TAx=x^Tx$,于是等价于对对称矩阵$S=E-A^TA$,总有$x^TSx=0$.按照实数域上对称矩阵的合同标准型,这个对称矩阵$S$的正负惯性指数必然都是0,否则会取非0数,导致$S=0$,于是$A^TA=0$.
	
\end{proof}

于是,看到之前所证明的标准正交基的过渡矩阵是正交矩阵/酉矩阵,这个命题的映射角度描述是,实/复内积空间上的线性变换是正交/酉变换,当且仅当这个变换把每个标准正交基都映射为标准正交基.

正交补.在定义双线性函数或者共轭双线性函数时定义过左右正交补的概念,并且指出对于对称或者斜对称的双线性函数,以及对于共轭对称或者斜共轭对称的共轭双线性函数,左右正交补是一致的.但是一般对称双线性函数或者共轭对称的共轭双线性函数上,子空间和正交补可能会有非平凡的交集的.不过在正定条件下这种情况并不存在,事实上在正定条件下可以证明更多:给定实或复内积空间$V$上的子空间$U$,那么总有直和分解$V=U\oplus U^{\perp}$.
\begin{proof}
	
	设$U$作为线性子空间的维数是$k$,那么内积限制在$U$上得到了一个$k$维内积空间,取它的标准正交基$\{\zeta_1,\zeta_2,\cdots,\zeta_k\}$,现在把它延拓为$V$上的标准正交基$\{\zeta_1,\zeta_2,\cdots,\zeta_n\}$,设$\{\zeta_{k+1},\zeta_{k+2},\cdots,\zeta_n\}$生成的子空间是$W$,于是$V=U\oplus W$,最后验证$W=U^{\perp}$.
	
\end{proof}

另外,同样的会有$V=U^{\perp}\oplus(U^{\perp})^{\perp}$,再结合$U\subset (U^{\perp})^{\perp}$,看到在有限维内积空间上子空间$U$总有$U=(U^{\perp})^{\perp}$.
\newpage
\subsection{内积空间上的标准型}

接下来要讨论内积空间上的线性变换的矩阵表示,如何选取特定的标准正交基使得它具有简单形式.类似线性空间上的相抵和相似问题,内积空间上的线性变换也归结为两种问题.第一种,对于两个内积空间之间的线性变换,如何分别选取两组标准正交基使得矩阵具有简单形式,用矩阵语言说,给定$m\times n$的实或复矩阵$A$,如何找到$m$阶和$n$阶的正交或酉矩阵$P,Q$,使得$PAQ$具有简单形式,这就是奇异值分解定理,会看到这样的简单形式就是对角矩阵,其中对角元是所谓的奇异值.第二种,同一个内积空间上的线性变换,如何选取同一组标准正交基使得矩阵具有简单形式,矩阵语言就是,给定$n$阶实或复矩阵$A$,如何找到正交或者酉矩阵$P$,使得$P^TAP$或者$P^*AP$具有简单形式.即实复方阵在正交相似和酉相似关系下的简单形式.这个问题会给出两个回答,对一般的复矩阵和实矩阵证明它们可以分别酉相似和正交相似到上三角矩阵和准上三角矩阵,对于一种特殊的矩阵复或实正规矩阵证明它们可以酉相似和正交相似到对角矩阵和准对角矩阵.

称两个$m\times n$的实矩阵$A,B$是正交相抵或者正交等价的,如果存在$m$阶正交矩阵$P$和$n$阶正交矩阵$Q$使得$A=PBQ$,称两个$m\times n$的复矩阵$A,B$是酉相抵或者酉等价的,如果存在$m$阶酉矩阵$U$和$n$阶酉矩阵$V$使得$A=UBV$.称两个实方阵$A,B$是正交相似的,如果存在正交矩阵$P$满足$A=P^TBP$,称两个复方阵$A,B$是酉相似的,如果如果存在酉矩阵$U$使得$A=UBU^*$.注意到正交/酉相似是正交/酉相抵的特殊情况.并且上述四个关系都是同型矩阵上的等价关系

对酉/正交等价的第一个观察是,正交/酉等价的矩阵具有相同的全部元素的模长平方和.注意到对任意的$m\times n$的矩阵$A$,它的的元素模长和就是$\mathrm{tr}(AA^*)$.于是倘若$B=UAV$,其中$U,V$是酉矩阵,那么有$\mathrm{tr}(BB^*)=\mathrm{tr}(UAVV^*A^*U^*)=\mathrm{tr}(U(AA^*U^*))=\mathrm{tr}(AA^*U^*U)=\mathrm{tr}(AA^*)$.

曾经证明过域上一个方阵的特征多项式可裂当且仅当它可以相似上三角化.这里来证明,对于实数域上的方阵$A$,设它的实特征值为$\lambda_1,\lambda_2,\cdots,\lambda_t$,复特征值为$a_j+ib_j,j=1,2,\cdots,s$,它们都计了代数重数,那么$2s+t=n$就是阶数.则$A$可以相似为如下准上三角型.特别的,证明过程告诉实线性变换总会有1维或者2维不变子空间.
$$\left(\begin{array}{cccccc}
\left(\begin{array}{cc}
a_1&-b_1\\
b_1&a_1\end{array}\right)&&&&&\\
&\ddots&&&*&\\
&&\left(\begin{array}{cc}
a_s&-b_s\\
b_s&a_s\end{array}\right)&&&\\
&&&\lambda_1&&\\
&0&&&\ddots&\\
&&&&&\lambda_t\end{array}\right)$$
\begin{proof}
	
	来对阶数$n$归纳.对$n=1$是平凡的,现在假设$n\ge2$,假设对小于$n$的阶数的方阵命题成立.取一个$n$阶实方阵$A$,如果$A$有实特征值$\lambda$,取特征向量是$\alpha_1$,把它扩充成一组基$\{\alpha_1,\alpha_2,\cdots,\alpha_n\}$,设凑成的可逆矩阵为$P$,那么$P^{-1}AP=P^{-1}A(\alpha_1,\cdots,\alpha_n)=P^{-1}(\lambda\alpha_1,\cdots,A\alpha_n)$,按照$P^{-1}P=E$得到$P^{-1}\alpha_1=e_1$,于是$P^{-1}AP$的第一列是向量$\lambda e_1$.于是$P^{-1}AP=\left(\begin{array}{cc}
	\lambda&*\\
	0&A_1\end{array}\right)$.按照归纳假设,存在$n-1$阶的$Q_1$使得$Q_1^{-1}A_1Q_1$是所要求的形式.那么取$Q=P\left(\begin{array}{cc}
	1&0\\
	0&Q_1\end{array}\right)$,就得到$Q^{-1}AQ$是所要求的形式.注意到二阶对角块对应于共轭复特征值,一阶对角元对应实特征值,所以对角块在不计顺序意义下是唯一的.
	
	倘若$A$没有实特征值,那么它有一对共轭复特征值$a+bi$,$b\not=0$.设$a+bi$的特征向量是$x+yi$,断言向量$x,y$是线性无关的,首先,$x,y$都不是零向量,否则,比方说$y=0$,那么有$Ax=(a+bi)x$,这迫使$b=0$矛盾,对$x=0$是同理的.接下来如果$x=ky$,其中$k$是实数,那么有$A(x+yi)=(k+i)Ay=(a+bi)(k+i)y$,导致$Ay=(a+bi)y$,同样得到$b=0$矛盾.现在展开$A(x+yi)=(a+bi)(x+yi)$得到$Ax=ax-by,Ay=ay+bx$.于是$\{x,y\}$生成的二维子空间是$A$的不变子空间,现在把$\{x,y\}$扩充为一组基,凑成一个可逆矩阵$P$,那么得到$P^{-1}AP$具有形式$\left(\begin{array}{cc}
	A&*\\
	0&A_1\end{array}\right)$,其中$A=\left(\begin{array}{cc}
	a&-b\\
	b&a\end{array}\right)$,于是同样按照归纳假设把$A_1$相似为要求的形式,就会完成归纳.
	
\end{proof}

在正交相似和酉相似下,实复矩阵也可以得到相应的上三角化结论,这样的上三角矩阵称为Schur型.一个复矩阵可以酉相似为上三角矩阵.一个实矩阵可以正交相似为准上三角型,其中对角块是一阶和二阶的,一阶的是实特征值,二阶矩阵未必具有上述简单形式,但是它的特征值就是原矩阵的一对共轭特征值.简要解释下这个证明.在之前的相似上三角化的情况下,都是对阶数归纳证明,当选出一个特征向量或者选出实矩阵一个复特征值对应的特征向量的实虚部后,不直接扩充为一组基,而是扩充为一组标准正交基,那么按照Schmidt正交化过程,向量组生成的子空间是不变的,于是导致矩阵同样会正交相似或者酉相似于上三角型或者准上三角形.但是由于正交化过程会改变向量,于是在实情况下矩阵表示不再是上述简单形式$\left(\begin{array}{cc}
a&-b\\
b&a\end{array}\right)$,而是一个并不确定的二阶矩阵,但是它的特征值固定的是$a\pm ib$.

Schur不等式.给定一个复方阵$A=(a_{ij})$,它可以取为实矩阵,它的全体特征值计重数意义下记作$\{\lambda_i,1\le i\le n\}$,$A$在复数域上可以酉相似上三角化,其中对角元就是特征值的排列.知道酉相似是酉相抵的特例,而酉相抵的矩阵具有相同的元素模长平方和,由此得到:$\sum_{i,j=1}^{n}|a_{ij}|^2\ge\sum_{i=1}^{n}|\lambda_i|^2$.

Schur不等式取等号当且仅当复方阵可以酉对角化.事实上一方面,酉相似不改变矩阵的项的模长平方和,如果复方阵可以酉对角化,那么二者的元素模长平方和不变,于是得到Schur不等式取等号.反过来,如果取等号,那么酉相似得到的上三角化方阵不得不是一个对角方阵.

为了引出正规矩阵,先来介绍伴随变换.在内积空间上每个线性变换都附带着一个伴随变换.首先,证明过无论实数域上的双线性函数还是复数域上的共轭双线性函数$f$,它非退化的一个等价描述是,对$V$上任意一个线性函数$g$,存在一个向量$\gamma$满足$g=f(-,\gamma)$.注意到正定的情况下双线性函数必须是非退化的.于是,对于内积空间上的线性变换$f$,如果取定向量$\beta$,有$g(-)=(f(-),\beta)$是一个线性函数,于是存在向量$\beta'$满足$(f(-),\beta)=(-,\beta')$.另外这样的$\beta'$必然是唯一的,否则如果对任意向量$\alpha$总有$(\alpha,\beta')=(\alpha,\beta'')$,那么得到$\beta'-\beta''\in V^{\perp}$.但是知道正定条件下$V^{\perp}=\{0\}$.

由此,得到了一个$V$上的映射$f^*:\beta\mapsto\beta'$.验证它是$V$上线性映射,称这个线性映射是$f$的伴随映射.按照定义$(f(\alpha),\beta)=(\alpha,f^*(\beta))$,容易验证一些基本的性质:$(f+g)^*=f^*+g^*$;$(\lambda f)^*=\lambda(f^*)$;$(f\circ g)^*=g^*\circ f^*$;$(f^*)^*=f$.

给定线性变换$f$在一组标准正交基下的矩阵是$A$,那么$f$的伴随变换$f^*$在相同的标准正交基下的矩阵表示,在实情况下是取转置$A^T$,在复情况下是取共轭转置$A^*$.
\begin{proof}
	
	以实情况为例.记$A=(a_{ij})$.取这组标准正交基为$\{\xi_1,\xi_2,\cdots,\xi_n\}$,于是$f(\xi_i)=\sum_{k=1}^{n}a_{ki}\xi_k$.现在记$f^*$在这组基下的矩阵表示为$B=(b_{ij})$,那么有$f^*(\xi_i)=\sum_{k=1}^{n}b_{ki}\xi_k$.按照定义,有$(f(\xi_i),\xi_j)=(\xi_i,f^*(\xi_j))$,这得到$a_{ji}=b_{ij}$,完成证明.
	
\end{proof}

在内积空间$V$上,如果子空间$U$是线性变换$f$的不变子空间,那么$U^{\perp}$是伴随变换$f^*$的不变子空间.事实上任取$x\in U^{\perp}$,任取$y\in U$,那么$(y,f^*(x))=(f(y),x)=0$.

如果内积空间上的线性变换$f$和伴随变换可交换,即满足$ff^*=f^*f$,就称$f$是正规变换.转化维矩阵语言.如果复矩阵$A$满足$A^*A=AA^*$就称为正规矩阵,实矩阵$A$满足$A^TA=AA^T$就称为实正规矩阵,那么实正规矩阵就是项都是实数的正规矩阵.所遇到的全部特殊的矩阵,正交矩阵,酉矩阵,对称矩阵,Hermite矩阵都是正规矩阵的特例.

内积空间上的线性变换$f$是正规变换,当且仅当对任意向量$x$,有$|f(x)|=|f^*(x)|$.首先一方面无论是实复情况,从正规变换说明$(f(x),f(x))=(f^*(x),f^*(x))$.另一方面,对于实情况有$2(f(x),f(y))=(f(x+y),f(x+y))-(f(x-y),f(x-y))=(f^*(x+y),f^*(x+y))-(f^*(x-y),f^*(x-y))=(f^*(x),f^*(y))$.而对于复情况稍微麻烦一点,用到等式:
$$4(x,y)=(x+y,x+y)-(x-y,x-y)+i(x+iy,x+iy)-i(x-iy,x-iy)$$

如果一个正规矩阵是分块上三角的,那么它必然是分块对角的,并且每个对角块都是正规矩阵.即如果$A=\left(\begin{array}{cc}
A_1&A_2\\
0&A_3\end{array}\right)$是实或复的正规矩阵,那么必然有$A_2=0$并且$A_1,A_3$都是正规矩阵.为此,计算$AA^*$和$A^*A$的左上角分块,得到$A_1A_1^*+A_2A_2^*=A_1^*A_1$,等式两边取迹,得到$\mathrm{tr}(A_2A_2^*)=0$,但是我们知道无论实复情况,$\mathrm{tr}(BB^*)$表示的是$B$的全体元素的模长平方和,它为0当且仅当矩阵$B$是零矩阵.于是看到$A_2=0$,也就得到$A_1,A_3$都是正规矩阵.

对于正规变换$f$,如果$W$是$f$的不变子空间,那么$W^{\perp}$也是$f$的不变子空间.事实上取$x\in W^{\perp}$,假如$f(x)\not\in W^{\perp}$,那么按照$V=W\oplus W^{\perp}$得到$f(x)=y+z,y\in W,z\in W^{\perp}$,那么按照$W^{\perp}$是$f^*$的不变子空间,得到$(x,f^*(z))=0$,于是$(f(x),z)=0$,导致$(z,z)=0$,于是$z=0$.

下面给出正规矩阵的等价描述,先来看复数域的情况:
\begin{enumerate}
	\item $A$是正规矩阵.
	\item $A$可以酉对角化.
	\item Schur不等式取等,即$\sum_ {i,j=1}^{n}|a_{ij}|^2=\sum_{i=1}^{n}|\lambda_i|^2$.
	\item $A$具有$n$个两两正交的特征向量.
\end{enumerate}
\begin{proof}
	
	1和2的等价性.1推2,首先复矩阵$A$可以酉相似上三角化.但是在上面已经证明了如果一个准上三角阵是正规矩阵,那么它必然是准对角矩阵,据此得到一个正规矩阵是上三角矩阵那么它必然是对角矩阵,于是正规矩阵可以酉对角化.反过来2推1,注意到如果$U^*AU=D$是对角矩阵,其中$U$是酉矩阵,那么有$AA^*=UDU^*UD^*U^*=UDD^*U^*=UD^*DU^*=A^*A$.
	
	2和3的等价性已经证明过了.2和4的等价性.如果$U^*AU$是对角矩阵$D$,于是等价于说$AU=UD$,于是$U$的列向量组就是$n$个两两正交的特征向量.反过来,如果$A$有两两正交的$n$个特征向量,那么把它们单位化,即模长为1,然后作为列向量组构成一个酉矩阵$U$,那么就有$U^*AU$是对角矩阵.
	
\end{proof}

于是看到复正规矩阵必然酉相似于对角矩阵,其中对角元是全部计重数意义下的特征值.这也告诉,两个同阶正规矩阵是相似的当且仅当是酉相似的当且仅当它们具有相同的特征多项式.

指出过之前遇到的几乎全部特殊的矩阵都是正规矩阵的特例,这里从映射角度解释这些特殊矩阵,这些变换都是正规变换的特例.给定复内积空间$V$上的线性变换$f$:
\begin{enumerate}
	\item 称$f$为酉变换,如果它满足$ff^*=f^*f=\mathrm{Id}$,于是等价于矩阵表示是酉矩阵,等价于:
	$$(f(x),f(y))=(x,y),\forall x,y\in V$$
	\item 称$f$为自伴变换,如果它满足$f=f^*$,于是等价于矩阵表示是Hermite矩阵,等价于:$$(f(x),y)=(x,f(y)),\forall x,y\in V$$
	\item 称$f$为斜自伴变换,如果满足$f=-f^*$,于是等价于矩阵表示是斜Hermite矩阵,等价于$$(f(x),y)=-(x,f(y)),\forall x,y,\in V$$
\end{enumerate}

那么这三种矩阵的酉相似标准型,就是它们全部特征值构成的对角矩阵.为此只要探究出这三种矩阵的特征值的性质,就得到相应标准型:
\begin{enumerate}
	\item 酉矩阵的特征值的模长都是1,为此,注意到如果$Ax=\lambda x$,那么取共轭转置得到$x^*A^*=\overline{\lambda}x^*$,于是把第二个式子左乘到上式,得到$x^*A^*Ax=|\lambda|^2x^*x$,按照$A^*A=E$和$x\not=0$,就得到$|\lambda|^2=1$.于是,酉矩阵的酉相似标准型为如下矩阵,其中$e^{i\theta_j},1\le j\le n$是它的全部特征值.
	$$\mathrm{diag}\{e^{i\theta_1},e^{i\theta_2},\cdots,e^{i\theta_n}\}$$
	\item Hermite矩阵的特征值必然是实数.为此,注意到如果有$Ax=\lambda x$,那么一方面左乘$x^*$得到$x^*Ax=\lambda x^*x$,另一方面对原式取共轭转置再右乘$x$得到$x^*Ax=\overline{\lambda}x^*x$,结合$x\not=0$,看到$\lambda=\overline{\lambda}$,于是$\lambda$是实数.于是,Hermite矩阵的酉相似标准型为一个实对角矩阵,它的对角元$\lambda_1\ge\lambda_2\ge\cdots\ge\lambda_n$是它的全部特征值.
	$$\mathrm{diag}\{\lambda_1,\lambda_2,\cdots,\lambda_n\}$$
	\item 斜Hermite的特征值必然是0或者纯虚数.这个证明是同上的,于是斜Hermite矩阵的酉相似标准型如下,其中$\lambda_i$是实数,对角元是它的全部特征值.
	$$\mathrm{diag}\{i\lambda_1,i\lambda_2,\cdots,i\lambda_r,\textbf{0}_{n-r}\}$$
\end{enumerate}

复正规矩阵还存在一个有趣的等价描述.一个复矩阵$A$是正规矩阵,当且仅当它的共轭转置$A^*$可以表示为$A$的多项式.事实上一方面如果$A^*=f(A)$,那么必然有$A$和$A^*$可交换.另一方面,假设$A$是正规矩阵,设$A$的全部不同特征值是$\lambda_i,1\le i\le s$,设$A$的$\lambda_i$的特征子空间为$V_i$,于是按照酉对角化,得到了直和分解$V=\mathbb{C}^n=\oplus_{1\le i\le s}V_i$.设$A$在$V_i$的投影为$E_i$,那么有$A=\sum_{1\le i\le s}\lambda_iE_i$.并且每个$E_i$满足$E_i^2=E_i$和$i\not=j$时$E_iE_j=0$.据此得到对任意的正整数$k$有$A^k=\sum_{1\le i\le s}\lambda_i^kE_i$,于是对任意复数域多项式$f$有$f(A)=\sum_{1\le i\le s}f(\lambda_i)E_i$.现在取多项式$p_j(\lambda)=\prod_{i\not=j}\frac{\lambda-\lambda_i}{\lambda_j-\lambda_i}$,于是当$i\not=j$时候有$p_j(\lambda_i)=0$,并且$p_j(\lambda_j)=1$.于是$p_j(A)=\sum_{k=1}^{s}p_j(\lambda_k)E_k=E_j$.于是每个$E_i=p_i(A)$,最后取$p(x)=\sum_{k=1}^{s}\overline{\lambda_k}p_k(x)$,得到$p(A)=A^*$.

引理.如果复矩阵$B$和正规矩阵$A$可交换,那么$B$和$A^*$也可交换.事实上,不妨设$A$是对角矩阵$\mathrm{diag}\{\lambda_1E_1,\cdots,\lambda_sE_s\}$,否则选取合适的酉矩阵$U$,以$U^*AU$和$U^*BU$分别代替$A,B$.那么对$B$做相同的分块$B=(B_{ij})$为$s\times s$的分块矩阵.那么按照可交换性,得到$i\not=j$的时候$B_{ij}=0$,于是$B$是准对角矩阵$\mathrm{diag}\{B_1,\cdots,B_s\}$.那么$B$自然也就和$A^*=\mathrm{diag}\{\overline{\lambda_1}E_1,\cdots,\overline{\lambda_s}E_s\}$交换.

特别的,如果$A,B$都是复正规矩阵,并且$AC=CB$,那么有$A^*C=CB^*$.事实上取$X=\left(\begin{array}{cc}
A&0\\
0&B\end{array}\right)$和$Y=\left(\begin{array}{cc}
0&C\\
0&0\end{array}\right)$,那么$X$是正规矩阵,套用上述引理就得证.

现在来讨论实的正规矩阵.按照定义,实正规矩阵就是实的复正规矩阵.先来给出实正规矩阵正交相似标准型.给定实正规矩阵$A$,如果设$A$的全部实特征值是$\lambda_1,\lambda_2,\cdots,\lambda_t$,成对出现的复特征值是$a_k+ib_k,1\le k\le s$.那么$A$正交相似于如下矩阵:
$$\mathrm{diag}\left\{\left(\begin{array}{cc}
a_1&-b_1\\
b_1&a_1\end{array}\right),\cdots,\left(\begin{array}{cc}
a_s&-b_s\\
b_s&a_s\end{array}\right),\lambda_1,\cdots,\lambda_t\right\}$$

简要说明下证明过程.和实矩阵正交相似准上三角化的证明几乎相同.也是对阶数归纳.如果存在实特征值,取一个实特征向量,那么它生成了一维不变子空间,但是这里不直接把这个向量扩充为一组基,而是扩充为一组标准正交基,按照正规矩阵的不变子空间的正交补也是不变子空间,就得到正规矩阵正交相似为一个准对角矩阵,继续操作下去.如果不存在实特征值,取一个复特征向量,取实虚部,那么这是线性无关的两个向量,否则会导致存在实特征值,于是这生成了一个二维不变子空间.接下来不直接把这两个向量扩充成基,而是按照Schmidt正交化过程,转化为一组标准正交基,那么这保证了新的基的前两个向量生成的二维不变子空间和之前的相同.于是在这组基下正规矩阵正交相似于准对角型,接下来继续归纳.

类似于复正规矩阵的情况,实正规矩阵同样包含了三类已经见过的特殊实矩阵,这里描述它们对应的线性变换,以及相应的正交相似标准型.给定实内积空间$V$上的线性变换$f$.
\begin{enumerate}
	\item 称$f$是正交变换,如果满足$ff^*=f^*f=\mathrm{id}$,于是等价于说矩阵表示是正交矩阵,等价于说满足$(f(x),f(y))=(x,y),\forall x,y\in V$.按照正交矩阵的复特征值模长1,看到正交矩阵的实特征值只能是$\pm1$,于是正交矩阵的正交相似标准型为:
	$$\mathrm{diag}\left\{E_s,-E_t,\left(\begin{array}{cc}
	\cos\theta_1&\sin\theta_1\\
	-\sin\theta_1&\cos\theta_1\end{array}\right),\cdots,\left(\begin{array}{cc}
	\cos\theta_p&\sin\theta_p\\
	-\sin\theta_p&\cos\theta_p\end{array}\right)\right\},\theta_i\in \mathbb{R},s+t+2p=n,s,t,p\ge0$$
	\item 称$f$是自伴变换,如果满足$f=f^*$,等价于说矩阵表示是对称矩阵,等价于说$(f(x),y)=(x,f(y)),\forall x,y\in V$.按照对称矩阵只有实特征值,看到对称矩阵必然会正交相似为对角矩阵.
	$$\mathrm{diag}\{\lambda_1,\lambda_2,\cdots,\lambda_n\}$$
	\item 称$f$是斜自伴变换,如果满足$f=-f^*$,等价于说矩阵表示是斜对称矩阵,等价于说$(f(x),y)=-(x,f(y)),\forall x,y\in V$.按照斜对称矩阵的实特征值只能是0,非实特征值只能是纯虚数,于是斜对称矩阵特征值如果是$\pm b_1,\pm b_2,\cdots,\pm b_p$和$n-2p$个0,那么它的正交相似标准型为:
	$$\mathrm{diag}\left\{\left(\begin{array}{cc}
	0&b_1\\
	-b_1&0\end{array}\right),\cdots,\left(\begin{array}{cc}
	0&b_p\\
	-b_p&0\end{array}\right),\textbf{0}_{n-2p}\right\},b_i\in \mathbb{R}$$
\end{enumerate}

接下来进入二次型理论.称实数域上方阵$A$诱导的二次型是指函数$Q:R^n\to R,x\mapsto x^TAx$.那么斜对称矩阵诱导的二次型即零函数,按照实矩阵总可以唯一的表示为实对称矩阵和实斜对称矩阵的和,看到一个矩阵$A$诱导的二次型和$\frac{A+A^T}{2}$诱导的二次型相同.而后者是对称矩阵,这说明为了探究实二次型,总是只需要考虑对称矩阵诱导的实二次型,或者说,只关注对称双线性函数下的正定性.今后提及实二次型,只考虑被对称矩阵诱导的二次型.

称复数域上矩阵$A$诱导的二次型是指函数$Q:C^n\to R,x\mapsto x^*Ax$.注意约定了取值是实数,那么一方面,Hermite矩阵诱导的复二次型的取值总是实数,而另一方面,实际上可以诱导出取实数的复二次型的矩阵$A$只能是Hermite矩阵,注意到对任意向量$x,y$,有$(x+y)^*A(x+y)$是实数,结合$x^*Ax,y^*Ay$是实数得到$x^*Ay+y^*Ax$是实数,取$x=e_k,y=e_j$得到$a_{kj}+a_{jk}$是实数,再取$x=ie_k,y=e_j$得到$-ia_{kj}+ia{jk}$是实数,这得到$a_ {jk}=\overline{a_{kj}}$.于是$A$是Hermite矩阵.

类似于实矩阵可以唯一表示为对称和斜对称矩阵之和,关于复矩阵有Toeplitz分解.一个复矩阵$A$可以唯一的写作$H+K$,其中$H$是Hermite矩阵,$K$是斜Hermite矩阵,事实上$H=\frac{A+A^*}{2},K=\frac{A-A^*}{2}$.当阶数为1的时候这是最熟知的复数定义.于是Herimite矩阵某种意义上是实数的推广.

二次型具有多种映射角度的解释.在双线性函数那里,对一个实对称双线性函数或者复共轭对称双线性函数$f$,在取一组基下,就有$Q(\alpha)=f(\alpha,\alpha)$转化为矩阵表示$x^*Ax$.这里$A$是实对称矩阵或者复的Hermite矩阵.另外,还有一种做法是,取实或复内积空间$V$上的一个自伴变换$f$,那么考虑$Q(\alpha)=(f(\alpha),\alpha)$,如果任取$V$上一组标准正交基,$f$和$\alpha$在基下的矩阵表示分别是$A$和$x$,那么就有$Q(\alpha)=x^*Ax$.其中$A$是实对称矩阵或者复Hermite矩阵.

称二次型$Q(x)=x^*Ax$是半正定的,如果对任意的向量$x$总有$Q(x)\ge0$,如果还要求取零当且仅当$x=0$,那么就称$Q(x)$是正定的.可以类似定义半负定和负定.那么,已经在双线性函数那里证明了对一般的实对称矩阵或者复Hermite矩阵,它的合同或者复合同标准型是由$\pm1$和0构成的对角矩阵,其中$1$和$-1$的个数完全被矩阵所决定,称为惯性定理.并且看到了正定当且仅当矩阵合同或者复合同标准型是单位矩阵,半正定当且仅当合同或者复合同标准型的对角元不能出现$-1$.类似的有负定和半负定的等价描述.这里指出,这个正负惯性指数,实际上就是相应的实对称矩阵或者复Hermite矩阵的全部特征值中正实数和负实数的个数.事实上注意到正交或者酉相似是合同或者复合同的特例,于是考虑正交或者酉相似标准型就得到结论.

二次型的化简问题.如何通过线性代换把二次型化作不含交叉项的形式,即$a_1x_1^2+a_2x_2^2+\cdots+a_rx_r^2$的形式.这个问题转化为矩阵语言就是说,给定二次型$x^*Ax$,如何找到可逆复矩阵$P$,使得$P^*AP$是对角形式.按照正规矩阵的酉相似标准型定理.已经看到存在这样的酉矩阵$P$,并且它就是由$A$的$n$个两两正交的特征向量作为列向量组构成的酉矩阵.于是在实际操作中,对每个特征子空间求一组基,再拿Schmidt正交化过程得到特征子空间的一组标准正交基.这里来说明一件事,对于Hermite矩阵或者实对称矩阵$A$,它的不同特征值的特征向量已经是正交的了,这就省去很多工作.事实上取$A$的两个不同特征值$a,b$,取特征向量分别是$x,y$,那么有$x^*Ay=y^*Ax$,$x^*Ay=bx^*y$,$y^*Ax=ay^*x$,这导致$ax^*y=bx^*y$,得到$x^*y=0$.于是,只要在每个特征子空间找一组标准正交基,把它们并起来就是整个空间的标准正交基.

关于二次型还有一个重要定理是Rayleigh商定理.给定内积空间$V$上的自伴变换$f$,我们称$R(\alpha)=\frac{(f(\alpha),\alpha)}{(\alpha,\alpha)}$是自伴变换的Rayleigh商.这是一个取非零向量的函数,并且一个非零向量的取值和这个向量单位化的向量的取值是相同的.于是Rayleigh商可以看作是定义域为全体单位向量的函数.

Rayleigh商的矩阵描述.取定一组标准正交基下,那么自伴变换对应于对称矩阵或者Hermite矩阵,于是此时Rayleigh商具有表示$R(x)=\frac{x^*Ax}{x^*x}$.其中$x$取非0的$\mathbb{R}^n$或者$\mathbb{C}^n$中的元.或者,可以等价的写作$R(x)=x^*Ax$,其中$x$取单位向量.

所谓Rayleigh商定理是指,Rayleigh商的最大最小值,就是诱导这个二次型的对称或者Hermite矩阵的最大最小特征值,或者说就是对应自伴映射的最大最小特征值.知道无论实复的情况,自伴变换的特征值总是实数.另外取最值的情况就是带入最大最小特征值相应的特征向量.定理的证明是容易的,只要按照正交相似或者酉相似标准型,可以取正交或者酉矩阵$U$,使得$A=U^*DU$,这里$D$是由特征值作为对角元的对角矩阵.于是Rayleigh商就等价于$R(x)=\frac{x^*U^*DUx}{x^*U^*Ux}=\frac{y^*Dy}{y^*y}=R'(y)$.如果记$D=\mathrm{diag}\{\lambda_1,\cdots,\lambda_n\}$是按从大到小排列的,那么$R'(y)=\frac{\sum_{k=1}^{n}\lambda_ky_k^2}{\sum_{k=1}^{n}y_k^2}$,这个最大最小值自然就是$\lambda_1$和$\lambda_n$.

称实对称矩阵或者复Hermite矩阵是正定/半正定的,如果所诱导的实或复二次型是正定/半正定的.接下来给出正定和半正定的一些充要条件.

给定实对称方阵$S$或复Hermite矩阵$H$,那么如下条件等价.
\begin{enumerate}
	\item $S$是正定实对称矩阵/$H$是正定Hermite矩阵.
	\item $S$的全部特征值都是正实数/$H$的全部特征值都是正实数.
	\item 存在可逆实方阵$M$满足$S=M^TM$/存在可逆复方阵$P$满足$H=P^*P$.
	\item 方阵$S$的全部主子式都是正的/方阵$H$的每个主子式都是正实数.
	\item 方阵$S$的全部顺序主子式都是正的/方阵$H$的每个顺序主子式都是正实数.
	\item 对每个$1\le k\le n$,$S$的所有$k$阶主子式的和都是正的/对每个$1\le k\le n$,$H$的所有$k$阶主子式的和都是正的.
\end{enumerate}

给定实对称方阵$S$或者复Hermite矩阵$H$,那么如下条件等价.
\begin{enumerate}
	\item $S$是半正定实对称矩阵/$H$是半正定Hermite矩阵.
	\item $S$的全部特征值都是非负实数/$H$的全部特征值都是非负实数.
	\item 存在同阶实方阵$M$满足$S=M^TM$/存在同阶复方阵$P$满足$S=P^*P$.
	\item 方阵$S$的全部主子式都是非负的/方阵$H$的全部主子式都是非负的.
	\item 对每个$1\le k\le n$,$S$的所有$k$阶主子式的和都是非负的/对每个$1\le k\le n$,$H$的所有$k$阶主子式的和都是非负的.
\end{enumerate}

接下来介绍奇异值分解,它是实方阵或者复方阵在正交相抵或者酉相抵下的标准型.给定$m\times n$的矩阵$A$,那么$A^*A$和$AA^*$都是半正定Hermite矩阵并且具有相同的非0特征值以及代数重数.把全部非0特征值在计重数意义下从大到小排列为$\lambda_1,\lambda_2,\cdots,\lambda_r$,那么称$\{\sqrt{\lambda_1},\sqrt{\lambda_2},\cdots,\sqrt{\lambda_r}\}$为$A$的全部奇异值.奇异值分解定理:对$m\times n$的矩阵$A$,设奇异值为$\{\eta_1,\eta_2,\cdots,\eta_r\}$,那么存在$m$阶正交或者酉矩阵$P$和$n$阶正交或者酉矩阵$Q$,使得$PAQ$具有如下形式.并且奇异值完全决定了正交相抵或者酉相抵关系.
$$\mathrm{diag}\{\eta_1,\eta_2,\cdots,\eta_r,\textbf{0}_{(m-r)\times(n-r)}\}$$
\begin{proof}
	
	知道$A^*A$是$n$阶半正定Hermite或者对称矩阵,于是它的非零特征值都是正实数,于是存在正交或者酉矩阵$O$使得:
	$$A^*A=O\mathrm{diag}\{\eta_1^2,\cdots,\eta_r^2,\textbf{0} _{n-r}\}O^*$$
	
	现在记$O=(O_1,O_2)$,其中$O_1$是$n\times r$的子矩阵,记$D=\mathrm{diag}\{\eta_1,\cdots,\eta_r\}$,那么有:
	$$\left(\begin{array}{c}
	O_1^*\\
	O_2\end{array}\right)A^*A\left(O_1,O_2\right)=\left(\begin{array}{cc}
	O_1^*A^*AO_1&O_1^*A^*AO_2\\
	O_2^*A^*AO_1&O_2^*A^*AO_2\end{array}\right)$$
	
	于是得到$O_1^*A^*AO_1=D$,$O_2^*A^*AO_2=0$,从后一个等式得到$AO_2=0$.另外从$O^*O=E_n$得到$O_1^*O_1=E_r$,$O_2^*O_2=E_{n-r}$,$O_1O_1^*+O_2O_2^*=E_n$,左乘$A$结合$AO_2=0$得到$AO_1O_1^*=A$.现在取$P_1=AO_1D^{-1}$是$m\times r$的矩阵,并且有$P_1^*P_1=E_r$,于是$P_1$是$r$个两两正交的$m$维单位向量作为列向量组构成的矩阵.现在把$P_1$的列向量组扩充为一组标准正交基,得到$m$阶正交或者酉矩阵$P=(P_1,P_2)$.于是有:
	$$P\left(\begin{array}{cc}
	D&0\\
	0&0\end{array}\right)O^*=(P_1,P_2)\left(\begin{array}{cc}
	D&0\\
	0&0\end{array}\right)\left(\begin{array}{c}
	O_1^*\\
	O_2^*\end{array}\right)=P_1DO_1^*=A$$
	
	这就完成了存在性.接下来证明唯一性.设$m\times n$的实或复矩阵$A,B$是正交相抵或者酉相抵的,也就是存在正交或者酉矩阵$O_1,O_2$满足$B=O_1AO_2$.那么得到$B^*B=O_2^*A^*AO_2$,于是$B^*B$和$A^*A$具有相同的非零特征值及其重数,于是$A,B$具有相同的奇异值.这就说明了唯一性.
	
\end{proof}

矩阵的极分解.
\begin{enumerate}
	\item 任意$n$阶实方阵$A$可以分解为一个半正定对称方阵$S$右乘一个实正交方阵$O$,即$A=SO$,其中半正定部分是被$A$唯一确定的.
	\item 任意$n$阶实方阵$A$可以分解为一个半正定对称方阵$S_1$左乘一个实正交方阵$O_1$,即$A=O_1S_1$,其中半正定部分是被矩阵$A$唯一决定的.
	\item 任意$n$阶复方阵$A$可以分解为一个半正定Hermite矩阵$S$右乘一个酉矩阵$U$,即$A=SU$,其中半正定部分是被$A$唯一决定的.
	\item 任意$n$阶复方阵$A$可以分解为一个半正定Hermite矩阵$S_1$左乘一个酉矩阵$U_1$,即$A=U_1S_1$,其中半正定部分是被矩阵$A$唯一确定的.
\end{enumerate}
\begin{proof}
	
	只对第一条做证明.交换位置的情况只要对$A$取转置即可.对于复情况的证明是类似的.设$A$的全部奇异值是$\mu_1,\mu_2,\cdots,\mu_r$,于是存在同阶的正交矩阵$O_1,O_2$使得:
	$$A=O_1\mathrm{diag}\{\mu_1,\mu_2,\cdots,\mu_r,\textbf{0} _{n-r}\}O_2$$
	
	把中间对角矩阵记作$D$,那么显然有$A=(O_1DO_1^T)(O_1O_2)$,这就证明了这种表示的存在性.接下来说明分解中半正定部分是唯一的.假如有$A=SO=S_1O_1$两种分解,那么得到$AA^T=S^2=S_1^2$是半正定矩阵,按照下面的定理就得到$S=S_1$.
	
\end{proof}

给定半正定实对称矩阵$A$,对每个正整数$m$,存在唯一的一个半正定实对称矩阵$A_1$使得$A_1^m=A$.复情况就是把上述半正定实对称矩阵改为半正定Hermite矩阵.另外这里的证明过程实际上得到了这样一个结论:如果实或复方阵$A$和一个半正定对称矩阵或半正定Hermite矩阵的某个$m$次幂可交换,那么$A$和这个半正定矩阵可交换.
\begin{proof}
	
	复的情况类似,只来证明实情况.假如有两个半正定对称矩阵$A_1,A_2$满足$A_1^m=A_2^m=A$,按照标准型定理,存在正交矩阵$O_1,O_2$使得$A_i=O_i^TD_iO_i$,其中$i=1,2$.不妨设$D_1,D_2$是对角元从大到小排列的对角矩阵.于是得到$O_1^TD_1^mO_1=O_2^TD_2^mO_2$,于是$A$相似于$D_1^m$和$D_2^m$,但这两个都是对角矩阵,对角元都是从大到小排序,这说明$D_1^m=D_2^m$,于是$D_1=D_2$.记$O_2O_1^T=P=(p_{ij})$,那么有$PD_1^m=D_2^mP$.并且必然有$D_1^m=D_2^m$,设这个$m$次幂的对角元依次为$\lambda_1,\cdots,\lambda_n$,于是从上述等式,考虑$(ij)$元,得到$p_{ij}\lambda_j=\lambda_ip_{ij}$.如果$\lambda_i=\lambda_j$,那么显然得到$p_{ij}\lambda_j^{1/m}=\lambda_i^{1/m}p_{ij}$;如果$\lambda_i\not=\lambda_j$,那么必然有$p_{ij}=0$,此时也必然会有等式$p_{ij}\lambda_j^{1/m}=\lambda_i^{1/m}p_{ij}$成立,综上得到$PD=DP$,于是$A_1=A_2$.
	
\end{proof}

有了极分解来给出复正规矩阵相似则酉相似的另一证明.倘若存在复可逆矩阵$P$满足$T^{-1}AT=B$,那么按照极分解,就有$T=SU$,其中$S$是正定Hermite矩阵,$U$是酉矩阵,于是得到$U^{-1}(S^{-1}AS)U=B$,如果可以证明$A$和$S$可交换,那么就结束.为此注意到$S$是正定Hermite矩阵$PP^*$的平方根,于是只要证$A$和$PP^*$可交换,那么就有$A$和$S$可交换.最后注意到从$AP=PB$得到$A^*P=PB^*$,于是有$APP^*=PBP^*=PP^*A$,完成证明.

利用极分解还可以说明,两个实矩阵如果是酉相似的,那么它们必然是正交相似的.对此要注意到,如果有实矩阵$A,B$和酉矩阵$U=P+iQ$满足$AU=UB$,那么取实虚部得到对任意实数$t$有$A(P+tQ)=(P+tQ)B$,那么首先$Q$如果是零矩阵,就有$U=P$是正交矩阵,于是$A,B$已经正交相似.如果$Q$不是零矩阵,那么$f(t)=|P+tQ|$是关于$t$的非常数多项式,于是必然存在实数$t$使得$f(t)$非零,也就是$M=P+tQ$是实的可逆矩阵,满足$AM=MB$.接下来的证明是同上的.
\newpage
\section{一些补充}
\subsection{同时上三角化}

给定一族$V$上的线性变换,如果存在一组基下所有线性变换的矩阵表示是上三角矩阵,就称这族线性变换可以同时上三角化.转化为矩阵语言,就是说给定一族域$F$上的同阶方阵,如果存在同一个可逆矩阵,它作为过渡矩阵会让每个方阵都相似于上三角矩阵,就称这族矩阵可以同时上三角化.本节的目的就是探究一些能够使得矩阵族上三角化的条件.

先来探究两个矩阵的情况.首先如果可上三角化矩阵$A,B$可以同时上三角化.设$P^{-1}AP,P^{-1}BP$都是上三角矩阵,不妨设它们分别的(1,1)元位置是$a_{11},b_{11}$,设$P$的第一个列向量是$\alpha$,那么有:$A\alpha=a_{11}\alpha,B\alpha=b_{11}\alpha$,于是如果$A,B$可以同时上三角化,那么必然有$A,B$具有公共的特征向量.于是,需要给两个矩阵$A,B$添加的,能够使得它们同时上三角化的条件,至少要强于$A,B$具有公共特征向量.

可上三角化矩阵如果可交换,那么就有公共特征向量.事实上如果$A,B$可交换,则$A$的每个特征子空间总是$B$的不变子空间,取$A$的一个特征子空间$V$,把$B$限制在这个不变子空间上,因为$B$的特征多项式可裂,于是$B$限制在$V$上也是可裂的,特别的它也有特征子空间,任取这个特征子空间中的非0向量就是要找的公共特征向量.

注意上述结论基于的是特征多项式可以分解为一次因式的乘积,特别的如果矩阵所在的基域是代数闭域那么该命题总成立,在一般域上一个矩阵甚至未必有特征值.不过我们可以证明,实数域上的两个奇数阶矩阵$A,B$可交换,那么它们具有公共的实特征向量.这是因为实数域上的奇数次矩阵必然有实特征值,并且非实数的复特征值成对出现且代数重数相同,于是必然存在一个实特征值的代数重数是奇数.此时我们可以用矩阵$A$的该实特征值(不妨记作$a$)的根子空间$V$取代上述证明中的特征子空间.我们知道根子空间的维数就是代数重数,于是$V$是奇数维.另外$AB=BA$同样可说明$V$是$B$的不变子空间.现在把$B$限制在$V$上,这个新的限制映射$B|V$是奇数维空间上的,于是可以取特征值$b$,记特征子空间为$W$.现在任取$W$中的非0向量$v$,按照$v$位于根子空间$V$中,于是可取一个最小的正整数$r$满足$v\in\ker(A-aE)^r$,记$w=(A-aE)^{r-1}$,按照$r$的最小性说明$w\not=0$.另外有$(A-aE)w=0$,于是$w$是$A$的属于$a$的特征向量.最后按照交换条件得到$Bw=B(A-aE)^{r-1}v(A-aE)^{r-1}Bv=(A-aE)^{r-1}(bv)=bw$,即$w$是$B$的属于$b$的特征向量.这就说明了$w$是公共特征向量.

事实上可交换这个条件的确能使得两个矩阵同时上三角化.:\textbf{给定域$F$上两个可上三角化的矩阵,如果它们可交换,那么它们可以同时上三角化}.如果$AB=BA$,按照上一段知道$A,B$存在公共的特征向量$\alpha$,把它扩充为一组基,并且作为一个矩阵$P$的列向量组,那么$P$可逆,并且满足:
$$P^{-1}AP=\left(\begin{array}{cc}
a_{11}&*\\
0&A_1
\end{array}\right),P^{-1}BP=\left(\begin{array}{cc}
b_{11}&*\\
0&B_1
\end{array}\right)$$

但是知道$A,B$可上三角化而且可交换说明$A_1,B_1$可上三角化而且可交换,因此可以对阶数归纳,按照归纳假设得到$A_1,B_1$可以同时上三角化,设用于同时相似的可逆矩阵是$P_1$,那么取$Q=P diag{1,P_1}$,就有$Q^{-1}AQ,Q^{-1}BQ$是同时上三角化的.

可同时对角化和可交换是等价的:\textbf{给定域$F$上的两个可对角化矩阵$A,B$,那么$A,B$可交换当且仅当它们可以同时对角化}.一方面如果$A,B$可以同时对角化,那么它们必然可交换,现在来证明另一侧.因为$A$是可对角化的,所以有可逆矩阵$P$使得$P^ {-1}AP=\mathrm{diag}\{\lambda_1I_{r_1},\lambda_2I_{r_2},\cdots,\lambda_tI_{r_t}\}$
,其中$\lambda_i$两两不同,那么$P^{-1}BP$和这个分块对角矩阵可交换.于是我们$B$是相应的分块对角矩阵$B=diag\{B_1,B_2,\cdots,B_t\}$,但是知道$B$可对角化等价于每个$B_i$可对角化,于是存在$Q_i$使得$Q_i^{-1}B_iQ_i$是对角矩阵,于是取$M=P\mathrm{diag}\{Q_1,Q_2,\cdots,Q_t\}$,有$M^ {-1}AM,M^{-1}BM$是对角矩阵.得证.

已经给出了两个矩阵同时对角化同时上三角化和可交换的关系.现在把情况推广至矩阵族.称一个矩阵族是交换的矩阵族,如果任意两个矩阵可交换.

在上述的证明里全部运用到两个可上三角化矩阵可交换则它们具有公共特征向量.不出意料,这个结果要推广至可交换的矩阵族.将给出两个证明.

给定一个可上三角化的交换的矩阵族$\mathscr{F}$,来证明所有矩阵存在一个公共的特征向量.考虑全体$\mathscr{F}$不变子空间构成的集合$S$,也就是说$S$是由这样的$F^n$子空间$V$构成,使得每个$\mathscr{F}$中的矩阵$A$有$V$是$A$不变子空间.那么这个$S$是非空的,因为$F^n\in S$,现在取$S$中一个维数最小的空间$W$,设维数是$m$,任取$A\in\mathscr{F}$,那么知道$A$限制在$W$上存在特征子空间,也就是有$x_0\in W$使得$Ax_0=\lambda x_0$,那么考虑子空间$W'=\{x\in W:Ax=\lambda x\}$,它是$W$的非0子空间,对任意的$B\in\mathscr{F}$,任意$x\in W'$,有$A(Bx)=\lambda(Bx)$,于是$W'$是$\mathscr{F}$不变子空间,于是$W'=W$,这就说明了$W$中每个非0向量都是$\mathscr{F}$中每个矩阵的特征向量.

现在给出第二个证明.首先把二元情况按照归纳推广至有限元的情况.现在对于一个矩阵可能无限的矩阵族$\mathscr{F}$,自然不能常规归纳到整个$\mathscr{F}$,但是注意到全部矩阵构成一个有限维线性空间$V_0$,尽管$\mathscr{F}$未必是它的线性子空间,但是它生成的空间是$V_0$的线性子空间,于是只要取$\mathscr{F}$中的一个极大的线性无关组,按照前面已经证明的有限情况成立,知道这个极大线性无关组是满足存在公共特征向量的,但是$\mathscr{F}$任意一个矩阵都是这个线性无关组的线性组合,于是仍然保证这个公共特征向量仍然是它的特性向量.于是得证.

已经证明了核心引理,现在可以给出推广了.即,\textbf{一个可交换的矩阵族如果每个矩阵都可以上三角化,那么它们可以同时上三角化};\textbf{一个可交换的矩阵族如果每个矩阵都可以对角化,那么它们可以同时对角化}.证明只要模仿之前情况的证明即可.

给定一族矩阵或者线性变换$\mathscr{A}$,称$V$的一个子空间是$\mathscr{A}$不变子空间,如果这个子空间关于每个$\mathscr{A}$中的线性变换都是不变子空间.如果这族线性变换存在非平凡的不变子空间,即既不是零空间也不是全空间的不变子空间,就称$\mathscr{A}$是可约的线性变换族或者矩阵族,否则就称不可约的.

总结一下证明这个命题所用到的思路.分为两步,第一步是找到一个公共特征向量,为此,我们需要满足特定一种性质$P$(在上述例子里这个性质就是可交换)的线性变换族或者矩阵族$\mathscr{A}$总是可约的,这使得存在非平凡的不变子空间,还需要这种特定性质$P$要传递给这个不变子空间,于是,继续对限制在这个不变子空间上的满足性质$P$的线性变换族继续取出非平凡不变子空间,一直操作下去,最后总归得到一个一维的不变子空间,它的非零向量就是矩阵族的公共特征向量.第二步,是利用公共特征向量不断的归纳下去,严格说,是对公共的特征向量$x$,取$W=(x)$,再取商空间$V/W$,把线性变换族限制到这个商空间上,需要这族线性变换所满足的性质$P$可以传递给商空间$V/W$.这就保证了可以继续归纳下去.

于是看到,一族线性变换满足性质$P$,它如果满足如下三个条件就可以保证它可约同时上三角化:对任意线性空间$V$上满足性质$P$的一族线性变换,总是可约的,即有非平凡不变子空间;性质$P$可以传递给线性变换限制在不变子空间上得到的线性变换族;性质$P$可约传递给线性变换限制在全空间商去一个一维公共不变子空间构成的线性变换族.而实际上后两个条件可以归结为一个条件,称性质$P$是被商所继承的,如果对任意的一族满足性质$P$的线性变换,任意的两个不变子空间$M,N$,总有线性变换族限制在商空间$M/N$上是满足性质$P$的.看到如果$N$取零空间就保证上述第二个条件,而取$M=V$就保证上述第三个条件.至此可以给出上三角化引理:

上三角化引理.给定关于矩阵族的一个可以被商所继承的性质$P$,如果每个满足性质$P$的阶数大于1的矩阵族总是可约的,那么每个满足性质$P$的矩阵族都是可以同时上三角化的.
\begin{proof}
	
	知道同时上三角化等价于说存在一个公共不变子空间链$\{0\}=M_0\subset M_1\subset\cdots\subset M_n=V$,使得每个商$M_{i+1}/M_i$都是一维的.现在我们取满足性质$P$的不变子空间链的一个极大链$\{0\}=M_0\subset M_1\subset\cdots\subset M_m=V$.只要证明每个$M_{i+1}/M_i$都是一维的.这就能保证$m=n$.假如存在一个$M_{k}/M_{k-1}$的维数大于1,那么线性变换族限制在商$M_{k}/M_{k-1}$上会存在一个非平凡的不变子空间$L/M_{k-1}$,这导致链可以延拓为$\cdots\subset M_{k-1}\subset L\subset M_k\subset\cdots$.和极大性矛盾.
	
\end{proof}

注意被商所继承这个条件,转化为矩阵语言是说,如果一族矩阵同时相似于同型的准上三角矩阵,那么全体第$i$个对角块构成的矩阵族还满足之前的性质.其中$i$取遍1到对角块个数.那么可交换这个条件自然是被商所继承的,另外一个可交换矩阵族总有1维的不变子空间,于是按照上三角化引理直接得到特征多项式可裂的矩阵构成的交换族,必然是同时上三角化的.如果域取做代数闭域,可以省去特征多项式可裂这一条件.

再比如,如果特征多项式可裂的同阶方阵$A,B$满足$AB=0$,证明$A,B$可以同时上三角化.容易看出这个性质$AB=0$是被商继承的,于是只要找出$A,B$的一个公共非平凡不变子空间,就得到了$A,B$的一个公共非平凡不变子空间,从上三角化引理就得证.为此注意到,如果$A=0$,那么$B$的任意特征向量都是公共特征向量,而如果$A$可逆,就有$B=0$也是同理的.现在如果$A\not=0$也不是可逆的,就取$\ker A$,它是非平凡的$A,B$的公共不变子空间.

再比如有\textbf{Laffey定理}:如果$A,B$特征多项式可裂,或者域是代数闭域,$AB-BA$的秩不超过1,则$A,B$可以同时上三角化.这个秩条件也是被商继承的,于是只要找$A,B$的非平凡不变公共子空间.为此考虑$A$的一个特征值$\lambda$,则不妨设$A\not=\lambda E$,否则结论平凡,接下来就有$\ker(A-\lambda E)$和$\mathrm{im}(A-\lambda E)$都是非平凡的$A$不变子空间,证明它们至少有一个是$B$不变子空间就结束.

在引出内积空间后介绍了两种特殊的相似关系,实数域上的正交相似和复数域上的酉相似.并且证明了实矩阵总可以正交相似准上三角化,复矩阵总可以酉相似对角化.这里推广下可交换和酉上三角化与正交上三角化的关系:
\begin{enumerate}
	\item 实数域上一个可交换矩阵族如果每个矩阵可以正交相似上三角化,那么这个矩阵族可以同时正交相似上三角化
	\item 实数域上一个可交换矩阵族如果每个矩阵可以正交相似对角化(实矩阵可正交相似对角化当且仅当它是对称矩阵),等价于这个矩阵族可以同时正交相似对角化
	\item 复数域上一个可交换矩阵族如果每个矩阵可以酉相似上三角化,那么这个矩阵族可以同时酉相似上三角化
	\item 复数域上一个可交换矩阵族如果每个矩阵可以酉相似对角化(复矩阵酉相似对角化当且仅当它是正规矩阵),等价于这个矩阵族可以同时酉相似对角化
\end{enumerate}

关于证明.回顾之前情况的证明里,当取公共特征向量之后把它扩充为一组基,只要把这一步改为扩充为实数域或者复数域上的标准正交基,其他证明不变即可.这里给出第二条对于两个矩阵情况的证明:

给定$n$阶对称实矩阵$A,B$,满足$AB=BA$,证明存在正交矩阵$T$使得$T^{-1}AT$和$T^{-1}BT$同时是对角矩阵.
\begin{proof}
	
	首先证明,可交换的实对称矩阵$A,B$具有公共的特征向量.按照$A$是对称矩阵知它有实特征值$\lambda$,记特征子空间是$W\subset\mathbb{R}^n$.断言$W$是$B$的不变子空间,为此只要注意到$\forall x\in W$有$A(Bx)=B(Ax)=\lambda(Bx)$.现在把$A$限制在$W$上,按照$A$是$\mathbb{R}^n$上的自伴变换得到$(Ax,y)=(x,Ay),\forall x,y\in W$,于是$A$限制在$W$上也是自伴变换,于是$A\mid_{W}$有特征向量$\alpha_1$,于是这个$\alpha_1$是$A,B$公共的特征向量.
	
	为证原命题,对阶数归纳.当$n=1$时没什么需要证的.现在假设$n\ge2$,假设对$n-1$的情况已经成立,现在取两个$n$阶可交换实对称矩阵$A,B$.按照上述引理$A,B$有公共特征向量$\alpha_1$,不妨设这个向量的模长是1,否则除以$|\alpha_1|$不改变它是公共特征向量这一事实.设$A\alpha_1=a\alpha_1$,$B\alpha_1=b\alpha_1$.设$\alpha_1$生成的一维子空间是$W$,那么有$\mathbb{R}^n=W\oplus W^{\perp}$.取$W^{\perp}$上的一组标准正交基$\{\alpha_2,\cdots,\alpha_n\}$.记矩阵$(\alpha_1,\alpha_2,\cdots,\alpha_n)$为$T_1$,那么这是一个正交矩阵,并且有:
	$$T_1^{-1}AT_1=\left(\begin{array}{cc}
	a&\textbf{0}\\
	\textbf{0}&A_1\end{array}\right);T_1^{-1}BT_1=\left(\begin{array}{cc}
	a&\textbf{0}\\
	\textbf{0}&B_1\end{array}\right)$$
	
	于是从$T_1^{-1}AT_1$和$T_1^{-1}BT_1$可交换得到$A_1,B_1$可交换,它们是$n-1$阶对称矩阵,按照归纳假设,存在$n-1$阶正交矩阵$T_2$满足$T_2^{-1}A_1T_2$和$T_2^{-1}B_1T_2$同时是对角矩阵.最后取$T=T_1\left(\begin{array}{cc}
	1&0\\
	0&T_2\end{array}\right)$是正交矩阵.就得到$T^{-1}AT$和$T^{-1}BT$同时是对角矩阵,完成归纳.
	
\end{proof}
\newpage
\subsection{矩阵的幂零-对角化分解}

幂零-对角化分解定理:任意一个代数闭域上矩阵可以分解为一个幂零矩阵加上一个可对角化矩阵,并且两部分可交换.这种分解是唯一的,并且两部分都可以写作原矩阵的多项式.

首先这种分解的存在性是容易的.设矩阵$A$有Jordan标准型分解:$A=SJS^{-1}$,那么将$J$分解为这样两部分的和:一部分是保留对角元,其余位置均为0,另一部分是保留除对角元的上三角部分,其余位置均为0,不妨写作$J=J_D+J_N$,那么有:$A=SJ_DS^ {-1}+SJ_NS^{-1}$记作$A_D+A_N$,那么$A_D$是可对角化矩阵,$A_N$是幂零矩阵(因为取$A$Jordan标准型里最大的Jordan块的阶数$s$,就有$A_N^s=0$).因为$A_D,A_N$可以同时上三角化所以它们可交换(或者,因为$\lambda E$和$J_r(0)$必然可交换所以$A_D,A_N$可交换),这就得到了存在性.

现在来说明$A_D,A_N$可以取为$A$的多项式.取$A$的极小多项式为$m_A(x)$.那么在代数闭域上有唯一分解:
$$m_A(x)=\prod_{i=1}^{s}(x-\lambda_i)^n_i$$

记$f_i(x)=m_A(x)/(x-\lambda_i)^n_i$,于是有$(f_1,f_2,\cdots,f_s)=1$,按照Bezout定理,得到一组多项式$g_i,i=1,2,\cdots,s$,使得:
$$1=\sum_{i=1}^{s}f_i(x)g_i(x)$$

记矩阵$A_i=f_i(A)g_i(A)$,那么$A_i,i=1,2,\cdots,s$是一组两两乘积为0的幂等矩阵.即,因为$m(x)$整除$f_i(x)g_i(x)f_j(x)g_j(x),i\not=j$,于是$A_iA_j=0$.并且把$x=A$带入$1=\sum_{i=1}^{s}f_i(x)g_i(x)$得到$E=\sum_{i=1}^{s}A_i$,左乘以某个$A_i$得到:$A_i=A_i^2$,即每个$A_i$是幂等矩阵.

现在可以来构造对角化部分和幂零部分了!取:
$$p_D(x)=\sum_{i=1}^{s}\lambda_if_i(x)g_i(x)$$
$$p_N(x)=\sum_{i=1}^{s}(x-\lambda_i)f_i(x)g_i(x)$$

那么首先,$p_D(x)+p_N(x)=x\sum_{i=1}^{s}f_i(x)g_i(x)=x$,于是$A=p_D(A)+p_N(A)$,只需说明$p_D(A)$是可对角化的,$p_N(A)$是幂零的.

幂零部分:$p_N(A)=\sum_{i=1}^{s}(A-\lambda_i E)A_i$,归纳结合$A_i$是两两乘积为0的幂等矩阵可以得出$p_N^j(A)=\sum_{i=1}^{s}(A-\lambda_i E)^jA_i$(注意每个$A_i$都是$A$的多项式,所以$A_i$和$(A-\lambda_j E)^t$可交换),于是如果设极小多项式$m_A(x)$唯一分解中一次项因式的最大数是$d$,那么$p_N^d(A)=0$.从而$p_N(A)$幂零.

可对角化部分:注意到$\lambda_j E-p_D(A)=\lambda_i\sum_{i=1}^{s}A_i-\sum_{i=1}^{s}\lambda_iA_i
=\sum_{i=1}^{s}\left(\lambda_j-\lambda_i\right)A_i$.结合$A_i$两两乘积为0并且幂等,得到:
$$\prod_{j=1}^{s}(\lambda_j E-p_D(A))=
\sum_{i=1}^{s}\prod_{j=1}^{s}\left(\lambda_j-\lambda_i\right)A_i=0$$

这说明$\prod_{i=1}^{s}(x-\lambda_i)$是$p_D(A)$的零化多项式,于是它的极小多项式必然是不同一次因式的乘积,从而可对角化.

至此完成了可对角化-幂零分解的存在性并且证明了两部分可以取作$A$的多项式,最后来证明分解的唯一性.

取可被多项式表达的分解为$A=A_D+A_N$,即$A_D$可对角化,$A_N$幂零,并且$A_D,A_N$可交换,而且存在多项式$p,q$使得$A_D=p(A),A_N=q(A)$.现在假设还存在一种分解$A=B+C$,其中$B$可对角化,$C$幂零,$B,C$可交换.那么$B,C$都和$B+C=A$可交换,于是$B,C$都和$A$的多项式可交换,这也就是说$B,C,A_D,A_N$两两可交换,所以四个矩阵可以同时上三角化,于是矩阵$X=A_D-B=C-A_N$即可以对角化也是幂零的,这说明$X$只能是0矩阵,于是$A_D=B,A_N=C$,完成了唯一性的证明.
\newpage
\subsection{循环变换和无损矩阵}

从循环子空间的定义抽象出这样一个概念:称非平凡的线性空间$V$上一个线性变换$f$是循环变换,如果$V$是关于$f$是一个循环空间,也就是说,存在一个向量$\alpha$,使得全空间被$\{\alpha,f\alpha,\cdots\}$生成.把可以这样生成全空间的向量$\alpha$称为循环向量.注意这个定义已经要求了$f$必然不是0映射.

那么首先,已经证明过向量$\alpha$生成的循环子空间的维数就是它的极小多项式的次数$n$,并且$\{\alpha,f(\alpha),\cdots,f^{n-1}(\alpha)\}$是一组基.于是,按照$f$的极小多项式必然是$\alpha$的零化多项式,导致$f$的极小多项式被$\alpha$的极小多项式整除,这导致$f$的极小多项式的次数就是空间维数,于是导致它和特征多项式相同.即证明了,线性空间上的循环变换必然满足特征多项式等于极小多项式.

事实上这个命题反过来也是成立的.为此来证明一个定理.无限域$F$上线性空间$V$上给定一个线性变换$A$,那么存在一个$V$中元素,它关于$A$的极小多项式恰好就是$A$的极小多项式.为证明这个命题需要所谓的\textbf{覆盖定理}:一个无限域上的线性空间,不会被有限个真子空间所覆盖.先看覆盖定理如何推出这个结论:知道任意一个非0元素$v$的极小多项式$m_v(x)$整除$f$的极小多项式$m(x)$,但是$m(x)$只有有限个因子$m_i(x)$,考虑有限个子空间$V_i=\{v\in V:m_i(A)v=0\}$,它们的并覆盖全空间,于是必然存在一个$V_i=V$,那么就有$m(x)=m_i(x)$.现在证明覆盖定理,对一个真子空间,必然不会覆盖整个空间,现在归纳假设,如果证明了小于$k$个的真子空间不会覆盖整个空间,来证明$k$个真子空间不会覆盖整个空间.设有$k$个真子空间$V_i,1\le i\le k$使得$V\subset\cup V_i$,不妨设$V_{k}$和$\cup_{1\le i\le k-1}V_i$之间没有包含关系,否则化作$k-1$或者$1$的情况,那么可以取一个向量$\alpha\in V_k,\not\in\cup_{1\le i\le k-1}V_i$,也可以取一个向量$\beta\in\cup_{1\le i\le k-1}V_i,\not\in V_k$,那么考虑集合$\{\lambda\alpha+\beta,\lambda\in F,\lambda\not=0\}$,这是一个无限的集合,但是这个集合中的元素不会在$\cup_ {1\le i\le k-1}V_i$里,否则得到$\alpha\in\cup_{1\le i\le k-1}V_i$.而且不会存在两个元素同时属于$V_k$里,否则有$\lambda_2(\lambda_1\alpha+\beta)-\lambda_1(\lambda_2\alpha+\beta)\in V_k$,推出$\beta\in V_k$同样矛盾.于是这个集合只有有限个元素,矛盾.

于是按照这个定理,当特征多项式等于极小多项式的时候,存在一个向量$\alpha$,它的极小多项式就是$f$的极小多项式,并且次数是空间维数,于是$\alpha$生成的关于$f$的循环子空间就是整个空间.于是$f$是循环变换.即证明了,线性变换是循环变换当且仅当它的特征多项式等于极小多项式.

来分析一下这个条件.极小多项式是第$n$个不变因子,并且知道不变因子的乘积是特征多项式,于是这个条件等价于说,前$n-1$个不变因子都是1.如果把域限制为代数闭域,那么可以说的更多:此时矩阵的每个特征值对应的不变因子只有一个.知道一个特征值的不变因子的个数就是特征值的几何重数,于是看到在代数闭域上要求特征多项式等于极小多项式,等价于要求每个特征值的几何重数都是1.并且,在这个情况下可以推出,每个特征值唯一的那个Jordan块的阶数就是对应的代数重数.

把代数闭域上能成为循环变换的矩阵表示的矩阵,称为无损矩阵.于是,无损矩阵的等价定义有:
\begin{enumerate}
	\item 特征多项式等于极小多项式.
	\item 每个特征值的几何重数都是1.
\end{enumerate}

特殊的无损矩阵有:Jordan块,具有阶数个不同的特征值的矩阵.

按照矩阵的秩就是它的Jordan标准型中,阶数减去0对应的Jordan块的个数,于是一个$n$阶无损矩阵的秩要么是满秩的,要么是$n-1$.

无损矩阵还有一个有趣的等价描述.首先,对一个矩阵$A$,看到它的多项式都是和自身可交换的,那么这些是否是全部和他可交换的矩阵?这是未必的,比如单位矩阵,它和每一个矩阵都可交换,但是它的多项式只能是数量矩阵.但是对于无损矩阵,这是成立的!即每个和无损矩阵可交换的矩阵都能表示为该矩阵的多项式.

\begin{proof}
	
	注意只需对Jordan标准型证明这个结论即可.首先记无损矩阵的Jordan标准型是$J=$diag$\{J_ {n_1}(\lambda_1),J_{n_2}(\lambda_2),\cdots,J_{n_s}(\lambda_s)\}$,并且$\lambda_i$两两不同,那么如果存在一个矩阵$A$和$J$可交换,把$A$共形的分块,记作$A=(A_{ij}),1\le i,j\le s$,那么考虑等式$AJ=JA$两边的$(i,j),i\not=j$元得到:$A_{ij}J_{n_j}=J_{n_i}A_{ij}$,这说明$A_ {ij}$是矩阵方程$XJ_{n_j}=J_{n_i}X$的解,但是因为$J_{n_i}$和$J_{n_j}$没有公共特征值,导致这个方程只有0解,于是得出,如果$A$和$J$可交换,那么$A$必然是共形的分块对角矩阵.记作diag$\{A_1,\cdots,A_s\}$,也就是说每个$A_i$和$J_{n_i}$可交换.于是,为了证明原命题,只需对Jordan块证明这个结论即可.一旦得到存在多项式$f_i$使得$f_i(J_ {n_i})=A_i$,取$g_i(x)=\prod_{j\not=i}(x-\lambda_i)^{n_j}$,从$(g_1,g_2,\cdots,g_s)=1$得到一组多项式$h_1,h_2,\cdots,h_s$有$1=\sum_{i=1}^{s}g_ih_i$,于是取$f=\sum_{i=1}^{s}f_ig_ih_i$就得到$f(J_{n_i})=A_i$.
	
	注意到和$J_r(\lambda)$可交换的矩阵必然是上三角的Toeplitz矩阵,也就是每个主/次对角线上取同一个元素的矩阵,注意到第$i$个次对角线取1的矩阵恰好就是$J_r(0)^i$.这就保证了和Jordan块可交换的矩阵都是Jordan块的多项式.于是完成了证明.
	
\end{proof}

从空间角度可以直接说明,和循环变换$\phi$可交换的变换都能表示为自身多项式.事实上,设$v$是循环向量,那么任何向量都可以表示为$v,\phi(v),\phi^2(v),\cdots$的线性组合,对任意可交换的变换$\psi$,就设存在一个多项式$g$使得$\psi(v)=g(\phi)v$,那么断言$\psi=g(\phi)$,因为对每个向量$m$,取多项式$f$使得$f(\phi)v=m$,那么$\psi(m)=\psi f (\phi)(v)=f(\phi)(\psi(v))=f(\phi)m$.按照这个证明可以说,循环变换的一个可交换变换被它在循环向量处的取值完全决定.

逆命题,如果一个矩阵可交换矩阵都是它的多项式,那么这是无损矩阵.实际上可以直接算出和矩阵可交换矩阵的空间的维数,并且看出这个维数总是不小于阶数的.如果取Jordan型矩阵$J=diag\{J_1,J_2,\cdots,J_r\}$,取一个可交换矩阵$B$,把它共形的分解为$B=(B_{ij})$,于是$JB=BJ$得到$J_iB_{ij}=B_{ij}J_j$.所以如果$J_i,J_i$特征值不同那么$B_{ij}=0$,如果特征值相同,看到$B_{ij}$是一个Toeplitz型矩阵,那么需要确定这个矩阵要用到参数个数是$\min\{J_i\text{阶数},J_j\text{阶数}\}$.这个数可以当作$J_i,J_j$对应初等因子的最大公因式的次数,经过计算,如果设$r_i$是第$i$个的不变因子的次数,可以得到可交换矩阵空间维数是:
$$r_n+3r_{n-1}+5r_{n-2}+\cdots+(2n-1)r_1$$

那么这个维数将会不小于$\sum r_i=n$,但是知道一个矩阵的全部多项式构成的空间的维数不会超过$n$,所以如果一个矩阵可交换矩阵空间恰好就是它的全部多项式,那么$r_1=r_2=\cdots=r_{n-1}=0$,$r_n=n$,于是矩阵特征多项式和极小多项式相同,于是是无损矩阵.

注意上述证明中我们实际上给出了特征多项式在基域上可裂的方阵的可交换空间的维数,并且证明了这个维数不会小于$n$,这一事实还可以更简单的证明:考虑域$k$上的$n$阶方阵$A$,考虑$M_n(k)$上的线性变换$X\mapsto XA-AX$,那么可交换空间就是这个线性变换的核.为证明核的维数不小于$n$,只需证明像空间的维数不超过$n^2-n$.取$M_{n}(k)$的一组基$\{E_{ij},1\le i,j\le n\}$,那么注意到$E_{ij}A-AE_{ij}$的对角线元素都为零,于是像空间包含于$\{E_{ij},i\not=j\}$生成的子空间中,所以维数不超过$n^2-n$.

至此已经给出了无损矩阵的三个等价定义,现在来给出最后一个.给定$n$次多项式$p(t)=t^n+a_{n-1}t^{n-1}+\cdots+a_1t+a_0$,它的友矩阵记作$n$阶方阵:
$$\left(\begin{array}{ccccc}
0&&&&-a_0\\
1&0&&&-a_1\\
&1&\ddots&&\vdots\\
&&\ddots&0&-a_{n-2}\\
0&&&1&-a_{n-1}\end{array}\right)$$

记$F^n$上标准基为$\{e_1,e_2,\cdots,e_n\}$,注意到$Ae_i=e_{i+1}=A^ie_1,i=1,2,\cdots,n-1$,并且$Ae_n=(A^n-p(A))e_1$.于是看到$p(A)e_1=0$,于是归纳得到$p(A)e_i=0,\forall 1\le i\le n$,这就说明恒有$p(A)=0$.于是$p(t)$是$A$的一个阶数次的零化多项式.现在如果存在一个次数低于$n$的零化多项式$q(t)=t^m+b_ {m-1}t^{m-1}+\cdots+b_0$,有$0=q(A)e_1=e_{m+1}+b_{m-1}e_m+\cdots+b_0e_1=0$,这是矛盾的.于是$p(t)$就是它友矩阵的最小多项式,也是特征多项式.于是看到多项式的友矩阵总是一个无损矩阵,反过来给定无损矩阵$A$,它的特征多项式的友矩阵记作$B$,那么$A,B$的极小多项式都是$p_A(x)$,并且每个特征值都是恰好有唯一一个以代数重数为阶数的Jordan块,于是$A,B$相似,也就是说,无损矩阵恰好就是和友矩阵相似的矩阵.

从上述证明的过程,看到两个无损矩阵相似当且仅当特征多项式相同.
\newpage
\subsection{整数方阵的相似关系}

给定两个$n$阶整数方阵$A,B$,称它们是整相似的,如果存在整数可逆方阵$T$使得$T^{-1}AT=B$.证明过无限域上的扩域不改变相似关系,但是扩环情况是会改变的.举例来讲:
$$\left(\begin{array}{cc}
1&1\\
0&-1\end{array}\right);\left(\begin{array}{cc}
1&0\\
0&-1\end{array}\right)$$

这两个矩阵作为复数域上矩阵都是可对角化矩阵并且特征多项式相同,导致它们在复数域上相似,从而也在有理数域上相似,但是,作为整数环上的方阵并不是相似的.这是因为,倘若相似,那取$\mod2$下,第二个矩阵变成有限域$\mathbb{F}_2$的单位矩阵,和它相似的只能是自身,这就矛盾了.

为了进一步探究整相似,来回顾一些数论知识.称$K$是数域,如果它是$\mathbb{Q}$的有限扩张,设$O_K$是$K$上的整数环,它就是$\mathbb{Z}$在$K$中的正规化.$O_K$作为群是秩为$n=[K:\mathbb{Q}]$的自由交换群.于是$n$阶可逆整数方阵可以理解为这个自由交换群上的基变换矩阵.另外,如果取$O_K$的非零理想$I$,任取$I$的非零元$\alpha$,以及任取$O_K$作为自由交换群的一组基$\{e_1,e_2,\cdots,e_n\}$,那么$\{\alpha e_1,\alpha e_2,\cdots,\alpha e_n\}$必然是线性无关的,于是$I$作为交换群必然是秩为$n$的$O_K$作为自由交换群的子群,于是按照子群结构定理,存在$O_K$的一组基$\{z_1,z_2,\cdots,z_n\}$和不变因子组$s_1\mid s_2\mid\cdots\mid s_n$使得$\{s_1z_1,\cdots,s_nz_n\}$是$I$的一组基.

代数数论里知道在整环的全体非零理想上可以定义一个等价类,$I_1$和$I_2$等价当且仅当存在整环中的元$a,b\not=0$使得$aI_1=bI_2$,把等价类称为理想类,全体等价类上具有乘法,并且以单位理想作为幺元,于是它构成一个幺半群群,当整环是戴德金整环时,特别的如果是$O_K$的情况,这时候理想类构成一个群,称为理想类群.数域的类群的阶总是有限的,称为数域$K$或者整环$O_K$的类数,这是经典的类数有限定理.类数1的情况等价于$O_K$是一个PID.

按照本原元定理,可以取数域$K$中一个合适的代数数$\theta$使得$K=\mathbb{Q}(\theta)$,并且有$\mathbb{Z}[\theta]$是$O_K$的子环.如果取$O_K$的一组基$\{\omega_1,\omega_2,\cdots,\omega_n\}$,那么左乘$\theta$是$O_K$上的一个交换群同态,它对应的矩阵记作$B=(b_{ij})$.那么$B$的特征多项式就是代数数$\theta$在$\mathbb{Q}$上的不可约多项式.

现在来说明一个整数矩阵必然可以整相似于准上三角矩阵,其中每个对角块的特征多项式对应于原矩阵的特征多项式的一个不可约因式.取整数方阵$A$,取$A$的特征多项式的一个不可约因式$p(x)$,设它是$k$次的,取$p(x)$在$\mathbb{C}$中的一个根是$\theta$,它是$A$的落在$K=\mathbb{Q}(\theta)$的特征值,于是有$n\times 1$的向量$x\in O_K^n$使得$Ax=\theta x$.另外如果取$O_K$的一组基$\omega=(\omega_1,\cdots,\omega_k)^T$,那么存在$n\times k$的整数矩阵$C$使得$x=C\omega$.另外存在$k$阶整数方阵$B$使得$\theta\omega=B\omega$.于是我们有$Ax=AC\omega=\theta C\omega=C\theta\omega=CB\omega$.于是$(AC-CB)\omega=0$,记$C$的Smith标准型是$C=USV$,其中$U,V$是相应阶数的可逆整数方阵,记$U^{-1}AU=\left(\begin{array}{cc}
A_{11}&A_{12}\\
A_{21}&A_{22}\end{array}\right)$,其中$A_{11}$是$k$阶子矩阵.那么得到$A_{11}z=\theta z$,$A_{21}=0$.这说明$A_{11}$以$p(x)$为特征多项式.继续归纳下去就得到结论.

例如,如果整数方阵$A$满足$A^2=E$,上面的结论告诉$A$必然可以整相似于一个上三角矩阵,它的对角元是$\pm1$.注意到一个对角元全部为1的上三角矩阵如果满足平方是$E$,那么它必然就是单位矩阵.于是这样的矩阵$A$,存在整数可逆方阵$Q$满足$Q^{-1}AQ=\left(\begin{array}{cc}
E_r&T\\
0&-E_s\end{array}\right)$.可以取可逆整数方阵$U_1,V_1$使得$U_1^{-1}TV_1=S$是Smith标准型,那么得到:
$$\left(\begin{array}{cc}
U_1&0\\
0&V_1\end{array}\right)^{-1}Q^{-1}AQ\left(\begin{array}{cc}
U_1&0\\
0&V_1\end{array}\right)=\left(\begin{array}{cc}
E_r&S\\
0&-E_s\end{array}\right)$$

但是还有:
$$\left(\begin{array}{cc}
E_r&X\\
0&-E_s\end{array}\right)^{-1}\left(\begin{array}{cc}
E_r&S\\
0&-E_s\end{array}\right)\left(\begin{array}{cc}
E_r&X\\
0&-E_s\end{array}\right)=\left(\begin{array}{cc}
E_r&S+2X\\
0&-E_s\end{array}\right)$$

通过选取$X$为适当的对角矩阵,可以要求$S$的对角元是1或0,导致$A$相似于:
$$B_m=\left(\begin{array}{cc}
E_r&\left(\begin{array}{cc}
E_m&0\\
0&0\end{array}\right)\\
0&-E_s\end{array}\right)$$

其中$0\le m\le\min\{r,s\}$.通过$\mod2$转化为域$\mathbb{F}_2$中的矩阵,我们看到任意两个$m$取值,$B_m$都是不相似的.于是它们两两不是整相似的,即满足$A^2=E$的整数方阵的相似类有$\min\{r,s\}$个.

这个例子告诉整数环上方阵的相似关系要比域的情况复杂.在域的情况里可以用不变因子组完全刻画相似关系,但是在整数环上,会看到即便是特征多项式本身是不可约多项式的情况,相似类的个数也是不唯一的.

现在假设整数方阵$A$具有$n$次不可约的特征多项式$f(x)$,设$\theta\in\mathbb{C}$是它的一个根.取$K=\mathbb{Q}(\theta)$,于是有分量在$\mathbb{Z}[\theta]$中的特征向量$v\not=0$,$Av=\theta v$.记$v$的$n$个分量生成的$\mathbb{Z}[\theta]$的子群是$I$.那么按照$\theta v=Av$,这个$I$实际上是$\mathbb{Z}[\theta]$的理想.现在$\mathbb{Z}[\theta]$是一个整环,于是可以定义理想类.断言可以构造从以不可约多项式$f(x)$为特征多项式的整数方阵的等价类,到$\mathbb{Z}[\theta]$的理想类的映射.这个映射就是把一个方阵的属于$\theta$的特征向量的各分量生成的$\mathbb{Z}[\theta]$的理想所诱导的两个等价关系下等价类之间的映射.先来验证这个映射定义的良性,为此如果$A$是以$f(x)$为特征多项式的整数方阵,它的属于$\theta$的特征向量$v$各分量生成了$\mathbb{Z}[\theta]$的理想$I$,现在给定整数可逆方阵$V$,使得$\omega$是分量在$\mathbb{Z}[\theta]$中的$V^{-1}AV$的属于$\theta$的特征向量,那么$V\omega=\alpha v$,其中$\alpha\in\mathbb{Q}[\theta]$,就有$\alpha=\beta/\gamma$,其中$\beta,\gamma\in\mathbb{Z}[\theta]$,设$\omega$的$n$个分量生成的理想是$J$,那么就有$\gamma J=\beta I$,于是$I,J$在同一个理想类.

接下来说明诱导的映射是单射.为此,如果有理想$I$被$A$的特征向量$v$的各分量生成,任取$I$的理想类中另一个理想$I'$,可以找到代数整元$\alpha,\beta$,满足$\alpha I=\beta I'$.那么可以取$I'$中一组基构成的向量$v'$,使得存在一个可逆整数方阵$U$满足$\alpha Uv'=\beta v$,这导致$v'$是$U^{-1}AU$的一个特征向量.最后说明诱导映射是满射.为此,任取一个理想$I$,取它作为自由交换群的一组基,以$\theta$左乘是一个交换群同态,这个同态的矩阵表示是一个以$f(x)$为特征多项式的矩阵.而这组基凑成的列向量就是生成$I$的特征向量.

综上,得到\textbf{Latimer–MacDuffee定理}:给定不可约首一$n$次整系数多项式$f(x)$,取$f(x)$在$\mathbb{C}$中的一个根$\theta$,那么$n$阶的以$f(x)$为特征多项式的整数方阵的整相似等价类个数,就是整环$\mathbb{Z}[\theta]$的理想类的个数.由于它是$O_K$的子环,其中$K=\mathbb{Q}(\theta)$,于是这个理想类的个数必然是有限的.
\newpage
\subsection{张量积}

本节我们要介绍的是线性空间的张量积和矩阵的张量积.线性空间的张量积我们给出两种定义方式,一种是借助多重线性映射来描述的具体构造,一种是按照泛映射性质的抽象构造.我们就考虑实数域$\mathbb{R}$上的线性空间.

设$V_1,V_2,\cdots,V_k$和$W$都是$\mathbb{R}$有限维线性空间.一个映射$f:V_1\times V_2\times\cdots\times V_k\to W$称为多重线性映射,如果它对每个分量(即固定其余分量的取值)都是线性映射.全体这样的多重线性映射构成的集合记作$\mathrm{L}(V_1,V_2,\cdots,V_k;W)$.

张量积的定义.设$V_1,\cdots,V_s,W_1,\cdots,W_t$都是多重线性函数(即终端都是$\mathbb{R}$).设$f\in\mathrm{L}(V_1,V_2,\cdots,V_s;\mathbb{R})$和$g\in\mathrm{L}(W_1,W_2,\cdots,W_t;\mathbb{R})$.那么我们定义$f\otimes g$是$V_1\times\cdots\times V_s\times W_1\times\cdots\times W_t$上的多重线性函数,满足$(v_1,\cdots,v_s,w_1,\cdots,w_t)\mapsto f(v_1,\cdots,v_s)g(w_1,\cdots,w_t)$.

$\mathrm{L}(V_1,V_2,\cdots,V_k;\mathbb{R})$是$\mathbb{R}$有限维线性空间,如果记$\dim V_i=n_i$,那么它的维数是$n_1\cdots n_k$.如果记$\{E_1^{j},\cdots,E_{n_j}^j\}$是$V_j$的一组基,它的对偶基记作$\{\varepsilon_j^1,\cdots,\varepsilon_j^{n_j}\}$,那么这个空间的一组基为$\mathscr{B}=\{\varepsilon_1^{i_1}\otimes\cdots\varepsilon_k^{i_k}\mid1\le i_1\le n_1,\cdots,1\le i_k\le n_k\}$.
\begin{proof}
	
	容易验证$\mathrm{L}(V_1,V_2,\cdots,V_k;\mathbb{R})$是$\mathbb{R}$线性空间.下面仅需验证$\mathscr{B}$生成其中所有多重线性函数,并且$\mathscr{B}$是线性无关组.
	
	先证明它生成了整个空间:任取$V_1\times\cdots\times V_k$上的多重线性函数$f$,对指标组$\{i_1,i_2,\cdots,i_k\}$,我们记$a_{i_1,\cdots,i_k}=f(E_{i_1}^1,E_{i_2}^2,\cdots,E_{i_k}^k)$.那么我们断言有$f=\sum_{i_1,\cdots,i_k}a_{i_1,\cdots,i_k}\varepsilon_1^{i_1}\otimes\cdots\otimes\varepsilon_k^{i_k}$.因为任取$(v_1,v_2,\cdots,v_k)\in V_1\times V_2\times\cdots\times V_k$,记$v_j=\sum_{i_j}v_j^{i_j}E_{i_j}^j$,那么有:
	\begin{align*}
	\sum_{i_1,\cdots,i_k}a_{i_1,\cdots,i_k}\varepsilon_1^{i_1}\otimes\cdots\times\varepsilon_k^{i_k}(v_1,v_2,\cdots,v_k)&=\sum_{i_1,\cdots,i_k}a_{i_1,\cdots,i_k}\varepsilon_1^{i_1}(v_1)\cdots\varepsilon_k^{i_k}(v_k)\\&=\sum_{i_1,\cdots,i_k}a_{i_1,\cdots,i_k}v_1^{i_1}\cdots v_k^{i_k}\\&=f(\sum_{i_1}v_1^{i_1}E_{i_1}^1,\cdots,\sum_{i_k}v_k^{i_k}E_{i_k}^k)=f(v_1,v_2,\cdots,v_k)
	\end{align*}
	
	再证明$\mathscr{B}$是线性无关的:如果有$\sum_{i_1,\cdots,i_k}a_{i_1,\cdots,i_k}\varepsilon_1^{i_1}\otimes\cdots\otimes\varepsilon_k^{i_k}=0$.把它作用到向量组$(E_{j_1}^1,\cdots,E_{j_k}^k)$上得到$a_{j_1,\cdots,j_k}=0$,这得到线性无关性.
\end{proof}

抽象张量积.线性空间是模的特例,模的张量积定义完全可以放在线性空间上,给定$\mathbb{R}$线性空间$V_1,V_2,\cdots,V_k$,它们的张量积被如下泛映射性质所描述:对每个$\mathbb{R}$模$W$和每个多重线性映射$f:V_1\times V_2\times\cdots\times V_k\to W$,都存在唯一的$\mathbb{R}$模同态$\widetilde{f}:V_1\otimes\cdots\otimes V_k\to W$使得如下图表交换.
$$\xymatrix{V_1\otimes\cdots\otimes V_k\ar[rr]^{\widetilde{f}}&&W\\V_1\times\cdots\times V_k\ar[u]\ar[urr]_f&&}$$

两种张量积的等价性.给定$\mathbb{R}$有限维线性空间$V_1,V_2,\cdots,V_k$,那么存在典范(没有依赖基的选取)的同构:
$$V_1^*\otimes\cdots\otimes V_k^*\cong\mathrm{L}(V_1,\cdots,V_k;\mathbb{R})$$
$$\omega_1\otimes\cdots\otimes\omega_k\mapsto\left((v_1,v_2,\cdots,v_k)\mapsto\omega^1(v_1)\cdots\omega^k(v_k)\right)$$
\begin{proof}
	
	先构造多重线性映射$\Phi:V_1^*\times\cdots\times V_k^*\to\mathrm{L}(V_1,V_2,\cdots,V_k;\mathbb{R})$为$\Phi(\omega^1,\cdots,\omega^k)=\omega^1\otimes\cdots\otimes\omega^k$.于是按照张量积的泛性质,存在$\mathbb{R}$模同态$\widetilde{\Phi}:V_1^*\otimes\cdots\otimes V_k^*\to\mathrm{L}(V_1,V_2,\cdots,V_k;\mathbb{R})$.并且这个映射把抽象张量积的一组基映射为后者的一组基,这说明$\widetilde{\Phi}$是同构,并且这个同构不依赖基的选取.
\end{proof}

















给定两个线性空间$V_1,V_2$,设分别的一组基为$\{v_1^{1},\cdots,v_n^1\}$,$\{v_2^2,\cdots,v_q^2\}$.那么$V_1\otimes V_2$就是以$\{v_i^1\otimes v_j^2,1\le i\le n,1\le j\le q\}$为基的一个线性空间,所以它的维数是$nq$.再取维数分别为$m,p$的线性空间$W_1,W_2$,设一组基分别为$\{w_1^1,\cdots,w_m^1\}$和$\{w_1^2,\cdots,w_p^2\}$.那么同样可以定义$W_1\otimes W_2$为一个$mp$维线性空间.现在如果给定两个线性变换$f:V_1\to W_1$和$g:V_2\to W_2$,定义两个线性变换$f,g$的一个有序基,称为张量积,为$f\otimes g:V_1\otimes V_2\to W_1\otimes W_2$,$(v_1\otimes v_2)\mapsto(f(v_1)\otimes g(v_2))$.线性映射的张量积的确是一个线性映射.

期望从上述线性变换的张量积的定义诱导出矩阵的一种新的积.为此,设在上述基下$f,g$的矩阵表示分别为$A,B$,那么$A$是$m\times n$的矩阵,$B$是$p\times q$的矩阵,按照线性变换的矩阵表示,把它们的(有序的)Kronecker积定义为:
$$\mathbf{A}\otimes\mathbf{B} = \begin{bmatrix} a_{11} \mathbf{B} & \cdots & a_{1n}\mathbf{B} \\ \vdots & \ddots & \vdots \\ a_{m1} \mathbf{B} & \cdots & a_{mn} \mathbf{B} \end{bmatrix}$$

具体的写就是:

$$\mathbf{A}\otimes\mathbf{B} = \begin{bmatrix}
a_{11} b_{11} & a_{11} b_{12} & \cdots & a_{11} b_{1q} &
\cdots & \cdots & a_{1n} b_{11} & a_{1n} b_{12} & \cdots & a_{1n} b_{1q} \\
a_{11} b_{21} & a_{11} b_{22} & \cdots & a_{11} b_{2q} &
\cdots & \cdots & a_{1n} b_{21} & a_{1n} b_{22} & \cdots & a_{1n} b_{2q} \\
\vdots & \vdots & \ddots & \vdots & & & \vdots & \vdots & \ddots & \vdots \\
a_{11} b_{p1} & a_{11} b_{p2} & \cdots & a_{11} b_{pq} &
\cdots & \cdots & a_{1n} b_{p1} & a_{1n} b_{p2} & \cdots & a_{1n} b_{pq} \\
\vdots & \vdots & & \vdots & \ddots & & \vdots & \vdots & & \vdots \\
\vdots & \vdots & & \vdots & & \ddots & \vdots & \vdots & & \vdots \\
a_{m1} b_{11} & a_{m1} b_{12} & \cdots & a_{m1} b_{1q} &
\cdots & \cdots & a_{mn} b_{11} & a_{mn} b_{12} & \cdots & a_{mn} b_{1q} \\
a_{m1} b_{21} & a_{m1} b_{22} & \cdots & a_{m1} b_{2q} &
\cdots & \cdots & a_{mn} b_{21} & a_{mn} b_{22} & \cdots & a_{mn} b_{2q} \\
\vdots & \vdots & \ddots & \vdots & & & \vdots & \vdots & \ddots & \vdots \\
a_{m1} b_{p1} & a_{m1} b_{p2} & \cdots & a_{m1} b_{pq} &
\cdots & \cdots & a_{mn} b_{p1} & a_{mn} b_{p2} & \cdots & a_{mn} b_{pq}
\end{bmatrix}$$