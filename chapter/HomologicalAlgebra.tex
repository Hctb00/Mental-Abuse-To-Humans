\chapter{同调代数}
\section{一般理论}
\subsection{链同伦范畴}

链复形范畴.设$\mathscr{A}$是加性范畴.
\begin{enumerate}
	\item 称一族对象和态射$(X^n,d^n)_{n\in\mathbb{Z}}$满足$d^n$是$x^n\to X^{n+1}$的态射,通常称为微分映射,并且$d^{n+1}\circ d^n=0,\forall n\in\mathbb{Z}$,就称$X^{\bullet}=(X^n,d^n)_{n\in\mathbb{Z}}$是上链复形:
	$$\xymatrix{\cdots\ar[r]&X^n\ar[r]^{d^n}&X^{n+1}\ar[r]^{d^{n+1}}&X^{n+2}\ar[r]&0}$$
	
	对偶的我们把一族对象和态射$(X_n,d_n)_{n\in\mathbb{Z}}$称为链复形,如果满足$d_n$是$X_n\to X_{n-1}$的态射,通常称为微分映射,使得$d_n\circ d_{n+1}=0,\forall n\in\mathbb{Z}$:
	$$\xymatrix{\cdots\ar[r]&X_{n+1}\ar[r]^{d_{n+1}}&X_n\ar[r]^{d_n}&X_{n-1}\ar[r]&\cdots}$$
	
	链复形和上链复形没有本质区别,使用哪个只是一种习惯.
	\item 给定两个$\mathscr{A}$上的复形$(C_*,d_*)$和$(C_*',d_*')$,称一族态射$f=(f_n:C_n\to C_n')$为$(C_*,d_*)\to(C_*',d_*')$的链映射,如果满足交换图:
	$$\xymatrix{\cdots\ar[r]&C_{n+1}\ar[r]^{d_{n+1}}\ar[d]^{f_{n+1}}&A_n\ar[r]^{d_n}\ar[d]^{f_n}&A_{n-1}\ar[r]\ar[d]^{f_{n-1}}&\cdots\\\cdots\ar[r]&C_{n+1}'\ar[r]^{d_{n+1}'}&C_n'\ar[r]^{d_n'}&C_{n-1}'\ar[r]&\cdots}$$
	
	当给定一个链复形$\cdots\to A\to B\to C\to\cdots$时,实际隐含约定了一个从$\mathbb{Z}$到其上全部对象的逐项依次对应,即隐含约定了存在某个整数$m$使得这里$A$是第$m$项,$B$是第$m-1$项,$C$是第$m-2$项.这才使得提及两个链复形之间的链映射时是没有歧义的,此时链复形第$m$个分量将会是这两个复形中第$m$个分量之间的态射.
	\item 加性范畴$\mathscr{A}$上的全体链复形和链态射构成一个加性范畴,记作$\textbf{Comp}(\mathscr{A})$:
	\begin{itemize}
		\item 它是预加性范畴:即Hom集总存在交换群结构使得链映射的复合是双线性的.同一个Hom集中的两个链映射$f=(f_n),g=(g_n)$的加法就约定为$f+g=(f_n+g_n)$.零态射恰好就是零态射构成的链映射$(0_n)$.
		\item 零对象.取$\mathscr{A}$中的零对象0,那么$\textbf{Comp}$中的零对象为$\textbf{0}:\to0\to0\to\cdots$.
		\item 二元对称积.给定两个复形$(C_*,d_*)$和$(C_*',d_*')$,它们的积和余积总存在,并且同构的(事实上预加性范畴上二元积存在等价于二元余积存在,并且它们是同构的,即二元对称积).它就是$(C_n\oplus C_n',d_n\oplus d_n')$.
	\end{itemize}
    \item 设$(X_n,d_n)$是复形,如果只存在有限个指标使得$X_n\not=0$,就称复形是有界的;如果$n$足够大时$X_n=0$,就称复形是上有界的;如果$-n$足够大时$X_n=0$,就称复形是下有界的;如果$n<0$时$X_n=0$就称复形是正项复形;如果$n>0$时$X_n=0$就称复形是负项复形,负项复形就表示为$0\to X_0\to X_{-1}\to X_{-2}\to\cdots$,但是我们通常就记作上链形式$0\to C^0\to C^1\to C^2\to\cdots$.这些特殊复形构成的完全子范畴都是加性范畴,分别称为有界复形范畴$\textbf{Comp}^b(\mathscr{A})$;上有界复形范畴$\textbf{Comp}^-(\mathscr{A})$;下有界复形范畴$\textbf{Comp}^+(\mathscr{A})$;正项复形范畴$\textbf{Comp}^{\ge0}(\mathscr{A})$;负项复形范畴$\textbf{Comp}^{\le0}(\mathscr{A})$.
	\item 单满态射和子商对象有如下刻画.特别的如果$\mathscr{A}$是阿贝尔范畴,那么链复形范畴$\textbf{Comp}$也是阿贝尔范畴.这实际是充要的,因为总可以把$\mathscr{A}$中的对象等同于复形$(A,1_A)=\xymatrix{A\ar[r]^{1_A}&A}$,把$\mathscr{A}$中的态射$f:A\to B$等同于链映射$(f_n=f):(A,1_A)\to(B,1_B)$.
	\begin{itemize}
		\item 链映射$f=(f_n)$是单态射当且仅当每个$f_n$是$\mathscr{A}$中的单态射;它是满态射当且仅当每个$f_n$是$\mathscr{A}$中的满态射.于是复形$C_*$的子复形可视为一个单链态射$i:A_*\to C_*$;商复形可视为一个满链态射$j:C^*\to B^*$.
		\item 特别的,对于模范畴,给定链复形$(C_*,d_*)$,它的子复形$(A_*,d_*')$即,每个$A_n$是$C_n$的子对象,满足$d_n(A_n)\subset A_{n-1}$,此时$d_n'$约定为$d_n$在$A_n$上的限制;关于子复形$(A_*,d_*')$的商复形定义为$(C_*/A_*,e_*)$,其中对象为商$C_n/A_n$,微分定义为典范映射$e_n:C_n/A_n\to C_{n-1}/A_{n-1}$,即$x+A_n\mapsto d_n(x)+A_{n-1}$.从子复形定义中的$d_n(A_n)\subset A_{n-1}$说明这个映射定义良性.
		\item 特别的,对于模范畴,链态射$f=(f_n):(C_*,d_*)\to(C_*',d_*')$的核$\ker f$定义$(C_*,d_*)$的子复形$\to\ker f_{n+1}\to\ker f_n\to\ker f_{n-1}\to\cdots$,其中微分$\ker f_n\to\ker f_{n-1}$定义为$d_n$在$\ker f_n$的限制.$f$的像$\mathrm{im}f$定义为$(C_*',d_*')$的子复形$\to\mathrm{im}f_{n+1}\to\mathrm{im}f_n\to\mathrm{im}f_{n-1}\to\cdots$,其中微分$\mathrm{im}f_n\to\mathrm{im}f_{n-1}$定义为$d_n'$在$\mathrm{im}f_n$上的限制.$f$的余核$\mathrm{coker}f$定义为$(C_*',d_*')$的商复形$\to C_{n+1}'/\mathrm{im}f_{n+1}\to C_n'/\mathrm{im}f_n\to C_{n-1}'/\mathrm{im}f_{n-1}\to\cdots$.其中微分$C_n'/\mathrm{im}f_n\to C_{n-1}'/\mathrm{im}f_{n-1}$定义为$x+\mathrm{im}f_n\mapsto d_n'(x)+\mathrm{im}f_{n-1}$.
	\end{itemize}
	\item 加性函子.因为加性函子会把零对象映射为零对象,零态射映射为零态射.于是加性函子作用在一个复形上仍为复形,这导致两个加性范畴之间的加性函子会诱导出相应链复形范畴上的加性函子.
	\item 复形的复形.设$A=(A_*,\alpha_*)$,$B=(B_*,\beta_*)$,$C=(C_*,\gamma_*)$是加性范畴$\mathscr{A}$上的三个链复形,取序列$$\xymatrix{A\ar[r]^f&B\ar[r]^g&C}$$其中$f,g$是链映射,那么这是复形等价于讲$g\circ f=0$,也即所有$g_n\circ f_n=0$.另外上述复形是正合的,无非是等价于讲每个分量上在$B$处是正合的,也即如下交换图中每列在中间处正合:
		$$\xymatrix{
		\cdots\ar[r]&A_{n+1}\ar[r]^{\alpha_{n+1}}\ar[d]^{f_{n+1}}&A_n\ar[r]^{\alpha_n}\ar[d]^{f_n}&A_{n-1}\ar[r]\ar[d]^{f_{n-1}}&\cdots\\
		\cdots\ar[r]&B_{n+1}\ar[r]^{\beta_{n+1}}\ar[d]^{g_{n+1}}&B_n\ar[r]^{\beta_n}\ar[d]^{g_n}&B_{n-1}\ar[r]\ar[d]^{g_{n-1}}&\cdots\\
		\cdots\ar[r]&C_{n+1}\ar[r]^{\gamma_{n+1}}&C_n\ar[r]^{\gamma_n}&C_{n-1}\ar[r]&\cdots}$$
\end{enumerate}

同调函子.给定阿贝尔范畴$\mathscr{A}$,下面构造一列$\textbf{Comp}(\mathscr{A})\to\mathscr{A}$的函子列$\{H_n,n\in\mathbb{Z}\}$称为同调函子.
\begin{enumerate}
	\item 对每个复形$C=(C_*,d_*)$,按照代数拓扑的叫法,称$C_n$中的元为$n-$链,称$Z_n(C)=\ker d_n$中的元为$n-$圈,称$B_n(C)=\mathrm{im}d_{n+1}$中的元为$n-$边.那么复形条件要求了全部$n-$边是$n-$圈,也即$B_n(C)\subset Z_n(C),n\in\mathbb{Z}$.定义$C$的第$n$个同调为商对象为$H_n(C)=Z_n(C)/B_n(C)$.特别的,$z\in Z_n(C)$在$H_n(C)$中的像记作$\mathrm{cls}(z)$,称为$z$所在的同调类,它实际也就是$z+B_n(C)$.
	\item 几个基本例子.
	\begin{enumerate}
		\item 明显的,一个复形是正合列当且仅当处处同调为零.
		\item 如果复形$(C_*,d_*)$的微分$d_n$恒为零态射.此时$Z_n(C)=C_n$,$B_n(C)=0$,于是每个$H_n(C)=C_n$.
		\item 给定态射$f:A\to B$,把它拼接为正项复形$C=\to0\to0\to A\to B\to0$.那么$H_0(C)=B/\mathrm{im}f=\mathrm{coker}f$,$H_1(C)=\ker f$,对于其它的$n\not=0,1$,有$H_n(C)=0$.
	\end{enumerate}
	\item 链映射诱导的态射,不妨考虑$\textbf{Ab}$的情况.给定链映射$f=(f_n):C=(C_*,d_*)\to C'=(C_*',d_*')$.定义$H_n(f):H_n(C)\to H_n(C')$为$\mathrm{cls}(z_n)\mapsto\mathrm{cls}(f_nz_n)$.先验证这个定义的良性.只需验证如果$z\in B_n(C)$,那么$f_n(z)\in B_n(C')$,这从链映射满足的下述交换图是直接得到的:记$z=d_{n+1}(y)$,那么$f_n(z)=f_n\circ d_{n+1}(y)=d_n'\circ f_{n+1}(y)\in B_n(C')$.
	$$\xymatrix{C_{n+1}\ar[r]^{d_{n+1}}\ar[d]_{f_{n+1}}&C_n\ar[d]_{f_n}\\C_{n+1}'\ar[r]_{d_{n+1}'}&C_n'}$$
	\item 容易验证$H_n(-)$在链映射上满足$H_n(1_C)=1_{H_n(C)}$,$H_n(f\circ g)=H_n(f)\circ H_n(g)$,$H_n(f+g)=H_n(f)+H_n(g)$.于是$H_n(-)$是$\textbf{Comp}(\mathscr{A})\to\mathscr{A}$的加性函子.
	\item 复形上的同调函子和正合加性函子可交换.对阿贝尔范畴$\mathscr{A}$,取一个复形$(C_*,d_*)$,设$T:\mathscr{A}\to\mathscr{B}$是正合的加性函子,对任意$n\in\mathbb{Z}$,恒有同构$H_n(TC_*,Td_*)\cong TH_n(C_*,d_*)$.
	\begin{proof}
		
		考虑短正合列$0\to\mathrm{im}d_{n+1}\to\ker d_n\to H_n(C_*)\to0$.将正合加性函子$T$作用其上,得到短正合列$0\to T(\mathrm{im}d_{n+1})\to T(\ker d_n)\to TH_n(C_*)\to0$.按照$T$是正合加性函子,它和像可交换,也和核可交换,即有$T(\mathrm{im}d_ {n+1})=\mathrm{im}Td_{n+1}$和$T(\ker d_n)=\ker Td_n$.于是按照短五引理,得到$H_n(TC_*)\cong TH_n(C_*)$.
	\end{proof}
\end{enumerate}

长正合列定理(同调代数基本定理).定理断言链复形的短正合列诱导出一个正合列,蛇形引理将会是这个定理的一种特例,另外我们会证明连接映射会存在某种自然性.设$C',C,C''$是阿贝尔范畴$\mathscr{A}$上的三个复形.存在链映射$i,p$使得它们构成一个复形的短正合列.
$$\xymatrix{0\ar[r]&C'\ar[r]^i&C\ar[r]^p&C''\ar[r]&0}$$
\begin{enumerate}
	\item 连接映射.先定义$\partial_n^*:Z_n(C'')\to H_{n-1}(C')$是这样的映射,任取$z_n''\in Z_n''$,按照$p_n$满射知存在$c_n\in C_n$使得$p_n(c_n)=c_n''$.于是$0=d_n''(z_n'')=p_{n-1}(d_n(c_n))$,于是$d_n(c_n)\in\ker p_{n-1}=\mathrm{im} i_{n-1}$.于是存在唯一的$c_{n-1}'\in C_{n-1}'$使得$i_{n-1}(c_{n-1}')=d_n(c_n)$.就定义$z_n''\mapsto\mathrm{cls}(c_{n-1}')$.
	
	要证明这个定义良性还需要说明两件事,首先说明$c_{n-1}'\in Z_{n-1}'$.因为$d_{n-1}(i_{n-1}(c_{n-1}'))=d_{n-1}(d_n(c_n))=0$,于是$i_{n-2}(d_{n-1}'(c_{n-1}'))=0$,按照$i_{n-2}$是单射,就得到$c_{n-1}'\in\ker d_{n-1}'=Z_{n-1}'$.
	
	其次是说明这个定义的良性,也即取值不随一开始的$c_n$的选取而改变.为此假设$y\in C_n$还满足$p_n(y)=p_n(c_n)=z_n''$.于是$y-c_n\in\ker p_n=\mathrm{im}i_n$.于是存在$c_n'\in C_n'$使得$i_n(c_n')=y-c_n$.接下来作用$d_n$得到$d_n(y)-d_n(c_n)=d_n(i_n(c_n'))=i_{n-1}(d_n'(c_n'))$.记唯一的$x\in C_{n-1}'$使得$i_{n-1}(x)=d_n(y)$,于是得到$i_{n-1}(x-c_n)=i_{n-1}(d_n'(c_n'))$,按照$i_{n-1}$是单射,得到$x-c_n=d_n'(c_n')\in B_n'$,于是$\mathrm{cls}(x)=\mathrm{cls}(c_n')$.
	
	容易验证$\partial^*$是交换群同态.最后为说明它诱导了典范同态$\partial:H_n(C'')\to H_n(C')$,只需注意到$B_n(C'')$落在$\ker\partial^*$中.事实上这个映射的构造和蛇形引理中连接映射的构造是一样的.
	$$\xymatrix{
		&\ar[d]&\ar[d]&\ar[d]&\\
		0\ar[r]&C_{n}'\ar[r]^{i_{n}}\ar[d]_{d_{n}'}&C_{n}\ar@/^/[d]\ar[r]^{p_{n}}\ar[d]_{d_n}&C_{n}''\ar@/^/[l]\ar[r]\ar[d]_{d_n''}&0\\
		0\ar[r]&C_{n-1}'\ar[r]^{i_{n-1}}\ar[d]_{d_{n-1}'}&C_{n-1}\ar@/^/[l]\ar[r]^{p_{n-1}}\ar[d]_{d_{n-1}}&C_{n-1}''\ar[r]\ar[d]_{d_{n-1}''}&0\\
		0\ar[r]&C_{n-2}'\ar[r]^{i_{n-2}}\ar[d]&C_{n-2}\ar[r]^{p_{n-2}}\ar[d]&C_{n-2}''\ar[d]\ar[r]&0\\
		&&&&}$$
	\item 长正合列定理.$\xymatrix{0\ar[r]&C'\ar[r]^i&C\ar[r]^p&C''\ar[r]&0}$是阿贝尔范畴$\mathscr{A}$上复形的短正合列,那么它诱导了长正合列:
	$$\xymatrix{\ar[r]&H_{n+1}(C'')\ar[r]^{\partial_{n+1}}&H_n(C')\ar[r]^{H_n(i)}&H_n(C)\ar[r]^{H_n(p)}&H_n(C'')\ar[r]^{\partial_n}&H_{n-1}(C')\ar[r]&}$$
	\begin{proof}
		
		同样只需考虑$\textbf{Ab}$的情况,要验证如下六件事:
		\begin{enumerate}
			\item $\mathrm{im}H_n(i)\subset\ker H_n(p)$.这从$H_n(p)\circ H_n(i)=H_n(p\circ i)=H_n(0)=0$直接得证.
			\item $\ker H_n(p)\subset\mathrm{im}H_n(i)$.设$\mathrm{cls}(z)\in\ker H_n(p)$,那么$\mathrm{cls}(p_n(z))=0$,即存在$c_{n+1}''\in C_{n+1}''$使得$p_n(z)=d_{n+1}''(c_{n+1}'')$.而按照$p_{n+1}$是满射,存在$c_{n+1}\in C_{n+1}$使得$c_{n+1}''=p_{n+1}(c_{n+1})$.
			
			于是$p_n(z)=d_{n+1}''(p_{n+1}(c_{n+1}))=p_n(d_{n+1}(c_{n+1}))$.这导致$z-d_{n+1}(c_{n+1})\in\ker p_n=\mathrm{im}i_n$,于是存在$c_n'\in C_n'$使得$i_n(c_n')=z-d_{n+1}(c_{n+1})$.而$c_n'$实际上在$Z_n(C')$中,因为$i_{n-1}(d_n'(c_n'))=d_n(z)-d_n(d_{n+1}(c_{n+1}))=0$.这导致$H_n(i)(\mathrm{cls}(c_n'))=\mathrm{cls}(i_n(c_n'))=\mathrm{cls}(z-d_{n+1}(c_{n+1}))=\mathrm{cls}(z)$.即$\mathrm{cls}(z)\in\mathrm{im}H_n(i)$.
			\item $\mathrm{im}H_n(p)\subset\ker\partial_n$.设$H_n(p)(\mathrm{cls}(c_n))=\mathrm{cls}(p_n(c_n))\in\ker H_n(p)$,其中$c_n\in Z_n(C)$.于是$d_n(c_n)=0$,于是唯一的满足$i_{n-1}(c_{n-1}')=d_n(c_n)$的$c_{n-1}'\in C_{n-1}'$是零元,这导致$\partial_n(\mathrm{cls}(p_n(c_n)))=0$.
			\item $\ker\partial_n\subset\mathrm{im}H_n(p)$.设$c_n''\in Z_n(C'')$,使得$\partial_n(\mathrm{cls}(c_n''))=0$.也即取$c_n\in C_n$使得$p_n(c_n)=c_n''$,那么唯一的使得$i_{n-1}(c_{n-1}')=d_n(c_n)$的$c_{n-1}'\in C_{n-1}'$落在$B_{n-1}(C')$中.也即存在$c_n'\in C_n'$使得$c_{n-1}'=d_n'(c_n')$.于是有$d_n(i_n(c_n'))=i_{n-1}(d_n'(c_n'))=d_n(c_n)$.这说明$i_n(c_n')-c_n\in Z_n(C)$中.于是$H_n(p)(\mathrm{cls}(c_n-i_n(c_n')))=\mathrm{cls}(c_n'')$,即$\mathrm{cls}(c_n'')\in\mathrm{im}H_n(p)$中.
			\item $\mathrm{im}\partial_{n+1}\subset\ker H_n(i)$.任取$c_{n+1}''\in Z_{n+1}(C'')$,设$c_{n+1}\in C_{n+1}$使得$p_{n+1}(c_{n+1})=c_{n+1}''$.设唯一的$c_n'\in C_n'$满足了$i_n(c_n')=d_{n+1}(c_{n+1})$.于是$\mathrm{cls}(c_n')$是$\mathrm{im}\partial_{n+1}$中的任意元.那么$i_n(c_n')=d_{n+1}(c_{n+1})\in B_n(C)$,于是$\mathrm{cls}(c_n')\in\ker H_n(i)$.
			\item $\ker H_n(i)\subset\mathrm{im}\partial_{n+1}$.取$\mathrm{cls}(c_n')\in\ker H_n(i)$,也即$c_n'\in Z_n(C')$使得$i_n(c_n')\in B_n(C')$.于是可取$c_{n+1}\in C_{n+1}$使得$i_n(c_n')=d_{n+1}(c_{n+1})$.于是$d_{n+1}''(p_{n+1})(c_{n+1})=p_n(d_{n+1}(c_{n+1}))=p_n(i_n(c_n'))=0$,于是$p_{n+1}(c_{n+1})\in Z_{n+1}(C'')$.最后$\partial_{n+1}(\mathrm{cls}(p_{n+1}(c_{n+1})))=\mathrm{cls}(c_n')$,这就得证.
		\end{enumerate}
	$$\xymatrix{
		&\ar[d]&\ar[d]&\ar[d]&\\
		0\ar[r]&C_{n+1}'\ar[r]^{i_{n+1}}\ar[d]_{d_{n+1}'}&C_{n+1}\ar[r]^{p_{n+1}}\ar[d]_{d_{n+1}}&C_{n+1}''\ar[r]\ar[d]_{d_{n+1}''}&0\\
		0\ar[r]&C_{n}'\ar[r]^{i_{n}}\ar[d]_{d_{n}'}&C_{n}\ar[r]^{p_{n}}\ar[d]_{d_{n}}&C_{n}''\ar[r]\ar[d]_{d_{n}''}&0\\
		0\ar[r]&C_{n-1}'\ar[r]^{i_{n-1}}\ar[d]&C_{n-1}\ar[r]^{p_{n-1}}\ar[d]&C_{n-1}''\ar[d]\ar[r]&0\\
		&&&&}$$
	\end{proof}
    \item 现在我们解释蛇形引理如何作为长正合列定理的特例.取三个复形$C',C,C''$分别为长度仅为1的复形:$C'=\xymatrix{A'\ar[r]^f&B'}$,$C'=\xymatrix{A\ar[r]^g&B}$,$C''=\xymatrix{A''\ar[r]^h&B''}$.于是按照本定理它诱导了一个长正合列.现在注意到$H_0(C')=\mathrm{coker}f$,$H_0(C)=\mathrm{coker}g$,$H_0(C'')=\mathrm{coker}h$,$H_1(C')=\ker f$,$H_1(C)=\ker g$,$H_1(C'')=\ker h$,当$n\not=0,1$时三个复形的同调都是零,于是诱导的长正合列即蛇形引理的结论:
    $$\xymatrix{0\ar[r]&\ker f\ar[r]&\ker g\ar[r]&\ker h\ar[r]&\mathrm{coker}f\ar[r]&\mathrm{coker}g\ar[r]&\mathrm{coker}h\ar[r]&0}$$
    \item 连接映射的自然性.设$\mathscr{A}$是阿贝尔范畴,给定$\textbf{Comp}(\mathscr{A})$中两个短正合列之间的链映射(复形的复形之间的链映射):
    $$\xymatrix{0\ar[r]&C'\ar[r]^i\ar[d]_f&C\ar[r]^p\ar[d]_g&C''\ar[r]\ar[d]_h&0\\0\ar[r]&A'\ar[r]_j&A\ar[r]_q&A''\ar[r]&0}$$
    
    那么两个复形的短正合了诱导的长正合列之间存在链映射$(H(f),H(g),H(h))$:
    $$\xymatrix{\ar[r]&H_n(C')\ar[d]_{H_n(f)}\ar[r]^{H_n(i)}&H_n(C)\ar[r]^{H_n(p)}\ar[d]_{H_n(g)}&H_n(C'')\ar[r]^{\partial_n}\ar[d]_{H_n(h)}&H_{n-1}(C')\ar[r]\ar[d]_{H_{n-1}(f)}&\\\ar[r]&H_n(A')\ar[r]_{H_n(j)}&H_n(A)\ar[r]_{H_n(q)}&H_n(A'')\ar[r]_{\partial_n'}&H_{n-1}(A')\ar[r]&}$$
\end{enumerate}

链同伦范畴.
\begin{enumerate}
	\item 链复形之间的$p$次映射.给定两个复形$C=(C_n)$和$D=(D_n)$,对$p\in\mathbb{Z}$,称$s=(s_n)$是$C\to D$的$p$次映射,如果每个$s_n$是$C_n\to D_{n+p}$的态射.注意$p$次映射不需要和微分映射可交换.例如链映射是零次的映射,而微分$d=(d_n)$是$-1$次的映射.
	\item 链映射之间的同伦关系.考虑两个链映射:$f,g:(C,d)\to(C',d')$,称它们是同伦的,如果存在次数为1的链映射$s=(s_n):C\to C'$(不要求和微分映射可交换)满足$f_n-g_n=d_{n+1}'\circ s_n+s_{n-1}\circ d_n,\forall n\in\mathbb{Z}$.这里的$s=(s_n)$也会称为同伦(链)映射.
	$$\xymatrix{\cdots\ar[rr]&&C_{n+1}\ar[rr]^{d_{n+1}}\ar@/_/@<0.1pt>[dd]_{f_{n+1}}\ar@/^/@<-0.1pt>[dd]^{g_{n+1}}&&C_n\ar[ddll]_{s_n}\ar[rr]^{d_n}\ar@/_/@<0.1pt>[dd]_{f_n}\ar@/^/@<-0.1pt>[dd]^{g_n}&&A_{n-1}\ar[ddll]_{s_{n-1}}\ar[rr]\ar@/_/@<0.1pt>[dd]_{f_{n-1}}\ar@/^/@<-0.1pt>[dd]^{g_{n-1}}&&\cdots\\&&&&&&&&\\\cdots\ar[rr]&&C_{n+1}'\ar[rr]_{d_{n+1}'}&&C_n'\ar[rr]_{d_n'}&&C_{n-1}'\ar[rr]&&\cdots}$$
	\item 验证同伦是链复形之间Hom集上的等价关系.自反性只需取$s=(0_n)$;对称性只需注意到$t=(t_n)=(-s_n)$满足$g_n-f_n=d_{n+1}'\circ t_n+t_{n-1}\circ d_n$;最后说明传递性,如果$f=(f_n)$和$g=(g_n)$经$s=(s_n)$同伦,$g=(g_n)$和$h=(h_n)$经$t=(t_n)$同伦,那么$s+t=(s_n+t_n)$就满足$f_n-h_n=d_{n+1}'\circ(s_n+t_n)+(s_{n-1}+t_{n-1})\circ d_n,\forall n\in\mathbb{Z}$.
	\item 同伦关系不仅是等价关系,它还保复合,即如果有同伦关系$f^1\sim f^2:(C,d)\to(C',d')$(不妨设经同伦映射$s=(s_n)$),还有同伦关系$g^1\sim g^2:(C',d')\to(C'',d'')$(不妨设经同伦映射$t=(t_n)$),那么在复合有意义的前提下总有$g_1\circ f_1\sim g_2\circ f_2$.
	\begin{proof}事实上取$u_n=g_{n+1}^2\circ s_n+t_n\circ f_n^1$,那么有:
		\begin{align*}
		d_{n+1}''\circ u_n+u_{n-1}\circ d_n&=d_{n+1}''\circ g_{n+1}^2\circ s_n+d_{n+1}''\circ t_n\circ f_n^1+g_n^2\circ s_{n-1}\circ d_n+t_{n-1}\circ f_{n-1}^1\circ d_n\\
		&=g_n^2\circ d_{n+1}'\circ s_n+g_n^2\circ s_{n-1}\circ d_n+d_{n+1}''\circ t_n\circ f_n^1+t_{n-1}\circ d_n'\circ f_n^1\\
		&=g_n^2\circ(f_n^2-f_n^1)+(g_n^2-g_n^1)\circ f_n^1\\
		&=g^2_n\circ f^2_n-g^1_n\circ f^1_n
		\end{align*}
	\end{proof}
    \item 链映射如果和零映射同伦,则称为零伦链映射.$A\to B$的全体零伦链映射$\mathrm{nll}(A,B)$是$\mathrm{Hom}_{\textbf{Comp}(\mathscr{A})}(A,B)$的子群.定义如下范畴,对象类与$\textbf{Comp}(\mathscr{A})$一致,但是每个态射集取如下商群.我们验证了同伦关系是保复合的,于是这的确定义了一个范畴,称为链同伦范畴,记作$\textbf{K}(\mathscr{A})$.这个范畴仍然是加性范畴.
    $$\mathrm{Hom}_{\textbf{K}(\mathscr{A})}(A,B)=\frac{\mathrm{Hom}_{\textbf{Comp}(\mathscr{A})}(A,B)}{\mathrm{nll}(A,B)}$$
    \item 链映射如果和同构链映射同伦,则称为同伦等价链映射.两个链复形之间如果存在同伦等价链映射,也称它们是同伦等价的.设$C$是复形,如果$\mathrm{Hom}(C,C)$的伦型恰有一个,换句话说同伦等价类只有一个,换句话说$C$自身的全体链映射互相同伦,这也等价于零链映射与恒等链映射同伦,即存在$s$使得$1=d\circ s+s\circ d$,此时就称$C$是可缩复形.这实际上是链同伦范畴中的零对象.
    \item 如果两个链映射$f,g:(C,d)\to(C',d')$同伦,那么它们诱导了相同的同调上的态射,即$\forall n\in\mathbb{Z}$恒有$H_n(f)=H_n(g):H_n(C_n)\to H_n(C_n')$.于是同调函子可视为链同伦范畴为源端的函子.
    \begin{proof}
    	
    	事实上任取$z\in Z_n(C)$,有$f_n(z)-g_n(z)=d_{n+1}'\circ s_n(z)+s_{n-1}\circ d_n(z)=d_{n+1}'(s_n(z))\in B_n(C')$,即有$\mathrm{cls}(f_n(z))=\mathrm{cls}(g_n(z))$.
    \end{proof}
    \item 特别的,零伦链映射诱导的同调映射都是零态射;同伦等价诱导的同调映射都是同构.链复形可缩则可推出正合,但反过来未必.
    \item 如果一个链复形同伦等价于有界链复形,上/下有界链复形,就分别称它为同伦有界链复形,同伦上/下有界链复形.这些种类的链复形构成的链同伦范畴的完全子范畴分别称为同伦有界链复形范畴$K^b(\mathscr{A})$,同伦上/下有界链复形范畴$K^-(\mathscr{A})$和$K^+(\mathscr{A})$.
\end{enumerate}

设$\mathscr{A}$是加性范畴,下面证明$\textbf{K}(\mathscr{A})$是一个三角范畴.这里我们把复形取为上指标的.
\begin{enumerate}
	\item 平移算子.定义$\textbf{K}(\mathscr{A})$上的平移算子$[1]$为$X\mapsto X[1]$,满足$X[1]^n=X^{n+1}$,$d[1]^n=d^{n+1}$.
	\item 映射锥.设$u:X\to Y$是链映射,它的映射锥定义为链复形$\mathrm{con}(u)=X[1]\oplus Y$,也即$\mathrm{con}(u)^n=X^{n+1}\oplus Y^n$.微分定义为$X^{n+1}\oplus Y^n\to X^{n+1}\oplus Y^{n+2}$为$(x^{n+1},y^n)\mapsto(-dx^{n+1},u(x^{n+1})+dy^n)$.
	\item 好三角.给定链映射$u$,我们断言在同伦意义下有如下映射序列,它称为$u$诱导的好三角,换句话讲如果把$u$替换为同伦的链映射$v$,那么定义出来的好三角应该在链同伦范畴中同构.
	$$\xymatrix{X\ar[r]^u&Y\ar[r]^{\left(\substack{0\\1}\right)}&\mathrm{con}(u)\ar[r]^{(1,0)}&X[1]}$$
	\begin{proof}
		
		设同伦映射$s:u\simeq v$,构造链映射$f:\mathrm{con}(u)\to\mathrm{con}(v)$为:
		$$f=\left(\begin{array}{cc}1&0\\s&1\end{array}\right):(x^{n+1},y^n)\mapsto(x^{n+1},sx^{n+1}+y^n)$$
		
		验证它和微分映射可交换,所以它的确是链映射.它是可逆的链映射因为它的逆映射就是:
		$$\left(\begin{array}{cc}1&0\\-s&1\end{array}\right)$$
		
		验证有如下交换图表,所以$u,v$定义的好三角是同构的.
		$$\xymatrix{X\ar[r]^u\ar[d]_{1_X}&Y\ar[r]^{\left(\substack{0\\1}\right)}\ar[d]_{1_Y}&\mathrm{con}(u)\ar[r]^{(0,1)}\ar[d]_f&X[1]\ar[d]_{1_{X[1]}}\\X\ar[r]_v&Y\ar[r]_{\left(\substack{0\\1}\right)}&\mathrm{con}(v)\ar[r]_{(0,1)}&X[1]}$$
	\end{proof}
    \item 定义$\textbf{K}(\mathscr{A})$中一个六元序列$(X,Y,Z,u,v,w)$是好三角,如果存在链映射$u:A\to B$使得这个三角同构于$A\to B\to\mathrm{con}(f)\to A[1]$.我们断言这满足三角范畴的四条公理.
    \begin{enumerate}
    	\item $(\mathrm{TR}1)$.这条公理包含三个条件,前两条要求和好三角同构的三角仍然是好三角,每个链映射$f:A\to B$都可以延拓为一个好三角$A\to B\to C\to A[1]$.这两条是平凡的.第三条要求$$\xymatrix{X\ar[r]^{1_X}&X\ar[r]^0&0\ar[r]&X[1]}$$是好三角:
    	\begin{proof}
    		
    		归结为证明有如下同伦交换图表,两边小方格的交换性在$\textbf{Comp}(\mathscr{A})$中已经成立.中间的小方格交换是在链同伦范畴中成立的.而这件事归结为证明$\mathrm{con}(1_X)$是零伦链映射.构造$s:X^{n+1}\oplus X^n\to X^n\oplus X^{n-1}$为$(x^{n+1},x^n)\mapsto(x^n,0)$,那么有$1_{\mathrm{con}(1_X)}=sd+ds$,其中$d$是$\mathrm{con}(1_X)$的微分映射.于是$\mathrm{con}(u)$是可缩链复形.
    	\end{proof}
        \item $(\mathrm{TR}2)$.顺时针旋转.设$u:X\to Y$是链映射,我们要证明如下三角是好三角:
        $$\xymatrix{Y\ar[r]^{\left(\substack{0\\1}\right)}&\mathrm{cone}(u)\ar[r]^{(1,0)}&X[1]\ar[r]^{-u[1]}&Y[1]}$$
        \begin{proof}
        	
        	按照定义有$\mathrm{con}\left(\substack{0\\1}\right)=Y[1]\oplus\mathrm{con}(u)=Y[1]\oplus X[1]\oplus Y$.微分映射为:
        	$$\left(\begin{array}{ccc}-\mathrm{d}_Y&&\\&-\mathrm{d}_X&\\1&u&\mathrm{d}_Y\end{array}\right):(y,x,y_1)\mapsto(-\mathrm{d}y,-\mathrm{d}x,y+ux+\mathrm{d}y_1)$$
        	
        	先证明有如下交换图表:
        	$$\xymatrix{Y\ar[r]^{\left(\substack{0\\1}\right)}\ar@{=}[d]&\mathrm{con}(u)\ar@{=}[d]\ar[r]^{(1,0)}&X[1]\ar[d]_{(-u[1],1,0)}\ar[r]^{-u[1]}&Y[1]\ar@{=}[d]\\Y\ar[r]_{\left(\substack{0\\1}\right)}&\mathrm{con}(u)\ar[r]_{\left(\substack{0\\1}\right)}&\mathrm{con}\left(\substack{0\\1}\right)\ar[r]_{(1,0)}&Y[1]}$$
        	
        	这里左侧和右侧的小方格在$\textbf{Comp}(\mathscr{A})$中已经交换.至于中间的小方格,我们要证明的是$\mathrm{con}(u)\to\mathrm{con}\left(\substack{0\\1}\right)$的两个链映射$f:(x,y)\mapsto(-ux,x,0)$和$g:(x,y)\mapsto(0,x,y)$是链同伦的.为此只需取:
        	$$s:\mathrm(con)(u)\mapsto\mathrm{con}\left(\substack{0\\1}\right)$$
        	$$s^n:X^{n+1}\oplus Y^n\to Y^n\oplus X^n\oplus Y^{n-1}$$
        	$$(x,y)\mapsto(y,-\mathrm{d}x,\mathrm{d}y)$$
        	
        	那么就有:
        	\begin{align*}
        		(sd+ds)(x,y)&=sd(x,y)+ds(x,y)\\&=s(-dx,ux+dy)+d(y,-dx,dy)\\&=(ux+dy,0,d(u(x)))+(-dy,0,-udx+y)\\&=(ux,0,y)\\&=g(x,y)-f(x,y)
        	\end{align*}
        
            再证明这里$\varphi=(-u[1],1,0)$是同伦等价.设$\psi:\mathrm{con}\left(\substack{0\\1}\right)\to X[1]$为$(0,1,0)$,那么$\psi\circ\varphi$是$X[1]$上的恒等链映射.另一方面要证明$\varphi\circ\psi:(y,x,y')\mapsto(-ux,x,0)$同伦于$\mathrm{con}\left(\substack{0\\1}\right)$上的恒等链映射.也即证明$\alpha:(y,x,y')\mapsto(y+ux,0,y')$是零伦链映射.为此取$s:Y^{n+1}\oplus X^{n+1}\oplus Y^n\mapsto Y^n\oplus X^n\oplus Y^{n-1}$为$(y,x,y')\mapsto(y',0,0)$.那么有$(sd+ds)(y,x,y')=(y+ux,0,y')$.
        \end{proof}
        \item $(\mathrm{TR}3)$.设有链复形的如下同伦交换图表:
        $$\xymatrix{X\ar[rr]^u\ar[d]_f&&Y\ar[d]^g\\X'\ar[rr]^{u'}&&Y'}$$
        
        那么存在$h:\mathrm{con}(u)=X[1]\oplus Y\to\mathrm{con}(u')=X'[1]\oplus Y'$使得如下同伦图表交换.
        $$\xymatrix{X\ar[r]^u\ar[d]_f&Y\ar[r]^{\left(\substack{0\\1}\right)}\ar[d]_g&\mathrm{con}(u)\ar@{-->}[d]_h\ar[r]^{(1,0)}&X[1]\ar[d]_{f[1]}\\X'\ar[r]_u&Y'\ar[r]_{\left(\substack{0\\1}\right)}&\mathrm{con}(u')\ar[r]_{(1,0)}&X'[1]}$$
        \begin{proof}
        	
        	直接取$h=\left(\begin{array}{cc}f[1]&0\\0&g\end{array}\right)$就使得右侧两个小方格在$\textbf{Comp}(\mathscr{A})$中已经交换.
        \end{proof}
        \item $(\mathrm{TR}4)$.设$u:X\to Y$和$v:Y\to Z$是两个链映射,考虑如下实线交换图表,我们要证明存在虚线链映射使得第三列也是好三角.
        $$\xymatrix{X\ar@{=}[d]\ar[r]^u&Y\ar[r]^{\left(\substack{0\\1}\right)}\ar[d]_v&\mathrm{con}(u)\ar[r]^{(1,0)}\ar@{-->}[d]&X[1]\ar@{=}[d]\\X\ar[r]^{vu}&Z\ar[r]\ar[d]_{\left(\substack{0\\1}\right)}&\mathrm{con}(vu)\ar[r]\ar@{-->}[d]&X[1]\ar[d]^{u[1]}\\&\mathrm{con}(v)\ar@{=}[r]\ar[d]_{(1,0)}&\mathrm{con}(v)\ar[r]^{(1,0)}\ar[d]^{\left(\substack{00\\10}\right)}&\\&Y[1]\ar[r]_{\left(\substack{0\\1}\right)}&\mathrm{con}(u)[1]&}$$
        \begin{proof}
        	
        	第三列只要取如下三角:
        	$$\xymatrix{\mathrm{con}(u)\ar[rr]^{\left(\begin{array}{cc}1_{X[1]}&0\\0&v\end{array}\right)}&&\mathrm{con}(vu)\ar[rr]^{\left(\begin{array}{cc}u[1]&0\\0&1_Z\end{array}\right)}&&\mathrm{con}(v)\ar[rr]^{\left(\begin{array}{cc}0&0\\1&0\end{array}\right)}&&\mathrm{con}(u)[1]\\X[1]\oplus Y&&X[1]\oplus Z&&Y[1]\oplus Z&&X[2]\oplus Y[1]}$$
        	
        	就使得四个以虚线为边的小方格在$\textbf{Comp}(\mathscr{A})$中已经交换.最后这个三角是好三角因为它是如下两个好三角的直和:
        	$$\xymatrix{X[1]\ar[r]&X[1]\ar[r]&0\ar[r]&X[2]}$$
        	$$\xymatrix{Y\ar[r]^v&Z\ar[r]&\mathrm{con}(v)\ar[r]&Y[1]}$$
        \end{proof}
    \end{enumerate}
\end{enumerate}

映射筒$\mathrm{cyl}(u)$.我们把$\textbf{K}(\mathscr{A})$中的三角核与三角余核称为同伦核与同伦余核,那么一个链映射$u:X\to Y$的同伦余核就是映射锥$\xymatrix{Y\ar[r]^{\left(\substack{0\\1}\right)}&\mathrm{con}(u)}$.链映射$u$的同伦核就是$(-1,0):\mathrm{con}(u)[-1]\to X$.这个同伦核的映射锥定义为$u$的映射筒,记作$\mathrm{cyl}(u)$.它也称为同伦余像.
\begin{enumerate}
	\item 按照定义,有$\mathrm{cyl}(u)=X[1]\oplus Y\oplus X$,它的微分映射为:
	$$\left(\begin{array}{ccc}-\mathrm{d}_X&0&0\\u&\mathrm{d}_Y&0\\-1_X&0&\mathrm{d}_X\end{array}\right):X^{n+1}\oplus Y^n\oplus X^n\to X^{n+2}\oplus Y^{n+1}\oplus X^{n+1}$$
	$$(x,y,x')\mapsto(-dx,ix+dy,-x+dx')$$
	\item 设$u:X\to Y$是链映射,那么如下三角是一个好三角:
	$$\xymatrix{X\ar[rr]^{\left(\substack{0\\0\\1}\right)}&&\mathrm{cyl}(u)\ar[rr]^{\left(\substack{100\\010}\right)}&&\mathrm{con}(u)\ar[rr]^{(1,0)}&&X[1]}$$
	
	事实上有如下三角同构,其中$(0,1,u):\mathrm{cyl}(u)\to Y$的同伦逆是$\left(\substack{0\\1\\0}\right):Y\to\mathrm{cyl}(u)$.
	$$\xymatrix{X\ar[rr]^{\left(\substack{0\\0\\1}\right)}\ar@{=}[d]&&\mathrm{cyl}(u)\ar[d]_{(0,1,u)}\ar[rr]^{\left(\substack{100\\010}\right)}&&\mathrm{con}(u)\ar@{=}[d]\ar[rr]^{(1,0)}&&X[1]\ar@{=}[d]\\X\ar[rr]_u&&Y\ar[rr]^{\left(\substack{0\\1}\right)}&&\mathrm{con}(u)\ar[rr]^{(1,0)}&&X[1]}$$
	\begin{proof}
		
		左侧和右侧小方格在$\textbf{Comp}(\mathscr{A})$中已经交换.中间小方格在同伦意义下交换等价于证明$\mathrm{cyl}(u)\to\mathrm{con}(u)$的两个映射$f:(x',y,x)\mapsto(x',y)$和$g:(x',y,x)\mapsto(0,y+ux)$是同伦的.为此取$s:X^{n+1}\oplus Y^n\oplus X^n\to X^n\oplus Y^{n-1}$为$(x',y,x)\mapsto(-x,0)$,那么有$(ds+sd)(x',y,x)=(x',-ux)=f(x',y,x)-g(x',y,x)$.
		
		\qquad
		
		再验证$\alpha=(0,1,u)$的确是同伦逆映射.取$\beta=\left(\substack{0\\1\\0}\right):Y\to\mathrm{cyl}(u)$,那么$\alpha\circ\beta=1_Y$.另一方面$\beta\circ\alpha$是$(x',y,x)\mapsto(0,y+ux,0)$,为证明它同伦于恒等链映射,归结为证明$(x',y,x)\mapsto(-x',ux,-x)$是零伦链映射.为此只要取$s:X^{n+1}\oplus Y^n\oplus X^n\mapsto X^n\oplus Y^{n-1}\oplus X^{n-1}$为$(x',y,x)\mapsto(x,0,0)$,那么有$(ds+sd)(x',y,x)=(-x',ux,-x)$.
	\end{proof}
    \item 推论.设$\xymatrix{0\ar[r]&X\ar[r]^u&Y\ar[r]^v&Z\ar[r]&0}$是阿贝尔范畴$\mathscr{A}$上链复形构成的短正合列,取$$(0,v):\mathrm{con}(u)\to Z$$那么这实际上是拟同构,并且有如下链复形短正合列的同态:
    $$\xymatrix{0\ar[r]&X\ar[r]^{\left(\substack{0\\0\\1}\right)}\ar@{=}[d]&\mathrm{cyl}(u)\ar[r]^{\left(\substack{100\\010}\right)}\ar[d]_{(0,1,u)}&\mathrm{con}(u)\ar[r]\ar[d]_{(0,v)}&0\\0\ar[r]&X\ar[r]_u&Y\ar[r]_v&Z\ar[r]&0}$$
    \begin{proof}
    	
    	复形短正合列之间的同态诱导了长正合列之间的同态.我们解释过这里$(0,1,u)$是同伦等价,所以是拟同构,于是按照五引理说明$(0,v)$也是拟同构.
    \end{proof}
\end{enumerate}

同伦像.设$u:X\to Y$是链映射,我们定义过它的同伦余核就是$\left(\substack{0\\1}\right):Y\to\mathrm{con}(u)$,它的映射锥的平移$\mathrm{con}\left(\substack{0\\1}\right)[-1]$定义为$u$的同伦像,记作$\mathrm{Him}(u)$.
\begin{enumerate}
	\item 设$v:Y\to Z$是链映射,那么有如下同伦好三角:
	$$\xymatrix{\mathrm{con}(v)[-1]\ar[rr]^{\left(\substack{00\\10\\01}\right)}&&\mathrm{Him}(v)\ar[rr]^{(1,0,0)}&&Z\ar[rr]^{\left(\substack{0\\1}\right)}&&\mathrm{con}(v)}$$
	
	并且有如下同伦交换图表,其中$(0,-1,0)$是同伦等价,它的同伦逆是$\left(\substack{u\\-1\\0}\right)$.
	$$\xymatrix{\mathrm{con}(v)[-1]\ar@{=}[d]\ar[rr]^{\left(\substack{00\\10\\01}\right)}&&\mathrm{Him}(v)\ar[d]_{(0,-1,0)}\ar[rr]^{(1,0,0)}&&Z\ar@{=}[d]\ar[rr]^{\left(\substack{0\\1}\right)}&&\mathrm{con}(v)\ar@{=}[d]\\\mathrm{con}(v)[-1]\ar[rr]_{(-1,0)}&&Y\ar[rr]_v&&Z\ar[rr]_{\left(\substack{0\\1}\right)}&&\mathrm{con}(v)}$$
	\item 推论.设$\xymatrix{0\ar[r]&X\ar[r]^u&Y\ar[r]^v&Z\ar[r]&0}$是阿贝尔范畴$\mathscr{A}$上链复形构成的短正合列,取$\left(\substack{-u\\0}\right):X\to\mathrm{con}(v)[-1]$,那么这是拟同构,并且有如下交换图表:
	$$\xymatrix{0\ar[r]&X\ar[r]^u\ar[d]_{\left(\substack{-u\\0}\right)}&Y\ar[d]_{\left(\substack{v\\-1\\0}\right)}\ar[r]^v&Z\ar[r]\ar@{=}[d]&0\\0\ar[r]&\mathrm{con}(v)[-1]\ar[r]_{\left(\substack{00\\10\\01}\right)}&\mathrm{Him}(v)\ar[r]_{(1,0,0)}&Z\ar[r]&0}$$
\end{enumerate}

同伦版本的同调代数基本定理.设$\mathscr{A}$是阿贝尔范畴.
\begin{enumerate}
	\item 设$(X,Y,Z,u,v,w)$是$\textbf{K}(\mathscr{A})$中的好三角,对每个$n$有$\xymatrix{\mathrm{H}^n(X)\ar[r]^{\mathrm{H}^n(u)}&\mathrm{H}^n(Y)\ar[r]^{\mathrm{H}^n(v)}&\mathrm{H}^n(Z)}$是$\mathscr{A}$中的正合列.
	\begin{proof}
		
		我们解释过存在链映射$u'$使得有如下同伦交换图表,其中垂直同态都是同伦等价.
		$$\xymatrix{X\ar[r]^u\ar[d]&Y\ar[r]^v\ar[d]&Z\ar[r]^w\ar[d]&X[1]\\X'\ar[r]_{\left(\substack{0\\0\\1}\right)}&\mathrm{cyl}(u')\ar[r]_{\left(\substack{100\\010}\right)}&\mathrm{con}(u')\ar[r]_{(1,0)}&X[1]}$$
		
		但是这里$\xymatrix{0\ar[r]&X'\ar[r]_{\left(\substack{0\\0\\1}\right)}&\mathrm{cyl}(u')\ar[r]_{\left(\substack{100\\010}\right)}&\mathrm{con}(u')\ar[r]&0}$是链复形的短正合列.所以它诱导了同调群的长正合列.特别的有正合列$\xymatrix{\mathrm{H}^n(X')\ar[r]&\mathrm{H}^n(\mathrm{cyl}(u'))\ar[r]&\mathrm{H}^n(\mathrm{con}(u'))}$.同调可视为$\textbf{K}(\mathscr{A})\to\textbf{Ab}$的加性函子,所以有如下交换图表,但是同伦等价是拟同构,所以如下交换图表中下行是正合列得到上行是正合列,这就得证.
		$$\xymatrix{\mathrm{H}^n(X)\ar[r]\ar[d]&\mathrm{H}^n(Y)\ar[r]\ar[d]&\mathrm{H}^n(Z)\ar[d]\\\mathrm{H}^n(X')\ar[r]&\mathrm{H}^n(\mathrm{cyl}(u'))\ar[r]&\mathrm{H}^n(\mathrm{con}(u'))}$$
	\end{proof}
    \item 设$(X,Y,Z,u,v,w)$是$\textbf{K}(\mathscr{A})$中的好三角,那么有如下上同调群的长正合列,特别的这里连接映射是$w$诱导的.
    $$\xymatrix{\cdots\ar[r]&\mathrm{H}^{n-1}(Z)\ar[r]&\mathrm{H}^n(X)\ar[r]&\mathrm{H}^n(Y)\ar[r]&\mathrm{H}^n(Z)\ar[r]&\mathrm{H}^{n+1}(X)\ar[r]&\cdots}$$
    \item 设有如下好三角的三角同态:
    $$\xymatrix{X\ar[r]^u\ar[d]_f&Y\ar[r]^v\ar[d]_g&Z\ar[r]^w\ar[d]_h&X[1]\ar[d]_{f[1]}\\X'\ar[r]^{u'}&Y\ar[r]^{v'}&Z\ar[r]^{w'}&X'[1]}$$
    
    那么有$\mathscr{A}$中如下长正合列之间的同态.
    $$\xymatrix{\cdots\ar[r]&\mathrm{H}^n(X)\ar[r]^{\mathrm{H}^n(u)}\ar[d]_{\mathrm{H}^n(f)}&\mathrm{H}^n(Y)\ar[r]^{\mathrm{H}^n(v)}\ar[d]_{\mathrm{H}^n(g)}&\mathrm{H}^n(Z)\ar[r]^{\mathrm{H}^n(w)}\ar[d]_{\mathrm{H}^n(h)}&\mathrm{H}^{n+1}(X)\ar[r]\ar[d]_{\mathrm{H}^{n+1}(f)}&\cdots\\\cdots\ar[r]&\mathrm{H}^n(X')\ar[r]_{\mathrm{H}^n(u')}&\mathrm{H}^n(Y')\ar[r]_{\mathrm{H}^n(v')}&\mathrm{H}^n(Z')\ar[r]_{\mathrm{H}^n(w')}&\mathrm{H}^{n+1}(X')\ar[r]&\cdots}$$
    \item 设$(X,Y,Z,u,v,w)$是$\textbf{K}(\mathscr{A})$中的好三角,那么$u$是拟同构当且仅当$Z$是零调复形,也等价于$\mathrm{con}(u)$是零调复形.
    \begin{proof}
    	
    	如果$Z$是零调复形,考虑这个好三角诱导的长正合列,就得到$u$是一个拟同构.反过来如果$u$是拟同构,为了证明$Z$是零调复形,只需验证$\mathscr{A}$中的如下正合列如果$X\to Y$和$A\to B$是同构,那么$C$是零对象:同构$X\to Y$的像是它本身,于是$Y\to Z$的核是同构,于是$Y\to Z$是零态射,同理有$Z\to A$是零态射.但是零态射的像是零态射,零态射的核是同构,一个零态射是同构当且仅当它的源端核终端都是零对象,所以$Z$是零对象.
    	$$\xymatrix{X\ar[r]&Y\ar[r]&Z\ar[r]&A\ar[r]&B}$$
    \end{proof}
    \item 设$\mathscr{A}$是阿贝尔范畴,那么$\textbf{K}^b(\mathscr{A})$,$\textbf{K}^{-,b}(\mathscr{A})$,$\textbf{K}^{+,b}(\mathscr{A})$都是$\textbf{K}(\mathscr{A})$的三角子范畴.
\end{enumerate}
\newpage
\subsection{导出函子}

比较定理.
\begin{enumerate}
	\item 给定阿贝尔范畴$\mathscr{A}$,任取态射$f:A\to A'$,考虑如下图表,其中每个$P_n$和$P_n'$是投射对象,并且第一行是正合列(即为$A'$的投射预解),那么$f$可延拓为链复形$\{P_n\}$和$\{P_n'\}$之间的链映射.
	$$\xymatrix{
		\cdots\ar[r]&P_2\ar[r]^{d_2}\ar[d]^{f_2}&P_1\ar[r]^{d_1}\ar[d]^{f_1}&P_0\ar[r]^{\varepsilon}\ar[d]^{f_0}&A\ar[r]\ar[d]^{f}&0\\
		\cdots\ar[r]&P_2'\ar[r]_{d_2'}&P_1'\ar[r]_{d_1'}&P_0'\ar[r]_{\varepsilon'}&A'\ar[r]&0}$$
	\begin{proof}
		
		归纳的构造态射$f_n$.对$n=0$,按照$\varepsilon':P_0'\to A'$是满态射,而$P_0$是投射对象,于是存在态射$f_0:P_0\to P_0'$提升了态射$f\circ\varepsilon:P_0\to A'$.也即构造了$f_0$满足图表交换.
		$$\xymatrix{&P_0\ar[dl]_{f_0}\ar[d]^{f\circ\varepsilon}&\\P_0'\ar[r]_{\varepsilon'}&A'\ar[r]&0}$$
		
		现在假设已经构造了$f_n:P_n\to P_n'$,考虑如下交换图:
		$$\xymatrix{P_{n+1}\ar[r]^{d_{n+1}}&P_n\ar[r]^{d_n}\ar[d]_{f_n}&P_{n-1}\ar[d]_{f_{n-1}}\\P_{n+1}'\ar[r]_{d_{n+1}'}&P_n'\ar[r]_{d_n'}&P_{n-1}'}$$
		
		按照$d_n'\circ f_n\circ d_{n+1}=f_{n-1}\circ d_n\circ d_{n+1}=0$,得到$\mathrm{im}f_n\circ d_{n+1}\subset\ker d_n'=\mathrm{im}d_{n+1}'$.于是可考虑如下图表,其中$d_{n+1}'$为$P_{n+1}'\to\mathrm{im}d_{n+1}'$的满态射,于是$f_n\circ d_{n+1}:P_{n+1}\to\mathrm{im}d_{n+1}'$可提升为态射$f_{n+1}$,也即$f_{n+1}$满足图表交换.
		$$\xymatrix{&P_{n+1}\ar[dl]_{f_{n+1}}\ar[d]^{f_n\circ d_{n+1}}&\\P_{n+1}'\ar[r]_{d_{n+1}'}&\mathrm{im}d_{n+1}'\ar[r]&0}$$
	\end{proof}
	\item 上一条中延拓了$f$的任意两个链映射是同伦的.称这个同伦意义下唯一的链映射为态射$f:A\to A'$的提升.
	\begin{proof}
		
		设$h=(f,h_0,h_1,\cdots)$是另一个提升了$f$的链映射.需要构造态射列$s_n:P_n\to P_{n+1}',n\ge-1$满足$h_n-f_n=d_{n+1}'\circ s_n+s_{n-1}\circ d_n$.初始步骤是简单的,只要取$s_{-2}=s_{-1}=0$,就得到$f-f=0$满足上述等式.对于归纳步骤,假设已经构造了$s_{-1},s_0,\cdots,s_n$,先证明$\mathrm{im}(h_{n+1}-f_{n+1})\subset\mathrm{im}d_{n+2}'$,也即证明$d_{n+1}'(h_{n+1}-f_{n+1}-s_nd_{n+1})=0$,这是因为:
		\begin{align*}
		d_{n+1}'(h_{n+1}-f_{n+1}-s_nd_{n+1})&=d_{n+1}'(h_{n+1}-f_{n+1})-d_{n+1}'s_nd_{n+1}\\
		&=d_{n+1}'(h_{n+1}-f_{n+1})-(h_n-f_n-s_{n-1}d_n)d_{n+1}\\
		&=d_{n+1}'(h_{n+1}-f_{n+1})-(h_n-f_n)d_{n+1}=0
		\end{align*}
		
		现在考虑如下图表,其中$d_{n+2}'$是$P_{n+2}'\to\mathrm{im}d_{n+2}'$的满态射,于是$h_{n+1}-f_{n+1}-s_nd_{n+1}:P_{n+1}\to\mathrm{im}d_{n+2}'$可提升为态射$s_{n+1}:P_{n+1}\to P_{n+2}'$,这就构造$s_{n+1}$满足了$h_{n+1}-f_{n+1}=s_nd_{n+1}+d_{n+2}'s_{n+1}$.
	\end{proof}
	\item 特别的,任取态射$f:A\to A'$,任取$A$和$A'$分别的投射预解$P_A$和$P'_{A'}$,那么$f$可延拓为两个预解之间的链映射,并且这样延拓的链映射总是互相同伦的.
	\item 对偶的,给定态射$g:A'\to A$,考虑如下图表,其中每个$E^n$和$X^n$是内射对象,并且第二行是正合列,那么存在链态射延拓了$g$,并且这样的延拓在同伦意义下唯一.
	$$\xymatrix{0\ar[r]&A\ar[r]&E^0\ar[r]&E^1\ar[r]&E^2\ar[r]&\\0\ar[r]&A'\ar[r]\ar[u]^g&X^0\ar[r]\ar[u]&X^1\ar[r]\ar[u]&X^2\ar[r]\ar[u]&}$$
	\item 特别的,给定态射$g:A'\to A$,任取$A'$和$A$的内射预解,那么$g$在同伦意义下可唯一延拓为两个内射预解之间的链映射.
\end{enumerate}
\subsubsection{左导出函子}

左导出函子.设$\mathscr{A}$和$\mathscr{C}$都是阿贝尔范畴,其中$\mathscr{A}$上具有足够多的投射对象.取加性共变函子$T:\mathscr{A}\to\mathscr{C}$,它的左导出函子是一列函子$L_nT:\mathscr{A}\to\mathscr{C},n\ge0$.
\begin{enumerate}
	\item 对$\mathscr{A}$上每个对象$A$,取$P_A:\xymatrix{\ar[r]&P_2\ar[r]^{d_2}&P_1\ar[r]^{d_1}&P_0\ar[r]^{\varepsilon}&A\ar[r]&0}$是$A$的一个固定的投射预解.称$\xymatrix{\ar[r]&P_2\ar[r]^{d_2}&P_1\ar[r]^{d_1}&P_0\ar[r]&0}$为它对应的简化投射预解.设$$TP_A:\xymatrix{\ar[r]&TP_2\ar[r]^{Td_2}&TP_1\ar[r]^{Td_1}&TP_0\ar[r]&0}$$是共变加性函子$T$作用在简化投射预解上得到的$\mathscr{C}$中的一个复形.取同调,定义$(L_nT)(A)=H_n(TP)=\ker(Td_n)/\mathrm{im}Td_{n+1}$.
	\item 接下来定义$L_nT$作用在态射上的定义.任取态射$f:A\to A'$,按照比较定理,存在同伦意义下唯一的链映射$f_*:P_A\to P_{A'}$延拓了$f$.就定义$L_nT(f)=H_n(Tf_*)$,换句话讲,对每个$z+\mathrm{im}Td_{n+1}\in H_n(TP_A)$,定义它在$L_nT(f)$下的像为$T(f_n)(z)+\mathrm{im}Td_{n+1}'$.
	\item 首要需要说明的是,$L_nT(A)$的定义不依赖于$A$的投射预解的选取.换句话讲,如果对每个$\mathscr{A}$中的对象$A$选定两个投射预解$P_A$和$P'_A$,按照投射预解族$\{P_A\}$定义的函子列记作$L_nT$,按照投射预解族$\{P'_A\}$定义的函子列记作$L'_nT$,那么对每个$n\ge0$都有自然同构$L_nT\cong L'_nT$.
	\begin{proof}
		
		设$P_A=(P_n)$和$P'_A=(P_n')$,它们是$A$的投射预解,那么有如下交换图,其中两行均为正合列:$$\xymatrix{
			\ar[r]&P_2\ar[r]&P_1\ar[r]&P_0\ar[r]&A\ar[r]\ar[d]_{1_A}&0\\
			\ar[r]&P_2'\ar[r]&P_1'\ar[r]&P_0'\ar[r]&A\ar[r]&0}$$
		
		按照比较定理.存在链映射$l:P_A\to P_A'$提升了$1_A$.将共变加性函子$T$作用在这个交换图上,得到了$TP$和$TP'$之间的链映射$Tl$提升了$1_{TA}$.链映射$Tl$诱导了同调之间的态射$\tau_{n,A}:L_nT(A)\to L'_nT(A)$.我们断言固定$n$时$\tau_{n,A}$当$A$变动时是一个自然同构.
		
		为说明$\tau_{n,A}$是同构,按照同伦链映射诱导了相同的同调映射,于是只需说明链映射$Tl$是一个同伦等价,为此考虑如下交换图,即把两个$A$的投射预解$P$和$P'$交换位置,按照比较定理仍然存在链映射$k:P'\to P$提升了$1_A$.于是比较定理中延拓在同伦意义下的唯一性得到$kl$同伦于$1_{P}$,$lk$同伦于$1_{P'}$.再将加性共变函子$T$作用其上,得到$Tl$和$Tk$是在同伦意义下互为逆映射的链映射,于是$Tl$诱导的同调映射总是同构,也即$\tau_{n,A},\forall n,A$总是同构.
		$$\xymatrix{
			\ar[r]&P_2'\ar[r]&P_1'\ar[r]&P_0'\ar[r]&A\ar[r]\ar[d]_{1_A}&0\\
			\ar[r]&P_2\ar[r]&P_1\ar[r]&P_0\ar[r]&A\ar[r]&0}$$
		
		为说明$\tau_{n,A}$在固定$n$时是自然同构,需要验证的是对每个态射$f:A\to B$,总有如下交换图:
		$$\xymatrix{
			L_nT(A)\ar[d]_{L_nT(f)}\ar[r]^{\tau_A}&L_n'T(A)\ar[d]^{L_n'T(f)}\\
			L_nT(B)\ar[r]_{\tau_B}&L_n'T(B)}$$
		
		为此设$L_nT$定义中$A$的投射预解选取的是$(P_n)$,$B$的投射预解选取的是$(Q_n)$;在$L'_nT$定义中$A$的投射预解选取的是$(P'_n)$,$B$的投射预解选取的是$(Q_n')$.先考虑顺时针的两个映射的复合.考虑如下交换图,那么比较定理提供了提升$f1_A=f$的$(P_n)\to (Q_n')$的链映射,它诱导的同调之间的映射即顺时针两个映射的复合:
		$$\xymatrix{
			\ar[r]&P_1\ar[r]&P_0\ar[r]&A\ar[r]\ar[d]^{1_A}&0\\
			\ar[r]&P_1'\ar[r]&P_0'\ar[r]&A\ar[r]\ar[d]^{f}&0\\
			\ar[r]&Q_1'\ar[r]&Q_0'\ar[r]&B\ar[r]&0}$$
		
		同理考虑如下交换图,按照比较定理得到的提升$1_Bf=f$的$(P_n)\to(Q_n')$的链映射,所诱导的同调映射就是逆时针两个映射的复合:
		$$\xymatrix{\ar[r]&P_1\ar[r]&P_0\ar[r]&A\ar[r]\ar[d]^{f}&0\\\ar[r]&Q_1\ar[r]&Q_0\ar[r]&B\ar[r]\ar[d]^{1_B}&0\\\ar[r]&Q_1'\ar[r]&Q_0'\ar[r]&B\ar[r]&0}$$
		
		比较定理说明了$(P_n)\to(Q_n')$的提升了$f:A\to B$的链映射在同伦意义下是唯一的,另外同伦的链映射诱导了相同的同调映射,这就说明了顺时针两个映射的复合就等于逆时针两个映射的复合.这就证明了$\tau_A$构成了自然同构.
	\end{proof}
    \item 于是阿贝尔范畴之间的加性共变函子$T:\mathscr{A}\to\mathscr{C}$,其中$\mathscr{A}$上具有足够多的投射对象,那么就构造出一列函子$L_nT:\mathscr{A}\to\mathscr{C},n\ge0$.并且$L_nT,n\ge0$总是加性函子.此即$T$的左导出函子列.这里左的含义是,构造的时候运用的是投射预解,此时零对象习惯写在了右侧,于是仅有左侧具有信息,于是这样构造出来的函子称为"左"导出函子.
\end{enumerate}

左导出函子的一些基本性质.
\begin{enumerate}
	\item Horseshoe引理.在阿贝尔范畴上给定短正合列$0\to A'\to A\to A''\to0$,任取$A''$和$A'$的投射预解分别为$P''=(P_n'')$和$P'=(P_n')$.那么两个预解的直和构成了$A$的一个投射预解,并且这三个预解构成了一个提升了初始短正合列的复形的短正合列.
	$$\xymatrix{&\ar[d]&&\ar[d]&\\&P_1'\ar[d]&&P_1''\ar[d]&\\&P_0'\ar[d]_{\varepsilon'}&&P_0''\ar[d]^{\varepsilon''}&\\0\ar[r]&A'\ar[r]_i&A\ar[r]_q&A''\ar[r]&0}$$
	\begin{proof}
		
		对$n$归纳构造.先是初始步骤,取$P_0=P_0'\oplus P_0''$,我们来证明存在下图表的交换图,使得所有行所有列都是短正合列.$P_0$是投射对象的二元积,于是它是投射对象.构造$i_0:P_0'\to P_0$和$q_0:P_0\to P_0''$是阿贝尔范畴中积和余积定义中的典范映射,这使得中间第二行是一个短正合列.接下来按照$P_0''$是投射对象,于是$\varepsilon''$可以提升为态射$\sigma:P_0''\to A$.现在定义$\varepsilon:P_0\to A$为$(x',x'')\mapsto i\varepsilon'(x')+\sigma(x'')$.这是一个态射并且满足后两行的图表交换.并且短五引理得到$\varepsilon$是满态射.最后取$K_0'$,$K_0$和$K_0''$分别是$\varepsilon'$,$\varepsilon$和$\varepsilon''$的核,于是它们可以拼接为下图形式的交换图,满足全部行与列都是短正合列.
		$$\xymatrix{&0\ar[d]&0\ar[d]&0\ar[d]&\\0\ar[r]&K_0'\ar[r]\ar[d]&K_0\ar[r]\ar[d]&K_0''\ar[r]\ar[d]&0\\0\ar[r]&P_0'\ar[r]^{i_0}\ar[d]_{\varepsilon'}&P_0\ar[r]^{q_0}\ar[d]_{\varepsilon}&P_0''\ar[dl]^{\sigma}\ar[r]\ar[d]^{\varepsilon''}&0\\0\ar[r]&A'\ar[r]\ar[d]&A\ar[r]\ar[d]&A''\ar[d]\ar[r]&0\\&0&0&0&}$$
		
		现在假设已经构造了$P_0,P_1,\cdots,P_n$使得从下数前$n$行构成的图表交换,并且$K_n'=\ker d_n'$,$K_n=\ker d_n$和$K_n''=\ker d_n''$,模仿初始步骤的证明,取$P_{n+1}=P_{n+1}'\oplus P_{n+1}''$会得到如下交换图,满足所有行与所有列都是短正合列,这就构造出了$A$的投射预解以及预解之间的提升了$i$和$q$的链映射.
		$$\xymatrix{&0\ar[d]&0\ar[d]&0\ar[d]&\\0\ar[r]&K_{n+1}'\ar[r]\ar[d]&K_{n+1}\ar[r]\ar[d]&K_{n+1}''\ar[r]\ar[d]&0\\0\ar[r]&P_{n+1}'\ar[r]\ar[d]&P_{n+1}\ar[r]\ar[d]&P_{n+1}''\ar[r]\ar[d]&0\\0\ar[r]&K_n'\ar[r]\ar[d]&K_n\ar[r]\ar[d]&K_n''\ar[d]\ar[r]&0\\&0&0&0&}$$
	\end{proof}
    \item 投射对象.给定投射对象$P$,那么$\to0\to0\to P\to P\to0$是它的投射预解,于是任取加性函子$T$,它作用在简化投射预解上为$\to0\to0\to TP\to0$,于是有$L_0T(P)=TP$和$L_nT(P)=0,\forall n\ge1$.
    \item 左导出函子的长正合列.取阿贝尔范畴中的短正合列$0\to A'\to A\to A''\to0$.任取$A'$和$A''$的投射预解,按照Horseshoe引理,取对应投射对象的直和就得到$A$的一个投射预解,并且三个投射预解构成了一个复形的短正合列.按照加性函子保分离短正合列,说明将阿贝尔范畴之间的加性函子$T:\mathscr{A}\to\mathscr{C}$作用在这个复形的短正合列上仍然得到复形的短正合列.按照复形的短正合列诱导长正合列,就得到左导出函子的长正合列:
    $$\xymatrix{\cdots\ar[r]&L_nTA'\ar[r]^{L_nTi}&L_nTA\ar[r]^{L_nTp}&L_nTA''\ar[r]^{\delta_n}&L_{n-1}TA'\ar[r]&\cdots}$$
    $$\xymatrix{\ar[r]&L_0T(A')\ar[r]&L_0T(A)\ar[r]&L_0T(A'')\ar[r]&0}$$
    \item 第零个左导出函子$L_0T$与右正合性.首先按照定义,$L_0T(A)$事实上就是$\mathrm{coker}Td_1=TP_0/\mathrm{im}Td_1$.考虑上述诱导的长正合列的终端,说明了即便$T$本身未必是右正合函子,但是$L_0T$总会是一个右正合函子.反过来如果$T$本身是右正合函子,那么有自然同构$T\cong L_0T$.
    \begin{proof}
    	
    	任取对象$A$,任取它的投射预解$P:\xymatrix{\ar[r]&P_1\ar[r]^{d_1}&P_0\ar[r]^{\varepsilon}&A\ar[r]&0}$.按照$T$的右正合性,得到正合列$\xymatrix{\ar[r]&TP_1\ar[r]^{Td_1}&TP_0\ar[r]^{T\varepsilon}&TA\ar[r]&0}$,于是$L_0T(A)=\mathrm{coker}Td_1=TP_0/\mathrm{im}Td_1=TP_0/\ker T\varepsilon$,按照同构定理,存在自然同构$TP_0/\ker T\varepsilon\cong TA$,记作$\sigma_A$,于是$L_0T(A)\cong TA$.最后验证$\sigma_A$是自然同构.
    \end{proof}
    \item 零调预解.设$F:\mathscr{A}\to\mathscr{B}$是右正合加性函子,设$\mathscr{A}$具有足够多的投射对象,称对象$P$是关于$F$零调的,如果$L_iF(P)=0,\forall i\ge1$.设$J$是对象$A$的零调预解,即由零调对象构成的预解,设它的简化复形为$J^*$,那么有$L_iF(A)\cong H_i(FJ^*)$.
\end{enumerate}

公理化描述.
\begin{enumerate}
	\item 给定阿贝尔范畴之间的两个加性函子列$(T_n)_{n\ge0}$和$(T_n')_{n\ge0}$:$\mathscr{A}\to\mathscr{B}$,假设$\mathscr{A}$具有足够多的投射对象.如果它们满足如下三条,那么就有每个$T_n$都自然同构于$T_n'$.
	\begin{itemize}
		\item 对每个短正合列$0\to A\to B\to C\to0$,分别存在这两个函子列的长正合列,并且连接态射都是自然的.
		\item $T_0$自然同构于$T_0'$.
		\item 对每个投射对象和每个$n\ge1$都有$T_n(P)=T_n'(P)=0$.
	\end{itemize}
    \item 给定阿贝尔范畴之间的两个加性函子列$(T_n)_{n\ge0}$和$(T_n')_{n\ge0}$:$\mathscr{A}\to\mathscr{B}$,假设存在$\mathscr{A}$的一类对象$\mathscr{X}$,满足对每个$\mathscr{A}$中的对象$A$,存在$\mathscr{X}$中的对象$F$到$A$的满态射.如果它们满足如下三个条件,那么就有每个$T_n$自然同构于$T_n',\forall n\ge0$.
    \begin{itemize}
    	\item 对每个$\mathscr{A}$中的短正合列$0\to A\to B\to C\to0$.那么分别存在这两列函子的长正合列,并且连接态射是自然的.
    	\item $T_0$自然同构于$T_0'$.
    	\item 对每个对象$P\in\mathscr{X}$和每个$n\ge1$都有$T_n(P)=T_n'(P)=0$.
    \end{itemize}
    \item 上述几种描述中的核心内容都是第一条,即短正合列诱导长正合列.Grothendieck把这一性质抽象为一个定义.阿贝尔范畴之间的一列加性函子$\{T_n\}$称为同调$\delta$函子,如果每个短正合列可诱导出连接映射自然的长正合列.有的文献会要求$T_0$是右正合的.
    \item 两列同调$\delta$函子之间的态射$(T_n)_{n\ge0}\to(H_n)_{n\ge0}$是指一列自然变换$\tau_n:T_n\to H_n$,满足对任意短正合列$0\to A\to B\to C\to0$诱导的两个长正合列,存在如下交换图,这个定义蕴含了要求同调$\delta$函子之间的态射要和连接同态可交换.
    $$\xymatrix{T_n(C)\ar[r]^{\delta}\ar[d]_{\tau_{n,C}}&T_{n-1}(A)\ar[d]^{\tau_{n-1,A}}\\H_n(C)\ar[r]_{\delta}&H_{n-1}(A)}$$
    \item 函子的同调延拓.给定阿贝尔范畴的加性函子$F:\mathscr{A}\to\mathscr{B}$,称一个同调$\delta$函子$\{T_n:\mathscr{A}\to\mathscr{B}\}$是$F$的同调延拓,如果存在自然同构$F\cong T_0$.例如$\{\mathrm{Tor}_n^R(-,M)\}$是$-\otimes_RM$的同调延拓.关于同调延拓有如下结论:
    \begin{enumerate}
    	\item 给定两个从$\mathscr{A}\to\textbf{Ab}$的同调$\delta$函子$\{T_n\},\{H_n\}$,满足对任意投射对象$P$有$H_n(P)=0,\forall n\ge1$.并且存在自然变换$\tau_0:T_0\to H_0$.那么存在唯一的同调$\delta$函子之间的态射$\tau:(T_n)\to(H_n)$.并且如果$\tau_0$是自然同构,那么$\tau_n,n\ge1$是自然同构.
    	\item 特别的,给定右正合加性函子$F:\mathscr{A}\to\textbf{Ab}$.那么存在唯一的$F$的同调延拓满足$H_n(P)=0,\forall n\ge1,\forall P$投射,此即$F$的左导出函子列.
    	\item 特别的,如果取$F$是张量函子,这一条即$\mathrm{Tor}$函子的公理描述(见下节):一列从右$R$模范畴到$\textbf{Ab}$的函子$\{T_n,n\ge0\}$,如果满足对任意右$R$模的短正合列可诱导由自然的连接同态相连的长正合列;$T_0$自然同构于某个$-\otimes_RM$,其中$M$是一个左$R$模;对任意投射右$R$模$P$总有$T_n(P)=0,\forall n\ge1$.那么对每个$n\ge0$有自然同构$T_n\cong\mathrm{Tor}_n^R(-,M)$.
    \end{enumerate}
    \begin{proof}
    	
    	只需证明第一个结论,思路是归纳构造自然变换列$\{\tau_n:T_n\to H_n,n\ge0\}$.其中初始步骤$\tau_0$的存在性由条件保证.接下来考虑归纳步骤,假设已经构造了自然变换$\tau_{n-1}:T_{n-1}\to H_{n-1}$.现在任取对象$C$,取投射对象$P$和它到$C$的满态射,这就得到一个短正合列$0\to K\to P\to C\to0$.按照同调$\delta$函子的定义以及归纳假设,得到如下交换图:
    	$$\xymatrix{&T_n(C)\ar[r]\ar[d]&T_{n-1}(K)\ar[r]\ar[d]&T_{n-1}(P)\ar[d]\\H_n(P)\ar[r]&H_n(C)\ar[r]&H_{n-1}(K)\ar[r]&H_{n-1}(P)}$$
    	
    	这两行都是正合列,按照条件还有$H_n(P)=H_{n-1}(P)=0$,于是存在同构$H_n(C)\cong H_{n-1}(K)$,那么可构造$\tau_{n,C}:T_n(C)\to H_n(C)$为图表中顺时针旋转的三个态射的复合,其中第三个态射要取逆,那么$\tau_{n,C}$使得图表交换.下面验证$\tau_n=(\tau_{n,C})$是自然变换.为此需要先验证这个定义良性,即定义出的$\tau_{n,C}$不依赖于$C$的投射表现的选取.
    	
    	引理.给定短正合列$0\to A'\to A\to A''\to0$,给定态射$f:C\to A''$,那么诱导了如下交换图:
    	$$\xymatrix{T_n(C)\ar[r]^{T_n(f)}\ar[d]_{\tau_{n,C}}&T_n(A'')\ar[r]^{\partial}&T_{n-1}(A')\ar[d]^{\tau_{n-1,A'}}\\H_n(C)\ar[r]_{H_n(f)}&H_n(A'')\ar[r]_{\partial}&H_{n-1}(A')}$$
    	
    	引理的证明.从$P$是投射对象结合短五引理,得到两个短正合列之间的态射:
    	$$\xymatrix{0\ar[r]&K\ar[r]\ar[d]_g&P\ar[r]\ar[d]&C\ar[r]\ar[d]^f&0\\0\ar[r]&A'\ar[r]&A\ar[r]&A''\ar[r]&0}$$
    	
    	考虑如下图表,首先按照$\tau_{n,C}$的定义,上方的梯形是交换的;按照$\{H_n\}$是一个同调$\delta$函子,得到中间的小方形是交换的;再按照归纳假设$\tau_{n-1}$是自然变换,得到右侧的梯形是交换的;最后按照$\{T_n\}$是同调$\delta$函子,得到大方形是交换的.于是图表交换性说明引理成立.
    	$$\xymatrix{T_n(C)\ar[rrr]^{\partial}\ar[ddd]_{T_n(f)}\ar[dr]_{\tau_{n,C}}&&&T_{n-1}(K)\ar[ddd]^{\tau_{n-1,K}}\ar[dl]^{\tau_{n-1,K}}\\&H_n(C)\ar[r]^{\partial}\ar[d]_{H_n(f)}&H_{n-1}(K)\ar[d]^{H_{n-1}(g)}&\\&H_n(A'')\ar[r]_{\partial}&H_{n-1}(A')&\\T_n(A'')\ar[rrr]_{\partial}&&&T_{n-1}(A')\ar[ul]_{\tau_{n-1,A'}}}$$
    	
    	现在说明$\tau_{n,C}$的定义不依赖于投射表现的选取.假设$0\to K'\to P'\to C\to0$是另一个短正合列,满足$P'$是投射对象.把引理中的短正合列取为这个新的短正合列,再取$f=1_C$,于是结论中的交换图里有$T_n(f)=1$,$H_n(f)=1$,于是由新的短正合列所定义的$\tau_{n-1,C}':T_n(A'')\to H_n(A'')$需要使得引理结论中的图表交换,这就得到$\tau_{n,C}=\tau_{n,C}'$.
    	
    	接下来证明$\tau_n$是自然变换.任取$\mathscr{A}$中的对象$X_1,X_2$,取投射表现$0\to K_i\to P_i\to X_i\to0,i=1,2$.也即$P_i$都是投射对象.任取态射$f:X_1\to X_2$,于是可定义$\tau_{n,X_i}:T_n(X_i)\to H_n(X_i)$.下面在引理中取短正合列$0\to K_2\to P_2\to X_2\to0$和态射$f:X_1\to X_2$,于是得到如下图表的大矩体交换,另外按照$\tau_{n,X_2}$的构造知右侧小方形是交换的.最后说明左侧的小方形是交换的,为此注意到按照$H_n(P_2)=0$,导致下行的连接同态$\partial$是单射,于是从大矩体的交换性得到左侧小方形是交换的,这就证明了$\tau_n=\{\tau_{n,C}\}$是自然变换,并且和连接同态交换.
    	$$\xymatrix{T_n(X_1)\ar[r]^{T_n(f)}\ar[d]_{\tau_{n,X_1}}&T_n(X_2)\ar[r]^{\partial}\ar[d]^{\tau_{n,X_2}}&T_{n-1}(K_2)\ar[d]^{\tau_{n-1,K_2}}\\H_n(X_1)\ar[r]_{H_n(f)}&H_n(X_2)\ar[r]_{\partial}&H_{n-1}(K_2)}$$
    	
    	最后注意到如果$\tau_0$是自然同构,那么在归纳步骤中从$\tau_{n-1,K}$是同构得到所构造的$\tau_{n,C}$是同构,因而构造的$\tau_n=\{\tau_{n,C}\}$是自然同构.
    \end{proof}
\end{enumerate}
\subsubsection{共变右导出函子}

给定阿贝尔范畴之间的加性共变函子$T:\mathscr{A}\to\mathscr{B}$,其中$\mathscr{A}$具有足够多的内射对象.$T$的右导出函子是一列函子$R^nT:\mathscr{A}\to\mathscr{B},n\ge0$.
\begin{enumerate}
	\item 对$\mathscr{A}$上每个对象$A$,取$\xymatrix{0\ar[r]&B\ar[r]^{\eta}&E^0\ar[r]^{d^0}&E^1\ar[r]^{d^1}&E^2\ar[r]&}$是$B$的一个固定的内射预解.称$\xymatrix{0\ar[r]&E^0\ar[r]^{d^0}&E^1\ar[r]^{d^1}&E^2\ar[r]&}$为它对应的简化内射预解.设$$TE_B:\xymatrix{0\ar[r]&TE^0\ar[r]^{Td^0}&TE^1\ar[r]^{Td^1}&TE^2\ar[r]&}$$是共变加性函子$T$作用在简化内射预解上得到的$\mathscr{B}$中的一个复形.取同调,定义$(R^nT)(B)=H^n(TE_B)=\ker(Td^n)/\mathrm{im}Td^{n-1}$.
	\item 接下来定义$R^nT$作用在态射上的定义.任取态射$f:B\to B'$,按照比较定理,存在同伦意义下唯一的链映射$f^*:E_B\to E_{B'}$延拓了$f$.就定义$R^nT(f)=H^n(Tf^*)$,换句话讲,对每个$z+\mathrm{im}Td^{n}\in H^n(TE_B)$,定义它在$R^nT(f)$下的像为$T(f^n)(z)+\mathrm{im}T(d^{n-1})'$.
	\item 和投射情况一样,右导出函子的定义不依赖于内射预解族的选取.右导出函子$\{R^nT,n\ge0\}$总是加性函子.这里右的含义是,构造的时候运用的是内射预解,此时零对象习惯写在了左侧,于是仅有右侧具有信息,于是这样构造出来的函子称为"右"导出函子.
\end{enumerate}

按照之前投射情况的命题的对偶证明,得到一系列结果:
\begin{enumerate}
	\item Horseshoe引理的内射情况.给定阿贝尔范畴上的短正合列$0\to B'\to B\to B''\to0$,任取$B'$和$B''$的内射预解$E'=((E^n)')$和$E''=((E^n)'')$,那么两个预解的直和构成了$B$的一个内射预解,并且这三个预解构成了一个提升了初始短正合列的复形的短正合列.
	$$\xymatrix{0\ar[r]&B'\ar[r]\ar[d]&B\ar[r]&B''\ar[r]\ar[d]&0\\&(E^0)'\ar[d]&&(E^0)''\ar[d]&\\&(E^1)'\ar[d]&&(E^1)''\ar[d]&\\&&&&}$$
	\item 内射对象.给定内射对象$Q$,那么$0\to Q\to Q\to0\to0\to$是它的一个内射预解,任取加性函子$T$,它作用在简化内射预解上为$0\to TQ\to0\to0\to$,于是有$R^0T(Q)=TQ$和$R^nT(Q)=0,\forall n\ge1$.
	\item 右导出函子的长正合列.取阿贝尔范畴中的短正合列$0\to B'\to B\to B''\to0$.任取$B'$和$B''$的内射预解,按照Horseshoe引理,取对应内射对象的直和就得到$A$的一个内射预解,并且三个内射预解构成了一个复形的短正合列.按照加性函子保分离短正合列,说明将阿贝尔范畴之间的加性函子$T:\mathscr{A}\to\mathscr{C}$作用在这个复形的短正合列上仍然得到复形的短正合列.按照复形的短正合列诱导长正合列,就得到左导出函子的长正合列:
	$$\xymatrix{0\ar[r]&R^0T(B')\ar[r]&R^0T(B)\ar[r]&R^0T(B'')\ar[r]&}$$
	$$\xymatrix{\cdots\ar[r]&R^nTB'\ar[r]^{R^nTi}&R^nTB\ar[r]^{R^nTp}&R^nTB''\ar[r]^{\delta^n}&R^{n+1}TB'\ar[r]&\cdots}$$
	\item 具有足够多内射对象前提下第零个右导出函子$R^0T$与左正合性.按照定义,$R^0T(B)$事实上就是$\ker Td^0$.考虑上述诱导的长正合列的前端,说明了即便$T$本身未必是左正合函子,但是$R^0T$总会是一个左正合函子.反过来如果$T$本身是左正合函子,那么有自然同构$T\cong R^0T$.
\end{enumerate}

给定两个阿贝尔范畴之间的两组加性函子列$(F^n)_{n\ge0}$和$((F^n)')_{n\ge0}$:$\mathscr{A}\to\mathscr{B}$,假设$\mathscr{A}$具有足够多的内射对象.如果它们满足如下三条,那么就有每个$F^n$都自然同构于$(F^n)'$.
\begin{enumerate}
	\item 对每个短正合列$0\to A\to B\to C\to0$,分别存在这两个函子列的长正合列,并且连接态射都是自然的.
	\item $F^0$自然同构于$(F^0)'$.
	\item 对每个内射对象和每个$n\ge1$,有$E^n(P)=(E^n)'(P)=0$.
\end{enumerate}

可以把这个定理进行推广.给定阿贝尔范畴之间的一列加性函子$\{F^n:\mathscr{A}\to\mathscr{B}\}$.给定$\mathscr{A}$的一类对象$\mathscr{X}$,称$A$具有足够多的$\mathscr{X}$对象,如果对每个$\mathscr{A}$中的对象$A$,存在$A$到$\mathscr{X}$中某个对象的单态射.称这列加性函子是$\mathscr{X}$可去的,如果$\forall n\ge1,\forall X\in\mathscr{X}$有$F^n(X)=0$.现在给定两列加性函子$F^n,(F^n)':\mathscr{A}\to\mathscr{B}$,满足如下三个性质,那么必然有$F^n$自然同构于$(F^n)',\forall n\ge0$.
\begin{enumerate}
	\item 对每个$\mathscr{A}$中的短正合列$0\to A\to B\to C\to0$.那么分别存在这两列函子的长正合列,并且连接态射是自然的.
	\item $E^0$自然同构于$(E^0)'$.
	\item 这两列函子都是$\mathscr{X}$可去的.
\end{enumerate}

阿贝尔范畴之间的一列加性函子$\{T^n\}$称为上同调$\delta$函子,如果每个短正合列可诱导出连接映射自然的长正合列.两列上同调$\delta$函子之间的态射$(H^n)_{n\ge0}\to(T^n)_{n\ge0}$是指一列自然变换$\tau^n:H^n\to T^n$,满足对任意短正合列$0\to A\to B\to C\to0$诱导的两个长正合列,存在如下交换图:
$$\xymatrix{H^n(C)\ar[r]^{\delta}\ar[d]_{\tau^n_{C}}&H^{n+1}(A)\ar[d]^{\tau^{n+1}_{A}}\\T^n(C)\ar[r]_{\delta}&T^{n+1}(A)}$$

函子的上同调延拓.给定阿贝尔范畴的加性函子$F:\mathscr{A}\to\mathscr{B}$,称一个上同调$\delta$函子$\{T^n:\mathscr{A}\to\mathscr{B}\}$是$F$的上同调延拓,如果存在自然同构$F\cong T^0$.例如$\{\mathrm{Ext}^n_R(A,-)\}$是$\mathrm{Hom}_R(A,-)$的上同调延拓.关于上同调延拓有如下结论:
\begin{enumerate}
	\item 给定两个从$\mathscr{A}\to\textbf{Ab}$的上同调$\delta$函子$\{T^n\},\{H^n\}$,满足对任意内射对象$E$有$H^n(E)=0,\forall n\ge1$.并且存在自然变换$\tau^0:T^0\to H^0$.那么存在唯一的上同调$\delta$函子之间的态射$\tau:(T^n)\to(H^n)$.并且如果$\tau^0$是自然同构,那么$\tau^n,n\ge1$是自然同构.
	\item 特别的,给定左正合加性函子$F:\mathscr{A}\to\textbf{Ab}$.那么存在唯一的$F$的上同调延拓满足$H^n(P)=0,\forall n\ge1,\forall P$内射,此即$F$的右导出函子列.由此可得到$\mathrm{Ext}$的公理描述(见下节).
\end{enumerate}
\subsubsection{逆变右导出函子}

按照导出函子左右的含义,为得到逆变函子的右导出函子,就需要找对象的投射预解而不是内射预解.给定阿贝尔范畴之间的加性逆变函子$T:\mathscr{A}\to\mathscr{B}$,其中$\mathscr{A}$具有足够多的投射对象.$T$的右导出函子是一列逆变函子$R^nT:\mathscr{A}\to\mathscr{B},n\ge0$.
\begin{enumerate}
	\item 对$\mathscr{A}$上每个对象$A$,取$\xymatrix{\ar[r]&P_2\ar[r]^{d_2}&P_1\ar[r]^{d_1}&P_0\ar[r]^{\varepsilon}&A\ar[r]&0}$是$A$的一个固定的投射预解.称$\xymatrix{\ar[r]&P_2\ar[r]^{d_2}&P_1\ar[r]^{d_1}&P_0\ar[r]&0}$为它对应的简化投射预解.设$$TP_A:\xymatrix{0\ar[r]&TP_0\ar[r]^{Td_1}&TP_1\ar[r]^{Td_2}&TP_2\ar[r]&}$$是逆变加性函子$T$作用在简化投射预解上得到的$\mathscr{B}$中的一个复形.取同调,定义$(R^nT)(A)=H^n(TP_A)=\ker(Td_{n+1})/\mathrm{im}Td_n$.
	\item 接下来定义$R^nT$作用在态射上的定义.任取态射$f:A\to A'$,按照比较定理,存在同伦意义下唯一的链映射$f^*:P_A\to P_{A'}$延拓了$f$.就定义$R^nT(f)=H^n(Tf^*)$,换句话讲,对每个$z+\mathrm{im}Td_{n}\in H^n(TP_A)$,定义它在$R^nT(f)$下的像为$T(f_n)(z)+\mathrm{im}T(d_n)'$.
	\item 和投射情况一样,右导出函子的定义不依赖于投射预解族的选取.另外右导出函子$\{R^nT,n\ge0\}$总是加性函子.
\end{enumerate}

按照之前投射情况的命题的对偶证明,得到一系列结果:
\begin{enumerate}
	\item 投射对象.对投射对象$P$,那么$\to0\to0\to P\to P\to0$是它的一个投射预解,任取逆变加性函子$T$,它作用在简化投射预解上为$0\to TP\to0\to0\to$,于是有$R^0T(P)=TP$和$R^nT(P)=0,\forall n\ge1$.
	\item 右导出函子的长正合列.取阿贝尔范畴中的短正合列$0\to A'\to A\to A''\to0$.任取$A'$和$A''$的投射预解,按照Horseshoe引理,取对应投射对象的直和就得到$A$的一个投射预解,并且三个投射预解构成了一个复形的短正合列.按照加性函子保分离短正合列,说明将阿贝尔范畴之间的加性函子$T:\mathscr{A}\to\mathscr{C}$作用在这个复形的短正合列上仍然得到复形的短正合列.按照复形的短正合列诱导长正合列,就得到左导出函子的长正合列:
	$$\xymatrix{0\ar[r]&R^0T(A'')\ar[r]&R^0T(A)\ar[r]&R^0T(A')\ar[r]&}$$
	$$\xymatrix{\cdots\ar[r]&R^nTA''\ar[r]^{R^nTp}&R^nTA\ar[r]^{R^nTi}&R^nTA'\ar[r]^{\delta^n}&R^{n+1}TA''\ar[r]&\cdots}$$
	\item 具有足够多投射对象前提下第零个右导出函子$R^0T$与左正合性.按照定义,$R^0T(B)$事实上就是$\ker Td_0$.考虑上述诱导的长正合列的前端,说明了即便$T$本身未必是左正合函子,但是$R^0T$总会是一个左正合函子.反过来如果$T$本身是左正合函子,那么有自然同构$T\cong R^0T$.
\end{enumerate}

给定两个阿贝尔范畴之间的两组逆变加性函子列$(G^n)_{n\ge0}$和$((G^n)')_{n\ge0}$:$\mathscr{A}\to\mathscr{B}$,假设$\mathscr{A}$具有足够多的投射对象.如果它们满足如下三条,那么就有每个$G^n$都自然同构于$(G^n)'$.
\begin{enumerate}
	\item 对每个短正合列$0\to A\to B\to C\to0$,分别存在这两个函子列的长正合列,并且连接态射都是自然的.
	\item $G^0$自然同构于$(G^0)'$.
	\item 对每个投射对象和每个$n\ge1$,有$G^n(P)=(G^n)'(P)=0$.
\end{enumerate}

可以把这个定理进行推广.给定阿贝尔范畴之间的一列加性函子$\{G^n:\mathscr{A}\to\mathscr{B}\}$.给定$\mathscr{A}$的一类对象$\mathscr{X}$,称$A$具有足够多的$\mathscr{X}$对象,如果对每个$\mathscr{A}$中的对象$A$,存在$\mathscr{X}$中某个对象到$A$的满态射.称这列加性函子是$\mathscr{X}$可去的,如果$\forall n\ge1,\forall X\in\mathscr{X}$有$G^n(X)=0$.现在给定两列加性函子$G^n,(G^n)':\mathscr{A}\to\mathscr{B}$,满足如下三个性质,那么必然有$G^n$自然同构于$(G^n)',\forall n\ge0$.
\begin{enumerate}
	\item 对每个$\mathscr{A}$中的短正合列$0\to A\to B\to C\to0$.那么分别存在这两列函子的长正合列,并且连接态射是自然的.
	\item $G^0$自然同构于$(G^0)'$.
	\item 这两列函子都是$\mathscr{X}$可去的.
\end{enumerate}

阿贝尔范畴之间的一列逆变加性函子$\{H^n\}$称为逆变上同调$\delta$函子,如果每个短正合列可诱导出连接映射自然的长正合列.两列上同调$\delta$函子之间的态射$(H^n)_{n\ge0}\to(T^n)_{n\ge0}$是指一列自然变换$\tau^n:H^n\to T^n$,满足对任意短正合列$0\to A\to B\to C\to0$诱导的两个长正合列,存在如下交换图:
$$\xymatrix{H^{n+1}(A)\ar[r]^{\delta}\ar[d]_{\tau^{n+1}_{A}}&H^n(C)\ar[d]^{\tau^n_{C}}\\T^{n+1}(A)\ar[r]_{\delta}&T^n(C)}$$

函子的上同调延拓.给定阿贝尔范畴的逆变加性函子$F:\mathscr{A}\to\mathscr{B}$,称一个逆变上同调$\delta$函子$\{T^n:\mathscr{A}\to\mathscr{B}\}$是$F$的逆变上同调延拓,如果存在自然同构$F\cong T^0$.例如$\{\mathrm{Ext}^n_R(-,A)\}$是$\mathrm{Hom}_R(-,A)$的逆变上同调延拓.关于上同调延拓有如下结论:
\begin{enumerate}
	\item 给定两个从$\mathscr{A}\to\textbf{Ab}$的逆变上同调$\delta$函子$\{T^n\},\{H^n\}$,满足对任意投射对象$E$有$H^n(E)=0,\forall n\ge1$.并且存在自然变换$\tau^0:T^0\to H^0$.那么存在唯一的上同调$\delta$函子之间的态射$\tau:(T^n)\to(H^n)$.并且如果$\tau^0$是自然同构,那么$\tau^n,n\ge1$是自然同构.
	\item 特别的,给定逆变左正合加性函子$F:\mathscr{A}\to\textbf{Ab}$.那么存在唯一的$F$的逆变上同调延拓满足$H^n(P)=0,\forall n\ge1,\forall P$内射,此即$F$的逆变右导出函子列.特别的,由此可得到$\mathrm{Ext}$的公理描述的第二版本(见下节).
\end{enumerate}
\newpage
\subsection{Tor和Ext}

张量函子的导出函子,函子$\mathrm{Tor}_n^R$.
\begin{enumerate}
	\item 取环$R$,取左$R$模$B$,考虑加性函子$T=-\otimes_RB$,这是从右$R$模范畴到\textbf{Ab}的函子,如果$B$是一个$(R,S)$双边模,那么这是从右$R$模范畴到左$S$模范畴的加性函子.如果$R$是交换的,那么这就是$R$模范畴自身上的加性函子.按照模范畴上具有足够多的投射对象,$T$的左导出函子存在,记作$\mathrm{Tor}_n^R(-,B)=L_nT(-)$.
	\item 按照定义,给定右$R$摸$A$和左$R$模$B$,为计算$\mathrm{Tor}_n^R(A,B)$,先任取$A$的投射预解$P=(P_n,d_n)$,那么有:
	$$\mathrm{Tor}_n^R(A,B)=H_n(P\otimes_RB)=\frac{\ker(d_n\otimes1_B)}{\mathrm{im}(d_{n+1}\otimes1_B)}$$
	\item 考虑另一个张量函子$T=A\otimes_R-$,其中$A$是一个右$R$模.同样的当$A$具有某些双侧模结构或者$R$本身交换等情况下,函子$T$可视为左$R$模范畴到\textbf{Ab}或者某个右$S$模范畴,或者干脆交换环$R$上模范畴自身的加性函子.它的左导出函子记作$\mathrm{tor}_n^R(A,-)=L_nT(-)$.
	\item 同样的按照定义,给定右$R$模$A$和左$R$模$B$,为计算$\mathrm{tor}_n^R(A,B)$,先任取$B$的投射预解$Q=(Q_n,d_n)$,那么有:
	$$\mathrm{tor}_n^R(A,B)=H_n(A\otimes Q)=\frac{\ker(1_A\otimes d_n)}{\mathrm{im}(1_A\otimes d_{n+1})}$$
\end{enumerate}

实际上$\mathrm{Tor}$和$\mathrm{tor}$是一致的.即对任意右$R$模$A$和左$R$模$B$,恒有$\mathrm{Tor}_n^R(A,B)\cong\mathrm{tor}_n^R(A,B),\forall n\ge0$.
\begin{enumerate}
	\item 引理1.设阿贝尔范畴$\mathscr{A}$具有足够多投射对象.$\xymatrix{P:\ar[r]&P_2\ar[r]^{d_2}&P_1\ar[r]^{d_1}&P_0\ar[r]^{\varepsilon}&0}$是对象$A$的一个投射预解.记syzygy列为$K_0=\ker\varepsilon$,$K_n=\ker d_n,n\ge1$.那么有同构:
	$$L_{n+1}T(A)\cong L_nT(K_0)\cong L_{n-1}T(K_1)\cong\cdots\cong L_1T(K_{n-1})$$
	\begin{proof}
		
		按照$P$的正合性,有$K_0=\ker\varepsilon=\mathrm{im}d_1$.于是$\xymatrix{Q=\ar[r]&P_2\ar[r]^ {d_2}&P_1\ar[r]^{d_1}&K_0\ar[r]&0}$是$K_0$的投射预解.按照同调不依赖投射预解的选取,得到$L_nT(K_0)=L{n+1}T(A)$.按照归纳容易验证剩余的内容.
	\end{proof}
    \item 特别的,取定左$R$模$B$,那么对右$R$模$A$,任取投射预解$P$,一样定义$\{K_n\}$,就得到:$$\mathrm{Tor}^R_{n+1}(A,B)=\mathrm{Tor}^R_n(K_0,B)=\cdots=\mathrm{Tor}^R_1(K_{n-1},B)$$
    \item 特别的,取右$R$模$A$,那么对左$R$模$B$,任取投射预解$P$,一样定义$\{V_n\}$,就得到:$$\mathrm{tor}^R_{n+1}(A,B)=\mathrm{tor}^R_n(A,V_0)=\cdots=\mathrm{tor}^R_1(A,V_{n-1})$$
	\item 引理2,给定阿贝尔范畴的如下交换图,其中行和列都是正合列,那么有$\ker f\cong\ker a$,$\ker h\cong\ker b$.
	$$\xymatrix{
		&\ker f\ar[r]\ar[d]^i&0\ar[d]&\ker h\ar[d]&\\
		\ker a\ar[r]^j\ar[d]&L'\ar[r]^a\ar[d]^f&M'\ar[r]\ar[d]^g&N'\ar[r]\ar[d]^h&0\\
		0\ar[r]&L\ar[d]\ar[r]&M\ar[d]\ar[r]&N\ar[r]\ar[d]&0\\
		\ker b\ar[r]&L''\ar[r]^b\ar[d]&M''\ar[r]\ar[d]&N''\ar[d]\ar[r]&0\\
		&0&0&0&
	}$$
	\begin{proof}
		
		考虑态射$f,g,h$,按照蛇形引理得到正合列$\ker g\to\ker h\to\mathrm{coker}f\to\mathrm{coker g}$.其中$\ker g=0$,而$\mathrm{coker}f=L''$,$\mathrm{coker}g=M''$,那么$\mathrm{coker}f\to\mathrm{coker}g$可以取为$b$.于是得到正合列$0\to\ker h\to L''\to M''$,这得到$\ker h\cong\ker b$.
		
		现在设$i,j$都是包含映射.注意到$fj=0$,于是$\ker a=\mathrm{im}j\subset\ker f=\mathrm{im}i$.再注意到$ai=0$得到$\ker f=\mathrm{im}i\subset\ker a=\mathrm{im}j$,于是$\mathrm{im}i=\mathrm{im}j$,于是$\ker f\cong\ker a$.
	\end{proof}
	\item 取右$R$模$A$和左$R$模$B$,分别取它们的投射预解为:
	$$P_A:\xymatrix{\ar[r]&P_1\ar[r]^{d_1}&P_0\ar[r]^{\varepsilon}&A\ar[r]&0};Q_B:\xymatrix{\ar[r]&Q_1\ar[r]^{d_1'}&Q_0\ar[r]^{\varepsilon'}&B\ar[r]&0}$$
	
	那么对任意$n\ge0$有$H_n(P_A\otimes_RB)\cong H_n(A\otimes_RQ_B)$,也即$\forall n\ge0$有$\mathrm{Tor}_n^R(A,B)\cong\mathrm{tor}_n^R(A,B)$.
	\begin{proof}
		
		$n=0$的情况,按照两个张量函子都是右正合的,于是有$\mathrm{Tor}_0^R(A,B)\cong\mathrm{tor}_0^R(A,B)\cong A\otimes_RB$.现在把预解$P_A$中每个微分映射$d_n:P_n\to P_{n-1}$分解为经syzygy的两个态射:
		$$\xymatrix{0\ar[dr]&&0&&0\ar[dr]&&0&&&&\\&K_3\ar[dr]^{i_4}\ar[ur]&&&&K_1\ar[ur]\ar[dr]^{\varepsilon_2}&&&&&\\\cdots\ar[ur]\ar[rr]&&P_3\ar[dr]_{i_3}\ar[rr]^{d_3}&&P_2\ar[ur]^{i_2}\ar[rr]^{d_2}&&P_1\ar[dr]_{i_1}\ar[rr]^{d_1}&&P_0\ar[r]^{\varepsilon}&A\ar[r]&0\\&&&K_2\ar[dr]\ar[ur]_{\varepsilon_3}&&&&K_0\ar[dr]\ar[ur]_{\varepsilon_1}&&&\\&&0\ar[ur]&&0&&0\ar[ur]&&0&&}$$
		
		那么我们证明过全部斜向的序列$0\to K_i\to P_i\to K_{i-1}\to0$都是短正合列.其中可记$K_{-1}=A$,那么第一项$0\to K_0\to P_0\to A\to0$依然是短正合列.同理记预解$Q_B$的syzygy为$\{V_n\}$就得到一族短正合列$0\to V_j\to Q_j\to V_{j-1}\to0,j\ge0$,其中同样约定$V_{-1}=B$.按照张量函子可视为双侧函子,于是下图对任意固定的$i,j\ge-1$都是交换图,其中$X,Y,Z,W$是相应模同态的核.这个交换图实际上每一行与每一列都是正合的,一部分原因是张量函子总是右正合函子,另一部分原因是$X,Y,Z,W$本身就取了适当的核.最后中间行中间列都是短正合列因为$P_i$和$Q_j$都是平坦模:
		$$\xymatrix{&X\ar[r]\ar[d]&0\ar[d]&W\ar[d]&\\Y\ar[r]\ar[d]&K_i\otimes V_j\ar[r]\ar[d]&K_i\otimes Q_i\ar[r]\ar[d]&K_i\otimes V_{j-1}\ar[r]\ar[d]&0\\0\ar[r]&P_i\otimes V_j\ar[d]\ar[r]&P_i\otimes Q_j\ar[d]\ar[r]&P_i\otimes V_{j-1}\ar[r]\ar[d]&0\\Z\ar[r]&K_{i-1}\otimes V_j\ar[r]\ar[d]&K_{i-1}\otimes Q_j\ar[r]\ar[d]&K_{i-1}\otimes V_{j-1}\ar[d]\ar[r]&0\\&0&0&0&}$$
		
		注意到$W=\mathrm{Tor}_1(K_ {i-1},V_{j-1})$,$X=\mathrm{Tor}_1(K_{i-1},V_j)$,$Y=\mathrm{tor}_1(K_i,V_{j-1})$,$Z=\mathrm{tor}_1(K_{i-1},V_{j-1})$.按照引理2得到:
		$$\mathrm{Tor}_1(K_{i-1},V_{j-1})\cong \mathrm{tor}_1(K_{i-1},V_{j-1}),i,j\ge-1$$
		$$\mathrm{Tor}_1(K_{i-1},V_{j})\cong \mathrm{tor}_1(K_{i},V_{j-1}),i,j\ge-1$$
		
		特别的,对$i=j=0$,注意到$K_{-1}=A$,$V_{-1}=B$.于是这就是命题在$n=1$时成立.现在按照引理1,有:
		$$\mathrm{tor}_{n+1}(A,B)\cong \mathrm{tor}_1(A,V_{n-1})=\mathrm{tor}_1(K_{-1},V_{n-1})$$
		$$\mathrm{Tor}_{n+1}(A,B)\cong \mathrm{Tor}_1(K_{n-1},B)=\mathrm{Tor}_1(K_{n-1},V_{-1})$$
		
		于是为证$n+1$的情况,只需反复带入$X\cong Y$,结合$n=1$已经得证:
		$$\mathrm{tor}_{n+1}(A,B)\cong\mathrm{tor}_1(K_{-1},V_{n-1})$$
		$$\mathrm{Tor}_{1}(K_{-1},V_{n-1})\cong\mathrm{tor}_1(K_{0},V_{n-2})$$
		$$\mathrm{Tor}_{1}(K_0,V_{n-2})\cong\mathrm{tor}_1(K_{1},V_{n-3})$$
		$$\cdots$$
		$$\mathrm{Tor}_{1}(K_{n-2},V_0)\cong\mathrm{tor}_1(K_{n-1},V_{-1})$$
		$$\mathrm{Tor}_{1}(K_{n-1},V_{-1})\cong\mathrm{Tor}_{n+1}(A,B)$$
	\end{proof}
    \item 借助谱序列.记函子$A\otimes_R-$的左导出函子列是$\mathrm{Tor}_n^R(A,-)$,记函子$-\otimes_RB$的左导出函子列是$\mathrm{tor}_n^R(-,B)$.那么有自然同构$\mathrm{Tor}_n^R(A,B)\cong\mathrm{tor}_n^R(A,B)$.
    \begin{proof}
    	
    	取$A$和$B$的简化投射预解分解为$\textbf{P}_A$和$\textbf{Q}_B$.记张量积构成的双复形$M_{s,t}=P_s\otimes Q_t$.先求复形$M_{s,\bullet}$的同调,得到:
    	$$(^1E)^1_{s,t}=\mathrm{H}_t(M_{s,\bullet})=\left\{\begin{array}{cc}\{0\}&q>0\\P_s\otimes B\end{array}\right.$$
    	
    	再求同调得到累次同调,也即谱序列第二项:
    	$$(^1)E^2_{s,t}=\mathrm{H}_p\mathrm{H}_q(M_{\bullet,\bullet})=\left\{\begin{array}{cc}\{0\}&q>0\\\mathrm{tor}_s^R(A,B)\end{array}\right.$$
    
        这个谱序列塌陷在$s$轴,于是有:
        $$\mathrm{H}_n(\textbf{P}_A\otimes\textbf{Q}_B)\cong(^1E)^2_{n,0}=\mathrm{tor}_n^B(A,B)$$
    
        同理有$\mathrm{H}_n(\textbf{P}_A\otimes\textbf{Q}_B)\cong(^2E)^2_{0,n}=\mathrm{Tor}_n^B(A,B)$.这得证.
\end{proof}
\end{enumerate}

$\mathrm{Tor}$函子的一些基本性质.
\begin{enumerate}
	\item 按照张量函子是右正合函子,总有自然同构$\mathrm{Tor}_1^R(A,B)\cong A\otimes_RB$.
	\item 如果$A$是平坦右$R$模或者$B$是平坦左$R$模,那么恒有$\mathrm{Tor}_n^R(A,B)=0,\forall n\ge1$.
	\item 短正合列诱导的长正合列.给定右$R$模的短正合列$0\to A'\to A\to A''\to0$,任取左$R$模$B$则诱导了$\mathrm{Tor}$函子的长正合列:
	$$\xymatrix{\mathrm{Tor}_2^R(A'',B)\ar[r]&\mathrm{Tor}_1^R(A',B)\ar[r]&\mathrm{Tor}_1^R(A,B)\ar[r]&\mathrm{Tor}_1^R(A'',B)\ar[r]&}$$
	$$\xymatrix{A'\otimes_RB\ar[r]&A\otimes_RB\ar[r]&A''\otimes_RB\ar[r]&0}$$
	\item 短正合列诱导的长正合列,版本2.给定左$R$模的短正合列$0\to B'\to B\to B''\to0$,任取右$R$模$A$则诱导了$\mathrm{Tor}$函子的长正合列:
	$$\xymatrix{\mathrm{Tor}_2^R(A,B'')\ar[r]&\mathrm{Tor}_1^R(A,B')\ar[r]&\mathrm{Tor}_1^R(A,B)\ar[r]&\mathrm{Tor}_1^R(A,B'')\ar[r]&}$$
	$$\xymatrix{A\otimes_RB'\ar[r]&A\otimes_RB\ar[r]&A\otimes_RB''\ar[r]&0}$$
	\item 同调的连接映射的自然性的推论.给定右$R$模的交换图如下,其中两行都是短正合列:
	$$\xymatrix{0\ar[r]&A'\ar[r]^i\ar[d]_f&A\ar[r]^p\ar[d]_g&A''\ar[r]\ar[d]_h&0\\0\ar[r]&C'\ar[r]_j&C\ar[r]_q&C''\ar[r]&0}$$
	
	那么任取左$R$模$B$则诱导了如下交换图,其中两行都是正合列:
	$$\xymatrix{\mathrm{Tor}_n^R(A'',B)\ar[r]^{i_*}\ar[d]_{f_*}&\mathrm{Tor}_n^R(A,B)\ar[r]^{p_*}\ar[d]_{g_*}&\mathrm{Tor}_n^R(A'',B)\ar[r]^{\partial_n}\ar[d]_{h_*}&\mathrm{Tor}_{n-1}^R(A',B)\ar[d]_{f_*}\\\mathrm{Tor}_n^R(C',B)\ar[r]_{j_*}&\mathrm{Tor}_n^R(C,B)\ar[r]_{q_*}&\mathrm{Tor}_n^R(C'',B)\ar[r]_{\partial_n'}&\mathrm{Tor}_{n-1}^R(C',B)}$$
	\begin{proof}
		
		取$A'$和$A''$的投射预解,它们的直和是$A$的投射预解,于是短正合列的正合性诱导了简化投射预解的短正合列,将$T=-\otimes_RB$作用其上,按照加性函子和二元积可交换,得到加性函子作用在分离短正合列上仍然为分离短正合列,即$T$作用在这个复形的短正合列上仍然是短正合列,于是连接映射的自然性就得到结论.
	\end{proof}
    \item 同调的连接映射的自然性的推论,版本2.给定左$R$模的交换图如下,其中两行都是短正合列:
    $$\xymatrix{0\ar[r]&B'\ar[r]^i\ar[d]_f&B\ar[r]^p\ar[d]_g&B''\ar[r]\ar[d]_h&0\\0\ar[r]&C'\ar[r]_j&C\ar[r]_q&C''\ar[r]&0}$$
    
    那么任取右$R$模$A$则诱导了如下交换图,其中两行都是正合列:
    $$\xymatrix{\mathrm{Tor}_n^R(A,B')\ar[r]^{i_*}\ar[d]_{f_*}&\mathrm{Tor}_n^R(A,B)\ar[r]^{p_*}\ar[d]_{g_*}&\mathrm{Tor}_n^R(A,B'')\ar[r]^{\partial_n}\ar[d]_{h_*}&\mathrm{Tor}_{n-1}^R(A,B')\ar[d]_{f_*}\\\mathrm{Tor}_n^R(A,C')\ar[r]_{j_*}&\mathrm{Tor}_n^R(A,C)\ar[r]_{q_*}&\mathrm{Tor}_n^R(A,C'')\ar[r]_{\partial_n'}&\mathrm{Tor}_{n-1}^R(A,C')}$$
\end{enumerate}

Tor的公理.给定一列从右$R$模范畴到\textbf{Ab}的共变加性函子$\{T_n,n\ge0\}$,满足如下三个要求,那么对每个$n\ge0$有自然同构$T_n\cong\mathrm{Tor}_n^R(-,M)$.
\begin{enumerate}
	\item 对任意右$R$模的短正合列$0\to A\to B\to C\to0$,存在自然的连接同态$\Delta_n$满足如下长正合列:
	$$\xymatrix{\ar[r]&T_{n+1}\ar[r]^{\Delta_{n+1}}&T_n(A)\ar[r]&T_n(B)\ar[r]&T_n(C)\ar[r]^{\Delta_n}&T_{n-1}(A)\ar[r]&}$$
	\item $T_0$自然同构于$-\otimes_RM$,其中$M$是某个左$R$模.
	\item 对任意投射右$R$模$P$有$T_n(P)=\{0\},n\ge1$.
\end{enumerate}

Tor的公理,版本2.给定一列从左$R$模范畴到\textbf{Ab}的共变加性函子$\{T_n,n\ge0\}$,满足如下三个要求,那么对每个$n\ge0$有自然同构$T_n\cong\mathrm{Tor}_n^R(M,-)$.
\begin{enumerate}
	\item 对任意左$R$模的短正合列$0\to A\to B\to C\to0$,存在自然的连接同态$\Delta_n$满足如下长正合列:
	$$\xymatrix{\ar[r]&T_{n+1}\ar[r]^{\Delta_{n+1}}&T_n(A)\ar[r]&T_n(B)\ar[r]&T_n(C)\ar[r]^{\Delta_n}&T_{n-1}(A)\ar[r]&}$$
	\item $T_0$自然同构于$M\otimes_R-$,其中$M$是某个右$R$模.
	\item 对任意投射左$R$模$P$有$T_n(P)=\{0\},n\ge1$.
\end{enumerate}

Hom函子的导出函子.函子$\mathrm{Ext}$.
\begin{enumerate}
	\item 取环$R$,取左$R$模$A$,考虑共变加性函子$T=\mathrm{Hom}_R(A,-)$,这是从左$R$模范畴到\textbf{Ab}的函子,如果$A$是一个$(R,S)$双边模,那么这是从左$R$模范畴到左$S$模范畴的加性函子.如果$R$是交换的,那么这就是$R$模范畴自身上的加性函子.按照模范畴上具有足够多的内射对象,$T$的右导出函子存在,记作$\mathrm{Ext}^n_R(A,-)=R^nT(-)$.
	\item 按照定义,给定做$R$摸$A$和$B$,为计算$\mathrm{Tor}_n^R(A,B)$,先任取$B$的内射预解$P=(E^n,d^n)$,那么有:
	$$\mathrm{Ext}^n_R(A,B)=H^n(\mathrm{Hom}_R(A,E_B))=\frac{\ker(\mathrm{Hom}(1_A,d^n))}{\mathrm{im}(\mathrm{Hom}(1_A,d^{n-1}))}$$
	\item 取环$R$,取左$R$模$B$,考虑逆变加性函子$T=\mathrm{Hom}_R(-,B)$.它的右导出函子记作$\mathrm{ext}_R^n(-,B)=R^nT$.按照定义,为计算$\mathrm{ext}_R^n(A,B)$,先任取$A$的投射预解$P=(P_n,d_n)$,那么有:
	$$\mathrm{ext}_R^n(A,B)=H^n(\mathrm{Hom}_R(P_A,B))=\frac{\ker(\mathrm{Hom}(d_n,1_B))}{\mathrm{im}(\mathrm{Hom}(d_{n-1},1_B))}$$
	\item 和$\mathrm{Tor}$的情况一样,两种Hom函子诱导出的右导出函子是相同的,即对任意左$R$模$A$和$B$,恒有$\mathrm{Ext}_R^n(A,B)\cong\mathrm{ext}_R^n(A,B),\forall n\ge0$.
\end{enumerate}

$\mathrm{Ext}$函子的一些基本性质.
\begin{enumerate}
	\item 按照Hom函子是左正合函子,总有自然同构$\mathrm{Ext}_R^1(A,B)\cong\mathrm{Hom}_R(A,B)$.
	\item 如果$A$是投射左$R$模或者$B$是内射左$R$模,那么恒有$\mathrm{Ext}_n^R(A,B)=0,\forall n\ge1$.
	\item 短正合列诱导的长正合列.给定左$R$模的一个短正合列$0\to B'\to B\to B''\to0$,任取左$R$模$A$则诱导了长正合列:
	$$\xymatrix{0\ar[r]&\mathrm{Hom}_R(A,B')\ar[r]&\mathrm{Hom}_R(A,B)\ar[r]&\mathrm{Hom}_R(A,B'')}$$
	$$\xymatrix{\mathrm{Ext}^1_R(A,B')\ar[r]&\mathrm{Ext}^1_R(A,B)\ar[r]&\mathrm{Ext}^1_R(A,B'')\ar[r]&\mathrm{Ext}^2_R(A,B')\ar[r]&}$$
	\item 短正合列诱导的长正合列,版本2.给定左$R$模的一个短正合列$0\to A'\to A\to A''\to0$,任取左$R$模$B$则诱导了长正合列:
	$$\xymatrix{0\ar[r]&\mathrm{Hom}_R(A'',B)\ar[r]&\mathrm{Hom}_R(A,B)\ar[r]&\mathrm{Hom}_R(A',B)}$$
	$$\xymatrix{\mathrm{Ext}^1_R(A'',B)\ar[r]&\mathrm{Ext}^1_R(A,B)\ar[r]&\mathrm{Ext}^1_R(A',B)\ar[r]&\mathrm{Ext}^2_R(A'',B)\ar[r]&}$$
	\item 同调的连接映射的自然性的推论.给定左$R$模的交换图如下,其中两行都是短正合列:
	$$\xymatrix{0\ar[r]&B'\ar[r]^i\ar[d]_f&B\ar[r]^p\ar[d]_g&B''\ar[r]\ar[d]_h&0\\0\ar[r]&C'\ar[r]_j&C\ar[r]_q&C''\ar[r]&0}$$
	
	那么任取左$R$模$A$则诱导了如下交换图,其中两行都是正合列:
	$$\xymatrix{\mathrm{Ext}^n_R(A,B')\ar[r]^{i_*}\ar[d]_{f_*}&\mathrm{Ext}^n_R(A,B)\ar[r]^{p_*}\ar[d]_{g_*}&\mathrm{Ext}^n_R(A,B'')\ar[r]^{\partial_n}\ar[d]_{h_*}&\mathrm{Ext}^{n+1}_R(A,B')\ar[d]_{f_*}\\\mathrm{Ext}^n_R(A,C')\ar[r]_{j_*}&\mathrm{Ext}^n_R(A,C)\ar[r]_{q_*}&\mathrm{Ext}^n_R(A,C'')\ar[r]_{\partial_n'}&\mathrm{Ext}^{n+1}_R(A,C')}$$
	\item 同调的连接映射的自然性的推论,版本2.给定左$R$模的交换图如下,其中两行都是短正合列:
	$$\xymatrix{0\ar[r]&A'\ar[r]^i\ar[d]_f&A\ar[r]^p\ar[d]_g&A''\ar[r]\ar[d]_h&0\\0\ar[r]&C'\ar[r]_j&C\ar[r]_q&C''\ar[r]&0}$$
	
	那么任取左$R$模$B$则诱导了如下交换图,其中两行都是正合列:
	$$\xymatrix{\mathrm{Ext}^n_R(A'',B)\ar[r]^{i_*}\ar[d]_{f_*}&\mathrm{Ext}^n_R(A,B)\ar[r]^{p_*}\ar[d]_{g_*}&\mathrm{Ext}^n_R(A',B)\ar[r]^{\partial_n}\ar[d]_{h_*}&\mathrm{Ext}^{n+1}_R(A'',B)\ar[d]_{f_*}\\\mathrm{Ext}^n_R(C'',B)\ar[r]_{j_*}&\mathrm{Ext}^n_R(C,B)\ar[r]_{q_*}&\mathrm{Ext}^n_R(C',B)\ar[r]_{\partial_n'}&\mathrm{Ext}^{n+1}_R(C'',B)}$$
\end{enumerate}

$\mathrm{Ext}$的公理.给定一列从$R$模范畴到Ab的共变加性函子$F^n,n\ge0$,满足如下三个条件,那么每个$F^n$自然同构于$\mathrm{Ext}_R^n(M,-)$.
\begin{enumerate}
	\item 对任意左$R$模的短正合列$0\to A\to B\to C\to0$,存在连接同态$\Delta_n$满足如下长正合列:
	$$\xymatrix{
		\ar[r]&F^{n-1}(C)\ar[r]^{\Delta_{n-1}}&F^n(A)\ar[r]&F^n(B)\ar[r]&F^n(C)\ar[r]^{\Delta_n}&F^{n+1}(A)\ar[r]&
	}$$
	\item $F^0$自然同构于$\mathrm{Hom}_R(M,-)$,其中$M$是某个左$R$模.
	\item 对任意内射左$R$模$E$有$F^n(E)=0,n\ge1$.
\end{enumerate}

$\mathrm{Ext}$的公理,版本2.给定一列从$R$模范畴到Ab的逆变加性函子$G^n,n\ge0$,满足如下三个条件,那么每个$G^n$自然同构于$\mathrm{Ext}_R^n(-,M)$.
\begin{enumerate}
	\item 对任意左$R$模的短正合列$0\to A\to B\to C\to0$,存在连接同态$\Delta_n$满足如下长正合列:
	$$\xymatrix{\ar[r]&G^{n-1}(A)\ar[r]^{\Delta}&G^n(C)\ar[r]&G^n(B)\ar[r]&G^n(A)\ar[r]^{\Delta}&G^{n+1}(C)\ar[r]&}$$
	\item $G^0$自然同构于$\mathrm{Hom}_R(-,M)$,其中$M$是某个左$R$模.
	\item 对任意投射左$R$模$P$有$G^n(P)=0,n\ge1$.
\end{enumerate}
\subsubsection{Tor的性质}

首先如果给定左$R$模$B$和右$R$模$A$,那么有同构:$\mathrm{Tor}_n^R(A,B)\cong\mathrm{Tor}_n^{R^{\mathrm{op}}}(B,A),\forall n\ge0$.于是当$R$是交换环的时候,有$\mathrm{Tor}_n^R(A,B)\cong\mathrm{Tor}_n^R(B,A),\forall n\ge0$.
\begin{proof}
	
	我们知道一个左$R$模是右$S=R^{\mathrm{op}}$模,右$R$模是左$S$模.现在任取右$R$模$A$的简化投射预解$P_A$,那么$P_A$可视为$S$上左模$A$的投射预解.下面考虑典范链映射$t=(t_n):P_A\otimes_RB\to B\otimes_SP_A$,其中$t_n:P_n\otimes_RB\to B\otimes_SP_n$为$x_n\otimes b\mapsto b\otimes x_n$.于是每个$t_n$是交换群同构.于是链映射$t$是同构.按照同构的链复形具有同构的同调,得到结论.
\end{proof}

平坦性的Tor准则.给定一个右$R$模$F$,如下三个条件等价.
\begin{enumerate}
	\item $F$是平坦模.
	\item 对任意左$R$模$M$和任意$n\ge1$有$\mathrm{Tor}_n^R(F,M)=0$.
	\item 对任意左$R$模$M$有$\mathrm{Tor}_1^R(F,M)=0$.
\end{enumerate}
\begin{proof}
	
	1推2,如果$F$是平坦模,那么$F\otimes_R-$是正合函子,于是任取$M$的投射预解$P_M$就有$F\otimes_RP_M$是正合列,这就得到$\mathrm{Tor}_n^R(F,M)=0,\forall n\ge1$.
	
	2推3是直接的,最后证明3推1,任取单同态$i:A\to B$,把它拼接为一个短正合列$0\to A\to B\to B/A\to0$.这个短正合列诱导了$\mathrm{Tor}$的长正合列,导致有如下正合列,这就说明$1_F\otimes i$是单射,于是$F$是平坦模.
	$$\xymatrix{0=\mathrm{Tor}_1^R(F,B/A)\ar[r]&F\otimes A\ar[r]^{1\otimes i}&F\otimes B}$$
\end{proof}

$\mathrm{Tor}$函子与平坦预解.我们知道投射模都是平坦模,于是投射预解都是平坦预解.接下来我们$\mathrm{Tor}$函子可经平坦预解求得.换句话讲,任取右$R$模$A$和左$R$模$B$,分别取$A,B$的平坦预解为$F,G$,对应的简化预解记作$F_A$和$G_B$,那么对所有$n\ge0$恒有:
$$H_n(F_A\otimes_RB)\cong\mathrm{Tor}_n^R(A,B)\cong H_n(A\otimes_RG_B)$$
\begin{proof}
	
	我们只需证明$H_n(F_A\otimes_RB)\cong\mathrm{Tor}_n^R(A,B)$,另一侧的证明只要考虑反环$R^{\mathrm{op}}$.$n=0$的情况是直接的,设平坦预解的终端为$\xymatrix{\ar[r]&F_1\ar[r]^{d_1}&F_0\ar[r]&A\ar[r]&0}$,于是$H_0(F_A\otimes_RB)=\mathrm{coker}(d_1\otimes1_B)=A\otimes_RB=\mathrm{Tor}_0^R(A,B)$.
	
	下面证明$n=1$.考虑如下交换图,其中$Y=\ker d_0=\mathrm{im}d_1$,$i$是包含映射:
    $$\xymatrix{F_2\ar[r]^{d_2}&F_1\ar[rr]^{d_1}\ar[dr]_{d_1'}&&F_0\\&&Y\ar[ur]^i&}$$

    将$-\otimes_RB$作用其上,得到$\mathrm{im}d_1\otimes1=\mathrm{im}i\otimes1$,另外张量函子的又正合性保证了$d_1'\otimes1_B$是满同态.现在考虑如下交换图:
    $$\xymatrix{F_2\otimes_RB\ar[r]^{d_2\otimes1}&F_1\otimes_RB\ar[r]^{d_1\otimes1}\ar[d]_{\alpha}&F_0\otimes_RB\\&F_1\otimes_RB/\mathrm{im}(d_2\otimes1)\ar@/_/[ur]^{\delta}\ar[d]_{\beta}&\\&F_1\otimes_RB/\ker(d_1\otimes1)\ar@/_/[uur]_{\gamma}&}$$

    其中$\alpha$是典范的商映射,而从$\mathrm{im}(d_2\otimes1_B)\subset\ker(d_1\otimes1_B)$可定义典范满映射$\beta$.$\gamma$是由同态定理得到的单射.取$\delta=\gamma\circ\beta$,那么它使得上述图表交换.那么我们断言有$\ker\delta=\frac{\ker(d_1\otimes1_B)}{\mathrm{im}(d_2\otimes1_B)}=H_1(F_A\otimes B)$.这件事是因为如下一般事实:如果$N$是$M$的子模并且有如下交换图表,那么有$\ker f/N=\ker h$.
    $$\xymatrix{M\ar[rr]^f\ar[d]_g&&X\\M/N\ar@/_0.5pc/[urr]_h&&}$$
    
    从$\alpha,\beta$满射得到$\mathrm{im}(d_1\otimes1_B)=\mathrm{im}\delta=\mathrm{im}\gamma$,
    
    【】
    
    $\mathrm{im}(d_1\otimes1)=\mathrm{im}i\otimes1$.最后考虑同调的长正合列$\mathrm{Tor} _1^R(F_0,B)\to\mathrm{Tor}_1^R(A,B)\to Y\otimes_RB\to F_0\otimes_RB$.按照$F_0$平坦得到第一个为0.于是$\mathrm{Tor} _1^R(A,B)\sim\ker(i\otimes1)\sim H_1(F_A\otimes B)$.最后,如果$n\ge1$成立,考虑正合列$\mathrm{Tor} _{n+1}^R(F_0,B)\to\mathrm{Tor}_{n+1}^R(A,B)\to\mathrm{Tor}_n^R(Y,B)\to \mathrm{Tor}_n^R(F_0,B)$.按照两端为0,得到$\mathrm{Tor} _{n+1}(A,B)\sim\mathrm{Tor}_n(Y,B)$.注意到$F'=\to F_2\to F_1\to Y\to0$是$Y$的平坦预解,按照归纳假设得到$H_n(F'_Y\otimes B)\sim\mathrm{Tor}_n(Y,B)$.再结合$H_n(F'_Y\otimes B)=H_{n+1}(F_A\otimes B)$,完成归纳.
\end{proof}

$\mathrm{Tor}$与直和可交换.给定一族左$R$模$\{B_k,k\in K\}$,那么对每个右$R$模$A$和任意$n\ge0$有如下第一个自然同构式,类似的还有第二个自然同构式:
$$\mathrm{Tor}_n^R(A,\oplus_{k\in K}B_k)\cong\oplus_{k\in K}\mathrm{Tor}_n^R(A,B_k),n\ge0$$
$$\mathrm{Tor}_n^R(\oplus_{i\in I}A_i,B)\cong\oplus_{i\in I}\mathrm{Tor}_n^R(A_i,B),n\ge0$$
\begin{proof}
	
	我们只来证明第一式.对$n$归纳,首先$n=0$的情况即张量积与直和的自然的可交换性.现在假设对$n$已经成立,对每个$k\in K$,取短正合列$0\to N_k\to P_k\to B_k\to0$,其中$P_k$是投射模,那么这诱导了直和的短正合列$0\to\oplus_kN_k\to\oplus_kP_k\to\oplus_kB_k\to0$.另外$\oplus_kP_k$是投射模.考虑短正合列诱导的长正合列,得到如下交换图:
	$$\xymatrix{\mathrm{Tor}_{n+1}(A,\oplus_kP_k)\ar[r]&\mathrm{Tor}_{n+1}(A,\oplus_kB_k)\ar[d]_{\sigma}\ar[r]^{\delta}&\mathrm{Tor}_n(A,\oplus_kN_k)\ar[r]\ar[d]_{\tau}&\mathrm{Tor}_n(A,\oplus_kP_k)\ar[d]\\\oplus_k\mathrm{Tor}_{n+1}(A,P_k)\ar[r]&\oplus_k\mathrm{Tor}_{n+1}(A,B_k)\ar[r]_{\delta'}&\oplus_k\mathrm{Tor}_n(A,N_k)\ar[r]&\oplus_k\mathrm{Tor}_n(A,P_k)}$$
	
	其中$\tau$是归纳假设中的自然同构,短五引理得到唯一的同构$\sigma$,接下来要类似同调$\delta$函子唯一延拓定理中那样,先证明这里的$\sigma$良性,即不随选取的短正合列族的改变而改变,再证明$\sigma$的自然性.
	
	不过我们也可以直接运用同调$\delta$函子唯一延拓定理.注意$\mathrm{Tor}_n^R(-,\oplus_{k\in K}B_k)$无非就是唯一延拓了右正合函子$-\otimes_R(\oplus_{k\in K}B_k)$的同调$\delta$函子.因而我们只需说明$\oplus_{k\in K}\mathrm{Tor}_n^R(-,B_k)$是延拓了函子$-\otimes_R(\oplus_{k\in K})B_k$的同调$\delta$函子.
	
	如果$P$是投射左模,那么有$\oplus_{k\in K}\mathrm{Tor}_n^k(P,B_k)=0,\forall n\ge1$;另外当$n=0$时有自然同构$-\otimes_R(\oplus_{k\in K}B_k)\cong\oplus_{k\in K}-\otimes_RB_k$;最后需要说明短正合列总诱导出连接映射自然的长正合列:任取短正合列$0\to A'\to A\to A''\to0$,按照$\mathrm{Tor}_n^R$的同调$\delta$函子,它诱导了长正合列.把长正合列对$k\in K$取直和,就得到长正合列$\oplus_k\mathrm{Tor}_{n+1}(A,P_k)\to\oplus_k\mathrm{Tor}_{n+1}(A,B_k)\to\oplus_k\mathrm{Tor}_n(A,N_k)\to\oplus_k\mathrm{Tor}_n(A,P_k)$.这里连接映射是以$K$为指标集的自然的连接映射的直和,因而必然自然.这就完成证明.
\end{proof}

具体例子,交换群的$\mathrm{Tor}_1^{\mathbb{Z}}$.首先对任意交换群$G$有$\mathrm{Tor}_1^{\mathbb{Z}}(\mathbb{Z}/n,B)\cong B[n]=\{b\in B\mid nb=0\}$.据此可以计算当$A,B$均为有限生成交换群时的$\mathrm{Tor}_1^{\mathbb{Z}}(A,B)$:先按照上一定理,直和与$\mathrm{Tor}$可交换,因而只需计算$A$为循环群的情况,当$A=\mathbb{Z}$时它也就是平坦模,于是此时$\mathrm{Tor}_1^{\mathbb{Z}}(A,B)=0$,而当$A$是有限循环群时只要运用本段的定理.
\begin{proof}
	
	先取短正合列$\xymatrix{0\ar[r]&\mathbb{Z}\ar[r]^{\mu_n}&\mathbb{Z}\ar[r]&\mathbb{Z}/n\ar[r]&0}$.它诱导了长正合列$\mathrm{Tor}_1^{\mathbb{Z}}(\mathbb{Z},B)\to\mathrm{Tor}_1^{\mathbb{Z}}(\mathbb{Z}/n,B)\to\mathbb{Z}\otimes_{\mathbb{Z}}B\to\mathbb{Z}\otimes_{\mathbb{Z}}B$.其中第一项为零,于是$\mathrm{Tor}_1^{\mathbb{Z}}(\mathbb{Z}/n,B)$即映射$\mu_n\otimes1_B$的核.而按照典范同构$\mathbb{Z}\otimes_{\mathbb{Z}}B\cong B$为$n\otimes b\mapsto nb$,得到这个核作为$B$的子群即$B[n]$,这就完成证明.注意本质上这个证明运用到了$R\otimes_RB\cong B$是一个自然同构,严格用交换图表证明即存在唯一的同构$\alpha$使得图表交换:
	$$\xymatrix{0\ar[r]&B[n]\ar[r]\ar[d]_{\alpha}&B\ar[r]^{\mu_n}\ar[d]_{\tau_B}&B\ar[d]_{\tau_B}\\0\ar[r]&\mathrm{Tor}_1(Z_n,B)\ar[r]&Z\otimes B\ar[r]_{1\otimes\mu_n}&Z\otimes B}$$
\end{proof}

$\mathrm{Tor}$函子与正向极限可交换.给定以有向集为指标集的左$R$模的正向系统$\{B_i,\phi_j^i\}$,那么对任意右$R$模$A$,对任意$n\ge0$,有自然同构$\mathrm{Tor} _n^R(A,\lim_{\rightarrow}B_i)\cong\lim_{\rightarrow}\mathrm{Tor}_n^R(A,B_i)$.
\begin{proof}
	
	和直和情况一样,这个证明同样可以套用同调$\delta$函子唯一延拓定理中的证明,即具体构造同构$\tau_n$.还可以用该定理推出,即证明$\lim_{\rightarrow}\mathrm{Tor}_n^R(-,B_i)$是延拓了右正合函子$-\otimes_R(\lim_{\rightarrow}B_i)$的同调$\delta$函子.
\end{proof}

整环上的$\mathrm{Tor}$函子.给定整环$R$,取商域为$Q$,取$K=Q/R$,那么$\mathrm{Tor}_1^R(K,-)$是自然同构于挠子模函子的,即把$R$模$A$映射为它的挠子模,把模同态映射为限制在挠子模的同态.另外恒有$\mathrm{Tor}_n^R(K,A)=0,\forall n\ge2$.这也是$\mathrm{Tor}$函子名字的由来.
\begin{enumerate}
	\item 先证明对挠模$A$恒有自然同构$\mathrm{Tor}_1^R(K,A)\cong A$.事实上考虑短正合列$0\to R\to Q\to K\to0$诱导的长正合列,有$\mathrm{Tor}_1^R(Q,A)\to\mathrm{Tor}_1^R(K,A)\to R\otimes_RA\to Q\otimes_RA$.按照$Q$是平坦$R$模得到第一项为零,按照$Q$是可除$R$模而$A$是挠模得到最后一项为零.于是有$\mathrm{Tor}_1^R(K,A)\cong R\otimes_RA\cong A$.最后注意到这个同构是连接映射,它的自然性是熟悉的.
	\item 对每个$R$模$A$,有$\mathrm{Tor}_n^R(K,A)=0,\forall n\ge2$.先从短正合列诱导出长正合列$\mathrm{Tor}_n^R(Q,A)\to\mathrm{Tor}_n^R(K,A)\to\mathrm{Tor}_{n-1}^R(R,A)$,从$R$和$Q$都是平坦$R$模得到第一项和最后一项都是零,于是有$\mathrm{Tor}_n^R(K,A)=0,\forall n\ge2$.
	\item 对每个无挠$R$模$A$恒有$\mathrm{Tor}_1^R(K,A)=0$.这个证明借助这样一个事实,对无挠模$A$,总有短正合列$0\to A\to V\to T\to0$,其中$V$是$Q$上线性空间(也即$R$上无挠可除模),而$T$是挠模.那么$V$是$R$上平坦模,诱导的长正合列得到$\mathrm{Tor}_2^R(K,T)\to\mathrm{Tor}_1^R(K,A)\to\mathrm{Tor}_1^R(K,V)$.第二条说明这里第一项为零,而$V$平坦得到最后一项为零,这就得到$\mathrm{Tor}_1^R(K,A)=0$.
	\item 最后证明本段的命题.即总有$\mathrm{Tor}_1^R(K,-)$自然同构于挠子模函子.
	\begin{proof}
	
	先任取$R$模$A$,得到短正合列$0\to tA\to A\to A/tA\to0$,其中$A/tA$是无挠模.这诱导了长正合列,于是得到$\mathrm{Tor}_2^R(K,A/tA)\to\mathrm{Tor}_1^R(K,tA)\to\mathrm{Tor}_1^R(K,A)\to\mathrm{Tor}_1^R(K,A/tA)$.第二条和第三条说明这里第一项和最后一项为零,这就得到同构$\mathrm{Tor}_1^R(K,A)\cong\mathrm{Tor}_1^R(K,tA)$.它是自然的同样因为这是连接映射.最后第一条说明有自然同构$\mathrm{Tor}_1^R(K,tA)\cong tA$,于是它们的复合得到自然同构$\mathrm{Tor}_1^R(K,A)\cong tA$.
	\end{proof}
\end{enumerate}

挠模的等价描述.设$R$是整环,设商域$Q$,那么$R$模$A$是挠模当且仅当$Q\otimes_RA=0$.
\begin{proof}
	
	考虑短正合列$0\to R\to Q\to K\to0$,其中$K=Q/R$.这诱导了长正合列$\mathrm{Tor}_1^R(Q,A)\to\mathrm{Tor}_1^R(K,A)\to R\otimes_RA\to Q\otimes_RA\to K\otimes_RA\to0$.按照$Q$是平坦$R$模得到第一项为零,这就得到一个正合列$0\to tA\to A\to Q\otimes_RA$.
	
	现在如果$A$是挠模,我们知道$Q\otimes_RA=0$.反过来如果这个张量积为零,上述正合列导致必然有$tA=A$,也即$A$是挠模.
\end{proof}

本段给出的性质同样可以作为$\mathrm{Tor}$函子名字的由来.给定整环$R$,对任意$R$模$A,B$以及任意$n\ge1$,恒有$\mathrm{Tor}_n^R(A,B)$是挠模.
\begin{proof}
	
	引理1.先证明$B$是挠模的时候$\mathrm{Tor}_n^R(A,B),n\ge0$总是挠模.首先$n=0$的情况即张量积,此时每个生成元$a\otimes b$都是挠元,于是$A\otimes_RB$是挠模.现在取短正合列$0\to N\to P\to A\to0$,其中$P$是投射模.考虑它诱导的长正合列.
	
	对$n=1$的情况,从$0=\mathrm{Tor}_1^R(P,B)\to\mathrm{Tor}_1^R(A,B)\to N\otimes_RB$,按照$N\otimes_RB$是挠模,于是子模$\mathrm{Tor}_1^R(A,B)$是挠模.
	
	现在假设$n$的情况得证,对$n+1$的情况,从$0=\mathrm{Tor}_{n+1}^R(P,B)\to\mathrm{Tor}_{n+1}^R(A,B)\to\mathrm{Tor}_n^R(N,B)\to\mathrm{Tor}_n^R(P,B)=0$得到中间两项同构,即得到$\mathrm{Tor}_{n+1}^R(A,B)$是挠模.
	
	引理2.我们断言正合列$\xymatrix{A\ar[r]^f&B\ar[r]^g&C}$满足$A,C$是挠模时候有$B$是挠模.为此取$K=\ker g=\mathrm{im}f$是$B$的子模,那么$K$是挠模,因为任取$a\in A$,则有非零元$r\in R$使得$ra=0$,导致$rf(a)=0$,于是$K=\mathrm{im}f$中每个元都是挠元,于是$K$是挠模.再考虑$B/K$,按照同构定理有$B/K=B/\ker g\cong\mathrm{im}g$,按照后者是挠模$C$的子模,于是它也是挠模.最后从$K$和$B/K$是挠模推出$B$是挠模:任取$b\in K$,那么存在非零元$r_1\in R$使得$r_1(b+K)=K$,即$r_1b=k\in K$,再结合$K$是挠模有非零元$r_2\in R$使得$r_2r_1b=0$,注意$R$是整环说明$r_2r_1\not=0$,于是$B$是挠模.
	
	引理3.我们证明$B$是无挠模的时候总有$\mathrm{Tor}_n^R(A,B),n\ge1$是挠模.可取短正合列$0\to B\to V\to Y\to0$,其中$V$是$Q$上线性空间,而$T=V/B$是挠模.这诱导了长正合列得到了$\mathrm{Tor}_{n+1}^R(A,T)\to\mathrm{Tor}_n^R(A,B)\to\mathrm{Tor}_n^R(A,V)$.按照$V$是平坦模得到这里最后一项是零,于是中间项是第一项的商,而引理1说明第一项是挠模,于是中间项是挠模.
	
	接下来证明原命题.设$B$是任意的$R$模.取短正合列$0\to tB\to B\to B/tB\to0$,这诱导了长正合列得到了$\mathrm{Tor}_n^R(A,tB)\to\mathrm{Tor}_m^R(A,B)\to\mathrm{Tor}_n^R(A,B/tB)$,其中$n\ge1$.按照引理1和引理3得到两侧的模都是挠模,引理2得到了$\mathrm{Tor}_1^R(A,B)$总是挠模.
\end{proof}

$\mathrm{Tor}$函子和分式化可交换.设$S$是交换环$R$的乘性闭子集,对任意$n\ge0$和任意$R$模$A,B$,有自然同构:$S^{-1}\mathrm{Tor}_n^R(A,B)\cong\mathrm{Tor}_n^{S^{-1}R}(S^{-1}A,S^{-1}B)$.这个结论本质上是因为正合函子与同调函子可交换,而$S^{-1}R\otimes_R-$是正合函子.它的对偶,$\mathrm{Ext}$函子和分式化未必可交换.
\begin{proof}
	
	固定$R$模$A$,首先$n=0$的情况即张量积和分式化的可交换性.现在取$B$的简化投射预解$P_B$,那么按照$S^{-1}R\otimes_R-$是正合加性函子,说明$S^{-1}(P_B)$是$S^{-1}B$上的简化投射预解,并且保投射模.于是这诱导了复形的同构$S^{-1}(A\otimes_RP_B)\cong S^{-1}A\otimes_{S^{-1}R}S^{-1}(P_B)$.于是它们诱导了相同的同调群,也即对每个$n\ge0$有$\mathrm{Tor}_n^{S^{-1}R}(S^{-1}A,S^{-1}B)=H_n(S^{-1}(A\otimes_RP_B))\cong S^{-1}H_n(A\otimes_RP_B)\cong S^{-1}\mathrm{Tor}_n^R(A,B)$.
\end{proof}

设$R$是交换诺特环,如果$A,B$均为有限生成$R$模,那么$\mathrm{Tor}_n^R(A,B)$对任意$n\ge0$总是一个有限生成$R$模.
\begin{proof}
	
	引理.设$R$是左诺特环,设$A$是有限生成左$R$模,那么存在$A$的投射预解$P=(P_n)$使得每个$P_n$都是有限生成模.事实上按照$A$是有限生成的,可取有限生成自由模$P_0$和满同态$\varepsilon:P_0\to A$.按照$R$是左诺特环,于是$\ker\varepsilon$是有限生成的,于是存在有限生成自由模$P_1$和满同态$e_1:P_1\to\ker\varepsilon$,取$d_1=\varepsilon\circ e_1:P_1\to P_0$,那么存在正合列$0\to\ker d_1\to P_1\to P_0\to A\to0$.归纳的构造$P_n$,这就得到$A$的一个每项都有限生成的自由预解.
	
	现在证明原命题.$n=0$的情况是张量积的基本性质.现在取$A$的每一项都有限生成的投射预解$P=(P_n)$.按照每个$P_n\otimes_RB$是有限生成的,于是子模$\ker(d_n\otimes1_B)$有限生成,于是商$H_n(P\otimes_R B)=\mathrm{Tor}_n^R(A,B)$是有限生成的.
\end{proof}
\subsubsection{Ext的性质}

$\mathrm{Ext}$函子与直和直积.这个结论和$\mathrm{Hom}$函子是一致的,这个双函子对二一个分量是共变的,此时会把积映射为极限;对第一个分量是逆变的,此时会把余积映射为对应的积:设$\{A_i\}$,$\{B_j\}$,$A,B$均为左$R$模,则对每个$n\ge0$都有如下自然同构,特别的,$\mathrm{Ext}$函子与有限直和可交换.
$$\mathrm{Ext}_R^n(\oplus_iA_i,B)\cong\prod_i\mathrm{Ext}_R^n(A_i,B)$$
$$\mathrm{Ext}_R^n(A,\prod_jB_j)\cong\prod_j\mathrm{Ext}_R^n(A,B_j)$$

$\mathrm{Ext}$函子和正向逆向极限都未必可交换,直和直积情况的证明也是没法模仿的,理由是投射模不保正向极限以及内射模不保逆向极限.

具体例子.交换群上的$\mathrm{Ext}_{\mathbb{Z}}^1$.首先对任意交换群$B$有$\mathrm{Ext}_{\mathbb{Z}}^1(\mathbb{Z}/n,B)\cong B/nB$.据此可计算当$A,B$均为有限生成交换群时的$\mathrm{Ext}_{\mathbb{Z}}^1(A,B)$:先按照与直和的可交换性,只需计算$A$是循环群的情况.如果$A=\mathbb{Z}$,它是投射交换群,于是此时$\mathrm{Ext}_{\mathbb{Z}}^1(A,B)=0$;如果$A$是有限循环群只需运用本段定理.

涉及到$\mathrm{Ext}$函子的另一个有趣的内容是模的延拓.给定左$R$模$A,C$,称$A$的经$C$的延拓是指一个短正合列$0\to A\to B\to C\to0$.特别的,如果这样的延拓是分离短正合列,就称为分离延拓.另外注意一下短正合列分离是比$B\cong A\oplus C$更强的条件:分离是指这个短正合列同构于$0\to A\to A\oplus C\to C\to0$.

关于延拓的第一件事是,如果$\mathrm{Ext}_R^1(C,A)=0$,则每个$A$的经$C$的延拓都是分离延拓.后文我们会证明这件事实际上是充要的,并且这实际上给出了$\mathrm{Ext}_R^1$的一个具体的描述.
\begin{proof}
	
	任取$A$的经$C$的延拓$0\to A\to B\to C\to0$,那么函子$\mathrm{Hom}_R(C,-)$诱导了长正合列,得到$\mathrm{Hom}_R(C,B)\to\mathrm{Hom}_R(C,C)\to\mathrm{Ext}_R^1(C,A)$.条件得到最后一项为零,导致$p:B\to C$诱导的$p_*$是满态射,于是存在$s:C\to B$使得$p\circ s=1_C$,即原短正合列分离.
\end{proof}

如果两个$A$经$C$的延拓都是分离延拓,那么它们是等价的.事实上给定两个模的延拓$0\to A\to B\to C\to0$和$0\to A\to B'\to C\to0$,按照短正合列分离的等价描述,存在如下交换图,于是二者是等价的:
$$\xymatrix{0\ar[r]&A\ar[r]\ar[d]_{1_A}&B\ar[r]\ar[d]&C\ar[r]\ar[d]^{1_C}&0\\\ar[r]&A\ar[r]^{i_A}\ar[d]_{1_A}&A\oplus C\ar[d]\ar[r]^{\pi_C}&C\ar[r]\ar[d]^{1_C}&0\\0\ar[r]&A\ar[r]&B'\ar[r]&C\ar[r]&0}$$

投射模与内射模关于$\mathrm{Ext}$的准则.
\begin{enumerate}
	\item 左$R$模$P$是投射模当且仅当对任意左$R$模$B$有$\mathrm{Ext}_R^1(P,B)=0$,当且仅当对任意左$R$模$B$和任意$n\ge1$有$\mathrm{Ext}_R^n(P,B)=0$.
	\item 左$R$模$E$是内射模当且仅当对任意左$R$模$A$有$\mathrm{Ext}_R^1(A,E)=0$,当且仅当对任意左$R$模$A$和任意$n\ge1$有$\mathrm{Ext}_R^n(A,E)=0$.
\end{enumerate}
\begin{proof}
	
	第一个命题.从$P$投射可推出另外两个命题.现在设对任意左$R$模$B$有$\mathrm{Ext}_R^1(P,B)=0$.于是上面定理说明每个短正合列$0\to B\to X\to P\to0$分离,特别的$P$是自由模的直和项,于是它是投射模.第二个命题是类似的,从$\mathrm{Ext}_R^1(A,E)=0$得到短正合列$0\to E\to B\to A\to0$总是分离的,于是$E$是内射模.
\end{proof}

给定两个左$R$模$A,C$,记全体$A$的经$C$的延拓的全体记作$S(C,A)$.在其上定义一个等价关系为,两个短正合列$\xi:0\to A\to B\to C\to0$和$\xi':0\to A\to B'\to C\to0$等价当且仅当存在$\phi:B\to B'$使得如下图表交换:
$$\xymatrix{0\ar[r]&A\ar[r]^i\ar[d]_{1_A}&B\ar[r]^p\ar[d]_{\phi}&C\ar[r]\ar[d]_{1_C}&0\\0\ar[r]&A\ar[r]_{i'}&B'\ar[r]_{p'}&C\ar[r]&0}$$

这里做两个注解.首先按照短五引理,如果两个短正合列等价,中间的映射$\phi$必然是同构,于是这个关系的确是一个等价关系.第二件事是,等价关系是非常依赖短正合列中的两个映射$i$和$p$的.即便两个短正合列都是$A$经$C$的延拓,并且中间项是同构的,也未必会有两个延拓等价.这里我们给出一个交换群的例子.取奇素数$n$,取$A$是$n$阶循环群$\langle a\rangle$,取$B=B'$为$n^2$阶循环群$\langle g\rangle$.取$i:A\to B$为$a\mapsto ng$,取$C=B/\mathrm{im}i$,取$p$是商的典范映射.再取$i'$为$a\mapsto 2ng$,于是$\mathrm{im}i=\mathrm{im}i'=\{0,ng,2ng,\cdots,(n-1)ng\}$,于是可取$p'=p$.这就构造了两个短正合列.但是如果存在$\phi:B\to B$使得图表交换,那么交换性迫使$\phi(ng)=2ng$.假设$\phi(g)=kg$,那么$\phi(ng)=kng$,于是$(k-2)n\equiv0(\mod n^2)$,导致$k=2+tn$,于是$\phi$为$g\mapsto(2+tn)g$.但是右侧的小方形交换得到$g-(2+tn)g\in\mathrm{im}i$,但这是不可能的.于是它们不是等价的延拓.

把$S(C,A)$关于上述等价关系的全体等价类构成的集合记作$e(C,A)$.我们接下来要做的是构造$e(C,A)$和$\mathrm{Ext}_R^1(C,A)$之间的比较直观的双射.
\begin{enumerate}
	\item 先来构造双射$\psi:e(C,A)\to\mathrm{Ext}_R^1(C,A)$.给定延拓$\xi:0\to A\to B\to C\to0$,取$C$的投射预解$P=(P_n,d_n)$,按照比较定义,存在如下交换图:
	$$\xymatrix{\ar[r]&P_2\ar[r]^{d_2}\ar[d]&P_1\ar[r]^{d_1}\ar[d]_{\alpha_1}&P_0\ar[r]\ar[d]_{\alpha_0}&C\ar[d]_{1_C}\ar[r]&0\\\ar[r]&0\ar[r]&A\ar[r]&B\ar[r]&C\ar[r]&0}$$
	
	这里的$\alpha_1:P_1\to A$满足$\alpha_1\circ d_2=0$,也即$\alpha_1\in\ker d_2^*$,就定义$\psi$为把$\xi$所在的等价类$[\xi]$映射为$\mathrm{cls}(\alpha_1)$.为证明$\psi$的定义良性,需要验证这个定义不依赖于延拓$1_C$的上述链映射的选取,也不依赖于等价类$[\xi]$中模的延拓的具体选取.
	
	第一件事,假设$(\alpha_n'):P\to\xi$是第二个提升了$1_C$的链映射,那么比较定理说明$(\alpha_n)$和$(\alpha_n')$是同伦的.设同伦为$(s_n)$,那么有$\alpha_1'-\alpha=s_0\circ d_1$,于是$\mathrm{cls}(\alpha_1)=\mathrm{cls}(\alpha_1')$.第二件事,考虑如下交换图,那么第二行与第三行的等价性得到它们提供了相同的$\alpha_1$.
	$$\xymatrix{\ar[r]&P_2\ar[r]^{d_2}\ar[d]&P_1\ar[r]^{d_1}\ar[d]_{\alpha_1}&P_0\ar[r]\ar[d]_{\alpha_0}&C\ar[d]_{1_C}\ar[r]&0\\\ar[r]&0\ar[d]\ar[r]&A\ar[d]_{1_A}\ar[r]&B\ar[d]_{\beta}\ar[r]&C\ar[d]_{1_C}\ar[r]&0\\\ar[r]&0\ar[r]&A\ar[r]&B'\ar[r]&C\ar[r]&0}$$
	\item 这一条证明,如果延拓$\xi$是分离的短正合列,那么它在$\psi$下的像是零.事实上考虑如下交换图,按照$\xi$是分离短正合列,存在同态$j:C\to B$使得$p\circ j=1_C$,于是此时的$\alpha_1=0$,于是$[\xi]$映射到了零元.
	$$\xymatrix{\ar[r]&P_2\ar[r]^{d_2}\ar[d]&P_1\ar[r]^{d_1}\ar[d]_0&P_0\ar[r]\ar[d]_{j\circ\varepsilon}&C\ar[d]_{1_C}\ar[r]&0\\\ar[r]&0\ar[r]&A\ar[r]&B\ar[r]_p&C\ar[r]&0}$$
	\item 我们接下来要证明$\psi$是双射.为此先证明如下引理:设$X_1$经$C$的一个延拓为$0\to X_1\to X_0\to C\to0$,给定同态$h:X_1\to A$,考虑如下交换图:
	$$\xymatrix{0\ar[r]&X_1\ar[r]^j\ar[d]_h&X_0\ar[r]^{\varepsilon}&C\ar[r]\ar[d]^{1_C}&0\\&A&&C&}$$
	
	那么诱导了如下交换图,其中两行均为短正合列,并且任意补全的第二行短正合列是等价的$A$经$C$的延拓.
	$$\xymatrix{0\ar[r]&X_1\ar[r]^j\ar[d]_h&X_0\ar[r]^{\varepsilon}\ar[d]^{\beta}&C\ar[r]\ar[d]^{1_C}&0\\0\ar[r]&A\ar[r]_i&B\ar[r]_{\eta}&C\ar[r]&0}$$
	\begin{proof}
		
		先证明存在性.取$B$为$j$和$h$的纤维积,即取$S=\{(hx_1,-jx_1)\mid x_1\in X_1\}$为$A\otimes X_0$的子模,然后定义$B=(A\otimes X_0)/S$.取$i:A\to B$为$a\mapsto(a,0)+S$和$\beta:X_0\to B$为$x_0\mapsto(0,x_0)+S$.那么$i$是单射并且左侧的方形交换.再取$\eta:B\to C$为$(a,x_0)+S\mapsto\varepsilon x_0$,验证这个定义良性,右侧的方形交换,并且下行是短正合列.这就证明了存在性.
		
		接下来证明作为模的延拓在等价意义下的唯一性.先取第二个补全的交换图为:
		$$\xymatrix{0\ar[r]&X_1\ar[r]^j\ar[d]_h&X_0\ar[r]^{\varepsilon}\ar[d]^{\beta'}&C\ar[r]\ar[d]^{1_C}&0\\0\ar[r]&A\ar[r]_{i'}&B'\ar[r]_{\eta'}&C\ar[r]&0}$$
		
		现在取同态$\theta:B\to B'$为$(a,x_0)+S\mapsto i'a+\beta'x_0$,得到如下图表交换,即二者作为模的延拓是等价的.
		$$\xymatrix{0\ar[r]&A\ar[r]^i\ar[d]_{1_A}&B\ar[r]^{\eta}\ar[d]^{\theta}&C\ar[r]\ar[d]^{1_C}&0\\0\ar[r]&A\ar[r]_{i'}&B'\ar[r]_{\eta'}&C\ar[r]&0}$$
	\end{proof}
    \item 上述引理存在对偶结论.任取$A$经$Y_1$的延拓为$0\to A\to Y_0\to Y_1\to0$,取同态$k:C\to Y_1$,那么它诱导了如下交换图,其中两行均为短正合列,并且任取补全的第一行短正合列是等价的$A$经$C$的延拓.
    $$\xymatrix{0\ar[r]&A\ar[r]\ar[d]_{1_A}&B\ar[d]\ar[r]&C\ar[d]^k\ar[r]&0\\0\ar[r]&A\ar[r]&Y_0\ar[r]_p&Y_1\ar[r]&0}$$
    \item 现在证明$\psi$是双射.为此我们来直接构造它的逆映射$\theta:\mathrm{Ext}_R^1(C,A)\to e(C,A)$.取$C$的投射预解$P=(P_n,d_n)$.选取一个1余圈$\alpha_1:P_1\to A$,也即$d_2^*(\alpha_1)=\alpha_1\circ d_2=0$.那么这诱导了一个同态$\alpha_1':P_1/\mathrm{im}d_2\to A$.现在考虑短正合列$0\to P_1/\mathrm{im}d_2\to P_0\to C\to0$,按照引理,存在如下交换图,其中两行都是短正合列:
    $$\xymatrix{0\ar[r]&P_1/\mathrm{im}d_2\ar[r]\ar[d]_{\alpha_1'}&P_0\ar[r]\ar[d]^{\beta}&C\ar[r]\ar[d]^{1_C}&0\\0\ar[r]&A\ar[r]_i&B\ar[r]&C\ar[r]&0}$$
    
    按照引理,补全的下行短正合列在延拓等价意义下唯一,就定义$\theta$把$\mathrm{cls}(\alpha_1)$映射为这个唯一的$e(C,A)$中的等价类.这个定义的良性还需要一些说明,即说明它不依赖于1余圈$\alpha_1$在同调类中的选取.假设$\alpha_1'$是另一个$\mathrm{cls}(\alpha_1)$中代表的选取,也即存在$s:P_0\to A$使得$\alpha_1'=\alpha_1+s\circ d_1$.那么有如下交换图,于是按照下行短正合列没有变化,说明$\alpha_1$和$\alpha_1'$诱导的$\theta$的取值是相同的,即定义良性.
    $$\xymatrix{P_2\ar[r]^{d_2}\ar[d]_0&P_1\ar[r]^{d_1}\ar[d]_{\alpha_1+sd_1}&P_0\ar[r]\ar[d]^{\beta+is}&C\ar[r]\ar[d]^{1_C}&0\\0\ar[r]&A\ar[r]_i&B\ar[r]&C\ar[r]&0}$$
    
    下面只需验证$\psi$和$\theta$互为逆映射.任取$\mathrm{cls}(\alpha_1)\in\mathrm{Ext}_R^1(C,A)$,那么$\theta(\mathrm{cls}(\alpha_1))$即如下交换图下行短正合列所在的$e(C,A)$中的等价类:
    $$\xymatrix{P_2\ar[r]^{d_0}\ar[d]&P_1\ar[r]\ar[d]_{\alpha_1}&P_0\ar[r]\ar[d]^{\alpha_0}&C\ar[r]\ar[d]^{1_C}&0\\0\ar[r]&A\ar[r]&B\ar[r]&C\ar[r]&0}$$
    
    这说明$\psi(\theta(\mathrm{cls}(\alpha_1)))=\mathrm{cls}(\alpha_1)$,即有$\psi\circ\theta=1$.另一方面,任取延拓$\xi:0\to A\to B\to C\to0$,它可以补全引理中以$0\to P_1/\mathrm{im}d_2\to P_0\to C\to0$和$\alpha_1':P_1/\mathrm{im}d_2\to A$的交换图,于是$\xi$落在$\theta\circ\psi([\xi])$中,也即$[\xi]=\theta\circ\psi([\xi])$,也即$\theta\circ\psi=1$.
    $$\xymatrix{0\ar[r]&P_1/\mathrm{im}d_2\ar[r]^{d_1'}\ar[d]_{\alpha_1'}&P_0\ar[r]\ar[d]&C\ar[r]\ar[d]^{1_C}&0\\0\ar[r]&A\ar[r]&B\ar[r]&C\ar[r]&0}$$
    \item 存在$e(C,A)$上的二元运算Baer和,使得$e(C,A)$上具备交换群结构,并且上述双射实际上是一个交换群同构.这个证明见Rotman同调代数428页.本质上讲,$\mathrm{Ext}$函子度量的是$\mathrm{Hom}$函子正合性的损失.如果我们只引入延拓等价类构成的集合$e(C,A)$以及其上的Baer和,同样可以引出$\mathrm{Ext}_R^1(C,A)$,并且这种操作回避了预解的选取,也完全不需要投射对象与内射对象的概念.这件事符合于Mac Lane的观点,即$\mathrm{Ext}$函子的定义应该按照公理化定义,而预解仅仅用于计算.随后Yoneda把$\mathrm{Ext}_R^1(C,A)$视为$e(C,A)$这个观点进行了推广,即$\mathrm{Ext}_R^n(C,A)$视为全体$0\to A\to B_1\to\cdots\to B_n\to C\to0$在某种特定的等价关系下的等价类集合.
\end{enumerate}

这里我们来证明上文中承诺的逆命题.每个$A$经$C$的延拓都是分离的当且仅当$\mathrm{Ext}_R^1(C,A)=0$.其中充分性已经得证.对于必要性,按照$\mathrm{Ext}_R^1(C,A)\cong e(C,A)$如果$S(C,A)$中每个延拓都是分离的,我们证明过分离的延拓都是两两等价的,这就说明$e(C,A)$是单元集合,于是有$\mathrm{Ext}_R^1(C,A)=0$.这个定理说明,$\mathrm{Ext}_R^1(C,A)$中的非零元描述的是非分离的$A$经$C$的延拓.

例子.我们介绍过交换群上$\mathrm{Ext}_R^1$的计算方法.特别的,对素数$p$有$\mathrm{Ext}_{\mathbb{Z}}^1(\mathbb{Z}/p,\mathbb{Z}/p)=\mathbb{Z}/p$,于是全体延拓$0\to\mathbb{Z}/p\to B\to\mathbb{Z}/p\to0$的等价类恰好有$p$个.但是按照$|B|=p^2$,按照交换群结构定理,同构意义下只能有$\mathbb{Z}/p^2$和$\mathbb{Z}/p\oplus\mathbb{Z}/p$两种情况.因而存在中间项同构的两个模的延拓,但是它们不是等价的延拓.

我们指出过$\mathrm{Ext}$函子和分式化一般不会交换.这里给出它们的一些结论,其中核心结论是,给定诺特交换环$R$,取乘性闭子集$S$,如果$A$是有限生成$R$模,那么总有$S^{-1}R$模的自然同构:
$$S^{-1}\mathrm{Ext}_R^n(A,B)\cong\mathrm{Ext}_{S^{-1}R}^n(S^{-1}A,S^{-1}B)$$
\begin{enumerate}
	\item 设$R$是诺特交换环,设$A,B$是有限生成$R$模,那么和$\mathrm{Tor}$的情况一样,诺特条件可以使得有限生成模取到分量均有限生成的投射预解和内射预解,进而得到每个$\mathrm{Ext}_R^n(A,B)$总是有限生成$R$模.
	\item 设$S$是交换环$R$的乘性闭子集,设$A,B$均为$R$模,并且$A$是有限表现模,那么存在$S^{-1}R$模的自然同构:
	$$\tau_{A,B}:S^{-1}\mathrm{Hom}_R(A,B)\cong\mathrm{Hom}_{S^{-1}R}(S^{-1}A,S^{-1}B)$$
	\begin{proof}
		
		我们要构造的$\tau_{A,B}$是如下两个自然同构的复合:
		$$\xymatrix{S^{-1}\mathrm{Hom}_R(A,B)\ar[r]^{\varphi}&\mathrm{Hom}_R(A,S^{-1}B)\ar[r]^{\theta_A}&\mathrm{Hom}_{S^{-1}R}(S^{-1}A,S^{-1}B)}$$
		
		先构造$\theta_A$.考虑自由$R$模$A=R^n$,选取一组基$\{a_1,a_2,\cdots,a_n\}$,那么$S^{-1}A$是以$\{a_1/1,a_2/1,\cdots,a_n/1\}$的自由$S^{-1}R$模.构造$\theta_{R^n}$为把$R$模同态$f:A\to S^{-1}B$映射为$S^{-1}R$模同态$\widetilde{f}$,满足$\widetilde{f}(a_i/s)=f(a_i)/s$.此时$\theta_{R^n}$是一个自然的$R$模同构.
		
		现在设$A$是有限表示$R$模,那么存在正合列$R^t\to R^n\to A\to0$,分别作用左正合函子$\mathrm{Hom}_R(-,B')$和$\mathrm{Hom}_{S^{-1}R}(-,B')$,其中$B'=S^{-1}B$.那么得到唯一的自然同构$\theta_A$使得如下图表交换.这里$\theta_A$具体写出来就是,如果$f:A\to B'$是$R$模同态,那么$\theta_A(f)=\widetilde{f}$为$a/s\mapsto f(a)/s$.容易验证$\theta_A$是自然的.
		$$\xymatrix{0\ar[r]&\mathrm{Hom}_R(A,B')\ar[r]\ar[d]_{\theta_A}&\mathrm{Hom}_R(R^n,B')\ar[r]\ar[d]_{\theta_{R^n}}&\mathrm{Hom}_R(R^t,B')\ar[d]_{\theta_{R^t}}\\0\ar[r]&\mathrm{Hom}_{R'}(A',B')\ar[r]&\mathrm{Hom}_{R'}((R')^n,B')\ar[r]&\mathrm{Hom}_{R'}((R')^t,B')}$$
		
		再构造$\varphi_A$为,设$f:A\to B$是$R$模同态,设$s\in S$,定义$\varphi_A$把$f/s$映射为$f_s:a\mapsto f(a)/s$.换句话讲$\varphi_A$就是$R$双线性映射$S^{-1}R\times\mathrm{Hom}_R(A,B)\to\mathrm{Hom}_R(A,S^{-1}B)$为$(r/s,f)\mapsto rf_s$所诱导的$R$模同态$S^{-1}\mathrm{Hom}_R(A,B)=S^{-1}R\otimes_R\mathrm{Hom}_R(A,B)\to\mathrm{Hom}_R(A,S^{-1}B)$.如果$A$是有限生成自由模那么$\varphi_A$是同构.现在对于一般的有限表示模$A$,选取正合列$R^t\to R^n\to A\to0$,有如下交换图表,按照局部化是正合函子,以及$Hom(-,B)$是左正合的,得到这两行都是正合列,按照$\varphi_{R^n}$和$\varphi_{R^t}$都是同构,得到$\varphi_A$是同构.容易验证$\varphi_A$的自然性.
		$$\xymatrix{0\ar[r]&S^{-1}\mathrm{Hom}_R(A,B)\ar[r]\ar[d]_{\varphi_A}&S^{-1}\mathrm{Hom}_R(R^n,B)\ar[r]\ar[d]_{\varphi_{R^n}}&S^{-1}\mathrm{Hom}_R(R^t,B)\ar[d]_{\varphi_{R^t}}\\0\ar[r]&\mathrm{Hom}_{R}(A,S^{-1}B)\ar[r]&\mathrm{Hom}_{R}(R^n,S^{-1}B)\ar[r]&\mathrm{Hom}_{R}(R^t,S^{-1}B)}$$
	\end{proof}
    \item 这里我们举例说明上一条中$A$未必有限表现的时候结论不成立.如果取$R=\mathbb{Z}$,取$S$是全体$\mathbb{Z}$中非零元构成的乘性闭子集,那么$S^{-1}\mathbb{Z}=\mathbb{Q}$.于是有如下不同构式,按照$\mathrm{Hom}_{\mathbb{Z}}(\mathbb{Q},\mathbb{Z})=0$说明左侧为零.但是按照$\mathbb{Q}\otimes_{\mathbb{Z}}\mathbb{Q}=S^{-1}\mathbb{Z}\otimes_{\mathbb{Z}}\mathbb{Q}=S^{-1}\mathbb{Q}=\mathbb{Q}$,和$\mathbb{Q}\otimes_{\mathbb{Z}}\mathbb{Z}=\mathbb{Q}$得到右侧为$\mathrm{Hom}_{\mathbb{Q}}(\mathbb{Q},\mathbb{Q})=\mathbb{Q}$.这就说明它们不同构.
    $$\mathbb{Q}\otimes_{\mathbb{Z}}\mathrm{Hom}_{\mathbb{Z}}(\mathbb{Q},\mathbb{Z})\not\cong\mathrm{Hom}_{\mathbb{Q}}(\mathbb{Q}\otimes_{\mathbb{Z}}\mathbb{Q},\mathbb{Q}\otimes_{\mathbb{Z}}\mathbb{Z})$$
    \item 给定诺特交换环$R$,取乘性闭子集$S$,如果$A$是有限生成$R$模,那么总有$S^{-1}R$模的自然同构:
    $$S^{-1}\mathrm{Ext}_R^n(A,B)\cong\mathrm{Ext}_{S^{-1}R}^n(S^{-1}A,S^{-1}B)$$
    \begin{proof}
    	
    	由于$R$诺特并且$A$是有限生成模,于是可取$A$的投射预解使得每个分量都是有限生成的投射模.按照诺特环上有限生成模必然是有限表现模,于是上述引理说明了存在自然同构$\tau_{A,B}:S^{-1}\mathrm{Hom}_R(A,B)\to\mathrm{Hom}_{S^{-1}R}(S^{-1}A,S^{-1}B)$.这诱导了复形的同构$S^{-1}(\mathrm{Hom}_R(P_A,B))\cong\mathrm{Hom}_{S^{-1}R}(S^{-1}(P_A),S^{-1}B)$.等式两侧取同调,右侧即$\mathrm{Ext}_{S^{-1}R}^n(S^{-1}A,S^{-1}B)$.而左侧按照分式化函子是正合函子,它和同调函子可交换,于是左侧取同调自然同构于先取同调再取分式化,即$S^{-1}\mathrm{Ext}_R^n(A,B)$.
    \end{proof}
    \item 推论.投射是一个局部性质.如果$R$是诺特交换环,$A$是它的有限生成模,那么$A$是投射$R$模当且仅当对每个极大理想$m$都有$A_m$是投射$R_m$模.
    \begin{proof}
    	
    	按照张量与直和可交换,得到投射模的局部化仍然是投射模.反过来对任意$A$模$B$和极大理想$m$有$\mathrm{Ext}_R^1(A,B)_m\cong\mathrm{Ext}_{R_m}^1(A_m,B_m)=\{0\}$,这就说明对任意$B$有$\mathrm{Ext}_R^1(A_m,B_m)=\{0\}$,于是$A$是投射模.
    \end{proof}
\end{enumerate}
\subsubsection{泛系数定理}

泛系数定理的四个版本.
\begin{enumerate}
	\item 同调版本1.设$R$是环,$A$是左$R$模,设$(\textbf{K},\mathrm{d})$是由平坦右$R$模构成的复形,设边界构成的子复形$(\textbf{B},\mathrm{d})$也都是平坦右$R$模.那么对每个$n\ge0$,都有如下短正合列.其中$\lambda_n:\mathrm{cls}(z)\otimes a\mapsto\mathrm{cls}(z\otimes a)$.并且这里$\lambda_n$和$\mu_n$都是自然的.
	$$\xymatrix{0\ar[r]&\mathrm{H}_n(\textbf{K})\otimes_RA\ar[r]^{\lambda_n}&\mathrm{H}_n(\textbf{K}\otimes_RA)\ar[r]^{\mu_n\quad}&\mathrm{Tor}_1^R(\mathrm{H}_{n-1}(\textbf{K}),A)\ar[r]&0}$$
	\begin{proof}
		
		我们有短正合列$\xymatrix{0\ar[r]&Z_n\ar[r]^{i_n}&K_n\ar[r]^{\mathrm{d}_n'}&B_{n-1}\ar[r]&0}$.按照$B_{n-1}$是平坦模得到$\mathrm{Tor}_1^R(B_{n-1},A)$,结合诱导的长正合列,得到如下正合列:
		$$\xymatrix{0\ar[r]&Z_n\otimes_RA\ar[r]^{i_n\times1}&K_n\otimes_RA\ar[r]^{\mathrm{d}_n'\otimes1}&B_{n-1}\otimes_RA\ar[r]&0}$$
		
		所以我们得到一个链复形的短正合列,其中$\textbf{B}[-1]$是指$\textbf{B}[-1]_n=B_{n-1}$:
		$$\xymatrix{0\ar[r]&\textbf{Z}\otimes_RA\ar[r]^{i\times1}&\textbf{K}\otimes_RA\ar[r]^{\mathrm{d}'\otimes1}&\textbf{B}[-1]\otimes_RA\ar[r]&0}$$
		
		这个复形的短正合列就诱导了同调群的长正合列:
		$$\xymatrix{\mathrm{H}_{n+1}(\textbf{B}[-1]\otimes_RA)\ar[r]^{\partial_{n+1}}&\mathrm{H}_n(\textbf{Z}\otimes_RA)\ar[r]^{(i_n\otimes1)_*}&\mathrm{H}_n(\textbf{K}\otimes_RA)\ar[r]&}$$
		$$\xymatrix{\mathrm{H}_n(\textbf{B}[-1]\otimes_RA)\ar[r]^{\partial_n}&\mathrm{H}_{n-1}(\textbf{Z}\otimes_RA)}$$
		
		这里$\textbf{Z}$和$\textbf{B}[-1]$的微分算子都是零.所以有$\mathrm{H}_{n+1}(\textbf{B}[-1]\otimes_RA)=B_n\otimes_RA$和$\mathrm{H}_n(\textbf{Z}\otimes_RA)=Z_n\otimes_RA$.所以上述长正合列为:
		$$\xymatrix{B_n\otimes_RA\ar[r]^{\partial_{n+1}}&Z_n\otimes_RA\ar[r]&\mathrm{H}_n(\textbf{K}\otimes_RA)\ar[r]&}$$
		$$\xymatrix{B_{n-1}\otimes_RA\ar[r]^{\partial_n}&Z_{n-1}\otimes_RA}$$
		
		这得到如下短正合列,其中$\alpha_n:z\otimes a+\mathrm{im}\partial_{n+1}\mapsto\mathrm{cls}(i_nz\otimes a)$.
		$$\xymatrix{0\ar[r]&\mathrm{coker}\partial_{n+1}\ar[r]^{\alpha_n}&\mathrm{H}_n(\textbf{K}\otimes_RA)\ar[r]^{\beta_n}&\ker\partial_n\ar[r]&0}$$
		
		连接映射$\partial_{n+1}$按照定义是按照如下交换图,对任意的$b\in B_n$,存在$k\in K_{n+1}$使得$b=\mathrm{d}_{n+1}k$.那么$\partial_{n+1}$为$b\otimes a\mapsto k\otimes a\mapsto b\otimes a\mapsto (i_n\otimes1)^{-1}(b\otimes a)=b\otimes a$.换句话讲$\partial_{n+1}=j_n\otimes1$,其中$j_n:B_n\subset Z_n$是包含映射.
		$$\xymatrix{&K_{n+1}\otimes_RA\ar[r]^{\mathrm{d}'_{n+1}\otimes1}\ar[d]^{\mathrm{d}_{n+1}\otimes1}&B_n\otimes_RA\\Z_n\otimes_RA\ar[r]^{i_n\otimes1}&K_n\otimes_RA&}$$
		
		于是之前的短正合列就变为如下形式,那么$\alpha_n:\mathrm{cls}(z\otimes a)+\mathrm{im}(j_n\otimes1)\mapsto\mathrm{cls}(i_nz\otimes a)$.容易验证$\alpha_n$和$\beta_n$都是自然的.
		$$\xymatrix{0\ar[r]&\mathrm{coker}(j_n\otimes1)\ar[r]^{\alpha_n}&\mathrm{H}_n(\textbf{K}\otimes_RA)\ar[r]^{\beta_n}&\ker(j_{n-1}\otimes1)\ar[r]&0}$$
		
		按照$B_n,Z_n$都是平坦模,所以$0\to B_n\to Z_n\to\mathrm{H}_n(\textbf{K})\to0$是$\mathrm{H}_n(\textbf{K})$的平坦预解,所以$0\to B_n\to Z_n\to0$就是$\mathrm{H}_n(\textbf{K})$的简化平坦预解,我们解释过$\mathrm{Tor}$可以经平坦预解计算,所以张量$A$再取同调就得到$\mathrm{H}_1=\ker(j_{n-1}\otimes1)=\mathrm{Tor}_1^R(\mathrm{H}_{n-1}(\textbf{K}),A)$和$\mathrm{H}_0=\mathrm{coker}(j_n\otimes1)=\mathrm{Tor}_0^R(\mathrm{H}_n(\textbf{K}),A)=\mathrm{H}_n(\textbf{K})\otimes_RA$.带入上述短正合列就得到:
		$$\xymatrix{0\ar[r]&\mathrm{H}_n(\textbf{K})\otimes_RA\ar[r]&\mathrm{H}_n(\textbf{K}\otimes_RA)\ar[r]&\mathrm{Tor}_1^R(\mathrm{H}_{n-1}(\textbf{K},A))\ar[r]&0}$$
		
		这里单射$\lambda_n:\mathrm{H}_n(\textbf{K})\otimes_RA\to\mathrm{H}_n(\textbf{K}\otimes_RA)$为$\mathrm{cls}(z)\otimes a\mapsto\mathrm{cls}(z\otimes a)$.
	\end{proof}
    \item 同调版本2.如果$R$是右遗传环,设$A$是左$R$模,设$(\textbf{K},\mathrm{d})$是投射右$R$模构成的复形,那么对每个$n\ge0$都有如下分裂的短正合列,其中$\lambda_n$为$\mathrm{cls}(z)\otimes a\mapsto\mathrm{cls}(z\otimes a)$,并且$\lambda_n$和$\mu_n$都是自然的(这里分裂得到的中间项同构于直和不是自然的).
    $$\xymatrix{0\ar[r]&\mathrm{H}_n(\textbf{K})\otimes_RA\ar[r]^{\lambda_n}&\mathrm{H}_n(\textbf{K}\otimes_RA)\ar[r]^{\mu_n\quad}&\mathrm{Tor}_1^R(\mathrm{H}_{n-1}(\textbf{K}),A)\ar[r]&0}$$
    \begin{proof}
    	
    	对于右遗传环,投射右模的子模总是投射右模.所以按照版本1就得到依旧有这里的短正合列以及$\lambda_n$的表达式.下面只要证明这个短正合列是分裂的.
    	
    	首先$\xymatrix{0\ar[r]&Z_n\ar[r]^{i_n}&K_n\ar[r]^{\mathrm{d}_n'}&B_{n-1}\ar[r]&0}$总是分裂的,因为$B_{n-1}$是投射右模,这里$\mathrm{d}_n':K_n\to B_{n-1}$是$\mathrm{d}_n$缩小终端得到的映射.分裂短正合列张量任意模都是分裂短正合的,于是有分裂短正合列:
    	$$\xymatrix{0\ar[r]&Z_n\otimes A\ar[r]^{i_n\otimes1}&K_n\otimes A\ar[r]^{\mathrm{d}_n'\otimes1}&B_{n-1}\otimes A\ar[r]&0}$$
    	
    	有包含关系$\mathrm{im}(d_{n+1}\otimes1)\subset\mathrm{im}(i_n\otimes1)\subset\ker(\mathrm{d}_n\otimes1)\subset K_n\otimes A$.那么有$\mathrm{im}(i_n\otimes1)$是$\ker(\mathrm{d}_n\otimes1)$的直和项(这是因为如果$M=S\oplus T$和$S\subset N\subset M$,那么有$N=S\oplus(N\cap T)$).类似的有$\mathrm{im}(i_n\otimes1)/\mathrm{im}(\mathrm{d}_{n+1}\otimes1)$是$\ker(\mathrm{d}_n\otimes1)/\mathrm{im}(\mathrm{d}_{n+1}\otimes1)=\mathrm{H}_n(\textbf{K}\otimes_RA)$的直和项.现在有$\mathrm{d}_{n+1}=i_n\circ j_n\circ\mathrm{d}'_{n+1}$.其中$j_n$是包含映射$B_n\subset Z_n$.按照$\mathrm{im}(fg)=f(\mathrm{im}g)$就得到:
    	$$\mathrm{im}(\mathrm{d}_{n+1}\otimes1)=(i_n\otimes1)\mathrm{im}(j_n\circ\mathrm{d}'_{n+1}\otimes1)=(i_n\otimes1)\circ(j_n\otimes1)\mathrm{im}(\mathrm{d}'_{n+1}\otimes1)$$
    	
    	按照$\mathrm{d}_{n+1}'$是满射,并且$-\otimes_RA$是右正合的,得到$\mathrm{im}(\mathrm{d}'_{n+1}\otimes1)=B_n\otimes_RA$.于是有$\mathrm{im}(i_n\otimes1)/\mathrm{im}(\mathrm{d}_{n+1}\otimes1)=Z_n\otimes_RA/\mathrm{im}(j_n\otimes1)=\mathrm{coker}(j_n\otimes1)=\mathrm{H}_n(\textbf{K})\otimes_RA$.所以有$\mathrm{H}_n(\textbf{K})\otimes_RA$是$\mathrm{H}_n(\textbf{K}\otimes_RA)$的直和项.这就说明命题中的短正合列是分裂的.
    \end{proof}
    \item 上同调版本1.设$R$是环,设$A$是左$R$模,设$(\textbf{K},\mathrm{d})$是由投射左$R$模构成的复形,并且边界构成的子复形$\textbf{B}$也都是投射模,那么对任意$n\ge0$有如下短正合列.并且这里$\lambda_n$和$\mu_n$都是自然的.
    $$\xymatrix{0\ar[r]&\mathrm{Ext}^1_R(\mathrm{H}_{n-1}(\textbf{K}),A)\ar[r]^{\lambda_n}&\mathrm{H}^n(\mathrm{Hom}_R(\textbf{K},A))\ar[r]^{\mu_n}&\mathrm{Hom}_R(\mathrm{H}_n(\textbf{K}),A)\ar[r]&0}$$
    \item 上同调版本2.设$R$是左遗传环,设$A$是左$R$模,设$(\textbf{K},\mathrm{d})$是由投射左$R$模构成的复形,那么对每个$n\ge0$都有如下分裂的短正合列(这里分裂不是自然的).
    $$\xymatrix{0\ar[r]&\mathrm{Ext}^1_R(\mathrm{H}_{n-1}(\textbf{K}),A)\ar[r]^{\lambda_n}&\mathrm{H}^n(\mathrm{Hom}_R(\textbf{K},A))\ar[r]^{\mu_n}&\mathrm{Hom}_R(\mathrm{H}_n(\textbf{K}),A)\ar[r]&0}$$
\end{enumerate}

在代数拓扑中的运用.
\begin{enumerate}
	\item 设$\mathrm{H}_n(X)$是空间$X$上的同调群.设$A$是一个阿贝尔群,定义$X$的系数在$A$中的同调群为$\mathrm{H}_n(X,A)=\mathrm{H}_n(\mathrm{S}_*(X)\otimes_{\mathbb{Z}}A)$,定义$X$的系数在$A$中的上同调群为$\mathrm{H}^n(X,A)=\mathrm{H}_{-n}(\mathrm{Hom}_{\mathbb{Z}}(\mathrm{S}_*(X),A))$.
	\item 设$X$是拓扑空间$A$是阿贝尔群,那么对每个$n\ge0$都有:
	$$\mathrm{H}_n(X,A)\cong\left(\mathrm{H}_n(X)\otimes_{\mathbb{Z}}A\right)\oplus\left(\mathrm{Tor}_1^{\mathbb{Z}}(\mathrm{H}_{n-1}(X),A)\right)$$
	\begin{proof}
		
		因为这里$\mathrm{S}_*(X)$的项都是自由阿贝尔群,并且$\mathbb{Z}$是遗传环.
	\end{proof}
    \item 推论.如果$\mathrm{H}_{n-1}(X)$或者$A$是无挠阿贝尔群,那么$\mathrm{Tor}_1^{\mathbb{Z}}(\mathrm{H}_{n-1}(X),A)=0$,于是有:
    $$\mathrm{H}_n(X,A)\cong\mathrm{H}_n(X)\otimes_{\mathbb{Z}}A$$
    \item 如果$X$是拓扑空间,$A$是阿贝尔群,那么对每个$n\ge0$都有同构:
    $$\mathrm{H}^(X,A)\cong\mathrm{Hom}_{\mathbb{Z}}(\mathrm{H}_n(X),A)\oplus\mathrm{Ext}_{\mathbb{Z}}^1(\mathrm{H}_{n-1}(X),A)$$
    \item 推论.如果$\mathrm{H}_{n-1}(X)$是自由的或者$A$是可除交换群,那么$\mathrm{Ext}_{\mathbb{Z}}^1(\mathrm{H}_{n-1}(X),A)=0$,于是有:
    $$\mathrm{H}^n(X,A)\cong\mathrm{Hom}_{\mathbb{Z}}(\mathrm{H}_n(X),A)$$
\end{enumerate}
\newpage
\subsection{同调维数}

投射等价和syzygy.
\begin{enumerate}
	\item 投射等价.给定两个$A$模$M$和$N$,称它们是投射等价的,如果存在投射$A$模$P$和$Q$,使得$A\oplus P\cong B\oplus Q$.容易验证投射等价是$A$模上的一个等价关系,记$A$模$M$所在的等价类为$[M]$.
	\item 在全体等价类构成的集合$G$上定义一个交换的二元运算$[M]+[N]=[M\oplus N]$,按照投射模的基本性质,一个等价类$[M]$可逆当且仅当$[M]$是加法幺元,当且仅当$M$是投射模.
	\item Scanuel引理.设$M$是一个左$A$模,
	\begin{enumerate}
		\item 假设有如下两个短正合列,其中$P,Q$都是投射模,那么有$K\oplus Q\cong P\oplus L$,特别的,$K$和$L$是投射等价的.于是给定一个模$M$,从某个投射模到$M$的满射的核在投射等价意义下是唯一的.
		$$\xymatrix{0\ar[r]&K\ar[r]&P\ar[r]&M\ar[r]&0}$$
		$$\xymatrix{0\ar[r]&L\ar[r]&Q\ar[r]&M\ar[r]&0}$$
		\item 假设有如下两个长正合列,其中全体$P_i$和$Q_i$都是投射模,那么有同构式$K\oplus Q_{n-1}\oplus P_{n-2}\oplus\cdots\cong L\oplus P_{n-1}\oplus Q_{n-2}\oplus\cdots$.
		$$\xymatrix{0\ar[r]&K\ar[r]&P_{n-1}\ar[r]&\cdots\ar[r]&P_1\ar[r]&P_0\ar[r]&M\ar[r]&0}$$
		$$\xymatrix{0\ar[r]&L\ar[r]&Q_{n-1}\ar[r]&\cdots\ar[r]&Q_1\ar[r]&Q_0\ar[r]&M\ar[r]&0}$$
	\end{enumerate}
    \item 上一条的引理说明,尽管给定$A$模$M$可取不同的投射预解$(P_n,d_n)$,记$\ker d_n=K_n$,称为第$n$个syzygy,那么固定$n$的时候$K_n$在投射等价意义下是唯一的.
    \item 投射等价和Ext函子.如果$A$模$M$和$M'$是投射等价的,那么总有$\mathrm{Ext}_R^n(M,N)\cong\mathrm{Ext}_R^n(M',N)$.
\end{enumerate}

投射维数.给定左$A$模$M$,称它的投射维数不超过$n$,如果它满足如下互相等价的条件中的任意一个,此时记作$\mathrm{pd}_A(M)\le n$.如果对每个自然数$n$命题都不成立,就称$M$的投射维数是无穷,记作$\mathrm{pd}_A(M)=\infty$.
\begin{enumerate}
	\item 存在模$M$的长度为$n$的投射预解,即有如下正合列,其中$P_i$都是投射$A$模:
	$$\xymatrix{0\ar[r]&P_n\ar[r]&\cdots\ar[r]&P_1\ar[r]&P_0\ar[r]&M\ar[r]&0}$$
	\item $\mathrm{Ext}_A^i(M,N)=0$对任意$i\ge n+1$和任意左$A$模$N$成立.
	\item $\mathrm{Ext}_A^{n+1}(M,N)=0$对任意左$A$模$N$成立.
	\item 对$M$的任意投射预解,第$n-1$个syzygy总是投射的.我们解释过对于不同投射预解,固定$i$的时候第$i$个syzygy是投射等价的,于是这一条也等价于把"任意"改成"存在".
\end{enumerate}
\begin{proof}
	
	4推1推2推3是平凡的,下面仅需验证3推4.任取$M$的一个投射预解,记第$i$个syzygy为$K_i$,那么我们证明过$\mathrm{Ext}_A^{n+1}(M,N)=\mathrm{Ext}_A^1(K_{n-1},N)$,于是条件3说明$K_{n-1}$是投射模.
\end{proof}

对偶的,Schanuel引理对内射情况同样成立,我们仍然可以定义内射等价和cosyzygy的联系,并且定义内射维数:给定左$A$模$M$,称它的内射维数不超过$n$,如果它满足如下互相等价的条件中的任意一个,此时记作$\mathrm{id}_A(M)\le n$.如果对每个自然数$n$命题都不成立,就称$M$的内射维数是无穷,记作$\mathrm{id}_A(M)=\infty$.
\begin{enumerate}
	\item 存在模$M$的长度为$n$的内射预解,即有如下正合列,其中$E^i$都是投射$A$模:
	$$\xymatrix{0\ar[r]&M\ar[r]&E^0\ar[r]&E^1\ar[r]&\cdots\ar[r]&E^n\ar[r]&0}$$
	\item $\mathrm{Ext}_A^i(N,M)=0$对任意$i\ge n+1$和任意左$A$模$N$成立.
	\item $\mathrm{Ext}_A^{n+1}(N,M)=0$对任意左$A$模$N$成立.
	\item 对$M$的任意内射预解,第$n-1$个cosyzygy总是内射的.和投射情况一样,对于不同内射预解,固定$i$的时候第$i$个cosyzygy是内射等价的,于是这一条也等价于把"任意"改成"存在".
	\item 按照内射维数和投射维数都可以用Ext函子来描述,容易验证当$M$遍历左(右)$A$模时,$\mathrm{id}_A(M)$的上确界恰好是左(右)整体维数.换句话讲,按照投射维数和内射维数的上确界定义出来的左右整体维数是相同的.
\end{enumerate}

整体维数.当$M$取遍左$A$模时,$\mathrm{pd}_A(M)$的上确界恰好是使得$\mathrm{Ext}_A^{n+1}(M,N)=0$对任意$A$模$M,N$成立的最小自然数$n$.这也等价于是$N$取遍左$A$模时$\mathrm{id}_A(N)$的上确界.(左)投射维数和(左)内射维数的这个公共上确界称为环$A$的(左)整体维数,注意整体维数可以取到无穷.环的左右整体维数记作$\mathrm{lD}(A)$和$\mathrm{rD}(A)$.
\begin{enumerate}
	\item 按照定义环$A$的左整体维数$\le n$当且仅当$\mathrm{Ext}_A^{n+1}(M,N)=0$对任意左$A$模$M,N$成立,当且仅当$\mathrm{Ext}_A^k(M,N)=0$对任意左$A$模$M,N$和任意$k\ge n+1$成立.
	\item 环的左整体维数为0当且仅当右整体维数为0当且仅当环是半单环.事实上环的左整体维数为零等价于每个模都是投射模,我们证明过这是半单环的等价描述.
	\item 环的左(右)整体维数不超过1当且仅当环是左(右)遗传环.
	\begin{proof}
		
		必要性,设环$A$的左整体维数不超过1,任取环$A$的左理想$I$,考虑短正合列$0\to I\to R\to R/I\to0$,那么$I$作为第0个syzygy是投射的,于是$A$的左理想都是投射模,于是$A$是左遗传环.充分性,设$A$是左遗传环,任取$A$模$M$,取投射模$P$到$M$的满射的核为$K$,我们证明过左遗传环上投射模的子模投射,于是这里$K$是投射的,于是$\mathrm{pd}_A(M)\le1$,于是$A$的左整体维数不超过1.
	\end{proof}
	\item 尽管环的左整体维数为0等价于右整体维数为0,但是一般来讲环的左整体维数未必和右整体维数相同.事实上已经存在右遗传环不是左遗传环的例子,Kaplansky构造过一个环的右整体维数为1,但是左整体维数为2;Small构造过一个环的右整体维数为1,但是左整体维数为3.甚至对任意的$1\le m\le n\le\infty$,存在这样的环左整体维数是$n$而右整体维数是$m$.
	\item 整体维数有时会涉及到一些集合论困难.例如设环是可数个域的直积,那么它的整体维数$\le2$,取2当且仅当连续统假设成立.
	\item 整体维数的等价描述.给定环$A$,那么$A$的左(右)整体维数是$\mathrm{pd}_A(A/I)$的上确界,这里$I$取遍$A$的左(右)理想.这个结论说明当$M$跑遍有限生成$A$模时,$\mathrm{pd}_A(M)$的上确界已经是整体维数.
	\begin{proof}
		
		假设这里的上确界是无穷,那么自然有左整体维数是无穷.现在假设对每个左理想$I$都有$\mathrm{pd}_A(A/I)\le n$,于是对每个左理想$I$和每个左模$N$都有$\mathrm{Ext}_A^{n+1}(A/I,N)=0$.下面仅需验证对每个左$A$模$N$都有$\mathrm{id}_A(N)\le n$:任取$N$的内射预解式,记第$n-1$个cosyzygy为$V^{n-1}$,我们证明过$\mathrm{Ext}_A^{n+1}(A/I,N)=\mathrm{A/I,V^{n-1}}$,于是条件说明了$V^{n-1}$是内射的,于是$\mathrm{id}_A(N)\le n$.
	\end{proof}
    \item 给定一族左$A$模$\{M_i,i\in I\}$,那么$\mathrm{pd}(\oplus_iM_i)=\sup_i\mathrm{pd}_A(M_i)$.这个结论说明如果$\mathrm{lD}(A)=\infty$,那么必然存在一个左$A$模$M$使得$\mathrm{pd}_A(M)=\infty$.
\end{enumerate}

平坦维数.给定$A$模$M$的平坦预解$F_*\to M\to0$,称$F_n\to F_{n-1}$的核为这个预解的第$n$个syzygy,其中$n\ge1$.
\begin{enumerate}
	\item Schanuel引理对平坦预解是不成立的,考虑交换群的两个短正合列$0\to\mathbb{Z}\to\mathbb{Q}\to\mathbb{Q}/\mathbb{Z}\to0$和$0\to S\to F\to\mathbb{Q}/\mathbb{Z}\to0$.其中$F$是一个自由交换群,它的子群$S$还是无挠的.我们解释过交换群是平坦的当且仅当无挠的.于是这两个短正合列都是$\mathbb{Q}/\mathbb{Z}$的平坦预解.但是不能有$\mathbb{Z}\oplus F\cong\mathbb{Q}\oplus S$,因为左侧是一个自由交换群,它的直和项理应是投射交换群,但是$\mathbb{Q}$不是投射的.
	\item 给定右$A$模$M$,称它的平坦维数$\mathrm{fd}_A(M)$不超过$n$,如果它满足如下等价条件中的任意一个:
	\begin{enumerate}
		\item 存在$M$的长度为$n$的平坦预解,换句话讲存在平坦预解$F_*\to M\to0$满足$k\ge n+1$时$F_k=0$.
		\item $\mathrm{Tor}_k^A(M,N)=0$对任意左$A$模$N$和任意$k\ge n+1$成立.
		\item $\mathrm{Tor}_{n+1}^A(M,N)=0$对任意左$A$模$N$成立.
		\item $A$的每个平坦预解的第$n-1$个syzygy都是平坦的.
	\end{enumerate}
    \begin{proof}
    	
    	1推2因为我们解释过计算$\mathrm{Tor}$函子可以用平坦预解,这里只要取1中给定的平坦预解即可.2推3是平凡的.3推4因为任取一个$n-1$次的syzygy为$Y_{n-1}$,那么我们解释过$\mathrm{Tor}_1^A(Y_{n-1},N)=\mathrm{Tor}_{n+1}^A(M,N)=0$,导致$Y_{n-1}$是平坦的.最后4推1因为任取平坦预解$F_*\to M\to0$,那么$0\to T_{n-1}\to F_{n-1}\to\cdots\to F_0\to M\to0$也是$M$的一个平坦预解.
    \end{proof}
\end{enumerate}

弱同调维数.考虑所有右$A$模的平坦维数的上确界,它就是最大的自然数$n$使得$\mathrm{Tor}_{n+1}^A(M,N)=0$对任意左$A$模$N$和任意右$A$模$M$成立.于是这也是所有左$A$模的平坦维数的上确界.这个公共的上确界称为环$A$的弱同调维数,记作$\mathrm{wD}(A)$.
\begin{enumerate}
	\item $\mathrm{wD}(A)=0$当且仅当$A$的所有模都是平坦模,我们解释过这样的环是冯诺依曼正则环.弱同调维数不超过1等价于讲平坦$A$模的子模都是平坦的,对于整环情况,这个条件等价于讲它是Pr\"ufer环.
	\item 按照平坦预解也是投射预解,一个右/左$A$模$M$的平坦维数总不超过投射维数.于是特别的环的弱同调维数总不超过左和右整体维数.弱维数可能不是整体维数,比方说取一个不是半单环的冯诺依曼正则环.
	\item 按照$\mathrm{Tor}$函子和分式化可交换,得到分式化会让弱维数变小.即如果$S$是交换环$A$的乘性闭子集,那么$\mathrm{wD}(S^{-1}A)\le\mathrm{wD}(A)$.
	\item 弱维数$\mathrm{wD}(A)$恰好是当$I$跑遍左理想时$A/I$的左平坦维数的上确界,也恰好是当$I$跑遍右理想时$A/I$的右平坦维数的上确界.
	\item 设$A$是右诺特环,设$M$是有限生成右$A$模,那么$M$的平坦维数恰好是投射维数.另外我们解释过整体维数和弱维数都可以视为商模$A/I$的投射维数和平坦维数的上确界,这里$A/I$总是有限生成模,于是得到右诺特环的右整体维数恰好是弱维数.于是对于同时是左诺特和右诺特的环,它的左右整体维数和弱维数都是相同的.
	\begin{proof}
		
		我们已经证明过一般情况下弱维数不超过右整体维数.现在不妨设$A$的弱维数是$n<\infty$,否则没什么需要证的.此时可取$M$的投射预解$P_*\to M\to0$使得每个$P_n$都是有限生成的.这也是一个平坦预解,于是它的第$n-1$个syzygy(记作$Y_{n-1}$)是平坦的,按照诺特环上有限生成平坦模是投射的,得到$Y_{n-1}$是投射模,于是按照定义$M$的投射维数不超过$n$.
	\end{proof}
\end{enumerate}

同调维数与短正合列.给定左$A$模的短正合列$0\to M_1\to M_2\to M_3\to0$.下面这些结论几乎都可以用长正合列定理得到.
\begin{enumerate}
	\item 如果这三个模$M_1,M_2,M_3$中有两个模的投射维数是有限的,那么第三个模的投射维数也是有限的.这个命题对内射维数同样成立.
	\item 
	\begin{itemize}
		\item 如果$\mathrm{pd}_A(M_1)<\mathrm{pd}_A(M_2)$,那么$\mathrm{pd}_A(M_3)=\mathrm{pd}_A(M_2)$.
		\item 如果$\mathrm{pd}_A(M_1)>\mathrm{pd}_A(M_2)$,那么$\mathrm{pd}_A(M_3)=\mathrm{pd}_A(M_1)+1$.
		\item 如果$\mathrm{pd}_A(M_1)=\mathrm{pd}_A(M_2)$,那么$\mathrm{pd}_A(M_3)\le\mathrm{pd}_A(M_1)+1$.
	\end{itemize}
\end{enumerate}

我们接下来要证明希尔伯特syzygy定理:给定环$A$,设$A[x]$是它的一元多项式环,那么$\mathrm{lD}(A[x])=\mathrm{lD}(A)+1$.为此先证明一侧的不等式:$\mathrm{lD}(A[x])\le1+\mathrm{lD}(A)$.
\begin{enumerate}
	\item 设$M$是左$A$模,记$M[x]=A[x]\otimes_AM$,这可以视为$M$系数的多项式构成的$A[x]$上的模.对每个左$A$模$M$都有$\mathrm{pd}_A(M)=\mathrm{pd}_{A[x]}(M[x])$.
	\begin{proof}
		
		如果$\mathrm{pd}_A(M)\le n$,那么有长度$n$的投射预解$0\to P_n\to\cdots\to P_0\to M\to0$.按照$A[x]$是自由右$A$模,它也是平坦的,于是张量后得到正合列$0\to A[x]\otimes_AP_n\to\cdots\to A[x]\otimes_AM\to0$.这里每个$A[x]\otimes_AP_i$都是投射$A[x]$模,于是这说明$\mathrm{pd}_{A[x]}(M[x])\le n$.
		
		如果$\mathrm{pd}_{A[x]}(M[x])\le n$,那么有$A[x]$模的投射预解$0\to Q_n\to\cdots\to Q_0\to M[x]\to0$.这里每个$Q_i$都是某个$A[x]$自由模$F_i$的直和项,可记$F_i=Q_i\oplus S_i$,那么这里$F_i$视为$A$模时也是自由模.于是把这个正合列视为$A$模的正合列时是$A$模$M[x]\cong M^{\oplus\mathbb{N}}$的投射预解.一般的给定$A$模族$\{M_i,i\in I\}$,有$\mathrm{pd}_A(\oplus_iM_i)=\sup_{i\in I}\mathrm{pd}_A(M_i)$,这说明$\mathrm{pd}_A(M)=\mathrm{pd}_A(M[x])\ge n$.
	\end{proof}
    \item 如果$\mathrm{lD}(A)=\infty$,那么$\mathrm{lD}(A[x])=\infty$.
    \begin{proof}
    	
    	按照一族模的直和的投射维数等于分量投射维数的上确界,说明当$\mathrm{lD}(A)=\infty$时,存在$A$模$M$的投射维数是无穷,于是上一条得到$M[x]$是投射维数无穷的$A[x]$模,这说明$\mathrm{lD}(A[x])=\infty$.
    \end{proof}
    \item 对每个环$A$,有$\mathrm{lD}(A[x])\le1+\mathrm{lD}(A)$.
    \begin{proof}
    	
    	首先无穷情况我们已经解释了这个不等式取等号.现在设$\mathrm{lD}(A)$是有限数$n$,任取$A[x]$模$M$,把它视为$A$模,我们证明过$\mathrm{pd}_{A[x]}(M[x])=\mathrm{pd}_A(M)\le n$.现在我们断言有短正合列$0\to M[x]\to M[x]\to M\to0$.于是得到$\mathrm{pd}_{A[x]}(M)\le1+\mathrm{pd}_{A[x]}(M[x])\le 1+\mathrm{pd}_A(M)$.
    	
    	这里$\phi:M[x]\to M$定义为$x^i\otimes m\mapsto x^im$.首先这明显是一个满同态,记$\ker\phi=K$,只需验证$K\cong M[x]$.构造$f:\sum_ix^i\otimes m_i\mapsto\sum_ix^i(1\otimes x-x\otimes1)m_i$验证是同构即可.
    \end{proof}
\end{enumerate}

为了证明另一侧的不等式,我们来引入正则元的概念,并考虑模去正则元后维数的变化.设$A$是环,设$M$是$A$模,取$x\in Z(A)$,称它是$M$上的正则元,如果数乘$x$这个映射是$M$上的单射,换句话讲从$xm=0$推出$m=0$.特别的,$x\in A$是$A$上的正则元当且仅当它在中心并且它是非零因子.
\begin{enumerate}
	\item Rees定理.设$A$是环,设$x\in Z(A)$是$A$上的非单位非零因子,记$A^*=A/xA$,如果$M$是$A$模使得$x$是$M$上的正则元,那么对每个$A^*$模$N^*$和每个$n\ge0$都有:
	$$\mathrm{Ext}_{A^*}^n(N^*,M/xM)\cong\mathrm{Ext}_A^{n+1}(N^*,M)$$
	\item 设$A$是环,取$x\in Z(A)$是$A$的非单位非零因子元,记$A^*=A/xA$,如果$\mathrm{lD}(A^*)=n<\infty$,那么$\mathrm{lD}(A)\ge\mathrm{lD}(A^*)+1$.
	\begin{proof}
		
		设$N^*$是一个左$A^*$模满足$\mathrm{pd}_{A^*}(N^*)=n$,那么我们解释过存在左自由$A^*$模$F^*$使得$\mathrm{Ext}_{A^*}^n(N^*,F^*)\not=0$,设$Q$是左自由$A$模使得它和$F^*$具有相同的基,并且满足$F^*\cong Q/xQ$,那么按照上一条得到$\mathrm{Ext}_{A^*}^n(N^*,F^*)\cong\mathrm{Ext}_A^{n+1}(N^*,Q)$.于是$\mathrm{pd}_A(N^*)\ge n+1$,于是有$\mathrm{lD}(A)\ge n+1=1+\mathrm{lD}(A^*)$.
	\end{proof}
    \item 希尔伯特syzygy定理.对环$A$,总有$\mathrm{lD}(A[x])=\mathrm{lD}(A)+1$.于是归纳得到$\mathrm{lD}(A[x_1,x_2,\cdots,x_n])=\mathrm{lD}(A)+n$.
    \begin{proof}
    	
    	我们之前已经证明了无限维情况等式成立,也证明了一侧的不等式$\mathrm{lD}(A[x])\le\mathrm{lD}(A)+1$.现在设$\mathrm{lD}(A)=n<\infty$.记$S=A[x]$,取$S^*=S/xS=A[x]/(x)\cong A$,上一条就得到另一侧的不等式$\mathrm{lD}(A[x])\ge\mathrm{lD}(A)+1$.
    \end{proof}
\end{enumerate}

正则元和投射维数.
\begin{enumerate}
	\item 设$A$是环,$x\in Z(A)$非零因子,设$M$是左$A$模,设$x$是$M$上正则元,记$A^*=A/(x)$.
	\begin{itemize}
		\item 如果$\mathrm{pd}_A(M)=n<\infty$,那么$\mathrm{pd}_{A^*}(M/xM)\le n-1$.
		\begin{proof}
			
			我们有$\mathrm{Ext}_A^{n+1}(N,M)=0$对任意左$A$模$N$成立.于是$\mathrm{Ext}_A^{n+1}(N^*,M)=0$对任意左$A^*$模$N^*$成立.于是按照Rees定理得到$\mathrm{Ext}_{A^*}^n(N^*,M/xM)=0$对任意$A^*$模$N^*$成立,于是$\mathrm{pd}_{A^*}(M/xM)\le n-1$.
		\end{proof}
	    \item 如果$\mathrm{pd}_{A^*}(M/xM)=n<\infty$,那么$\mathrm{pd}_A(M)\ge n+1$.这个证明同上,也是一步Rees定理.
	\end{itemize}
    \item 设$A$是环,$x\in Z(A)$非单位非零因子,记$A^*=A/(x)$,设$M^*$是左$A^*$模,如果$\mathrm{pd}_{A^*}(M^*)=n<\infty$,那么$\mathrm{pd}_A(M^*)=n+1$.
    \begin{proof}
    	
    	一个左$A^*$模恰好就是被$x$零化的左$A$模.我们来对$n\ge0$归纳.当$n=0$时$M^*$即投射左$A^*$模,于是它是某个$A^*$自由模$F^*$的直和项.按照$x$非零因子,得到短正合列$0\to A\to A\to A^*\to0$,这里$A\to A$取为左乘$x$这个映射.于是有$\mathrm{pd}_A(A^*)\le1$.按照投射维数和直和的关系,得到$\mathrm{pd}_A(M^*)\le1$.倘若$\mathrm{pd}_A(M^*)=0$,导致$M^*$是投射$A$模,于是它是某个自由$A$模的直和项,但是这样它理应不被非零因子$x$零化,这就矛盾,于是$\mathrm{pd}_A(M^*)=1$.
    	
    	现在设$n\ge1$,那么存在左$A^*$模的短正合列$0\to K^*\to F^*\to M^*\to0$,其中$F^*$是自由$A^*$模,$K^*$是满同态$F^*\to M^*$的核.此时按照短正合列上投射维数的关系,得到$\mathrm{pd}_{A^*}(M^*)=n-1$,于是归纳假设得到$\mathrm{pd}_A(M^*)=n$.
    	
    	如果$n=1$,按照短正合列上投射维数的关系,只能有$\mathrm{pd}_{A^*}(K^*)=1$和$\mathrm{pd}_A(K^*)\le2$.取$A$上自由模$F$使得有$A$模的短正合列$0\to L\to F\to M^*\to0$.按照$xM^*=0$,得到$xF\subset\ker(F\to M^*)=L$,于是这诱导了一个$A^*$模的短正合列$0\to L/xF\to F/xF\to M^*\to0$.这里$F/xF$是$A^*$自由模,导致$\mathrm{pd}_{A^*}(L/xF)=\mathrm{pd}_{A^*}(M^*)-1=0$.于是$L/xF$是投射$A^*$模,于是短正合列$0\to xF/xL\to L/xL\to L/xF\to0$分裂.而$M^*\cong F/L\cong xF/xL$,于是$M^*$是$L/xL$的直和项.倘若$L/xL$是投射左$A$模,那么$L/xL$是投射左$A^*$模,导致直和项$M^*$是投射$A^*$模和$n=1$矛盾,于是$L$不是投射$A$模,于是$\mathrm{pd}_A(M^*)=1+\mathrm{pd}_A(L)\ge2$.这就得到$\mathrm{pd}_A(M^*)=2$.
    	
    	最后假设$n\ge2$,我们解释了此时有$\mathrm{pd}_A(K^*)=n$,于是$A$模的短正合列$0\to K^*\to F^*\to M^*\to0$得到$\mathrm{pd}_A(M^*)=\mathrm{pd}_A(K^*)+1=n+1$,完成归纳.
    \end{proof}
\end{enumerate}

稳定自由模(stably free).一个$A$模$M$称为稳定自由模,如果它是有限生成的,并且存在有限生成自由模$F_1$和$F_2$使得$F_1\oplus M\cong F_2$.
\begin{enumerate}
	\item 有限生成自由模自然总是稳定自由模.
	\item 如果$M$和$N$都是稳定自由模,那么$M\oplus N$也是稳定自由模.但是一族无穷个稳定自由模的直和未必是稳定自由的.
	\item 稳定自由模总是投射模,因为它是自由模的直和项.但是反过来投射模未必是稳定自由模,例如$A=\mathbb{Z}/6$,有$A=\mathbb{Z}/2\oplus\mathbb{Z}/3$,但是$\mathbb{Z}/2$不会是稳定自由模,因为计算元素个数就得出矛盾.
	\item Eilenberg证明过每个投射模都有自由的直和补,于是定义中我们必须约定有限生成.
\end{enumerate}

称一个$A$模$M$具有长度$\le n$的有限自由预解(简称FFR,finite free resolution),如果存在自由预解$0\to F_n\to\cdots\to F_0\to M\to0$,使得每个$F_i$都是有限生成自由$A$模.
\begin{enumerate}
	\item 按照定义,稳定自由模必然是有限生成的,也是有限表示模.
	\item 一个有限生成左投射$A$模$P$具有FFR当且仅当它是稳定自由的,此时它具有长度$\le1$的FFR.
	\begin{proof}
		
		一方面如果$P$是稳定自由的,设有限生成自由模$F$使得$P\oplus F$也是有限生成自由模.于是此时$0\to F\to F\oplus P\to P\to0$是一个长度为1的FFR.另一方面如果$P$具有FFR,记作$0\to F_n\to\cdots\to F_0\to P\to0$,其中每个$F_i$是有限生成自由模.我们对$n\ge0$归纳证明$P$是稳定自由模.如果$n=0$,那么$P\cong F_0$本身是有限生成自由模,于是它是稳定自由的.设$n\ge1$,记$K$为这个长度为$n+1$的FFR的第0个syzygy,那么这个FFR可拆为两个正合列$0\to F_{n+1}\to\cdots\to F_1\to K\to0$和$0\to K\to F_0\to P\to0$.第一个正合列说明$K$具有长度为$n$的FFR.第二个短正合列说明$F_0\cong P\oplus K$,于是$K$是有限生成的投射模,于是按照归纳假设有$K$是稳定自由模,于是存在有限生成自由模$N$使得$K\oplus N$是有限生成自由模,导致$P\oplus(K\oplus N)\cong F_0\oplus N$是有限生成自由模,于是$P$是稳定自由的.
	\end{proof}
    \item 引理.设左$A$模$M$具有投射预解$0\to P_n\to\cdots\to P_0\to M\to0$,这里连接映射记作$d_n$,$P_0\to M$记作$\varepsilon$,满足每个$P_i$是稳定自由模,那么$M$具有长度$\le n+1$的FFR.
    \begin{proof}
    	
    	我们对$n\ge0$归纳.对于$n=0$,有$M\cong P_0$是稳定自由模,上一条得到它具有长度$\le1$的FFR.设$n\ge1$,按照$P_0$是稳定自由的,存在有限生成自由模$F$使得$P_0\oplus F$也是有限生成自由模.我们构造正合列:
    	$$\xymatrix{0\ar[r]&P_n\ar[r]&\cdots\ar[r]&P_2\ar[r]^{d_2'}&P_1\oplus F\ar[r]^{d_1\oplus 1_F}&P_0\oplus F\ar[r]^{\varepsilon'}&M\ar[r]&0}$$
    	
    	这里$d_2'$定义为$p_2\mapsto(d_2p_2,0)$,$\varepsilon'$定义为$(p_0,f)\mapsto\varepsilon(p_0)$.现在$\ker\varepsilon'$具有长度为$n-1$的稳定自由预解,归纳假设得到它具有长度为$n$的FFR.把这个FFR复合上$0\to\ker\varepsilon'\to P_0\oplus F\to M\to0$得到$M$的长度为$n+1$的FFR.
    \end{proof}
    \item 给定左诺特环$A$上左模的短正合列$0\to M_1\to M_2\to M_3\to0$,如果$M_1,M_2,M_3$中有两个模具有FFR,那么第三个模也具有FFR.
    \begin{proof}
    	
    	具有FFR的模是有限生成的,按照环是诺特环,短正合列中任意两个模如果是有限生成的,那么第三个必然也是有限生成的.按照Horseshoe引理,如果取$M_1$和$M_3$的自由预解为$F_*'\to M_1\to0$和$F_*''\to M_3\to0$,那么可取$M_2$的自由预解$F_*'\oplus F_*''\to M_2\to0$.于是如果$M_1$和$M_3$具有FFR,那么$M_2$具有FFR.
    	
    	现在设$M_2$和$M_3$具有FFR,记$M_3$具有长度不超过$n$的FFR,那么它的第$n$个syzygy(记作$K_n''$)是稳定自由的,【】
    \end{proof}
\end{enumerate}

我们来抽象化"具有FFR"这个性质,在证明下面Serre定理过程中要用到这些内容.给定一个左$A$模的族$\mathscr{S}$,满足对任意短正合列$0\to M_1\to M_2\to M_3\to0$,如果$M_1,M_2,M_3$中某两个落在$\mathscr{S}$中,那么第三个模也落在$\mathscr{S}$中.如果$A$是左诺特环,那么全部具有FFR的左$A$模满足这个性质.这个性质记作$(\ast)$.
\begin{enumerate}
	\item 如果$\{\mathscr{S}_i,i\in I\}$是一族满足上述条件的左$A$模族(模族的族),那么$\cap_i\mathscr{S}_i$也满足上述条件.于是我们可以定义一个模族$\mathscr{S}$生成的满足$(\ast)$的模族为全部包含这个模族的满足上述条件的模族的交.它称为这个模族生成的$(\ast)$模族,记作$\tau(\mathscr{S})$.
	\item 给定包含零模的模族$\mathscr{S}$,记$\sigma(\mathscr{S})$表示这样的模族,一个左$A$模在$\sigma(\mathscr{S})$中当且仅当存在一个包含这个模的短正合列,使得另外两个模在$\mathscr{S}$中.记$\sigma^{n+1}(\mathscr{S})=\sigma(\sigma^n(\mathscr{S}))$.我们断言这是递增的模族,因为从$0\to M\to M\to0\to0$得到当$\mathscr{S}$非空时有$0\in\sigma(\mathscr{S})$,而$M,0\in\mathscr{S}$保证了$M\in\sigma(\mathscr{S})$.这就说明了$\mathscr{S}\subset\sigma(\mathscr{S})$.另外容易验证$\cup_{n\ge0}\sigma^n(\mathscr{S})=\tau(\mathscr{S})$.
	\item 如果$A$是左诺特的,设模族$\mathscr{S}$包含零模,并且每个模都具有FFR,那么$\tau(\sigma)$中每个模都具有FFR.
\end{enumerate}

我们接下来要证明Serre定理:如果$k$是域,那么$k[x_1,x_2,\cdots,x_n]$的每个有限生成投射模都是稳定自由的.设$M$是左$A$模,它的一个子集$X\subset M$称为系数闭的,如果$x\in X$推出$rx\in X,\forall r\in A$.特别的子模总是系数闭子集.对系数闭子集$X$,我们记$\mathscr{A}(X)=\{\mathrm{Ann}(x)\mid x\in X,x\not=0\}$.
\begin{enumerate}
	\item 设$A$是交换诺特环,设$M$是非零的有限生成$A$模,设$X\subset M$是非空的系数闭子集.
	\begin{itemize}
		\item $\mathscr{A}(X)$中的极大元存在,并且是素理想.
		\begin{proof}
			
			按照$A$是诺特环,说明$\mathscr{A}(X)$的极大元存在.任取极大元$I=\mathrm{Ann}(x)$,假设有$a,b\in A$使得$ab\in I$但$b\not\in I$,也即$abx=0$但$bx\not=0$,那么有$I\subset I+A(a)\subset\mathrm{Ann}(bx)$,这里$bx\in X$.极大性说明$I=Aa+I$,也即$a\in I$.
		\end{proof}
	    \item 存在递降的子模链$M=M_0\supset M_1\supset\cdots\supset M_n=\{0\}$,使得每个$M_i/M_{i+1}\cong A/p_i$,这里$p_i$是素理想.
	    \begin{proof}
	    	
	    	按照诺特条件,$\mathscr{A}(M)$具有极大元,记作$p_1=\mathrm{Ann}(x_1)$,这是一个素理想,记$M_1=\langle x_1\rangle$,于是$M_1\cong A/\mathrm{Ann}(x_1)\cong A/p_1$.继续取$\mathscr{A}(M/M_1)$的极大元$p_2=\mathrm{Ann}(x_2+M_1)$,这是素理想,记$M_2=\langle x_1,x_2\rangle$,此时$M_2/M_1\cong A/\mathrm{Ann}(x_2+M_1)=A/p_2$,反复操作下去,按照$M$是有限生成的,这个操作必然会终止,此时终止到的$M_n$必然是零模,否则可以继续操作下去,这就得证.
	    \end{proof}
	\end{itemize}
    \item 设$A$是诺特交换环,如果每个有限生成$A$模都具有FFR,那么每个有限生成$A[x]$模也都具有FFR.
    \begin{proof}
    	
    	设$\mathscr{S}$是所有形如$M=A[x]\otimes_AB$,其中$B$是有限生成$A$模,的$A[x]$模构成的模族.按照条件这里$M$具有FFR,即存在$A$模的正合列$0\to F_m\to\cdots\to F_0\to B\to0$,其中每个$F_i$都是有限生成自由$A$模.按照$A[x]$是平坦$A$模,张量这个模仍然保持正合性.于是$A[x]\otimes_AB=M$作为$A[x]$模具有FFR.于是按照我们之前证明的结论,这里$\tau(\mathscr{S})$中的$A[x]$模都具有FFR.我们仅需验证每个有限生成$A[x]$模都落在$\tau(\mathscr{S})$中.
    	
    	假设$\mathrm{Ann}(M)\cap A\not=\{0\}$,那么任取$0\not=m\in M$,都有$\mathrm{Ann}(m)\cap A\not=\{0\}$.记$I=\mathrm{Ann}(m)\cap A$,就有$A/I\cong Am$.按照$A[x]$是平坦$A$模,得到$0\to A[x]\otimes_AI\to A[x]\to A[x]\otimes_AAm\to0$.这里$A[x]\otimes_AI\cong A[x]I$,于是$A[x]I\not=0$.现在$A[x]/A[x]I\cong A[x]\otimes_AAm$是$M$的一个循环子模,记作$\langle m_1\rangle,m_1\in M$,并且它落在$\mathscr{S}$中.于是$\mathrm{Ann}(m_1)\cong A[x]I$.于是$\mathrm{Ann}(x_1)\cap A\not=\{0\}$.继续对$A[x]$模$M/\langle m_1\rangle$操作,可取$m_2+\langle m_1\rangle\in M/\langle m_1\rangle$,使得$\mathrm{Ann}(m_2+\langle m_1\rangle)\cap A\not=\{0\}$,并且$\langle m_1,m_2\rangle/\langle m_1\rangle\in\mathscr{S}$.于是$\langle m_1,m_2\rangle\in\mathscr{S}$和$\mathrm{Ann}(m_1,m_2)\cap A\not=\{0\}$.按照$M$是有限生成的,这个操作必然会终止.于是我们证明了如果有限生成模$M$满足$\mathrm{Ann}(M)\cap A\not=\{0\}$,那么$M\in\tau(\mathscr{S})$.
    	
    	现在按照上一条结论,存在子模链$M=M_0\supset\cdots M_n=\{0\}$,使得每个$M_i/M_{i+1}\cong A[x]/p_i$,其中$p_i$是$A[x]$的素理想.一旦我们证明了每个整环$A[x]/p$都在$\tau(\sigma)$中,那么反复归纳下去得到每个$M_i\in\tau(\sigma)$,最后就得到$M\in\tau(\sigma)$.并且按照上一段的讨论,可不妨设$\mathrm{Ann}(A[x]/p)\cap A=p\cap A=\{0\}$,这等价于讲$A$和$A[x]$都是整环.现在任取$0\not=f(x)\in A[x]$,有$A[x]$模的短正合列$0\to A[x]f(x)\to p\to p/(f)\to0$.整环条件说明$A[x]f(x)\cong A[x]$,而$\mathrm{Ann}(p/(f(x)))\not=\{0\}$,于是$p\in\tau(\sigma)$,于是又导致$A[x]/p\in\tau(\sigma)$,完成证明.
    \end{proof}
    \item 如果$k$是域,那么$k[x_1,x_2,\cdots,x_n]$的每个有限生成模都具有FFR.这个结论就可以推出Serre定理,因为我们证明过具有FFR的投射模是稳定自由的.
    \begin{proof}
    	
    	对$n\ge1$归纳.如果$n=1$,此时$k[x]$是PID,它的有限生成自由模的子模还是有限生成自由的,于是此时$k[x]$有限生成模总有长度$\le1$的FFR.归纳步骤即上一条结论.
    \end{proof}
\end{enumerate}

我们接下来讨论交换诺特局部环上的整体维数.核心定理是这样的环上整体维数有限当且仅当它是所谓的正则局部环,此时整体维数即Krull维数.
\begin{enumerate}
	\item 和局部化维数的关系.设$A$是交换诺特环,对有限$A$模$M$,有$\mathrm{pd}_A(M)=\sup_m\{\mathrm{pd}_{A_m}(M_m)\}$,这里$m$跑遍$A$的全部极大理想.于是这件事也说明整体维数$\mathrm{D}$满足$\mathrm{D}(A)=\sup_m\{\mathrm{D}(A_m)\}$.
	\begin{proof}
		
		如果$P$是投射$A$模,那么$P\otimes_AA_m$是投射$A_m$模,于是$\mathrm{pd}_A(M)\ge\mathrm{pd}_{A_m}(M_m)$直接对投射预解张量$A_m$得到.为证另一侧的不等式,注意到诺特条件下弱维数和整体维数相同,并且$\mathrm{Tor}$函子和局部化可交换,再结合零模是一个局部性质,就得证.
	\end{proof}
    \item 局部环的投射维数.如果$(A,m,k)$是一个局部环,如果$M$是有限$A$模,那么$\mathrm{pd}_A(M)\le n$当且仅当$\mathrm{Tor}_{n+1}^A(M,k)=0$.于是这件事说明对于诺特局部环$A$上的有限模$M$,有:
    $$\mathrm{pd}_A(M)=\sup\{i\mid\mathrm{Tor}_i^A(M,k)\not=0\}$$
    \begin{proof}
    	
    	设$\mathrm{pd}_A(M)\le n$,我们有$\mathrm{fd}_A(M)\le n$,于是对任意$A$模$N$都有$\mathrm{Tor}_{n+1}^A(M,N)=0$,于是特别的$\mathrm{Tor}_{n+1}^A(M,k)=0$.反过来我们对$n\ge0$归纳.对于$n=0$,我们需要从$\mathrm{Tor}_1^A(M,k)=0$推出$M$是投射模.我们来证明它甚至是有限自由模:按照投射覆盖的内容,可取有限生成自由模$F$,存在满射$F\to M$,使得它的核$N$满足$N\subset mF$.按照$\mathrm{Tor}_1^A(M,k)=0$,说明存在短正合列$0\to N\otimes_Ak\to F\otimes_Ak\to M\otimes_Ak\to0$.于是我们得到如下交换图表:
    	$$\xymatrix{0\ar[r]&N\otimes_Ak\ar[r]^{i\otimes1}\ar[d]_{\tau_N}&F\otimes_Ak\ar[d]^{\tau_F}\\&N/mN\ar[r]_{\overline{i}}&F/mF}$$
    	
    	这里$i$是包含映射$N\to F$.按照$i\otimes1$是单射,得到$\overline{i}$是单射.但是按照$N\subset mF$,得到$\overline{i}$是零映射,这只能有$N/mN=0$,于是按照NAK引理得到$N=0$,于是$F\cong M$是有限自由模.
    	
    	假设命题对$n$成立.如果$\mathrm{Tor}_{n+2}^A(M,k)=0$,取$M$的投射预解$P_*$,记它第$n$个syzygy为$Y_n$.按照$P_*$同样是一个平坦预解,得到$\mathrm{Tor}_{n+2}^A(M,k)\cong\mathrm{Tor}_1^A(Y_n,k)=0$.于是$n=0$的步骤证明了$Y_n$是有限自由模,这导致$\mathrm{pd}_A(M)\le n+1$.
    \end{proof}
    \item 局部环的整体维数.设$(A,m,k)$是局部环,那么整体维数$\mathrm{D}(A)\le n$当且仅当$\mathrm{Tor}_{n+1}^A(k,k)=0$.结合上一条得到$\mathrm{D}(A)=\mathrm{pd}_A(k)$.
    \begin{proof}
    	
    	必要性是直接的.反过来如果$\mathrm{Tor}_{n+1}^A(k,k)=0$,那么上一条得到$\mathrm{pd}_A(k)\le n$.于是对任意有限$A$模$M$都有$\mathrm{Tor}_{n+1}^A(M,k)=0$,我们证明过局部环上这个条件等价于$\mathrm{pd}_A(M)\le n$.最后按照当$M$跑遍有限生成投射$A$模的时候$\sup\{\mathrm{pd}_A(M)\}$就是$A$的整体维数,得到$\mathrm{D}(A)\le n$.
    \end{proof}
\end{enumerate}

嵌入维数和正则局部环.我们把交换环$A$的Krull维数记作$\dim A$.给定交换局部环$(A,m,k)$,把$m/m^2$作为$k=A/m$线性空间的维数记作$V(A)$,称为$A$的嵌入维数.按照NAK引理的推论,嵌入维数恰好是$m$极小生成元集的元素个数.按照Krull主理想定理(或者说它的一般形式),一个局部环的嵌入维数必然不超过Krull维数.一旦诺特局部环$A$上这两个维数相同,就称$A$是正则局部环.

正则序列.设$A$是交换环,设$M$是$A$模,一个$A$中的序列$\{x_1,x_2,\cdots,x_n\}$称为$M$正则序列,如果$x_1$在$M$上正则(我们定义过正则元,此即映射$M\to M$,$m\mapsto x_1m$是单射),并且每个$x_{i+1}$都是$M/(x_1,x_2,\cdots,x_i)M$上的正则元.

我们接下来证明正则局部环和整体维数的关系.
\begin{enumerate}
	\item 设$(A,m,k)$是诺特局部环,取$x\in m-m^2$,记$A^*=A/(x)$和$m^*=m/(x)$,那么$V(A^*)=V(A)-1$.
	\begin{proof}
		
		取$\{y_1^*,y_2^*,\cdots,y_t^*\}$为$m^*$的极小生成元集,取$y_i^*$在$m$中的一个提升$y_i$,那么$\{x,y_1,y_2,\cdots,y_t\}$生成了$m$.我们只需验证它是极小生成元集合,按照NAK引理,等价于验证它们在$m/m^2$中是$A/m=k$基.
		
		假设$rx+\sum_ir_iy_i\in m^2$,其中$r,r_i\in A$,在$\mathrm{mod}(x)$下有$\sum_ir_i^*y_i^*\in (m^*)^2$,于是按照$\{y_1^*,\cdots,y_t^*\}$是一组$A^*$基,得到$r_i^*\in m^*$,于是$r_i\in m$,于是$rx\in m^2$,但是$x\not\in m^2$,导致$r\in m$.
	\end{proof}
	\item 一个诺特局部环是正则的当且仅当它唯一的极大理想可被一个正则序列生成,并且此时正则序列的长度恰好是嵌入维数和Krull维数.
	\item 设$(A,m,k)$是局部环,设$M$是一个有限$A$模,设$x\in m$在$M$上正则,如果$\mathrm{pd}_A(M)=n<\infty$,那么$\mathrm{pd}_A(M/xM)=n+1$.
	\begin{proof}
		
		按照$x$在$M$上正则,存在短正合列$0\to M\to M\to M/xM\to0$,这里$M\to M$取为左称$x$的这个模同态.它张量$k$诱导了$\mathrm{Tor}$函子的长正合列.当$i>n+1$时有$\mathrm{Tor}_i^A(M,k)=\mathrm{Tor}_{i-1}^A(M,k)=0$,导致$\mathrm{Tor}_i^A(M/xM,k)=0$,于是$\mathrm{pd}_A(M/xM)\le n+1$.当$i=n+1$时有正合列:
		$$\xymatrix{0=\mathrm{Tor}_{n+1}^A(M,k)\ar[r]&\mathrm{Tor}_{n+1}^A(M/xM,k)\ar[r]&\mathrm{Tor}_n^A(M,k)\ar[r]^{\phi}&\mathrm{Tor}_n^A(M,k)}$$
		
		这里$\phi$也是数乘$x$诱导的同态,但是$x$在$k$上是零,导致$\phi$是零映射,正合性导致$\mathrm{Tor}_{n+1}^A(M/xM,k)\cong\mathrm{Tor}_n^A(M,k)$非零,这说明只能有$\mathrm{pd}_A(M/xM)=n+1$.
	\end{proof}
    \item 设$(A,m,k)$是诺特局部环.
    \begin{itemize}
    	\item 如果$m-m^2$完全由零因子构成,那么存在非零元$a\in A$零化$m$.
    	\begin{proof}
    		
    		设$Z$为环$A$的所有零因子构成的集合,那么它是全部有限个伴随素理想的并,按照条件有$m-m^2\subset Z\subset p_1\cup p_2\cup\cdots\cup p_n$.如果我们能够证明$m\subset\cup_ip_i$,那么素理想avoidance引理得到$m$恰好是某个$p_i=\mathrm{Ann}(a)$,于是$a\in m$就满足$am=\{0\}$.
    		
    		于是问题归结为证明$m^2\subset\cup_ip_i$.由NAK引理得$m/m^2\not=0$,否则$m=0$没什么需要证的.于是存在$x\in m-m^2\subset\cup_ip_i$.任取$y\in m^2$,对每个整数$s\ge1$,有$x+y^s\in m-m^2\subset\cup_ip_i$,于是对每个$s$都有一个指标$j(s)$使得$x+y^s\in p_{j(s)}$.但是这里指标只有有限个,而$s$可以取无限个,于是存在正整数$t>s$使得$x+y^s,x+y^t\in p_j$,于是$y^s(1-y^{t-s})\in p_j$,但是$1-y^{t-s}$是单位,导致$y\in p_j$.这就说明了$m^2\subset\cup_ip_i$.
    	\end{proof}
        \item 如果$0<\mathrm{D}(A)=n<\infty$,那么$m-m^2$中存在非零因子.
        \begin{proof}
        	
        	按照上一条,只需说明不存在非零的$a\in A$使得$am=\{0\}$.假设存在这样的$0\not=a\in A$,取$\mu:A\to A$为数乘$a$的同态,那么$m\subset\ker\mu$,这个包含关系不能是严格的,否则$\ker\mu=A$,但是幺元明显不在$\ker\mu$中.于是$\ker\mu=m$.
        	
        	于是$k=A/m\cong A/\ker\mu\cong\mathrm{im}\mu=Aa$.考虑短正合列$0\to Aa\to A\to A/Aa\to0$,我们证明过此时要么$\mathrm{pd}(A/Aa)=\mathrm{pd}(Aa)+1$,要么$\mathrm{A/Aa}=0$.在前者情况下,有$\mathrm{pd}(A/Aa)>\mathrm{pd}(k)$,和我们证明过的局部环的整体维数必然是剩余类域的投射维数相矛盾.对于后者情况有$0=\mathrm{pd}(Aa)=\mathrm{pd}(k)=\mathrm{D}(A)$,这和条件矛盾.
        \end{proof}
    \end{itemize}
    \item 定理.诺特局部环$(A,m,k)$是正则的当且仅当$\mathrm{D}(A)$是有限的,并且此时有$\mathrm{D}(A)=V(A)=\dim A$.
\end{enumerate}
\newpage
\subsection{谱序列}

分次模.$A$上一个$\mathbb{Z}$分次模是指$A$模$M=\oplus_{n\in\mathbb{Z}}M_n$.这个信息等价于给定以$\mathbb{Z}$为指标集的$A$模族$\{M_n\}$.我们把$\mathbb{Z}$分次模就简称为分次模.两个$A$分次模之间的次数$a\in\mathbb{Z}$的同态$f:M\to N$是指一族模同态$\{f_m:M_n\to N_{m+n}\}$,次数记作$\deg f$.
\begin{enumerate}
	\item 给定分次模的同态$\xymatrix{M\ar[r]^f&N\ar[r]^g&P}$,那么有$\deg gf=\deg g+\deg f$.分次模和分次模同态构成范畴,我们有:
	$$\mathrm{Hom}(M,N)=\cup_{a\in\mathbb{Z}}\left(\prod_{n\in\mathbb{Z}}\mathrm{Hom}(M_n,N_{a+n})\right)$$
	\item 例如复形的连接映射可视为一个次数为$-1$的分次模同态,于是复形可视为一个分次模上赋予满足$f_{n-1}\circ f_n=0,\forall n\in\mathbb{Z}$的分次模同态$\{f_n\}$.再比如链映射是一个次数为零的满足和复形连接映射可交换的分次模同态.再比如同伦映射是一个次数为1的满足和复形的连接映射可交换的分次模同态.
	\item 
	\begin{itemize}
		\item 给定分次模$M$,它的子模是$M'=\{M_n'\subset M_n\}$,这个子模的商模是$M/M'=\{M_n/M_n'\}$,此时有零次的典范包含映射$i:M'\to M$和零次的典范商映射$\pi:M\to M/M'$.
		\item 次数为$a$的分次模的同态$f:M\to N$的核$\ker f$定义为$\{\ker f_n\subset M_n\}$,余核$\mathrm{coker}f$定义为$\{M_n/\mathrm{Im}f_{n-a}\}$,像$\mathrm{Im}f$定义为$\{\mathrm{Im}f_{n-a}\subset M_n\}$.
		\item $\xymatrix{M\ar[r]^f&N\ar[r]^g&P}$称为正合的如果$\ker g=\mathrm{Im}f$,换句话讲如果记$\deg f=a$,那么有$\mathrm{Im}f_{n-a}=\ker g_n$.
	\end{itemize}
	\item 给定如下分次模的正合三角形,也即每个位置都是正合的(例如复形的正合三角形),其中分次模同态$f,g,h$的次数分别记作$a,b,c$.
	$$\xymatrix{A\ar[rr]^{f}&&B\ar[dl]^{g}\\&C\ar[ul]^{h}&}$$
	
	这个信息等价于对每个整数$p$都有如下长正合列:
	$$\xymatrix{\cdots\ar[r]&B_{p-b-c}\ar[r]^{g}&C_{p-c}\ar[r]^h&A_p\ar[r]^f&B_{p+a}\ar[r]^g&C_{p+a+b}\ar[r]^h&A_{p+a+b+c}\ar[r]&\cdots}$$
\end{enumerate}

双分次模.一个双分次模是指指标集为$\mathbb{Z}\times\mathbb{Z}$的一族$A$模.双分次模通常记作$M_{\bullet\bullet}$.如果$M,N$是两个双分次模,记$(a,b)\in\mathbb{Z}\times\mathbb{Z}$,一个$(a,b)$次的分次模同态$f:M\to N$是指一族模同态$\{f_{p,q}:M_{p,q}\to M_{p+a,q+b},\forall(p,q)\in\mathbb{Z}\times\mathbb{Z}\}$.
\begin{enumerate}
	\item 如果$f:M\to N$次数是$(a,b)$,$g:N\to P$次数是$(p,q)$,那么$gf$的次数是$(a+p,b+q)$.双分次模和同态依旧构成范畴,我们有:
	$$\mathrm{Hom}(M,N)=\cup_{(a,b)\in\mathbb{Z}^2}\left(\prod_{(p,q)\in\mathbb{Z}^2}\mathrm{Hom}(M_{p,q},N_{p+a,q+b})\right)$$
	\item 
	\begin{itemize}
		\item 给定双分次模$M$,它的子模是$M'=\{M_{p,q}'\subset M_{p,q}\}$,这个子模的商模是$M/M'=\{M_{p,q}/M_{p,q}'\}$,此时有$(0,0)$次的典范包含映射$i:M'\to M$和$(0,0)$次的典范商映射$\pi:M\to M/M'$.
		\item 次数为$(a,b)$的分次模的同态$f:M\to N$的核$\ker f$定义为$\{\ker f_{p,q}\subset M_{p,q}\}$,余核$\mathrm{coker}f$定义为$\{M_{p,q}/\mathrm{Im}f_{p-a,q-b}\}$,像$\mathrm{Im}f$定义为$\{\mathrm{Im}f_{p-a,q-b}\subset M_{p,q}\}$.
		\item $\xymatrix{M\ar[r]^f&N\ar[r]^g&P}$称为正合的如果$\ker g=\mathrm{Im}f$,换句话讲如果记$\deg f=(a,b)$,那么有$\mathrm{Im}f_{p-a,q-b}=\ker g_{p,q}$.
	\end{itemize}
	\item 给定如下分次模的正合三角形,也即每个位置都是正合的,其中分次模同态$f,g,h$的次数分别记作$(a,a'),(b,b'),(c,c')$.
	$$\xymatrix{A\ar[rr]^{f}&&B\ar[dl]^{g}\\&C\ar[ul]^{h}&}$$
	
	这个信息等价于对每个整数对$(p,q)\in\mathbb{Z}^2$都有如下长正合列:
	$$\xymatrix{\cdots\ar[r]&C_{p-c,q-c'}\ar[r]^h&A_{p,q}\ar[r]^f&B_{p+a,q+a'}\ar[r]^g&C_{p+a+b,q+a'+b'}\ar[r]^h&A_{p+a+b+c,q+a'+b'+c'}\ar[r]&\cdots}$$
\end{enumerate}

双复形和全复形.
\begin{enumerate}
	\item 双复形(bicomplex).复形可视为一个分次模上赋予一个次数为$-1$的分次模同态$\{f_n\}$作为链接映射,满足$f_{n-1}\circ f_n=0$.双复形是指一个双分次模上赋予两个次数分别为$(-1,0)$和$(0,-1)$的分次模同态$d',d''$,满足$d'\circ d'=d''\circ d''=0$和$d'_{p,q-1}d''_{p,q}+d''_{p-1,q}d'_{p,q}=0$(最后一个等式是反对称性).
	\item 全复形(total complex).给定双复形$M$,它的全复形$\mathrm{Tot}(M)$定义为$\mathrm{Tot}(M)_n=\oplus_{p+q=n}M_{p,q}$.微分映射$D_n:\mathrm{Tot}(M)_n\to\mathrm{Tot}(M)_{n-1}$是两个直和之间的映射,在终端的$(p,q)$分量,其中$p+q=n-1$处取的是$d'_{p+1,q}+d''_{p,q+1}$,其中$d'_{p+1,q}$把源端的$(p+1,q)$分量中的元映射过来,$d''_{p,q+1}$把源端的$(p,q+1)$分量中的元映射过来.所以我们可以简记作$D_n=\oplus_{p+q=n-1}(d'_{p+1,q}+d''_{p,q+1})$.当然要验证$D_n$满足$D\circ D=0$:
	\begin{proof}
		\begin{align*}
		D\circ D&=\sum_{p,q}(d'+d'')(d'+d'')\\&=\sum d'd'+\sum(d'd''+d''d')+\sum d''d''\\&=0
		\end{align*}
	\end{proof}
    \item 如果$M=(M_{p,q})$是双分次模,$d'$是次数为$(-1,0)$的同态,$d''$是次数为$(0,-1)$的同态,它们分别使得$M$的行和列都是复形,如果这些映射使得$M$构成一个交换图表,把它变成一个双复形只要做一个符号变化:取$\Delta''_{p,q}=(-1)^pd''_{p,q}$,那么有$\Delta''\circ\Delta''=0$,并且有如下等式,于是$(M,d',\Delta'')$是双复形.
    \begin{align*}
    d'_{p,q-1}\Delta''_{p,q}+\Delta''_{p-1,q}d'_{p,q}&=(-1)^pd'_{p,q}+(-1)^{p-1}d''_{p-1,q}d'_{p,q}\\&=(-1)^p(d'_{p,q-1}d''_{p,q}-d''_{p-1,q}d'_{p,q})\\&=0
    \end{align*}
    \item 复形的张量积.设$R$是环,设$\textbf{A}=(A_n,\Delta_n')$和$\textbf{B}=(B_n,\Delta_n'')$是两个右$R$模构成的正复形(即指标为负的项都是零),定义双复形$(M_{p,q},d',d'')$为$M_{p,q}=A_p\otimes_RB_q$,$d'_{p,q}=\Delta'_p\otimes 1_{B_q}$和$d''_{p,q}=(-1)^p1_{A_p}\otimes\Delta_q''$.这个双复形满足$p,q$但凡有一个小于0则$M_{p,q}=0$,我们不妨非正式的称这样的双复形为第一象限双复形.这个双复形的全复形称为$\textbf{A}$和$\textbf{B}$的张量积,记作$\textbf{A}\otimes_R\textbf{B}$,它满足$(\textbf{A}\otimes_R\textbf{B})_n=\oplus_{p+q=n}$,并且形如$(a_p\otimes b_q)_{p+q=n}$的元素生成了整个$(\textbf{A}\otimes_R\textbf{B})_n$.微分映射$D_n:(\textbf{A}\otimes_R\textbf{B})_n\to(\textbf{A}\otimes_R\textbf{B})_{n-1}$就是由$(a_p\otimes b_q)_{p+q=n}\mapsto(\Delta'a_{s+1}\otimes b_t+(-1)^sa_s\otimes\Delta''b_{t+1})_{s+t=n-1}$.
    \item 双复形的转置.如果$(M,d',d'')$是双复形.它的转置定义为双复形$(M^t,\delta',\delta'')$,其中$M^t_{p,q}=M_{q,p}$,$\delta'_{p,q}=d_{q,p}'$和$\delta''_{p,q}=d''_{q,p}$.但是全复形是不变的,有$\mathrm{Tot}(M^t)_n=\mathrm{Tot}(M)_n$和$D_n^t=\sum(\delta'+\delta'')=\sum(d'+d'')=D_n$.
\end{enumerate}

$$\textbf{滤过复形}\Rightarrow\textbf{正合对}\Rightarrow\textbf{谱序列}$$

微分对象和谱序列.阿贝尔范畴中的一个微分对象是指一个对$(A,d)$,其中$A$是对象,$d\in\mathrm{Hom}(A,A)$,使得$d\circ d=0$.定义微分对象$(A,d)$的圈是对象$\ker d$,边界是对象$\mathrm{im}d$,同调是对象$H(A,d)=\ker d/\mathrm{im}d$.
\begin{enumerate}
	\item 复形是一个微分对象.记复形$(A_n,d_n)$,取$A=\oplus_nA_n$和$d=\oplus_nd_n$,那么$(A,d)$是一个微分对象.
	\item 谱序列定义为一列微分对象$\{(E^n,d^n)\mid n\ge1\}$,使得每个$H(E^n,d^n)=E^{n+1},\forall n\ge1$.后面我们具体构造的滤过复形的谱序列都是双分次模(所以会有$E^r_{p,q}$).
	\item 谱序列的极限项.如果$(E^r,d^r)_{r\ge1}$是谱序列,按照定义$E^2$是$(E^1,d^1)$的同调,所以有$E^2=Z^2/B^2$,满足$B^2\subset Z^2\subset E^1$,再考虑$E^3=\mathrm{H}(E^2,d^2)$,按照定义有$E^3=Z^3/B^3$,其中$B^3\subset Z^3\subset E^2=Z^2/B^2$,于是我们可以把$B^3$和$Z^3$替换为$Z^2$的包含$B^2$的适当子群,使得$E^3=Z^3/B^3$不变.那么就有$B^2\subset B^3\subset Z^3\subset Z^2\subset E^1$.归纳就得到:$$B^2\subset\cdots\subset B^r\subset Z^r\subset\cdots\subset Z^2\subset E^1,E^r=Z^r/B^r$$
	
	定义$Z^{\infty}=\cap_rZ^r$和$B^{\infty}=\cup_rB^r$.定义该谱序列的极限项是$E^{\infty}=Z^{\infty}/B^{\infty}$.
	\item 谱序列$\{E^r,d^r\}$有$E^{n+1}=E^n$当且仅当$Z^{n+1}=Z^n$和$B^{n+1}=B^n$.于是如果$E^{r+1}=E^r$对$r\ge n$成立,那么有$E^n=E^{\infty}$.
\end{enumerate}

正合对和诱导的谱序列.
\begin{enumerate}
	\item 一个正合对(exact couple)是指五元对$(D,E,\alpha,\beta,\gamma)$,其中$D,E$是对象,$\alpha,\beta,\gamma$是使得如下图表为正合三角形的态射.
	$$\xymatrix{D\ar[rr]^{\alpha}&&D\ar[dl]^{\beta}\\&E\ar[ul]^{\gamma}&}$$
	\item 两个正合对之间的态射是指两个态射$(\kappa:D\to D',\lambda:E\to E'):(D,E,\alpha,\beta,\gamma)\to(D',E',\alpha',\beta',\gamma')$使得如下图表交换:
	$$\xymatrix{D\ar[rr]^{\alpha}\ar[dd]^{\kappa}&&D\ar[dl]^{\beta}\ar[dd]^{\kappa}\\&E\ar[ul]^{\gamma}\ar[dd]^{\lambda}&\\D'\ar'[r][rr]^{\alpha'}&&D'\ar[dl]^{\beta'}\\&E\ar[ul]^{\gamma'}&}$$
	\item 正合对的导出对(derived couple).设$(D,E,\alpha,\beta,\gamma)$是正合对,定义它的导出对$(D^2,E^2,\alpha^2,\beta^2,\gamma^2)$如下.
	\begin{itemize}
		\item $d^1=\beta\gamma$是$E=E^1$上的微分,定义$E^2$就是同调$\mathrm{H}(E,d^1)$.
		\item 定义$D^2=\mathrm{im}\alpha\subset D$.
		\item 定义$\alpha^2$是$\alpha$诱导的,即$\alpha$在$D^2$上的限制.
		\item 定义$\beta^2:D^2\to E^2$是$\beta\alpha^{-1}$诱导的:即对$D^2$中的元$\alpha(x)$,定义$\beta_1(\alpha(x))=[\beta(x)]$.这不依赖$x$选取是因为$\alpha(x_1)=\alpha(x_2)$得到$x_1-x_2\in\ker\alpha=\mathrm{im}\gamma$,所以$x_1-x_2=\gamma(y)$,得到$\beta(x_1)-\beta(x_2)=\beta\gamma(y)=d_1(y)$.
		\item 定义$\gamma^2:E^2\to D^2$是$\gamma$诱导的:任取$[z]\in E^2$,那么$z\in\ker d^1=\ker \beta\gamma$,于是$\gamma(z)\in\ker\beta=\mathrm{im}\alpha$,就定义$\gamma^2([z])=\gamma(z)\in D^2$.这不依赖代表元$z$的选取是因为如果$z_1-z_2=d^1(y)=\beta\gamma(y)$,那么$\gamma(z_1)=\gamma(z_2)$.
		\item 定义$d^2=\beta^2\gamma^2$,它是$\beta\alpha^{-1}\gamma:E\to E$诱导的同调之间的映射$E^2\to E^2$,这是$E^2$上的微分态射,于是$(E^2,d^2)$是微分对象.
	\end{itemize}
	
	原正合对视为1阶导出对,它的导出对视为2阶导出对,归纳的定义$r$阶导出对.那么$(E^r,d^r)$是微分对象,这个序列称为正合对$(D,E,\alpha,\beta,\gamma)$诱导的谱序列.
	$$\xymatrix{D\ar[rr]_{\alpha}^{(a,a')}&&D\ar[dl]_{\beta}^{(b,b')}\\&E\ar[ul]_{\gamma}^{(c,c')}&}\qquad\xymatrix{D^2\ar[rr]_{\alpha^2}^{(a,a')}&&D^2\ar[dl]_{\beta^2}^{(b-a.b'-a')}\\&E^2\ar[ul]_{\gamma^2}^{(c,c')}&}$$
	\item 记$(D,E,\alpha,\beta,\gamma)$的$r$阶导出列$(D^r,E^r,\alpha^r,\beta^r,\gamma^r)$满足:
	\begin{itemize}
		\item $D^r=\mathrm{im}\alpha^{\circ(r-1)}$.
		\item $E^r=\frac{\gamma^{-1}(D^r)}{\beta(\ker\alpha^{\circ(r-1)})}$.
		\item $\alpha^r:D^r\to D^r$是被$\alpha$诱导的,即它是$\alpha$在$D^r$上的限制.
		\item $\beta^r:D^r\to E^r$是被$\beta\alpha^{-\circ(r-1)}$诱导的,即对$\alpha^{\circ(r-1)}(x)\in D^r$,定义它在$\beta^r$下的像是$[\beta(x)]$.
		\item $\gamma^r:E^r\to D^r$是被$\gamma$诱导的,即对$[z]\in E^r$,其中$z\in\gamma^{-1}(D^r)$,有$\gamma(z)=\alpha^{\circ(r-1)}(x)\in D_r$,就定义$\gamma^r([z])=\gamma(z)$.
		\item $d^r=\beta^r\gamma^r:E^r\to E^r$是被$\beta\alpha^{-\circ(r-1)}\gamma$诱导的.
	\end{itemize}
\end{enumerate}

滤过复形的谱序列.我们要构造的依赖于滤过的谱序列是一列微分双分次对象$(E^r,d^r)_{r\ge1}$,使得$E^{r+1}=\mathrm{H}(E^r,d^r),\forall r\ge1$.
\begin{enumerate}
	\item 谱序列的思路.设$\textbf{C}$是链复形,给定它的子复形$\textbf{C}'$,我们可以用同调$\mathrm{H}_{\bullet}(\textbf{C}')$和$\mathrm{H}_{\bullet}(\textbf{C}/\textbf{C}')$来描述$\mathrm{H}_{\bullet}(\textbf{C})$.如果我们给出$\textbf{C}$的一个滤过$\{F^p\textbf{C}\mid p\in\mathbb{Z}\}$,那么有理由相信肯定能用商的同调$\mathrm{H}_{\bullet}(F^{p+1}\textbf{C}/F^p\textbf{C})$来描述$\mathrm{H}_{\bullet}(\textbf{C})$.
	\item 设$(F^p\textbf{C})_{p\in\mathbb{Z}}$是复形$\textbf{C}$的滤过:
	$$\cdots\subset F^{p-1}\textbf{C}\subset F^p\textbf{C}\subset F^{p+1}\textbf{C}\subset\cdots\subset\textbf{C}$$
	
	我们来构造他诱导的正合对$(D^1,E^1,\alpha,\beta,\gamma)$.其中$D^1,E^1$都是双分次模,$\alpha,\beta,\gamma$都是双分次同态.
	$$\xymatrix{D^1\ar[rr]^{\alpha:(1,-1)}&&D^1\ar[dl]^{\beta:(0,0)}\\&E^1\ar[ul]^{\gamma:(-1,0)}&}$$
	
	把复形$F^p\textbf{C}$简记作$F^p$.对每个$p\in\mathbb{Z}$都有如下短正合列,其中$j^{p-1}$是包含映射,$v^p$是典范的商映射:
	$$\xymatrix{0\ar[r]&F^{p-1}\ar[r]^{j^{p-1}}&F^p\ar[r]^{v^p}&F^p/F^{p-1}\ar[r]&0}$$
	
	它诱导了如下正合三角形:
	$$\xymatrix{H_n(F^{p-1})\ar[rr]&&H_n(F^p)\ar[dl]\\&H_n(F^p/F^{p-1})\ar[ul]^{n\mapsto n-1}&}$$
	
	记$n-p=q$,取$D^1_{p,q}=H_{p+q}(F^p)$,取$E^1_{p,q}=H_{p+q}(F^p/F^{p-1})$,那么有:
	$$\xymatrix{D^1\ar[rr]^{\alpha:(1,-1)}&&D^1\ar[dl]^{\beta:(0,0)}\\&E^1\ar[ul]^{\gamma:(-1,0)}&}$$
	
	这些同态具体写出来就是:
	\begin{itemize}
		\item $\alpha^1_{p,q}:H_{p+q}(F^p)\to H_{p+q}(F^{p+1})$就是包含映射$F^p\textbf{C}\to F^{p+1}\textbf{C}$诱导的同调上的同态.
		\item $\beta^1_{p,q}:H_{p+q}(F^p)\to H_{p+q}(F^p/F^{p+1})$就是商映射$F^p\textbf{C}\to F^p\textbf{C}/F^{p-1}\textbf{C}$诱导的同调上的同态.
		\item $\gamma^1_{p,q}:H_{p+q}(F^p/F^{p-1})\to H_{p+q-1}(F^{p-1})$就是短正合列$0\to F^{p-1}\to F^p\to F^p/F^{p-1}\to0$诱导的同调的长正合列的连接映射.
		\item $d^1_{p,q}:H_{p+q}(F^p/F^{p-1})\to H_{p+q-1}(F^{p-1}/F^{p-2})$是复合映射$\beta^1\gamma^1$,它也是短正合列$\xymatrix{0\ar[r]&F^{p-1}/F^{p-2}\ar[r]&F^p/F^{p-2}\ar[r]&F^p/F^{p-1}\ar[r]&0}$所诱导的连接映射.
	\end{itemize}
	\item 设$(F^p)$是复形$\textbf{C}$的滤过,它诱导的正合对记作$(D,E,\alpha,\beta,\gamma)$,它的$r$阶导出对记作$(D^r,E^r,\alpha^r,\beta^r,\gamma^r)$.记$\beta^r\gamma^r=d^r$,那么$(E^r,d^r)_{r\ge1}$是一个谱序列,它称为滤过复形$(\textbf{C},(F^p))$的谱序列.
	$$\xymatrix{D\ar[rr]_{\alpha}^{(1,-1)}&&D\ar[dl]_{\beta}^{(0,0)}\\&E\ar[ul]_{\gamma}^{(-1,0)}&}\qquad\xymatrix{D^r\ar[rr]_{\alpha^r}^{(1,-1)}&&D^r\ar[dl]_{\beta^r}^{(1-r.r-1)}\\&E^r\ar[ul]_{\gamma^r}^{(-1,0)}&}$$
	\begin{itemize}
		\item 微分$d^r$的次数是$(-r,r-1)$,并且它被$\beta\alpha^{1-r}\gamma$诱导.
		\item $E_{p,q}^{r+1}=\ker d^r_{p,q}/\mathrm{im}d^r_{p+r,q-r+1}$.
		\item $D_{p,q}^r=\mathrm{im}(\alpha_{p-1,q+1})(\alpha_{p-2,q+2})\cdots(\alpha_{p-r+1,q+r-1})=\mathrm{im}(j^{p-1}j^{p-2}\cdots j^{p-r+1})_*$.即它是$(j^{p-1}j^{p-2}\cdots j^{p-r+1})_*:\mathrm{H}_n(F^{p-r+1})\to\mathrm{H}_n(F^p)$的像.
	\end{itemize}
    \item 设谱序列$\{(E^r,d^r)\}$中的$E^r$都是双分次对象.设有一列对象$\{H_n\mid n\in\mathbb{Z}\}$,设每个$H_n$上赋予了滤过$F^pH_n$,我们称谱序列$\{(E^r,d^r)\}$收敛到带滤过的对象列$\{H_n\}$,如果存在同构$E^{\infty}_{p,q}\cong F^pH_{p+q}/F^{p-1}H_{p+q},\forall p,q\in\mathbb{Z}$.
    \item 诱导滤过是滤过复形的同调上的自然滤过.设$(F^p)$是复形$\textbf{C}$的滤过,每个包含映射$i^p:F^p\subset\textbf{C}$都诱导了同调之间的同态$\mathrm{H}_{\bullet}(F^p)\to\mathrm{H}_{\bullet}(\textbf{C})$.按照$F^p\subset F^{p+1}$,就得到$\mathrm{im}i^p_*\subset\mathrm{im}i^{p+1}_*$,所以$(\mathrm{im}i^p_*)$是$\mathrm{H}_{\bullet}(\textbf{C})$上的滤过,它称为$(F^p)$在同调$\{H_n\}$上的诱导滤过.我们把$\mathrm{im}i_{n,*}^p$也记作$F^p\mathrm{H}_n(\textbf{C})$.
    \item 设$(F^p)$是复形$\textbf{C}$上的滤过,设滤过诱导的谱序列为$(E^r,d^r)_{r\ge1}$,称这个谱序列收敛到同调$\mathrm{H}=\mathrm{H}_{\bullet}(\textbf{C})$,如果对任意整数$p,q$都有:
    $$E_{p,q}^{\infty}\cong F^p\mathrm{H}_{p+q}(\textbf{C})/F^{p-1}\mathrm{H}_{p+q}(\textbf{C})$$
\end{enumerate}

但是复形上任取滤过,诱导的谱序列未必是收敛到同调的,为此需要加上一些条件.分次模$M=(M_n)$上的滤过$(F^p)$称为有界的,如果对每个$n$,都存在和$n$有关的指标$s$和$t$,使得$F^sM_n=\{0\}$和$F^tM_n=M_n$.
\begin{enumerate}
	\item 如果复形$\textbf{C}$的滤过$(F^p)$是有界的,那么它在$\mathrm{H}_{\bullet}(\textbf{C})$上的诱导滤过也是有界的,并且具有相同的界.
	\begin{proof}
		
		假设对$n$有$F^sM_n=0$和$F^tM_n=M_n$,那么$F^s\mathrm{H}_n=\mathrm{im}(i_*^s)_n=0$和$F^t\mathrm{H}_n=\mathrm{im}(i_*^t)_n=\mathrm{H}_n$.所以就有有限的链:
		$$\{0\}=F^s\mathrm{H}_n\subset F^{s+1}\mathrm{H}_n\subset\cdots\subset F^t\mathrm{H}_n=\mathrm{H}_n$$
	\end{proof}
    \item 设$(F^p)$是复形$\textbf{C}$是有界滤过,设$(E^r,d^r)_{r\ge1}$是该滤过的谱序列.那么对每对固定的$p,q$,当$r$足够大的时候都有$E_{p,q}^{\infty}=E_{p,q}^r$.此时有谱序列收敛到同调.
    \begin{proof}
    	
    	按照$(F^p)$是有界滤过,可设函数$s(n),t(n)$满足当$p<s(n)$时$F^p\textbf{C}_n=0$,当$p>t(n)$时$F^p\textbf{C}_n=\textbf{C}_n$.于是在$p>t(n)$时按照定义有$E_{p,q}^1=\mathrm{H}_{p+q}(F^p/F^{p-1})=0$.按照$E_{p,q}^r$是$E^1_{p,q}$的子模商,得到$E_{p,q}^r=0,\forall r\ge1$.同理得到$p<s(n)$的时候有$E_{p,q}^r=0,\forall r\ge1$.
    	
    	我们知道$d^r$的次数是$(-r,r-1)$,当固定$(p,q)$时,选取足够大的$r$使得$p-r<s(n)$,那么导致$d^r(E_{p,q}^r)\subset E_{p-r,q+r-1}^r=0$.所以有$\ker d^r_{p,q}=E_{p,q}^r$.我们还可以让$r$更大一些使得$E_{p+r,*}^r=0$,那么导致$\mathrm{im}d_{p+r,*}^r=0$.于是有$E_{p,q}^{r+1}=\ker d_{p,q}^r/\mathrm{im}d^r_{p+r,*}=E_{p,q}^r/0=E_{p,q}^r$.这说明固定$p,q$时当$r$足够大的时候$E_{p,q}$是固定的.
    	
    	我们解释过$\alpha^r$的次数为$(1,-1)$,$\beta$的次数为$(1-r,r-1)$,$\gamma^r$的次数为$(-1,0)$.考虑$r$阶导出对,得到如下正合列:
    	$$\xymatrix{D_{p+r-2,q+2-r}^r\ar[r]^{\alpha^r}&D_{p+r-1,q+1-r}^r\ar[r]^{\beta^r}&E_{p,q}^r\ar[r]^{\gamma^r}&D_{p-1,q}^r}$$
    	
    	我们解释过有如下等式,其中$n=p+q$:
    	$$D_{p+r-1,q+1-r}^r=\mathrm{im}(j^{p+r-2}\cdots j^p)_*\subset\mathrm{H}_n(F^{p+r-1})$$
    	$$D_{p+r-2,q+2-r}^r=\mathrm{im}(j^{p+r-3}\cdots j^{p-1})_*\subset\mathrm{H}_n(F^{p+r-2})$$
    	
    	当$r$足够大时就有$F^{p+r-1}=\textbf{C}$,此时$j^{p+r-2}\cdots j^p$就是包含映射$i^p:F^p\subset\textbf{C}$.所以有$D_{p+r-1,q+1-r}^r=\mathrm{im}i^p_*=F^p\mathrm{H}_n$.类似的有$D_{p+r-2,q+2-r}^r=F^{p-1}\mathrm{H}_n$.所以之前的四项正合列就变成:
    	$$\xymatrix{F^{p-1}\mathrm{H}_n(\textbf{C})\ar[r]&F^p\mathrm{H}_n(\textbf{C})\ar[r]&E_{p,q}^r\ar[r]&D_{p-1,q}^r}$$
    	
    	这里第一个映射就是包含映射.最后一项$D_{p-1,q}^r$是包含映射$F^{p-r}\subset F^{p-1}$诱导的同调的同态,但是只要让$r$足够大就有$F^{p-1-r}=0$,导致$D_{p-1,q}^r=0$.最后正合性就得到:
    	$$F^p\mathrm{H}_n(\textbf{C})/F^{p-1}\mathrm{H}_n(\textbf{C})\cong E_{p,q}^r=E_{p,q}^{\infty}$$
    \end{proof}
\end{enumerate}

如下命题中我们总设$\{(E^r,d^r)\}$是某个有界滤过复形$\{F^p\textbf{C}\}$的谱序列,于是我们证明了该谱序列就收敛到同调$\{F^pH_n\}$.
\begin{enumerate}
	\item 如果存在正整数$r$,使得$E^r$塌陷在$p$轴上(或者塌陷在$q$轴上),此即对任意$q\not=0$都有$E^r_{p,q}=0$.那么有$E_{p,q}^2=E_{p,q}^3=\cdots=E_{p,q}^{\infty}$.
	\begin{proof}
		
		我们解释过$E^{r+r'}_{p,q}$是$E_{p,q}^r$的子商对象(即两个子对象的商),于是如果$q\not=0$,那么有$E_{p,q}^{r+r'}=0$.现在$E_{p,0}^{r+1}=\ker d_{p,0}^r/\mathrm{im}d_{p+r,-r+1}^r$.这里$d_{p,0}^r=0$因为终端不在$p$轴,所以有$\ker d_{p,0}^r=E_{p,0}^r$.类似的$d_{p+r,-r+1}^r=0$,所以有$E_{p,0}^{r+1}=E_{p,0}^r$,归纳得到结论.
	\end{proof}
    \item 如果存在正整数$r$使得$E^r$塌陷在$p$轴,那么有$E_{n,0}^r\cong\mathrm{H}_n,\forall n\in\mathbb{Z}$;如果$E^r$塌陷在$q$轴,那么有$E_{0,n}^r\cong\mathrm{H}_n,\forall n\in\mathbb{Z}$.
    \begin{proof}
    	
    	如果谱序列塌陷在$p$轴,那么当$p\le n-1,p\ge n+1$的时候$q=n-p$非零,导致$E_{p,q}^r=0$.所以上一条说明$p\le n-1,p\ge n+1$的时候有$0=E_{p,q}^r=E_{p,q}^{\infty}=F^p\mathrm{H}_n/F^{p-1}\mathrm{H}_n$.按照条件就有:$$\{0\}=F^{-1}\mathrm{H}_n=F^0\mathrm{H}_n=\cdots=F^{n-1}\mathrm{H}_n,F^nH_n=F^{n+1}H_n=\cdots=H_n$$
    	
    	所以有$\mathrm{H}_n=F^n\mathrm{H}_n/F^{n-1}\mathrm{H}_n\cong E_{n,0}^r$.
    \end{proof}
    \item 设$E^2$塌陷在$q=0$和$q\ge n$上(即除此之外的$q$和任意$p$总满足$E^2_{p,q}=0$).那么:
    \begin{itemize}
    	\item 对任意$0\le i\le n-1$都有$E^2_{i,0}\cong H^i$.
    	\item 有正合列$\xymatrix{0\ar[r]&E^2_{n,0}\ar[r]&H_n\ar[r]&E^2_{0,n}\ar[r]&E^2_{n+1,0}\ar[r]&H_{n-1}}$
    \end{itemize}
    \begin{proof}
    	
    	
    \end{proof}
    \item 推论.特别的,如果$E^2$是第一象限的,此即对$p<0$或$q<0$总有$E^2_{p,q}=0$,那么$E^2_{0,0}=H_0$,并且有正合列:$$\xymatrix{0\ar[r]&E^2_{1,0}\ar[r]&H_1\ar[r]&E^2_{0,1}\ar[r]&E^2_{2,0}\ar[r]&H_0}$$
    \item 设$E^2$塌陷在$p=0$和$p=t$上,那么有:
    \begin{itemize}
    	\item $E^2=E^3=\cdots=E^t$.
    	\item $E^{t+1}=E^{t+2}=\cdots=E^{\infty}$.
    	\item 有短正合列$\xymatrix{0\ar[r]&E^{t+1}_{0,n}\ar[r]&H_n\ar[r]&E^{t+1}_{t,n-t}\ar[r]&0}$.
    	\item 有长正合列:
    	$$\xymatrix{\cdots\ar[r]&H_{n+1}\ar[r]&E^2_{t,n+1-t}\ar[r]&E_{0,n}^2\ar[r]&H_n\ar[r]&E^2_{t,n-t}\ar[r]&E_{0,n-1}^2\ar[r]&\cdots}$$
    \end{itemize}
    \begin{proof}
    	
    	
    \end{proof}
\end{enumerate}

全复形上的两个滤过.设$(M,d',d'')$是双复形.
\begin{itemize}
	\item 它的全复形的第一滤过定义为:
	$$\cdots\subset(^1F)^{p-1}\mathrm{Tot}(M)\subset(^1F)^p\mathrm{Tot}(M)\subset (^1F)^{p+1}\mathrm{Tot}(M)\subset\cdots$$
	$$(^1F)^p\mathrm{Tot}(M)_n=\oplus_{i\le p}M_{i,n-i}=M_{p,q}\oplus M_{p-1,q+1}\oplus M_{p-2,q+2}\oplus\cdots$$
	\item 它的全复形的第二过滤定义为:
	$$\cdots\subset(^2F)^{p-1}\mathrm{Tot}(M)\subset(^2F)^p\mathrm{Tot}(M)\subset (^2F)^{p+1}\mathrm{Tot}(M)\subset\cdots$$
	$$(^2F)^p\mathrm{Tot}(M)_n=\oplus_{j\le p}M_{n-j,j}=M_{q,p}\oplus M_{q+1,p-1}\oplus M_{q+2,p-2}\oplus\cdots$$
\end{itemize}
\begin{enumerate}
	\item 设$(M,d',d'')$是双复形,它的转置双复形是$(M^t,d'',d')$.那么全复形$\mathrm{Tot}(M,d',d'')$的第二滤过就是$\mathrm{Tot}(M^t,d'',d')$的第一滤过.
	\item 如果$M$是第一象限双复形,它的全复形的第一滤过和第二滤过就都是有界的滤过.诱导的谱序列分别记作$^1E^r$和$^2E^r$,那么固定$p,q$时当$r$足够大时有$^1E_{p,q}^r=^1E_{p,q}^{\infty}$和$^2E_{p,q}^r=^2E_{p,q}^{\infty}$.并且这两个谱序列都收敛到全复形的同调.
	\item 第一累次同调(first iterated homology).设$(M,d',d'')$是双复形,它的全复形的第一滤过记作$(F^p)$.我们有:
	$$(F^p)_n=M_{p,q}\oplus M_{p-1,q+1}\oplus M_{p-2,q+2}\oplus\cdots$$
	$$(F^{p-1})_n=M_{p-1,q+1}\oplus M_{p-2,q+2}\oplus\cdots$$
	
	所以按照定义就有$(F^p/F^{p-1})_n=M_{p,q}$,微分映射$(F^p/F^{p-1})_n\to(F^p/F^{p-1})_{n-1}$就是$\overline{D}_n:a_n+(F^{p-1})_n\mapsto D_na_n+(F^{p-1})_{n-1}$.其中$D_n$是全复形上的微分映射.有$D_na_n=(d'_{p,q}+d''_{p,q})a_n\in M_{p-1,q}\oplus M_{p,q-1}$.这里$M_{p-1,q}\subset(F^{p-1})_n$.所以$\overline{D}_n$就是在$\mathrm{mod}(F^{p-1})_{n-1}$下把$a_n$映射为$d''_{p,q}a_n$.所以有:
	$$E_{p,q}^1=\mathrm{H}_n(F^p/F^{p-1})=\frac{\ker\overline{D}_n}{\mathrm{im}\overline{D}_{n+1}}\cong\frac{\ker d''_{p,q}}{\mathrm{im}d''_{p,q+1}}=\mathrm{H}_q(M_{p,*})$$
	
	换句话讲固定$p$的时候$E_{p,q}^1$就是把双复形$(M,d',d'')$视为一个复形$(M_{p,*},d'')$再取同调.另外这里$\mathrm{H}_q(M_{p,*})$中的元素可以表示为$\mathrm{cls}(z)$,其中$z\in M_{p,q}$并且$d''z=0$.接下来固定$q$,考虑如下序列:
	$$\cdots,\mathrm{H}_q(M_{p+1,*}),\mathrm{H}_q(M_{p,*}),\mathrm{H}_q(M_{p-1,*}),\cdots$$
	
	如果定义$\overline{d'}_p:\mathrm{H}_q(M_{p,*})\to\mathrm{H}_q(M_{p-1},*)$为$\mathrm{cls}(z)\mapsto\mathrm{cls}(d'_{p,q}z)$.这使得上述序列构成一个复形.这个复形在指标$p$处的同调记作$\mathrm{H}_p'\mathrm{H}_q''(M)$.它是第一滤过诱导的累次同调,称为第一累次同调.$\mathrm{H}_p'\mathrm{H}_q''(M)$中的元素可以表示为$\mathrm{cls}(\overline{d}'z)$,其中$z\in M_{p,q}$满足$d''z=0$.类似可定义双复形$M$的第二累次同调$\mathrm{H}''_p\mathrm{H}'_q(M)$.
	\item 上述讨论已经说明$^1E_{p,q}^1=\mathrm{H}_q(M_{p,*})$.这里我们证明第一累次同调就是第一滤过诱导的谱序列中的第二项:$^1E_{p,q}^2=\mathrm{H}'_p\mathrm{H}''_q(M)$.类似的第二累次同调就是第二滤过诱导的谱序列中的第二项.
	\begin{proof}
		
		按照定义有$d^1:\mathrm{H}_{p+q}(F^p/F^{p-1})\to\mathrm{H}_{p+q-1}(F^{p-1}/F^{p-2})$是如下图表中的连接映射:
		$$\xymatrix{&&M_{p-1,q+1}\oplus M_{p,q}\ar[r]^{\pi}\ar[d]^{\overline{D}}&M_{p,q}\ar[r]&0\\0\ar[r]&M_{p-1,q}\ar[r]^i&M_{p,q-1}\oplus M_{p-1,q}&&}$$
		
		我们断言$d^1$把$\mathrm{cls}(z)\mapsto\mathrm{cls}(\overline{d'}z)\in\mathrm{H}_p'\mathrm{H}_q''(M)$.这是因为选取$z\in M_{p,q}$满足$d''_{p,q}z=0$,按照$\overline{D}:(a_{p-1,q},a_{p,q})\mapsto(d''a_{p-1,q}+d'a_{p,q},d''a_{p,q})$.选取$\pi^{-1}(z)=(0,z)$,就得到$\overline{D}(0,z)=(d'_{p,q}z,0)$.所以有:
		$$d^1\mathrm{cls}(z)=\mathrm{cls}(i^{-1}\overline{D}\pi^{-1}z)=\mathrm{cls}(\overline{d'}z)\in\mathrm{H}'_p\mathrm{H}''_q(M)$$
	\end{proof}
\end{enumerate}

Cartan-Eilenberg预解.如下命题的对偶版本(内射版本)均是成立的.
\begin{enumerate}
	\item 称阿贝尔范畴$\mathscr{A}$上的一个复形$(\textbf{C},d)$是分裂的(split),如果对任意整数$n$,它诱导的如下短正合列都是分裂的:
	$$\xymatrix{0\ar[r]&Z_n\ar[r]&C_n\ar[r]^{d_n}&B_{n-1}\ar[r]&0}$$
	$$\xymatrix{0\ar[r]&B_n\ar[r]&Z_n\ar[r]&H_n\ar[r]&0}$$
	\item 引理.设$f:A\to B$是阿贝尔范畴中的一个态射,对整数$k$,记$\Sigma^k(f)$表示这样一个复形,它在位置$k$是$A$,位置$k-1$是$B$,微分映射是$f$,其余位置的对象都是零,微分映射也都是0.那么我们断言,如果$P$是阿贝尔范畴中的投射对象,那么$\Sigma^k(1_P)$是复形范畴中的投射对象.
	\begin{proof}
		
		任取复形之间的满态射$(\textbf{C},d)\to(\textbf{C}',d')$,考虑如下图表,按照$P$是投射对象,就有$f$提升到态射$h:P\to C_k$.取$h':P\to C_{k-1}$为$d\circ h$,那么有$g'\circ h'=g'\circ d\circ h=d'\circ g\circ h=d'\circ f=f'$.所以所有面都是交换的,这样就构造了$\Sigma^k(1_P)\to(\textbf{C}',d')$的提升链态射$\Sigma^k(1_P)\to(\textbf{C},d)$.
		$$\xymatrix@!{&P\ar[dr]^{1_P}\ar[ddl]_h\ar[dd]^f&\\&&P\ar[ddl]^{h'}\ar[dd]^{f'}\\C_k\ar[r]^g\ar[dr]_d&C_k'\ar[dr]^{d'}&\\&C_{k-1}\ar[r]_{g'}&C_{k-1}'}$$
	\end{proof}
    \item 设$(\textbf{C},d)$是分裂复形,设$\delta_k$是包含映射$B_k(\textbf{C})\to Z_k(\textbf{C})$,那么有复形的同构:$$\textbf{C}\cong\oplus_{k\in\mathbb{Z}}\Sigma^k(\delta_k)$$
    \item 设$(\textbf{C},d)$是每一项都为投射对象的分裂复形,那么它是$\textbf{Comp}(\mathscr{A})$中的投射对象.实际上逆命题也成立,下面会给出证明.
    \begin{proof}
    	
    	按照分裂条件有$\textbf{C}\cong\oplus_{k\in\mathbb{Z}}\Sigma^k(\delta_k)$.按照每个$0\to Z_k\to C_k\to B_{k-1}\to0$是分裂的,说明$\Sigma^k(Z_k\to C_k)$是$\Sigma^k(1_{C_k})$的直和项,我们解释过$\Sigma^k(1_{C_k})$是投射复形,另外投射对象的直和项是投射的,所以$\Sigma^k(Z_k\to C_k)$也是投射复形,所以$Z_k$是投射对象.所以$\Sigma^k(1_{Z_k})$也是投射复形.再按照每个$0\to B_k\to Z_k\to H_k\to0$是分裂短正合列,说明$\Sigma^k(B_k\to Z_k)$是投射复形$\Sigma^k(1_{Z_k})$的直和项,所以$\Sigma^k(\delta_k)$也是投射复形,所以投射复形的直和$\textbf{C}\cong\oplus_{k\in\mathbb{Z}}\Sigma^k(\delta_k)$是投射复形.
    \end{proof}
    \item 设$f:(\textbf{C},d)\to(\textbf{C}',d')$是链映射,它的锥定义为一个复形$\mathrm{cone}(f)$,满足$\mathrm{cone}(f)_n=C_{n-1}\oplus C_n'$.连接映射定义为:
    $$D_n:(c_{n-1},c_n')\mapsto(-d_{n-1}c_{n-1},d_n'c_n'-f_{n-1}c_{n-1})$$
    \item 设$f:(\textbf{C},d)\to(\textbf{C}',d')$是链映射,那么有如下复形的短正合列,其中$i:c'\mapsto(0,c')$,$j:(c,c')\mapsto -c$.它诱导的长正合列中的连接映射$H_n(\textbf{C}[-1])=H_{n-1}(\textbf{C})\to H_{n-1}(\textbf{C}')$就是链映射$f$诱导的映射.
    $$\xymatrix{0\ar[r]&\textbf{C}'\ar[r]^i&\mathrm{cone}(f)\ar[r]^j&\textbf{C}[-1]\ar[r]&0}$$
    \item 关于锥有如下结论.设$(\textbf{C},d)$是阿贝尔范畴中的复形.
    \begin{itemize}
    	\item $\mathrm{cone}(1_{\textbf{C}})$是零调复形,即是正合的.
    	\begin{proof}
    		
    		没什么需要证的,如果$y\in C_n,x\in C_{n-1}$满足$dx=0$和$x=dy$,那么$D(y,0)=(x,y)$.
    	\end{proof}
    	\item 如果有次数$+1$的链映射$s:\textbf{C}\to\textbf{C}$满足$d=d\circ s\circ d$,那么$\textbf{C}$是分裂复形.
    	\begin{proof}
    		
    		证明$\xymatrix{0\ar[r]&Z_n\ar[r]&C_n\ar[r]^d&B_{n-1}\ar[r]&0}$是分裂的.因为取$s$在$B_{n-1}$上的限制记作$t$,那么有$dt(b)=dtd(c)=d(c)=b$,于是$dt=1_{B_{n-1}}$,所以短正合列是分裂的.
    		
    		\qquad
    		
    		证明$\xymatrix{0\ar[r]&B_n\ar[r]&Z_n\ar[r]^p&H_n\ar[r]&0}$是分裂的.定义$u:H_n\to Z_n$为$[z]\mapsto z-dsz$,这个定义不依赖于代表元选取是因为如果$z_1-z_2=d(c)$,那么$(z_1-dsz_1)-(z_2-dsz_2)=d(c)-dsd(c)=0$.另外有$pu([z])=[z-dsz]=[z]$,于是$pu=1_{H_n}$,所以短正合列是分裂的.
    	\end{proof}
    	\item $\mathrm{cone}(1_{\textbf{C}})$是分裂复形.
    	\begin{proof}
    		
    		直接构造次数$+1$的链映射$s:\mathrm{cone}(1)\to\mathrm{cone}(1)$为$(c_{n-1},c_n)\mapsto(-c_n,0)$,它满足$dsd=d$,上一条就说明这是分裂复形.
    	\end{proof}
    \end{itemize}
    \item 链映射$f:(\textbf{C},d)\to(\textbf{C}',d')$是拟同构(即诱导了同调的同构)当且仅当$\mathrm{cone}(f)$是零调的.
    \begin{proof}
    	
    	一方面如果$f$是拟同构,考虑$0\to\textbf{C}\to\mathrm{cone}(f)\to\textbf{C}[-1]\to0$诱导的长正合列,我们解释过连接映射是$f$诱导的同调之间的同态,这可以得到$\mathrm{cone}(f)$零调是因为,如果有如下正合列,并且$f,g$是同构,那么$C=0$:
    	$$\xymatrix{A\ar[r]^f&B\ar[r]&C\ar[r]&D\ar[r]^g&E}$$
    	
    	另一方面如果$\mathrm{cone}(f)$是零调的,考虑$0\to\textbf{C}\to\mathrm{cone}(f)\to\textbf{C}[-1]\to0$诱导的长正合列就说明连接映射都是同构,但是这个连接映射就是被$f$诱导的.
    \end{proof}
    \item 阿贝尔范畴上的一个复形是投射复形当且仅当它是由投射对象构成的分裂复形.
    \begin{proof}
    	
    	我们已经证明过充分性,对于必要性,假设$(\textbf{C},d)$是投射复形,即复形范畴中的投射对象.那么$\textbf{C}[-1]$也是投射对象,于是短正合列$0\to\textbf{C}\to\mathrm{cone}(1)\to\textbf{C}[-1]\to0$是分裂短正合列,但是我们解释过$\mathrm{cone}(1)$是分裂复形,它的直和项就也是分裂复形.另外我们证明$C_k$是阿贝尔范畴中的投射对象,任取满态射$A\to B$,记$\Sigma^k(A)$表示位置$k$为$A$,其余位置都是零对象,微分映射都是零态射的复形.任取态射$C_k\to B$,它可以零延拓为链映射$\textbf{C}\to\Sigma^k(B)$,所以有链映射的提升$\textbf{C}\to\Sigma^k(A)$,于是有态射$C_k\to A$提升了$C_k\to B$.这就说明$C_k$是投射对象.
    \end{proof}
    \item 如果阿贝尔范畴$\mathscr{A}$具有足够多的投射对象,那么$\textbf{Comp}(\mathscr{A})$也是.类似的完全子范畴$\textbf{Comp}_{\ge0}(\mathscr{A})$和$\textbf{Comp}^{\le0}(\mathscr{A})$都具有足够多投射对象.
    \begin{proof}
    	
    	取复形$(\textbf{C},d)$,按照$\mathscr{A}$上具有足够多投射对象,有投射对象$P_n$和满态射$g_n:P_n\to C_n$.考虑$G_n:\Sigma^n(1_{P_n})\to\textbf{C}$是如下链映射:
    	$$\xymatrix{\ar[r]&0\ar[r]\ar[d]&P_n\ar[r]^{1_{P_n}}\ar[d]_{g_n}&P_n\ar[r]\ar[d]^{d_ng_n}&0\ar[r]\ar[d]&\\\ar[r]&C_{n+1}\ar[r]&C_n\ar[r]_{d_n}&C_{n-1}\ar[r]&C_{n-2}\ar[r]&}$$
    	
    	于是有以投射对象为源端的态射$\Sigma=\oplus_{n\in\mathbb{Z}}\Sigma^n(1_{P_n})\to\textbf{C}$,这是满态射,得证.
    \end{proof}
    \item 设$(\textbf{C},d)$是阿贝尔范畴上的复形,它的一个Cartan-Eilenberg投射预解是指复形范畴上的一个正合列:
    $$\xymatrix{\ar[r]&M_{\bullet,q}\ar[r]&\cdots\ar[r]&M_{\bullet,1}\ar[r]&M_{\bullet,0}\ar[r]&\textbf{C}\ar[r]&0}$$
    $$\xymatrix{&\ar[d]&\ar[d]&\ar[d]&\ar[d]&\\\cdots\ar[r]&M_{2,2}\ar[r]\ar[d]&M_{2,1}\ar[r]\ar[d]&M_{2,0}\ar[r]\ar[d]&C_2\ar[d]\ar[r]&0\\\cdots\ar[r]&M_{1,2}\ar[r]\ar[d]&M_{1,1}\ar[r]\ar[d]&M_{1,0}\ar[r]\ar[d]&C_1\ar[r]\ar[d]&0\\\cdots\ar[r]&M_{0,2}\ar[r]\ar[d]&M_{0,1}\ar[r]\ar[d]&M_{0,0}\ar[r]\ar[d]&C_0\ar[r]\ar[d]&0\\\cdots\ar[r]&M_{-1,2}\ar[r]\ar[d]&M_{-1,1}\ar[r]\ar[d]&M_{-1,0}\ar[r]\ar[d]&C_{-1}\ar[r]\ar[d]&0\\\cdots\ar[r]&M_{-2,2}\ar[r]\ar[d]&M_{-2,1}\ar[r]\ar[d]&M_{-2,0}\ar[r]\ar[d]&C_{-2}\ar[r]\ar[d]&0\\&&&&&}$$
    
    满足每一行都是投射预解,并且相邻两行之间的链映射就诱导了如下复形都是投射预解:
    $$\xymatrix{\cdots\ar[r]&Z_{p,1}\ar[r]&Z_{p,0}\ar[r]&Z_p(\textbf{C})\ar[r]&0}$$
    $$\xymatrix{\cdots\ar[r]&B_{p,1}\ar[r]&B_{p,0}\ar[r]&B_p(\textbf{C})\ar[r]&0}$$
    $$\xymatrix{\cdots\ar[r]&H_{p,1}\ar[r]&H_{p,0}\ar[r]&H_p(\textbf{C})\ar[r]&0}$$
    
    那么首先这的确是一个投射预解,因为按照条件有如下短正合列都是分裂的,就说明每个$M_{\bullet,q}$都是由投射对象构成的分裂复形,我们解释过这等价于讲它是投射复形.
    $$\xymatrix{0\ar[r]&Z_{p,q}\ar[r]&M_{p,q}\ar[r]&B_{p-1,q}\ar[r]&0}$$
    $$\xymatrix{0\ar[r]&B_{p,q}\ar[r]&Z_{p,q}\ar[r]&H_{p,q}\ar[r]&0}$$
    \item 我们断言如果阿贝尔范畴上具有足够多投射对象,那么它的每个复形都存在Cartan-Eilenberg投射预解.
    \begin{proof}
    	
    	取复形$(\textbf{C},d)$,对每个整数$p$就有如下短正合列:
    	$$\xymatrix{0\ar[r]&B_p\ar[r]&Z_p\ar[r]&H_p\ar[r]&0}$$
    	$$\xymatrix{0\ar[r]&Z_p\ar[r]&C_p\ar[r]&B_{p-1}\ar[r]&0}$$
    	
    	任取$B_p$和$H_p$的投射预解为$B_{p,\bullet}$和$H_{p,\bullet}$,按照Horseshoe引理做直和就得到$Z_p$的投射预解$Z_{p,\bullet}=B_{p,\bullet}\oplus H_{p,\bullet}$和$C_p$的投射预解$M_{p,\bullet}=B_{p,\bullet}\oplus H_{p,\bullet}\oplus B_{p-1,\bullet}$.我们取$d_{p,q}:M_{p,q}=B_{p,q}\oplus H_{p,q}\oplus B_{p-1,q}\to M_{p-1,q}=B_{p-1,q}\oplus H_{p-1,q}\oplus B_{p-2,q}$为$(a,b,c)\mapsto(c,0,0)$.于是有如下交换图表:
    	$$\xymatrix{M_{p,q}\ar[rr]\ar[d]_{d_{p,q}}&&M_{p,q-1}\ar[d]^{d_{p,q-1}}\\M_{p-1,q}\ar[rr]&&M_{p-1,q-1}}$$
    	
    	并且有$\ker d_{p,q}=Z_{p,q}$和$\mathrm{im}d_{p,q}=B_{p-1,q}$.这说明$M_{\bullet,\bullet}$构成了Cartan-Eilenberg投射预解.
    \end{proof}
\end{enumerate}

Grothendieck谱序列.设$F:\mathscr{A}\to\mathscr{B}$是阿贝尔范畴之间的加性函子,如果$\mathscr{A}$具有足够多的投射对象,称对象$A\in\mathscr{A}$是左$F$零调的,如果$p\ge1$时左导出函子列满足$L_pF(A)=0$;如果$\mathscr{A}$具有足够多内射对象,称对象$A\in\mathscr{A}$是右$F$零调的,如果$p\ge1$时右导出函子列满足$R^pF(A)=0$.
\begin{enumerate}
	\item 设$\xymatrix{\mathscr{A}\ar[r]^G&\mathscr{B}\ar[r]^F&\mathscr{C}}$是阿贝尔范畴之间的共变加性函子,设$\mathscr{A}$和$\mathscr{B}$和$\mathscr{C}$具有足够多的内射对象,设$F$是左正合的,对$\mathscr{A}$的每个内射对象$A$有$GA$是右$F$零调的.那么有第三象限谱序列$E_{\infty}^{p,q}=E_2^{p,q}=(R^pF)(R^qG)A$收敛到$R^n(FG)A$.
	\begin{proof}
		
		给定对象$A\in\mathscr{A}$,选取内射预解$0\to A\to E^0\to E^1\to\cdots$,把$G$作用在简化内射预解上得到$GE^A=0\to GE^0\to GE^1\to GE^2\to\cdots$.于是可取它的Cartan-Eilenberg内射预解$M^{\bullet,\bullet}$:
		$$\xymatrix{&0\ar[d]&0\ar[d]&0\ar[d]&\\0\ar[r]&GE^0\ar[r]\ar[d]&M^{0,0}\ar[r]\ar[d]&M^{0,1}\ar[r]\ar[d]&\cdots\\0\ar[r]&GE^1\ar[r]\ar[d]&M^{1,0}\ar[r]\ar[d]&M^{1,1}\ar[r]\ar[d]&\cdots\\0\ar[r]&GE^2\ar[r]\ar[d]&M^{2,0}\ar[r]\ar[d]&M^{2,1}\ar[r]\ar[d]&\cdots\\0\ar[r]&GE^3\ar[r]\ar[d]&M^{3,0}\ar[r]\ar[d]&M^{3,1}\ar[r]\ar[d]&\cdots\\&&&&}$$
		
		这可以理解为第三象限双复形.作用函子$F$,得到双复形$FM^{\bullet,\bullet}$,下面计算它的累次同调.先求行复形的同调,第$p$行第$q$个同调就是$(R^qF)(GE^p)$,按照条件当$q\ge1$时总有$(R^qF)(GE^p)=0$.而当$q=0$时按照$F$是左正合的就有$R^0F=F$,所以求行复形的同调,当$q>0$时恒为零,当$q=0$时是复形$0\to FG(E^0)\to FG(E^1)\to\cdots$.再求同调就得到第一谱序列:
		$$(^1E)_2^{p,q}=\left\{\begin{array}{cc}R^p(FG)(A)&q=0\\\{0\}&q>0\end{array}\right.$$
		
		这个谱序列塌陷在$p$轴,说明有$H^n(\mathrm{Tot}(FM))\cong R^n(FG)A,\forall n$.下面求第二谱序列,为此要先求列复形的同调,为此要回顾Cartan-Eilenberg内射预解中列方向映射的定义.我们有如下分裂短正合列:
		$$\xymatrix{0\ar[r]&Z^{q,p}\ar[r]^i&M^{q,p}\ar[r]^{\delta}&B^{q+1,p}\ar[r]&0}$$
		$$\xymatrix{0\ar[r]&B^{q,p}\ar[r]^j&Z^{q,p}\ar[r]&H^{q,p}\ar[r]&0}$$
		
		那么列方向的映射$d$就是$i\circ j\circ\delta$.于是求列复形的同调就是$\ker Fd/\mathrm{im}Fd$.但是这里从$i,j$是单射得到$Fi,Fj$是单射(因为短正合列是分裂的,函子是加性就保证和直和可交换).于是$\ker Fd=\ker F(ij\delta)=\ker F\delta=\mathrm{im}Fi=Fi(FZ)$.还有$\mathrm{im}Fd=(Fd)(FM)=(Fi)[(Fj)(F\delta)(FM)]=(Fi)(Fj)(FB)$.于是有:
		$$\ker Fd/\mathrm{im}Fd=Fi(FZ)/(Fi)[(Fj)(FB)]\cong FZ/(Fj)(FB)=\mathrm{coker}Fj\cong FH$$
		
		最后一个同构因为$0\to B\to Z\to H\to0$分裂得到$0\to FB\to FZ\to FH\to0$也是正合的.综上对双复形$FM^{\bullet,\bullet}$求行复形的同调得到的是$FH^{q,p}$.按照定义$H^{q,\bullet}$是$H^q(GE^A)=(R^qG)A$的内射预解,所以再作用$F$取同调得到的是$(R^pF)(R^qG)A$.最后按照第二谱序列收敛到同调,我们证明了$H^n(\mathrm{Tot}(FM))\cong R^n(FG)A,\forall n$,说明第二谱序列也收敛到$R^n(FG)A$,也即$E_{\infty}^{p,q}=E_2^{p,q}=(R^pF)(R^qG)A$收敛到$R^n(FG)A$.
	\end{proof}
	\item 设$\xymatrix{\mathscr{A}\ar[r]^G&\mathscr{B}\ar[r]^F&\mathscr{C}}$是阿贝尔范畴之间的共变加性函子,设$\mathscr{A}$和$\mathscr{B}$和$\mathscr{C}$具有足够多的投射对象,设$F$是右正合的,对$\mathscr{A}$的每个投射对象$A$有$GA$是左$F$零调的.那么有第一象限谱序列$E^{\infty}_{p,q}=E^2_{p,q}=(L_pF)(L_qG)A$收敛到$L_n(FG)A$.
	\item 设$\xymatrix{\mathscr{A}\ar[r]^G&\mathscr{B}\ar[r]^F&\mathscr{C}}$是阿贝尔范畴之间的加性函子,$F$是逆变的$G$是共变的,设$\mathscr{A}$具有足够多的投射对象,设$\mathscr{B}$具有足够多的内射对象,设$\mathscr{C}$具有足够多的投射对象.设$F$是逆变左正合的,对$\mathscr{A}$的每个投射对象$A$有$GA$是右$F$零调的.那么有第三象限谱序列$E_{\infty}^{p,q}=E_2^{p,q}=(R^pF)(L_qG)A$收敛到$R^n(FG)A$.
	\item 设$\xymatrix{\mathscr{A}\ar[r]^G&\mathscr{B}\ar[r]^F&\mathscr{C}}$是阿贝尔范畴之间的逆变加性函子,设$\mathscr{A}$具有足够多的内射对象,设$\mathscr{B}$和$\mathscr{C}$具有足够多的投射对象.设$F$是左正合的,并且对任意$\mathscr{A}$中的内射对象$A$有$GA$是右$F$零调的.那么有第一象限谱序列$E^{\infty}_{p,q}=E^2_{p,q}=(R^pF)(R^qG)A$收敛到$R^n(FG)A$.
\end{enumerate}
\newpage
\section{层上同调}
\subsection{预层和层}
\subsubsection{基本内容}
\begin{enumerate}
	\item 预层.一个拓扑空间$X$上的取值在范畴$\mathscr{A}$中的预层,是指$X$上的开集范畴到范畴$\mathscr{A}$的逆变函子$\mathscr{F}$.这里$X$上的开集范畴定义为,它的对象是全体$X$上的开集,给定两个开集$U,V$,态射集$\mathrm{Hom}(U,V)$取单点集当且仅当$U\subset V$,否则$\mathrm{Hom}(U,V)$是空集.换句话讲:
	\begin{enumerate}[(1)]
		\item 对$X$的每个开集$U$,$\mathscr{F}(U)$是$\mathscr{A}$中的对象,它称为预层$\mathscr{F}$在$U$处的截面.提到$\mathscr{F}$的截面时指的是全集$X$的截面,它称为整体截面.截面通常也会记作$\Gamma(U,\mathscr{F})$或者$H^0(U,\mathscr{F})$.
		\item 对$X$中每个开集的包含关系$U\subset V$,唯一的存在$\mathscr{A}$中的态射$\mathrm{res}_{V,U}:\mathscr{F}(V)\to \mathscr{F}(U)$,它称为限制映射.限制映射满足两件事,首先$\mathrm{res}_{U,U}$是$\mathscr{F}(U)$上的恒等态射;另外如果有开集的包含链$U\subset V\subset W$,那么限制映射满足复合关系:
		$$\xymatrix{\mathscr{F}(W)\ar[rr]^{\mathrm{res}_{W,V}}\ar[dr]_{\mathrm{res}_{W,U}}&&\mathscr{F}(V)\ar[dl]^{\mathrm{res_{V,U}}}\\ &\mathscr{F}(U)&}$$
	\end{enumerate}
	\item 预层的态射.既然预层本质上是逆变函子,预层之间的态射理应定义为逆变函子之间的自然变换.严格讲即,如果约定了同一个范畴$\mathscr{A}$,并且给出两个$X$上的预层$\mathscr{F}_1,\mathscr{F}_2$,那么我们称预层之间的态射$\varphi:\mathscr{F}_1\to \mathscr{F}_2$,就是指一族被$X$上开集$U$作为指标的$\mathscr{A}$中的态射$\varphi(U):\mathscr{F}_1(U)\to \mathscr{F}_2(U)$,使得如下图表交换.预层和相应态射构成了范畴.我们把取值在范畴$\mathscr{C}$的全体$X$上预层和态射构成的范畴记作$\textbf{pSh}(X,\mathscr{C})$.
	$$\xymatrix{\mathscr{F}_1(V)\ar[d]_{\mathrm{res}_{V,U}}\ar[rr]^{\varphi(V)}&&\mathscr{F}_2(V)\ar[d]^{\mathrm{res}_{V,U}}\\\mathscr{F}_1(U)\ar[rr]_{\varphi(U)}&&\mathscr{F}_2(U)}$$
	
	空间$X$的取值在$\mathscr{A}$的预层构成的范畴记作$\textbf{pSh}(\mathscr{A})$.
	\item 层.
	\begin{enumerate}[(1)]
		\item 层是在预层的基础上再满足如下公理:任取开集$U$,任取$U$的开覆盖$\{U_i,i\in I\}$,记限制映射$p_i=\mathrm{res}_{U,U_i}$和$p_{ij}=\mathrm{res}_{U_i,U_i\cap U_j}$,任取$\mathscr{A}$中的对象$T$,那么有如下典范双射:
		$$\mathrm{Hom}_{\mathscr{A}}(T,\mathscr{F}(U))\cong\{(f_i\in\mathrm{Hom}_{\mathscr{A}}(T,\mathscr{F}(U_i)))_{i\in I}\mid\forall i,j\in I,p_{ij}\circ f_i=p_{ji}\circ f_j\}$$
		$$f\mapsto(p_i\circ f)_{i\in I}$$
		
		层之间的态射约定为它们作为预层之间的态射,于是在指定范畴$\mathscr{A}$后,层范畴是预层范畴的完全子范畴.空间$X$的取值在$\mathscr{A}$的层构成的范畴记作$\textbf{Sh}(\mathscr{A})$.
		\item 如果$\mathscr{A}$是集合范畴,或者是群环模等代数结构的范畴,这个公理等价于如下两条公理,也即约定了可以把局部的截面粘合为整体截面,其中一个是约定粘合的存在性,一个是约定粘合的唯一性.
		\begin{enumerate}[(a)]
			\item 恒等公理(粘合的唯一性).如果$\{U_i\}_{i\in I}$是开集$U$的开覆盖,并且$\forall a_1,a_2\in\mathscr{F}(U)$,满足对任意$i\in I$有$\mathrm{res}_{U,U_i}a_1=\mathrm{res}_{U,U_i}a_2$,那么有$a_1=a_2$.满足这一条的预层成为可分预层.
			\item 粘合公理(粘合的存在性).如果$\{U_i\}_{i\in I}$是开集$U$的开覆盖,并且给定$a_i\in\mathscr{F}(U_i),i\in I$,满足$\mathrm{res}_{U,U_i\cap U_j}a_i=\mathrm{res}_{U,U_i\cap U_j}a_j$对任意$i,j\in I$成立,那么存在一个$a\in \mathscr{F}(U)$满足$\mathrm{res}_{U,U_i}a=a_i,\forall i\in I$.
		\end{enumerate}
		\item $X$上的取值在交换群范畴$\textbf{Ab}$的预层和层分别称为阿贝尔预层和阿贝尔层.对于阿贝尔预层,它的层公理具有正合性描述:考虑如下图表,其中$d_0$限制在每个$\mathscr{F}(U_i)$上是限制映射$\mathrm{res}_{U,U_i}$.再定义$d_1$是$\mathrm{res}_{U_i,U_i\cap U_j}$,定义$d_2$是$\mathrm{res}_{U_j,U_i\cap U_j}$.恒等公理等价于讲$\mathscr{F}(U)$处是正合的,换句话说$d_0$是单射;粘合公理等价于讲两个映射$d_1,d_2$的等化子就是$d_0$的像,这可以理解为$\prod_i\mathscr{F}(U_i)$处的正合性.
		$$\xymatrix{
			\{\cdot\}\ar[r]&\mathscr{F}(U)\ar[r]^{d_0}&\prod_i \mathscr{F}(U_i)\ar[r]^{d_1-d_2}&\prod_{i,j}\mathscr{F}(U_{ij})}$$
	\end{enumerate}
	\item 如果$\mathscr{A}$表示群范畴,环范畴,模范畴之一,那么$X$的取值在$\mathscr{A}$上的预层或者层,复合上遗忘函子$\mathscr{A}\to\textbf{Sets}$自动也是一个预层或者层.反过来如果$X$的取值在$\mathscr{A}$中的预层$\mathscr{F}$视为取值在$\textbf{Sets}$的预层是一个层,那么$\mathscr{F}$也是一个层.但是不是所有范畴$\mathscr{A}$都满足这件事.
	\item 层在开集上的限制.设$\mathscr{F}$是空间$X$上的预层或层,设$U\subset X$是开集,定义$U$上开集$V$的截面就是$\mathscr{F}(V)$,这定义了$U$上的一个预层或层,称为$\mathscr{F}$在开集$U$上的限制,记作$\mathscr{F}\mid_U$.
	\item 子预层和商预层.我们考虑取值在$\mathscr{A}=$群范畴,环范畴,模范畴等代数结构范畴上的预层.
	\begin{enumerate}
		\item 子预层.设$\mathscr{F}$是$X$上的取值在$\mathscr{A}$的预层.它的子预层是指$X$上的预层$\mathscr{G}$,使得对每个开集$U\subset X$都有$\mathscr{G}(U)$是$\mathscr{F}(U)$的子群,满足只要$U\subset V$,就有$\mathrm{res}^\mathscr{F}_{V,U}(\mathscr{G}(V))\subset\mathscr{G}(U)$,这个映射作为$\mathscr{G}$上的限制映射.
		\item 商预层.定义预层$\mathscr{F}$关于子预层$\mathscr{G}$的商预层为这样一个预层$\mathscr{F}/\mathscr{G}$,它在$U$处的截面是商群$\mathscr{F}(U)/\mathscr{G}(U)$.如果$U\subset V$,定义限制映射$\mathrm{res}_{V,U}$为,经$\mathscr{F}$上限制映射$\mathscr{F}(V)\to \mathscr{F}(U)$所诱导的$\mathscr{F}(V)/\mathscr{G}(V)\to \mathscr{F}(U)/\mathscr{G}(U)$.这个定义良性是由子预层的定义所保证的.
		\item 层的子预层未必是一个层,不过这样的子预层必然满足粘合的唯一性.
		\item 给定预层之间的态射$\varphi:\mathscr{F}\to\mathscr{G}$,它的像定义为$\mathscr{G}$的子预层$\mathrm{im}\varphi$,它在$U$上的截面为$\mathrm{im}\varphi(U)$.
	\end{enumerate}
	\item 层的粘合.设$\mathscr{A}$上总存在逆向极限,设$\mathscr{U}=\{U_i\mid i\in I\}$是空间$X$的开覆盖,对每个指标$i$存在$U_i$上的取值在$\mathscr{A}$的层$\mathscr{F}_i$.满足对任意指标$i,j$,有层同构$\varphi_{ij}:\mathscr{F}_i\mid U_i\cap U_j\cong\mathscr{F}_j\mid U_i\cap U_j$,并且满足如下两个条件,那么存在$X$上同构意义下唯一的取值在$\mathscr{A}$中的层$\mathscr{F}$和一族层同构$\psi_i:\mathscr{F}\mid U_i\cong\mathscr{F}_i$.使得在$U_i\cap U_j$上总有$\psi_i=\psi_j$.这称为把$\{\mathscr{F}_i\}$经$\varphi_{ij}$粘合而成的层$\mathscr{F}$.
	\begin{enumerate}[(1)]
		\item $\forall i$有$\varphi_{ii}=\mathrm{id}$.
		\item 对任意指标$i,j,k$,在$U_i\cap U_j\cap U_k$上有$\varphi_{ik}=\varphi_{jk}\circ\varphi_{ij}$.
	\end{enumerate}
	\item 预层的茎.设范畴$\mathscr{A}$的正向极限总存在,设$\mathscr{F}$是空间$X$的取值在$\mathscr{A}$的层,任取$x\in X$.
	\begin{enumerate}[(1)]
		\item 记$x$的全部开邻域为$\{U_i,i\in I\}$,定义偏序$i\le j$如果$U_j\subseteq U_i$,则这是一个有向偏序集(此即任取指标$i,j$,则存在指标$k$使得$i\le k$和$j\le k$),于是$\{(u_i)_{i\in I},(\mathrm{res}_{i,j}:U_i\to U_j)_{i\ge j}\}$构成$\mathscr{A}$的正向系统,它的极限称为$\mathscr{F}$在点$x\in X$的茎(stalk),记作$\mathscr{F}_x$.
		\item 如果$U$是任意一个包含$x$的开集,任取$s\in\mathscr{F}(U)$,把$s$在正向极限$\mathscr{F}_x$中的像记作$s_x$,称为截面$s$在点$x$处的芽(germ).
	\end{enumerate}
\end{enumerate}
\subsubsection{阿贝尔层}
\begin{enumerate}
	\item 在阿贝尔层上,茎的性质可以决定层的性质:
	\begin{enumerate}[(1)]
		\item 如果$\varphi:\mathscr{F}\to\mathscr{G}$是$\textbf{pSh}(\textbf{Ab})$中的态射,任取$\mathscr{F}_x$中的一个元$x$,它的一个表示记作$s\in\mathscr{F}(U)$,那么$\varphi(U)(s)_x$不依赖于这个表示的选取,这定义了一个阿贝尔群同态,记作$\varphi_x$,称为$\varphi$在点$x$处的芽.于是特别的,固定$x\in X$,那么$\mathscr{F}\mapsto\mathscr{F}_x$和$\varphi\mapsto\varphi_x$是$\textbf{pSh}(\textbf{Ab})\to\textbf{Ab}$的函子,称为茎函子.
		\item 设$\mathscr{F}$是$X$上的可分阿贝尔预层,那么对任意开集$U$和任意$s,s'\in \mathscr{F}(U)$,有$s=s'$当且仅当$\forall p\in U$有$s_p=s'_p$.
		\begin{proof}
			
			充分性,对每个点$p\in U$,按照$s_p=s'_p$,可取$p$的开邻域$U_p$包含在$U$中,满足$\mathrm{res}_{U,U_p}(s)=\mathrm{res}_{U,U_p}(s')$.于是从唯一性公理得到$s=s'$.
		\end{proof}
		\item 如果$\mathscr{F}$是$X$上的阿贝尔预层,$\mathscr{G}$是$X$上的可分阿贝尔预层,两个态射$f,g:\mathscr{F}\to\mathscr{G}$满足诱导的每个茎之间的态射都相同$f_p=g_p,p\in X$,那么有$f=g$.特别的,层之间的态射被诱导的全体茎之间的态射所决定.
		\begin{proof}
			
			任取开集$U$,任取$s\in\mathscr{F}(U)$,需要验证的是$f(U)(s)=g(U)(s)$.任取$p\in U$,从$f_p(s_p)=g_p(s_p)$得到$(f(U)(s))_p=(g(U)(s))_p$.于是$f(U)(s)=g(U)(s)$.
		\end{proof}
		\item 给定$X$上两个阿贝尔层之间的态射$\varphi:\mathscr{F}\to\mathscr{G}$,那么$\varphi$是层的同构当且仅当对每个$p\in X$所诱导的茎之间的态射$\varphi_p$是阿贝尔群同构.
		\begin{proof}
			
			如果$\varphi$有逆态射$\psi$,按照函子性,每个$\psi_p$就是$\varphi_p$的逆态射,这就得到必要性.下面证明充分性,假设每个$\varphi_p$都是阿贝尔群同构,需要证明的是对每个开集$U$,有$\varphi(U)$是$\mathscr{F}(U)\to\mathscr{G}(U)$的同构.
			
			证明$\varphi(U)$是单射.假设$s\in\mathscr{F}(U)$满足$\varphi(U)(s)=0$,那么对每个$p\in U$有$(\varphi(U)(s))_p=0$,既然$\varphi_p$是单射,这就得到每个$s_p=0,p\in X$,于是按照层上的茎决定了截面中的元,得到$s=0$.
			
			证明$\varphi(U)$是满射.任取$t\in\mathscr{G}(U)$,按照每个$\varphi_p$是满射,于是可以取$s_p\in \mathscr{F}_p$使得$\varphi_p(s_p)=t_p$.设$s_p$的一个代表为$s(P)\in \mathscr{F}(V_p)$.于是$\varphi(V_p)(s(P))$和$\mathrm{res}_{U,V_p}(t)$是$\mathscr{G}(V_p)$中的两个元,它们在$p$处的芽相同,可令$V_p$缩小为合适的开集使得$\varphi(V_p)(s(P))=\mathrm{res}_{U,V_p}(t)$.现在$U$被全体$\{V_p,p\in U\}$所覆盖,按照粘合性公理,存在$s\in \mathscr{F}(U)$使得$\mathrm{res}_{U,V_p}(s)=s(P)$,最后按照截面上的元被茎所决定,得到$\varphi(U)(s)=t$,得到满射性.
		\end{proof}
	\end{enumerate}
	\item 支集.
	\begin{enumerate}[(1)]
		\item 设$\mathscr{F}$是空间$X$上的阿贝尔层,设$s$是开集$U$上的截面,定义$s$的支集$\mathrm{Supp}(s)=\{p\in U\mid s_p\not=0\}$.那么$\mathrm{Supp}(s)$是$U$上的闭集.
		\begin{proof}
			
			如果$s_p=0$,按照以有向集为指标集的正向极限的性质,就存在$p$的某个开邻域$V\subseteq U$使得$s\mid_V=0$,于是对每个$q\in V$有$s_q=0$,于是$\{q\in U\mid s_q=0\}$是$U$的开子集.
		\end{proof}
		\item 设$\mathscr{F}$是空间$X$上的阿贝尔层,它的支集定义为$\mathrm{Supp}(\mathscr{F})=\{p\in X\mid\mathscr{F}_p\not=0\}$.这未必是闭集.
	\end{enumerate}
	\item 层化.
	\begin{enumerate}
		\item 设$\mathscr{F}$是空间$X$上的阿贝尔预层,它的层化指的是$X$上的一个阿贝尔层$\widetilde{\mathscr{F}}$和一个预层态射$i_{\mathscr{F}}:\mathscr{F}\to\widetilde{\mathscr{F}}$,满足对任意层$\mathscr{G}$和一个预层态射$\varphi:\mathscr{F}\to\mathscr{G}$,都存在唯一的层态射$\widetilde{\varphi}:\widetilde{\mathscr{F}}\to\mathscr{G}$使得如下图表交换:
		$$\xymatrix{\widetilde{\mathscr{F}}\ar[rr]^{\exists! \widetilde{\varphi}}&&\mathscr{G}\\\mathscr{F}\ar[u]^{i_{\mathscr{F}}}\ar@/_1pc/[urr]_{\varphi}&&}$$
		\begin{proof}
			
			对任意开集$U\subseteq X$,定义:
			$$\widetilde{\mathscr{F}}(U)=\{(s_x)\in\prod_{x\in U}\mathscr{F}_x\mid\forall x\in U,\exists x\text{的开邻域} W\subseteq U,\text{和} t\in\mathscr{F}(W),\text{使得}\forall w\in W,s_w=t_w\}$$
			
			如果$U\subseteq V$,定义$\widetilde{F}(V)\to\widetilde{F}(U)$就是典范的投影映射$\prod_{x\in V}\mathscr{F}_x\to\prod_{x\in U}\mathscr{F}_x$.容易验证这是一个层.预层态射$i_{\mathscr{F}}:\mathscr{F}\to\widetilde{\mathscr{F}}$就取为$t\in\mathscr{F}(U)$映射为$(t_x)_{x\in U}\in\prod_{x\in U}\mathscr{F}_x$.
			
			\qquad
			
			最后验证泛性质.任取$X$上的层$\mathscr{G}$和预层态射$\varphi:\mathscr{F}\to\mathscr{G}$,那么$\widetilde{\varphi}$如果取为把$(s_x)_{x\in U}\in\widetilde{\mathscr{F}}(U)$映射为$\varphi_x(s_x)$所对应的(粘合得到的)$\mathscr{G}(U)$中的元可以保证图表交换.它的唯一性是因为满足这个等式的$\widetilde{\varphi}$必须满足$\widetilde{\varphi}_x=\varphi_x$,而阿贝尔层之间的态射被茎同态唯一决定.
		\end{proof}
		\item 按照上述证明,层化态射$i_{\mathscr{F}}$的茎同态都是同构.特别的如果$\mathscr{F}$本身已经是层,那么层化态射$i_{\mathscr{F}}$是同构.
		\item 层化的泛性质等价于讲包含函子$j:\textbf{Sh}(X)\to\textbf{pSh}(X)$右伴随于层化函子,更具体地讲,任取$X$上的预层$\mathscr{F}$和层$\mathscr{G}$总有如下典范同构.特别的,这说明$\textbf{Sh}(X)$是$\textbf{pSh}(X)$的反射子范畴.
		$$\mathrm{Hom}_{\textbf{Sh}(X)}(\Gamma\Phi(\mathscr{F}),\mathscr{G})\cong\mathrm{Hom}_{\textbf{pSh}(X)}(\mathscr{F},i(\mathscr{G}))$$
	\end{enumerate}
	\item 阿贝尔(预)层范畴是阿贝尔范畴.
	\begin{enumerate}[(a)]
		\item 我们解释过包含函子右伴随于层化函子.这使得$\textbf{pSh}(X)$是$\textbf{Sh}(X)$的反射子范畴(就是一个完全充足子范畴,使得包含函子存在左伴随函子,这样的左伴随函子称为反射).在这个条件下,一个由层构成的图表在层范畴中的极限自然同构于它在预层范畴中的极限;一个由层构成的图表在层范畴中的余极限自然同构于它在预层范畴中的余极限的层化.
		\item 先验证阿贝尔(预)层范畴是加性范畴.
		\begin{enumerate}[(1)]
			\item 零对象,或者说空集作为指标集的直和与直积.全体截面均取平凡群的预层称为零层(它同样是一个层),这是层范畴和预层范畴上的零对象.
			\item 零态射,预层范畴和层范畴上的零态射恰好就是每个截面上是交换群之间的零同态的态射.
			\item Hom集合的加性.设$\mathscr{F},\mathscr{G}$是$X$上的两个预层,任取$f,g\in\mathrm{Hom}(\mathscr{F},\mathscr{G})$,那么取$(f+g)(U):\mathscr{F}(U)\to\mathscr{G}(U)$为$s\mapsto f(U)(s)+g(U)(s)$.其中零元为把$\mathscr{F}(U)$中每个元都映射为$\mathscr{G}(U)$中零元的自然变换;$f$的逆元$-f$为$-f(U):-f(U)(s)\in\mathscr{G}(U)$的自然变换.容易验证这个加法下复合满足分配律,即$q\circ(f+g)=q\circ f+q\circ g$和$(f+g)\circ p=f\circ p+g\circ p$.
			\item 二元对偶积.设$\mathscr{F}$和$\mathscr{G}$是$X$上的两个阿贝尔预层,定义$X$上的预层$\mathscr{F}\oplus\mathscr{G}$为,在每个开集$U$上的截面群定义为$(\mathscr{F}\oplus\mathscr{G})(U)=\mathscr{F}(U)\oplus\mathscr{G}(U)$.限制映射自然的定义为$\mathrm{res}_{U,V}^\mathscr{F}\oplus\mathrm{res}_{U,V}^\mathscr{G}$.定义预层态射$i_f:\mathscr{F}\to \mathscr{F}\oplus\mathscr{G}$为$s\mapsto(s,0)$;$i_{\mathscr{G}}:\mathscr{G}\to\mathscr{F}\oplus\mathscr{G}$为$t\mapsto(0,t)$;$\pi_\mathscr{F}:\mathscr{F}\oplus\mathscr{G}\to\mathscr{F}$为$(s,t)\mapsto s$;$\pi_\mathscr{G}:\mathscr{F}\oplus\mathscr{G}\to\mathscr{G}$为$(s,t)\mapsto t$.那么可验证$\pi_\mathscr{F}\circ i_\mathscr{F}=1_\mathscr{F}$,$\pi_\mathscr{G}\circ i_\mathscr{G}=1_\mathscr{G}$,$\pi_\mathscr{G}\circ i_\mathscr{F}=\pi_\mathscr{F}\circ i_\mathscr{G}=0$,$i_\mathscr{F}\circ\pi_\mathscr{F}+i_\mathscr{G}\circ\pi_\mathscr{G}=1_{\mathscr{F}\oplus\mathscr{G}}$.于是$(\mathscr{F}\oplus\mathscr{G},i_\mathscr{F},i_\mathscr{G},\pi_\mathscr{F},\pi_\mathscr{G})$是预层范畴和层范畴上的二元对偶积.
		\end{enumerate}
		\item 直和与直积.
		\begin{enumerate}[(1)]
			\item 给定一族$X$上的预层$\{\mathscr{F}_i\}$,它的直积定义为$(\prod_i\mathscr{F}_i)(U)=\prod_i\mathscr{F}_i(U)$,限制映射定义为在每个分量上取$\mathscr{F}_i$的限制映射.这个构造也是层范畴上的直积.
			\item 给定一族$X$上的预层$\{\mathscr{F}_i\}$,它的直和定义为$(\oplus_i\mathscr{F}_i)(U)=\oplus_i\mathscr{F}_i(U)$,限制映射定义为在每个分量上取$\mathscr{F}_i$的限制映射.如果每个$\mathscr{F}_i$都是层,这样构造出来的直和未必是层,因为截面不是总能粘合为直和中的整体截面(粘合后未必分量只有有限个不为零).它的层化是层范畴中的直和.
		\end{enumerate}
		\item 核与余核.
		\begin{enumerate}[(1)]
			\item 预层范畴上态射的核总是存在的.给定预层的态射$\varphi:\mathscr{F}\to\mathscr{G}$,它的核$K$为$\mathscr{F}$的子预层,满足$K(U)=\ker\varphi(U)$.
			\item 设$f:\mathscr{F}\to\mathscr{G}$是预层的态射,其中$\mathscr{F}$是层而$\mathscr{G}$是可分层,那么核$\ker f$是一个层.于是在层范畴上态射的核同样是总存在的.
			\begin{proof}
				
				验证唯一性公理,即对开集$U$,和$U$的任意开覆盖$\{V_i\}$,如果$x\in K(U)=\ker f(U)$,满足$x\mid_{V_i}=0$,那么有$x=0$.但是$K(U)$实际上是$\mathscr{F}(U)$的子集,于是按照$\mathscr{F}$上有唯一性公理,就得到$x=0$.
				
				验证粘合性公理,即对开集$U$,和$U$的任意开覆盖$\{V_i,i\in I\}$,对每个$i\in I$取了一个元$x_i\in K(V_i)$,满足如果$i,j\in I$,就有$x_i\mid_{V_i\cap V_j}=x_j\mid_{V_i\cap V_j}$,需要验证的是存在元$x\in K(U)$使得$x\mid_{V_i}=x_i$.但是同样的按照$K(U)$实际上是$\mathscr{F}(U)$的子集,从$\mathscr{F}$上的粘合性公理得到存在$x\in \mathscr{F}(U)$满足$x\mid_{V_i}=x_i$.最后仅需要验证$x\in K(U)$中.这是因为$x'=\mathscr{F}(U)(x)$满足$\mathrm{res}_{U,V_i}(x')=\mathscr{F}(V_i)(x_i)=0$,于是从$\mathscr{G}$上的唯一性公理得到$x'=0$,也即$x\in K(U)$.
			\end{proof}
			\item 预层范畴上的余核总是存在的.给定预层之间的态射$f:\mathscr{F}\to\mathscr{G}$,它的余核定义为$\mathscr{G}$关于子预层$\mathrm{im}\mathscr{F}$的商预层$\mathrm{Coker}f$.这个定义满足余核的泛映射性质:如果预层$\mathscr{H}$和态射$g:\mathscr{G}\to\mathscr{H}$满足$g\circ f=0$,那么存在唯一的态射$\mathrm{coker}f\to\mathscr{H}$使得复合$\mathscr{G}\to\mathrm{coker}f\to\mathscr{H}$就是$g:\mathscr{G}\to\mathscr{H}$.
			\item 层范畴上的余核总是存在的.它是在预层范畴上取余核后的层化.
			\item 预层范畴上态射$\varphi:\mathscr{F}\to\mathscr{G}$的像就取为$\mathscr{G}$的子预层$\mathrm{im}\varphi$.如果是层范畴上态射的像,取为它作为预层范畴上态射的像的层化.另外按照层化的泛性质,有唯一诱导的层态射$\mathrm{im}\varphi^+\to\mathscr{G}$,这个态射是单的,所以在层的情况下像也可以视为终端的子层.
			\item 无论是预层还是层上的核,余核,像,都与茎函子可交换.以核为例:设$\varphi:\mathscr{F}\to\mathscr{G}$是预层(或者层)之间的态射,那么对每个$x\in X$都有$(\ker\varphi)_x=\ker(\varphi_x)$.
			\begin{proof}
				
				选取$t\in(\ker\varphi)_x$,那么有$x$的开邻域$U$和$s\in\mathscr{F}(U)$,使得$s_x=t$并且$\varphi(U)(s)=0$.于是有$\varphi_x(s_x)=0_x=0$,所以$t\in\ker(\varphi_x)$,这说明$(\ker\varphi)_x\subset\ker(\varphi_x)$.反过来任取$s_x\in\ker\varphi_x$,选取$s_x$的代表元$(s,V)$,记$t=\varphi(V)(s)$,那么$t_x=0$,所以存在$x$的足够小的开邻域$V\subset U$使得$t\mid V=0$.那么$\varphi(V)(s\mid V)=0$,这说明$s_x\in(\ker\varphi)_x$.
			\end{proof}
		\end{enumerate}
		\item 单满态射,同构及等价刻画.
		\begin{enumerate}[(1)]
			\item 给定预层之间的态射$f:\mathscr{F}\to\mathscr{G}$,那么如下前三条等价,并且它们可推出第四条,并且在预层$\mathscr{F}$可分的情况下是等价于第四条的:
			\begin{enumerate}[(a)]
				\item $\ker f=0$.
				\item 对每个开集$U\subset X$,有$\mathscr{F}(U)$是单射.
				\item $\mathscr{F}$是预层范畴上的一个单态射,即从$\mathscr{F}\circ g=\mathscr{F}\circ h$总能推出$g=h$,按照阿贝尔预层范畴是加性范畴,这也等价于讲从$\mathscr{F}\circ k=0$总能推出$k=0$.
				\item 对每个点$x\in X$,有$\mathscr{F}_x$是单射.
			\end{enumerate}
			\begin{proof}
				
				(a)和(b)的等价性是直接的.现在证明(a)推(c),假设$\mathscr{F},k$满足$\mathscr{F}\circ k=0$,按照核的泛映射性质,$k:H\to \mathscr{F}$可分解为态射的复合$H\to\ker f\to \mathscr{F}$,按照$\ker f=0$,就得到$k=0$.下面证明(c)推(a),为此只要验证$0\to \mathscr{F}$满足核的泛映射性质.也即对任意的态射$g:H\to \mathscr{F}$,满足$\mathscr{F}\circ g=0$,那么存在唯一的态射$g':H\to 0$使得$0\circ g'=g$.但是按照$\mathscr{F}\circ 0=\mathscr{F}\circ g$,从$\mathscr{F}$是单态射得到$g=0$,因而唯一的$g'=0$.
				
				对任意预层$\mathscr{F}$验证(b)推(d):设$t\in \mathscr{F}_x$满足$\mathscr{F}_x(t)=0$,不妨设$t$的一个代表为$s\in \mathscr{F}(U)$.于是$(\mathscr{F}(U)(s))_x=0$,于是存在开集$V\subset U$满足$0=\mathrm{res}_{U,V}(\mathscr{F}(U)(s))=\mathscr{F}(V)(s\mid_V)$.但是按照$\mathscr{F}(U)$是单射,得到$s\mid_V=0$,因而$t=0$.
				
				对可分预层$\mathscr{F}$验证(d)推(b):假设$s\in \mathscr{F}(U)$满足$\mathscr{F}(U)(s)=0\in\mathscr{G}(U)$.那么对每个$x\in U$有$\mathscr{F}_x(s_x)=(\mathscr{F}(U)(s))_x=0$,按照条件$\mathscr{F}_x$是单射,得到$s_x=0,\forall x\in U$,于是唯一性公理得到$s=0$.
			\end{proof}
			\item 给定预层之间的态射$f:\mathscr{F}\to\mathscr{G}$,那么如下三条等价:
			\begin{enumerate}[(1)]
				\item $\mathrm{Coker}f=0$.
				\item 对每个开集$U\subset X$,有$\mathscr{F}(U)$是满射.
				\item $\mathscr{F}$是预层范畴上的一个满态射,即从$g\circ \mathscr{F}=h\circ \mathscr{F}$总能推出$g=h$,按照阿贝尔预层范畴是加性范畴,这也等价于讲从$k\circ \mathscr{F}=0$总能推出$k=0$.
			\end{enumerate}
			\item 给定层之间的态射$f:\mathscr{F}\to\mathscr{G}$,那么如下五条等价,并且上一条中的三个条件总可以推出这四条中的任一:
			\begin{enumerate}[(1)]
				\item $\mathrm{SCoker}f=0$.
				\item $\mathrm{im}f=\mathscr{G}$.
				\item 对每个点$x\in X$,有$(\mathrm{Coker}f)_x=0$.
				\item 对每个点$x\in X$,有$\mathscr{F}_x$是满射.
				\item $\mathscr{F}$是层范畴上的一个满态射.
			\end{enumerate} 
			\item 设$f:\mathscr{F}\to\mathscr{G}$是预层之间的态射,那么前三条等价,另外如果$\mathscr{F},G$都是层,那么如下五条均等价:
			\begin{enumerate}[(1)]
				\item $\mathscr{F}$是预层的同构.
				\item 对每个开集$U$有$\mathscr{F}(U)$是双射.
				\item $\mathscr{F}$是单态射和预层的满态射.
				\item $\mathscr{F}$是单态射和层的满态射.
				\item 对任意$x\in X$有$\mathscr{F}_x$是同构.
			\end{enumerate}
			\item 例子.设$X=\mathbb{C}-\{0\}$,设$\mathscr{F}$是$X$上的全纯函数层,设$\mathscr{G}$是$X$上的可逆全纯函数层.考虑层态射$\alpha:\mathscr{F}\to\mathscr{G}$为,把开集$U\subseteq X$上的全纯函数$f$映射为$U$上的可逆全纯函数$\exp(f)$.那么$\alpha$是一个满态射,但是$\alpha(X)$不是满射,因为$X$上的恒等映射不能写成某个全纯函数的$\exp$.
		\end{enumerate}
		\item 关于正向极限(范畴余极限).我们解释过层的范畴极限仍然是层,于是特别的层的逆向极限总是层.对于正向极限我们有如下结论:如果$\{\mathscr{F}_i\}$是诺特空间$X$上层的正向系统,那么正向极限也是层.特别的有截面函子与正向极限可交换.
		\item 综上阿贝尔层范畴和阿贝尔预层范畴都是阿贝尔范畴.
	\end{enumerate}
	\item 正合性.
	\begin{enumerate}[(a)]
		\item 像.在一般阿贝尔范畴上态射的像定义为余核的核.阿贝尔预层范畴上态射$\mathscr{F}$的像就是我们之前定义的$\mathrm{Im}\mathscr{F}$,阿贝尔层范畴上态射$\mathscr{F}$的像称为层像,记作$\mathrm{SIm}\mathscr{F}$.
		\item 正合性.给定一族预层/层$\mathscr{F}_i$和一族预层/层态射$\mathscr{F}_i$,表示为$\xymatrix{\cdots\ar[r]&\mathscr{F}_{i+1}\ar[r]^{\mathscr{F}_{i+1}}&\mathscr{F}_i\ar[r]^{\mathscr{F}_i}&\mathscr{F}_{i-1}\ar[r]&\cdots}$.称它在$\mathscr{F}_i$处正合,如果满足$\mathrm{PIm}\mathscr{F}_{i+1}=\ker f_i$($\mathrm{SIm}\mathscr{F}_{i+1}=\ker f_i$).称它是正合列,如果在每个$\mathscr{F}_i$处都正合.预层上的正合性可用截面的正合性描述,层上的正合性可用茎的正合性描述:
		\begin{enumerate}[(1)]
			\item $\mathscr{F}\to\mathscr{G}\to\mathscr{H}$是预层的正合列当且仅当对每个开集$U$,有$\mathscr{F}(U)\to\mathscr{G}(U)\to\mathscr{H}(U)$是阿贝尔群的正合列.
			\item $\mathscr{F}\to\mathscr{G}\to\mathscr{H}$是层的正合列当且仅当对每个点$x\in X$,有$\mathscr{F}_x\to\mathscr{G}_x\to\mathscr{H}_x$是阿贝尔群的正合列.
			\item 如果层的列$\mathscr{F}\to\mathscr{G}\to\mathscr{H}$是预层范畴上的正合列,那么它也是层范畴上的正合列.注意这仅说明任意开集$U$上有正合列$\mathscr{F}(U)\to\mathscr{G}(U)\to\mathscr{H}(U)$可推出层$\mathscr{F}\to\mathscr{G}\to\mathscr{H}$是正合的,但是反过来不成立.
		\end{enumerate}
		\item 正合性的一些例子.
		\begin{enumerate}[(1)]
			\item $\xymatrix{0\ar[r]&\mathscr{F}\ar[r]^f&\mathscr{G}}$是正合列等价于讲$f$是单态射.
			\item $\xymatrix{\mathscr{F}\ar[r]^f&\mathscr{G}\ar[r]&0}$是正合列等价于讲$g$是满态射.
			\item 对任意态射$f:\mathscr{F}\to\mathscr{G}$,在预层范畴上总有$\xymatrix{0\ar[r]&\ker f\ar[r]&\mathscr{F}\ar[r]&\mathscr{G}\ar[r]&\mathrm{PIm}\mathscr{F}\ar[r]&0}$;在层范畴上总有$\xymatrix{0\ar[r]&\ker f\ar[r]&\mathscr{F}\ar[r]&\mathscr{G}\ar[r]&\mathrm{SIm}\mathscr{F}\ar[r]&0}$.
			\item $\xymatrix{0\ar[r]&\mathscr{F}\ar[r]^f&\mathscr{G}\ar[r]^g&\mathscr{H}}$是正合列等价于讲$\mathscr{F}$是$g$的核.
			\item $\xymatrix{\mathscr{F}\ar[r]^f&\mathscr{G}\ar[r]^g&\mathscr{H}\ar[r]&0}$是正合列等价于讲$g$是$\mathscr{F}$的余核.
			\item $\xymatrix{0\ar[r]&\mathscr{F}\ar[r]&\mathscr{G}\ar[r]&0}$是正合列等价于讲$\mathscr{F}$和$\mathscr{G}$同构.
		\end{enumerate}
		\item 层化函子和包含函子的正合性.
		\begin{enumerate}[(1)]
			\item 它们都是加性函子,这等价于它们都和二元对偶积可交换.
			\item 从阿贝尔层范畴到阿贝尔预层范畴的包含函子是左正合的.这是因为它是一个右伴随函子.
			\item 层化函子是正合的.它是右正合的因为是左伴随函子,它是左正合的因为它与核可交换.
		\end{enumerate}
		\item 截面函子.固定$X$的一个开子集$U$,构造预层范畴上的截面函子$\Gamma(U,-):\textbf{pSh}(X)\to\textbf{Ab}$为,对每个$X$上的预层$\mathscr{F}$,定义$\Gamma(U,\mathscr{F})=\mathscr{F}(U)$,对每个态射$\varphi:\mathscr{F}\to\mathscr{G}$,定义$\Gamma(U,\varphi)=\varphi(U):\mathscr{F}(U)\to\mathscr{G}(U)$.这个函子也可以构造在层范畴上.
		\begin{enumerate}[(1)]
			\item 无论是预层范畴还是层范畴上的截面函子,它都是加性的,也即都和二元对偶积可交换.
			\item 预层范畴上的截面函子$\Gamma(U,-):\textbf{pSh}(X)\to\textbf{Ab}$是正合的.于是特别的预层上的截面函子和所有范畴极限和范畴余极限可交换.
			\item 层范畴上的截面函子$\Gamma(U,-):\textbf{Sh}(X)\to\textbf{Ab}$是左正合的.这是因为截面函子与核可交换.于是特别的层上的截面函子和所有范畴极限可交换.
			\item 如果$X$是诺特空间,那么一个阿贝尔层的正向系统的极限仍然是一个层,于是按照预层上截面函子是正合的,就说明在$X$是诺特空间的前提下截面函子还与正向极限可交换.
		\end{enumerate}
	\end{enumerate}
\end{enumerate}
\subsubsection{拓扑基上的预层}

设$\mathscr{A}$上总存在逆向极限.
\begin{enumerate}
	\item 拓扑基上的预层.设$X$是拓扑空间,设$\mathscr{B}$是一组拓扑基.称$\mathscr{B}$上的一个取值在$\mathscr{A}$的预层$\mathscr{F}$,是指对每个$B\in\mathscr{B}$,定义$\mathscr{F}(B)$是$\mathscr{C}$中的一个对象.使得只要两个基元素有包含关系$U\subset V$,那么指定一个态射$\mathrm{res}_{V,U}:\mathscr{F}(V)\to\mathscr{F}(U)$.满足:$\mathrm{res}_{U,U}=\mathrm{id}_{\mathscr{F}(U)}$,以及$\mathrm{res}_{W,U}=\mathrm{res}_{V,U}\circ\mathrm{res}_{W,V}$,其中$U\subset V\subset W$.
	\item 如果$\mathscr{F}$是拓扑基$\mathscr{B}$上的预层,那么我们总可以定义$X$上的预层$\mathscr{F}'$为,对任意开子集$U\subseteq X$,取$\mathscr{F}'(U)=\varprojlim\mathscr{F}(B)$,其中$B\in\mathscr{B}$跑遍包含于$U$的基元素.如果开集$U\subset V\subset X$,限制映射是逆向极限之间唯一满足如下图表交换的同态:
	$$\xymatrix{\varprojlim\limits_{W\subset V,W\in\mathscr{B}}\mathscr{F}(W)\ar[d]&&\varprojlim\limits_{W\subset U,W\in\mathscr{B}}\mathscr{F}(W)\ar@/^1pc/[dll]\ar[ll]_{\mathrm{res}_{U,V}'}\\F(W);W\subset V,W\in\mathscr{B}&&}$$
	\item 我们断言这样定义出来的预层$\mathscr{F}'$是层,当且仅当$\mathscr{F}$是$\mathscr{B}$上的层,也即:对任意$B\in\mathscr{B}$和$\mathscr{B}$中元素所构成的开覆盖$B=\cup_{i\in I}B_i$,记限制$p_i=\mathrm{res}_{B,B_i}$和$p_{ij}^V=\mathrm{res}_{B_i,V}$,其中$V\subseteq B_i\cap B_j$是$\mathscr{B}$中的元,对$\mathscr{A}$的任意对象$T$,都有如下典范双射:
	$$\mathrm{Hom}_{\mathscr{A}}(T,\mathscr{F}(B))\cong\{(f_i\in\mathrm{Hom}_{\mathscr{A}}(T,\mathscr{F}(B_i)))_{i\in I}\mid\forall i,j\in I,\forall U_i\cap U_j\supseteq V\in\mathscr{B},p_{ij}^V\circ f_i=p_{ji}^V\circ f_j\}$$
	\begin{proof}
		
		必要性明显成立.我们先证明如果$\mathscr{B}'\subseteq\mathscr{B}$是另一个拓扑基,把$\mathscr{F}$限制在这个拓扑基上,所延拓得到的$X$上的预层$\mathscr{F}''$和$\mathscr{F}'$是典范同构的:任取开子集$U\subseteq X$,按照$\mathscr{F}''(U)$是$\mathscr{F}'(U)$对应的逆向系统的子系统的极限,于是有典范态射$\mathscr{F}'(U)\to\mathscr{F}''(U)$,并且如果$U\in\mathscr{B}$则这个典范态射是同构.于是$\mathscr{F}''(U)$就满足$\mathscr{F}'(U)$的泛性质,于是它们典范同构.
		
		\qquad
		
		下面设$U\subseteq X$是任意开集,设$\{U_i,i\in I\}$是$U$的任意开覆盖,考虑$\mathscr{B}$的那些至少包含在一个$U_i$中的基元素构成的子集族$\mathscr{B}'$,这也是拓扑基,于是按照上一段,不妨设$\mathscr{B}$本身已经满足每个基元素都包含在某个$U_i$中,那么此时$\mathscr{B}$满足层公理就得到$\mathscr{F}'$满足层公理.
	\end{proof}
	\item 取$X$的一组拓扑基$\mathscr{B}$,设$\mathscr{F}$是$\mathscr{B}$上的一个层,那么它可以在同构意义下唯一的延拓为$X$上的一个层$\mathscr{F}'$.它在开子集$U\subset X$上的截面定义为一个逆向极限:
	$$\mathscr{F}'(U)=\lim\limits_{\substack{\leftarrow\\ V\in\mathscr{B},V\subset U}}\mathscr{F}(V)$$
	\item 态射.设$\mathscr{F},\mathscr{G}$是拓扑基$\mathscr{B}$上的两个预层,它们之间的态射$\varphi:\mathscr{F}\to\mathscr{G}$就定义为一族态射$\{\mathscr{F}(V)\to\mathscr{G}(V),V\in\mathscr{B}\}$,使得它们和限制映射可交换.于是按照逆向系统之间的态射诱导了逆向极限之间的态射,$\varphi$就诱导了一个预层态射$\varphi':\mathscr{F}'\to\mathscr{G}'$.
	\item 茎.如果$\mathscr{A}$还总存在正向极限,任取$x\in X$,那么$\mathscr{B}$中的那些包含$x$的基元素是$X$的包含$x$的全体开子集上的共尾子集族,这导致有典范同构$\mathscr{F}_x\cong\mathscr{F}_x'$;类似的如果$\varphi:\mathscr{F}\to\mathscr{G}$是$\mathscr{B}$上两个预层之间的态射,那么有典范同构$\varphi_x\cong\varphi'_x$.
	\item 层和层态射已经被它在拓扑基上的限制完全决定:如果$\mathscr{F}_0$是$X$上的层,设它限制在拓扑基$\mathscr{B}$上为层$\mathscr{F}$,那么按照逆向极限的泛性质,$\mathscr{F}$诱导的$X$上的层$\mathscr{F}'$和最初的$\mathscr{F}_0$是典范同构的.类似的如果$\varphi_0:\mathscr{F}_0\to\mathscr{G}_0$是$X$上的层态射,它限制在拓扑基$\mathscr{B}$上得到一个层态射$\varphi:\mathscr{F}\to\mathscr{G}$,那么它诱导的$X$上的层态射$\varphi'$和最初的$\varphi_0$是典范同构的.
\end{enumerate}
\subsubsection{顺像函子}

给定拓扑空间之间的连续映射$\varphi:X\to Y$,对$X$上的每个$\textbf{A}$值预层$\mathscr{F}$,定义$Y$上的预层$\varphi_*(\mathscr{F})$为,开集$U\subset Y$上的截面定义为$\mathscr{F}(\varphi^{-1}(U))$;如果$Y$上有开集$V\subset U$,定义限制映射为$\mathrm{res}_{U,V}=\mathrm{res}_{\varphi^{-1}(U),\varphi^{-1}(V)}$.给定$X$上两个预层之间的态射$f:\mathscr{F}\to\mathscr{G}$,定义诱导的态射$\varphi_*f:\varphi_*\mathscr{F}\to\varphi_*G$为$\varphi_*\mathscr{F}(U)=\mathscr{F}(\varphi^{-1}(U))$.称$\varphi_*$是顺像函子.
\begin{enumerate}
	\item $\varphi_*$是从$\textbf{pSh}(X)\to\textbf{pSh}(Y)$的加性函子,也即和二元对偶积可交换.
	\item 如果$\mathscr{F}$是$X$上的层,那么$\varphi_*\mathscr{F}$是$Y$上的层,于是$\varphi$也是从$\mathrm{Sh}(X)\to\mathrm{Sh}(Y)$的加性函子.这个证明是直接的,因为$\varphi_*\mathscr{F}$的截面实际上都是被层$\mathscr{F}$提供的,可直接验证粘合的存在性和唯一性.
	\item 单点空间上的阿贝尔层范畴等价于它上面的阿贝尔预层范畴,等价于交换群范畴$\textbf{Ab}$.于是如果取$Y$是单点集合,取$\varphi:X\to Y$是唯一的映射,那么此时$\varphi_*$就是$X$上的整体截面函子.
	\item 层范畴上的前推函子总是左正合的.事实上任取层范畴上的左正合列$0\to \mathscr{F}\to\mathscr{G}\to\mathscr{H}$,对每个$Y$中的开集$V$,记$\varphi^{-1}(V)=U$是$X$中的开集.按照截面函子是左正合的,得到$0\to \mathscr{F}(U)\to\mathscr{G}(U)\to\mathscr{G}(U)$是交换群上的正合列,于是前推函子是左正合的.
	\item 关于茎映射.任取$x\in X$,任取$\varphi(x)$的开邻域$V$,那么有典范映射$\Gamma(\varphi^{-1}(V),\mathscr{F})\to\mathscr{F}_x$,按照正向极限的泛性质这就诱导了$\varphi_x:(\varphi_*\mathscr{F})_{\varphi(x)}\to\mathscr{F}_x$.它具有函子性,具体地讲,如果$u:\mathscr{F}_1\to\mathscr{F}_2$是$X$上两个预层之间的态射,那么有如下交换图表.另外如果$\varphi$是拓扑嵌入,那么$\varphi_x$是同构.
	$$\xymatrix{(\varphi_*\mathscr{F}_1)_{\varphi(x)}\ar[rr]^{\varphi_x}\ar[d]_{(\varphi_*u)_{\varphi(x)}}&&(\mathscr{F}_1)_x\ar[d]^{u_x}\\(\varphi_*\mathscr{F}_2)_{\varphi(x)}\ar[rr]^{\varphi_x}&&(\mathscr{F}_2)_x}$$
	\item 关于支集.考虑阿贝尔预层,$\varphi_*\mathscr{F}$的支集包含在$\overline{\varphi(X)}$中,但是通常它不包含在$\varphi(X)$中.
	\item 例子:摩天大楼层.设$X$是拓扑空间,任取一个点$x$,任取一个阿贝尔群$A$,定义摩天大楼层$(x_*A)$为,如果开集$U$包含了点$x$则定义截面为$A$,否则定义截面是平凡群.限制映射自然的定义为要么是恒等映射$1_A$,要么是到平凡群的平凡映射.这是一个层,它在闭包$\overline{\{x\}}$中的点的茎是$A$,其余都是平凡群.可以理解为它真正的信息只保留在$x$附近,所以称为摩天大楼层.如果记$i$是闭子空间$Y=\overline{\{x\}}$到$X$的包含映射,设$\mathscr{F}$是$Y$上关于$A$的常值层,那么$X$上的摩天大楼层就是前推层$i_*\mathscr{F}$.
\end{enumerate}
\subsubsection{逆像函子}

\begin{enumerate}
	\item 预层之间的$\varphi$态射.设$\varphi:X\to Y$是连续映射,设$\mathscr{F}$和$\mathscr{G}$分别是$X$和$Y$上的$\mathscr{A}$值预层.称$Y$上的预层态射$u:\mathscr{G}\to\varphi_*\mathscr{F}$为$\mathscr{G}\to\mathscr{F}$的$\varphi$态射,全体这样的态射构成的集合记作$\mathrm{Hom}_{\varphi}(\mathscr{G},\mathscr{F})$.一个$\varphi$态射$u:\mathscr{G}\to\mathscr{F}$完全等价于如下信息:对任意一组$(U,V)$,其中$U\subseteq X$和$V\subseteq Y$都是开集,并且满足$\varphi(U)\subseteq V$,都取定一个态射$u_{U,V}:\mathscr{G}(V)\to\mathscr{F}(U)$(也即典范态射的复合$\xymatrix{\mathscr{G}(V)\ar[r]&\mathscr{F}(\varphi^{-1}(V))\ar[r]&\mathscr{F}(U)}$),使得只要$(U',V')\subseteq(U,V)$就有如下交换图表:
	$$\xymatrix{\mathscr{G}(V)\ar[rr]^{u_{U,V}}\ar[d]&&\mathscr{F}(U)\ar[d]\\\mathscr{G}(V')\ar[rr]^{u_{U',V'}}&&\mathscr{F}(U')}$$
	\item 层之间的$\varphi$态射.设$\mathscr{A}$上总存在逆向极限,设$\mathfrak{B}$和$\mathfrak{B}'$分别是$X$和$Y$的一组拓扑基,那么定义层之间的$\varphi$态射$u$,只需对满足$U\in\mathfrak{B}$,$V\in\mathfrak{B}'$,$\varphi(U)\subseteq V$的$(U,V)$定义态射$u_{U,V}$,使得上一条中的图表交换即可.
	\item $\varphi$态射的茎映射.设$\mathscr{A}$上总存在正向极限,任取$x\in X$和$\varphi(x)$的开邻域$V$,任取$\varphi$态射$\mathscr{G}\to\mathscr{F}$,那么有典范的复合映射$\mathscr{G}(V)\to\mathscr{F}(\varphi^{-1}(V))\to\mathscr{F}_x$.进而诱导了态射$\mathscr{G}_{\varphi(x)}\to\mathscr{F}_x$.
	\item $\varphi$态射的复合.设$\mathscr{F},\mathscr{G},\mathscr{H}$分别是空间$X,Y,Z$上的$\mathscr{A}$值预层.设$\varphi:X\to Y$和$\psi:Y\to Z$是连续映射.设$u:\mathscr{G}\to\varphi_*\mathscr{F}$和$v:\mathscr{H}\to\psi_*\mathscr{G}$分别是$\varphi$态射和$\psi$态射.那么它们的复合就典范的定义为$w:\xymatrix{\mathscr{H}\ar[r]^{v}&\psi_*\mathscr{G}\ar[r]^{\psi_*u}&\psi_*\varphi_*\mathscr{F}}$.
	\item 综上我们可以定义如下范畴:它的对象是二元组$(X,\mathscr{F})$,其中$X$是拓扑空间,$\mathscr{F}$是其上$\mathscr{A}$值预层.$(X,\mathscr{F})\to(Y,\mathscr{G})$的态射定义为二元组$(\varphi,u)$,其中$\varphi:X\to Y$是连续映射,$u:\mathscr{G}\to\mathscr{F}$是$\varphi$态射.
	\item 逆像.设$\varphi:X\to Y$是连续映射,设$\mathscr{G}$是$Y$上的$\mathscr{A}$值预层,它在$\varphi$下的逆像指的是二元组$(\mathscr{G}',\rho)$,其中$\mathscr{G}'$是$X$上的$\mathscr{A}$值层,而$\rho:\mathscr{G}\to\mathscr{G}'$是一个$\varphi$态射,使得对$X$上任意$\mathscr{A}$值层$\mathscr{F}$,都有如下映射是双射.逆像记作$\varphi^{-1}\mathscr{G}$,按照定义它典范的带着一个态射$\rho:\mathscr{G}\to\varphi_*\varphi^{-1}\mathscr{G}$.
	$$\mathrm{Hom}_X(\mathscr{G}',\mathscr{F})\to\mathrm{Hom}_Y(\mathscr{G},\varphi_*\mathscr{F})$$
	$$v\mapsto\varphi_*(v)\circ\rho$$
	
	换句话讲$(\mathscr{G}',\rho)$满足如下泛性质:对任意$X$上的层$\mathscr{F}$,以及一个$Y$上的层态射$u:\mathscr{G}\to\varphi^{-1}\mathscr{F}$,都存在$X$上唯一的态射$\widetilde{u}:\mathscr{G}'\to\mathscr{F}$,使得如下图表交换:
	$$\xymatrix{\varphi_*\mathscr{G}'\ar[rr]^{\exists_!\varphi_*\widetilde{u}}&&\varphi_*\mathscr{F}\\\mathscr{G}\ar[u]^{\rho}\ar@/_1pc/[urr]_u&&}$$
	\item 如果$Y$上的任意预层在$\varphi$下都有逆像,此时逆像就是$\textbf{pSh}(Y)\to\textbf{Sh}(X)$的函子,并且它限制在层范畴上是顺像的伴随函子,也即总有自然同构:
	$$\mathrm{Hom}_X(\varphi^{-1}\mathscr{G},\mathscr{F})\cong\mathrm{Hom}_Y(\mathscr{G},\varphi_*\mathscr{F})$$
	\item 设$w:\mathscr{G}_1\to\mathscr{G}_2$是$Y$上预层的态射,设$u:\mathscr{G}_2\to\mathscr{F}$是$\varphi$态射,考虑如下交换图表,如果我们把$u:\mathscr{G}\to\varphi_*\mathscr{F}$在逆像泛性质中唯一对应的$\varphi^{-1}\mathscr{G}\to\mathscr{F}$记作$u^{\#}$,那么如下交换图表得到$(u\circ w)^{\#}=u^{\#}\circ\varphi^{-1}(w)$.
	$$\xymatrix{\varphi_*\varphi^{-1}\mathscr{G}_1\ar[r]^{\varphi_*\varphi^{-1}w}&\varphi_*\varphi^{-1}\mathscr{G}_2\ar[r]^{\varphi^{-1}u^{\#}}&\varphi_*\mathscr{F}\\\mathscr{G}_1\ar[u]^{\rho}\ar[r]^w&\mathscr{G}_2\ar[u]_{\rho}\ar@/_1pc/[ur]_u&}$$
	\item 设$\mathscr{F}$是$X$上的层,按照逆像的泛性质,存在唯一的态射$\varphi^{-1}\varphi_*\mathscr{F}\to\mathscr{F}$使得如下图表交换,这个唯一的态射记作$\sigma_{\mathscr{F}}$.那么按照上一条,任意$\varphi$态射$\mathscr{G}\to\mathscr{F}$都有$u^{\#}=\sigma_{\mathscr{F}}\circ\varphi^{-1}(u)$.
	$$\xymatrix{\varphi_*\varphi^{-1}\varphi_*\mathscr{F}\ar[rr]^{\varphi_*\sigma_{\mathscr{F}}}&&\varphi_*\mathscr{F}\\\varphi_*\mathscr{F}\ar@/_1pc/@{=}[urr]\ar[u]^{\rho}&&}$$
	\item 考虑代数结构范畴,我们给出逆像的具体构造.设连续映射$\varphi:X\to Y$,设$\mathscr{G}$为$Y$上的一个层.
	\begin{enumerate}[(1)]
		\item 定义$X$上的一个预层$\varphi^+\mathscr{G}$为,对每个开集$U\subset X$上的截面群定义为$\lim\limits_{\substack{\rightarrow\\\varphi(U)\subset V}}\mathscr{G}(V)$.如果$U_1\subset U_2$,那么每个满足$\varphi(U_2)\subset V$的开集$V$自然也满足$\varphi(U_1)\subset V$,于是这自然诱导了正向极限之间的同态$\lim\limits_{\substack{\rightarrow\\\varphi(U_2)\subset V}}\mathscr{G}(V)\to\lim\limits_{\substack{\rightarrow\\\varphi(U_1)\subset V}}\mathscr{G}(V)$,这作为限制映射.
		\item 即便$\mathscr{G}$是$Y$上的层,预层$\varphi^+\mathscr{G}$也未必是$X$上的层.把$\varphi^+\mathscr{G}$的层化记作$\varphi^{-1}\mathscr{G}$.这个层有如下更具体地描述:对任意开集$U\subseteq X$,有$\varphi^{-1}\mathscr{G}(U)=\{(s'_x)_{x\in U}\mid s'_x\in\mathscr{G}_{\varphi(x)},\text{对任意}x\in U,\text{可以找到}\varphi(x)\in Y\text{的开邻域}V\text{和}X\text{的开邻域}W\subseteq\varphi^{-1}(V)\cap U,\text{以及一个元素}s\in\mathscr{G}(V),\text{使得}\forall z\in W,s'_z=s_{\varphi(z)}\}$.
		\item 设$\alpha:\mathscr{F}\to\mathscr{G}$是$Y$上的层态射,那么$\alpha(V):\mathscr{F}(V)\to\mathscr{G}(V)$是两个正向系统之间的同态,它诱导的正向极限的同态$\lim\limits_{\substack{\rightarrow\\\varphi(U)\subset V}}\mathscr{F}(V)\to\lim\limits_{\substack{\rightarrow\\\varphi(U)\subset V}}\mathscr{G}(V)$.这是两个预层之间的态射,记作$\varphi^+\alpha$,它提升为两个层化之间的态射,记作$\varphi^{-1}\alpha$.
		\item $\varphi^+\mathscr{G}$的茎满足$(f^+\mathscr{G})_x=\mathscr{G}_{f(x)}$,按照层化不改变茎说明$(\varphi^{-1}\mathscr{G})_x=\mathscr{G}_{f(x)}$.这说明$\varphi^{-1}$是正合函子.
		$$(\varphi^+\mathscr{G})_x=\lim\limits_{\substack{\rightarrow\\x\in U}}\lim\limits_{\substack{\rightarrow\\\varphi(U)\subset V}}\mathscr{G}(V)=\lim\limits_{\substack{\rightarrow\\\varphi(x)\in V}}\mathscr{G}(V)=\mathscr{G}_{\varphi(x)}$$
		\item 按照上一条有$\mathrm{Supp}\varphi^{-1}\mathscr{G}=\varphi^{-1}(\mathrm{Supp}\mathscr{G})$.
		\item 把$f_*$视为$\textbf{pSh}(X)\to\textbf{pSh}(Y)$的函子,那么$f^+$左伴随于$f_*$,即对任意$Y$上的预层$\mathscr{G}$和$X$上的预层$\mathscr{F}$,有自然同构:
		$$\mathrm{Hom}_{\textbf{pSh}(X)}(f^+\mathscr{G},\mathscr{F})\cong\mathrm{Hom}_{\textbf{pSh}(Y)}(\mathscr{G},f_*\mathscr{F})$$
		\begin{proof}
			
			设$\varphi$是$f^+\mathscr{G}\to\mathscr{F}$的预层之间的态射,我们需要构造它诱导的预层之间的态射$\varphi':\mathscr{G}\to f_*\mathscr{F}$:任取$Y$的开子集$V$,按照$f(f^{-1}(V))\subset V$,说明$\mathscr{G}(V)$出现在$f^+\mathscr{G}(f^{-1}(V))$定义中的正向系统中,于是存在典范同态$\mathscr{G}(V)\to f^+\mathscr{G}(f^{-1}(V))$.再复合上$\varphi(f^{-1}(V)):f^+\mathscr{G}(f^{-1}(V))\to\mathscr{F}(f^{-1}(V))=f_*\mathscr{F}(V)$,得到$\mathscr{G}(V)\to f_*\mathscr{F}(V)$.
			
			反过来设$\psi:\mathscr{G}\to f_*\mathscr{F}$是预层之间的态射,需要构造它诱导的预层之间的态射$\psi':f^{-1}\mathscr{G}\to\mathscr{F}$:取$X$的开子集$U$,取$s\in f^+\mathscr{G}(U)$,如果开集$V\subset Y$满足$f(U)\subset V$,并且存在$s_V\in\mathscr{G}(V)$在正向极限$f^+\mathscr{G}$中的像是$s$,那么$\psi(V)(s_V)\in f_*\mathscr{F}(V)=\mathscr{F}(f^{-1}(V))$,就设$\psi'(U)(s)$是$\psi(V)(s_V)$在$U$上的限制.
			
			验证这两个构造互为逆映射,验证自然性得证.
		\end{proof}
		\item 把$f_*$视为$\textbf{Sh}(X)\to\textbf{Sh}(Y)$的函子,那么$f^{-1}$左伴随于$f_*$,即对任意$Y$上的层$\mathscr{G}$和$X$上的层$\mathscr{F}$,有自然同构:
		$$\mathrm{Hom}_{\textbf{Sh}(X)}(f^{-1}\mathscr{G},\mathscr{F})\cong\mathrm{Hom}_{\textbf{Sh}(Y)}(\mathscr{G},f_*\mathscr{F})$$
		\begin{proof}
			
			我们有如下自然同构,其中第一个同构是层化的泛性质,第二个同构是上一条结论,第三个同构是因为这里的预层都是层.
			$$\mathrm{Hom}_{\textbf{Sh}(X)}(f^{-1}\mathscr{G},\mathscr{F})\cong\mathrm{Hom}_{\textbf{pSh}(X)}(f^+\mathscr{G},\mathscr{F})\cong\mathrm{Hom}_{\textbf{pSh}(Y)}(\mathscr{G},f_*\mathscr{F})\cong\mathrm{Hom}_{\textbf{Sh}(Y)}(\mathscr{G},f_*\mathscr{F})$$
		\end{proof}
	\end{enumerate}
\end{enumerate}
\subsubsection{常值层}

\begin{enumerate}
	\item 空间$X$上的$\mathscr{A}$值预层$\mathscr{F}$指的是对任意非空开集$U\subseteq X$有$\mathscr{F}(X)\to\mathscr{F}(U)$是同构.常值预层的层化称为常值层.局部常值层指的是存在$X$的开覆盖$\{U_i\}$,使得$\mathscr{F}\mid_{U_i}$是常值层.
	\item 设$A$是集合或者阿贝尔群,空间$X$上关于$A$的常值预层就是指只要$U\subset X$是非空开集,那么$\mathscr{F}(U)=A$,并且所有非空集开集之间的限制映射都定义为恒等映射.它的层化$\widetilde{\mathscr{F}}$具有如下描述:对非空开集$U\subseteq X$,定义$\widetilde{\mathscr{F}}$是在赋予$A$离散拓扑后,全体$U\to A$的连续映射构成的集合.
	\item 如果$X$是不可约空间,那么其上常值预层等价于常值层等价于局部常值层.
	\begin{proof}
		
		不可约空间保证了两个非空开集的交一定非空,那么此时常值预层本身就已经满足粘合的唯一性和存在性.设$\mathscr{F}$是$X$上的局部常值层,那么存在$X$的由非空开集构成的开覆盖$\{U_i\}$使得$\mathscr{F}\mid_{U_i}$是常值层,进而也是常值预层.但是由于$U_i\cap U_j$非空,导致$\mathscr{F}$上所有非空开集的截面是一致的,进而它是常值预层.
	\end{proof}
\end{enumerate}
\subsubsection{伪离散层}

\begin{enumerate}
	\item 设空间$X$有一个由拟紧开集构成的拓扑基$\mathfrak{B}$.设$\mathscr{F}$是$X$上的集合层,如果对开集$U\subseteq X$赋予$\mathscr{F}(U)$离散拓扑,那么$\mathscr{F}$可以视为拓扑空间预层.我们断言存在$X$上的拓扑空间层$\widetilde{\mathscr{F}}$,使得它在任意开集$U\in\mathfrak{B}$上对应的拓扑空间恰好是离散空间$\mathscr{F}(U)$.按照我们之前解释的拓扑基上的层可以延拓为全空间上的层,这个结论归结为说$\mathscr{F}$限制在$\mathfrak{B}$上是一个拓扑空间层.这里我们证明更强的结论:只要$U\subseteq X$是拟紧开集,那么$\widetilde{\mathscr{F}}(U)$就是离散空间$\mathscr{F}(U)$.
	\begin{proof}
		
		我们来证明更一般的,如果$U\subseteq X$是拟紧开集,如果$\{U_i\}$是$U$的由$\mathfrak{B}$中元素构成的开覆盖,那么$\mathscr{F}(U)$上的使得所有映射$\mathscr{F}(U)\to\mathscr{F}(U_i)$连续的最粗拓扑恰好是离散拓扑.
		
		\qquad
		
		首先可以取有限个指标$t$使得$U=\cup_tU_t$.任取$s\in\mathscr{F}(U)$,记它限制到$U_t$上是$s_t$.按照限制映射是连续的,这些$s_t$的原像的交集是$s$的一个开邻域.但是由于$\mathscr{F}$是集合层,满足粘合唯一性,于是这个交集是单点集.
	\end{proof}
	\item 如果$U\subseteq X$是非拟紧的开集,那么$\widetilde{\mathscr{F}}(U)$的底空间仍然是$\mathscr{F}(U)$,但是它的拓扑未必是离散的.
	\item 称$\widetilde{\mathscr{F}}$是集合层$\mathscr{F}$的伪离散层.类似的如果考虑阿贝尔层和环层,一样可以定义相应的伪离散层.
	\item 如果$\mathscr{F}$是集合层,$\mathscr{H}$是它的子层,它们的伪离散层分别记作$\widetilde{\mathscr{F}}$和$\widetilde{\mathscr{H}}$,那么对任意开集$U\subseteq X$,都有$\widetilde{\mathscr{H}}(U)$是$\widetilde{\mathscr{F}}(U)$的闭子集.这件事是因为$\widetilde{\mathscr{H}}(U)=\mathscr{H}(U)$是$\mathscr{H}(V)$在$\mathscr{F}(U)\to\mathscr{F}(V)$下的原像的交,其中$V$跑遍包含在$U$中的基元素.
\end{enumerate}
\newpage
\subsection{环空间和模层}
\subsubsection{环空间}
\begin{enumerate}
	\item 环空间是指对二元组$(X,\mathscr{O}_X)$,其中$X$是一个拓扑空间,$\mathscr{O}_X$是$X$上的一个(交换)环层称为结构层.给定两个环空间$(X,\mathscr{O}_X)$和$(Y,\mathscr{O}_Y)$.我们也会把环空间简记为$X$,约定它的结构层记号是$\mathscr{O}_X$.一个态射$(X,\mathscr{O}_X)\to(Y,\mathscr{O}_Y)$是指一个连续映射$f:X\to Y$,以及一个$f$态射$\mathscr{O}_Y\to\mathscr{O}_X$.这是指对每对满足$f(U)\subset V$的$X,Y$分别的开子集$U,V$,都指定一个同态$f^{\#}_{U,V}:\mathscr{O}_Y(V)\to\mathscr{O}_X(U)$,满足对任意开子集链$U'\subset U\subset X$和$V'\subset V\subset Y$,并且满足$f(U')\subset V'$和$f(U)\subset V$,那么总有如下图表交换:
	$$\xymatrix{\mathscr{O}_Y(V)\ar[rr]^{f^{\#}_{V,U}}\ar[d]_{\mathrm{res}_{V,V'}}&&\mathscr{O}_X(U)\ar[d]^{\mathrm{res}_{U,U'}}\\\mathscr{O}_Y(V')\ar[rr]_{f^{\#}_{V',U'}}&&\mathscr{O}_X(U')}$$
	
	按照顺像和逆像的伴随性,环空间之间的态射也等价于一个连续映射$f:X\to Y$和一个层态射$f^{-1}\mathscr{O}_Y\to\mathscr{O}_X$.
	\item 设$(f,f^{\#}):(X,\mathscr{O}_X)\to(Y,\mathscr{O}_Y)$和$(g,g^{\#}):(Y,\mathscr{O}_Y)\to(Z,\mathscr{O}_Z)$是环空间之间的态射,它们的复合定义为$(h,h^{\#})$,其中$h=g\circ f$,$h^{\#}=\xymatrix{\mathscr{O}_Z\ar[r]^{g^{\#}}&g_*\mathscr{O}_Y\ar[r]^{g_*(f^{\#})}&h_*\mathscr{O}_X}$.
	\begin{enumerate}[(1)]
		\item 我们之前解释过$(g\circ f)^{\#}=g^{\#}\circ\varphi^{-1}(f^{\#})$,于是从$f^{\#}$和$g^{\#}$是单态射/满态射可以推出$(g\circ f)^{\#}$是单态射/满态射.
		\item 如果$f$是拓扑嵌入(连续单映射),$f^{\#}$是满态射,那么$(f,f^{\#})$是环空间范畴中的单态射.
		\item 环空间之间的态射$(f,f^{\#})$是同构当且仅当$f$是同胚和$f^{\#}$是同构.
	\end{enumerate}
	\item 设$(X,\mathscr{O}_X)$是环空间,设$i:Y\subseteq X$是子空间,其上典范的结构层取为$\mathscr{O}_X\mid_Y=i^*\mathscr{O}_X$.称单态射$(Y,i^{-1}\mathscr{O}_X)\to(X,\mathscr{O}_X)$为典范嵌入.如果$(X,\mathscr{O}_X)\to(Z,\mathscr{O}_Z)$是任意态射,它在$X$的子空间$Y$上的限制就是指这个态射复合上上述典范嵌入.
	\item 设$(X,\mathscr{O}_X)$是环空间,这里$\mathscr{O}_X$是阿贝尔层,它在点$x\in X$的茎$\lim\limits_{\substack{\rightarrow\\x\in U}}\mathscr{O}_X(U)$记作$\mathscr{O}_{X,x}$,这是一个交换环,它的乘法定义为$[s,U][t,V]=[s\mid_{U\cap V}t\mid_{U\cap V},U\cap V]$.另外我们之前解释过$\varphi$态射诱导的茎映射,由此可定义环空间之间的态射诱导了茎映射.
	\item 例子.
	\begin{enumerate}[(1)]
		\item 局部环空间.称环空间$(X,\mathscr{O}_X)$是局部环空间,如果它还满足每个点$p\in X$处的茎$\mathscr{O}_{X,p}$都是一个局部环.此时$\mathscr{O}_{X,p}$模去极大理想$m_p$这个域称为$p$处的剩余类域,记作$\kappa(p)$.局部环空间之间的态射是环空间之间的态射,并且满足诱导的茎之间的同态总是局部环同态(局部环之间的同态$f:A\to B$称为局部环同态,如果满足$\mathscr{F}(m_A)\subset m_B$).
		\item $k$值函数空间.设$k$是一个域,称环空间$(X,\mathscr{O}_X)$是$k$值函数空间,如果每个截面环$\mathscr{O}_X(U)$是由某些$U\to k$的函数构成的环.这里限制映射定义为$k$值函数在更小的开集上的限制.两个$k$值函数空间之间的态射$\varphi:(X,\mathscr{O}_X)\to(Y,\mathscr{O}_Y)$是一个连续映射$\varphi:X\to Y$,使得对$\mathscr{O}_Y(V)$中的每个$k$值函数$s$,都有$s\circ\varphi\in\mathscr{O}_X(\varphi^{-1}(V))$.这样定义的态射$\varphi$诱导的$\mathscr{O}_Y(V)\to\mathscr{O}_X(\varphi^{-1}(V))$自然的是一个环同态.
	\end{enumerate}
	\item 环空间的粘合.设$\{(X_i,\mathscr{O}_i)\mid i\in I\}$是一族环空间,对任意有序的指标对$(i,j)$,取定$X_i$的开子集$X_{ij}$以及一个环空间同构$\varphi_{ij}:(X_{ij},\mathscr{O}_i\mid_{X_{ij}})\cong(X_{ji},\mathscr{O}_j\mid_{X_{ji}})$.满足如下两个条件:
	\begin{enumerate}[(1)]
		\item 对任意指标$i$有$X_{ii}=X_i$和$\varphi_{ii}=1_{X_i}$.
		\item 对任意指标对$(i,j,k)$,有$\varphi_{ij}\mid_{X_{ij}\cap X_{ik}}$是同构:
		$$(X_{ij}\cap X_{ik},\mathscr{O}_i\mid_{X_{ij}\cap X_{ik}})\cong(X_{ji}\cap X_{jk},\mathscr{O}_j\mid_{X_{ji}\cap X_{jk}})$$
		
		并且满足余圈条件:
		$$\varphi_{ij}\mid_{X_{ij}\cap X_{ik}}=\varphi_{ik}\mid_{X_{ik}\cap X_{ij}}\circ\varphi_{kj}\mid_{X_{kj}\cap X_{ki}}$$
	\end{enumerate}
	
	那么这些$\{X_i\}$首先可以粘合为一个拓扑空间$X$,进而这些$\mathscr{O}_i$可以粘合为$X$上的环层.
\end{enumerate}
\subsubsection{模层和代数层}
\begin{itemize}
	\item 给定环空间$(X,\mathscr{O}_X)$,$X$上的一个阿贝尔预层$F$称为$\mathscr{O}_X$模预层,如果对每个开集$U\subset X$,阿贝尔群$F(U)$都具备一个$\mathscr{O}_X(U)$模结构,并且对任意开集$V\subset U$,模结构和限制映射是可交换的,即如下图表交换.如果$F$是层,就称它是$\mathscr{O}_X$模层.
	$$\xymatrix{\mathscr{O}_X(U)\times F(U)\ar[rr]\ar[d]&&F(U)\ar[d]\\\mathscr{O}_X(V)\times F(V)\ar[rr]&&F(V)}$$
	\item 两个$\mathscr{O}_X$模预层$F$和$G$之间的态射是指阿贝尔预层态射$\varphi:F\to G$,使得对每个开集$U\subset X$都有$\varphi(U)$是$F(U)\to G(U)$的$\mathscr{O}_X(U)$模同态.
	\item $\mathscr{O}_X$模预层和态射构成范畴,$\mathscr{O}_X$模层和态射构成范畴.但是我们更关注层的情况,把后一个范畴就记作$\textbf{Mod}(\mathscr{O}_X)$.
	\item 环空间$(X,\mathscr{O}_X)$上的一个$\mathscr{O}_X$代数层就是把模层定义中的模都改为代数.更具体地讲,一个$\mathscr{O}_X$代数层是指一个$\mathscr{O}_X$模层$\mathscr{A}$,以及一个$\mathscr{O}_X$模层态射$\varphi:\mathscr{A}\otimes_{\mathscr{O}_X}\mathscr{A}\to\mathscr{A}$(作为乘法)和一个整体截面$e\in\mathscr{A}(X)$(作为幺元),满足如下信息:
	\begin{enumerate}[(a)]
		\item 结合律:
		$$\xymatrix{\mathscr{A}\otimes_{\mathscr{O}_X}\mathscr{A}\otimes_{\mathscr{O}_X}\mathscr{A}\ar[rr]^{\varphi\otimes1}\ar[d]_{1\otimes\varphi}&&\mathscr{A}\otimes_{\mathscr{O}_X}\mathscr{A}\ar[d]^{\varphi}\\mathscr{A}\otimes_{\mathscr{O}_X}\mathscr{A}\ar[rr]^{\varphi}&&\mathscr{A}}$$
		\item 幺元:对任意开集$U\subseteq X$和任意截面$s\in\mathscr{A}(U)$,都有$\varphi((e\mid_U)\otimes s)=\varphi(s\otimes(e\mid_U))=s$.
		\item (如果定义交换代数层)交换性:记$\sigma:\mathscr{A}\otimes_{\mathscr{O}_X}\mathscr{A}\to\mathscr{A}\otimes_{\mathscr{O}_X}\mathscr{A}$表示典范对称.
		$$\xymatrix{\mathscr{A}\otimes_{\mathscr{O}_X}\mathscr{A}\ar[rr]^{\sigma}\ar[dr]_{\varphi}&&\mathscr{A}\otimes_{\mathscr{O}_X}\mathscr{A}\ar[dl]^{\varphi}\\&\mathscr{A}&}$$
	\end{enumerate}
	
	把代数层范畴记作$\textbf{Alg}(\mathscr{O}_X)$.
	\item 环空间$(X,\mathscr{O}_X)$上的代数层$\mathscr{A}$称为分次代数层,如果存在子模层直和分解$\mathscr{A}=\oplus_{n\ge0}\mathscr{A}_n$,满足对任意$m,n\ge0$有$\mathscr{A}_n\mathscr{A}_m\subseteq\mathscr{A}_{m+n}$.如果$\mathscr{O}_X$本身具备分次结构,它的模层$\mathscr{F}$称为分次模层,如果存在子模层的直和分解$\mathscr{F}=\oplus_{n\ge0}\mathscr{F}_n$,满足对任意$m,n\ge0$有$\mathscr{O}_{X,n}\mathscr{F}_m\subseteq\mathscr{F}_{m+n}$.这里定义的分次结构是$\mathbb{N}$分次结构,当然还可以定义其它的分次结构.
\end{itemize}
\begin{enumerate}
	\item 茎函子.设$\mathscr{F}$是$\mathscr{O}_X$模预层,茎$\mathscr{F}_x=\varinjlim\mathscr{F}(U)$自然的具备$\mathscr{O}_{X,x}$模结构,即$[a,V][m,U]=(a\mid U\cap V m\mid U\cap V,U\cap V)$,这是$\textbf{Mod}(\mathscr{O}_X)\to\textbf{Ab}$的函子;如果$\mathscr{F}$是$\mathscr{O}_X$代数预层,那么$\mathscr{F}_x$自然的具备$\mathscr{O}_{X,x}$代数结构.另外茎的定义是一个正向极限,它总是和余极限可交换的.
	\item 层化.设$\mathscr{F}$是$\mathscr{O}_X$模预层,它的层化记作$F^*$,那么$F^*$自然具备一个$\mathscr{O}_X$模层结构,并且这个结构和层化的典范态射$F\to F^*$是交换的(下图).类似的结论对代数预层成立.
	$$\xymatrix{\mathscr{O}_X(U)\times F(U)\ar[rr]\ar[d]&&F(U)\ar[d]\\\mathscr{O}_X(U)\times F^*(U)\ar[rr]&&F^*(U)}$$
	\begin{proof}
		
		回顾一下层化的具体构造.对开集$U\subset X$,层化$F^*(U)$是$\prod_{x\in U}F_x$中这样的元$(t_x)_{x\in U}$构成,满足对任意$x\in U$,都存在开邻域$W_x\subset U$,存在$s\in F(W_x)$使得$y\in W_x$时总有$t_y=s_y$.定义$F^*(U)$上的$\mathscr{O}_X(U)$模结构为$(s,(t_x))\mapsto(s_xt_x)$.另外层化的典范态射$F(U)\to F^*(U)$就是把$s\in F(U)$映射为$(s_x)_{x\in U}$.容易验证这个映射和模结构可交换.
	\end{proof}
	\item 子模层.设$\mathscr{F}$是一个$\mathscr{O}_X$模层,一个$\mathscr{O}_X$模层$\mathscr{G}$是它的子模层,如果每个截面$G(U)$是$F(U)$的$\mathscr{O}_X(U)$子模.特别的,$\mathscr{O}_X$视为自身模层时,它的子模层称为理想层.
	\item 商模层.设$\mathscr{G}$是$\mathscr{O}_X$模层$\mathscr{F}$的子模层,预层$U\mapsto\mathscr{F}(U)/\mathscr{G}(U)$是$\mathscr{O}_X$预层,它的层化是$\mathscr{O}_X$模层,称为$\mathscr{F}$关于$\mathscr{G}$的商模层,记作$\mathscr{F}/\mathscr{F}$.另外茎上有$(\mathscr{F}/\mathscr{G})_x=\mathscr{F}_x/\mathscr{G}_x$.
	\item 类似阿贝尔层的情况,我们可以验证$\mathscr{O}_X$模层范畴上的二元对偶积,核与余核,直和直积,单满态射的等价描述等所有内容,并且最终得到这个范畴是阿贝尔范畴.另外和阿贝尔层的情况一样的,模层范畴是模预层范畴的反射子范畴,导致模层在预层范畴里取范畴极限都仍然是模层,而取范畴余极限则还需要取层化.
	\item 正如交换群范畴实际上就是$\mathbb{Z}$模范畴,阿贝尔层范畴实际上就是$\underline{\mathbb{Z}}$模层范畴,其中$\underline{\mathbb{Z}}$是$X$关于$\mathbb{Z}$的常值层.
	\begin{itemize}
		\item 先定义模层结构.任取非空开集$U\subset X$,我们解释过$\underline{\mathbb{Z}}(U)$就是全体$U\to\mathbb{Z}$的局部常值函数构成的集合.任取这样的一个函数$f$,任取$X$上的一个阿贝尔层$\mathscr{F}$,任取$a\in\mathscr{F}(U)$.我们需要定义$f$在$a$上的作用.$f$在$U$的不同开子集上可能取值不同,导致无法把$f$直接作用在$a$上.于是我们把$U$做分割,使得每个小开集上$f$的作用是统一的,即:对每个整数$n$,记$U_n=f^{-1}(\{n\})$,按照定义这是一个开集,并且$U=\cup_{n\in\mathbb{Z}}U_n$是一个无交并.现在我们有理由约定$f$作用在每个$a\mid_{U_n}$上为$n(a\mid_{U_n})$,粘合这些元得到一个$b\in\mathscr{F}(U)$,就定义$(f,a)\mapsto b$.
		\item 现在验证我们构造的$\underline{\mathbb{Z}}(U)\times\mathscr{F}(U)\to\mathscr{F}(U)$的确是模层结构.比方说设$f,g\in\underline{\mathbb{Z}}(U)$和$a\in\mathscr{F}(U)$,验证$fg(a)=f(g(a))$:设$U_{m,n}\subset U$是这样的开子集,在其上有$f$取值$m$,$g$取值$n$,那么$U=\cup_{m,n\in\mathbb{Z}}U_{m,n}$是一个无交并.设$a_{m,n}=a\mid_{U_{m,n}}$,那么按照定义有$fg(a)$是$\{mna_{m,n}\mid m,n\in\mathbb{Z}\}$的粘合.现在设$b_n=n(a\mid\cup_mU_{m,n})$,那么$\{b_n,n\in\mathbb{Z}\}$粘合为$g(a)$.设$c_m=mg(a)\mid\cup_nU_{m,n}$,那么$\{c_m,m\in\mathbb{Z}\}$粘合为$f(g(a))$.但是每个$c_m$是$\{mna_{m,n}\mid n\in\mathbb{Z}\}$的粘合,于是$f(g(a))$也是$\{mna_{m,n}\mid m,n\in\mathbb{Z}\}$的粘合,按照粘合的唯一性,就得到$fg(a)=f(g(a))$.
	\end{itemize}
	\item 子模层生成的代数层.设$\mathscr{A}$是一个$\mathscr{O}_X$代数层,设$\mathscr{F}\subseteq\mathscr{A}$是子模层,定义由$\mathscr{F}$生成的$\mathscr{A}$代数子层为代数预层$U\mapsto\mathscr{G}(U)$的层化,其中$\mathscr{G}(U)$是$\mathscr{A}(U)$的由子模$\mathscr{F}(U)$生成的子代数.这个子代数层也就是$\{\mathscr{F}^{\otimes n}\to\mathscr{A},n\ge0\}$的像的和生成的$\mathscr{A}$的$\mathscr{O}_X$子代数层.
	\item 分次环层和分次模层.一个分次环层指的是这样一个环层$\mathscr{A}$,它可以表示为一族阿贝尔层$\{\mathscr{A}_n,n\in\mathbb{Z}\}$的直和,并且满足$\mathscr{A}_n\mathscr{A}_m\subseteq\mathscr{A}_{n+m}$;如果$\mathscr{A}$是上述分次环层,一个分次$\mathscr{A}$模层指的是这样的一个$\mathscr{A}$模层$\mathscr{F}$,它可以表示为一族阿贝尔层$\{\mathscr{F}_n,n\in\mathbb{Z}\}$的直和,并且满足$\mathscr{A}_n\mathscr{F}_m\subseteq\mathscr{F}_{n+m}$.
	\item 零化子.$\mathscr{O}_X$模层$\mathscr{F}$的零化子定义为典范模层同态$\mathscr{O}_X\to\mathrm{HOM}_{\mathscr{O}_X}(\mathscr{F},\mathscr{F})$的核$\mathscr{I}$,这个同态是把截面$s\in\mathscr{O}_X(U)$映射为$\mathrm{Hom}_{\mathscr{O}_U}(\mathscr{F}\mid_U,\mathscr{F}\mid_U)$的数乘$s$的映射.
\end{enumerate}
\subsubsection{模层范畴上的张量函子和Hom函子}

设$(X,\mathscr{O}_X)$是环空间.
\begin{enumerate}
	\item 给定两个$\mathscr{O}_X$模层$\mathscr{F}$和$\mathscr{G}$.定义预层$\mathscr{H}$为$\mathscr{H}(U)=\mathscr{F}(U)\otimes_{\mathscr{O}_X(U)}\mathscr{G}(U)$,它一般不是层,它的层化称为这两个模层的张量积,记作$\mathscr{F}\otimes_{\mathscr{O}_X}\mathscr{G}$.
	\item $\mathscr{H}_p\cong\mathscr{F}_p\otimes_{\mathscr{O}_{X,p}}\mathscr{G}_p,\forall p\in X$.
	\begin{proof}
		
		首先要证是同构的这两个东西都是$\mathscr{O}_{X,p}$模.任取$p$的开邻域$U$,典范映射$\mathscr{O}_X(U)\to\mathscr{O}_{X,p}$诱导了$\mathscr{F}_p\otimes_{\mathscr{O}_{X,p}}\mathscr{G}_p$上的$\mathscr{O}_X(U)$模结构.映射$\alpha_U':\mathscr{F}(U)\times\mathscr{G}(U)\to\mathscr{F}_p\otimes_{\mathscr{O}_{X,p}}\mathscr{G}_p$,$(s,t)\mapsto s_p\otimes t_p$是$\mathscr{O}_X(U)$模上的双线性映射,于是存在唯一的$\mathscr{O}_X(U)$模同态$\alpha_U:s\otimes t\mapsto s_p\otimes t_p$使得如下图表交换:
		$$\xymatrix{\mathscr{F}(U)\otimes_{\mathscr{O}_X(U)}\mathscr{G}(U)\ar[rrr]^{\alpha_U}&&&\mathscr{F}_p\otimes_{\mathscr{O}_{X,p}}\mathscr{G}_p\\\mathscr{F}(U)\times\mathscr{G}(U)\ar[u]\ar@/_1pc/[urrr]_{\alpha_U'}&&&}$$
		
		现在把$\{\alpha_U,p\in U\}$视为一族交换群同态.假设$p\in V\subset U$,那么有如下交换图:
		$$\xymatrix{\mathscr{F}(U)\otimes_{\mathscr{O}_X(U)}\mathscr{G}(U)\ar[dr]_{\alpha_U}\ar[rr]^{\mathrm{res}_{U,V}}^{\mathscr{H}}&&\mathscr{F}(V)\otimes_{\mathscr{O}_X(V)}\mathscr{G}(V)\ar[dl]^{\alpha_V}\\&\mathscr{F}_p\otimes_{\mathscr{O}_{X,p}}\mathscr{G}_p&}$$
		
		于是按照$\mathscr{H}_p$的泛性质,存在交换群同态$\alpha:\mathscr{H}_p\to\mathscr{F}_p\otimes_{\mathscr{O}_{X,p}}\mathscr{G}_p$为把等价类$[s\otimes t\in\mathscr{H}(U)]\mapsto s_p\otimes t_p$.可验证这实际上是一个$\mathscr{O}_{X,p}$模同态.
		$$\xymatrix{\mathscr{H}_p\ar[rr]^{\alpha}&&\mathscr{F}_p\otimes_{\mathscr{O}_{X,p}}\mathscr{G}_p\\\mathscr{F}(U)\otimes_{\mathscr{O}_X(U)}\mathscr{G}(U)\ar[u]\ar@/_1pc/[urr]_{\alpha_U}&&}$$
		
		现在构造逆同态.设$\beta':\mathscr{F}_p\times\mathscr{G}_p\to\mathscr{H}_p$为$\mathscr{O}_{X,p}$模上的双线性映射$(s_p,t_p)\mapsto[s\mid_{U\cap V}\otimes t\mid_{U\cap V}\in\mathscr{H}(U\cap V)]$,其中$s\in\mathscr{F}(U)$和$t\in\mathscr{G}(V)$.于是存在同态$\beta:\mathscr{F}_p\otimes_{\mathscr{O}_{X,p}}\mathscr{G}_p\to\mathscr{H}_p$使得如下图表交换,这个映射即$s_p\otimes t_p\mapsto[s\mid_{U\cap V}\otimes t\mid_{U\cap V}\in\mathscr{H}(U\cap V)]$.可验证这也是一个$\mathscr{O}_{X,p}$模同态.
		$$\xymatrix{\mathscr{F}_p\otimes_{\mathscr{O}_{X,p}}\mathscr{G}_p\ar[rr]^{\beta}&&\mathscr{H}_p\\\mathscr{F}_p\times\mathscr{G}_p\ar[u]\ar@/_1pc/[urr]_{\beta'}&&}$$
		
		最后验证$\alpha$和$\beta$互为逆映射:
		$$[s\otimes t\in\mathscr{H}(U)]\mapsto s_p\otimes t_p\mapsto[s\otimes t\in\mathscr{H}(U)]$$
		$$(s,U)_p\otimes (t,V)_p\mapsto[s\mid_{U\cap V}\otimes t\mid_{U\cap V}]\mapsto s_p\otimes t_p$$
	\end{proof}
	\item $\mathscr{F}\otimes_{\mathscr{O}_X}-$和$-\otimes_{\mathscr{O}_X}\mathscr{F}$都是$\textbf{Mod}(\mathscr{O}_X)\to\textbf{Mod}(\mathscr{O}_X)$的右正合函子.于是特别的,它和范畴余极限可交换.
	\item 如果$\mathscr{F}$是$\mathscr{O}_X$模层,$\mathscr{I}$是$\mathscr{O}_X$理想层,记$\mathscr{I}\otimes_{\underline{\mathbb{Z}}}\mathscr{F}\to\mathscr{F}$的层像为$\mathscr{I}\mathscr{F}$.那么按照茎函子和像(余核的核)可交换,就有$(\mathscr{I}\mathscr{F})_x=\mathscr{I}_x\mathscr{F}_x$.这个层$\mathscr{I}\mathscr{F}$也是预层$U\mapsto\mathscr{I}(U)\mathscr{F}(U)$的层化.
	\item 设$\mathscr{F}$和$\mathscr{G}$是两个$\mathscr{O}_X$模层.定义一个$\mathscr{O}_X$模层$\mathrm{HOM}_{\mathscr{O}_X}(\mathscr{F},\mathscr{G})$为,在每个开集$U$上的截面群是$\mathrm{Hom}_{\mathscr{O}_X\mid U}(\mathscr{F}\mid U,\mathscr{G}\mid U)$.这是一个阿贝尔层(粘合的存在性和唯一性是层态射粘合的存在性和唯一性),现在定义它的模层结构,对任意开集$U$定义:
	$$\mathscr{O}_X(U)\times\mathrm{Hom}_{\mathscr{O}_X\mid U}(\mathscr{F}\mid U,\mathscr{G}\mid U),(s,\varphi)\mapsto\psi$$
	$$\psi(V):\mathscr{F}(V)\to\mathscr{G}(V),t\mapsto s\mid_V\cdot\varphi(V)(t),V\subset U$$
	\item $\mathrm{HOM}_{\mathscr{O}_X}(-,\mathscr{G})$是$\textbf{Mod}(\mathscr{O}_X)\to\textbf{Mod}(\mathscr{O}_X)$的左正合函子.特别的,它和范畴极限可交换.
	\begin{proof}
		
		设$\xymatrix{\mathscr{F}_1\ar[r]^{\varphi}&\mathscr{F}_2\ar[r]^{\psi}&\mathscr{F}_3\ar[r]&0}$是$\mathscr{O}_X$模层的正合列,我们要证明有如下正合列:
		$$\xymatrix{0\ar[r]&\mathrm{HOM}_{\mathscr{O}_X}(\mathscr{F}_3,\mathscr{G})\ar[r]^{\varphi^*}&\mathrm{HOM}_{\mathscr{O}_X}(\mathscr{F}_2,\mathscr{G})\ar[r]^{\psi^*}&\mathrm{HOM}_{\mathscr{O}_X}(\mathscr{F}_1,\mathscr{G})}$$
		
		如果我们能证明在每个截面上这是一个正合列,就得到它是一个正合列(尽管反过来是不对的,如果它是正合列推不出在所有截面上是正合列).任取开集$U\subseteq X$,我们要证明有如下正合列:
		$$\xymatrix{0\ar[r]&\mathrm{HOM}_{\mathscr{O}_U}(\mathscr{F}_3\mid_U,\mathscr{G}\mid_U)\ar[r]^{\varphi^*}&\mathrm{HOM}_{\mathscr{O}_U}(\mathscr{F}_2\mid_U,\mathscr{G}\mid_U)\ar[r]^{\psi^*}&\mathrm{HOM}_{\mathscr{O}_U}(\mathscr{F}_1\mid_U,\mathscr{G}\mid_U)}$$
		\begin{itemize}
			\item $\ker\psi^*=\{0\}$.设$\alpha:\mathscr{F}_3\mid_U\to\mathscr{F}_2\mid_U$是$\mathscr{O}_U$模层之间的态射,满足$\psi^*(\alpha)=\alpha\circ\psi=0$.于是对任意$x\in U$就有$\alpha_x\circ\psi_x=0$.但是$\psi$是满态射,每个$\psi_x$是满同态,就导致$\alpha_x=0,\forall x\in U$,于是$\alpha=0$.
			\item $\mathrm{im}\psi^*\subseteq\ker\varphi^*$.这是因为$\psi\circ\varphi=0$.
			\item $\ker\varphi^*\subseteq\mathrm{im}\psi^*$.设$\alpha:\mathscr{F}_2\mid_U\to\mathscr{G}\mid_U$是$\mathscr{O}_U$模层态射,满足$\alpha\circ\varphi=0$.由于$\mathscr{F}_3\mid_U$是$\varphi\mid_U$的余核,就有$\alpha$要经$\psi\mid_U$分解,也即$\alpha\in\mathrm{im}\psi^*$.
		\end{itemize}
	\end{proof}
	\item 一般来讲,关于茎只能得到如下典范映射,它一般既不是单射也不是满射.但是如果$F$是局部有限表示模层,这个典范映射是同构.
	$$\mathrm{HOM}_{\mathscr{O}_X}(\mathscr{F},\mathscr{G})_x\to\mathrm{Hom}_{\mathscr{O}_{X,x}}(\mathscr{F}_x,\mathscr{G}_x)$$
	\begin{proof}
		
		先在不添加任何条件的前提下构造这个典范态射.任取$x$的开邻域$U$,任取$\varphi\in\mathrm{HOM}_{\mathscr{O}_X}(\mathscr{F},\mathscr{G})(U)=\mathrm{HOM}_{\mathscr{O}_U}(\mathscr{F}\mid_U,\mathscr{G}\mid_U)$.对任意开集$V$满足$x\in V\subseteq U$,有$\mathscr{O}_X(V)$模同态$\varphi(V):\mathscr{F}(V)\to\mathscr{G}(V)$.它们和限制同态可交换,这构成一个正向系统,取极限得到$\varphi_x:\mathscr{F}_x=\lim\limits_{\rightarrow}\mathscr{F}(V)\to\lim\limits_{\rightarrow}\mathscr{G}(V)=\mathscr{G}_x$,这是一个$\mathscr{O}_{X,x}$模同态.
		
		\qquad
		
		现在设$\mathscr{F}$是有限表示模层,于是存在自然数$m,n$和一个正合列:
		\begin{equation}\label{f.r}
			\xymatrix{\mathscr{O}_X^m\ar[r]&\mathscr{O}_X^n\ar[r]&\mathscr{F}\ar[r]&0}
		\end{equation}
		
		按照$\mathrm{HOM}_{\mathscr{O}_X}(-,\mathscr{G})$是左正合函子,得到正合列:
		$$\xymatrix{0\ar[r]&\mathrm{HOM}_{\mathscr{O}_X}(\mathscr{F},\mathscr{G})\ar[r]&\mathrm{HOM}_{\mathscr{O}_X}(\mathscr{O}_X^n,\mathscr{G})\ar[r]&\mathrm{HOM}_{\mathscr{O}_X}(\mathscr{O}_X^n,\mathscr{G})}$$
		
		取stalk是一个正合函子,于是有正合列:
		$$\xymatrix{0\ar[r]&\mathrm{HOM}_{\mathscr{O}_X}(\mathscr{F},\mathscr{G})_x\ar[r]&\mathscr{G}_x^n\ar[r]&\mathscr{G}_x^m}$$
		
		另一方面,对\ref{f.r}取stalk,得到正合列:
		$$\xymatrix{\mathscr{O}_{X,x}^m\ar[r]&\mathscr{O}_{X,x}^n\ar[r]&\mathscr{F}_x\ar[r]&0}$$
		
		再作用左正合函子$\mathrm{Hom}_{\mathscr{O}_{X,x}}(-,\mathscr{G}_x)$得到:
		$$\xymatrix{0\ar[r]&\mathrm{Hom}_{\mathscr{O}_{X,x}}\ar[r]&\mathscr{G}_x^n\ar[r]&\mathscr{G}_x^m}$$
		
		于是$\mathrm{Hom}_{\mathscr{O}_{X,x}}$和$\mathrm{HOM}_{\mathscr{O}_X}(\mathscr{F},\mathscr{G})_x$都是$\mathscr{G}_x^n\to\mathscr{G}_x^m$的核,所以它们典范同构.
	\end{proof}
	\item 伴随性.设$\mathscr{F},\mathscr{G},\mathscr{H}$是$\mathscr{O}_X$模层.
	\begin{enumerate}
		\item 我们有自然的$\Gamma(X,\mathscr{O}_X)$模同构:
		$$\mathrm{Hom}_{\mathscr{O}_X}(\mathscr{F}\otimes_{\mathscr{O}_X}\mathscr{G},\mathscr{H})\cong\mathrm{Hom}_{\mathscr{O}_X}(\mathscr{F},\mathrm{HOM}_{\mathscr{O}_X}(\mathscr{G},\mathscr{H}))$$
		\item 我们有自然的$\mathscr{O}_X$模层同构:
		$$\mathrm{HOM}_{\mathscr{O}_X}(\mathscr{F}\otimes_{\mathscr{O}_X}\mathscr{G},\mathscr{H})\cong\mathrm{HOM}_{\mathscr{O}_X}(\mathscr{F},\mathrm{HOM}_{\mathscr{O}_X}(\mathscr{G},\mathscr{H}))$$
		\item 我们有如下自然的$\mathscr{O}_X$模层态射,如果$\mathscr{F}$或者$\mathscr{H}$是局部自由模层,那么这是同构.
		$$\mathrm{HOM}_{\mathscr{O}_X}(\mathscr{F},\mathscr{G})\otimes_{\mathscr{O}_X}\mathscr{H}\cong\mathrm{HOM}_{\mathscr{O}_X}(\mathscr{F},\mathscr{G}\otimes_{\mathscr{O}_X}\mathscr{H})$$
	\end{enumerate}
\end{enumerate}
\subsubsection{模层和代数层的顺像}

$\mathscr{O}_X$模层的顺像(direct image).设$\varphi:(X,\mathscr{O}_X)\to(Y,\mathscr{O}_Y)$是环空间之间的态射,它诱导了顺像函子$\varphi_*:\textbf{Mod}(\mathscr{O}_X)\to\textbf{Mod}(\mathscr{O}_Y)$.
\begin{itemize}
	\item 设$\mathscr{F}$是一个$\mathscr{O}_X$模层.我们定义过$Y$上的阿贝尔层$\varphi_*\mathscr{F}$,它在开集$V\subset Y$上的截面群定义为$\mathscr{F}(\varphi^{-1}(V))$.这个$\varphi_*\mathscr{F}$实际上是$\mathscr{O}_Y$模层:
	$$\mathscr{O}_Y(V)\times\mathscr{F}(f^{-1}(V))\to\mathscr{F}(\varphi^{-1}(V)),(s,t)\mapsto\varphi^{\#}(V)(s)\cdot t$$
	\item 设$\alpha:\mathscr{F}\to\mathscr{G}$是$\mathscr{O}_X$模层之间的态射,它诱导了$\varphi_*\mathscr{F}\to\varphi_*\mathscr{G}$的态射$\alpha_*$定义为$\alpha_*(V)=\alpha(\varphi^{-1}(V))$.
\end{itemize}
\begin{enumerate}
	\item 顺像和张量积.设$\mathscr{F}$和$\mathscr{G}$是两个$\mathscr{O}_X$模层,对任意开集$V\subseteq Y$,有典范映射:
	$$\mathscr{F}(\varphi^{-1}(V),\mathscr{F})\times\mathscr{G}(\varphi^{-1}(V))\to\mathscr{F}\otimes_{\mathscr{O}_X}\mathscr{G}(\varphi^{-1}(V))$$
	
	这个映射是$\mathscr{O}_X(\varphi^{-1}(V))$双线性的,进而诱导了同态:
	$$\varphi_*\mathscr{F}(V)\otimes_{\mathscr{O}_Y(V)}\varphi^*\mathscr{G}(V)\to\varphi_*(\mathscr{F}\otimes_{\mathscr{O}_X}\mathscr{G})(V)$$
	
	它和限制映射是交换的,于是诱导了函子性的$\mathscr{O}_Y$模层态射:
	$$(\varphi_*\mathscr{F})\otimes_{\mathscr{O}_Y}(\varphi_*\mathscr{G})\to\varphi_*(\mathscr{F}\otimes_{\mathscr{O}_X}\mathscr{G})$$
	
	这个态射一般未必是单态射也未必是满态射.如果$\mathscr{H}$是第三个$\mathscr{O}_X$模层,那么有如下交换图表:
	$$\xymatrix{(\varphi_*\mathscr{F})\otimes_{\mathscr{O}_Y}(\varphi_*\mathscr{G})\otimes_{\mathscr{O}_Y}(\varphi_*\mathscr{G})\ar[rr]\ar[d]&&\varphi_*(\mathscr{F}\otimes_{\mathscr{O}_X}\mathscr{G})\otimes_{\mathscr{O}_Y}(\varphi^*\mathscr{H})\ar[d]\\(\varphi_*\mathscr{F})\otimes_{\mathscr{O}_Y}\varphi_*(\mathscr{G}\otimes_{\mathscr{O}_X}\mathscr{H})\ar[rr]&&\varphi_*(\mathscr{F}\otimes_{\mathscr{O}_X}\mathscr{G}\otimes_{\mathscr{O}_X}\mathscr{H})}$$
	\item $\varphi$态射的张量积.设$\mathscr{F}_1,\mathscr{F}_2$是两个$\mathscr{O}_X$模层,$\mathscr{G}_1,\mathscr{G}_2$是两个$\mathscr{O}_Y$模层,设$u_i:\mathscr{G}_i\to\mathscr{F}_i$是$\varphi$态射,其中$i=1,2$.取复合态射$\xymatrix{\mathscr{G}_1\otimes_{\mathscr{O}_Y}\mathscr{G}_2\ar[r]^{u_1\otimes u_2}&(\varphi_*\mathscr{F}_1)\otimes_{\mathscr{O}_Y}(\varphi_*\mathscr{F}_2)\ar[r]&\varphi_*(\mathscr{F}_1\otimes_{\mathscr{O}_X}\mathscr{F}_2)}$,把它记作$u$,这是一个$\varphi$态射,称为$u_1$和$u_2$的张量积.它满足$u^{\#}=(u_1)^{\#}\otimes(u_2)^{\#}$.
	\item 顺像和Hom.类似的如果$\mathscr{F}$和$\mathscr{G}$是两个$\mathscr{O}_X$模层,那么有如下典范态射,它一般也不是单态射或者满态射:
	$$\varphi_*\mathrm{HOM}_{\mathscr{O}_X}(\mathscr{F},\mathscr{G})\to\mathrm{Hom}_{\mathscr{O}_Y}(\varphi_*\mathscr{F},\varphi_*\mathscr{G})$$
	\item 代数层的顺像.$\varphi_*\mathscr{O}_X$的$\mathscr{O}_Y$模层结构和自身的环层结构使得它是$\mathscr{O}_Y$代数层.如果$\mathscr{A}$是$\mathscr{O}_X$代数层,考虑如下复合态射:
	$$(\varphi_*\mathscr{A})\otimes_{\mathscr{O}_Y}(\varphi_*\mathscr{A})\to\varphi_*(\mathscr{A}\otimes_{\mathscr{O}_X}\mathscr{A})\to\varphi_*\mathscr{A}$$
	
	这定义了$\varphi_*\mathscr{A}$上的乘法,进而$\varphi_*\mathscr{A}$是一个$\mathscr{O}_Y$代数层.于是顺像也是$\textbf{Alg}(\mathscr{O}_X)\to\textbf{Alg}(\mathscr{O}_Y)$的函子.
\end{enumerate}
\subsubsection{模层和代数层的逆像}

$\mathscr{O}_X$模层的逆像(inverse image).设$\varphi:(X,\mathscr{O}_X)\to(Y,\mathscr{O}_Y)$是环空间之间的态射,它诱导了逆像函子$\varphi^*:\textbf{Mod}(\mathscr{O}_Y)\to\textbf{Mod}(\mathscr{O}_X)$.
\begin{itemize}
	\item 设$\mathscr{G}$是一个$\mathscr{O}_Y$模层,那么$\varphi^{-1}\mathscr{G}$是$X$上的阿贝尔层,并且它自然的具备一个$\varphi^{-1}\mathscr{O}_Y$模层结构.态射$\varphi^{\#}:\varphi^{-1}\mathscr{O}_Y\to\mathscr{O}_X$提供了$\mathscr{O}_X$上的$\varphi^{-1}\mathscr{O}_Y$模层结构.我们定义$\mathscr{G}$在$\varphi$下的逆像模层为$(\varphi^{-1}\mathscr{G})\otimes_{\varphi^{-1}\mathscr{O}_Y}\mathscr{O}_X$,记作$\varphi^*\mathscr{G}$.
	\item 设$v:\mathscr{G}_1\to\mathscr{G}_2$是$\mathscr{O}_Y$模层之间的态射,那么$\varphi^{-1}(v)\otimes1$是$\varphi^*\mathscr{G}_1\to\varphi^*\mathscr{G}_2$的模层态射,记作$\varphi^*(v)$.
\end{itemize}
\begin{enumerate}
	\item 之前解释过$\varphi^{-1}$是正合函子,这里$\varphi^*$只是右正合函子,因为我们前面解释过张量函子是右正合的.于是特别的它和正向极限可交换.另外按照$\varphi^{-1}$是加性函子,张量$\mathscr{O}_X$就得到$\varphi^*$也是加性函子(也即它和有限直和可交换,结合无限直和是有限直和的正向极限,就得到它也和任意直和可交换).
	\item 茎.我们有$(\varphi^*\mathscr{G})_x\cong\mathscr{O}_{X,x}\otimes_{\mathscr{O}_{Y,\varphi(x)}}\mathscr{G}_{\varphi(x)}$.另外这说明$\mathrm{Supp}\varphi^*\mathscr{G}\subseteq\varphi^{-1}(\mathrm{Supp}\mathscr{G})=\mathrm{Supp}(\varphi^{-1}\mathscr{G})$.
	\item 逆像和张量积.设$\mathscr{G}_1$和$\mathscr{G}_2$是两个$\mathscr{O}_Y$模层,我们有典范的$\varphi^{-1}\mathscr{O}_Y$模层态射:
	$$(\varphi^{-1}\mathscr{G}_1)\otimes_{\varphi^{-1}\mathscr{O}_Y}(\varphi^{-1}\mathscr{G}_2)\to\varphi^*(\mathscr{G}_1\otimes_{\mathscr{O}_Y}\mathscr{G}_2)$$
	
	由于这个态射在茎上是同构,于是这个映射是同构.进而张量$\mathscr{O}_X$得到自然同构:
	$$(\varphi^*\mathscr{G}_1)\otimes_{\mathscr{O}_X}(\varphi^*\mathscr{G}_2)\cong\varphi^*(\mathscr{G}_1\otimes_{\mathscr{O}_Y}\mathscr{G}_2)$$
	\item 代数层的逆像.如果$\mathscr{A}$是一个$\mathscr{O}_Y$代数层,它相当于模层结构的基础上赋予一个带一些条件(结合律,幺元等)的模层态射$\mathscr{A}\otimes_{\mathscr{O}_Y}\mathscr{A}\to\mathscr{A}$.于是上一条中的同构允许我们赋予$\varphi^*\mathscr{A}$典范的$\mathscr{O}_X$代数层结构.特别的,有$\varphi^*\mathscr{O}_Y=\mathscr{O}_X$.类似的如果$\mathscr{A}$是一个$\mathscr{O}_Y$代数层,$\mathscr{F}$是$\mathscr{A}$模层,它的模层结构也就是一个模层态射$\mathscr{A}\otimes_{\mathscr{O}_Y}\mathscr{F}\to\mathscr{F}$,按照上一条就诱导了典范的$(\varphi^*\mathscr{A})\otimes_{\mathscr{O}_X}(\varphi^*\mathscr{F})\to\varphi^*\mathscr{F}$,于是提供了$\varphi^*\mathscr{F}$上的$\varphi^*\mathscr{A}$模层结构.于是逆像也是$\textbf{Alg}(\mathscr{O}_Y)\to\textbf{Alg}(\mathscr{O}_X)$的函子.
	\item 关于理想层.设$\mathscr{I}$是$\mathscr{O}_Y$的理想层.由于函子$\varphi^{-1}$是正合的,于是$\varphi^{-1}\mathscr{I}$是$\varphi^{-1}\mathscr{O}_Y$的理想层.进而诱导了$\mathscr{O}_Y$模层态射$\varphi^*\mathscr{I}=(\varphi^{-1}\mathscr{I})\otimes_{\varphi^{-1}\mathscr{O}_Y}\mathscr{O}_X\to\mathscr{O}_X$,这个模层态射的像$\varphi^{\#}(\varphi^{-1}\mathscr{I})\mathscr{O}_X$记作$(\varphi^*\mathscr{I})\mathscr{O}_X$或者$\mathscr{I}\mathscr{O}_X$.那么有$(\mathscr{I}\mathscr{O}_X)_x=\varphi^{\#}_x(\mathscr{I}_{\varphi(x)})\mathscr{O}_{X,x}$.
	\item 伴随性.设$\varphi:(X,\mathscr{O}_X)\to(Y,\mathscr{O}_Y)$是环空间之间的态射.设$\mathscr{F}$是$\mathscr{O}_X$模层,$\mathscr{G}$是$\mathscr{O}_Y$模层,那么有自然的同构:
	$$\mathrm{Hom}_{\mathscr{O}_X}(\varphi^*\mathscr{G},\mathscr{F})\cong\mathrm{Hom}_{\mathscr{O}_Y}(\mathscr{G},\varphi_*\mathscr{F})$$
	
	这个自然同构也可以写作如下形式,取整体截面得到上面的同构.
	$$\varphi_*\mathrm{HOM}_{\mathscr{O}_X}(\varphi^*\mathscr{G},\mathscr{F})\cong\mathrm{HOM}_{\mathscr{O}_Y}(\mathscr{G},\varphi_*\mathscr{F})$$
	
	另外如果把模层改为代数层,也有类似的伴随性.
	\begin{proof}
		
		只需证明有如下自然同构:
		$$\mathrm{Hom}_{\mathscr{O}_X}(\varphi^*\mathscr{G},\mathscr{F})\cong\mathrm{Hom}_{\varphi^{-1}\mathscr{O}_Y}(\varphi^{-1}\mathscr{G},\mathscr{F})\cong\mathrm{Hom}_{\mathscr{O}_Y}(\mathscr{G},\varphi_*\mathscr{F})$$
		
		这里第一个双射来自如下交换图,第二个双射来自阿贝尔层范畴上$\varphi^{-1}$和$\varphi_*$的伴随性.
		$$\xymatrix{\mathscr{O}_X\otimes_{\varphi^{-1}\mathscr{O}_Y}\varphi^{-1}\mathscr{G}\ar[rr]&&\mathscr{F}\\\mathscr{O}_X\times\varphi^{-1}\mathscr{G}\ar[u]\ar@/_1pc/[urr]&&}$$
	\end{proof}
	\item 按照上一条的伴随性,恒等态射对应于如下两个典范态射:$\rho_{\mathscr{G}}:\mathscr{G}\to\varphi_*\varphi^*\mathscr{G}$和$\sigma_{\mathscr{F}}:\varphi^*\varphi_*\mathscr{F}\to\mathscr{F}$.
	\item 逆像和Hom.设$\mathscr{F}$和$\mathscr{G}$是两个$\mathscr{O}_Y$模层,设$V\subseteq Y$是开子集,记$U=\varphi^{-1}(V)$,有如下$\mathscr{O}_Y(V)$模同态:
	$$\mathrm{Hom}_{\mathscr{O}_Y\mid_V}(\mathscr{F}\mid_V,\mathscr{G}\mid_V)\to\mathrm{Hom}_{\mathscr{O}_X\mid_U}\left((\varphi^{-1}\mathscr{F})\mid_U,(\varphi^{-1}\mathscr{G})\mid_U\right)$$
	$$v\mapsto\varphi^{-1}v$$
	
	其中右侧这个集合具备$\varphi^{-1}\mathscr{O}_Y$模结构,它按照$\mathscr{O}_Y(V)\to\varphi^{-1}\mathscr{O}_Y(U)$具备$\mathscr{O}_Y(V)$模结构.这个同态和限制映射是可交换的,于是有如下函子性的态射:
	$$\mathrm{HOM}_{\mathscr{O}_Y}(\mathscr{F},\mathscr{G})\to\varphi_*\mathrm{HOM}_{\mathscr{O}_X}(\varphi^*\mathscr{F},\varphi^*\mathscr{G})$$
	
	按照伴随性它对应于如下函子性的态射:
	$$\varphi^*\mathrm{HOM}_{\mathscr{O}_Y}(\mathscr{F},\mathscr{G})\to\mathrm{HOM}_{\mathscr{O}_X}(\varphi^*\mathscr{F},\varphi^*\mathscr{G})$$
\end{enumerate}
\subsubsection{截面生成的模层}
\begin{enumerate}
	\item 设$\mathscr{F}$是一个$\mathscr{O}_X$模层,那么如下映射是一个$\mathscr{O}_X(X)$模同构:
	$$\mathrm{Hom}_{\mathscr{O}_X}(\mathscr{O}_X,\mathscr{F})\cong\mathscr{F}(X),\varphi\mapsto\varphi(X)(1)$$
	\begin{proof}
		
		直接构造它的逆映射.任取$s\in\mathscr{F}(X)$,构造$\varphi_s(U):\mathscr{O}_X(U)\to\mathscr{F}(U)$为$t\mapsto t\cdot(s\mid_U)$.一方面任取$s\in\mathscr{F}(X)$,那么$\varphi_s(X)(1)=1\cdot(s\mid_X)=s$.
		
		另一方面,任取$\mathscr{O}_X$模层态射$\varphi:\mathscr{O}_X\to\mathscr{F}$,记$\varphi(X)(1)=s$,那么$\varphi(U)(t)=\varphi(U)(t\cdot1)=t\varphi(U)(1)$,按照如下图表得到$\varphi(U)(1)=\varphi(X)(1)\mid_U=s\mid_U$,于是$\varphi(U)(t)=t\cdot(s\mid_U)$.
		$$\xymatrix{\mathscr{O}_X(X)\ar[rr]^{\varphi(X)}\ar[d]&&\mathscr{F}(X)\ar[d]\\\mathscr{O}_X(U)\ar[rr]_{\varphi(U)}&&\mathscr{F}(U)}$$
	\end{proof}
	\item 设$I$是一个指标集,记$\mathscr{F}^{(I)}=\oplus_{I}\mathscr{F}$和$\mathscr{F}^I=\prod_I\mathscr{F}$.有典范同构:
	$$\mathrm{Hom}_{\mathscr{O}_X}(\mathscr{O}_X^{(I)},\mathscr{F})\cong\mathrm{Hom}_{\mathscr{O_X}}(\mathscr{O}_X,\mathscr{F})^I\cong\mathscr{F}(X)^I$$
	\item 给定一个模层$\mathscr{F}$,一般来讲未必存在指标集$I$使得存在满态射$\mathscr{O}_X^{(I)}\to\mathscr{F}$.一旦存在这样的满态射,它在上一条中同构于一族整体截面$\{s_i\}$,就称$\mathscr{F}$是被这族截面生成的.这等价于讲对每个$x\in X$都有$\mathscr{O}_{X,x}$模$\mathscr{F}_x$被$(s_{i,x})_{i\in I}$生成.
	\item 我们证明过逆像函子是右正合的.设$\mathscr{G}$是环空间$Y$上的被一族截面$\{s_i\}$生成的模层.设$\varphi:X\to Y$是环空间之间的态射,那么模层的逆像$\mathscr{F}=\varphi^*(\mathscr{G})$是被截面族$\{\varphi(X)(s_i)\}$生成的$X$上模层.
	\item 设$\varphi:X\to Y$是环空间的态射,设$\mathscr{F}$是能被一族整体截面生成的$\mathscr{O}_X$模层,那么按照典范态射$\varphi^*\varphi_*\mathscr{F}\to\mathscr{F}$把$s\otimes1$映为$s$,对任意整体截面$s\in\mathscr{F}(X)$成立(因为$\varphi^{-1}\varphi_*\mathscr{F}(X)=\mathscr{F}(X)$).于是此时这个典范态射一定是满态射.
\end{enumerate}
\subsubsection{有限模层}

一个$\mathscr{O}_X$模层$\mathscr{F}$称为有限的,如果对每个$x\in X$,都存在开邻域$U$,使得$\mathscr{F}\mid U$被有限个截面生成,换句话讲局部上总有一个满态射$(\mathscr{O}_X\mid U)^n\to\mathscr{F}\mid U$.
\begin{enumerate}
	\item 如果$\mathscr{F}$是有限$\mathscr{O}_X$模层,那么对任意$x\in X$都有$\mathscr{F}_x$是有限$\mathscr{O}_{X,x}$模.这件事的逆命题一般不对.
	\item 设$\mathscr{F}$是有限$\mathscr{O}_X$模层,设$U\subseteq X$是开集,设$x\in U$,设$s_1,\cdots,s_p\in\mathscr{F}(U)$,使得它们生成了整个$\mathscr{F}_x$,那么存在开集$x\in V\subseteq U$,使得对任意$y\in V$都有$s_1,\cdots,s_p$生成了整个$\mathscr{F}_y$.
	\begin{proof}
		
		任取$y\in U$,由于$\mathscr{F}$是有限模层,存在$x$的足够小的开邻域上的有限个截面$t_1,\cdots,t_q$,使得对这个足够小的开邻域上的任意点$y$,总有这些截面生成了整个$\mathscr{F}_y$.按照$s_1(x),\cdots,s_p(x)$生成了整个$\mathscr{F}_x$,于是存在$x$足够小的开邻域上的一族截面$f_{ij}$使得$t_i(x)=\sum_{j=1}^pf_{ij}(x)s_j(x)$.于是这些等式在$x$足够小的开邻域上是截面的等式,这个开邻域满足$V$的要求.
	\end{proof}
	\item 上一条可说明有限模层的支集一定是闭集.类似可证明如果$u:\mathscr{F}\to\mathscr{G}$是源端为有限模层的态射,如果$x\in X$使得$u_x=0$,那么存在$x$的开邻域$U$使得对任意$y\in U$有$u_y=0$.
	\item 如果$\mathscr{F}$是有限模层,把$\mathscr{F}_x$能被$r$个元在$\mathscr{O}_{X,x}$上生成的点构成的集合记作$X_r$,那么这总是$X$的开集.如果$X$是局部环空间,这里$X_r=\{x\in X\mid\dim_{\kappa(x)}\mathscr{F}_x/\mathfrak{m}_x\mathscr{F}_x\}$.
	\item 设$X$是拟紧空间,设$u:\mathscr{F}\to\mathscr{G}$是终端为有限模层的满态射,设$\mathscr{F}$是一个有向集作为指标集的模层正向系统$\{\mathscr{F}_i\}$的极限.那么总可以找到一个指标$\alpha$使得典范的$\mathscr{F}_{\alpha}\to\mathscr{G}$是满态射.
	\begin{proof}
		
		任取$x\in X$,记$U_x$是它的开邻域,使得存在有限个截面$s_1,\cdots,s_n\in\mathscr{G}(U_x)$,满足对任意$y\in U_x$都有$s_1,\cdots,s_n$生成了整个$\mathscr{G}_y$.按照$u$是满态射,又可以找到$x$的开邻域$V_x\subseteq U_x$以及截面$t_1,\cdots,t_n\in\mathscr{F}(V_x)$满足$u(t_i)=s_i\mid_{V_x}$.选取足够大的指标$\alpha$可以保证$t_1,\cdots,t_n\in\mathscr{F}_{\alpha}(V_x)$.最后选取$\{V_x\mid x\in X\}$的有限子覆盖,选取这些指标$\alpha$中最大的作为新的$\alpha$即可.
	\end{proof}
	\item 设$X$是拟紧空间,设$\mathscr{F}$是有限$\mathscr{O}_X$模层,并且可由整体截面生成,那么它一定能被有限个整体截面生成.
	\item 有限模层的商,有限直和,逆像都是有限模层.
\end{enumerate}
\subsubsection{有限表示模层}

一个$\mathscr{O}_X$模层$\mathscr{F}$称为有限表示模层,如果对每个$x\in X$都存在开邻域$U$,使得$\mathscr{F}\mid U$同构于某个态射$(\mathscr{O}_X\mid U)^p\to(\mathscr{O}_X\mid U)^q$的余核.
\begin{enumerate}
	\item 有限表示模层必然是有限生成模层;有限表示模层的逆像也是有限表示的.
	\item 有限表示模层的定义是局部上总存在一个有限表示.模范畴的情况下,$A$模$M$存在一个有限表出等价于每个满同态$A^n\to M\to0$的核都是有限生成的,但是这个结论对于模层是不成立的.事实上即便是$\mathscr{O}_X$作为自身的模层也未必有这个性质.
	\item 设$\mathscr{F}$和$\mathscr{G}$是有限表示模层,设$x\in X$,取定一个$\mathscr{O}_{X,x}$同构$f:\mathscr{F}_x\cong\mathscr{G}_x$.那么存在$x$的开邻域$U$,和一个$\mathscr{O}_U$模层同构$\varphi:\mathscr{F}\mid_U\cong\mathscr{G}\mid_U$,使得$\varphi$在$x$上诱导的stalk映射就是预先取定的同构$\mathscr{F}_x\cong\mathscr{G}_x$.
	\begin{proof}
		
		按照上一条,存在$x$的开邻域$U$,和一个$\varphi\in\mathrm{HOM}_{\mathscr{O}_X}(\mathscr{F},\mathscr{G})(U)$,使得$\varphi$在$x$处诱导的stalk同态就是$f$.同理存在$x$的开邻域$V$,和一个$\psi\in\mathrm{HOM}_{\mathscr{O}_X}(\mathscr{G},\mathscr{F})(V)$,使得$\psi$在$x$处诱导的stalk同态是$f^{-1}$.不妨设$U=V$,因为可以取$U\cap V$代替$U$和$V$.那么$\psi\circ\varphi$是$\mathscr{F}\mid_U\to\mathscr{F}\mid_U$是态射,并且在$x$处诱导的stalk同态是$\mathscr{F}_x$上的恒等映射,于是可取一个更小的$x$的开邻域$V$使得$\psi\circ\varphi$在$V$上是$\mathscr{F}\mid_V$的恒等态射.同理可取$V$中更小的开集$W$使得$\varphi\circ\psi$在$W$上是$\mathscr{F}\mid_W$的恒等态射.于是此时$\varphi\mid_W$和$\psi\mid_W$互为逆态射.
	\end{proof}
	\item 设$\mathscr{F}$是环空间$X$上的有限生成模层,那么它是有限表示模层当且仅当对任意开子集$U\subseteq X$和任意$\mathscr{O}_U$模层的如下短正合列,其中$\mathscr{G}$是有限生成模层,都有$\mathscr{F}'$是有限生成模层.
	$$\xymatrix{0\ar[r]&\mathscr{F}'\ar[r]&\mathscr{G}\ar[r]&\mathscr{F}\mid_U\ar[r]&0}$$
\end{enumerate}
\subsubsection{(拟)凝聚层}

设$X$是环空间.
\begin{itemize}
	\item 一个$\mathscr{O}_X$模层$\mathscr{F}$称为拟凝聚层,如果存在$X$的开覆盖$\{U_i\}$,使得每个$\mathscr{F}\mid_{U_i}$,都是某个自由模层之间态射$\mathscr{O}_X^{(I)}\mid_{U_i}\to\mathscr{O}_X^{(J)}\mid_{U_i}$的余核.
	\item 一个$\mathscr{O}_X$模层$\mathscr{F}$称为凝聚层,如果它是有限模层,并且对任意开集$U\subseteq X$,任意有限自由模层为源端的态射$\mathscr{O}_X^n\mid_U\to\mathscr{F}\mid_U$的余核都是有限模层.
	\item 一个$\mathscr{O}_X$上的代数层称为(拟)凝聚的,如果它作为模层是(拟)凝聚层.
\end{itemize}
\begin{enumerate}
	\item $\mathscr{O}_X$是自身的拟凝聚层;拟凝聚层的直和是拟凝聚层;按照逆像函子是右正合的,拟凝聚层的逆像是拟凝聚层;有限模层未必是拟凝聚层.
	\item 凝聚层的性质.
	\begin{enumerate}[(1)]
		\item 凝聚层一定是有限表示模层,但是反过来一般不对,甚至$\mathscr{O}_X$也未必是凝聚$\mathscr{O}_X$模层.比方说,按照如下结论,$\mathscr{O}_X$是凝聚层就必须满足$\mathscr{O}_{X,x}$的任意两个有限理想的交必须是有限理想.不过如果$\mathscr{O}_X$本身的确是凝聚层,那么此时有限表示$\mathscr{O}_X$模层一定是凝聚层.
		
		\qquad
		
		如果$\mathscr{O}_X$本身是凝聚层,设$\mathscr{F}$是凝聚$\mathscr{O}_X$模层,设$x\in X$,设$M$是$\mathscr{F}_x$的有限子模层,那么存在$x$的开邻域$U$和$\mathscr{F}\mid_U$的一个凝聚$\mathscr{O}_X\mid_U$子模层$\mathscr{G}$满足$\mathscr{G}_x=M$.
		\item 凝聚层的有限子模层是凝聚层.
		\item 凝聚层的有限直和是凝聚层.
		\item 如果$0\to\mathscr{F}\to\mathscr{G}\to\mathscr{H}\to0$是$\mathscr{O}_X$模层的正合列,那么这三个模层任意两个是凝聚层都能推出第三个也是凝聚层.特别的,凝聚层之间态射的核,余核,像都是凝聚层.
		\item 设$\mathscr{F},\mathscr{G}$是某个$\mathscr{O}_X$模层的凝聚子模层,那么$\mathscr{F}+\mathscr{G}$和$\mathscr{F}\cap\mathscr{G}$都是凝聚层.
		\item 如果$\mathscr{F}$和$\mathscr{G}$是凝聚$\mathscr{O}_X$模层,那么$\mathscr{F}\otimes_{\mathscr{O}_X}\mathscr{G}$和$\mathrm{HOM}_{\mathscr{O}_X}(\mathscr{F},\mathscr{G})$都是凝聚层.
	\end{enumerate}
	\item 设$\mathscr{O}_X$是自身的凝聚层,设$x\in X$,设$M$是有限表示$\mathscr{O}_{X,x}$模层,那么存在$x$的开邻域$U$和一个凝聚$\mathscr{O}_X\mid_U$模层$\mathscr{F}$,使得$\mathscr{F}_x\cong M$.
	\begin{proof}
		
		这件事就是因为有典范同构$(\mathrm{HOM}_{\mathscr{O}_X}(\mathscr{O}_X^p,\mathscr{O}_X^q))_x\cong\mathrm{Hom}_{\mathscr{O}_{X,x}}(\mathscr{O}_{X,x}^p,\mathscr{O}_{X,x}^q)$.
	\end{proof}
	\item 设$\mathscr{O}_X$是自身的凝聚层,设$\mathscr{I}$是它的一个凝聚理想层,那么一个$\mathscr{O}_X/\mathscr{I}$模层$\mathscr{F}$是凝聚的当且仅当它作为$\mathscr{O}_X$模层是凝聚的.特别的此时$\mathscr{O}_X/\mathscr{I}$本身是凝聚环层.
	\item 设$f:X\to Y$是环空间之间的态射,设$\mathscr{O}_X$是自身的凝聚层,那么对任意凝聚$\mathscr{O}_Y$层$\mathscr{G}$,都有$f^*\mathscr{G}$是凝聚$\mathscr{O}_X$层.
	\begin{proof}
		
		任取$x\in X$,存在$f(x)$在$Y$中的开邻域$V$,使得$\mathscr{G}\mid_V$是某个模层态射$u:\mathscr{O}_Y^p\mid_V\to\mathscr{O}_Y^q\mid_V$的余核.记$U=f^{-1}(V)$,按照$f_U^*$是右正合的,就得到$(f^*\mathscr{G})\mid_U=f_U^*(\mathscr{G}\mid_V)$是$f^*_U(u):\mathscr{O}_X^p\mid_U\to\mathscr{O}_X^q\mid_U$的余核.
	\end{proof}
	\item 设$Y\subseteq X$是闭子空间,典范嵌入记作$j:Y\to X$,如果它们的结构层满足$\mathscr{O}_X=j_*\mathscr{O}_Y$,那么一个$\mathscr{O}_Y$模层$\mathscr{G}$是有限的/拟凝聚的/凝聚的,当且仅当$j_*\mathscr{G}$是有限的/拟凝聚的/凝聚的$\mathscr{O}_X$模层.【】	
\end{enumerate}
\subsubsection{局部自由模层}

一个$\mathscr{O}_X$模层$\mathscr{F}$称为局部自由的,如果对每个$x\in X$,存在它的开邻域$U$,使得$\mathscr{F}\mid_U$作为$\mathscr{O}_X\mid_U$模层同构于$\mathscr{O}_X^{(I)}$.如果这里$I$总可取有限集合,就称$\mathscr{F}$是有限局部自由的.
\begin{enumerate}
	\item 秩和秩函数.如果$\mathscr{F}\mid_U\cong\mathscr{O}_X^{(I)}\mid_U$,按照茎函子和余极限可交换,说明有$\mathscr{O}_{X,x}$模同构$\mathscr{F}_x\cong\mathscr{O}_{X,x}^{(I)}$.这说明这里$I$的基数是只依赖于$\mathscr{F}$和$x$,和这里的$U$无关,这个基数称为$\mathscr{F}$在$x$处的秩,记作$\mathrm{rank}_{\mathscr{F}}(x)$.注意这里开集$U$上每个点关于$\mathscr{F}$的秩是相同的,于是秩函数$\mathrm{rank}_{\mathscr{F}}(-)$是$X$上的(取值为基数的)局部常值函数.于是如果$X$是连通的,那么局部自由模层的秩就是固定的,此时这个固定的秩就称为这个模层的秩.称秩1局部自由模层为可逆模层或者可逆层.
	\item 设$\mathscr{O}_X$是自身的凝聚层,设$\mathscr{F}$是凝聚$\mathscr{O}_X$模层.设$x\in X$满足$\mathscr{F}_x$是秩$n$自由$\mathscr{O}_{X,x}$模,那么可以找到$x$的开邻域$U$使得$\mathscr{F}\mid_U$是秩$n$局部自由模层.
	\item 如果$\mathscr{F}$是秩为$n$的局部自由$\mathscr{O}_X$模层,$\mathscr{G}$是秩为$m$的局部自由$\mathscr{O}_X$模层,那么$\mathrm{HOM}_{\mathscr{O}_X}(\mathscr{F},\mathscr{G})$是秩为$mn$的局部自由$\mathscr{O}_X$模层.特别的,秩为$n$的局部自由$\mathscr{O}_X$模层$\mathscr{F}$的对偶模层$\mathscr{F}^{\vee}=\mathrm{HOM}_{\mathscr{O}_X}(\mathscr{F},\mathscr{O}_X)$是秩为$n$的局部自由模层.
	\item 设$f:X\to Y$是环空间之间的态射,设$\mathscr{F}$是有限秩局部自由/可逆$\mathscr{O}_X$模层,那么$f^*\mathscr{F}$也是有限秩局部自由/可逆$\mathscr{O}_Y$模层.
	\item 设$f:X\to Y$是环空间之间的态射,设$\mathscr{F}$是$\mathscr{O}_X$模层,我们之前解释过总有典范态射$\mathrm{HOM}_{\mathscr{O}_Y}(\mathscr{F},\mathscr{O}_Y)\to f_*\mathrm{HOM}_{\mathscr{O}_X}(f^*\mathscr{F},\mathscr{O}_X)$,这对应了一个态射$f^*(\mathscr{F}^{\vee})\to(f^*\mathscr{F})^{\vee}$.并且如果$\mathscr{F}$是局部自由的,那么这个典范态射是同构.我们之前还解释过逆像函子总满足$(f^*\mathscr{F})\otimes_{\mathscr{O}_X}(f^*\mathscr{G})\cong f^*(\mathscr{F}\otimes_{\mathscr{O}_Y}\mathscr{G})$.综上对可逆$\mathscr{O}_Y$模层$\mathscr{L}$和任意整数$n$都有$f^*(\mathscr{L}^{\otimes n})$典范同构于$(f^*\mathscr{L})^{\otimes n}$.
	\item 分次环$\Gamma_*(X,\mathscr{L})$.设$X$是环空间,设$\mathscr{L}$是可逆$\mathscr{O}_X$模层,把阿贝尔群$\bigoplus_{n\in\mathbb{Z}}\Gamma(X,\mathscr{L}^{\otimes n})$记作$\Gamma_*(X,\mathscr{L})$,或者不引起歧义的时候简记作$\Gamma_*(\mathscr{L})$.它还具备一个分次环结构:设$s_n\in\Gamma(X,\mathscr{L}^{\otimes n})$和$s_m\in\Gamma(X,\mathscr{L}^{\otimes m})$,它们的乘积定义为$\mathscr{L}^{\otimes(n+m)}$中的$s_n\otimes s_m$.于是$\Gamma_*(X,\bullet)$是$X$上可逆模层范畴到分次环范畴的函子.
	\item 分次模$\Gamma_*(\mathscr{L},\mathscr{F})$.设$X$是环空间,设$\mathscr{L}$是$X$上的可逆层,设$\mathscr{F}$是$X$上的模层,定义$\Gamma_*(\mathscr{L},\mathscr{F})=\bigoplus_{n\in\mathbb{Z}}\Gamma_*(X,\mathscr{F}\otimes_{\mathscr{O}_X}\mathscr{L}^{\otimes n})$.它自然的具备$\Gamma_*(\mathscr{L})$分次模结构,也即对$s_n\in\Gamma(X,\mathscr{L}^{\otimes n})$和$u_m\in\Gamma(X,\mathscr{F}\otimes\mathscr{L}^{\otimes m})$,定义$s_nu_m$为$s_n\otimes u_m$在$\mathscr{L}^{\otimes n}\otimes\mathscr{F}\otimes\mathscr{L}^{\otimes m}\cong\mathscr{F}\otimes\mathscr{L}^{\otimes(n+m)}$中的像.$\Gamma_*(\mathscr{L},\bullet)$是$\textbf{Mod}(\mathscr{O}_X)$到$\Gamma_*(\mathscr{L})$分次模范畴的函子.
	\item 设$f:X\to Y$是环空间之间的态射,设$\mathscr{L}$是$Y$上的可逆层,设$\mathscr{F}$是$Y$上的模层.
	\begin{enumerate}[(1)]
		\item 典范态射$\rho:\mathscr{L}^{\otimes n}\to f_*f^*(\mathscr{L}^{\otimes n})$定义了一个阿贝尔群同态$\Gamma(X,\mathscr{L}^{\otimes n})\to\Gamma(Y,f^*(\mathscr{L}^{\otimes n}))$,进而定义了分次环同态$\Gamma_*(X,\mathscr{L})\to\Gamma_*(Y,f^*\mathscr{L})$.
		\item 类似的典范态射$\rho:\mathscr{F}\otimes\mathscr{L}^{\otimes n}\to f_*f^*(\mathscr{F}\otimes\mathscr{L}^{\otimes n})$定义了阿贝尔群同态$\Gamma(X,\mathscr{F}\otimes\mathscr{L}^{\otimes n})\to\Gamma(Y,f^*(\mathscr{F}\otimes\mathscr{L}^{\otimes n}))$,进而定义了分次模同态$\Gamma_*(\mathscr{L},\mathscr{F})\to\Gamma_*(f^*\mathscr{L},f^*\mathscr{F})$(这是不同分次环上的分次模之间的同态,这两个分次环之间有分次环同态).
	\end{enumerate}
	\item 设$X$是环空间,有$\mathscr{O}_X$模层的短正合列$\xymatrix{0\ar[r]&\mathscr{E}\ar[r]^j&\mathscr{G}\ar[r]^p&\mathscr{F}\ar[r]&0}$,其中$\mathscr{F}$是有限秩局部自由模层,那么这个短正合了是局部分裂的,换句话讲对任意$x\in X$都可以找到$x$的开邻域$U$使得$\mathscr{G}\mid_U$典范同构于$\mathscr{E}\mid_U\oplus\mathscr{F}\mid_U$.
	\begin{proof}
		
		问题是局部的,不妨设$\mathscr{F}=\mathscr{O}_X^n$,取它的典范整体截面$e_1,\cdots,e_n$,按照$p$是满态射,可以找到$x$的开邻域$U$以及$s_1,\cdots,s_n\in\mathscr{G}(U)$使得$p(s_i)=e_i\mid_U$.定义$f:\mathscr{F}\mid_U\to\mathscr{G}\mid_U$为$s_1,\cdots,s_n$确定的态射,那么有$p\circ f=1_{\mathscr{F}\mid_U}$.
	\end{proof}
	\item 设$f:X\to Y$是环空间之间的态射,设$\mathscr{F}$是$\mathscr{O}_X$模层,设$\mathscr{L}$是有限秩局部自由$\mathscr{O}_Y$模层,考虑如下典范态射的复合,我们断言这是一个同构.
	$$\xymatrix{(f_*\mathscr{F})\otimes_{\mathscr{O}_Y}\mathscr{L}\ar[r]^{1\otimes\rho}&(f_*\mathscr{F})\otimes_{\mathscr{O}_Y}(f_*f^*\mathscr{L})\ar[r]^{\alpha}&f_*(\mathscr{F}\otimes_{\mathscr{O}_X}f^*\mathscr{L})}$$
	\begin{proof}
		
		不妨设$\mathscr{L}=\mathscr{O}_Y^n$,按照$f^*$和$f_*$都是加性函子,又归结为设$n=1$,此时这个同构归结为$f^*\mathscr{O}_Y=\mathscr{O}_X$.
	\end{proof}
	\item 设$X$是局部环空间,设$\mathscr{L}$是可逆$\mathscr{O}_X$模层,设$f$是$\mathscr{L}$的一个整体截面,则如下条件互相等价,并且我们把满足这个条件的$x\in X$构成的集合记作$X_f$,它就是$f$的不取零的点构成的集合.
	\begin{enumerate}[(1)]
		\item $f_x$是$\mathscr{L}_x$的生成元.
		\item $f_x\not\in\mathfrak{m}_x\mathscr{L}_x$.
		\item 存在$\mathscr{L}^{\vee}$在$x$的某个开邻域$V$上的截面$g$,使得$f\otimes g$在$\Gamma(V,\mathscr{O}_X)$中的典范的像就是该截面环的幺元.
	\end{enumerate}
	\begin{proof}
		
		问题是局部的,不妨设$\mathscr{L}=\mathscr{O}_X$.此时当然有(1)和(2)等价.(3)推(2)是平凡的.最后(2)推(3):如果$f(x)\not=0$,那么$f_x$是$\mathscr{O}_{X,x}$中的可逆元,设$f_xg_x=1$,那么可以找到$\mathscr{L}$在$x$足够小的开邻域$V$上的截面$g$使得$fg=1$,这得到(3).
	\end{proof}
	\item 设$X$是局部环空间,设$\mathscr{L}$和$\mathscr{L}'$是两个可逆$\mathscr{O}_X$模层,设$f,g$分别是它们的整体截面,那么$f\otimes g$是$\mathscr{L}\otimes_{\mathscr{O}_X}\mathscr{L}'$的整体截面,此时有$X_f\cap X_g=X_{f\otimes g}$.
\end{enumerate}
\subsubsection{皮卡群}
\begin{enumerate}
	\item 设$X$是环空间.可逆层的同构类按照张量积构成一个群,称为$X$的皮卡群,记作$\mathrm{Pic}(X)$.
	\begin{itemize}
		\item 如果$\mathscr{F}$和$\mathscr{G}$都是可逆$\mathscr{O}_X$模层,那么$\mathscr{F}\otimes_{\mathscr{O}_X}\mathscr{G}$也是可逆层.
		\item 有典范同构$\mathscr{F}^{\vee}\otimes_{\mathscr{O}_X}\mathscr{F}\cong\mathrm{HOM}_{\mathscr{O}_X}(\mathscr{F},\mathscr{F})$.但是对于秩1局部自由模层$\mathscr{F}$,总有$\mathrm{HOM}_{\mathscr{O}_X}(\mathscr{F},\mathscr{F})\cong\mathscr{O}_X$.这说明可逆模层$\mathscr{F}$总满足$\mathscr{F}^{\vee}\otimes_{\mathscr{O}_X}\mathscr{F}\cong\mathscr{O}_X$.于是皮卡群的幺元是$\mathscr{O}_X$,乘法逆元是对偶模层.
	\end{itemize}
	\item 我们有典范群同构$\mathrm{Pic}(X)\cong\mathrm{H}^1(X,\mathscr{O}_X^*)$.
	\begin{proof}
		
		对开集$U\subseteq X$,有$\Gamma(U,\mathscr{O}_X^*)$就是$\mathscr{O}_X\mid_U$作为自身模层的自同构群.设$\mathfrak{U}=\{U_i\}$是$X$的开覆盖,给定$\check{\mathrm{H}}^1(\mathfrak{U},\mathscr{O}_X^*)$中的一个元,那么对开覆盖中的任意指标$i,j$,都给出了$\mathscr{O}_X\mid_{U_i\cap U_j}$上的自同构$\theta_{ij}$,并且这些自同构满足余圈条件.于是这些信息可以粘合为$X$上的一个可逆模层,这就给出了一个映射$\varphi_{\mathfrak{U}}:\check{\mathrm{H}}^1(\mathfrak{U},\mathscr{O}_X^*)\to\mathrm{Pic}(X)$.
		
		\qquad
		
		如果$\mathfrak{B}$是比$\mathfrak{U}$更精细的开覆盖,那么有如下交换图表,这诱导了映射$\varphi:\mathrm{H}^1(X,\mathscr{O}_X^*)\to\mathrm{Pic}(X)$.
		$$\xymatrix{\check{\mathrm{H}}^1(\mathfrak{U},\mathscr{O}_X^*)\ar[rr]\ar[dr]_{\varphi_{\mathfrak{U}}}&&\check{\mathrm{H}}^1(\mathfrak{B},\mathscr{O}_X^*)\ar[dl]^{\varphi_{\mathfrak{B}}}\\&\mathrm{Pic}(X)&}$$
		
		$\varphi$是满射:任取可逆$\mathscr{O}_X$模层$\mathscr{L}$,可以找到$X$的开覆盖$\mathfrak{U}=\{U_i\}$,使得$\mathscr{L}\mid_{U_i}$同构于$\mathscr{O}_X\mid_{U_i}$,于是$\mathscr{L}$落在$\varphi_{\mathfrak{U}}$的像中.
		
		\qquad
		
		$\varphi$是单射:因为$\mathrm{H}^1(X,\mathscr{O}_X^*)=\varinjlim_{\mathfrak{U}}\check{\mathrm{H}}^1(\mathfrak{U},\mathscr{O}_X^*)$,于是归结为证明每个$\varphi_{\mathfrak{U}}$都是单射.而这归结于粘合的唯一性.
		
		\qquad
		
		最后验证$\varphi$是群同态:设$\mathscr{L}$和$\mathscr{L}'$是两个可逆$\mathscr{O}_X$模层.那么存在$X$的开覆盖$\mathfrak{U}=\{U_i\}$,使得$\mathscr{L}\mid_{U_i}$和$\mathscr{L}'\mid_{U_i}$都同构于$\mathscr{O}_X\mid_{U_i}$.如果$\mathscr{L}$和$\mathscr{L}'$对应的余圈分别是$\{\theta_{ij}\}$和$\{\theta_{ij}'\}$,那么$\{\theta_{ij}\theta_{ij}'\}$就对应于$\mathscr{L}\otimes_{\mathscr{O}_X}\mathscr{L}'$.
	\end{proof}
	\item 设$f:X\to Y$是环空间之间的态射,我们解释过可逆层的逆像还是可逆层,于是$f^*$是$\mathrm{Pic}(Y)\to\mathrm{Pic}(X)$的函子.按照上一条的同构它对应于一个同态$\mathrm{H}^1(Y,\mathscr{O}_Y^*)\to\mathrm{H}^1(X,\mathscr{O}_X^*)$.
\end{enumerate}
\subsubsection{平坦性}

设$f:X\to Y$是环空间之间的态射,设$\mathscr{F}$是$\mathscr{O}_X$模层.
\begin{itemize}
	\item 称$\mathscr{F}$在点$x\in X$是$f$平坦的或者$\mathscr{F}$在点$x\in X$是$Y$平坦的,如果$\mathscr{F}_x$在$\mathscr{O}_{Y,f(x)}\to\mathscr{O}_{X,x}$下作为$\mathscr{O}_{Y,f(x)}$模是平坦的.称$\mathscr{F}$是$f$平坦模层,如果它在$X$的每个点都是$f$平坦的.
	\item 称$\mathscr{F}$在点$x\in X$平坦,如果$\mathscr{F}_x$作为$\mathscr{O}_{X,x}$模是平坦的;称$\mathscr{F}$是平坦模层,如果它在$X$上处处平坦.这是上一条中取$X=Y$和$f=\mathrm{id}_X$.
	\item 称$f$在点$x\in X$平坦,如果$\mathscr{O}_{Y,f(x)}\to\mathscr{O}_{X,x}$是平坦同态;称$f$是平坦态射或者$Y$在$X$上平坦,如果$f$在$X$上处处平坦.这也是第一条中取$\mathscr{F}=\mathscr{O}_X$.
\end{itemize}
\begin{enumerate}
	\item 设$X$是局部环空间,设$\mathscr{F}$是$\mathscr{O}_X$模层,如下命题互相等价:
	\begin{enumerate}
		\item $\mathscr{F}$是局部自由的有限生成模层.
		\item $\mathscr{F}$是有限表示模层,并且对任意$x\in X$有$\mathscr{F}_x$是自由$\mathscr{O}_{X,x}$模.
		\item $\mathscr{F}$是平坦模层并且是有限表示模层.
	\end{enumerate}
	\item 设$f:X\to Y$是环空间之间的态射,设$\mathscr{F}$是$\mathscr{O}_X$模层.
	\begin{enumerate}[(1)]
		\item 设$x\in X$,设$\mathscr{F}$在点$x$是$f$平坦的,那么$(f^*(\bullet)\otimes_{\mathscr{O}_X}\mathscr{F})_x$是$\textbf{Mod}(\mathscr{O}_Y)\to\textbf{Ab}$的正合函子.
		\item 如果$\mathscr{F}$是$f$平坦的,那么$f^*(\bullet)\otimes_{\mathscr{O}_X}\mathscr{F}$是$\textbf{Mod}(\mathscr{O}_Y)\to\textbf{Mod}(\mathscr{O}_X)$的正合函子.特别的如果$f$是平坦态射,那么$f^*$是$\textbf{Mod}(\mathscr{O}_Y)\to\textbf{Mod}(\mathscr{O}_X)$的正合函子.
		\item 反过来设$\mathscr{O}_Y$是自身的凝聚层,设$x\in X$,设$\mathscr{F}$是$\mathscr{O}_X$模层,如果对$y=f(x)$的任意开邻域$V$都有$(f^*(\bullet)\otimes_{\mathscr{O}_X}\mathscr{F})_x$是$\textbf{Mod}(\mathscr{O}_Y\mid_V)\to\textbf{Ab}$的正合函子,那么$\mathscr{F}$在$x$处是$f$平坦的.
		\begin{proof}
			
			要证明$\mathscr{O}_{Y,y}$的任意有限生成理想$J$都有$J\otimes_{\mathscr{O}_{Y,y}}\mathscr{F}_x\to\mathscr{F}_x$是单射.但是$\mathscr{O}_Y$是自身凝聚层保证存在$y$的开邻域$V$和一个$\mathscr{O}_Y\mid_V$凝聚理想层$\mathscr{I}$使得$\mathscr{I}_y=J$.于是这个单射是因为$\mathscr{I}\to\mathscr{O}_Y$是单射.
		\end{proof}
	\end{enumerate}
	\item 设$f:X\to Y$是环空间之间的平坦态射,如果$\mathscr{F}$是有限表示$\mathscr{O}_Y$模层,$\mathscr{G}$是$\mathscr{O}_Y$模层,那么如下典范态射是一个同构:
	$$f^*\mathrm{HOM}_{\mathscr{O}_Y}(\mathscr{F},\mathscr{G})\cong\mathrm{HOM}_{\mathscr{O}_X}(f^*\mathscr{F},f^*\mathscr{G})$$
	\begin{proof}
		
		问题是局部的,不妨设有正合列$\mathscr{O}_Y^p\to\mathscr{O}_Y^q\to\mathscr{F}\to0$,按照$f^*$和$\mathrm{HOM}(\bullet,\mathscr{G})$总是左正合的,利用短五引理,归结为证明$\mathscr{F}=\mathscr{O}_Y$的情况,此时同构是平凡的.
	\end{proof}
	\item 环空间之间的态射$f:X\to Y$称为忠实平坦态射,如果它是满射,并且对任意$x\in X$都有$\mathscr{O}_{Y,f(x)}\to\mathscr{O}_{X,x}$是忠实平坦同态.如果$f$是局部环空间之间的态射,按照局部环上的平坦等价于忠实平坦,就有$f$是忠实平坦态射当且仅当它是平坦满射.
	\begin{enumerate}[(1)]
		\item 如果$f$是环空间之间的忠实平坦态射,那么$f^*$是正合忠实函子.
		\item 设$f:X\to Y$是环空间之间的忠实平坦态射,设$\mathscr{G}$是$\mathscr{O}_Y$模层,那么$\mathscr{G}$是$\mathscr{O}_Y$平坦模层当且仅当$f^*\mathscr{G}$是$\mathscr{O}_X$平坦模层.
	\end{enumerate}
\end{enumerate}
\newpage
\subsection{层上同调}
\subsubsection{作为导出函子}

内射对象.设$(X,\mathscr{O}_X)$是环空间,那么$\textbf{Mod}(\mathscr{O}_X)$是具有足够多的内射对象的阿贝尔范畴.特别的$\textbf{Sh}(X,\textbf{Ab})$(因为它就是$\overline{\mathbb{Z}}$模层范畴)是具有足够多内射对象的阿贝尔范畴.另外我们解释下一般是无法期待具有足够多的投射层的:如果空间是局部连通Hausdorff空间,并且没有孤立点,其上的阿贝尔投射层只有零层.
\begin{proof}
	
	任取$\mathscr{O}_{X,x}$内射模$A$,任取$x\in X$,记$j:\{x\}\to X$是包含映射,那么$j_*(A)$是$X$上摩天大楼层,即对覆盖点$x$的开集定义截面是$A$,对不覆盖点$x$的开集定义截面环是零模,限制映射就要么取为$A$上的恒等映射,要么取零同态.任取$\mathscr{O}_X$模层$G$,总有Hom集之间的如下自然同构.于是按照茎函子是正合的,说明$j_*(A)$是$\textbf{Mod}(\mathscr{O}_X)$上的内射层.另外内射层的直积(即作为预层的直积再取层化)也是内射层.
	$$\mathrm{Hom}_{\mathscr{O}_{X,x}}(G_x,A)\cong\mathrm{Hom}_{\textbf{Mod}(\mathscr{O}_X)}(G,j_*(A))$$
	
	设$F$是$\mathscr{O}_X$模层,按照模范畴上总具有足够多的投射模,可取终端为内射模的嵌入$F_x\to I_x$.这诱导的$j_*(F_x)\to j_*(I_x)$自然是单态射,因为每个截面都是单同态.于是诱导的积态射$\prod_{x\in X}j_*(F_x)\to\prod_{x\in X}j_*(I_x)$是单态射.
	
	\qquad
	
	现在定义层态射$\varphi^x:F\to j_*(F_x)$为当$x\in U$时$\varphi^x(U)$是典范同态$F(U)\to F_x$,当$x\not\in U$时$\varphi^x(U)$是零同态.那么这个层态射诱导的$x$处的茎同态是$F_x$上的恒等映射.这些$\varphi^x$诱导了层态射$\varphi:F\to\prod_{x\in X}j_*(F_x)$.这是单同态因为取$x\in X$处的茎同态得到$\varphi_x$是单射.综上复合态射$F\to\prod_{x\in X}j_*(F_x)\to\prod_{x\in X}j_*(I_x)$是单态射,并且$\prod_{x\in X}j_*(I_x)$是内射层,证毕.
	$$\xymatrix{\prod_{x\in X}(x_*(F_x))\ar@/_1pc/[drr]_{\pi_x}&&F\ar[ll]_{\varphi}\ar[d]^{\varphi^x}\\&&x_*(F_x)}$$
\end{proof}

层上同调的定义.
\begin{itemize}
	\item 设$X$是一个拓扑空间,整体截面函子$\Gamma(X,-)$是$\textbf{Sh}(X,\textbf{Ab})\to\textbf{Ab}$的函子,它把$X$上的阿贝尔层$F$映射为整体截面群$\Gamma(X,F)$,把阿贝尔层态射$\varphi:F\to G$映射为整体截面之间的交换群同态$\Gamma(X,\varphi)=\varphi(X)$.这个函子在阿贝尔层范畴上是左正合的,它一般不是右正合函子.这个函子的右导出函子列称为$X$上的层上同调函子,记作$H^q(X,F)$.按照左正合性就有$H^0(X,F)=\Gamma(X,F)$.
	\item 设$(X,\mathscr{O}_X)$是环空间,函子$\Gamma(X,-)$是$\textbf{Mod}(\mathscr{O}_X)\to\mathscr{O}_X(X)-\textbf{Mod}$的左正合函子,它的右导出函子列称为环空间$X$上的模层上同调函子.我们会证明如果把$X$视为拓扑空间,把上同调结果视为阿贝尔群,那么这个导出函子列和上一条定义的是相同的.
	\item $X$上的一个层$F$称为零调的(acyclic),如果$H^q(X,F)=0,\forall q\ge1$.例如内射层自然总是零调层.
\end{itemize}

松弛层.一个阿贝尔层$F$称为松弛层(flasque),如果对每个开集$U\subset X$,每个截面$s\in F(U)$可以延拓为一个整体截面.阿贝尔层$F$的松弛预解是指如下正合列,使得每个$L^q,q\ge0$都是松弛层.
$$\xymatrix{0\ar[r]&F\ar[r]&L^0\ar[r]&L^1\ar[r]&\cdots}$$
\begin{enumerate}
	\item 松弛层的延拓条件等价于讲每个限制映射$\mathrm{res}_{U,V}$都是满射.于是如果$U\subset X$是开子集,$F$是$X$上的松弛层,那么限制层$F\mid U$也是松弛层.
	\item Godement预解.设$F$是环空间$X$上的模层,定义它的Godement层为$G^0F(U)=\prod_{x\in U}F_x$,限制映射自然定义为限制分量,明显的$G^0(F)$总是松弛层也是$X$上的模层.并且有典范的模层单同态$F\to G^0(F)$为$F(U)\to\prod_{x\in X}F_x$,$s\mapsto s_x$.假设已经构造了如下正合列,其中每个$G^iF$都是松弛层.
	$$\xymatrix{0\ar[r]&F\ar[r]^{d^{-1}}&G^0\mathscr{F}\ar[r]^{d^0}&G^1F\ar[r]^{d^1}&\cdots\ar[r]&G^{q-1}F\ar[r]^{d^{q-1}}&G^qF}$$
	
	取松弛层$G^{q+1}F=G^0(\mathrm{coker}d^{q-1})$,定义$d^q:G^qF\to G^{q+1}F$为复合$G^qF\to\mathrm{coker}d^{q-1}\to G^0(\mathrm{coker}d^{q-1})=G^{q+1}F$.那么$\ker d^q=\mathrm{im}d^{q-1}$.这个松弛预解称为$F$的Godement预解.于是我们证明了每个模层都存在松弛预解.
	\item 我们断言Godement预解构造中的$G^i$都是模层范畴上的正合加性函子.
	\begin{proof}
		
		给定模层的短正合列$0\to F_1\to F_2\to F_3\to0$,我们归纳证明如下总是短正合列:
		$$\xymatrix{0\ar[r]&G^iF_1\ar[r]&G^iF_2\ar[r]&G^iF_3\ar[r]&0}$$
		$$\xymatrix{0\ar[r]&\mathrm{coker}d^{i-1}_{F_1}\ar[r]&\mathrm{coker}d^{i-1}_{F_2}\ar[r]&\mathrm{coker}d^{i-1}_{F_3}\ar[r]&0}$$
		
		先证明$i=0$的两个短正合列.任取开子集$U\subset X$,按照茎函子是正合的,于是有如下正合列.这说明$0\to G^0F_1\to G^0F_2\to G^0F_3\to0$是正合的.
		$$\xymatrix{0\ar[r]&G^0F_1(U)\ar[r]&G^0F_2(U)\ar[r]&G^0F_3(U)\ar[r]&0}$$
		
		于是得到如下短正合列之间的交换图表:
		$$\xymatrix{0\ar[r]&F_1\ar[r]\ar[d]_{d^{-1}_{F_1}}&F_2\ar[r]\ar[d]_{d^{-1}_{F_2}}&F_3\ar[r]\ar[d]_{d^{-1}_{F_3}}&0\\0\ar[r]&G^0F_1\ar[r]&G^0F_2\ar[r]&G^0F_3\ar[r]&0}$$
		
		这里$d_{F_i}^{-1}$都是单射,所以蛇形引理说明有如下短正合列:
		$$\xymatrix{0\ar[r]&\mathrm{coker}d^{-1}_{F_1}\ar[r]&\mathrm{coker}d^{-1}_{F_2}\ar[r]&\mathrm{coker}d^{-1}_{F_3}\ar[r]&0}$$
		
		现在假设两个短正合列对指标$i$成立.按照$G^0$是正合的得到:
		$$\xymatrix{0\ar[r]&G^0\mathrm{coker}d^{i-1}_{F_1}\ar[r]&G^0\mathrm{coker}d^{i-1}_{F_2}\ar[r]&G^0\mathrm{coker}d^{i-1}_{F_3}\ar[r]&0}$$
		
		于是有如下短正合列之间的交换图表:
		$$\xymatrix{0\ar[r]&\mathrm{coker}d^{i-1}_{F_1}\ar[r]\ar[d]_{d^{i}_{F_1}}&\mathrm{coker}d^{i-1}_{F_2}\ar[r]\ar[d]_{d^{i}_{F_2}}&\mathrm{coker}d^{i-1}_{F_1}\ar[r]\ar[d]_{d^{i}_{F_3}}&0\\0\ar[r]&G^{i+1}F_1\ar[r]&G^{i+1}F_2\ar[r]&G^{i+1}F_3\ar[r]&0}$$
		
		垂直态射都是单的,按照蛇形引理就得到如下短正合列,这完成归纳.
		$$\xymatrix{0\ar[r]&\mathrm{coker}d^{i}_{F_1}\ar[r]&\mathrm{coker}d^{i}_{F_2}\ar[r]&\mathrm{coker}d^{i}_{F_3}\ar[r]&0}$$
	\end{proof}
	\item 内射模层总是松弛层.
	\begin{proof}
		
		设$F$是环空间$X$上的内射模层,那么有短正合列$0\to F\to G^0F\to F_1\to0$,按照$F$是内射模层,这个短正合列是分裂的,于是有$G^0F=F\oplus F_1$.但是从$G_0F$是松弛层,即它的限制映射总是满射,得到直和分量的限制映射也必须是满射,也即$F$是松弛层.
	\end{proof}
	\item 设$\xymatrix{0\ar[r]&F_1\ar[r]^i&F_2\ar[r]^{\varphi}&F_3\ar[r]&0}$是阿贝尔层的短正合列,并且$F_1$是松弛层,那么有短正合列$0\to\Gamma(X,F_1)\to\Gamma(X,F_2)\to\Gamma(X,F_3)\to0$(我们解释过层上截面函子一般不是正合的).这个结论对模层也是成立的.
	\begin{proof}
		
		我们解释过层上截面函子是左正合的,于是这里仅需验证诱导的$\Gamma(X,F_2)\to\Gamma(X,F_3)$是满态射,这个映射是$\varphi:s\mapsto\varphi(X)(s)$.任取$s_3\in F_3(X)$,定义$S=\{(U,s)\mid U\subset X\text{是开集},s\in F_2(U),\varphi(s)=s_3\mid U\}$.赋予$S$上偏序为$(U,s)\le(U_1,s_1)$当且仅当$U\subset U_1$并且$s_1\mid U=s$.容易验证链都有上界,于是Zorn引理说明存在极大元$(U_0,s_0)$.如果$U_0=X$,那么$s_0$就是$s_3$在$\varphi_X$下的原像,这就得到满射性.
		
		\qquad
		
		假设存在$x\in X-U_0$,按照$\varphi:F_2\to F_3$是满的层态射,于是它在茎上也是满同态,于是存在$x$的开邻域$V$和截面$t\in F_2$使得$\varphi(t)=s_3\mid V$,于是有$s-t\mid V\cap U_0\in F_1(V\cap U_0)$,假设$U_0\cap V=\emptyset$,那么直接粘合得到定义域更大的$S$中的元,和$(U_0,s_0)$的极大性矛盾.假设$U_0\cap V$非空,按照$F_1$是松弛层,存在$r\in F_1(X)$延拓了$s-t\mid V\cap U_0$,于是$s=t+r\mid V\cap U_0\in F_2(U_0\cap V)$,于是我们可以把$t$和$s_0$粘合得到$U_0\cup V$上的函数在$S$中,这也和极大性矛盾.
	\end{proof}
	\item 设$\xymatrix{0\ar[r]&F_1\ar[r]^i&F_2\ar[r]^{\varphi}&F_3\ar[r]&0}$是模层的短正合列,如果$F_1$和$F_2$都是松弛层,那么$F_3$也是松弛层.
	\begin{proof}
		
		短五引理的推论,如果下图中$\rho_1,\rho_2$是满射,那么$\rho_3$是满射.
		$$\xymatrix{0\ar[rr]&&F_1(X)\ar[rr]\ar[d]_{\rho_1}&&F_2(X)\ar[rr]\ar[d]_{\rho_2}&&F_3(X)\ar[d]^{\rho_3}\ar[rr]&&0\\0\ar[rr]&&F_1(U)\ar[rr]&&F_2(U)\ar[rr]&&F_3(U)\ar[rr]&&0}$$
	\end{proof}
	\item 松弛层在模层范畴中是零调的.于是松弛预解得到的上同调也是层上同调.
	\begin{proof}
		
		设$F$是松弛层,选取内射模层$F_1$使得有短正合列$0\to F\to F_1\to F_2\to0$.我们解释了内射模层总是松弛层,结合上一条得到$F_1$和$F_2$都是松弛层.这个短正合列诱导了如下长正合列:
		$$\xymatrix{0\ar[r]&H^0(X,F)\ar[r]&H^0(X,F_1)\ar[r]&H^0(X,F_2)\ar[r]&H^1(X,F)\ar[r]&H^1(X,F_1)=0}$$
		
		按照我们解释过这里$\Gamma(X,F_1)\to\Gamma(X,F_2)$是满射,说明$H^1(X,F)=0$.那么这个等式对任意的松弛模层$F$成立.再考虑长正合列的如下部分,归纳假设$H^{i-1}(X,-)=0$对任意松弛模层成立,就得到$H^i(X,-)=0$对任意松弛模层成立.
		$$\xymatrix{0=H^{i-1}(X,F_1)\ar[r]&H^{i-1}(X,F_2)\ar[r]&H^{i}(X,F)\ar[r]&H^i(X,F_1)=0}$$
	\end{proof}
	\item 设$(X,\mathscr{O}_X)$是环空间,那么$\textbf{Sh}(X,\textbf{Ab})$和$\textbf{Mod}_{\mathscr{O}_X}$上截面函子的右导出函子是相同的.这说明如果记$A=\Gamma(X,\mathscr{O}_X)$,那么$\mathscr{O}_X$模层$F$视为阿贝尔群得到的上同调群总是自然的具备一个$A$模结构.
	\begin{proof}
		
		计算模层上的层上同调需要内射模层预解,但是我们解释了内射模层都是松弛层,我们还解释了松弛预解得到的上同调是阿贝尔层上的层上同调,于是它们是相同的.
	\end{proof}
\end{enumerate}

零延拓层和函子$j_!$.
\begin{enumerate}
	\item 零延拓层.设$U\subset X$是开子集,设$F$是$U$上的模层,定义它在$X$上的零延拓层$j_!(F)$为如下预层的层化:对包含于$U$内的开集,设截面为$F(U)$,其余的开集截面取零.
	\begin{enumerate}
		\item $j_!(F)$在$x\in U$的点处的茎就是$F_x$,其余点的茎都是零.这说明$j_!$是正合函子.
		\item 这个函子和模层的限制函子$\textbf{Mod}(\mathscr{O}_X)\to\textbf{Mod}(\mathscr{O}_U)$是互相伴随的:
		$$\mathrm{Hom}_{\mathscr{O}_X}(j_!(F),G)\cong\mathrm{Hom}_{\mathscr{O}_U}(F,G\mid U)$$
		\item $\mathscr{O}_U$本身可以视为$j_!(\mathscr{O}_X\mid U)$.
		\item 如果记包含映射$i:X-U\to X$,设$F$是$X-U$上的层,那么$i_*F$和$j_!(F)$在stalk上恰好相反,如果$x\in U$那么$(i_*F)_x=0$,如果$x\in X-U$那么$(i_*F)_x=F_x$.那么如果$F$是$x$上的层,我们有如下$X$上层的短正合列:
		$$\xymatrix{0\ar[r]&j_!(F\mid_U)\ar[r]&F\ar[r]&i_*(F\mid_{X-U})\ar[r]&0}$$
	\end{enumerate}
	\item 设$(X,\mathscr{O}_X)$是环空间,我们已经证明过内射$\mathscr{O}_X$模层总是松弛层.这里给出另一种证明.
	\begin{proof}
		
		任取内射模层$\mathscr{I}$,任取开子集$V\subset U\subset X$,那么有模层的单同态$0\to\mathscr{O}_V\to\mathscr{O}_U$,作用正合函子$\mathrm{Hom}_{\mathscr{O}_X}(-,\mathscr{I})$,按照$\mathrm{Hom}_{\mathscr{O}_X}(\mathscr{O}_U,\mathscr{I})=\mathscr{I}(U)$,得到正合列$\mathscr{I}(U)\to\mathscr{I}(V)\to0$,这说明$\mathscr{I}$是松弛层.
	\end{proof}
	\item 内射层在开子集上的限制.设$(X,\mathscr{O}_X)$是环空间,设$I$是$\mathscr{O}_X$内射模层,射$U\subset X$是开子集,那么$I\mid U$是$\mathscr{O}_U$内射模层.特别的,对空间$X$上的阿贝尔层$F$,对任意开子集$U\subset X$,有$H^i(U,F)\cong H^i(U,F\mid U)$.
	\begin{proof}
		
		按照$j_!$是正合函子,从$\mathscr{Hom}_{\mathscr{O}_X}(j!-,I)$是正合函子得到$\mathrm{Hom}_{\mathscr{O}_U}(-,I\mid U)$是正合函子,于是$I\mid U$是内射模层.于是如果$I^*$是$F$的内射预解,得到$I^*\mid U$是$F\mid U$的内射预解,就得到典范同构:
		$$H^i(U,F\mid U)\cong H^i((I^*\mid U)(U))=H^i(I^*(U))=H^i(U,F)$$
	\end{proof}
\end{enumerate}

我们接下来主要证明这样一个定理:(Grothendieck)设$X$是$n$维诺特空间,那么对每个$i>n$和每个$X$上的阿贝尔层$\mathscr{F}$,有$H^i(X,\mathscr{F})=0$.
\begin{enumerate}
	\item 设$X$是诺特空间,其上松弛层的正向极限总是松弛层.
	\begin{proof}
		
		设$\{\mathscr{F}_j\}$是松弛层的一个正向系统,对任意开集$V\subset U$,每个$\mathscr{F}_j(U)\to\mathscr{F}_j(V)$都是满射.按照正向极限是正合函子(前提是有向集作为指标集),得到$\lim\limits_{\rightarrow}\mathscr{F}_j(U)\to\lim\limits_{\rightarrow}\mathscr{F}_j(V)$是满射.但是对于诺特空间有$\lim\limits_{\rightarrow}\mathscr{F}_j(U)=(\lim\limits_{\rightarrow}\mathscr{F}_j)(U)$,这说明正向极限也是松弛的.
	\end{proof}
	\item 设$X$是诺特空间,设$\{F_j\}$是阿贝尔层的一个正向系统,那么对每个$i\ge0$存在自然同构:
	$$\lim\limits_{\substack{\rightarrow\\j}}H^i(X,F_j)\cong H^i(X,\lim\limits_{\substack{\rightarrow\\j}}F_j)$$
	\begin{proof}
		
		首先有向指标集$I$可视为一个范畴.于是$\lim\limits_{\rightarrow}H^i(X,-)$和$H^i(X,\lim\limits_{\rightarrow}-)$都是$\textbf{Sh}(X,\textbf{Ab})^{\mathscr{I}}\to\textbf{Ab}$的函子.它们满足如下四件事保证了它们是自然同构的:
		\begin{itemize}
			\item 如果一个正向系统由松弛层构成,就称为松弛系统.我们断言对任意正向系统$\{F_i\}$,都可以单嵌入到一个松弛系统中.这是因为我们之前构造的Godement预解中的单射$F\to G^0F$是自然变换,所以$\{G^0F_i\}$就是期望的松弛系统.
			\item $\lim\limits_{\rightarrow}H^i(X,-)$和$H^i(X,\lim\limits_{\rightarrow}-)$都是上同调函子.这是因为$I$是有向范畴的时候,正向极限函子是正合函子.
			\item $n=0$的时候有自然同构$\lim\limits_{\rightarrow}\Gamma(X,-)\cong\Gamma(X,\lim\limits_{\rightarrow}-)$.而这是因为诺特空间上阿贝尔层在预层范畴中的正向极限已经就是层,而截面函子$\Gamma$本身就是预层范畴上的正合函子.
			\item 最后要验证对松弛系统$\{F_i\}$都有当$i\ge1$时$\lim\limits_{\rightarrow}H^i(X,F_i)=H^i(X,\lim\limits_{\rightarrow}F_i)=0$.而这是因为我们解释了诺特空间上松弛系统的正向极限也是松弛的.
		\end{itemize}
	\end{proof}
	\item 设$Y$是$X$的闭子集,设$F$是$Y$上的阿贝尔层,设$j:Y\to X$是包含映射,那么$j_*F$是$X$上的层.
	\begin{itemize}
		\item 如果$p\in Y$,那么$(j_*F)_x=F_x$,如果$p\not\in Y$有$(j_*F)_x=0$.
		\item 对任意$i\ge0$总有$H^i(Y,F)=H^i(X,j_*F)$.
	\end{itemize}
	\item 如果$Y\subset X$是闭子集,对$X$上的层$X$,记$F_Y=j_*(F\mid Y)$,其中$j:Y\subset X$是包含映射,$F\mid Y$定义为$F$在$Y$上的回拉(即$l^{-1}(F)$,其中$l:Y\subset X$是包含映射).如果$U\subset X$是开子集,对$X$上的层$F$,记$F_U=i_!(F\mid U)$,其中$i:U\subset X$是包含映射.于是如果$Y$是闭子集,$U=X-Y$是开子集,考虑茎就得到短正合列:
	$$\xymatrix{0\ar[r]&F_U\ar[r]&F\ar[r]&F_Y\ar[r]&0}$$
	\item 我们证明承诺的定理.设$X$是$n$维诺特空间,那么对每个$i>n$和每个$X$上的阿贝尔层$\mathscr{F}$,有$H^i(X,\mathscr{F})=0$.
	\begin{proof}
		
		第一步,归结为$X$不可约的情况.如果$X$是可约的,选取它的不可约分支$Y$,取$U=X-Y$,那么有短正合列$0\to F_U\to F\to F_Y\to0$.这诱导了长正合列,倘若我们解决了不可约的情况,问题就归结为$H^i(X,F_U)=0,i>n$.记闭嵌入$h:\overline{U}\subset X$,按照$F_U$被$\overline{U}$支撑,得到同构$F_U\cong h_*h^{-1}F_U$,于是就有$H^i(X,F_U)=H^i(X,h_*h^{-1}F_U)=H^i(\overline{U},h^{-1}F_U)$.但是$\overline{U}$是$X$的真闭子集,它的不可约分支个数相比$X$一定严格减小1,所以对分支个数归纳假设就归结为证$X$不可约的情况.
		
		\qquad
		
		第二步,解决零维不可约的情况.此时$X$不能存在非空非全集的闭子集,所以$X$仅有的非空开集是全集,所以$\Gamma(X,-)$和茎函子是一致的,所以它是正合函子,所以有$H^i(X,F)=0,\forall i\ge1,\forall F$.
		
		\qquad
		
		第三步,把问题归结为单个局部截面生成的情况.称$F$的局部截面是指某个开子集上的截面,设$\alpha=\{s_i\in F(U_i)\mid i=1,2,\cdots,n\}$为$F$的有限个局部截面.定义这有限个局部截面生成的$F$的子层为,$F$的所有包含这有限个局部截面的子层的交.它还可以理解为层态射$\oplus_{1\le i\le n}(j_i)_!\mathbb{Z}_{U_i}\to F$的像,这里$(j_i)_!\mathbb{Z}_{U_i}$表示$U_i$上的$\mathbb{Z}$常值层零延拓到整个$X$上.层态射$(j_i)_!\mathbb{Z}_{U_i}\to F$就定义为$\mathbb{Z}\to F(U_i)$,$1\mapsto s_i$的层化态射.具体的讲,设$s\in F(U)$,我们来证明$j_!\mathbb{Z}_U\to F$的像是$F_{\{s\}}$:首先我们解释$\mathbb{Z}_U\to F\mid U$具体是什么,任取$V\subset U$上$\mathbb{Z}_U$的截面,这是一个局部常值映射$f:V\to\mathbb{Z}$,设它在$V$的开子集$V_n$上取值为$n$,我们定义$f$在$F(V)$中的像就是$ns\mid_{V_n}\in F(V_n)$在$F(V)$中的唯一粘合.于是这样的截面必然落在每个包含局部截面$s$的子层$F'$中.于是这个像落在$F_{\{s\}}$中.另外按照层的一个子预层的层化就是所有包含它的子层的交,就得到这个像恰好是$F_{\{s\}}$.类似可得到$F_{\alpha}$就是$\oplus_{1\le i\le n}(j_i)_!\mathbb{Z}_{U_i}\to F$的像.
		
		\qquad
		
		这个概念的意义在于当$\alpha$跑遍$F$的所有有限局部截面集合时,$F_{\alpha}$和自然的包含映射$F_{\alpha}\to F_{\beta},\alpha\subset\beta$构成了一个有向集作为指标集的正向系统,而它的极限就是$F$本身(类似于模总是它有限生成子模的正向极限).于是按照诺特空间上层上同调和正向极限可交换,归结为设$F$可被有限个局部截面生成的情况.如果$\alpha'$是$\alpha$的子集,那么有短正合列$0\to F_{\alpha'}\to F_{\alpha}\to F_{\alpha-\alpha'}\to0$(这是因为每个$F_{\alpha}$是像的有限直和,在层化之前作为预层的这些直和必然满足短正合列,而层化是正合函子).按照诱导的长正合列以及对$\alpha$的元素个数做归纳,归结为证明$F$是被单个局部截面生成的情况.
		
		\qquad
		
		第四步,把问题归结为$F=\mathbb{Z}_U$的情况.如果$F$是被单个局部截面生成的,设这单个局部截面所在的开子集是$U$,那么存在层的满态射$\mathbb{Z}_U\to F$,记核为$K$,那么有层的短正合列$0\to K\to\mathbb{Z}_U\to F\to0$.于是归结为证明$K$和$\mathbb{Z}_U$的情况.如果$K=0$那么问题已经归结为$F=\mathbb{Z}_U$的情况.如果$K\not=0$,每个$K_x$都是$\mathbb{Z}$的子群,设$d$是所有$K_x$中出现的最小的正整数,那么使得$K_x=d\mathbb{Z}$的点$x\in U$构成了一个开子集$V$,于是有短正合列$0\to\mathbb{Z}_V\to K\to K/\mathbb{Z}_V\to0$.但是这里$K/\mathbb{Z}_V$被闭集$\overline{U-V}$支撑,按照$X$是不可约的,这个闭子集的维数严格小于$\dim X$,归纳假设就得到$H^i(X,K/\mathbb{Z}_V)=0,\forall i>n$.于是问题归结为证明对任意开子集$U\subset X$,有$H^i(X,\mathbb{Z}_U)=0,\forall i>n$成立.
		
		\qquad
		
		第五步,证明$\mathbb{Z}_U$的情况.取$Y=X-U$是闭集,我们解释过存在短正合列$0\to\mathbb{Z}_U\to\mathbb{Z}_X\to\mathbb{Z}_Y\to0$.按照诱导的长正合列,问题归结为证明$H^i(X,\mathbb{Z}_X)=H^i(X,\mathbb{Z}_Y)=0,\forall i>n$.按照$X$是不可约的,有$\dim Y<\dim X$,于是归纳假设可得到$H^i(X,\mathbb{Z}_Y)=0,\forall i>n$.最后按照$X$是不可约的,其上关于$\mathbb{Z}$的常值预层本身已经是常值层(按照非空开集的交总是非空的,得到粘合存在性).于是它是松弛层,于是有$H^i(X,\mathbb{Z}_X)=0,\forall i\ge1$.完成证明.
	\end{proof}
\end{enumerate}

高阶前推函子.设$f:X\to Y$是连续映射,前推函子$f_*:\textbf{Sh}(X,\textbf{Ab})\to\textbf{Sh}(Y,\textbf{Ab})$是左正合的,并且$\textbf{Sh}(X,\textbf{Ab})$具有足够多的内射对象.$f_*$的右导出函子列记作$R^if_*:\textbf{Sh}(X,\textbf{Ab})\to\textbf{Sh}(Y,\textbf{Ab})$.
\begin{enumerate}
	\item 设$i\ge0$,设$F$是$X$上的阿贝尔层,设$f:X\to Y$是连续映射,那么$R^if_*(F)$是$Y$上如下预层的层化:
	$$V\mapsto H^i(f^{-1}(V),F)$$
	\begin{proof}
		
		任取$F$的内射预解$F\to I^*$,按照定义就有$R^if_*F=\ker(f_*I^i\to f_*I^{i+1})/\mathrm{im}(f_*I^{i-1}\to f_*I^i)$.于是$R^if_*F$就是如下预层的层化:
		\begin{align*}
			V&\mapsto\frac{\ker\left(f_*I^i(V)\to f_*I^{i+1}(V)\right)}{\mathrm{im}\left(f_*I^{i-1}(V)\to f_*I^i(V)\right)}\\&=\frac{\ker\left(I^i(f^{-1}(V))\to I^{i+1}(f^{-1}(V))\right)}{\mathrm{im}\left(I^{i-1}(f^{-1}(V))\to I^i(f^{-1}(V))\right)}\\&\cong H^i(f^{-1}(V),F)
		\end{align*}
	\end{proof}
	\item 如果$F$是$X$上的松弛层,按照松弛层在开子集上的限制仍是松弛层,说明它的前推$f_*F$总是松弛层,并且总有$R^if_*(F)=0,\forall i\ge1$成立.
	\item 设$f:(X,\mathscr{O}_X)\to(Y,\mathscr{O}_Y)$是环空间之间的态射,把$f_*$视为连续映射诱导的左正合函子$$\textbf{Sh}(X,\textbf{Ab})\to\textbf{Sh}(Y,\textbf{Ab})$$和视为$$\textbf{Mod}(\mathscr{O}_X)\to\textbf{Mod}(\mathscr{O}_Y)$$的左正合函子得到的右导出函子如果视为终端为$\textbf{Ab}$的函子是相同的.
	\begin{proof}
		
		计算$f_*:\textbf{Mod}(\mathscr{O}_X)\to\textbf{Mod}(\mathscr{O}_Y)$的右导出函子要用内射模层预解,但是内射模层总是松弛层,并且用松弛预解计算的同调群是$f_*:\textbf{Sh}(X,\textbf{Ab})\to\textbf{Sh}(Y,\textbf{Ab})$的右导出函子列,所以二者相同.
	\end{proof}
\end{enumerate}

延拓群和延拓层($\mathrm{Ext}$).
\begin{itemize}
	\item 设$F$是$\mathscr{O}_X$模层,函子$\mathrm{Hom}_{\mathscr{O}_X}(F,-)$是$\textbf{Mod}(\mathscr{O}_X)\to\textbf{Ab}$的左正合函子,它的右导出函子列记作$\mathrm{Ext}^i_{\mathscr{O}_X}(F,-)$,称为延拓群.
	\item 设$F$是$\mathscr{O}_X$模层,函子$\mathrm{HOM}_{\mathscr{O}_X}(F,-)$是$\textbf{Mod}(\mathscr{O}_X)\to\textbf{Mod}(\mathscr{O}_Y)$的左正合函子,它的右导出函子列记作$\mathrm{EXT}^i_{\mathscr{O}_X}(F,-)$,称为延拓层.
\end{itemize}
\begin{enumerate}
	\item 先解释$\mathrm{HOM}$的左正合性.如果有模层的左正合列:
	$$\xymatrix{0\ar[r]&G_1\ar[r]&G_2\ar[r]&G_3}$$
	
	任取开集$U\subset X$,那么有模层限制的左正合列:
	$$\xymatrix{0\ar[r]&G_1\mid_U\ar[r]&G_2\mid_U\ar[r]&G_3\mid_U}$$
	
	按照小$\mathrm{Hom}$的左正合性,任取模层$F$,就得到:
	$$\xymatrix{0\ar[r]&\mathrm{Hom}_U(F\mid_U,G_1\mid_U)\ar[r]&\mathrm{Hom}_U(F\mid_U,G_2\mid_U)\ar[r]&\mathrm{Hom}_U(F\mid_U,G_3\mid_U)}$$
	
	也即左正合列:
	$$\xymatrix{0\ar[r]&\mathrm{HOM}_X(F,G_1)(U)\ar[r]&\mathrm{HOM}_X(F,G_2)(U)\ar[r]&\mathrm{HOM}_X(F,G_3)(U)}$$
	
	这说明在预层范畴上下式已经是正合的,所以层范畴上自然也是:
	$$\xymatrix{0\ar[r]&\mathrm{HOM}_X(F,G_1)\ar[r]&\mathrm{HOM}_X(F,G_2)\ar[r]&\mathrm{HOM}_X(F,G_3)}$$
	\item 设$(X,\mathscr{O}_X)$是环空间,设$F,G$是$\mathscr{O}_X$模层,对每个$i\ge0$,有$\mathrm{EXT}^i_X(F,G)$是预层$U\mapsto\mathrm{Ext}^i_X(F\mid U,G\mid U)$的层化.
	\item 对开集$U\subset X$,有典范同构:
	$$\mathrm{EXT}^i_X(F,G)\mid U\cong\mathrm{EXT}^i_U(F\mid U,G\mid U)$$
	\item 例如$\mathrm{Hom}_{\mathscr{O}_X}(\mathscr{O}_X,-)$自然同构于整体截面函子$\Gamma(X,-)$;函子$\mathrm{HOM}_{\mathscr{O}_X}(\mathscr{O}_X,-)$自然同构于恒等函子.于是有:
	\begin{itemize}
		\item $\mathrm{EXT}^0_X(\mathscr{O}_X,G)=G$.
		\item $\mathrm{EXT}_X^i(\mathscr{O}_X,G)=0,\forall i\ge1$.
		\item $\mathrm{Ext}^i_X(\mathscr{O}_X,G)\cong H^i(X,G),\forall i\ge0$.
	\end{itemize}
	\item 如果$0\to G_1\to G_2\to G_3\to0$是$\mathscr{O}_X$模层的短正合列,按照导出函子列是同调$\delta$函子,就有延拓群和延拓层的如下长正合列:
	$$\xymatrix{\cdots\ar[r]&\mathrm{Ext}^i_X(F,G_1)\ar[r]&\mathrm{Ext}^i_X(F,G_2)\ar[r]&\mathrm{Ext}^i_X(F,G_3)\ar[r]&\mathrm{Ext}^{i+1}_X(F,G_1)\ar[r]&\cdots}$$
	
	不平凡的是如果$0\to F_1\to F_2\to F_3\to0$是$\mathscr{O}_X$模层的短正合列,仍有延拓群和延拓层的如下长正合列(因为按理说$\mathrm{Ext}_X^i(-,-)$不是双函子的导出函子列,所以本不该诱导这个位置的长正合列).
	$$\xymatrix{\cdots\ar[r]&\mathrm{Ext}^i_X(F_3,G)\ar[r]&\mathrm{Ext}^i_X(F_2,G)\ar[r]&\mathrm{Ext}^i_X(F_1,G)\ar[r]&\mathrm{Ext}^{i+1}_X(F_1,G)\ar[r]&\cdots}$$
	\begin{proof}
		
		取$G$的内射预解$G\to I^*$,那么有如下复形的短正合列,它取长正合列得到结论.
		$$\xymatrix{0\ar[r]&\mathrm{Hom}_X(F_3,I^*)\ar[r]&\mathrm{Hom}_X(F_2,I^*)\ar[r]&\mathrm{Hom}_X(F_1,I^*)\ar[r]&0}$$
	\end{proof}
	\item 设$F$是$\mathscr{O}_X$模层,设$L_*\to F$是有限秩局部自由模层构成的预解,那么对任意$\mathscr{O}_X$模层$G$有延拓群或者延拓层的典范同构:
	$$\mathrm{EXT}^i_X(F,G)\cong H^i(\mathrm{HOM}_{\mathscr{O}_X}(L_*,G))$$
	\begin{proof}
		
		这个证明要借助谱序列.记$I^*$是$G$的简化内射预解,考虑双复形$M_{s,t}=\mathrm{HOM}_X(L_s,I^t)$,按照$L_s$是局部有限秩自由层,说明$\mathrm{HOM}_X(L_s,-)$是正合函子,所以有:
		$$(^1E)_1^{s,t}=H^t(M_{s,\bullet})=\left\{\begin{array}{cc}\{0\}&t\ge1\\\mathrm{HOM}_X(L_s,G)&t=0\end{array}\right.$$
		
		再求同调得到第一谱序列:
		$$(^1E)_2^{s,t}=H^sH^t(M_{s,\bullet})=\left\{\begin{array}{cc}\{0\}&t\ge1\\H^s(\mathrm{HOM}_X(L_*,G))&t=0\end{array}\right.$$
		
		这个谱序列塌陷在$s$轴,说明有同构:
		$$H^n(\mathrm{Tot}(M_{s,t}))\cong(^1E)_2^{n,0}=H^n(\mathrm{HOM}_X(L_*,G))$$
		
		下面求第二谱序列,首先按照$\mathrm{HOM}_X(-,I^t)$是正合的,得到:
		$$(^2E)_1^{s,t}=H^s(M_{\bullet,t})=\left\{\begin{array}{cc}\{0\}&t\ge1\\\mathrm{HOM}_X(F,I^t)&t=0\end{array}\right.$$
		
		再求同调得到第二谱序列:
		$$(^2E)_2^{s,t}=H^tH^s(M_{\bullet,t})=\left\{\begin{array}{cc}\{0\}&t\ge1\\\mathrm{EXT}_X^s(F,G)&t=0\end{array}\right.$$
		
		这个谱序列塌陷在$s$轴,说明有同构:
		$$H^n(\mathrm{Tot}(M_{s,t}))\cong(^2E)_2^{0,n}=\mathrm{EXT}_X^n(F,G)$$
		
		综上得到自然同构:$$\mathrm{EXT}_X^n(F,G)\cong H^n(\mathrm{HOM}_X(L_*,G))$$
	\end{proof}
	\item 设$F,G$是$\mathscr{O}_X$模层,称$F$关于$G$的一个延拓是指短正合列$0\to G\to H\to F\to0$.两个这样的延拓称为同构,如果存在层同构$H_1\to H_2$使得有如下图表交换:
	$$\xymatrix{0\ar[r]&G\ar[r]\ar@{=}[d]&H_1\ar[r]\ar[d]&F\ar[r]\ar@{=}[d]&0\\0\ar[r]&G\ar[r]&H_2\ar[r]&F\ar[r]&0}$$
	
	我们断言存在从$\mathrm{Ext}_{\mathscr{O}_X}^1(F,G)$到$F$关于$G$的所有延拓的同构类的双射.
	\begin{proof}
		
		选定$G$的内射预解$\xymatrix{0\ar[r]&G\ar[r]&I^0\ar[r]^{d^0}&I^1\ar[r]^{d^1}&\cdots}$.按照定义有如下等式,其中商理解为作为预层的商的层化:
		\begin{align*}
			\mathrm{Ext}_{\mathscr{O}_X}^1(F,G)&=\frac{\ker\left(\mathrm{Hom}_{\mathscr{O}_X}(F,I^1)\to\mathrm{Hom}_{\mathscr{O}_X}(F,I^2)\right)}{\mathrm{im}\left(\mathrm{Hom}_{\mathscr{O}_X}(F,I^0)\to\mathrm{Hom}_{\mathscr{O}_X}(F,I^1)\right)}\\&=\frac{\mathrm{Hom}_{\mathscr{O}_X}(F,Z^1(I^*))}{\mathrm{im}\left(\mathrm{Hom}_{\mathscr{O}_X}(F,I^0)\to\mathrm{Hom}_{\mathscr{O}_X}(F,I^1)\right)}
		\end{align*}
		
		设$F$关于$G$的所有延拓的所有等价类构成的集合为$S$.我们来构造映射$\Phi:\mathrm{Ext}^1_{\mathscr{O}_X}(F,G)\to S$如下:任取$e\in\mathrm{Ext}^1_{\mathscr{O}_X}(F,G)=\frac{\mathrm{Hom}_{\mathscr{O}_X}(F,Z^1(I^*))}{\mathrm{im}\left(\mathrm{Hom}_{\mathscr{O}_X}(F,I^0)\to\mathrm{Hom}_{\mathscr{O}_X}(F,I^1)\right)}$.任取代表元$e':F\to Z^1(I^*)$.取典范映射$d^0:I^0\to Z^1(I^*)$和$e':F\to Z^1(I^*)$的纤维积为$I^0\oplus_{e'}F$,它是$I^0\oplus F$的子层,由满足$d^0(s)=e'(t)$的$(s,t)$构成.在考虑恒等态射$1_G:G\to G$和零态射$G\to F$诱导的$G\to I^0\oplus F$,那么这是单态射,因为它复合$I^0\oplus F\to I^0$得到单态射$G\to I^0$.验证上一行是短正合列,我们得到如下短正合列之间的交换图.就定义$\Phi(e)$为上一行短正合列所在的等价类.
		$$\xymatrix{0\ar[r]&G\ar[r]\ar@{=}[d]&I^0\oplus _{e'}F\ar[r]\ar[d]&F\ar[r]\ar[d]^{e'}&0\\0\ar[r]&G\ar[r]&I^0\ar[r]^{d^0}&Z^1(I^*)\ar[r]&0}$$
		
		我们需要验证这个定义不依赖于代表元$e'$的选取.如果还存在一个代表元$e'':F\to Z^1(I^*)$,那么存在$f:F\to I^0$使得$e'-e''=d^0f$.定义$I^0\oplus_{e'}F\to I^0\oplus_{e''}F$为$(s,t)\mapsto(s-f(t),t)$.那么有如下交换图表,短五引理说明这时诱导的延拓是同构的.
		$$\xymatrix{0\ar[r]&G\ar[r]\ar@{=}[d]&I^0\oplus_{e'}F\ar[r]\ar[d]&F\ar[r]\ar@{=}[d]&0\\0\ar[r]&G\ar[r]&I^0\oplus_{e''}F\ar[r]&F\ar[r]&0}$$
		
		定义$\Psi:S\to\mathrm{Ext}_{\mathscr{O}_X}^1(F,G)$如下:任取等价类$E\in S$,任取代表元$0\to F\to H\to F\to0$,我们可以把$G\to I^0$提升为$\psi:H\to I^0$,于是存在同态$\overline{\psi}:F\to Z^1(I^*)$使得如下图表交换:
		$$\xymatrix{0\ar[r]&G\ar@{=}[d]\ar[r]&H\ar[r]\ar[d]^{\psi}&F\ar[r]\ar[d]^{\overline{\psi}}&0\\0\ar[r]&G\ar[r]&I^0\ar[r]&Z^1(I^*)\ar[r]&0}$$
		
		定义$\Psi(E)$为$\overline{\psi}:F\to Z^1(I^*)$在$\frac{\mathrm{Hom}_{\mathscr{O}_X}(F,Z^1(I^*))}{\mathrm{im}\left(\mathrm{Hom}_{\mathscr{O}_X}(F,I^0)\to\mathrm{Hom}_{\mathscr{O}_X}(F,I^1)\right)}$中的像.我们断言这个定义不依赖于提升$\psi$和$\overline{\psi}$的选取.假设还存在提升$\psi':H\to I^0$,那么$\psi-\psi'$在$\ker(G\to H)$上平凡,导致可定义$\delta:F\to I^0$使得$\overline{\psi}-\overline{\psi}'=d^0\delta$.于是$\overline{\psi}$和$\overline{\psi}'$相差$\mathrm{im}\left(\mathrm{Hom}_{\mathscr{O}_X}(F,I^0)\to\mathrm{Hom}_{\mathscr{O}_X}(F,I^1)\right)$中的元.
		
		\qquad
		
		还要验证这个定义不依赖于$E$的代表元的选取.假设还有代表元$0\to G\to H'\to F\to0$.于是存在$\alpha:H'\to H$使得两个代表元作为短正合列是同构的,于是得到如下交换图表,导致$\psi\circ\alpha:H'\to I^0$是$G\to I^0$的提升,按照我们之前证明的不依赖提升的选取,说明$\overline{\psi}$也是短正合列代表元$0\to G\to H'\to F\to0$在$\Psi$下的像,这完成证明.
		$$\xymatrix{0\ar[r]&G\ar[r]\ar@{=}[d]&H'\ar[r]\ar[d]^{\alpha}&F\ar[r]\ar@{=}[d]&0\\0\ar[r]&G\ar[r]\ar@{=}[d]&H\ar[r]\ar[d]^{\psi}&F\ar[r]\ar[d]^{\overline{\psi}}&0\\0\ar[r]&G\ar[r]&I^0\ar[r]&Z^1(I^*)\ar[r]&0}$$
		
		最后验证$\Phi$和$\Psi$互为逆映射.验证$\Psi\circ\Phi=1$是直接的,验证$\Phi\circ\Psi=1$只要注意如下交换图表:
		$$\xymatrix{0\ar[r]&G\ar[r]\ar@{=}[d]&H\ar[d]\ar[r]&F\ar[r]\ar@{=}[d]&0\\0\ar[r]&G\ar[r]\ar@{=}[d]&I^0\oplus_{e'}F\ar[r]\ar[d]&F\ar[d]^{\overline{\psi}}\ar[r]&0\\0\ar[r]&G\ar[r]&I^0\ar[r]&Z^1(I^*)\ar[r]&0}$$
	\end{proof}
	\item 设$L$是$X$上的有限秩局部自由模层,设$\mathscr{I}$是$X$上内射模层,那么$\mathscr{L}\otimes\mathscr{I}$也是内射模层.
	\begin{proof}
		
		我们要证明函子$\mathrm{Hom}(-,L\otimes I)$是正合的,这个函子同构于$\mathrm{Hom}(-\otimes L^{\vee},I)$,其中$L^{\vee}$是对偶模层$\mathrm{HOM}_{\mathscr{O}_X}(L,\mathscr{O}_X)$,它也是一个有限秩局部自由模层,于是$-\otimes L^{\vee}$是正合函子,复合上正合函子$\mathrm{Hom}_{\mathscr{O}_X}(-,I)$就还是正合的.
	\end{proof}
	\item 设$L$是局部自由有限秩模层,设$L^{\vee}$是它的对偶,对任意模层$F,G$,就有如下成立:
	$$\mathrm{Ext}^i(F\otimes L,G)\cong\mathrm{Ext}^i(F,L^{\vee}\otimes G)$$
	$$\mathrm{EXT}^i(F\otimes L,G)\cong\mathrm{EXT}^i(F,L^{\vee}\otimes G)\cong\mathrm{EXT}^i(F,G)\otimes L^{\vee}$$
	\begin{proof}
		
		我们之前给出过$i=0$的情况.对于一般情况,第一个同构式的两侧是$\textbf{Mod}(\mathscr{O}_X)\to\textbf{Ab}$的同调$\delta$函子,第二个同构式的两侧是$\textbf{Mod}(\mathscr{O}_X)\to\textbf{Mod}(\mathscr{O}_X)$的同调$\delta$函子.因为$G$取内射模层的时候这些式子都是零,按照同调$\delta$函子唯一性的定理得到它们典范同构.
	\end{proof}
\end{enumerate}
\subsubsection{\v{C}ech上同调}

\v{C}ech复形和上同调.
\begin{enumerate}
	\item 设$\mathscr{U}=\{U_i\mid i\in I\}$是拓扑空间$X$上的开覆盖.对有限个指标$i_0,i_1,\cdots,i_p\in I$,记$U_{i_0,i_1,\cdots,i_p}=U_{i_0}\cap U_{i_1}\cap\cdots\cap U_{i_p}$.如果$\mathscr{F}$是$X$上的阿贝尔层,记$\check{C}^p(\mathscr{U},\mathscr{F})=\prod_{(i_0,\cdots,i_p)\in I^{p+1}}\mathscr{F}(U_{i_0,\cdots,i_p})$.
	\item 一个元素$a\in\check{C}^p(\mathscr{U},\mathscr{F})$相当于$I^{p+1}$上的取值在$\mathscr{F}$上的映射.它称为关于$\mathscr{U}$的取值在$\mathscr{F}$的一个$p$余链.
	\item 定义微分算子$\mathrm{d}:\check{C}^p(\mathscr{U},\mathscr{F})\to\check{C}^{p+1}(\mathscr{U},\mathscr{F})$为如下,那么有$\mathrm{d}\circ\mathrm{d}=0$.
	$$(\mathrm{d}a)(i_0,\cdots,i_{n+1})=\sum_{j=0}^{n+1}(-1)^ja(i_0,\cdots,\hat{i_j},\cdots,i_{n+1})\mid U_{i_0,\cdots,i_{n+1}}$$
	\item 定义如下复形为$\mathscr{F}$关于开覆盖$\mathscr{U}$的\v{C}ech复形.这个复形的上同调群(不是简化的)称为$\mathscr{F}$的关于开覆盖$\mathscr{U}$的\v{C}ech上同调群,记作$\check{H}^p(\mathscr{U},\mathscr{F})$.
	$$\xymatrix{0\ar[r]&\check{C}^0(\mathscr{U},\mathscr{F})\ar[r]^{\mathrm{d}}&\check{C}^1(\mathscr{U},\mathscr{F})\ar[r]^{\mathrm{d}}&\check{C}^2(\mathscr{U},\mathscr{F})\ar[r]&\cdots}$$
\end{enumerate}

交错形式和全序形式的\v{C}ech上同调.
\begin{itemize}
	\item 一个$p$余链$a\in\check{C}^p(\mathscr{U},\mathscr{F})$称为交错的,如果它满足对任意$\{0,1,\cdots,p\}$上的双射$\sigma$都有$a(i_{\sigma(0)},\cdots,i_{\sigma(p)})=\mathrm{sgn}(\sigma)a(i_0,\cdots,i_p)$.这个条件可以推出只要有$j<k$使得$i_j=i_k$,那么有$a(i_0,\cdots,i_n)=0$.
	\item 全体交错$p$余链构成$\check{C}^p(\mathscr{U},\mathscr{F})$的子群,记作$\check{C}^{\mathrm{alt},p}(\mathscr{U},\mathscr{F})$.我们断言微分算子可以限制在交错子群上,即有$\mathrm{d}(\check{C}^{\mathrm{alt},p})\subset\check{C}^{\mathrm{alt},p+1}(\mathscr{U},\mathscr{F})$.于是得到如下复形,称为$\mathscr{F}$关于$\mathscr{U}$的交错\v{C}ech复形,它的上同调群称为交错\v{C}ech上同调群.
	$$\xymatrix{0\ar[r]&\check{C}^{\mathrm{alt},0}(\mathscr{U},\mathscr{F})\ar[r]^{\mathrm{d}}&\check{C}^{\mathrm{alt},1}(\mathscr{U},\mathscr{F})\ar[r]^{\mathrm{d}}&\check{C}^{\mathrm{alt},2}(\mathscr{U},\mathscr{F})\ar[r]&\cdots}$$
	\item 赋予指标集$I$上全序.定义$\check{C}^{\mathrm{tot},p}(\mathscr{U},\mathscr{F})=\prod_{i_0<\cdots<i_p}\mathscr{F}(U_{i_0,\cdots,i_p})$.这是$\check{C}^p(\mathscr{U},\mathscr{F})$的子群.微分算子自然的限制为$\check{C}^{\mathrm{tot},p}(\mathscr{U},\mathscr{F})\to\check{C}^{\mathrm{tot},p+1}(\mathscr{U},\mathscr{F})$的微分映射.于是得到如下复形.
	$$\xymatrix{0\ar[r]&\check{C}^{\mathrm{tot},0}(\mathscr{U},\mathscr{F})\ar[r]^{\mathrm{d}}&\check{C}^{\mathrm{tot},1}(\mathscr{U},\mathscr{F})\ar[r]^{\mathrm{d}}&\check{C}^{\mathrm{tot},2}(\mathscr{U},\mathscr{F})\ar[r]&\cdots}$$
\end{itemize}
\begin{enumerate}
	\item 验证定义中微分算子可以限制在交错子群上.
	\begin{proof}
		
		因为对称群中的每个双射都可以分解为若干对换的复合,所以我们只要验证对换满足交错定义中的等式即可.设$\sigma$是把$j,j+1$交换,其余恒等的映射.设$a$是一个交错$p$余圈.那么有:
		\begin{align*}
			&(\mathrm{d}a)(i_0,\cdots,i_{j+1},i_j,\cdots,i_{p+1})\\&=\sum_{k=0}^{j-1}(-1)^ka(i_0,\cdots,\hat{i_k},\cdots,i_{j+1},i_j,\cdots,i_{n+1})\\&+(-1)^ja(i_0,\cdots,\hat{i_{j+1}},i_j,\cdots,i_{n+1})+(-1)^{j+1}a(i_0,\cdots,i_{j+1},\hat{i_j},\cdots,i_{n+1})\\&+\sum_{k=j+2}^{p+1}(-1)^ka(i_0,\cdots,i_{j+1},i_j,\cdots,\hat{i_k},\cdots,i_{n+1})\\&=\sum_{k=0}^{j-1}(-1)^{k+1}a(i_0,\cdots,\hat{i_k},\cdots,i_j,i_{j+1},\cdots,i_{n+1})\\&+(-1)^ja(i_0,\cdots,i_j,\hat{i_{j+1}},\cdots,i_{n+1})+(-1)^{j+1}a(i_0,\cdots,\hat{i_j},i_{j+1},\cdots,i_{n+1})\\&+\sum_{k=j+2}^{p+1}(-1)^{k+1}a(i_0,\cdots,i_j,i_{j+1},\cdots,\hat{i_k},\cdots,i_{n+1})\\&=-(\mathrm{d}a)(i_0,\cdots,i_j,i_{j+1},\cdots,i_{n+1})
		\end{align*}
	\end{proof}
	\item 交错形式和全序形式的\v{C}ech复形是同构的,于是特别的它们具有相同的上同调列.
	$$\xymatrix{\check{C}^{\mathrm{alt},i}(\mathscr{U},\mathscr{F})\ar[rr]^{\mathrm{d}}\ar[d]_{\sim}&&\check{C}^{\mathrm{alt},i+1}(\mathscr{U},\mathscr{F})\ar[d]^{\sim}\\\check{C}^{\mathrm{tot},i}(\mathscr{U},\mathscr{F})\ar[rr]^{\mathrm{d}}&&\check{C}^{\mathrm{tot},i+1}(\mathscr{U},\mathscr{F})}$$
	\item 典范的包含映射$\check{C}^{\mathrm{tot},p}(\mathscr{U},\mathscr{F})\subset\check{C}^p(\mathscr{U},\mathscr{F})$是一个链映射,并且它是一个拟同构,即诱导了上同调群之间的同构.于是三种\v{C}ech复形求出的上同调群都是相同的,今后我们都用$\check{H}^p(\mathscr{U},\mathscr{F})$表示该上同调群.
	\begin{proof}
		
		固定指标集$I$上的一个全序.对每个$n\ge0$,定义$K_n(I)$表示以$I^{n+1}$中的元生成的自由阿贝尔群;定义$K_n'(I)$表示以$I^{n+1}$中的满足$i_0<\cdots<i_n$的序$(i_0,\cdots,i_n)$生成的自由阿贝尔群.那么$\partial_n:K_n(I)\to K_{n-1}(I)$,$(i_0,\cdots,i_n)\mapsto\sum_{k=0}^n(-1)^k(i_0,\cdots,\hat{i_k},\cdots,i_n)$同时是$\{K_n(I)\}$和$\{K_n'(I)\}$上的微分算子,这就得到两个复形$K(I)=(K_n(I),\partial_n)$和$K'(I)=(K_n'(I),\partial_n')$.存在包含链映射$i:K'(I)\to K(I)$.下面构造链映射$f=(f_n):K(I)\to K'(I)$为:
		$$f_n(i_0,\cdots,i_n)=\left\{\begin{array}{cc}0&\exists k\not=j,i_k=i_j\\\mathrm{sgn}(\sigma)(i_{\sigma(0)},\cdots,i_{\sigma(n)})&i_{\sigma(0)}<\cdots<i_{\sigma(n)}\end{array}\right.$$
		
		那么$f\circ i=1_{K'(I)}$.另一方面我们断言$i\circ f$链同伦于$1_{K(I)}$的.一旦这个断言成立,我们可以构造$i_n,f_n$为$\check{C}^p(\mathscr{U},\mathscr{F})$和$\check{C}^{\mathrm{tot},p}(\mathscr{U},\mathscr{F})$之间的同态,使得$i,f$是互为拟逆的链映射,这就得证.
		
		\qquad
		
		最后我们证明断言.首先证明$\mathrm{H}_n(K(I))=0,\forall n\ge1$(当然这并不能说明$\mathrm{H}^n(\check{C}^n(\mathscr{U},\mathscr{F}))=0$).这是因为任取$i\in I$,当$n\ge1$时构造$g_n:K_n(I)\to K_{n+1}(I)$为$(i_0,\cdots,i_n)\mapsto(i,i_0,\cdots,i_n)$.此时有$\partial_{n+1}g_n+g_{n-1}\partial_n=1_{K_n(I)}$,这就导致$\mathrm{H}_n(K(I))=0,\forall n\ge1$.
		
		\qquad
		
		下面归纳构造$h_k$.首先$i_0f_0=1_{K_0(I)}$,取$h_0=0$.假设我们构造了$h_k,0\le k<n$,使得$\partial_{k+1}h_k+h_{k-1}\partial_k=i_kf_k-1_{K_k(I)}$对$0\le k<n$成立,并且$h_k(i_0,\cdots,i_k)$是由满足$\{j_0,\cdots,j_{k+1}\}\subset\{i_0,\cdots,i_k\}$的$(j_0,\cdots,j_{k+1})\in I^{k+2}$的线性组合构成.那么任取$(i_0,\cdots,i_n)\in I^{n+1}$,就有$(i_nf_n-1_{K_n}-h_{n-1}\partial_n)(i_0,\cdots,i_n)$是由满足$\{j_0,\cdots,j_n\}\subset\{i_0,\cdots,i_n\}$的$(j_0,\cdots,j_n)$的线性组合,把它视为$K_n(\{i_0,\cdots,i_n\})$中的元,按照:
		\begin{align*}
			&\partial_n(i_nf_n-1_{K_n}-h_{n-1}\partial_n)(i_0,\cdots,i_n)\\=&(i_{n-1}f_{n-1}-1_{K_{n-1}})\partial_n(i_0,\cdots,i_n)-\partial_nh_{n-1}\partial_n(i_0,\cdots,i_n)\\=&(\partial_nh_{n-1}+h_{n-2}\partial_{n-1})\partial_n(i_0,\cdots,i_n)-\partial_nh_{n-1}\partial_n(i_0,\cdots,i_n)\\=&0
		\end{align*}
		
		结合我们证明的$\mathrm{H}_n(K(\{i_0,\cdots,i_n\}))=0$,说明存在$c\in K_{n+1}(\{i_0,\cdots,i_n\})$使得:
		$$\partial_{n+1}(c)=(i_nf_n-1_{K_n}-h_{n-1}\partial_n)(i_0,\cdots,i_n)$$
		
		就定义$h_n(i_0,\cdots,i_n)=c$,这满足条件.
	\end{proof}
\end{enumerate}

一些推论.
\begin{enumerate}
	\item 给定空间$X$的开覆盖$\mathscr{U}$,那么对任意阿贝尔层$\mathscr{F}$都有$\check{H}^0(\mathscr{U},\mathscr{F})\cong\Gamma(X,\mathscr{F})$.
	\begin{proof}
		
		我们有$\check{H}^0(\mathscr{U},\mathscr{F})=\ker(d:C^0(\mathscr{U},\mathscr{F})\to C^1(\mathscr{U},\mathscr{F}))$.任取$\alpha\in C^0$,等价于给定一族$\{\alpha_i\in\mathscr{F}(U_i)\}$,并且对任意$i<j$有$(d\alpha)_{ij}=
		\alpha_j-\alpha_i$,于是$d\alpha=0$说明$\alpha_i,\alpha_j$在$U_i\cap U_j$上限制相同,于是等价于它粘合得到的元素$\alpha'\in\Gamma(X,\mathscr{F})$.于是$\ker d=\Gamma(X,\mathscr{F})$.
	\end{proof}
	\item 取$\mathscr{U}$是由单个全集构成的开覆盖,对任意阿贝尔层$\mathscr{F}$有$\check{H}^q(\mathscr{U},\mathscr{F})=0,\forall q\ge1$.如果开覆盖$\mathscr{U}$由$n$个开集构成,那么$\check{H}^p(\mathscr{U},\mathscr{F})=0,\forall p\ge n$.
	\item 固定一个开覆盖$\mathscr{U}$时\v{C}ech上同调列$\{\check{H}^q(\mathscr{U},-)\}$如果视为$\textbf{Sh}(X)\to\textbf{Ab}$的函子列,它并不是一个同调$\delta$函子,更具体的讲给定阿贝尔层的短正合列不能得到\v{C}ech上同调群的长正合列.例如按照上一条取$\mathscr{U}$是由单个全集构成的开覆盖,但是截面函子只是左正合的,于是长正合列不能成立.但是如果视为$\textbf{pSh}(X)\to\textbf{Ab}$的函子列它的确是一个同调$\delta$函子.
\end{enumerate}

加细和不依赖开覆盖的\v{C}ech上同调.
\begin{enumerate}
	\item 空间$X$上的开覆盖$\mathscr{V}=\{V_j\mid j\in J\}$称为开覆盖$\mathscr{U}=\{U_i\mid i\in I\}$的加细(refinement),如果存在映射$\sigma:J\to I$使得总有$V_j\subset U_{\sigma(j)}$.这个映射称为加细映射(refining map).
	\item 如果两个开覆盖之间存在加细关系,那么这里加细映射未必是唯一的.另外明显的子覆盖是原覆盖的一个加细.
	\item 如果开覆盖$\mathscr{V}$是开覆盖$\mathscr{U}$的加细,选取加细映射$\sigma:J\to I$,那么可以定义同态$\sigma:\check{C}^p(\mathscr{U},\mathscr{F})\to\check{C}^p(\mathscr{V},\mathscr{F})$为$\sigma(a)(j_0,\cdots,j_p)=a(\sigma(j_0),\cdots,\sigma(j_p))\mid V_{j_0,\cdots,j_p}$.这个映射和微分算子可交换,所以它是链映射,它诱导了同调群之间的同态$\sigma^*:\check{H}^p(\mathscr{U},\mathscr{F})\to\check{H}^p(\mathscr{V},\mathscr{F})$.我们断言这个同调同态不依赖于加细映射$\sigma$选取.
	\begin{proof}
		
		如果$\sigma_1:I\to J$也满足$U_i\subset V_{\sigma_1(i)}$.构造$h:\check{C}^n(\mathscr{V},\mathscr{F})\to\check{C}^{n-1}(\mathscr{U},\mathscr{F})$为,如果$s\in\check{C}^n(\mathscr{V},\mathscr{F})$,定义$h(s)(i_0,\cdots,i_{n-1})=\sum_{k=0}^{n-1}(-1)^ks(\sigma(i_0),\cdots,\sigma(i_k),\sigma_1(i_{k+1}),\cdots,\sigma_1(i_k))$.可验证有$h\circ d+d\circ h=\sigma-\sigma_1$,于是$\sigma^*=\sigma^*_1$.
	\end{proof}
	\item 如果开覆盖$\mathscr{U}$是开覆盖$\mathscr{V}$的加细,并且开覆盖$\mathscr{V}$是开覆盖$\mathscr{U}$的加细,那么加细所诱导的同调群之间的同态$\check{H}^p(\mathscr{U},\mathscr{F})\to\check{H}^p(\mathscr{V},\mathscr{F})$是同构.
	\begin{proof}
		
		设$\mathscr{U}=\{U_i\mid i\in I\}$和$\mathscr{V}=\{V_j\mid j\in J\}$.取加细映射$\sigma:I\to J$和加细映射$\tau:J\to I$.那么$\tau\circ\sigma$是$I\to I$的加细映射.我们解释过诱导的同调同态不依赖具体加细映射的选取,所以这个复合映射诱导的同调映射理应与$I\to I$的恒等映射诱导的同调映射相同,于是有$\tau^*\circ\sigma^*=(\tau\circ\sigma)^*=\mathrm{id}$.同理有$\sigma^*\circ\tau^*=\mathrm{id}$.这说明诱导的同调映射都是同构.
	\end{proof}
	\item 考虑空间$X$上的所有开覆盖,两个开覆盖如果互相是加细,就称它们是加细等价的,那么这是一个等价关系.上一条就是说固定$p$和$\mathscr{F}$的时候,$\check{H}^p(\mathscr{U},\mathscr{F})$只依赖于开覆盖$\mathscr{U}$所在的等价类.
	\item 有向类上的正向系统和正向极限的概念:
	\begin{itemize}
		\item 设$\mathscr{K}$是一个类,赋予一个偏序关系$\le$后称它为偏序类.称偏序类是一个有向类,如果对任意$k,k'\in\mathscr{K}$,都存在$k''\in\mathscr{K}$使得$k\le k''$和$k'\le k''$.称子类$\mathscr{L}$在$\mathscr{K}$中共尾(cofinal),如果对任意$k\in\mathscr{K}$,都存在$l\in\mathscr{L}$满足$k\le l$.类似有向集上正向系统的定义,我们可以定义有向类上正向系统.
		\item 设$\mathscr{K}$是有向类,设$\mathscr{C}$是一个余完备范畴,设$\{A_k,\varphi_j^k\}$是以$\mathscr{K}$为指标类的$\mathscr{C}$中的正向系统.如果$\mathscr{L}$和$\mathscr{M}$都是在$\mathscr{K}$中共尾的子集,那么有$\lim\limits_{\substack{\rightarrow\\\mathscr{L}}}A_k\cong\lim\limits_{\substack{\rightarrow\\\mathscr{M}}}A_k$.
		\begin{proof}
			
			这个命题对于指标类构成集合的时候是成立的,这里为了证明$\mathscr{L}$和$\mathscr{M}$得到同构的正向极限,只要把有向类替换为集合$\mathscr{L}\cup\mathscr{M}$即可.
		\end{proof}
	\end{itemize}
	\item 设$\mathscr{F}$是空间上的阿贝尔层,当$\mathscr{U}$跑遍所有开覆盖时$\check{H}^p(\mathscr{U},\mathscr{F})$构成一个真类$\mathscr{K}$.在$\mathscr{K}$定义一个偏序为,如果$X,Y\in\mathscr{K}$,$X\le Y$定义为,存在开覆盖$\mathscr{V}$加细了开覆盖$\mathscr{U}$,使得$X=\check{H}^p(\mathscr{U},\mathscr{F})$和$Y=\check{H}^p(\mathscr{V},\mathscr{F})$.它有如下共尾子集:
	\begin{itemize}
		\item 全体没有重复开集的开覆盖构成了$\mathscr{K}$的共尾的子集,记作$\mathscr{H}$.这里共尾是因为任取一个开覆盖$\mathscr{U}$,删去它所有重复的多余的开集,得到的没有重复开集的开覆盖记作$\mathscr{U'}$,那么明显的有$\mathscr{U}$和$\mathscr{U}'$是加细等价的.而$\mathscr{H}$是集合是因为它是是$X$上幂集的幂集的子集.
		\item $\mathscr{K}$的所有加细等价类中任取一个代表元构成的也是$\mathscr{K}$的共尾子集.
		\item 如果空间本身是拟紧的,那么所有有限开覆盖构成$\mathscr{K}$的共尾子集.
	\end{itemize}
	
	我们之前解释过有向类共尾子集$\mathscr{M}$的正向极限在同构意义下唯一,就定义这个正向极限为空间$X$的关于阿贝尔层$\mathscr{F}$的(不依赖开覆盖的)\v{C}ech上同调群:
	$$\check{H}^p(X,\mathscr{F})=\lim\limits_{\substack{\rightarrow\\\mathscr{M}}}\check{H}^p(\mathscr{U},\mathscr{F})$$
\end{enumerate}

层上同调和\v{C}ech上同调的联系.
\begin{enumerate}
	\item 我们解释过对任意开覆盖$\mathscr{U}$总有$\check{H}^0(\mathscr{U},\mathscr{F})=\Gamma(X,\mathscr{F})$.左侧中的元素是指对开覆盖的每个开集$U_i$,取$s_i\in\mathscr{F}(U_i)$,满足$s_i\mid_{U_i\cap U_j}=s_j\mid_{U_i\cap U_j}$.这等同于粘合得到的整体截面.如果有开覆盖的加细$\mathscr{U}\le\mathscr{V}$,加细映射诱导的上同调同态就把$s=\{s_i\in\mathscr{F}(U_i)\}$映射为$t=\{t_j=s_{\alpha(j)}\mid_{V_j}\in\mathscr{F}(V_{\alpha(j)})\}$,但是$t$和$s$粘合是相同的整体截面,这说明用来定义$\check{H}^0(X,\mathscr{F})$的正向系统中的映射都是$\Gamma(X,\mathscr{F})$上的恒等态射,于是有$\check{H}^0(X,\mathscr{F})=\Gamma(X,\mathscr{F})$.
	\item 引理.取$X$的开覆盖$\mathscr{U}$,取$X$上的阿贝尔层$\mathscr{F}$,构造$\mathscr{C}^p(\mathscr{U},\mathscr{F})=\prod_{i_0<i_1<\cdots<i_p}f_*(\mathscr{F}\mid U_{i_0,i_1,\cdots,i_p})$.这里映射$f$是随分量变动的,$f_*(\mathscr{F}\mid U_{i_0,i_1,\cdots,i_p})$是指包含映射$f:U_{i_0,i_1,\cdots,i_p}\to X$的前推函子,换句话讲对每个$X$的开子集$V$,有$f_*(\mathscr{F}\mid U_{i_0,i_1,\cdots,i_p})(V)=\mathscr{F}(V\cap U_{i_0,i_1,\cdots,i_p})$.限制映射和\v{C}ech上同调的定义相同.我们断言存在如下正合列:
	$$\xymatrix{0\ar[r]&\mathscr{F}\ar[r]^{\varepsilon\qquad}&\mathscr{C}^0(\mathscr{U},\mathscr{F})\ar[r]^d&\mathscr{C}^1(\mathscr{U},\mathscr{F})\ar[r]&\cdots}$$
	\begin{proof}
		
		任取开集$U$,任取开覆盖$\mathscr{U}$的指标$i\in I$,限制映射$\mathscr{F}\to f_*(\mathscr{F}\mid U_i),i\in I$即$\mathscr{F}(U)\to\mathscr{F}(U\cap U_{i_0,i_1,\cdots,i_p})$诱导了层同态$\varepsilon:\mathscr{F}\to\mathscr{C}^0$.层公理保证这是单态射.
		
		为了证明其他位置也都是正合的,对每个$p\ge1$和$x\in X$,我们来构造同伦映射$k:\mathscr{C}^p(\mathscr{U},\mathscr{F})_x\to\mathscr{C}^{p-1}(\mathscr{U},\mathscr{F})_x$.满足$d\circ k+k\circ d=\mathrm{id}_{\mathscr{C}^p_x}$.一旦这得证,说明这个正合列在取$x\in X$处的茎后得到的复形和自身是零伦的,于是必然有同调群处处为零,也即它是正合的.
		
		假设$x\in U_j$,任取$\alpha_x\in\mathscr{C}^p(\mathscr{U},\mathscr{F})_x$,它可以表示为$x$的某个开邻域$V$上的层$\mathscr{C}^p(\mathscr{U},\mathscr{F})$的截面$\alpha$.另外我们可以不妨设$V\subset U_j$.对每个$i_0<i_1<\cdots<i_{p-1}$,定义$(k\alpha)_{i_0,i_1,\cdots,i_{p-1}}=\alpha_{j,i_0,\cdots,i_{p-1}}$.这里我们约定对一个双射$\sigma$有$\alpha_{\sigma i_0,\sigma i_1,\cdots,\sigma i_p}=(-1)^{\mathrm{sgn}\sigma}\alpha_{i_0,i_1,\cdots,i_p}$.这个定义有意义因为$V\cap U_{i_0,i_1,\cdots,i_{p-1}}=V\cap U_{j,i_0,i_1,\cdots,i_{p-1}}$.最后容易验证对任意$p\ge1$和任意$\alpha\in\mathscr{C}_x^p$有$(d\circ k+k\circ d)(\alpha)=\alpha$.
	\end{proof}
	\item 设$\mathscr{U}$是$X$的开覆盖,设$\mathscr{F}$是$X$上的松弛层,那么对每个$p\ge1$有$\check{H}^p(\mathscr{U},\mathscr{F})=0$.于是对松弛层(特别的,对所有内射层)$\mathscr{F}$就有$\check{H}^p(X,\mathscr{F})=0,\forall p\ge1$.
	\begin{proof}
		
		首先如果$\mathscr{F}$是松弛层,那么每个$\mathscr{F}\mid U_{i_0,i_1,\cdots,i_p}$都是松弛层,松弛层的前推都是松弛层,松弛层的直积是松弛层,于是每个$\mathscr{C}^p(\mathscr{U},\mathscr{F})$是松弛层.于是上一条中的正合列是$\mathscr{F}$的松弛预解,可以用它计算标准层上同调.一方面按照松弛层的标准上同调总有$H^q=0,q\ge1$,另一方面按照$\Gamma(X,\mathscr{C}^q(\mathscr{U},\mathscr{F}))=C^q(\mathscr{U},\mathscr{F})$,得到它的$q\ge1$次同调群也是$\check{H}^q(\mathscr{U},\mathscr{F})$,于是这是零.
	\end{proof}
	\item 无论考虑阿贝尔层范畴还是模层范畴,\v{C}ech上同调的确是一个函子.但是它一般不是同调$\delta$函子,换句话讲短正合列未必总诱导出长正合列.另外我们之前解释过一个左正合函子的右导出函子列$\{H^p,p\ge1\}$被如下两个性质在自然同构意义下唯一决定:它们都在内射对象上取零,它们是同调$\delta$函子.于是空间$X$上的\v{C}ech上同调和层上同调一致当且仅当\v{C}ech上同调是一个同调$\delta$函子.例如如果$X$是仿紧空间,那么这两个互相等价的条件都是成立的.
	\item Leray定理.设$\mathscr{F}$是$X$上的阿贝尔层,设$\mathscr{U}$是一个开覆盖,如果$Y$是这个开覆盖的有限个开集的交,总有$H^p(Y,\mathscr{F}\mid_Y)=0,\forall p\ge1$.那么对任意$p\ge0$就有典范的同构:$$\check{H}^p(\mathscr{U},\mathscr{F})\cong\mathrm{H}^p(X,\mathscr{F})$$
	\begin{proof}
		
		取$\mathscr{F}$在阿贝尔层范畴中的内射预解$I^*$,考虑双复形$(K^{p,q})=(\check{C}^p(\mathscr{U},I^q))$.我们有$\mathrm{H}^q(K^{p,q})=\prod\mathrm{H}^q(U_{i_0,\cdots,i_p},\mathscr{F})=\check{C}^p(\mathscr{U},\mathrm{H}^q(-,\mathscr{F}))$,进而有$\mathrm{H}^p\mathrm{H}^q(K^{p,q})=\check{H}^p(\mathscr{U},\mathrm{H}^q(-,\mathscr{F}))$.并且按照条件当$p\ge1$时恒为零,于是这个双复形的第一谱序列塌陷在$q$轴.另一方面按照$I^q$是内射层,有$\mathrm{H}^p(K^{p,q})=0,\forall q\ge1$,进而得到:
		$$\mathrm{H}^p\mathrm{H}^q(K^{p,q})=\left\{\begin{array}{cc}0&q\ge1\\\mathrm{H}^p(X,\mathscr{F})&q=0\end{array}\right.$$
		
		于是第二谱序列塌陷在$p$轴,这说明有$\check{H}^p(\mathscr{U},\mathscr{F})=\mathrm{H}^p(K^{-,-})=\mathrm{H}^p(X,\mathscr{F})$.
	\end{proof}
	\item 引理.如果$\mathscr{F}$是$X$上的阿贝尔层,对每个$q\ge1$就有$\check{H}^0(X,\mathrm{H}^q(-,\mathscr{F}))=0$.注意$U\mapsto\mathrm{H}^q(U,\mathscr{F})$是一个预层,所以未必有$\check{H}^0(X,\mathrm{H}^q(-,\mathscr{F}))=\mathrm{H}^q(X,\mathscr{F})$.
	\begin{proof}
		
		首先任取$X$的开覆盖$\mathscr{U}=\{U_i\mid i\in I\}$.那么$\check{H}^0(\mathscr{U},\mathrm{H}^q(-,\mathscr{F}))$中的元就是一族$(s_i\in\mathrm{H}^q(U_i,\mathscr{F}))$,使得$s_i\mid_{U_i\cap U_j}=s_j\mid_{U_i\cap U_j}$.取$\mathscr{F}$的内射预解$I^*$,那么对$q\ge1$就有:
		$$\mathrm{H}^q(U_i\mathscr{F})=\ker(I^q(U_i)\to I^{q+1}(U_i))/\mathrm{im}(I^{q-1}(U_i)\to I^q(U_i))$$
		
		取$t_i\in\ker(I^q(U_i)\to I^{q+1}(U_i))$使得它在商中的像为$s_i$.按照$I^{q-1}\to I^q\to I^{q+1}$是正合的,可取$U_i$的开覆盖$\{U_{ij}\mid j\in J_i\}$,使得每个$t_i\mid_{U_{ij}}$落在$\mathrm{im}(I^{q-1}(U_{ij})\to I^q(U_{ij}))$.那么$s_i$在$\mathrm{H}^q(U_i,\mathscr{F})\to\mathrm{H}^q(U_{ij},\mathscr{F})$的像是零.当指标取遍$i\in I,j\in J_i$时全体$U_{ij}$构成的开覆盖$\mathscr{U}'$是$\mathscr{U}$的开加细,并且$s=(s_i)$在加细诱导的上同调同态$\check{H}^0(\mathscr{U},\mathrm{H}^q(-,\mathscr{F}))\to\check{H}^0(\mathscr{U}',\mathrm{H}^q(-,\mathscr{F}))$是零.于是按照正向极限的性质,在$q\ge1$时就有:
		$$\check{H}^0(X,\mathrm{H}^q(-,\mathscr{F}))=0$$
	\end{proof}
	\item 给定$X$上的阿贝尔层$\mathscr{F}$,存在典范的同构$\check{H}^1(X,\mathscr{F})\cong\mathrm{H}^1(X,\mathscr{F})$.另外尽管\v{C}ech上同调不能诱导长正合列,但是任给定阿贝尔层或者模层的短正合列$0\to\mathscr{F}'\to\mathscr{F}\to\mathscr{F}''\to0$总能诱导出如下六项的正合列:
	$$\xymatrix{0\ar[r]&\check{H}^0(X,\mathscr{F}')\ar[r]&\check{H}^0(X,\mathscr{F})\ar[r]&\check{H}^0(X,\mathscr{F}'')\ar[dll]\\&\check{H}^1(X,\mathscr{F}')\ar[r]&\check{H}^1(X,\mathscr{F})\ar[r]&\check{H}^1(X,\mathscr{F}'')}$$
	\begin{proof}
		
		第一件事是因为谱序列$E_2^{p,q}=\check{H}^p(\mathscr{U},\mathrm{H}^q(-,\mathscr{F}))$收敛到$\mathrm{H}^{p+q}(X,\mathscr{F})$,取正向极限得到谱序列$E_2^{p,q}=\check{H}^p(X,\mathrm{H}^q(-,\mathscr{F}))$收敛到$\mathrm{H}^{p+q}(X,\mathscr{F})$.这里这里$E_2^{p,q}$是第一象限的,所以有如下正合列:
		$$\xymatrix{0\ar[r]&E_2^{1,0}\ar[r]&\mathrm{H}^1\ar[r]&E_2^{0,1}\ar[r]&E_2^{2,0}\ar[r]&\mathrm{H}^2}$$
		
		但是上一条引理说明$E_2^{0,1}=0$,导致$\check{H}^1(X,\mathscr{F})=E_2^{1,0}\cong\mathrm{H}^1(X,\mathscr{F})$.
	\end{proof}
	\item Cartan定理.设$\mathscr{F}$是$X$上的阿贝尔层,设开覆盖$\mathscr{B}$构成$X$上一个拓扑基,满足如果$U$是$\mathscr{B}$中有限个开集的交,有$\check{H}^p(U,\mathscr{F})=0,\forall p\ge1$成立.那么对任意这样的$U$和$p\ge1$就总有$\mathrm{H}^p(U,\mathscr{F})=0$.特别的,按照Leray定理说明对任意开集$V$和任意的由$\mathscr{B}$中开集构成的$V$的开覆盖$\mathscr{U}$,总有$\check{H}^n(\mathscr{U},\mathscr{F})=\mathrm{H}^n(V,\mathscr{F}),\forall n\ge0$成立.特别的这说明总有典范同构:
	$$\check{H}^n(V,\mathscr{F})\cong\mathrm{H}^n(V,\mathscr{F})$$
	\begin{proof}
		
		记$U=U_1\cap\cdots\cap U_k$是$\mathscr{B}$中有限个开集的交,我们来对$q\ge1$归纳证明$\mathrm{H}^q(U,\mathscr{F})=0$.首先起始步有$\mathrm{H}^1(U,\mathscr{F})\cong\check{H}^1(U,\mathscr{F})=0$.假设命题对$0<q<n$成立.任取由$\mathscr{B}$中开集构成的$U$的开覆盖$\mathscr{U}$,任取$0<q<n$,任取$p$,总有$\check{C}^p(\mathscr{U},\mathrm{H}^q(-,\mathscr{F}))=0$.于是只要$0<q<n$就有$\check{H}^p(\mathscr{U},\mathrm{H}^q(-,\mathscr{F}))=0$,于是只要$0<q<n$就有$\check{H}^p(U,\mathrm{H}^q(-,\mathscr{F}))=0$.我们知道有谱序列$E_2^{p,q}=\check{H}^p(U,\mathrm{H}^q(-,\mathscr{F}))$收敛到$\mathrm{H}^{p+q}(U,\mathscr{F})$.从这个谱序列是第一象限的并且有$E_2^{p,q}=0,0<q<n$,就得到正合列$\xymatrix{0\ar[r]&E_2^{n,0}\ar[r]&\mathrm{H}^n\ar[r]&E_2^{0,n}}$.但是这里$E_2^{n,0}=\check{H}^n(U,\mathscr{F})$,$\mathrm{H}^n=\mathrm{H}^n(U,\mathscr{F})$,$E_2^{0,n}=0$.于是有$\mathrm{H}^n(U,\mathscr{F})\cong\check{H}^n(U,\mathscr{F})=0$.
	\end{proof}
\end{enumerate}
\subsubsection{非阿贝尔上同调}

挠子.
\begin{enumerate}
	\item 设$X$是拓扑空间,设$G$是$X$上的群层(未必交换),设$T$是$X$上的集合层,我们称$G$作用在$T$上,或者称$T$是一个$G$层,如果存在层态射$G\times T\to T$,使得对任意开子集$U\subseteq X$,有$G(U)\times T(U)\to T(U)$是群$G(U)$在集合$T(U)$上的一个左作用.如果$T$和$T'$是两个$G$层,一个层态射$\varphi:T\to T'$称为$G$层态射,如果对任意开集$U\subseteq X$有$\varphi(U):T(U)\to T'(U)$是一个$G(U)$同态,换句话讲它是一个层态射,并且如下图表交换:
	$$\xymatrix{G(U)\times T(U)\ar[rr]^{1\times\varphi(U)}\ar[d]&&G(U)\times T'(U)\ar[d]\\T(U)\ar[rr]^{\varphi(U)}&&T'(U)}$$
	
	于是全体$G$层和全体$G$层态射构成一个范畴,称为$G$层范畴.
	\item 一个$G$层$T$称为$G$挠子($G$-torsor),如果它满足如下两个条件:
	\begin{enumerate}
		\item 对任意开集$U\subseteq X$,有$G(U)$在$T(U)$上的群作用是单可迁的.(一般的如果群$G$作用在集合$S$上,我们称这个群作用是单可迁的(simply transitively),如果对任意$s_1,s_2\in S$,恰好存在唯一一个$g\in G$使得$gs_1=s_2$).
		\item 存在$X$的开覆盖$\{U_i\mid i\in I\}$使得$T(U_i)\not=\emptyset,\forall i$.
	\end{enumerate}
	\item 考虑群层$G$左平移作用在自身上,那么这是一个$G$挠子,称为平凡$G$挠子.一个$G$挠子$T$同构于平凡$G$挠子当且仅当$T(X)\not=\emptyset$.这是因为单可迁条件保证了只要存在$t\in T(X)$,那么$G(U)\to T(U)$为$g\mapsto gt\mid_U$总是一个双射.
	\item 全体$G$挠子的同构类记作$\mathrm{H}^1(X,G)$,这是一个带基点的集合,基点取为平凡$G$挠子所在的同构类.
\end{enumerate}

非阿贝尔的\v{C}ech上同调.设$X$是拓扑空间,设$G$是一个群层.
\begin{enumerate}
	\item 1-余圈.给定$X$的开覆盖$\mathscr{U}=\{U_i\mid i\in I\}$,称关于$\mathscr{U}$的一个$G$的1-余圈为$\theta=\{g_{ij}\in G(U_i\cap U_j)\mid i,j\in I\}$,满足如下余圈条件$g_{kj}g_{ji}=g_{ki}$(这里我们做如下约定,如果$s\in\Gamma(U,G)$和$t\in\Gamma(V,G)$,记$st$表示$s\mid_{U\cap V}t\mid_{U\cap V}\in\Gamma(U\cap V,G)$,记$s=t$表示$s\mid_{U\cap V}=t\mid_{U\cap V}$),并且约定$g_{ii}=1$,那么有$g_{ij}=g_{ji}^{-1},\forall i,j\in I$.
	\item 设$\theta=(g_{ij}),\theta'=(g_{ij}')$是两个关于$\mathscr{U}$的1-余圈,称它们在相同的上同调类,如果存在$h_i\in G(U_i),\forall i\in I$,使得$h_ig_{ij}=g_{ij}'h_j,\forall i,j\in I$成立.这是一个等价关系,等价类称为上同调类,全体上同调类构成的集合称为$G$关于$\mathscr{U}$的一阶\v{C}ech上同调,记作$\check{\mathrm{H}}^1(\mathscr{U},G)$.我们约定它是一个带基点集合,基点取为1-余圈$g_{ij}\equiv1,\forall i,j$所在的上同调类.
	\item 给定$X$的另一个开覆盖$\mathscr{V}=\{V_j\mid j\in J\}$,称$\mathscr{V}$是$\mathscr{U}$的加细(refinement),如果对任意$j\in J$,存在$i\in I$使得$V_j\subseteq U_i$,这等价于约定存在所谓的加细映射$\tau:J\to I$,满足$V_j\subseteq U_{\tau(j)}$.如果$\theta=(g_{ii'})$是关于$\mathscr{U}$的1-余圈,那么$\tau^*(\theta)_{jj'}=g_{\tau(j)\tau(j')}\mid_{V_j\cap V_j'}$是关于$\mathscr{V}$的1-余圈,并且这把在相同上同调类的1-余圈映射为在相同上同调类的1-余圈.于是加细诱导了如下保基点的映射:
	$$\tau^*:\check{\mathrm{H}}^1(\mathscr{U},G)\to\check{\mathrm{H}}^1(\mathscr{V},G)$$
	\item 考虑全体开覆盖上的加细偏序,我们期望可以取$\check{\mathrm{H}}^1(\mathscr{U},G)$的正向极限,但是遗憾的是一个空间上的全体开覆盖未必构成集合.解决这件事的方法是定义两个开覆盖$\mathscr{U}$和$\mathscr{V}$等价,如果它们互相是对方的加细,此时就有$\check{\mathrm{H}}^1(\mathscr{U},G)\cong\check{\mathrm{H}}^1(\mathscr{V},G)$.考虑开覆盖在加细下的全体等价类,这的确构成一个集合,对$\mathscr{U}$取正向极限定义为$G$在$X$上的一阶\v{C}ech上同调:
	$$\check{\mathrm{H}}^1(X,G)=\lim\limits_{\rightarrow}\check{\mathrm{H}}^1(\mathscr{U},G)$$
	
	特别的,如果$G$是阿贝尔层,那么这个定义吻合于阿贝尔层的\v{C}ech上同调.
	\item 设$T$是$G$挠子,设$\mathscr{U}=\{U_i\mid i\in I\}$是$G$挠子定义中的开覆盖,换句话讲总有$T(U_i)\not=\emptyset,\forall i\in I$.记$U_{ij}=U_i\cap U_j,\forall i,j\in I$.取$t_i\in T(U_i)$,由于$G$作用是可迁的,说明存在唯一的$g_{ij}\in G(U_{ij})$使得$g_{ij}t_j=t_i$.于是有$g_{kj}g_{ji}t_i=t_k=g_{ji}t_i$,于是有$g_{kj}g_{ji}=g_{ki}$,于是$\{g_{ij}\}$构成一个1-余圈,并且任意另一个由此得到的1-余圈和$\{g_{ij}\}$在相同的上同调类.换句话讲,如果记$\mathrm{H}^1(\mathscr{U},G)$为$T\in\mathrm{H}^1(X,G)$使得$T$在$\mathscr{U}$上处处非空,那么我们构造了如下保基点的集合映射:
	$$c_{G,\mathscr{U}}:\mathrm{H}^1(\mathscr{U},G)\to\check{\mathrm{H}}^1(\mathscr{U},G)$$
	
	我们断言这个映射是双射,进而取开覆盖等价类在加细下构成的正向极限就得到保基点的双射:
	$$c_G:\mathrm{H}^1(X,G)\to\check{\mathrm{H}}^1(X,G)$$
	\begin{proof}
		
		我们来构造$c_{G,\mathscr{U}}$的逆映射.给定1-余圈$(g_{ij})$,对开集$V\subseteq X$,我们定义$T(V)=\{(t_i)\in\prod_iG(U_i\cap V)\mid t_it_j^{-1}=g_{ij}\}$.限制映射定义为$(t_i)$在更小开集上的限制,这使得$T$是$X$上的层,那么$T$明显在$\mathscr{U}$上非空(否则取不到$g_{ij}$).我们定义$T$的$G$层结构为$g(t_i)_i=(t_ig^{-1})_i$,那么任取$k\in I$,任取开集$V\subseteq U_k$,有$G(V)\to T(V)$为$g\mapsto(g_{ik}g^{-1})$是$G\mid_{U_k}$层$G\mid_{U_k}\to T\mid_{U_k}$的同构,因为它的逆映射是$(t_i)_i\mapsto t_k^{-1}$.于是$T$是一个$G$挠子.倘若我们把$(g_{ij})$替换为在相同上同调类中的元$(h_ig_{ij}h_j^{-1})$,设它对应的$G$挠子为$T'$,那么$(t_i)_i\mapsto(h_it_i)_i$是$T\cong T'$的$G$层同构.于是我们的确定义了$\check{\mathrm{H}}^1(\mathscr{U},G)\to\mathrm{H}^1(\mathscr{U},G)$的映射.最后验证它们互为逆映射即可.
	\end{proof}
	\item 设$\varphi:G\to G'$是群层之间的态射.如果$(g_{ij})$是$G$上关于$\mathscr{U}$的1-余圈,那么$(\varphi(g_{ij}))$就是$G'$上的关于$\mathscr{U}$的1-余圈,于是这诱导了如下保基点的集合映射:
	$$\check{\mathrm{H}}^1(\varphi):\check{\mathrm{H}}^1(X,G)\to\check{\mathrm{H}}^1(X,G')$$
	\item 设$1\to G'\to G\to G''\to1$是$X$上群层的正合列,我们断言有如下带基点集合构成的正合列,对于带基点集合之间的映射,它的核定义为基点的原像,正合性的定义依旧是前一个映射的像是后一个映射的核.
	$$\xymatrix{1\ar[r]&G'(X)\ar[r]&G(X)\ar[r]&G''(X)\ar[r]^{\delta}&\check{\mathrm{H}}^1(X,G')\ar[r]&\check{\mathrm{H}}^1(X,G)\ar[r]&\check{\mathrm{H}}^1(X,G'')}$$
	
	这里连接映射$\delta:G''(X)\to\check{\mathrm{H}}^1(X,G')$是这样定义的:任取$g''\in G''(X)$,因为$G\to G''$是满态射,我们可以取到$X$的一个开覆盖$\mathscr{U}=\{U_i\mid i\in I\}$,使得存在$g_i\in G(U_i)$,满足$g_i$在$G''(U_i)$中的像是$g''\mid_{U_i}$.对任意$i,j\in I$,取$g_{ij}'\in G'(U_{ij})$是唯一的元素使得它在$G(U_{ij})$中的像是$g_ig_j^{-1}$.那么$(g_{ij}')$构成了$\mathscr{U}$上的1-余圈.如果一开始选取不同的$g_i\in G(U_i)$,得到的结果仍在相同的上同调类中.我们就定义$\delta(g'')$是这个上同调类.并且它把$0\in G''(X)$映射为$\check{\mathrm{H}}^1(X,G')$的基点.
\end{enumerate}
\newpage
\section{群上同调}
\subsection{基本构造}
\subsubsection{群上同调}
\begin{enumerate}
	\item $G$模.设$G$是一个群,一个$G$模是指一个交换群$A$,赋予了群作用$G\to\mathrm{Aut}(A)$.这等价于讲$A$是(未必交换的环)$\mathbb{Z}[G]$上的模.两个$G$模之间的同态是指一个交换群同态$f:A\to B$,使得$f(ga)=gf(a),\forall g\in G,a\in A$成立.$G$模范畴即$\mathbb{Z}[G]$模范畴,其上总具有足够多的投射对象和内射对象.
	\item 构造$\textbf{G-Mod}\to\textbf{G-Mod}$的函子$\mathscr{F}$为(或者可视为终端为$\textbf{Ab}$的函子,没有本质区别),把$G$模$A$映射为它的子模$A^G=\{a\in A\mid ga=a,\forall g\in G\}$,此即$A$的被$G$固定的极大子模,把$G$模同态$\phi:A\to B$映射为它在$A^G$上的限制$A^G\to B^G$.
	\item 把$\mathbb{Z}$视为平凡$G$模,存在自然同构$\mathscr{F}(-)\cong\mathrm{Hom}_G(\mathbb{Z},-)$.于是特别的$\mathscr{F}$是左正合函子.它的右导出函子列称为群上同调.
	\begin{proof}
		
		对$G$模$A$,构造$\tau_A:\mathrm{Hom}_G(\mathbb{Z},A)\to A^G$为$f\mapsto f(1)$,这里$f(1)\in A^G$是因为$\forall g\in G$有$gf(1)=f(g1)=f(1)$.这是单射因为从$f(1)=e$得到$f$是恒取$e$的映射.这是满射因为任取$a\in A^G$,可直接构造$G$模同态$\mathbb{Z}\to A$为$n\mapsto na$,它满足$f(1)=a$.任取$G$模同态$\phi:A\to B$,有如下交换图表,于是这是自然同构.
		$$\xymatrix{\mathrm{Hom}_G(\mathbb{Z},A)\ar[rr]^{\tau_A}\ar[d]_{\phi_*}&&A^G\ar[d]^{\phi^G}\\\mathrm{Hom}_G(\mathbb{Z},B)\ar[rr]&&B^G}$$
	\end{proof}
	\item 标准预解.取$P_n=\mathbb{Z}[G^{n+1}]\cong\mathbb{Z}[G]^{\oplus(n+1)}$,它的$G$模结构定义为$g(g_0,g_1,\cdots,g_n)=(gg_0,gg_1,\cdots,gg_n)$,于是$P_n$是$\mathbb{Z}[G]$模范畴中的自由对象.取连接同态$\partial_n:P_n\to P_{n-1}$为$$(g_0,g_1,\cdots,g_n)\mapsto\sum_{0\le i\le n}(-1)^i(g_0,\cdots,\hat{g_i},\cdots,g_n)$$再取$\varepsilon:P_0\to\mathbb{Z}$为把$\sum n_gg$映射为$\sum n_g$.这得到如下复形,这是正合的,于是它是$G$模$\mathbb{Z}$的自由预解.
	$$\xymatrix{\cdots\ar[r]&P_2\ar[r]^{\partial_2}&P_1\ar[r]^{\partial_1}&P_0\ar[r]^{\varepsilon}&\mathbb{Z}\ar[r]&0}$$
	\item 借助这个预解,对$G$模$M$,计算它的群上同调可以通过如下复形,其中$\mathrm{Hom}_G(P_n,M)$就是全体满足齐次条件$g\alpha(g_0,\cdots,g_n)=\alpha(gg_0,\cdots,gg_n)$的映射$\alpha:G^{n+1}\to M$构成的集合.这里$\partial_n^*$为把带齐次条件的映射$\alpha:G^n\to M$映射为带齐次条件的映射$(g_0,g_1,\cdots,g_{n+1})\mapsto\sum_{i=0}^{n+1}(-1)^i\alpha(g_0,\cdots,\widehat{g_i},\cdots,g_{n+1})$.
	$$\xymatrix{0\ar[r]&\mathrm{Hom}_G(P_0,M)\ar[r]^{\partial_1^*}&\mathrm{Hom}_G(P_1,M)\ar[r]^{\partial_2^*}\ar[r]&\mathrm{Hom}_G(P_2,M)\ar[r]&\cdots}$$
	\item Bar预解.对$G$模$M$,记$C^n(G,M)$为全体$G^n\to M$的映射构成的集合,那么有$\mathrm{Hom}_G(P_n,M)\to C^n(G,M)$的双射为$\alpha(g_0,\cdots,g_n)\mapsto\beta(g_1,g_2,\cdots,g_n)=\alpha(1,g_1,g_1g_2,\cdots,g_1g_2\cdots g_n)$,它的逆映射是$\beta(g_1,\cdots,g_n)\mapsto\alpha(g_0,g_1,\cdots,g_n)=g_0\beta(g_1/g_0,g_2/g_1,\cdots,g_n/g_{n-1})$.使得如下图表交换的唯一边界算子$d^n$是:
	\begin{align*}
		(d_n\beta)(g_1,g_2,\cdots,g_{n+1})&=g_1\beta(g_2,\cdots,g_{n+1})\\&+\sum_{j=1}^n(-1)^j\beta(g_1,\cdots,g_{j-1},g_jg_{j+1},g_{j+2},\cdots,g_{n+1})\\&+(-1)^{n+1}\alpha(g_1,\cdots,g_n)
	\end{align*}
	$$\xymatrix{\mathrm{Hom}_G(P_n,M)\ar[rr]^{\partial_{n+1}^*}\ar[d]_{\cong}&&\mathrm{Hom}_G(P_{n+1},M)\ar[d]^{\cong}\\C^n(G,M)\ar[rr]_{d^n}&&C^{n+1}(G,M)}$$
	
	换句话讲,我们可以定义$\mathbb{Z}$的所谓的Bar预解:记$B_0$表示由单个符号$[\qquad]$生成的自由$G$模.取$\varepsilon:B_0\to\mathbb{Z}$为$\sum n_gg\mapsto\sum n_g$.记$B_n,n\ge1$是由全体符号$[g_1\mid g_2\mid\cdots\mid g_n],g_i\in G$生成的自由$G$模(于是$B_n$可视为$\mathbb{Z}[G^{n+1}]$).定义映射$d_n:B_n\to B_{n-1}$为:
	\begin{align*}
		d_n:[g_1\mid g_2\mid\cdots\mid g_n]&\mapsto g_1[g_2\mid\cdots\mid g_n]\\&+\sum_{1\le i\le n-1}(-1)^i[g_1\mid\cdots\mid g_{i-1}\mid g_ig_{i+1}\mid\cdots\mid g_n]\\&+(-1)^n[g_1\mid\cdots\mid g_{n-1}]
	\end{align*}
	
	此时如下复形是正合列,于是它是$G$模$\mathbb{Z}$的自由预解.
	$$\xymatrix{\cdots\ar[r]&B_2\ar[r]^{d_2}&B_1\ar[r]^{d_1}&P_0\ar[r]^{\varepsilon}&\mathbb{Z}\ar[r]&0}$$
	
	标准预解和Bar预解是同构的.链同构可直接构造为$\tau_n:(g_0,g_1,\cdots,g_n)\mapsto g_0[g_0^{-1}g_1\mid g_1^{-1}g_2\mid\cdots\mid g_{n-1}^{-1}g_n]$和$\sigma_n:[g_1\mid\cdots\mid g_n]\mapsto(1,g_1,g_1g_2,\cdots,g_1g_2\cdots g_n)$.
	\item 借助Bar预解我们可以计算一些低阶的上同调群.
	\begin{itemize}
		\item 0余圈$C^0(G,M)\cong M$.这里$C^0(G,M)$就是全体从$\{e\}$到$M$的映射,也即典范同构于$M$本身.
		\item 1余边界$\mathrm{im}d^0=\{ga-a\}$.任取$a\in M$,那么$d^0a$是$G\to M$的映射$g\mapsto ga-a$.
		\item 1余圈$\ker d^1=\{\varphi:G\to M\mid\varphi(g_1g_2)=g_1\varphi(g_2)+\varphi(g_2),\forall g_1,g_2\in G\}$,其中的元素也称为crossed同态.
		\item 2余边界$\mathrm{im}d^1=\{\varphi:G\times G\to M\mid\varphi(g_1,g_2)=g_1\beta(g_2)-\beta(g_1g_2)+\beta(g_1),\beta:G\to M\}$.
		\item 2余圈$\ker d^2=\{\varphi:G\times G\to M\mid g_1\varphi(g_2,g_3)-\varphi(g_1g_2,g_3)+\varphi(g_1,g_2g_3)-f(g_1,g_2)\}$.
		\item $\mathrm{H}^0(G,M)=M^G$.
		\item $\mathrm{H}^1(G,M)=\{\varphi:G\to M\mid\varphi(g_1g_2)=g_1\varphi(g_2)+\varphi(g_1)\}/\{ga-a\mid a\in M\}$
	\end{itemize}
	\item 按照预解的构造,如果$G$是有限群,$G$模$A$是有限集合,那么所有$\mathrm{H}^q(G,A)$都是有限群.
	\item 有限循环群的上同调.
	\begin{enumerate}[(1)]
		\item 设$G$是$k$阶有限循环群,设生成元为$x$,记$D=x-1,N=1+x+\cdots+x^{k-1}$是$\mathbb{Z}[G]$中的两个元.那么有如下$G$模$\mathbb{Z}$的自由预解:
		$$\xymatrix{\cdots\ar[r]&\mathbb{Z}[G]\ar[r]^{\times D}&\mathbb{Z}[G]\ar[r]^{\times N}&\mathbb{Z}[G]\ar[r]^{\times D}&\mathbb{Z}[G]\ar[r]^{\varepsilon}&\mathbb{Z}\ar[r]&0}$$
		\begin{proof}
			
			首先$G$是有限循环群时$\mathbb{Z}[G]$是交换环,左乘和右乘是相同的映射.按照$\mathbb{Z}[G]$中有$ND=DN=0$,于是这是一个复形.这里$\varepsilon$是满射,于是$\mathbb{Z}$处正合.
			
			任取$\ker\varepsilon$中的元$y=\sum m_ix^i$,那么$\sum m_i=0$,导致$y=(x-1)z$,于是$y\in\mathrm{Im}D$.
			
			任取$\ker D$中的元$y=\sum m_ix^i$,那么$(x-1)y=0$,导致$m_{k-1}=m_0$,$m_0=m_1$,$\cdots$,$m_{k-2}=m_{k-1}$,于是$y\in\mathrm{Im}N$.
			
			任取$\ker N$中的元$y=\sum m_ix^i$,那么$0=\varepsilon Ny=\varepsilon(N)\varepsilon(y)=k\varepsilon(y)$,于是$\varepsilon(y)=0$,导致$y\in\mathrm{Im}D$.
		\end{proof}
		\item 如果$G$是$k$阶循环群,设$A$是$G$模,记$A[N]=\{a\in A\mid Na=0\}$,那么有:
		$$\mathrm{H}^0(G,A)=A^G,\mathrm{H}^{2n-1}(G,A)=A[N]/DA,\mathrm{H}^{2n}(G,A)=A^G/NA,n\ge1$$
		\item 特别的,如果$G$是平凡群,那么$H^n(G,A)=\{0\},n\ge1$,$H^0(G,A)=A$.
		\item 特别的,如果$G$是$k$阶有限循环群,并且$A$是平凡$G$模,那么:
		$$H^0(G,A)=A,H^{2n-1}(G,A)=A[k],H^{2n}(G,A)=A/kA,n\ge1$$
		\item 特别的,在上一条中取$G$平凡模$A=\mathbb{Z}$,那么$\mathbb{Z}(G)$的整体维数是无穷的.
	\end{enumerate}
\end{enumerate}
\subsubsection{群同调}
\begin{enumerate}
	\item 构造$\textbf{G-Mod}\to\textbf{Ab}$的函子$\mathscr{G}$为,把$G$模$A$映射为$A$的被$G$固定的极大商模$A_G=A/DA$,其中$DA$是被全体$ga-a,g\in G,a\in A$生成的子模.如果记$I_G$是$\mathbb{Z}[G]$的由所有$g-e$生成的理想,那么$DA=I_GA$.把$G$模同态$\phi:A\to B$映射为诱导的同态$\phi_G:A/DA\to B/DB$,这里定义良性是因为$\phi(DA)\subset DB$:$\phi(ga-a)=g\phi(a)-\phi(a)\in DB$.
	\item 把$\mathbb{Z}$视为平凡$G$模,那么存在自然同构$\mathscr{G}(-)\cong\mathbb{Z}\otimes_G-$.于是特别的$\mathscr{G}$是右正合函子.它的左导出函子列称为群同调.
	\begin{proof}
		
		对$G$模$A$,构造$\tau_A:A_G\to\mathbb{Z}\otimes_GA$为$a+DA\mapsto1\otimes a$.这个定义良性是因为对$g\in G,a\in A$有$1\otimes(ga-a)=1\otimes ga-1\otimes a=(g1)\otimes a-1\otimes a=0$.它是同构因为有逆映射$m\otimes a\mapsto ma+DA$.最后容易验证自然同构的交换图表.
	\end{proof}
	\item 于是求$\mathbb{Z}\otimes_{\mathbb{Z}[G]}-$的左导出函子依旧可以考虑$G$模$\mathbb{Z}$的自由预解,记$P_n=\mathbb{Z}[G^{n+1}]$,那么有:
	$$\xymatrix{\cdots\ar[r]&P_2\ar[r]^{\partial_2}&P_1\ar[r]^{d_1}&P_0\ar[r]^{\varepsilon}&\mathbb{Z}\ar[r]&0}$$
	
	再将这个张量函子作用其上得到:
	$$\xymatrix{\cdots\ar[r]&P_2\otimes_GA\ar[r]^{d_2}&P_1\otimes_GA\ar[r]^{d_1}&P_0\otimes_GA\ar[r]&0}$$
	
	这里$P_n\otimes_GA$中的元可以视为$G^{n+1}\to A$的带齐次条件的映射$\alpha$,齐次条件指的是$g\alpha(g_0,\cdots,g_n)=\alpha(gg_0,\cdots,gg_n)$.另外$\alpha$还要满足只对有限个$G^{n+1}$中的元不取零.边界算子可以表示为:
	$$d_n\alpha:G^n\to M$$
	$$(g_0,\cdots,g_{n-1})\mapsto\sum_{i=0}^n(-1)^i\sum_{g\in G}\alpha(g_0,\cdots,g_{i-1},g,g_{i+1},\cdots,g_{n-1})$$
	\item Bar预解.和上同调的情况一样,$P_n\otimes_GA$中的元可以表示为不带齐次条件的$G^n\to A$的映射,满足只在有限个$G^n$的点上不取零,全体这样的映射构成的集合记作$C_0^n(G,A)$.那么此时存在唯一的边界算子$d_n'$使得如下图表交换:
	$$\xymatrix{P_n\otimes_GA\ar[rr]^{d_n}\ar[d]_{\cong}&&P_{n-1}\otimes_GA\ar[d]^{\cong}\\C_0^n(G,A)\ar[rr]_{d_n'}&&C_0^{n-1}(G,A)}$$
	
	即对$\alpha\in C_0^n(G,A)$有:
	\begin{align*}
		d_n'(\alpha)(g_1,g_2,\cdots,g_{n-1})&=\sum_{g\in G}g^{-1}x(g,g_1,\cdots,g_{n-1})\\&+\sum_{0\le j\le n-1}(-1)^j\sum_{g\in G}x(g_1,\cdots,g_jg,g^{-1},g_{j+1},\cdots,g_{n-1})\\&+(-1)^n\sum_{g\in G}x(g_1,\cdots,g_{n-1},g)
	\end{align*}
	\item $\mathrm{H}_1(G,\mathbb{Z})=I_G/I_G^2=G^{\mathrm{ab}}$.
	\begin{proof}
		
		考虑短正合列$0\to I_G\to\mathbb{Z}[G]\to\mathbb{Z}\to0$,它诱导的长正合列为:
		$$\xymatrix{\cdots\ar[r]&H_1(G,\mathbb{Z}[G])\ar[r]&H_1(G,\mathbb{Z})\ar[r]^{\phi}&I_G/I_G^2\ar[r]&\mathbb{Z}[G]/I_G\mathbb{Z}[G]\ar[r]&\cdots}$$
		
		这里$I_G/I_G^2\to\mathbb{Z}[G]/I_G$是零映射,导致$\phi$是满射,而$H_1(G,\mathbb{Z}[G])=0$得到$\phi$是单射.最后有$G/[G,G]\cong I_G/I_G^2$,构造为$s\mapsto s-e$.
	\end{proof}
	
	特别的有$G_0^0(G,A)=A$和$G_0^1(G,A)=\{\alpha:G\to A\text{只在一个有限集合上不取零}\}$.对$\alpha\in G_0^1(G,A)$,有$\mathrm{d}_1'(\alpha)=\sum_{g\in G}(g^{-1}\alpha(g)-\alpha(g))$,于是特别的$\mathrm{H}_0(G,A)=A/I_GA$吻合于定义.
	\item 按照预解的构造,如果$G$是有限群,$G$模$A$是有限集合,那么所有$\mathrm{H}_q(G,A)$都是有限群.
	\item 有限循环群的同调.
	\begin{enumerate}[(1)]
		\item 如果$G$是有限循环群,设生成元为$x$,记$D=x-1,N=1+x+\cdots+x^{k-1}$是$\mathbb{Z}[G]$中的两个元.那么有如下$G$模$\mathbb{Z}$的自由预解:
		$$\xymatrix{\cdots\ar[r]&\mathbb{Z}[G]\ar[r]^{\times D}&\mathbb{Z}[G]\ar[r]^{\times N}&\mathbb{Z}[G]\ar[r]^{\times D}&\mathbb{Z}[G]\ar[r]^{\varepsilon}&\mathbb{Z}\ar[r]&0}$$
		
		由此可直接计算它的同调群为:
		$$\mathrm{H}_0(G,A)=A/DA;\mathrm{H}_{2n-1}(G,A)=A^G/NA;\mathrm{H}_{2n}(G,A)=A^{N=1}/DA$$
		\item 如果$G$是$k$阶循环群,$A$是平凡$G$模,那么有:
		$$\mathrm{H}_0(G,A)=A;\mathrm{H}_{2n-1}(G,A)=A/kA;\mathrm{H}_{2n}(G,A)=A^{N=1}$$
	\end{enumerate}
\end{enumerate}
\subsubsection{诱导模和余诱导模}

设$G$是一个群,设$H$是子群,我们有典范的遗忘函子$\mathscr{F}:\textbf{Mod}(G)\to\textbf{Mod}(H)$.它的左右伴随函子都存在,左伴随函子记作$\mathrm{Ind}_H^G$,同构于形如$\mathrm{Ind}_H^GA$的$G$模称为$H$诱导模,右伴随函子记作$\mathrm{coInd}_H^G$,同构于形如$\mathrm{coInd}_H^GA$的$G$模称为$H$余诱导模.如果$H=\{e\}$,省去角标$H$,并且这两种模简称诱导模和余诱导模.
\begin{enumerate}
	\item 具体构造.设$A$是$H$模.
	\begin{enumerate}[(1)]
		\item $\mathrm{Ind}_H^G(A)=\mathbb{Z}[G]\otimes_HA$,它的$G$模结构定义为$g'(g\otimes a)=g'g\otimes a$.
		\item $\mathrm{coInd}_H^G(A)=\{f:G\to A\mid f(hg)=hf(g),\forall g\in G,h\in H\}=\mathrm{Hom}_H(\mathbb{Z}[G],A)$.这是一个阿贝尔群,它的$G$模结构定义为$(gf)(g')=f(g'g),\forall g,g'\in G$.它称为$H$模$A$的余诱导$G$模.
		\item 特别的,$\mathrm{Ind}^G(A)=\mathbb{Z}[G]\otimes_{\mathbb{Z}}A$相当于是$\mathrm{coInd}^G(A)=\mathrm{Hom}_{\mathbb{Z}}(\mathbb{Z}[G],A)$的子群,由那些只在$G$有限项不取零的映射$f$构成.
	\end{enumerate}
	
	伴随性即:设$H$是$G$的子群,有限制函子$\mathscr{F}:\textbf{G-Mod}\to\textbf{H-Mod}$为把$G$模直接视为$H$模.
	\begin{itemize}
		\item $\mathscr{F}$左伴随于函子$\mathrm{coInd}^G_H$.即对任意$G$模$A$和$H$模$B$有:
		$$\mathrm{Hom}_G(A,\mathrm{coInd}^G_H(B))\cong\mathrm{Hom}_H(\mathscr{F}(A),B)$$
		
		特别的,取$H=\{e\}$得到:
		$$\mathrm{Hom}_G(A,\mathrm{coInd}^G(B))\cong\mathrm{Hom}_{\mathbb{Z}}(A,B)$$
		\item $\mathscr{F}$右伴随于函子$\mathrm{Ind}^G_H$.即对任意$G$模$A$和$H$模$B$有:
		$$\mathrm{Hom}_G(\mathrm{Ind}^G_H(A),B)\cong\mathrm{Hom}_H(A,\mathscr{F}(B))$$
		
		特别的,取$H=\{e\}$得到:
		$$\mathrm{Hom}_G(\mathrm{Ind}^G(A),B)\cong\mathrm{Hom}_{\mathbb{Z}}(A,B)$$
	\end{itemize}
	\begin{proof}
		\begin{align*}
			\mathrm{Hom}_G(A,\mathrm{coInd}^G_H(B))&\cong\mathrm{Hom}_G(A,\mathrm{Hom}_H(\mathbb{Z}[G],B))\\&\cong\mathrm{Hom}_H(A\otimes_G\mathbb{Z}[G],B)\\&\cong\mathrm{Hom}_H(A,B)
		\end{align*}
		\begin{align*}
			\mathrm{Hom}_G(\mathrm{Ind}^G_H(A),B)&\cong\mathrm{Hom}_G(A\otimes_H\mathbb{Z}[G],B)\\&\cong\mathrm{Hom}_H(A,\mathrm{Hom}_G(\mathbb{Z}[G],B))\\&\cong\mathrm{Hom}_H(A,B)
		\end{align*}
	\end{proof}
	\item 如果$G$是有限群,那么诱导$G$模和余诱导$G$模是一致的概念:存在$G$模同构$\varphi:\mathrm{Hom}_H(\mathbb{Z}[G],A)\cong\mathbb{Z}[G]\otimes_HA$为,对$f\in\mathrm{Hom}_H(\mathbb{Z}[G],A)$,记$\varphi(f)=\sum_{\overline{g}\in G/H}g\otimes f(g^{-1})$.这个表示不依赖于$\overline{g}$的提升元$g$的选取.由于$\overline{g}\in G/H$构成了$\mathbb{Z}[H]$模$\mathbb{Z}[G]$的一组基,$\varphi$是一个阿贝尔群同构,它是$G$模同构是因为:
	\begin{align*}
		\varphi(g'f)&=\sum_{\overline{g}\in G/H}g\otimes(g'f)(g^{-1})=\sum_{\overline{g}\in G/H}g\otimes f(g^{-1}g')\\&=\sum_{\overline{g}\in G/H}g'g\otimes f(g^{-1})=g'\varphi(f)
	\end{align*}
	此时一个$G$模$M$是诱导$G$模当且仅当它存在阿贝尔子群$N$,使得$M=\oplus_{g\in G}gN$.特别的如果$H\le G$是子群,那么一个诱导$G$模一定也是诱导$H$模.
	\item 设$G$是有限群,取加性双函子$T:\textbf{Ab}\times\textbf{Ab}\to\textbf{Ab}$.如果$A,B$是$G$模,定义$T(A,B)$上的$G$模结构为:任取$g\in G$,按照$A$和$B$上的$G$模结构,就定义了$g_A\in\mathrm{Hom}(A,A)$和$g_B\in\mathrm{Hom}(B,B)$.取$g$在$T(A,B)$上的作用是$T(g_A,g_B):T(A,B)\to T(A,B)$.例如我们可以取:
	$$A\otimes_{\mathbb{Z}}B,\mathrm{Hom}_{\mathbb{Z}}(A,B),\mathrm{Tor}^{\mathbb{Z}}(A,B),\mathrm{Ext}_{\mathbb{Z}}(A,B)$$
	我们断言如果$A$是诱导$G$模或者相对投射模,那么对任意$G$模$B$都有$T(A,B)$是诱导$G$模或者相对投射模.
	\begin{proof}
		
		问题归结为设$A$是诱导$G$模,那么存在$A$的子群$A'$使得$A=\oplus_{g\in G}gA'$.进而有$T(A,B)=\oplus_{g\in G}T(gA',B)$.
	\end{proof}
	\item 每个$G$模$A$都是某个诱导$G$模$B$的商,并且作为$\mathbb{Z}$模有$A$是$B$的直和项.如果$G$模$A$是某个诱导$G$模作为$G$模的直和项,我们就称$A$是相对投射的或者弱投射的$G$模.
	\begin{proof}
		
		如果把$A$作为阿贝尔群记作$A_0$,考虑$\pi:\mathbb{Z}[G]\otimes_{\mathbb{Z}}A_0\to A$为$g\otimes a\mapsto ga$,这是满同态.于是$A$是诱导$G$模$\mathbb{Z}[G]\otimes_{\mathbb{Z}}A_0$的商.作为$\mathbb{Z}$模是直和项是因为有阿贝尔群同态$\varphi:A\to\mathbb{Z}[G]\otimes_{\mathbb{Z}}A_0$为$a\mapsto 1\otimes a$,使得$\pi\circ\varphi=1_A$.
	\end{proof}
	\item 每个$G$模$A$都是某个余诱导$G$模$B$的子模,并且作为$\mathbb{Z}$模有$A$是$B$的直和项.如果$G$模$A$是某个余诱导$G$模作为$G$模的直和项,我们就称$A$是相对内射的或者弱内射的$G$模.
	\begin{proof}
		
		设$A$是一个$G$模,记$I=\mathrm{Hom}_{\mathbb{Z}}(\mathbb{Z}[G],A)$.定义$A\to\mathrm{Ind}^G(A)$为把$a\in A$对应为一个映射$f_a:G\to A$为$f_a(g)=ga$.这是一个$G$模同态因为$f_{ga}(g')=g'ga=g(g'a)=gf_a(g')$.它是单射因为如果$ga=0,\forall g\in G$,取$g=1$就有$a=0$.最后为证明$A$是$I$的作为交换群的直和项,只要构造交换群同态$I\to A$的映射$f\mapsto f(1)$,满足$A\to I\to A$是$1_A$.
	\end{proof}
	\item 
	\begin{enumerate}[(1)]
		\item 内射$G$模总是弱内射$G$模.如果$A$是内射$G$模,存在余诱导$G$模$I$使得$A$嵌入到$I$中,按照内射模的性质有$A$是$I$的直和项,于是$A$是弱内射$G$模.
		\item 投射$G$模总是弱投射$G$模.如果$A$是投射$G$模,存在从诱导$G$模$B$到$A$的满$G$模同态,按照投射模的性质有$A$是$B$的直和项,于是$A$是弱投射$G$模.
	\end{enumerate}
	\item 推论.$\mathrm{coInd}_H^G$把$\textbf{H-Mod}$上的内射对象映射为$\textbf{G-Mod}$上的内射对象.对偶的$\mathrm{Ind}_H^G$把$\textbf{H-Mod}$上的投射对象映射为$\textbf{G-Mod}$上的投射对象.特别的,如果$I$是可除交换群,那么$\mathrm{coInd}^G(I)$是内射$G$模;如果$P$是自由阿贝尔群,那么$\mathrm{Ind}^G(P)$是投射$G$模.
	\item 设$H$是$G$的子群,那么每个内射$G$模都是内射$H$模.特别的,内射$G$模一定是内射交换群(也即可除交换群),这件事还可以从如下典范同构直接得到:
	一个内射$G$模一定是内射交换群(也即可除交换群).
	$$\mathrm{Hom}_{\mathbb{Z}}(-,A)=\mathrm{Hom}_G(-\otimes_{\mathbb{Z}}\mathbb{Z}[G],A)$$
	\begin{proof}
		
		设$A$是内射$G$模,那么$A$可以嵌入到余诱导$G$模$\mathrm{coInd}^G(I)=\mathrm{Hom}_{\mathbb{Z}}(\mathbb{Z}[G],I)$中,这里阿贝尔群$I$可以嵌入到某个可除交换群$J$中(阿贝尔群范畴上的内射对象就是可除交换群),导致$\mathrm{coInd}^G(I)$可以嵌入到$\mathrm{coInd}^G(J)$,于是我们不妨直接设$I$是可除交换群.由于$A$是内射的,于是它作为$G$模是$\mathrm{coInd}^G(I)$的直和项,所以一旦我们证明$\mathrm{Hom}_{\mathbb{Z}}(\mathbb{Z}[G],I)$作为$H$模是内射的,就得到它的直和项$A$也是内射$H$模.但是这里$\mathbb{Z}[G]$是若干$\mathbb{Z}[H]$的直和,按照$\mathrm{Hom}$第一个分量的直和提出来可以变成直积,于是$\mathrm{Hom}_{\mathbb{Z}}(\mathbb{Z}[G],I)$是若干$\mathrm{Hom}_{\mathbb{Z}}(\mathbb{Z}[H],I)$的直积,但是这里$\mathrm{Hom}_{\mathbb{Z}}(\mathbb{Z}[H],I)$是内射$H$模,内射模的直积是内射模,这就得到$\mathrm{Hom}_{\mathbb{Z}}(\mathbb{Z}[G],I)$是内射$H$模.
	\end{proof}
	\item 如果$A$是相对内射$G$模,那么$\mathrm{H}^q(G,A)=0,\forall q\ge1$;如果$A$是相对投射$G$模,那么$\mathrm{H}_q(G,A)=0,\forall q\ge1$.
	\begin{proof}
		
		我们知道Ext函子满足$\mathrm{H}^q(G,A\oplus B)\cong\mathrm{H}^q(G,A)\oplus\mathrm{H}^q(G,B)$.于是问题归结为对余诱导$G$模$A$有$\mathrm{H}^q(G,A)=0,\forall q\ge1$.我们知道$\mathrm{H}^q(G,A)$就是如下复形的同调:
		$$\xymatrix{0\ar[r]&\mathrm{Hom}_G(P_0,A)\ar[r]&\mathrm{Hom}_G(P_1,A)\ar[r]&\cdots}$$
		
		但是我们有:
		\begin{align*}
			\mathrm{Hom}_G(P_n,A)&=\mathrm{Hom}_G(P_n,\mathrm{Hom}_{\mathbb{Z}}(\mathbb{Z}[G],A_0))\\&=\mathrm{Hom}_{\mathbb{Z}}(P_n\otimes_G\mathbb{Z}[G],A_0)\\&=\mathrm{Hom}_{\mathbb{Z}}(P_n,A_0)
		\end{align*}
		
		于是这个复形的同调就是$\mathrm{H}^q(G,A)=\mathrm{Ext}_{\mathbb{Z}}^q(\mathbb{Z},A_0)$,这在$q\ge1$时总是平凡群.
	\end{proof}
	\item 有$G$模的短正合列$0\to A\to I\to A'\to0$,其中$I$是余诱导$G$模,那么有:
	$$\mathrm{H}^q(G,A')=\mathrm{H}^{q+1}(G,A),\forall q\ge1$$
\end{enumerate}
\subsubsection{基群的变换}
\begin{itemize}
	\item 设$f:G'\to G$是群同态,设$A$是一个$G$模,那么$A$可以典范的视为一个$G'$模,即对$s'\in G'$,定义$s'a=f(s')a$.把作为$G'$模的$A$记作$f_*A$.我们有如下链复形之间的同态:
	$$\xymatrix{0\ar[r]&\mathrm{Hom}_G(\mathbb{Z}[G],A)\ar[r]\ar[d]&\mathrm{Hom}_G(\mathbb{Z}[G^2],A)\ar[r]\ar[d]&\cdots\\0\ar[r]&\mathrm{Hom}_{G'}(\mathbb{Z}[G'],f_*A)\ar[r]&\mathrm{Hom}_{G'}(\mathbb{Z}[(G')^2],f_*A)\ar[r]&\cdots}$$
	
	这里垂直同态是把$G^n\to A$的映射$\alpha$对应到$\alpha\circ f:(G')^n\to f_*A$.那么这个链同态诱导了复形之间的同态$f^*_q:\mathrm{H}^q(G,A)\to\mathrm{H}^q(G',f_*A)$.它在$q=0$时就是包含映射$A^G\subseteq(f_*A)^{G'}$.
	\item 设$f:G'\to G$是群同态,设$A$是$G$模,$A'$是$G'$模,设$g:A\to A'$是一个交换群同态.称$g$是一个$f$同态,如果对任意$s'\in G$和任意$a\in A$有$g(f(s')a)=s'g(a)$,这等价于讲$g$是$G'$模同态$f_*A\to A'$.那么$g$诱导了同态$\mathrm{H}^q(G',f_*A)\to\mathrm{H}^q(G',A')$,复合上之前的同态得到$(f,g)_q^*:\mathrm{H}^q(G,A)\to\mathrm{H}^q(G',A')$.
\end{itemize}
\begin{enumerate}
	\item 回顾一下如果$\{T_n\}$和$\{H_n\}$是同一个阿尔贝范畴上的两个同调$\delta$函子,并且对投射对象$P$都有$T_n(P)=H_n(P)=0,\forall n\ge1$.那么对任意自然变换$\tau_0:T_0\to H_0$,都可以唯一的延拓为同调$\delta$函子之间的态射(此即一族自然变换$\tau_n:T_n\to H_n$,使得和长正合列的连接同态可交换).
	\item 上同调的res同态.设$H$是$G$的子群,设$f:H\to G$是包含映射,这诱导了同态$\mathrm{H}^q(G,A)\to\mathrm{H}^q(H,A)$,称为限制(restriction)同态,记作$\mathrm{Res}_n$.它就是把$G^n\to A$的映射限制成$H^n\to A$的映射构成的链同态诱导的同调之间的同态.我们断言$\{\mathrm{res}_n\}$就是$\mathrm{res}_0:\mathrm{H}^0(G,-)\to\mathrm{H}^0(H,-)$唯一延拓的同调$\delta$函子之间的态射.
	\begin{proof}
		
		这些$\mathrm{res}_n$明显是自然变换,于是我们只需证明如果$0\to A\to B\to C\to0$是$G$模的短正合列,那么有交换图表:
		$$\xymatrix{\mathrm{H}^n(G,C)\ar[rr]^{\delta}\ar[d]_{\mathrm{res}}&&\mathrm{H}^{n+1}(G,A)\ar[d]^{\mathrm{res}}\\\mathrm{H}^n(H,C)\ar[rr]_{\delta}&&\mathrm{H}^{n+1}(H,C)}$$
		
		为此只要用如下交换图表:
		$$\xymatrix{&0\ar[r]&\mathrm{C}^n(G,A)\ar[r]\ar[dl]\ar[ddr]&\mathrm{C}^n(G,B)\ar[r]\ar[dl]\ar[ddr]&\mathrm{C}^n(G,C)\ar[r]\ar[dl]\ar[ddr]&0&&\\0\ar[r]&\mathrm{C}^{n+1}(G,A)\ar[r]\ar[ddr]&\mathrm{C}^{n+1}(G,B)\ar[r]\ar[ddr]&\mathrm{C}^{n+1}(G,C)\ar[r]\ar[ddr]&0&&&\\&&0\ar[r]&\mathrm{C}^n(H,A)\ar[r]\ar[dl]&\mathrm{C}^n(H,B)\ar[r]\ar[dl]&\mathrm{C}^n(H,C)\ar[r]\ar[dl]&0\\&0\ar[r]&\mathrm{C}^{n+1}(H,A)\ar[r]&\mathrm{C}^{n+1}(H,B)\ar[r]&\mathrm{C}^{n+1}(H,C)\ar[r]&0&}$$
		
	\end{proof}
	\item 上同调的inf同态.设$H$是$G$的正规子群,设$A$是$G$模,那么$A^H$是$G/H$模,并且$A^H\subseteq A$是$f:G\to G/H$同态,于是我们得到同态$\mathrm{H}^q(G/H,A^H)\to\mathrm{H}^q(G,A)$,称为inflation同态,记作$\mathrm{Inf}$.它就是把$(G/H)^n\to A^H$的映射对应为$G^n\to(G/H)^n\to A^H\subseteq A$的映射,构成的链同态诱导的同调之间的同态.
	\item 同调的res同态.设$H$是$G$的有限指数子群,设$f:H\to G$是包含映射,设$A$是$G$模,那么$f_*A$就是把$A$限制为$H$模.定义$\mathrm{H}_0(G,A)=A_G\to\mathrm{H}_0(H,f_*A)=A_H$为,$a\mapsto\sum_{\overline{g}\in G/H}g^{-1}a$.它延拓为上同调函子之间的态射$\{\mathrm{H}_q(G,-)\}\to\{\mathrm{H}_q(H,f_*-)\}$.这称为同调的res同态.
	\item 设$G=G'$,$A=A'$,取$t\in G$,定义$f:G\to G$为内自同构$s\mapsto tst^{-1}$,定义$g:A\to A$为$a\mapsto t^{-1}a$.它们是兼容的,于是诱导了同态$\sigma_t:\mathrm{H}^q(G,A)\to\mathrm{H}^q(G,A)$,我们断言这实际上是恒等映射.
	\begin{proof}
		
		首先$n=0$时是平凡的.下面设$n>0$,如果$n-1$的情况成立,把$G$模$A$嵌入到余诱导$G$模$I$中,记$B=I/A$,那么有如下交换图表,于是从$\sigma_{n-1}$是恒等映射得到$\sigma_n$也是恒等映射.
		$$\xymatrix{\mathrm{H}^{n-1}(G,B)\ar[r]^{\delta}\ar[d]_{\sigma_{n-1}}&\mathrm{H}^n(G,A)\ar[r]\ar[d]^{\sigma_n}&0\\\mathrm{H}^{n-1}(G,B)\ar[r]^{\delta}&\mathrm{H}^n(G,A)\ar[r]&0}$$
	\end{proof}
	\item 设$H$是$G$的正规子群,任取$s\in G$,记$f_s:H\to H$为$t\mapsto sts^{-1}$.记$H$是$A$模,取$g_s:A\to A$为$a\mapsto s^{-1}a$.那么$f_s$和$g_s$是兼容的,于是诱导了同态$\mathrm{H}^q(H,A)\to\mathrm{H}^q(H,A)$.于是我们定义了$G$在每个$\mathrm{H}^q(H,A)$上的作用.上一条就是说正规子群$H$在$\mathrm{H}^q(H,A)$上的作用总是平凡的,这就诱导了$G/H$在每个$\mathrm{H}^q(H,A)$上的群作用.
	\item Shapiro引理.设$H$是$G$的子群,设$A'$是一个$H$模,那么$A=\mathrm{coInd}_H^G(A')=\mathrm{Hom}_H(\mathbb{Z}[G],A')$是$G$模,定义$g:A'\to A$为把$\varphi\mapsto\varphi(1)$.那么$g$是$f:H\subseteq G$同态.于是诱导了$\mathrm{H}^q(G,A')\to\mathrm{H}^q(H,A)$.我们断言这些都是同构.
	\begin{proof}
		
		取$H$模$A'$的内射预解$I^*\to A'$,那么$\mathrm{H}^q(H,A')$就是对简化内射预解$I^*\to0$作用函子$M\mapsto M^H$再取同调.但是对任意$H$模$M$,我们有$(\mathrm{coInd}_H^G(M))^G=M^H$,于是$\mathrm{H}^q(H,A')$也就是对$\mathrm{coInd}(I)^*\to0$作用函子$M\to M^G$再取同调.由于$\mathbb{Z}[G]$是$\mathbb{Z}[H]$范畴上的自由模,于是$\mathrm{coInd}_H^G(-)$是$\textbf{H-Mod}\to\textbf{G-Mod}$的正合函子,并且它把内射对象映射为内射对象,于是$\mathrm{coInd}(I)^*\to A$也是$G$模范畴上$A$的内射预解,它作用函子$M\to M^G$再取同调就是$\mathrm{H}^q(G,A)$.
	\end{proof}
	\item res-inf正合列.
	\begin{enumerate}
		\item 设$H$是$G$的正规子群,设$A$是$G$模,那么$A^H$是$G/H$模,我们有如下正合列:
		$$\xymatrix{0\ar[r]&\mathrm{H}^1(G/H,A^H)\ar[r]^{\mathrm{inf}}&\mathrm{H}^1(G,A)\ar[r]^{\mathrm{res}}&\mathrm{H}^1(H,A)}$$
		\begin{proof}
			\begin{itemize}
				\item $\ker\mathrm{inf}=0$:设$f:G/H\to A^H$是一个余圈,并且它在$\mathrm{H}^1(G,A)$中为零,此即如果把$f$复合为$\widetilde{f}:G\to G/H\to A^H\subseteq A$,那么存在$a\in A$使得$\widetilde{f}(s)=sa-a$.由于$\widetilde{f}(s)$只依赖于$s$在$G/H$中的陪集,于是对任意$t\in H$有$sa-a=sta-a$,也即$ta=a,\forall t\in H$,于是$a\in A^H$,于是$f$本身在余边界中.
				\item $\mathrm{im}\mathrm{inf}\subseteq\ker\mathrm{res}$:考虑映射$f:G/H\to A^H$,它先扩充为$\widetilde{f}:G\to G/H\to A^H\subseteq A$再限制在$H$上得到$\widetilde{f}=0$,这个链映射诱导的同调之间的同态就是$\mathrm{res}\circ\mathrm{inf}$,但是这个链映射是零同态,所以它诱导的同调之间的同态是零.
				\item $\ker\mathrm{res}\subseteq\mathrm{im}\mathrm{inf}$:设$f:G\to A$是余圈,使得限制在$H$上的上同调类为零,此即存在$a\in A$,使得对任意$t\in H$有$f(t)=ta-a$.取$G\to A$的映射$g(s)=sa-a,\forall s\in G$,用$f-g$替代$f$,不妨设$f$满足$f(t)=0,\forall t\in H$.由于$f$是余圈所以要满足$f(st)=f(s)+sf(t)$.如果取$t\in H$,说明$f$在$G/H$的每个陪集类上取值是固定的.如果取$s\in H$,那么$f(t)=f(st)=sf(t)$,于是$f$的取值落在$A^H$中,这就说明$f$诱导了一个$G/H\to A^H$的余圈.
			\end{itemize}
		\end{proof}
		\item 依旧设$H$是$G$的正规子群,设$A$是$G$模,设有正整数$q$使得$\mathrm{H}^i(H,A)=0,\forall 1\le i\le q-1$.那么有如下正合列:
		$$\xymatrix{0\ar[r]&\mathrm{H}^q(G/H,A^H)\ar[r]^{\mathrm{inf}}&\mathrm{H}^q(G,A)\ar[r]^{\mathrm{res}}&\mathrm{H}^q(H,A)}$$
		
		并且对$1\le i\le q-1$总有$\mathrm{inf}:\mathrm{H}^i(G/H,A^H)\to\mathrm{H}^i(G,A)$都是同构.
		\begin{proof}
			
			我们对$q$归纳,$q=1$即上一条.下面设$q\ge2$.取$B=\mathrm{Hom}_{\mathbb{Z}}(\mathbb{Z}[G],A)$为$A$作为阿贝尔群的余诱导$G$模.我们有典范的$G$模单同态$A\to B$为把$a\in A$映射为$B$中的映射$\varphi_a:t\mapsto ta$.记$C=B/A$,我们有$G$模的正合列$0\to A\to B\to C\to0$.由于$B$是诱导$G$模,它也是诱导$H$模,我们有$\mathrm{H}^i(H,C)=\mathrm{H}^{i+1}(H,A)=0,\forall 1\le i\le q-1$.另外按照条件有$\mathrm{H}^1(H,A)=0$,于是有短正合列$0\to A^H\to B^H\to C^H\to0$.这里$B^H=\mathrm{Hom}_{\mathbb{Z}}(\mathbb{Z}[G/H],A)$是$G/H$余诱导模.于是这两个短正合列诱导了如下交换图表,这里$\delta$是长正合列的连接同态:
			$$\xymatrix{0\ar[r]&\mathrm{H}^{q-1}(G/H,C^H)\ar[r]\ar[d]_{\delta}&\mathrm{H}^{q-1}(G,C)\ar[r]\ar[d]_{\delta}&\mathrm{H}^{q-1}(H,C)\ar[d]_{\delta}\\0\ar[r]&\mathrm{H}^q(G/H,A^H)\ar[r]&\mathrm{H}^q(G,A)\ar[r]&\mathrm{H}^q(H,A)}$$
			
			按照归纳假设,第一行是正合列,由于$B^H$是诱导$G/H$模,$B$是诱导$G$模也是诱导$H$模,它们的上同调在$\ge1$时都是零,于是连接同态$\delta$都是同构,这里垂直的同态都是同构,于是第二行也是正合列.最后我们实际上证明了对$1\le i\le q$总有这个正合列,结合条件$\mathrm{H}^i(H,A)=0,\forall 1\le i\le q-1$,就得到$1\le i\le q-1$时$\mathrm{inf}:\mathrm{H}^i(G/H,A^H)\to\mathrm{H}^i(G,A)$都是同构.
		\end{proof}
	\end{enumerate}
	\item Hochschild-Serre谱序列.设$H$是$G$的正规子群,设$A$是$G$模,那么我们解释过$\mathrm{H}^q(H,A)$可视为一个$G/H$模.我们有谱序列收敛:
	$$\mathrm{E}_2^{s,t}=\mathrm{H}^s(G/H,\mathrm{H}^t(H,A))\Rightarrow\mathrm{H}^{s+t}(G,A)$$
	
	这可以延拓res-inf正合列的结论:条件$\mathrm{H}^i(H,A)=0,\forall 1\le i\le q-1$也即$\mathrm{E}_2^{s,t}$塌陷在$t=0$和$t\ge q$上,此时我们有如下正合列:
	$$\xymatrix{0\ar[r]&\mathrm{H}^q(G/H,A^H)\ar[r]&\mathrm{H}^q(G,A)\ar[r]&\mathrm{H}^q(H,A)^{G/H}\ar[r]&\mathrm{H}^{q+1}(G/H,A^H)\ar[r]&\mathrm{H}^{q+1}(G,A)}$$
	\item 上同调的余限制.设$H$是$G$的子群,设$A$是$G$模,我们之前构造了限制同态$\mathrm{res}:\mathrm{H}^q(G,A)\to\mathrm{H}^q(H,A)$.这里我们设$[G:H]$有限,构造反方向的余限制$\mathrm{cor}:\mathrm{H}^q(H,A)\to\mathrm{H}^q(G,A)$.首先对$q=0$,构造$A^H\to A^G$为$a\mapsto\sum_{\overline{g}\in G/H}ga$,记作$\mathrm{N}_{G/H}$.它可以唯一的延拓为$\{\mathrm{H}^i(H,f_*-)\}$到$\{\mathrm{H}^i(G,-)\}$的态射.定义为余限制同态.
	\item 设$[G:H]=n$,那么$\mathrm{cor}\circ\mathrm{res}=n$,即$\mathrm{H}^i(G,A)$上的数乘$n$的同态.
	\begin{proof}
		
		对于$n=0$,如果$a\in A^G$,那么$\mathrm{cor}\circ\mathrm{res}(a)=\sum_{\overline{g}\in G/H}ga=[G:H]a=na$.接下来由于$\mathrm{cor}\circ\mathrm{res}-n$是$\{\mathrm{H}^n(G,-)\}\to\{\mathrm{H}^n(G,-)\}$作为上同调$\delta$函子之间的态射,它必须是唯一的,但是它在零维情况是零映射,于是任意维上都是零映射,于是$\mathrm{cor}\circ\mathrm{res}=n$.
	\end{proof}
	\item 同调的余限制.设$H$是$G$的子群,设$A$是$G$模,取$\mathrm{cor}_0:\mathrm{H}_0(H,A)\to\mathrm{H}_0(G,A)$为典范映射$A/I_HA\to A/I_GA$诱导的同调$\delta$函子之间的态射$\mathrm{cor}_n:\mathrm{H}_n(H,A)\to\mathrm{H}_n(G,A)$.
	\item transfer.设$H$是$G$的有限指数子群,考虑$\mathbb{Z}$作为平凡$G$模,我们称一阶群同态的限制同态$\mathrm{res}_1:\mathrm{H}_1(G,\mathbb{Z})\to\mathrm{H}_1(H,\mathbb{Z})$为transfer,记作$\mathrm{Ver}$.我们下面给出transfer的具体表示.设$R$是子群$H$的右陪集代表元集合,并且约定包含幺元,对任意$s\in G$和$t\in R$,存在唯一的$t'\in R$和$s_t\in H$,使得$ts=s_tt'$.记$G'$表示$G$的换位子群,那么$\mathrm{Ver}:G/G'\to H/H'$可以表示为$s(\mathrm{mod}G')\mapsto\prod_{t\in R}s_t(\mathrm{mod}H')$.
	\begin{proof}
		
		我们有如下交换图表,其中$\mathrm{N}$是把$x\in I_G$映射为$\sum_{\overline{g}\in G/H}g^{-1}x=\sum_{t\in R}tx$,而$i$是典范包含映射
		$$\xymatrix{G/G'\ar[rr]^{\cong}\ar[dd]_{\mathrm{Ver}}&&I_G/I_G^2\ar[drr]^{\mathrm{N}}&&\\&&&&I_G/I_HI_G\\H/H'\ar[rr]^{\cong}&&I_H/I_H^2\ar[urr]_i&&}$$
		
		任取$s\in G$,它映入$I_G/I_G^2$就是$s-1$,那么有:
		\begin{align*}
			\mathrm{N}(s-1)&\equiv\sum_{t\in R}ts-\sum_{t\in R}t(\mathrm{mod}I_HI_G)\\&\cong\sum_ts_tt'-\sum_tt'(\mathrm{mod}I_HI_G)\\&\cong\sum_t(s_t-1)(t'-1)+\sum_t(s_t-1)(\mathrm{mod}I_HI_G)\\&\cong\sum_t(s_t-1)(\mathrm{mod}I_HI_G)
		\end{align*}
		
		而这里$\sum_t(s_t-1)$在$\mathrm{mod}I_H^2$下就对应于$\prod_{t\in R}s_t$.
	\end{proof}
\end{enumerate}
\subsubsection{非阿贝尔群的上同调}

设$A$是一个未必阿贝尔的群,群$G$在$A$上的(左)作用就是指一个群同态$\rho:G\to\mathrm{Aut}(A)$,对$s\in G$和$a\in A$,我们记$sa=\rho(s)a$.于是有$s(ab)=(sa)(sb)$和$(s_1s_2)a=s_1(s_2a)$.两个$G$模之间的同态定义为一个群同态$\varphi:A\to B$,满足和群作用可交换,即$\forall s\in G$和$\forall a\in A$有$\varphi(ga)=g\varphi(a)$.
\begin{enumerate}
	\item 
	\begin{itemize}
		\item 依旧定义$\mathrm{H}^0(G,A)=A^G=\{a\in A\mid g(a)=a,\forall g\in G\}$,这是$A$的子群,具备群结构.
		\item 定义1余圈是一个映射$f:G\to A$,满足$f(st)=f(s)s(f(t))$.两个1余圈$f,g$称为上同调的,如果存在$a\in A$使得$g(s)=a^{-1}f(s)s(a),\forall s\in G$成立.这定义了1余圈集合上的一个等价关系,商集合记作$\mathrm{H}^1(G,A)$.尽管它未必是阿贝尔群,我们把它理解成一个带基点的集合,基点取为单位1余圈$f:s\mapsto1_A,\forall s\in G$所在的上同调类.
	\end{itemize}
	\item 我们构造的$\mathrm{H}^0(G,-)$和$\mathrm{H}^1(G,-)$依旧具有函子性.如果$\varphi:A\to B$是$G$模同态.我们定义:
	\begin{itemize}
		\item $\varphi^0:\mathrm{H}^0(G,A)\to\mathrm{H}^0(G,B)$就是$\varphi$限制在$A^G\to B^G$上.像集落在$B^G$上是因为$s\varphi(a)=\varphi(sa)=\varphi(a)$.这是一个群同态.
		\item $\varphi^1:\mathrm{H}^1(G,A)\to\mathrm{H}^1(G,B)$.对1余圈$f:G\to A$,定义$\varphi^1(f)=\varphi\circ f$,这依旧是$B$上的1余圈是因为$\varphi(f(st))=\varphi(f(s)s(f(t)))=\varphi(f(s))s\varphi(f(t))$.并且它把基点映射为基点.这是一个保基点的集合映射.
	\end{itemize}
	\item 设$\xymatrix{1\ar[r]&A\ar[r]^i&B\ar[r]^p&C\ar[r]&1}$是未必阿贝尔的$G$模的短正合列,那么$i(A)$是$B$的正规子群,我们索性就把$A$视为$B$的正规子群.构造连接映射$\delta:\mathrm{H}^0(G,C)\to\mathrm{H}^1(G,A)$如下:设$c\in C^G$,选取$b\in B$使得$p(b)=c$.对任意$s\in G$,有$p(s(b)b^{-1})=1$,于是总有$b^{-1}s(b)\in A$.我们定义$f:G\to A$为$f(s)=b^{-1}s(b)$.我们断言$f$是余圈,并且$f$所在的上同调类不依赖于$c$的原像$b$的选取:如果$p(b')=p(b)=c$,那么$b'b^{-1}\in\ker p=A$,于是存在$a\in A$使得$b'=ba$,如果$b'$定义的余圈是$f'$,那么$f'(s)=(b')^{-1}s(b')=a^{-1}b^{-1}s(b)s(a)=a^{-1}f(s)s(a)$,于是$f'$和$f$在相同的上同调类.再验证$f$是余圈:$f(st)=b^{-1}st(b)=b^{-1}s(b)s(b^{-1}t(b))=f(s)s(f(t))$.
	\item 设$A$包含在群$B$的中心,那么$A$是阿贝尔群,记号$\mathrm{H}^2(G,A)$的含义是明确的.依旧设$\xymatrix{0\ar[r]&A\ar[r]^i&B\ar[r]^p&C\ar[r]&1}$是$G$模的短正合列.我们来构造保基点的映射$\Delta:\mathrm{H}^1(G,C)\to\mathrm{H}^2(G,A)$:设$f:G\to C$是余圈,对任意$s\in G$取$g(s)\in B$使得$p(g(s))=f(s)$,上一段证明了$g(st)\equiv g(s)s(g(t))(\mathrm{mod}A),\forall s,t\in G$.定义$h:G^2\to A$为$h(s,t)=g(s)s(g(t))g(st)^{-1}$.我们断言$h$是2余圈,并且把$f$替换为相同上同调类中的1余圈$f'$得到的$h'$和$h$在相同的上同调类,并且改变$g(s)\in p^{-1}(f(s))$定义出来的$h'$和$h$也在相同的上同调类.
	\begin{itemize}
		\item 验证$h$是2余圈:此即证明$s(h(t,u))h(s,tu)=h(st,u)h(s,t)$,按照$A$是阿贝尔的,等价于证明$h(s,t)^{-1}h(s,tu)h(st,u)^{-1}s(h(t,u))=1$.左边展开就是:
		$$g(st)s(g(t))^{-1}s(g(tu))st(g(u))^{-1}g(st)^{-1}s(h(t,u))$$
		
		按照$A$在$B$的中心里,说明$s(h(t,u))$和$b$中的元可交换,于是上式变成:
		$$g(st)s(g(t))^{-1}s(h(t,u))s(g(tu))st(g(u))^{-1}g(st)^{-1}$$
		$$=g(st)s(g(t))^{-1}s\left(g(t)t(g(u))g(tu)^{-1}\right)s(g(tu))st(g(u))^{-1}g(st)^{-1}=1$$
		\item 把$f$替换为相同上同调类中的$f'$,那么存在$c\in C$使得$f'(s)=c^{-1}f(s)s(c)$.取$c$在$B$中的提升$b$,也即$p(b)=c$,取$g'(s)=b^{-1}g(s)s(b)$,我们断言$h'(s,t)=h(s,t)$:
		\begin{align*}
			h'(s,t)&=g'(s)s(g'(t))g'(st)^{-1}\\&=b^{-1}g(s)s(b)s(b^{-1}g(t)t(b))(b^{-1}g(st)st(b))^{-1}\\&=b^{-1}h(s,t)b\\&=h(s,t)
		\end{align*}
		\item 最后如果$g'(s)$也是$f(s)$在$p$下的原像,此即存在$a_s\in A$使得$g'(s)=a_sg(s)$.于是余圈$h(s,t)$就要替换为$h'(s,t)=g'(s)s(g'(t))g'(st)^{-1}=a_sg(s)s(a_t)s(g(t))g(st)^{-1}a_{st}^{-1}$.由于$A$落在$B$的中心,于是:
		$$h'(s,t)=a_ss(a_t)a_{st}^{-1}h(s,t)$$
		
		这里$a_ss(a_t)a_{st}^{-1}$是一个2余边界,所以$h(s,t)$和$h'(s,t)$落在相同的上同调类.
	\end{itemize}
	\item 设$\xymatrix{1\ar[r]&A\ar[r]^i&B\ar[r]^p&C\ar[r]&1}$是$G$模的短正合列,带基点的集合上就可以讨论核与余核,核就是基点的原像集.
	\begin{enumerate}
		\item 我们有如下正合列:
		$$\xymatrix{1\ar[r]&\mathrm{H}^0(G,A)\ar[r]^{i_0}&\mathrm{H}^0(G,B)\ar[r]^{p_0}&\mathrm{H}^0(G,C)\ar[r]^{\delta}&\mathrm{H}^1(G,A)\ar[r]^{i_1}&\mathrm{H}^1(G,B)\ar[r]^{p_0}&\mathrm{H}^1(G,C)}$$
		\item 如果额外的还有$A$落在$B$的中心,那么有如下正合列:
		$$\xymatrix{1\ar[r]&\mathrm{H}^0(G,A)\ar[r]^{i_0}&\mathrm{H}^0(G,B)\ar[r]^{p_0}&\mathrm{H}^0(G,C)\ar[r]^{\delta}&\mathrm{H}^1(G,A)\ar[r]^{i_1}&\mathrm{H}^1(G,B)}$$
		$$\xymatrix{\ar[r]^{p_0}&\mathrm{H}^1(G,C)\ar[r]^{\Delta}&\mathrm{H}^2(G,A)}$$
	\end{enumerate}
	\begin{proof}
		\begin{itemize}
			\item $\mathrm{H}^0(G,A)$处的正合性.此即$i_0$是单射,它是包含映射$A\to B$限制的$A^G\to B^G$,于是它是单射.
			\item $\mathrm{H}^0(G,B)$处的正合性.一方面函子性保证$p_0\circ i_0=1$.反过来如果$b\in B^G$满足$p_0(b)=0$,那么$b\in A\cap B^G=A^G$.
			\item $\mathrm{H}^0(G,C)$处的正合性.任取$c=p_0(b)$,其中$b\in B^G$,我们解释过$\delta(c)$是一个1余圈$f:G\to A$,满足$f(s)=b^{-1}s(b)$,但是这里$b\in B^G$,导致$f(s)=1$,于是$f$落在基点所在的上同调类;反过来如果$c\in C^G$,满足$\delta(c)$作为映射$f:G\to A$是落在基点所在上同调类的,此即存在$b\in p^{-1}(c)$使得$1=b^{-1}s(b),\forall s\in G$,于是$b\in B^G$,于是$c\in\mathrm{im}p_0$.
			\item $\mathrm{H}^1(G,A)$处的正合性.取$c\in C^G$,那么$\delta(c)$是1余圈$f:G\to A$,满足$f(s)=b^{-1}s(b)$,其中$b\in p^{-1}(c)$.如果把它视为$G\to B$的映射,它本身是1余边界.反过来如果$f:G\to A$是一个1余圈,使得它视为$G\to B$的映射是1余边界,此即存在$b\in B$使得$f(s)=b^{-1}s(b)$,于是如果记$p(b)=c$,那么$f=\delta(c)$.
			\item $\mathrm{H}^1(G,B)$处的正合性.一方面函子性保证$p_1\circ i_1=1$.反过来如果1余圈$f:G\to B$满足$p\circ f$是零调的,那么存在$c\in C$使得$p\circ f(s)=c^{-1}s(c)$,取$b\in B$使得$p(b)=c$.那么对每个$s\in G$,从$p(f(s))=p(b^{-1}s(b))$得到$p(bf(s)s(b)^{-1})=1$,于是$h(s)=bf(s)s(b)^{-1}\in A$,我们断言$h$是$A$上的1余圈,即$h(st)=h(s)s(h(t))$,展开等价于$f(st)=f(s)sf(t)$,但是由于$f$是$B$上的1余圈知这成立.于是$f(s)=b^{-1}h(s)s(b)$,这说明$f$在$B$上和一个$A$中的1余圈在相同的上同调类,于是$f$落在$i_1$的像中.
			\item $\mathrm{H}^1(G,C)$处的正合性(此时我们约定$A$在$B$的中心里).取$B$上的1余圈$f:G\to B$,那么$p\circ f$是$C$上的1余圈,那么$\Delta(p\circ f)$是映射$h:G\times G\to A$,对任意$s\in G$,取$g(s)\in B$使得$p(g(s))=p(f(s))$,这里我们可以干脆取$g=f$本身,那么$h(s,t)=f(s)s(f(t))f(st)^{-1}$,这是一个2余边界,于是$p_1$的像落在$\Delta$的核中.反过来如果$f(s)$是$C$上的一个1余圈,满足$\Delta(f)$是零调的.如果取$g(s)\in B$使得$p(g(s))=f(s)$,那么就是说$h(s,t)=g(s)s(g(t))g(st)^{-1}$是零调的,于是它可以表示为$k(s)s(k(t))k(st)^{-1}$,其中$k:G\to A$的映射.如果记$g'(s)=k(s)^{-1}g(s)$,那么这是$B$上的一个1余圈,它也满足$p(g'(s))=p(f(s)^{-1}g(s))=p(g(s))=f(s)$,于是$f(s)$在$p_1$的像中.
		\end{itemize}
	\end{proof}
	\item 存在这样的例子,使得$\Delta$的像集不构成$\mathrm{H}^2(G,A)$的子群.
\end{enumerate}
\newpage
\subsection{低阶上同调和群的延拓}

\newpage
\subsection{有限群的上同调和同调}
\subsubsection{Tate上同调}

设$G$是有限群,记$N=\sum_{g\in G}g$是$\mathbb{Z}[G]$中的元(这必须要求$G$是有限群),它称为$G$的范数.如果$A$是$G$模,记$A\to A$的映射$a\mapsto Na$的核为$_NA$.定义$\mathrm{N}^*:\mathrm{H}_0(G,A)=A/I_GA\to\mathrm{H}^0(G,A)=A^G$为$a\mapsto Na$.那么$\ker\mathrm{N}^*=_NA/I_GA$,$\mathrm{coker}\mathrm{N}^*=A^G/NA$.我们定义Tate上同调为:
$$\widehat{\mathrm{H}}^n(G,A)=\left\{\begin{array}{cc}\mathrm{H}^n(G,A)&n\ge1\\A^G/NA&n=0\\_NA/I_GA&n=-1\\\mathrm{H}_{-n-1}(G,A)&n\le -2\end{array}\right.$$
\begin{enumerate}
	\item 设$0\to A\to B\to C\to0$是$G$模的短正合列,那么它诱导了Tate上同调的长正合列:
	$$\xymatrix{\cdots\ar[r]&\widehat{\mathrm{H}}^{-2}(G,C)\ar[r]^{\delta}&\widehat{\mathrm{H}}^{-1}(G,A)\ar[r]&\widehat{\mathrm{H}}^{-1}(G,B)\ar[r]&\widehat{\mathrm{H}}^{-1}(G,C)}$$
	$$\xymatrix{\ar[r]^{\delta}&\widehat{\mathrm{H}}^0(G,A)\ar[r]&\widehat{\mathrm{H}}^0(G,B)\ar[r]&\widehat{\mathrm{H}}^0(G,C)\ar[r]^{\delta}&\widehat{\mathrm{H}}^1(G,A)\ar[r]&\cdots}$$
	
	于是$\widehat{\mathrm{H}}^n(G,-)$就是加性函子$A\mapsto A^G/NA$的同调延拓和上同调延拓.
	\begin{proof}
		
		我们有如下交换图表:
		$$\xymatrix{\mathrm{H}_1(G,C)\ar[r]^{\delta_*}\ar[d]&\mathrm{H}_0(G,A)\ar[r]\ar[d]_{\mathrm{N}_A^*}&\mathrm{H}_0(G,B)\ar[r]\ar[d]_{\mathrm{N}_B^*}&\mathrm{H}_0(G,C)\ar[r]\ar[d]_{\mathrm{N}_C^*}&0\ar[d]\\0\ar[r]&\mathrm{H}^0(G,A)\ar[r]&\mathrm{H}^0(G,B)\ar[r]&\mathrm{H}^0(G,C)\ar[r]_{\delta^*}&\mathrm{H}^1(G,A)}$$
		
		按照蛇形引理,这得到如下正合列:
		$$\xymatrix{0\ar[r]&\mathrm{im}\delta_*\ar[r]&\widehat{\mathrm{H}^{-1}}(G,A)\ar[r]&\widehat{\mathrm{H}^{-1}}(G,B)\ar[r]&\widehat{\mathrm{H}^{-1}}(G,C)\ar[r]&\widehat{\mathrm{H}^0}(G,A)}$$
		$$\xymatrix{\ar[r]&\widehat{\mathrm{H}^0}(G,B)\ar[r]&\widehat{\mathrm{H}^0}(G,C)\ar[r]&\mathrm{im}\delta^*\ar[r]&0}$$
		
		按照原本的群同调和群上同调,有如下正合列:
		$$\xymatrix{\cdots\ar[r]&\widehat{\mathrm{H}^{-2}}(G,A)\ar[r]&\widehat{\mathrm{H}^{-2}}(G,B)\ar[r]&\widehat{\mathrm{H}^{-2}}(G,C)\ar[r]&\mathrm{im}\delta_*\ar[r]&0}$$
		$$\xymatrix{0\ar[r]&\mathrm{im}\delta^*\ar[r]&\widehat{\mathrm{H}^1}(G,A)\ar[r]&\widehat{\mathrm{H}^1}(G,B)\ar[r]&\widehat{\mathrm{H}^1}(G,C)\ar[r]&\cdots}$$
		
		它们就可以拼凑成整个长正合列.
	\end{proof}
	\item 对任意$G$模$A$和$B$有$\widehat{\mathrm{H}}^n(G,A\oplus B)=\widehat{\mathrm{H}}^n(G,A)\oplus\widehat{\mathrm{H}}^n(G,B)$.
	\begin{proof}
		
		在$n\ge1$和$n\le-2$时此即$\mathrm{Ext}$函子和有限直和可交换.对于$n=0$和$-1$的情况,这件事可以从$_NA$,$I_GA$,$A^G$,$NA$都和二元直和可交换得到.
	\end{proof}
	\item 如果$A$是相对投射的$G$模(因为这里$G$是有限群,相对投射和相对内射的概念是一致的),那么$\widehat{\mathrm{H}}^n(G,A)=0,\forall n\in\mathbb{Z}$.
	\begin{proof}
		
		因为$A$同时是相对投射和相对内射的,于是当$i\ge1$时有$\mathrm{H}^i(G,A)=0$和$\mathrm{H}_i(G,A)=0$.于是问题归结为证明$\widehat{\mathrm{H}}^0(G,A)=\widehat{\mathrm{H}}^{-1}(G,A)=0$.又因为相对投射是诱导模的直和项,上一条保证了只需验证$A$是诱导模的情况.此时$A=\mathbb{Z}[G]\otimes_{\mathbb{Z}}A_0=\oplus_ggA_0$,其中$A_0$是$A$的某个子群.于是$A$中的元可以唯一的表示为$\sum_gg(a_g)$,其中$a_g\in A_0$.
		
		\qquad
		
		先证$n=0$的情况,也即$A^G=NA$.任取$a=\sum_gg(a_g)\in A$.那么$a\in A^G$当且仅当$a_g$都是相同的元,记作$x$,那么$a=Nx$,于是$a\in NA$.再证$n=-1$的情况,也即$_NA=I_GA$.任取$a=\sum_gg(a_g)\in A$,它落在$_NA$中当且仅当$Na=0$,也即$\sum_{h\in G}\sum_{g\in G}hga_g=0$,换元得到$\sum_ss(\sum_ha_{sh^{-1}})$.但是当$h$跑遍$G$时$sh^{-1}$也跑遍$G$,于是$\sum_ga_g=0$,于是$a\in I_GA$.
	\end{proof}
	\item 设$G$是有限群,一个$G$模$A$称为上同调平凡模,如果对它的任意子群$H$,对任意整数$n$都有$\widehat{\mathrm{H}}^n(H,A)=0$.那么相对投射模(在$G$是有限群时它和相对内射模一致)总是上同调平凡模.
	\begin{proof}
		
		我们之前定义过相对投射模是指诱导模的直和项.并且在$G$是有限群的时候对任意子群$H\le G$,一个诱导$G$模也是诱导$H$模.结合上一条得证.
	\end{proof}
	\item 设$0\to X_1\to\cdots\to X_n\to0$是$G$模的正合列.如果$X_i$中有$n-1$个是上同调平凡$G$模,那么剩余的那个也是.
	\begin{proof}
		
		记$N_i=\ker(X_i\to X_{i+1})$,$N_0=N_{n+1}=0$.那么这个长正合列可以分解为$n+1$个短正合列:
		$$\xymatrix{0\ar[r]&N_i\ar[r]&X_i\ar[r]&N_{i+1}\ar[r]&0},0\le i\le n$$
		设对$i\not=q$都有$X_i$是上同调平凡$G$模.那么从$0\le i\le q-1$这些短正合列得到$N_1,\cdots,N_q$是上同调平凡$G$模;从$q+1\le i\le n$这些短正合列得到$N_{q+1},\cdots,N_n$是上同调平凡的.进而$i=q$这个短正合列得到$X_q$是上同调平凡模.
	\end{proof}
\end{enumerate}
\subsubsection{Tate上同调的限制和余限制}

设$G$是有限群.
\begin{enumerate}
	\item 限制映射.设$H$是$G$的子群,设$A$是$G$模,它可以自然的视为一个$H$模.我们要定义$\mathrm{res}:\widehat{\mathrm{H}}^n(G,A)\to\widehat{\mathrm{H}}^n(H,A)$.对$n\ge1$此即群上同调的res映射;对$n\le-2$此即群同调的res映射.我们补充定义$n=0$和$n=-1$的情况.
	\begin{itemize}
		\item 首先有典范包含映射$A^G\subseteq A^H$,并且$N_GA\subseteq N_HA$,于是包含映射诱导了$A^G/N_GA\to A^H/N_HA$.此为$n=0$的情况.
		\item 我们定义过零阶同调的res同态$A_G\to A_H$为$a\mapsto\sum_{\overline{g}\in G/H}g^{-1}a$.它把$_{N_G}A$中的元映射到$_{N_H}A$中,因为如果$a\in A$满足$\sum_gga=0$,那么有$\sum_{h\in H}\sum_{\overline{g}\in G/H}hg^{-1}a=\sum_{g\in G}ga=0$.于是它诱导了$_{N_G}A/I_GA\to_{N_H}A/I_HA$.此即$n=-1$的情况.
	\end{itemize}
	\item $\mathrm{res}$是上同调$\delta$函子之间的态射,换句话讲它是自然的,并且和长正合列中的连接同态可交换.
	\begin{proof}
		
		设$0\to A\to B\to C\to0$是$G$模的长正合列,我们要证明有如下交换图表:
		$$\xymatrix{\widehat{\mathrm{H}}^n(G,C)\ar[rr]^{\delta}\ar[d]_{\mathrm{res}}&&\widehat{\mathrm{H}}^{n+1}(G,A)\ar[d]^{\mathrm{res}}\\\widehat{\mathrm{H}}^n(H,C)\ar[rr]_{\delta}&&\widehat{\mathrm{H}}^{n+1}(H,C)}$$
		
		这在$n\ge1$和$n\le-2$时已经得证了.对于$n=-1$,图表变成:
		$$\xymatrix{_{N_G}C/I_GC\ar[rr]^{\delta}\ar[d]_{\mathrm{res}}&&A^G/N_GA\ar[d]^{\mathrm{res}}\\_{N_H}C/I_HC\ar[rr]_{\delta}&&A^H/N_HA}$$
		
		任取$x\in _{N_G}C$,它在$_{N_G}C/I_GC$中的像记作$\overline{x}$,那么$\delta{\overline{x}}$是这样得到的:先把$c$提升到$b\in B$,再取$N_G(b)$在$A^G/N_GA$中的像,它在$A^G/N_HA$中的像就是$\mathrm{res}(\delta(\overline{x}))$.另一方面,$\mathrm{res}(\overline{x})$可以表示为$\sum_{i\in I}s_ix$,其中$s_i$是$G/H$右陪集的代表元集,那么$\delta(\mathrm{res}(\overline{x}))$就是$N_H(\sum_is_ib)$在$A^H/N_HA$中的像,也就是$\mathrm{N}_G(b)$在$A^H/N_HA$中的像.
	\end{proof}
	\item 余限制映射.依旧设$H$是$G$的子群,设$A$是$G$模,我们来构造余限制$\mathrm{cor}:\widehat{\mathrm{H}}^n(H,A)\to\widehat{\mathrm{H}}^n(G,A)$.对$n\ge1$此即群上同调的余限制,对$n\le-2$此即群同调的余限制.我们来定义$n=0$和$n=-1$的情况.
	\begin{itemize}
		\item $n=0$:考虑$A^H\to A^G$的映射$a\mapsto\sum_{\overline{g}\in G/H}ga$,这把$N_HA$映射到$N_GA$,于是诱导了$A^H/N_HA\to A^G/N_GA$.
		\item $n=-1$:考虑典范包含映射$_{N_H}A\to_{N_G}A$,它把$I_HA$映入$I_GA$,于是诱导了$_{N_H}A/I_HA\to _{N_G}A/I_GA$.
	\end{itemize}
	\item $\mathrm{cor}$是上同调$\delta$函子之间的态射,换句话讲它是自然的,并且和长正合列中的连接同态可交换.
	\item 设$[G:H]=n$,那么$\mathrm{cor}\circ\mathrm{res}=n$,也即$\widehat{\mathrm{H}}^n(G,A)$上的数乘$n$的同态.
	\begin{proof}
		
		首先$n=0$的时候这是显然的,因为对$a\in A^G$有$\sum_{\overline{g}\in G/H}ga=|G/H|a=na$.接下来由于$\mathrm{cor}\circ\mathrm{res}-n$是上同调$\delta$函子之间的态射,它是唯一的,它在零维情况是零映射,所以任意维上同调上也是零映射.
	\end{proof}
	\item 设$|G|=m$,那么每个$\widehat{\mathrm{H}}^n(G,A)$都被$m$零化.
	\begin{proof}
		
		这是因为如果在上一条中取$H=\{e\}$,那么数乘$m$的映射可以分解为$\mathrm{cor}\circ\mathrm{res}$,但是$A^{\{e\}}/N_{\{e\}}A=\{0\}$,迫使数乘$m$的映射都是零映射.
	\end{proof}
	\item 如果$A$是有限生成阿贝尔群,那么$\widehat{\mathrm{H}}^n(G,A)$是有限群.这是因为有限生成阿贝尔群的挠子群是有限群.
\end{enumerate}
\subsubsection{有限循环群的Tate上同调}

Herbrand商.设$A$是阿贝尔群,设$f,g$是$A$上的自同构,满足$f\circ g=g\circ f=0$,那么有$\mathrm{im}g\subseteq\ker f$和$\mathrm{im}f\subseteq\ker g$.我们定义$(f,g)$的Herbrand商为:
$$\frac{[\ker f:\mathrm{im}g]}{[\ker g:\mathrm{im}f]}$$
\begin{enumerate}
	\item 如果$G$是有限循环群,它的Tate上同调群是周期为2的:
	$$\widehat{\mathrm{H}}^{2n}(G,A)=A^G/\mathrm{N}A,\widehat{\mathrm{H}}^{2n-1}(G,A)=_NA/I_GA$$
	
	此时$G$模的短正合列$\xymatrix{0\ar[r]&A\ar[r]^i&B\ar[r]^j&C\ar[r]&0}$诱导了如下正合六边形,其中$f_3$为$c\mapsto(j^{-1}c)^{\sigma-1}$,$f_6$为$c\mapsto\mathrm{N}(j^{-1}c)$:
	$$\xymatrix{&\widehat{\mathrm{H}}^0(G,A)\ar[rr]^{f_1}&&\widehat{\mathrm{H}}^0(G,B)\ar[dr]^{f_2}&\\\widehat{\mathrm{H}}^1(G,C)\ar[ur]^{f_6}&&&&\widehat{\mathrm{H}}^0(G,C)\ar[dl]^{f_3}\\&\widehat{\mathrm{H}}^1(G,B)\ar[ul]^{f_5}&&\widehat{\mathrm{H}}^1(G,A)\ar[ll]^{f_4}&}$$
	\item 设$G$是有限循环群,设$A$是$G$模,记$G$的生成元是$\sigma$,记$D=\sigma-1$和$N=1+\sigma+\cdots\sigma^{n-1}$,那么有$D\circ N=N\circ D=0$.并且有$\ker D=A^G$,$\mathrm{im}N=\mathrm{N}_GA$,$\ker N=_NA$,$\mathrm{im}D=I_GA$.于是此时$(D,N)$的Herbrand商为$h(A)=|\widehat{\mathrm{H}}^0(G,A)|/|\widehat{\mathrm{H}}^1(G,A)|$.我们断言如果$A$是有限群,那么Herbrand商$h(A)$总是1.
	\begin{proof}
		
		考虑数乘$N$的映射$A\to NA$,它的核就是$_NA$.考虑数乘$D$的映射$A\to DA$,它的核是$A^G$:一方面如果$a\in\ker D$,那么$(x-1)a=0$,于是$ax=a$,归纳得到$ax^i=a,\forall i$,于是$a\in A^G$;另一方面如果$a\in A^G$,那么自然有$xa=a$,得到$(x-1)a=0$.于是$|\widehat{\mathrm{H}}^1(G,A)|=|_NA|/|DA|$和$|\widehat{\mathrm{H}}^0(G,A)|=|A^G|/|NA|$.但是我们证明了$|_NA||NA|=|DA||A^G|=|A|$.
	\end{proof}
	\item 设$G$是有限循环群,给定$G$模的短正合列$0\to A\to B\to C\to0$,那么如果$h(G,A)$,$h(G,B)$,$h(G,C)$中有两个有定义,那么第三个也有定义,并且此时满足$h(G,B)=h(G,A)h(G,C)$.
	\begin{proof}
		
		记$f_i$的像的阶数为$n_i$,其中$1\le i\le6$,那么这个正合六边形可以分解为六个短正合列,于是有如下等式,得到所求等式.
		$$\#\widehat{\mathrm{H}}^0(G,A)=n_6n_1;\#\widehat{\mathrm{H}}^0(G,B)=n_1n_2;\#\widehat{\mathrm{H}}^0(G,C)=n_2n_3$$
		$$\#\widehat{\mathrm{H}}^1(G,A)=n_3n_4;\#\widehat{\mathrm{H}}^1(G,B)=n_4n_5;\#\widehat{\mathrm{H}}^1(G,C)=n_5n_6$$
	\end{proof}
	\item 推论.如果$f:A\to B$是有限循环群$G$上的模之间的同态,如果$\ker f$或者$\mathrm{coker}f$是有限的,那么只要$h(G,A)$和$h(G,B)$之一有定义,另一个就也有定义,并且有$h(G,A)=h(G,B)$.
\end{enumerate}
\subsubsection{cup积}
\begin{enumerate}
	\item 群上同调的情况.设$A,B$是两个$G$模,定义阿贝尔群$A\otimes_{\mathbb{Z}}B$上的$G$模结构为$g(a\otimes b)=ga\otimes gb$.记$\mathrm{C}^n(G,A)$表示齐次余链群,也即所有$G^{n+1}\to A$的满足齐次条件的映射构成的集合.我们构造如下映射:
	$$\mathrm{C}^p(G,A)\times\mathrm{C}^q(G,B)\to\mathrm{C}^{p+q}(G,A\otimes_{\mathbb{Z}}B)$$
	$$(a,b)\mapsto a\cup b$$
	$$(a\cup b)(g_0,g_1,\cdots,g_{p+q})=a(g_0,\cdots,g_p)\otimes b(g_p,\cdots,g_{p+q})$$
	
	我们断言这个映射满足$\mathrm{d}(a\cup b)=(\mathrm{d}a)\cup b+(-1)^pa\cup (\mathrm{d}b)$,于是:
	\begin{itemize}
		\item 如果$a$和$b$都是余圈,那么$a\cup b$也是余圈.
		\item 如果$a=\mathrm{d}a_0$是余边界,$b$是余圈,把$a=a_0$和$b=b$带入,得到$\mathrm{d}(a_0\cup b)=a\cup b$,于是此时$a\cup b$也是余边界.
	\end{itemize}
	
	于是对任意$p,q\ge0$,我们的映射诱导了如下双线性映射,它称为群上同调的cup积.
	$$\mathrm{H}^p(G,A)\times\mathrm{H}^q(G,B)\to\mathrm{H}^{p+q}(G,A\otimes_{\mathbb{Z}}B)$$
	\item 特别的,$p=q=0$时此即映射$A^G\times B^G\to(A\otimes_{\mathbb{Z}}B)^G$,$(a,b)\mapsto a\otimes b$.
	\item Tate上同调的cup积.设$G$是有限群,存在唯一的同态族:
	$$\widehat{\mathrm{H}}^p(G,A)\otimes_{\mathbb{Z}}\widehat{\mathrm{H}}^q(G,B)\to\widehat{\mathrm{H}}^{p+q}(G,A\otimes_{\mathbb{Z}}B)$$
	
	其中$p,q$取遍$\ge0$的整数,$A,B$取遍$G$模.满足如下四个条件:
	\begin{enumerate}
		\item 固定$p,q$时,这些同态构成双函子之间的自然变换.
		\item $p=q=0$时cup积取为$A^G\otimes_{\mathbb{Z}}B^G\to(A\otimes_{\mathbb{Z}}B)^G$诱导的商映射.
		\item 如果$0\to A\to A'\to A''\to0$是$G$模的短正合列,如果张量$G$模$B$后仍然是短正合列,那么有如下交换图表:
		$$\xymatrix{\widehat{\mathrm{H}}^p(G,A'')\otimes_{\mathbb{Z}}\widehat{\mathrm{H}}^q(G,B)\ar[rr]^{\mathrm{cup}}\ar[d]_{\delta\otimes1}&&\widehat{\mathrm{H}}^{p+q}(G,A''\otimes_{\mathbb{Z}}B)\ar[d]^{\delta}\\\widehat{\mathrm{H}}^{p+1}(G,A)\otimes_{\mathbb{Z}}\widehat{\mathrm{H}}^q(G,B)\ar[rr]^{\mathrm{cup}}&&\widehat{\mathrm{H}}^{p+q+1}(G,A\otimes_{\mathbb{Z}}B)}$$
		\item 如果$0\to B\to B'\to B''\to0$是$G$模的短正合列,如果张量$G$模$A$后仍然是短正合列,那么有如下交换图表:
		$$\xymatrix{\widehat{\mathrm{H}}^p(G,A)\otimes_{\mathbb{Z}}\widehat{\mathrm{H}}^q(G,B'')\ar[rr]^{\mathrm{cup}}\ar[d]_{1\otimes\delta}&&\widehat{\mathrm{H}}^{p+q}(G,A\otimes_{\mathbb{Z}}B'')\ar[d]^{(-1)^p\delta}\\\widehat{\mathrm{H}}^p(G,A)\otimes_{\mathbb{Z}}\widehat{\mathrm{H}}^{q+1}(G,B)\ar[rr]^{\mathrm{cup}}&&\widehat{\mathrm{H}}^{p+q+1}(G,A\otimes_{\mathbb{Z}}B)}$$
	\end{enumerate}
	
	【下面几条的证明都是dimension shift】
	\item 结合律:
	$$\xymatrix{\left(\widehat{\mathrm{H}}^p(G,A)\otimes_{\mathbb{Z}}\widehat{\mathrm{H}}^q(G,B)\right)\otimes_{\mathbb{Z}}\widehat{\mathrm{H}}^r(G,C)\ar[rr]^{\cong}\ar[d]_{\mathrm{cup}\otimes1}&&\widehat{\mathrm{H}}^p(G,A)\otimes_{\mathbb{Z}}\left(\widehat{\mathrm{H}}^q(G,B)\otimes_{\mathbb{Z}}\widehat{\mathrm{H}}^r(G,C)\right)\ar[d]^{1\otimes\mathrm{cup}}\\\widehat{\mathrm{H}}^{p+q}(G,A\otimes_{\mathbb{Z}}B)\otimes_{\mathbb{Z}}\widehat{\mathrm{H}}^r(G,C)\ar[d]_{\mathrm{cup}}&&\widehat{\mathrm{H}}^p(G,A)\otimes_{\mathbb{Z}}\widehat{\mathrm{H}}^{q+r}(G,B\otimes_{\mathbb{Z}}C)\ar[d]^{\mathrm{cup}}\\\widehat{\mathrm{H}}^{p+q+r}(G,A\otimes_{\mathbb{Z}}B\otimes_{\mathbb{Z}}C)\ar[rr]^{\mathrm{id}}&&\widehat{\mathrm{H}}^{p+q+r}(G,A\otimes_{\mathbb{Z}}B\otimes_{\mathbb{Z}}C)}$$
	\item 交换律:对$\overline{a}\in\widehat{\mathrm{H}}^p(G,A)$和$\overline{b}\in\widehat{\mathrm{H}}^q(G,B)$,有:
	$$\overline{a}\cup\overline{b}=(-1)^{pq}(\overline{b}\cup\overline{a})$$
	\item 考虑Tate上同调.设$A,B$都是$G$模,如果$a_p$是$A$上的一个$p$余圈,$b_q$是$B$上的一个$q$余圈,用$\overline{a_p}$和$\overline{b_q}$表示它们所在的上同调类,那么有:
	$$\overline{a_0}\cup\overline{b_q}=\overline{a_0\otimes b_q};\overline{a_p}\cup\overline{b_0}=\overline{a_p\otimes b_0}$$
	\item 设$f:A\to A'$和$g:B\to B'$是$G$模同态,对$\overline{a}\in\widehat{\mathrm{H}}^p(G,A)$和$\overline{b}\in\widehat{\mathrm{H}}^q(G,B)$,我们有:
	$$f^*(\overline{a})\cup g^*(\overline{b})=(f\otimes g)^*(\overline{a}\cup\overline{b})\in\widehat{\mathrm{H}}^{p+q}(G,A'\otimes B')$$
	\item 设$A,B$是$G$模,设$H\subseteq G$是子群,设$\overline{a}\in\widehat{\mathrm{H}}^p(G,A)$和$\overline{b}\in\widehat{\mathrm{H}}^q(G,B)$,那么有:
	$$\mathrm{res}(\overline{a}\cup\overline{b})=\mathrm{res}(\overline{a})\cup\mathrm{res}(\overline{b})\in\widehat{\mathrm{H}}^{p+q}(H,A\otimes B)$$
	
	设$\overline{a}\in\widehat{\mathrm{H}}^p(G,A)$和$\overline{b}\in\widehat{\mathrm{H}}^q(H,B)$,那么有:
	$$\mathrm{cor}((\mathrm{res}\overline{a})\cup\overline{b})=\overline{a}\cup\mathrm{cor}(\overline{b})\in\widehat{\mathrm{H}}^{p+q}(G,A\otimes B)$$
	\begin{proof}
		
		按照dimension shift只需验证$p=q=0$的情况.第一条是平凡的,至于第二条设$a\in A^G$和$b\in B^H$,那么有:
		\begin{align*}
			\mathrm{cor}(\mathrm{res}(\overline{a}\cup\overline{b}))&=\mathrm{cor}(a\otimes b+\mathrm{N}_H(A\otimes B))\\&=\sum_{\overline{g}\in G/H}g(a\otimes b)+\mathrm{N}_G(A\otimes B)\\&=\sum_{\overline{g}\in G/H}a\otimes gb+\mathrm{N}_G(A\otimes B)\\&=\overline{a}\cup\mathrm{cor}(\overline{b})
		\end{align*}
	\end{proof}
\end{enumerate}
\subsubsection{$p$群的上同调}
\begin{enumerate}
	\item 设$G$是$p$群(我们总指有限$p$群),设$A$是$G$模,满足$pA=0$,那么如下命题互相等价:
	\begin{enumerate}[(a)]
		\item $A=0$.
		\item $\mathrm{H}^0(G,A)=0$.
		\item $\mathrm{H}_0(G,A)=0$.
	\end{enumerate}
	\begin{proof}
		
		(c)推(a):这件事是因为此时$I_G$是幂零理想,以及$A=I_GA$.
	\end{proof}
	\item 设$G$是$p$群,设$A$是$G$模,满足$pA=0$,设$\mathrm{H}_1(G,A)=0$,那么$A$是$\mathbb{F}_p[G]$上的自由模.
	\begin{proof}
		
		我们有$\mathrm{H}_0(G,A)=A/I_GA$是$\mathbb{F}_p$线性空间.取一组基$\{x_i\}$,把它们提升为一组$\{a_i\}\subseteq A$.设$\{a_i\}$生成了子模$A_1\subseteq A$,记$A'=A/A_1$,那么从$A'/I_GA'=0$,上一条说明$A_1=A$,也即$\{a_i\}$生成了整个$A$.于是存在满同态$L=\oplus a_i\mathbb{F}_p[G]\to A$,把核记作$K$.这个同态诱导的$L/I_GL\to A/I_GA$是同构,考虑如下正合列:
		$$\xymatrix{0\ar@{=}[r]&\mathrm{H}_1(G,A)\ar[r]&\mathrm{H}_0(G,K)\ar[r]&\mathrm{H}_0(G,L)\ar[r]^{\cong}&\mathrm{H}_0(G,A)}$$
		进而有$\mathrm{H}_0(G,K)=0$,进而上一条说明$K=0$.
	\end{proof}
	\item 设$G$是$p$群,设$A$是$G$模,满足$pA=0$,那么如下命题互相等价:
	\begin{enumerate}[(a)]
		\item 存在一个整数$q$使得$\widehat{\mathrm{H}}^q(G,A)=0$.
		\item $A$是上同调平凡$G$模.
		\item $A$是诱导$G$模.
		\item $A$是自由$\mathbb{F}_p[G]$模.
	\end{enumerate}
	\begin{proof}
		
		明显有(d)推(c)推(b)推(a).只需证明(a)推(d):反复取短正合列:
		$$\xymatrix{0\ar[r]&A_{i+1}\ar[r]&I_i\ar[r]&A_i\ar[r]&0}$$
		其中$I_i$都是诱导$G$模,$A_0=A$,并且这里总有$pA_i=0$,记$B=A_{q+1}$,那么对任意整数$n$就有$\widehat{\mathrm{H}}^n(G,A)=\widehat{\mathrm{H}}^{n-q-2}(G,B)$.于是特别的有$\mathrm{H}_1(G,B)=0$,进而上一条得到$B$是$\mathbb{F}_p[G]$上的自由模,于是它的全部Tate上同调为零,特别的有$\mathrm{H}_1(G,A)=\widehat{\mathrm{H}}^{-2}(G,A)=\widehat{\mathrm{H}}^{-q-4}(G,B)=0$.再用一次上一条结论就得到$A$是自由$\mathbb{F}_p[G]$模.
	\end{proof}
	\item 设$G$是$p$群,设$A$是$G$模,并且$p$-扭部分平凡(此即阶数为$p$的元$a\in A$只有0).那么如下命题互相等价:
	\begin{enumerate}[(a)]
		\item $\widehat{\mathrm{H}}^q(G,A)=0$对某两个相邻正整数成立.
		\item $A$是上同调平凡模.
		\item $\mathbb{F}_p[G]$模$A/pA$是自由的.
	\end{enumerate}
	\begin{proof}
		
		(a)推(c):按照$A$的$p$-扭部分平凡,有短正合列$\xymatrix{0\ar[r]&A\ar[r]^p&A\ar[r]&A/pA\ar[r]&0}$,进而有如下长正合列:
		$$\xymatrix{\widehat{\mathrm{H}}^q(G,A)\ar[r]^p&\widehat{\mathrm{H}}^q(G,A)\ar[r]&\widehat{\mathrm{H}}^q(G,A/pA)\ar[r]&\widehat{\mathrm{H}}^{q+1}(G,A)\ar[r]^p&\widehat{\mathrm{H}}^{q+1}(G,A)}$$
		设$\widehat{\mathrm{H}}^q(G,A)=\widehat{\mathrm{H}}^{q+1}(G,A)=0$,那么这个正合列得到$\widehat{\mathrm{H}}^q(G,A/pA)=0$.于是上一条得到$A/pA$是自由模.
		
		(c)推(b):此时对任意$q$有$\widehat{\mathrm{H}}^q(G,A/pA)=0$,于是数乘$p$诱导的$\widehat{\mathrm{H}}^q(G,A)$上的自同态总是同构.但是按照$G$是有限$p$群,这个自同态是幂零的,迫使对任意$q$有$\widehat{\mathrm{H}}^q(G,A)=0$.这件事对$G$的任意子群$H$都成立,于是$A$是上同调平凡模.最后(b)推(a)是平凡的.
	\end{proof}
\end{enumerate}
\subsubsection{有限群的上同调}

设$G$是有限群.
\begin{enumerate}
	\item 设$p$是素数,设$G_p$是$G$的一个Sylow $p$子群.那么对任意$G$模$A$和任意整数$n$,限制同态:
	$$\mathrm{res}:\widehat{\mathrm{H}}^n(G,A)\to\widehat{\mathrm{H}}^n(G_p,A)$$
	限制在准素部分$\widehat{\mathrm{H}}^n(G,A)(p)$上(此为阶数是$p$次幂的元构成的子群)是单射.
	\begin{proof}
		
		任取$x\in\ker(\mathrm{res})$,记$[G:G_p]=q$,则$qx=\mathrm{cor}\circ\mathrm{res}(x)=0$.倘若$x$的阶数是$p^r$,那么从$(q,p)=1$得到$x=0$.
	\end{proof}
	\item 推论.设$A$是$G$模,设$n$是一个正整数,如果对任意素数$p$都有$\widehat{\mathrm{H}}^n(G_p,A)=0$,这里$G_p$是$G$的某个Sylow $p$子群,那么有$\widehat{\mathrm{H}}^n(G,A)=0$.
	\begin{proof}
		
		这是因为$\widehat{\mathrm{H}}^n(G,A)$限制在任意$p$准素部分都是零(从$G$有限得到这个上同调群是扭群,每个元的阶数都有限,所以准素部分都为零得到它是零群).
	\end{proof}
	\item 引理.设$G$是$p$群,设$A$是一个$\mathbb{Z}$自由的$G$模,满足$G$模$A$是上同调平凡模.那么对任意无扭$G$模$B$,有$G$模$N=\mathrm{Hom}_{\mathbb{Z}}(A,B)$是上同调平凡的.另外当我们证明了下一条后,这里$A$是投射$G$模,进而我们证明过此时$N$甚至是相对投射模.
	\begin{proof}
		
		$N$是无挠$G$模,此时$N$是上同调平凡模等价于验证$N/pN$是上同调平凡模.为此考虑数乘$p$诱导的短正合列:
		$$\xymatrix{0\ar[r]&B\ar[r]^p&B\ar[r]&B/pB\ar[r]&0}$$
		按照$A$是自由交换群,于是有如下短正合列:
		$$\xymatrix{0\ar[r]&N\ar[r]^p&N\ar[r]&\mathrm{Hom}_{\mathbb{Z}}(A,B/pB)\ar[r]&0}$$
		进而有$N/pN=\mathrm{Hom}_{\mathbb{Z}}(A/pA,B/pB)$.我们解释过$G$模$A$是上同调平凡模等价于$A/pA$是自由$\mathbb{F}_p[G]$模,特别的它是诱导$G$模,我们之前解释过此时$\mathrm{Hom}_{\mathbb{Z}}(A/pA,B/pB)$是上同调平凡模.
	\end{proof}
	\item 设$A$是一个$\mathbb{Z}$自由的$G$模,对任意素数$p$取一个$G$的Sylow $p$子群$G_p$.那么如下命题互相等价:
	\begin{enumerate}[(a)]
		\item 对任意素数$p$有$G_p$模$A$是上同调平凡模.
		\item $A$是$\mathbb{Z}[G]$投射模.
	\end{enumerate}
	\begin{proof}
		
		(b)推(a):对$G$的任意子群$H$,有$\mathbb{Z}[G]$是自由$\mathbb{Z}[H]$(左)模,结合投射模是自由模的直和项,就得到$A$也是每个$G_p$投射模,特别的每个$G_p$模$A$是上同调平凡模(因为投射模是相对投射模).
		
		(a)推(b):把$A$写做某个自由$G$模$L$的商:
		$$\xymatrix{0\ar[r]&N\ar[r]&L\ar[r]&A\ar[r]&0}$$
		按照$A$是自由$\mathbb{Z}$模,我们有短正合列:
		$$\xymatrix{0\ar[r]&\mathrm{Hom}_{\mathbb{Z}}(A,N)\ar[r]&\mathrm{Hom}_{\mathbb{Z}}(A,L)\ar[r]&\mathrm{Hom}_{\mathbb{Z}}(A,A)\ar[r]&0}$$
		这里$N$作为自由交换群的子群仍然是自由的,特别的它是无扭的,于是上一条引理得到$G_p$模$\mathrm{Hom}_{\mathbb{Z}}(A,N)$是上同调平凡模.进而按照上面推论有$\mathrm{H}^1(G,\mathrm{Hom}_{\mathbb{Z}}(A,N))=0$.进而我们有满同态:
		$$\mathrm{Hom}_G(A,L)\to\mathrm{Hom}_G(A,A)$$
		特别的,$1_A$延拓为一个$G$同态$A\to L$,进而$A$是自由$G$模$L$的直和项,于是$A$是投射$G$模.
	\end{proof}
	\item 设$A$是$G$模,那么如下命题互相等价:
	\begin{enumerate}[(a)]
		\item 对任意素数$p$,有$\widehat{\mathrm{H}}^q(G_p,A)=0$对某两个相邻整数$q$成立(这个$q$依赖于$p$).
		\item $A$是上同调平凡$G$模.
		\item $A$的投射维数有限,也即存在自然数$k$以及$A$的一个$G$模上的有限投射预解:
		$$\xymatrix{0\ar[r]&P_k\ar[r]&P_{k-1}\ar[r]&\cdots\ar[r]&P_0\ar[r]&A\ar[r]&0}$$
		\item $A$的投射维数$\le1$,此即存在$A$的如下$G$模上的投射预解:
		$$\xymatrix{0\ar[r]&P_1\ar[r]&P_0\ar[r]&A\ar[r]&0}$$
	\end{enumerate}
	\begin{proof}
		
		(d)推(c)平凡,(c)推(b)是因为这样的正合列的中间项里$k+1$项是上同调平凡$G$模就推出剩余的那项也是上同调平凡$G$模.(b)推(a)也平凡.最后(a)推(d):取自由$G$模$L$到$A$的满同态,得到短正合列:
		$$\xymatrix{0\ar[r]&K\ar[r]&L\ar[r]&A\ar[r]&0}$$
		这里$L$是自由$G$模,而$\mathbb{Z}[G]$是自由交换群,于是$L$也是自由交换群,进而$K$也是自由交换群.于是按照上一条得到$K$是投射$G$模.
	\end{proof}
	\item 设$A,B$是$G$模,其中$A$是上同调平凡的.那么$A\otimes_{\mathbb{Z}}B$或者$\mathrm{Hom}_{\mathbb{Z}}(A,B)$或者$\mathrm{Hom}_{\mathbb{Z}}(B,A)$是上同调平凡$G$模,当且仅当$\mathrm{Tor}_1^{\mathbb{Z}}(A,B)$或者$\mathrm{Ext}_{\mathbb{Z}}^1(A,B)$或者$\mathrm{Ext}_{\mathbb{Z}}^1(B,A)$是上同调平凡$G$模.
	\begin{proof}
		
		按照上一条可取$A$的投射预解$0\to P_1\to P_0\to A\to0$,于是有长正合列:
		$$\xymatrix{0\ar[r]&\mathrm{Tor}^{\mathbb{Z}}_1(A,B)\ar[r]&P_1\otimes_{\mathbb{Z}}B\ar[r]&P_0\otimes_{\mathbb{Z}}B\ar[r]&A\otimes_{\mathbb{Z}}B\ar[r]&0}$$
		由于$P_0,P_1$是投射$G$模,所以这里$P_1\otimes B$和$P_0\otimes B$是上同调平凡模,进而按照上面正合列就有$\mathrm{Tor}(A,B)$是上同调平凡模当且仅当$A\otimes B$是上同调平凡模.$\mathrm{Hom}$的情况是类似的.
	\end{proof}
	\item 推论.如果$A$是上同调平凡$G$模,如果$A$或者$B$是无扭模,那么$A\otimes_{\mathbb{Z}}B$是上同调平凡模.
	\begin{proof}
		
		因为此时$\mathrm{Tor}^{\mathbb{Z}}_1(A,B)=0$再结合上一条.
	\end{proof}
\end{enumerate}
\subsubsection{有限群的同调}

设$G$是有限群.
\begin{enumerate}
	\item 设$G$模$A$是内射交换群,那么$A$是上同调平凡$G$模当且仅当$A$是$G$内射模.
	\begin{proof}
		
		充分性平凡.必要性:把$A$嵌入到$G$内射模$I$中,于是有短正合列:
		$$\xymatrix{0\ar[r]&A\ar[r]&I\ar[r]&R\ar[r]&0}$$
		按照$A$是内射交换群,就有短正合列:
		$$\xymatrix{0\ar[r]&\mathrm{Hom}_{\mathbb{Z}}(R,A)\ar[r]&\mathrm{Hom}_{\mathbb{Z}}(I,A)\ar[r]&\mathrm{Hom}_{\mathbb{Z}}(A,A)\ar[r]&0}$$
		按照$A$是上同调平凡的,以及$\mathrm{Hom}_{\mathbb{Z}}^1(R,A)=0$,于是$\mathrm{Hom}_{\mathbb{Z}}(R,A)$是上同调平凡$G$模.特别的$\mathrm{H}^1(G,\mathrm{Hom}_{\mathbb{Z}}(R,A))=0$,进而有满同态:
		$$\mathrm{Hom}_G(I,A)\to\mathrm{Hom}_G(A,A)$$
		特别的$1_A$可以延拓为一个$G$同态$I\to A$,于是$A$是内射$G$模$I$的直和项,进而它是内射模.
	\end{proof}
	\item 一个$G$模$A$是上同调平凡模,当且仅当它的内射维数有限,当且仅当它的内射维数$\le1$.
	\begin{proof}
		
		设$A$是上同调平凡$G$模,把它嵌入到一个内射$G$模$I_0$中,诱导了如下短正合列:
		$$\xymatrix{0\ar[r]&A\ar[r]&I_0\ar[r]&R\ar[r]&0}$$
		按照$A$是上同调平凡$G$模,得到$R$也是.我们知道$I_0$是内射$G$模得到它是内射(可除)交换群,进而商$R$也是内射(可除)交换群,于是上一条得到$R$是内射$G$模.
	\end{proof}
\end{enumerate}
\subsubsection{Tate-Nakayama定理}
\begin{enumerate}
	\item 比较定理.设$G$是有限群,设$f:A'\to A$是$G$同态.对任意素数$p$,取一个Sylow $p$子群$G_p\subseteq G$.设存在整数$n_p$使得同态:
	$$f_i^*:\widehat{\mathrm{H}}^i(G_p,A')\to\widehat{\mathrm{H}}^i(G_p,A)$$
	在$i=n_p$时是满射,在$i=n_p+1$时是双射,在$i=n_p+2$时是单射.再设$B$是$G$模,满足$\mathrm{Tor}^{\mathbb{Z}}_1(A,B)=\mathrm{Tor}^{\mathbb{Z}}_1(A',B)=0$.那么典范同态:
	$$\widehat{\mathrm{H}}^i(H,A'\otimes_{\mathbb{Z}}B)\to\widehat{\mathrm{H}}^i(H,A\otimes_{\mathbb{Z}}B)$$
	对任意子群$H\subseteq G$和任意整数$i$都是同构.特别的,此时
	$$\widehat{\mathrm{H}}^i(H,A')\to\widehat{\mathrm{H}}^i(H,A)$$
	对任意子群$H\subseteq G$和任意整数$i$都是同构.
	\begin{proof}
		
		记$\overline{A'}=\mathrm{Ind}^G(A')$,记典范$G$嵌入$i:A'\to\overline{A'}$.再记$A^*=A\oplus\overline{A'}$.记$G$单同态$\theta=(f,i):A'\to A^*$.再记$\mathrm{coker}\theta=A''$.于是有短正合列:$$\xymatrix{0\ar[r]&A'\ar[r]^{\theta}&A^*\ar[r]&A''\ar[r]&0}$$
		按照$\overline{A'}$是上同调平凡$G$模,就有$A$的上同调和$A^*$的上同调一致.条件告诉我们对$q=n_p,n_p+1$有:$$\widehat{\mathrm{H}}^q(G_p,A'')=0$$
		我们解释过此时有$A''$是上同调平凡$G$模.另外按照$A'$是交换群$\overline{A'}$的直和项,于是$A'$也是交换群$A^*$的直和项.但是这里交换群$A^*$是$A$和若干$A'$的直和,于是从条件得到$\mathrm{Tor}_1^{\mathbb{Z}}(A^*,B)=0$,进而有$\mathrm{Tor}_1^{\mathbb{Z}}(A'',B)=0$.进而有$A''\otimes_{\mathbb{Z}}B$是上同调平凡$G$模.考虑如下短正合列:
		$$\xymatrix{0\ar[r]&A'\otimes_{\mathbb{Z}}B\ar[r]&A^*\otimes_{\mathbb{Z}}B\ar[r]&A''\otimes_{\mathbb{Z}}B\ar[r]&0}$$
		就得到典范同构:
		$$\widehat{\mathrm{H}}^i(H,A'\otimes_{\mathbb{Z}}B)\cong\widehat{\mathrm{H}}^i(H,A^*\otimes_{\mathbb{Z}}B)$$
		我们还有典范同构:
		$$\widehat{\mathrm{H}}^i(H,A^*\otimes_{\mathbb{Z}}B)\cong\widehat{\mathrm{H}}^i(H,A\otimes_{\mathbb{Z}}B)$$
		得证.
	\end{proof}
	\item 设$G$是有限群,设$A,B,C$是给定的$G$模,设$\varphi:A\times B\to C$是$G$双线性同态.取整数$q$,取$a\in\widehat{\mathrm{H}}^q(G,A)$,取子群$H\le G$,记$a_H=\mathrm{res}_{G/H}(a)$,取$G$模$D$,考虑如下cup积和双线性$G$同态的复合:
	$$\widehat{\mathrm{H}}^q(H,A)\otimes_{\mathbb{Z}}\widehat{\mathrm{H}}^n(H,B\otimes_{\mathbb{Z}}D)\to\widehat{\mathrm{H}}^{n+q}(H,A\otimes_{\mathbb{Z}}B\otimes_{\mathbb{Z}}D)\to\widehat{\mathrm{H}}^{n+q}(H,C\otimes_{\mathbb{Z}}D)$$
	那么对$a_H$做cup积诱导了如下同态:
	$$f(n,H,D):\widehat{\mathrm{H}}^n(H,B\otimes_{\mathbb{Z}}D)\to\widehat{\mathrm{H}}^{n+q}(H,C\otimes_{\mathbb{Z}}D)$$
	如果对任意素数$p$和任意Sylow $p$子群$G_p\le G$,存在整数$n_p$使得:
	$$f(n,G_p,\mathbb{Z})\left\{\begin{array}{cc}\text{是满射}&n=n_p\\\text{是双射}&n=n_p+1\\\text{是单射}&n=n_p+2\end{array}\right.$$
	那么对任意$n$,任意子群$H$和任意满足如下条件的$G$模$D$,都有$f(n,H,D)$是双射.
	$$\mathrm{Tor}^{\mathbb{Z}}_1(B,D)=\mathrm{Tor}^{\mathbb{Z}}_1(C,D)=0$$
	\begin{proof}
		
		先证明$q=0$的情况.$a\in\widehat{\mathrm{H}}^0(G,A)=A^G$.记$G$同态$f:B\to C$为$b\mapsto\varphi(a,b)$,那么此时$f(n,H,D)$就是被$f\otimes1:B\otimes D\to C\otimes D$诱导的上同调群同态.此时问题归结为上一条比较定理.
		
		再设$q\ge1$,假设$q-1$的时候成立.把$A$嵌入到$\overline{A}=\mathrm{Ind}^G(A)$,记$A_1=\overline{A}/A$.类似定义$C_1=\overline{C}/C$.取$\varphi_1:A_1\times B\to C_1$.那么$a\in\widehat{\mathrm{H}}^q(G,A)$可以表示为$a=\delta(a_1)$,其中$a_1\in\widehat{\mathrm{H}}^{q-1}(G,A_1)$.那么这个$a_1$定义了同构:
		$$f_1(n,H,D)=\widehat{\mathrm{H}}^n(H,B\otimes D)\to\widehat{\mathrm{H}}^{n+q-1}(H,C_1\otimes D)$$
		把它复合上如下同构:
		$$\delta:\widehat{\mathrm{H}}^{n+q-1}(H,C_1\otimes D)\to\widehat{\mathrm{H}}^{n+q}(H,C\otimes D)$$
		就是$f(n,H,D)$,所以这是同构.
	\end{proof}
	\item (Tate-Nakayama定理)设$G$是有限群,设$A$是$G$模,设$a\in\mathrm{H}^2(G,A)$.对任意素数$p$取Sylow $p$子群$G_p\le G$.再设:
	\begin{enumerate}[(1)]
		\item $\mathrm{H}^1(G_p,A)=0$.
		\item $\mathrm{H}^2(G_p,A)$被$\mathrm{res}_{G/G_p}(a)$生成,并且阶数$=|G_p|$.
	\end{enumerate}
	那么对任意整数$n$,任意子群$H\le G$,任意满足$\mathrm{Tor}^{\mathbb{Z}}_1(A,D)=0$的$G$模$D$,和$a_H=\mathrm{res}_{G/H}(a)$做cup积诱导的如下同态是一个同构:
	$$\widehat{\mathrm{H}}^n(H,D)\to\widehat{\mathrm{H}}^{n+2}(g,A\otimes D)$$
	\begin{proof}
		
		在上一条中取$B=\mathbb{Z}$,$C=A$,$q=2$和$\varphi:A\times\mathbb{Z}\to A$是$(x,n)\mapsto nx$.取$n_p=-1$,验证条件:
		\begin{itemize}
			\item $f(-1,G_p,\mathbb{Z}):\widehat{\mathrm{H}}^{-1}(G_p,\mathbb{Z})\to\widehat{\mathrm{H}}^1(G_p,A)$是满射是因为(1).
			\item $f(0,G_p,\mathbb{Z}):\widehat{\mathrm{H}}^0(G_p,\mathbb{Z})\to\widehat{\mathrm{H}}^2(G_p,A)$是同构是因为(2).
			\item $f(1,G_p,\mathbb{Z}):\widehat{\mathrm{H}}^1(G_p,\mathbb{Z})\to\widehat{\mathrm{H}}^3(G_p,A)$是单射是因为$\mathrm{H}^1(G_p,\mathbb{Z})=0$.
		\end{itemize}
	\end{proof}
\end{enumerate}

\newpage
\subsection{射影有限群的上同调}
\subsubsection{离散$G$模}

\begin{enumerate}
	\item 设$G$是射影有限群,称$G$连续作用在离散交换群$A$上,如果$G\times A\to A$是连续的.这个条件等价于讲$A$的每个元的稳定子都是$G$的开子群.我们称此时$A$是离散$G$模,换句话讲,一个$G$模$A$是离散的当且仅当$A$的每个元的稳定子都是$G$的开子群.离散$G$模构成的范畴记作$\textbf{DMod}(G)$.
	\begin{proof}
		
		一方面如果$G\times A\to A$连续,任取$a\in A$,那么$G\to A$,$g\mapsto ga$是连续的,于是开子集$\{a\}$的原像$G_a=\{g\in G\mid ga=a\}$是开子群.反过来如果$\{G_a\mid a\in A\}$都是开子群.设$ga=a'$,那么$gG_a$是$G$的开子集,满足$gG_a\times\{a\}$的像是$a'$,于是$G\times A\to A$连续.
	\end{proof}
	\item 离散$G$模的上同调.这个上同调依旧可以定义为函子$\textbf{DMod}(G)\to\textbf{Ab}$,$A\mapsto A^G$的右导出函子列.但是此时计算上同调就有困难:如果$G$是无穷阶射影有限群,$\mathbb{Z}[G]$作为$G$模就不是离散的,因为稳定子是平凡子群.事实上此时$\textbf{DMod}(G)$甚至没有足够多的投射对象.我们用链的语言定义:对离散$G$模$A$
	\begin{enumerate}[(1)]
		\item 定义$n$余链集合$C^n_{c}(G,A)$是连续映射(也即局部常值映射,鉴于$A$离散)$G^n\to A$构成的集合.
		\item 定义边界映射$C^n(G,A)\to C^{n+1}_c(G,A)$和之前一样.
		\item 定义上同调群是复形$(C^n_c(G,A))$的上同调群.
	\end{enumerate}
	由此定义的上同调列的确是一个上同调$\delta$函子.并且有$\mathrm{H}^0(G,A)=A^G$.任取离散$G$模$A$,任取$\textbf{DMod}(G)$中的内射对象$I$和一个单射$A\to I$,那么对任意开子群$U\subseteq G$,$I^U$总是$\textbf{Mod}(G/U)$的内射对象,进而下一条得到:
	$$\mathrm{H}^q(G,I)=\varinjlim\mathrm{H}^q(G/U,I^U)=0$$
	也即这个上同调列是effaceable的,于是这个上同调列的确也是$\textbf{DMod}(G)\to\textbf{Ab}$,$A\mapsto A^G$的右导出函子列.
	\item 设$\{G_i\}$是射影有限群的射影系统,设$\{A_i\}$是离散$G_i$模构成的归纳系统,那么对任意$q\ge0$就有:
	$$\mathrm{H}^q(G,A)=\varinjlim\mathrm{H}^q(G_i,A_i)$$
	\begin{proof}
		
		问题归结为证明如下典范同态是同构:
		$$u_q:\varinjlim C^q_c(G_i,A_i)\to C^q_c(G,A)$$
	\end{proof}
	\item 推论.
	\begin{enumerate}[(1)]
		\item 如果$A$是离散$G$模,那么对任意$q\ge0$有:
		$$\mathrm{H}^q(G,A)=\varinjlim\mathrm{H}^q(G/U,A^U)$$
		其中$U$跑遍$G$的开正规子群.
		\item 如果$A$是离散$G$模,那么对任意$q\ge0$有:
		$$\mathrm{H}^q(G,A)=\varinjlim\mathrm{H}^q(G,B)$$
		其中$B$跑遍$G$模$A$的有限生成子模.
		\item 对$q\ge1$,总有$\mathrm{H}^q(G,A)$是扭群.
	\end{enumerate}
	\item 限制和余限制.设$G$是射影有限群,设$H$是闭子群,依旧可以定义限制映射:
	$$\mathrm{res}:\mathrm{H}^q(G,A)\to\mathrm{H}^q(H,A)$$
	和余限制映射:
	$$\mathrm{cor}:\mathrm{H}^q(H,A)\to\mathrm{H}^q(G,A)$$
	如果$H\le G$是指数为$n$的开子群,那么依旧有:
	$$\mathrm{cor}\circ\mathrm{res}=n$$
	\item 余诱导模.设$H$是射影有限群$G$的闭子群,设$A$是离散$H$模,那么依旧可以定义余诱导模$\mathrm{coInd}^H_G(A)$.它是全部$G\to A$的连续映射$f$,满足对任意$g\in G$和$h\in H$有$f(hg)=hf(g)$.但是诱导模没法定义,理由依旧是$G$是无限阶射影有限群时$\mathbb{Z}[G]$不再是离散$G$模.此时依旧有Shapiro定理:典范映射$\mathrm{coInd}^G_G(A)\to A$,$f\mapsto f(1_G)$是$H\to G$同态,进而诱导了同态$\mathrm{H}^q(G,\mathrm{coInd}^H_G(A))\to\mathrm{H}^q(H,A)$,那么这是一个同构.
	\begin{enumerate}[(1)]
		\item 特别的,取$H=\{1\}$,得到余诱导模$\mathrm{coInd}_G$的$\ge1$次上同调都为零.
		\item 如果$A$是离散$G$模,那么限制映射就是$G$模同态$\mathrm{coInd}_G(A)\to A$诱导的上同调同态:
		$$\mathrm{res}:\mathrm{H}^q(G,A)\to\mathrm{H}^q(G,\mathrm{coInd}_G^H(A))=\mathrm{H}^q(H,A)$$
		\item 设$H\le G$是开子群,设$A$是离散$G$模,那么有典范满同态$\pi:\mathrm{coInd}_G^H(A)\to A$为$f\mapsto\sum_{\overline{g}\in G/H}\overline{g}f(g^{-1})$,它诱导的上同调同态就是余限制:
		$$\mathrm{cor}:\mathrm{H}^q(H,A)=\mathrm{H}^q(G,\mathrm{coInd}_G^H(A))\to\mathrm{H}^q(G,A)$$
	\end{enumerate}
	\item 对正规闭子群$H\le G$,有Hochschild–Serre谱序列依旧成立:设$H$是射影有限群$G$的闭正规子群,设$A$是离散$G$模,那么$G/H$连续作用在$\mathrm{H}^q(H,A)$上.进而有谱序列:
	$$\mathrm{H}^p(G/H,\mathrm{H}^q(H,A))\Rightarrow\mathrm{H}^{p+q}(G,A)$$
	特别的,这说明有长正合列:
	$$\xymatrix{0\ar[r]&\mathrm{H}^1(G/H,A^H)\ar[r]&\mathrm{H}^1(G,A)\ar[r]&\mathrm{H}^1(H,A)^{G/H}\ar[r]&\mathrm{H}^2(G/H,A^H)\ar[r]&\mathrm{H}^2(G,A)}$$
	\item 依旧可以定义上同调群的cup积,特别的有:
	$$\mathrm{res}(x\cup y)=\mathrm{res}(x)\cup\mathrm{res}(y)$$
	$$\mathrm{cor}(x\cup\mathrm{res}(y))=\mathrm{cor}(x)\cup y$$
\end{enumerate}
\subsubsection{上同调维数}

设$p$是素数,设$G$是射影有限群.
\begin{itemize}
	\item $G$的$p$-上同调维数,记作$\mathrm{cd}_p(G)$,是最小的正整数$n$,使得对任意离散$G$扭模$A$和任意$q>n$,都有$\mathrm{H}^q(G,A)\{p\}=0$(这里一个交换群$G$,记它的$p$准素子群$G\{p\}$为它的全部$p$次幂阶元构成的子群,有限群时候这就是$G$的Sylow-$p$子群,称一个交换群$G$是$p$-准素的如果$G=G\{p\}$).如果不存在这样的正整数$n$,我们记$\mathrm{cd}_p(G)=+\infty$.定义$G$的上同调维数$\mathrm{cd}(G)=\sup_p\mathrm{cd}_p(G)$.
	\item $G$的严格$p$-上同调维数,记作$\mathrm{scd}_p(G)$,是最小的正整数$n$,使得对任意离散$G$模$A$和任意$q>n$(去掉了扭模条件),都有$\mathrm{H}^q(G,A)\{p\}=0$.同样的如果不存在这样的正整数$n$就记$\mathrm{scd}_p(G)=+\infty$.定义$G$的严格上同调维数为$\mathrm{scd}(G)=\sup_p\mathrm{scd}_p(G)$.
\end{itemize}
\begin{enumerate}
	\item 对离散$G$扭模$A$总有$\mathrm{H}^q(G,A\{p\})=\mathrm{H}^q(G,A)\{p\}$.于是特别的$\mathrm{cd}_p(G)\le n$当且仅当对任意$p$准素离散$G$模$A$(此时它自动是$G$扭模)和任意$q>n$有$\mathrm{H}^q(G,A)=0$.另外这个命题在去掉扭模条件时不成立,例如:
	$$\mathrm{H}^2(\mathbb{Z}/p\mathbb{Z},\mathbb{Z})\{p\}=\mathbb{Z}/p\mathbb{Z};\mathrm{H}^2(\mathbb{Z}/p\mathbb{Z},\mathbb{Z}\{p\})=0$$
	\begin{proof}
		
		如果$A$是离散$G$扭模,那么有分解:
		$$\mathrm{H}^q(G,A)\cong\oplus_p\mathrm{H}^q(G,A\{p\})$$
		任取$f\in\mathrm{H}^q(G,A\{p\})$,那么$f$是$G^q\to A\{p\}$的连续映射,按照$G^q$是紧集,它的像集是离散空间的紧子集,从而是有限集,于是可以取一个足够大的次幂$p^r$使得$p^rf=0$,从而$f\in\mathrm{H}^q(G,A)\{p\}$.这迫使$\mathrm{H}^q(G,A\{p\})$就是$\mathrm{H}^q(G,A)$的$p$准素子群.
	\end{proof}
	\item 设$G$是射影有限群,设$p$是素数,设$n$是自然数,那么$\mathrm{cd}_p(G)\le n$当且仅当$\mathrm{H}^{n+1}(G,A)=0$对任意$p$扭(被$p$零化)的离散单$G$模成立,这里单的意思是它没有非平凡的离散$G$子模.另外对于非平凡的单$G$模$A$,它是$p$扭群和它是$p$准素群是等价的,因为$A[p]$是$A$的非平凡子$G$模.
	\begin{proof}
		
		必要性是平凡的.充分性:先证明对有限$p$准素的离散$G$模$A$总有$\mathrm{H}^{n+1}(G,A)=0$.为此对$|A|$做归纳.不妨设$A\not=0$,如果$A$是单$G$模,那么$A[p]=A$,此时条件(3)保证$\mathrm{H}^{n+1}(G,A)=0$.如果$G$不是单$G$模,那么有短正合列$0\to A_1\to A\to A_2\to0$,其中$A_1$和$A_2$是阶数严格小于$|A|$的$p$-准素有限群.所以归纳和长正合列假设导致$\mathrm{H}^{n+1}(G,A)=0$.
		
		\qquad
		
		进而按照离散$p$准素$G$模是有限离散$p$准素$G$模的正向极限,就得到对$p$准素离散$G$模$A$总有$\mathrm{H}^{n+1}(G,A)=0$.这就证明了(2)对$q=n+1$成立.接下来对$q$归纳,把$A$嵌入到余诱导模$\mathrm{coInd}_G(A)$中,这仍然是一个$p$准素群,那么$\mathrm{coInd}_G(A)/A$是$p$准素群,考虑如下短正合列,取长正合列就归纳下去.
		$$\xymatrix{0\ar[r]&A\ar[r]&\mathrm{coInd}_G(A)\ar[r]&\mathrm{coInd}_G(A)/A\ar[r]&0}$$
	\end{proof}
	\item 设$G$是射影$p$群,我们断言$p$扭离散单$G$模同构于$\mathbb{Z}/p\mathbb{Z}$上赋予平凡$G$作用.进而上一条说明$\mathrm{cd}_p(G)\le n$当且仅当$\mathrm{H}^n(G,\mathbb{Z}/p\mathbb{Z})=0$.
	\begin{proof}
		
		设$A$是$p$扭离散单$G$模.任取非零元$a\in A$,它生成的$G$子模记作$M$,那么$a$的稳定子是一个开子群$H$,于是存在满同态$\mathbb{F}_p[G/H]\to M$.特别的此时$M$是有限集.进而按照$A$是单模就有$A=M$.于是$A$是$G/H$模.换句话讲归结为设$G$是有限$p$群.那么此时$A^G$非平凡,进而依旧因为它是单模得到$A=A^G$,也即$A$是平凡$G$模.按照$A$被$p$零化并且是单模,迫使它是$\mathbb{Z}/p\mathbb{Z}$.
	\end{proof}
	\item 设$\mathrm{cd}_p(G)\le n$,设$A$是离散$p$可除$G$模($p$可除指的是数乘$p:A\to A$是满射).那么对任意$q>n$有$\mathrm{H}^q(G,A)\{p\}=0$.特别的,如果离散$G$模$A$是可除交换群,那么对任意$q>n$有$\mathrm{H}^q(G,A)=0$.
	\begin{proof}
		
		考虑短正合列$\xymatrix{0\ar[r]&A[p]\ar[r]&A\ar[r]^p&A\ar[r]&0}$,它诱导了如下正合列:
		$$\xymatrix{\mathrm{H}^q(G,A[p])\ar[r]&\mathrm{H}^q(G,A)\ar[r]^p&\mathrm{H}^q(G,A)}$$
		当$q\ge n+1$时有$\mathrm{H}^q(G,A[p])=0$,于是$\mathrm{H}^q(G,A)$上数乘$p$是单射,这导致$\mathrm{H}^q(G,A)$的$p$准素部分为零.
	\end{proof}
	\item 我们有$\mathrm{scd}_p(G)\in\{\mathrm{cd}_p(G),\mathrm{cd}_p(G)+1\}$.
	\begin{proof}
		
		明显有$\mathrm{scd}_p(G)\ge\mathrm{cd}_p(G)$.只需证明$\mathrm{scd}_p(G)\le\mathrm{cd}_p(G)+1$.我们有典范短正合列:
		$$\xymatrix{0\ar[r]&A[p]\ar[r]&A\ar[r]^{\alpha}&pA\ar[r]&0}$$
		$$\xymatrix{0\ar[r]&pA\ar[r]^{\beta}&A\ar[r]&A/pA\ar[r]&0}$$
		这里$\alpha$和$\beta$的复合是数乘$p$.如果$q>\mathrm{cd}_p(G)+1$,按照$A[p]$和$A/pA$都是$p$准素群,就有$\mathrm{H}^q(G,N)=\mathrm{H}^{q-1}(G,Q)=0$.于是如下两个同态都是单射:
		$$\mathrm{H}^q(G,A)\to\mathrm{H}^q(G,pA)$$
		$$\mathrm{H}^q(G,pA)\to\mathrm{H}^q(G,A)$$
		进而它们的复合是单射,也即$\mathrm{H}^q(G,A)$上数乘$p$是单射,于是$\mathrm{H}^q(G,A)\{p\}=0$,于是$\mathrm{scd}_p(G)\le\mathrm{cd}_p(G)+1$.
	\end{proof}
	\item 例子.
	\begin{enumerate}
		\item $\mathrm{cd}_p(\mathbb{Z}_p)=1$.
		\begin{proof}
			
			按照循环群的上同调,有$A_n=\mathrm{H}^2(\mathbb{Z}/p^n,\mathbb{Z}/p)=\mathbb{Z}/p$.但是$A_n\to A_{n+1}$是数乘$p$的映射,于是$\mathrm{H}^2(\mathbb{Z}_p,\mathbb{Z}/p)=\varinjlim_n\mathrm{H}^2(\mathbb{Z}/p^n,\mathbb{Z}/p)=0$.于是按照$\mathbb{Z}_p$是射影$p$群,得到$\mathrm{cd}_p(\mathbb{Z}_p)\le1$.又因为$\mathbb{Z}/p$是离散模,有:
			$$\mathrm{H}^1(\mathbb{Z}_p,\mathbb{Z}/p)=\mathrm{Hom}_{\mathrm{cont}}(\mathbb{Z}_p,\mathbb{Z}_p)=\mathrm{Hom}_{\mathrm{cont}}(\mathbb{Z}_p/p\mathbb{Z}_p,\mathbb{Z}/p)=\mathbb{Z}/p\not=0$$
		\end{proof}
		\item 对任意素数$p$有$\mathrm{cd}_p(\widehat{\mathbb{Z}})=1$.另外有$\mathrm{H}^2(G,\mathbb{Z})=\mathrm{H}^1(G,\mathbb{Q}/\mathbb{Z})=\mathbb{Q}/\mathbb{Z}$,于是$\mathrm{scd}_p(\widehat{\mathbb{Z}})=2$.
	\end{enumerate}
	\item 引理.设$G$是射影有限群,设$A$是离散$G$模,设$p$是素数,设$H\le G$是闭子群.
	\begin{enumerate}[(1)]
		\item 如果$p$不整除超自然数$[G:H]$,那么对任意$q\ge1$有限制映射的限制$\mathrm{res}:\mathrm{H}^q(G,A)\{p\}\to\mathrm{H}^q(H,A)$是单射.
		\item 如果额外的还有$H$是开子群,$\mathrm{cd}_p(G)=n$或者$\mathrm{scd}_p(G)=n$,那么对任意离散扭$G$模$A$或者对任意离散$G$模$A$有余限制映射$\mathrm{cor}:\mathrm{H}^n(H,A)\{p\}\to\mathrm{H}^n(G,A)\{p\}$是满射.
	\end{enumerate}
	\begin{proof}
		
		(1):如果$[G:H]=m$是有限的,那么有$\mathrm{cor}\circ\mathrm{res}(x)=mx$,所以如果$x\in\ker\mathrm{res}\cap\mathrm{H}^q(G,A)\{p\}$,按照$(m,p)=1$就得到$x=0$.对于非有限的情况,从$\mathrm{H}^q(G,A)=\varinjlim_U\mathrm{H}^q(G/U,A^U)$就归结到$G$有限的情况.
		
		\qquad
		
		(2):设$I=\mathrm{coInd}_G^H(A)$,我们定义过典范同态$\pi:I\to A$为$f\mapsto\sum_{\overline{g}\in G/H}gf(g^{-1})$.任取$a\in A$,那么$f:h\mapsto ha,h\in H,g\mapsto0,g\not\in H$就把$f$映为$a$,于是$\pi$是满射.
		
		\qquad
		
		如果$A$是扭群,那么$B=\ker\pi$也是扭群.于是$\mathrm{H}^{n+1}(G,B)\{p\}=0$.结合$\mathrm{cor}$就是同态$\mathrm{H}^n(G,I)=\mathrm{H}^n(H,A)\to\mathrm{H}^n(G,A)$.于是$0\to B\to I\to A\to0$诱导的长正合列上作用正合函子$\{p\}$就得到结论(扭交换群范畴上$\{p\}$是正合函子).
	\end{proof}
	\item 设$H$是射影有限群$G$的闭子群,那么有:
	$$\mathrm{cd}_p(H)\le\mathrm{cd}_p(G);\mathrm{scd}_p(H)\le\mathrm{scd}_p(G)$$
	如果如下两个条件之一成立,那么这两个不等式取等号.
	\begin{enumerate}[(1)]
		\item 超自然数$[G:H]$和$p$互素.
		\item $H$是$G$的开子群,并且$\mathrm{cd}_p(G)<+\infty$.
	\end{enumerate}
	\begin{proof}
		
		我们来证明上同调维数的情况,严格上同调维数的情况是类似的.这个不等式就是因为当$A$是离散$H$模/$p$准素时$\mathrm{coInd}_G^H(A)$是离散$G$模/$p$准素,以及Shapiro引理$\mathrm{H}^q(G,\mathrm{coInd}_G^H(A))=\mathrm{H}^q(H,A)$.如果$p$不整除$[G:H]$,那么上一条引理的(1)就保证该不等式取等.
		
		\qquad
		
		如果$H\le G$是开子群,并且$\mathrm{cd}_p(G)=n$是有限数.不妨设$n\ge1$,那么存在某个扭离散$G$模$A$使得$\mathrm{H}^n(G,A)\{p\}\not=0$.于是上一条引理的(2)就保证$\mathrm{H}^n(H,A)\{p\}\not=0$,于是该不等式取等.
	\end{proof}
	\item 推论.
	\begin{enumerate}[(1)]
		\item 设$G_p$是$G$的一个Sylow-$p$子群,那么有:
		$$\mathrm{cd}_p(G)=\mathrm{cd}_p(G_p)=\mathrm{cd}(G_p)$$
		$$\mathrm{scd}_p(G)=\mathrm{scd}_p(G_p)=\mathrm{scd}(G_p)$$
		\begin{proof}
			
			这两个等式链中的第一个等式都来自于上一条.第二个等式归结为证明如果$p,q$是不同的素数,那么$\mathrm{cd}_q(G_p)=\mathrm{scd}_q(G_p)=0$.而这是因为任取离散$G$模$A$,那么总有$\mathrm{H}^1(G_p,A\{q\})=\mathrm{Hom}_{\mathrm{cont}}(G_p,A\{q\})=0$(第一个等号是因为作用是平凡的).
		\end{proof}
		\item $\mathrm{cd}_p(G)=0$当且仅当$G$的阶数(作为超自然数)和$p$互素.
		\begin{proof}
			
			上一条说明归结为设$G$是射影$p$群.此时$\mathrm{H}^1(G,\mathbb{Z}/p)=\mathrm{Hom}_{\mathrm{cont}}(G,\mathbb{Z}/p)$非零当且仅当$G$非平凡.
		\end{proof}
		\item 如果$G$是有限群,如果$p\not\mid|G|$,那么$\mathrm{cd}_p(G)=0$;如果$p\mid|G|$,那么$\mathrm{cd}_p(G)=+\infty$.所以上同调维数这个概念只对无限群有实际意义.
		\begin{proof}
			
			按照推论(1)依旧可以设$G$是有限$p$群.如果$G$是平凡群自然有$\mathrm{cd}_p(G)=0$.如果$G$非平凡,那么$\{e\}$是它的开子群,倘若$\mathrm{cd}_p(G)\not=+\infty$,按照上一条定理的(2),就有$\mathrm{cd}_p(G)=\mathrm{cd}_p(\{e\})=0$矛盾,因为至少有$\mathrm{H}^1(G,\mathbb{Z}/p\mathbb{Z})=\mathrm{Hom}_{\textbf{Ab}}(G,\mathbb{Z}/p\mathbb{Z})$非平凡.
		\end{proof}
		\item 如果$\mathrm{cd}_p(G)\not=0,\infty$,那么$p$在$|G|$(作为超自然数)的次数是无穷.
		\begin{proof}
			
			依旧按照推论的(1)可以设$G$是射影$p$群.如果$G$是有限群,上一条导致$\mathrm{cd}_p(G)=0$或$+\infty$,矛盾.
		\end{proof}
		\item 设$\mathrm{cd}_p(G)=n$是有限的,那么$\mathrm{scd}_p(G)=n$当且仅当对$G$的任意开子群$H$都有$\mathrm{H}^{n+1}(H,\mathbb{Z})=0$.进而对$\mathrm{cd}(G)$有相同的结论.
		\begin{proof}
			
			必要性是直接的,充分性:按照Shapiro引理,对任意形如$A=\mathrm{coInd}_G^H(\mathbb{Z}^r)=\mathbb{Z}[G/H]^r$的$G$模都有$\mathrm{H}^{n+1}(G,A)=0$.设$M$是离散有限生成$G$模,那么存在正规开子群$H$在$M$上的作用平凡,于是有同构$M\cong\mathbb{Z}[G/H]^r/N$.它的长正合列结合$\mathrm{H}^{n+2}(G,N)\{p\}=0$以及$\mathrm{H}^{n+1}(G,\mathbb{Z}[G/H]^r)=0$就得到$\mathrm{H}^{n+1}(G,M)\{p\}=0$.最后对$M$取正向极限就得到对任意离散$G$模$M$都有$\mathrm{H}^{n+1}(G,M)=0$.
		\end{proof}
	\end{enumerate}
	\item 如果$G$是$p$无扭射影有限群,$H$是开子群,那么有:
	$$\mathrm{cd}_p(G)=\mathrm{cd}_p(H);\mathrm{scd}_p(G)=\mathrm{scd}_p(H)$$
	结合前面的结论,这件事归结为证明从$\mathrm{cd}_p(H)<+\infty$推出$\mathrm{cd}_p(G)<+\infty$.
	\item 设$H\le G$是闭正规子群,那么有不等式:
	$$\mathrm{cd}_p(G)\le\mathrm{cd}_p(H)+\mathrm{cd}_p(G/H)$$
	$$\mathrm{scd}_p(G)\le\mathrm{scd}_p(H)+\mathrm{scd}_p(G/H)$$
	\begin{proof}
		
		设$A$是离散$G$扭模,我们有Hochschild–Serre谱序列:
		$$\mathrm{E}_2^{p,q}=\mathrm{H}^p(G/H,\mathrm{H}^q(H,A))\Rightarrow\mathrm{H}^{p+q}(G,A)$$
		取$n>\mathrm{cd}_p(H)+\mathrm{cd}_p(G/H)$,那么如果$i+j=n$,就总有$\mathrm{E}_2^{i,j}$的$p$准素部分为零.进而谱序列的性质告诉我们$\mathrm{H}^n(G,A)\{p\}=0$.
	\end{proof}
\end{enumerate}
\subsubsection{$\mathrm{cd}_p(G)\le1$和提升性质}
\begin{enumerate}
	\item 设$\xymatrix{1\ar[r]&P\ar[r]&E\ar[r]^{\pi}&W\ar[r]&1}$是射影有限群的短正合列(延拓).称射影有限群$G$关于这个短正合列(延拓)满足提升性质,如果对任意同态$f:G\to W$,均可以提升为一个同态$f':G\to E$使得如下图表交换:
	$$\xymatrix{&&&G\ar[d]^f\ar@{-->}[dl]_{f'}&\\1\ar[r]&P\ar[r]&E\ar[r]^{\pi}&W\ar[r]&1}$$
	换句话讲,如果取$E_f=\{(g,e)\in G\times E\mid f(g)=\pi(e)\}$是$\pi$和$f$的纤维积,那么有如下短正合列的同态,此时上述条件就等价于讲上一行短正合列是分裂的,也即$E_f\to G$有连续右逆$\alpha:G\to E_f$.
	$$\xymatrix{1\ar[r]&P\ar@{=}[d]\ar[r]&E_f\ar[r]\ar[d]&G\ar[d]^f\ar[r]&1\\1\ar[r]&P\ar[r]&E\ar[r]^{\pi}&W\ar[r]&1}$$
	\item 设$G$是射影有限群,设$p$是素数,如下命题互相等价:
	\begin{enumerate}[(a)]
		\item $\mathrm{cd}_p(G)\le1$.
		\item 对任意如下短正合列,其中$E$有限,$P$是$p$扭交换群,都有$G$关于该短正合列满足提升性质.
		$$\xymatrix{1\ar[r]&P\ar[r]&E\ar[r]^{\pi}&W\ar[r]&1}$$
		\item 任意的$G$关于有限交换$p$扭$p$群的延拓都分裂.
		\item 对任意如下短正合列,其中$P$是射影$p$群,都有$G$关于该短正合列满足提升性质.
		$$\xymatrix{1\ar[r]&P\ar[r]&E\ar[r]^{\pi}&W\ar[r]&1}$$
		\item 任意的$G$关于射影$p$群的延拓都分裂.
	\end{enumerate}
	\begin{proof}
		
		(d)和(e)等价就是上一条.(c)推(b)也是上一条.(b)推(c):任取如下延拓,其中$P$是$p$扭有限交换$p$群
		$$\xymatrix{1\ar[r]&P\ar[r]&E_0\ar[r]&G\ar[r]&1}$$
		取$E_0$的开正规子群$H$满足$H\cap P=1$【?】.那么投影$E_0\to G$就把$H$恒等于$G$的一个开正规子群.取$E=E_0/H$和$W=G/H$,那么有短正合列:
		$$\xymatrix{1\ar[r]&P\ar[r]&E\ar[r]&W\ar[r]&1}$$
		于是(b)说明$G\to W$可以延拓为$G\to E$.又因为有纤维积图表:
		$$\xymatrix{E_0\ar[rr]\ar[d]&&G\ar[d]\\E\ar[rr]&&W}$$
		于是$G\to E$又可以延拓为$G\to E_0$.这得到最开始的这个短正合列是分裂的.
		
		\qquad
		
		按照在$A$有限的时候$\mathrm{H}^2(G,A)$是$G$关于$A$的延拓的等价类,得到(a)和(c)等价【】.
		
		
	\end{proof}
	\item 推论.自由射影$p$群$F(I)$总满足$\mathrm{cd}_p(F(I))\le1$.
	\begin{proof}
		
		取延拓$1\to P\to E\to F(I)\to1$.取连续同态$u:F(I)\to E$【】.
	\end{proof}
\end{enumerate}




\newpage
\subsection{NOT NOW}















给定左$R$模$A,B$,那么一般来讲$\mathrm{Hom}_R(A,B)$只是一个交换群,并不会是一个$R$模,除非给$A$或$B$赋予双边模结构.同样的如果给右$R$模$A$和左$R$模$B$,那么$A\otimes_RB$也只会是交换群.但是来证明,当$R=Z[G]$的时候,模结构是存在的.

给定群$G$,取左$Z[G]$模$A,B$,约定$\mathrm{Hom}_G(A,B)$上的对角作用是指,对$g\in G,\phi:A\to B$和任意$a\in A$有$(g\phi)(a)=g\phi(g^{-1}a)$.给定右$G$模$M$和左$G$模$B$,约定对角作用为$g(m\otimes b)=gm\otimes gb$.

给定群$G$,和一个$G$模$K$,称固定点子模为$K$的如下子模:$K^G=\{a\in K\mid xa=a,\forall x\in G\}$.那么$K^G$就是$K$的极大的$G$平凡子模.如果存在$G$映射$\phi:K\to L$,取$a\in K^G$,那么$\forall x\in G$有$xa=a$.于是$x\phi(a)=\phi(xa)=\phi(a)$,于是$\phi(a)\in L^G$. 记$\phi^G=\phi\mid K^G$.

固定点函子$\mathrm{Fix}^G$是$Z[G]$模上的把$K$映射到$K^G$,把$\phi$映射到$\phi^G$的加性函子.先解释如何把$Z$当做一个平凡$Z[G]$模,即,对任意$m\in Z$和任意的$\sum a_ig_i\in Z[G]$,有$(\sum a_ig_i)m=\sum a_i(g_im)=\sum a_im=(\sum a_i)m$.那么有自然同构$\mathrm{Fix} ^G\sim \mathrm{Hom}_G(Z,-)$,特别的,看到$\mathrm{Fix}^G$是左正合的.证明.来构造$\mathrm{Hom}_G(Z,K)\to K^G$的同构$\tau_K$.即对任意的$G$映射$f:Z\to K$,记$\tau_K(f)=f(1)$.若$x\in G$,那么有$xf(1)=f(x1)=f(1)$,于是$f(1)\in K^G$,于是这的确是一个同态.现在通过构造$\tau_K$的逆来证明它是同构.若$a\in K^G$,那么存在一个$Z$映射$f_a:Z\to K$满足$f_a(1)=a$.但是按照$xa=a,\forall x\in G$,看到$f_a$是一个$G$映射,并且是$\tau_K$的逆.最后验证自然同构只要验证交换图:
$$\xymatrix{
	\mathrm{Hom}_G(Z,K)\ar[d]_{\phi_*}\ar[r]^{\tau_K}&K^G\ar[d]^{\phi^G}\\
	\mathrm{Hom}_G(Z,L)\ar[r]_{\tau_L}&L^G
}$$

给定群$G$和一个$G$模$K$,把$G$的系数$K$的上同调群定义如下,其中$Z$是平凡$G$模:
$$H^n(G,K)=\mathrm{Ext}_{Z[G]}^n(Z,K)$$

那么为了计算群上同调,需要探究$Z$的$G$投射预解,先考虑$Z[G]$到$Z$的满同态.

存在$G$上短正合列$\xymatrix{0\ar[r]&\ker\varepsilon\ar[r]&Z[G]\ar[r]^ {\varepsilon}&Z\ar[r]&0}$.其中$\varepsilon:Z[G]\to Z$为$\sum_{x\in G}m_xx\mapsto\sum_{x\in G}m_x$.那么$\varepsilon$既是环同态又是$G$模同态.另外$\ker\varepsilon$是$Z[G]$的一个双边理想.

给上述$\varepsilon$一个名字,称作扩充映射,称$\ker\varepsilon$为扩充理想.

扩充理想作为交换群是一个以$\{x-1\mid x\in G-\{1\}\}$为基的自由交换群.证明.记$u\sum_{x\in G}m_xx\in Z[G]$,那么$u$在扩充理想中当且仅当$\sum_ {x\in G}m_x=0$.于是$u=u-(\sum m_x)1=\sum_{x\in G^*=G-\{1\}}m_x(x-1)$.现在假设$\sum_{x\in G^*}m_x(x-1)=0$,那么$\sum_ {x\in G^*}m_xx-(\sum_{x\in G^*}m_x)1$,但是$Z[G]$是以$\{x\in G\}$为基的自由交换群,于是全部$m_x=0,x\in G^*$.

回顾在群的扩张中,记$\mathrm{Der}(G,A)$表示的是$G$模$A$上全体导数构成的群.现在来说明它可以看作$G$模之间的$\mathrm{Hom}$集.即,如果$Q$是扩充理想,那么存在自然同构$\tau:\mathrm{Hom} _G(Q,-)\to\mathrm{Der}(G,-)$.即$\tau_A$为把$Q\to A$的$G$模$f$,映射到$f':G\to A,x\mapsto f(x-1)$.

有限循环群的上同调群.取$G=<x>$为一个有限的循环群,设阶数是$k$,定义$Z[G]$中的元$D=x-1$,$N=1+x+\cdots+x^{k-1}$.那么存在如下$G$自由预解,其中$D,N$表示的是乘以该元素的$G$映射:
$$\xymatrix{\ar[r]&Z[G]\ar[r]^D&Z[G]\ar[r]^N&Z[G]\ar[r]^D&Z[G]\ar[r]^{\varepsilon}&Z\ar[r]&0}$$

若$A$是$G$模,其中$G$是有限循环群,取$A_N=\{a\in A\mid Na=0\}$,那么把$\mathrm{Hom}_G(Z,-)$作用于上述预解,就得到:
$$H^0(G,A)=A^G;H^{2n-1}(G,A)=A_N/DA;H^{2n}(G,A)=A^G/NA$$

特别的,如果$A$是$G$平凡模,其中$G$是有限循环群,那么:
$$H^0(G,A)=A;H^{2n-1}(G,A)=A_k;H^{2n}(G,A)=A/kA$$

特别的,取$A=Z$,有:
$$H^0(G,Z)=Z;H^{2n-1}(G,A)=0;H^{2n}(G,A)=Z_k$$

现在开始来证明所定义的群上同调在一阶和二阶情况下和从群的扩张得到的群上同调是一致的.

记$B_0$为以单个元$[]$为基的自由$G$模,于是$B_0\sim Z[G]$.对于$n\ge1$,记$B_n$为以$G^n$为基的自由$G$模,把$G^n$这个基中的元素记作$[x_1\mid\cdots\mid x_n]$而不是用$(x_1,\cdots,x_n)$.

回顾在定义一阶二阶群上同调的时候所用到的东西:
\begin{align*}
	\text{因式集} &: 0=xf(y,z)-f(xy,z)+f(x,yz)-f(x,y) \\
	\text{余2边界} &: f(x,y)=xh(y)-h(xy)+h(x) \\
	\text{导数}&: 0=xd(y)-d(xy)+f(x) \\
	\text{主导数} &: d(x)=xa_0-a_0
\end{align*}

就让这些函数来作为预解的低阶边界算子.即取:
$$\xymatrix{B_3\ar[r]^{d_3}&B_2\ar[r]^{d_2}&B_1\ar[r]^{d_1}&B_0\ar[r]^{\varepsilon}&Z\ar[r]&0}$$

其中$\varepsilon$是扩充映射,而$d_i$为:
$$d_3[x\mid y\mid z]=x[y\mid z]-[xy\mid z]+[x\mid yz]-[x\mid y]$$
$$d_2[x\mid y]=x[y]-[xy]+[x]$$
$$d_1[x]=x[]-[]$$
$$\varepsilon[]=1$$

那么这是一个正合列.现在把逆变函子$\mathrm{Hom}_G(-,K)$作用在这个正合列上,由此得到一个正合列:
$$\xymatrix{\mathrm{Hom}_G(B_2,K)&\mathrm{Hom}_G(B_1,K)\ar[l]^{d_2^*}&\mathrm{Hom}_G(B_0,K)\ar[l]^{d_1^*}}$$

那么按照定义,有$\mathrm{Ext}^1(Z,K)=\frac{\ker d_2^*}{\mathrm{im}d_1^*}$.知道给定一个$G$映射$g:B_1\to K$等价于给定它在基上的限制,并且对任意的$G\to K$的函数都可以延拓为一个从$B_1\to K$的$G$映射,于是可以把$\mathrm{Der}(G,K)$当作$\mathrm{Hom} _G(B_1,K)$的子群.现在如果$g:G\to K$落在$\ker d_2^*$中,那么$g$必然是一个导数.反过来如果$\delta$是一个导数,那么它必然落在$\ker d_2^*$中,于是看到$\ker d_2^*=\mathrm{Der}(G,K)$.类似证明$\mathrm{PDer}(G,K)=\mathrm{im}d_1^*$.于是得到了:
$$\mathrm{Ext}^1(Z,K)=\frac{\mathrm{Der}(G,K)}{\mathrm{PDer}(G,K)}$$

但是类似的方法是不能证明二阶上同调群一致的.会通过改变$Z$的投射预解来证明二阶上同调群一致.在给出不同的$Z$的预解之前,先说明,对于三阶上同调群$H^3(Q,K)$是存在群论描述的,但是对于阶数超过3的上同调群,正如Robinson所说,“尽管对于阶数超过3的上同调群的群论描述是存在的,但是这些描述并没法合适的应用在群论上”.不过高阶上同调群可以用来定义一个群的上同调维数.

现在定义$G$平凡模$Z$的Bar预解.所谓Bar预解是指预解:
$$\xymatrix{&\ar[r]&B_3\ar[r]^{d_3}&B_2\ar[r]^{d_2}&B_1\ar[r]^{d_1}&B_0\ar[r]^{\varepsilon}&Z\ar[r]&0}$$

其中$B_0$仍然是以单个元$[]$为基的自由$G$模,$\varepsilon$仍然是扩充映射,$B_n$仍然是以$G^n$为基的自由模,其中边界映射$d_n:B_n\to B_{n-1}$推广了之前的$d_1,d_2,d_3$,它满足:
$$d_n:[x_1\mid\cdots\mid x_n]\mapsto x_1[x_2\mid\cdots\mid x_n]
+\sum_{i=1}^{n-1}(-1)^i[x_1\mid\cdots\mid x_ix_{i+1}\mid\cdots\mid x_n]+(-1)^n[x_1\mid\cdots\mid x_{n-1}]$$

为了证明二阶群上同调和在群扩张中定义的二阶群上同调吻合,来定义正规Bar预解.记$U_n\subset B_n$是倍全体$[x_1\mid\cdots\mid x_n]$中至少有一个$x_i=1$的元生成的自由子模,那么$d_n(U_n)\subset U_{n-1}$,于是得到Bar预解$B(G)$的一个子复形$U(G)$.记$B^*(G)=B(G)/U(G)$为正规Bar预解.那么$B_n^*=B_n/U_n$是以全体$[x_1\mid\cdots\mid x_n]^*,x_i\not=1$为基的自由$G$模.这是一个$G$自由预解.利用这个预解,得到:
$$\mathrm{Ext}^2(Z,K)=\frac{\ker d_3^*}{\mathrm{im} d_2^*}=\frac{Z^2(G,K)}{B^2(G,K)}$$

引理.如果$G$是$m$阶群,那么$mH^n(G,K)=0$.取Bar预解$B(G)$.如果$f:B_n\to K$,取$g:B_{n-1}\to K$为:$$g[x_1\mid\cdots\mid x_{n-1}]=\sum_{x\in G}f[x_1\mid\cdots\mid x_{n-1}\mid x]$$

那么如果$d_nf=0$,证明$0=gd_{n-1}+m(-1)^nf$,于是$mf\in d_{n-1}^*g$是余边界,得证.

如果$G$是有限群,$A$是有限生成$G$模,那么上同调群$H^n(G,A),n\ge0$都是有限群.证明,按照$G$有限群,看到群环$Z[G]$是有限生成的交换群,于是每个有限生成$G$模作为交换群都是有限生成的.于是$A$和$B(G)$中每个群$B_n$都是有限生成的交换群.现在$\mathrm{Hom}_G(B_n,A)\subset\mathrm{Hom}_Z(B_n,A)$,而后者是有限生成交换群,于是每个$\mathrm{Hom}_G(B_n,A)$都是有限生成交换群,这导致每个$\ker d_n^*$都是有限生成交换群,于是每个$H^n(G,A)$都是有限生成交换群,结合上述引理,看到每个$H^n(G,A)$是有限群.

给定有限群$G$,阶数$mn$,其中$(m,n)=1$,如果$K$是它的交换正规$n$阶子群,那么$G$是$K$和$G/K$的半直积,并且$K$的任意两个补是共轭的.证明.取$Q=G/K$,那么$Q$的阶数$m$,按照在群的扩张一节最后一个定理,只需证明$H^2(Q,K)=H^1(Q,K)=0$.但是对任意$q\ge0$,知道$mH^q(Q,K)=0$.按照$(m,n)=1$,并且$K$是$n$阶有限交换群,看到$\mu:K\to K,a\mapsto ma$是自同构,于是诱导了$H^q(Q,K)$上的自同构,即乘以$m$,这导致$H^q(Q,K)=0$.

事实上上述条件$K$交换可以去掉,这就是Schur-Zassenhaus定理.给定有限群$G$,阶数$mn$,其中$(m,n)=1$,如果$K$是一个$n$阶正规子群,那么$G$是$K$和$G/K$的半直积,并且任意两个$K$的补是共轭的.来简述证明过程,证明$G$是半直积需要给出一系列正规化,最后只需要$K$交换的情况.至于证明$K$的补都共轭是困难得多的,需要先证明当$K$或$Q$可解的时候$K$的补是两两共轭的,接下来$|Q|,|K|$共轭说明必有一个奇数阶,Feit-Thompson定理告诉奇数阶群都可解,然后完成证明.





给定两个群$G,H$,倘若存在从$G$到$Aut(H)$ 的映射$\theta_g,g\in G$,那么可以在$H\times G$中定义乘法为$(h_1,g_1) (h_2,g_2)=(h_1\theta_{g_1}(h_2),g_1g_2)$.容易验证这是一个群,称为$G$和$H$关于$\theta$的半直积.特别的,如果$\theta$本身是一个平凡映射,即每个$\theta_g$都是$H$上的恒等映射,那么半直积就是常规的直积.另外,对于一个群$G$,记$H$是一个子群,$N$是一个正规子群,那么存在从$H$到$Inn(N)$ 的自然的映射,即$\forall h\in H$,定义$\theta_h(n)=hnh^{-1}$,这得到一个半直积,倘若$H\cap N=\{e\}$并且$G=HN$,那么有$G$同构于该半直积,事实上可以说明所有的半直积都是这个形式,即对于任意群$H$和$G$,它们分别作为半直积的子群和正规子群,再按照上述构造.

非交换群的正合列和交换的情况相同,就是要求$im d_{n+1}=\ker d_n,n\in Z$,那么这时候必然有$\ker d_n$是$G_n$的一个正规子群.
$$\xymatrix{
	\ar[r]&G_{n+1}\ar[r]^{d_{n+1}}&G_n\ar[r]^{d_n}&G_{n-1}\ar[r]&
}$$

给定群$K,Q$,这里不要求交换性.那么$K$的关于$Q$的扩张是指一个短正合列:
$$\xymatrix{
	1\ar[r]&K\ar[r]^i&E\ar[r]^p&Q\ar[r]&1
}$$

那么这时候必然有$K$是$E$的正规子群.另外给定群$E$和任意一个正规子群$K$,那么$E$可以被当作$K$通过$E/K$扩张而成的.所谓的扩张问题是指逆问题:给定两个群$K,Q$,寻找全部的扩张.也就是寻找这样的群$E$,它以$K$为正规子群,并且$Q=E/K$.Schreier通过列举$E$的全部可能的乘法表解决了这个问题.

延拓问题的重要性来自Jordan-Holder定理.定理断言,对任意群$E$,存在合成列,即:
$$E=K_0\ge K_1\ge\cdots\ge K_n=\{1\}$$

其中$k_{i-1}/K_i$记作$Q_i$是单群.那么从$K_n=1$得到$Q_{n}=K_{n-1}$.倘若扩张问题可以解决,那么看到$K_{n-2}$可以直接从$K_{n-1}$被$Q_{n-1}$的延拓得到.这也就是说$E$可以被全体$Q_n,Q_{n-1},\cdots,Q_1$得到.知道全体有限单群已经于2005年被完全分类.于是全部有限群可以同构延拓问题来得到.

尽管延拓的定义使用于未必交换的群,接下来约定$K$总是一个交换群,并且用以加法记号,而$E$同样用以加法记号,尽管$E$可能未必是交换群.总以乘法记号表示$Q$上的运算.

给定延拓$\xymatrix{0\ar[r]&K\ar[r]&E\ar[r]^p&Q\ar[r]&1}$,称这个延拓的一个提升是指一个从$Q$到$E$的集合映射$l$,也就是说它未必是同态,满足$pl=1_Q$,并且$l(1)=0$.给定群$E$的子群$K$,称$K$的一个右截线是指,$K$在$E$中全体右陪集各取一个元构成的集合,另外约定陪集$K$取的元素是0.

那么任给右截线$T$,可以构造延拓的一个提升.对任意$x\in Q$,按照$p$是满射知道存在$E$中的一个元$l(x)$使得$pl(x)=x$.于是$l$是提升.反过来,任给一个提升$l$,我们断言$l(Q)$就是$K$的一个右截线.取陪集$K+e$,那么记$p(e)=x\in Q$,那么$p(e-l(x))=1$,于是$e-l(x)\in K$,那么$K+e=K+l(x)$.于是每个右陪集存在一个表示元素落在$l(Q)$中.最后需要证明$l(Q)$中不存在两个元素表示同一个右陪集.否则有$K+l(x)=K+l(y),x,y\in Q$,那么就有$a\in K$使得$a+l(x)=l(y)$,那么用$p$作用上,因为$p(a)=1$得到$x=y$,于是$l(x)=l(y)$.

回顾群论知识,记$Aut(E)$表示群$G$的全部自同构以复合作为运算构成的群.按照元素共轭得到的自同构称为内自同构,全体内自同构构成群,记作$Inn(G)$,那么它是$Aut(G)$的正规子群,另外总有$G/Z(G)\sim Inn(G)$.把商$Aut(E)/Inn(E)$记作$Out(E)$,称为$E$上的外自同构群.

给定延拓$\xymatrix{0\ar[r]&K\ar[r]&E\ar[r]^p&Q\ar[r]&1}$,其中$K$交换,并且$l:Q\to E$是一个提升:
\begin{enumerate}
	\item 对任意的$x\in Q$,考虑$K$上的共轭映射$\theta_x:a\mapsto l(x)+a-l(x)$,那么这个映射不依赖于提升$l(x)$的选取.事实上如果$l'(x)$是另一个提升,那么$b=l'(x)-l(x)\in K$,那么有$l'(x)+a-l'(x)=l(x)+b+a-b-l(x)=l(x)+a-l(x)$.
	\item 那么上述$\theta$构成了从$Q$到$Aut(K)$的一个同态$x\mapsto\theta_x$.证略.
	\item $K$是一个左$Z[Q]$模,其中数乘是$xa=\theta_x(a)=l(x)+a-l(x)$.
\end{enumerate}

给定群$Q$和一个交换群$K$,称一个$Q$模是指一个左$Z[Q]$模.会把$Hom_ {Z[Q]}(A,B)$简单写作$Hom_Q(A,B)$,把张量积简单写作$M\otimes_QN$.

称交换群$K$通过延拓$\xymatrix{0\ar[r]&K\ar[r]&E\ar[r]^p&Q\ar[r]&1}$称为$Q$模,如果对任意$x\in Q$和$a\in K$有$xa=l(x)+a-l(x)$,特别的,称$Q$模$K$平凡,如果$xa=a,\forall x\in Q,a\in K$.

如果$K$通过延拓$\xymatrix{0\ar[r]&K\ar[r]&E\ar[r]^p&Q\ar[r]&1}$成为$Q$模,那么$K$是平凡的$Q$模当且仅当$K\subset Z(E)$.充分性易证,对于必要性,看到$l(Q)$中元和$K$中元交换,但是$E$中任意元可以表示为$b+l(y),b\in K,y\in Q$,于是$a\in K$必然和$E$每个元交换,于是$K\subset Z(E)$.

半直积.$K$的由$Q$生成的最简单的延拓是半直积.给的那个延拓$\xymatrix{0\ar[r]&K\ar[r]&E\ar[r]^p&Q\ar[r]&1}$,如果存在提升$j:Q\to E$是同态,那么就称延拓是分离的,此时称$E$是$K$到$Q$的半直积,记作$K \rtimes Q$.

给定加法记号群$E$和交换的正规子群$K$,有$E/K\sim Q$,那么如下条件等价:
\begin{enumerate}
	\item $E$是$K$和$Q$的半直积.
	\item 存在$E$的子群$C$,满足$C\sim Q$,并且$K\cap C=0$和$K+C=E$.这时候称$C$是$K$的补.
	\item 对每个$e\in E$存在唯一的分解$e=a+x$,其中$a\in K,x\in C$.
\end{enumerate}

知道对于两个群的直积,这两个群可以作为正规子群.称上述构造为半直积的理由是,半直积可以理解为是一个正规子群和一个子群的类似直积的东西.举例来讲,考虑$S_3$的正规子群$K=<(1,2,3)>$和子群$Q=<(1,2)>$,那么$Q$非正规子群,而半直积$K\rtimes Q=S_3$.

给定$Q$模$K$,那么存在一个分离的延拓$\xymatrix{0\ar[r]&K\ar[r]&K\rtimes Q\ar[r]^p&Q\ar[r]&1}$恰好诱导了这个$Q$模结构.其中$K\rtimes Q$作为集合就是$K\times Q$,而它的运算定义为$(a,x)+(b,y)=(a+xb,xy)$.证明.容易验证在这个运算下$K\rtimes Q$构成群.现在构造映射$p:K\rtimes Q\to Q$为$(a,x)\mapsto x$.于是$p$是满射,核为$\ker p=\{(a,1)\mid a\in K\}$,取$i:K\to K\rtimes Q$为$a\mapsto (a,1)$.于是得到了上述延拓.现在构造$j:Q\to K\rtimes Q$为$x\mapsto(0,x)$,那么看到这个延拓是分裂的.最后验证这个延拓诱导了$Q$模$K$.若$x\in Q$,那么$x$的每个提升都有形式$l(x)=(b,x),b\in K$,那么$(b,x)+(a,1)-(b,x)=(xa,1)$,得证.

若$K$是交换群,$E$是$K$和某个群$Q$的半直积,那么存在$K$上的一个$Q$模结构被半直积的这个延拓所实现.证明.把$K$和$Q$当作$E$的子群.那么$K$正规子群并且$Q$是$E$的补.约定对$a\in K,x\in Q$有$xa=x+a-x$.对任意$e\in E$有唯一分解$e=a+x,a\in K,x\in Q$.那么存在$\phi:E\to K\rtimes Q$为$a+x\mapsto (a,x)$.这是双射.验证它是同态,从而是同构.

现在来解决延拓问题.已经知道给定延拓,等价于存在$K$上的一个$Q$模结构,于是可以设$K$就是一个$Q$模,然后寻找所有能够诱导出这个模结构的延拓.假设存在延拓$\xymatrix{0\ar[r]&K\ar[r]&E\ar[r]^p&Q\ar[r]&1}$,取一个提升$l:Q\to E$,那么由于$K$是正规子群,有$l(Q)$是左右截线.并且每个$R$中元可以唯一的写作$a+l(x)$.于是对任意$x,y\in Q$,有$l(xy)$和$l(x)+l(y)$对应同一个$K$的陪集,于是$f(x,y)=l(x)+l(y)-l(xy)\in K$.给定延拓和一个提升$l$,称这个$l:Q\times Q\to K$为一个因式集(余环).那么当延拓是分裂延拓的时候,提升可以取为同态,这时候因式集就是0.也就是说,因式集可以度量一个延拓和分裂延拓的不同.

给定群$Q$,$K$是一个$Q$模,取延拓$\xymatrix{0\ar[r]&K\ar[r]&E\ar[r]^p&Q\ar[r]&1}$诱导了这个$K$上的$Q$模结构.如果$l:Q\to E$是一个提升并且$f:Q\times Q\to E$是对应的因式集.那么:
\begin{enumerate}
	\item 对任意$x,y\in Q$,有$f(1,y)=f(x,1)=0$.
	\item $f$满足余环恒等式,对任意$x,y,z\in Q$有:
	$$f(x,y)+f(xy,z)=xf(y,z)+f(x,yz)$$
\end{enumerate}

事实上上述两个条件是一个$f:Q\times Q\to K$成为因式集的充要条件.证明.取$E$作为集合是$K\times Q$.约定运算为$(a,x)+(b,y)=(a+xb+f(x,y),xy)$.验证这构成了群,其中余环恒等式使得结合律成立.并且看到如果$f(x,y)=0$有$E\sim K\rtimes Q$.另外逆元是$- (a,x)=(-x^{-1}a-x^{-1}f(x,x^{-1}),x^{-1})$.取$p:E\to Q$为$(a,x)\mapsto x$.于是$p$是一个满同态.取$i:K\to E$为$a\mapsto (a,1)$,那么这得到了延拓$\xymatrix{0\ar[r]&K\ar[r]&E\ar[r]^p&Q\ar[r]&1}$.现在证明这个延拓实现了$Q$模$K$结构.这需要证明对任意提升$l$有$xa=l(x)+a-l(x),\forall a\in K,x\in Q$.记$l(x)=(b,x)$易证.最后需要证明存在提升诱导了这个因式集$f$.取$l(x)=(0,x),x\in Q$.于是有$l(x)+l(y)-l(xy)=(f(x,y),1)$.

给定群$K,Q$和一个因式集$f$,把上述构造的延拓$R$记作$Gr(K,Q,f)$.现在证明,每个延拓都可以被某个因式集这样生成.也就是说,如果存在群$Q$,$Q$模$K$,取延拓$$\xymatrix{0\ar[r]&K\ar[r]&E\ar[r]^p&Q\ar[r]&1}$$

实现了这个模结构,那么存在一个因式集$f:Q\times Q\to K$满足$E\sim Gr(K,Q,f)$.证明,取一个提升$l:Q\to E$,取对应的因式集$f$,那么对任意$e\in E$存在唯一的分解$e=a+l(x)$,定义$\phi:E\to Gr(K,Q,f)$为$e=a+l(x)\mapsto(a,x)$.这是一个良性定义的双射,最后验证是同态即可.

至此通过因式集构造了全体从一个$Q$模$K$到$Q$的延拓.但是因式集依赖于提升的选取.现在给出不同提升诱导的因式集的关系:如果给定群$Q$和$Q$模$K$,给定延拓
$$\xymatrix{0\ar[r]&K\ar[r]&E\ar[r]^p&Q\ar[r]&1}$$

给定两个提升$l$和$l'$,记诱导的因式集分别为$f$和$f'$,那么存在一个函数$h:Q\to K$,满足$h(1)=0$,并且$\forall x,y\in Q$有$f'(x,y)-f(x,y)=xh(y)-h(xy)+h(x)$.证明.对任意$x\in Q$有$l(x)$和$l'(x)$落在相同的$K$的陪集中,取$h(x)=l'(x)-l(x)$.于是$h(1)=0$,那么有$l'(x)+l'(y)=h(x)+xh(y)+l(x)+l(y)=h(x)+xh(y)+f(x,y)-h(xy)+l'(xy)$,结合$f'(x,y)=l'(x)+l'(y)-l'(xy)$得到等式.

给定群$Q$和$Q$模$K$,称函数$g:Q\times Q\to K$是余边界,如果存在$h:Q\to K$满足$h(1)=0$,并且$\forall x,y\in Q$有$g(x,y)=xh(y)-h(xy)+h(x)$.那么已经看到给出两个余圈$f,f'$,那么$f'-f$是余边界.

现在定义,给定群$Q$和$Q$模$K$,记$Z^2(Q,K)$为全体余圈构成的集合,记$B^2(Q,K)$为全体余边界构成的集合.约定$Z^2$和$B^2$上的加法为$f+f': (x,y)\mapsto f(x,y)+f'(x,y)$.那么看到$B^2$是$Z^2$的子群.于是定义二阶群上同调为:$$H^2(Q,K)=\frac{Z^2(Q,K)}{B^2(Q,K)}$$

给定$Q$模$K$,称两个延拓是等价的,如果各存在一个余圈$f,f'$满足$f'-f$是余边界,等价于说这两个余圈在二阶上同调群中对应相同的元.

给定$Q$模$K$的两个延拓$E,E'$,那么他们等价当且仅当存在同构$\gamma:E\to E'$使得如下图表成立:
$$\xymatrix{
	0\ar[r]&K\ar[r]^i\ar[d]_{1_K}&E\ar[r]^{p}\ar[d]_{\gamma}&Q\ar[r]\ar[d]_{1_Q}&1\\
	0\ar[r]&K\ar[r]^{i'}&E'\ar[r]^{p'}&Q\ar[r]&1\\
}$$

证明.如果两个延拓是等价的,取两个提升为$l:Q\to E$和$l':Q\to E$.取对应的余圈为$f,f'$.那么等价意味着存在一个函数$h:Q\to K$满足$h(1)=0$和$f(x,y)-f'(x,y)=xh(y)-h(xy)+h(x)$.对每个$e\in E$有唯一分解$e=a+l(x)$.对每个$e'$又有唯一分解$e'=a'+l'(x')$.构造映射$\gamma:E\to E'$为$a+l(x)\mapsto a+h(x)+l'(x)$.这个映射使得上述图表交换,最后验证是同态即可.反过来,如果存在$E\to E'$的同构$\gamma$使得上述图表交换.于是$\forall a\in K$有$x=pl(x)=p'\gamma l(x)$.于是$\gamma l$是$Q\to E'$的提升.于是$\gamma f$是$\gamma l$确定的因式集.现在按照图表交换有$\gamma f=f$,$\forall x,y\in Q$.于是$f$同样是$E'$的因式集.另外如果$f'$是任意的另一个因式集,那么$f'-f\in B^2$,于是延拓等价.

给定$Q$模$K$,那么全体延拓的等价类构成的集合和二阶上同调群$H^2(Q,K)$同构.其中0映射到分裂延拓.证明,延拓$E$的等价类记作$[E]$.可以定义从二阶上同调群到全体延拓等价类的映射为,把$f+B^2$映射到$[Gr(K,Q,f)]$,那么如果$f-g\in B^2$,已经证明了$[Gr(K,Q,f)]=[Gr(K,Q,g)]$,于是定义良性并且是单射.为了证明满射,只要注意到任取等价类$[E]$,又已经在前文证明了存在因式集$f$满足$[E]=[Gr(K,Q,f)]$.于是$[E]=\phi(f+B^2)$.得证.

特别的,看到如果$Q$模$K$的二阶上同调群是0,那么每个延拓都同构于半直积.

已经完成了二阶群上同调的描述,现在来讨论一阶群上同调.

给定$Q$模$K$的一个延拓$E$,那么$E$的一个稳定自同构是指自同构$\phi$满足如下交换图:
$$\xymatrix{
	0\ar[r]&K\ar[r]^i\ar[d]_{1_K}&E\ar[r]^{p}\ar[d]_{\phi}&Q\ar[r]\ar[d]_{1_Q}&1\\
	0\ar[r]&K\ar[r]^{i}&E\ar[r]^{p}&Q\ar[r]&1\\
}$$

全体$E$的稳定自同构在复合运算下构成的群记作$Stab(Q,K)$.将会看到这个群不依赖于$E$的选取!只会和$Q$和$K$有关.

取$Q$模$K$的延拓$\xymatrix{0\ar[r]&K\ar[r]&E\ar[r]^p&Q\ar[r]&1}$,取提升$l:Q\to E$,那么每个$E$上的稳定自同构具有形式$\phi(a+l(x))=a+d(x)+l(x)$,这里$d:Q\to K$不受提升$l$的选取,并且这个公式定义了一个稳定自同构当且仅当$\forall x,y\in Q$有$d(xy)=d(x)+xd(y)$.证明.如果$\phi$是稳定的自同构,那么对任意$a\in K$有$\phi(a)=a$,并且$p\circ\phi=p$.现在按照$E$中元素的唯一分解,我妈记$\phi(l(x))=d(x)+l(y)$,那么看到$x=pl(x)=p\phi l(x)=p(d(x)+ly)=y$.于是得到了$\phi(a+l(x))=a+d(x)+l(x)$.现在来证明$d$的定义不受$l$的选取而改变.如果有另一个提升$l':Q\to E$,那么$\phi(l'(x))=d'(x)+l'(x)$,取$k(x)=l'(x)-l(x)\in K$,那么$pl'(x)=x=pl(x)$.于是$d'(x)=d(x)$.下面验证$d$满足的等式,取$l$诱导的余圈为$f$,那么一方面有$\phi(l(x)+l(y))=\phi(f(x,y)+l(xy))=f(x,y)+d(xy)+l(xy)$,另一方面有$\phi(l(x)+l(y))=\phi(l(x))+\phi(l(y))=d(x)+xd(y)+f(x,y)+l(xy)$,于是得到$d(xy)=d(x)+xd(y)$.最后逆命题直接验证即可.

给上述函数$d$一个名字,如果$K$是$Q$模,那么一个导数,定义为$d:Q\to K$满足$d(xy)=xd(y)+d(x)$.那么$Q$模$K$的全体导数在逐项相加运算下构成一个交换群,记作$Der(Q,K)$.那么如果$K$是平凡$Q$模,有$Der(Q,K)=Hom(Q,K)$.另外总有$d(1)=0$.

$Stab(Q,K)$和$Der(Q,K)$是同构的.事实上如果取$Q$模$K$,对任一个延拓$E$,给定一个稳定自同构$\phi$,按照之前的定理记$\phi(l(x))=d(x)+l(x)$,那么从$\phi$到$d(x)$是一个良性定义.并且这是一个同态.来构造逆同态来证明它是同构.取一个导数$d$,定义稳定自同构是$\phi(a+l(x))=a+d(x)+l(x)$,得证.

按照这个同构可以得到两件事,第一,$Stab(Q,K)$不依赖于$E$的选取.第二,$Stab(Q,K)$是交换群.这两件事从稳定自同构的原始定义并不能直接的得到,尤其这个群的运算是复合,一般复合运算并不会交换.

称一个导数$d_0:Q\to K$是主导数,如果存在$a_0\in K$满足$d_0(x)=xa_0-a_0$.那么全体主导数构成了$Der(Q,K)$的子群$PDer(Q,K)$.给定$Q$模$K$上一个延拓$E$,称一个稳定自同构是内稳定自同构,如果它是稳定的内自同构,全体内稳定自同构构成的子群记作$Inn(Q,K)$.那么这两个新概念实际上是对应的.即$\phi:E\to E$是一个被元$a_0\in K$诱导的稳定的内自同构等价于$\phi(a+l(x))=a+xa_0-a_0+l(x)$.由此得到:
$$\frac{Stab(Q,K)}{Inn(Q,K)}\sim\frac{Der(Q,K)}{PDer(Q,K)}$$

把这个群称为$Q$模$K$的一阶群上同调,记作$H^1(Q,K)$.

给定$Q$模$K$的分裂延拓$E$,给定$K$的任意两个补$C,C'$,如果$H^1(Q,K)=0$,那么$C$和$C'$是共轭的.证明.$E=K\rtimes Q$.存在提升单同态$l:Q\to E$的像为$C$,又存在提升单同态$l':Q\to E$的像为$C'$.取提升对应的因式集$f,f'$都为0,于是$f'-f=0$,于是存在映射$h:Q\to K$满足$h(x)=l'(x)-l(x)$,那么$xh(y)-h(xy)+h(x)=0$,于是$h$是一个导数.既然一阶群上同调是0,这说明$h$是一个主导数,于是$l'(x)-l(x)=xa_0-a_0,\forall x\in Q$.于是$l(x)=a_0-xa_0+l'(x)=a_0+l'(x)-a_0$,于是$l'$和$l$关于$a_0$共轭,但是$im l=C,im l'=C'$,于是$C$和$C'$经过$a_0$共轭.

最后指出,Schur-Zassenhaus定理提供了一个使得二阶群上同调为0的条件,即$Q,K$是有限群并且阶数互素的时候.在这种情况下,延拓$E$总是半直积,如果$C$和$C'$是$K$的两个补,那么定理告诉二者是共轭的.








