\chapter{复分析}

数学上能够称为基本定理的内容并不多,这样的定理一是因为历史因素——它曾作为主要问题被探究,更重要的是它可以作为一套理论的核心或者说基石.柯西积分定理就是这样的基本定理,它当之无愧的是复分析的基石.由它可得出大量的具有深度的推论,并且定理自身也有许多值得细究的地方.

在复平面上可以定义函数在曲线上的积分,一个自然的问题是在什么条件下积分不依赖于两点之间曲线的选取.柯西积分定理回答的就是这个问题,粗略的讲,定理断言如果函数是定义在特定区域上的解析函数,那么区域内的任意一条闭曲线上函数的积分为0.

为了描述柯西积分定理的条件,一些严格定义需要明确.首先是一个没有表面上那么浅显的问题:如何严格描述一个闭曲线围起的点集.直觉告诉我们,一个不自交的闭曲线在复平面的补集由两个连通分支构成,其中一个是有界集,另一个是无界集,这里有界集中的点就称为该曲线的内点.这是Jordan曲线定理,它的证明并没有和直觉上一样容易,事实上这个定理的证明困扰了数学家很多年.另一种解决这个定义的办法是借助缠绕数.给定复平面上分段光滑(可微)闭曲线$C$,对任意$C$补集中的点$z$,定义$z$的缠绕数为$\mathrm{Ind}_C(z)=\frac{1}{2\pi i}\int_C\frac{\mathrm{d}\zeta}{\zeta-z}$.在闭曲线补集上,每个连通分支中点的缠绕数是固定的,由此可定义不自交的闭曲线的内点就是缠绕数不为0的点.

在复分析的早期历史,人们通过一些简单的反例意识到柯西积分定理对区域的要求需要不能有"洞",这个条件称作单连通,它的严格描述为:一个区域称为单连通的,如果这个区域内的每个不自交闭曲线的内部都包含于这个区域.现代人们对单连通的描述不再借助Jordan曲线定理,而是抓住不自交闭曲线会连续收缩为一个点这个思想,也即同伦定义:一个区域称为单连通区域,如果任意两个区域内的具有相同初始点的闭曲线,存在连续的同伦变换.

至此我们可以严格描述柯西积分定理的条件:如果$f(z)$是一个单连通区域$D$上的解析函数,$C$是$D$中的一条可求长的简单闭曲线,那么$f(z)$在$C$上的积分为0.

值得一提的是在19世纪初,高斯和泊松就已经分别在思考复平面上曲线积分蕴含的思想,但是他们都没有相关论文.柯西作为第一个发表关于这个定理的重要论文,使得这个定理以他命名.

在1822年柯西发表的论文中证明了积分定理在曲线取为长方形边界这种特殊情况下是成立的.但是在柯西早期工作的证明中默认了$f(z)$导函数的存在性,甚至默认了导函数的连续性.克莱因认为对此的一种解释是,和柯西时代的其他大多数人一样,柯西认为一个连续函数总是可导的,并且导函数不连续只有在原函数不连续的情况下才可能发生.

随后在1846年柯西发表了积分定理的另一种证明,这个证明借助了全微分的概念,以及我们如今称为格林定理的内容.格林定理最初被格林于1828年发表于一个相对私人的小册子上,直到1850年才发表到数学期刊上.柯西本人是否独立于格林的工作而得到格林定理我们不得而知.这是柯西对积分定理所发表的最后一篇重要论文.

接下来对积分定理的理解做出贡献的人物是黎曼.在黎曼短暂的40年生命中,他在数学尤其是复分析上做出了大量杰出的工作.在1851年黎曼给出了复变函数在一点处可微需要满足的条件,也即满足Cauchy-Riemann方程,反过来他证明了如果复变函数满足方程并且实虚部的偏导数连续,则复变函数在该点可微.结合格林定理,黎曼改进了积分定理的描述:如果$f(z)$在一个不自交闭曲线上和内部解析并且有连续导数,那么在该曲线上积分为0.黎曼这个版本的积分定理摆脱了全微分这个条件的限制,使得积分定理可以运用到更多的函数上.

至此19世纪结束.从现代的观点看,由于未知的内容过多,整个19世纪人们对复分析的探究是具体化想法优于严格化证明.并且尽管整个19世纪存在大量关于复分析思想的思考,但是同一时代人们的思想和处理方法比较相互独立.在20世纪初Goursat的工作回归了理论的基础,从这开始整套复分析理论开始走向统一.

在1900年Goursat证明了黎曼版本的积分定理的条件中,导函数的连续性是可以去掉的.他对此提供了一种全新的证明思路,即分割区域为一类规则的小区域和靠近曲线的一类不规则的小区域,再分别估计规则的和不规则的区域上边界的积分.但是同时Goursat也添加了一些比较麻烦的条件使得他的证明得以正确.这导致在实用角度上大量的函数都无法运用该定理.

Pringsheim随后改进了Goursat的证明思路,他先用反复四等分三角形的方式证明对三角形边界的情况成立,接下来用三角形边界的情况说明多边形的边界的情况成立,接下来用内接多边形的边界逼近曲线说明一般情况成立.他的证明有一个瑕疵,和Goursat的证明一样,他们都回避了单连通这个条件,而是用"函数在曲线上和内部都解析"这个条件取代.

接下来来在20世纪中期出现了很多拓扑角度的对柯西积分定理的证明,而Pringsheim的证明依旧被认为是最自然最简洁的证明,因此这个证明收录到大量复分析教材中.
\newpage
\section{全纯函数的分析性质}
\subsection{全纯与解析}

我们先来粗略整理一下$\mathbb{C}$上的拓扑.首先$\mathbb{C}$的拓扑定义为度量$d(x_1+iy_1,x_2+iy_2)=\sqrt{(x_1-x_2)^2+(y_1-y_2)^2}$所诱导的拓扑.那么在$x+iy\mapsto (x,y)$的对应下,$\mathbb{C}$和$\mathbb{R}^2$是等距同构的,换句话说如果$z_1=x_1+iy_1$和$z_2=x_2+iy_2$,那么$d(z_1,z_2)=d((x_1,y_1),(x_2,y_2))$.于是从拓扑意义上或者度量空间意义上讲,这两个空间都是完全等价的.

这一段我们罗列$\mathbb{C}$的一些拓扑性质,它们都是欧氏空间本身已经具备的性质.$\mathbb{C}$是完备的度量空间;它是局部紧致的Hausdorff空间,并且它的紧子集可以用有界闭集来完全刻画;它是连通空间,并且连通开子集有属于自己的名字"区域";它的开子集是连通的当且仅当是道路连通的;它是第二可数空间;最后$\mathbb{C}\to\mathbb{C}$的映射可以记作$f(z)=f(x+yi)=u(x,y)+iv(x,y)$,这里实际上是直接把$\mathbb{C}$等同于了$\mathbb{R}^2$,那么$f$连续当且仅当$u,v$都是连续函数.

序列和级数.$\mathbb{C}$上还可以讨论序列收敛,按照Hausdorff条件说明序列收敛的时候仅会收敛直一个点.设$z_n\in\mathbb{C}$,无穷和式$\sum_{n\ge1}z_n$称为级数,称它是收敛的,如果部分和$S_n=\sum_{k=1}^nz_n$序列收敛到$z_0\in\mathbb{C}$.按照空间的完备性,这等价于讲,对任意的$\varepsilon>0$,存在正整数$N$使得$n\ge N$和任意$r\ge1$,有$|z_{n+1}+\cdots+z_{n+r}|<\varepsilon$.这也称为柯西收敛准则.另外如果$\sum_{n\ge1}|z_n|$收敛,则称$\sum_{n\ge1}z_n$绝对收敛,于是按照柯西收敛准则,绝对收敛级数必然收敛.如果一个级数收敛但不绝对收敛,则称它是条件收敛的.

函数项级数.如果$\{f_n(z)\}$是一个函数列,则称$\sum_{n\ge1}f_n(z)$是函数项级数.使得级数收敛的$z$的取值构成的集合称为函数项级数的收敛点集.注意它和幂级数函数的收敛域概念不同.函数项级数的一个重要性质叫一致收敛性,它可以保证函数项的绝大多数公共的分析性质传递给收敛函数.称函数项级数$\sum_{n\ge1}f_n(z)$在收敛点集的子集$E$上一致收敛,如果对任意$\varepsilon>0$,存在正整数$N$使得$n\ge N$的时候,对任意$x\in E$总有$|f(z)-S_n(z)|<\varepsilon$.

一致收敛的准则和实情况是一样的,这里我们简单列举两个:
\begin{enumerate}
	\item 柯西收敛准则:$\sum_{n\ge1}f_n(z)$在$x\in E$上一致收敛当且仅当对任意$\varepsilon>0$,存在正整数$N$使得对任意$n\ge N$和任意$r\ge1$,总有$|f_{n+1}(z)+\cdots+f_{n+r}(z)|<\varepsilon$.
	\item 维尔斯特拉斯判别法:若$|f_n(z)|\le M_n,x\in E$,并且正项级数$\sum_{n\ge1}M_n$收敛,那么函数项级数$\sum_{n\ge1}f_n(z)$在$E$上一致收敛.
\end{enumerate}

和实情况一样,一致收敛不仅把函数项的公共的连续/可微/可积性质传递给极限函数,还允许在特定连续性可微性要求下的极限号/可导号积分号与求和号可交换.这个结果我们会在后文完整给出.

幂级数函数.给定复数$a$,和复数列$\{c_n\}$称为系数列,一个幂级数函数是指形如$\sum_{n\ge0}c_n(z-a)^n$的函数项级数.通过简单的平移变换,可以不妨讨论幂级数函数$\sum_{n\ge0}c_nz^n$.幂级数对收敛域的刻画和实情况是类似的:任一幂级数都有收敛半径$R$,在圆盘$|z|<R$内幂级数绝对且内闭一致收敛,在圆盘的闭包外处处发散,并且这里的收敛半径就是上极限$\limsup_{n\to+\infty}\sqrt[n]{|c_n|}$的倒数.
\begin{proof}
	
	这一段证明,如果幂级数在点$z_0\not=0$处收敛,则幂级数在$|z|<|z_0|$上绝对收敛并且内闭一致收敛.按照$\sum_{n\ge0}c_nz_0^n$收敛,说明有上界$|c_nz_0^n|\le M$,于是当$|z|<|z_0|$时候有$c_nz^n=c_nz_0^n\cdot\left(\frac{z}{z_0}\right)^n$,后者作为函数项的级数是绝对收敛的,前者通项一致有界,按照Abel-Dirichlet判别法,说明幂级数在$|z|<|z_0|$上绝对收敛.内闭一致收敛则是直接考虑闭圆盘$|z|\le k|z_0|,k\in(0,1)$,那么$|c_nz^n|\le Mk^n$,按照维尔斯特拉斯判别法知在闭圆盘上一致收敛.
	
	这一段说明收敛半径$R$的存在性.倘若幂级数处处收敛,则$R=+\infty$;倘若除0以外处处发散,则$R=0$.现在设幂级数在点$z_1$处收敛,在点$z_1'$处发散.倘若$|z_1|=|z_1'|$,那么$R$就是这个模长,此时不可能存在$|z|\le R$以外的点处收敛,否则会导致$z_1'$处收敛.倘若$|z_1|\not=|z_1'|$,那么必然有$|z_1|<|z_1'|$,否则同样的矛盾会产生.现在反复将连接$z_1,z_1'$的直线段二等分,按照闭区间套会得到一个点$z_2$,可验证$R=|z_2|$满足要求.
	
	最后说明收敛半径的表达式.为此考虑实幂级数$\sum_{n\ge0}|c_n|x^n$,数学分析理论告诉我们这个实幂级数的收敛半径就是我们想得到的$R$的表达式,把它记作$R_1$,设原本复幂级数的收敛半径是$R_2$,于是只需说明$R_1=R_2$.如果$0<r<R_1$,那么$\sum_{n\ge0}|c_n|r^n$收敛,导致$\sum_{n\ge0}c_nr^n$收敛,于是得到$R_1\le R_2$.另一方面如果$0<r<R_2$,上一段说明$\sum_{n\ge0}|c_n|r^n$收敛,于是$r\le R_1$,得到$R_2\le R_1$,综上$R_1=R_2$.
\end{proof}

我们通常也会考虑局部紧致空间$\mathbb{C}$的单点紧致化$S^2$,或者记作$\overline{\mathbb{C}}$,会把添加的点记作$\infty$.那么这一点的开邻域基可以取做$\{z\in\mathbb{C}\mid |z|>n\},n\ge1$.称$\mathbb{C}$的一个区域$D$是单连通区域,如果$\overline{\mathbb{C}}-D$是连通集,否则会称$D$是多连通区域.另外如果$D$在$\overline{\mathbb{C}}$中的全体连通分支的基数为$\alpha$,就称$D$是一个$\alpha$连通区域.

现在讨论可微性.给定$\mathbb{C}$上的开集$U$,称复值(复变)函数$f:U\to\mathbb{C}$在点$x_0\in U$处可导,如果存在有限极限$\lim_{h\to0}\frac{f(z_0+h)-f(z_0)}{h}$.此时称这个极限为$f$在$x_0$处的导数,记作$f'(z_0)$.

\begin{enumerate}
	\item 和实情况相同,复值函数可导点必然连续.这从$f(z+h)-f(z)=\frac{f(z+h)-f(z)}{h}\cdot h$直接看出.
	\item 和实情况相同,两个复值函数在一点可导,那么它们的和与积都在该点可导,满足:$(f+g)'(z_0)=f'(z_0)+g'(z_0)$和$(fg)'(z_0)=f'(z_0)g(z_0)+f(z_0)g'(z_0)$.另外如果$f,g$在$z_0$可导并且$g'(z_0)\not=0$,那么$f/g$在$z_0$可导并且满足$(f/g)'(z_0)=\frac{g(z_0)f'(z_0)-g'(z_0)f(z_0)}{g^2(z_0)}$.
	\item 和实情况相同,可导函数的复合满足链式法则,即如果$f$在$z_0$可导,$g$在$f(z_0)$可导,那么$g\circ f$在$z_0$可导,并且满足$(g\circ f)'(z_0)=g'(f(z_0))f'(z_0)$.
\end{enumerate}

若复变函数在开集$U\subset\mathbb{C}$上处处可导,则称$f$在$U$上可导或可微.下一段我们会解释实复情况下可微是有区别的,为了区分实复情况会把复变函数的可微性改称为全纯性.

存在$\mathbb{C}$到$\mathbb{R}^2$的同胚为$a+bi\mapsto(a,b)$.于是一个复变函数$f$可以理解为$\mathbb{R}^2$上的映射$F$,即设$f(a+bi)=u(a,b)+iv(a,b)$,则$F(a,b)=(u(a,b),v(a,b))$.在这个对应下,实复情况的可微的定义是不同的.对于实情况,$\mathbb{R}^2$是一个二维的线性空间,这时一点可微视为存在线性逼近时,用于逼近的线性映射可以取任意的$2\times2$矩阵.对于复情况,它需要视为一维线性空间,此时可微理解为存在线性逼近时,用于逼近的线性映射是数乘复数.如果取一组基$\{1,i\}$,那么数乘复数对应的需要有形式$\left(\begin{array}{cc}a&b\\ -b&a\end{array}\right)$.于是复情况下的可微是实情况下的特例,因为复情况下能够线性逼近的线性变换的种类被做了限制.反过来如果$\mathbb{R}^2$上的映射$F$在一点处是连续可微的,并且该点的全微分具有上述矩阵形式,那么对应的复变函数$f$在该点是可微的.

整理一下上述讨论,我们得到全纯的CR方程条件.复变函数$f(x+yi)=u(x,y)+iv(x,y)$在一点$z=x+yi$全纯,则$F$在$\mathbb{R}^2$的点$(x,y)$处满足如下Cauchy-Riemann方程,并且满足$f'(z)=\frac{\partial u}{\partial x}-i\frac{\partial u}{\partial y}$.反过来设$F$在$(x,y)$处是连续可微的,那么$f$在$z=x+yi$处是全纯的.
$$\frac{\partial u}{\partial x}=\frac{\partial v}{\partial y};\frac{\partial u}{\partial y}=-\frac{\partial v}{\partial x}$$
\begin{proof}
	
	第一个命题.设$f'(z)=a+bi$,设$w=h+ik$,按照全纯性得到等式$f(z+w)-f(z)=f'(z)w+\tau(w)w$,其中$\lim_{w\to0}\tau(w)=0$.拆开上述等式,得到:
	$$F(x+h,y+k)-F(x,y)=(ah-bk,bh+ak)+\tau_1(h,k)h+\tau_2(h,k)k$$
	
	其中$\lim_{(h,k)\to0}\tau_i(h,k)=0,i=1,2$.这说明$f$在点$z=x+yi$全纯时有$F$在$(x,y)$处可微,并且全微分为$\left(\begin{array}{cc}a&-b\\ b&a\end{array}\right)$.这里$a=\frac{\partial u}{\partial x}=\frac{\partial v}{\partial y}$,$b=-\frac{\partial u}{\partial y}=\frac{\partial v}{\partial x}$.于是$F$满足CR方程,并且$f'(z)$具有命题中的表示.
	
	第二个命题.如果连续可微的函数$u,v$满足CR方程,取$t$为命题中的$f'(z)$,反向写上述推导得到$f$在$z$处全纯并且导数是$t$.
\end{proof}

至此我们给出了全纯函数的复可微性,接下来讨论全纯函数的曲线积分.给定实直线上的闭区间$[a,b]$,复平面上的曲线是指一个映射$\gamma:[a,b]\to\mathbb{C}$,根据不同的目的会要求这个映射满足不同的条件,这里我们要求复平面上的曲线是$C^1$的,即连续可微的,换句话讲,如果记$\gamma(t)=\gamma_1(t)+\gamma_2(t)i$,这里$\gamma_1$和$\gamma_2$都是实值函数,那么要求$\gamma_1$和$\gamma_2$都是连续可微的.称点$\gamma(a)$是曲线的起点,$\gamma(b)$是曲线的终点.称复平面上的一个点是该曲线上的点,如果它可以表示为$\gamma(t_0),t_0\in[a,b]$.

把曲线的导数记作$\gamma'(t)=\gamma'_1(t)+\gamma'_2(t)i$.那么曲线的和积商的导数满足我们熟知的形式.另外如果$\psi$是实直线上两个闭区间之间的可微映射$[c,d]\to[a,b]$,那么对$[a,b]$上的曲线$\phi$,有$\phi\circ\psi$是$[c,d]$上的曲线.另外如果$f:U\to\mathbb{C}$是全纯函数,那么对任意曲线$\gamma:[a,b]\to\mathbb{C}$,有$f\circ\gamma$也是$[a,b]$上的曲线,并且满足链式法则$(f\circ\gamma)'(t)=f'(\gamma(t))\gamma'(t)$.

有限个曲线拼凑而成的映射称为分段连续可微曲线.即曲线$\gamma$由一组有限个曲线$\{\gamma_1,\gamma_2,\cdots,\gamma_n\}$构成,设每个$\gamma_i$是$[a_i,b_i]$上的曲线,定义$\gamma_i(b_i)=\gamma_{i+1}(a_{i+1})$.称$\gamma_1(a_1)$是分段曲线$\gamma$的起点,$\gamma_n(b_n)$是它的终点.

称复平面的一个开集$U$是连通的,如果任意两个点$a,b\in U$存在分段曲线相连,即存在分段曲线以$a$为起点以$b$为终点,并且这个曲线的像包含于开集中.这里我们解释下这个新的连通性和拓扑中连通性的关系.在拓扑中称一个空间是连通的,如果它不存在非平凡的开闭子集.称空间是道路连通的,如果任意两个点存在连续曲线将他们相连.一般空间上道路连通推出连通,但反过来未必成立.复平面$\mathbb{C}$同胚于$\mathbb{R}^2$,欧式空间的开子集作为子空间,连通和道路连通是一致的.于是$\mathbb{C}$上开子集的拓扑连通性等价于拓扑道路连通性(即要求连续曲线相连).那么我们所定义的分段曲线相连是连续曲线相连的特例,于是这里新定义的连通性推出拓扑意义的连通性.下面会证明实际上二者在开集上是等价的,另外我们强调上述两种连通性只在开集上等价,对于其他类型的子集未必成立.
\begin{proof}
	
	假设$U$是拓扑意义的连通空间.需要证明$U$中任意两个点有分段连续可微的曲线将他们相连.取定$z_0\in U$,记$V$为所有$U$中的可以用分段曲线与$z_0$相连的点构成的集合.那么对$V$中每个点$z$,$z$存在足够小的开球邻域$B_{\varepsilon}(z)$包含于$W$中,这个小开邻域中每个点都可以与$z$用直线段相连,直线度自然是连续可微的,这说明了$V$中每个点都是开集.同理可说明$U$中不能与$z_0$相连的点构成了开集$W$,于是$U=V\cup W$是不交的两个开集,连通性就说明$W=\varnothing$.
\end{proof}

如果一个全纯函数$f$在一个连通开集$U$上处处导数为0,那么这个函数是常值函数.
\begin{proof}
	
	任取$U$中两个点$z,w$,连通性说明存在分段曲线连接两个点.先假设存在曲线$\gamma:[a,b]\to U$连接这两个点,那么$f\circ\gamma$是$[a,b]$上的可微的函数,链式法则说明导数恒为0,于是有$f(z)=f(\gamma(a))=f(\gamma(b))=f(w)$.
	
	现在设连接$z,w$的是分段曲线$\gamma=\{\gamma_1,\gamma_2,\cdots,\gamma_n\}$,其中每个$\gamma_i$设为$[a_i,b_i]$上的曲线.那么按照上一段所证得到$f(\gamma_i(a_i))=f(\gamma_i(b_i))$,另外有$f(\gamma_i(b_i))=f(\gamma_{i+1}(a_{i+1}))$,于是得到$f(z)=f(w)$.即$f$是常值函数.
\end{proof}

现在定义曲线积分.给定开集$U$上的全纯函数$f$,以及一条像集包含于$U$的曲线$\gamma:[a,b]\to U$.从欧式空间中的经验入手,要对$[a,b]$做分割$T:a=t_0<t_1<\cdots<t_n=b$,称所有小区间$[t_i,t_{i+1}],0\le i\le n-1$长度的最大值为分割的模长$|T|$,在每个小区间$[t_i,t_{i+1}]$中取点$\xi_i,0\le i\le n-1$,记$z_i=\gamma(t_i)$,考虑和$\sum_{k=0}^nf(\gamma(\xi_k))(z_{i+1}-z_i)$,如果当$|T|\to0$时该和式趋于一个复数,就称它为$f$在曲线$\gamma$上的线积分,记作$\int_{\gamma}f(z)\mathrm{d}z$.

数学分析中的知识告诉我们,如果这个极限存在,那么它可以表示为第二型曲线积分的形式,即设$f(z)=u(x,y)+iv(x,y)$,那么有$\int_{\gamma}f(z)\mathrm{d}z=\int_{\gamma}u\mathrm{d}x-v\mathrm{d}y+i\int_{\gamma}v\mathrm{d}x+u\mathrm{d}y$.如果$\gamma$是可求长曲线,即常值1函数在其上积分存在,并且$u,v$都连续,则第二型曲线积分存在.也就是说如果$\gamma$是可求长曲线,并且$f(z)$在$\gamma$上连续,则复曲线积分存在.

特别的,分段可微的曲线必然是可求长曲线,另外在这个条件下第二型曲线积分还可以转化为常义积分,此时整理可得$\int_{\gamma}f(z)\mathrm{d}z=\int_a^bf(\gamma(t))\gamma'(t)\mathrm{d}t$.

复曲线积分的性质.
\begin{enumerate}
	\item 设$\gamma:[a,b]\to U$是分段可微的曲线,那么可定义它的逆曲线为$\gamma\circ\tau:[a,b]\to U$,其中$\tau:[a,b]\to[a,b]$为$t\mapsto b+a-t$.把逆曲线记作$\gamma^{-1}$,那么有:$$\int_{\gamma^{-1}}f(z)\mathrm{d}z=-\int_{\gamma}f(z)\mathrm{d}z$$
	\item 线性,$\int_{\gamma}\left(af(z)+bg(z)\right)\mathrm{d}z=a\int_{\gamma}f(z)\mathrm{d}z+b\int_{\gamma}g(z)\mathrm{d}z$.
	\item 曲线积分只和曲线的像集有关,和参数的选取无关.事实上如果有$s:[c,d]\to[a,b]$可微,那么有:$$\int_a^bf(\gamma(t))\gamma'(t)\mathrm{d}t=\int_c^df(\gamma(t(s)))\gamma'(t(s))t'(s)\mathrm{d}s=\int_c^df(\overline{\gamma}(s))\overline{\gamma}'(s)\mathrm{d}s$$
	\item $\left|\int_{\gamma}f(z)\mathrm{d}z\right|\le ML$,这里$M=\sup_{z\in\gamma}|f(z)|$,而$L$是$\gamma$的长度.
	\item 复积分不再有中值定理,这个本质是因为$\mathbb{C}$不再是全序集.例如$\int_0^{2\pi}e^{i\theta}\mathrm{d}\theta=0$,但是$e^{i\theta}$不取0.
\end{enumerate}

在介绍解析函数前,需要给出幂级数函数的全纯性.设幂级数$f(z)=\sum a_n(z-z_0)^n$的收敛半径为$r$,那么:
\begin{enumerate}
	\item 按照幂级数收敛半径的公式,幂级数$g(z)=\sum na_nz^{n-1}$具有相同的收敛半径$r$.
	\item 幂级数$f(z)$在开圆盘$B_r(z_0)$上全纯,并且导函数就是$g(z)$.
	\begin{proof}
		
		不妨设$z_0=0$,对一般情况只要复合一个平移即可,对任意$|z_0|<r$,可取足够小的$\delta>0$使得$|z_0|+\delta<r$.任取复数$h$满足$|h|<\delta$.记$f(z)$的部分和与余项分别是$S_N$和$E_N$,即$f(z)=S_N(z)+E_N(z)$.那么有:
		$$\frac{f(z_0+h)-f(z_0)}{h}-g(z_0)=\left(\frac{S_N(z_0+h)-S_N(z_0)}{h}-S_N'(z_0)\right)+\left(S_N'(z_0)-g(z_0)\right)+\left(\frac{E_N(z_0+h)-E_N(z_0)}{h}\right)$$
		
		现在预先取定正实数$\varepsilon$,对上述三个式子做估计.
		
		首先是第三个式子,它可放缩为$\sum_{n=N+1}^{+\infty}|a_n|\left|\frac{(z_0+h)^n-z_0^n}{h}\right|\le\sum_{n\ge N+1}|a_n|nr^{n-1}$.由于$g(z)$是在$|z|<r$上绝对收敛的,于是可以取正整数$N_1$使得$N>N_1$的时候有第三项$<\frac{\varepsilon}{3}$.
		
		对于第二项,由于$\lim_{N\to+\infty}S_N'(z_0)=g(z_0)$,于是可取足够大的正整数$N_2$使得$N>N_2$的时候有第二型$<\frac{\varepsilon}{3}$.
		
		最后是第一项,在取定$N>\max\{N_1,N_2\}$的前提下,有第一项在$h\to0$的时候极限为0,于是可取$\delta>\delta'>0$使得$|h|<\delta'$的时候有第一项$<\frac{\varepsilon}{3}$.综上就得到$f(z)$在$B_r(0)$上全纯,并且导函数就是$g(z)$.
	\end{proof}
    \item 上一条说明幂级数是任意阶可导的,并且求高阶导数总可以逐项求导.于是解析函数总是光滑函数.在实情况下我们知道可微弱于光滑弱于解析,并且每一个都推不出后一个.在介绍Cauchy定理后我们会证明复情况下的可微函数的性质非常好:全纯性和解析性是等价的.
\end{enumerate}

解析函数.设复变函数$f$在点$z_0$的某个开邻域上有定义,称$f$在$z_0$处复解析,如果存在幂级数$\sum_{n\ge0}a_n(z-z_0)^n$和一个半径$r>0$,使得幂级数在$|z-z_0|<r$上绝对收敛,并且在这个区域内有$f(z)=\sum_{n\ge0}a_n(z-z_0)^n$.如果$f$在开集$U$上处处解析,就称$f$是$U$上的解析函数.另外$f$在点$z_0$处复解析也会说作$f$在$z_0$处可幂级数展开.
\begin{enumerate}
	\item 按照幂级数的唯一性定理,同一个函数在一点处的幂级数展开存在则唯一.事实上如果在一点处展开为$f(z)=\sum_{n\ge0}a_n(z-z_0)^n$,按照幂级数可逐项求导,每个$a_n$必然具有形式$\frac{f^{(n)}(z_0)}{n!}$.
	\item 如果$S$未必是$\mathbb{C}$的开集,称$f$在$S$处复解析,如果存在一个定义在包含了$S$的某个开集$U$上的复解析函数,使得它在$S$上的限制是$f$.
	\item 两个复解析函数的和,积,商,复合,在适当做约定的时候都是复解析函数,不再赘述.
\end{enumerate}

复解析这个性质实际上并不是属于单点的性质,而是一旦成立则在收敛域的内点处处满足的性质.即,给定幂级数$f(z)=\sum_{n\ge0}a_nz^n$,设收敛半径为$r>0$,那么$f$在开圆盘$D(0,r)$上处处解析.
\begin{proof}
	
	取$|z_0|<r$,需要证明$f$在$z_0$处复解析.取$s>0$使得$|z_0|+s<r$,那么$z=z_0+(z-z_0)$,于是$f(z)=\sum_{n\ge0}a_n\left(\sum_{k=0}^n\left(\begin{array}{c} n\\ k\end{array}\right)z_0^{n-k}(z-z_0)^k\right)$.若$|z-z_0|<s$,那么$|z_0|+|z-z_0|<r$,于是下述级数收敛:
	$$\sum_{n\ge0}|a_n|(|z_0|+|z-z_0|)^n=\sum_{n\ge0}|a_n|\left(\left(\begin{array}{c} n\\ k\end{array}\right)|z_0|^{n-k}|z-z_0|^k\right)$$
	
	于是可以交换$f(z)$上述展开式的求和顺序,得到$f(z)=\sum_{k\ge0}\left(\sum_{n\ge k}a_n\left(\begin{array}{c} n\\ k\end{array}\right)z_0^{n-k}\right)(z-z_0)^k$.
\end{proof}

解析函数零点的孤立性,或者称为解析函数的唯一性定理.设$U$是复平面中的连通开集(连通开集通常称为区域),
\begin{enumerate}
	\item 若$f$在$U$上解析,并且非常值,那么每个零点都是孤立点,即每个零点存在开邻域上恰好只有自身一个零点,换句话说零点集是离散点集.
	\item $f,g$是$U$上解析函数,设$S$是$U$的非离散子集,满足在其上恒有$f(z)=g(z)$,那么在整个$U$上恒有$f(z)=g(z)$.
\end{enumerate}
\begin{proof}
	
	从1推2只需考虑解析函数$f-g$,于是我们只需证明1.设全体满足存在开邻域上恒取0的点构成的$U$的子集为$S$,那么$S$自然是开集.为说明它的闭集,只需说明它的补集是开集,假设$U$中的点$a$不在$S$中,在点$a$处做幂级数展开,如果这一点取值非0,那么在足够小的开邻域上有$f(z)=f(a)+(z-a)g(z)$,于是当$|z-a|$足够小的时候$f(z)$总不取0,于是这个开邻域属于$S$的补集;如果这一点取0,那么可设在$a$的足够小的开邻域上满足$f(z)=a_m(z-a)^m(1+(z-a)h(z-a))$,这里当$|z-a|$足够小的时候括号中的函数不取0,于是在$a$的这个足够小的开邻域中除了点$a$以外都不取0,于是这个开邻域包含于$S$的补集.至此我们说明了$S$是既开又闭的子集,按照连通性,$S$要么是整个$U$要么是空集.按照条件非常值,于是$U$是空集,也就是说如果存在零点,那么零点只能是孤立点.
\end{proof}

从唯一性定理可得出解析函数环没有非平凡零因子这个性质,也即如果区域$U$上两个解析函数满足$fg\equiv0$,那么要么$f\equiv0$要么$g\equiv0$.事实上只要取$U$上的一个收敛点列$\{a_n\}$,记极限$a_0\in U$,那么$f(a_n)g(a_n)\equiv0$得到要么存在子列上恒有$f\equiv0$,要么存在子列上恒有$g\equiv0$,按照唯一性定理,它们分别得到$f$或$g$在$U$上恒为0.

设$f$为开集$U$上的复解析函数,称它是$U$上的解析同构,如果$V=f(U)$也是开集,并且存在解析函数$g:V\to U$满足$f\circ g=\mathrm{id}_V$和$g\circ f=\mathrm{id}_U$.称$f$在点$z_0$处是局部解析同构,如果存在包含$z_0$的开集$U$使得$f$限制在$U$上是解析同构.

我们接下来要证明的定理是复情况的局部和整体逆映射定理.先来证明局部情况:如果$f$是开集$U$上的解析函数,如果在点$z_0\in U$处导数非0,则$f$在该点是局部解析同构的.逆映射定理存在多种证明,这里我们先来形式构造逆幂级数函数,再证明逆的收敛性.

\begin{proof}

首先设$f(x)$是常数项为0,一次项系数$a_1$非0的形式幂级数函数,先来证明存在唯一的形式幂级数$g(x)$满足$f(g(x))=x$,并且此时实际上还满足$g(f(x))=x$.

设$f(x)=a_1x-\sum_{n\ge2}a_nx^n$.需要找到形式幂级数$g(x)=\sum_{n\ge1}b_nx^n$使得$f(g(x))=x$.寻找这个$g(x)$需要解方程$a_1g(x)-a_2g^2(x)-\cdots=x$的各系数构成的方程组.这些方程满足$a_1b_1=1$,和$a_1b_n=P_n(a_2,\cdots,a_n,b_1,\cdots,b_{n-1})$.这里$P_n$是正整数系数的多元多项式.可设:
$$P_n(a_2,\cdots,a_n,b_1,\cdots,b_{n-1})=a_2P_{2,n}(b_1,\cdots,b_{n-1})+\cdots+a_nP_{n,n}(b_1,\cdots,b_{n-1})$$

其中每个$P_{k,n}$都是正整数系数的多项式,于是可以归纳的唯一的依次解出$b_1,b_2,\cdots$.这就说明了满足$f(g(x))=x$的形式幂级数$g(x)$的存在性和唯一性.

接下来需要说明这样唯一的$g(x)$还满足$g(f(x))=x$.为此,按照刚刚所证,存在唯一的形式幂级数$h(x)$满足$g(h(x))=x$,那么这样的$h$必然也是常数项为0,一次项系数非0的形式幂级数函数.于是得到$g(f(x))=g(f(g(h(x))))=g(h(x))=x$.

接下来我们设上述$f$是收敛的幂级数函数,现在来证明上述唯一的形式幂级数$g$也是收敛的幂级数.不妨约定$a_1=1$,否则可以以收敛的幂级数函数$a_1^{-1}f(x)$代替$f(x)$.现在取$f^*(x)=x-\sum_{n\ge2}a_n^*x^n$.这里每个$a_n^*$是满足$|a_n|\le a_n^*$的非负实数.取$\phi(x)=\sum_{n\ge1}c_nx^n$为$f^*(x)$的形式逆幂级数函数.那么$c_1=1$,并且有$c_n=P_n(a_2^*,\cdots,a_n^*,c_1,\cdots,c_{n-1})$.这里$P_n$和之前的定义相同.归纳可以说明每个$c_n\ge0$.并且有$|b_n|\le c_n$.

现在只要取特定的$f^*$,使得形式逆$\phi$能够容易的计算,并且能够容易得出它有正的收敛半径,这就说明了$g$的收敛性.按照$f$的收敛性,可取足够大的实数$A$使得$n\ge0$时候总有$|a_n|\le A^n$,于是干脆把$a_n^*$取为几何数列$A^n$.此时有$f^*(x)=x-\frac{A^2x^2}{1-Ax}$.按照定义$\phi(x)$满足$f^*(\phi(x))=x$,这解出$\phi(x)$的下述表达式,它是收敛幂级数的复合,于是它收敛.
$$\phi(x)=\frac{1+Ax-\sqrt{(1+Ax)^2-4x(A^2+A)}}{2(A^2+A)}$$

最后来证明逆映射定理.先设$z_0=0$和$f(z_0)=0$.于是$f$在0处解析,于是在以0为中心的一个开圆盘$D$上可展开为收敛的幂级数函数.于是按照上述所证,可取$0$为中心的一个开圆盘$V_0$使得形式逆映射在$V_0$上收敛,并且$g(V_0)\subset D$.再取$U_0=f^{-1}(V_0)$,记$f$在$U_0$上的限制为$f_0$,于是限制函数$f_0$和$g_0$互为逆映射,这就完成了$z_0=f(z_0)=0$的情况.

最后对任意的幂级数$f=\sum a_n(z-z_0)^n$,记$w=z-z_0$和$F(w)=f(z)-f(z_0)=\sum_{n\ge1}a_nw^n$.那么此时有$F(0)=0$,于是得到$F$的局部逆映射$G(w)$,再取$w_0=f(z_0)$和$g(w)=G(w-w_0)+z_0$,于是$g$是$f$的一个局部逆映射,完成证明.
\end{proof}

整体解析同构定理:若$f$在开集$U$上解析,并且单射,记$V=f(U)$,那么$f$是$U$到$V$的解析同构,另外此时$f$的导数处处非0.
\begin{proof}
	
	先承认一个引理:一个开集$U$上单的解析函数不能存在一点导数为0.在后文中我们会用Rouch\'e定理给出它的证明.
	
	此时$f$是$U$到$V$的双射,于是可定义逆映射$g:V\to U$.任取$z_0\in U$,按照双射说明$f'(z_0)\not=0$,于是按照局部逆映射定理,存在它的开邻域$U_0$使得$f$在$U_0$上的限制是解析同构.这说明了$g$在$f(z_0)$处是解析的.让$z_0$遍历$U$,就得到$g$是解析函数.
\end{proof}
\newpage
\subsection{Cauchy定理}

我们从道路无关性和原函数的关系讲起.给定开集$U$和$U$上的连续函数$f$,倘若$f$在$U$上有原函数$g$,即$g'=f$,那么$f$在$U$上曲线积分的值只依赖于起点和终点,即总有$\int_{\gamma}f=g(\beta)-g(\alpha)$.
\begin{proof}
	
	如果连接$\alpha$和$\beta$的曲线$\gamma$是可微连续的,那么由链式法则得到:
	$$\int_{\gamma}f(z)\mathrm{d}z=\int_a^bg'(\gamma(t))\gamma'(t)\mathrm{d}t=g(\gamma(t))\mid_a^b=g(\gamma(b))-g(\gamma(a))$$
	
	如果连接两个点的曲线是分段可微连续的,那么在每个可微连续的分度上运用上一段结论即可.
\end{proof}

反过来,连通开集上曲线积分和道路无关可说明原函数的存在性:设$U$是连通开集,设$f$是$U$上的连续函数,如果$f$在$U$的任意闭曲线上的积分都是0,那么$f$在$U$上存在原函数.
\begin{proof}
	
	取定一点$z_0\in U$,对每个点$z\in U$,可定义函数$g$为$g(z)=\int_{\gamma}f$,这里$\gamma$为任意一条在$U$内的起点为$z_0$终点为$z$的曲线.按照闭曲线的积分总是0,这个函数的定义是良性的.接下来只需说明$g$可导并且导函数为$f$,为此注意:
	$$\frac{g(z+h)-g(z)}{h}=\frac{1}{h}\int_{\gamma}f(t)\mathrm{d}t$$
	
	这里$\gamma$是任意一条连接$z$和$z+h$的$U$中的曲线,于是我们不妨让$h$足够小使得$z+h$在$z$的某个包含于$U$的开圆盘$D$中,此时连接$z$和$z+h$的曲线可以直接取为直线段,于是此时有$\lim_{h\to0}\frac{1}{h}\int_{z}^{z+h}f(t)\mathrm{d}t=f(z)$,完成证明.	
\end{proof}

我们接下来讨论全纯函数存在局部的原函数.给定连通开集$U$,设$f$是$U$上的全纯函数.任取$z_0\in U$,那么可取足够小半径的以$z_0$为圆心的开圆盘$D$包含于$U$.一个自然的,在$D$上定义$f$的原函数的方式为取$g(z)=\int_{\gamma}f(t)\mathrm{d}t$,这里$\gamma$取以$z_0$为起点,以$z$为终点的$D$中的一条曲线.

但是这个定义明显的依赖于曲线的选取.为此我们可以限制曲线的种类,取分段曲线为$z_0$先平行于实轴移动到和$z$的实部相同的位置,再垂直于实轴平移到$z$点.也可以先垂直于实轴平移到和$z$虚部相同的点,再平行于实轴平移到$z$点.我们接下来说明这两种曲线的选取得到的积分值相同.

称复平面上的一个矩体,是指在常义矩体的基础上要求矩体的四条边均平行于实轴或虚轴.复平面的矩体$R$定义为全部矩体内部的点和边界点构成的集合,于是矩体总是闭集.当提及闭矩体的边界时,我们不仅指作为点集意义上的边界,还包含了约定边界作为(分段)曲线是逆时针方向的.此时边界记作$\partial R$.

Goursat定理.设$R$是复平面上的闭矩体,设$f$是$R$上的全纯函数(这是在要求$f$是某个包含了$R$的开集上的全纯函数).那么有$\int_{\partial R}f=0$.另外,模仿下面证明可以看出,如果取$R$为三角形,结论是同样成立的.
\begin{proof}
	
	将闭矩体$R$两组相对的边的中点相连,这划分出四个小闭矩体$R_i,i=1,2,3,4$.于是得到$\int_{\partial R}f=\sum_{i=1}^4\int_{\partial R_i}f$,按照绝对值不等式,说明必然存在某个$R_i$满足$\left|\int_{\partial R_i}f\right|\ge\frac{1}{4}\left|\int_{\partial R}f\right|$.把这个小闭矩体记作$R^1$.继续将$R^1$按照相同方式分为四个小闭矩体,继续取满足上述要求的小闭矩体为$R^2$,这就得到一列闭矩体列$R^1\supset R^2\supset\cdots$.满足总有$\left|\int_{\partial R^{n+1}}f\right|\ge\frac{1}{4}\left|\int_{\partial R^n}f\right|$.于是得到$\left|\int_{\partial R^n}f\right|\ge\frac{1}{4^n}\left|\int_{\partial R}f\right|$.另外如果记$R=R^0$,记$\partial R^n$的长度为$L_n$,那么有$L_{n+1}=\frac{1}{2}L_n$,于是$L_n=\frac{1}{2^n}L_0$.由于$R^n$的直径趋于0,于是闭矩体套$R^n$的交$\cap_{n\ge0}R^n$恰好由一个点构成,这个点记作$z_0$.
	
	现在取$R^n$的中心点为$z_n$,那么$\{z_n\}$是柯西列,它的极限就是$z_0$.由于$f$在$z_0$处全纯,于是有足够小的以$z_0$为中心的开圆盘$V$满足$z\in V$总有:
	$$f(z)=f(z_0)+f'(z_0)(z-z_0)+(z-z_0)h(z)$$
	
	这里$h(z)$在$z_0$处的极限为0.设$n\ge N$时可使得$R^n$落在$V$中,于是此时上式可对$\partial R^n$积分,得到:
	$$\int_{\partial R^n}f(z)\mathrm{d}z=\int_{\partial R^n}(z-z_0)h(z)\mathrm{dz}$$
	
	经放缩得:$$\frac{1}{4^n}\left|\int_{\partial R}f\right|\le\left|\int_{\partial R^n}f\right|=\left|\int_{\partial R^n}(z-z_0)h(z)\mathrm{d}z\right|\le\frac{1}{2^n}L_0\mathrm{diam} R^n\sup_{R^n}|h(z)|$$
	
	按照$\mathrm{diam} R^n=\frac{1}{2^n}\mathrm{diam} R$,上式得到$\left|\int_{\partial R}f\right|\le L_0\mathrm{diam}R\sup_{R^n}|h(z)|$.这在$n\to+\infty$时右侧趋于0,这就得到积分为0.
\end{proof}

有了Goursat定理,我们可以证明全纯函数局部存在原函数.设全纯函数$f$定义在开集$U$上,那么对每个点$z_0\in U$,取足够小半径的以$z_0$为中心的开圆盘$D$包含于$U$,那么在$D$上可定义函数$g$为$g(z_1)=\int_{z_0}^{z_1}f$,这里积分曲线取做以$z_0$和$z_1$为一对能由对角线相连的复平面上闭矩体的边界构成的分段曲线.那么$g$是$D$上的全纯函数,并且满足$g'(z)=f(z)$.并且,原函数存在说明$D$上函数的曲线积分和路径无关.
\begin{proof}
	
	Goursat定理已经告诉我们$g$的定义是良性的.只需说明$g$在$D$上的导函数是$f$.为此任取$z_1\in D$,任取模长足够小的复数$h$满足$z_1+h\in D$,那么$g(z_1+h)-g(z_1)$是$f$在$z_1\to z_1+\Re h\to z_1+h$的曲线积分,这个曲线记作$\gamma$.记$h(z)=f(z)-f(z_1)$,那么有:
	$$g(z_1+h)-g(z_1)=\int_{\gamma}f(z_1)\mathrm{d}z+\int_{\gamma}h(z)\mathrm{d}z$$
	
	右侧第一项即$hf(z_1)$.最后只需验证$\lim_{h\to0}\left|\frac{1}{h}\int_{\gamma}h(z)\mathrm{d}z\right|=0$,为此只需做放缩:
	$$\left|\frac{1}{h}\int_{\gamma}h(z)\mathrm{d}z\right|\le\frac{1}{|h|}\left(|\Re h|+|\Im h|\right)\sup_{\gamma}|h(z)|$$
	
	注意到$\lim_{z\to z_1}h(z)=0$,这就完成证明.
\end{proof}

同伦.给定两个开集$U$中的连续曲线$\gamma_0$和$\gamma_1$,要求它们具有相同的起点和终点,并且它们都是定义在$[a,b]$上的曲线,即满足$\gamma_0(a)=\gamma_1(a)=\alpha$和$\gamma_0(b)=\gamma_1(b)=\beta$.称这两个曲线在$U$中同伦,如果存在连续映射$F:[a,b]\times[0,1]\to U$,满足$F(t,0)=\gamma_0(t)$和$G(t,1)=\gamma_1(t)$,并且总有$F(0,s)=\alpha$和$F(1,s)=\beta$.于是对每个$s\in[0,1]$,都有一条连续曲线$F(t,s)$.注意,两个分段可微的曲线称为同伦的,只要求了$F$是连续的,并没有要求,比方说,固定$s$时总有$F(t,s)$是分段可微的曲线.

柯西积分定理的同伦形式.
\begin{enumerate}
	\item 设$f$在开集$U$上全纯,设$\gamma_0$和$\gamma_1$是两个具有相同起点和相同终点的分段光滑曲线,如果它们在$U$中同伦,那么有积分等式:
	$$\int_{\gamma_0}f(z)\mathrm{d}z=\int_{\gamma_1}f(z)\mathrm{d}z$$
	\item 设$f$在开集$U$上全纯,设$\gamma_0$和$\gamma_1$是两个闭的分段光滑曲线,如果它们在$U$中同伦,那么有积分等式:
	$$\int_{\gamma_0}f(z)\mathrm{d}z=\int_{\gamma_1}f(z)\mathrm{d}z$$
\end{enumerate}
\begin{proof}
	
	只来证明第一条.我们的核心思路是如果$s_1,s_2\in[0,1]$的差足够小,可以修改$\gamma_{s_i}$为分段光滑的曲线$\gamma_{s_i}'$,使得相邻的两个曲线$\gamma_{s_i}'$与$\gamma_{s_{i+1}}'$上积分相同,这就会得到$\gamma_0$上积分和$\gamma_1$上积分相同.
	
	记同伦为$F:[0,1]\times[a,b]\to U$,记$F(s,t)=\gamma_s(t)$,那么对每个$s\in[0,1]$有$\gamma_s$是$U$中的连续曲线.$F$的像是紧集,记作$K$,于是$U$的补集和$K$的距离是一个有限正数.于是可取$\varepsilon>0$使得$K$中每个点为圆心,半径为$3\varepsilon$的开圆盘都包含于$U$中.
	
	按照紧集上连续函数是一致连续的,于是可取$\delta>0$,使得只要$s_1,s_2\in[0,1]$满足$|s_1-s_2|<\delta$,就有$\sup_{t\in[a,b]}\left|\gamma_{s_1}(t)-\gamma_{s_2}(t)\right|<\varepsilon$.
	
	现在取定$s_1,s_2\in[0,1]$使得$|s_1-s_2|<\delta$.在$[a,b]$中取若干固定点$a=t_0<t_1<\cdots<t_{n+1}=b$,记$z_i=\gamma_{s_1}(t_i)$和$w_i=\gamma_{s_2}(t_i)$.要求$|t_i-t_{i+1}|$足够小,使得相邻指标的四个点$z_i,z_{i+1},w_i,w_{i+1}$总能被一个半径为$2\varepsilon$的开圆盘$D_i$覆盖.并且全体$D_i,0\le i\le n$覆盖了两条曲线.
	
	现在注意我们要求了$\gamma_1$和$\gamma_0$是分段可微的曲线,并没有要求$\gamma_{s}$总是分段可微的,于是倘若上述$s$取$s_1,s_2$时对应的连续曲线并不是分段光滑的,则我们把取定的曲线上的点$z_0,z_1,\cdots,z_{n+1}$相邻的两两相连,这可以保证$z_i$与$z_{i+1}$相连的直线段包含于$D_i$中,把这个曲线记作$\gamma_s'$.
	
	现在设$f$在开圆盘$D_i$上的原函数为$F_i$,注意在$D_i\cap D_{i+1}$上$F_i$与$F_{i+1}$都是$f$的原函数,于是它们相差一个常数,结合$z_{i+1},w_{i+1}\in D_i\cap D_{i+1}$,说明:
	$$F_{i+1}(z_{i+1})-F_i(z_{i+1})=F_{i+1}(w_{i+1})-F_i(w_{i+1})$$
	
	对$i$求和就导致$f$在$\gamma_1$和$\gamma_0$上积分相同.最后我们把$[0,1]$划分为若干小闭区间,使得每个小闭区间的长度小于$\delta$,那么每个端点对应的曲线上的积分就和相邻端点对应的曲线的积分相同,这就导致$f$在$\gamma_0$和$\gamma_1$上的曲线积分相同.
\end{proof}

单连通.称复平面的一个区域$U$是单连通的,如果任意两个$U$中具有相同起点和相同终点的连续曲线都是同伦的.那么首先,凸开集$U$总是单连通集,这是因为对任意连续曲线$\gamma_0,\gamma_1:[a,b]\to U$,可以直接取同伦为$F:[0,1]\times[a,b]\to U$,$(s,t)\mapsto(1-s)\gamma_0(t)+s\gamma_1(t)$.特别的,开圆盘总是单连通集.按照我们之前给出的原函数存在性和柯西积分定理的联系,可得:
\begin{enumerate}
	\item 单连通区域上的全纯函数总有原函数.
	\item 单连通区域上的全纯函数在任一条闭曲线上的积分为0.
\end{enumerate}

为证明接下来引出的柯西积分定理的另一版本,还需要一个引理:设$f$是区域$D$上的连续函数,设$\gamma$是$D$内的可求长曲线,则对任意的$\varepsilon>0$,存在一条$D$中的折线$P$,满足$P$和$\gamma$具有相同的起点和终点,并且$P$的全部折点都在$\gamma$上.最后还满足:
$$\left|\int_{\gamma}f(z)\mathrm{d}z-\int_{P}f(z)\mathrm{d}z\right|<\varepsilon$$
\begin{proof}
	
	$\partial D$是闭集,$\gamma$是紧集,并且二者不交,于是$d(\gamma,\partial D)$是一个正数$\rho$.取有界区域$G$,满足$\gamma\subset\overline{G}\subset D$,那么$f$在$\overline{G}$上一致连续,于是对给定的$\varepsilon>0$,可取$\delta>0$使得只要$z_1,z_2\in\overline{G}$并且$|z_1-z_2|<\delta$,那么有$|f(z_1)-f(z_2)|<\frac{\varepsilon}{2L}$,其中$L$是$\gamma$的长度.取$\eta=\min\{\rho,\delta\}$,在$\gamma$上从起点至终点依次取分点$\{z_0,z_1,\cdots,z_n\}$,其中$z_0$和$z_n$分别是曲线的起点和终点,使得每段$z_{k-1}z_k$的曲线长度都小于$\eta$.将$z_{k-1}$和$z_k$相连,这就得到一条折点均在$\gamma$上,并且具有相同起点和相同终点的折现$P$.由于相邻两点的直线段长度小于$\rho$,说明$P$在$D$内.
	
	最后估计积分差.记从$z_{k-1}$到$z_k$的曲线为$\gamma_k$,记直线段为$P_k$,那么有:
	\begin{align*}
	\left|\sum\int_{\gamma_k}f(z)\mathrm{d}z-\sum\int_{P_k}f(z)\mathrm{d}z\right|&\le\sum\left|\int_{\gamma_k}f(z)\mathrm{d}z-f(z_{k-1})(z_k-z_{k-1})\right|+\left|\int_{P_k}f(z)\mathrm{d}z-f(z_{k-1})(z_k-z_{k-1})\right|\\
	&\le L\cdot\frac{\varepsilon}{2L}+L\cdot\frac{\varepsilon}{2L}=\varepsilon
	\end{align*}
\end{proof}

柯西积分定理的经典形式.如果$D$是$\mathbb{C}$的单连通区域,$f$是$D$上的全纯函数,那么对$D$内任一可求长闭曲线$\gamma$,均有$f$在其上积分为0.注意分段可微曲线总是可求长曲线.
\begin{proof}
	
	我们已经说明过对三角形的情况柯西积分定理成立,于是由于$D$是单连通区域,它包含的任一三角形的边界上的积分总为0.接下来考虑一个多边形边界的情况,总可以把多边形某些顶点相连使得多边形划分为若干两两的交只包含于边界的小三角形的并,那么$f$在多边形边界(取了曲线方向)上的积分等同于在每个小三角边界上积分的和,于是依然是0.
	
	接下来设$\gamma$是$D$内任一条可求长闭曲线,那么对任意$\varepsilon>0$可取引理中的折线段满足引理中的不等式,而闭折线的积分我们已经证明了为0,于是$f$在$\gamma$上的积分可以任意小,于是积分为0.
\end{proof}

局部柯西积分公式.设$D$是复平面上的一个开圆盘,$f$是$\overline{D}$上的全纯函数,设$\overline{D}$的边界取逆时针方向构成曲线$\gamma$,那么对任意$z_0\in D$,有:
$$f(z_0)=\frac{1}{2\pi i}\int_{\gamma}\frac{f(z)}{z-z_0}\mathrm{d}z$$
\begin{proof}
	
	记$C_r$是以$z_0$为圆心半径为$r$的开圆盘,那么可取$r$足够小使得$C_r\subset D-\{z_0\}=U_0$.记$C_r$的边界作为曲线为$\eta$.我们先来说明$\gamma$和$\eta$是同伦的.首先$\eta$可以表示为$\eta(t)=z_0+s\frac{\gamma(t)-z_0}{|\gamma(t)-z_0|}$.于是可直接取同伦为$F(s,t)=s\eta(t)+(1-s)\gamma(t)$.并且$\Im f\subset U_0$.
	
	记$g(z)=\frac{f(z)-f(z_0)}{z-z_0}$,它在$U_0$上全纯,于是按照同伦形式的柯西积分定理,得到$\int_{\gamma}g(z)\mathrm{d}z=\int_{\eta}g(z)\mathrm{d}z$.眼下我们还不能直接说明$g(z)$在$C_r$上全纯(在给出零点的黎曼定理后知这成立),所以不能直接从柯西积分定理得到上述等式右侧积分为0.不过按照$f$在紧集$\overline{C_r}$上绝对值有上界$M$,可得$\left|\int_{C_r}g(z)\mathrm{d}z\right|\le 2M\pi r$,另$r\to0$得到这个积分为0.于是得到:
	$$\int_{\gamma}\frac{f(z)}{z-z_0}\mathrm{d}z=\int_{\gamma}\frac{f(z_0)}{z-z_0}\mathrm{d}z=f(z_0)\int_{C_r}\frac{1}{z-z_0}\mathrm{d}z=f(z_0)2\pi i$$
\end{proof}

借助局部柯西积分公式,我们终于可以说明开集上的全纯函数总是解析函数:设$f$在闭圆盘$\overline{D}_R(z_0)$上全纯,记边界取逆时针作为曲线为$C_R$,则$f$在点$z_0$处可展开为幂级数$f(z)=\sum_{n\ge0} a_n(z-z_0)^n$,这里系数项具有表示:
$$a_0=\frac{1}{n!}f^{(n)}(z_0)=\frac{1}{2\pi i}\int_{C_R}\frac{f(\zeta)}{(\zeta-z_0)^{n+1}}\mathrm{d}\zeta$$
\begin{proof}
	
	局部柯西积分公式告诉我们对$z\in D_{R}(z_0)$,有$f(z)=\frac{1}{2\pi i}\frac{f(\zeta)}{\zeta-z}\mathrm{d}\zeta$.取$0<s<R$,我们断言$f$在圆心$z_0$半径$s$上有幂级数展开.对$C_R$上的点$\zeta$,总有展开:
	$$\frac{1}{\zeta-z}=\frac{1}{\zeta-z_0}\left(\frac{1}{1-\frac{z-z_0}{\zeta-z_0}}\right)=\frac{1}{\zeta-z_0}\left(1+\frac{z-z_0}{\zeta-z_0}+\left(\frac{z-z_0}{\zeta-z_0}\right)^2+\cdots\right)$$
	
	由于$\left|\frac{z-z_0}{\zeta-z_0}\right|\le\frac{s}{R}<1$,说明上述等式在$|z-z_0|\le s$时是一致并绝对收敛的.于是按照局部柯西积分公式,得到:
	$$f(z)=\sum_{n\ge0}\frac{1}{2\pi i}\int_{\gamma}\frac{f(\zeta)}{(\zeta-z_0)^{n+1}}\mathrm{d}\zeta (z-z_0)^n$$
	
	最后按照幂级数系数项的唯一性,以及$a_n=\frac{1}{n!}f^{(n)}(z_0)$得到$a_n$的表达式.
\end{proof}

注意,这个结论还告诉我们,如果$f$是定义在$U$上的解析函数,任取一点$a\in U$,那么在点$a$处做的局部幂级数展开的收敛半径,恰好就是$a$到最近的$f$的奇点之间的距离.

有时我们会把上述结论中的表达式$f^{(n)}(z)=\frac{n!}{2\pi i}\int_{C_R}\frac{f(\zeta)}{(\zeta-z_0)^{n+1}}\mathrm{d}\zeta$称为局部柯西积分公式.注意这里本质的内容是,如果$\gamma$是开集$U$上的一条分段可微的曲线,如果$z$在$\gamma$的补集中,而$g$是$\gamma$上的连续函数,则可以定义积分函数$f(z)=\int_{\gamma}\frac{g(\zeta)}{\zeta-z}\mathrm{d}z\zeta$,那么$f(z)$是一个在$\gamma$补集中的解析函数,并且导数具有公式:$$f^{(n)}(z)=n!\int_{\gamma}\frac{g(\zeta)}{(\zeta-z)^{n+1}}\mathrm{d}\zeta$$

对局部柯西积分公式换元,可得平均值公式:若函数$f(z)$在$|z-z_0|<R$内解析,则对任意$r\in(0,R)$,有公式:$$f(z_0)=\frac{1}{2\pi}\int_0^{2\pi}f(z_0+re^{i\theta})\mathrm{d}\theta$$

为了描述整体柯西积分公式,先需要缠绕数的概念.给定复平面中的闭曲线$\gamma$,对任意不在$\gamma$上的点$z_0$,定义$z_0$关于$\gamma$的缠绕数为$W(\gamma,z_0)=\frac{1}{2\pi i}\int_{\gamma}\frac{1}{z-z_0}\mathrm{d}z$.

关于缠绕数的一些基本性质:
\begin{enumerate}
	\item 缠绕数总是整数.
	\begin{proof}
		
		设$\gamma=\{\gamma_1,\cdots,\gamma_n\}$,其中每个$\gamma_i$都是$C^1$的曲线.按照有界闭区间总是微分同胚的,于是可以约定$\gamma_i$是定义在$[a_i,b_i]$的曲线,并且满足$b_i=a_{i+1},1\le i\le n-1$.于是$\gamma$是定义在$[a,b]=[a_1,b_n]$上的曲线.
		
		现在定义$F(t)=\int_a^t\frac{\gamma'(t)}{\gamma(t)-z_0}\mathrm{d}t$,它是$[a,b]$上的连续函数,并且在除去$a_i,b_i$的点上可微,在这些点上的导函数为$F'(t)=\frac{\gamma'(t)}{\gamma(t)-z_0}$.这个等式使得$e^{-F(t)}(\gamma(t)-z_0)$的导函数为0,导致有$\gamma(t)-z_0=Ce^{F(t)}$.按照$\gamma$是闭曲线,得到$\gamma(a)=\gamma(b)$.按照$z_0$不在$\gamma$上得到$C\not=0$,于是得到$e^{F(a)}=e^{F(b)}$.于是存在一个整数$k$满足$F(b)=F(a)+2\pi ik$.而$F(a)=0$,于是$F(b)=2\pi ik$.这就说明$W(\gamma,z_0)=k$.
	\end{proof}
	\item 缠绕数的定义式$f(z_0)=\frac{1}{2\pi i}\int_{\gamma}\frac{1}{z-z_0}\mathrm{d}z$是关于$z_0$的连续函数.
	\begin{proof}
		
		我们需要证明的是,给定不在曲线上的点$z_0$,要证明当$z$趋于$z_0$时有$\int_{\gamma}\left(\frac{1}{\zeta-z}-\frac{1}{\zeta-z_0}\right)\mathrm{d}\zeta$趋于0.注意函数$|z_0-\gamma(t)|$是紧集上不取0的连续函数,于是它有最小值$s>0$.当$z$足够接近$z_0$时,按照$|z-\gamma(t)|\ge|z_0-\gamma(t)|-|z-z_0|$可以使得$|z-\gamma(t)|\ge\frac{s}{2}$.于是当$z$足够接近$z_0$的时候有估计式:
		$$\left|\int_{\gamma}\frac{1}{\zeta-z}-\frac{1}{\zeta-z_0}\mathrm{d}\zeta\right|=\int_{\gamma}\left|\frac{z-z_0}{(\zeta-z)(\zeta-z_0)}\right|\le\frac{4}{s^2}|z-z_0|L(\gamma)$$
		
		这里$L(\gamma)$表示$\gamma$的长度,这就得到连续性.
	\end{proof}
	\item 根据上两条,在$\gamma$补集的每个连通分支上点的缠绕数是一个定值.另外如果$S$是$\gamma$补集中的一个无界的连通子集,则$S$中点的缠绕数恒为0.
	\begin{proof}
		
		我们只需说明最后一个断言,这是因为当$\zeta$模长足够大的时候,$\frac{1}{|\zeta-z_0|}$会足够小,但是缠绕数的积分定义的值是整数,这就只能是0.
	\end{proof}
\end{enumerate}

同源.给定开集$U$中的两条闭曲线$\gamma,\eta$,称它们是在$U$上同源的,如果对任意$U$补集中的点$z_0$,总有$W(\gamma,z_0)=W(\eta,z_0)$.称$U$中的闭曲线$\gamma$同源于0,如果对任意$U$补集中的点$z_0$总有$W(\gamma,z_0)=0$.

开集$U$上同伦的两条闭曲线必然是同源的.这是因为同源定义是两个曲线上积分相同,而这里的积分式$\frac{1}{\zeta-z}$在$z\not\in U$时总是全纯的,所以由柯西积分定理的经典形式直接得到这个结论.

链.开集$U$上全体曲线生成的自由交换群中的元称为开集$U$上的链.换句话说$U$中的链由一列曲线$\gamma_1,\cdots,\gamma_n$和一列整数$m_1,\cdots,m_n$构成,记住$\gamma=m_1\gamma_1+\cdots+m_n\gamma_n$.称一个链是闭链,如果构成它的全部曲线都是闭曲线.定义链上的积分为:$\int_{\gamma}f=\sum m_i\int_{\gamma_i}f$.于是我们可以定义一个点$z_0$关于不包含这个点的链$\gamma$上的缠绕数为$W(\gamma,z_0)=\frac{1}{2\pi i}\frac{1}{z-z_0}\mathrm{d}z$.最后类似的可以定义两个链同源,也可以定义一个链同源于0.

至此我们可以给出柯西积分定理的同源版本:设$\gamma$是开集$U$中的一条闭链,并且它同源于0,设$f$是$U$上的全纯函数,则有$f$在$\gamma$上的积分为0.
\begin{proof}
	
	证明见gtm103的147页,从149页开始还提供了第二个证明.
\end{proof}

为了证明一般版本的柯西积分公式,先来证明:设$\gamma$是开集$U$中的同源于0的闭链.那么存在$U$中不同的点$z_1,\cdots,z_n$,设$\overline{D}_i$是以$z_i$为圆心的包含于$U$的闭圆盘,设它的边界取逆时针方向为$\gamma_i$,并且约定半径足够小使得$\gamma_i$两两不交.取$m_i=W(\gamma,z_i)$,记$U_0=U-\{z_1,\cdots,z_n\}$,那么在$U_0$上$\gamma$同源于$\sum m_i\gamma_i$.另外如果$f$是$U_0$上的全纯函数,那么有积分等式$\int_{\gamma}f=\sum_{i=1}^nm_i\int_{\gamma_i}f$.
\begin{proof}
	
	记链$C=\gamma-\sum m_i\gamma_i$,任取$U$补集中的点$z_0$,那么由于$z_0$在每个$\gamma_i$外侧,于是有$W(C,z_0)=0$.另外如果$i=k$则有$W(\gamma_i,z_k)=1$,如果$i\not=k$则有$W(\gamma_i,z_k)=0$,于是总有$W(C,z_k)=W(\gamma,z_k)-m_k=0$.于是$C$同源于0.再按照同源版本的柯西积分定理,得到命题最后的积分等式.
\end{proof}

一般版本的柯西积分公式.设$\gamma$为$U$中的同源于0的闭链,设$f$是$U$上的全纯函数,设$z_0\in U$并且$z_0\not\in\gamma$,那么有如下公式.特别的,如果$\gamma$本身是不自交的闭曲线(也称为简单闭曲线),则系数$W(\gamma,z_0)=1$,此时即简单闭曲线情况下的柯西积分公式.
$$\frac{1}{2\pi i}\int_{\gamma}\frac{f(z)}{z-z_0}\mathrm{d}z=W(\gamma,z_0)f(z_0)$$
\begin{proof}
	
	$f$在$z_0$处可以展开为幂级数$f(z)=a_0+a_1(z-z_0)+\cdots$.设$C_r$为以$z_0$为圆心以$r$为半径的开圆盘,则可取$r$足够小使得$\overline{C_r}\subset U$.按照上一个定理,左边的积分可以转化为$C_r$的边界取逆时针的积分再乘以$z_0$的缠绕数.得到:
	$$\frac{1}{2\pi i}\int_{\gamma}\frac{f(z)}{z-z_0}\mathrm{d}z=W(\gamma,z_0)\frac{1}{2\pi i}\int_{C_r}\sum_{n\ge0}a_n(z-z_0)^{n-1}\mathrm{d}z=W(\gamma,z_0)f(z_0)$$
\end{proof}

这里我们指出柯西积分定理和柯西积分公式是本质相同的内容,我们已经看到柯西积分定理推出了柯西积分公式.反过来在柯西积分公式中直接取全纯函数$F(z)=(z-z_0)f(z)$,那么得到:
$$\frac{1}{2\pi i}\int_{\gamma}f(z)\mathrm{d}z=\frac{1}{2\pi i}\int_{\gamma}\frac{F(z)}{z-z_0}\mathrm{d}z=F(z_0)W(\gamma,z_0)=0$$
\newpage
\subsection{Cauchy定理的应用}
\subsubsection{一些基本推论}

首先,柯西积分公式提供了导函数的局部上阶:$|f^{(n)}(z_0)|=\left|\frac{n!}{2\pi i}\int_{C_R}\frac{f(z)}{z-z_0}\mathrm{d}z\right|\le\frac{n!M}{R^n}$.这里$C_R$是以$z_0$为圆心半径$r$的开圆盘$D_R$的圆周取逆时针,而$M$是$|f|$在圆周$C_R$上的最大值.这通常称为柯西积分不等式.

柯西积分不等式可以得出刘维尔定理:称一个全纯函数是整函数,如果它的定义域是整个复平面,刘维尔定理是说,有界的整函数必然是一个常值函数.为此只需注意到柯西积分不等式说明$|f'(z)|\le\frac{M}{R}$,这里$M$是$|f|$在整个复平面上的上界,令$R\to+\infty$即可.

Picard给出了关于整函数的比刘维尔定理更为深刻的定理,即Picard小定理:一个非常值的整函数要么取到整个复数域,要么取到整个复数域除去一个单点.

代数学基本定理可以用刘维尔定理得出.设$f$是复系数的非常数多项式,那么$f$至少包含$\mathbb{C}$中的一个根.
\begin{proof}
	
	记$f(z)=a_nz^n+\cdots+a_0$,其中$a_n\not=0$.假设$f$没有复根,则$g(z)=\frac{1}{f(z)}$是整个复平面上的解析函数.我们断言$g(z)$有界,按照刘维尔定理就说明$g$是常值函数,这和$f$非常数矛盾.
	
	将$f(z)$表示为$a_nz^n\left(1+\frac{b_1}{z}+\cdots+\frac{b_n}{z^n}\right)$.当$|z|\to+\infty$时有$|f(z)|\to+\infty$,于是$|g(z)|\to0$.于是对足够大的正数$R$,有$|g(z)|$在$B_0(R)$以外有上界,但是$|g(z)|$还在紧集$\overline{B_0(R)}$上连续,于是也有上界,导致$|g(z)|$在整个复平面上有上界.
\end{proof}

我们证明过解析函数的零点孤立性定理,已经得出了全纯和解析的等价性,于是我们有:如果$f,g$同为区域$U$上的全纯函数,它们在$U$的一个非离散子集$A$上满足$f(a)=g(a),a\in A$,换句话说,它们在$U$的一个包含了至少自身一个聚点的子集$A$上满足$f(a)=g(a),a\in A$.那么在$U$上恒有$f(x)=g(x)$.

Morera定理.设$f$是定义在开集$U$上的函数,满足对$U$内任意的分段$C^1$的闭曲线$\gamma$总有$f$在$\gamma$上的积分为0.则$f$是$U$上的解析函数.注意按照经典版本的柯西积分定理的证明,这个定理条件也可以改为,对任意$U$内的三角形边界的曲线上积分总为0,那么也可以说明$f$是解析函数.
\begin{proof}
	
	事实上,无论哪种条件,都说明了$f$在$U$上存在原函数,这个原函数就是一个全纯函数,但是我们证明过全纯和解析的等价性,于是$f$也是解析函数.
\end{proof}

注意Morera定理要求的是$U$中任意闭曲线上积分为0,而解析函数满足的积分为0还要添加闭曲线围住的点都在$U$内.事实上并不是所有解析函数都有原函数,不过我们曾证明过单连通域上这总是成立的.

这里我们可以给出复版本的一致收敛条件下,函数列的公共分析性质传递给极限函数这一结论.我们放在这里主要是因为可导性的传递需要借助Morera定理来证明,其余两个则仅需$\varepsilon-\delta$语言证明.
\begin{enumerate}
	\item 若复变连续函数列$\{f_n\}$在开集$U$上内闭一致收敛于极限函数$f(z)$,则$f(z)$是$U$上的连续函数.事实上这个命题可以和实情况一样推广为对$z$的极限符号和对$n$的极限符号在满足特定一致收敛条件时可交换.
	\item 若$\{f_n(z)\}$均在曲线$\gamma$上连续,并且函数列在$\gamma$上一致收敛于极限函数$f(z)$,那么有$\int_{\gamma}f(z)\mathrm{d}z=\lim_{n\to+\infty}\int_{\gamma}f_n(z)\mathrm{d}z$.换句话讲满足特定一致收敛条件时积分符号和极限符号可交换.
	\item 设$\{f_n\}$是开集$U$上的一列解析函数,满足对$U$的任意紧子集$K$,都有$f_n$一致收敛于同一个函数$f$,那么$f$是解析函数.另外,此时在每个紧子集上$\{f_n'\}$同样一致收敛于$f'$.
\end{enumerate}
\begin{proof}
	
	只需证明第三个断言,为此不妨设$\{f_n\}$在整个$U$上一致收敛于$f$.那么只要说明对每个正数$\delta$,有$\{f_n'\}$在$U_{\delta}=\{z\in U\mid \overline{B_{\delta}(z)}\subset U\}$上一致收敛于$f'$即可.
	
	记$F_n=f_n-f$,按照柯西积分公式,对每个$z\in U_{\delta}$,有$F_n'(z)=\frac{1}{2\pi i}\int_{C_{\delta}(z)}\frac{F_n(\zeta)}{(\zeta-z)^2}\mathrm{d}\zeta$,于是得到估计式:
	\begin{align*}
	|F_n'(z)|&\le\frac{1}{2\pi}\left|\int_{C_{\delta}(z)}\frac{F_n(\zeta)}{(\zeta-z)^2}\mathrm{d}\zeta\right|\\
	&\le\frac{1}{2\pi}\sup_{\zeta\in U}|F(\zeta)|\frac{1}{\delta^2}2\pi\delta=\frac{1}{\delta}\sup_{\zeta\in U}|F(\zeta)|
	\end{align*}
	
	这里$|F(\zeta)|$在$U$上有最大值是由一致收敛性提供的.这个不等式说明了$\{f_n'\}$在$U_{\delta}$上一致收敛于$f'$.
\end{proof}

解析函数的最大模原理.给定连通开集$U\subset\mathbb{C}$,设$f$是$U$上的解析函数,如果在点$z_0\in U$处取$|f|$在$U$上的最大值,那么$f$在整个$U$上是常值函数.
\begin{proof}
	
	假设$f$非常值,但在$U$上取到$f(U)$的最大值.设这个最大值为$M$,现在设$U_1=\{z\in U\mid |f(z)|=M\}$和$U_2=\{z\in U\mid |f(z)|<M\}$.那么$U=U_1\cup U_2$是无交并,并且$U_1$非空,并且$U_2$是开集.我们来说明$U_1$同样是开集.
	
	任取$z_0\in U_1$,按照平均值公式,得到$f(z_0)=\frac{1}{2\pi}\int_0^{2\pi}f(z_0+re^{i\theta})\mathrm{d}\theta$.这里$r$是满足$\overline{B_r(z_0)}\subset U$的任意正实数.任取一个满足的正实数$R$,那么对$0\le r<R$,就有$\int_0^{2\pi}\left(M-|f(z_0+re^{i\theta})|\right)\mathrm{d}\theta\le0$.被积函数是连续函数,说明被积函数恒为0,也就是说对任意的$\theta\in[0,2\pi]$和任意的$r\in[0,R)$,恒有$|f(a+re^{i\theta})|=M$,也即$z_0$的开邻域$B_{R}(z_0)$包含于$U_1$,于是$U_1$是开集.
	
	于是按照连通性,从$U=U_1\cup U_2$是开集的无交并,并且$U_1$非空,得到$U_2$是空集,这说明$f$是$U$上的模长恒定的解析函数.对模长恒等式求偏导数,结合Cauchy-Riemann方程,就得到$f$是常值函数,这就矛盾.
\end{proof}

最大模原理可以说明一个结论:如果$f$是连通开集$U$上的解析函数,并且可以连续延拓到$\overline{U}$上,那么$f$在$\overline{U}$上绝对值的最大值必然在边界点取到.

习题.设$f_1,f_2,\cdots,f_m$均在区域$U$上解析,若$f=\sum_{k=1}^m|f_k(z)|$在$U$的内点取最大值,证明全部$f_k$都是常值函数.
\begin{proof}
	
	设$f(z)$在点$a\in U$处取到$U$上的最大值.记$\arg f_k(a)e^{i\theta_k}=0,1\le k\le m$.记$g_k=f_k(z)e^{i\theta_k}$,记$g=\sum_{k=1}^mg_k$.那么$g_k$以及$g$都是解析函数.并且满足$|g(z)|\le f(z)\le f(a)$,但是$|g(z)|=f(a)$,说明$g$在区域$U$内取最大模,导致$g$是常值函数,即$g(z)\equiv f(a)$.说明恒有$g_1(z)+\cdots+g_m(z)=|g_1(z)|+\cdots+|g_m(z)|$,对$m$归纳可说明每个$g_k$满足$\mathrm{im}g_k\equiv0$,结合Cauchy-Riemann方程得到$g_k$均为常值函数,于是$f_k$均为常值函数.
\end{proof}

习题.非常值的整函数的实虚部作为实值函数必然无上下界.
\begin{proof}
	
	以实部有上界为例.设$f(z)=u(x,y)+iv(x,y)$,倘若$(x,y)\in\mathbb{R}^2$时恒有$u(x,y)\le M$,考虑解析函数$e^{f(z)}$,那么有$|e^{f(z)}|=e^{u(x,y)}\le e^{M}$.于是按照刘维尔定理,说明$e^{f(z)}$是常值函数,导致它的导函数$e^{f(z)}f'(z)$为0,但是$e^{z}$总不会取0,说明恒有$f'(z)$为0,即$f(z)$是常值函数.其他情况可以分别构造$e^{if(z)}$,$e^{-f(z)}$,$e^{-if(z)}$来证明.
\end{proof}

\subsubsection{洛朗级数}

洛朗(Laurent)级数是指形如$f(z)=\sum_{n=-\infty}^{+\infty}a_n(z-z_0)^n$的级数.称一个洛朗级数在$z$处收敛,如果两部分级数$f^+(z)=\sum_{n\ge 0}a_n(z-z_0)^n$和$f^-(z)=\sum_{n<0}a_n(z-z_0)^n$均在$z$处收敛.

首先要处理的是收敛域的问题.定义中的$f^+(z)$就幂级数,它的收敛域是某个开圆盘$B_R(z_0)$,而$f^-(z)$若做代换$t=1/z$,则可视为关于$t$的幂级数,它的收敛域就是某个开圆盘$B_{\frac{1}{r}}(z_0)$,于是$f^-(z)$的收敛域会是它的补集$\{z\in\mathbb{C}\mid |z-z_0|>r\}$.倘若$r<R$,则洛朗级数的收敛域为圆环$r<|z-z_0|<R$.注意和幂级数的情况类似,收敛域并不是指收敛点构成的集合,收敛域的边界点也可能是使得级数收敛的点.

我们已经看到了洛朗级数的收敛域必然是一个圆环,也即$r<|z-z_0|<R$.现在我们说明,在一个圆环$D:r<|z-z_0|<R$上解析的函数$f$,必然可以表示为洛朗级数$\sum_{n=-\infty}^{+\infty}a_n(z-z_0)^n$,并且这里展开式是唯一的,系数项具有公式:
$$a_n=\frac{1}{2\pi i}\int_{|\zeta-z_0|=\rho}\frac{f(\zeta)}{(\zeta-z_0)^{n+1}}\mathrm{d}\zeta,\rho\in(r,R)$$
\begin{proof}
	
	首先,系数项表达式里不同的$\rho$的选取,所得到的圆周闭曲线在$D$中同伦,于是积分值相同,即这个定义是良性的.
	
	任取$D$中的点$z$.取定两个$D$中的圆周$\gamma_1:|\zeta-z_0|=\rho_1$和$\gamma_2:|\zeta-z_0|=\rho_2$,使得$\rho_1<|z-z_0|<\rho_2$.那么按照同源版本的柯西积分公式,得到$f(z)=\frac{1}{2\pi i}\int_{\gamma_1-\gamma_2}\frac{f(\zeta)}{\zeta-z}\mathrm{d}\zeta$.我们断言这两部分的积分分别就是洛朗级数定义中的$f^+$和$f^-$.
	
	倘若$\zeta\in\gamma_1$,则有$|\frac{\zeta-z_0}{z-z_0}|<1$,于是得到$\frac{1}{\zeta-z}=-\sum_{n\ge1}\frac{(\zeta-z_0)^{n-1}}{(z-z_0)^n}$.并且这个级数在$\zeta\in\gamma_1$上一致收敛.
	
	同理若$\zeta\in\gamma_2$,则$|\frac{z-z_0}{\zeta-z_0}|<1$,此时有$\frac{1}{\zeta-z}=\sum_{n\ge0}\frac{(z-z_0)^n}{(\zeta-z_0)^{n+1}}$.
	
	于是带入上述积分式,得到满足命题中系数项公式的洛朗级数,这就完成了展开式存在性的证明.最后我们说明唯一性,设$f$在$D$上还有洛朗级数展开式$f(z)=\sum_{n=-\infty}^{+\infty}$.对两种洛朗展开式乘以$\frac{1}{(z-z_0)^{m+1}},m\in\mathbb{Z}$再逐项积分,利用$\int_{|z-z_0|=\rho}(z-z_0)^k\mathrm{d}z$只有在$k=-1$时取非0数$2\pi i$,其余的$k$均取0,就得到$a_n=a_n'$,完成证明.
\end{proof}
\subsubsection{孤立奇点}

给定复平面中的点$z_0$,如果存在以它为圆心的开圆盘$D$,使得$f$在$U=D-\{z_0\}$上解析,则称$z_0$是$f$的孤立奇点.本节的目标是将孤立奇点划分为三类,并分别给出它们的等价刻画.

首先,定义中的$U$实际上是个圆盘,只不过这里$r$取了0:$0<|z-z_0|<R$.因此$f$可以在$U$上做洛朗展开,一个合理的期望是洛朗展开可以反映该奇点的性质,我们会看到这的确是可行的.

可去奇点.不言而喻的,这是指可以去掉的奇点,即$f$可以延拓为$D$上的解析函数,一个孤立奇点$z_0$是可去奇点具有如下等价描述:
\begin{enumerate}
	\item $f$可以延拓为$D$上的解析函数.
	\item $\lim_{z\to z_0}f(z)$是有限复数.
	\item $f$在$z_0$的某个去心开邻域上有界.
	\item $f$在圆环$U$上的洛朗展开式中全体负次数项系数为0.
\end{enumerate}
\begin{proof}
	
	1推2和2推3是直接的.3推4,不妨设$D$的半径足够小使得$f$在$D-\{z_0\}$上满足$|f|\le M$.那么按照洛朗展开式的系数公式,对$n\ge1$有:
	$$|c_{-n}|=\left|\frac{1}{2\pi i}\int_{|z-z_0|=\rho}f(z)(z-z_0)^{n-1}\mathrm{d}z\right|\le\frac{1}{2\pi}Mr^{n-1}2\pi r=Mr^n$$
	
	这里$r$是$D$的半径,令$r\to0$就得到$c_{-n}\equiv0$.最后4推1,此时洛朗级数在扩充定义$f(z_0)=a_0$时就是幂级数,而我们知道幂级数一定是解析函数,这就说明了$f$可以延拓到$D$.
\end{proof}

极点.一个孤立奇点$z_0$称为极点,如果它满足如下等价描述的任意一个:
\begin{enumerate}
	\item $\lim_{z\to z_0}|f(z)|=+\infty$.
	\item 存在唯一的正整数$m$使得$f(z)=\frac{g(z)}{(z-z_0)^m}$,其中$g(z)$是$D$上的解析函数,并且满足$g(z_0)\not=0$.这里的$m$称为该极点的次数.另外$z_0$是$f$的$m$次极点当且仅当$z_0$是$\frac{1}{f}$的$m$次零点.
	\item $f$在$U$上的洛朗级数只有有限多个负次数项系数不为0.
\end{enumerate}
\begin{proof}
	
	1推2,首先$f$不会在$U$上常值,否则和极限式矛盾.并且零点集不以$z_0$为聚点,否则也和极限式矛盾.于是可不妨取$D$的半径足够小,使得$f$在$U=D-\{z_0\}$上没有零点.于是$\frac{1}{f(z)}$是$U$上的解析函数,并且$\lim_{z\to z_0}\frac{1}{f(z)}=0$.于是$z_0$是$\frac{1}{f(z)}$的可去奇点.可把$\frac{1}{f(z)}$延拓为$D$上的解析函数,则$z_0$是延拓后函数的零点,设这个零点的次数为$m$,也就是$\frac{1}{f(z)}=(z-z_0)^mh(z)$,这里$h(z)$是$U$上解析函数,并且$h(z_0)\not=0$.于是有$f(z)=\frac{g(z)}{(z-z_0)^m}$.这里$g(z)=\frac{1}{h(z)}$.最后说明$m$的唯一性,否则有$\frac{g(z)}{(z-z_0)^m}=\frac{h(z)}{(z-z_0)^n}$,不妨设$m>n$,两边乘以$(z-z_0)^m$,让$z\to z_0$,得到$g(z_0)=0$和要求矛盾.
	
	2推3是直接的,因为把$g(z)$做幂级数展开,再逐项除以$(z-z_0)^m$即得.3推1也是直接的,因为非负次数项极限为0,有限个负次数项极限是无穷.
\end{proof}

极点的孤立性.利用零点的孤立性定理可以得到极点的孤立性.解析函数在开集$U$上的极点集是离散点集,即极点集不会包含自身的聚点.这是因为,倘若$z_0\in U$是极点集的聚点,不妨设$U$连通,不妨设$f$不在$U$上常值,于是$f$在$U$上的零点集是孤立点集,并且$z_0$不会是零点集的聚点,否则和极限为无穷远点矛盾.于是可取以$z_0$为圆心的足够小半径的开圆盘$D$,使得$D$不含$f$的零点.现在在$D$上可定义函数$g(z)$,在$f$的非极点上取$1/f(z)$,在极点上取0.那么$g(z)$在$D-\{z_0\}$上是解析的,并且可以说明$z_0$是$g$的可去奇点,导致$g(z)$在$D$上解析,按照零点的孤立性定理得到$g(z)\equiv0$,这导致$f$在$z_0$的一个去心开邻域中没有定义,这矛盾.

本性奇点.按照前两种情况的等价描述,直接可以得到如下两命题之间的等价性,称满足任一条的孤立奇点为本性奇点:
\begin{enumerate}
	\item $\lim_{z\to z_0}f(z)$不存在,并且$\lim_{z\to z_0}|f(z)|$也不趋于无穷.
	\item $f$在$U$上的洛朗展开式中负次数项系数非0的项有无穷个.
\end{enumerate}

这里我们做一个注解.有时会把复平面的无穷远看作一个单点,这实际上是把复平面单点紧致化为球面,在这个观点下,极限式$\lim_{z\to z_0}|f(z)|=+\infty$等价于说$f(z)$趋于了这个无穷远点.于是,在这个观点下,谈及收敛概念时要包含趋于无穷远点这个情况,此时可去奇点等价于$\lim_{z\to z_0}f(z)$收敛到有限复数;极点等价于$\lim_{z\to z_0}f(z)$收敛到无穷远点;本性奇点等价于$\lim_{z\to z_0}f(z)$不存在,换句话说,这个极限既不收敛到有限复数,也不收敛到无穷远点.

本性奇点具有一些非常复杂的性质,我们可以从下面威尔斯特拉斯定理初步体会:如果$z_0$是$f$的本性奇点,那么对$z_0$的任一$f$在其上解析的去心开邻域$U$,总有$f(U)$在$\mathbb{C}$中稠密.
\begin{proof}
	
	等价于证明,给定正实数$\delta$,使得$f$在圆环$D:0<|z-z_0|<\delta$上解析,给定有限复数$A$,给定正实数$\varepsilon$,那么必然存在一个复数$z\in D$,满足$|f(z)-A|<\varepsilon$.
	
	现在假设这个命题不成立,也就是说存在一个复数$A$,存在一个正实数$\varepsilon_0$,使得对任意的满足$0<|z-z_0|<\delta$的$z$,总有$|f(z)-A|\ge\varepsilon_0$.于是在$D$上有$F(z)=\frac{1}{f(z)-A}$解析并且有界.这说明$z_0$是$F(z)$的可去奇点.设$\lim_{z\to z_0}F(z)=a$,那么如果$a=0$则$z_0$是$f$的极点,如果$a\not=0$则$z_0$是$f$的可去奇点,这都和条件矛盾.
\end{proof}

Picard证明了一个比维尔斯特拉斯定理更为复杂的结论,即Picard大定理:解析函数在本性奇点的去心开邻域内无数次取到每个有限复数,至多可能除去一个例外值.

这里我们借助孤立奇点的分类给出皮卡小定理的理解.皮卡小定理本质上给出了整函数的三种分类:取值为单一复数的整函数;取值为全体复数域的整函数;取值为全体复数域除去一个单点的整函数.它们分别对应的,就是把无穷远点作为孤立奇点的三种类型.

我们称一个定义在无穷远的去心邻域中的解析函数$f(z)$在无穷远点是某个类型的孤立奇点,如果函数$f(1/z)$在$z=0$处是相应类型的孤立奇点.换句话讲,假设正实数$R$满足$f(z)$在$R<|z|<+\infty$上解析,那么$f(z)$在其上可以展开为洛朗级数$f(z)=\sum_{n<0}c_nz^n+\sum_{n\ge0}c_nz^n$.那么$f(1/z)$在$z=0$处的洛朗展开式就是$f(1/z)=\sum_{n<0}c_nz^{-n}+\sum_{n\ge0}c_nz^{-n}$.于是此时,无穷远是$f(z)$的可去奇点等价于$f(z)$的洛朗展开中正次数项都是0;是极点等价于$f(z)$的洛朗展开中只有有限个正次数项的系数不为0;是本性奇点等价于$f(z)$的洛朗展开中正次数项系数不为0的有无穷个.

现在倘若无穷远点是整函数的可去奇点,那么整函数有上界,本质上刘维尔定理就是在刻画这一类情况,此时整函数是常值函数,对应的是整函数分类中的第一类.倘若无穷远点是整函数的极点,按照$f(z)$的洛朗展开式,此时按照整函数条件,负次数项系数均为0,于是此时整函数是一个非常值的多项式函数,这对应的是整函数分类中的第二类,也即恰好取到整个复数域的函数,本质上这是代数基本定理所刻画的情况.最后倘若无穷远点是整函数的本性奇点,此时整函数可以描述为非多项式的整函数,这称作超越整函数,皮卡小定理本质上刻画了这一种情况,它断言超越整函数的取值恰好是复数域除去一个单点.

习题.单叶整函数必然是线性函数.
\begin{proof}
	
	我们只需说明这样的函数不能是次数大于1的多项式函数和超越整函数.为此,首先按照代数学基本定理,次数大于1的多项式函数不是单叶的.现在考虑超越整函数,我们知道此时无穷远点是它的本性奇点,于是可取无穷远的一个去心开邻域$U$,按照维尔斯特拉斯定理,$f(U)$会在$\mathbb{C}$中稠密.但是我们可以取一个和$U$不交的开集$V\subset\mathbb{C}$,按照开映射定理,得到$f(V)$是$\mathbb{C}$中的开集,于是稠密性说明$f(V)$和$f(U)$有交集,这和单射矛盾.
\end{proof}
\subsubsection{亚纯函数}

我们已经证明了开集$U$上一个解析函数的孤立奇点是离散的.而复平面上一个离散点集至多是可数集,这是因为对每个正整数$n$,闭圆盘$\overline{B_n(0)}$至多和给定的离散点集相交于有限个点,再对$n\in\mathbb{N}$取并即说明是至多可数集.如果$f$在开集$U$的除了一个由极点构成的离散点集以外处处全纯,就称$f$是$U$上的亚纯函数.

我们可以对扩充复平面上的整函数和亚纯函数做出刻画.
\begin{enumerate}
	\item 扩充复平面上的整函数是常值函数.事实上,函数在无穷远点处连续说明存在$r>0$,使得$|z|>r$的时候恒有$|f(z)-f(\infty)|<1$,这导致$|f(z)|<|f(\infty)|+1$,说明$|z|>r$时候$f$有界,而$|z|\le r$时由于是紧集上的连续函数,也是有界的,这得到$f$在$\mathbb{C}$上有界,有界整函数自然是常值函数,最后按照连续性得到无穷远点取值也和其他的点相同.
	\item 扩充复平面上的亚纯函数是有理函数.
	\begin{proof}
		
		首先扩充复平面是紧集,导致离散子集只能是有限点集,设$f$在复平面上的极点为$z_1,z_2,\cdots,z_k$.设$f$在$z_i$处洛朗展开式中全体负次数项构成的级数为$\phi_i$,非负次数项构成的级数为$\psi_i$.无论$\infty$处是不是极点,我们都记$f$在$\infty$处的洛朗级数的非负次数项构成的级数记作$\psi$,负次数项构成的级数记作$\phi$.现在记$F(z)=f(z)-\phi(z)-\sum_{i=1}^n\psi_i(z)$,它在$\mathbb{C}-\{z_1,z_2,\cdots,z_n\}$上解析.并且在这些点$z_i$和$\infty$处的极限都有限,于是它们是$F$的可去奇点,于是$F$可以延拓为扩充复平面上的解析函数,于是$F(z)\equiv C$,这里$\phi(z)$是有理函数,导致$f(z)$是有理函数.
	\end{proof}
\end{enumerate}
\subsubsection{留数定理}

设$z_0$为解析函数$f$的孤立奇点,那么在$z_0$的某个去心开圆盘上有洛朗展开式$f(z)=\sum_{n=-\infty}^{+\infty}a_n(z-z_0)^n$.称$a_{-1}=\frac{1}{2\pi i}\int_{|z-z_0|=\rho}f(z)\mathrm{d}z$为$f$在$z_0$处的留数,记作$\mathrm{res}_{z_0}f$.

留数定理实际上就是同源版本的柯西积分定理以及留数定义的简单推导:设$U$是开集,$\gamma$为$U$中的一个同源于0的闭链,设$f$是$U$上除了有限个极点$\{z_1,z_2,\cdots,z_n\}$的解析函数,记$z_i$关于$\gamma$的缠绕数为$m_i$,那么有如下留数公式.特别的,如果$\gamma$是不自交的$C^1$曲线,则所有缠绕数均为1.
$$\int_{\gamma}f=2\pi i\sum_{i=1}^{n}m_i\mathrm{Res}_{z_i}f$$

在实际应用时,熟悉如下两个结论是有益的:
\begin{enumerate}
	\item 如果$f$在$z_0$处是一次极点,$g$在$z_0$处解析,那么有:
	$$\mathrm{Res}_{z_0}(fg)=g(z_0)\mathrm{Res}_{z_0}(f)$$
	\item 如果$f(z_0)=0$,$f'(z_0)\not=0$,那么$\frac{1}{f}$在$z_0$处是一次极点,并且有:$$\mathrm{Res}_{z_0}\left(\frac{1}{f}\right)=\frac{1}{f'(z_0)}$$
\end{enumerate}

例子.柯西积分定理/留数定理证明Cayley-Hamilton定理.
\begin{proof}
	
	给定$n$阶复方阵$A$,定义$f(z)=\det(zE-A)$,那么$f(z)$是$\mathbb{C}$上的多项式函数,于是它是解析函数,另外它可以表示为$f(z)=\sum_{k=0}^n\frac{f^{(k)}(0)}{k!}z^k$.现在考虑$\mathbb{C}\to\mathbb{C}^{n^2}$的映射$g(z)=f(z)(zE-A)^{-1}$.取$\gamma:|z|=R>0$使得曲线上不包含任一特征值,为计算积分$\frac{1}{2\pi i}\int_{\gamma}g(z)\mathrm{d}z$,先注意到:
	$$g(z)=(f(0)+f'(0)z+\cdots+\frac{f^{(n)}(0)}{n!})(1+z^{-1}A+z^{-2}A^2+\cdots)=\left(\sum_{k=0}^n\frac{f^{(k)}(0)}{k!}A^k\right)z^{-1}+R(z)$$
	
	因此$g(z)$在该点的留数为$\frac{1}{2\pi i}\sum_{k=0}^n\frac{f^{(k)}(0)}{k!}A^k$,也就是$A$带入到特征多项式中.另一方面由于$g(z)$就是$zE-A$的伴随矩阵,它的分量都是多项式,都在$\mathbb{C}$上解析,说明积分为0,这就得到$A$带入到特征多项式中为0.
\end{proof}

接下来介绍幅角原理.首先注意到一个等式$\frac{(fg)'}{fg}=\frac{f'}{f}+\frac{g'}{g}$.于是如果$z_0$是$f$的$n$次零点,也就是说$f(z)=(z-z_0)^ng(z)$,其中$g(z)$是解析函数并且满足$g(z_0)\not=0$.那么有$\frac{f'(z)}{f(z)}=\frac{n}{z-z_0}+G(z)$,这里$G(z)=\frac{g'(z)}{g(z)}$,它在$z_0$足够小的开邻域上是解析函数.同理,如果$z_0$是$f$的$n$次极点,那么在$z_0$足够小的去心开邻域中会有$\frac{f'(z)}{f(z)}=\frac{-n}{z-z_0}+H(z)$,这里$H(z)$是$z_0$这个足够小的去心邻域上的解析函数.

于是上一段的讨论说明了,如果$f$是开集$U$上的亚纯函数,那么如果$z_0$是$f$的极点或者零点,则有$z_0$是$\frac{f'}{f}$的一次极点,并且如果$z_0$是$f$的极点,那么它作为$f$极点的次数就是$\frac{f'}{f}$留数的相反数;如果$z_0$是$f$的零点,那么它作为$f$的零点的次数就是$\frac{f'}{f}$的留数.

这就证明了幅角原理:如果$\gamma$是开集$U$上的同源于0的闭链,设$f$是$U$上的亚纯函数,并且只有有限个极点和零点$\{z_1,z_2,\cdots,z_n\}$,设$m_i$是$z_i$关于$\gamma$的缠绕数,设$t_i$的绝对值是$z_i$作为零点或极点的次数,它的符号约定为零点取正极点取负,那么有公式:
$$\int_{\gamma}\frac{f'(z)}{f(z)}\mathrm{d}z=2\pi i\sum_{k=1}^nm_kt_k$$

特别的,如果$\gamma$是$U$中不自交的闭$C^1$曲线,那么$\frac{1}{2\pi i}\frac{f'(z)}{f(z)}\mathrm{d}z$就是$C$内部的$f$的零点个数(计次数意义下)减去极点个数(计次数意义下).

Rouch\'e定理.设$U$是开集,$\gamma$是$U$中的同源于0的闭曲线,并且$\gamma$内部非空且包含于$U$中.设$f,g$是$U$上的两个解析函数,满足$|f(z)-g(z)|<|f(z)|$对任意$z\in\gamma$成立,那么$f,g$在$\gamma$的内部中具有相同个数的零点(计重数).
\begin{proof}
	
	注意条件中的不等式说明了$f,g$都不在$\gamma$上有零点.条件告诉我们$\left|\frac{g(z)}{f(z)}-1\right|<1$对任意$z\in\gamma$成立.于是函数$F=g/f$在曲线上的取值落在1为圆心半径为1的开圆盘中.这说明$0$关于曲线$F\circ\gamma$的缠绕数为0.不妨设$\gamma$定义在闭区间$[a,b]$上,那么有:
	$$0=W(F\circ\gamma,0)=\int_{F\circ\gamma}\frac{1}{z}\mathrm{d}z=\int_a^b\frac{F'(\gamma(t))}{F(\gamma(t))}\gamma'(t)\mathrm{d}t=\int_{\gamma}\left(\frac{g'}{g}-\frac{f'}{f}\right)\mathrm{d}z$$
	
	于是从幅角原理立刻得出结论.
\end{proof}

Rouch\'e定理可以推出开映射定理.一个区域$U$上的解析函数$f$如果不是常值函数,那么它是开映射,即把开集映射为开集.
\begin{proof}
	
	取$f$像集中的一个点$w_0=f(z_0)$,需要证明$w_0$在复平面中存在一个开邻域包含于像集.先取足够小的正数$\delta$使得$|z-z_0|\le\delta$落在$U$中,另外按照$f(z)-w_0$的零点的孤立性,还可以约定这个闭圆盘上除了点$z_0$以外处处不满足$f(z)=w_0$.特别的,圆周$|z-z_0|=\delta$上处处有$f(z)\not=w_0$.于是圆周这个紧集上连续函数$|f(z)-w_0|$就有最小值$\varepsilon>0$.现在假设$|w-w_0|<\varepsilon$,取$F(z)=f(z)-w_0$和$G(z)=w_0-w$,那么在圆周上总有$|F(z)|>|G(z)|$.按照Rouch\'e定理,得到$F(z)$和$F(z)+G(z)=f(z)-w$具有相同个数的零点,而前者有零点,这就说明后者也有零点,于是$w_0$的一个足够小的开邻域包含于$f$的像集中,说明$f$是开映射.
\end{proof}

Rouch\'e定理还可以直接证明代数学基本定理.不妨设一个非常数的多项式为$f(z)=z^n+a_1z^{n-1}+\cdots+a_n$,那么有$\lim_{z\to\infty}\frac{f(z)}{z^n}=1$,于是存在足够大的实数$R>0$,满足$|z|\ge R$时候有$\left|\frac{f(z)}{z^n}-1\right|<1$,于是得到$|z|=R$的时候有$|f(z)-z^n|<|z^n|$,这说明$f(z)$和$z^n$在$|z|<R$内有相同个数的零点,于是$f(z)$在$|z|<R$内有$n$个根.

开映射定理可以直接说明最大模原理,因为倘若$f$非常值并且在一个区域$D$内的点$z_0$取到绝对值的最大值,但是$f(D)$同样是开集,导致存在$f(z_0)$足够小的开圆盘落在像集中,这和$|f(z_0)|$取$f(D)$绝对值的最大值矛盾.
\newpage
\subsection{调和函数}

称一个二元实值函数$u$是调和函数,如果它是$C^2$的函数,并且满足$\frac{\partial^2u}{\partial x^2}+\frac{\partial^2u}{\partial y^2}=0$.我们可以定义拉普拉斯算符$\Delta=\left(\frac{\partial}{\partial x}\right)^2+\left(\frac{\partial}{\partial y}\right)^2$,那么调和函数的第二个条件等价于$\Delta u=0$.

按照Cauchy-Riemann方程,一个解析函数的实部和虚部总是调和函数.反过来我们会证明,一个调和函数局部上是某个解析函数的实部.

这一段我们定义两个关于复变函数的微分算子$\frac{\partial}{\partial z}$和$\frac{\partial}{\partial\overline{z}}$.我们知道有等式$2x=z+\overline{z}$和$2iy=z-\overline{z}$.于是为了使得链式法则成立,合理的定义是:
$$\frac{\partial}{\partial z}=\frac{1}{2}\left(\frac{\partial}{\partial x}+i\frac{\partial}{\partial y}\right);\frac{\partial}{\partial\overline{z}}=\frac{1}{2}\left(\frac{\partial}{\partial x}-i\frac{\partial}{\partial y}\right)$$

在这个新的算子记号下,我们可验证Cauchy-Riemann方程等价于$\frac{\partial f}{\partial\overline{z}}=0$.于是一个复变函数$f$解析当且仅当$\frac{\partial f}{\partial\overline{z}}=0$,可以说解析函数本质上是只为关于变量$z$的复变函数.另外按照偏导数均可交换,我们看到拉普拉斯算符可以写作$\Delta=4\frac{\partial}{\partial z}\frac{\partial}{\partial\overline{z}}=4\frac{\partial}{\partial\overline{z}}\frac{\partial}{\partial z}$.并且如果把$f$的实部记作$u$,那么导函数的表达式为$f'(z)z=2\frac{\partial u}{\partial z}=\frac{\partial u}{\partial x}-i\frac{\partial u}{\partial y}$.

设$U$是一个单连通的开子集,$u$是$U$上的调和函数,则存在$U$上的解析函数$f$使得它以$u$为实部.唯一性结论:满足上述条件的$U$上的解析函数总在$U$上相差一个纯虚数.
\begin{proof}
	
	取$h(z)=2\frac{\partial u}{\partial z}=\frac{\partial u}{\partial x}-i\frac{\partial u}{\partial y}$.那么按照调和函数的定义,$h(z)$是$C^1$的函数.另外$\frac{\partial h}{\partial\overline{z}}=\frac{1}{2}\Delta u=0$,于是$h$实际上是$U$上的解析函数.由于约定了$U$是单连通域,于是$h$在$U$上有原函数,记作$f$,那么$f'(z)=h(z)$.记$f$的实部为$u_1$,按照$f$导数的公式,$u$和$u_1$具有相同的一阶偏导数.处处偏导数为0的函数必然是常值的.说明$u$和$u_1$仅相差一个常数,记$u=u_1+C$.现在取$g=f-C$,于是$g$是$U$上的解析函数并且实部为$u$.
	
	为了证明唯一性,只需证明,如果$f,g$都是区域$U$上的解析函数,并且具有相同的实部,则它们相差一个纯虚数.为此只要考虑差$h=f-g$,它是解析函数,实部恒为0,于是按照刘维尔定理,它是常值函数,此时必然是纯虚数.
\end{proof}

调和函数是光滑的,即$C^{\infty}$函数.
\begin{proof}
	
	设$u$是开集$U$上的调和函数,任取点$z_0\in U$,则可取$D=B_r(z_0)\subset U$,那么$D$是一个单连通域,于是存在$D$上的解析函数$f$使得它以$u$为实部.但是我们知道解析函数是任意阶连续可导的,于是$f$的实部$u$也是任意阶偏导数都存在且连续的.?
\end{proof}




\newpage
\section{全纯函数的几何性质}
\subsection{共形映射}

给定复平面的一个开集$U$,给定$U$内的一个可微曲线$\gamma:[a,b]\to U$,那么可记$\gamma(t)=x(t)+iy(t)$,这里$x(t),y(t)$是$[a,b]\to\mathbb{R}$的可微函数.此时曲线的导数就是$\gamma'(t)=x'(t)+iy'(t)$.如果$f$是$U$上的解析函数,那么$f\circ\gamma$同样是一个可微曲线,它的导数就是$\frac{d}{dt}f(\gamma(t))=f'(\gamma(t))\gamma'(t)$.

几何上讲,如果曲线$\gamma(t)$在一点$t_0$处的导数非0,那么$\gamma'(t_0)$是曲线在点$\gamma(t_0)$处的切向量.现在设有两条可微曲线$\gamma_1,\gamma_2$经过点$z_0\in U$,设$z_0=\gamma_1(t_1)=\gamma_2(t_2)$.如果点$z_0$处两个曲线的切向量都非0,则称两个切向量$\gamma_1'(t_1)$和$\gamma_2'(t_2)$的夹角为两个曲线在交点$z_0$处的夹角.

现在把解析函数$f$复合到这两个曲线上,即得到两个新可微曲线$f\circ\gamma_1$和$f\circ\gamma_2$.它们都经过点$f(z_0)$,按照链式法则,对应的切向量分别是$f'(z_0)\gamma_1'(t_1)$和$f'(z_0)\gamma_2'(t_2)$.我们断言,如果$f'(z_0)\not=0$,那么$\gamma_i,i=1,2$在交点$z_0$处的夹角,等于$f\circ\gamma_i,i=1,2$在交点$f(z_0)$处的夹角.几何上讲,两个切向量均乘以了一个相同的非0复数$f'(z_0)$,这相当于数乘了这个复数的模长和旋转了这个复数的角度,因此夹角不会改变.下面我们严格给出描述.

记非0复数$z=a+bi,w=c+di$,定义复数的内积为$(z,w)=\Re{z\overline{w}}=ac+bd$.也即这个内积就是把$\mathbb{C}$视为$\mathbb{R}^2$后的常义内积.此时两个复数视为以原点为初始点的向量时的夹角$\theta(z,w)$就满足$\cos\theta(z,w)=\frac{(z,w)}{|z||w|}$.这说明$\sin\theta(z,w)=\cos\left(\theta(z,w)-\frac{\pi}{2}\right)=\frac{(z,-iw)}{|z||w|}$.现在设$f'(z_0)=\alpha$,那么有:
$$(\alpha z,\alpha w)=\Re(\alpha z\overline{\alpha}\overline{w})=\alpha\overline{\alpha}\Re(z\overline{w})=|\alpha|^2(z,w)$$

这就说明了$\cos\theta(\alpha z,\alpha w)=\cos\theta(z,w)$和$\sin\theta(\alpha z,\alpha w)=\sin\theta(z,w)$.也就严格给出了上述断言的证明.

称保曲线在交点上夹角的函数为复平面上的保角映射.于是我们看到导函数不为0的解析函数总是一个保角映射.如果一个欧氏空间上的映射同时保定向和局部的夹角,则称它是共形映射.这里保定向指的是,所诱导的切空间之间的映射,保基的定向.在复平面中同样可以定义共形映射,并且开集$U$上的复变函数$f$是共形映射当且仅当它是导数在$U$上处处不取0的解析函数.注意这样的解析函数的共轭函数同样会保曲线交点的夹角,但是它不会保定向.

于是按照上述定义,复平面上我们提及共形映射指的是导函数处处不取0的解析函数.但是有的教材里会定义共形映射是指开集上的单解析函数.下面一个结论会说明这个新定义可以推出上一段的定义.不过第一个定义不会推出第二个定义,例如考虑$f(z)=e^z$.另外结合开映射定理和整体解析同构定理,这个新定义下会迫使共形映射是开集$U$到开集$V$上的解析双射,并且逆映射同样是解析函数.在本文中我们运用第二个定义.

如果$f:U\to\mathbb{C}$是单的解析函数,那么$f'(z)$在$U$上处处不为0.反证,假设存在$z_0\in U$使得$f'(z_0)=0$,那么在$z_0$附近有幂级数展开式$f(z)-f(z_0)=a(z-z_0)^k+G(z)$,其中$a\not=0$,并且$k\ge2$,并且$G(z)$在点$z_0$处至少是$k+1$次的零点.取模长足够小的复数$w$,记$F(z)=a(z-z_0)^k-w$,则$f(z)-f(z_0)-w=F(z)+G(z)$.那么在一个更小的$z_0$的开邻域内就有$|G(z)|<|F(z)|$.另外$F$在这个开邻域内至少有两个不同的零点,Rouch\'e定理告诉我们这说明$f(z)-f(z_0)-w$在这个足够小的$z_0$的开邻域内至少有两个不同的零点,这就和单射条件矛盾.

给定开集上的共形映射$f:U\to\mathbb{C}$,那么我们已经知道了此时$f(U)=V$也是开集,并且$f:U\to V$是解析同构,它的逆映射$g:V\to U$也是解析函数.于是两个开集之间存在共形映射是全体开集上的一个等价关系,这个关系称为共形等价.一个自然的问题就是如何判断两个开集是否共形等价.对于这个问题有一个重要结论是黎曼映射定理,定理断言非整个复平面的单连通区域总是和单位开圆盘之间存在共形映射.换句话讲单连通区域之间总是共形等价的.本节我们先来给出共形映射的一些基本例子.

施瓦兹引理.设$D$是以0为圆心的单位开圆盘,设$f:D\to D$是解析函数,满足$f(0)=0$,那么有$|f(z)|\le|z|,z\in D$,并且倘若存在一个$z_0\in D,z_0\not=0$使得这个不等式取等号,那么存在一个模长为1的复数$\alpha$满足$f(z)=\alpha z$.
\begin{proof}
	
	设$f$在$0$处的幂级数展开为$f(z)=a_1z+\cdots$.于是$f(z)/z$是解析函数.如果$|z|=r<1$,那么有$|f(z)/z|<1/r$.按照最大模原理,得到$|z|\le r<1$的时候总有$|f(z)/z|<1/r$.固定$z\in D$令$r$趋于1得到$|f(z)|\le|z|$总成立.
	
	现在假设存在$z_0$使得$|f(z_0)|=|z_0|$.按照最大模原理说明$f(z)/z$只能在$D$上是常值函数,这就说明存在模长1的复数$\alpha$满足$f(z)=\alpha z,\forall z\in D$.
\end{proof}

用施瓦兹引理能够描述开圆盘上的全体共形映射.首先给定模长小于1的复数$\alpha$,记$g_{\alpha}(z)=\frac{\alpha-z}{1-\overline{\alpha}z}$.那么$g_{\alpha}(z)$是定义在单位闭圆盘$|z|\le1$上的解析函数.并且如果$|z|=1$,则$|g_{\alpha}(z)|=1$.按照最大模原理,得到$|z|\le1$时有$|g_{\alpha}(z)|\le1$.再按照开映射定义,说明$|z|<1$的时候$|g_{\alpha}(z)|<1$.于是$g_{\alpha}(z)$是单位开圆盘$D:|z|<1$到自身的解析函数.另外容易验证$g_{\alpha}\circ g_{\alpha}=\mathrm{id}_D$.于是$g_{\alpha}$总是$D\to D$的共形映射.

现在我们描述$D$到自身的全部共形映射.设$f:D\to D$是以0为圆心的单位开圆盘上的共形映射,设有$f(\alpha)=0$,那么存在一个实数$\phi$满足$f(z)=e^{i\phi}\frac{\alpha-z}{1-\overline{\alpha}z}$.特别的,满足$f(0)=0$的$D$的共形映射只有$f(z)=e^{i\phi}z$,也即旋转变换.
\begin{proof}
	
	取上一段中的$g_{\alpha}$记作$g$,那么$h=f\circ g^{-1}=f\circ g$是$D$自身的一个共形映射,并且它把0映射为0.于是按照施瓦兹引理的第一部分,说明$|z|<1$的时候有$|h(z)|\le|z|$.而对$h^{-1}$同样运用施瓦兹引理的第一部分,得到$|z|\le|h(z)|,|z|<1$,这说明有$|h(z)|=|z|,|z|<1$,再运用施瓦兹引理的第二部分,得出$h(z)=e^{i\phi}z$.
\end{proof}

给定一个开区域$U$,那么$U$到自身的全部共形映射在复合运算下构成了一个群,这里姑且记作$\mathrm{Aut}(U)$.假设有开区域$U$到开区域$V$的共形映射$f$,那么一点代数知识告诉我们自同构群满足关系:$\mathrm{Aut}(U)=f^{-1}\mathrm{Aut}(V)f$.所以一旦我们知道黎曼映射定理,并且找到单连通区域到$D$的一个共形映射,结合已经得到了$\mathrm{Aut}(D)$,这就使得我们可以表示出单连通区域自身上的全部共形映射.

这里以上半复平面这个单连通区域为例.记$H=\{z=x+yi\mid y>0\}$是上半复平面,则解析函数$f(z)=\frac{z-i}{z+i}$是$H$到单位开圆盘$D$的共形映射.这个验证是直接的,它的逆映射是$g(z)=-i\frac{z+1}{z-1}$.据此我们可以直接得出$\mathrm{Aut}(H)$的完全刻画:给定一个二阶实矩阵$A=(a_{ij})$,设行列式1,那么$f_M(z)=\frac{a_{11}z+a_{12}}{a_{21}z+a_{22}}$是$H$自身上的共形映射.反过来任意$H$自身上的共形映射具有这个形式.另外$f_M$和$f_{M'}$是相同的共形映射当且仅当$M'=\pm M$.
\newpage
\subsection{解析延拓}

解析延拓是指把开集上的解析函数延拓为更大的开集上的解析函数.我们通常会要求这两个开集都是连通集,于是按照零点孤立性定理,这样的解析延拓如果存在则是唯一的.

施瓦兹对称原理.设$U^+$是上半复平面中的一个连通开集,并且设$U^+$的边界中包含了实轴上的一个开区间$I$.记$U^-=\{\overline{z}\mid z\in U^+\}$是$U^+$关于实轴的对称集.记$U=U^+\cup I\cup U^-$并假设它是开集.
\begin{enumerate}
	\item 若$f$是$U$上的复变函数,在$U^+$和$U^-$上解析,在$I$上连续,那么$f$实际上是$U$上的解析函数.
	\item 若$f$是$U^+\cup I$上的复变函数,在$U^+$上解析,在$I$上连续并且取实数,那么存在$f$在$U$上的解析延拓$F$,满足$F(z)=\overline{f(\overline{z})}$.
\end{enumerate}
\begin{proof}
	
	先来说明1推出2.可直接按命题中定义$U$上的函数$F$,验证Cauchy-Riemann方程可知$F$在$U^-$上同样解析,按照$f$在$I$上取实数,得到$F$在$I$上连续,于是按照第一个命题就说明$F$是解析函数.
	
	按照Morera定理,只需验证对任一$U$内的可求长简单闭曲线$l$有$\int_lf(z)\mathrm{d}z=0$.为此,设落在$U^+\cup I$中的那段可求长曲线为$l^+$,落在$U^-\cup I$中的那段为$l^-$,那么有$\int_lf(z)\mathrm{d}z$可以分解为$f(z)$在$l^+$和$I$凑成的定向简单闭曲线上的积分,加上$f(z)$在$l^-$和$I$凑成的定向简单闭曲线上的积分.按照$f(z)$分别在$U^+$和$U^-$上解析,并且分别在$U^+\cup I$和$U^-\cup I$上连续,说明上述两个积分为0,于是和也是0.
\end{proof}
\newpage
\subsection{黎曼映射定理}

本节最后我们整理一个关于非单连通域共形变换为非单连通域的例子.首先我们注意共形变换作为解析双射,它必然也是一个同胚,而单连通性是拓扑不变性,这说明了不会存在非单连通域共形变换为单连通域.其次,在复平面上单连通域具有这样一个等价描述,一个开集是单连通域当且仅当它在复平面中的补集是连通的.由此可定义多连通域,称一个开集是$n$连通域,如果它在复平面中的补集具有$n$个连通分支.多联通域上同样具有类似黎曼映射定理的结论,以二连通区域为例,每个二连通区域都可共形变换变换为圆环$1<|z|<R$.这里的$R$被原二联通区域所决定,不能任意选取.事实上对$1<R_1<R_2$,总有$1<|z|<R_1$和$1<|z|<R_2$不是共形等价的.


